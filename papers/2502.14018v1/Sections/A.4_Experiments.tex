\newpage
\section{Implementations of the used HSTs}
\label{app:experiments}

We build KD trees, Cover trees using the C++ library \emph{mlpack4} \citep{mlpack}, and build HST-DPO trees using the C++ code from \url{https://github.com/yzengal/ICDE21-HST}.


\section{Used Datasets}
Table \ref{tbl:dataset_overview} lists the datasets on which we validated and compared the proposed framework SHIP to other competitors.

\section{Tables -- Runtimes, ARI, and NMI}
Table \ref{tbl:runtimes_full} shows the runtimes, Table \ref{tbl:ari_full} the ARI values, and Table \ref{tbl:nmi_full} the NMI values for all datasets. 


\begin{table}[t]
\small
\centering
\setlength\tabcolsep{2pt} % default value: 6pt
\begin{tabular}{llccccc}
\toprule
\multicolumn{2}{l}{} & 
\textbf{Dataset} & 
%\textbf{\# Lines} & 
\textbf{Languages} & 
\\
\midrule
\multicolumn{4}{l}{\textbf{Slot Filling}} \\
& Task - train 
& \multirow{1}{*}{\shortstack[l]{MASSIVE}}
%& 57$k$
& \{ar, en, es, ru, zh\}
\\
\cmidrule{4-5}
& \multirow{2}{*}{\shortstack[l]{Task - test}}
& \multirow{2}{*}{\shortstack[l]{MASSIVE}}
%& 2974 per lang.
& supervised + \{af, az, cy, de, el,  \\
&
&
&
fr, hi, is, ja, jv, sw, th, tl, tr, ur\}
\\
%\midrule
\cmidrule{4-5}
& \multirow{3}{*}{\shortstack[l]{Alignment}}
& \multirow{3}{*}{\shortstack[l]{Tatoeba}}
%& ?
& low-res.: \{cy, jv, jp, sw, tl\}-en
\\
&
&
& mid-res.: \{el, hi, th, tr\}-en
\\
&
&
& high-res.: \{ar, es, ru, zh\}-en
\\
\midrule
\multicolumn{4}{l}{\textbf{Machine Translation}} \\
& Task - train 
& ALMA
%& 117$k$
&  \{cs, de, is, ru, zh\}$ \leftrightarrow$ en
\\
& Task - test
& WMT 23
& supervised + \{he, ja, uk\} $\leftrightarrow$ en
\\
& Alignment
& \multicolumn{2}{c}{{(same as ``Task - train'')}}
\\
\midrule
\multicolumn{4}{l}{\textbf{JSON Generation} (challenge task)}  \\
& Task - train 
& UNER
%& 117$k$
&  \{en, pt, zh\}
\\
& Task - test
& UNER
%& 117$k$
&  supervised + \{da, hr, sk, sr, sv\}
\\
& Alignment
& Tatoeba
%& 117$k$
&  \{da, sv\}-en
\\
\midrule
\multicolumn{4}{l}{\textbf{Semantic Alignment Evaluation}} \\
&
Alignment &
FLoRes-200 &
$N(N-1)$ pairs for $N$ lang.
\\
% \multicolumn{4}{l}{\textbf{Grammatical gender control} (feminine/masculine)} \\
% & train 
% &  \multicolumn{2}{l}{en$\rightarrow$es}
% & 194
% \\
% &
% test (supervised)
% & 
%  \multicolumn{2}{l}{en$\rightarrow$es}
% & 552-556 %\verb|~|
% \\
% &
% test (new tgt)
% &  \multicolumn{2}{l}{en$\rightarrow$\{it, fr\}}
% & 515-546
% \\
% &
% test (new src$+$tgt)
% & \multicolumn{2}{l}{\{es, fr\}$\rightarrow$it, \{es, it\}$\rightarrow$fr}
% & 271-365
% \\
\bottomrule
\end{tabular}
\caption{\label{tab:data_overview} 
Dataset overview. 
More details in \autoref{sec:appendix_dataset_details}.
}
\end{table}


%%% Full Runtimes Table
\newpage

\begin{largetable}[htb]
    \centering
    \caption{
        Runtimes of various parts of our clustering framework SHIP over ten runs, compared with standard clustering methods. 
        Time is given in minutes, seconds, and milliseconds [min:sec.ms]. 
        The first four columns (``build'') refer to the computation time of the different ultrametric trees. The next two columns state the runtimes of computing one hierarchy ($k$-means) on the DC tree. The second column here shows the timing for the heuristic from Section \ref{app:ties_heuristic}.
        Note that these runtimes are not correlated to the chosen ultrametric.
        The subsequent three columns state the runtimes of different partitioning methods, i.e., `Stability', `Median of Elbows (MoE)', and the `Elbow' method on various hierarchies.
    }
    \label{tbl:runtimes_full}
    \vspace*{0.5em}
    \renewcommand{\arraystretch}{1.2}
    \resizebox{1\linewidth}{!}{
\begin{NiceTabular}{|rl|ccccccccc|cccccc|}
\CodeBefore
    \rowcolors{4}{gray!20}{white}[cols=2-*]
\Body
\toprule
& & \multicolumn{4}{c}{\textbf{build ultrametric}} & \multicolumn{2}{c}{\textbf{build hierarchy}} & \multicolumn{3}{c}{\textbf{partitioning}} & & & \multicolumn{2}{c}{\textbf{competitors}} & & \\
& & \multirow{2}{*}{KD tree} & \multirow{2}{*}{Cover tree} & \multirow{2}{*}{HST-DPO} & \multirow{2}{*}{DC tree} & \multirow{2}{*}{$k$-means} & $k$-means & \multirow{2}{*}{Stability} & \multirow{2}{*}{MoE} & \multirow{2}{*}{Elbow} & $k$-means & $k$-means & \multirow{2}{*}{SCAR} & \multirow{2}{*}{Ward} & AMD- & \multirow{2}{*}{DPC}\\
& \textbf{Dataset} & & & & & & (heur.) & & & & ($k=\text{GT}$) & ($k=500$) & & & DBSCAN & \\
\midrule
\parbox[t]{2mm}{\multirow{17}{*}{\rotatebox[origin=c]{90}{Density-based 2D-Data}}}

& Boxes & 00:00.013 & 00:00.059 & 00:09.679 & 00:14.239 & 00:00.033 & 00:00.498 & 00:00.005 & 00:00.134 & 00:00.002 & 00:00.114 & 00:01.217 & 00:01.481 & 00:10.808 & 01:04.670 & 12:10.358 \\
& D31 & 00:00.002 & 00:00.004 & 00:00.207 & 00:00.327 & 00:00.004 & 00:00.020 & 00:00.000 & 00:00.015 & 00:00.000 & 00:00.289 & 00:00.509 & 00:00.620 & 00:00.153 & 00:00.878 & 00:12.816 \\
& 3-spiral & 00:00.000 & 00:00.000 & 00:00.002 & 00:00.009 & 00:00.000 & 00:00.001 & 00:00.000 & 00:00.000 & 00:00.000 & 00:00.095 & 00:00.131 & 00:00.023 & 00:00.002 & 00:00.025 & 00:00.091 \\
& aggregation & 00:00.000 & 00:00.000 & 00:00.024 & 00:00.035 & 00:00.001 & 00:00.004 & 00:00.000 & 00:00.004 & 00:00.000 & 00:00.227 & 00:00.204 & 00:00.038 & 00:00.010 & 00:00.121 & 00:00.663 \\
& chainlink & 00:00.000 & 00:00.001 & 00:00.042 & 00:00.045 & 00:00.001 & 00:00.008 & 00:00.000 & 00:00.004 & 00:00.000 & 00:00.133 & 00:00.171 & 00:00.049 & 00:00.012 & 00:00.159 & 00:01.128 \\
& cluto-t4-8k & 00:00.004 & 00:00.017 & 00:01.359 & 00:01.898 & 00:00.012 & 00:00.164 & 00:00.001 & 00:00.046 & 00:00.000 & 00:01.344 & 00:01.514 & 00:00.417 & 00:01.233 & 00:06.470 & 01:32.488 \\
& cluto-t5-8k & 00:00.005 & 00:00.013 & 00:01.319 & 00:01.836 & 00:00.010 & 00:00.198 & 00:00.001 & 00:00.042 & 00:00.000 & 00:00.682 & 00:01.758 & 00:00.378 & 00:01.256 & 00:05.942 & 01:33.225 \\
& cluto-t7-10k & 00:00.006 & 00:00.019 & 00:02.100 & 00:02.848 & 00:00.014 & 00:00.252 & 00:00.002 & 00:00.058 & 00:00.000 & 00:02.532 & 00:02.052 & 00:00.455 & 00:02.058 & 00:10.670 & 02:39.063 \\
& cluto-t8-8k & 00:00.005 & 00:00.014 & 00:01.328 & 00:01.728 & 00:00.012 & 00:00.158 & 00:00.001 & 00:00.046 & 00:00.000 & 00:01.930 & 00:01.899 & 00:00.327 & 00:01.394 & 00:09.212 & 01:39.524 \\
& complex8 & 00:00.001 & 00:00.003 & 00:00.129 & 00:00.244 & 00:00.003 & 00:00.011 & 00:00.000 & 00:00.012 & 00:00.000 & 00:00.565 & 00:00.592 & 00:00.221 & 00:00.120 & 00:01.295 & 00:10.900 \\
& complex9 & 00:00.001 & 00:00.004 & 00:00.185 & 00:00.304 & 00:00.003 & 00:00.014 & 00:00.000 & 00:00.012 & 00:00.000 & 00:00.745 & 00:00.682 & 00:00.175 & 00:00.158 & 00:01.376 & 00:12.483 \\
& compound & 00:00.000 & 00:00.000 & 00:00.004 & 00:00.011 & 00:00.000 & 00:00.001 & 00:00.000 & 00:00.000 & 00:00.000 & 00:00.066 & 00:00.127 & 00:00.036 & 00:00.003 & 00:00.031 & 00:00.204 \\
& dartboard1 & 00:00.000 & 00:00.002 & 00:00.027 & 00:00.051 & 00:00.001 & 00:00.003 & 00:00.000 & 00:00.004 & 00:00.000 & 00:00.210 & 00:00.270 & 00:00.033 & 00:00.016 & 00:00.138 & 00:01.346 \\
& diamond9 & 00:00.001 & 00:00.004 & 00:00.183 & 00:00.310 & 00:00.003 & 00:00.017 & 00:00.000 & 00:00.012 & 00:00.000 & 00:00.321 & 00:00.751 & 00:00.194 & 00:00.153 & 00:01.107 & 00:12.540 \\
& jain & 00:00.000 & 00:00.000 & 00:00.004 & 00:00.006 & 00:00.000 & 00:00.001 & 00:00.000 & 00:00.000 & 00:00.000 & 00:00.005 & 00:00.045 & 00:00.024 & 00:00.002 & 00:00.028 & 00:00.101 \\
& pathbased & 00:00.000 & 00:00.000 & 00:00.002 & 00:00.006 & 00:00.000 & 00:00.001 & 00:00.000 & 00:00.000 & 00:00.000 & 00:00.013 & 00:00.039 & 00:00.023 & 00:00.001 & 00:00.024 & 00:00.061 \\
& smile1 & 00:00.000 & 00:00.002 & 00:00.024 & 00:00.036 & 00:00.001 & 00:00.005 & 00:00.000 & 00:00.004 & 00:00.000 & 00:00.321 & 00:00.319 & 00:00.048 & 00:00.017 & 00:00.139 & 00:01.392 \\

\midrule
\parbox[t]{2mm}{\multirow{8}{*}{\rotatebox[origin=c]{90}{Tabular Data}}}

& Synth\_low & 00:00.040 & 00:00.087 & 00:04.264 & 00:01.090 & 00:00.007 & 00:00.128 & 00:00.001 & 00:00.027 & 00:00.000 & 00:00.351 & 00:03.754 & 00:00.367 & 00:00.999 & 00:02.879 & 00:33.945 \\
& Synth\_high & 00:00.034 & 00:00.030 & 00:04.075 & 00:00.878 & 00:00.006 & 00:00.117 & 00:00.001 & 00:00.024 & 00:00.000 & 00:00.782 & 00:04.199 & 00:00.455 & 00:00.982 & 00:04.573 & 00:33.785 \\
& Mice & 00:00.003 & 00:00.006 & 00:00.126 & 00:00.052 & 00:00.001 & 00:00.013 & 00:00.000 & 00:00.004 & 00:00.000 & 00:00.173 & 00:00.876 & 00:00.054 & 00:00.073 & 00:00.482 & 00:01.423 \\
& airway & 00:00.008 & 00:00.027 & 00:04.054 & 00:04.997 & 00:00.021 & 00:00.592 & 00:00.003 & 00:00.085 & 00:00.001 & 00:00.122 & 00:00.392 & 00:00.651 & 00:05.243 & 00:16.822 & 06:46.726 \\
& lactate & 00:00.025 & 00:00.161 & 00:39.433 & 00:47.731 & 00:00.068 & 00:07.184 & 00:00.012 & 00:00.275 & 00:00.004 & 00:00.126 & 00:01.997 & 00:02.612 & 00:59.992 & 07:43.293 & 69:34.012 \\
& HAR & 00:00.341 & 00:11.053 & 01:55.260 & 00:23.144 & 00:00.014 & 00:01.936 & 00:00.002 & 00:00.056 & 00:00.000 & 00:02.248 & 00:08.478 & 00:00.658 & 00:18.448 & 00:30.129 & 03:45.745 \\
& letterrec. & 00:00.024 & 00:00.322 & 00:14.396 & 00:09.076 & 00:00.022 & 00:01.103 & 00:00.002 & 00:00.087 & 00:00.001 & 00:03.386 & 00:03.521 & 00:02.133 & 00:10.309 & 00:37.055 & 11:42.910 \\
& PenDigits & 00:00.015 & 00:00.067 & 00:04.625 & 00:02.566 & 00:00.013 & 00:00.416 & 00:00.002 & 00:00.053 & 00:00.000 & 00:01.001 & 00:02.725 & 00:00.490 & 00:03.281 & 00:09.254 & 03:21.409 \\

\midrule
\parbox[t]{2mm}{\multirow{6}{*}{\rotatebox[origin=c]{90}{Image Data}}}

& COIL20 & 00:01.026 & 00:03.658 & 01:04.304 & 00:14.817 & 00:00.001 & 00:00.379 & 00:00.000 & 00:00.004 & 00:00.000 & 00:01.637 & 00:13.524 & 00:00.310 & 00:08.941 & 00:02.378 & 00:11.710 \\
& COIL100 & 00:36.361 & 03:52.397 & 82:50.274 & 14:50.661 & 00:00.010 & 00:22.517 & 00:00.001 & 00:00.039 & 00:00.000 & 00:32.122 & 01:59.225 & 00:07.964 & 12:04.037 & 02:58.177 & 14:18.780 \\
& cmu\_faces & 00:00.021 & 00:00.046 & 00:00.682 & 00:00.238 & 00:00.001 & 00:00.010 & 00:00.000 & 00:00.002 & 00:00.000 & 00:00.206 & 00:00.661 & 00:00.116 & 00:00.064 & 00:00.346 & 00:00.515 \\
& OptDigits & 00:00.019 & 00:00.335 & 00:03.442 & 00:01.420 & 00:00.006 & 00:00.162 & 00:00.001 & 00:00.024 & 00:00.000 & 00:00.502 & 00:01.154 & 00:00.290 & 00:00.974 & 00:03.073 & 00:44.270 \\
& USPS & 00:00.117 & 00:02.924 & 00:45.038 & 00:08.670 & 00:00.011 & 00:01.843 & 00:00.002 & 00:00.045 & 00:00.000 & 00:02.709 & 00:08.493 & 00:01.433 & 00:06.092 & 00:29.259 & 01:48.607 \\
& MNIST & 00:10.649 & 16:24.220 & 131:03.099 & 37:05.095 & 00:00.120 & 03:21.012 & 00:00.030 & 00:00.491 & 00:00.008 & 00:04.320 & 03:29.969 & 00:15.352 & 19:02.498 & 17:21.183 & - \\

\bottomrule
\CodeAfter
  \tikz \draw [dotted, thick] 
  (1-|7) -- (last-|7)
  (1-|9) -- (last-|9);
\end{NiceTabular}}
    \renewcommand{\arraystretch}{1}
\end{largetable}



%%% Full ARI Table
\newpage
\begin{largetable}[htb]
    \centering
    \caption{
        ARI values for various combinations within our clustering framework SHIP over ten runs, compared with standard clustering methods. Noise points were handled as singleton clusters. This shows that the results yield high-quality clusterings when using the DC tree as an ultrametric.
    }
    \label{tbl:ari_full}
    \vspace*{0.5em}
    \renewcommand{\arraystretch}{1.2}
    \resizebox{1\linewidth}{!}{
\begin{NiceTabular}{|rl|ccccccccccccccc|ccccc|}%[color-inside]
\CodeBefore
    \rowcolors{4}{gray!20}{white}[cols=2]
\Body
\toprule
& & \multicolumn{6}{c}{DC tree} & \multicolumn{3}{c}{HST-DPO} & \multicolumn{3}{c}{Cover tree} & \multicolumn{3}{c}{KD tree} & \multicolumn{5}{c}{\textbf{competitors}} \\ 
& & $k$-center & $k$-median & $k$-means & $k$-center & $k$-median & $k$-means & $k$-center & $k$-median & $k$-means & $k$-center & $k$-median & $k$-means & $k$-center & $k$-median & $k$-means & Eucl. & \multirow{2}{*}{SCAR} & \multirow{2}{*}{Ward} & AMD- & \multirow{2}{*}{DPC} \\ 
& \textbf{Dataset} & GT & GT & GT & Stability & MoE & Elbow & Stability & MoE & Elbow & Stability & MoE & Elbow & Stability & MoE & Elbow & $k$-means &  &  & DBSCAN & \\
\midrule
\parbox[t]{2mm}{\multirow{17}{*}{\rotatebox[origin=c]{90}{Density-based 2D-Data}}}
& Boxes & \cellcolor{Green!48}$66.5$ & \cellcolor{Green!69}$\bm{99.3}$ & \cellcolor{Green!69}$\bm{99.3}$ & \cellcolor{Green!63}$90.1$ & \cellcolor{Green!69}$\bm{99.3}$ & \cellcolor{Green!68}$\underline{97.9}$ & \cellcolor{Green!6}$2.9 \pm 0.1$ & \cellcolor{Green!34}$45.5 \pm 7.8$ & \cellcolor{Green!28}$35.5 \pm 9.3$ & \cellcolor{Green!6}$2.6$ & \cellcolor{Green!32}$42.1 \pm 4.7$ & \cellcolor{Green!20}$24.2 \pm 1.6$ & \cellcolor{Green!6}$2.3$ & \cellcolor{Green!28}$36.9$ & \cellcolor{Green!19}$21.7$ & \cellcolor{Green!65}$93.5 \pm 4.3$ & \cellcolor{Green!5}$0.1 \pm 0.1$ & \cellcolor{Green!67}$95.8$ & \cellcolor{Green!46}$63.9$ & \cellcolor{Green!21}$25.9$  \\
& D31 & \cellcolor{Green!47}$65.9$ & \cellcolor{Green!65}$\bm{93.7}$ & \cellcolor{Green!65}$\bm{93.7}$ & \cellcolor{Green!56}$79.7$ & \cellcolor{Green!32}$42.7$ & \cellcolor{Green!58}$82.9$ & \cellcolor{Green!31}$41.4 \pm 6.7$ & \cellcolor{Green!27}$33.9 \pm 11.1$ & \cellcolor{Green!34}$45.7 \pm 8.7$ & \cellcolor{Green!35}$46.5 \pm 1.8$ & \cellcolor{Green!45}$62.0 \pm 5.4$ & \cellcolor{Green!49}$67.7 \pm 3.2$ & \cellcolor{Green!30}$38.7$ & \cellcolor{Green!43}$59.2$ & \cellcolor{Green!53}$74.9$ & \cellcolor{Green!64}$\underline{92.0 \pm 2.7}$ & \cellcolor{Green!32}$41.7 \pm 5.4$ & \cellcolor{Green!64}$\underline{92.0}$ & \cellcolor{Green!61}$86.4 \pm 0.1$ & \cellcolor{Green!17}$18.5$  \\
& 3-spiral & \cellcolor{Green!40}$55.2$ & \cellcolor{Green!68}$\underline{98.1}$ & \cellcolor{Green!68}$\underline{98.1}$ & \cellcolor{Green!69}$\bm{98.6}$ & \cellcolor{Green!68}$\underline{98.1}$ & \cellcolor{Green!68}$\underline{98.1}$ & \cellcolor{Green!9}$6.6 \pm 1.3$ & \cellcolor{Green!8}$5.8 \pm 4.2$ & \cellcolor{Green!9}$6.2 \pm 3.0$ & \cellcolor{Green!13}$13.8 \pm 2.0$ & \cellcolor{Green!14}$14.7 \pm 1.8$ & \cellcolor{Green!15}$16.3 \pm 1.9$ & \cellcolor{Green!11}$9.6$ & \cellcolor{Green!8}$5.9$ & \cellcolor{Green!11}$9.6$ & \cellcolor{Red!5}$-0.6$ & \cellcolor{Red!5}$-0.3 \pm 0.1$ & \cellcolor{Green!5}$0.0 \pm 0.1$ & \cellcolor{Green!24}$30.6$ & \cellcolor{Green!68}$97.4 \pm 7.8$  \\
& aggregation & \cellcolor{Green!56}$79.2$ & \cellcolor{Green!69}$\bm{99.2}$ & \cellcolor{Green!69}$\bm{99.2}$ & \cellcolor{Green!57}$80.9$ & \cellcolor{Green!57}$80.9$ & \cellcolor{Green!57}$80.9$ & \cellcolor{Green!21}$24.8 \pm 2.7$ & \cellcolor{Green!38}$52.1 \pm 13.3$ & \cellcolor{Green!38}$51.7 \pm 12.5$ & \cellcolor{Green!17}$18.6 \pm 1.6$ & \cellcolor{Green!35}$46.4 \pm 4.3$ & \cellcolor{Green!33}$43.8 \pm 7.4$ & \cellcolor{Green!17}$18.9$ & \cellcolor{Green!33}$43.8$ & \cellcolor{Green!24}$29.8$ & \cellcolor{Green!53}$74.7 \pm 1.6$ & \cellcolor{Green!18}$20.4 \pm 6.3$ & \cellcolor{Green!57}$80.6 \pm 0.9$ & \cellcolor{Green!68}$\underline{97.8}$ & \cellcolor{Green!52}$73.4$  \\
& chainlink & \cellcolor{Green!70}$\bm{100.0}$ & \cellcolor{Green!70}$\bm{100.0}$ & \cellcolor{Green!70}$\bm{100.0}$ & \cellcolor{Green!70}$\bm{100.0}$ & \cellcolor{Green!70}$\bm{100.0}$ & \cellcolor{Green!70}$\bm{100.0}$ & \cellcolor{Green!10}$8.0 \pm 5.1$ & \cellcolor{Green!13}$12.9 \pm 5.6$ & \cellcolor{Green!12}$11.1 \pm 4.7$ & \cellcolor{Green!8}$6.1 \pm 0.4$ & \cellcolor{Green!12}$12.1 \pm 2.9$ & \cellcolor{Green!10}$8.2 \pm 0.5$ & \cellcolor{Green!8}$5.2$ & \cellcolor{Green!9}$7.2$ & \cellcolor{Green!9}$7.0$ & \cellcolor{Green!11}$9.5 \pm 0.4$ & \cellcolor{Green!6}$2.4 \pm 1.5$ & \cellcolor{Green!23}$\underline{28.0}$ & \cellcolor{Green!13}$12.4$ & \cellcolor{Green!10}$8.1$  \\
& cluto-t4-8k & \cellcolor{Green!5}$0.4$ & \cellcolor{Green!56}$79.8$ & \cellcolor{Green!61}$\underline{87.0}$ & \cellcolor{Green!61}$\bm{87.2}$ & \cellcolor{Green!54}$75.5$ & \cellcolor{Green!44}$61.5$ & \cellcolor{Green!7}$3.4 \pm 0.3$ & \cellcolor{Green!25}$32.2 \pm 6.1$ & \cellcolor{Green!22}$27.6 \pm 8.4$ & \cellcolor{Green!6}$2.8 \pm 0.1$ & \cellcolor{Green!19}$22.9 \pm 2.1$ & \cellcolor{Green!17}$19.3 \pm 3.6$ & \cellcolor{Green!7}$3.1$ & \cellcolor{Green!20}$23.7$ & \cellcolor{Green!14}$14.1$ & \cellcolor{Green!34}$45.4 \pm 2.0$ & \cellcolor{Green!5}$0.6 \pm 0.5$ & \cellcolor{Green!37}$49.8$ & \cellcolor{Green!57}$80.1$ & \cellcolor{Green!33}$43.6$  \\
& cluto-t5-8k & \cellcolor{Green!5}$0.1$ & \cellcolor{Green!53}$74.8$ & \cellcolor{Green!55}$77.9$ & \cellcolor{Green!61}$\underline{86.8}$ & \cellcolor{Green!48}$66.7$ & \cellcolor{Green!47}$65.0$ & \cellcolor{Green!7}$3.9 \pm 0.3$ & \cellcolor{Green!37}$50.3 \pm 6.7$ & \cellcolor{Green!34}$45.2 \pm 10.1$ & \cellcolor{Green!7}$4.3 \pm 0.1$ & \cellcolor{Green!32}$42.0 \pm 7.5$ & \cellcolor{Green!23}$29.0 \pm 6.6$ & \cellcolor{Green!7}$3.2$ & \cellcolor{Green!29}$37.5$ & \cellcolor{Green!19}$22.3$ & \cellcolor{Green!53}$74.2 \pm 0.3$ & \cellcolor{Green!5}$0.3 \pm 0.2$ & \cellcolor{Green!53}$73.9$ & \cellcolor{Green!61}$\bm{87.4}$ & \cellcolor{Green!32}$42.2 \pm 2.7$  \\
& cluto-t7-10k & \cellcolor{Green!5}$1.0$ & \cellcolor{Green!49}$68.8$ & \cellcolor{Green!49}$68.8$ & \cellcolor{Green!62}$\bm{88.9}$ & \cellcolor{Green!51}$71.4$ & \cellcolor{Green!28}$36.9$ & \cellcolor{Green!6}$2.8 \pm 0.1$ & \cellcolor{Green!21}$25.1 \pm 4.3$ & \cellcolor{Green!17}$19.4 \pm 4.4$ & \cellcolor{Green!6}$2.8 \pm 0.2$ & \cellcolor{Green!17}$18.9 \pm 2.4$ & \cellcolor{Green!13}$13.5 \pm 0.5$ & \cellcolor{Green!6}$2.3$ & \cellcolor{Green!16}$17.8$ & \cellcolor{Green!12}$10.9$ & \cellcolor{Green!26}$33.5 \pm 1.5$ & \cellcolor{Red!5}$-0.1 \pm 0.9$ & \cellcolor{Green!31}$41.2$ & \cellcolor{Green!62}$\underline{87.8}$ & \cellcolor{Green!25}$31.0$  \\
& cluto-t8-8k & \cellcolor{Green!46}$63.4$ & \cellcolor{Green!56}$\underline{79.9}$ & \cellcolor{Green!57}$\bm{80.1}$ & \cellcolor{Green!46}$63.4$ & \cellcolor{Green!52}$72.4$ & \cellcolor{Green!47}$65.1$ & \cellcolor{Green!7}$3.8 \pm 0.3$ & \cellcolor{Green!21}$26.0 \pm 4.5$ & \cellcolor{Green!18}$20.5 \pm 4.0$ & \cellcolor{Green!7}$3.9 \pm 0.2$ & \cellcolor{Green!18}$21.3 \pm 3.7$ & \cellcolor{Green!15}$15.5 \pm 0.8$ & \cellcolor{Green!6}$3.0$ & \cellcolor{Green!19}$23.0$ & \cellcolor{Green!13}$13.4$ & \cellcolor{Green!27}$35.3 \pm 1.7$ & \cellcolor{Red!5}$-0.1 \pm 0.2$ & \cellcolor{Green!26}$33.5$ & \cellcolor{Green!54}$76.9 \pm 0.1$ & \cellcolor{Green!24}$30.2$  \\
& complex8 & \cellcolor{Green!64}$90.9$ & \cellcolor{Green!63}$90.2$ & \cellcolor{Green!69}$\bm{99.0}$ & \cellcolor{Green!64}$\underline{91.2}$ & \cellcolor{Green!26}$33.2$ & \cellcolor{Green!23}$28.0$ & \cellcolor{Green!11}$9.6 \pm 1.4$ & \cellcolor{Green!23}$28.8 \pm 7.9$ & \cellcolor{Green!23}$27.9 \pm 5.3$ & \cellcolor{Green!11}$9.9 \pm 0.7$ & \cellcolor{Green!23}$28.7 \pm 1.7$ & \cellcolor{Green!19}$23.0 \pm 3.4$ & \cellcolor{Green!10}$8.8$ & \cellcolor{Green!22}$27.6$ & \cellcolor{Green!20}$24.3$ & \cellcolor{Green!29}$37.2 \pm 1.6$ & \cellcolor{Green!7}$4.2 \pm 1.6$ & \cellcolor{Green!28}$36.7$ & \cellcolor{Green!39}$53.3$ & \cellcolor{Green!24}$29.7$  \\
& complex9 & \cellcolor{Green!65}$\bm{93.1}$ & \cellcolor{Green!59}$\underline{83.3}$ & \cellcolor{Green!59}$\underline{83.3}$ & \cellcolor{Green!65}$\bm{93.1}$ & \cellcolor{Green!65}$\bm{93.1}$ & \cellcolor{Green!44}$61.4$ & \cellcolor{Green!10}$8.0 \pm 0.9$ & \cellcolor{Green!22}$26.7 \pm 6.8$ & \cellcolor{Green!20}$24.3 \pm 4.5$ & \cellcolor{Green!9}$7.5 \pm 0.7$ & \cellcolor{Green!21}$25.2 \pm 2.5$ & \cellcolor{Green!16}$17.8 \pm 3.4$ & \cellcolor{Green!9}$7.2$ & \cellcolor{Green!18}$20.5$ & \cellcolor{Green!15}$16.6$ & \cellcolor{Green!29}$37.6 \pm 2.2$ & \cellcolor{Red!6}$-1.6 \pm 1.9$ & \cellcolor{Green!32}$42.2$ & \cellcolor{Green!57}$81.3 \pm 0.3$ & \cellcolor{Green!18}$21.2$  \\
& compound & \cellcolor{Green!55}$78.4$ & \cellcolor{Green!47}$65.2$ & \cellcolor{Green!47}$65.2$ & \cellcolor{Green!57}$80.7$ & \cellcolor{Green!53}$74.0$ & \cellcolor{Green!60}$\underline{85.3}$ & \cellcolor{Green!25}$31.1 \pm 6.8$ & \cellcolor{Green!33}$44.0 \pm 9.4$ & \cellcolor{Green!32}$42.8 \pm 6.9$ & \cellcolor{Green!25}$32.1 \pm 5.0$ & \cellcolor{Green!35}$47.1 \pm 4.8$ & \cellcolor{Green!35}$47.3 \pm 4.8$ & \cellcolor{Green!24}$30.6$ & \cellcolor{Green!39}$52.4$ & \cellcolor{Green!30}$38.6$ & \cellcolor{Green!39}$53.4 \pm 0.9$ & \cellcolor{Green!22}$27.2 \pm 7.2$ & \cellcolor{Green!40}$55.1 \pm 0.1$ & \cellcolor{Green!64}$\bm{91.1}$ & \cellcolor{Green!43}$59.1$  \\
& dartboard1 & \cellcolor{Green!70}$\bm{100.0}$ & \cellcolor{Green!70}$\bm{100.0}$ & \cellcolor{Green!70}$\bm{100.0}$ & \cellcolor{Green!70}$\bm{100.0}$ & \cellcolor{Green!70}$\bm{100.0}$ & \cellcolor{Green!70}$\bm{100.0}$ & \cellcolor{Green!8}$5.3 \pm 5.3$ & \cellcolor{Green!9}$7.6 \pm 8.0$ & \cellcolor{Green!10}$8.7 \pm 9.1$ & \cellcolor{Green!13}$12.6 \pm 0.4$ & \cellcolor{Green!15}$16.2 \pm 1.9$ & \cellcolor{Green!16}$\underline{17.6 \pm 1.0}$ & \cellcolor{Green!10}$8.9$ & \cellcolor{Green!14}$14.3$ & \cellcolor{Green!14}$13.9$ & \cellcolor{Green!6}$2.8 \pm 9.2$ & \cellcolor{Green!14}$14.8 \pm 4.0$ & \cellcolor{Green!6}$2.8 \pm 1.5$ & \cellcolor{Green!5}$0.0$ & \cellcolor{Green!5}$0.8 \pm 0.1$  \\
& diamond9 & \cellcolor{Green!49}$68.6$ & \cellcolor{Green!69}$\bm{99.7}$ & \cellcolor{Green!69}$\bm{99.7}$ & \cellcolor{Green!59}$83.6$ & \cellcolor{Green!50}$70.4$ & \cellcolor{Green!34}$45.1$ & \cellcolor{Green!19}$22.6 \pm 8.2$ & \cellcolor{Green!32}$41.9 \pm 12.4$ & \cellcolor{Green!33}$44.4 \pm 13.1$ & \cellcolor{Green!14}$14.6 \pm 0.6$ & \cellcolor{Green!43}$58.8 \pm 7.7$ & \cellcolor{Green!34}$45.9 \pm 4.1$ & \cellcolor{Green!12}$11.0$ & \cellcolor{Green!33}$44.3$ & \cellcolor{Green!24}$30.0$ & \cellcolor{Green!68}$98.1 \pm 5.6$ & \cellcolor{Green!8}$5.2 \pm 1.4$ & \cellcolor{Green!69}$\underline{99.6}$ & \cellcolor{Green!67}$96.6$ & \cellcolor{Green!14}$15.1$  \\
& jain & \cellcolor{Green!5}$1.0$ & \cellcolor{Green!70}$\bm{100.0}$ & \cellcolor{Green!70}$\bm{100.0}$ & \cellcolor{Green!65}$\underline{92.5}$ & \cellcolor{Green!37}$50.7$ & \cellcolor{Green!35}$46.3$ & \cellcolor{Green!10}$9.0 \pm 2.5$ & \cellcolor{Green!17}$18.5 \pm 10.7$ & \cellcolor{Green!14}$14.5 \pm 4.0$ & \cellcolor{Green!9}$7.5 \pm 1.9$ & \cellcolor{Green!14}$14.7 \pm 2.1$ & \cellcolor{Green!14}$15.1 \pm 2.3$ & \cellcolor{Green!10}$7.7$ & \cellcolor{Green!16}$17.4$ & \cellcolor{Green!12}$11.7$ & \cellcolor{Green!25}$31.5 \pm 1.2$ & \cellcolor{Red!8}$-5.1 \pm 5.7$ & \cellcolor{Green!38}$51.5$ & \cellcolor{Green!59}$83.4$ & \cellcolor{Green!7}$3.2 \pm 0.1$  \\
& pathbased & \cellcolor{Green!38}$51.6$ & \cellcolor{Green!40}$54.8$ & \cellcolor{Green!40}$54.8$ & \cellcolor{Green!43}$\bm{58.7}$ & \cellcolor{Green!40}$54.8$ & \cellcolor{Green!42}$\underline{58.3}$ & \cellcolor{Green!19}$22.9 \pm 4.5$ & \cellcolor{Green!24}$29.6 \pm 7.8$ & \cellcolor{Green!27}$35.0 \pm 7.0$ & \cellcolor{Green!18}$21.1 \pm 5.9$ & \cellcolor{Green!23}$29.1 \pm 6.8$ & \cellcolor{Green!22}$26.3 \pm 10.0$ & \cellcolor{Green!16}$17.7$ & \cellcolor{Green!21}$24.9$ & \cellcolor{Green!21}$25.4$ & \cellcolor{Green!34}$46.1 \pm 0.1$ & \cellcolor{Green!22}$27.6 \pm 9.3$ & \cellcolor{Green!36}$48.3$ & \cellcolor{Green!34}$45.7 \pm 0.8$ & \cellcolor{Green!30}$38.8$  \\
& smile1 & \cellcolor{Green!70}$\bm{100.0}$ & \cellcolor{Green!70}$\bm{100.0}$ & \cellcolor{Green!70}$\bm{100.0}$ & \cellcolor{Green!70}$\bm{100.0}$ & \cellcolor{Green!70}$\bm{100.0}$ & \cellcolor{Green!70}$\bm{100.0}$ & \cellcolor{Green!13}$13.3 \pm 15.5$ & \cellcolor{Green!23}$28.9 \pm 29.0$ & \cellcolor{Green!25}$30.9 \pm 31.0$ & \cellcolor{Green!15}$15.5 \pm 1.3$ & \cellcolor{Green!48}$66.7 \pm 1.0$ & \cellcolor{Green!48}$\underline{67.6 \pm 1.2}$ & \cellcolor{Green!14}$15.3$ & \cellcolor{Green!45}$63.0$ & \cellcolor{Green!47}$66.1$ & \cellcolor{Green!40}$54.6$ & \cellcolor{Green!9}$7.3 \pm 1.9$ & \cellcolor{Green!41}$55.5 \pm 0.1$ & \cellcolor{Green!24}$29.9$ & \cellcolor{Green!26}$32.5 \pm 0.7$  \\
\midrule
\parbox[t]{2mm}{\multirow{8}{*}{\rotatebox[origin=c]{90}{Tabular Data}}}
& Synth\_low & \cellcolor{Green!66}$\underline{94.2}$ & \cellcolor{Green!59}$84.0$ & \cellcolor{Green!59}$84.0$ & \cellcolor{Green!67}$\bm{95.4}$ & \cellcolor{Green!28}$35.9$ & \cellcolor{Green!29}$37.2$ & \cellcolor{Green!27}$34.8 \pm 16.4$ & \cellcolor{Green!40}$55.1 \pm 16.6$ & \cellcolor{Green!40}$54.7 \pm 14.6$ & \cellcolor{Green!13}$13.2 \pm 2.3$ & \cellcolor{Green!40}$55.2 \pm 8.7$ & \cellcolor{Green!49}$67.9 \pm 6.8$ & \cellcolor{Green!13}$12.8$ & \cellcolor{Green!36}$49.2$ & \cellcolor{Green!57}$80.0$ & \cellcolor{Green!41}$55.9 \pm 13.2$ & \cellcolor{Green!5}$1.4 \pm 1.0$ & \cellcolor{Green!45}$62.5$ & \cellcolor{Green!26}$32.4$ & \cellcolor{Green!5}$0.0$  \\
& Synth\_high & \cellcolor{Green!66}$\underline{94.1}$ & \cellcolor{Green!59}$84.6$ & \cellcolor{Green!59}$84.6$ & \cellcolor{Green!67}$\bm{95.4}$ & \cellcolor{Green!20}$23.5$ & \cellcolor{Green!50}$70.5$ & \cellcolor{Green!22}$27.3 \pm 20.2$ & \cellcolor{Green!28}$36.4 \pm 15.0$ & \cellcolor{Green!39}$53.8 \pm 10.3$ & \cellcolor{Green!13}$13.0 \pm 0.4$ & \cellcolor{Green!21}$24.8 \pm 1.6$ & \cellcolor{Green!48}$67.6 \pm 3.4$ & \cellcolor{Green!10}$8.0$ & \cellcolor{Green!27}$35.0$ & \cellcolor{Green!35}$46.3$ & \cellcolor{Green!36}$47.9 \pm 8.2$ & \cellcolor{Green!6}$2.9 \pm 0.7$ & \cellcolor{Green!51}$71.7$ & \cellcolor{Green!5}$0.0$ & \cellcolor{Green!5}$0.0$  \\
& Mice & \cellcolor{Green!5}$1.1$ & \cellcolor{Green!22}$\bm{26.7}$ & \cellcolor{Green!18}$\underline{20.0}$ & \cellcolor{Green!5}$0.2$ & \cellcolor{Green!12}$11.6$ & \cellcolor{Green!12}$11.9$ & \cellcolor{Green!11}$10.7 \pm 1.6$ & \cellcolor{Green!11}$10.0 \pm 2.0$ & \cellcolor{Green!12}$10.8 \pm 1.4$ & \cellcolor{Green!11}$9.7 \pm 2.7$ & \cellcolor{Green!13}$13.5 \pm 1.9$ & \cellcolor{Green!12}$11.8 \pm 1.1$ & \cellcolor{Green!10}$8.6$ & \cellcolor{Green!12}$11.0$ & \cellcolor{Green!11}$9.9$ & \cellcolor{Green!14}$14.5 \pm 1.1$ & \cellcolor{Green!13}$12.7 \pm 3.1$ & \cellcolor{Green!15}$16.4$ & \cellcolor{Green!5}$0.0$ & \cellcolor{Green!6}$3.0$  \\
& airway & \cellcolor{Green!20}$23.5$ & \cellcolor{Green!42}$58.1$ & \cellcolor{Green!43}$58.9$ & \cellcolor{Green!29}$38.0$ & \cellcolor{Green!47}$\bm{65.9}$ & \cellcolor{Green!43}$58.8$ & \cellcolor{Green!5}$0.9 \pm 0.1$ & \cellcolor{Green!23}$28.8 \pm 11.9$ & \cellcolor{Green!18}$20.5 \pm 8.1$ & \cellcolor{Green!5}$0.8$ & \cellcolor{Green!16}$18.2 \pm 2.4$ & \cellcolor{Green!12}$12.0 \pm 1.4$ & \cellcolor{Green!5}$0.7$ & \cellcolor{Green!15}$16.9$ & \cellcolor{Green!10}$8.7$ & \cellcolor{Green!30}$39.9 \pm 2.0$ & \cellcolor{Red!5}$-0.9 \pm 0.5$ & \cellcolor{Green!33}$43.7$ & \cellcolor{Green!25}$31.7$ & \cellcolor{Green!47}$\underline{65.1}$  \\
& lactate & \cellcolor{Green!8}$4.9$ & \cellcolor{Green!45}$61.7$ & \cellcolor{Green!45}$62.1$ & \cellcolor{Green!31}$41.0$ & \cellcolor{Green!31}$41.0$ & \cellcolor{Green!48}$\underline{67.5}$ & \cellcolor{Green!6}$1.9 \pm 1.8$ & \cellcolor{Green!9}$6.3 \pm 1.5$ & \cellcolor{Green!7}$3.5 \pm 1.0$ & \cellcolor{Green!5}$0.1$ & \cellcolor{Green!7}$4.1 \pm 0.6$ & \cellcolor{Green!6}$1.7 \pm 0.2$ & \cellcolor{Green!5}$0.1$ & \cellcolor{Green!7}$4.4$ & \cellcolor{Green!6}$2.3$ & \cellcolor{Green!23}$28.6 \pm 1.1$ & \cellcolor{Green!5}$1.5 \pm 1.0$ & \cellcolor{Green!23}$27.7$ & \cellcolor{Green!51}$\bm{71.5}$ & \cellcolor{Green!5}$0.0$  \\
& HAR & \cellcolor{Green!5}$0.7$ & \cellcolor{Green!35}$47.3$ & \cellcolor{Green!35}$47.0$ & \cellcolor{Green!24}$30.0$ & \cellcolor{Green!35}$46.9$ & \cellcolor{Green!39}$\bm{52.8}$ & \cellcolor{Green!18}$21.5 \pm 1.9$ & \cellcolor{Green!27}$35.1 \pm 6.0$ & \cellcolor{Green!25}$31.3 \pm 8.5$ & \cellcolor{Green!14}$14.7 \pm 8.8$ & \cellcolor{Green!14}$14.2 \pm 4.7$ & \cellcolor{Green!11}$9.6 \pm 2.2$ & \cellcolor{Green!5}$1.5$ & \cellcolor{Green!6}$2.2$ & \cellcolor{Green!6}$1.6$ & \cellcolor{Green!34}$46.0 \pm 4.5$ & \cellcolor{Green!8}$5.5 \pm 3.2$ & \cellcolor{Green!36}$\underline{49.1}$ & \cellcolor{Green!5}$0.0$ & \cellcolor{Green!26}$33.2$  \\
& letterrec. & \cellcolor{Green!5}$-0.0$ & \cellcolor{Green!11}$9.6$ & \cellcolor{Green!10}$8.9$ & \cellcolor{Green!12}$12.1$ & \cellcolor{Green!15}$\underline{16.6}$ & \cellcolor{Green!16}$\bm{17.9}$ & \cellcolor{Green!8}$4.9 \pm 0.5$ & \cellcolor{Green!10}$7.9 \pm 0.9$ & \cellcolor{Green!9}$7.6 \pm 0.9$ & \cellcolor{Green!8}$5.8 \pm 0.2$ & \cellcolor{Green!9}$7.2 \pm 0.6$ & \cellcolor{Green!9}$6.2 \pm 0.3$ & \cellcolor{Green!7}$3.3$ & \cellcolor{Green!8}$5.4$ & \cellcolor{Green!7}$4.3$ & \cellcolor{Green!13}$12.9 \pm 0.6$ & \cellcolor{Green!5}$0.4 \pm 0.1$ & \cellcolor{Green!14}$14.7 \pm 0.9$ & \cellcolor{Green!10}$7.9$ & \cellcolor{Green!5}$-0.0$  \\
& PenDigits & \cellcolor{Green!5}$0.1$ & \cellcolor{Green!49}$68.3$ & \cellcolor{Green!50}$70.1$ & \cellcolor{Green!48}$66.4$ & \cellcolor{Green!52}$\underline{73.1}$ & \cellcolor{Green!54}$\bm{75.4}$ & \cellcolor{Green!10}$8.9 \pm 1.5$ & \cellcolor{Green!27}$33.9 \pm 7.6$ & \cellcolor{Green!23}$28.6 \pm 6.5$ & \cellcolor{Green!10}$8.0 \pm 0.8$ & \cellcolor{Green!12}$12.0 \pm 0.6$ & \cellcolor{Green!10}$8.9 \pm 0.5$ & \cellcolor{Green!7}$4.5$ & \cellcolor{Green!9}$6.2$ & \cellcolor{Green!7}$4.5$ & \cellcolor{Green!40}$55.3 \pm 3.2$ & \cellcolor{Green!5}$0.9 \pm 0.3$ & \cellcolor{Green!40}$55.2$ & \cellcolor{Green!41}$55.6$ & \cellcolor{Green!23}$28.8 \pm 1.1$  \\
\midrule
\parbox[t]{2mm}{\multirow{6}{*}{\rotatebox[origin=c]{90}{Image Data}}}
& COIL20 & \cellcolor{Green!42}$58.4$ & \cellcolor{Green!52}$72.8$ & \cellcolor{Green!54}$\underline{75.5}$ & \cellcolor{Green!57}$\bm{81.2}$ & \cellcolor{Green!52}$72.8$ & \cellcolor{Green!52}$72.6$ & \cellcolor{Green!30}$39.9 \pm 2.4$ & \cellcolor{Green!29}$38.4 \pm 9.0$ & \cellcolor{Green!33}$44.2 \pm 4.8$ & \cellcolor{Green!35}$46.4 \pm 4.4$ & \cellcolor{Green!35}$46.6 \pm 2.1$ & \cellcolor{Green!36}$47.7 \pm 2.0$ & \cellcolor{Green!19}$22.5$ & \cellcolor{Green!15}$16.9$ & \cellcolor{Green!17}$18.5$ & \cellcolor{Green!42}$58.2 \pm 2.8$ & \cellcolor{Green!26}$33.5 \pm 2.0$ & \cellcolor{Green!49}$68.6$ & \cellcolor{Green!30}$39.2$ & \cellcolor{Green!28}$35.9 \pm 0.1$  \\
& COIL100 & \cellcolor{Green!17}$19.6$ & \cellcolor{Green!50}$\underline{70.2}$ & \cellcolor{Green!50}$69.9$ & \cellcolor{Green!57}$\bm{80.1}$ & \cellcolor{Green!48}$66.8$ & \cellcolor{Green!50}$70.0$ & \cellcolor{Green!31}$40.6 \pm 4.7$ & \cellcolor{Green!27}$34.3 \pm 12.6$ & \cellcolor{Green!33}$43.2 \pm 6.2$ & \cellcolor{Green!33}$44.6 \pm 4.2$ & \cellcolor{Green!35}$46.6 \pm 1.5$ & \cellcolor{Green!37}$50.1 \pm 1.2$ & \cellcolor{Green!20}$23.6$ & \cellcolor{Green!19}$22.5$ & \cellcolor{Green!20}$24.4$ & \cellcolor{Green!41}$56.1 \pm 1.4$ & \cellcolor{Green!15}$16.7 \pm 0.8$ & \cellcolor{Green!44}$61.4$ & \cellcolor{Green!14}$14.2$ & \cellcolor{Green!5}$0.2$  \\
& cmu\_faces & \cellcolor{Green!20}$24.2$ & \cellcolor{Green!44}$60.2$ & \cellcolor{Green!44}$61.4$ & \cellcolor{Green!44}$60.2$ & \cellcolor{Green!41}$56.6$ & \cellcolor{Green!48}$\bm{66.5}$ & \cellcolor{Green!9}$7.1 \pm 3.2$ & \cellcolor{Green!15}$16.1 \pm 10.2$ & \cellcolor{Green!22}$26.6 \pm 20.4$ & \cellcolor{Green!10}$8.6 \pm 3.1$ & \cellcolor{Green!29}$37.1 \pm 4.1$ & \cellcolor{Green!27}$34.2 \pm 2.1$ & \cellcolor{Green!12}$11.3$ & \cellcolor{Green!12}$12.2$ & \cellcolor{Green!12}$12.2$ & \cellcolor{Green!39}$53.2 \pm 4.7$ & \cellcolor{Green!30}$38.5 \pm 2.9$ & \cellcolor{Green!45}$\underline{61.6}$ & \cellcolor{Green!5}$0.7$ & \cellcolor{Green!5}$0.6$  \\
& OptDigits & \cellcolor{Green!8}$5.0$ & \cellcolor{Green!53}$\underline{74.8}$ & \cellcolor{Green!53}$\underline{74.8}$ & \cellcolor{Green!40}$55.3$ & \cellcolor{Green!55}$\bm{77.0}$ & \cellcolor{Green!55}$\bm{77.0}$ & \cellcolor{Green!28}$36.4 \pm 4.9$ & \cellcolor{Green!27}$34.3 \pm 7.7$ & \cellcolor{Green!27}$34.1 \pm 5.4$ & \cellcolor{Green!31}$40.9 \pm 3.5$ & \cellcolor{Green!18}$20.9 \pm 2.3$ & \cellcolor{Green!16}$18.1 \pm 2.4$ & \cellcolor{Green!7}$4.1$ & \cellcolor{Green!6}$2.9$ & \cellcolor{Green!6}$2.9$ & \cellcolor{Green!44}$61.3 \pm 6.6$ & \cellcolor{Green!14}$14.4 \pm 4.1$ & \cellcolor{Green!53}$74.6 \pm 2.4$ & \cellcolor{Green!46}$63.2$ & \cellcolor{Green!5}$0.0$  \\
& USPS & \cellcolor{Green!6}$1.8$ & \cellcolor{Green!41}$\underline{55.8}$ & \cellcolor{Green!40}$54.0$ & \cellcolor{Green!26}$33.7$ & \cellcolor{Green!24}$29.3$ & \cellcolor{Green!24}$29.3$ & \cellcolor{Green!14}$15.0 \pm 1.9$ & \cellcolor{Green!22}$26.8 \pm 4.3$ & \cellcolor{Green!20}$23.9 \pm 3.5$ & \cellcolor{Green!12}$12.0 \pm 1.7$ & \cellcolor{Green!10}$8.7 \pm 1.0$ & \cellcolor{Green!12}$11.2 \pm 1.5$ & \cellcolor{Green!6}$3.0$ & \cellcolor{Green!6}$2.7$ & \cellcolor{Green!7}$3.9$ & \cellcolor{Green!38}$52.3 \pm 1.7$ & \cellcolor{Green!6}$2.9 \pm 0.9$ & \cellcolor{Green!46}$\bm{63.9}$ & \cellcolor{Green!5}$0.0$ & \cellcolor{Green!18}$21.0$  \\
& MNIST & \cellcolor{Green!5}$0.3$ & \cellcolor{Green!37}$50.1$ & \cellcolor{Green!38}$\underline{51.0}$ & \cellcolor{Green!17}$19.7$ & \cellcolor{Green!32}$41.7$ & \cellcolor{Green!34}$46.0$ & \cellcolor{Green!12}$12.3 \pm 2.8$ & \cellcolor{Green!9}$7.3 \pm 2.3$ & \cellcolor{Green!9}$7.6 \pm 2.0$ & \cellcolor{Green!12}$11.1 \pm 1.7$ & \cellcolor{Green!8}$5.4 \pm 0.6$ & \cellcolor{Green!8}$5.4 \pm 0.6$ & \cellcolor{Green!5}$1.4$ & \cellcolor{Green!5}$0.2$ & \cellcolor{Green!5}$0.3$ & \cellcolor{Green!28}$36.9 \pm 1.0$ & \cellcolor{Green!5}$1.3 \pm 0.4$ & \cellcolor{Green!39}$\bm{52.7}$ & \cellcolor{Green!5}$0.0$ & \cellcolor{Red!0}-  \\
\bottomrule

\CodeAfter
  \tikz \draw [dotted, thick] 
  (1-|9) -- (last-|9)
  (1-|12) -- (last-|12)
  (1-|15) -- (last-|15);
  \tikz \draw [dotted]
  (2-|4) -- (4-|4)
  (2-|5) -- (4-|5)
  (2-|6) -- (4-|6)
  (2-|7) -- (4-|7)
  (2-|8) -- (4-|8)
  (2-|10) -- (4-|10)
  (2-|11) -- (4-|11)
  (2-|13) -- (4-|13)
  (2-|14) -- (4-|14)
  (2-|16) -- (4-|16)
  (2-|17) -- (4-|17);
\end{NiceTabular}}

    \renewcommand{\arraystretch}{1}
\end{largetable}


\newpage
\begin{largetable}[htb]
    \centering
    \caption{
        NMI values for various combinations within our clustering framework SHIP over ten runs, compared with standard clustering methods. Noise points were handled as singleton clusters. This shows that the results yield high-quality clusterings when using the DC tree as an ultrametric.
    }
    \label{tbl:nmi_full}
    \vspace*{0.5em}
    \renewcommand{\arraystretch}{1.2}
    \resizebox{1\linewidth}{!}{
\begin{NiceTabular}{|rl|ccccccccccccccc|ccccc|}%[color-inside]
\CodeBefore
    \rowcolors{4}{gray!20}{white}[cols=2]
\Body
\toprule
& & \multicolumn{6}{c}{DC tree} & \multicolumn{3}{c}{HST-DPO} & \multicolumn{3}{c}{Cover tree} & \multicolumn{3}{c}{KD tree} & \multicolumn{5}{c}{\textbf{competitors}} \\ 
& & $k$-center & $k$-median & $k$-means & $k$-center & $k$-median & $k$-means & $k$-center & $k$-median & $k$-means & $k$-center & $k$-median & $k$-means & $k$-center & $k$-median & $k$-means & Eucl. & \multirow{2}{*}{SCAR} & \multirow{2}{*}{Ward} & AMD- & \multirow{2}{*}{DPC} \\ 
& \textbf{Dataset} & GT & GT & GT & Stability & MoE & Elbow & Stability & MoE & Elbow & Stability & MoE & Elbow & Stability & MoE & Elbow & $k$-means &  &  & DBSCAN & \\
\midrule
\parbox[t]{2mm}{\multirow{17}{*}{\rotatebox[origin=c]{90}{Density-based 2D-Data}}}
& Boxes & \cellcolor{Green!61}$86.6$ & \cellcolor{Green!69}$\bm{99.2}$ & \cellcolor{Green!69}$\bm{99.2}$ & \cellcolor{Green!66}$95.0$ & \cellcolor{Green!69}$\bm{99.2}$ & \cellcolor{Green!68}$\underline{98.3}$ & \cellcolor{Green!37}$50.6 \pm 0.3$ & \cellcolor{Green!53}$74.0 \pm 1.9$ & \cellcolor{Green!51}$71.5 \pm 2.7$ & \cellcolor{Green!38}$51.1 \pm 0.2$ & \cellcolor{Green!53}$75.0 \pm 1.5$ & \cellcolor{Green!49}$69.1 \pm 0.5$ & \cellcolor{Green!39}$52.7$ & \cellcolor{Green!52}$73.1$ & \cellcolor{Green!49}$68.0$ & \cellcolor{Green!66}$95.3 \pm 1.9$ & \cellcolor{Green!10}$8.8 \pm 0.8$ & \cellcolor{Green!68}$97.2$ & \cellcolor{Green!47}$65.6$ & \cellcolor{Green!37}$50.3$  \\
& D31 & \cellcolor{Green!59}$83.2$ & \cellcolor{Green!67}$\bm{95.8}$ & \cellcolor{Green!67}$\bm{95.8}$ & \cellcolor{Green!62}$87.7$ & \cellcolor{Green!58}$81.8$ & \cellcolor{Green!66}$94.0$ & \cellcolor{Green!54}$75.7 \pm 2.4$ & \cellcolor{Green!50}$69.7 \pm 6.4$ & \cellcolor{Green!54}$76.4 \pm 4.1$ & \cellcolor{Green!56}$78.6 \pm 0.5$ & \cellcolor{Green!59}$84.4 \pm 2.1$ & \cellcolor{Green!60}$86.1 \pm 1.1$ & \cellcolor{Green!56}$79.1$ & \cellcolor{Green!58}$82.3$ & \cellcolor{Green!61}$87.6$ & \cellcolor{Green!67}$\underline{95.7 \pm 0.9}$ & \cellcolor{Green!57}$81.5 \pm 1.8$ & \cellcolor{Green!66}$95.1$ & \cellcolor{Green!63}$90.7$ & \cellcolor{Green!44}$60.0$  \\
& 3-spiral & \cellcolor{Green!51}$71.0$ & \cellcolor{Green!67}$\underline{96.7}$ & \cellcolor{Green!67}$\underline{96.7}$ & \cellcolor{Green!68}$\bm{97.6}$ & \cellcolor{Green!67}$\underline{96.7}$ & \cellcolor{Green!67}$\underline{96.7}$ & \cellcolor{Green!20}$24.0 \pm 3.6$ & \cellcolor{Green!13}$13.2 \pm 8.6$ & \cellcolor{Green!16}$17.0 \pm 6.5$ & \cellcolor{Green!28}$35.7 \pm 4.1$ & \cellcolor{Green!30}$39.1 \pm 7.7$ & \cellcolor{Green!34}$44.8 \pm 2.9$ & \cellcolor{Green!22}$26.7$ & \cellcolor{Green!16}$17.0$ & \cellcolor{Green!22}$26.7$ & \cellcolor{Green!5}$0.1$ & \cellcolor{Green!5}$0.2 \pm 0.1$ & \cellcolor{Green!5}$0.6 \pm 0.1$ & \cellcolor{Green!34}$45.2$ & \cellcolor{Green!67}$95.9 \pm 12.4$  \\
& aggregation & \cellcolor{Green!61}$86.6$ & \cellcolor{Green!69}$\bm{98.8}$ & \cellcolor{Green!69}$\bm{98.8}$ & \cellcolor{Green!62}$88.9$ & \cellcolor{Green!62}$88.9$ & \cellcolor{Green!62}$88.9$ & \cellcolor{Green!43}$59.6 \pm 2.2$ & \cellcolor{Green!49}$68.0 \pm 7.7$ & \cellcolor{Green!51}$71.3 \pm 5.8$ & \cellcolor{Green!43}$59.2 \pm 1.3$ & \cellcolor{Green!54}$76.0 \pm 2.5$ & \cellcolor{Green!53}$74.9 \pm 3.2$ & \cellcolor{Green!45}$62.1$ & \cellcolor{Green!52}$73.8$ & \cellcolor{Green!49}$67.8$ & \cellcolor{Green!60}$85.0 \pm 1.5$ & \cellcolor{Green!39}$53.6 \pm 4.5$ & \cellcolor{Green!64}$91.9 \pm 0.2$ & \cellcolor{Green!67}$\underline{96.7}$ & \cellcolor{Green!59}$83.6$  \\
& chainlink & \cellcolor{Green!70}$\bm{100.0}$ & \cellcolor{Green!70}$\bm{100.0}$ & \cellcolor{Green!70}$\bm{100.0}$ & \cellcolor{Green!70}$\bm{100.0}$ & \cellcolor{Green!70}$\bm{100.0}$ & \cellcolor{Green!70}$\bm{100.0}$ & \cellcolor{Green!19}$22.6 \pm 9.4$ & \cellcolor{Green!20}$23.2 \pm 10.5$ & \cellcolor{Green!21}$25.2 \pm 11.8$ & \cellcolor{Green!25}$31.5 \pm 0.8$ & \cellcolor{Green!30}$\underline{38.9 \pm 2.6}$ & \cellcolor{Green!27}$35.3 \pm 0.5$ & \cellcolor{Green!25}$31.3$ & \cellcolor{Green!22}$27.5$ & \cellcolor{Green!26}$33.5$ & \cellcolor{Green!9}$7.1 \pm 0.3$ & \cellcolor{Green!12}$11.4 \pm 2.9$ & \cellcolor{Green!28}$36.6$ & \cellcolor{Green!27}$34.7$ & \cellcolor{Green!22}$26.6 \pm 0.2$  \\
& cluto-t4-8k & \cellcolor{Green!6}$2.6$ & \cellcolor{Green!59}$\underline{83.6}$ & \cellcolor{Green!61}$\bm{87.1}$ & \cellcolor{Green!57}$81.5$ & \cellcolor{Green!57}$81.0$ & \cellcolor{Green!54}$76.4$ & \cellcolor{Green!33}$44.2 \pm 0.3$ & \cellcolor{Green!38}$51.3 \pm 2.8$ & \cellcolor{Green!39}$53.5 \pm 2.5$ & \cellcolor{Green!33}$44.3 \pm 0.2$ & \cellcolor{Green!41}$56.4 \pm 1.0$ & \cellcolor{Green!41}$56.0 \pm 1.5$ & \cellcolor{Green!35}$46.4$ & \cellcolor{Green!41}$56.3$ & \cellcolor{Green!39}$53.2$ & \cellcolor{Green!43}$58.5 \pm 0.4$ & \cellcolor{Green!11}$10.0 \pm 1.2$ & \cellcolor{Green!45}$62.8$ & \cellcolor{Green!54}$76.8$ & \cellcolor{Green!36}$48.5$  \\
& cluto-t5-8k & \cellcolor{Green!8}$4.7$ & \cellcolor{Green!56}$\underline{79.9}$ & \cellcolor{Green!58}$\bm{82.1}$ & \cellcolor{Green!55}$78.3$ & \cellcolor{Green!54}$75.4$ & \cellcolor{Green!53}$74.3$ & \cellcolor{Green!34}$45.0 \pm 0.4$ & \cellcolor{Green!47}$65.2 \pm 3.2$ & \cellcolor{Green!46}$63.9 \pm 3.8$ & \cellcolor{Green!35}$46.8 \pm 0.3$ & \cellcolor{Green!46}$63.6 \pm 2.9$ & \cellcolor{Green!43}$59.1 \pm 2.4$ & \cellcolor{Green!35}$46.2$ & \cellcolor{Green!43}$59.3$ & \cellcolor{Green!41}$56.6$ & \cellcolor{Green!56}$79.6 \pm 0.2$ & \cellcolor{Green!12}$10.9 \pm 0.8$ & \cellcolor{Green!56}$79.7$ & \cellcolor{Green!55}$77.1$ & \cellcolor{Green!41}$55.7 \pm 3.4$  \\
& cluto-t7-10k & \cellcolor{Green!8}$5.0$ & \cellcolor{Green!57}$80.5$ & \cellcolor{Green!57}$80.4$ & \cellcolor{Green!59}$\bm{83.7}$ & \cellcolor{Green!58}$82.4$ & \cellcolor{Green!50}$69.7$ & \cellcolor{Green!34}$45.8 \pm 0.4$ & \cellcolor{Green!37}$49.8 \pm 3.1$ & \cellcolor{Green!38}$52.0 \pm 1.3$ & \cellcolor{Green!35}$47.0 \pm 0.4$ & \cellcolor{Green!39}$53.7 \pm 1.4$ & \cellcolor{Green!40}$55.3 \pm 0.7$ & \cellcolor{Green!35}$47.5$ & \cellcolor{Green!39}$53.7$ & \cellcolor{Green!39}$53.8$ & \cellcolor{Green!42}$57.2 \pm 0.7$ & \cellcolor{Green!12}$11.6 \pm 0.9$ & \cellcolor{Green!46}$64.1$ & \cellcolor{Green!58}$\underline{82.6}$ & \cellcolor{Green!33}$43.6$  \\
& cluto-t8-8k & \cellcolor{Green!55}$77.3$ & \cellcolor{Green!60}$\underline{85.7}$ & \cellcolor{Green!60}$85.6$ & \cellcolor{Green!56}$79.3$ & \cellcolor{Green!60}$\bm{86.0}$ & \cellcolor{Green!57}$80.2$ & \cellcolor{Green!34}$45.9 \pm 0.3$ & \cellcolor{Green!38}$50.9 \pm 4.5$ & \cellcolor{Green!38}$52.2 \pm 3.4$ & \cellcolor{Green!36}$47.9 \pm 0.5$ & \cellcolor{Green!40}$54.0 \pm 1.6$ & \cellcolor{Green!41}$55.7 \pm 0.6$ & \cellcolor{Green!35}$47.6$ & \cellcolor{Green!42}$57.7$ & \cellcolor{Green!40}$54.4$ & \cellcolor{Green!41}$56.5 \pm 1.7$ & \cellcolor{Green!13}$12.8 \pm 1.2$ & \cellcolor{Green!40}$55.3$ & \cellcolor{Green!50}$70.0 \pm 0.1$ & \cellcolor{Green!33}$43.1$  \\
& complex8 & \cellcolor{Green!65}$93.8$ & \cellcolor{Green!66}$\underline{94.4}$ & \cellcolor{Green!69}$\bm{98.8}$ & \cellcolor{Green!64}$91.3$ & \cellcolor{Green!51}$70.9$ & \cellcolor{Green!49}$68.3$ & \cellcolor{Green!38}$50.8 \pm 0.8$ & \cellcolor{Green!37}$50.7 \pm 8.3$ & \cellcolor{Green!41}$55.7 \pm 1.0$ & \cellcolor{Green!39}$53.2 \pm 1.1$ & \cellcolor{Green!42}$58.4 \pm 1.1$ & \cellcolor{Green!43}$59.4 \pm 1.5$ & \cellcolor{Green!40}$55.1$ & \cellcolor{Green!43}$59.4$ & \cellcolor{Green!45}$61.6$ & \cellcolor{Green!43}$59.4 \pm 1.1$ & \cellcolor{Green!23}$28.6 \pm 2.6$ & \cellcolor{Green!42}$58.1$ & \cellcolor{Green!47}$65.3 \pm 0.2$ & \cellcolor{Green!34}$46.0 \pm 0.1$  \\
& complex9 & \cellcolor{Green!67}$\underline{96.8}$ & \cellcolor{Green!65}$93.0$ & \cellcolor{Green!65}$93.0$ & \cellcolor{Green!67}$\bm{96.9}$ & \cellcolor{Green!67}$\bm{96.9}$ & \cellcolor{Green!60}$85.4$ & \cellcolor{Green!38}$51.6 \pm 0.9$ & \cellcolor{Green!39}$53.1 \pm 7.2$ & \cellcolor{Green!41}$56.8 \pm 3.3$ & \cellcolor{Green!39}$52.9 \pm 1.0$ & \cellcolor{Green!45}$61.9 \pm 2.7$ & \cellcolor{Green!45}$62.2 \pm 1.3$ & \cellcolor{Green!41}$55.5$ & \cellcolor{Green!40}$55.2$ & \cellcolor{Green!41}$56.5$ & \cellcolor{Green!46}$63.5 \pm 1.8$ & \cellcolor{Green!20}$23.6 \pm 2.2$ & \cellcolor{Green!50}$70.5$ & \cellcolor{Green!63}$89.5 \pm 0.3$ & \cellcolor{Green!31}$41.1$  \\
& compound & \cellcolor{Green!53}$74.5$ & \cellcolor{Green!59}$84.0$ & \cellcolor{Green!59}$84.0$ & \cellcolor{Green!61}$\underline{86.4}$ & \cellcolor{Green!57}$80.6$ & \cellcolor{Green!64}$\bm{91.2}$ & \cellcolor{Green!41}$56.4 \pm 3.0$ & \cellcolor{Green!42}$58.3 \pm 7.5$ & \cellcolor{Green!43}$59.6 \pm 5.0$ & \cellcolor{Green!45}$62.6 \pm 2.8$ & \cellcolor{Green!49}$68.5 \pm 2.1$ & \cellcolor{Green!49}$68.8 \pm 2.2$ & \cellcolor{Green!44}$61.3$ & \cellcolor{Green!50}$70.2$ & \cellcolor{Green!47}$65.2$ & \cellcolor{Green!51}$71.8 \pm 0.2$ & \cellcolor{Green!38}$51.6 \pm 6.3$ & \cellcolor{Green!52}$73.3$ & \cellcolor{Green!55}$77.0$ & \cellcolor{Green!52}$72.9$  \\
& dartboard1 & \cellcolor{Green!70}$\bm{100.0}$ & \cellcolor{Green!70}$\bm{100.0}$ & \cellcolor{Green!70}$\bm{100.0}$ & \cellcolor{Green!70}$\bm{100.0}$ & \cellcolor{Green!70}$\bm{100.0}$ & \cellcolor{Green!70}$\bm{100.0}$ & \cellcolor{Green!30}$38.5 \pm 5.1$ & \cellcolor{Green!22}$26.4 \pm 8.3$ & \cellcolor{Green!24}$30.3 \pm 10.6$ & \cellcolor{Green!37}$49.8 \pm 0.6$ & \cellcolor{Green!32}$42.5 \pm 6.4$ & \cellcolor{Green!39}$\underline{53.0 \pm 1.8}$ & \cellcolor{Green!32}$42.2$ & \cellcolor{Green!27}$35.2$ & \cellcolor{Green!29}$37.9$ & \cellcolor{Green!7}$3.5 \pm 10.6$ & \cellcolor{Green!29}$37.1 \pm 5.8$ & \cellcolor{Green!8}$5.7 \pm 1.9$ & \cellcolor{Green!5}$0.0$ & \cellcolor{Green!27}$34.0 \pm 0.1$  \\
& diamond9 & \cellcolor{Green!61}$86.8$ & \cellcolor{Green!69}$\bm{99.6}$ & \cellcolor{Green!69}$\bm{99.6}$ & \cellcolor{Green!59}$83.3$ & \cellcolor{Green!60}$85.7$ & \cellcolor{Green!54}$76.7$ & \cellcolor{Green!42}$58.3 \pm 1.1$ & \cellcolor{Green!46}$64.1 \pm 8.4$ & \cellcolor{Green!49}$67.9 \pm 8.6$ & \cellcolor{Green!45}$61.6 \pm 0.5$ & \cellcolor{Green!57}$80.1 \pm 2.6$ & \cellcolor{Green!54}$76.0 \pm 1.2$ & \cellcolor{Green!43}$59.5$ & \cellcolor{Green!50}$70.2$ & \cellcolor{Green!49}$68.4$ & \cellcolor{Green!69}$\underline{99.0 \pm 3.1}$ & \cellcolor{Green!29}$37.1 \pm 3.6$ & \cellcolor{Green!69}$\bm{99.6}$ & \cellcolor{Green!67}$96.3$ & \cellcolor{Green!32}$41.7$  \\
& jain & \cellcolor{Green!5}$1.2$ & \cellcolor{Green!70}$\bm{100.0}$ & \cellcolor{Green!70}$\bm{100.0}$ & \cellcolor{Green!56}$\underline{78.7}$ & \cellcolor{Green!50}$69.7$ & \cellcolor{Green!46}$63.9$ & \cellcolor{Green!23}$28.4 \pm 2.5$ & \cellcolor{Green!26}$32.5 \pm 6.9$ & \cellcolor{Green!26}$32.4 \pm 4.4$ & \cellcolor{Green!24}$30.2 \pm 2.8$ & \cellcolor{Green!30}$39.0 \pm 3.0$ & \cellcolor{Green!30}$39.0 \pm 2.3$ & \cellcolor{Green!25}$32.2$ & \cellcolor{Green!29}$37.8$ & \cellcolor{Green!28}$36.5$ & \cellcolor{Green!28}$36.4 \pm 0.7$ & \cellcolor{Green!9}$7.4 \pm 5.1$ & \cellcolor{Green!37}$50.5$ & \cellcolor{Green!37}$49.3$ & \cellcolor{Green!18}$20.2 \pm 0.2$  \\
& pathbased & \cellcolor{Green!45}$62.2$ & \cellcolor{Green!47}$\underline{66.0}$ & \cellcolor{Green!47}$\underline{66.0}$ & \cellcolor{Green!43}$59.1$ & \cellcolor{Green!47}$\underline{66.0}$ & \cellcolor{Green!48}$\bm{66.4}$ & \cellcolor{Green!32}$42.4 \pm 3.7$ & \cellcolor{Green!30}$39.5 \pm 7.2$ & \cellcolor{Green!35}$46.4 \pm 6.0$ & \cellcolor{Green!33}$44.2 \pm 3.6$ & \cellcolor{Green!38}$51.9 \pm 4.2$ & \cellcolor{Green!37}$50.6 \pm 4.9$ & \cellcolor{Green!31}$41.4$ & \cellcolor{Green!33}$44.0$ & \cellcolor{Green!34}$46.1$ & \cellcolor{Green!40}$54.6 \pm 0.1$ & \cellcolor{Green!29}$37.3 \pm 9.1$ & \cellcolor{Green!41}$56.6$ & \cellcolor{Green!33}$43.1 \pm 0.7$ & \cellcolor{Green!33}$43.6$  \\
& smile1 & \cellcolor{Green!70}$\bm{100.0}$ & \cellcolor{Green!70}$\bm{100.0}$ & \cellcolor{Green!70}$\bm{100.0}$ & \cellcolor{Green!70}$\bm{100.0}$ & \cellcolor{Green!70}$\bm{100.0}$ & \cellcolor{Green!70}$\bm{100.0}$ & \cellcolor{Green!32}$42.9 \pm 9.7$ & \cellcolor{Green!32}$42.5 \pm 22.5$ & \cellcolor{Green!33}$44.1 \pm 24.2$ & \cellcolor{Green!37}$50.4 \pm 1.2$ & \cellcolor{Green!52}$73.0 \pm 1.1$ & \cellcolor{Green!52}$\underline{73.7 \pm 1.0}$ & \cellcolor{Green!39}$53.0$ & \cellcolor{Green!48}$66.7$ & \cellcolor{Green!50}$69.7$ & \cellcolor{Green!44}$60.8$ & \cellcolor{Green!22}$26.5 \pm 2.8$ & \cellcolor{Green!44}$61.2 \pm 0.1$ & \cellcolor{Green!33}$43.5$ & \cellcolor{Green!37}$49.4 \pm 4.2$  \\
\midrule
\parbox[t]{2mm}{\multirow{8}{*}{\rotatebox[origin=c]{90}{Tabular Data}}}
& Synth\_low & \cellcolor{Green!61}$86.7$ & \cellcolor{Green!62}$\bm{89.2}$ & \cellcolor{Green!62}$\bm{89.2}$ & \cellcolor{Green!61}$\underline{87.5}$ & \cellcolor{Green!43}$59.4$ & \cellcolor{Green!46}$63.2$ & \cellcolor{Green!46}$63.8 \pm 8.1$ & \cellcolor{Green!52}$72.5 \pm 10.5$ & \cellcolor{Green!52}$72.7 \pm 10.0$ & \cellcolor{Green!42}$57.9 \pm 1.1$ & \cellcolor{Green!54}$76.2 \pm 3.7$ & \cellcolor{Green!56}$79.8 \pm 1.6$ & \cellcolor{Green!43}$59.8$ & \cellcolor{Green!54}$76.6$ & \cellcolor{Green!57}$80.7$ & \cellcolor{Green!55}$77.5 \pm 7.0$ & \cellcolor{Green!20}$24.0 \pm 1.3$ & \cellcolor{Green!57}$80.7$ & \cellcolor{Green!38}$51.8 \pm 0.5$ & \cellcolor{Green!5}$0.0$  \\
& Synth\_high & \cellcolor{Green!61}$86.6$ & \cellcolor{Green!61}$\underline{87.1}$ & \cellcolor{Green!61}$\underline{87.1}$ & \cellcolor{Green!61}$\bm{87.5}$ & \cellcolor{Green!33}$43.2$ & \cellcolor{Green!57}$81.1$ & \cellcolor{Green!45}$61.6 \pm 6.3$ & \cellcolor{Green!46}$64.2 \pm 14.2$ & \cellcolor{Green!50}$70.6 \pm 4.5$ & \cellcolor{Green!44}$60.9 \pm 0.2$ & \cellcolor{Green!48}$66.3 \pm 0.6$ & \cellcolor{Green!54}$76.3 \pm 0.7$ & \cellcolor{Green!43}$58.5$ & \cellcolor{Green!50}$70.0$ & \cellcolor{Green!50}$69.3$ & \cellcolor{Green!51}$71.4 \pm 5.1$ & \cellcolor{Green!18}$21.4 \pm 1.1$ & \cellcolor{Green!59}$84.1$ & \cellcolor{Green!5}$0.0$ & \cellcolor{Green!5}$0.0$  \\
& Mice & \cellcolor{Green!18}$21.2$ & \cellcolor{Green!32}$43.0$ & \cellcolor{Green!30}$39.5$ & \cellcolor{Green!10}$8.0$ & \cellcolor{Green!21}$25.6$ & \cellcolor{Green!24}$29.5$ & \cellcolor{Green!24}$29.6 \pm 2.8$ & \cellcolor{Green!17}$19.6 \pm 3.4$ & \cellcolor{Green!20}$23.4 \pm 2.2$ & \cellcolor{Green!32}$42.2 \pm 2.5$ & \cellcolor{Green!28}$36.2 \pm 3.6$ & \cellcolor{Green!34}$\bm{45.1 \pm 2.9}$ & \cellcolor{Green!28}$35.5$ & \cellcolor{Green!23}$28.3$ & \cellcolor{Green!26}$32.5$ & \cellcolor{Green!21}$25.7 \pm 1.3$ & \cellcolor{Green!23}$29.2 \pm 2.2$ & \cellcolor{Green!23}$27.9$ & \cellcolor{Green!5}$0.0$ & \cellcolor{Green!34}$\underline{44.9}$  \\
& airway & \cellcolor{Green!35}$46.9$ & \cellcolor{Green!49}$68.3$ & \cellcolor{Green!49}$\underline{68.5}$ & \cellcolor{Green!40}$54.5$ & \cellcolor{Green!50}$\bm{70.7}$ & \cellcolor{Green!49}$68.3$ & \cellcolor{Green!26}$32.6 \pm 0.2$ & \cellcolor{Green!41}$56.4 \pm 4.3$ & \cellcolor{Green!39}$53.0 \pm 3.8$ & \cellcolor{Green!26}$33.0 \pm 0.1$ & \cellcolor{Green!39}$53.6 \pm 1.2$ & \cellcolor{Green!36}$48.9 \pm 0.7$ & \cellcolor{Green!26}$33.8$ & \cellcolor{Green!38}$52.2$ & \cellcolor{Green!35}$47.3$ & \cellcolor{Green!46}$63.7 \pm 1.0$ & \cellcolor{Green!9}$6.7 \pm 1.0$ & \cellcolor{Green!46}$63.9$ & \cellcolor{Green!31}$40.9$ & \cellcolor{Green!42}$57.7$  \\
& lactate & \cellcolor{Green!12}$11.8$ & \cellcolor{Green!45}$62.3$ & \cellcolor{Green!45}$\underline{62.5}$ & \cellcolor{Green!38}$51.9$ & \cellcolor{Green!39}$53.5$ & \cellcolor{Green!46}$\bm{63.1}$ & \cellcolor{Green!23}$28.7 \pm 4.5$ & \cellcolor{Green!29}$38.2 \pm 1.5$ & \cellcolor{Green!27}$35.0 \pm 1.4$ & \cellcolor{Green!20}$24.2 \pm 0.1$ & \cellcolor{Green!29}$37.6 \pm 0.8$ & \cellcolor{Green!26}$33.4 \pm 0.6$ & \cellcolor{Green!21}$25.0$ & \cellcolor{Green!29}$38.0$ & \cellcolor{Green!27}$35.0$ & \cellcolor{Green!39}$52.9 \pm 1.6$ & \cellcolor{Green!7}$3.4 \pm 1.0$ & \cellcolor{Green!38}$52.0$ & \cellcolor{Green!29}$38.0$ & \cellcolor{Green!5}$0.0$  \\
& HAR & \cellcolor{Green!12}$11.8$ & \cellcolor{Green!47}$\underline{65.3}$ & \cellcolor{Green!46}$63.9$ & \cellcolor{Green!33}$43.7$ & \cellcolor{Green!47}$\bm{65.9}$ & \cellcolor{Green!46}$63.8$ & \cellcolor{Green!31}$40.9 \pm 1.0$ & \cellcolor{Green!38}$51.9 \pm 5.3$ & \cellcolor{Green!38}$52.1 \pm 4.2$ & \cellcolor{Green!30}$39.6 \pm 1.9$ & \cellcolor{Green!34}$45.8 \pm 1.4$ & \cellcolor{Green!33}$44.3 \pm 0.7$ & \cellcolor{Green!24}$29.3$ & \cellcolor{Green!24}$29.6$ & \cellcolor{Green!24}$29.3$ & \cellcolor{Green!44}$60.2 \pm 2.2$ & \cellcolor{Green!20}$24.6 \pm 6.3$ & \cellcolor{Green!45}$62.9$ & \cellcolor{Green!5}$0.0$ & \cellcolor{Green!41}$55.5$  \\
& letterrec. & \cellcolor{Green!5}$1.4$ & \cellcolor{Green!32}$42.8$ & \cellcolor{Green!33}$44.0$ & \cellcolor{Green!41}$\bm{56.5}$ & \cellcolor{Green!39}$53.6$ & \cellcolor{Green!40}$\underline{54.5}$ & \cellcolor{Green!32}$42.2 \pm 2.2$ & \cellcolor{Green!26}$33.0 \pm 3.9$ & \cellcolor{Green!28}$36.3 \pm 4.1$ & \cellcolor{Green!38}$51.9 \pm 0.9$ & \cellcolor{Green!38}$51.1 \pm 1.3$ & \cellcolor{Green!39}$53.2 \pm 0.5$ & \cellcolor{Green!34}$45.2$ & \cellcolor{Green!31}$40.5$ & \cellcolor{Green!33}$43.6$ & \cellcolor{Green!27}$35.3 \pm 0.5$ & \cellcolor{Green!17}$19.4 \pm 1.6$ & \cellcolor{Green!31}$40.3 \pm 0.9$ & \cellcolor{Green!37}$49.3$ & \cellcolor{Green!37}$49.7$  \\
& PenDigits & \cellcolor{Green!9}$7.5$ & \cellcolor{Green!55}$77.4$ & \cellcolor{Green!56}$78.6$ & \cellcolor{Green!50}$70.3$ & \cellcolor{Green!58}$\underline{81.7}$ & \cellcolor{Green!58}$\bm{82.8}$ & \cellcolor{Green!37}$50.6 \pm 0.6$ & \cellcolor{Green!42}$57.6 \pm 2.7$ & \cellcolor{Green!43}$59.3 \pm 1.6$ & \cellcolor{Green!38}$51.5 \pm 0.5$ & \cellcolor{Green!42}$57.8 \pm 0.4$ & \cellcolor{Green!41}$56.2 \pm 0.3$ & \cellcolor{Green!36}$49.1$ & \cellcolor{Green!37}$50.2$ & \cellcolor{Green!37}$49.6$ & \cellcolor{Green!49}$68.0 \pm 1.0$ & \cellcolor{Green!16}$18.4 \pm 1.7$ & \cellcolor{Green!52}$72.7$ & \cellcolor{Green!43}$59.9$ & \cellcolor{Green!34}$46.0 \pm 0.6$  \\
\midrule
\parbox[t]{2mm}{\multirow{6}{*}{\rotatebox[origin=c]{90}{Image Data}}}
& COIL20 & \cellcolor{Green!57}$80.7$ & \cellcolor{Green!61}$86.9$ & \cellcolor{Green!62}$\underline{88.0}$ & \cellcolor{Green!62}$\bm{89.1}$ & \cellcolor{Green!61}$86.9$ & \cellcolor{Green!61}$86.8$ & \cellcolor{Green!49}$68.8 \pm 1.1$ & \cellcolor{Green!48}$67.4 \pm 8.4$ & \cellcolor{Green!51}$70.9 \pm 3.4$ & \cellcolor{Green!51}$71.6 \pm 1.5$ & \cellcolor{Green!54}$75.9 \pm 0.8$ & \cellcolor{Green!54}$76.1 \pm 0.8$ & \cellcolor{Green!37}$50.0$ & \cellcolor{Green!38}$51.7$ & \cellcolor{Green!39}$53.1$ & \cellcolor{Green!55}$77.8 \pm 1.9$ & \cellcolor{Green!46}$64.6 \pm 0.7$ & \cellcolor{Green!58}$82.6$ & \cellcolor{Green!47}$65.9$ & \cellcolor{Green!46}$64.5 \pm 0.1$  \\
& COIL100 & \cellcolor{Green!56}$79.7$ & \cellcolor{Green!63}$89.5$ & \cellcolor{Green!63}$\underline{89.6}$ & \cellcolor{Green!63}$\bm{90.7}$ & \cellcolor{Green!62}$88.9$ & \cellcolor{Green!63}$\underline{89.6}$ & \cellcolor{Green!54}$76.7 \pm 0.9$ & \cellcolor{Green!52}$73.5 \pm 7.4$ & \cellcolor{Green!55}$78.1 \pm 3.3$ & \cellcolor{Green!55}$78.3 \pm 0.5$ & \cellcolor{Green!58}$82.2 \pm 0.3$ & \cellcolor{Green!58}$82.9 \pm 0.3$ & \cellcolor{Green!44}$61.3$ & \cellcolor{Green!45}$62.9$ & \cellcolor{Green!46}$64.1$ & \cellcolor{Green!59}$83.2 \pm 0.6$ & \cellcolor{Green!52}$73.1 \pm 0.4$ & \cellcolor{Green!61}$86.2$ & \cellcolor{Green!51}$72.1$ & \cellcolor{Green!49}$68.3$  \\
& cmu\_faces & \cellcolor{Green!51}$71.2$ & \cellcolor{Green!59}$84.0$ & \cellcolor{Green!60}$\underline{85.4}$ & \cellcolor{Green!56}$78.9$ & \cellcolor{Green!57}$81.5$ & \cellcolor{Green!61}$\bm{87.3}$ & \cellcolor{Green!25}$31.5 \pm 6.6$ & \cellcolor{Green!33}$43.3 \pm 15.2$ & \cellcolor{Green!38}$52.2 \pm 23.2$ & \cellcolor{Green!27}$34.2 \pm 5.1$ & \cellcolor{Green!56}$78.6 \pm 0.8$ & \cellcolor{Green!55}$78.0 \pm 0.5$ & \cellcolor{Green!33}$43.4$ & \cellcolor{Green!33}$44.2$ & \cellcolor{Green!34}$45.0$ & \cellcolor{Green!54}$75.7 \pm 2.5$ & \cellcolor{Green!47}$64.9 \pm 1.6$ & \cellcolor{Green!57}$81.5$ & \cellcolor{Green!19}$22.6$ & \cellcolor{Green!46}$63.5$  \\
& OptDigits & \cellcolor{Green!24}$29.5$ & \cellcolor{Green!58}$83.0$ & \cellcolor{Green!58}$83.0$ & \cellcolor{Green!45}$62.3$ & \cellcolor{Green!59}$\bm{84.5}$ & \cellcolor{Green!59}$\bm{84.5}$ & \cellcolor{Green!44}$61.2 \pm 1.3$ & \cellcolor{Green!45}$62.2 \pm 2.6$ & \cellcolor{Green!45}$61.9 \pm 1.6$ & \cellcolor{Green!45}$62.7 \pm 1.5$ & \cellcolor{Green!43}$58.8 \pm 0.9$ & \cellcolor{Green!42}$57.7 \pm 0.9$ & \cellcolor{Green!29}$37.6$ & \cellcolor{Green!30}$38.6$ & \cellcolor{Green!30}$39.3$ & \cellcolor{Green!51}$72.2 \pm 3.5$ & \cellcolor{Green!37}$50.1 \pm 4.3$ & \cellcolor{Green!59}$\underline{83.4 \pm 1.2}$ & \cellcolor{Green!48}$66.8$ & \cellcolor{Green!32}$42.1$  \\
& USPS & \cellcolor{Green!18}$21.0$ & \cellcolor{Green!48}$\underline{67.6}$ & \cellcolor{Green!48}$67.3$ & \cellcolor{Green!35}$46.9$ & \cellcolor{Green!39}$52.6$ & \cellcolor{Green!39}$52.9$ & \cellcolor{Green!35}$47.0 \pm 0.8$ & \cellcolor{Green!35}$47.2 \pm 1.9$ & \cellcolor{Green!35}$47.6 \pm 1.9$ & \cellcolor{Green!35}$47.3 \pm 0.9$ & \cellcolor{Green!37}$50.1 \pm 0.5$ & \cellcolor{Green!37}$50.7 \pm 0.5$ & \cellcolor{Green!28}$36.0$ & \cellcolor{Green!28}$36.5$ & \cellcolor{Green!29}$37.5$ & \cellcolor{Green!44}$60.8 \pm 1.0$ & \cellcolor{Green!25}$31.2 \pm 1.9$ & \cellcolor{Green!54}$\bm{76.1}$ & \cellcolor{Green!5}$0.0$ & \cellcolor{Green!32}$42.0$  \\
& MNIST & \cellcolor{Green!14}$14.8$ & \cellcolor{Green!45}$61.6$ & \cellcolor{Green!45}$\underline{62.9}$ & \cellcolor{Green!29}$38.2$ & \cellcolor{Green!42}$58.3$ & \cellcolor{Green!45}$62.6$ & \cellcolor{Green!30}$39.1 \pm 1.6$ & \cellcolor{Green!30}$39.7 \pm 0.5$ & \cellcolor{Green!31}$40.0 \pm 0.6$ & \cellcolor{Green!31}$40.9 \pm 0.8$ & \cellcolor{Green!32}$41.9 \pm 0.4$ & \cellcolor{Green!32}$42.2 \pm 0.4$ & \cellcolor{Green!21}$25.6$ & \cellcolor{Green!21}$25.0$ & \cellcolor{Green!21}$25.0$ & \cellcolor{Green!36}$49.1 \pm 0.7$ & \cellcolor{Green!17}$19.2 \pm 2.7$ & \cellcolor{Green!49}$\bm{68.2}$ & \cellcolor{Green!5}$0.0$ & \cellcolor{Red!0}-  \\
\bottomrule


\CodeAfter
  \tikz \draw [dotted, thick] 
  (1-|9) -- (last-|9)
  (1-|12) -- (last-|12)
  (1-|15) -- (last-|15);
  \tikz \draw [dotted]
  (2-|4) -- (4-|4)
  (2-|5) -- (4-|5)
  (2-|6) -- (4-|6)
  (2-|7) -- (4-|7)
  (2-|8) -- (4-|8)
  (2-|10) -- (4-|10)
  (2-|11) -- (4-|11)
  (2-|13) -- (4-|13)
  (2-|14) -- (4-|14)
  (2-|16) -- (4-|16)
  (2-|17) -- (4-|17);
\end{NiceTabular}}

    \renewcommand{\arraystretch}{1}
\end{largetable}
