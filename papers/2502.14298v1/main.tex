%\documentclass{uai2025} % for initial submission
\documentclass[accepted]{uai2025} % after acceptance, for a revised version; 
% also before submission to see how the non-anonymous paper would look like 
                        
%% There is a class option to choose the math font
% \documentclass[mathfont=ptmx]{uai2025} % ptmx math instead of Computer
                                         % Modern (has noticeable issues)
% \documentclass[mathfont=newtx]{uai2025} % newtx fonts (improves upon
                                          % ptmx; less tested, no support)
% NOTE: Only keep *one* line above as appropriate, as it will be replaced
%       automatically for papers to be published. Do not make any other
%       change above this note for an accepted version.

% FROM UAI TEM}PLATE
%% Choose your variant of English; be consistent
\usepackage[american]{babel}
% \usepackage[british]{babel}
%% Some suggested packages, as needed:
\usepackage{natbib} % has a nice set of citation styles and commands
    \bibliographystyle{plainnat}
    \renewcommand{\bibsection}{\subsubsection*{References}}
\usepackage{mathtools} % amsmath with fixes and additions
% \usepackage{siunitx} % for proper typesetting of numbers and units
\usepackage{booktabs} % commands to create good-looking tables
\usepackage{tikz} % nice language for creating drawings and diagrams

%% Provided macros
% \smaller: Because the class footnote size is essentially LaTeX's \small,
%           redefining \footnotesize, we provide the original \footnotesize
%           using this macro.
%           (Use only sparingly, e.g., in drawings, as it is quite small.)

%% Self-defined macros
\newcommand{\swap}[3][-]{#3#1#2} % just an example


% FROM ICML TEMPLATE
\usepackage{microtype}
\usepackage{graphicx}
\usepackage{subfigure}
\usepackage{booktabs} % for professional tables
\usepackage{multirow}
% If your build breaks (sometimes temporarily if a hyperlink spans a page)
% please comment out the following usepackage line and replace
% \usepackage{icml2025} with \usepackage[nohyperref]{icml2025} above.
\usepackage{hyperref}
\usepackage{amsmath}
\usepackage{amssymb}
\usepackage{mathtools}
\usepackage{amsthm}
\usepackage[capitalize,noabbrev]{cleveref}
% Todonotes is useful during development; simply uncomment the next line
%    and comment out the line below the next line to turn off comments
%\usepackage[disable,textsize=tiny]{todonotes}
\usepackage[textsize=tiny]{todonotes}

% FROM Neurips TEMPLATE
\usepackage[utf8]{inputenc} % allow utf-8 input
\usepackage[T1]{fontenc}    % use 8-bit T1 fonts
\usepackage{url}            % simple URL typesetting
% \usepackage{booktabs}       % professional-quality tables
\usepackage{amsfonts}       % blackboard math symbols
\usepackage{nicefrac}       % compact symbols for 1/2, etc.
% \usepackage{microtype}      % microtypography
%\usepackage{xcolor}         % colors
\usepackage{natbib}

% CUSTOM (UNCOMMENT WHEN IT IS NEEDED)
% \usepackage{algorithm}
% \usepackage{algpseudocode}
% \usepackage{hyperref}       % hyperlinks
% \usepackage{multirow}
% \usepackage[table,svgnames, dvipsnames]{xcolor} % \rowcolor
% \usepackage{amsmath} 
% \usepackage{amsthm}%foor proof env
% \usepackage{pifont} %for \xmark
% \usepackage{cleveref} % for \Cref
\usepackage{comment}
% \usepackage{multirow,rotating}
\usepackage{bbm}
% \usepackage{wrapfig}
% \usepackage[%
    %font={small,sf},
    %labelfont=bf,
    %format=hang,    
    %format=plain,
    %margin=0pt,
    %width=0.8\textwidth,
% ]{caption}
% \usepackage[list=true]{subcaption}
% \usepackage[noframe]{showframe}%use \AddToShipoutPicture*{\ShowFramePicture} to show margins on page add [noframe] to disable
\usepackage{placeins} % for floatbarrier
\usepackage{calligra} % for calligraphic lower case letters
\DeclareMathAlphabet{\mathcalligra}{T1}{calligra}{m}{n}

\usepackage{cancel} %strikethrough in math

\def\russ#1{{\color{blue}{\bf [russ:} {\it{#1}}{\bf ]}}}
%%%%% NEW MATH DEFINITIONS %%%%%

\usepackage{amsmath, amsfonts, bm}

% Mark sections of captions for referring to divisions of figures
\newcommand{\figleft}{{\em (Left)}}
\newcommand{\figcenter}{{\em (Center)}}
\newcommand{\figright}{{\em (Right)}}
\newcommand{\figtop}{{\em (Top)}}
\newcommand{\figbottom}{{\em (Bottom)}}
\newcommand{\captiona}{{\em (a)}}
\newcommand{\captionb}{{\em (b)}}
\newcommand{\captionc}{{\em (c)}}
\newcommand{\captiond}{{\em (d)}}

% Highlight a newly defined term
\newcommand{\newterm}[1]{{\bf #1}}


% Figure reference, lower-case.
\def\figref#1{figure~\ref{#1}}
% Figure reference, capital. For start of sentence
\def\Figref#1{Figure~\ref{#1}}
\def\twofigref#1#2{figures \ref{#1} and \ref{#2}}
\def\quadfigref#1#2#3#4{figures \ref{#1}, \ref{#2}, \ref{#3} and \ref{#4}}
% Section reference, lower-case.
\def\secref#1{section~\ref{#1}}
% Section reference, capital.
\def\Secref#1{Section~\ref{#1}}
% Reference to two sections.
\def\twosecrefs#1#2{sections \ref{#1} and \ref{#2}}
% Reference to three sections.
\def\secrefs#1#2#3{sections \ref{#1}, \ref{#2} and \ref{#3}}
% Reference to an equation, lower-case.
\def\eqref#1{equation~\ref{#1}}
% Reference to an equation, upper case
\def\Eqref#1{Equation~\ref{#1}}
% A raw reference to an equation---avoid using if possible
\def\plaineqref#1{\ref{#1}}
% Reference to a chapter, lower-case.
\def\chapref#1{chapter~\ref{#1}}
% Reference to an equation, upper case.
\def\Chapref#1{Chapter~\ref{#1}}
% Reference to a range of chapters
\def\rangechapref#1#2{chapters\ref{#1}--\ref{#2}}
% Reference to an algorithm, lower-case.
\def\algref#1{algorithm~\ref{#1}}
% Reference to an algorithm, upper case.
\def\Algref#1{Algorithm~\ref{#1}}
\def\twoalgref#1#2{algorithms \ref{#1} and \ref{#2}}
\def\Twoalgref#1#2{Algorithms \ref{#1} and \ref{#2}}
% Reference to a part, lower case
\def\partref#1{part~\ref{#1}}
% Reference to a part, upper case
\def\Partref#1{Part~\ref{#1}}
\def\twopartref#1#2{parts \ref{#1} and \ref{#2}}

\def\ceil#1{\lceil #1 \rceil}
\def\floor#1{\lfloor #1 \rfloor}
\def\1{\bm{1}}
\newcommand{\train}{\mathcal{D}}
\newcommand{\valid}{\mathcal{D_{\mathrm{valid}}}}
\newcommand{\test}{\mathcal{D_{\mathrm{test}}}}

\def\eps{{\epsilon}}


% Random variables
\def\reta{{\textnormal{$\eta$}}}
\def\ra{{\textnormal{a}}}
\def\rb{{\textnormal{b}}}
\def\rc{{\textnormal{c}}}
\def\rd{{\textnormal{d}}}
\def\re{{\textnormal{e}}}
\def\rf{{\textnormal{f}}}
\def\rg{{\textnormal{g}}}
\def\rh{{\textnormal{h}}}
\def\ri{{\textnormal{i}}}
\def\rj{{\textnormal{j}}}
\def\rk{{\textnormal{k}}}
\def\rl{{\textnormal{l}}}
% rm is already a command, just don't name any random variables m
\def\rn{{\textnormal{n}}}
\def\ro{{\textnormal{o}}}
\def\rp{{\textnormal{p}}}
\def\rq{{\textnormal{q}}}
\def\rr{{\textnormal{r}}}
\def\rs{{\textnormal{s}}}
\def\rt{{\textnormal{t}}}
\def\ru{{\textnormal{u}}}
\def\rv{{\textnormal{v}}}
\def\rw{{\textnormal{w}}}
\def\rx{{\textnormal{x}}}
\def\ry{{\textnormal{y}}}
\def\rz{{\textnormal{z}}}

% Random vectors
\def\rvepsilon{{\mathbf{\epsilon}}}
\def\rvtheta{{\mathbf{\theta}}}
\def\rva{{\mathbf{a}}}
\def\rvb{{\mathbf{b}}}
\def\rvc{{\mathbf{c}}}
\def\rvd{{\mathbf{d}}}
\def\rve{{\mathbf{e}}}
\def\rvf{{\mathbf{f}}}
\def\rvg{{\mathbf{g}}}
\def\rvh{{\mathbf{h}}}
\def\rvu{{\mathbf{i}}}
\def\rvj{{\mathbf{j}}}
\def\rvk{{\mathbf{k}}}
\def\rvl{{\mathbf{l}}}
\def\rvm{{\mathbf{m}}}
\def\rvn{{\mathbf{n}}}
\def\rvo{{\mathbf{o}}}
\def\rvp{{\mathbf{p}}}
\def\rvq{{\mathbf{q}}}
\def\rvr{{\mathbf{r}}}
\def\rvs{{\mathbf{s}}}
\def\rvt{{\mathbf{t}}}
\def\rvu{{\mathbf{u}}}
\def\rvv{{\mathbf{v}}}
\def\rvw{{\mathbf{w}}}
\def\rvx{{\mathbf{x}}}
\def\rvy{{\mathbf{y}}}
\def\rvz{{\mathbf{z}}}

% Elements of random vectors
\def\erva{{\textnormal{a}}}
\def\ervb{{\textnormal{b}}}
\def\ervc{{\textnormal{c}}}
\def\ervd{{\textnormal{d}}}
\def\erve{{\textnormal{e}}}
\def\ervf{{\textnormal{f}}}
\def\ervg{{\textnormal{g}}}
\def\ervh{{\textnormal{h}}}
\def\ervi{{\textnormal{i}}}
\def\ervj{{\textnormal{j}}}
\def\ervk{{\textnormal{k}}}
\def\ervl{{\textnormal{l}}}
\def\ervm{{\textnormal{m}}}
\def\ervn{{\textnormal{n}}}
\def\ervo{{\textnormal{o}}}
\def\ervp{{\textnormal{p}}}
\def\ervq{{\textnormal{q}}}
\def\ervr{{\textnormal{r}}}
\def\ervs{{\textnormal{s}}}
\def\ervt{{\textnormal{t}}}
\def\ervu{{\textnormal{u}}}
\def\ervv{{\textnormal{v}}}
\def\ervw{{\textnormal{w}}}
\def\ervx{{\textnormal{x}}}
\def\ervy{{\textnormal{y}}}
\def\ervz{{\textnormal{z}}}

% Random matrices
\def\rmA{{\mathbf{A}}}
\def\rmB{{\mathbf{B}}}
\def\rmC{{\mathbf{C}}}
\def\rmD{{\mathbf{D}}}
\def\rmE{{\mathbf{E}}}
\def\rmF{{\mathbf{F}}}
\def\rmG{{\mathbf{G}}}
\def\rmH{{\mathbf{H}}}
\def\rmI{{\mathbf{I}}}
\def\rmJ{{\mathbf{J}}}
\def\rmK{{\mathbf{K}}}
\def\rmL{{\mathbf{L}}}
\def\rmM{{\mathbf{M}}}
\def\rmN{{\mathbf{N}}}
\def\rmO{{\mathbf{O}}}
\def\rmP{{\mathbf{P}}}
\def\rmQ{{\mathbf{Q}}}
\def\rmR{{\mathbf{R}}}
\def\rmS{{\mathbf{S}}}
\def\rmT{{\mathbf{T}}}
\def\rmU{{\mathbf{U}}}
\def\rmV{{\mathbf{V}}}
\def\rmW{{\mathbf{W}}}
\def\rmX{{\mathbf{X}}}
\def\rmY{{\mathbf{Y}}}
\def\rmZ{{\mathbf{Z}}}

% Elements of random matrices
\def\ermA{{\textnormal{A}}}
\def\ermB{{\textnormal{B}}}
\def\ermC{{\textnormal{C}}}
\def\ermD{{\textnormal{D}}}
\def\ermE{{\textnormal{E}}}
\def\ermF{{\textnormal{F}}}
\def\ermG{{\textnormal{G}}}
\def\ermH{{\textnormal{H}}}
\def\ermI{{\textnormal{I}}}
\def\ermJ{{\textnormal{J}}}
\def\ermK{{\textnormal{K}}}
\def\ermL{{\textnormal{L}}}
\def\ermM{{\textnormal{M}}}
\def\ermN{{\textnormal{N}}}
\def\ermO{{\textnormal{O}}}
\def\ermP{{\textnormal{P}}}
\def\ermQ{{\textnormal{Q}}}
\def\ermR{{\textnormal{R}}}
\def\ermS{{\textnormal{S}}}
\def\ermT{{\textnormal{T}}}
\def\ermU{{\textnormal{U}}}
\def\ermV{{\textnormal{V}}}
\def\ermW{{\textnormal{W}}}
\def\ermX{{\textnormal{X}}}
\def\ermY{{\textnormal{Y}}}
\def\ermZ{{\textnormal{Z}}}

% Vectors
\def\vzero{{\bm{0}}}
\def\vone{{\bm{1}}}
\def\vmu{{\bm{\mu}}}
\def\vtheta{{\bm{\theta}}}
\def\va{{\bm{a}}}
\def\vb{{\bm{b}}}
\def\vc{{\bm{c}}}
\def\vd{{\bm{d}}}
\def\ve{{\bm{e}}}
\def\vf{{\bm{f}}}
\def\vg{{\bm{g}}}
\def\vh{{\bm{h}}}
\def\vi{{\bm{i}}}
\def\vj{{\bm{j}}}
\def\vk{{\bm{k}}}
\def\vl{{\bm{l}}}
\def\vm{{\bm{m}}}
\def\vn{{\bm{n}}}
\def\vo{{\bm{o}}}
\def\vp{{\bm{p}}}
\def\vq{{\bm{q}}}
\def\vr{{\bm{r}}}
\def\vs{{\bm{s}}}
\def\vt{{\bm{t}}}
\def\vu{{\bm{u}}}
\def\vv{{\bm{v}}}
\def\vw{{\bm{w}}}
\def\vx{{\bm{x}}}
\def\vy{{\bm{y}}}
\def\vz{{\bm{z}}}

% Elements of vectors
\def\evalpha{{\alpha}}
\def\evbeta{{\beta}}
\def\evepsilon{{\epsilon}}
\def\evlambda{{\lambda}}
\def\evomega{{\omega}}
\def\evmu{{\mu}}
\def\evpsi{{\psi}}
\def\evsigma{{\sigma}}
\def\evtheta{{\theta}}
\def\eva{{a}}
\def\evb{{b}}
\def\evc{{c}}
\def\evd{{d}}
\def\eve{{e}}
\def\evf{{f}}
\def\evg{{g}}
\def\evh{{h}}
\def\evi{{i}}
\def\evj{{j}}
\def\evk{{k}}
\def\evl{{l}}
\def\evm{{m}}
\def\evn{{n}}
\def\evo{{o}}
\def\evp{{p}}
\def\evq{{q}}
\def\evr{{r}}
\def\evs{{s}}
\def\evt{{t}}
\def\evu{{u}}
\def\evv{{v}}
\def\evw{{w}}
\def\evx{{x}}
\def\evy{{y}}
\def\evz{{z}}

% Matrix
\def\mA{{\bm{A}}}
\def\mB{{\bm{B}}}
\def\mC{{\bm{C}}}
\def\mD{{\bm{D}}}
\def\mE{{\bm{E}}}
\def\mF{{\bm{F}}}
\def\mG{{\bm{G}}}
\def\mH{{\bm{H}}}
\def\mI{{\bm{I}}}
\def\mJ{{\bm{J}}}
\def\mK{{\bm{K}}}
\def\mL{{\bm{L}}}
\def\mM{{\bm{M}}}
\def\mN{{\bm{N}}}
\def\mO{{\bm{O}}}
\def\mP{{\bm{P}}}
\def\mQ{{\bm{Q}}}
\def\mR{{\bm{R}}}
\def\mS{{\bm{S}}}
\def\mT{{\bm{T}}}
\def\mU{{\bm{U}}}
\def\mV{{\bm{V}}}
\def\mW{{\bm{W}}}
\def\mX{{\bm{X}}}
\def\mY{{\bm{Y}}}
\def\mZ{{\bm{Z}}}
\def\mBeta{{\bm{\beta}}}
\def\mPhi{{\bm{\Phi}}}
\def\mLambda{{\bm{\Lambda}}}
\def\mSigma{{\bm{\Sigma}}}

% Tensor
\DeclareMathAlphabet{\mathsfit}{\encodingdefault}{\sfdefault}{m}{sl}
\SetMathAlphabet{\mathsfit}{bold}{\encodingdefault}{\sfdefault}{bx}{n}
\newcommand{\tens}[1]{\bm{\mathsfit{#1}}}
\def\tA{{\tens{A}}}
\def\tB{{\tens{B}}}
\def\tC{{\tens{C}}}
\def\tD{{\tens{D}}}
\def\tE{{\tens{E}}}
\def\tF{{\tens{F}}}
\def\tG{{\tens{G}}}
\def\tH{{\tens{H}}}
\def\tI{{\tens{I}}}
\def\tJ{{\tens{J}}}
\def\tK{{\tens{K}}}
\def\tL{{\tens{L}}}
\def\tM{{\tens{M}}}
\def\tN{{\tens{N}}}
\def\tO{{\tens{O}}}
\def\tP{{\tens{P}}}
\def\tQ{{\tens{Q}}}
\def\tR{{\tens{R}}}
\def\tS{{\tens{S}}}
\def\tT{{\tens{T}}}
\def\tU{{\tens{U}}}
\def\tV{{\tens{V}}}
\def\tW{{\tens{W}}}
\def\tX{{\tens{X}}}
\def\tY{{\tens{Y}}}
\def\tZ{{\tens{Z}}}


% Graph
\def\gA{{\mathcal{A}}}
\def\gB{{\mathcal{B}}}
\def\gC{{\mathcal{C}}}
\def\gD{{\mathcal{D}}}
\def\gE{{\mathcal{E}}}
\def\gF{{\mathcal{F}}}
\def\gG{{\mathcal{G}}}
\def\gH{{\mathcal{H}}}
\def\gI{{\mathcal{I}}}
\def\gJ{{\mathcal{J}}}
\def\gK{{\mathcal{K}}}
\def\gL{{\mathcal{L}}}
\def\gM{{\mathcal{M}}}
\def\gN{{\mathcal{N}}}
\def\gO{{\mathcal{O}}}
\def\gP{{\mathcal{P}}}
\def\gQ{{\mathcal{Q}}}
\def\gR{{\mathcal{R}}}
\def\gS{{\mathcal{S}}}
\def\gT{{\mathcal{T}}}
\def\gU{{\mathcal{U}}}
\def\gV{{\mathcal{V}}}
\def\gW{{\mathcal{W}}}
\def\gX{{\mathcal{X}}}
\def\gY{{\mathcal{Y}}}
\def\gZ{{\mathcal{Z}}}

% Sets
\def\sA{{\mathbb{A}}}
\def\sB{{\mathbb{B}}}
\def\sC{{\mathbb{C}}}
\def\sD{{\mathbb{D}}}
% Don't use a set called E, because this would be the same as our symbol
% for expectation.
\def\sF{{\mathbb{F}}}
\def\sG{{\mathbb{G}}}
\def\sH{{\mathbb{H}}}
\def\sI{{\mathbb{I}}}
\def\sJ{{\mathbb{J}}}
\def\sK{{\mathbb{K}}}
\def\sL{{\mathbb{L}}}
\def\sM{{\mathbb{M}}}
\def\sN{{\mathbb{N}}}
\def\sO{{\mathbb{O}}}
\def\sP{{\mathbb{P}}}
\def\sQ{{\mathbb{Q}}}
\def\sR{{\mathbb{R}}}
\def\sS{{\mathbb{S}}}
\def\sT{{\mathbb{T}}}
\def\sU{{\mathbb{U}}}
\def\sV{{\mathbb{V}}}
\def\sW{{\mathbb{W}}}
\def\sX{{\mathbb{X}}}
\def\sY{{\mathbb{Y}}}
\def\sZ{{\mathbb{Z}}}

% Entries of a matrix
\def\emLambda{{\Lambda}}
\def\emA{{A}}
\def\emB{{B}}
\def\emC{{C}}
\def\emD{{D}}
\def\emE{{E}}
\def\emF{{F}}
\def\emG{{G}}
\def\emH{{H}}
\def\emI{{I}}
\def\emJ{{J}}
\def\emK{{K}}
\def\emL{{L}}
\def\emM{{M}}
\def\emN{{N}}
\def\emO{{O}}
\def\emP{{P}}
\def\emQ{{Q}}
\def\emR{{R}}
\def\emS{{S}}
\def\emT{{T}}
\def\emU{{U}}
\def\emV{{V}}
\def\emW{{W}}
\def\emX{{X}}
\def\emY{{Y}}
\def\emZ{{Z}}
\def\emSigma{{\Sigma}}

% entries of a tensor
% Same font as tensor, without \bm wrapper
\newcommand{\etens}[1]{\mathsfit{#1}}
\def\etLambda{{\etens{\Lambda}}}
\def\etA{{\etens{A}}}
\def\etB{{\etens{B}}}
\def\etC{{\etens{C}}}
\def\etD{{\etens{D}}}
\def\etE{{\etens{E}}}
\def\etF{{\etens{F}}}
\def\etG{{\etens{G}}}
\def\etH{{\etens{H}}}
\def\etI{{\etens{I}}}
\def\etJ{{\etens{J}}}
\def\etK{{\etens{K}}}
\def\etL{{\etens{L}}}
\def\etM{{\etens{M}}}
\def\etN{{\etens{N}}}
\def\etO{{\etens{O}}}
\def\etP{{\etens{P}}}
\def\etQ{{\etens{Q}}}
\def\etR{{\etens{R}}}
\def\etS{{\etens{S}}}
\def\etT{{\etens{T}}}
\def\etU{{\etens{U}}}
\def\etV{{\etens{V}}}
\def\etW{{\etens{W}}}
\def\etX{{\etens{X}}}
\def\etY{{\etens{Y}}}
\def\etZ{{\etens{Z}}}

% The true underlying data generating distribution
\newcommand{\pdata}{p_{\rm{data}}}
% The empirical distribution defined by the training set
\newcommand{\ptrain}{\hat{p}_{\rm{data}}}
\newcommand{\Ptrain}{\hat{P}_{\rm{data}}}
% The model distribution
\newcommand{\pmodel}{p_{\rm{model}}}
\newcommand{\Pmodel}{P_{\rm{model}}}
\newcommand{\ptildemodel}{\tilde{p}_{\rm{model}}}
% Stochastic autoencoder distributions
\newcommand{\pencode}{p_{\rm{encoder}}}
\newcommand{\pdecode}{p_{\rm{decoder}}}
\newcommand{\precons}{p_{\rm{reconstruct}}}

\newcommand{\laplace}{\mathrm{Laplace}} % Laplace distribution

\newcommand{\E}{\mathbb{E}}
\newcommand{\Ls}{\mathcal{L}}
\newcommand{\R}{\mathbb{R}}
\newcommand{\emp}{\tilde{p}}
\newcommand{\lr}{\alpha}
\newcommand{\reg}{\lambda}
\newcommand{\rect}{\mathrm{rectifier}}
\newcommand{\softmax}{\mathrm{softmax}}
\newcommand{\sigmoid}{\sigma}
\newcommand{\softplus}{\zeta}
\newcommand{\KL}{D_{\mathrm{KL}}}
\newcommand{\Var}{\mathrm{Var}}
\newcommand{\standarderror}{\mathrm{SE}}
\newcommand{\Cov}{\mathrm{Cov}}
% Wolfram Mathworld says $L^2$ is for function spaces and $\ell^2$ is for vectors
% But then they seem to use $L^2$ for vectors throughout the site, and so does
% wikipedia.
\newcommand{\normlzero}{L^0}
\newcommand{\normlone}{L^1}
\newcommand{\normltwo}{L^2}
\newcommand{\normlp}{L^p}
\newcommand{\normmax}{L^\infty}

\newcommand{\parents}{Pa} % See usage in notation.tex. Chosen to match Daphne's book.

\DeclareMathOperator*{\argmax}{arg\,max}
\DeclareMathOperator*{\argmin}{arg\,min}

\DeclareMathOperator{\sign}{sign}
\DeclareMathOperator{\Tr}{Tr}
\let\ab\allowbreak
\newcommand{\higher}{$\uparrow$}


% \newcommand{\xmark}{\ding{55}}%
% \DeclareMathOperator{\sign}{sgn}
%\newcommand{\todo}[1]{\textcolor{red}{TODO: #1}}
\Crefname{equation}{}{}
% \Crefname{figure}{Fig.}{Figs.}
%\Crefname{tabular}{Tab.}{Tabs.}
% \Crefname{appendix}{App.}{App.}
% \Crefname{section}{Sec.}{Sec.}
\Crefname{proposition_app}{Proposition}{Propositions}

\def\russ#1{{\color{blue}{\bf [russ:} {\it{#1}}{\bf ]}}}
\def\cheng#1{{\color{magenta}{\bf [cs:} {\it{#1}}{\bf ]}}}

\def\deltat{\hat{\delta}}



\title{Generalization Certificates for Adversarially Robust Bayesian Linear Regression}


% \makeatletter
% \renewcommand\AB@affilsepx{, \protect\Affilfont}
% \renewcommand\AB@affilsepx{\quad \protect\Affilfont}
% \makeatother

% The standard author block has changed for UAI 2025 to provide
% more space for long author lists and allow for complex affiliations
%
% All author information is authomatically removed by the class for the
% anonymous submission version of your paper, so you can already add your
% information below.
%
% Add authors
%\author[1]{\href{mailto:<jj@example.edu>?Subject=Your UAI 2025 paper}{Mahalakshmi Sabanayagam}{}}
\author[1]{Mahalakshmi Sabanayagam}
\author[3]{Russell Tsuchida\thanks{work partially done while at Data61, CSIRO}}
\author[4,5]{Cheng Soon Ong}
\author[1,2]{Debarghya Ghoshdastidar}
% Add affiliations after the authors
\affil[1]{%
    School of Computation, Information and Technology\\
    % ${^2}$Munich Data Science Institute\\
    Technical University of Munich\\
    Germany
}
\affil[2]{%
    Munich Data Science Institute\\
    Technical University of Munich\\
    Germany
}
\affil[3]{%
    Monash University\\
    %Australia\\
    $^4$Data61, CSIRO\\
    %Australia\\
    $^5$College of Systems and Society\\
    Australian National University\\
    Australia
}
% \affil[4]{%
%     Data61, CSIRO\\
%     Australia
% }
% \affil[5]{
%     College of Systems and Society\\
%     Australian National University
% }
\affil[ ]{\texttt{sabanaya@in.tum.de, russell.tsuchida@monash.edu, chengsoon.ong@anu.edu.au, ghoshdas@in.tum.de}}
  \begin{document}
  \DeclareFontShape{T1}{calligra}{m}{n}{<->s*[2.5]callig15}{}
\maketitle

\begin{abstract}
\vspace{-0.2cm}
  Adversarial robustness of machine learning models is critical to ensuring reliable performance under data perturbations. Recent progress has been on point estimators, and this paper considers distributional predictors. First, using the link between exponential families and Bregman divergences, we formulate an adversarial  Bregman divergence loss as an adversarial negative log-likelihood. Using the geometric properties of Bregman divergences, we  compute the adversarial perturbation for such models in closed-form. Second, under such losses, we introduce \emph{adversarially robust posteriors}, by exploiting the optimization-centric view of generalized Bayesian inference. Third, we derive the \emph{first} rigorous generalization certificates in the context of an adversarial extension of Bayesian linear regression by leveraging the PAC-Bayesian framework. Finally, experiments on real and synthetic datasets demonstrate the superior robustness of the derived adversarially robust posterior over Bayes posterior, and also validate our theoretical guarantees.
  %Third, we study generalization certificates in the context of an adversarial extension of Bayesian linear regression. 
  %Leveraging the PAC-Bayesian framework, we derive novel generalization risk certificates 
  %for both standard Bayes and adversarially robust posteriors, considering both standard and adversarial negative log likelihood loss functions. 
  %Our analysis provides the \emph{first} rigorous non-trivial generalization guarantees for adversarially robust probabilistic models. 
  %we evaluate the robustness Bayes and adversarially robust posteriors on real data, and validate the generalization certificates on synthetic data. %Experimental evaluation of the bounds shows the non-vacuousness of the certificates.
\end{abstract}
\vspace{-0.2cm}

\section{Introduction}

Video generation has garnered significant attention owing to its transformative potential across a wide range of applications, such media content creation~\citep{polyak2024movie}, advertising~\citep{zhang2024virbo,bacher2021advert}, video games~\citep{yang2024playable,valevski2024diffusion, oasis2024}, and world model simulators~\citep{ha2018world, videoworldsimulators2024, agarwal2025cosmos}. Benefiting from advanced generative algorithms~\citep{goodfellow2014generative, ho2020denoising, liu2023flow, lipman2023flow}, scalable model architectures~\citep{vaswani2017attention, peebles2023scalable}, vast amounts of internet-sourced data~\citep{chen2024panda, nan2024openvid, ju2024miradata}, and ongoing expansion of computing capabilities~\citep{nvidia2022h100, nvidia2023dgxgh200, nvidia2024h200nvl}, remarkable advancements have been achieved in the field of video generation~\citep{ho2022video, ho2022imagen, singer2023makeavideo, blattmann2023align, videoworldsimulators2024, kuaishou2024klingai, yang2024cogvideox, jin2024pyramidal, polyak2024movie, kong2024hunyuanvideo, ji2024prompt}.


In this work, we present \textbf{\ours}, a family of rectified flow~\citep{lipman2023flow, liu2023flow} transformer models designed for joint image and video generation, establishing a pathway toward industry-grade performance. This report centers on four key components: data curation, model architecture design, flow formulation, and training infrastructure optimization—each rigorously refined to meet the demands of high-quality, large-scale video generation.


\begin{figure}[ht]
    \centering
    \begin{subfigure}[b]{0.82\linewidth}
        \centering
        \includegraphics[width=\linewidth]{figures/t2i_1024.pdf}
        \caption{Text-to-Image Samples}\label{fig:main-demo-t2i}
    \end{subfigure}
    \vfill
    \begin{subfigure}[b]{0.82\linewidth}
        \centering
        \includegraphics[width=\linewidth]{figures/t2v_samples.pdf}
        \caption{Text-to-Video Samples}\label{fig:main-demo-t2v}
    \end{subfigure}
\caption{\textbf{Generated samples from \ours.} Key components are highlighted in \textcolor{red}{\textbf{RED}}.}\label{fig:main-demo}
\end{figure}


First, we present a comprehensive data processing pipeline designed to construct large-scale, high-quality image and video-text datasets. The pipeline integrates multiple advanced techniques, including video and image filtering based on aesthetic scores, OCR-driven content analysis, and subjective evaluations, to ensure exceptional visual and contextual quality. Furthermore, we employ multimodal large language models~(MLLMs)~\citep{yuan2025tarsier2} to generate dense and contextually aligned captions, which are subsequently refined using an additional large language model~(LLM)~\citep{yang2024qwen2} to enhance their accuracy, fluency, and descriptive richness. As a result, we have curated a robust training dataset comprising approximately 36M video-text pairs and 160M image-text pairs, which are proven sufficient for training industry-level generative models.

Secondly, we take a pioneering step by applying rectified flow formulation~\citep{lipman2023flow} for joint image and video generation, implemented through the \ours model family, which comprises Transformer architectures with 2B and 8B parameters. At its core, the \ours framework employs a 3D joint image-video variational autoencoder (VAE) to compress image and video inputs into a shared latent space, facilitating unified representation. This shared latent space is coupled with a full-attention~\citep{vaswani2017attention} mechanism, enabling seamless joint training of image and video. This architecture delivers high-quality, coherent outputs across both images and videos, establishing a unified framework for visual generation tasks.


Furthermore, to support the training of \ours at scale, we have developed a robust infrastructure tailored for large-scale model training. Our approach incorporates advanced parallelism strategies~\citep{jacobs2023deepspeed, pytorch_fsdp} to manage memory efficiently during long-context training. Additionally, we employ ByteCheckpoint~\citep{wan2024bytecheckpoint} for high-performance checkpointing and integrate fault-tolerant mechanisms from MegaScale~\citep{jiang2024megascale} to ensure stability and scalability across large GPU clusters. These optimizations enable \ours to handle the computational and data challenges of generative modeling with exceptional efficiency and reliability.


We evaluate \ours on both text-to-image and text-to-video benchmarks to highlight its competitive advantages. For text-to-image generation, \ours-T2I demonstrates strong performance across multiple benchmarks, including T2I-CompBench~\citep{huang2023t2i-compbench}, GenEval~\citep{ghosh2024geneval}, and DPG-Bench~\citep{hu2024ella_dbgbench}, excelling in both visual quality and text-image alignment. In text-to-video benchmarks, \ours-T2V achieves state-of-the-art performance on the UCF-101~\citep{ucf101} zero-shot generation task. Additionally, \ours-T2V attains an impressive score of \textbf{84.85} on VBench~\citep{huang2024vbench}, securing the top position on the leaderboard (as of 2025-01-25) and surpassing several leading commercial text-to-video models. Qualitative results, illustrated in \Cref{fig:main-demo}, further demonstrate the superior quality of the generated media samples. These findings underscore \ours's effectiveness in multi-modal generation and its potential as a high-performing solution for both research and commercial applications.
\vspace{-3mm}
\section{Preliminaries}
\label{sec:preliminaries}
\subsection{Formulation of Collaborative Perception}
\label{sec:formulation}

% \vspace{-3mm}
In this section, we formulate collaborative perception and give the pipeline of our CP system. Specifically, let $\mathcal{X}^N$ denote the set of $N$ CAVs in the CP system. CAVs in $\mathcal{X}$ can be divided into two categories: the ego CAV and helping CAVs. The ego CAV is the one that needs to perceive its surrounding environment, while helping CAVs are the ones that send their complementary sensing information to the ego CAV to help it enhance its perception performance.
Thus, each CAV can be an ego one and helping one, depending on its role in a perception process. We assume that each CAV is equipped with a feature encoder $f_\mathtt{{encoder}}(\cdot)$, a feature aggregator $f_\mathtt{{agg}}(\cdot)$, and a feature decoder $f_\mathtt{{decoder}}(\cdot)$. For the $i$-th CAV in the set $\mathcal{X}$, the raw observation is denoted as $\mathbf{O}_i$ (such as camera images and LiDAR point clouds), and the final perception results are denoted as $\mathbf{Y}_i$. The CP pipeline of the $i$-th CAV can be described as follows.
\begin{enumerate}
    % \vspace{-3mm}
    \setlength{\itemsep}{0pt}
    \setlength{\parskip}{0pt}
    \setlength{\parsep}{0pt}
    \item \textit{Observation Encoding}: Each CAV encodes its raw observation $\mathbf{O}_j$ into an initial feature map $\mathbf{F}_j = f_\mathtt{{encoder}}(\mathbf{O}_j)$, where $j \in \mathcal{X}^N$.
    \item \textit{Intermediate Feature Transmission}: Helping CAVs transmit their intermediate features to the ego CAV: $\mathbf{F}_{j\rightarrow i}=\mathbf{\Gamma}_{j\rightarrow i}(\mathbf{F}_j),\  j\in \mathcal{X}^N, j\neq i,$
    where $\mathbf{\Gamma}_{j\rightarrow i}(\cdot)$ denotes a transmitter that conveys the $j$-th CAV's intermediate feature $\mathbf{F}_j$ to the ego CAV, while performing a spatial transformation. $\mathbf{F}_{j\rightarrow i}$ is the spatially aligned feature in the $i$-th CAV's coordinate.
    \item \textit{Feature Aggregation}: The ego CAV receives all the intermediate features and fuses them into a unified observational feature $\mathbf{F}_\mathtt{fused}=f_\mathtt{agg}(\mathbf{F}_{0\rightarrow i}, \{\mathbf{F}_{j\rightarrow i}\}_{j\neq i,\  j\in \mathcal{X}^N})$.
    \item \textit{Perception Decoding}: Finally, the ego CAV decodes the unified observational feature $\mathbf{F}_\mathtt{fused}$ into the final perception results $\mathbf{Y}=f_\mathtt{decoder}(\mathbf{F}_\mathtt{fused})$.
    % \vspace{-4mm}
\end{enumerate}


\begin{figure*}[t]
    % \vspace{-5mm}
    \centering
    % \fbox{\rule{0pt}{1.8in} \rule{0.9\linewidth}{0pt}}
    \includegraphics[width=.9\linewidth]{fig/CPDataGenerationPipeline.png}
    \vspace{-3mm}
    \caption{\textbf{Automatic Data Generation and Annotation Pipeline.} We first train a robust LiDAR collaborative object detector. Then, we discard the detection head and decoder and only keep the backbone as the intermediate feature generator. The data generation pipeline is shown in (a), (b), and (c), where (a) is the intermediate feature generation, (b) is the attack implementation, and (c) is the pair generation and saving.}
    \label{fig:data_generation}
    \vspace{-5mm} 
\end{figure*}



\subsection{Adversarial Threat Model}

Our focus is on the operation of an  intermediate-fusion collaboration scheme, where an attacker introduces designed adversarial perturbations into the intermediate features to mislead the perception of the ego CAV. Since an attacker participates in the collaborative system with local perception model installation, we assume they have white-box access to the model parameters. The attack procedure in each frame follows four sequential phases.
\begin{enumerate}
    \vspace{-4mm}
    \setlength{\itemsep}{0pt}
    \setlength{\parskip}{0pt}
    \setlength{\parsep}{0pt}
    \item \textit{Local Perception Phase}: All agents, including the malicious one, process their sensing data independently and extract intermediate features using feature encoders. 
    \vspace{-1mm}
    \begin{equation}
    \mathbf{F}_k = f_\mathtt{encoder}(\mathbf{O}_k), \quad k \in \mathcal{X}^N
    \end{equation}
    \vspace{-1mm}
    This phase operates in parallel without inter-agent communication.

    \item \textit{Feature Communication Phase}: All agents broadcast their extracted features through the network. Malicious agent $k$ collects feature information $\{\mathbf{F}_{j\rightarrow i}\}$ from other agents. Feature-level transmission ensures minimal communication overhead compared to raw sensor data exchange.

    \item \textit{Attack Generation Phase}: A malicious agent executes the attack by first perturbing its local features and then propagating them through the collaborative perception pipeline described in Section \ref{sec:formulation}.
    The attacker aims to optimize the perturbation $\delta$ through an iterative process. The optimization objective is formulated as:
    \vspace{-1mm}
    \begin{equation}
        \vspace{-2mm}
        \begin{aligned}
        \mathop{\arg\max}_{\delta} \mathcal{L}(\mathbf{Y}^\delta, \mathbf{Y}^\mathtt{gt}),
        \quad \mathtt{s.t.}\quad  \|\delta\|\leq \Delta
        \end{aligned}
    \end{equation}
    where $\Delta$ bounds the perturbation magnitude to maintain attack stealthiness. The total loss function is designed to aggregate adversarial losses over all object proposals, targeting both classification and localization aspects:
    \vspace{-1mm}
    \begin{equation}
        \vspace{-2mm}
        \mathcal{L}(\mathbf{Y}^\delta, \mathbf{Y}^\mathtt{gt}) = \sum_{p \in \mathbf{Y}^\delta} \mathcal{L}_\mathtt{adv}(p, p^\mathtt{gt})
    \end{equation}
    For each proposal $p$ with the highest confidence class $c = \mathop{\arg\max}\{p_i\}$, we leverage a class-specific adversarial loss following \citep{tuAdversarialAttacksMultiAgent2021}:
    \begin{equation*}
        % \vspace{-2mm}
        \mathcal{L}_\mathtt{adv}(p', p) = \begin{cases}
            -\log(1 - p'_c)\cdot\eta & c \neq k,\ p_c > \tau_1\\
            -\lambda p'_c\log(1 - p'_c) & c = k,\ p_c > \tau_2\\
            0 & \text{otherwise}
        \end{cases}
    \end{equation*}
    where $\eta$ represents the IoU between perturbed and original proposals to consider spatial accuracy, $\tau_1$ and $\tau_2$ are confidence thresholds for different attack scenarios, $\lambda$ balances the importance of different attack objectives, and $k$ denotes the background class.

    \item \textit{Defense and Final Perception Phase}: The ego vehicle integrates all received feature information, including potentially corrupted ones, to complete the final object detection task. Note that we focus exclusively on CP-specific vulnerabilities, excluding physical sensor attacks (e.g., LiDAR or GPS spoofing), which are general threats to CAVs. We also assume communication channels are secured with proper cryptographic protection.
\end{enumerate}

\section{Adversarially robust generalized linear models}
\label{sec:adversarially-robust-posterior}

In this section, we derive the adversarially robust posterior $q_\delta(\theta)$ when the likelihood is respectively a Gaussian likelihood, and more generally an exponential family likelihood.
The result in~\Cref{lm:adv_loss_closed_form} may be of independent interest for studying adversarially robust models even in the setting of point-estimation.
It allows, for example, an adversarially robust extension of  logistic regression (binary-valued data), Poisson regression (count-valued data), and exponential or gamma regression (positive-valued data).
More generally, any generalized linear model~\citep{mccullagh1989generalized} with canonical link function may be adversarialized. 

\paragraph{Adversarial negative log likelihood} 
We consider adversarial losses $ {\ell}_\delta(\theta, \mathcal{D})$ of the form
\iffalse
\begin{align*}
     {\ell}_\delta(\theta, \mathcal{D}) &= \sum_{i=1}^n \max_{\|  \widetilde{x}_i - x_i \|_2 \leq \delta} - \log p\big(y_i \mid  f_\theta(\widetilde{x}_i) \big) \numberthis \label{eq:nll_exponentialfam}\\
     &= \sum_{i=1}^n \max_{\|  \widetilde{x}_i - x_i \|_2 \leq \delta} d_\psi\big(f_\theta(x_i), y_i^\ast \big) + C(y) \numberthis \label{eq:adv_loss}
\end{align*} 
\fi
\begin{align*}
     {\ell}_\delta\big(\theta, (x,y) \big) &=  \max_{\|  \widetilde{x} - x \|_2 \leq \delta} - \log p\big(y \mid  f_\theta(\widetilde{x}) \big) \numberthis \label{eq:nll_exponentialfam}\\
     &=  \max_{\|  \widetilde{x} - x \|_2 \leq \delta} d_\psi\big(f_\theta(\widetilde{x}), y^\ast \big) + C(y), \numberthis \label{eq:adv_loss}
\end{align*} 
where $\delta$ controls the allowable perturbation in the features. 
%The corresponding adversarially robust posterior $ {q}_\delta(\theta)$ is the Gibbs posterior with loss $\mathcal{L}(\theta, \mathcal{D}) =  {\ell}_\delta(\theta, \mathcal{D})$. 
%This formulation is motivated by its equivalence to the adversarial training objective as proved in the following lemma. 
\iffalse
We motivate probabilistic adversarial losses through the link between Bregman divergences and exponential families.
\begin{lemma}[Equivalence between adversarial loss and adversarial training] 
The point estimate obtained by optimizing $\arg\min_{\theta}  {\ell}_\delta(\theta, \mathcal{D})$ is equivalent to performing adversarial training with objective $\arg \min_{\theta} \sum_{i=1}^n \max_{\| \widetilde{x}_i -x_i\|_2 \leq \delta} d_\psi\big(y_i, f_\theta(x_i)\big)$.
\label{lm:adv_loss_adv_training}
\end{lemma}
\fi
Considering a linear predictor and Gaussian likelihood (squared error Bregman divergence) allows us to derive the robust loss in closed-form. 
\begin{lemma}[Robust loss in closed-form for Gaussian likelihood] \label{lm:adv_loss_closed_form_gaussian}
Under a linear predictor $f_\theta(x)=\theta^\top x$ and in the case where the exponential family is a Gaussian family,
$$ \ell_\delta(\theta, \mathcal{D}) = \frac{n}{2}\log \lp 2\pi \sigma^2 \rp + \frac{1}{2\sigma^2} \Big\| |Y - X\theta| + \delta \|\theta\| 1_n \Big\|^2,$$
and the adversarial perturbation of the sample $x$ is $\widetilde{x} = \delta \sign(\theta^\top x - y) \frac{\theta}{\Vert \theta \Vert_2} + x = \delta \sign(\theta^\top \widetilde{x} - y) \frac{\theta}{\Vert \theta \Vert_2} + x$.
\end{lemma}
More generally, a linear predictor with any exponential family likelihood (Bregman divergence) allows us to derive the robust loss in closed-form.
% (\Cref{lm:adv_loss_closed_form}). %as stated in \Cref{lm:adv_loss_closed_form}. \Cref{lm:adv_loss_adv_training,lm:adv_loss_closed_form} are proved in \Cref{app:adv_loss}. 
\iffalse
\begin{lemma}[Adversarial loss in closed-form] Under a linear predictor $f_\theta(x) = \theta^\top x$ and Gaussian likelihood, the closed form for the robust loss is
$$ {\ell}_\delta(\theta, \mathcal{D}) = \frac{n}{2}\log \lp 2\pi \sigma^2 \rp + \frac{1}{2\sigma^2} \Big\| |Y - X\theta| + \delta \|\theta\| 1_n \Big\|^2.$$
Here the adversarial perturbation of sample $x_i$ is $\widetilde{x}_i = x_i - \delta \sign(y_i-x_i^T\theta)\frac{\theta}{\|\theta\|}$. 
\label{lm:adv_loss_closed_form}
\end{lemma} \Cref{lm:adv_loss_adv_training,lm:adv_loss_closed_form} are proved in \Cref{app:adv_loss}. 
%\todo{should we consider the constant term in $\ell$ and $ {\ell}$?}
\fi

\begin{lemma}[Robust loss in closed-form for exponential family likelihood]
\label{lm:robust-expfam-loss}
Under a linear predictor $f_\theta(x)=\theta^\top x$ and an exponential family likelihood, the robust loss is
    \begin{align*}
        &\phantom{{}={}} \ell_{\delta}\
\big(\theta, (x, y) \big) \\
&= \max_{s \in \{-1, 1\}}\psi(s \delta \Vert \theta \Vert_2 + \theta^\top x) - \psi(\theta^\top x) - y s \delta \Vert \theta \Vert_2 \\
&\phantom{{}=\max_{s \in \{-1, 1\}}}+ d_\psi(\theta^\top x, y^\ast) + C(y),
    \end{align*}
and the adversarial perturbation of the sample $x$ is $\widetilde{x} = \delta \sign \big(\nabla(\psi(\theta^\top \widetilde{x}) - y \big) \frac{\theta}{\Vert \theta \Vert_2} + x$.
\label{lm:adv_loss_closed_form}
\end{lemma}
Note that in the case of a general exponential family likelihood, a trivial maximization problem over $s \in \{-1, 1\}$ must be solved.
All other terms in~\Cref{lm:adv_loss_closed_form} are available in closed-form. 
In practice, this optimization problem can be solved by simply evaluating the objective for $s=1$ and $s=-1$, and picking the result with the highest value.
\Cref{lm:adv_loss_closed_form,lm:adv_loss_closed_form_gaussian} are proved in \Cref{app:adv_loss}. 

\iffalse
\begin{lemma}[Equivalence between adversarial loss and adversarial training] 
The point estimate obtained by optimizing $\arg\min_{\theta}  {\ell}_\delta(\theta, \mathcal{D})$ is equivalent to performing adversarial training with objective $\arg \min_{\theta} \sum_{i=1}^n \max_{\| \widetilde{x}_i -x_i\|_2 \leq \delta} \lp  \widetilde{x}_i^\top \theta - y_i \rp^2$.
\label{lm:adv_loss_adv_training}
\end{lemma}
\fi


\iffalse
\begin{lemma}[Equivalence between adversarial loss and adversarial training] 
The point estimate obtained by optimizing $\arg\min_{\theta}  {\ell}_\delta(\theta, \mathcal{D})$ is equivalent to performing adversarial training with objective $\arg \min_{\theta} \sum_{i=1}^n \max_{\| \widetilde{x}_i -x_i\|_2 \leq \delta} \lp  \widetilde{x}_i^\top \theta - y_i \rp^2$.
\label{lm:adv_loss_adv_training}
\end{lemma}
\fi

We are now ready to define our robust posterior for Bayesian generalized linear models.
\begin{corollary}[Robust posterior]    \label{cor:robust_posterior}
    The Gibbs posterior~\eqref{eq:gibbs_posterior} obtained by setting the loss $\mathcal{L}$ to be an adversarially perturbed exponential family NLL~\eqref{eq:nll_exponentialfam} (or equivalently, an adversarially perturbed Bregman divergence~\eqref{eq:adv_loss}) under a linear model $f_\theta(x) = \theta^\top x$ is given by
    \begin{align*}
        q_\delta(\theta) = \frac{\exp\Big( -\sum_{i=1}^N\ell_\delta\big( \theta, (x_i, y_i)\big) \Big) \pi(\theta) }{\int \exp\Big( - \sum_{i=1}^N\ell_\delta\big( \theta', (x_i, y_i)\big) \Big) \pi(\theta') d\theta'},
    \end{align*}
    where $\ell_\delta\big( \theta, (x_i, y_i)\big) $ is as in~\Cref{lm:adv_loss_closed_form}, or in the special case of a Gaussian (or squared loss),~\Cref{lm:adv_loss_closed_form_gaussian}.
\end{corollary}
We note that this notion of a robust posterior is not the only choice, however this choice does lead to tractable losses derived from adversarial likelihoods, and also allows us to derive generalization guarantees in \Cref{sec:pac_bayes_certs}.
See \Cref{app:adv_loss} for a discussion of other choices.
\section{Standard and Adversarial Generalization Certificates}
\label{sec:pac_bayes_certs}
In this section, we focus on Bayesian linear regression (i.e. a robustified squared error loss or Gaussian NLL) for the robust posterior in~\Cref{cor:robust_posterior}.
We consider labels generated using a true parameter $\theta^*$,  $y_i=x_i^\top \theta^* + \epsilon_i$ where $\mathbb{E}[x_i] = 0, \, \mathbb{E}[\|x_i\|^2] = \sigma_x^2$ and $\epsilon_i \sim \mathcal{N}(0,\sigma^2)$. %Recall that $X \in \mathbb{R}^{n\times d}$ is the feature matrix, and $Y \in \mathbb{R}^n$ is the label vector. 

\paragraph{PAC-Bayesian generalization certificates} 
Unlike traditional generalization bounds based on uniform convergence such as VC-dimension \citep{vapnik1971uniform}, Rademacher complexity \citep{shalev2014understanding}, and information-theory \citep{zhang2006information}, PAC-Bayes \citep{mcallester1998some} focuses on Bayesian predictors %, which make predictions by drawing parameters from a learned posterior distribution 
rather than a single deterministic hypothesis class. This perspective allows PAC-Bayes to provide \emph{data-dependent} generalization guarantees, 
%PAC-Bayesian theorems provide data-driven generalization bounds 
that are computed on training samples without relying on the test data. 
As such, all certificates computed in this section depend on the data $X$, $Y$ in a non-obvious way.
%Furthermore, the guarantees are uniformly valid for all $\theta \sim \rho$, %\russ{something wrong with this notation here $\rho(\theta)$ is a nonnegative number, not a set}.
%thus, providing numerical certificates on the generalization capabilities of the model. 
%To derive PAC-Bayesian bounds, we assume the prior to be Gaussian distribution $\theta \sim \mathcal{N}(0, \sigma^2_p I)$ where $I$ is identity matrix.
While other approaches based on uniform convergence or information theory result in a worst-case guarantee, PAC-Bayesian offers fine-grained analysis by taking advantage of informed prior choice leading to a tighter certificate. 
%only in an estimate of the generalization, and not certificates. 
%\russ{Is this related to ``oracle'' versus ``empirical'' bounds, or is that something different?}
%The PAC-Bayes bounds provide certificates unlike the others which only give an estimate of the generalization. Others include Rademacher, uniform convergence, VC theory..

\paragraph{Data and constants in bounds} In addition to the data $X, Y$, each of the bounds in this section also rely on constants such as the training and testing perturbation allowances $\delta$ and $\deltat$.
We intuitively describe the role of these data and constant terms following the presentation of each of the bounds, even though the main purpose of the bounds is as computable data-dependent certificates.

\paragraph{Standard and adversarial generalization}
We derive certificates for both standard and adversarial generalization (columns in \Cref{tab:overview}). In order to formalize this, let us define the expected and empirical errors for standard loss $\ell$ as $R(\theta)=\expectation{(x,y) \sim \mathcal{P}}{\ell\big(\theta, (x,y)\big)}$ and $r(\theta)=\frac{1}{n} \mathcal{L}(\theta, \mathcal{D})$, respectively. Similarly, the expected and empirical adversarial errors under perturbation $\delta$ are defined as $ {R}_\delta(\theta)=\expectation{(x,y) \sim \mathcal{P}}{ {\ell}_\delta\big(\theta, (x,y)\big)}$ and $ {r}_\delta(\theta)=\frac{1}{n}  \mathcal{L}_\delta(\theta, \mathcal{D})$, respectively. 
In the context of Bayesian inference, the standard generalization risk certificate quantifies the expected loss of the posterior $\rho(\theta)$ on unperturbed test data $x$: $\expectation{\theta \sim \rho}{R(\theta)}$. %, i.e., how well the posterior $\rho$ performs on non-perturbed test data in the worst-case. 
Similarly, the adversarial generalization certificate quantifies the performance
%on adversarially perturbed test data:
of the test data under adversarial perturbation $\deltat$: % is obtained by bounding the error on adversarially perturbed test data, i.e., 
$\expectation{\theta \sim \rho}{ {R}_{\deltat}(\theta)}$. %, i.e., how well the posterior $\rho$ performs on adversarially perturbed test data. 
% We derive the bounds on these quantities (thereby certificates) considering both standard and adversarially robust posteriors, $q(\theta)$ and $ {q}(\theta)$, respectively.
We derive the certificates by bounding the respective quantity for both the standard Bayes posterior $q(\theta)$ and adversarially robust posterior $ {q}_\delta(\theta)$.
Note that we allow the perturbation $\delta$ used for inference (i.e. calculation of $ {q}_\delta(\theta)$) to be different to the perturbation $\deltat$ at test-time.

\subsection{Cumulant generating function}% of losses}

To derive the standard and adversarial generalization certificates, we leverage the PAC-Bayesian theorem for any loss with bounded cumulant generating function (CGF) in \citet{banerjee2021information}. 
We state the result in \Cref{th:pac_bayes_bounded_cgf}, which requires a bounded CGF.
We then show that the CGFs of the standard and adversarial losses corresponding with Gaussian NLLs are bounded in \Cref{lm:cgf_bounds_std_adv_loss}. 
%Recently, \citet{banerjee2021information} derived tighter PAC-Bayesian bounds for any loss with bounded cumulant generating function (CGF). 
\begin{theorem}[Theorem 6 in \citet{banerjee2021information}] \label{th:pac_bayes_bounded_cgf}
Consider data $\mathcal{D}$ and any loss $\ell \big(\theta, (x,y)\big)$ with its corresponding expected and empirical generalization errors $R(\theta)=\expectation{(x,y) \sim \mathcal{P}}{\ell \big(\theta, (x,y)\big)}$ and $r(\theta)=\frac{1}{n} \mathcal{L}(\theta, \mathcal{D})$, respectively.  %, and a random variable $Z=\expectation{}{\ell } - \ell $. 
Let the CGF of the loss $\psi(t) = \log \expectation{}{\exp \lp t \lp \expectation{}{\ell } - \ell  \rp \rp}$  be bounded, where for some constant $c>0$, $t \in (0,1/c)$. 
%\russ{What is $c$?}
Then, we have, with probability at least $1-\beta$, for all densities 
$\rho(\theta)$,
    $$\expectation{\theta \sim \rho}{R(\theta)}
\leq 
\expectation{\theta \sim \rho}{r(\theta)} 
+ \frac{1}{t} \left[ \frac{\mathrm{KL}({\rho} \| \pi) + \log \frac{1}{\beta}}{n} + \psi(t) \right].$$
\end{theorem}
Before stating the CGF bound for the losses, we first define a \textit{sub-gamma} random variable. A random variable with variance $s^2$ and scale parameter $c$ is said to be sub-gamma if its CGF $\psi$ satisfies the following upper bound: 
$$\psi(t) \leq \frac{s^2t^2}{2(1-ct)} \text{ for all } 0 < t < 1/c.$$ 
We state the CGF bounds for both standard and adversarial losses in \Cref{lm:cgf_bounds_std_adv_loss} and provide the proof in \Cref{app:cgf_std_adv_loss}.

% \russ{What is sub-gamma?}
\begin{lemma}%[CGF of standard and adversarial losses are bounded] 
[CGF bounds for standard and adversarial losses] \label{lm:cgf_bounds_std_adv_loss}
%The standard and adversarial losses are sub-gamma distributed and their CGFs are bounded as 
%The CGF of both standard and adversarial losses are bounded as 
%$\psi(t) \leq \frac{s^2t^2}{2(1-ct)}$ for all $0<t<1/c$.
The standard and adversarial losses are both sub-gamma with the following variance $s^2$ and scale factor $c$. 
In the case of standard loss, 
%$$s^2 = \frac{2}{t} \lp \sigma^2_p \sigma_x^2d + \sigma_x^2 \|\theta^*\|^2 + \sigma^2 \lp 1-2t\sigma^2_p \sigma_x^2\rp \rp, c=2\sigma^2_p \sigma_x^2.$$ %and $t \in (0,\frac{1}{2\sigma^2_p \sigma_x^2}).$
\begin{align}
    c&=\frac{\sigma^2_p \sigma_x^2}{\sigma^2}, \quad %\nonumber \\
    s^2 = \frac{1}{t} \lp cd - ct +1 + \frac{\sigma_x^2 \|\theta^*\|^2}{\sigma^2}\rp.
    %s^2 &= \frac{2}{t} \lp \sigma^2_p \sigma_x^2 \lp d - 2\sigma^2t \rp + \sigma_x^2 \|\theta^*\|^2 + \sigma^2 \rp. 
    \label{eq:cgf_std_loss}
\end{align}
For adversarial loss with $\deltat$ perturbation, 
%$$s^2 = \frac{4}{t} \lp \sigma_p^2d \lp \sigma_x^2 + \delta^2 \rp + \sigma_x^2 \|\theta^*\|^2 + 2\sigma^2 \lp 1-4\sigma^2_pt\lp \sigma_x^2+\delta^2 \rp \rp \rp,$$ $$c=4\sigma^2_p\lp \sigma_x^2+\delta^2 \rp.$$ % and $t\in \lp 0, \frac{1}{4\sigma^2_p\lp \sigma_x^2+\delta^2 \rp} \rp$.
\begin{align}
c&=\frac{2\sigma^2_p\lp \sigma_x^2+\deltat^2 \rp}{\sigma^2}, \nonumber \\
s^2 &= \frac{2}{t} \lp cd - ct + 1 + \frac{\sigma_x^2 \|\theta^*\|^2}{\sigma^2} \rp.
\label{eq:cgf_adv_loss}
\end{align}
\end{lemma}

Using the sub-gamma property of the losses and applying their CGF bounds in \Cref{th:pac_bayes_bounded_cgf}, we derive the standard and adversarial generalizations of Bayes posterior $q(\theta)$ in \Cref{ss:bayes_cert}, % \Cref{thm:std_post_std_loss,thm:std_post_adv_loss}, 
and robust posterior $q_\delta(\theta)$ in \Cref{ss:robust_cert}, and present the proofs in \Cref{app:pac_bounds_proof_robust_posterior}. % \Cref{thm:adv_post_std_loss,thm:adv_post_adv_loss_gen,thm:adv_post_adv_loss}. %\Cref{thm:std_post_std_loss,thm:std_post_adv_loss,thm:adv_post_std_loss,thm:adv_post_adv_loss_gen,thm:adv_post_adv_loss}. 
Each of the bounds depends on parameters $c$ and $s^2$, and intuitively larger values of either lead to worse bounds, as the losses are subject to higher variability.

\subsection{Generalization certificates for Bayes posterior}
\label{ss:bayes_cert}
Using the sub-gamma property of the standard loss, the certificate for the standard generalization of the Bayes posterior $q(\theta)$ is derived in \citet{germain2016pac} by setting the free variable $t$ in CGF to $1$. We restate this result in \Cref{thm:std_post_std_loss}, expressing it explicitly in terms of the data. This contrasts with the formulation in \citet[Corollary 5]{germain2016pac}, where the bound is expressed in terms of the posterior (which in turn depends on the data). %Nevertheless, the bounds essentially capture the same behavior 
%Fundamentally, both bounds capture the same generalization behavior, with the difference being that our formulation explicitly highlights the dependence on data. % and enables 
%which is consistent with the result in \citet[Corollary 5]{germain2016pac}.

\begin{theorem}[Standard generalization of Bayes posterior, adapted from \citet{germain2016pac}] \label{thm:std_post_std_loss}
Consider $c$ and $s$ as defined in \Cref{eq:cgf_std_loss} with $t=1$, $\sigma_p^2 < \frac{\sigma^2}{\sigma_x^2}$, $W_d= I_d + \frac{\sigma_p^2}{\sigma^2} X^\top X$ and $W_n= I_n + \frac{\sigma_p^2}{\sigma^2} XX^\top$. Then, with probability at least $1-\beta$, we have the following certificate for standard generalization of the Bayes posterior $q(\theta)$: 
\begin{align}
\expectation{\theta \sim q}{R(\theta)} &\leq \frac{1}{n} \log \sqrt{\det \lp W_d \rp} + \frac{1}{2n\sigma^2} Y^\top W_n^{-1} Y  \nonumber \\
&\qquad + \frac{1}{n} \log \frac{1}{\beta}  +\frac{s^2}{2(1-c)}.
\label{eq:pac_bayes_std_post_std_loss}
\end{align}
%holding with probability at least $1-\beta$.
%where $M= I + 2\sigma_p^2 XX^\top$.
%holds with probability at least $1-\beta$. % where $s,c$ are as defined in \Cref{eq:cgf_std_loss}. %and $\sigma_p^2 < \frac{1}{2\sigma_x^2}$.
\end{theorem}
Increasing sub-gamma variability parameters $s^2$ and $c$ increase the bound~\eqref{eq:pac_bayes_std_post_std_loss}, as expected. 
Informally, the term depending on $\log \det W_d$ is a sum of $d$ $\log$ eigenvalues of $W_d$, and if $X^\top X$ is low-rank, most of these $\log$ eigenvalues are close to $0$. 
Hence the first term decreases like $1/n$.
The other data dependent term is essentially the product of $Y^\top$ and the average training error of ridge regression, which should be small if the dataset is large and the model is well-specified.
%Subsequently, we derive the other cases in the following and establish new results, especially for adversarial loss and adversarially robust posterior. %While the standard generalization of the adversarial posterior might not be a realistic valid setting, we provide the result for completion.

Next, we derive the adversarial generalization certificate for Bayes posterior similar to standard generalization.

\begin{theorem}[Adversarial generalization of Bayes posterior]\label{thm:std_post_adv_loss}
Consider $c$ and $s$ as defined in \Cref{eq:cgf_adv_loss} with $t=1$, $\sigma_p^2 < \frac{\sigma^2}{2 \lp \sigma_x^2 + \deltat^2\rp}$, $W_d= I_d + \frac{\sigma_p^2}{\sigma^2} X^\top X$ and $W_n= I_n + \frac{\sigma_p^2}{\sigma^2} XX^\top$. Then, with probability at least $1-\beta$, we have the following certificate for adversarial generalization of the Bayes posterior ${q}(\theta)$: 
\begin{align}
&\expectation{\theta\sim q}{  {R}_{\deltat}(\theta)} \leq \frac{2}{n} \log \sqrt{\det \lp W_d \rp} + \frac{1}{n\sigma^2} Y^\top W_n^{-1} Y \nonumber \\
&\qquad \qquad + \frac{1}{n} \log \frac{1}{\beta}  +\frac{s^2}{2(1-c)} + \frac{d\deltat^2 \sigma_p^2}{\sigma^2-2n\deltat^2 \sigma_p^2}. 
\label{eq:pac_bayes_std_post_adv_loss}
\end{align}
\end{theorem}

While \Cref{thm:std_post_std_loss,thm:std_post_adv_loss} are upper bounds and are incomparable, we note that the main difference in \Cref{thm:std_post_adv_loss} is the additional constant term dependent on the perturbation radius $\deltat$ and the data-dependent terms are scaled by $2$. 
The additional constant term captures the effect of testing the model adversarially, increasing the bound. 
This term behaves linearly in $\deltat$ for small $\deltat$.
%The constants $c$ and $s$ in \Cref{thm:std_post_adv_loss} are specific to adversarial loss derived in \Cref{eq:cgf_adv_loss}.
%Proofs are presented in \Cref{app:pac_bounds_proof_robust_posterior}.

\subsection{Generalization certificates for robust posterior}
\label{ss:robust_cert}

First we derive the standard generalization of robust posterior. While this setting may not be of practical interest, we provide the result for completeness.

\begin{theorem}[Standard generalization of robust posterior] \label{thm:adv_post_std_loss}
Consider $c$ and $s$ as defined in \Cref{eq:cgf_std_loss} with $t=1$, $\sigma_p^2 < \frac{\sigma^2}{\sigma_x^2}$, 
$k_\delta = \frac{2n\delta^2\sigma_p^2}{\sigma^2} + 1$, 
% $Z_2= I \lp 4n\delta^2\sigma_p^2 + 1 \rp + 4\sigma_p^2 XX^\top$, and 
$U_d= k_\delta I_d + \frac{2\sigma_p^2}{\sigma^2} X^\top X$,
$U_n= k_\delta I_n + \frac{2\sigma_p^2}{\sigma^2} XX^\top$, 
% $Z_3= I \lp 4n\delta^2\sigma_p^2 + 1 \rp + 2\sigma_p^2 XX^\top$,
$V_d= k_\delta I_d + \frac{\sigma_p^2}{\sigma^2} X^\top X$, and $V_n= k_\delta I_n + \frac{\sigma_p^2}{\sigma^2} X X^\top$. Then, with probability at least $1-\beta$, we have the following certificate for standard generalization of the robust posterior ${q}_\delta(\theta)$: 
\begin{align}
\expectation{  \theta \sim {q}_\delta}{R(\theta)}  &\leq \frac{2}{n} \log \sqrt{\det(U_d)} + \frac{2}{n k_\delta \sigma^2}Y^\top U_n^{-1} Y \nonumber \\
&\quad - \frac{1}{n} \log \sqrt{\det \lp V_d \rp} -  \frac{k_\delta}{n \sigma^2} Y^\top V_n^{-1} Y \nonumber \\
&\quad + \frac{1}{n} \log \frac{1}{\beta}  +\frac{s^2}{2(1-c)}. 
\label{eq:pac_bayes_adv_post_std_loss}
\end{align}
\end{theorem}
The terms involving $\det U_d$ and $\det V_d$ are sums of $d$ $\log$ eigenvalues divided by $n$, so they scale like $1 / n$. 
As in the previous bounds, the remaining data dependent bounds resemble the product of $Y^\top$ and the average error of ridge regression with an effective regularization parameter.
%Using a different analysis, we derive a tighter bound 

For the robust posterior, we consider the cases $\deltat=\delta$ and $\deltat\neq\delta$ separately. We derive a tighter bound for the special case when the allowed adversarial perturbation at train and test-time are the same, i.e., $\deltat = \delta$. 
\begin{theorem}[Adversarial generalization of robust posterior with $\deltat=\delta$] \label{thm:adv_post_adv_loss}
Consider $c$ and $s$ as defined in \Cref{eq:cgf_adv_loss} with $t=1$, $\sigma_p^2 < \frac{\sigma^2}{2 \lp \sigma_x^2 + \deltat^2\rp}$,
$k_\delta = \frac{2n\delta^2\sigma_p^2}{\sigma^2} + 1$, 
% $Z_2= I \lp 4n\delta^2\sigma_p^2 + 1 \rp + 4\sigma_p^2 XX^\top$, and 
$U_d= k_\delta I_d + \frac{2\sigma_p^2}{\sigma^2} X^\top X$, and $U_n= k_\delta I_n + \frac{2\sigma_p^2}{\sigma^2} XX^\top$. 
% $Z= I \lp 4n\delta^2\sigma_p^2 + 1 \rp + 4\sigma_p^2 XX^\top$, 
Then, with probability at least $1-\beta$, we have the following certificate for adversarial generalization of the robust posterior $  {q}_\delta(\theta)$: 
\begin{align}
\expectation{  \theta \sim {q}_\delta}{  {R}_\delta(\theta)}  &\leq \frac{1}{n} \log \sqrt{\det \lp U_d \rp} + \frac{1}{n k_\delta \sigma^2} Y^\top U_n^{-1} Y  \nonumber \\
&\qquad + \frac{1}{n} \log \frac{1}{\beta}  +\frac{s^2}{2(1-c)}. \label{eq:pac_bayes_adv_post_adv_loss}
\end{align}
\end{theorem}
Compared with \Cref{thm:adv_post_std_loss}, \Cref{thm:adv_post_adv_loss} only includes $1$ times the $U_d$ and $U_n$ dependent terms, instead of $2$ times the $U_d$ and $U_n$ dependent terms minus the $V_d$ and $V_d$ dependent terms. 
Empirically, in \Cref{sec:exp}, we find that this leads to a favorable bound.
%Each of these can be understood as generalization bounds of standard Bayes regression using an effective regularisation parameter.




Finally, using a different analysis we derive the adversarial generalization focusing on a general setting where the adversarial  perturbation radius at test-time $\deltat$ is not the same as the radius used to learn the posterior $\delta$.
\begin{theorem}[Adversarial generalization of robust posterior] \label{thm:adv_post_adv_loss_gen}
Consider $c$ and $s$ as defined in \Cref{eq:cgf_adv_loss} with $t=1$, $\sigma_p^2 < \frac{1}{4 \lp \sigma_x^2 + \deltat^2\rp}$,
$k_\delta = \frac{2n\delta^2\sigma_p^2}{\sigma^2} + 1$, 
$U_d= k_\delta I_d + \frac{2\sigma_p^2}{\sigma^2} X^\top X$, and $U_n= k_\delta I_n + \frac{2\sigma_p^2}{\sigma^2} XX^\top$. Then, with probability at least $1-\beta$, we have the certificate for adversarial generalization of the robust posterior $  {q}_\delta(\theta)$: 
\begin{align}
&\expectation{  \theta \sim {q}_\delta}{  {R}_{\deltat}(\theta)} \leq \frac{2}{n} \log \sqrt{\det \lp U_d \rp} + \frac{2}{n k_\delta \sigma^2} Y^\top U_n^{-1} Y  \nonumber \\
&\,\, + \frac{1}{n} \log \frac{1}{\beta}  +\frac{s^2}{2(1-c)} + \frac{\lp \deltat^2 -\delta^2 \rp \sigma_p^2 d}{\sigma^2-2n\lp \deltat^2 -\delta^2 \rp \sigma_p^2}.
\label{eq:pac_bayes_adv_post_adv_loss_gen}
\end{align}
\end{theorem}

%Proofs are presented in \Cref{app:pac_bounds_proof_robust_posterior}.

%\subsection{Proof sketch}
%\label{ss:proof_sketch}
%\todo[inline]{add it at the end depending on the page constaint}
\section{Experiments}
\label{sec:experiments}
The experiments are designed to address two key research questions.
First, \textbf{RQ1} evaluates whether the average $L_2$-norm of the counterfactual perturbation vectors ($\overline{||\perturb||}$) decreases as the model overfits the data, thereby providing further empirical validation for our hypothesis.
Second, \textbf{RQ2} evaluates the ability of the proposed counterfactual regularized loss, as defined in (\ref{eq:regularized_loss2}), to mitigate overfitting when compared to existing regularization techniques.

% The experiments are designed to address three key research questions. First, \textbf{RQ1} investigates whether the mean perturbation vector norm decreases as the model overfits the data, aiming to further validate our intuition. Second, \textbf{RQ2} explores whether the mean perturbation vector norm can be effectively leveraged as a regularization term during training, offering insights into its potential role in mitigating overfitting. Finally, \textbf{RQ3} examines whether our counterfactual regularizer enables the model to achieve superior performance compared to existing regularization methods, thus highlighting its practical advantage.

\subsection{Experimental Setup}
\textbf{\textit{Datasets, Models, and Tasks.}}
The experiments are conducted on three datasets: \textit{Water Potability}~\cite{kadiwal2020waterpotability}, \textit{Phomene}~\cite{phomene}, and \textit{CIFAR-10}~\cite{krizhevsky2009learning}. For \textit{Water Potability} and \textit{Phomene}, we randomly select $80\%$ of the samples for the training set, and the remaining $20\%$ for the test set, \textit{CIFAR-10} comes already split. Furthermore, we consider the following models: Logistic Regression, Multi-Layer Perceptron (MLP) with 100 and 30 neurons on each hidden layer, and PreactResNet-18~\cite{he2016cvecvv} as a Convolutional Neural Network (CNN) architecture.
We focus on binary classification tasks and leave the extension to multiclass scenarios for future work. However, for datasets that are inherently multiclass, we transform the problem into a binary classification task by selecting two classes, aligning with our assumption.

\smallskip
\noindent\textbf{\textit{Evaluation Measures.}} To characterize the degree of overfitting, we use the test loss, as it serves as a reliable indicator of the model's generalization capability to unseen data. Additionally, we evaluate the predictive performance of each model using the test accuracy.

\smallskip
\noindent\textbf{\textit{Baselines.}} We compare CF-Reg with the following regularization techniques: L1 (``Lasso''), L2 (``Ridge''), and Dropout.

\smallskip
\noindent\textbf{\textit{Configurations.}}
For each model, we adopt specific configurations as follows.
\begin{itemize}
\item \textit{Logistic Regression:} To induce overfitting in the model, we artificially increase the dimensionality of the data beyond the number of training samples by applying a polynomial feature expansion. This approach ensures that the model has enough capacity to overfit the training data, allowing us to analyze the impact of our counterfactual regularizer. The degree of the polynomial is chosen as the smallest degree that makes the number of features greater than the number of data.
\item \textit{Neural Networks (MLP and CNN):} To take advantage of the closed-form solution for computing the optimal perturbation vector as defined in (\ref{eq:opt-delta}), we use a local linear approximation of the neural network models. Hence, given an instance $\inst_i$, we consider the (optimal) counterfactual not with respect to $\model$ but with respect to:
\begin{equation}
\label{eq:taylor}
    \model^{lin}(\inst) = \model(\inst_i) + \nabla_{\inst}\model(\inst_i)(\inst - \inst_i),
\end{equation}
where $\model^{lin}$ represents the first-order Taylor approximation of $\model$ at $\inst_i$.
Note that this step is unnecessary for Logistic Regression, as it is inherently a linear model.
\end{itemize}

\smallskip
\noindent \textbf{\textit{Implementation Details.}} We run all experiments on a machine equipped with an AMD Ryzen 9 7900 12-Core Processor and an NVIDIA GeForce RTX 4090 GPU. Our implementation is based on the PyTorch Lightning framework. We use stochastic gradient descent as the optimizer with a learning rate of $\eta = 0.001$ and no weight decay. We use a batch size of $128$. The training and test steps are conducted for $6000$ epochs on the \textit{Water Potability} and \textit{Phoneme} datasets, while for the \textit{CIFAR-10} dataset, they are performed for $200$ epochs.
Finally, the contribution $w_i^{\varepsilon}$ of each training point $\inst_i$ is uniformly set as $w_i^{\varepsilon} = 1~\forall i\in \{1,\ldots,m\}$.

The source code implementation for our experiments is available at the following GitHub repository: \url{https://anonymous.4open.science/r/COCE-80B4/README.md} 

\subsection{RQ1: Counterfactual Perturbation vs. Overfitting}
To address \textbf{RQ1}, we analyze the relationship between the test loss and the average $L_2$-norm of the counterfactual perturbation vectors ($\overline{||\perturb||}$) over training epochs.

In particular, Figure~\ref{fig:delta_loss_epochs} depicts the evolution of $\overline{||\perturb||}$ alongside the test loss for an MLP trained \textit{without} regularization on the \textit{Water Potability} dataset. 
\begin{figure}[ht]
    \centering
    \includegraphics[width=0.85\linewidth]{img/delta_loss_epochs.png}
    \caption{The average counterfactual perturbation vector $\overline{||\perturb||}$ (left $y$-axis) and the cross-entropy test loss (right $y$-axis) over training epochs ($x$-axis) for an MLP trained on the \textit{Water Potability} dataset \textit{without} regularization.}
    \label{fig:delta_loss_epochs}
\end{figure}

The plot shows a clear trend as the model starts to overfit the data (evidenced by an increase in test loss). 
Notably, $\overline{||\perturb||}$ begins to decrease, which aligns with the hypothesis that the average distance to the optimal counterfactual example gets smaller as the model's decision boundary becomes increasingly adherent to the training data.

It is worth noting that this trend is heavily influenced by the choice of the counterfactual generator model. In particular, the relationship between $\overline{||\perturb||}$ and the degree of overfitting may become even more pronounced when leveraging more accurate counterfactual generators. However, these models often come at the cost of higher computational complexity, and their exploration is left to future work.

Nonetheless, we expect that $\overline{||\perturb||}$ will eventually stabilize at a plateau, as the average $L_2$-norm of the optimal counterfactual perturbations cannot vanish to zero.

% Additionally, the choice of employing the score-based counterfactual explanation framework to generate counterfactuals was driven to promote computational efficiency.

% Future enhancements to the framework may involve adopting models capable of generating more precise counterfactuals. While such approaches may yield to performance improvements, they are likely to come at the cost of increased computational complexity.


\subsection{RQ2: Counterfactual Regularization Performance}
To answer \textbf{RQ2}, we evaluate the effectiveness of the proposed counterfactual regularization (CF-Reg) by comparing its performance against existing baselines: unregularized training loss (No-Reg), L1 regularization (L1-Reg), L2 regularization (L2-Reg), and Dropout.
Specifically, for each model and dataset combination, Table~\ref{tab:regularization_comparison} presents the mean value and standard deviation of test accuracy achieved by each method across 5 random initialization. 

The table illustrates that our regularization technique consistently delivers better results than existing methods across all evaluated scenarios, except for one case -- i.e., Logistic Regression on the \textit{Phomene} dataset. 
However, this setting exhibits an unusual pattern, as the highest model accuracy is achieved without any regularization. Even in this case, CF-Reg still surpasses other regularization baselines.

From the results above, we derive the following key insights. First, CF-Reg proves to be effective across various model types, ranging from simple linear models (Logistic Regression) to deep architectures like MLPs and CNNs, and across diverse datasets, including both tabular and image data. 
Second, CF-Reg's strong performance on the \textit{Water} dataset with Logistic Regression suggests that its benefits may be more pronounced when applied to simpler models. However, the unexpected outcome on the \textit{Phoneme} dataset calls for further investigation into this phenomenon.


\begin{table*}[h!]
    \centering
    \caption{Mean value and standard deviation of test accuracy across 5 random initializations for different model, dataset, and regularization method. The best results are highlighted in \textbf{bold}.}
    \label{tab:regularization_comparison}
    \begin{tabular}{|c|c|c|c|c|c|c|}
        \hline
        \textbf{Model} & \textbf{Dataset} & \textbf{No-Reg} & \textbf{L1-Reg} & \textbf{L2-Reg} & \textbf{Dropout} & \textbf{CF-Reg (ours)} \\ \hline
        Logistic Regression   & \textit{Water}   & $0.6595 \pm 0.0038$   & $0.6729 \pm 0.0056$   & $0.6756 \pm 0.0046$  & N/A    & $\mathbf{0.6918 \pm 0.0036}$                     \\ \hline
        MLP   & \textit{Water}   & $0.6756 \pm 0.0042$   & $0.6790 \pm 0.0058$   & $0.6790 \pm 0.0023$  & $0.6750 \pm 0.0036$    & $\mathbf{0.6802 \pm 0.0046}$                    \\ \hline
%        MLP   & \textit{Adult}   & $0.8404 \pm 0.0010$   & $\mathbf{0.8495 \pm 0.0007}$   & $0.8489 \pm 0.0014$  & $\mathbf{0.8495 \pm 0.0016}$     & $0.8449 \pm 0.0019$                    \\ \hline
        Logistic Regression   & \textit{Phomene}   & $\mathbf{0.8148 \pm 0.0020}$   & $0.8041 \pm 0.0028$   & $0.7835 \pm 0.0176$  & N/A    & $0.8098 \pm 0.0055$                     \\ \hline
        MLP   & \textit{Phomene}   & $0.8677 \pm 0.0033$   & $0.8374 \pm 0.0080$   & $0.8673 \pm 0.0045$  & $0.8672 \pm 0.0042$     & $\mathbf{0.8718 \pm 0.0040}$                    \\ \hline
        CNN   & \textit{CIFAR-10} & $0.6670 \pm 0.0233$   & $0.6229 \pm 0.0850$   & $0.7348 \pm 0.0365$   & N/A    & $\mathbf{0.7427 \pm 0.0571}$                     \\ \hline
    \end{tabular}
\end{table*}

\begin{table*}[htb!]
    \centering
    \caption{Hyperparameter configurations utilized for the generation of Table \ref{tab:regularization_comparison}. For our regularization the hyperparameters are reported as $\mathbf{\alpha/\beta}$.}
    \label{tab:performance_parameters}
    \begin{tabular}{|c|c|c|c|c|c|c|}
        \hline
        \textbf{Model} & \textbf{Dataset} & \textbf{No-Reg} & \textbf{L1-Reg} & \textbf{L2-Reg} & \textbf{Dropout} & \textbf{CF-Reg (ours)} \\ \hline
        Logistic Regression   & \textit{Water}   & N/A   & $0.0093$   & $0.6927$  & N/A    & $0.3791/1.0355$                     \\ \hline
        MLP   & \textit{Water}   & N/A   & $0.0007$   & $0.0022$  & $0.0002$    & $0.2567/1.9775$                    \\ \hline
        Logistic Regression   &
        \textit{Phomene}   & N/A   & $0.0097$   & $0.7979$  & N/A    & $0.0571/1.8516$                     \\ \hline
        MLP   & \textit{Phomene}   & N/A   & $0.0007$   & $4.24\cdot10^{-5}$  & $0.0015$    & $0.0516/2.2700$                    \\ \hline
       % MLP   & \textit{Adult}   & N/A   & $0.0018$   & $0.0018$  & $0.0601$     & $0.0764/2.2068$                    \\ \hline
        CNN   & \textit{CIFAR-10} & N/A   & $0.0050$   & $0.0864$ & N/A    & $0.3018/
        2.1502$                     \\ \hline
    \end{tabular}
\end{table*}

\begin{table*}[htb!]
    \centering
    \caption{Mean value and standard deviation of training time across 5 different runs. The reported time (in seconds) corresponds to the generation of each entry in Table \ref{tab:regularization_comparison}. Times are }
    \label{tab:times}
    \begin{tabular}{|c|c|c|c|c|c|c|}
        \hline
        \textbf{Model} & \textbf{Dataset} & \textbf{No-Reg} & \textbf{L1-Reg} & \textbf{L2-Reg} & \textbf{Dropout} & \textbf{CF-Reg (ours)} \\ \hline
        Logistic Regression   & \textit{Water}   & $222.98 \pm 1.07$   & $239.94 \pm 2.59$   & $241.60 \pm 1.88$  & N/A    & $251.50 \pm 1.93$                     \\ \hline
        MLP   & \textit{Water}   & $225.71 \pm 3.85$   & $250.13 \pm 4.44$   & $255.78 \pm 2.38$  & $237.83 \pm 3.45$    & $266.48 \pm 3.46$                    \\ \hline
        Logistic Regression   & \textit{Phomene}   & $266.39 \pm 0.82$ & $367.52 \pm 6.85$   & $361.69 \pm 4.04$  & N/A   & $310.48 \pm 0.76$                    \\ \hline
        MLP   &
        \textit{Phomene} & $335.62 \pm 1.77$   & $390.86 \pm 2.11$   & $393.96 \pm 1.95$ & $363.51 \pm 5.07$    & $403.14 \pm 1.92$                     \\ \hline
       % MLP   & \textit{Adult}   & N/A   & $0.0018$   & $0.0018$  & $0.0601$     & $0.0764/2.2068$                    \\ \hline
        CNN   & \textit{CIFAR-10} & $370.09 \pm 0.18$   & $395.71 \pm 0.55$   & $401.38 \pm 0.16$ & N/A    & $1287.8 \pm 0.26$                     \\ \hline
    \end{tabular}
\end{table*}

\subsection{Feasibility of our Method}
A crucial requirement for any regularization technique is that it should impose minimal impact on the overall training process.
In this respect, CF-Reg introduces an overhead that depends on the time required to find the optimal counterfactual example for each training instance. 
As such, the more sophisticated the counterfactual generator model probed during training the higher would be the time required. However, a more advanced counterfactual generator might provide a more effective regularization. We discuss this trade-off in more details in Section~\ref{sec:discussion}.

Table~\ref{tab:times} presents the average training time ($\pm$ standard deviation) for each model and dataset combination listed in Table~\ref{tab:regularization_comparison}.
We can observe that the higher accuracy achieved by CF-Reg using the score-based counterfactual generator comes with only minimal overhead. However, when applied to deep neural networks with many hidden layers, such as \textit{PreactResNet-18}, the forward derivative computation required for the linearization of the network introduces a more noticeable computational cost, explaining the longer training times in the table.

\subsection{Hyperparameter Sensitivity Analysis}
The proposed counterfactual regularization technique relies on two key hyperparameters: $\alpha$ and $\beta$. The former is intrinsic to the loss formulation defined in (\ref{eq:cf-train}), while the latter is closely tied to the choice of the score-based counterfactual explanation method used.

Figure~\ref{fig:test_alpha_beta} illustrates how the test accuracy of an MLP trained on the \textit{Water Potability} dataset changes for different combinations of $\alpha$ and $\beta$.

\begin{figure}[ht]
    \centering
    \includegraphics[width=0.85\linewidth]{img/test_acc_alpha_beta.png}
    \caption{The test accuracy of an MLP trained on the \textit{Water Potability} dataset, evaluated while varying the weight of our counterfactual regularizer ($\alpha$) for different values of $\beta$.}
    \label{fig:test_alpha_beta}
\end{figure}

We observe that, for a fixed $\beta$, increasing the weight of our counterfactual regularizer ($\alpha$) can slightly improve test accuracy until a sudden drop is noticed for $\alpha > 0.1$.
This behavior was expected, as the impact of our penalty, like any regularization term, can be disruptive if not properly controlled.

Moreover, this finding further demonstrates that our regularization method, CF-Reg, is inherently data-driven. Therefore, it requires specific fine-tuning based on the combination of the model and dataset at hand.
%\section{Discussion of Assumptions}\label{sec:discussion}
In this paper, we have made several assumptions for the sake of clarity and simplicity. In this section, we discuss the rationale behind these assumptions, the extent to which these assumptions hold in practice, and the consequences for our protocol when these assumptions hold.

\subsection{Assumptions on the Demand}

There are two simplifying assumptions we make about the demand. First, we assume the demand at any time is relatively small compared to the channel capacities. Second, we take the demand to be constant over time. We elaborate upon both these points below.

\paragraph{Small demands} The assumption that demands are small relative to channel capacities is made precise in \eqref{eq:large_capacity_assumption}. This assumption simplifies two major aspects of our protocol. First, it largely removes congestion from consideration. In \eqref{eq:primal_problem}, there is no constraint ensuring that total flow in both directions stays below capacity--this is always met. Consequently, there is no Lagrange multiplier for congestion and no congestion pricing; only imbalance penalties apply. In contrast, protocols in \cite{sivaraman2020high, varma2021throughput, wang2024fence} include congestion fees due to explicit congestion constraints. Second, the bound \eqref{eq:large_capacity_assumption} ensures that as long as channels remain balanced, the network can always meet demand, no matter how the demand is routed. Since channels can rebalance when necessary, they never drop transactions. This allows prices and flows to adjust as per the equations in \eqref{eq:algorithm}, which makes it easier to prove the protocol's convergence guarantees. This also preserves the key property that a channel's price remains proportional to net money flow through it.

In practice, payment channel networks are used most often for micro-payments, for which on-chain transactions are prohibitively expensive; large transactions typically take place directly on the blockchain. For example, according to \cite{river2023lightning}, the average channel capacity is roughly $0.1$ BTC ($5,000$ BTC distributed over $50,000$ channels), while the average transaction amount is less than $0.0004$ BTC ($44.7k$ satoshis). Thus, the small demand assumption is not too unrealistic. Additionally, the occasional large transaction can be treated as a sequence of smaller transactions by breaking it into packets and executing each packet serially (as done by \cite{sivaraman2020high}).
Lastly, a good path discovery process that favors large capacity channels over small capacity ones can help ensure that the bound in \eqref{eq:large_capacity_assumption} holds.

\paragraph{Constant demands} 
In this work, we assume that any transacting pair of nodes have a steady transaction demand between them (see Section \ref{sec:transaction_requests}). Making this assumption is necessary to obtain the kind of guarantees that we have presented in this paper. Unless the demand is steady, it is unreasonable to expect that the flows converge to a steady value. Weaker assumptions on the demand lead to weaker guarantees. For example, with the more general setting of stochastic, but i.i.d. demand between any two nodes, \cite{varma2021throughput} shows that the channel queue lengths are bounded in expectation. If the demand can be arbitrary, then it is very hard to get any meaningful performance guarantees; \cite{wang2024fence} shows that even for a single bidirectional channel, the competitive ratio is infinite. Indeed, because a PCN is a decentralized system and decisions must be made based on local information alone, it is difficult for the network to find the optimal detailed balance flow at every time step with a time-varying demand.  With a steady demand, the network can discover the optimal flows in a reasonably short time, as our work shows.

We view the constant demand assumption as an approximation for a more general demand process that could be piece-wise constant, stochastic, or both (see simulations in Figure \ref{fig:five_nodes_variable_demand}).
We believe it should be possible to merge ideas from our work and \cite{varma2021throughput} to provide guarantees in a setting with random demands with arbitrary means. We leave this for future work. In addition, our work suggests that a reasonable method of handling stochastic demands is to queue the transaction requests \textit{at the source node} itself. This queuing action should be viewed in conjunction with flow-control. Indeed, a temporarily high unidirectional demand would raise prices for the sender, incentivizing the sender to stop sending the transactions. If the sender queues the transactions, they can send them later when prices drop. This form of queuing does not require any overhaul of the basic PCN infrastructure and is therefore simpler to implement than per-channel queues as suggested by \cite{sivaraman2020high} and \cite{varma2021throughput}.

\subsection{The Incentive of Channels}
The actions of the channels as prescribed by the DEBT control protocol can be summarized as follows. Channels adjust their prices in proportion to the net flow through them. They rebalance themselves whenever necessary and execute any transaction request that has been made of them. We discuss both these aspects below.

\paragraph{On Prices}
In this work, the exclusive role of channel prices is to ensure that the flows through each channel remains balanced. In practice, it would be important to include other components in a channel's price/fee as well: a congestion price  and an incentive price. The congestion price, as suggested by \cite{varma2021throughput}, would depend on the total flow of transactions through the channel, and would incentivize nodes to balance the load over different paths. The incentive price, which is commonly used in practice \cite{river2023lightning}, is necessary to provide channels with an incentive to serve as an intermediary for different channels. In practice, we expect both these components to be smaller than the imbalance price. Consequently, we expect the behavior of our protocol to be similar to our theoretical results even with these additional prices.

A key aspect of our protocol is that channel fees are allowed to be negative. Although the original Lightning network whitepaper \cite{poon2016bitcoin} suggests that negative channel prices may be a good solution to promote rebalancing, the idea of negative prices in not very popular in the literature. To our knowledge, the only prior work with this feature is \cite{varma2021throughput}. Indeed, in papers such as \cite{van2021merchant} and \cite{wang2024fence}, the price function is explicitly modified such that the channel price is never negative. The results of our paper show the benefits of negative prices. For one, in steady state, equal flows in both directions ensure that a channel doesn't loose any money (the other price components mentioned above ensure that the channel will only gain money). More importantly, negative prices are important to ensure that the protocol selectively stifles acyclic flows while allowing circulations to flow. Indeed, in the example of Section \ref{sec:flow_control_example}, the flows between nodes $A$ and $C$ are left on only because the large positive price over one channel is canceled by the corresponding negative price over the other channel, leading to a net zero price.

Lastly, observe that in the DEBT control protocol, the price charged by a channel does not depend on its capacity. This is a natural consequence of the price being the Lagrange multiplier for the net-zero flow constraint, which also does not depend on the channel capacity. In contrast, in many other works, the imbalance price is normalized by the channel capacity \cite{ren2018optimal, lin2020funds, wang2024fence}; this is shown to work well in practice. The rationale for such a price structure is explained well in \cite{wang2024fence}, where this fee is derived with the aim of always maintaining some balance (liquidity) at each end of every channel. This is a reasonable aim if a channel is to never rebalance itself; the experiments of the aforementioned papers are conducted in such a regime. In this work, however, we allow the channels to rebalance themselves a few times in order to settle on a detailed balance flow. This is because our focus is on the long-term steady state performance of the protocol. This difference in perspective also shows up in how the price depends on the channel imbalance. \cite{lin2020funds} and \cite{wang2024fence} advocate for strictly convex prices whereas this work and \cite{varma2021throughput} propose linear prices.

\paragraph{On Rebalancing} 
Recall that the DEBT control protocol ensures that the flows in the network converge to a detailed balance flow, which can be sustained perpetually without any rebalancing. However, during the transient phase (before convergence), channels may have to perform on-chain rebalancing a few times. Since rebalancing is an expensive operation, it is worthwhile discussing methods by which channels can reduce the extent of rebalancing. One option for the channels to reduce the extent of rebalancing is to increase their capacity; however, this comes at the cost of locking in more capital. Each channel can decide for itself the optimum amount of capital to lock in. Another option, which we discuss in Section \ref{sec:five_node}, is for channels to increase the rate $\gamma$ at which they adjust prices. 

Ultimately, whether or not it is beneficial for a channel to rebalance depends on the time-horizon under consideration. Our protocol is based on the assumption that the demand remains steady for a long period of time. If this is indeed the case, it would be worthwhile for a channel to rebalance itself as it can make up this cost through the incentive fees gained from the flow of transactions through it in steady state. If a channel chooses not to rebalance itself, however, there is a risk of being trapped in a deadlock, which is suboptimal for not only the nodes but also the channel.

\section{Conclusion}
This work presents DEBT control: a protocol for payment channel networks that uses source routing and flow control based on channel prices. The protocol is derived by posing a network utility maximization problem and analyzing its dual minimization. It is shown that under steady demands, the protocol guides the network to an optimal, sustainable point. Simulations show its robustness to demand variations. The work demonstrates that simple protocols with strong theoretical guarantees are possible for PCNs and we hope it inspires further theoretical research in this direction.
\begin{figure*}[t]
  \centering
    \includegraphics[width=1\linewidth]{visuals/final_registration.png}
    \caption{Target measurement process on low-cost scan data using ICP and Coloured ICP. (1) Initialisation: The source point cloud (checkerboard) is misaligned with the target point cloud. (2) Initial Registration using Point-to-Plane ICP: Standard ICP leads to suboptimal registration. (3) Final Registration using Coloured ICP: Colour information is incorporated after pre-processing with RANSAC and Binarisation with Otsu Thresholding for real data, resulting in improved alignment.}
    \label{fig:Registration_visualisation}
\end{figure*}

\subsection{Iterative Closest Point (ICP) Algorithm}
The Iterative Closest Point (ICP) algorithm has been a fundamental technique in 3D computer vision and robotics for point cloud. Originally proposed by \cite{besl_method_1992}, ICP aims to minimise the distance between two datasets, typically referred to as the source and the target. The algorithm operates in an iterative manner, identifying correspondences by matching each source point with its nearest target point \citep{survey_ICP}. It then computes the rigid transformation, usually involving both rotation and translation, to achieve the best alignment of these matched points \citep{survey_ICP}. This process is repeated until convergence, where the change in the alignment parameters or the overall alignment error becomes smaller than a predefined threshold.

One key advantage of the ICP framework lies in its simplicity: the algorithm is conceptually straightforward, and its basic version is relatively easy to implement. However, traditional ICP can be sensitive to local minima, often requiring a good initial alignment \citep{zhang2021fast}. Furthermore, outliers, noise, and partial overlaps between datasets can significantly degrade its performance \citep{zhang2021fast, bouaziz2013sparse}. Over the years, various modifications and improvements \citep{gelfand2005robust, rusu2009fast, aiger20084, gruen2005least, fitzgibbon2003robust} have been proposed to mitigate these issues. Among the most common strategies are robust cost functions \citep{fitzgibbon2003robust}, weighting schemes for correspondences \citep{rusu2009fast}, and more sophisticated techniques \citep{gelfand2005robust, bouaziz2013sparse} to reject outliers. 

In addition, there is significant interest in integrating additional information into the ICP pipeline. Instead of solely relying on geometric cues such as point coordinates or surface normals, recent approaches have proposed incorporating colour (RGB) or intensity data to enhance correspondence accuracy. These methods \citep{park_colored_2017, 5980407}, commonly known as "Colored ICP" employ differences in pixel intensities or colour values as additional constraints. This is particularly beneficial in situations where geometric attributes alone are inadequate for accurate alignment or where surfaces possess complex texture patterns that can assist in the matching process.

\subsection{Applications of Target Measurement}

One approach relies on the use of physical checkerboard targets for registration. \cite{fryskowska2019} analyse checkerboard target identification for terrestrial laser scanning. They propose a geometric method to determine the target centre with higher precision, demonstrating that their approach can reduce errors by up to 6 mm compared to conventional automatic methods.

\cite{becerik2011assessment} examines data acquisition errors in 3D laser scanning for construction by evaluating how different target types (paper, paddle, and sphere) and layouts impact registration accuracy in both indoor and outdoor environments and presents guidelines for optimal target configuration.

\citet{Liang2024} propose to use Coloured ICP to measure target centres for checkerboard targets, similar to our investigation. They use data from a survey-grade terrestrial laser scanner. Their intended application is structural bridge monitoring purposes. They report an average accuracy of the measurement below 1.3 millimetres.

Where targets cannot be placed in the scene, the intensity information form the scanner can still be used to identify distinctive points. For point cloud data that is captured with a regular pattern, standard image processing can be used in a similar way to target detection. For example, \citet{wendt_automation_2004} proposes to use the SUSAN operator on a co-registered image from a camera, \citet{bohm_automatic_2007} proposes to use the SIFT operator on the LIDAR reflectance directly and \citet{theiler_markerless_2013} propose to use a Difference-of-Gaussian approach on the reflectance information.
Most of these methods extract image features to find reliable 3D correspondences for the purpose of registration.

In the following we describe our approach to the measurement of the target centre. In contrast to most proposed methods above we focus on unordered point clouds, where raster-based methods are not available, and low-cost sensors, where increased measurement noise and outliers are expected. As we are not aware of a commercial reference solution to this problem, we start by conducting a series of synthetic experiments to explore the viability and accuracy potential of the approach.



%The reviewed studies primarily rely on physical targets or target-free methods and do not utilise 3D synthetic point cloud checkerboards. In contrast, our approach introduces synthetic point cloud checkerboards, which offer controlled and consistent target geometry and reduce variability caused by physical targets. This innovation has significant potential for commercialisation and industrial application.

\section{Conclusion}
In this work, we propose a simple yet effective approach, called SMILE, for graph few-shot learning with fewer tasks. Specifically, we introduce a novel dual-level mixup strategy, including within-task and across-task mixup, for enriching the diversity of nodes within each task and the diversity of tasks. Also, we incorporate the degree-based prior information to learn expressive node embeddings. Theoretically, we prove that SMILE effectively enhances the model's generalization performance. Empirically, we conduct extensive experiments on multiple benchmarks and the results suggest that SMILE significantly outperforms other baselines, including both in-domain and cross-domain few-shot settings.

\clearpage

\iffalse
\section{Back Matter}
There are a some final, special sections that come at the back of the paper, in the following order:
\begin{itemize}
  \item Author Contributions (optional)
  \item Acknowledgements (optional)
  \item References
\end{itemize}
They all use an unnumbered \verb|\subsubsection|.

For the first two special environments are provided.
(These sections are automatically removed for the anonymous submission version of your paper.)
The third is the ‘References’ section.
(See below.)

(This ‘Back Matter’ section itself should not be included in your paper.)


\begin{contributions} % will be removed in pdf for initial submission 
					  % (without ‘accepted’ option in \documentclass)
                      % so you can already fill it to test with the
                      % ‘accepted’ class option
    Briefly list author contributions. 
    This is a nice way of making clear who did what and to give proper credit.
    This section is optional.

    H.~Q.~Bovik conceived the idea and wrote the paper.
    Coauthor One created the code.
    Coauthor Two created the figures.
\end{contributions}
\fi

\begin{acknowledgements} 
This work was partially done when MS visited Data61, CSIRO, and 
the authors would like to thank Peter Caley for facilitating the visit to Australia. Additionally, this research has been supported by the TUM Georg Nemetschek Institute Artificial Intelligence for the Built World. 
\end{acknowledgements}


% References
\bibliography{citations}

\newpage

\onecolumn

\title{Generalization Certificates for Adversarially Robust Bayesian Linear Regression\\(Supplementary Material)}
%\maketitle

\section*{Generalization Certificates for Adversarially Robust Bayesian Linear Regression (Supplementary Material)}

\appendix
\subsection{Lloyd-Max Algorithm}
\label{subsec:Lloyd-Max}
For a given quantization bitwidth $B$ and an operand $\bm{X}$, the Lloyd-Max algorithm finds $2^B$ quantization levels $\{\hat{x}_i\}_{i=1}^{2^B}$ such that quantizing $\bm{X}$ by rounding each scalar in $\bm{X}$ to the nearest quantization level minimizes the quantization MSE. 

The algorithm starts with an initial guess of quantization levels and then iteratively computes quantization thresholds $\{\tau_i\}_{i=1}^{2^B-1}$ and updates quantization levels $\{\hat{x}_i\}_{i=1}^{2^B}$. Specifically, at iteration $n$, thresholds are set to the midpoints of the previous iteration's levels:
\begin{align*}
    \tau_i^{(n)}=\frac{\hat{x}_i^{(n-1)}+\hat{x}_{i+1}^{(n-1)}}2 \text{ for } i=1\ldots 2^B-1
\end{align*}
Subsequently, the quantization levels are re-computed as conditional means of the data regions defined by the new thresholds:
\begin{align*}
    \hat{x}_i^{(n)}=\mathbb{E}\left[ \bm{X} \big| \bm{X}\in [\tau_{i-1}^{(n)},\tau_i^{(n)}] \right] \text{ for } i=1\ldots 2^B
\end{align*}
where to satisfy boundary conditions we have $\tau_0=-\infty$ and $\tau_{2^B}=\infty$. The algorithm iterates the above steps until convergence.

Figure \ref{fig:lm_quant} compares the quantization levels of a $7$-bit floating point (E3M3) quantizer (left) to a $7$-bit Lloyd-Max quantizer (right) when quantizing a layer of weights from the GPT3-126M model at a per-tensor granularity. As shown, the Lloyd-Max quantizer achieves substantially lower quantization MSE. Further, Table \ref{tab:FP7_vs_LM7} shows the superior perplexity achieved by Lloyd-Max quantizers for bitwidths of $7$, $6$ and $5$. The difference between the quantizers is clear at 5 bits, where per-tensor FP quantization incurs a drastic and unacceptable increase in perplexity, while Lloyd-Max quantization incurs a much smaller increase. Nevertheless, we note that even the optimal Lloyd-Max quantizer incurs a notable ($\sim 1.5$) increase in perplexity due to the coarse granularity of quantization. 

\begin{figure}[h]
  \centering
  \includegraphics[width=0.7\linewidth]{sections/figures/LM7_FP7.pdf}
  \caption{\small Quantization levels and the corresponding quantization MSE of Floating Point (left) vs Lloyd-Max (right) Quantizers for a layer of weights in the GPT3-126M model.}
  \label{fig:lm_quant}
\end{figure}

\begin{table}[h]\scriptsize
\begin{center}
\caption{\label{tab:FP7_vs_LM7} \small Comparing perplexity (lower is better) achieved by floating point quantizers and Lloyd-Max quantizers on a GPT3-126M model for the Wikitext-103 dataset.}
\begin{tabular}{c|cc|c}
\hline
 \multirow{2}{*}{\textbf{Bitwidth}} & \multicolumn{2}{|c|}{\textbf{Floating-Point Quantizer}} & \textbf{Lloyd-Max Quantizer} \\
 & Best Format & Wikitext-103 Perplexity & Wikitext-103 Perplexity \\
\hline
7 & E3M3 & 18.32 & 18.27 \\
6 & E3M2 & 19.07 & 18.51 \\
5 & E4M0 & 43.89 & 19.71 \\
\hline
\end{tabular}
\end{center}
\end{table}

\subsection{Proof of Local Optimality of LO-BCQ}
\label{subsec:lobcq_opt_proof}
For a given block $\bm{b}_j$, the quantization MSE during LO-BCQ can be empirically evaluated as $\frac{1}{L_b}\lVert \bm{b}_j- \bm{\hat{b}}_j\rVert^2_2$ where $\bm{\hat{b}}_j$ is computed from equation (\ref{eq:clustered_quantization_definition}) as $C_{f(\bm{b}_j)}(\bm{b}_j)$. Further, for a given block cluster $\mathcal{B}_i$, we compute the quantization MSE as $\frac{1}{|\mathcal{B}_{i}|}\sum_{\bm{b} \in \mathcal{B}_{i}} \frac{1}{L_b}\lVert \bm{b}- C_i^{(n)}(\bm{b})\rVert^2_2$. Therefore, at the end of iteration $n$, we evaluate the overall quantization MSE $J^{(n)}$ for a given operand $\bm{X}$ composed of $N_c$ block clusters as:
\begin{align*}
    \label{eq:mse_iter_n}
    J^{(n)} = \frac{1}{N_c} \sum_{i=1}^{N_c} \frac{1}{|\mathcal{B}_{i}^{(n)}|}\sum_{\bm{v} \in \mathcal{B}_{i}^{(n)}} \frac{1}{L_b}\lVert \bm{b}- B_i^{(n)}(\bm{b})\rVert^2_2
\end{align*}

At the end of iteration $n$, the codebooks are updated from $\mathcal{C}^{(n-1)}$ to $\mathcal{C}^{(n)}$. However, the mapping of a given vector $\bm{b}_j$ to quantizers $\mathcal{C}^{(n)}$ remains as  $f^{(n)}(\bm{b}_j)$. At the next iteration, during the vector clustering step, $f^{(n+1)}(\bm{b}_j)$ finds new mapping of $\bm{b}_j$ to updated codebooks $\mathcal{C}^{(n)}$ such that the quantization MSE over the candidate codebooks is minimized. Therefore, we obtain the following result for $\bm{b}_j$:
\begin{align*}
\frac{1}{L_b}\lVert \bm{b}_j - C_{f^{(n+1)}(\bm{b}_j)}^{(n)}(\bm{b}_j)\rVert^2_2 \le \frac{1}{L_b}\lVert \bm{b}_j - C_{f^{(n)}(\bm{b}_j)}^{(n)}(\bm{b}_j)\rVert^2_2
\end{align*}

That is, quantizing $\bm{b}_j$ at the end of the block clustering step of iteration $n+1$ results in lower quantization MSE compared to quantizing at the end of iteration $n$. Since this is true for all $\bm{b} \in \bm{X}$, we assert the following:
\begin{equation}
\begin{split}
\label{eq:mse_ineq_1}
    \tilde{J}^{(n+1)} &= \frac{1}{N_c} \sum_{i=1}^{N_c} \frac{1}{|\mathcal{B}_{i}^{(n+1)}|}\sum_{\bm{b} \in \mathcal{B}_{i}^{(n+1)}} \frac{1}{L_b}\lVert \bm{b} - C_i^{(n)}(b)\rVert^2_2 \le J^{(n)}
\end{split}
\end{equation}
where $\tilde{J}^{(n+1)}$ is the the quantization MSE after the vector clustering step at iteration $n+1$.

Next, during the codebook update step (\ref{eq:quantizers_update}) at iteration $n+1$, the per-cluster codebooks $\mathcal{C}^{(n)}$ are updated to $\mathcal{C}^{(n+1)}$ by invoking the Lloyd-Max algorithm \citep{Lloyd}. We know that for any given value distribution, the Lloyd-Max algorithm minimizes the quantization MSE. Therefore, for a given vector cluster $\mathcal{B}_i$ we obtain the following result:

\begin{equation}
    \frac{1}{|\mathcal{B}_{i}^{(n+1)}|}\sum_{\bm{b} \in \mathcal{B}_{i}^{(n+1)}} \frac{1}{L_b}\lVert \bm{b}- C_i^{(n+1)}(\bm{b})\rVert^2_2 \le \frac{1}{|\mathcal{B}_{i}^{(n+1)}|}\sum_{\bm{b} \in \mathcal{B}_{i}^{(n+1)}} \frac{1}{L_b}\lVert \bm{b}- C_i^{(n)}(\bm{b})\rVert^2_2
\end{equation}

The above equation states that quantizing the given block cluster $\mathcal{B}_i$ after updating the associated codebook from $C_i^{(n)}$ to $C_i^{(n+1)}$ results in lower quantization MSE. Since this is true for all the block clusters, we derive the following result: 
\begin{equation}
\begin{split}
\label{eq:mse_ineq_2}
     J^{(n+1)} &= \frac{1}{N_c} \sum_{i=1}^{N_c} \frac{1}{|\mathcal{B}_{i}^{(n+1)}|}\sum_{\bm{b} \in \mathcal{B}_{i}^{(n+1)}} \frac{1}{L_b}\lVert \bm{b}- C_i^{(n+1)}(\bm{b})\rVert^2_2  \le \tilde{J}^{(n+1)}   
\end{split}
\end{equation}

Following (\ref{eq:mse_ineq_1}) and (\ref{eq:mse_ineq_2}), we find that the quantization MSE is non-increasing for each iteration, that is, $J^{(1)} \ge J^{(2)} \ge J^{(3)} \ge \ldots \ge J^{(M)}$ where $M$ is the maximum number of iterations. 
%Therefore, we can say that if the algorithm converges, then it must be that it has converged to a local minimum. 
\hfill $\blacksquare$


\begin{figure}
    \begin{center}
    \includegraphics[width=0.5\textwidth]{sections//figures/mse_vs_iter.pdf}
    \end{center}
    \caption{\small NMSE vs iterations during LO-BCQ compared to other block quantization proposals}
    \label{fig:nmse_vs_iter}
\end{figure}

Figure \ref{fig:nmse_vs_iter} shows the empirical convergence of LO-BCQ across several block lengths and number of codebooks. Also, the MSE achieved by LO-BCQ is compared to baselines such as MXFP and VSQ. As shown, LO-BCQ converges to a lower MSE than the baselines. Further, we achieve better convergence for larger number of codebooks ($N_c$) and for a smaller block length ($L_b$), both of which increase the bitwidth of BCQ (see Eq \ref{eq:bitwidth_bcq}).


\subsection{Additional Accuracy Results}
%Table \ref{tab:lobcq_config} lists the various LOBCQ configurations and their corresponding bitwidths.
\begin{table}
\setlength{\tabcolsep}{4.75pt}
\begin{center}
\caption{\label{tab:lobcq_config} Various LO-BCQ configurations and their bitwidths.}
\begin{tabular}{|c||c|c|c|c||c|c||c|} 
\hline
 & \multicolumn{4}{|c||}{$L_b=8$} & \multicolumn{2}{|c||}{$L_b=4$} & $L_b=2$ \\
 \hline
 \backslashbox{$L_A$\kern-1em}{\kern-1em$N_c$} & 2 & 4 & 8 & 16 & 2 & 4 & 2 \\
 \hline
 64 & 4.25 & 4.375 & 4.5 & 4.625 & 4.375 & 4.625 & 4.625\\
 \hline
 32 & 4.375 & 4.5 & 4.625& 4.75 & 4.5 & 4.75 & 4.75 \\
 \hline
 16 & 4.625 & 4.75& 4.875 & 5 & 4.75 & 5 & 5 \\
 \hline
\end{tabular}
\end{center}
\end{table}

%\subsection{Perplexity achieved by various LO-BCQ configurations on Wikitext-103 dataset}

\begin{table} \centering
\begin{tabular}{|c||c|c|c|c||c|c||c|} 
\hline
 $L_b \rightarrow$& \multicolumn{4}{c||}{8} & \multicolumn{2}{c||}{4} & 2\\
 \hline
 \backslashbox{$L_A$\kern-1em}{\kern-1em$N_c$} & 2 & 4 & 8 & 16 & 2 & 4 & 2  \\
 %$N_c \rightarrow$ & 2 & 4 & 8 & 16 & 2 & 4 & 2 \\
 \hline
 \hline
 \multicolumn{8}{c}{GPT3-1.3B (FP32 PPL = 9.98)} \\ 
 \hline
 \hline
 64 & 10.40 & 10.23 & 10.17 & 10.15 &  10.28 & 10.18 & 10.19 \\
 \hline
 32 & 10.25 & 10.20 & 10.15 & 10.12 &  10.23 & 10.17 & 10.17 \\
 \hline
 16 & 10.22 & 10.16 & 10.10 & 10.09 &  10.21 & 10.14 & 10.16 \\
 \hline
  \hline
 \multicolumn{8}{c}{GPT3-8B (FP32 PPL = 7.38)} \\ 
 \hline
 \hline
 64 & 7.61 & 7.52 & 7.48 &  7.47 &  7.55 &  7.49 & 7.50 \\
 \hline
 32 & 7.52 & 7.50 & 7.46 &  7.45 &  7.52 &  7.48 & 7.48  \\
 \hline
 16 & 7.51 & 7.48 & 7.44 &  7.44 &  7.51 &  7.49 & 7.47  \\
 \hline
\end{tabular}
\caption{\label{tab:ppl_gpt3_abalation} Wikitext-103 perplexity across GPT3-1.3B and 8B models.}
\end{table}

\begin{table} \centering
\begin{tabular}{|c||c|c|c|c||} 
\hline
 $L_b \rightarrow$& \multicolumn{4}{c||}{8}\\
 \hline
 \backslashbox{$L_A$\kern-1em}{\kern-1em$N_c$} & 2 & 4 & 8 & 16 \\
 %$N_c \rightarrow$ & 2 & 4 & 8 & 16 & 2 & 4 & 2 \\
 \hline
 \hline
 \multicolumn{5}{|c|}{Llama2-7B (FP32 PPL = 5.06)} \\ 
 \hline
 \hline
 64 & 5.31 & 5.26 & 5.19 & 5.18  \\
 \hline
 32 & 5.23 & 5.25 & 5.18 & 5.15  \\
 \hline
 16 & 5.23 & 5.19 & 5.16 & 5.14  \\
 \hline
 \multicolumn{5}{|c|}{Nemotron4-15B (FP32 PPL = 5.87)} \\ 
 \hline
 \hline
 64  & 6.3 & 6.20 & 6.13 & 6.08  \\
 \hline
 32  & 6.24 & 6.12 & 6.07 & 6.03  \\
 \hline
 16  & 6.12 & 6.14 & 6.04 & 6.02  \\
 \hline
 \multicolumn{5}{|c|}{Nemotron4-340B (FP32 PPL = 3.48)} \\ 
 \hline
 \hline
 64 & 3.67 & 3.62 & 3.60 & 3.59 \\
 \hline
 32 & 3.63 & 3.61 & 3.59 & 3.56 \\
 \hline
 16 & 3.61 & 3.58 & 3.57 & 3.55 \\
 \hline
\end{tabular}
\caption{\label{tab:ppl_llama7B_nemo15B} Wikitext-103 perplexity compared to FP32 baseline in Llama2-7B and Nemotron4-15B, 340B models}
\end{table}

%\subsection{Perplexity achieved by various LO-BCQ configurations on MMLU dataset}


\begin{table} \centering
\begin{tabular}{|c||c|c|c|c||c|c|c|c|} 
\hline
 $L_b \rightarrow$& \multicolumn{4}{c||}{8} & \multicolumn{4}{c||}{8}\\
 \hline
 \backslashbox{$L_A$\kern-1em}{\kern-1em$N_c$} & 2 & 4 & 8 & 16 & 2 & 4 & 8 & 16  \\
 %$N_c \rightarrow$ & 2 & 4 & 8 & 16 & 2 & 4 & 2 \\
 \hline
 \hline
 \multicolumn{5}{|c|}{Llama2-7B (FP32 Accuracy = 45.8\%)} & \multicolumn{4}{|c|}{Llama2-70B (FP32 Accuracy = 69.12\%)} \\ 
 \hline
 \hline
 64 & 43.9 & 43.4 & 43.9 & 44.9 & 68.07 & 68.27 & 68.17 & 68.75 \\
 \hline
 32 & 44.5 & 43.8 & 44.9 & 44.5 & 68.37 & 68.51 & 68.35 & 68.27  \\
 \hline
 16 & 43.9 & 42.7 & 44.9 & 45 & 68.12 & 68.77 & 68.31 & 68.59  \\
 \hline
 \hline
 \multicolumn{5}{|c|}{GPT3-22B (FP32 Accuracy = 38.75\%)} & \multicolumn{4}{|c|}{Nemotron4-15B (FP32 Accuracy = 64.3\%)} \\ 
 \hline
 \hline
 64 & 36.71 & 38.85 & 38.13 & 38.92 & 63.17 & 62.36 & 63.72 & 64.09 \\
 \hline
 32 & 37.95 & 38.69 & 39.45 & 38.34 & 64.05 & 62.30 & 63.8 & 64.33  \\
 \hline
 16 & 38.88 & 38.80 & 38.31 & 38.92 & 63.22 & 63.51 & 63.93 & 64.43  \\
 \hline
\end{tabular}
\caption{\label{tab:mmlu_abalation} Accuracy on MMLU dataset across GPT3-22B, Llama2-7B, 70B and Nemotron4-15B models.}
\end{table}


%\subsection{Perplexity achieved by various LO-BCQ configurations on LM evaluation harness}

\begin{table} \centering
\begin{tabular}{|c||c|c|c|c||c|c|c|c|} 
\hline
 $L_b \rightarrow$& \multicolumn{4}{c||}{8} & \multicolumn{4}{c||}{8}\\
 \hline
 \backslashbox{$L_A$\kern-1em}{\kern-1em$N_c$} & 2 & 4 & 8 & 16 & 2 & 4 & 8 & 16  \\
 %$N_c \rightarrow$ & 2 & 4 & 8 & 16 & 2 & 4 & 2 \\
 \hline
 \hline
 \multicolumn{5}{|c|}{Race (FP32 Accuracy = 37.51\%)} & \multicolumn{4}{|c|}{Boolq (FP32 Accuracy = 64.62\%)} \\ 
 \hline
 \hline
 64 & 36.94 & 37.13 & 36.27 & 37.13 & 63.73 & 62.26 & 63.49 & 63.36 \\
 \hline
 32 & 37.03 & 36.36 & 36.08 & 37.03 & 62.54 & 63.51 & 63.49 & 63.55  \\
 \hline
 16 & 37.03 & 37.03 & 36.46 & 37.03 & 61.1 & 63.79 & 63.58 & 63.33  \\
 \hline
 \hline
 \multicolumn{5}{|c|}{Winogrande (FP32 Accuracy = 58.01\%)} & \multicolumn{4}{|c|}{Piqa (FP32 Accuracy = 74.21\%)} \\ 
 \hline
 \hline
 64 & 58.17 & 57.22 & 57.85 & 58.33 & 73.01 & 73.07 & 73.07 & 72.80 \\
 \hline
 32 & 59.12 & 58.09 & 57.85 & 58.41 & 73.01 & 73.94 & 72.74 & 73.18  \\
 \hline
 16 & 57.93 & 58.88 & 57.93 & 58.56 & 73.94 & 72.80 & 73.01 & 73.94  \\
 \hline
\end{tabular}
\caption{\label{tab:mmlu_abalation} Accuracy on LM evaluation harness tasks on GPT3-1.3B model.}
\end{table}

\begin{table} \centering
\begin{tabular}{|c||c|c|c|c||c|c|c|c|} 
\hline
 $L_b \rightarrow$& \multicolumn{4}{c||}{8} & \multicolumn{4}{c||}{8}\\
 \hline
 \backslashbox{$L_A$\kern-1em}{\kern-1em$N_c$} & 2 & 4 & 8 & 16 & 2 & 4 & 8 & 16  \\
 %$N_c \rightarrow$ & 2 & 4 & 8 & 16 & 2 & 4 & 2 \\
 \hline
 \hline
 \multicolumn{5}{|c|}{Race (FP32 Accuracy = 41.34\%)} & \multicolumn{4}{|c|}{Boolq (FP32 Accuracy = 68.32\%)} \\ 
 \hline
 \hline
 64 & 40.48 & 40.10 & 39.43 & 39.90 & 69.20 & 68.41 & 69.45 & 68.56 \\
 \hline
 32 & 39.52 & 39.52 & 40.77 & 39.62 & 68.32 & 67.43 & 68.17 & 69.30  \\
 \hline
 16 & 39.81 & 39.71 & 39.90 & 40.38 & 68.10 & 66.33 & 69.51 & 69.42  \\
 \hline
 \hline
 \multicolumn{5}{|c|}{Winogrande (FP32 Accuracy = 67.88\%)} & \multicolumn{4}{|c|}{Piqa (FP32 Accuracy = 78.78\%)} \\ 
 \hline
 \hline
 64 & 66.85 & 66.61 & 67.72 & 67.88 & 77.31 & 77.42 & 77.75 & 77.64 \\
 \hline
 32 & 67.25 & 67.72 & 67.72 & 67.00 & 77.31 & 77.04 & 77.80 & 77.37  \\
 \hline
 16 & 68.11 & 68.90 & 67.88 & 67.48 & 77.37 & 78.13 & 78.13 & 77.69  \\
 \hline
\end{tabular}
\caption{\label{tab:mmlu_abalation} Accuracy on LM evaluation harness tasks on GPT3-8B model.}
\end{table}

\begin{table} \centering
\begin{tabular}{|c||c|c|c|c||c|c|c|c|} 
\hline
 $L_b \rightarrow$& \multicolumn{4}{c||}{8} & \multicolumn{4}{c||}{8}\\
 \hline
 \backslashbox{$L_A$\kern-1em}{\kern-1em$N_c$} & 2 & 4 & 8 & 16 & 2 & 4 & 8 & 16  \\
 %$N_c \rightarrow$ & 2 & 4 & 8 & 16 & 2 & 4 & 2 \\
 \hline
 \hline
 \multicolumn{5}{|c|}{Race (FP32 Accuracy = 40.67\%)} & \multicolumn{4}{|c|}{Boolq (FP32 Accuracy = 76.54\%)} \\ 
 \hline
 \hline
 64 & 40.48 & 40.10 & 39.43 & 39.90 & 75.41 & 75.11 & 77.09 & 75.66 \\
 \hline
 32 & 39.52 & 39.52 & 40.77 & 39.62 & 76.02 & 76.02 & 75.96 & 75.35  \\
 \hline
 16 & 39.81 & 39.71 & 39.90 & 40.38 & 75.05 & 73.82 & 75.72 & 76.09  \\
 \hline
 \hline
 \multicolumn{5}{|c|}{Winogrande (FP32 Accuracy = 70.64\%)} & \multicolumn{4}{|c|}{Piqa (FP32 Accuracy = 79.16\%)} \\ 
 \hline
 \hline
 64 & 69.14 & 70.17 & 70.17 & 70.56 & 78.24 & 79.00 & 78.62 & 78.73 \\
 \hline
 32 & 70.96 & 69.69 & 71.27 & 69.30 & 78.56 & 79.49 & 79.16 & 78.89  \\
 \hline
 16 & 71.03 & 69.53 & 69.69 & 70.40 & 78.13 & 79.16 & 79.00 & 79.00  \\
 \hline
\end{tabular}
\caption{\label{tab:mmlu_abalation} Accuracy on LM evaluation harness tasks on GPT3-22B model.}
\end{table}

\begin{table} \centering
\begin{tabular}{|c||c|c|c|c||c|c|c|c|} 
\hline
 $L_b \rightarrow$& \multicolumn{4}{c||}{8} & \multicolumn{4}{c||}{8}\\
 \hline
 \backslashbox{$L_A$\kern-1em}{\kern-1em$N_c$} & 2 & 4 & 8 & 16 & 2 & 4 & 8 & 16  \\
 %$N_c \rightarrow$ & 2 & 4 & 8 & 16 & 2 & 4 & 2 \\
 \hline
 \hline
 \multicolumn{5}{|c|}{Race (FP32 Accuracy = 44.4\%)} & \multicolumn{4}{|c|}{Boolq (FP32 Accuracy = 79.29\%)} \\ 
 \hline
 \hline
 64 & 42.49 & 42.51 & 42.58 & 43.45 & 77.58 & 77.37 & 77.43 & 78.1 \\
 \hline
 32 & 43.35 & 42.49 & 43.64 & 43.73 & 77.86 & 75.32 & 77.28 & 77.86  \\
 \hline
 16 & 44.21 & 44.21 & 43.64 & 42.97 & 78.65 & 77 & 76.94 & 77.98  \\
 \hline
 \hline
 \multicolumn{5}{|c|}{Winogrande (FP32 Accuracy = 69.38\%)} & \multicolumn{4}{|c|}{Piqa (FP32 Accuracy = 78.07\%)} \\ 
 \hline
 \hline
 64 & 68.9 & 68.43 & 69.77 & 68.19 & 77.09 & 76.82 & 77.09 & 77.86 \\
 \hline
 32 & 69.38 & 68.51 & 68.82 & 68.90 & 78.07 & 76.71 & 78.07 & 77.86  \\
 \hline
 16 & 69.53 & 67.09 & 69.38 & 68.90 & 77.37 & 77.8 & 77.91 & 77.69  \\
 \hline
\end{tabular}
\caption{\label{tab:mmlu_abalation} Accuracy on LM evaluation harness tasks on Llama2-7B model.}
\end{table}

\begin{table} \centering
\begin{tabular}{|c||c|c|c|c||c|c|c|c|} 
\hline
 $L_b \rightarrow$& \multicolumn{4}{c||}{8} & \multicolumn{4}{c||}{8}\\
 \hline
 \backslashbox{$L_A$\kern-1em}{\kern-1em$N_c$} & 2 & 4 & 8 & 16 & 2 & 4 & 8 & 16  \\
 %$N_c \rightarrow$ & 2 & 4 & 8 & 16 & 2 & 4 & 2 \\
 \hline
 \hline
 \multicolumn{5}{|c|}{Race (FP32 Accuracy = 48.8\%)} & \multicolumn{4}{|c|}{Boolq (FP32 Accuracy = 85.23\%)} \\ 
 \hline
 \hline
 64 & 49.00 & 49.00 & 49.28 & 48.71 & 82.82 & 84.28 & 84.03 & 84.25 \\
 \hline
 32 & 49.57 & 48.52 & 48.33 & 49.28 & 83.85 & 84.46 & 84.31 & 84.93  \\
 \hline
 16 & 49.85 & 49.09 & 49.28 & 48.99 & 85.11 & 84.46 & 84.61 & 83.94  \\
 \hline
 \hline
 \multicolumn{5}{|c|}{Winogrande (FP32 Accuracy = 79.95\%)} & \multicolumn{4}{|c|}{Piqa (FP32 Accuracy = 81.56\%)} \\ 
 \hline
 \hline
 64 & 78.77 & 78.45 & 78.37 & 79.16 & 81.45 & 80.69 & 81.45 & 81.5 \\
 \hline
 32 & 78.45 & 79.01 & 78.69 & 80.66 & 81.56 & 80.58 & 81.18 & 81.34  \\
 \hline
 16 & 79.95 & 79.56 & 79.79 & 79.72 & 81.28 & 81.66 & 81.28 & 80.96  \\
 \hline
\end{tabular}
\caption{\label{tab:mmlu_abalation} Accuracy on LM evaluation harness tasks on Llama2-70B model.}
\end{table}

%\section{MSE Studies}
%\textcolor{red}{TODO}


\subsection{Number Formats and Quantization Method}
\label{subsec:numFormats_quantMethod}
\subsubsection{Integer Format}
An $n$-bit signed integer (INT) is typically represented with a 2s-complement format \citep{yao2022zeroquant,xiao2023smoothquant,dai2021vsq}, where the most significant bit denotes the sign.

\subsubsection{Floating Point Format}
An $n$-bit signed floating point (FP) number $x$ comprises of a 1-bit sign ($x_{\mathrm{sign}}$), $B_m$-bit mantissa ($x_{\mathrm{mant}}$) and $B_e$-bit exponent ($x_{\mathrm{exp}}$) such that $B_m+B_e=n-1$. The associated constant exponent bias ($E_{\mathrm{bias}}$) is computed as $(2^{{B_e}-1}-1)$. We denote this format as $E_{B_e}M_{B_m}$.  

\subsubsection{Quantization Scheme}
\label{subsec:quant_method}
A quantization scheme dictates how a given unquantized tensor is converted to its quantized representation. We consider FP formats for the purpose of illustration. Given an unquantized tensor $\bm{X}$ and an FP format $E_{B_e}M_{B_m}$, we first, we compute the quantization scale factor $s_X$ that maps the maximum absolute value of $\bm{X}$ to the maximum quantization level of the $E_{B_e}M_{B_m}$ format as follows:
\begin{align}
\label{eq:sf}
    s_X = \frac{\mathrm{max}(|\bm{X}|)}{\mathrm{max}(E_{B_e}M_{B_m})}
\end{align}
In the above equation, $|\cdot|$ denotes the absolute value function.

Next, we scale $\bm{X}$ by $s_X$ and quantize it to $\hat{\bm{X}}$ by rounding it to the nearest quantization level of $E_{B_e}M_{B_m}$ as:

\begin{align}
\label{eq:tensor_quant}
    \hat{\bm{X}} = \text{round-to-nearest}\left(\frac{\bm{X}}{s_X}, E_{B_e}M_{B_m}\right)
\end{align}

We perform dynamic max-scaled quantization \citep{wu2020integer}, where the scale factor $s$ for activations is dynamically computed during runtime.

\subsection{Vector Scaled Quantization}
\begin{wrapfigure}{r}{0.35\linewidth}
  \centering
  \includegraphics[width=\linewidth]{sections/figures/vsquant.jpg}
  \caption{\small Vectorwise decomposition for per-vector scaled quantization (VSQ \citep{dai2021vsq}).}
  \label{fig:vsquant}
\end{wrapfigure}
During VSQ \citep{dai2021vsq}, the operand tensors are decomposed into 1D vectors in a hardware friendly manner as shown in Figure \ref{fig:vsquant}. Since the decomposed tensors are used as operands in matrix multiplications during inference, it is beneficial to perform this decomposition along the reduction dimension of the multiplication. The vectorwise quantization is performed similar to tensorwise quantization described in Equations \ref{eq:sf} and \ref{eq:tensor_quant}, where a scale factor $s_v$ is required for each vector $\bm{v}$ that maps the maximum absolute value of that vector to the maximum quantization level. While smaller vector lengths can lead to larger accuracy gains, the associated memory and computational overheads due to the per-vector scale factors increases. To alleviate these overheads, VSQ \citep{dai2021vsq} proposed a second level quantization of the per-vector scale factors to unsigned integers, while MX \citep{rouhani2023shared} quantizes them to integer powers of 2 (denoted as $2^{INT}$).

\subsubsection{MX Format}
The MX format proposed in \citep{rouhani2023microscaling} introduces the concept of sub-block shifting. For every two scalar elements of $b$-bits each, there is a shared exponent bit. The value of this exponent bit is determined through an empirical analysis that targets minimizing quantization MSE. We note that the FP format $E_{1}M_{b}$ is strictly better than MX from an accuracy perspective since it allocates a dedicated exponent bit to each scalar as opposed to sharing it across two scalars. Therefore, we conservatively bound the accuracy of a $b+2$-bit signed MX format with that of a $E_{1}M_{b}$ format in our comparisons. For instance, we use E1M2 format as a proxy for MX4.

\begin{figure}
    \centering
    \includegraphics[width=1\linewidth]{sections//figures/BlockFormats.pdf}
    \caption{\small Comparing LO-BCQ to MX format.}
    \label{fig:block_formats}
\end{figure}

Figure \ref{fig:block_formats} compares our $4$-bit LO-BCQ block format to MX \citep{rouhani2023microscaling}. As shown, both LO-BCQ and MX decompose a given operand tensor into block arrays and each block array into blocks. Similar to MX, we find that per-block quantization ($L_b < L_A$) leads to better accuracy due to increased flexibility. While MX achieves this through per-block $1$-bit micro-scales, we associate a dedicated codebook to each block through a per-block codebook selector. Further, MX quantizes the per-block array scale-factor to E8M0 format without per-tensor scaling. In contrast during LO-BCQ, we find that per-tensor scaling combined with quantization of per-block array scale-factor to E4M3 format results in superior inference accuracy across models. 

\section{Experimental Details}
\label{sec:app_exp}
In this section, we provide a detailed description of the evaluation process, divided into three parts: the construction and distribution details of \ourdataset~\ref{sec:app_exp_ifbench}, the evaluation dataset settings~\ref{sec:app_exp_evaluation}, and additional experimental results~\ref{sec:app_exp_more_res}.

\subsection{\ourdataset Details}
\label{sec:app_exp_ifbench}

\ourdataset is a benchmark designed to evaluate reward models for multi-constraint instruction-following. The dataset comprises $444$ carefully curated instances, each containing: an instruction with $3$ to $5$ multi-constraints, a chosen response satisfying all constraints, and a rejected response violating specific constraints. All instances were constructed using \texttt{gpt-4o-2024-11-20} version through the following systematic pipeline.

\looseness=-1
\paragraph{Instruction Construction} We sampled $500$ initial instructions from the Open Assistant~\cite{kopf2023openassistant}. To ensure clarity and simplicity, we constrained the initial instruction length to $5$ to $20$ words. Subsequently, we employed GPT-4o to generate five distinct categories of constraints for each initial instruction. It then autonomously selected $3$ to $5$ constraints and paraphrased them into $1$ to $2$ sentences. The paraphrased constraints were integrated into the initial instruction. Finally, we use GPT-4o to evaluate the final instructions and filter out those with internal contradictions, resulting in a final set of $444$ instructions.

\looseness=-1
\begin{itemize}
    \item {\bf Content Constraints: } Specify conditions governing response, including topic focus, detail depth, and content scope limitations.
    \item {\bf Style Constraints: } Control linguistic characteristics such as tone, sentiment polarity, empathetic expression, and humor.
    \item {\bf Length Constraints: } Dictate structural requirements including word counts, paragraph composition, and specific opening phrases.
    \item {\bf Keyword Constraints: } Enforce lexical constraints through keyword inclusion, prohibited terms, or character-level specifications.
    \item {\bf Format Constraints: } Define presentation standards that include specific formats such as JSON, Markdown, or Python, along with section organization and punctuation rules.
\end{itemize}


\paragraph{Response Construction} For each instruction, we generated $8$ candidate responses using GPT-4o with temperature $1.0$ to maximize diversity. The chosen response was selected as the unique candidate satisfying all constraints through automated verification. Rejected responses were systematically selected to ensure balanced distributions of unsatisfied constraint (UC) categories and counts. As shown in Figure~\ref{fig:IFbench}, instances are stratified by difficulty: simple (\#UC$\geq$3), normal (\#UC$=$2), and hard (\#UC$=$1), with detailed information of UC category distributions. Specifically, (a) shows the distribution by the number of unsatisfied constraints in the rejected responses, where the sum of all parts equals the total number of instances. (b) presents the distribution by the categories of all unsatisfied constraints, where the sum of all parts equals the total number of unsatisfied constraints.

\begin{figure}[!ht]
    \centering
    \subfigure[]{
    \includegraphics[width=0.45\linewidth]{figures/IFBench_F1.pdf} }
    \subfigure[]{
    \includegraphics[width=0.45\linewidth]{figures/IFBench_F2.pdf} }
    \caption{Proportion (\%) of data in \ourdataset based on the number of unsatisfied constraints per instance and the categories of all unsatisfied constraints. }
    \label{fig:IFbench}
\end{figure}

\begin{figure*}
    \centering
    \includegraphics[width=0.98\linewidth]{figures/gpt4o_best_of_n.pdf}
    \caption{Best-of-n results (\%) on TriviaQA, IFEval, and CELLO using the base reward model ArmoRM and \ourmethod to search. ``+Oracle'' denotes using the oracle setting of \ourmethod as mentioned in \cref{sec:exp_analysis}.}
    \label{fig:gpt4o_best_of_n}
\end{figure*}


\subsection{Evaluation Details}
\label{sec:app_exp_evaluation}

% 表3中各个数据集的evaluation setting,metric 
\paragraph{Best-of-N} For the TriviaQA, we sample $500$ instances from the validation split in \texttt{rc.nocontext} version. The model is prompted to generate direct answers, and we report the exact match accuracies. For the IFEval, we report the average accuracy across the strict prompt, strict instruction, loose prompt, and loose instruction settings. For the CELLO, we report the average score based on the official evaluation script. All three tasks are conducted under a zero-shot setting.

\paragraph{DPO Training}
For MT-Bench and CELLO, we employ FastChat\footnote{\url{https://github.com/lm-sys/FastChat/tree/main/fastchat/llm_judge}} and the official evaluation script respectively, to conduct the evaluations and report the average scores.
For the other tasks, we use the \texttt{lm-evaluation-harness}\footnote{\url{https://github.com/EleutherAI/lm-evaluation-harness}} for evaluation. Specifically, we adopt a 5-shot setting for the MMLU and MMLU-Pro tasks, while using a zero-shot setting for TriviaQA and TruthfulQA. Notably, for TruthfulQA, we use the \texttt{truthfulqa\_gen} setting. 

\subsection{More Results on Best-of-N}
\label{sec:app_exp_more_res}
We conduct best-of-n search experiments using \texttt{gpt-4o-2024-11-20} as the policy model, with the results presented in Figure~\ref{fig:gpt4o_best_of_n}. The results demonstrate that \ourmethod significantly improves best-of-n performance compared to the base reward model ArmoRM, even when applied to a more powerful policy model than \ourmethod.




\begin{table*}
    \centering
    \small
    \begin{adjustbox}{max width=1\linewidth}
    {
    \begin{tabular}{p{\linewidth}}
    \toprule
    % \textbf{Prompt For Router} \\
    % \midrule
   Given the following instruction, determine whether the following check in needed. \\
    \\
        \text{[Instruction]} \\
        \{instruction\} \\
    \\
        \text{[Checks]} \\
       \{ 
            ``name'': ``constraint check'', 
            ``desp'': ``A `constraint check' is required if the instruction contains any additional constraints or requirements on the output, such as length, keywords, format, number of sections, frequency, order, etc.'', 
            ``identifier'': ``[[A]]'' 
        \}, 
        \{  
            ``name'': ``factuality check'', 
            ``desp'': ``A `factuality check' is required if the generated response to the instruction potentially contains claims about factual information or world knowledge.'', 
            ``identifier'': ``[[B]]'' 
        \} \\
        \\
        If the instruction requires some checks, please output the corresponding identifiers (such as [[A]], [[B]]). \\
        Please do not output other identifiers if the corresponding checkers not needed. \\
    \bottomrule
    \end{tabular}
    }
    \end{adjustbox}
    \caption{Our prompt for the router, where the \{instruction\} part varies based on the input. }
    \label{tab:planner}
\end{table*}

\begin{table*}
    \centering
    \small
    \begin{adjustbox}{max width=1\linewidth}
    {
    \begin{tabular}{p{\linewidth}}
    \toprule
    % \textbf{Prompt For Difference Proposal} \\
    % \midrule
    \textbf{Prompt For Difference Proposal} \\
        \text{[Answers]} \\
        \{formatted\_answers\} \\
        \\
        \text{[Your Task]} \\
        Given the above responses, please identify and summarize one key points of contradiction or inconsistency between the claims. \\
        \\
        \text{[Requirements]} \\
        1. Return a Python list containing only the most significant differences between the two answers. \\
        2. Do not include any additional explanations, only output the list. \\
        3. If there are no inconsistencies, return an empty list. \\
    \midrule
    \textbf{Prompt For Query Generation} \\
    \text{[Original question that caused the inconsistency]} \\
        \{instruction\} \\
\\
        \text{[Inconsistencies]} \\
        \{inconsistencies\} \\
        \\
        \text{[Your Task]} \\
        To resolve the inconsistencies, We need to query search engine. For each contradiction, please generate a corresponding query that can be used to retrieve knowledge to resolve the contradiction.  \\
        \\
        \text{[Requirements]} \\
        1. Each query should be specific and targeted, aiming to verify or disprove the conflicting points.  \\
        2. Provide the queries in a clear and concise manner, returning a Python list of queries corrresponding to the inconsistencies. \\
        3. Do not provide any additional explanations, only output the list. \\
        \midrule
    \textbf{Prompt For Verification} \\
    Evaluate which of the two answers is more factual based on the supporting information. \\
        \text{[Support knowledge sources]}: \\
        \{supports\} \\
        \\
        \text{[Original Answers]}: \\
        \{formatted\_answers\} \\
        \\
        \text{[Remeber]} \\
        For each answer, provide a score between 1 and 10, where 10 represents the highest factual accuracy. Your output should only consist of the following: \\
        Answer A: [[score]] (Wrap the score of A with [[ and ]]) \\
        Answer B: <<score>> (Wrap the score of B with << and >>) \\
        Please also provide a compact explanation. \\
    \bottomrule
    \end{tabular}
    }
    \end{adjustbox}
    \caption{Our prompt for assessing factuality in verification agents, with the \{formatted\_answers\}, \{supports\}, \{inconsistencies\}, \{instruction\} and \{supports\} parts varying based on the input. }
    \label{tab:factuality_agent}
\end{table*}


\begin{table*}
    \centering
    \small
    \begin{adjustbox}{max width=1\linewidth}
    {
    \begin{tabular}{p{\linewidth}}
    \toprule
    % \textbf{Prompt For Difference Proposal} \\
    % \midrule
    \textbf{Prompt For Constraint Parsing} \\
       You are an expert in natural language processing and constraint checking. Your task is to analyze a given instruction and identify which constraints need to be checked. \\
        \\
        The `instruction' contains a specific task query along with several explicitly stated constraints. Based on the instructions, you need to return a list of checker names that should be applied to the constraints. \\
        \\
        Task Example: \\  
        Instruction: Write a 300+ word summary of the Wikipedia page ``https://en.wikipedia.org/wiki/Raymond\_III,\_Count\_of\_Tripol''. Do not use any commas and highlight at least 3 sections that have titles in markdown format, for example, *highlighted section part 1*, *highlighted section part 2*, *highlighted section part 3*.\\
        Response: \\
        NumberOfWordsChecker: 300+ word \\
        HighlightSectionChecker: highlight at least 3 sections that have titles in markdown format\\
        ForbiddenWordsChecker: Do not use any commas \\
        \\
        Task Instruction: \\
        \{instruction\} \\
        \\
        \#\#\# Your task: \\
        - Generate the appropriate checker names with corresponding descriptions from the original instruction description. \\
        - Return the checker names with their descriptions separated by `\textbackslash n'  \\
        - Focus only on the constraints explicitly mentioned in the instruction (e.g., length, format, specific exclusions).  \\
        - Do **not** generate checkers for the task query itself or its quality. \\
        - Do **not** infer or output constraints that are implicitly included in the instruction (e.g., general style or unstated rules). \\
        - Each checker should be responsible for checking only one constraint. \\
    \midrule
    \textbf{Prompt For Code Generation} \\
    You are tasked with implementing a Python function `check\_following' that determines whether a given `response' satisfies a constraint defined by a checker. The function should return `True' if the constraint is satisfied, and `False' otherwise. \\
\\
        \text{[Instruction to check]}: \\
        \{instruction\} \\
\\
        \text{[Specific Checker and Description]}: \\
        \{checker\_name\} \\
\\
        Requirements: \\
        - The function accepts only one parameter: `response' which is a Python string. \\
        - The function must return a boolean value (`True' or `False') based on whether the `response' adheres to the constraint described by the checker. \\
        - The function must not include any I/O operations, such as `input()' or `ArgumentParser'. \\
        - The Python code for each checker should be designed to be generalizable, e.g., using regular expressions or other suitable techniques. \\
        - Only return the exact Python code, with no additional explanations. \\
    \bottomrule
    \end{tabular}
    }
    \end{adjustbox}
    \caption{Our prompt for assessing instruction-following in verification agents, with the \{instruction\} and \{checker\_name\} parts varying based on the input. }
    \label{tab:if_agent}
\end{table*}

\end{document}
