\section*{Appendix}
\appendix

We include several appendices with derivations, additional details and results. In Appendix \ref{appa}, we provide derivations of propositions in Section \ref{section3} and \ref{section4}, equivalence between \textit{posterior sampling} and Euler-Maruyama discretization, and velocity prediction, respectively. In Appendix \ref{appb}, we compare the notations used in this paper and MRSDE \citep{luo2023mrsde}. In Appendix \ref{appc}, we list detailed algorithms of \ourmethod~with various orders and parameterizations. In Appendix \ref{appd}, we present details about datasets, settings and results in experiments. In Appendix \ref{appe}, we provide an in-depth discussion on determining the optimal NFE.

\section{Derivation Details}\label{appa}

\subsection{Proofs of Propositions}\label{appa1}

\textbf{Proposition 1.} Given an initial value $\boldsymbol{x}_s$ at time $s\in[0,T]$, the solution $\boldsymbol{x}_t$ at time $t\in[0,s]$ of Eq.(\ref{10}) is
\begin{equation}
    \boldsymbol{x}_t=\frac{\alpha_t}{\alpha_s}\boldsymbol{x}_s+(1-\frac{\alpha_t}{\alpha_s})\boldsymbol{\mu}+\alpha_t\int_s^tg^2(\tau)\frac{\boldsymbol{\epsilon}_\theta(\boldsymbol{x}_\tau,\tau)}{\alpha_{\tau}\sigma_{\tau}}\mathrm{d}\tau
    +\sqrt{-\int_s^t\frac{\alpha_t^2}{\alpha_\tau^2}g^2(\tau)\mathrm{d}\tau}\boldsymbol{z},
    \label{prop1}
\end{equation}
where $\alpha_t:=e^{-\int_0^tf(\tau)\mathrm{d}\tau}$ and $\boldsymbol{z}\sim \mathcal{N}(\boldsymbol{0},\boldsymbol{I})$.

\textit{Proof}. For SDEs in the form of Eq.(\ref{1}), Itô's formula gives the following conclusion:
\begin{equation}
    \mathrm{d}\psi(\boldsymbol{x},t)=\frac{\partial\psi(\boldsymbol{x},t)}{\partial t}\mathrm{d}t + \frac{\partial\psi(\boldsymbol{x},t)}{\partial \boldsymbol{x}}[f(\boldsymbol{x},t) \mathrm{d}t + g(t) \mathrm{d}w] + \frac12\frac{\partial^2\psi(\boldsymbol{x},t)}{\partial \boldsymbol{x}^2}g^2(t)\mathrm{d}t,
    \label{a1-1}
\end{equation}
where $\psi(\boldsymbol{x},t)$ is a differentiable function. And we define 
\begin{equation*}
    \psi(\boldsymbol{x},t)=\boldsymbol{x}e^{\int_0^tf(\tau)\mathrm{d}\tau}
\end{equation*}
By substituting $f(\boldsymbol{x},t)$ and $g(t)$ with the corresponding drift and diffusion coefficients in Eq.(\ref{10}), we obtain 
\begin{equation*}
\mathrm{d}\psi(\boldsymbol{x},t)=\boldsymbol{\mu}f(t)e^{\int_0^tf(\tau)\mathrm{d}\tau}\mathrm{d}t+e^{\int_0^tf(\tau)\mathrm{d}\tau}\left[ \frac{g^2(t)}{\sigma_t}\boldsymbol{\epsilon}_\theta(\boldsymbol{x}_t,t)\mathrm{d}t+g(t)\mathrm{d}\bar{\boldsymbol{w}}\right].
\end{equation*}
And we integrate both sides of the above equation from $s$ to $t$:
\begin{equation*}
\psi(\boldsymbol{x},t)-\psi(\boldsymbol{x},s)=\boldsymbol{\mu}(e^{\int_0^tf(\tau)\mathrm{d}\tau}-e^{\int_0^sf(\tau)\mathrm{d}\tau})
+\int_s^te^{\int_0^{\tau}f(\xi)\mathrm{d}\xi}g^2(\tau)\frac{\boldsymbol{\epsilon}_\theta(\boldsymbol{x}_\tau,\tau)}{\sigma_{\tau}}\mathrm{d}\tau
+\int_s^te^{\int_0^{\tau}f(\xi)\mathrm{d}\xi}g(\tau)\mathrm{d}\bar{\boldsymbol{w}}.
\end{equation*}
Note that $\bar{\boldsymbol{w}}$ is a standard Wiener process running backwards in time and we have the quadratic variation $(\mathrm{d}\bar{\boldsymbol{w}})^2=-\mathrm{d}\tau$. According to the definition of $\psi(\boldsymbol{x},t)$ and $\alpha_t$, we have
\begin{equation*}
\frac{\boldsymbol{x}_t}{\alpha_t}-\frac{\boldsymbol{x}_s}{\alpha_s}=\boldsymbol{\mu}\left(\frac1\alpha_t-\frac1\alpha_s\right)+\int_s^tg^2(\tau)\frac{\boldsymbol{\epsilon}_\theta(\boldsymbol{x}_\tau,\tau)}{\alpha_{\tau}\sigma_{\tau}}\mathrm{d}\tau
+\sqrt{-\int_s^t\frac{g^2(\tau)}{\alpha_{\tau}^2}\mathrm{d}\tau}\boldsymbol{z},
\end{equation*}
which is equivalent to Eq.(\ref{prop1}).

\textbf{Proposition 2.} Given an initial value $\boldsymbol{x}_s$ at time $s\in[0,T]$, the solution $\boldsymbol{x}_t$ at time $t\in[0,s]$ of Eq.(\ref{16}) is 
\begin{equation}
\boldsymbol{x}_t=\frac{\alpha_t}{\alpha_s}\boldsymbol{x}_s+(1-\frac{\alpha_t}{\alpha_s})\boldsymbol{\mu}+\alpha_t\int_s^t
\frac{g^2(\tau)}{2\alpha_\tau\sigma_\tau}\boldsymbol{\epsilon}_\theta(\boldsymbol{x}_\tau,\tau)\mathrm{d}\tau, \label{prop2}
\end{equation}
where $\alpha_t:=e^{-\int_0^tf(\tau)\mathrm{d}\tau}$.

\textit{Proof}. For ODEs which have a semi-linear structure as follows:
\begin{equation}
\frac{\mathrm{d}\boldsymbol{x}}{\mathrm{d}t}=P(t)\boldsymbol{x}+Q(\boldsymbol{x},t), \label{a1-2}
\end{equation}
the method of "variation of constants" gives the following solution:
\begin{equation*}
\boldsymbol{x}(t)=e^{\int_0^tP(\tau)\mathrm{d}\tau}\cdot \left[\int_0^tQ(\boldsymbol{x},\tau)e^{-\int_0^\tau P(r)\mathrm{d}r}\mathrm{d}\tau+C \right]. 
\end{equation*}
By simultaneously considering the following two equations
\begin{equation*}
\begin{cases}
    \boldsymbol{x}(t)=e^{\int_0^tP(\tau)\mathrm{d}\tau}\cdot \left[\int_0^tQ(\boldsymbol{x},\tau)e^{-\int_0^\tau P(r)\mathrm{d}r}\mathrm{d}\tau+C \right],\\
    \boldsymbol{x}(s)=e^{\int_0^sP(\tau)\mathrm{d}\tau}\cdot \left[\int_0^sQ(\boldsymbol{x},\tau)e^{-\int_0^\tau P(r)\mathrm{d}r}\mathrm{d}\tau+C \right],
\end{cases}
\end{equation*}
and eliminating $C$, we obtain
\begin{equation}
\boldsymbol{x}(t)=\boldsymbol{x}(s)e^{\int_s^tP(\tau)\mathrm{d}\tau}+ \int_s^tQ(\boldsymbol{x},\tau)e^{\int_\tau^t P(\xi)\mathrm{d}\xi}\mathrm{d}\tau.
\label{a1-3}
\end{equation}
Now we compare Eq.(\ref{16}) with Eq.(\ref{a1-2}) and let 
\begin{align*}
    P(t)&=-f(t)\\
   \text{and\;} Q(\boldsymbol{x},t)&=f(t)\boldsymbol{\mu}+\frac{g^2(t)}{2\sigma_t}\boldsymbol{\epsilon}_\theta(\boldsymbol{x}_t,t).
\end{align*}
Therefore, we can rewrite Eq.(\ref{a1-3}) as
\begin{align*}
\boldsymbol{x}_t&=\boldsymbol{x}_se^{-\int_s^tf(\tau)\mathrm{d}\tau}+ \int_s^te^{-\int_\tau^tf(\xi)\mathrm{d}\xi}
\left[f(\tau)\boldsymbol{\mu}+\frac{g^2(\tau)}{2\sigma_\tau}\boldsymbol{\epsilon}_\theta(\boldsymbol{x}_\tau,\tau)\right]\mathrm{d}\tau\\
&=\boldsymbol{x}_se^{-\int_s^tf(\tau)\mathrm{d}\tau}+\boldsymbol{\mu}(1-e^{-\int_s^tf(\tau)\mathrm{d}\tau})+\int_s^te^{-\int_\tau^tf(\xi)\mathrm{d}\xi}
\frac{g^2(\tau)}{2\sigma_\tau}\boldsymbol{\epsilon}_\theta(\boldsymbol{x}_\tau,\tau)\mathrm{d}\tau,
\end{align*}
which is equivalent to Eq.(\ref{prop2}).

\textbf{Proposition 3.} Given an initial value $\boldsymbol{x}_s$ at time $s\in[0,T]$, the solution $\boldsymbol{x}_t$ at time $t\in[0,s]$ of Eq.(\ref{24}) is 
\begin{equation}
\begin{aligned}
\boldsymbol{x}_t=\frac{\sigma_t}{\sigma_s}e^{-(\lambda_t-\lambda_s)}\boldsymbol{x}_s
+\boldsymbol\mu\left(1-\frac{\alpha_t}{\alpha_s}e^{-2(\lambda_t-\lambda_s)}-\alpha_t+\alpha_t e^{-2(\lambda_t-\lambda_s)}\right)\\
+2\alpha_t\int_{\lambda_s}^{\lambda_t}e^{-2(\lambda_t-\lambda)}\boldsymbol{x}_\theta(\boldsymbol{x}_\lambda,\lambda)\mathrm{d}\lambda
+\sigma_t\sqrt{1-e^{-2(\lambda_t-\lambda_s)}}\boldsymbol{z},
\label{prop3}
\end{aligned}
\end{equation}
where $\boldsymbol{z}\sim\mathcal{N}(\mathbf{0},\boldsymbol{I})$.

\textit{Proof}. According to Eq.(\ref{a1-1}), we define
\begin{align*}
u(t)=\frac{g^2(t)}{\sigma_t^2}-f(t)\\
\text{and}\quad\psi(\boldsymbol{x},t)=\boldsymbol{x}e^{\int_0^tu(\tau)\mathrm{d}\tau}.
\end{align*}
We substitute $f(\boldsymbol{x},t)$ and $g(t)$ in Eq.(\ref{a1-1}) with the corresponding drift and diffusion coefficients in Eq.(\ref{24}), and integrate both sides of the equation from $s$ to $t$:
\begin{equation}
\begin{aligned}
\boldsymbol{x}_t=\boldsymbol{x}_se^{\int_s^tu(\tau)\mathrm{d}\tau}
+\boldsymbol\mu\int_s^te^{\int_\tau^tu(\xi)\mathrm{d}\xi}\left[f(\tau)-\frac{g^2(\tau)}{\sigma^2_\tau}(1-\alpha_\tau)\right]\mathrm{d}\tau\\
-\int_s^te^{\int_\tau^tu(\xi)\mathrm{d}\xi}\left[\frac{g^2(\tau)}{\sigma^2_\tau}\alpha_\tau\boldsymbol{x}_\theta(\boldsymbol{x}_\tau,\tau)\right]\mathrm{d}\tau
+\int_s^te^{\int_\tau^tu(\xi)\mathrm{d}\xi}g(\tau)\mathrm{d}\bar{\boldsymbol{w}}.
\label{a1-4}
\end{aligned}
\end{equation}
We can rewrite $g(\tau)$ as Eq.(\ref{12}) and obtain
\begin{equation}
e^{\int_s^tu(\tau)\mathrm{d}\tau}=\exp{\int_s^t\left(-2\frac{\mathrm{d}\lambda_\tau}{\mathrm{d}\tau}-f(\tau)\right)\mathrm{d}\tau}
=\frac{\alpha_t}{\alpha_s}e^{-2(\lambda_t-\lambda_s)}
=\frac{\sigma_t}{\sigma_s}e^{-(\lambda_t-\lambda_s)}.
\label{a1-5}
\end{equation}
Next, we consider each term in Eq.(\ref{a1-4}) by employing Eq.(\ref{12}) and Eq.(\ref{a1-5}). Firstly, we simplify the second term:
\begin{align}
&\boldsymbol\mu\int_s^te^{\int_\tau^tu(\xi)\mathrm{d}\xi}\left[f(\tau)-\frac{g^2(\tau)}{\sigma^2_\tau}(1-\alpha_\tau)\right]\mathrm{d}\tau \nonumber\\
&=\boldsymbol\mu\int_s^t\frac{\sigma_t}{\sigma_\tau}e^{-(\lambda_t-\lambda_\tau)}\left[f(\tau)+2(1-\alpha_\tau)\frac{\mathrm{d}\lambda_\tau}{\mathrm{d}\tau}\right]\mathrm{d}\tau \nonumber\\
&=\boldsymbol\mu\sigma_te^{-\lambda_t}\int_s^t\frac{e^{\lambda_\tau}}{\sigma_\tau}\left[f(\tau)\mathrm{d}\tau+2(1-\alpha_\tau)\mathrm{d}\lambda_\tau\right] \nonumber\\
&=\boldsymbol\mu\sigma_te^{-\lambda_t}\int_s^t\frac{\alpha_\tau}{\sigma^2_\tau}\left[f(\tau)\mathrm{d}\tau+2\mathrm{d}\lambda_\tau-2\alpha_\tau\mathrm{d}\lambda_\tau\right] \nonumber\\
&=\boldsymbol\mu\sigma_te^{-\lambda_t}\int_s^t\frac{-\mathrm{d}\alpha_\tau}{\sigma^2_\tau}+\frac{2\alpha_\tau}{\sigma^2_\tau}\mathrm{d}\lambda_\tau-2\frac{\alpha^2_\tau}{\sigma^2_\tau}\mathrm{d}\lambda_\tau.
\label{a1-6}
\end{align}
Note that
\begin{equation}
\mathrm{d}\lambda_t=\mathrm{d}\left(\log{\frac{\alpha_t}{\sigma_\infty\sqrt{1-\alpha_t^2}}}\right)
=\frac{\mathrm{d}\alpha_t}{\alpha_t}+\frac{\alpha_t\mathrm{d}\alpha_t}{1-\alpha_t^2}=\frac{\mathrm{d}\alpha_t}{\alpha_t(1-\alpha_t^2)}.
\label{a1-7}
\end{equation}
Substitute Eq.(\ref{a1-7}) into Eq.(\ref{a1-6}) and we obtain
\begin{align}
&\boldsymbol\mu\int_s^te^{\int_\tau^tu(\xi)\mathrm{d}\xi}\left[f(\tau)-\frac{g^2(\tau)}{\sigma^2_\tau}(1-\alpha_\tau)\right]\mathrm{d}\tau \nonumber\\
&=\boldsymbol\mu\sigma_te^{-\lambda_t}\int_s^t\frac{-\mathrm{d}\alpha_\tau}{\sigma^2_\tau}
+\frac{2\mathrm{d}\alpha_\tau}{\sigma^2_\tau(1-\alpha^2_\tau)}
-2\frac{\alpha^2_\tau}{\sigma^2_\tau}\mathrm{d}\lambda_\tau \nonumber\\
&=\boldsymbol\mu\sigma_te^{-\lambda_t}\int_s^t\frac{1+\alpha^2_\tau}{\sigma^2_\infty(1-\alpha^2_\tau)^2}\mathrm{d}\alpha_\tau
-2e^{2\lambda_\tau}\mathrm{d}\lambda_\tau \nonumber\\
&=\boldsymbol\mu\sigma_te^{-\lambda_t}\int_s^t\frac{1}{\sigma^2_\infty}\mathrm{d}\left(\frac{\alpha_\tau}{1-\alpha^2_\tau}\right)
-2e^{2\lambda_\tau}\mathrm{d}\lambda \nonumber\\
&=\boldsymbol\mu\left(1-\frac{\alpha_t}{\alpha_s}e^{-2(\lambda_t-\lambda_s)}-\alpha_t+\alpha_te^{-2(\lambda_t-\lambda_s)}\right).
\label{a1-8}
\end{align}
Secondly, we rewrite the third term in Eq.(\ref{a1-4}) by employing Eq.(\ref{12}) and Eq.(\ref{a1-5}).
\begin{align}
-\int_s^te^{\int_\tau^tu(\xi)\mathrm{d}\xi}\left[\frac{g^2(\tau)}{\sigma^2_\tau}\alpha_\tau\boldsymbol{x}_\theta(\boldsymbol{x}_\tau,\tau)\right]\mathrm{d}\tau
&=-\int_s^t\frac{\sigma_t}{\sigma_\tau}e^{-(\lambda_t-\lambda_\tau)}\left[-2\frac{\mathrm{d}\lambda_\tau}{\mathrm{d}\tau}\alpha_\tau\boldsymbol{x}_\theta(\boldsymbol{x}_\tau,\tau)\right]\mathrm{d}\tau \nonumber\\
&=2\int_s^t \sigma_te^{2\lambda_\tau-\lambda_t}\boldsymbol{x}_\theta(\boldsymbol{x}_\tau,\lambda_\tau)\mathrm{d}\lambda_\tau \nonumber\\
&=2\alpha_t\int_{\lambda_s}^{\lambda_t}e^{-2(\lambda_t-\lambda)}\boldsymbol{x}_\theta(\boldsymbol{x}_\lambda,\lambda)\mathrm{d}\lambda.
\label{a1-9}
\end{align}
Thirdly, we consider the fourth term in Eq.(\ref{a1-4}) (note that $(\mathrm{d}\bar{\boldsymbol{w}})^2=-\mathrm{d}\tau$):
\begin{align}
\int_s^te^{\int_\tau^tu(\xi)\mathrm{d}\xi}g(\tau)\mathrm{d}\bar{\boldsymbol{w}}
&=\sqrt{-\int_s^te^{2\int_\tau^tu(\xi)\mathrm{d}\xi}g^2(\tau)\mathrm{d}\tau}\boldsymbol{z} \nonumber\\
&=\sqrt{-\int_s^t\frac{\sigma^2_t}{\sigma^2_\tau}e^{-2(\lambda_t-\lambda_\tau)}\left(-2\sigma^2_{\tau}\frac{\mathrm{d}\lambda_\tau}{\mathrm{d}\tau}\right)\mathrm{d}\tau}\boldsymbol{z} \nonumber\\
&=\sqrt{\sigma^2_t\int_s^t 2e^{2(\lambda_\tau-\lambda_t)}\mathrm{d}\lambda_\tau}\boldsymbol{z} \nonumber\\
&=\sigma_t\sqrt{1-e^{-2(\lambda_t-\lambda_s)}}\boldsymbol{z}.
\label{a1-10}
\end{align}
Lastly, we substitute Eq.(\ref{a1-5}) and Eq.(\ref{a1-8}-\ref{a1-10}) into Eq.(\ref{a1-4}) and obtain the solution as presented in Eq.(\ref{prop3}).

\textbf{Proposition 4.} Given an initial value $\boldsymbol{x}_s$ at time $s\in[0,T]$, the solution $\boldsymbol{x}_t$ at time $t\in[0,s]$ of Eq.(\ref{27}) is 
\begin{equation}
\boldsymbol{x}_t=\frac{\sigma_t}{\sigma_s}\boldsymbol{x}_s
+\boldsymbol\mu\left(1-\frac{\sigma_t}{\sigma_s}+\frac{\sigma_t}{\sigma_s}\alpha_s-\alpha_t \right)
+\sigma_t\int_{\lambda_s}^{\lambda_t}e^{\lambda}\boldsymbol{x}_\theta(\boldsymbol{x}_\lambda,\lambda)\mathrm{d}\lambda.
\label{prop4}
\end{equation}
\textit{Proof}. Note that Eq.(\ref{27}) shares the same structure as Eq.(\ref{a1-2}). Let
\begin{align*}
P(t)&=\frac{g^2(t)}{2\sigma^2_t}-f(t),\\
\text{and}\quad
Q(\boldsymbol{x},t)&=\left[f(t)-\frac{g^2(t)}{2\sigma_t^2}(1-\alpha_t)\right]\boldsymbol\mu-\frac{g^2(t)}{2\sigma_t^2}\alpha_t\boldsymbol{x}_\theta(\boldsymbol{x}_t,t).
\end{align*}
According to Eq.(\ref{12}), we first consider
\begin{align}
e^{\int_s^tP(\tau)\mathrm{d}\tau}&=\exp{\int_s^t\left[\frac{g^2(\tau)}{2\sigma^2_\tau}-f(\tau)\right]\mathrm{d}\tau}=\exp{\int_s^t-\mathrm{d}\lambda_\tau+\mathrm{d}\log{\alpha_\tau}} \nonumber\\
&=\exp{\int_s^t\mathrm{d}\log{\alpha_\tau}-\mathrm{d}\log{\frac{\alpha_\tau}{\sigma_\tau}}}=\exp{\int_s^t\mathrm{d}\log{\sigma_\tau}}=\frac{\sigma_t}{\sigma_s}.
\label{a1-11}
\end{align}
Then, we can rewrite Eq.(\ref{a1-3}) as
\begin{equation}
\boldsymbol{x}_t=\frac{\sigma_t}{\sigma_s}\boldsymbol{x}_s+
\boldsymbol{\mu}\int_s^t\frac{\sigma_t}{\sigma_\tau}\left[f(\tau)-\frac{g^2(\tau)}{2\sigma_\tau^2}(1-\alpha_\tau)\right]\mathrm{d}\tau
-\int_s^t\frac{\sigma_t}{\sigma_\tau}\frac{g^2(\tau)}{2\sigma_\tau^2}\alpha_\tau\boldsymbol{x}_\theta(\boldsymbol{x}_\tau,\tau)\mathrm{d}\tau.
\label{a1-12}
\end{equation}
Firstly, we consider the second term in Eq.(\ref{a1-12})
\begin{align}
&\boldsymbol{\mu}\int_s^t\frac{\sigma_t}{\sigma_\tau}\left[f(\tau)-\frac{g^2(\tau)}{2\sigma^2_\tau}(1-\alpha_\tau)\right]\mathrm{d}\tau \nonumber\\
&=\boldsymbol\mu\sigma_t\int_s^t\frac{1}{\sigma_\tau}\left[f(\tau)-\frac{g^2(\tau)}{2\sigma^2_\tau}+\frac{g^2(\tau)}{2\sigma^2_\tau}\alpha_\tau\right]\mathrm{d}\tau \nonumber\\
&=\boldsymbol\mu\sigma_t\left[\int_s^t\frac{1}{\sigma_\tau}\left(f(\tau)-\frac{g^2(\tau)}{2\sigma^2_\tau}\right)\mathrm{d}\tau +\int_s^t\frac{g^2(\tau)}{2\sigma^3_\tau}\alpha_\tau\mathrm{d}\tau \right] \nonumber\\
&=\boldsymbol\mu\sigma_t\left[-\int_s^t\frac{\mathrm{d}\log{\sigma_\tau}}{\sigma_\tau}
-\int_s^t\frac{\alpha_\tau}{\sigma_\tau}\mathrm{d}\lambda_\tau \right] \quad\text{(refer to Eq.(\ref{12}) and Eq.(\ref{a1-11})} \nonumber\\
&=\boldsymbol\mu\sigma_t\left[\int_s^t\mathrm{d}\left(\frac{1}{\sigma_\tau}\right)-\int_s^t\mathrm{d}e^{\lambda_\tau} \right] \nonumber\\
&=\boldsymbol{\mu}\left(1-\frac{\sigma_t}{\sigma_s}+\frac{\sigma_t}{\sigma_s}\alpha_s-\alpha_t\right).
\label{a1-13}
\end{align}
Secondly, we rewrite the third term in Eq.(\ref{a1-12})
\begin{equation}
-\int_s^t\frac{\sigma_t}{\sigma_\tau}\frac{g^2(\tau)}{2\sigma_\tau^2}\alpha_\tau\boldsymbol{x}_\theta(\boldsymbol{x}_\tau,\tau)\mathrm{d}\tau
=\sigma_t\int_s^te^{\lambda_\tau}\boldsymbol{x}_\theta(\boldsymbol{x}_\lambda,\lambda)\mathrm{d}\lambda_\tau.
\label{a1-14}
\end{equation}
By substituting Eq.(\ref{a1-13}) and Eq.(\ref{a1-14}) into Eq.(\ref{a1-12}), we can obtain the solution shown in Eq.(\ref{prop4}).

\subsection{Equivalence between Posterior Sampling and Euler-Maruyama Discretization}
\label{appa2}

The \textit{posterior sampling} \citep{luo2024posterior} algorithm utilizes the reparameterization of Gaussian distribution in Eq.(\ref{19}) and computes $\boldsymbol{x}_{i-1}$ from $\boldsymbol{x}_{i}$ iteratively as follows:
\begin{equation}
\begin{aligned}
    \boldsymbol{x}_{i-1}&=\tilde{\boldsymbol\mu}_{i}(\boldsymbol{x}_{i},\boldsymbol{x}_0)+\sqrt{\tilde{\beta}_{i}}\boldsymbol{z}_i,\\
    \tilde{\boldsymbol\mu}_{i}(\boldsymbol{x}_{i},\boldsymbol{x}_{0})&=\frac{(1-\alpha^2_{i-1})\alpha_{i}}{(1-\alpha^2_{i})\alpha_{i-1}}(\boldsymbol{x}_{i}-\boldsymbol\mu)+\frac{1-\frac{\alpha^2_{i}}{\alpha^2_{i-1}}}{1-\alpha^2_{i}}\alpha_{i-1}(\boldsymbol{x}_{0}-\boldsymbol\mu)+\boldsymbol\mu,\\
    \tilde{\beta}_{i}&=\frac{(1-\alpha^2_{i-1})(1-\frac{\alpha^2_{i}}{\alpha^2_{i-1}})}{1-\alpha^2_{i}},
    \label{a2-1}
\end{aligned}
\end{equation}
where $\boldsymbol{z}_i\sim\mathcal{N}(\boldsymbol{0},\boldsymbol{I}),$ $\alpha_{i}=e^{-\int_{0}^{i}f(\tau)\mathrm{d}\tau}$ and $\boldsymbol{x}_0=\left(\boldsymbol{x}_{i}-\boldsymbol\mu-\sigma_{i}{\boldsymbol\epsilon}_\theta(\boldsymbol{x}_{i},\boldsymbol\mu,t_i)\right)/\alpha_{i}+\boldsymbol\mu$. By substituting $\boldsymbol{x}_0$ into $\tilde{\boldsymbol\mu}_{i}$, we arrange the equation and obtain
\begin{equation}
\begin{aligned}
\tilde{\boldsymbol\mu}_{i}(\boldsymbol{x}_{i},\boldsymbol{x}_0)&=\frac{\alpha_{i-1}}{\alpha_{i}}\boldsymbol{x}_{i}+(1-\frac{\alpha_{i-1}}{\alpha_{i}})\boldsymbol{\mu}-\frac{\frac{\alpha_{i-1}}{\alpha_{i}}-\frac{\alpha_{i}}{\alpha_{i-1}}}{1-\alpha^2_{i}}\sigma_{i}\tilde{\boldsymbol\epsilon}_\theta(\boldsymbol{x}_{i},\boldsymbol\mu,t_i)\\
&=\frac{\alpha_{i-1}}{\alpha_{i}}\boldsymbol{x}_{i}+(1-\frac{\alpha_{i-1}}{\alpha_{i}})\boldsymbol{\mu}-\frac{\frac{\alpha_{i-1}}{\alpha_{i}}-\frac{\alpha_{i}}{\alpha_{i-1}}}{\sqrt{1-\alpha^2_{i}}}\sigma_\infty\tilde{\boldsymbol\epsilon}_\theta(\boldsymbol{x}_{i},\boldsymbol\mu,t_i).
\label{a2-2}
\end{aligned}
\end{equation}
We note that
\begin{equation}
\frac{\alpha_{i-1}}{\alpha_{i}}=e^{\int_{i-1}^{i}f(\tau)\mathrm{d}\tau}=1+\int_{i-1}^{i}f(\tau)\mathrm{d}\tau+o\left(\int_{i-1}^{i}f(\tau)\mathrm{d}\tau\right)
\approx1+f(t_{i})\Delta t_i,
\label{a2-3}
\end{equation}
where the high-order error term is omitted and $\Delta t_i:=t_i-t_{i-1}$. By substituting Eq.(\ref{a2-3}) into Eq.(\ref{a2-2}) and Eq.(\ref{a2-1}), we obtain
\begin{align}
\tilde{\boldsymbol\mu}_{i}(\boldsymbol{x}_{i},\boldsymbol{x}_0)&=(1+f(t_i)\Delta t_i)\boldsymbol{x}_i-f(t_i)\Delta t_i\boldsymbol\mu-\frac{2f(t_i)\Delta t_i\sigma_\infty}{\sqrt{1-\alpha^2_i}}\tilde{\boldsymbol\epsilon}_\theta(\boldsymbol{x}_i,\boldsymbol\mu,t_i),\label{a2-4}\\
\tilde{\beta}_i&=\frac{(1-\alpha^2_{i-1})(1-\frac{\alpha^2_i}{\alpha^2_{i-1}})}{1-\alpha^2_i}
\approx\frac{2f(t_i)\Delta t_i(1-\alpha^2_{i-1})}{1-\alpha^2_i}.
\label{a2-5}
\end{align}
On the other hand, the reverse-time SDE has been presented in Eq.(\ref{10}). Combining the assumption $g^2(t)/f(t)=2\sigma_\infty^2$ in Section \ref{section2.2} and the definition of $\sigma_t$ in Section \ref{section3.1}, the Euler–Maruyama descretization of this SDE is
\begin{align}
\boldsymbol{x}_{i-1}-\boldsymbol{x}_i&=-f(t_i)(\boldsymbol\mu-\boldsymbol{x}_i)\Delta t_i-\frac{g^2(t_i)}{\sigma_i}\tilde{\boldsymbol\epsilon}_\theta(\boldsymbol{x}_i,\boldsymbol\mu,t_i)\Delta t_i+g(t_i)\sqrt{\Delta t_i}\boldsymbol{z}_i, \nonumber\\
\therefore \boldsymbol{x}_{i-1}&=(1+f(t_i)\Delta t_i)\boldsymbol{x}_i-f(t_i)\Delta t_i\boldsymbol\mu-\frac{2\sigma_\infty^2f(t_i)}{\sigma_\infty\sqrt{1-\alpha^2_i}}\tilde{\boldsymbol\epsilon}_\theta(\boldsymbol{x}_i,\boldsymbol\mu,t_i)\Delta t_i+g(t_i)\sqrt{\Delta t_i}\boldsymbol{z}_i \nonumber\\
&=(1+f(t_i)\Delta t_i)\boldsymbol{x}_i-f(t_i)\Delta t_i\boldsymbol\mu-\frac{2f(t_i)\Delta t_i\sigma_\infty}{\sqrt{1-\alpha^2_i}}\tilde{\boldsymbol\epsilon}_\theta(\boldsymbol{x}_i,\boldsymbol\mu,t_i)+\sigma_\infty\sqrt{2f(t_i)\Delta t_i}\boldsymbol{z}_i \nonumber\\
&=\tilde{\boldsymbol\mu}_i(\boldsymbol{x}_i,\boldsymbol{x}_0)+\sigma_\infty\sqrt{\frac{1-\alpha^2_i}{1-\alpha^2_{i-1}}\tilde{\beta}_i}\boldsymbol{z}_i.
\end{align}
Thus, the \textit{posterior sampling} algorithm is a special Euler–Maruyama descretization of reverse-time SDE with a different coefficient of Gaussian noise.

\subsection{Derivations about velocity prediction}
\label{appa3}

Following Eq.(\ref{31}), We can define the \textit{velocity prediction} as
\begin{equation}
\boldsymbol{v}_\theta(t)=\boldsymbol{\mu}\sin\phi_t-\boldsymbol{x}_\theta(t)\sin\phi_t+\sigma_\infty\cos(\phi_t)\boldsymbol\epsilon_\theta(t). \label{a3-1}
\end{equation}
And we have the relationship between $\boldsymbol{x}_\theta(t)$ and $\boldsymbol{\epsilon}_\theta(t)$ as follows:
\begin{equation}
\boldsymbol{x}_t=\boldsymbol{x}_\theta(t)\cos{\phi_t}+\boldsymbol{\mu}(1-\cos{\phi_t})+\sigma_\infty\sin{(\phi_t)}\boldsymbol{\epsilon}_\theta(t).
\label{a3-2}
\end{equation}
In order to get $\boldsymbol{x}_\theta$ from $\boldsymbol{v}_\theta$, we rewrite Eq.(\ref{a3-1}) as
\begin{equation}
\boldsymbol{x}_\theta(t)\sin^2\phi_t=\boldsymbol{\mu}\sin^2\phi_t-\boldsymbol{v}_\theta(t)\sin\phi_t+\sigma_\infty\boldsymbol{\epsilon}_\theta(t)\sin\phi_t\cos\phi_t.
\end{equation}
Then we replace $\boldsymbol{\epsilon}_\theta(t)$ according to Eq.(\ref{a3-2})
\begin{align}
\boldsymbol{x}_\theta(t)\sin^2\phi_t&=\boldsymbol{\mu}\sin^2\phi_t-\boldsymbol{v}_\theta(t)\sin\phi_t+\left[\boldsymbol{x}_t-\boldsymbol{x}_\theta(t)\cos{\phi_t}-\boldsymbol{\mu}(1-\cos{\phi_t}) \right]\cos\phi_t \nonumber\\
&=(1-\cos\phi_t)\boldsymbol{\mu}-\boldsymbol{v}_\theta(t)\sin\phi_t+\boldsymbol{x}_t\cos\phi_t-\boldsymbol{x}_\theta(t)\cos^2\phi_t.
\end{align}
Arranging the above equation, we can obtain the transformation from $\boldsymbol{v}_\theta$ to $\boldsymbol{x}_\theta$, as shown in Eq.(\ref{32}).
Similarly, we can also rewrite Eq.(\ref{a3-1}) and replace $\boldsymbol{x}_\theta(t)$ as follows:
\begin{align}
\sigma_\infty\cos^2(\phi_t)\boldsymbol{\epsilon}_\theta(t)&=
\boldsymbol{v}_\theta(t)\cos\phi_t-\boldsymbol{\mu}\sin\phi_t\cos\phi_t+\boldsymbol{x}_\theta(t)\sin\phi_t\cos\phi_t \nonumber\\
&=\boldsymbol{v}_\theta(t)\cos\phi_t-\boldsymbol{\mu}\sin\phi_t\cos\phi_t+\sin\phi_t\left[\boldsymbol{x}_t-\boldsymbol{\mu}(1-\cos{\phi_t})-\sigma_\infty\sin{(\phi_t)}\boldsymbol{\epsilon}_\theta(t)\right] \nonumber\\
&=\boldsymbol{v}_\theta(t)\cos\phi_t-\boldsymbol{\mu}\sin\phi_t+\boldsymbol{x}_\theta(t)\sin\phi_t-\sigma_\infty\sin^2\phi_t\boldsymbol{\epsilon}_\theta(t).
\end{align}
Thus we obtain the transformation from $\boldsymbol{v}_\theta$ to $\boldsymbol{\epsilon}_\theta$, as presented in Eq.(\ref{33}).