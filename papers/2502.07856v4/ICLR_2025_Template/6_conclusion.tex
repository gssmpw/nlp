\section{Conclusion}

We have developed a the fast sampling algorithm of MR Diffusion. Compared with DPMs, MR Diffusion is different in SDE and thus not adaptable to existing training-free fast samplers. We propose \ourmethod~for acceleration of sampling of MR Diffusion. We solve the reverse-time SDE
and PF-ODE derived from MRSDE and find a semi-analytical solution. We adopt the methods of \textit{exponential integrators} to estimate the non-linear integral part. Abundant experiments demonstrate that our algorithm achieves small errors and fast convergence. Additionally, we visualize sampling trajectories and explain why the parameterization of noise prediction does not perform well in the case of small NFEs.

\textbf{Limitations and broader impact.} Despite the effectiveness of \ourmethod, our method is still inferior to distillation methods \citep{song2023consistency,luo2023lcm} within less than 5 NFEs. Additionally, our method can only accelerate sampling, but cannot improve the upper limit of sampling quality. 

\section*{Reproducibility Statement}

Our codes are based on the official code of MR Diffusion \citep{luo2023mrsde} and DPM-Solver \citep{lu2022dpmsolverplus}. And we use the checkpoints and datasets provided by MR Diffusion \citep{luo2023mrsde}. We will release them after the blind review.

\section*{Acknowledgement}

This work was supported in part by the National Natural Science Foundation of China (grant 92354307), the National Key Research and Development Program of China (grant 2024YFF0729202), the Strategic Priority Research Program of the Chinese Academy of Sciences (grant XDA0460305), and the Fundamental Research Funds for the Central Universities (grant E3E45201X2). This work was also supported by Alibaba Group through Alibaba Research Intern Program.
\clearpage