\section{Related Work}
\label{app:rw}
One of the core properties of emergent communication is that agents learn a communication protocol on their own, where coordination is needed to solve the underlying task. The study of emergent communication with neural agents started with continuous communication channels during training. \citet{sukhbaatar2016learning} propose a communication channel that shares continuous vectors where each agent receives a combination of all messages broadcasted by all other agents. \citet{jorge2016learning} proposed a recurrent version of the Lewis Game with continuous messages. While such works have good performances, they benefit from having differential communication channels, making it possible for gradients to pass through. More recent works used a formulation of the Lewis Game to study emergent communication, where messages contain discrete tokens but still allow for gradients to pass through the communication channel~\citet{havrylov2017emergence,mordatch2018emergence,guo2019emergence,chaabouni-etal-2020-compositionality,rita2022on}. As such, these methods employ a DIAL paradigm~\citep{foerster2016learning}, allowing for an end-to-end training scheme (across agents) by sampling discrete tokens using straight-through Gumbel-Softmax estimator~\citep{jang2016categorical}. From a language evolution perspective, these approaches do not fully align with the properties of human communication, which is discrete and undifferentiable.

Other approaches, similarly to our work, close the gap to human language by using discrete channels to communicate at training and execution times, meaning gradients do not flow through the channel. In this case, we have a RIAL approach~\citep{foerster2016learning} where each agent perceives others as part of the environment. In most cases, these works rely on Reinforce~\citep{williams1992simple} or on an actor-critic~\citep{konda1999actor} variation to model the Speaker and Listener, where both agents try to maximize the game's reward. Primarily,~\citet{foerster2016learning,lazaridou2017multiagent} developed discrete communication channels composed of only one symbol. The former study designed games where two agents simultaneously have the Speaker and Listener roles to classify images. Nevertheless, the architecture having independent agents performed poorly for the mentioned task, when compared to a DIAL approach. The latter work implemented a simpler version of the Lewis Game where the Listener discriminates between two images. The Speaker also has information about the images that the Listener will receive. \citet{choicompositional} further extended the latter game design creating agents that can handle messages composed of multiple discrete symbols, called the \textit{Obverter} technique. The authors accomplish this by modeling the Speaker to choose the message that maximizes the Speaker's understanding. This assumption roots the theory of mind~\citep{premack1978does}, where the Speaker assumes the Listener's mind and its own are identical. Although this work is an improvement from past works regarding the emulation of human language, there are still severe limitations. For example, the environment considered has only two effective degrees of freedom that must be modeled (shape and color of simple 3D shapes), limiting severely the input diversity.

New extensions to the Obverter focus on studying specific properties of human language, like compositionality or pragmatics. \citet{Ren2020Compositional} developed an iterated learning strategy for the Lewis Game, trying to create highly compositional languages as a consequence of developing a new language protocol by having several distinct learning phases for each agent. From a linguistics perspective, compositionality is crucial since it encourages the expression of complex concepts through simpler ones. Another approach tries to leverage a population of agents to study its effect on simplifying the emerged language~\citep{graesser2019emergent}. The authors use a simple visual task where agents communicate binary messages through a fixed number of rounds to match an image to a caption. The image depicts a simple shape with a specific color. Since each agent observes only part of the image, they must cooperate to solve the task. \citet{NEURIPS2019_b0cf188d} also consider a variation of the Lewis Game with a discrete communication channel. The main objective is to give another perspective on how to evaluate the structure of the resulting communication protocol, giving experimental evidence that compositional languages are easier to teach to new Listeners. This additional external pressure surfaces when the Speaker interacts with new Listeners during training. The proposed experiments continue in the same line of simplicity since inputs are categorical values with two attributes, messages contain only two tokens, and the number of candidates the Listener discriminates is only five objects.

As a succeeding work, \citet{chaabouni2022emergent} proposed scaling several dimensions of the Lewis Game to create a setup closer to simulating human communication. The scaled dimensions are: the number of candidates received by the Listener, the dataset of images used, and the number of learning agents. Scaling such dimensions makes the referential game more complex, promoting the generality and validity of the experimental results. Moreover, this study also suggests that compositionality is not a natural emergent factor of generalization of a language as the environment becomes more complex (e.g., using real-world images). This work can be seen as the initial starting point of our proposed work. In particular, we follow this game setup and extensively make a more challenging environment where we add an RL agent as the Listener that only has information about the game's outcome and not which candidate is the correct guess (as in the original implementation). This modification alone brings advantages to the generalization capability of the communication protocol, where the game accuracy proportionally increases with the number of candidates when testing on unseen images. Additionally, we introduce a stochastic communication channel aiming to increase environmental pressure to study a specific linguistics property, denoted conversational repair mechanisms, see Sec. 1. As a result, the agents seek to create a protocol where information is given redundantly to overcome the noise effect. This type of robust communication is a form of an implicit repair mechanism.

Closely related to our work,~\citet{nikolaus2023emergent} explore explicit conversational repair mechanisms. The method proposed also introduced noise in the communication channel. As a way to study other initiate repair mechanisms, the authors propose adding a feedback loop where information flows from the Listener to the Speaker. Although this is a positive development, several simplifications are present, which increase the dissimilarity to human languages. Not only is the message sent by the Listener a single binary token but this feedback loop is also triggered after every token message the Speaker sends. As such, this communication loop breaks the turn-taking aspect of the human dialogue, and the feedback information received by the Speaker is extremely reduced. Comparing against our work, both studies have their distinct focus since each explores a different conversational repair mechanism. Nonetheless, our experiment is more complicated since the only feedback the Speaker receives is the outcome of the game. As such, the coordination of a communication protocol becomes more challenging, where only by trial and error can the Speaker understand that giving redundant information is advantageous in such noisy conditions.

Other works in the literature also consider noise in the communication channel. \citet{tucker2021emergent,kucinski2021catalytic} apply a DIAL scheme, where the noise is sampled and added in the continuous space before applying the Gumbel-Softmax trick. As such, the problem is simplified by learning a continuous latent space robust to noise. More closely related to our work is the study presented by~\citet{ueda-washio-2021-relationship}. In such work, we are in an emergent discrete communication setting where the authors use a noisy communication channel to test the Zipf's law of abbreviation~\citep{zipf2013psycho}. As such, the goal is to investigate conditions for which common messages become shorter. In this context, the authors propose an adversarial setting to try to deceive the Listener by giving other plausible messages (different from the Speaker's message), which, as training progresses, promotes the usage of shorter messages. Our method differs from this work since we assume the communication channel can suffer external perturbations, mimicking the loss of information. In our case, we apply noise by changing the corresponding token to a pre-defined token called the \textit{unknown} token. Therefore, we are interested in evaluating the communication protocols' robustness to handle different noise levels at test time. Additionally, our referential games are more complex since we use datasets with natural images for discrimination instead of categorical inputs.