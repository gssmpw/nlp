
\tikzstyle{leaf}=[my-box,
	text=black, align=left,font=\normalsize,
	inner xsep=2pt,
	inner ysep=4pt,
	line width=0.8pt,
        minimum height=1cm,
]
\begin{figure}
    \centering
\begin{tikzpicture}[
    font=\tiny,
    align=center,
    % trim left=-3.5cm,
    % rounded corners,
    node distance=1cm and 1.5cm,
    quadrant/.style={draw, minimum width=3cm, minimum height=1cm, fill=yellow!8},
    empty/.style={draw, minimum width=3cm, minimum height=1cm},
    central/.style={circle, draw=orange!40, fill=orange!10, minimum size=2.5cm},
    axis/.style={thick}
]

% Quadrant Nodes
\node[quadrant, inner sep=0pt, anchor=west] (long-docvlm) at (-4.5, 2) {
    \begin{minipage}[t][2.5cm][t]{2.5cm} % 外框高度一致,顶部对齐
        \vspace{1.5em} % 标题与上边沿的间距
        \centering
        \textbf{Long DocVLM} % 标题
        \vspace{0.5em} % 标题与内容的间距

        % 内容部分,使用 tabular 控制间距
        \renewcommand{\arraystretch}{1.2} % 设置行间距
        
            \begin{tabular}{c}
            PDF-WuKong\\~\citep{xie2024wukong}\\
            mPLUG-DocOwl2\\~\citep{hu2024mplug}
            \end{tabular}
    \end{minipage}
};

\node[quadrant, inner sep=0pt, anchor=west] (high-resolution) at (-4.5, -1.0) {
    \begin{minipage}[t][2.5cm][t]{2.5cm} % 外框高度一致,顶部对齐
        \vspace{2em} % 标题与上边沿的间距
        \centering
        \textbf{High-Resolution} % 标题
        \textbf{ImageLLM}
        \vspace{0.5em} % 标题与内容的间距

        % 内容部分,使用 tabular 控制间距
        \renewcommand{\arraystretch}{1.2} % 设置行间距
        % \small{
            \begin{tabular}{c}
            Monkey~\citep{li2024monkey}\\
            Sphinx~\citep{lin2023sphinx}
            \end{tabular}
        % }
    \end{minipage}
};

\node[quadrant, inner sep=0pt, anchor=west] (high-resolution) at (-4.5, -4) {
    \begin{minipage}[t][2.5cm][t]{2.5cm} % 外框高度一致,顶部对齐
        \vspace{1.5em} % 标题与上边沿的间距
        \centering
        \textbf{Long CLIP} % 标题
        \vspace{0.5em} % 标题与内容的间距

        % 内容部分,使用 tabular 控制间距
        \renewcommand{\arraystretch}{1.2} % 设置行间距
        % \small{
            \begin{tabular}{c}
            LongCLIP\\~\citep{zhang2025long}\\
            VideoCLIP-XL\\~\citep{wang2024videoclip}
            \end{tabular}
        % }
    \end{minipage}
};

\node[quadrant, inner sep=0pt, anchor=north] (long-video) at (4, 3.25) {
    \begin{minipage}[t][8.5cm][t]{10cm} % 外框高度一致,顶部对齐
        \vspace{2.5em} % 标题与上边沿的间距
        \centering
        \textbf{LongVideo} % 标题
        \vspace{1.5em} % 标题与内容的间距
        
        % 绘制树状图
        \resizebox{0.9\textwidth}{!}{ % 动态调整宽度
            \begin{forest}
                forked edges,
                for tree={
                    grow=east,
                    reversed=true,
                    anchor=base west,
                    parent anchor=east,
                    child anchor=west,
                    base=center,
                    font=\normalsize,
                    rectangle,
                    draw=black, % hidden-draw,
                    rounded corners,
                    align=left,
                    text centered,
                    minimum width=4em,
                    minimum height=1cm,
                    edge+={darkgray, line width=1pt},
                    s sep=10pt,
                    inner xsep=2pt,
                    inner ysep=3pt,
                    line width=0.8pt,
                    ver/.style={rotate=90, child anchor=north, parent anchor=south, anchor=center, minimum width=18em, fill=gray!10},
                },
                where level=1{text width=20em, fill=green!10}{},
                where level=2{text width=20em, fill=orange!10}{},
                where level=3{text width=55em, fill=white}{},
                where level=4{text width=18em,}{},
                where level=5{text width=18em,}{},
                [
                    \textbf{LongVideo}, ver
                    [
                        \textbf{Input Adaptation}\\ \newline \textbf{(\S\ref{sec10_1})}
                        [
                            \textbf{Text-Only}
                            [
                                \quad {zhang2023simple}{,} LangRepo~\citep{kahatapitiya2024language}, leaf
                            ]
                        ]
                        [
                            \textbf{Image-Only}
                            [
                                \quad IG-VLM~\citep{kim2024image}{,} LongVA~\citep{zhang2406long}{,} \\ \quad IXC-2.5~\citep{zhang2024internlm}{,} FreeVA~\citep{wu2024freeva}, leaf
                            ]
                        ]
                        [
                            \textbf{Q-Former-based}
                            [
                                \quad MovieChat~\citep{song2024moviechat}{,} MA-LMM~\citep{he2024ma}{,} \\ \quad VidCompress~\citep{lan2024vidcompress}{,} TCR~\citep{korbar2025text}{,} \\ \quad Vista-LLaMA~\citep{ma2023vista}{,}  TimeChat~\citep{ren2024timechat} {,} \\ \quad LVChat~\citep{wang2024lvchat}{,}  Momentor~\citep{qianmomentor}, leaf
                            ]
                        ]
                        [
                            \textbf{Q-Former-free}
                            [
                                \quad LVNet~\citep{park2024too}{,} KeyVideoLLM~\citep{liang2024keyvideollm} \\ \quad Frame-Voyager~\citep{yu2024frame}{,} VideoStreaming~\citep{qian2024streaming}{,} \\ \quad SlowFast-LLaVA~\citep{xu2024slowfast}{,} TESTA~\citep{ren2023testa}{,} \\ \quad Videollama2~\citep{cheng2024videollama}, leaf
                            ]
                        ]
                    ]
                    [
                        \textbf{Architecture Adaptation}\\ \newline \textbf{(\S\ref{sec10_2})}
                        [
                            \textbf{Position Embedding}
                            [
                                \quad TC-LLaVA~\citep{gao2024tc}{,} V2PE~\citep{ge2024v2pe}{,} \\ \quad RoPE-Tie~\citep{kexuefm10040}{,} M-RoPE~\citep{wang2024qwen2,li2024giraffe}, leaf
                            ]
                        ]
                        [
                            \textbf{Cache Optimization}
                            [
                                \quad FastV~\citep{chen2025image}{,} PyramidDrop~\citep{xing2024pyramiddrop}\\ \quad ZipVL~\citep{he2024zipvl}{,} VL-Cache~\citep{tu2024vl}{,} \\ \quad Look-M~\citep{wan2024look}{,} ElasticCache~\citep{liu2025efficient}, leaf
                            ]
                        ]
                        [
                            \textbf{Architecture Innovation}
                            [
                                \quad S4ND~\citep{nguyen2022s4nd}{,} ViS4mer~\citep{islam2022vis4mer}{,} \\ \quad S5~\citep{wang2023s5}{,} VideoMambaSuite~\citep{chen2024video}{,} \\ \quad VideoMamba~\citep{li2025videomamba}{,} LongLLaVA~\citep{wang2024longllava}, leaf
                            ]
                        ]
                    ]
                    [
                        \textbf{Training and Evaluation}\\ \newline \textbf{(unfinished)} %\S\ref{sec10_3}
                        [
                            \textbf{Infrastructure}
                            [
                                \quad LWM~\citep{liu2024world}{,} LongVILA~\citep{xue2024longvila}{,}\\\quad RLT~\citep{choudhurydon}, leaf
                            ]
                        ]
                        [
                            \textbf{Training}
                            [
                                \quad TimeIT-125k~\citep{ren2024timechat}{,} Moment-10M~\citep{qianmomentor}{,}\\ \quad T2Vid~\citep{yin2024t2vid}{,} VISTA~\citep{ren2024vista}{,}\\ \quad Kangaroo~\citep{liu2024kangaroo}{,} Video-T3~\citep{li2024temporal}, leaf
                            ]
                        ]
                        [
                            \textbf{Evaluation}
                            [
                                \quad Egoschema~\citep{mangalam2023egoschema}{,} Video-MME~\citep{fu2024video}{,} \\ \quad MLVU~\citep{zhou2024mlvu}{,} MMBench-Video~\citep{fang2024mmbench}{,} \\ \quad LongVideoBench~\citep{wulongvideobench}{,} V-NIAH~\citep{zhang2406long} , leaf
                            ]
                        ]
                    ]
                ]
            \end{forest}
        }
    \end{minipage}
};

\end{tikzpicture}

	\caption{An overview of the long context in Multi-modal LLMs.}
    \label{fig:mllm}
\end{figure}

