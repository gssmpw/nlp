\tikzstyle{leaf}=[my-box,
	text=black, align=left,font=\normalsize,
	inner xsep=2pt,
	inner ysep=4pt,
	line width=0.8pt,
        minimum height=1cm,
]

\begin{figure}
    \centering
\begin{tikzpicture}[
    font=\tiny,
    align=center,
    % trim left=-3.5cm,
    % rounded corners,
    node distance=1cm and 1.5cm,
    quadrant/.style={draw, minimum width=3cm, minimum height=2.5cm, fill=yellow!8},
    central/.style={circle, draw=orange!40, fill=orange!10, minimum size=2.5cm},
    axis/.style={thick}
]


% Quadrant Nodes
\node[quadrant, inner sep=0pt, anchor=west] (efficient-attention-mechanisms) at (-3, 2.25) {
    \begin{minipage}[t][2.2cm][t]{6cm} % 外框高度一致,顶部对齐
        \vspace{1.5em} % 标题与上边沿的间距
        \centering
        {\small\textbf{Efficient Attention~(\S\ref{sec5_1})}} % 标题
        \vspace{0.5em} % 标题与内容的间距

        % 内容部分,使用 tabular 控制间距
        % {\tiny
        \renewcommand{\arraystretch}{1.2} % 设置行间距
        {
            \begin{tabular}{c}
                % DIFF Transformer\citep{ye2024differential} \\ 
                % DoubleSparse~\citep{yang2024post} \\ 
                MInference~\citep{jiang2024minference} \\
                RetrievalAttention~\citep{liu2024retrievalattention} \\
                MoBA~\citep{lu2025mobamixtureblockattention}{,} NSA~\citep{yuan2025native} \\
                LightningAttention Series~\citep{qin2024various,qin2024lightning}
            \end{tabular}
        }
    \end{minipage}
};

\node[quadrant, inner sep=0pt] (lstm-rwkv) at (6.5, 2.25) {
    \begin{minipage}[t][2.2cm][t]{6cm} % 外框高度一致,顶部对齐
        \vspace{1.5em} % 标题与上边沿的间距
        \centering
        {\small\textbf{LSTM-RWKV~(\S\ref{sec5_2})}} % 标题
        \vspace{0.5em} % 标题与内容的间距

        % 内容部分,使用 tabular 控制间距
        \renewcommand{\arraystretch}{1.2} % 设置行间距
        {
            \begin{tabular}{c}
                RWKV Series~\citep{peng2023rwkv, peng2024eagle} \\
                xLSTM~\citep{beck2024xlstm} \\ 
                HGRN Series~\citep{qin2024hierarchically,qin2024hgrn2}\\
                ConvLSTM~\citep{shi2015convolutional}
            \end{tabular}
        }
    \end{minipage}
};

\node[draw, fill=yellow!8, align=center, anchor=west] (SSM-Mamba) at (-3, -2) {
    \begin{minipage}[t][4.75cm][t]{12.25cm} % 外框高度一致,顶部对齐
        \vspace{2.5em} % 标题与上边沿的间距
        \centering
        \small{\textbf{SSM-Mamba~(\S\ref{sec5_3})}} % 标题
        \vspace{1.5em} % 标题与内容的间距
        
        % 绘制树状图
        \resizebox{0.9\textwidth}{!}{ % 动态调整宽度
            \begin{forest}
                forked edges,
                for tree={
                    grow=east,
                    reversed=true,
                    anchor=base west,
                    parent anchor=east,
                    child anchor=west,
                    base=center,
                    font=\normalsize,
                    rectangle,
                    draw=black, % hidden-draw,
                    rounded corners,
                    align=left,
                    text centered,
                    minimum width=4em,
                    minimum height=1cm,
                    edge+={darkgray, line width=1pt},
                    s sep=10pt,
                    inner xsep=2pt,
                    inner ysep=3pt,
                    line width=0.8pt,
                    ver/.style={rotate=90, child anchor=north, parent anchor=south, anchor=center, minimum width=12em, fill=gray!10},
                },
                where level=1{text width=10em, fill=green!10}{},
                where level=2{text width=42em, fill=white}{},
                where level=3{text width=40em, fill=white}{},
                where level=4{text width=18em,}{},
                where level=5{text width=18em,}{},
                [
                    \textbf{SSM-Mamba}, ver
                    [
                        \textbf{Before Mamba}
                        [
                            \quad HiPPO~\citep{gu2020hippo}{,} S4~\citep{gu2021efficiently}{,} H3~\citep{fu2022hungry}, leaf
                        ]
                    ]
                    [
                        \textbf{Mamba Family}
                        % [
                        %     \quad Mamba~\citep{gu2023mamba}{,}Mamba-2~\citep{daotransformers}, leaf
                        % ]
                        [
                            \quad Mamba~\citep{gu2023mamba}{,} Mamba-2~\citep{daotransformers}{,} \\ \quad The Mamba in the Llama~\citep{wang2024mamba} \\ \quad Falcon Mamba~\citep{zuo2024falcon}{,} DeciMamba~\citep{ben2024decimamba}{,} \\ \quad ReMamba~\citep{yuan2024remamba}{,} StuffedMamba~\citep{chen2024stuffed} , leaf
                        ]
                    ]
                    [
                        \textbf{Hybrid Model}
                        [
                            \quad Jamba Series~\citep{lieber2024jamba,team2024jamba}{,} Hymba~\citep{dong2024hymba}{,} \\ \quad RecurFormer~\citep{yan2024recurformer}{,} Attamba~\citep{akhauri2024attamba}, leaf
                        ]
                    ]
                ]
            \end{forest}
        }
    \end{minipage}
};

\end{tikzpicture}

    \caption{An overview of architecture innovation in long-context LLM.}
    \label{fig:arch_inno}
\end{figure}