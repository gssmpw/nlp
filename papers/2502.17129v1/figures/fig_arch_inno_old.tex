\tikzstyle{descript} = [text = black,align=center, minimum height=1.8cm, align=center, outer sep=0pt,font = \footnotesize]
\tikzstyle{activity} =[align=center,outer sep=1pt]

\begin{figure}
    \centering
\begin{tikzpicture}[
    font=\tiny,
    align=center,
    trim left=-3.5cm,
    rounded corners,
    node distance=1cm and 1.5cm,
    quadrant/.style={draw, minimum width=7.5cm, minimum height=3.2cm, fill=green!10},
    central/.style={circle, draw=orange!40, fill=orange!10, minimum size=2.5cm},
    axis/.style={thick}
]


% Quadrant Nodes
\node[quadrant, inner sep=0pt, anchor=west] (efficient-attention-mechanisms) at (-4, 2.25) {
    \begin{minipage}[t][3.2cm][t]{7.5cm} % 外框高度一致,顶部对齐
        \vspace{2.5em} % 标题与上边沿的间距
        \centering
        \large{\textbf{Efficient Attention Mechanisms}} % 标题
        \vspace{0.5em} % 标题与内容的间距

        % 内容部分,使用 tabular 控制间距
        \renewcommand{\arraystretch}{1.2} % 设置行间距
        \small{
            \begin{tabular}{c}
                Differential Transformer\citep{ye2024differential} \\ 
                SLAB~\citep{guo2024slab} \\ 
                SparQ~\citep{ribar2023sparq} \\ 
                MInference\citep{jiang2024minference}
            \end{tabular}
        }
    \end{minipage}
};

\node[quadrant, inner sep=0pt] (lstm-rwkv) at (7.5, 2.25) {
    \begin{minipage}[t][3.2cm][t]{7.5cm} % 外框高度一致,顶部对齐
        \vspace{2.5em} % 标题与上边沿的间距
        \centering
        \large{\textbf{LSTM-RWKV}} % 标题
        \vspace{0.5em} % 标题与内容的间距

        % 内容部分,使用 tabular 控制间距
        \renewcommand{\arraystretch}{1.2} % 设置行间距
        \small{
            \begin{tabular}{c}
                xLSTM~\citep{beck2024xlstm} \\ 
                ConvLSTM~\citep{shi2015convolutional} \\
                RWKV~\citep{peng2023rwkv, peng2024eagle}
            \end{tabular}
        }
    \end{minipage}
};

\node[draw, minimum width=15.3cm, minimum height=1cm, fill=green!10, align=center, rounded corners, anchor=west] (SSM-Mamba) at (-4, -3.35) {
    % 标题部分
    \vbox{
    \begin{minipage}[t][1cm][t]{8cm} % 固定标题区域高度为 1cm
        \centering
        \large{\textbf{SSM-Mamba}} % 标题内容加粗
    \end{minipage}

    % \vspace{0.5em} % 标题与时间轴的垂直间距

    % 时间轴部分
    \begin{minipage}[t][3cm][t]{15.25cm} % 时间轴部分高度为 3cm
        \centering
            \begin{tikzpicture}[very thick, black, scale=0.6]
                \small
                %% Coordinates for the first row (Top Time Axis)
                \coordinate (O1) at (-5,3); % First row origin
                \coordinate (P1) at (-2,3);
                \coordinate (P2) at (1,3);
                \coordinate (P3) at (4,3);
                \coordinate (P4) at (7,3);
                \coordinate (P5) at (10,3);
                \coordinate (P6) at (13,3);
                \coordinate (P7) at (16,3);

                %% Coordinates for the second row (Bottom Time Axis)
                \coordinate (O2) at (-5,-2); % Second row origin
                \coordinate (Q1) at (-2,-2);
                \coordinate (Q2) at (1,-2);
                \coordinate (Q3) at (4,-2);
                \coordinate (Q4) at (7,-2);
                \coordinate (Q5) at (10,-2);
                \coordinate (Q6) at (13,-2);
                \coordinate (Q7) at (16,-2);

                %% Main Event
                \fill[color=orange!30] (O1) rectangle ($(P4)+(0,1)$);
                \draw ($(O1)+(-6.5,0.5)$) node[activity, orange] {Before Mamba};
                
                \fill[color=red!30] (P4) rectangle ($(P5)+(0,1)$);
                \draw ($(P4)+(-11.25,0.5)$) node[activity, red] {Mamba};

                \fill[color=Green!30] (P5) rectangle ($(P7)+(1,1)$);
                \draw ($(P5)+(-9,0.5)$) node[activity, Green] {Improvements};

                \fill[color=Green!30] (O2) rectangle ($(Q4)+(0,1)$);
                \draw ($(O2)+(-6.5,0.5)$) node[activity, Green] {of Mamba};

                \fill[color=Plum!30] (Q4) rectangle ($(Q7)+(0,1)$);
                \draw ($(Q4)+(-8,0.5)$) node[activity, Plum] {Hybrid Architectures};
                
                %% Top Row Events
                \draw[-, thick] (O1) -- ($(P7)+(1,0)$); % Top row timeline
                \foreach \x/\text in {O1/{HiPPO}, P1/{LSSL}, P2/{S4}, P3/{S4D}, P4/{H3}}
                    \draw (\x) node[rounded corners, above=28pt, align=center, text width=2cm, orange] {\text};
                \foreach \x/\text in {P5/{Mamba}}
                    \draw (\x) node[above=28pt, align=center, text width=2cm, red] {\text};
                \foreach \x/\text in {P6/{Jamba}}
                    \draw (\x) node[above=28pt, align=center, text width=2cm, Plum] {\text};
                \foreach \x/\text in {O2/{Mamba-2}, Q1/{SMR}, Q2/{Mamba\\-PTQ}, Q3/{The Mamba In the Llama}, Q4/{ReMamba}}
                    \draw (\x) node[above=28pt, font=\small, align=center, text width=2cm, Green] {\text};
                \foreach \x/\text in {Q5/{RecurFormer}, Q6/{Hymba}, Q7/{Attamba}}
                    \draw (\x) node[above=28pt, align=center, text width=2cm, Plum] {\text};
                \foreach \x/\text in {O1/{\small 2020.08}, P1/{\small 2021.10}, P2/{\small 2021.11}, P3/{\small 2022.6}, P4/{\small 2022.12},  P5/{\small 2023.12}, P6/{\small 2024.03}, P7/{\small 2024.05}}
                    \draw (\x) node[xshift=-7.6cm, yshift=-8pt, align=center, text width=1cm] {\text};
                
                %% Bottom Row Events
                \draw[->, thick] ($(O2)-(0,0)$) -- ($(Q7)+(1,0)$); % Bottom row timeline
                \foreach \x/\text in {O2/{\small 2024.05}, Q1/{\small 2024.05}, Q2/{\small 2024.07}, Q3/{\small 2024.08}, Q4/{\small 2024.08}, Q5/{\small 2024.10}, Q6/{\small 2024.11}, Q7/{\small 2024.11}}
                    \draw (\x) node[xshift=-7.6cm, yshift=-8pt, align=center, text width=1cm] {\text};
                
                %% Ticks for Top and Bottom Rows
                \foreach \x in {O1,P1,P2,P3,P4,P5,P6,P7}
                    \draw (\x) -- ++(0,0.2); % Top row ticks
                \foreach \x in {O2,Q1,Q2,Q3,Q4,Q5,Q6,Q7}
                    \draw (\x) -- ++(0,0.2); % Bottom row ticks

                            %% path
            \path[->, color=orange] (O1) edge ($(O1)+(0,2)$)
            \path[->, color=orange] (P1) edge ($(P1)+(0,2)$)
            \path[->, color=orange] (P2) edge ($(P2)+(0,2)$)
            \path[->, color=orange] (P3) edge ($(P3)+(0,2)$)
            \path[->, color=orange] (P4) edge ($(P4)+(0,2)$)
            \path[->, color=red] (P5) edge ($(P5)+(0,2)$)
            \path[->, color=Plum] (P6) edge ($(P6)+(0,2)$)
            \path[->, color=Green] (O2) edge ($(O2)+(0,2)$)
            \path[->, color=Green] (Q1) edge ($(Q1)+(0,2)$)
            \path[->, color=Green] (Q2) edge ($(Q2)+(0,2)$)
            \path[->, color=Green] (Q3) edge ($(Q3)+(0,2)$)
            \path[->, color=Green] (Q4) edge ($(Q4)+(0,2)$)
            \path[->, color=Plum] (Q5) edge ($(Q5)+(0,2)$)
            \path[->, color=Plum] (Q6) edge ($(Q6)+(0,2)$)
            \path[->, color=Plum] (Q7) edge ($(Q7)+(0,2)$)
        \end{tikzpicture}
    \end{minipage}
    \vspace{1.5em} % 标题与时间轴的垂直间距
    
    }
};

\end{tikzpicture}

    \caption{TODO}
    \label{fig:arch_inno}
\end{figure}