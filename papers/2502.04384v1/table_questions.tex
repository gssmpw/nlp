\begin{table}[htbp]
\caption{Basic Shapes 1 Questions}
\label{table:questions-basic_shapes_1}
\begin{tabularx}{\textwidth}{@{}lX@{}}
\hline
\textbf{Shape} & \textbf{Question} \\ \hline
Circle & Write a Python code to generate GDSII for a circle on layer 0, radius = 10 mm, center at 0,0. \\ \hline
Donut & Generate a donut shape with 10 mm outer radius and 5 mm inner radius. Make the circle smoother by setting max distance between point 0.01mm. \\ \hline
Oval & Generate an oval with major axis of 20 mm, minor axis of 13 mm, on layer 0, center at 0,0. \\ \hline
Square & Generate a square with width 10 mm, put lower right corner of the square at 0,0. \\ \hline
Triangle & Generate a triangle with each edge 10 mm, center at 0,0. \\ \hline
Grid & Draw the GDSII for a grid: Grid on Layer 1, DATATYPE 4, 5 µm grid, and total width is 200 µm and height is 400 µm, placed at coordinates (100,800) nanometers. \\ \hline
\end{tabularx}
\end{table}

\begin{table}[htbp]
\caption{Basic Shapes 2 Questions}
\label{table:questions-basic_shapes_2}
\begin{tabularx}{\textwidth}{@{}lX@{}}
\hline
\textbf{Shape} & \textbf{Question} \\ \hline
Heptagon & Generate a Heptagon with each edge 10 mm, center at 0,0. \\ \hline
Octagon & Generate an Octagon with each edge 10 mm, center at 0,0. \\ \hline
Trapezoid & Generate a Trapezoid with upper edge 10 mm, lower edge 20 mm, height 8 mm, center at 0,0. \\ \hline
Hexagon & Generate a regular hexagon with each edge 10 mm, center at 0,0. \\ \hline
Pentagon & Generate a regular pentagon with each edge 10 mm, center at 0,0. \\ \hline
Text & Generate a GDS file with the text "Hello, GDS!" centered at (0,0), with a height of 5 mm, on layer 1. \\ \hline
\end{tabularx}
\end{table}

\begin{table}[htbp]
\caption{Advanced Shapes Questions}
\label{table:questions-advanced_shapes}
\begin{tabularx}{\textwidth}{@{}lX@{}}
\hline
\textbf{Shape} & \textbf{Question} \\ \hline
Arrow & Generate an Arrow pointing to the right with length 10 mm, make the body 1/3 width of the head, start at 0,0. \\ \hline
SquareArray & Generate a square array with 5*5 mm square, for 10 columns and 10 rows, each 20 mm apart, the lower left corner of the upper right square is at 0,0. \\ \hline
Serpentine & Generate a serpentine pattern with a path width of 1 µm, 15 turns, each segment being 50 µm long and tall, starting at (0,0), on layer 2, datatype 6. \\ \hline
RoundedSquare & Draw a 10*10 mm square, and do corner rounding for each corner with r=1 mm. \\ \hline
Spiral & Generate a Parametric spiral with r(t) = e\^{}(-0.1t), for 0 <= t <= 6pi, line width 1. \\ \hline
BasicLayout & 1. Draw a rectangular active region with dimensions 10 µm x 5 µm.
2. Place a polysilicon gate that crosses the active region vertically at its center, with a width of 1 µm.
3. Add two square contact holes, each 1 µm x 1 µm, positioned 1 µm away from the gate on either side along the active region. \\ \hline
\end{tabularx}
\end{table}

\begin{table}[htbp]
\caption{Complex Structures Questions}
\label{table:questions-complex_structures}
\begin{tabularx}{\textwidth}{@{}lX@{}}
\hline
\textbf{Shape} & \textbf{Question} \\ \hline
RectangleWithText & Generate a GDS with a 30*10 mm rectangle on layer 0 with a text "IBM Research" at the center of the rectangle. Put the text on layer 1. \\ \hline
MicrofluidicChip & Draw a design of a microfluidic chip. On layer 0, it is the bulk of the chip. It is a 30 * 20 mm rectangle. On layer 2 (via level), draw two circular vias, with 2 mm radius, and 20 mm apart horizontally. On layer 3 (channel level), draw a rectangular shaped channel (width = 1 mm) that connects the two vias at their center. \\ \hline
ViaConnection & Create a design with three layers: via layer (yellow), metal layer (blue), and pad layer (red). The via radius is 10 units, pad radius is 30 units, and metal connection width is 40 units with a total length of 600 units. Position the first via at (50, 150) and the second via at (550, 150). Ensure the metal connection fully covers the vias and leaves a margin of 10 units between the edge of the metal and the pads. Leave a space of 50 units between the vias and the edges of the metal connection. \\ \hline
FiducialCircle & Draw a 3.2 mm circle, with fiducial marks inside. The fiducial marks should be a "+" sign, with equal length and width. Each marker should be 200 um apart. There will be annotations next to each marker. Row: A -> Z, column: start from 1. \\ \hline
ComplexLayout & 1. Draw three rectangular active regions with dimensions 20 µm x 5 µm, positioned horizontally with 5 µm spacing between them.
2. Create a complex polysilicon gate pattern consisting of multiple vertical and horizontal lines, with widths of 0.5 µm, forming a grid-like structure.
3. Add several contact holes (each 1 µm x 1 µm) positioned at the intersections of the polysilicon gate pattern and the active regions. \\ \hline
DLDChip & Draw a deterministic lateral displacement chip - include channel that can hold the array has gap size = 225 nm, circular pillar size = 400 nm, width = 30 pillars, row shift fraction = 0.1, add an inlet and outlet 40 µm diameter before and after the channel, use a 20*50 µm bus to connect the inlet and outlet to the channel. \\ \hline
FinFET & Draw a FinFET with the following specifications:
- Fin width: 0.1 µm
- Fin height: 0.2 µm
- Fin length: 1.0 µm
- Gate length: 0.1 µm
- Source/drain length: 0.4 µm
- Source/drain extension beyond the fin: 0.2 µm
Use separate layers for the fin, gate, and source/drain regions. \\ \hline
\end{tabularx}
\end{table}

