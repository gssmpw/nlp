
\section{Agents in \Ours{}}
\label{sec:agent}


The primary objective of Programming with Pixels is to enable general-purpose SWE agents.
We evaluate state-of-the-art agents based on vision-language models in our environment.
Each agent operates in a turn-based manner, receiving a screenshot each turn and returning an action (keyboard or mouse action) to progress toward the goal. This design is the same as used in previous works~\cite{xie2024osworldbenchmarkingmultimodalagents,koh2024visualwebarenaevaluatingmultimodalagents} and we refer them to as \textit{computer-use agents}.
In practice, most vision-language models struggle with raw image inputs.
To mitigate this, we incorporate \textit{Set-of-Marks (SoM)}~\cite{yang2023setofmarkpromptingunleashesextraordinary},
in which the agent receives both the raw image and a parse of available interface element (e.g., buttons, text fields).
The agent then interacts with element IDs instead of raw pixel coordinates.

We evaluate two categories of computer-use agents. The first category only outputs keyboard and mouse clicks.
That is, the agent uses the following observation and action space:
\begin{itemize}[leftmargin=*] 
    \item $\mathbf{O}$: a screenshot and set-of-marks annotations. 
    \item $\mathbf{A}$: keyboard and mouse clicks with set-of-marks.
\end{itemize}


The second category of computer-use agents has access to file and bash commands supplied by the environment through an API. Furthermore, a screenshot is received only when the screenshot action is called, instead of receiving it every turn. These actions are provided in PwP in a similar design principle as Anthropic computer-use~\cite{anthropic2024developing} which consists of file operations such as `read file', `create file', and `string replace'. 
In summary, these agents use the following observation and action space:
\begin{itemize}[leftmargin=*]
    \item \textbf{O}: a screenshot and set-of-marks annotations and text output from tools if used.
    \item \textbf{A}: keyboard, mouse, and file and bash operation actions.
\end{itemize}

Finally, 
in Analysis~\ref{sec:assisted_analysis}, we create a tool-based agent design compatible with \ours{}. 
Specifically, we provide agents with domain-specific API calls that represent high-level actions (such as getting file structure) instead of UI interactions.
Each high-level action is implemented as a sequence of low-level actions executed in \ours{}.     
This setting lets us test current state-of-the-art SWE agent patterns (e.g.,~\citet{wang2024openhandsopenplatformai}) within the PwP environment. 


