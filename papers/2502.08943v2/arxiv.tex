% This must be in the first 5 lines to tell arXiv to use pdfLaTeX, which is strongly recommended.
\pdfoutput=1
% In particular, the hyperref package requires pdfLaTeX in order to break URLs across lines.

\documentclass[11pt]{article}

% Change "review" to "final" to generate the final (sometimes called camera-ready) version.
% Change to "preprint" to generate a non-anonymous version with page numbers.
\usepackage[preprint]{acl}

% Standard package includes
\usepackage{times}
\usepackage{latexsym}

% For proper rendering and hyphenation of words containing Latin characters (including in bib files)
\usepackage[T1]{fontenc}
% For Vietnamese characters
% \usepackage[T5]{fontenc}
% See https://www.latex-project.org/help/documentation/encguide.pdf for other character sets

% This assumes your files are encoded as UTF8
\usepackage[utf8]{inputenc}

% This is not strictly necessary, and may be commented out,
% but it will improve the layout of the manuscript,
% and will typically save some space.
\usepackage{microtype}

% This is also not strictly necessary, and may be commented out.
% However, it will improve the aesthetics of text in
% the typewriter font.
\usepackage{inconsolata}

%Including images in your LaTeX document requires adding
%additional package(s)
\usepackage{graphicx}

\usepackage{subfigure}
\usepackage{amsmath,amsthm}
\usepackage{amssymb}
\usepackage{changes}
\definechangesauthor[name=wenyuchen]{wc}
\definechangesauthor[name=wenbo,color=red]{wb}
% \newcommand\wbadded{\added[id=wb]}

% If the title and author information does not fit in the area allocated, uncomment the following
%
%\setlength\titlebox{<dim>}
%
% and set <dim> to something 5cm or larger.
\usepackage[english]{babel}
% \theoremstyle{plain}
\usepackage{mathtools}
\newtheorem{theorem}{Theorem}[section]
\newtheorem{lemma}[theorem]{Lemma}
\newenvironment{myproof}[1][\proofname]
{\proof[#1]\mbox{}}{\endproof}
\usepackage{multirow}
\usepackage{paracol}
\usepackage{booktabs} 
\usepackage{authblk}

\renewcommand{\thefootnote}{\fnsymbol{footnote}}
\title{Beyond the Singular: The Essential Role of Multiple Generations in Effective Benchmark Evaluation and Analysis}

% Author information can be set in various styles:
% For several authors from the same institution:
% \author{Author 1 \and ... \and Author n \\
%         Address line \\ ... \\ Address line}
% if the names do not fit well on one line use
%         Author 1 \\ {\bf Author 2} \\ ... \\ {\bf Author n} \\
% For authors from different institutions:
% \author{Author 1 \\ Address line \\  ... \\ Address line
%         \And  ... \And
%         Author n \\ Address line \\ ... \\ Address line}
% To start a separate ``row'' of authors use \AND, as in
% \author{Author 1 \\ Address line \\  ... \\ Address line
%         \AND
%         Author 2 \\ Address line \\ ... \\ Address line \And
%         Author 3 \\ Address line \\ ... \\ Address line}

% \author{Wenbo Zhang$^*$ \qquad
%   Hengrui Cai  \qquad
%     Wenyu Chen \\
%     University of California Irvine, Meta\\
%     \textit{}
%     }

\author{
 \textbf{Wenbo Zhang\textsuperscript{1, 2 $*$}},
 \textbf{Hengrui Cai\textsuperscript{1 $\dag$}},
 \textbf{Wenyu Chen\textsuperscript{2 $\dag$}}
\\
 \textsuperscript{1}University of California Irvine,
 \textsuperscript{2}Meta, Central Applied Science\\
  \texttt{\{wenbz13,hengrc1\}@uci.edu}, \texttt{wenyuchen@meta.com}
}

\begin{document}
\maketitle

\footnotetext{\textsuperscript{$*$} Work done during internship at Meta}
\footnotetext{\textsuperscript{$\dag$} Co-correspondence}




\begin{abstract}
Large language models (LLMs) have demonstrated significant utilities in real-world applications, exhibiting impressive capabilities in natural language processing and understanding. Benchmark evaluations are crucial for assessing the capabilities of LLMs as they can provide a comprehensive assessment of their strengths and weaknesses. However, current evaluation methods often overlook the inherent randomness of LLMs by employing deterministic generation strategies or relying on a single random sample, resulting in unaccounted sampling variance and unreliable benchmark score estimates. In this paper, we propose a hierarchical statistical model that provides a more comprehensive representation of the benchmarking process by incorporating both benchmark characteristics and LLM randomness. We show that leveraging multiple generations improves the accuracy of estimating the benchmark score and reduces variance. We also introduce $\mathbb P\left(\text{correct}\right)$, a prompt-level difficulty score based on correct ratios, providing fine-grained insights into individual prompts. Additionally, we create a data map that visualizes difficulty and semantic prompts, enabling error detection and quality control in benchmark construction.


% By leveraging multiple generations, we demonstrate that our approach yields more accurate benchmark score estimations with reduced variance. Furthermore, we introduce a prompt-level difficulty score, termed 'P-correct,' which is based on the correct ratio of multiple generations. This metric provides fine-grained insights into individual prompts, enabling the analysis of answer prompt-level questions. Furthermore, we create a data map that visualizes benchmark prompts through difficulty and generation semantics, demonstrating its utility in detecting labeling errors and offering potential for quality control in benchmark construction.
\end{abstract}

\section{Introduction}
In recent years, advanced large language models have demonstrated remarkable versatility across a wide range of tasks and domains, with their development continuing to accelerate. To effectively track their progress, numerous generative benchmark datasets have been curated to assess both their general and specialized capabilities. 

% \highlight[id=wc, comment=if needed for page limit we can shorten these]{For instance}, MMLU-Pro \citep{wang2024mmlu} is a comprehensive benchmark designed to evaluate proficient-level understanding and reasoning across 14 diverse disciplines, including mathematics, physics, chemistry, law, engineering, psychology, and health. Similarly, GSM8K \citep{cobbe2021training} comprises high-quality, linguistically diverse grade school math word problems crafted by human problem writers. During evaluations, LLMs generate a single response for each prompt in the benchmark, and the correctness of these responses is determined by comparing them to the ground truth answers. The final benchmark score is then calculated as the average of these individual scores.

There are two primary ways for generating responses from large language models (LLMs): greedy decoding and random sampling \citep{holtzman2019curious}. Greedy decoding selects the next token with the highest probability, resulting in a deterministic output. In contrast, random sampling, such as nucleus sampling \citep{holtzman2019curious}, incorporates randomness during decoding by sampling a token at each step based on a probability distribution. This approach leads to non-deterministic output. Current LLM benchmarks typically employ one of these methods; for instance, LiveBench \citep{white2024livebench} WildBench \citep{lin2024wildbench} and OpenLLM leaderboard \citep{open-llm-leaderboard-v1} use greedy decoding, while TrustLLM \citep{huang2024trustllm},  MT Bench \citep{zheng2023judging} and Alpaca Eval \citep{alpaca_eval} employ a non-deterministic sampling configuration. During evaluations, LLMs generate a single response for each prompt in the benchmark, and the correctness of these responses is determined by comparing them to the ground truth answers. The final benchmark score is then calculated as the average of these individual scores.

% \begin{figure*}[!t]
%   \includegraphics[width=0.48\linewidth]{figures/bootstrap.png} 
%   \caption {Bootstrapped confidence intervals for Llma3 70b on MUSR dataset.}
%  %need one graph to show the performance difference of greedy decoding and random genertaion.
% \end{figure*}

However, this presents challenges within the current generative-evaluation paradigm. Firstly, deterministic generation does not align with the real-world application of LLMs, where randomness is inherent. This misalignment can lead to biased estimations of LLM performance. Even with random generation, relying on a single generation can result in significant variance in benchmark scores, particularly when the sample size is small. Furthermore, a single generation is not sufficiently informative for individual prompts, as it cannot address prompt-level questions such as, "Which question is more challenging?" This limitation creates obstacles to understanding the overall composition of the benchmark data.

In this paper, we regard the benchmark as an estimation problem characterized by a statistical model and highlight the significance of incorporating multiple random generations in a principled way. We theoretically demonstrate that increasing the number of generations decreases the variance in benchmark score estimation. Moreover, by leveraging multiple samples, we introduce a fine-grained difficulty metric, $\mathbb P\left(\text{correct}\right)$, derived from the inherent latent parameters of our statistical model, to quantify the difficulty of individual prompts. This enables comparisons across different prompts. Additionally, we demonstrate that mislabeled or ambiguous prompts can be effectively detected using multiple generations, highlighting its potential as a tool in benchmark construction.









% One recent observation shows that the model can perform better when using greedy decoding than random sampling in some downstream tasks \citep{song2024good} which might lead to overestimate of the model performance is greedy decoding is used.

\section{Benchmarking Procedure is a Hierarchical Model}

In this section, we show that the benchmark is an estimation problem. Without loss of generality, we consider random sampling as the generation strategy where each token is randomly sampled from a token distribution conditional on previously generated tokens. We also assume the correctness of generations can be obtained using a judgment function, which can be accomplished either by comparing the response with ground truth or by determining whether it passes unit tests. 

% During benchmarking, each time the LLM is queried with a prompt, it returns a generated response along with a corresponding correctness indicator provided by the judging function. Inherently, every prompt possesses a latent property reflecting its difficulty, which can be expressed through a hierarchical statistical model. 

Given an LLM parameterized by parameters $\theta$, including both model parameters and sampling parameters, for example temperature $T$ and top $P$, etc.), and a benchmark dataset $\mathcal{D}=\{x_i\}_{i=1}^n$, we can define difficulty of the $i$-th prompt with respect to the LLM as a random variable drawn from the unknown benchmark difficulty distribution $\mathbb{P}(\mu, \sigma;\theta)$, with mean $\mu$ and standard deviation $\sigma$. Without loss of generality, with $k$ generations per prompt, we can then regard the benchmarking procedure as a hierarchical model as follows:
\begin{equation}
\begin{gathered}
p_i \sim \mathbb{P}(\mu, \sigma;\theta) \quad \text{for } i=1,\cdots,n\\
y_{i,j}  \sim \text{Bernoulli}(p_i) \quad \text{for } j=1,\cdots,k,
\end{gathered}
\label{stat model}
\end{equation}
where prompt difficulty $p_i$ is sampled from $\mathbb{P}(\mu, \sigma;\theta)$ and $p_i$ represents the probability that the LLM can correctly answer the $i$-th prompt., i.e., $\mathbb P\left(\text{A generated answer to $i$-th prompt is correct}\right)=p_i$. This represents a latent difficulty of prompts, We denote the he $k$-th generation of the $i-$th prompt as $z_{i,j}$ and then $y_{i,j}$ is the correctness indicator for it, where $y_{i,j}=1$ if it’s correct otherwise $y_{i,k}=0$. 

 Here both benchmark distribution $\mathbb{P}(\mu, \sigma;\mathcal{D})$ and $p_i$ are unknown needs to be estimated with $\{y_{i,j}\}_{j=1}^k$ for $i=1, \cdots, n$.

To estimate $p_i$ and $\mu$, we can use a straight forward method of moment estimators $\hat p_i = \frac{\sum_{j=1}^k y_{i,j}}{k}$, $\hat \mu = \frac{\sum_{i=1}^n \hat p_{i}}{n}=\frac{\sum_{i=1}^n\sum_{j=1}^k y_{i,j}}{nk}$. We observe that a widely used item response theory \citep{polo2024tinybenchmarks,madaan2024quantifying,ding2024easy2hard}, employed to model the difficulty of prompts, represents a specific parametrization of $\mathbb{P}(\mu, \sigma;\mathcal{D})$. Further elaboration on this can be found in Appendix \ref{irt section}.

Note that, when $k=1$, the benchmark score computed based on a single random generation is an estimation of $\mu$, which only utilizes a single generation which leads to a large variance. We can show this by explicitly calculating the variance of our estimators.

\begin{lemma}
    Given the  hierarchical model in (\ref{stat model}) and the moment estimators $$\hat \mu =\frac{\sum_{i=1}^n\sum_{j=1}^k y_{i,j}}{nk}.$$
    Then $\hat \mu$ is an unbiased estimator for $\mu$ and its variance equals:
    \begin{equation}
\text{Var}(\hat\mu)=\underbrace{\frac{1}{nk}\left(\mu-\mu^2-\sigma^2\right)}_{\text{Withth-prompt Variance}}+\underbrace{\frac{1}{n}\sigma^2}_{\text{Between-prompt Variance}}.
\label{var}
    \end{equation}
    \label{lemma}
\end{lemma}
% \vspace{-0.5cm}
Here, $\text{Var}(\hat\mu)$ can be decomposed into within-prompt variance and between-prompt variance. Both terms decrease as the number of benchmark data $n$ increases. However, since benchmark data is typically fixed, we analyze the influence of sampling in terms of $k$. Within-prompt variance captures the randomness in sampling $y_{ij}$ conditional on the $i-$th prompt, and it can be effectively reduced by increasing the number of samples $k$, converging to $0$ as $k \to \infty$. The between-prompt variance term, on the other hand, captures the variability of prompt difficulty $p_i$ across groups, reflecting the randomness of difficulty distribution $\mathbb{P}(\mu, \sigma;\theta)$, and thus remains unaffected by $k$.

We can further plug in sample variance $\hat\sigma^2=\frac{1}{n-1} \sum_{i=1}^n(\hat p_{i}-\frac{\sum_{i=1}^n \hat p_i}{n})^2$ and $\hat\mu$ into (\ref{var}) to get $\widehat{\text{Var}(\hat \mu)}$. Finally, based on the central limit theorem, a $95\%$ confidence interval can be constructed as:
\begin{equation}
\hat\mu \pm 1.96 \sqrt{\widehat{\text{Var}(\hat \mu)}}
\end{equation}

\begin{figure}[!htbp]
    \centering
        \subfigure[MMLU-Pro]{\includegraphics[width=0.46\columnwidth]{figures/MMLU_PRO_distribution.pdf}}
    \subfigure[GSM8K]{\includegraphics[width=0.46\columnwidth]{figures/GSM8K_distribution.pdf}}\\
    \subfigure[IFEval]{\includegraphics[width=0.46\columnwidth]{figures/IFEval_distribution.pdf}}
    \subfigure[MuSR]{\includegraphics[width=0.46\columnwidth]{figures/MUSR_distribution.pdf}}
    \\
    \caption{Distribution of $\mathbb P\left(\text{correct}\right)$ of $4$ benchmarks.}
    \label{fig:dist}
    \vspace{-5mm}
\end{figure}
\subsection{Prompt Level Difficulty: $\mathbb P\left(\text{correct}\right)$}
Our goal is to develop a granular, quantifiable measure of prompt difficulty, enabling us to gain a deeper understanding of their relative complexities. By quantifying prompt difficulty at the individual level, we can address fundamental questions such as: `Which prompts are most challenging?' and `How do different prompts compare in terms of difficulty?' A fine-grained understanding of prompt difficulty will provide valuable insights into the strengths and weaknesses of language models, as well as the composition of benchmark datasets, ultimately informing the development of more effective models and evaluation frameworks. 

We refer to $\mathbb P\left(\text{correct}\right)=p_i$ in (\ref{stat model}) and its estimation $\widehat{ \mathbb{P}}\left(\text{correct}\right)=\hat p_i = \frac{\sum_{j=1}^k y_{i,j}}{k}$. When the number of generations $k$ increases, it will converge to the true $\mathbb P\left(\text{correct}\right)$ and therefore more fine-grained. The probability of correctness $p_i$ can be interpreted as a difficulty score at the prompt level: the higher the $p_i$, the easier the prompt since the language model has a higher probability of generating a correct response. We demonstrate the use of difficulty scores in the analysis section.

% We also demonstrate that $P\left(\text{correct}\right) $ can be used to explain the effectiveness of self-consistency \citep{wang2022self}, a widely adopted test-time inference strategy. A formal statement and detailed discussion are provided in the Appendix \ref{self-cosnsitenct}.

% #Informally, \textbf{self-consistency tends to yield the correct answer when the true probability} $\mathbf{p_i}$ \textbf{exceeds} $\mathbf{1/2}$.

% With this notion of difficulty, we can easily compare the difficulty across prompt based their multiple generations. 





\section{Experiments}
\subsection{Experimental Setup}
\noindent \textbf{Benchmark.}
We choose multiple benchmarks which cover various capabilities of LLMs: MMLU-Pro \citep{wang2024mmlu}, GSM8K \citep{cobbe2021training}, MuSR \citep{sprague2023musr}, IFEval \citep{zhou2023instruction}. For MMLU-Pro, GSM8K, and MUSR, we use accuracy as the metric, while for IFEval, we utilize instance-level strict accuracy. Brief introduction of those benchmarks can be found in Appendix \ref{bench details}.



\noindent \textbf{LLM and Setup.}
We utilize four widely-used open-source LLMs: Llama 3.1 (8B and 70B Instruct) \citep{dubey2024llama}, Qwen 2.5 (7B Instruct) \citep{qwen2}, and Ministral (8B Instruct) \citep{jiang2023mistral}\footnote{Ministral models and analysis on Ministral output were run only by Wenbo Zhang on academic research systems.}. We evaluate both greedy decoding and random sampling on these models, with the latter using a temperature of $0.7$ and top-p of $1.0$. For each prompt across all benchmarks, we generate $50$ samples ($k=50$) using a 0-shot chain-of-thought prompting strategy.


\begin{table*}[htbp]

\centering

\caption{Results on four benchmark datasets with four open source LLMs. "n" is the number of prompts, "Greedy" denotes greedy decoding, "Sample (k=50)" is the random sample with $50$ generations and "$\Delta$ ($k=1$)" denotes the performance gap between the best and worst run with $1$ generation. We include both benchmark score and SE.}

\resizebox{1\textwidth}{!}{

\begin{tabular}{lccccccc}

\toprule \multirow{2}{*}{\textbf{Benchmark}} & \multirow{2}{*}{\textbf{n}} & \multicolumn{3}{c}{\textbf{Llama 3.1 8b Instruct}} & \multicolumn{3}{c}{\textbf{Llama3.1 70b Instruct}}  \\
\cmidrule(lr){3-5} \cmidrule(lr){6-8} 
& & \textbf{Greedy} & \textbf{Sample} ($k=50$)& $\Delta(k=1)$  & \textbf{Greedy} & \textbf{Sample} ($k=50$) & $\Delta (k=1)$ \\

\hline MMLU-Pro  & $12,187$ & $46.2$ ($0.45$) & $46.1$ ($0.39$) & $10.0$ & $63.8$ ($0.44$)  & $63.4$ ($0.40$) & $3.9$ \\

 GSM8K & $1,319$ & $86.1$ ($0.95$) & $85.6$ ($0.68$) & $18.6$ & $95.6$ ($0.56$) & $95.3$ ($0.45$) & $4.8$ \\

 IFEval & $541$ & $74.5$ ($1.87$) & $71.1$ ($1.51$) & $8.3$ & $82.6$ ($1.64$) & $80.2$ ($1.42$)  & $5.9$ \\

 MuSR & $756$ & $24.8$ ($1.65$)  & $29.0$ ($1.00$) & $8.2$ & $56.3$ ($1.80$)  & $57.9$ ($1.40$) & $5.4$ \\
\midrule \multirow{2}{*}{\textbf{Benchmark}} & \multirow{2}{*}{\textbf{n}} & \multicolumn{3}{c}{\textbf{Qwen 2.5 7B Instruct}} & \multicolumn{3}{c}{\textbf{Ministral 8B Instruct}}  \\
\cmidrule(lr){3-5} \cmidrule(lr){6-8} 
& & \textbf{Greedy} & \textbf{Sample ($k=50$)}& $\Delta (k=1)$  & \textbf{Greedy} & \textbf{Sample} ($k=50$) & $\Delta (k=1)$ \\

\hline MMLU-Pro  & $12,187$ & $53.3$ ($0.45$) & $53.0$ ($0.36$) & $1.3$ & $39.7$ ($0.44$)  & $36.3$ ($0.29$) & $1.5$ \\

 GSM8K & $1,319$ & $90.2$ ($0.82$) & $90.2$ ($0.65$) & $2.3$ & $86.1$ ($0.95$) & $84.9$ ($0.73$) & $3.1$ \\

 IFEval & $541$ & $72.6$ ($1.92$) & $71.2$ ($1.64$) & $5.9$ & $51.4$ ($2.15$) & $49.8$ ($1.65$)  & $5.6$ \\

MuSR & $756$ & $49.2$ ($1.82$)  & $50.9$ ($0.98$) & $8.3$ & $49.7$ ($1.82$)  & $50.8$ ($0.91$) & $8.6$ \\
\bottomrule
\end{tabular}
}

\label{benchmark score}

\end{table*}
\subsection{Main Results}
Our results are presented in Figures \ref{fig:dist} and \ref{fig:map}, as well as in Table \ref{benchmark score}. The key takeaways from these results are summarized below.


\noindent \textbf{Distribution of $\mathbb P\left(\text{correct}\right)$ show diffuse density in challenging tasks, behaving like random samplers.}
For the distribution of $\mathbb P\left(\text{correct}\right)$, we define stable behavior as a density distribution with high concentrations near $0$ and $1$, and lower density in between. Conversely, a distribution with a high density between $0$ and $1$ indicates high randomness. As shown in Figure \ref{fig:dist}, when confronted with benchmarks that require strong reasoning skills(MMLU-Pro, IFEval, and MuSR), all models display a diffuse density distribution over the support $\left[0,1\right]$. This suggests that LLMs resemble random samplers when handling prompts requiring strong reasoning, underscoring the complexity and sensitivity of their reasoning processes. In contrast, the simpler task GSM8K display densities with more pronounced tails and reduced uncertainty. A plausible explanation is that GSM8K is easier and involves shorter reasoning lengths, which in turn decreases the likelihood of diverse reasoning paths emerging. Additionally, we observe that the Llama 70B model exhibits the most stable performance across all benchmarks, suggesting that larger models may be capable of providing more stable reasoning.

% we observe that the Llama 8b model exhibits a right-skewed distribution, while the 70b model shows a left-skewed distribution in both MUSR and MMLU-Pro datasets. This trend suggests that larger models, which are generally more capable, encounter fewer challenging examples. Despite this, both distributions display a notable concentration in the middle, implying that large language models (LLMs) may sometimes function like random samplers.

\noindent \textbf{Estimation differs noticeably between greedy decoding and random sampling, with a single random generation being unstable.}
Table \ref{benchmark score} presents the benchmark scores, highlighting the performance differences between greedy decoding and random sampling. Notably, for GSM8K and MuSR, the absolute differences in benchmark score between these two methods for Llama3 8B are $3.4$ and $4.2$ respectively, indicating a relatively large performance gap. This discrepancy can also be observed in other models and datasets. Furthermore, we observe considerable variability with one generation, characterized by large values of $\Delta$($k=1$). This suggests that random sampling with limited generations is ineffective for benchmark evaluation, particularly for small datasets, aligning with our Lemma \ref{lemma}. We also compared the confidence intervals with the bootstrap confidence intervals commonly used in previous studies and found similar performance. This indicates that running the bootstrap procedure is unnecessary, as our method achieves comparable results and operates much faster without resampling. We also investigate how sampling parameters influence the $\mathbb P\left(\text{correct}\right)$ distribution and results are in Appendix \ref{vary T}.

\begin{figure}[htbp]
    \centering
    \includegraphics[width=\columnwidth]{figures/noise_labels_2.png}
    \caption{Data map for GSM8K with Llama 70b.}
    \label{fig:map}
\end{figure}

% \begin{figure}[htbp]
%     \centering
%         \subfigure[MMLU-Pro]{\includegraphics[width=0.48\columnwidth]{figures/llama3.1 70b_mmlu-pro_map.pdf}}
%         \hfill
%     \subfigure[GSM8K]{\includegraphics[width=0.48\columnwidth]{figures/llama3.1 70b_gsm8k_map.pdf}}
%     \caption{$\mathbb P\left(\text{correct}\right)$ vs $\mathbb S\left(\text{consistency}\right)$ for two benchmarks with LLama3.1 70b.}
%     \label{fig:map}
% \end{figure}
\noindent \textbf{Multiple generations can help detect labeling errors: a case study on GSM8K.}
Benchmark construction can involve label errors or ambiguous prompts, such as the approximately $5\%$ error rate in GSM8K. Manually cleaning large datasets is costly, but we found using multiple generations from advanced LLMs can help identify mislabeled or ambiguous prompts. Based on multiple generations, we can create a data map to visualize $\mathbb{P}(\text{correct})$ against $\mathbb{S}(\text{consistency})$, which measures the semantic consistency of generations. More details on this measure are provided in Appendix \ref{s_consis}. We hypothesize that prompts with low $\mathbb P(\text{correct})$ and high $\mathbb{S}(\text{consistency})$ may be mislabeled or ambiguous due to contradicting with the self-consistency\citep{wang2022self}. Self-consistency \citep{wang2022self,mitchell2022enhancing} leverages the intuition that a challenging reasoning problem typically admits multiple reasoning paths leading to its unique correct answer. To verify our hypothesis, we utilize the data map of Llama3 70B for GSM8K and selected prompts with $\mathbb{P}(\text{correct}) \leq 0.1$ and $\mathbb{S}(\text{consistency}) \geq -0.8$, totaling $18$ prompts. After manually reviewing the selected prompts, we found that $44.4\%$ prompts were either mislabeled or ambiguous. Examples are shown in the Appendix Figure \ref{example}. Our results demonstrate the potential of using data maps as a tool to clean datasets, which is similar to a prior study \citep{swayamdipta2020dataset} but it focuses on classification models rather than generative models. Notably, our findings utilize only a single LLM and a simple semantic metric. This highlights future research that will incorporate more models and improved semantic metrics for more accurate detection.

% If we consider large language models (LLMs) as random samplers that can explore various solutions to a challenging question, rather than adhering to a single solution path unless they are completely certain.  

% for the prompts of three benchmarks in Figure \ref{fig:map}. This difficulty vs consistency can indeed categorize prompts as several categories: 1. difficult/consistent. 2. difficult/inconsistent 3. easy/consistent. 4. easy/inconsistent. easy/inconsistent is not true since easy question means high accuracy and and answer is unique. One notable category is difficult/consistent, which shows that for some questions, the LLM consistently produces the same incorrect reasoning paths. This seems contradictory when considering the LLM as a sampler, which would normally explore various solutions to a challenging question rather than sticking to a single solution unless it is entirely certain. The most likely reason for this behavior is a discrepancy between the LLM's knowledge and reference answer or the choice error we mentioned in the previous part. Considering potential labeling errors in the construction of the benchmark dataset, we hypothesize that we can use consistency with incorrect answers as a method to detect label noise.


% Specifically, the results highlight a gap between greedy decoding and random sampling ($k=50$), suggesting that the former may not accurately represent the true capabilities of large language models (LLMs) in real-world applications where random sampling is commonly employed. 


% \begin{table*}[ht]
% \centering
% \caption{Results on four benchmark datasets with LLama3.1 8b and 70b Instruct. "n" is the number of prompts, "Greedy" denotes greedy decoding, "Sample(k=50)" is the random sample with $50$ generations and "$\Delta$($k=1$)" denotes the  performance gap between the best and worst run with $1$ generation.}
% \resizebox{1\textwidth}{!}{
% \begin{tabular}{l|c|ccc|ccc}
% \hline \multirow{2}{*}{Benchmark} & \multirow{2}{*}{n} & \multicolumn{3}{c|}{LLama 3.1 8b Instruct} & \multicolumn{3}{c}{LLama3.1 70b Instruct}  \\
% & & Greedy & Sample($k=50$)& $\Delta($k=1$)$  & Greedy & Sample($k=50$) & $\Delta$($k=1$) \\
% \hline MMLU-Pro  & 12187 & 46.2 & 46.1 (2.60) & 9.99 & 63.8 & 63.4 (0.98) & 3.86 \\
% \hline GSM8K &1319& 81.7 & 82.7 (3.75) & 19.5 & 90.8 & 91.4 (1.03) & 5.53 \\
% \hline MUSR &756 & 24.8 & 29.0 (1.92) & 8.20 & 56.3 & 57.9 (1.32) & 5.42 \\
% \hline IFEval &541 & - & - & - & - & - & - \\

% \hline
% \end{tabular}
% }
% \label{benchmark score}
% \end{table*}

\section{Conclusion}
In this paper, we investigate the role of multiple generations in enhancing LLM benchmark evaluation. Using a hierarchical model, we quantify prompt difficulty and visualize its distribution, revealing insights into benchmark data. Our findings about variance reduction, and labeling error detection emphasize the importance of leveraging multiple generations for robust benchmark evaluations.

\section*{Limitations}
While using multiple generations in benchmark evaluation is promising, it demands more computational resources during inference time. Future research could explore the minimal number of generations required for robust evaluation, potentially reducing within-prompt variance. Additionally, our statistical model assumes that all prompts are independently sampled from the benchmark difficulty distribution, which may not be accurate in practice, as prompts can originate from the same subjects or resources. Future work should consider incorporating the covariance structure into the estimation process. Another drawback is the detection of mislabeled prompts. Although our method efficiently reduces the effort needed to filter samples, the true positive rate is not high (around $50\%$). Potential research could leverage more sophisticated semantic metrics and model ensembles to better detect mislabeled or ambiguous prompts.

\section*{Ethic Statement}
Our work utilizes benchmark datasets to evaluate LLMs. All the datasets and LLMs are publicly available.

\section*{Acknowledgements}
We thank Christine Agarwal, Xu Chen, Aude Hofleitner, Jenny Hong, Kartik Khandelwal, Wesley Lee, Tony Lee, and Yilin Zhang for their helpful discussions.

% we explore the role of multiple generations in enhancing the evaluation of Large Language Model (LLM) benchmarks. By employing a hierarchical model, we illustrate the benchmarking process from a difficulty perspective, enabling us to quantify prompt difficulty and describe its distribution across the benchmark. Additionally, we visualize the consistency of generations with respect to difficulty, creating a map of prompts that provides valuable insights into the benchmark data. Our findings also reveal discrepancies between greedy decoding and random generations, as well as high variance in single-sample random generation, underscoring the importance of leveraging multiple generations to increase the robustness of evaluations.


% to do; add prompt level p-correct visualization





\bibliography{ref}

\clearpage
\appendix
\onecolumn
\section{Related Work}

\subsection{LLM Benchmark Evaluation}
Recent benchmark evaluations have significantly enhanced our understanding of Large Language Models (LLMs) and have driven further advancements in the field. Notable benchmarks like MMLU \citep{hendrycks2020measuring}, HELM \citep{liang2022holistic}, and BIG-bench \citep{srivastava2022beyond} have expanded assessments to include language generation, general knowledge understanding, and complex reasoning. Several other benchmarks assess the trustworthiness of large language models (LLMs) \citep{wang2023decodingtrust, huang2024trustllm, zhang2024defining} in terms of safety, bias, privacy, and hallucination, etc. Leaderboards like the OpenLLM Leaderboard \citep{open-llm-leaderboard-v1} facilitate performance comparisons across LLMs by evaluating a range of tasks, each targeting different capabilities, to provide a comprehensive assessment of LLMs. However, most benchmark evaluations, even on leaderboards, rely on a single output per example, either greedy decoding or random sampling. \citet{song2024good} also examines the performance gap between the two types of generation strategies and highlights the importance of randomness. There is also concurrent work by \citet{miller2024adding} that mentions using multiple generations to reduce variance, but their contribution is primarily conceptual. In contrast, we provide both theoretical support and empirical results. Additionally, we propose several benefits of using multiple generations, such as difficulty quantification and mislabeled prompt detection, which distinguish our work from theirs.

\subsection{Prompt Difficulty in Benchmark}
Understanding prompt-level difficulty is crucial for analyzing benchmark composition and some benchmark datasets include difficulty scores for each prompt provided by humans. For example, the MATH dataset \citep{hendrycks2measuring} offers a variety of high-school-level problems with a broad five-level difficulty rating. Similarly, the GPQA dataset \citep{rein2023gpqa} contains graduate-level multiple-choice questions rated on a 4-point scale by two experts. Recent studies \citep{ding2024easy2hard,polotinybenchmarks} also attempted to estimate difficulty scores of individual prompts using item response theory \citep{cai2016item,natesan2016bayesian} or Glicko-2 \citep{glickman2012example}, based on offline evaluation results from a pool of large language models (LLMs) or human participants. This approach seeks to provide an objective difficulty score by encompassing a diverse range of testers, including both humans and LLMs. However, this can lead to misalignment when focusing solely on a target LLM. A question that is easy for one model might be difficult for others, highlighting the inherently subjective nature of difficulty \citep{desender2017subjective}. Therefore, it is more relevant to consider the subjective difficulty specific to the target LLM.

\section{IRT is a special parametrization of $\mathbb P\left(\text{correct}\right)$}
\label{irt section}
 $\mathbb P\left(\text{correct}\right)$ is closely connected to item response theory. Many studies \citep{polo2024tinybenchmarks,madaan2024quantifying,ding2024easy2hard} utilize IRT to quantify the difficulty of prompts using multiple LLMs. One variation of the IRT model is the one-parameter logistic (1PL) model as defined below: 
 \begin{align}
     \mathbb{P}\left(y_{li}=1\mid \theta_l,b_i\right) = \frac{1}{1+\exp^{-\left(-\theta_l-b_i\right)}},
     \label{irt}
 \end{align}
 where $\mathbb{P}\left(y_{li}=1\mid \theta_l,b_i\right)$ is the probability that LLM $l$ can answer the $j$-th prompt correctly. $\theta_l$ represents the latent ability of LLM $l$, $b_i$ is the difficulty parameter of the $j$-th prompt.

We observe that when we focus on a single LLM, i.e., when LLM $l$ is fixed, $\mathbb{P}\left(y_{li}=1\mid \theta_l,b_i\right)$ coincides with the prompt difficulty $p_i$ defined in (\ref{stat model}). Consequently, the right-hand side of (\ref{irt}) can be viewed as a specific parametrization of the prompt difficulty using a logit link function. This implies that, theoretically, the maximum likelihood estimator of IRT and our method are equivalent via a sigmoid transformation. We use the 1PL model here for illustrative purposes, but this equivalence also holds when extended to models with more parameters.

% \section{When Self-Consistency is Effective}
% \label{self-cosnsitenct}
% Self-consistency refers to an inference time technique. It samples multiple generations and then utilize the majority vote to get the final results. 

% The following Lemma illustrates when a prompt can gain from self-consistency.




\section{Benchmark Details}
\label{bench details}
MMLU-Pro is a comprehensive benchmark tailored for advanced, multi-disciplinary language understanding and reasoning at the proficient level. The GSM8K dataset comprises linguistically diverse math word problems from grade school curricula, crafted by human experts. MuSR is a specialized dataset designed to assess language models' performance on multi-step soft reasoning tasks presented in natural language narratives. IFEval, meanwhile, provides verifiable instructions to test large language models' ability to follow instructions accurately. 

\section{Additional Results on Varying Temperature $T$}
\label{vary T}
\begin{figure}[!htbp]
    \centering
    \subfigure[GSM8K Llama 8B]{\includegraphics[width=0.48\textwidth]{figures/GSM8K_8b_vary_T_distribution.pdf}}
    \subfigure[GSM8K Llama 70B]
    {\includegraphics[width=0.48\textwidth]{figures/GSM8K_70b_vary_T_distribution.pdf}}
    \subfigure[MUSR Llama 8B]{\includegraphics[width=0.48\textwidth]{figures/MUSR_8b_vary_T_distribution.pdf}}
    \subfigure[MUSR Llama 70B]
    {\includegraphics[width=0.48\textwidth]{figures/MUSR_70b_vary_T_distribution.pdf}}
    \\
    \caption{Distribution of $\mathbb P\left(\text{correct}\right)$ for GSM8K and MUSR when varying temperature $T$.}
    \label{fig vary t}
\end{figure}

To investigate how temperature influences the $\mathbb{P}(\text{correct})$ distribution, we vary the sampling temperatures $T$ across $0.4$, $0.7$, and $1.0$ for the GSM8K and MUSR datasets using the Llama 8B and 70B models. The results are in Figure \ref{vary T}. We find that for the smaller 8B model, as $T$ increases, the distribution becomes more unstable with a more diffuse density. However, for the larger model, the $\mathbb{P}(\text{correct})$ is less sensitive to changes in $T$.

\section{Semantic Consistency for Responses: $\mathbb S\left(\text{consistency}\right)$}
\label{s_consis}
Apart from the correctness, we can also measure the difficulty of benchmark prompts by examining the semantic complexity from
multiple generations. This is because analyzing the nature of errors produced by LLMs can provide valuable insights into their decision-making processes. Specifically, it can help us determine whether LLMs tend to make consistent or varied mistakes, shedding light on their limitations and potential areas for improvement. 

We can group responses into multiple clusters based on their semantic meaning using bidirectional entailment predictions from a Natural Language Inference (NLI) model, such as DeBERTa or a prompted large language model (LLM). 

% Like the number of semantic set, which counts how many semantic meanings in the generations. 

One common metric for quantifying consistency is the number of semantic sets, originally developed for uncertainty quantification in LLMs. The number of semantic sets assumes that a higher number of distinct semantic sets corresponds to lower consistency. 

However, the number of semantic sets only considers the number of clusters, without taking into account the proportion of generations within each cluster. For instance, consider two scenarios with 8 generations and 2 clusters: one where 1 generation falls into the first cluster and 7 into the second, versus another where 4 generations fall into each cluster. While these scenarios clearly represent different levels of consistency, the semantic set metric fails to distinguish between them, highlighting the need for a more nuanced approach to evaluating consistency.

Here we utilize a metric called semantic set entropy to better account for the proportions of semantic clusters. Given a set of $k$ generations and cluster them into $C$ semantic sets, semantic set entropy can be represented as:
$$
\mathbb S\left(\text{consistency}\right) = \sum_{c=1}^C \text{Prop}_c\log \text{Prop}_c,
$$
where $\text{Prop}_c$ measures the proportion of generations in group $c$ and its empirical estimator $\widehat{\text{Prop}}_c=\frac{\text{\# generations in set c}}{m}$ with finite $m$ samples. This can be seen as negative semantic set entropy, the larger, more consistent.
\section{Proof of Lemma \ref{lemma}}
Restate of Lemma \ref{lemma}: 

\noindent Given the model \begin{equation}
\begin{gathered}
p_i \sim \mathbb{P}(\mu, \sigma;\theta) \quad \text{for } i=1,\cdots,n\\
y_{i,j}  \sim \text{Bernoulli}(p_i) \quad \text{for } j=1,\cdots,k,
\end{gathered}
\end{equation}
and the moment estimator $\hat \mu =\frac{\sum_{i=1}^n\sum_{j=1}^k y_{i,j}}{nk}.$
Then $\hat \mu$ is an unbiased estimator for $\mu$ and its variance equals
    \begin{equation*}
\text{Var}(\hat\mu)=\underbrace{\frac{1}{nk}\left(\mu-\mu^2-\sigma^2\right)}_{\text{Withth-prompt Variance}}+\underbrace{\frac{1}{n}\sigma^2}_{\text{Between-prompt Variance}}.
    \end{equation*}

\noindent \textit{Proof}:
Firstly we show $\hat\mu$ is an unbiased estimation of $\mu$, which can be directly show by the expectation:

\begin{align*}
\mathbb{E}\left[\hat \mu\right] &= \frac{\sum_{i=1}^n\sum_{j=1}^k y_{i,j}}{nk} \\
& = \frac{\sum_{i=1}^n\mathbb E\left[\sum_{j=1}^k y_{i,j}\right]}{nk}\\
&  \stackrel{(3)}{=}  \frac{\sum_{i=1}^n\mathbb E\left[\mathbb E\left[\sum_{j=1}^k y_{i,j}\mid p_i\right]\right]}{nk} \\
& = \frac{\sum_{i=1}^n k\mathbb E\left[p_i\right]}{nk}\\
& = \frac{\sum_{i=1}^n k\mathbb \mu}{nk}\\
& = \mu, 
\end{align*}

where $(3)$ utilizes the law of total expectation. Hence $\hat \mu$ is unbiased estimator of $\mu$.
The variance of $\hat \mu$ can be further shown:

\begin{align*}
\text{Var}\left(\hat\mu\right)&=\text{Var}\left(\frac{\sum_{i=1}^n\sum_{j=1}^k y_{i,j}}{nk}\right)\\
& = \frac{1}{n^2k^2 }\left(\sum_{i=1}^n\text{Var}\left(\sum_{j=1}^k y_{i,j}\right)\right)\\
& \stackrel{(3)}{=} \frac{1}{n^2k^2}\left(\sum_{i=1}^n \mathbb E \left[\text{Var}\left(\sum_{j=1}^k y_{ij}\mid p_i\right)\right]\right.\\ 
&\quad + \left. \text{Var} \left(\mathbb E \left(\sum_{j=1}^k y_{ij}\mid p_i\right)\right)\right)\\
& = \frac{1}{n^2k^2}\left(\sum_{i=1}^n \mathbb E\left[ k p_i\left(1-p_i\right)\right] + \text{Var}\left(kp_i\right) \right)\\
& = \frac{1}{n^2k^2} \left(nk \left(\mathbb E\left[p_i\right] - \mathbb E\left[p_i^2\right]\right)+ nk^2\text{Var}\left(p_i\right)\right)\\
&=\underbrace{\frac{1}{nk}\left(\mu-\mu^2-\sigma^2\right)}_{\text{Withth-prompt Variance}}+\underbrace{\frac{1}{n}\sigma^2}_{\text{Between-prompt Variance}}.
\end{align*}
where $(3)$ utilizes the low of total variance.


% \section{Case Study: Noise Prompts Detection in GSM8K}

\begin{figure}[!htbp]
    \centering
    \includegraphics[width=\columnwidth]{figures/example_2.png}
    \caption{Examples of detected mislabeled and ambiguous prompts in GSM8K.}
    \label{example}
\end{figure}





\end{document}
