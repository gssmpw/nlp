\section{Results}\label{sec:experiments}
\subsection{Integration of the Anafi Drone into a Gazebo Simulation}
Figure \ref{fig:anafi_blimp} shows the Gazebo-based airship simulation \cite{price2022} including a mirrored instance of the Anafi drone. 
\begin{figure}[thpb]
      \centering
		\includegraphics[scale=0.275]{./pics/copter_blimp.jpg}
\caption{\textit{Anafi} drone and airship set up jointly in a common Gazebo simulation environment. }
\label{fig:anafi_blimp}
 \end{figure}
Since Gazebo can have difficulties loading and rendering multiple complex meshes simultaneously, we neglect the display of rotating propellers to keep the model of the \textit{Anafi} drone as simple as possible.
 
 
\subsection{Evaluation of Tracking Controllers}
\subsubsection*{General}
For the evaluation and comparison of our MPC and PID controller's tracking capabilities of static and moving waypoints in simulation and the real world we design four different scenarios. i) Static waypoints, ii) a straight line trajectory at different speeds, iii)  different circular trajectories residing in the $x,y$-plane with different radii and  speeds and iv) a rectilinear periodic movement in the $z$-direction only. The reason is that these scenarios provide standardized movements allowing a fair comparison between the controllers.
\subsubsection{Static waypoints}
Figure \ref{fig:static_waypoints_sim} qualitatively compares in simulation the performance between our customized PID framework and the MPC for reaching static waypoints starting in hover flight. Two waypoints are examined. The difference between them is the yaw angle at which the waypoints should be reached. For the first waypoint $WP_0$ it is $\psi_{ref,world} = 0\si{rad}$, for the second waypoint $WP_1$ is $\psi_{ref,world} = \pi\si{rad}$ .
\begin{figure}[thpb]
      \centering
		\includegraphics[scale=0.45]{./pics/static_comparison.pdf}
\caption{Comparison between the performance of the PID-based controller framework and MPC in simulation. The waypoints are $WP_0=\left\lbrace 3,2,5,0,0,0,0\right\rbrace$ (PID: blue, MPC: purple) and $WP_1=\left\lbrace 3,2,5,0,0,0,180^{\circ}\right\rbrace$  (PID: orange, MPC: olive)}
\label{fig:static_waypoints_sim}
 \end{figure}
It can be seen that being a predictive controller type, the MPC shows significantly less overshoot than the PID controller framework. The large overshoot for $WP_{1}$ in the $y$-direction is due to the yaw rotation which is conducted by a separate control loop for both the cascaded PID controller framework and the MPC. This is why even the MPC can only react to the external disturbance as the yaw movement is not considered in the MPC formulation.
\begin{figure}[thpb]
      \centering
		\includegraphics[scale=0.45]{./pics/static_comparison_real_world.pdf}
\caption{Comparison between the performance of the PID-based controller framework and MPC in the real world. The waypoints are $WP_0=\left\lbrace 3,2,11,0,0,0,0\right\rbrace$ (PID: blue, MPC: purple) and $WP_1=\left\lbrace 3,2,11,0,0,0,180^{\circ}\right\rbrace$  (PID: orange, MPC: olive)}
\label{fig:static_waypoints_real}
 \end{figure}
 Figure \ref{fig:static_waypoints_real} illustrates the same scenario but in  the real world experiments. It can be seen that this time the MPC is not beneficial in comparison to the PID controller framework. However, this can be attributed to the windy conditions ($v_{wind} = 4\si{m/s}$ with considerably higher wind gusts possibly present when testing the MPC). Both controllers show slight oscillations when maintaining hover flight indicating the presence of wind gusts.
\subsubsection{Straight Line Trajectory}
The purpose of testing the tracking performance of the controllers for a target moving on a straight line is to evaluate the tracking error $e$. Against this background, moving targets were evaluated for different speeds as is summarized in Table \ref{tab:straight_line_evaluation}.
\begin{table}[]
\center
\small
\caption{ Mean tracking error in meters between drone and waypoint moving on a straight line for the different control frameworks and different target speeds in simulation.\label{tab:straight_line_evaluation}}
\begin{tabular}{@{}cccccc@{}}
\toprule
\thead{Speed$\rightarrow$\\ Contr.$\downarrow$} & $1\si{m/s}$ &  $2\si{m/s}$  &  $3\si{m/s}$ &  $4\si{m/s}$ & $5\si{m/s}$\\ 
\midrule 
        MPC &   0.16   &  0.36   &   0.59    & 0.82  & 1.41  \\
        PID &   1.32   &   2.67   &  4.07   &  5.32  &7.00\\
\bottomrule\\
\end{tabular}
\end{table}
It is clear that the MPC achieves a significantly better tracking accuracy, with the tracking error $e$ staying in the range of  $1\si{m}$ as compared to the  $6\si{m}$ of the PID.  Real world experiments were not conducted for this scenario due to safety issues with the restricted flight space and concerns about losing the Anafi drone out of sight. 
%This shows that the integral error states included in the MPC formulation are a suitable approach to reduce the tracking error thus making the MPC the better choice for tracking moving targets.
\subsubsection{Circular Trajectory} In this scenario the target moves in a circle at a fixed altitude with a constant yaw angle $\psi_{ref,world}=0\si{rad}$. Different radii and moving speeds were tested. This scenario was chosen to evaluate the tracking error $e$ under more complex conditions. The results are summarized in Table \ref{tab:circular_trajectory_evaluation_sim} for the experiments conducted in simulation and  the real world.

\setlength{\tabcolsep}{4pt}

%\begin{table}[]
%\center
%\small
%\caption{ Maximum tracking error $e_{max}$, mean tracking error $e_{\mu}$ and standard deviation $e_{\sigma}$ of the tracking error for different circular trajectories defined by the radius $r$ and the velocity $v$ of the tracked target for flights conducted in simulation and the real world. \label{tab:circular_trajectory_evaluation_sim}}
%\begin{tabular}{@{}cccccc@{}}
%\toprule
%\thead{\textcolor{black}{Param.$\rightarrow$} \\ \textcolor{black}{Contr.$\downarrow$}} & \textcolor{black}{$r [\si{m}]$} & \textcolor{black}{$v [\si{m/s}]$}  & \textcolor{black}{$e_{max} [\si{m}]$}  & \textcolor{black}{$e_{\mu} [\si{m}]$} & \textcolor{black}{$e_{\sigma} [\si{m}]$}  \\ 
%\midrule 
%\textcolor{black}{Simulation MPC} &  \textcolor{black}{ 1}     & \textcolor{black}{1}  &  \textcolor{black}{0.73}   &   \textcolor{black}{0.66}    &   \textcolor{black}{0.03}    \\
%\textcolor{black}{Simulation PID} &   \textcolor{black}{1}     & \textcolor{black}{1}  &  \textcolor{black}{1.38}    &   \textcolor{black}{1.32}    &   \textcolor{black}{0.03}   \\
%\textcolor{black}{Real world MPC} &   \textcolor{black}{1}     & \textcolor{black}{1}  & \textcolor{black}{ 0.72}    & \textcolor{black}{  0.57 }   &  \textcolor{black}{ 0.05}    \\
%\textcolor{black}{Real world PID} &   \textcolor{black}{1}     &\textcolor{black}{ 1}  & \textcolor{black}{ 1.69}    &   \textcolor{black}{1.29}    &  \textcolor{black}{ 0.21}   \\
%\midrule 
%\textcolor{black}{Simulation MPC} &   \textcolor{black}{2}     & \textcolor{black}{1}  &  \textcolor{black}{0.47}    &   \textcolor{black}{0.39}    &   \textcolor{black}{0.03}    \\
%\textcolor{black}{Simulation PID} &   \textcolor{black}{2}     & \textcolor{black}{1}  &  \textcolor{black}{1.37}    &   \textcolor{black}{1.25}    &   \textcolor{black}{0.05}   \\
%\textcolor{black}{Real world MPC} &   \textcolor{black}{2}     & \textcolor{black}{1}  & \textcolor{black}{ 0.61}   &   \textcolor{black}{0.44}    &  \textcolor{black}{0.07   } \\
%\textcolor{black}{Real world PID} &   \textcolor{black}{2}     & \textcolor{black}{1}  &  \textcolor{black}{1.54}   &   \textcolor{black}{1.33}    &  \textcolor{black}{0.07}   \\
%\midrule 
%\textcolor{black}{Simulation MPC} &   \textcolor{black}{2}     & \textcolor{black}{2}  & \textcolor{black}{ 1.63}    &   \textcolor{black}{1.44}    &   \textcolor{black}{0.10}    \\
%\textcolor{black}{Simulation PID }&   \textcolor{black}{2}     & \textcolor{black}{2}  &  \textcolor{black}{2.7}    &   \textcolor{black}{2.57}    &   \textcolor{black}{0.05}   \\
%%\midrule 
%%MPC &   2     & 3  &  3.34    &   3.34   &   0.13    \\
%%PID &   2     & 3  &  3.59    &   3.44    &   0.07   \\
%\textcolor{black}{Real world MPC} &   \textcolor{black}{2}     & \textcolor{black}{2}  & \textcolor{black}{ 1.39}    &   \textcolor{black}{1.11}    &   \textcolor{black}{0.12   } \\
%\textcolor{black}{Real world PID} &   \textcolor{black}{2}     & \textcolor{black}{2}  & \textcolor{black}{ 2.70}    &   \textcolor{black}{2.36 }   &   \textcolor{black}{0.09}   \\
%\midrule 
%\textcolor{black}{Simulation MPC} &   \textcolor{black}{3}     & \textcolor{black}{1}  &  \textcolor{black}{0.43}    &   \textcolor{black}{0.33}   &   \textcolor{black}{0.03}   \\
%\textcolor{black}{Simulation PID} &   \textcolor{black}{3}     & \textcolor{black}{1}  &  \textcolor{black}{1.49}     &   \textcolor{black}{1.33}    &   \textcolor{black}{0.06}   \\
%\textcolor{black}{Real world MPC} &   \textcolor{black}{3}     & \textcolor{black}{1}  & \textcolor{black}{ 0.59}    &   \textcolor{black}{0.35}   &   \textcolor{black}{0.07} \\
%\textcolor{black}{Real world PID} &   \textcolor{black}{3}     & \textcolor{black}{1}  & \textcolor{black}{ 1.56}    &  \textcolor{black}{ 1.36}   &   \textcolor{black}{0.07}   \\
%\midrule 
%\textcolor{black}{Simulation MPC} &   \textcolor{black}{3}     & \textcolor{black}{2}  & \textcolor{black}{ 1.37}    &   \textcolor{black}{1.19}   &   \textcolor{black}{0.07}    \\
%\textcolor{black}{Simulation PID} &  \textcolor{black}{ 3}     &\textcolor{black}{ 2}  &\textcolor{black}{  2.74}    &   \textcolor{black}{2.57}    &  \textcolor{black}{ 0.09}   \\
%\textcolor{black}{Real world MPC} &  \textcolor{black}{ 3}     & \textcolor{black}{2}  & \textcolor{black}{ 1.20}    &  \textcolor{black}{ 0.94 }  &   \textcolor{black}{0.09   } \\
%\textcolor{black}{Real world PID }&   \textcolor{black}{3}     & \textcolor{black}{2}  & \textcolor{black}{ 3.20}    &  \textcolor{black}{ 2.54 }  &   \textcolor{black}{0.27 }  \\
%\midrule 
%\textcolor{black}{Simulation MPC} &   \textcolor{black}{3}     & \textcolor{black}{3}  & \textcolor{black}{ 2.66}    &  \textcolor{black}{ 2.43}   &  \textcolor{black}{ 0.07}    \\
%\textcolor{black}{Simulation PID} &   \textcolor{black}{3}     &\textcolor{black}{ 3}  & \textcolor{black}{ 4.42}    &  \textcolor{black}{ 4.26}    &  \textcolor{black}{ 0.07}   \\
%\textcolor{black}{Real world MPC} &   \textcolor{black}{3}     & \textcolor{black}{3}  & \textcolor{black}{ 1.98}    &  \textcolor{black}{ 1.54 }   &   \textcolor{black}{0.14}    \\
%\textcolor{black}{Real world PID} &   \textcolor{black}{3}     & \textcolor{black}{3}  & \textcolor{black}{ 4.23}    &  \textcolor{black}{ 3.70  }  &   \textcolor{black}{0.23  } \\
%\bottomrule 
%\end{tabular} 
%\end{table}

\begin{table}[]
\center
\small
\caption{Maximum tracking error $e_{max}$, mean tracking error $e_{\mu}$, and standard deviation $e_{\sigma}$ of the tracking error for different circular trajectories defined by the radius $r$ and the velocity $v$ of the tracked target for flights conducted in simulation and the real world. \label{tab:circular_trajectory_evaluation_sim}}
\begin{tabular}{c cc ccc ccc}
\toprule
{Controller} & \multicolumn{2}{c}{{Circle}} & \multicolumn{3}{c}{{Simulation}} & \multicolumn{3}{c}{{Real World}} \\ 
\cmidrule(lr){1-1} \cmidrule(lr){2-3} \cmidrule(lr){4-6} \cmidrule(lr){7-9}
Name & $r [\si{m}]$ & $v [\si{m/s}]$ & $e_{max} [\si{m}]$ & $e_{\mu} [\si{m}]$ & $e_{\sigma} [\si{m}]$ & $e_{max} [\si{m}]$ & $e_{\mu} [\si{m}]$ & $e_{\sigma} [\si{m}]$ \\ 
\midrule 
MPC & 1 & 1 & 0.73 & 0.66 & 0.03 & 0.72 & 0.57 & 0.05 \\ 
PID & 1 & 1 & 1.38 & 1.32 & 0.03 & 1.69 & 1.29 & 0.21 \\ 
\midrule 
MPC & 2 & 1 & 0.47 & 0.39 & 0.03 & 0.61 & 0.44 & 0.07 \\ 
PID & 2 & 1 & 1.37 & 1.25 & 0.05 & 1.54 & 1.33 & 0.07 \\ 
\midrule 
MPC & 2 & 2 & 1.63 & 1.44 & 0.10 & 1.39 & 1.11 & 0.12 \\ 
PID & 2 & 2 & 2.70 & 2.57 & 0.05 & 2.70 & 2.36 & 0.09 \\ 
\midrule 
MPC & 3 & 1 & 0.43 & 0.33 & 0.03 & 0.59 & 0.35 & 0.07 \\ 
PID & 3 & 1 & 1.49 & 1.33 & 0.06 & 1.56 & 1.36 & 0.07 \\ 
\midrule 
MPC & 3 & 2 & 1.37 & 1.19 & 0.07 & 1.20 & 0.94 & 0.09 \\ 
PID & 3 & 2 & 2.74 & 2.57 & 0.09 & 3.20 & 2.54 & 0.27 \\ 
\midrule 
MPC & 3 & 3 & 2.66 & 2.43 & 0.07 & 1.98 & 1.54 & 0.14 \\ 
PID & 3 & 3 & 4.42 & 4.26 & 0.07 & 4.23 & 3.70 & 0.23 \\ 
\bottomrule 
\end{tabular} 
\end{table}










The MPC shows better performance than the PID across all scenarios, although at higher velocities the advantage of the MPC diminishes. The reason is that in these scenarios the  control commands produced by the MPC are bound by the maximum attitude angles of the drone thus limiting its manueverability. %For this reason, only a maximum velocity of $v=1\si{m/s}$  was tested tested for a circle with a radius of $r=1\si{m}$. 
The real world experiments confirm the simulation results, although they were conducted under windy conditions subjecting the drone to heavy disturbances. However, the standard deviation of the tracking error $e_{\sigma}$ of the MPC is considerably lower than than the one of the PID controller framework thus indicating that the tracking is managed in an evenly way.

\subsubsection{Vertical Rectilinear Periodic Trajectory}
The tracking capability in vertical direction is verified for a target conducting a rectilinear periodic trajectory (RPT) in vertical direction only. Different radii and velocities of the RPT are tested. 
Table \ref{tab:rpm_trajectory_evaluation_sim} summarizes the results for the experiments conducted in simulation and the real world.
\begin{table}[]
\center
\small
\caption{Maximum tracking error $e_{max}$, mean tracking error $e_{\mu}$, and standard deviation $e_{\sigma}$ of the tracking error for different rectilinear periodic trajectories (RPT) defined by the radius $r$ and the velocity $v$ of the tracked target for flights conducted in simulation and the real world. \label{tab:rpm_trajectory_evaluation_sim}}
\begin{tabular}{c cc ccc ccc}
\toprule
{Controller} & \multicolumn{2}{c}{RPT} & \multicolumn{3}{c}{Simulation} & \multicolumn{3}{c}{Real World} \\ 
\cmidrule(lr){1-1} \cmidrule(lr){2-3} \cmidrule(lr){4-6} \cmidrule(lr){7-9}
Name & $r [\si{m}]$ & $v [\si{m/s}]$ & $e_{max} [\si{m}]$ & $e_{\mu} [\si{m}]$ & $e_{\sigma} [\si{m}]$ & $e_{max} [\si{m}]$ & $e_{\mu} [\si{m}]$ & $e_{\sigma} [\si{m}]$ \\ 
\midrule 
MPC & 1 & 1 & 0.36 & 0.21 & 0.10 & 0.64 & 0.31 & 0.11 \\ 
PID & 1 & 1 & 0.96 & 0.60 & 0.29 & 1.08 & 0.68 & 0.26 \\ 
\midrule 
MPC & 2 & 1 & 0.25 & 0.13 & 0.06 & 0.37 & 0.18 & 0.07 \\ 
PID & 2 & 1 & 1.07 & 0.65 & 0.32 & 1.12 & 0.65 & 0.30 \\ 
\midrule 
MPC & 2 & 2 & 1.08 & 0.51 & 0.28 & 1.27 & 0.72 & 0.31 \\ 
PID & 2 & 2 & 1.91 & 1.20 & 0.57 & 2.10 & 1.31 & 0.54 \\ 
\bottomrule
\end{tabular} 
\end{table}



Similar to the circular trajectories, the results show that the prediction capability of the MPC  significantly improves the tracking performance compared to the PID controller framework. The reason why the tracking error is even more reduced by the MPC compared to the PID-based controller framework and also compared to the circular trajectory is that during the RPT the drone only moves in the vertical direction instead of the  longitudinal and lateral direction simultaneously, thus constituting a simpler motion. Both controllers perform slightly worse in the real world than in simulation although for the MPC the loss in performance is less pronounced. However, similar to the circular trajectories, the standard deviation of the tracking error $e_{\sigma}$ is considerably lower than the one of the PID controller framework across all scenarios indicating that the tracking performance of the MPC is maintaining a  more consistent quality.














