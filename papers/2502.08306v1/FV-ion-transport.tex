\documentclass[a4paper,oneside,9pt]{article}
\usepackage{blindtext}
\usepackage{authblk}
\usepackage{newclude}
\usepackage{graphicx}
\usepackage{float}
\usepackage{wrapfig}
\usepackage[utf8]{inputenc}
%\usepackage[ngerman]{babel}
\usepackage{amsmath}
\usepackage{hyperref}
\usepackage{amsfonts, amssymb, amsthm}
\usepackage{tikz}
\usepackage{scalerel}
\usetikzlibrary{angles,patterns,calc}
\usepackage{mathtools}
\usepackage{stmaryrd}
\usepackage[figure]{hypcap}
\usepackage{etoolbox}
\usepackage{enumerate}
\usepackage{bbm}
\usepackage{esint}
\usepackage{url}
\usepackage{hyperref}
\usepackage[hyphenbreaks]{breakurl}
\usepackage[a4paper, left=2cm, right=2cm, top=2cm, bottom=2cm]{geometry}
\usepackage[sorting=nty,citestyle=numeric-comp,bibstyle=ieee,maxbibnames=99, backend=bibtex, url=false, isbn=false, date=year, doi=false,eprint=true, natbib=true]{biblatex}
\usepackage{fancyhdr}
\addbibresource{references.bib}
\usepackage{yhmath}
\renewcommand*{\bibfont}{\normalfont\footnotesize}

\newcommand{\B}{\mathcal{B}}
\newcommand{\h}{\mathcal{H}}
\newcommand{\uh}{\mathbb{H}}
\newcommand{\D}{\mathbb{D}}
\newcommand{\Mult}{\operatorname{Mult}}
\newcommand{\N}{\mathbb{N}}
\newcommand{\R}{\mathbb{R}}
\newcommand{\C}{\mathbb{C}}
\newcommand{\Z}{\mathbb{Z}}
\newcommand{\M}{\mathcal{M}}
\newcommand{\norm}[1]{\left\|{#1}\right\|}
\newcommand{\abs}[1]{\left|{#1}\right|}
\newcommand{\dc}[1]{\mathopen{\scalebox{1.0}{$\lbrace\!\mkern-1mu\lbrace$}}#1\mathclose{\scalebox{1.0}{$\rbrace\!\mkern-1mu\rbrace$}}}
\newcommand{\jump}[1]{\ensuremath{\left\llbracket #1\right\rrbracket}}
\newcommand{\keywords}[1]{\textbf{Keywords:} #1}

\newtheorem{defn}{Definition}[section]
\newtheorem{thm}[defn]{Theorem}
\newtheorem{exm}[defn]{Example}
\newtheorem{nota}[defn]{Notation}
\newtheorem{lemma}[defn]{Lemma}
\newtheorem{cor}[defn]{Corollary}
\newtheorem{prop}[defn]{Proposition}
\newtheorem{remark}[defn]{Remark}
\newtheorem{ass}[defn]{Assumption}

\title{A posteriori error control for a finite volume scheme for a cross-diffusion model of ion transport}
\date{}
\begin{document}
	\author[1]{Arne Berrens\thanks{Corresponding author; berrens@mathematik.tu-darmstadt.de}}
	\author[1]{Jan Giesselmann\thanks{giesselmann@mathematik.tu-darmstadt.de}}
	\affil{\centering Department of Mathematics\\ Technical University of Darmstadt\\ Dolivostr. 15, 64293 Darmstadt, Germany}
	\maketitle
\begin{abstract}
	We derive a reliable a posteriori error estimate for a cell-centered finite volume scheme approximating a cross-diffusion system modeling ion transport through nanopores.
	To this end we derive an abstract stability framework that is independent of the numerical scheme and introduce a suitable (conforming) reconstruction of the numerical solution.
	The stability framework relies on some simplifying assumption that coincide with those made in weak uniqueness results for this system.
	This is the first a posteriori error estimate for a cross-diffusion system.
	Along the way, we derive a pointwise a posteriori error estimate for a finite volume scheme approximating the diffusion equation.
	We conduct numerical experiments showing that the error estimator scales with the same order as the true error.
\end{abstract}
	\keywords{cross-diffusion; ion transport; finite-volume approximation; a posteriori error estimates; diffusion equation}\\
	\textbf{AMS subject classifications} (2020): 65M15, 35K65, 35K51, 65M08, 35K40
\section{Introduction}
\subsection{Ion Transport Model}
\"a
We consider the ion transport model presented in \textcite{burger2012}.
Similar to the Poisson-Nernst-Planck equations given by \textcite{nernst1888} it models ionic species in a solvent without charge.
In contrast to the Poisson-Nernst-Planck equations, the ion transport model incorporates forces due to finite size of every particle.
This means in essence, that the transport of ions is affected by size exclusion effects.
It comes into play when modeling ion transport through nanopores or membranes, i.e. when the size of an individual ion is not negligible compared to the channel size.
Hence, due to size exclusion different ionic species influence their diffusion and Fick's law is no longer true in small geometries (see \cite{simpson2009,burger2012,massimini2024} and references therein).
Therefore, the ion transport model incorporates cross-diffusion terms other than the classical Poisson-Nernst-Planck equations.
Furthermore, the ion transport model incorporates the solvent concentration and the advection with it, i.e. the ion particles are dragged along with the movement of the solvent.
Incorporating the solvent concentration leads to the volume filling condition, namely that the volume fraction of the solvent is the remaining volume that is not occupied by the ions.
This is modeled in the second equation in \eqref{equations}.\\
In general, cross-diffusion systems are analytically and numerically challenging, since classical methods, like maximum or comparison principle are not applicable (see \cite{jungel2016,stara1995, massimini2024}).\\
The ion transport model includes the ion concentrations $(v_1,\dots,v_n)$, solvent concentration $v_0$ and electric potential $\Psi$ in a Lipschitz domain $\Omega\subset\R^d$.
The ion and solvent concentrations fulfill the equations for some $T>0$
	\begin{align}\label{equations}
		\begin{aligned}
			\partial_t v_i &=  D_i\text{div}(v_0\nabla v_i-v_i\nabla v_0+v_0v_i\beta z_i\nabla \Psi)\quad i=1,\dots,n\\
			v_{0} &= 1-\sum_{i=1}^n v_i,
		\end{aligned}
	\quad\text{in }\Omega\times(0,T)
	\end{align}
where $D_1,\dots,D_n>0$ are the diffusion coefficients, $\beta>0$ is the inverse thermal voltage and $z_i\in\R$ the charge of each particle of the $i$th species.
The electric potential satisfies
\begin{align}\label{eqn:phi}
	-\lambda^2\Delta\Psi = \sum_{i=1}^n z_iv_i+f \quad\text{in }\Omega\times(0,T),
\end{align}
where $\lambda>0$ is the permittivity, $f\in L^\infty(\Omega)$ is the permanent charge density.
We fix a decomposition $\Gamma_D,\Gamma_N$ of $\partial\Omega$ and prescribe
\begin{align}\label{boundaryCond}
	\begin{aligned}
	\left(v_0\nabla v_i-v_i\nabla v_0+v_0v_i\beta z_i\nabla\Psi\right)\cdot n &= 0 \text{ on } \Gamma_N,\quad v_i = v_i^D \text{ on } \Gamma_D,\quad i=1,\dots,n\\
	\nabla\Psi\cdot n &= 0 \text{ on } \Gamma_N,\quad\, \Phi = \Psi^D\text{ on } \Gamma_D
	\end{aligned}
\end{align}
and consider
\begin{align}\label{eqn:initial}
	v_i(\cdot,0) = v_i^0\quad\text{in }\Omega,\; i=0,\dots,n
\end{align}
for given functions $v_i^D,\Psi^D\in L^2(\Gamma_D)$ and $v_i^0\in L^\infty(\Omega)$.
We further assume, that $v_i^D$ and $\Psi^D$ are traces of functions in $L^\infty(\Omega)\cap H^1(\Omega)$.
We will discuss the regularity of the solution in Theorem \ref{thm:existence}.\\
The main goal of this work is to establish an a posteriori error estimate for a finite volume scheme approximating the system \eqref{equations}-\eqref{eqn:initial} under some simplifying assumption (see Assumptions \ref{assumptions}).
With these assumptions we can state an abstract error estimate, that is independent of the numerical scheme as long as the assumptions on the regularity of the numerical solution are fulfilled.
This is a first step towards a posteriori error analysis of cross diffusion system.
There are numerical schemes for cross-diffusion systems \cite{barrett2004,burger2020,jungel2021} and specifically for the ion transport model \cite{cances2019, gerstenmayer2018, gerstenmayerFiniteelement2018,cances2019}.
In \cite{gerstenmayerFiniteelement2018} the a priori convergence for a finite element scheme is shown and a priori convergence for a finite volume scheme is proven in \cite{cances2019} for a reduced model and in \cite{massimini2024} for the full model.
In this paper, we initiate the previously missing a posteriori analysis.
We use a cell centered finite volume scheme found in \cite{gerstenmayer2018}.
Finite volume schemes are commonly used for cross-diffusion systems, since it is easier to prove some important properties of the resulting numerical approximation like non-negativity and entropy dissipation compared to finite element schemes.
Therefore, we consider a finite volume scheme although it is usually harder to derive a posteriori error estimates for finite volume scheme than for finite element schemes.\\
Up to our knowledge there are no a posteriori error estimates for finite volume schemes approximating general non-linear parabolic systems with positive diffusion matrix.
In contrast, there are a posteriori error estimates for finite element schemes for non-linear parabolic systems see \cite{verfurth1998,sutton2020} and even ones allowing for degenerate diffusion coefficient in \cite{cances2020}.\\
In \cite{sutton2020} nonlinear diffusion systems are studied with symmetric and positive definite diffusion matrices that do only depend on space and time and not on the species $v_1,\dots,v_n$.
In \cite{verfurth1998} more general nonlinear parabolic systems are considered that also allow for species dependent diffusion matrices that are still symmetric and positive definite.
\textcite{cances2020} consider the nonlinear anisotropic Fokker-Planck equation for only one species.
\textcite{cochez2009} proved an a posteriori error estimate for a finite volume method approximating a non-linear steady-state diffusion problem.
\subsection{The Model}
The cross-diffusion system \eqref{equations} was analytically studied in \cite{jungel2015,jungel2016,gerstenmayer2018,burger2012,chen2018}.
The existence of weak solutions was shown in \cite{burger2012} for the stationary problem and in \cite{gerstenmayer2018,jungel2016,jungel2015} for the time dependent problem.
Uniqueness of weak solutions was established in \cite{gerstenmayer2018,jungel2016,chen2018}.
Furthermore, for long times every solution converges exponentially to a steady state solution (\cite{jungel2016}).\\
In \cite{jungel2016,massimini2024}, System \eqref{equations} is written in the form
\begin{align*}
	\partial_t v_i = \operatorname{div}\left(\sum_{j=1}^n\left(A_{i,j}(v)\nabla v_j+Q_{i,j}(v)\nabla\Psi\right)\right),\quad i=1,\dots,n
\end{align*}
with $v = (v_1,\dots,v_n)$ and $A(v),Q(v)\in\R^{n\times n}$
\begin{align*}
	(A(v))_{i,j} = 
	\begin{cases}
		D_i\left(1-\sum_{\substack{k=1\\i\neq k}}^n v_k\right) &\text{if } i=j\\
		D_iv_i &\text{if } i\neq j
	\end{cases},\quad Q_{i,j}(v)=v_0v_i\delta_{i,j}.
\end{align*}
The matrix $A(v)$ is called the diffusion matrix.
\textcite{gerstenmayer2018} proved a global existence of a weak solution by a compactness argument via the entropy variables.
In what follows, we write $\abs{\cdot}$ for the $d$-dimensional or $d-1$-dimensional Lebesgue measure.
For $\abs{\Gamma_D}>0$ we define the spaces
\begin{align*}
	H_D^1(\Omega) := \{u\in H^1(\Omega)\,:\,u = 0\quad\text{on }\Gamma_D\}
\end{align*}
and $H^1_D(\Omega)'$ the dual space of $H^1_D(\Omega)$.
We equip the space $H_D^1(\Omega)$ in the following with the norm
\begin{align*}
	\norm{\cdot}_{H^1_D(\Omega)} = \norm{\nabla \cdot}_{L^2(\Omega)}.
\end{align*}
This norm is equivalent to the classical $H^1$ norm on $H^1_D$ via the general Poincar\'e-Friedrichs inequality Theorem \ref{thm:friedirchsineq}.
We denote $\langle\cdot,\cdot\rangle$ the duality pairing of $H^1_D(\Omega)'$ and $H^1_D(\Omega)$.
The next theorem states the existence of a global bounded weak solutions to \eqref{equations} under suitable conditions.
\begin{thm}{\cite[Theorem 1]{gerstenmayer2018}}\label{thm:existence}
	Let $\Gamma_N\cap\Gamma_D=\emptyset$ with $\Gamma_N$ open and $|\Gamma_D|>0$.
	For all $T>0$, there exists a global bounded weak solution $v_0,\dots,v_n:(0,T)\times\Omega\to[0,1], \Psi:(0,T)\times\Omega\to\R$ of \eqref{equations}-\eqref{eqn:initial} with $\sum_{i=0}^nv_i=1$ such that
	\begin{align*}
		v_i\sqrt{v_0},\sqrt{v_0}&\in L^2(0,T;H^1(\Omega)),\quad \partial_t v_i\in L^2(0,T;H^1_D(\Omega)')\quad i=1,\dots,n,\\
		\Psi&\in L^2(0,T;H^1(\Omega))
	\end{align*}
	and
	\begin{align*}
		\int_0^T \langle\partial_t v_i,\phi_i\rangle dt+D_i\int_0^T\int_\Omega \left(v_0\nabla v_i-v_i\left(\nabla v_0+v_0\beta z_i\nabla\Psi\right)\right)\cdot\nabla \phi_i dxdt &= 0\\
		\lambda^2\int_0^T\int_\Omega \nabla \Psi\cdot\nabla\theta dxdt - \int_0^T\int_\Omega\left(\sum_{i=1}^n z_iv_i+f\right)\theta dxdt &=0
	\end{align*}
	for all $\phi_i,\theta\in L^2(0,T;H^1_D(\Omega))$, $i=1,\dots,n$.
	The boundary conditions \eqref{boundaryCond} are fulfilled in the sense of traces in $L^2(\Gamma_D)$.
\end{thm}
We assume later on (see Assumptions \ref{assumptions}), that $v_0$ is bounded away from $0$.
In this case, according to \cite[Remark 4]{gerstenmayer2018}, $v_i\in L^2(0,T;H^1(\Omega))$ for all $i=0,\dots,n$.\\
Similar to the uniqueness proof in \cite{gerstenmayer2018} and the convergence proof of \cite{cances2019}, we need the assumption $D_i=D$ and $z_i=z$ for $i=1,\dots,n$ and further assume for simplicity $D=1$ with same boundary \eqref{eqn:initial} and initial \eqref{boundaryCond} conditions.
In this case the equations
\begin{align}\label{eqn:general}
	\begin{split}
		\partial_t v_i &= \text{div}(v_0\nabla v_i-v_i\nabla v_0+v_0v_i\beta z\nabla \Psi)\quad i=1,\dots,n\\
		v_{0} &= 1-\sum_{i=1}^n v_i\\
		-\lambda^2\Delta\Psi &= z(1-v_0)
	\end{split}\quad\text{in }\Omega\times(0,T)
\end{align}
decouple in the sense that the solvent concentration $v_0$ solves
\begin{align}\label{eqn:solvent}
	\partial_t v_0=\operatorname{div}(\nabla v_0-v_0(1-v_0)z\beta\nabla \Psi).
\end{align}
We call \eqref{eqn:general} the general model for convenience.
Similar to \cite{cances2019,jungel2016}, we first consider the reduced model
\begin{align}\label{eqn:reduced}
	\begin{split}
		\partial_t v_i&=\operatorname{div}(v_0\nabla v_i-v_i\nabla v_0)\quad i=1,\dots,n\\
		v_0 &= 1-\sum_{i=1}^nv_i
	\end{split}\quad\text{in }\Omega\times(0,T)
\end{align}
obtained by setting $z=0$.
In the reduced model, the solvent concentration solves the diffusion equation
\begin{align}\label{eqn:solvent_reduced}
	\partial_t v_0 = \text{div}(\nabla v_0) = \Delta v_0\quad\text{in }\Omega\times(0,T).
\end{align}
\subsection{Main results and proof strategy}
\subsubsection{Main results}
We study a cell centered finite volume scheme for the system \eqref{eqn:general} and derive an a posteriori error estimate. 
For this we need a reconstruction of the finite volume solution denoted by $(\widehat{u_0},\dots,\widehat{u_n},\widehat{\Phi})$.
Let $(v_0,\dots,v_n,\Psi)$ be a weak solution to the system \eqref{eqn:general} in the sense of Theorem \ref{thm:existence} satisfying the following assumptions.
\begin{ass}\label{assumptions}\hfill
	\begin{enumerate}[(H1)]
		\item The domain $\Omega\subset\R^d$ is an open bounded domain with polygonal boundary.
			Furthermore, let $\Gamma_D,\Gamma_N$ be disjoint with $\partial\Omega=\Gamma_D\cup\Gamma_N$, $\Gamma_N$ open and $\abs{\Gamma_D}>0$.
		\item	Let $v_0^0,\dots,v_n^0\in L^\infty(\Omega)\cap H^1(\Omega)$ with $v_i^0>0$ for all $i=1,\dots,n$ and $\sum_{i=0}^n v_i^0 = 1$ on $\Omega$.
		\item Assume that there exists $\gamma>0$ such that 
			\begin{align*}
				\gamma &< v_0^0(x) \leq 1\quad\forall x\in\Omega,\quad\text{if $z=0$ and }\\
				\gamma &< v_0(x,t)\leq1\quad\forall x\in\Omega\text{ and }t\in[0,T] \quad\text{otherwise.}
			\end{align*}
	\end{enumerate}
\end{ass}
\begin{remark}
	\begin{enumerate}[(i)]
		\item The assumptions (H1) and (H2) guarantee that there exists a weak solution to \eqref{equations}-\eqref{eqn:initial} according to Theorem \ref{thm:existence}.
		\item Assumption (H3) ensures that the solution $v_0$ is always bounded from below by $\gamma$.
			This is important, because equation \eqref{equations} degenerates for $v_0=0$ and we do not get a bound on the error of the gradient in this case.
			However, $v_0$ models the solvent concentration and therefore it is physically meaningful to assume that $v_0$ is bounded away from $0$.
	\end{enumerate}
\end{remark}
The next theorem states the a posteriori error estimate for the general model \eqref{eqn:general}.
The constants and estimators showing up in the next theorem are made explicit in Section \ref{section:residual}.
\begin{thm}\label{thm:main_estimator}
	Let Assumption \ref{assumptions} hold for a weak solution $(v_0,\dots,v_n,\Psi)$ of \eqref{eqn:general} in the sense of Theorem \ref{thm:existence} and let $(\widehat{u_0},\dots,\widehat{u_n},\widehat{\Phi})$ be the reconstruction of a finite volume solution in the sense of Section \ref{section:scheme}.
	\begin{enumerate}[(i)]
		\item The difference $\widehat{u_0}-v_0$ satisfies
			\begin{multline*}
				\max_{t\in[0,T]}\norm{\widehat{u_0}-v_0}_{\Omega}^2+\norm{\nabla (\widehat{u_0}-v_0)}_{[0,T]\times\Omega}^2 \leq\\
				\left(2\norm{\widehat{u_0}^0-v_0^0}_{\Omega}^2+\sum_{j=0}^J\tau_j\left(\frac{\abs{z\beta}^2}{4}\left(\eta_{R,\Phi}^j\right)^2+4\left(\eta_{R,0}^j\right)^2\right)\right)\\
				\times\exp\left(2C_G^{\frac{2}{1-\theta}}(1+C_{F,2,\Gamma_D}^2)^{\frac{\theta}{1-\theta}}\mu\norm{\nabla \widehat{\Phi}}_{X(q)}^{\frac{2}{1-\theta}}+C_{F,2,\Gamma_D}^2\frac{\abs{z}^4\beta^2}{8\lambda^4}\right)=:\eta_2^J.
			\end{multline*}
		\item The $L^2$-in-time, $H^1$-in-space seminorm of $\widehat{\Phi}-\Psi$ is bounded by
			\begin{align*}
				\norm{\nabla (\widehat{\Phi}-\Psi)}_{[0,T]\times\Omega}^2 
				\leq C_{F,2,\Gamma_D}^2\frac{\abs{z}}{\lambda^2}\eta_2^J+\sum_{j=0}^J \tau_j\left(\eta_{R,\Phi}^j\right)^2.
			\end{align*}
		\item The error $\widehat{u_i}-v_i$ satisfies
			\begin{multline}
				\max_{t\in[0,T]} \norm{\widehat{u_i}-v_i}_{\Omega}^2+\norm{\sqrt{v_0}\nabla(\widehat{u_i}-v_i)}_{[0,T]\times\Omega}^2
				\leq
				\left(2\norm{\widehat{u_i}^0-v_i^0}^2_{\Omega}\right.\\
				+\frac{12}\gamma\eta_2^J\left(\norm{\nabla \widehat{u_i}}_{L^\infty(0,T;L^{\tilde{q}}(\Omega))}^2+2\left((1+C_S\sqrt{1+C_{F,2,\Gamma_D}})\abs{z\beta}\norm{\nabla \widehat{\Phi}}_{L^\infty(0,T;L^{\tilde{q}}(\Omega))}+C_{F,2,\Gamma_D}\frac{z^2\beta}{\lambda^2}\right)^2\right)\\
				\left.+\frac{12}\gamma C_{F,2,\Gamma_D}^2\sum_{j=0}^J\tau_j\left(\abs{z\beta}^2\left(\eta_{R,\Phi}^j\right)^2+\left(\eta_{R,i}^j\right)^2\right) \right)
				\exp\left(2C_G^{\frac{2}{1-\theta}}(1+C_{F,2,\Gamma_D}^2)^{\frac{\theta}{1-\theta}}\frac{\mu}{\gamma^{\frac{1+\theta}{1-\theta}}}\norm{F}_{X(q)}^{\frac{2}{1-\theta}}\right).
			\end{multline}
	\end{enumerate}
	With $F=\nabla \widehat{u_0}-\widehat{u_0}z\beta\nabla\widehat{\Phi}$ the constants $\theta = \frac d2-\frac dp$ and $\mu= \frac{1-\theta}{2}\left(\frac{1}{2(1+\theta)}\right)^\frac{\theta+1}{\theta-1}$.
\end{thm}
We also derive an a posteriori error estimate for the reduced model \eqref{eqn:reduced} in Theorem \ref{thm:error_estimator}.
For this, we first consider the solvent concentration $v_0$ that solves the diffusion equation \eqref{eqn:solvent_reduced}.
We establish a posteriori error estimates for $\widehat{u_0}-v_0$ in the $L^\infty([0,T]\times\Omega)$ norm under a further assumption on the domain $\Omega$ (see Assumption \ref{ass:uniform}) and for $\nabla (\widehat{u_0}-v_0)$ in the $L^2([0,T]\times\Omega)$ norm.\\
For the maximum norm error we use a similar technique to \cite{kyza2018,demlow2012,demlow2009}.
Such a maximum norm a posteriori error estimate is new for any finite volume scheme for the diffusion equation \eqref{eqn:solvent_reduced} and we think it is interesting in its own right.
Similar estimates have been established for various finite element schemes see \cite{demlow2009,demlow2012}.\\
\subsubsection{Proof strategy}
To derive the a posteriori error estimates, we state abstract stability frameworks for the ion concentrations first for the reduced model \eqref{eqn:reduced} and then for the general model \eqref{eqn:general}.
Both stability frameworks are independent of the numerical scheme and reconstruction used.
We only assume that $(u_0,\dots,u_n,\Phi)$ is a weak solution of the perturbed system
\begin{align}\label{eqn:fullperturbed}
	\begin{aligned}
	\partial_t u_i-\operatorname{div}(u_0\nabla u_i-u_i\nabla u_0+u_0u_i\beta z\nabla\Phi)&=R_i\quad i=1,\dots,n\\
	\sum_{i=0}^n u_i&= 1\\
	-\lambda^2\Delta\Phi -z(1-u_0)&=R_\Phi
	\end{aligned}
	\quad\text{on }\Omega\times(0,T)
\end{align}
with $R_1,\dots,R_\Phi\in L^2(0,T;H^1_D(\Omega)')$ and $u_0,\dots,u_n,\Phi$ satisfying some additional assumptions (see Assumption \ref{ass:perturbed}).
In this paper, the solution to the perturbed system is the reconstruction $\widehat{u_0},\dots,\widehat{u_n},\widehat{\Phi}$ given in Section \ref{section:scheme} and satisfies Assumption \ref{ass:perturbed} automatically.
Using the abstract stability estimate, Theorem \ref{thm:abstract_general}, for the general model, we only need to bound the residuals $R_0,\dots,R_n,R_\Phi$ in the $L^2(0,T;H^1_D(\Omega)')$ norm to obtain a reliable a posteriori error estimate for the model \eqref{eqn:general}.
For the reduced model we need to combine the estimates for the solvent concentration from Section \ref{section:heat} and the bounds for the residuals $R_0,\dots,R_n,R_\Phi$ in the $L^2(0,T;H^1_D(\Omega)')$ norm to get a reliable a posteriori error estimate via the abstract stability framework for the reduced model (see Theorem \ref{thm:abstract_error}).
This is done in Section \ref{section:residual} and relies mostly on the properties of the reconstruction.\\
We expect that the error $\widehat{u_i}-v_i$ converges linearly in the $L^2(0,T;H^1(\Omega))$ norm, since $\widehat{u_i}$ is manly composed of piecewise linear polynomials (see Section \ref{subsec:Morley}).
This linear convergence of the a posteriori estimator and error is observed in Section \ref{section:numerics}.\\
With Assumption (H3) in Assumption \ref{assumptions}, the diffusion matrix $A(u)$ becomes positive definite.
This makes it possible to use the stability framework from Section \ref{section:abstract_error} instead of the Gajewski metric or relative entropy (see Remark \ref{Whyl2}).
In this setting it is easier to obtain a reliable upper bound for the error of our finite volume scheme.\\
The paper is structured as follows: In Section \ref{section:ineq}, we introduce the triangular mesh and some inequalities used throughout this paper.
Next in Section \ref{section:scheme}, we present a cell centered finite volume scheme for the full system and a reconstruction of the finite volume solution similar to \cite{nicaise2005,nicaise2006}.\\
In Section \ref{section:heat}, we prove the bounds on $u_0-v_0$ and $\nabla(u_0-v_0)$ needed in the error estimator for the reduced model presented in Section \ref{section:abstract_error}.
After that, we present the stability framework for the reduced model \eqref{eqn:reduced} and the general model under the assumptions in Assumptions \ref{assumptions}.
The bounds for the residuals $R_0,\dots,R_n,R_\Phi$ in the $L^2(0,T;H^1_D(\Omega)')$ and the a posteriori error estimator for the general and reduced system are stated in Section \ref{section:residual}.
In Section \ref{section:numerics}, the a posteriori error estimators are tested in numerical experiments.
\section{Notation and Inequalities}\label{section:ineq}
\subsection{Notation}
We use a classical definition of a regular admissible finite volume mesh of $\Omega$ found in \cite[Definition 9.1]{eymard2019} .
The definition stated only allows for simplices, but the scheme in Section \ref{subsec:FV} and the reconstruction in Section \ref{subsec:Morley} can be extended to rectangles in $d=2$ similar to \cite{nicaise2005}.
However, the reconstructions done in Section \ref{subsec:Morley} can be easily extended to rectangles and tetrahedra using the constructions in \cite{nicaise2005,nicaise2006}.
\begin{defn}{\cite[Def. 9.1]{eymard2019}}\label{defn:mesh}
	Let $\mathcal{T}$ be a set of simplices, $\mathcal{E}$ a family of edges with each $\sigma\in\mathcal{E}$ beeing a subset of $\overline{\Omega}$ and contained in a hyperplane of $\R^d$ and $\mathcal{P}$ a set of points in $\Omega$.
	The finite volume mesh $(\mathcal{T},\mathcal{E},\mathcal{P})$ is admissible if the following conditions are satisfied.
	\begin{enumerate}[(i)]
		\item $\bigcup_{K\in\mathcal{T}} \overline{K} = \overline{\Omega}$.
		\item For every $K,L\in\mathcal{T}$ with $K\neq L$ the $(d-1)$-dimensional Lebesgue measure of $\overline{K}\cap\overline{L}$ is 0 or there is $\sigma\in\mathcal{E}$ such that $\overline{\sigma}=\overline{K}\cap\overline{L}$.
		\item For every $K\in\mathcal{T}$, there is a subset $\mathcal{E}_K\subset\mathcal{E}$ such that $\partial K = \bigcup_{\sigma\in\mathcal{E}_K} \overline{\sigma}$ and $\mathcal{E} = \bigcup_{K\in\mathcal{T}}\mathcal{E}_K$.
		\item The family has the form $\mathcal{P} = (x_K)_{K\in\mathcal{T}}$ with $x_K\in \overline{K}$ and $x_K\neq x_L$ for cells $K\neq L$. Furthermore, for cells $K,L\in\mathcal{T}$ that share an edge $\overline{\sigma} = \overline{K}\cap\overline{L}$ the straight line from $x_K$ to $x_L$ is orthogonal to $\sigma$.
		\item Let $\sigma\in\mathcal{E}$ with $\sigma\subset\partial \Omega$.
			Then $\sigma\subset\Gamma_N$ or $\sigma\subset\Gamma_D$.
			We further divide the set of edges into
			\begin{align*}
				\mathcal{E} = \mathcal{E}^i\cup\mathcal{E}^N\cup\mathcal{E}^D,
			\end{align*}
			where $\mathcal{E}^i$ are the inner edges, $\mathcal{E}^N$ are the edges on the Neumann boundary and $\mathcal{E}^D$ are the edges on the Dirichlet boundary.
	\end{enumerate}
\end{defn}
In Definition \ref{defn:mesh}, assumption (iv) can be weakened if we use a different flux see Remark \ref{remark:scheme}.\\
We denote by $n_{K,\sigma}$ for the unit outer normal vector of $K$ on $\sigma\in\mathcal{E}_K$ and $\{a_i^K\,:\,i=0,\dots,d\}\subset\overline{K}$ the corners of $K$.
Furthermore, we denote $N_\partial := \max_{K\in\mathcal{T}}\abs{\mathcal{E}_K}$.\\
The jump of a sufficiently regular function $f$ across an edge $\sigma\in\mathcal{E}$ is given by
\begin{align*}
	\llbracket f\rrbracket(x) := \begin{cases}
									f|_K(x)\cdot n_{K,\sigma}+f|_L(x)\cdot n_{L,\sigma}&\text{ if }\sigma=\partial K\cap\partial L\\
									f|_K(x)\cdot n_{K,\sigma} &\text{ if } \sigma\subset\partial\Omega
								\end{cases}
\end{align*}
for $x\in \sigma$.
We also need to define the average on an edge by
\begin{align*}
	\{\!\!\{f\}\!\!\}(x) :=
	\begin{cases}
		\frac12\left(f|_K(x)+f|_L(x)\right) &\text{ if } \sigma=\partial K\cap\partial L\\
		f|_K(x) &\text{ if } \sigma \subset\partial\Omega
	\end{cases}
\end{align*}
for $x\in\sigma$.
The classical element and edge bubble function for any element $K\in\mathcal{T}$ and edge $\sigma\in\mathcal{E}$ are denoted by $b_K$ and $b_\sigma$ (for a definition see \cite[Lemma 4.3]{bartels2016}).\\
For every $K\in\mathcal{T}$ we set $h_K:=\operatorname{diam}(K)$ as the diameter of $K$ and for $\sigma\in\mathcal{E}$
\begin{align*}
	d_\sigma = 
	\begin{cases}
		\operatorname{dist}(x_K,x_L) &\text{if } \sigma=K\cap L\\
		\operatorname{dist}(x_K,\sigma)&\text{if } \sigma\subset \partial\Omega.	
	\end{cases}
\end{align*}
For each $K\in\mathcal{T}$ we define $\M_K:L^2(K)\to\R$ via
\begin{align*}
	\mathcal{M}_Kf = \frac1{\abs{K}}\int_K f(x)\;dx\quad\forall K\in\mathcal{T}, f\in L^2(K),\quad
\end{align*}
and for each $\sigma\in\mathcal{E}$ we define $M_\sigma:L^2(\sigma)\to\R$ via
\begin{align*}
	\mathcal{M}_\sigma f = \frac1{\abs{\sigma}}\int_\sigma f(x)\;dS(x)\quad\forall \sigma\in\mathcal{E}, f\in L^2(\sigma).
\end{align*}
For brevity, we write
\begin{align*}
	\norm{\cdot}_*:=\norm{\cdot}_{L^2(0,T;H^1_D(\Omega)')}, \quad
	\norm{\cdot}_M:=\norm{\cdot}_{L^2(M)},\quad
	X(q) := L^{\frac{2q}{q-d}}(0,T;L^q(\Omega)),
\end{align*}
for some measurable set $M$. 
For the time discretization, we fix a time $T$ and a partition $(t_j)_{j=0}^J$ with $t_0=0$ and $t_J = T$.
We denote the time step size by
\begin{align*}
	\tau_j = t_{j+1}-t_j\quad j=0,\dots,J-1.
\end{align*}
\subsection{Useful interpolation inequalities}
We need the following generalized version of the classical Poincar\'e-Friedrichs inequality.
A proof can be found in \cite[Theorem 3.1]{egert2015} and \cite[Section II.1.4]{teman1988}
\begin{thm}\label{thm:friedirchsineq}
	Let $\Omega\subset\R^d$ be an open domain and $\Gamma_1\subset\partial\Omega$ with $|\Gamma_1| >0$ and $1\leq q<\infty$.
	Then there is a constant $C_{F,q,\Gamma_1}$
	\begin{align*}
		\norm{u}_{L^q(\Omega)} \leq C_{F,q,\Gamma_1}\norm{\nabla u}_{L^q(\Omega)}
	\end{align*}
	for all $u\in H^1(\Omega)$ with $u|_{\Gamma_1}=0$.
\end{thm}
We later need a so called scaled trace theorem on each triangle.
A version of this inequality can be found in \cite[Lemma 1.49]{di_pietro_2012} for $p=2$.
A proof is stated for completeness and can be similarly found in \cite{di_pietro_2012}
\begin{lemma}\label{scaledTrace}
	For $p\in[1,\infty]$, there exits $C_\text{cti,p}>0$ such that for all $K\in \mathcal{T}$, $\sigma\in\mathcal{E}_K$ and $v\in W^{1,p}(K)$ it holds
	\begin{align*}
		\norm{v}_{L^p(\sigma)}^p 
		&\leq C_\text{cti,p}\left(p\norm{\nabla v}_{L^p(K)}+dh_K^{-1}\norm{v}_{L^p(K)}\right)\norm{v}_{L^p(K)}^{p-1}.
	\end{align*}
\end{lemma}
\begin{proof}
	Define the function
	\begin{align*}
		F_\sigma = \frac{\abs{\sigma}}{d\abs{K}}(x-a_\sigma)
	\end{align*}
	where $a_\sigma$ is the vertex of $K$ opposite to $\sigma$. Then,
	\begin{align*}
		\norm{v}_{L^p(\sigma)}^p 
		= \int_\sigma \abs{v}^p ds
		= \int_{\partial K} \abs{v}^p(F_\sigma\cdot n_K)\,dS
		= \int_{K} \operatorname{div}(\abs{v}^pF_\sigma) dx
		= \underbrace{\int_{K} p\abs{v}^{p-1}F_\sigma\cdot\nabla v dx}_{=:I}
		+ \underbrace{\int_K \abs{v}^p\operatorname{div}(F_\sigma) dx}_{=:II}.
	\end{align*}
	Applying Hölder's inequality to $I$ yields
	\begin{align*}
		\int_{K} p\abs{v}^{p-1}F_\sigma\cdot\nabla v dx
		\leq p\norm{\abs{v}^{p-1}}_{L^\frac p{p-1}(K)}\norm{F_\sigma\cdot\nabla v}_{L^p(K)}
		\leq p\frac{\abs{\sigma}h_K}{d\abs{K}}\norm{v}_{L^p(K)}^{p-1}\norm{\nabla v}_{L^p(K)}.
	\end{align*}
	For $II$, we again use Hölder's inequality
	\begin{align*}
		\int_K \abs{v}^p\operatorname{div}(F_\sigma) dx
		\leq \norm{\abs{v}^p}_{L^1(K)}\norm{\operatorname{div}(F_\sigma)}_{L^\infty(K)}
		= \frac{\abs{\sigma}}{\abs{K}}\norm{v}_{L^p(K)}^p.
	\end{align*}
	Combining the estimates for $I$ and $II$ yields
	\begin{align*}
		\norm{v}_{L^p(\sigma)}^p
		&\leq \frac{\abs{\sigma}h_K}{d\abs{K}}\left(p\norm{v}_{L^p(K)}\norm{\nabla v}_{L^p(K)}^{p-1}+dh_K^{-1}\norm{v}_{L^p(K)}^p\right)\\
		&\leq C_\text{cti}\left(p\norm{\nabla v}_{L^p(K)}+dh_K^{-1}\norm{v}_{L^p(K)}\right)\norm{v}_{L^p(K)}^{p-1}.
	\end{align*}
\end{proof}
Combining Lemma \ref{scaledTrace} with the classical Poincar\'e inequality (see \cite[Corollary 2.3]{bartels2016}) we prove an approximation inequality on edges.
We denote the Poincar\'e constant for the space $L^q$ by $C_{P,q}$.

\begin{thm}\label{pFaceInterpolation}
	For each $p\in[1,\infty]$ there exists $C_{\text{app},p}>0$ such that for all $K\in\mathcal{T}$, $\sigma\in\mathcal{E}_K$ and $v\in W^{1,p}(K)$ the following holds
	\begin{align*}
		\norm{v-\M_Kv}_{L^p(\sigma)} \leq C_{\text{app},p}h_K^{1-\frac1p}\norm{\nabla v}_{L^p(K)}.
	\end{align*}
\end{thm}
\begin{proof}
	Using Lemma \ref{scaledTrace} and Poicar\'e inequality we can bound 
	\begin{align*}
		\norm{v-\M_Kv}_{L^p(\sigma)}^p
		&\leq C_\text{cti}\left(p\norm{\nabla v}_{L^p(K)}+dh_K^{-1}\norm{v-\M_Kv}_{L^p(K)}\right)\norm{v-\M_Kv}_{L^p(K)}^{p-1}\\
		&\leq C_\text{cti}\left(p\norm{\nabla v}_{L^p(K)}+dC_{P,p}\norm{\nabla v}_{L^p(K)}\right)C_{P,p}^{p-1}h_K^{p-1}\norm{\nabla v}_{L^p(K)}^{p-1}\\
		&\leq C_\text{cti}C_{P,p}^{p-1}h^{p-1}_K\left(p+dC_{P,p}\right)\norm{\nabla v}_{L^p(K)}^{p}.
	\end{align*}
	Applying the $p$-th root on both sides yields the desired result.
\end{proof}

In the following, we need a version of the Gagliardo-Nirenberg inequality for bounded domains.
A more general version was first proved in \cite{nirenberg1959} and the stated version can be found in \cite[Theorem 5.2]{adams2003}.
In \cite{adams2003}, the domain $\Omega$ needs to satisfy the cone condition.
Since we assume here that $\Omega$ has a locally Lipschitz boundary, the cone condition is satisfied for $\Omega$.
We use it for the abstract stability framework in Section \ref{section:abstract_error} and present a simple proof such that we get an explicit bound on the constant $C_{G,p}$ that only depends on the constant $C_{S,q}$ in the Sobolev inequality for $q=\frac{2(d+p)}{d}$.
\begin{thm}\label{thm:Nirenberg}
	Let $1\leq p<\infty$ for $d\leq 2$ and $1\leq p<\frac{2d}{d-2}$ for $d>2$.
	There is a constant $C_{G,p}$ such that
	\begin{align*}
		\norm{u}_{L^p(\Omega)}\leq C_{G,p}\norm{u}_{H^1(\Omega)}^\theta\norm{u}_{L^2(\Omega)}^{1-\theta}\quad\forall u\in H^1(\Omega).
	\end{align*}
	with $\theta = \frac d2-\frac dp$.
	Furthermore, we can bound $C_{G,p}\leq C_{S,q}^\theta$ with $q=\frac{2(d+p)}{d}$.
\end{thm}
\begin{proof}
	Let $u\in H^1(\Omega)\cap L^p(\Omega)$.
	The $L^p$ interpolation inequality (see \cite[Theorem 2.11]{adams2003}) implies
	\begin{align*}
		\norm{u}_{L^p(\Omega)}\leq \norm{u}_{L^2(\Omega)}^{1-\theta}\norm{u}^\theta_{L^q(\Omega)}
	\end{align*}
	with $q=\frac{2(d+p)}{d}$.
	We apply the Sobolev inequality (see \cite[Theorem 4.12]{adams2003}) and obtain
	\begin{align*}
		\norm{u}_{L^p(\Omega)}\leq C_{S,q}^\theta\norm{u}_{H^1(\Omega)}^\theta\norm{u}_{L^2(\Omega)}^{1-\theta},
	\end{align*}
	where $C_{S,q}$ is the Sobolev imbedding constant.
\end{proof}
\section{Numerical scheme}\label{section:scheme}
We use the finite volume scheme from \cite{gerstenmayer2018} to solve \eqref{equations}.
We provide a posteriori error estimates by reconstructing the numerical solution, in a way similar to \cite{nicaise2005,nicaise2006}, and using the reconstructed solution in the stability framework of Section \ref{section:abstract_error}.
The basic ideas of the use of reconstructions in a posteriori error estimates can be found in \cite{makridakis2003,makridakis2007}.
There are also other finite volume schemes in the literature namely \cite{cances2019} and \cite{cances2024,cances2023}.
Since our goal is to initiate a posteriori error analysis for cross diffusion systems we opted to use the simpler finite volume scheme from \cite{gerstenmayer2018}.
In order to derive an a posteriori error estimate for the scheme from \cite{cances2023} we can combine the stability analysis from Section \ref{section:abstract_error} and the reconstruction presented below.
However, the use of this reconstruction would lead to an error estimator that is first order convergent, i.e. suboptimal for that scheme.
Deriving a reconstruction that provides a second order error estimator for the scheme from \cite{cances2023} is an interesting task but beyond the scope of this work.\\
\textcite{gerstenmayerFiniteelement2018} also describes a structure preserving finite element method for \eqref{equations}-\eqref{eqn:initial}.
It solves the equation in the entropy variables and therefore asks for higher regularity conditions than the finite volume method presented below and non-zero initial conditions.
\subsection{Finite Volume scheme}\label{subsec:FV}
In the following, we write $u_{i,K}^j$ for the approximation of $u_i$ on $K$ at time $t_j$.
We approximate the initial conditions as follows
\begin{align*}
	u_{i,K}^0 &= \M_K u_i^0,\quad\Phi_K^0 = \M_K\Phi^0
\end{align*}
for $i=0,\dots,n$ and $K\in\mathcal{T}$.
We recall the numerical scheme from \cite{gerstenmayer2018}. 
For $K\in\mathcal{T}$
\begin{align}\label{eqn:scheme}
	\abs{K}\frac{u_{i,K}^j-u_{i,K}^{j-1}}{\tau_j} + \sum_{\sigma\in\mathcal{E}_K} \mathcal{F}_{i,K,\sigma}^j &= 0\quad \forall i\in\{1,\dots,n\}\\
	u_{0,K}^j &= 1-\sum_{i=1}^nu_{i,K}^j\\
	-\lambda^2\sum_{\sigma\in\mathcal{E}_K} F_\sigma^K(\Phi^j) - \abs{K}\sum_{i=1}^n z_iu_{i,K}^j &= 0
	\label{eqn:scheme_phi}
\end{align}
with
\begin{align}
	\mathcal{F}^j_{i,K,\sigma} := -u_{0,\sigma}^jF_{\sigma}^K(u_i^j)+u_{i,\sigma}^j (F_{\sigma}^K(u_{0}^j)-u_{0,\sigma}^j\beta z_iF_{\sigma}^K(\Phi^j))\quad i=1,\dots,n
\end{align}
and
\begin{align*}
	F^K_{\sigma}(u_{i}^j) &:= \begin{cases}
		\frac{\abs{\sigma}}{d_\sigma}(u_{i,K}^j-u_{i,L}^j) &\text{if } \sigma = K\cap L\\
		0 &\text{if } \sigma \subset \Gamma_N\\
		\frac{\abs{\sigma}}{d_\sigma}u_{i,K}^j &\text{if } \sigma \subset \Gamma_D
	\end{cases},\quad
	u_{i,\sigma}^j
	:= \begin{cases}
		u_{i,K}^j &\text{if } \sigma\subset\Gamma_N\\
		\M_\sigma u_i^D &\text{if } \sigma\subset\Gamma_D\\
		\frac{u_{i,K}^j-u_{i,L}^j}{\log(u_{i,K}^j)-\log(u_{i,L}^j)}&\text{otherwise}
	\end{cases}\quad i=0,\dots,n.
\end{align*}
For the reconstruction of the solvent concentration we need a numerical equivalent of equation \eqref{eqn:solvent}.
We obtain such a description in the same way as we obtained equation \eqref{eqn:solvent}.
Summing \eqref{eqn:scheme} over $i$ yields
\begin{align}\label{eqn:scheme_0}
	\abs{K}\frac{u_{0,K}^j-u_{0,K}^{j-1}}{\tau_j}-\sum_{\sigma\in\mathcal{E}_K}F^K_{\sigma}(u_0^j)\left(u_{0,\sigma}^j+\sum_{i=1}^n u_{i,\sigma}^j\right)+\beta F^K_{\sigma}(\Phi^j)u_{0,\sigma}^j\left(\sum_{i=1}^n z_iu_{i,\sigma}^j\right)=0.
\end{align}
We write $u_{i,h}$ for the piecewise constant functions on $\Omega$ given by $u_{i,h}(x) = u_{i,K}$ for all $x\in K$ and $i=0,\dots,n$.
Similar we define $\Phi_h$ by $\Phi_h(x)=\Phi_K$ for all $x\in K$.
\begin{remark}\label{remark:scheme}
	\begin{enumerate}[(i)]
		\item If the mesh $\mathcal{T}$ does not satisfy the orthogonality criterion, one can use the flux defined in \cite{coudiere1999} and get a similar scheme.
			In the following, we use only the conservation of mass, namely $F_{\sigma,K}^j(f) = -F_{\sigma,L}^j(f)$ for $\sigma=\partial K\cap\partial L$.
			This is also the property of the flux used in \cite{nicaise2005,nicaise2006}.
		\item We use the logarithmic-mean to get an approximation of $u_i$ on the edges.
			\textcite{cances2019} use an upwind-scheme and \textcite{cances2023} use centered flux if $z_i=0$ ($i=1,\dots,n$).
			Using centered-flux, i.e. arithmetic mean, leads to similar results as using the logarithmic-mean in our tests.
			The same holds for the use of other means, e.g. the geometric mean.
	\end{enumerate}
\end{remark}
\subsection{Morley type reconstruction}\label{subsec:Morley}
We reconstruct the finite volume solution $(u_{0,h},\dots,u_{h,n},\Phi_h)$ defined in \eqref{eqn:scheme}-\eqref{eqn:scheme_phi}.
This spacial reconstruction is done elementwise, more precisely we use a Morley type reconstruction similar to \cite{nicaise2005,nicaise2006}.
For every time step, we first interpolate the finite volume solution with piecewise linear functions.
After that the integral of the outward normal derivative of the interpolant on every edge is adjusted to fit the numerical diffusive flux.\\
In \cite{nicaise2005} the reconstruction is applied to the Poisson equation and in \cite{nicaise2006} to more general elliptic equations.
The main difference to \cite{nicaise2006} is that we do not fix averages along edges in the reconstruction process.
In our tests we get a better convergence rate when we do not reconstruct with respect to the convection.
Thus, reconstructing with convection yields only convergence of order $\frac12$ whereas we obtain linear convergence, if we only reconstruct with respect to diffusion.
We give an explanation for this in Remark \ref{remark:convection}.
Not reconstructing with respect to convection leads to reconstruction error terms $R_{i,R}$ in the residual estimator in Theorem \ref{thm:Residual_bound}.
In our tests, these reconstruction error terms converge with linear order as expected.\\
First, we reconstruct the solvent concentration $u_{0,h}$ and electric potential $\Phi_h$, since they solve equations independent of the ion species.
We use the following polynomial space on $K\in\mathcal{T}$
\begin{align*}
	P_K &= \{q+pb_K\;:\;q\in\mathbb{P}_1(K),p\in\mathbb{P}_1(K)\}.
\end{align*}
For the solvent concentration and electric potential we use the degrees of freedom
\begin{align*}
	\Sigma_{K,0} &= \{v(a_i^K)\}_{i=1,2,3}\cup\left\{\int_\sigma\frac{\partial v}{\partial n_{K,\sigma}}ds\right\}_{\sigma\in\mathcal{E}_K}
\end{align*}
on every element $K\in\mathcal{T}$.
We define the Morley finite element space for the reconstructions of the $u_0^j$ and $\Phi^j$ by
\begin{align*}
	V_h^0 = \left\{v_h\in H^1_D(\Omega)\;:\;\right.&v_h|_K\in P_K\;\forall K\in\mathcal{T},\\
										&\int_\sigma \frac{\partial v_h|_K}{\partial n_{K,\sigma}} ds = -\int_\sigma \frac{\partial v_h|_L}{\partial n_{L,\sigma}} ds \;\forall \sigma\in\mathcal{E},K,L\in\mathcal{T}:\sigma = K\cap L,\\
										&\left.\int_\sigma \frac{\partial v_h|_K}{\partial n_{K,\sigma}} ds = 0 \;\forall K\in\mathcal{T}:\sigma\in\mathcal{E}_K\cap\mathcal{E}^N\right\}
\end{align*}
With the above notation on hand we can define the Morley reconstruction for the solvent concentration.
\begin{defn}\label{defn:Morley_solvent}
	Let the weights $w_L(a_i^K)$ ($i=1,\dots,3$ and $K,L\in\mathcal{T}$ with $a_i^K\in\overline{K}\cap\overline{L}$) be given according to \cite[Section 3.3]{coudiere1999}.
	\begin{enumerate}[(i)]
		\item For a finite volume solution $u_0^j$ of \eqref{eqn:scheme}-\eqref{eqn:scheme_phi} we define its Morley reconstruction $\widehat{u_0}^j$ as the unique element $\widehat{u_0}^j\in V_h^0$ satisfying
			\begin{align*}
				\widehat{u_0}^j(a_i^K) &= \sum_{L\in\mathcal{T}:a_i^K\in \overline{L}}w_L(a_i^K)u_0^j(L)\quad\forall K\in\mathcal{T},i\in\{1,2,3\}\\
				\int_\sigma \frac{\partial \widehat{u_0}^j|_K}{\partial n_{K,\sigma}}\, dS &= F_{\sigma}^K(u_{0}^j)\left(u_{0,\sigma}^j+\sum_{i=1}^n u_{i,\sigma}^j\right)\quad\forall K\in\mathcal{T},\sigma\in \mathcal{E}_K.
			\end{align*}
		\item For a finite volume solution $\Phi^j$ of \eqref{eqn:scheme}-\eqref{eqn:scheme_phi} we define its Morley reconstruction $\widehat{\Phi}^j$ as the unique element $\widehat{\Phi}^j\in V_h^0$ satisfying
			\begin{align*}
				\widehat{\Phi}^j(a_i^K) &= \sum_{L\in\mathcal{T}:a_i^K\in \overline{L}}w_L(a_i^K)\Phi^j(L)\quad\forall K\in\mathcal{T},i\in\{1,2,3\}\\
				\int_\sigma \frac{\partial \widehat{\Phi}^j|_K}{\partial n_{K,\sigma}}\, dS &= F_{\sigma}^K(\Phi^j)\quad\forall K\in\mathcal{T},\sigma\in \mathcal{E}_K.
			\end{align*}
	\end{enumerate}
\end{defn}
\begin{remark}
	\begin{enumerate}[(i)]
		\item
			The Morley type reconstruction for the solvent concentration is different from that presented in \cite{nicaise2005} and \cite{nicaise2006}.
			In \cite{nicaise2005} the $C^0$ reconstruction on triangles uses the polynomial base $q+pb_K$ for $q\in\mathbb{P}_2(K)$ instead of $\mathbb{P}_1(K)$ and use $\{p(m_i)\}_{i=1,2,3}$ (with $m_i$ the edge-midpoints of the triangle $K$) in addition to the degrees of freedom used here.
			For simplicity, we stick to the one presented in Definition \ref{defn:Morley_solvent}, since it uses less degrees of freedom and fulfills our needs.
		\item In the definition of $\widehat{u_0}^j$ we used $F_\sigma^K(u_0^j)\left(u_{0,\sigma}^j+\sum_{i=1}^nu_{i,\sigma}^j\right)$ instead of $F_\sigma^K(u_0^j)$ for the integral of the normal derivatives.
			In the proof of Theorem \ref{thm:H1_heat_bound} and Proposition \ref{prop:MorleyOrth}, we need that the integral of the normal derivatives is given by the numerical fluxes used in the finite volume scheme to derive the orthogonality condition \eqref{eqn:orth}.
			Recall that $\left(u_{0,\sigma}^j+\sum_{i=1}^nu_{i,\sigma}^j\right)\approx 1$.
	\end{enumerate}
\end{remark}
For the other species $u_i$ ($i=1,\dots,n$) we also use the polynomial space $P_K$, but with degrees of freedom
\begin{align*}
	\Sigma_K^j &= \{v(a_i^K)\}_{i=1,2,3}\cup\left\{\int_\sigma \widehat{u_0}^j\frac{\partial v}{\partial n_{K,\sigma}}ds\right\}_{\sigma\in\mathcal{E}_\sigma}
\end{align*}
on every triangle and time step $j=0,\dots,J$.
We now prove that $(K,P_K,\Sigma_K^j)$ defines a finite element for every $K\in\mathcal{T}$.
The proof is similar to \cite[Lemma 4.1]{nicaise2006} and \cite[Lemma 4.1]{nilssen2000}.
\begin{prop}
	If $\widehat{u_0}^j>0$ on $\overline{K}$, then the triple $(K,P_K,\Sigma_K^j)$ is a $C^0$-finite element.
\end{prop}
\begin{proof}
	Since $\dim P_K =6 =\operatorname{card}\Sigma_K$ it suffices to show that if $w=q+pb_K\in P_K$
	\begin{align*}
		\lambda(w) = 0\quad\forall\lambda\in\Sigma_K^j
	\end{align*}
	implies $w=0$.
	The condition $0=w(a_i^K) = q(a_i^K)+p(a_i^K)b_K(a_i^K) = q(a_i^K)$ and $q\in \mathbb{P}_1(K)$ implies $q=0$.\\
	We proceed similar to \cite[Lemma 4.1]{nilssen2000} to show $p=0$ on $K$.
	Let $\sigma\in\mathcal{E}_K$ be an arbritrary edge of $K$.
	Since the zero-set of $b_\sigma$ is a subset of the zero-set of $b_K$ we can factor out $b_\sigma$
	\begin{align*}
		b_K = b_\sigma \lambda_\sigma,
	\end{align*}
	with $\lambda_\sigma \neq 0$ in the interior of $\sigma$.
	The properties of $b_\sigma$ imply that $\frac{\partial b_\sigma}{\partial n_{\sigma,K}}<0$ on the interior of $\sigma$.
	For every edge $\sigma\in\mathcal{E}_K$ holds
	\begin{align*}
		0=\int_\sigma \widehat{u_0}^j\frac{\partial w}{\partial n_{\sigma,K}}\,dS
		=\int_\sigma \widehat{u_0}^j\frac{\partial pb_K}{\partial n_{\sigma,K}}\,dS
		=\int_\sigma \widehat{u_0}^jp\lambda_\sigma\frac{\partial b_\sigma}{\partial n_{\sigma,K}}\,dS.
	\end{align*}
	Thus, $\widehat{u_0}^jp$ has a root on the interior of every edge $\sigma\in\mathcal{E}_K$.
	Since $\widehat{u_0}^j>0$, we conclude, that $p\in\mathbb{P}_1(K)$ has a root on the interior of every edge $\sigma\in\mathcal{E}_K$.
	This implies $p=0$ and with the above $w=0$.
\end{proof}
We define the Morley finite element space for the reconstructions of $u_{i,h}^j$ by
\begin{align*}
	V_{j,h} = \left\{v_h\in H^1_D(\Omega):\right.&v_h|_K\in P_K\;\forall K\in\mathcal{T}\\
										&\int_\sigma \widehat{u_0}^j\frac{\partial v_h|_K}{\partial n_\sigma} ds = \int_\sigma \widehat{u_0}^j\frac{\partial v_h|_L}{\partial n_\sigma} ds \;\forall \sigma\in\mathcal{E},K,L\in\mathcal{T}:\sigma = K\cap L,\\
										&\left.\int_\sigma \widehat{u_0}^j\frac{\partial v_h|_K}{\partial n_\sigma} ds = 0 \;\forall T\in\mathcal{T}:\sigma\in\mathcal{E}_T\cap\partial\Omega\right\}.
\end{align*}
\begin{defn}
	The Morley type reconstruction $\widehat{u_i}^j$ for the species $u_i^j$ at time $j$ is the unique element $\widehat{u_i}^j\in V_{j,h}$ satisfying
	\begin{align*}
		\widehat{u_i}^j(a_i^K) &= \sum_{L\in\mathcal{T}:a_i^K\in \overline{L}}w_L(a_j^K)u_{i,L}^j\quad\forall K\in\mathcal{T},j\in\{1,2,3\}\\
		\int_\sigma \widehat{u_0}^j\frac{\partial \widehat{u_i}^j|_K}{\partial n_{K,\sigma}}\, dS &= u_{0,\sigma}^jF^K_\sigma(u_{i}^j)\quad\forall K\in\mathcal{T},\sigma\in \mathcal{E}_K,
	\end{align*}
	where we again use the weights $w_L(a_i^K)$ chosen according to \cite{coudiere1999}
\end{defn}
\begin{remark}\label{remark:convection}
	\textcite{nicaise2006} uses terms of the form $\alpha_\sigma b_\sigma b_K$ ($\alpha_\sigma\in\R$ for every edge $\sigma\in\mathcal{E}_K$) to add degrees of freedom corresponding to the convective fluxes.
	In our experiments, the convection term $u_{i,\sigma}^jF_\sigma^K(u_0^j)$ is large compared to $u_i^j$ the coefficient $\alpha_\sigma$ becomes large and introduces artificial bumps of the form $b_\sigma b_K$ in the middle of the triangle $K$.
	This yields to convergence rates of order $\frac12$ rather then linear convergence.
\end{remark}
For the reconstruction in time we use a linear interpolation between the time steps.
Namely, we denote the space-time reconstruction by
\begin{align*}
	\widehat{u_i}(t) := \frac{t-t_{j-1}}{\tau_j} \widehat{u_i}^j + \frac{t_j-t}{\tau_j} \widehat{u_i}^{j-1},
\end{align*}
with $t\in[t_{j-1},t_j]$.
We use a linear reconstruction in time, since it is easy and is sufficient for our needs.
\section{Error estimates for the diffusion equation}\label{section:heat}
The solvent concentration in the reduced model \eqref{eqn:reduced} solves the classical diffusion equation.
In Theorem \ref{thm:abstract_error}, we derive a bound of the difference $u_i-v_i$ that requires bounds for $\norm{u_0-v_0}_{L^\infty([0,T]\times\Omega)}$ and $\norm{\nabla (u_0-v_0)}_{L^2([0,T]\times\Omega)}$.
In this section, we derive bounds on the difference $u_0-v_0$ if $u_0$ is given by the reconstruction of the finite volume solution given in Section \ref{section:scheme}.
Following a similar strategy as in \cite{kyza2018} and \cite{demlow2023}, we first derive the $L^\infty$ bound with the help of Green's function for the Poisson equation on $\Omega$.
The proof presented below uses a uniform bound on the $W^{1,p}$ norm of Green's function on $\Omega$ for some $p^*\in[1,\infty)$ (see Assumption \ref{ass:uniform}).
For the $L^2$ bound of the gradient of $u_0-v_0$ we use the abstract error estimate from \cite[Proposition 4.5]{bartels2016}.
The residual is then estimated by the properties of the reconstruction similar to \cite{nicaise2005,nicaise2006}.\\
\subsection{$L^\infty$ norm estimates for the diffusion equation}
We look at the classical diffusion equation with Dirichlet and Neumann boundary conditions
\begin{align}\label{eqn:heat}
	\begin{split}
		\partial_t u&= \Delta u\quad \text{on }\Omega\\
		u &= u^D \quad \text{on }\Gamma_D\\
		\nabla u\cdot n &= 0\quad\text{on }\Gamma_N
	\end{split}
\end{align}
with $\Omega,\Gamma_D$ and $\Gamma_N$ as in Assumption \ref{assumptions}.
To show an upper error bound in the maximum norm for the reconstruction of the finite volume solution we follow the approach of \cite{kyza2018, demlow2012, demlow2023}.
In the upcoming discussion we need Green's function $G$ for the domain $\Omega$ satisfying
\begin{align*}
	-\Delta G(x;y) = \delta(y-x)\quad \forall x,y\in\Omega,
\end{align*}
with boundary conditions
\begin{align*}
	\nabla G\cdot n &= 0\quad\text{on }\Gamma_N\\
	G &=0\quad\text{on }\Gamma_D.\\
\end{align*}
This yields using Green's identity
\begin{align}\label{eqn:Green}
	u(x) = \int_\Omega \nabla G\cdot\nabla u\;dx
\end{align}
for $u\in H^1_D(\Omega)\cap W^{1,q}(\Omega)$ with $q> d$ in this case $W^{1,q}(\Omega)$ embedds into $C^0(\Omega)$.
In the following, we want to use an elliptic reconstruction in every time step $t_j$.
	We define the elliptic reconstruction $w^j$ for time step $j$ as the solution of
\begin{align}\label{def_ell_recon}
	\int_\Omega \nabla w^j\cdot\nabla f\,dx &= \int_\Omega A^jf\,dx\quad\forall f\in H^1_D(\Omega)
\end{align}
with
\begin{align}\label{eqn:def_A^j}
	A^j := \begin{cases}
			-\Delta_h \widehat{u_0}^0 &\text{if } j=0\\
			-\frac{u_0^j-u_0^{j-1}}{\tau_j} &\text{if } j>0.
		\end{cases}
\end{align}
and boundary conditions
\begin{align*}
	\nabla w^j\cdot n &= 0\quad\text{on }\Gamma_N\\
	w^j &= u^D\quad\text{on }\Gamma_D.
\end{align*}
In \eqref{eqn:def_A^j}, we use the discrete Lapalcian $\Delta_h:V_h^0\to V_h^0$ defined for $f_h\in V_h^0$ by (similar to \cite[Definition 4.13]{bartels2016})
\begin{align*}
	\left(-\Delta_h f_h,g_h\right) :=\left(\nabla f_h,\nabla g_h\right)\quad\forall g_h\in V_h^0.
\end{align*}
The reason for this is, that we cannot use $\Delta \widehat{u_0}^0$ for $w^0$, since $\Delta\widehat{u_0}^0\notin L^2(\Omega)$ and we need $w^0\in L^2(\Omega)$ in Lemma \ref{lemma:ParabolicBound}.\\
We split the error $e:=v_0-\widehat{u_0}$ into an elliptic error $\varepsilon := w-\widehat{u_0}$ and a parabolic error $\rho:=v_0-w$.
Since $\widehat{u_0}^j$ is a numerical approximation of the Poisson equation $-\Delta w = -A^j$, we can use elliptic error estimators to get a pointwise error bound.
The parabolic error $\rho$ solves the diffusion equation
\begin{align*}
	\partial_t \rho +\nabla \rho = R(t)-\partial_t\varepsilon
\end{align*}
with the temporal residual $R(t):=-\ell_{j-1}A^{j-1}-\ell_jA^j-\frac{\widehat{u_0}^j-\widehat{u_0}^{j-1}}{\tau_j}$ for $t\in[t_{j-1},t_j]$.
Therefore, we can use Duhamel's principle and the head semigroup to bound $\rho$.\\
First, we bound $\varepsilon$ pointwise using the properties of the Morley reconstruction and Green's function at each time level $t_j$.
We can represent the elliptic error at time $t_j$ by \eqref{eqn:Green} and obtain a pointwise bound by estimating the right hand side.
After that, we use the $L^\infty$-contractivity of the heat semigroup on $\Omega$ with homogeneous Neumann and Dirichlet boundary conditions to bound $\norm{\rho}_{L^\infty([0,T]\times\Omega)}$.
\begin{ass}\label{ass:uniform}\hfill
	To establish the maximum norm error bound, we need a uniform bound of the $W^{1,p^*}_D(\Omega)$ norm of Green's function from some $p^*\in[1,\infty)$, i.e. $G$ satisfies
	\begin{align*}
		\norm{\nabla_xG(x;y)}_{L^{p^*}(\Omega)}\leq C_{\text{Green},{p^*}}\quad\forall y\in\Omega.
	\end{align*}
\end{ass}
\begin{remark}\label{remark:uniform}
	Assumption \ref{ass:uniform} holds according to \cite[Theorem 1.1]{gurter1982}, for $d>2$ and $\Gamma_D =\partial\Omega$ and $1\leq p^*<\frac{d}{d-1}$.
	For $d=2$ Assumption \ref{ass:uniform} holds according to \cite[Theorem 2.12, Remark 2.19]{dong2009} and \cite[Theorem 2.2]{demlow2012} for $\partial\Omega=\Gamma_D$ and $1\leq p^*<\frac{d}{d-1}$.
	For $\Gamma_N=\partial\Omega$ and $1\leq p^*<\frac{d}{d-1}$, Assumption \ref{ass:uniform} is true according to \cite[Theorem 1]{mitrea2010}.
	In most circumstances the constant $C_{\text{Green},p}$ is not accessible. 
	In our numerical experiments in Section \ref{section:numerics}, we set $C_{\text{Green},p}=1$.
\end{remark}
We first prove a bound of $\varepsilon$ for every time step.
The next proposition establishes a quasi-orthogonality property for $\varepsilon$ to piecewiese constant functions similar to \cite[Lemma 5.3]{nicaise2005}.
Since we cannot use piecewiese constant functions as test functions in the weak formulation of the Poisson problem, the error cannot be orthogonal to those functions.
This property is central to the following considerations.
\begin{prop}\label{prop:MorleyOrth}
	Let $1\leq p<\infty$.
	The elliptic error $\varepsilon^j := w^j -\widehat{u_0}^j$ at time step $j$ satisfies
	\begin{multline*}
		\int_\Omega\nabla (w^j-\widehat{u_0}^j)\cdot\nabla f\,dx = \sum_{K\in \mathcal{T}}\left(\int_K (-A^j-\Delta \widehat{u_0}^j)(f-\M_K f) dx\right.\\
										\left.-\frac12\sum_{\sigma\in\mathcal{E}_K\setminus\mathcal{E}^D}\int_{\sigma} \jump{\nabla \widehat{u_0}^j}(f-\M_K f) dS\right) \quad \forall f\in W^{1,p}_D(\Omega)
	\end{multline*}
\end{prop}
\begin{proof}
	Let $f\in W^{1,p}_D(\Omega)$.
	Similar to \cite[Lemma 5.3]{nicaise2005}, we conclude
	\begin{align}\label{eqn:orth}
		\int_\Omega \nabla (w^j-\widehat{u_0}^j)\cdot\nabla f\,dx
		&=\sum_{K\in \mathcal{T}}\int_K (-A^j-\Delta \widehat{u_0}^j)(f-\M_K f) dx -\frac12\sum_{K\in\mathcal{T}}\sum_{\sigma\in\mathcal{E}_K\setminus\mathcal{E}^D}\int_\sigma\left\llbracket\nabla \widehat{u_0}^j\right\rrbracket f ds
	\end{align}
	Due to the conservation of mass, i.e. $\int_\sigma \nabla \widehat{u_0}^j\cdot n_{K,\sigma}\,dS = F_\sigma^K(u_0^j)=-F_\sigma^L(u_0^j)=-\int_\sigma \nabla \widehat{u_0}^j\cdot n_{L,\sigma}\,dS$ for $\sigma = \overline{K}\cap\overline{L}$, we get
	\begin{multline*}
		\int_\Omega\nabla (w^j-\widehat{u_0}^j)\cdot\nabla f\,dx=\sum_{K\in \mathcal{T}}\int_K (-A^j-\Delta \widehat{u_0}^j)(f-\M_K f)\, dx\\
		-\frac12\sum_{K\in\mathcal{T}}\sum_{\sigma\in\mathcal{E}_K\setminus\mathcal{E}^D}\int_\sigma\left\llbracket\nabla \widehat{u_0}^j\right\rrbracket (f-\M_K f) dS.
	\end{multline*}
\end{proof}
We can now estimate the elliptic error in the maximum norm for every time step.
For this we use Proposition \ref{prop:MorleyOrth} with $f=G$ and Theorem \ref{scaledTrace} to estimate the jump terms.
\begin{lemma}\label{lemma:EllipticBound}
	If $\Omega$ satisfies Assumption \ref{ass:uniform} with $p^*$, then elliptic error $\varepsilon^j$ at time $t_j$ can be bounded by
	\begin{align*}
		\norm{\varepsilon_j}_{L^\infty}\leq \eta_{S,q}^jC_{\text{Green},p}
	\end{align*}
	with $\frac1{p^*}+\frac1q=1$ and
	\begin{align*}
		\eta_{S,q}^j &:=
		\begin{dcases}
			2^{\frac1{p^*}}\left(\sum_{K\in\mathcal{T}}\left(C_{P,p^*}^qh_K^q\norm{A^j+\Delta \widehat{u_0}^j}_{L^q(K)}^q+\frac{C_{\text{app},p^*}^qN_\partial^{q-1}}{2^q}h_K\sum_{\sigma\in\mathcal{E}_K\setminus\mathcal{E}^D}\norm{\llbracket\nabla \widehat{u_0}^j\rrbracket}_{L^q(\sigma)}^q\right)\right)^\frac1q &\text{for }q<\infty\\[2\baselineskip]
			\max_{K\in\mathcal{T}}\left(C_{P,1}h_K\norm{A^j+\Delta \widehat{u_0}^j}_{L^\infty(K)}+\frac{C_{\text{app},1}}{2}\sum_{\sigma\in\mathcal{E}_K\setminus\mathcal{E}^D}\norm{\llbracket\nabla \widehat{u_0}^j\rrbracket}_{L^\infty(\sigma)}\right)&\text{for }q=\infty.
		\end{dcases}
	\end{align*}
\end{lemma}
\begin{proof}
	Using \eqref{eqn:Green} and Proposition \ref{prop:MorleyOrth} yields
	\begin{align*}
		(w^j-\widehat{u_0}^j)(y)
		&=\int_\Omega \nabla (w^j-\widehat{u_0}^j)\cdot\nabla G(\cdot;y)\,dx\\
		&=\sum_{K\in\mathcal{T}}\int_K (-A^j-\Delta \widehat{u_0}^j)(G-\M_K G) dx
		-\frac12\sum_{\sigma\in\mathcal{E}_K\setminus\mathcal{E}^D}\int_{\sigma} \left\llbracket\nabla \widehat{u_0}^j \right\rrbracket(G-\M_K G)\, dS\\
		&\leq \sum_{K\in\mathcal{T}}\norm{-A^j-\Delta \widehat{u_0}^j}_{L^q(K)}\norm{G-\M_KG}_{L^{p^*}(K)}
		+\frac12\sum_{\sigma\in\mathcal{E}_K\setminus\mathcal{E}^D}\norm{\left\llbracket\nabla \widehat{u_0}^j\right\rrbracket}_{L^q(\sigma)}\norm{G-\M_K G}_{L^{p^*}(\sigma)}.
	\end{align*}
	Using Theorem \ref{pFaceInterpolation} and the discrete Cauchy-Schwarz inequality we can estimate
	\begin{align*}
		\int_\Omega\nabla (w^j-\widehat{u_0}^j)\cdot\nabla G\,dx
		&\leq \sum_{K\in\mathcal{T}}\norm{-A^j-\Delta \widehat{u_0}^j}_{L^q(K)}C_{P,p^*}h_K\norm{\nabla G}_{L^{p^*}(K)}\\
		&+\frac12\sum_{\sigma\in\mathcal{E}_K\setminus\mathcal{E}^D}\norm{\left\llbracket\nabla \widehat{u_0}^j\right\rrbracket}_{L^q(\sigma)}C_{\text{app},p^*}h_K^{1-\frac1{p^*}}\norm{\nabla G}_{L^{p^*}(K)}\\
		&\leq2^{\frac1{p^*}}\left(\sum_{K\in\mathcal{T}}\left(C_{P,p^*}^qh_K^q\norm{A^j+\Delta \widehat{u_0}^j}_{L^q(K)}^q +\frac{C_{\text{app},p^*}^qN_\partial^{q-1}}{2^q} h_K\sum_{\sigma\in\mathcal{E}_K\setminus\mathcal{E}^D}\norm{\llbracket\nabla \widehat{u_0}^j\rrbracket}_{L^q(\partial K)}^q\right)\right)^{\frac1q}\\
		&\left(\sum_{K\in\mathcal{T}}\norm{\nabla G}_{L^{p^*}(K)}^{p^*}\right)^{\frac1{p^*}}\\
		&\leq \eta_{S,q}^j\norm{\nabla G}_{L^{p^*}(\Omega)}.
	\end{align*}
	Where we defined $N_\partial=\max_{K\in\mathcal{T}}\abs{\mathcal{E}_T}$ after Definition \ref{defn:mesh}.
	The proof for $q=\infty$ is completely analog.
\end{proof}
We now need to look at the parabolic error $\rho$.
For this we need the heat semigroup on $\Omega$ with homogeneous Neumann and Dirichlet conditions on $\Gamma_N$ and $\Gamma_D$ respectively denoted by $\left(e^{t\Delta}\right)_{t\geq0}$.
Furthermore, we note that
\begin{align*}
	\norm{e^{t\Delta}f}_{L^\infty(\Omega)}\leq \norm{f}_{L^\infty(\Omega)}
\end{align*}
for every $f\in L^\infty(\Omega)$ and $t>0$.
For more information see \cite[Chapter 4]{ouhabaz2009}.
Notice, that since $w$ and $v_0$ satisfy the boundary conditions exactly, we only need to consider homogeneous Dirichlet and Neumann boundary conditions.
Further, we define the temporal residual
\begin{align*}
	R(t) = -\ell_{j-1}A^{j-1}-\ell_{j}A^j-\frac{\widehat{u_0}^j-\widehat{u_0}^{j-1}}{\tau}\quad t\in[t_{j-1},t_j]
\end{align*}
with $\ell_{j-1} := \frac{t_j-t}{\tau_j}$ and $\ell_j:=\frac{t_{j-1}+t}{\tau_j}$.
\begin{lemma}\label{lemma:ParabolicBound}
	If $\Omega$ satisfies Assumption \ref{ass:uniform}, then the parabolic error satisfies for $t\in (t_{j-1},t_j)$ for $j=1,\dots,J$
	\begin{align*}
		\norm{\rho(t)} \leq \norm{\rho(t_{m-1})}_{L^\infty(\Omega)}+\eta_{T,\infty}^j+\tau_j\dot{\eta}_{S,q}^j C_{\text{Green},p^*}
	\end{align*}
	with $\frac1{p^*}+\frac1q=1$ and
	\begin{alignat*}{2}
		\eta_{T,\infty}^j &:= \int_{t_{j-1}}^{t_j}\norm{R(t)}_{L^\infty(\Omega)}\;dt = \int_{t_{j-1}}^{t_j}\norm{-\ell_{j-1}A^{j-1}-\ell_jA^j-\frac{\widehat{u_0}^j-\widehat{u_0}^{j-1}}{\tau_j}}_{L^\infty(\Omega)} dt&&\\
		\dot{\eta}_{S,q}^j&:= \tau_j^{-1}2^{\frac1{p^*}}\left(\sum_{K\in\mathcal{T}}\left(C_{P,p^*}^qh_K^q\norm{A^j-A^{j-1}+\Delta (\widehat{u_0}^j-\widehat{u_0}^{j-1})}_{L^q(K)}^q\right.\right.\\
			&\quad\left.\left.+\frac{C_{\text{app},p^*}^qN_\partial^{q-1}}{2^q}h_K\sum_{\sigma\in\mathcal{E}_K\setminus\mathcal{E}^D}\norm{\left\llbracket\nabla (\widehat{u_0}^j-\widehat{u_0}^{j-1})\right\rrbracket}_{L^q(\sigma)}^q\right)\right)^{\frac1q}&& \text{for }q<\infty\\
		\dot{\eta}_{S,\infty}^j&:=\tau_j^{-1}\max_{K\in\mathcal{T}}\left(C_{P,1}h_K\norm{A^j-A^{j-1}+\Delta (\widehat{u_0}^j-\widehat{u_0}^{j-1})}_{L^\infty(K)}\right.\\
			&\quad\left.+\frac{C_{\text{app},1}}{2}\sum_{\sigma\in\mathcal{E}_K\setminus\mathcal{E}^D}\norm{\left\llbracket\nabla (\widehat{u_0}^j-\widehat{u_0}^{j-1})\right\rrbracket}_{L^\infty(\sigma)}\right) &&\text{for } q=\infty.
	\end{alignat*}
\end{lemma}
\begin{proof}
	One can easily verify that
	\begin{align*}
		\rho_t-\Delta\rho = -\ell_{j-1}A^{j-1}-\ell_{j}A^j-\frac{\widehat{u_0}^j-\widehat{u_0}^{j-1}}{\tau_j}-\partial_t\varepsilon = R(t)-\partial_t\varepsilon
	\end{align*}
	for all $t\in (t_{j-1},t_j)$.
	Hence, using Duhamel's principle and the continuous semigroup $e^{t\Delta}$ we can express $\rho$ as
	\begin{align*}
		\rho(t) = e^{(t-t_j)\Delta}\rho(t_{j-1})+\int_{t_{j-1}}^te^{(t-s)\Delta}(R(s)-\partial_t\varepsilon(s))\;ds
	\end{align*}
	Using this expression we can bound
	\begin{align*}
		\norm{\rho(t)}_{L^\infty(\Omega)}
		&\leq \norm{\rho(t_{j-1})}_{L^\infty(\Omega)}+\int_{t_{j-1}}^t\norm{R(s)}_{L^\infty(\Omega)}+\norm{\partial_t\varepsilon(s)}_{L^\infty(\Omega)}\;ds\\
		&\leq \norm{\rho(t_{j-1})}_{L^\infty(\Omega)}+\eta_T^j+\tau_j\norm{\partial_t\varepsilon}_{L^\infty(\Omega)}.
	\end{align*}
	We only need to show $\norm{\partial_t\varepsilon}_{L^\infty(\Omega)}\leq\dot{\eta}_{S,q}^j$.
	To do this we write
	\begin{align*}
		\partial_t\varepsilon = \tau_j^{-1}(w^{j-1}-w^j-\widehat{u_0}^{j-1}+\widehat{u_0}^j)
	\end{align*}
	and use the same arguments as in Lemma \ref{lemma:EllipticBound} to arrive at
	\begin{multline*}
		\langle\nabla\partial_t\varepsilon,\nabla G \rangle
		\leq \tau_j^{-1}2^{\frac1{p^*}}\left(\sum_{K\in\mathcal{T}}\left(C_{P,p^*}^qh_K^q\norm{A^j-A^{j-1}+\Delta (\widehat{u_0}^j-\widehat{u_0}^{j-1})}_{L^q(K)}^q\right.\right.\\
		\left.\left.+\frac{C_{\text{app},p}^qN_\partial^{q-1}}{2^q}h_K\sum_{\sigma\in\mathcal{E}_K\setminus\mathcal{E}^D}\norm{\left\llbracket\nabla (\widehat{u_0}^j-\widehat{u_0}^{j-1})\right\rrbracket}_{L^q(\sigma)}^q\right)\right)^{\frac1q}\norm{\nabla G}_{L^{p^*}(\Omega)}
	\end{multline*}
	Combining the two estimates finishes the proof.
\end{proof}
\begin{thm}\label{thm:max_heat_bound}
	If $\Omega$ satisfies Assumption \ref{ass:uniform} with exponent $1\leq p^*<\infty$, then the $L^\infty(0,T;L^\infty(\Omega))$ norm of $e:=\widehat{u_0}-v_0$ satisfies
	\begin{align*}
		\max_{t\in[0,T]}\norm{e(t)}_{L^\infty(\Omega)} \leq \norm{e(0)}_{L^\infty(\Omega)}+\sum_{j=1}^J\eta_{T,\infty}^j+\sum_{j=1}^J\tau_j\dot{\eta}_{S,q}^jC_{\text{Green},p^*}+\max_{0\leq j\leq J} \eta_{S,q}^jC_{\text{Green},p^*}.
	\end{align*}
	With $\frac1{p^*}+\frac1q=1$ and the estimators $\eta_{T,\infty}^j,\dot{\eta}_{S,q}^j$ defined in Lemma \ref{lemma:ParabolicBound} and $\eta_{S,q}^j$ defined in Lemma \ref{lemma:EllipticBound}.
\end{thm}
\begin{proof}
	With Lemma \ref{lemma:ParabolicBound} and Lemma \ref{lemma:EllipticBound} it follows by induction that
	\begin{align*}
		\max_{t\in[0,T]}\norm{e(t)}_{L^\infty(\Omega)} 
		&\leq \max_{t\in [0,T]}\left(\norm{\rho(t)}_{L^\infty(\Omega)}+\norm{\varepsilon(t)}_{L^\infty(\Omega)}\right)\\
		&\leq \norm{\rho(t_{J-1})}_{L^\infty(\Omega)}+\eta_{T,\infty}^J+\tau_J\dot{\eta}_S^JC_{\text{Green},p^*}+\max_{0\leq j\leq J} \norm{\varepsilon_j}_{L^\infty(\Omega)}\\
		&\leq \norm{e(0)}_{L^\infty(\Omega)}+\sum_{j=1}^J\eta_{T,\infty}^j+\sum_{j=1}^J\tau_j\dot{\eta}_S^jC_{\text{Green},p^*}+\max_{0\leq j\leq J} \eta_{S,q}^jC_{\text{Green},p^*}
	\end{align*}
	where we also used
	\begin{align*}
		\norm{\rho(0)}_{L^\infty(\Omega)}
		\leq \norm{e(0)}_{L^\infty(\Omega)}+\norm{\varepsilon(0)}_{L^\infty(\Omega)}
		\leq \norm{e(0)}_{L^\infty(\Omega)}+\eta_{S,q}^0C_{\text{Green},p^*}.
	\end{align*}
\end{proof}
\subsection{Gradient error bounds}
We now derive an upper bound for $\norm{\nabla (\widehat{u_0}-v_0)(t)}_{L^2([0,T]\times\Omega)}$.
For this we use a classical stability framework for the diffusion equation.
The proof is similar to that in \cite[Proposition 4.5]{bartels2016}.
The bound on $e$ then only relies on the $L^2([0,T];H^1_D(\Omega)')$ norm of the residual $R_0:=-\sum_{i=1}^nR_i$.
\begin{thm}\label{thm:H1_heat_bound}
	The $L^2$-norm in space and time of $\nabla e=\nabla(\widehat{u_0}-v_0)$ can be bounded by
	\begin{multline*}
		\sup_{t\in[0,T]}\norm{e(t)}_\Omega^2+\int_0^T \norm{\nabla e(t)}_{L^2(\Omega)}^2 dt \\
		\leq 2\norm{\widehat{u_0}(0,\cdot)-v_0(0,\cdot)}_{L^2(\Omega)}^2+\sum_{j=0}^J\tau_j\left(\eta_{S,2}^j+C_{F,2,\Gamma_D}\norm{\partial_t(\widehat{u_0}^j-u_{0,h}^j)}_\Omega+\norm{\nabla(\widehat{u_0}^j-\widehat{u_0}^{j-1})}_{\Omega}\right)^2
	\end{multline*}
	with $\eta_{S,2}^j$ defined in Lemma \ref{lemma:EllipticBound}.
\end{thm}
\begin{proof}
	We use the abstract error framework from \cite[Proposition 4.5]{bartels2016}.
	Namely, the error $e:= v_0-\widehat{u_0}$ satisfies
	\begin{align*}
		\sup_{t\in[0,T]}\norm{e(t)}_{\Omega}^2+\norm{\nabla e}_{[0,T]\times\Omega}\leq2\norm{e(0)}_{\Omega}^2+\norm{R_0}_*^2.
	\end{align*}
	We now only need to bound $\norm{R_0}_*^2$.
	At every $t\in(t_{j-1},t_j)$ for $j=1,\dots,J$ is
	\begin{align*}
		\langle R_0(t), f\rangle
		&= \left( \partial_t \widehat{u_0}(t),f\right) +\left( \nabla \widehat{u_0}(t),\nabla f\right)\\
		&= \left( \partial_t (\widehat{u_0}-u_{0,h}),f\right)+\frac1{\tau_j}\left( u_{0,h}^j-u_{0,h}^{j-1},f\right) +\left( \ell_j\nabla \widehat{u_0}^j+\ell_{j-1}\widehat{u_0}^{j-1},\nabla f\right)\\
		&=\left( \nabla \left(\widehat{u_0}^j-w^j\right),\nabla f\right)+ \left( \partial_t (\widehat{u_0}-u_{0,h}),f\right)+\left( (\ell_j-1)\nabla \widehat{u_0}^j+\ell_{j-1}\nabla \widehat{u_0}^{j-1},\nabla f\right)\\
		&=: I+II+III
	\end{align*}
	We bound $I$ using Proposition \ref{prop:MorleyOrth}.
	We perform the same steps as proof of Lemma \ref{lemma:EllipticBound} to arrive at
	\begin{equation}
		\begin{aligned}\label{eqn:R}
			\left( \nabla \left(\widehat{u_0}^j-w^j\right),\nabla f\right)
			&= \sum_{K\in \mathcal{T}}\left(\int_K (-A^j-\Delta \widehat{u_0}^j)(f-\M_K f) dx\right.
			\left.-\frac12\sum_{\sigma\in\mathcal{E}_K\setminus\mathcal{E}^D}\int_{\sigma} \jump{\nabla \widehat{u_0}^j}(f-\M_K f) dS\right)\\
			&\leq\eta_{S,2}^j\norm{\nabla f}_{\Omega}.
		\end{aligned}
	\end{equation}
	For $II$ we use Hölder's inequality and the Poincaré-Friedrichs inequality (Theorem \ref{thm:friedirchsineq}) to derive
	\begin{align}\label{eqn:R_e}
		\left( \partial_t (\widehat{u_0}-u_{0,h}),f\right)
		\leq \norm{\partial_t (\widehat{u_0}-u_{0,h})}_{L^2(\Omega)}\norm{f}_{\Omega}
		\leq C_{F,2,\Gamma_D}\norm{\partial_t (\widehat{u_0}-u_{0,h})}_{\Omega}\norm{\nabla f}_{\Omega}
	\end{align}
	Notice, that \eqref{eqn:R_e} is not optimal in the sense that we estimate $\norm{\partial_t (\widehat{u_0}-u_{0,h})}_{H^{-1}_D(\Omega)}\leq\norm{\partial_t (\widehat{u_0}-u_{0,h})}_\Omega$.
	Since the $L^2$ norm of $\partial_t (\widehat{u_0}-u_{0,h})$ converges linearly similar to the rest of the estimator we may use the $L^2$ norm, which is easier to compute than the $H^1(D)'$ norm, without changing the convergence behavior of the overall estimator.\\
	For $III$ we use Hölder's inequality
	\begin{equation}
		\begin{aligned}\label{eqn:R_T}
			\left( (\ell_j-1)\nabla \widehat{u_0}^j+\ell_{j-1}\nabla \widehat{u_0}^{j-1},\nabla f\right)
			&\leq \norm{(\ell_j-1)\nabla \widehat{u_0}^j+\ell_{j-1}\nabla \widehat{u_0}^{j-1}}_{\Omega}\norm{\nabla f}_{\Omega}\\
			&\leq \norm{\nabla \widehat{u_0}^j-\nabla \widehat{u_0}^{j-1}}_{\Omega}\norm{\nabla f}_{\Omega}.
		\end{aligned}
	\end{equation}
	Combining estimates \eqref{eqn:R}-\eqref{eqn:R_T} yields the desired result.
\end{proof}
\section{Abstract error estimate}\label{section:abstract_error}
We first prove an abstract stability framework for the reduced model \eqref{eqn:reduced} with initial conditions \eqref{eqn:initial} and boundary conditions \eqref{boundaryCond}.
This means we bound the difference $u_i-v_i$ ($i=1,\dots,n$), with $(v_i)_{i=0}^n$ a weak solution in the sense of Theorem \ref{thm:existence} and $(u_i)_{i=0}^n$ a weak solution of a perturbed system \eqref{eqn:fullperturbed}.
This upper bound relies on the $L^\infty([0,T]\times\Omega)$ norm of $u_0-v_0$ and the $L^2([0,T]\times\Omega)$ norm of $\nabla(u_0-v_0)$, that were derived in Section \ref{section:heat}.\\
After that we present an abstract stability framework for the general model \eqref{eqn:general} with the same boundary conditions \eqref{boundaryCond}.
The upper bound for $u_i-v_i$ ($i=1,\dots,n$) follows in the same way as for the reduced model.
We also need an abstract stability framework for $u_0-v_0$ and $\Phi-\Psi$, since we cannot use the bound from Section \ref{section:heat}.\\
For the stability frameworks, we need the following assumptions on the weak solution $(u_0,\dots,u_n,\Phi)$ of a perturbed system \eqref{eqn:fullperturbed}.
Recall $X(q) = L^{\frac{2q}{q-d}}(0,T;L^q(\Omega))$.
\begin{ass}\label{ass:perturbed}
	\begin{enumerate}[(i)]
		\item Assume that the Dirichlet boundary conditions are exactly satisfied by the solution of the perturbed system, i.e.
			\begin{align*}
				u_i = v_i^D\quad \text{on }\Gamma_D.
			\end{align*}
		\item We assume that the solvent concentration fulfills $\nabla u_0\in X(q)$ for $q>d$.
			If $z\neq 0$, we assume additionally that $F:=\nabla u_0-u_0z\beta\nabla\Phi\in X(q)$ and $\nabla u_i\in L^\infty(0,T;L^{\tilde{q}}(\Omega))$ for some $q,\tilde{q}>2$ for $d=2$ and $q,\tilde{q}\geq 3$ for $d=3$.
	\end{enumerate}
\end{ass}
\begin{remark}
	\begin{enumerate}[(a)]
		\item Assumption \ref{ass:perturbed} (i) states, that the perturbed system satisfies the same Dirichlet boundary conditions as the original system.
			If we did not impose Assumption (i), we would have to bound
			\begin{align*}
				\langle u_0\partial_n u_i-u_i\partial_n u_0-v_0\partial_n v_i+v_i\partial_n v_0,u_i-v_i\rangle_{H^{-\frac12}(\Gamma_D),H^{\frac12}(\Gamma_D)}
			\end{align*}
			in the proof of Theorem \ref{thm:abstract_error} and Theorem \ref{thm:abstract_general}.
			It is unclear how this can be done.
			In most applications this assumption is not a severe restriction, since the Dirichlet boundary conditions are constant on connected subsets, that model connections to comparably large reservoirs with constant concentration.
		\item We do not assume that the Neumann boundary condition is exactly satisfied.
			The perturbed Neumann boundary conditions are incorporated in the right hand side $R_i\in H^1_D(\Omega)$. % and shows up in the residual bound in Section \ref{Section:Residual_bound}.
			Hence, we do not see the boundary error explicitly in this abstract setting.
		\item We need the additional regularity assumptions in (i) for the full model, since we cannot utilize the classical maximum regularity results for the diffusion equation.
	\end{enumerate}
\end{remark}
\subsection{Abstract stability framework for the reduced system}\label{subsec:stability_reduced}
We now establish a stability framework for the simplified ion transport model.
Let, in this subsection, $(v_i)_{i=0,\dots,n}$ be a weak solution of \eqref{eqn:reduced} and $(u_i)_{i=0,\dots,n}$ be a weak solution of the perturbed system
\begin{align}\label{eqn:simpert}
	&\begin{aligned}
	\partial_t u_i - \operatorname{div}(u_0\nabla u_i-u_0\nabla u_i) &= R_i\quad i=1,\dots,n\\
	\sum_{i=0}^n u_i &= 1
	\end{aligned}\quad \text{in }\Omega\times(0,T)
\end{align}
with $R_i\in L^2(0,T;H^1_D(\Omega)')$ and $(u_0,\dots,u_n)$ satisfying Assumption \ref{ass:perturbed}.
In Section \ref{section:residual}, the solution to the perturbed system will be a suitable reconstruction of a numerical solution.
In that case, $u_0$ satisfies Assumption \ref{ass:perturbed} if the Dirichlet boundary conditions are piecewise linear, since it is a continuous piecewise polynomial function (see Section \ref{section:scheme}).
For this abstract stability framework, it does not matter how the solution to the perturbed system is obtained, as long as it has the required regularity.
This abstract stability framework can be used with other numerical methods or different reconstructions, since we only impose minimal conditions on the residuals $R_i$ and the regularity conditions on $u_0,\dots,u_n$.\\
We write $\langle\cdot,\cdot\rangle$ for the duality pairing of $H^1_D(\Omega)'$ and $H^1_D(\Omega)$.
To obtain the bound, we subtract the weak formulations for \eqref{eqn:reduced} and \eqref{eqn:simpert} and test with $u_i-v_i$.
In the end, we use the classical Gronwall lemma to obtain a stability framework.
Note, that according to \cite[Section 5.9, Theorem 3.9]{evans2010} $v_i\in C([0,T],L^2(\Omega))$ for all $i=0,\dots,n$ and $T>0$.
With the positivity of the heat semigroup $(e^{t\Delta})_{t\geq0}$ follows that $\gamma<v_0(t,x)$ for all $(t,x)\in[0,T]\times\Omega$ (see \cite[Section 4.1]{ouhabaz2009}).
\begin{thm}\label{thm:abstract_error}
	Let $(v_0,\dots,v_n)$ be a weak solution of \eqref{equations}-\eqref{eqn:initial} as in Theorem \ref{thm:existence} and $(u_0,\dots,u_n)$ be a weak solution of \eqref{eqn:simpert}.
	Under Assumption \ref{assumptions} and \ref{ass:perturbed}, the difference $u_i-v_i$ for $i=1,\dots,n$ can be bounded as follows
	\begin{multline*}
		\max_{t\in[0,T]} \norm{u_i-v_i}_{\Omega}^2+\int_0^{T}\norm{\sqrt{v_0}\nabla(u_i-v_i)}_{\Omega}^2\,dt
		\leq\\
		\left(2\norm{u_i^0-v_i^0}^2_{\Omega}
		+\frac{12}\gamma\norm{v_0-u_0}_{L^\infty([0,T]\times\Omega)}^2\norm{\nabla u_i}_{[0,T]\times\Omega}^2
		+\frac{12}\gamma\norm{\nabla(u_0-v_0)}_{[0,T]\times\Omega}^2
		+\frac{12}\gamma \norm{R_i}_{*}^2 \right)\\
		\exp\left(2C_G^{\frac{2}{1-\theta}}(1+C_{F,2,\Gamma_D})^{\frac{\theta}{1-\theta}}\frac{\mu}{\gamma^{\frac{1+\theta}{1-\theta}}}\norm{\nabla u_0}_{X(q)}^{\frac{2}{1-\theta}}\right).
	\end{multline*}
	with $\theta = \frac d2-\frac dp$ and $\mu= \frac{1-\theta}{2}\left(\frac{1}{2(1+\theta)}\right)^\frac{\theta+1}{\theta-1}$.
\end{thm}
\begin{proof}
	Subtracting the weak formulations for $u_i$ and $v_i$ and testing with $u_i-v_i$ yields
	\begin{align*}
		\frac12\frac{d}{dt}\int_\Omega(u_i-v_i)^2\,dx
		= &-\int_\Omega \left(u_0\nabla u_i-u_i\nabla u_0-v_0\nabla v_i+v_i\nabla v_0\right)\cdot\left(\nabla (u_i-v_i)\right)\,dx\\
		&+\langle R_i,u_i-v_i\rangle.
	\end{align*}
	Rearranging terms and using Hölder's inequality yields
	\begin{equation}\label{eqn:abstract_err_gron}
		\begin{aligned}
		\frac12\frac{d}{dt}\norm{u_i-v_i}_{\Omega}^2
		\leq &-\norm{\sqrt{v_0}\nabla(u_i-v_i)}_{\Omega}^2+\norm{v_0-u_0}_{L^\infty(\Omega)}\norm{\nabla u_i}_{\Omega}\norm{\nabla(u_i-v_i)}_{\Omega}\\
		&+\norm{v_i}_{L^\infty(\Omega)}\norm{\nabla(u_0-v_0)}_{\Omega}\norm{\nabla(u_i-v_i)}_{\Omega}\\
		&+\norm{u_i-v_i}_{L^p(\Omega)}\norm{\nabla u_0}_{L^q(\Omega)}\norm{\nabla(u_i-v_i)}_{\Omega}\\
		&+\norm{R_i}_{H^1_D(\Omega)'}\norm{\nabla (u_i-v_i)}_{\Omega},
		\end{aligned}
	\end{equation}
	with $\frac1p+\frac1q=\frac12$ and $q>d$.
	To bound the $L^p$ norm of $u_i-v_i$ we use the Gagliardo-Nirenberg inequality (Theorem \ref{thm:Nirenberg}) with $\theta=\frac d2-\frac dp$
	\begin{equation}
	\begin{aligned}
		\norm{u_i-v_i}_{L^p(\Omega)}\label{eq:Nirenberg}
		&\leq C_G\norm{u_i-v_i}_{H^1(\Omega)}^\theta\norm{u_i-v_i}_{\Omega}^{1-\theta}\\
		&\leq C_G\left(1+C_{F,2,\Gamma_D}^2\right)^{\frac{\theta}{2}}\norm{u_i-v_i}_{\Omega}^{1-\theta}\norm{\nabla(u_i-v_i)}_{\Omega}^{\theta}.
	\end{aligned}
	\end{equation}
	Using \eqref{eqn:abstract_err_gron},\eqref{eq:Nirenberg} and Young's inequality we arrive at
	\begin{align*}
		\frac12\frac{d}{dt}\norm{u_i-v_i}_{\Omega}^2
		\leq &-\frac12\norm{\sqrt{v_0}\nabla(u_i-v_i)}_{\Omega}^2
		+\frac6{\gamma}\norm{v_0-u_0}_{L^\infty(\Omega)}^2\norm{\nabla u_i}_{\Omega}^2
		+\frac6{\gamma}\norm{v_i}_{L^\infty(\Omega)}^2\norm{\nabla(u_0-v_0)}_{\Omega}^2\\
		&+C_G^\frac2{1-\theta}(1+C_{F,2,\Gamma_D}^2)^{\frac{\theta}{1-\theta}}\frac{\mu}{\gamma^{\frac{1+\theta}{1-\theta}}}\norm{\nabla u_0}_{L^q(\Omega)}^{\frac{2}{1-\theta}}\norm{u_i-v_i}_{\Omega}^2
		+\frac6{\gamma}\norm{R_i}_{H^1_D(\Omega)'}^2,
	\end{align*}
	with $\mu := \frac{1-\theta}{2}\left(\frac{1}{2(1+\theta)}\right)^\frac{\theta+1}{\theta-1}$.
	The claim then follows with the classical Gronwall inequality.
	\begin{multline*}
		\max_{t\in[0,T]} \norm{u_i-v_i}_{\Omega}^2+\int_0^{T}\norm{\sqrt{v_0}\nabla(u_i-v_i)}_{\Omega}^2\,dt
		\leq\\
		\left(2\norm{u_i^0-v_i^0}^2_{\Omega}
		+\frac{12}\gamma\norm{v_0-u_0}_{L^\infty([0,T]\times\Omega)}^2\norm{\nabla u_i}_{[0,T]\times\Omega}^2
		+\frac{12}\gamma\norm{\nabla(u_0-v_0)}_{[0,T]\times\Omega}^2
		+\frac{12}\gamma \norm{R_i}_{*}^2 \right)\\
		\exp\left(2C_G^{\frac{2}{1-\theta}}(1+C_{F,2,\Gamma_D}^2)^{\frac{\theta}{1-\theta}}\frac{\mu}{\gamma^{\frac{1+\theta}{1-\theta}}}\norm{\nabla u_0}_{X(q)}^{\frac{2}{1-\theta}}\right).
	\end{multline*}
\end{proof}
\subsection{The general model}\label{abserror:general}
We now derive a stability framework for the general model.
For this let $(u_0,\dots,u_n,\Phi)$ be a weak solution of the perturbed system \eqref{eqn:fullperturbed} satisfying Assumption \ref{assumptions}.
Let $(v_0,\dots,v_n,\Psi)$ be a weak solution for the general model \eqref{eqn:general} in the sense of Theorem \ref{thm:existence}.
The main differences are that we now need a stability framework for the electric potential and that the solvent no longer solves the diffusion equation and we do not have a good bound for the difference $v_0-u_0$ in $L^\infty([0,T]\times\Omega)$.\\
We first derive a stability framework for the electric potential.
To do so, we note that the electric potential solves the Poisson equation.
Therefore, we can use the classical stability estimate for the Poisson equation obtained by the Poincar\'e inequality
\begin{align*}
	\norm{\nabla (\Phi-\Psi)}_{\Omega} \leq C_{F,2,\Gamma_D}\frac{\abs{z}}{\lambda^2}\norm{u_0-v_0}_{\Omega}+\frac1{\lambda^2}\norm{R_\Phi}_{H^1_D(\Omega)'}.
\end{align*}
For the ion concentrations we can perform similar steps as in Theorem \ref{thm:abstract_error} with $F:=\nabla u_0-z\beta u_0\nabla \Phi$ and $G:=\nabla v_0-z\beta v_0\nabla \Psi$ instead of $\nabla u_0$ and $\nabla v_0$ respectively.
Furthermore, to estimate $\norm{(u_0-v_0)\nabla u_i}_{[0,T]\times\Omega}$ without using $\norm{u_0-v_0}_{L^\infty([0,T]\times\Omega)}$ we utilize the Sobolev inequality
\begin{align*}
	\int_0^T \langle (u_0-v_0)\nabla u_i,\nabla (u_i-v_i)\rangle\,dt
	&\leq \int_0^T\norm{u_0-v_0}_{L^p(\Omega)}\norm{\nabla u_i}_{L^{\tilde{q}}(\Omega)}\norm{\nabla(u_i-v_i)}_{\Omega}\,dt\\
	&\leq C_S\norm{u_0-v_0}_{L^2(0,T;H^1(\Omega))}\norm{\nabla u_i}_{L^\infty(0,T;L^{\tilde{q}}(\Omega))}\norm{\nabla (u_i-v_i)}_{L^2(0,T;L^2(\Omega))},
\end{align*}
with $\frac1p+\frac1{\tilde{q}}=\frac12$ and $\tilde{q}>2$ for $d=2$ and $\tilde{q}\geq 3$ for $d=3$.
Therefore, here we need Assumption \ref{ass:perturbed} (ii), i.e. $\nabla u_i\in L^\infty(0,T;L^{\tilde{q}}(\Omega))$ instead of $L^2(0,T;L^2(\Omega))$.
This gives us the abstract error estimate
\begin{multline*}
	\max_{t\in[0,T]} \norm{u_i-v_i}_{\Omega}^2+\int_0^{T}\norm{\sqrt{v_0}\nabla(u_i-v_i)}_{\Omega}^2\,dt
	\leq\\
	\left(2\norm{u_i^0-v_i^0}^2_{\Omega}
	+\frac{12}\gamma\norm{\nabla (v_0-u_0)}_{[0,T]\times\Omega}^2\norm{\nabla u_i}_{L^\infty(0,T;L^q(\Omega))}^2
	+\frac{12}\gamma\norm{F-G}_{[0,T]\times\Omega}^2
	+\frac{12}\gamma \norm{R_i}_{*}^2 \right)\\
	\exp\left(2C_G^{\frac{2}{1-\theta}}(1+C_{F,2,\Gamma_D}^2)^{\frac{\theta}{1-\theta}}\frac{\mu}{\gamma^{\frac{1+\theta}{1-\theta}}}\norm{F}_{X(q)}^{\frac{2}{1-\theta}}\right)
\end{multline*}
with $\theta$ and $\mu$ from Theorem \ref{thm:abstract_error}.
Furthermore, we can bound the difference $F-G$ by the use of Sobolev inequality and Poincar\'e-Friedrichs inequality, Theorem \ref{thm:friedirchsineq}
\begin{align*}
	\norm{F-G}_{[0,T]\times\Omega}^2
	&\leq\int_0^T\left(\norm{\nabla (u_0-v_0)}_{\Omega}+\abs{z\beta}\norm{(u_0-v_0)\nabla\Phi}_{\Omega}+\abs{z\beta}\norm{v_0\nabla(\Phi-\Psi)}_{\Omega}\right)^2\,dt\\
	&\leq\int_0^T\left(\norm{\nabla (u_0-v_0)}_{\Omega}+\abs{z\beta}\norm{u_0-v_0}_{L^p(\Omega)}\norm{\nabla\Phi}_{L^{\tilde{q}}(\Omega)}+\abs{z\beta}\norm{\nabla(\Phi-\Psi)}_{\Omega}\right)^2\,dt\\
	&\leq2\norm{\nabla (u_0-v_0)}_{[0,T]\times\Omega}^2\left(1+C_S\sqrt{1+C_{F,2,\Gamma_D}^2}\abs{z\beta}\norm{\nabla\Phi}_{L^\infty(0,T;L^{\tilde{q}}(\Omega))} +C_{F,2,\Gamma_{D}}\frac{\abs{z}^2\beta}{\lambda^2}\right)^2\\
	&\;+2\frac{\abs{z\beta}^2}{\lambda^2}\norm{R_\Phi}_{*}^2,
\end{align*}
where we used $v_0\leq 1$ in the second estimate.
We now turn our attention to the solvent concentration.
Since the solvent concentration does not solve the diffusion equation in the general setting, we cannot use the classical stability framework here.
To obtain a stability framework for the solvent concentration, we test the weak formulation for the solvent concentration with $u_0-v_0$ and use Hölder's inequality and $v_0\leq$ to arrive at
\begin{align*}
	\frac12\frac{d}{dt}\norm{u_0-v_0}_{\Omega}^2
	&\leq -\norm{\nabla (u_0-v_0)}_{\Omega}^2+\abs{z\beta}\norm{u_0-v_0}_{L^p(\Omega)}\norm{\nabla\Phi}_{L^{q}(\Omega)}\norm{\nabla(u_0-v_0)}_{\Omega}\\
	&+\frac{\abs{z\beta}}4\norm{\nabla(\Phi-\Psi)}_{\Omega}\norm{\nabla(u_0-v_0)}_{\Omega}+\norm{R_0}_{H^1_D(\Omega)'}\norm{\nabla(u_0-v_0)}_\Omega.
\end{align*}
We need to bound the $L^p$ norm of the difference $u_0-v_0$.
In this case we cannot use the Sobolev inequality, because then we cannot apply Gronwall's inequality.
Therefore, we use the Gagliardo-Nirenberg inequality with $\theta =\frac d2-\frac dp$ similar to equation \eqref{eq:Nirenberg} to arrive at
\begin{align*}
	\frac12\frac{d}{dt}\norm{u_0-v_0}_{\Omega}^2
	&\leq -\frac12\norm{\nabla (u_0-v_0)}_{\Omega}^2
	+C_G^{\frac2{1-\theta}}\mu\norm{u_0-v_0}^2_{\Omega}\norm{\nabla\Phi}^{\frac2{1-\theta}}_{L^q(\Omega)}\\
	&+\frac{\abs{z\beta}^2}{8}\left(C_{F,2,\Gamma_D}^2\frac{\abs{z}^2}{\lambda^4}\norm{u_0-v_0}^2_{\Omega}+\norm{R_\Phi}^2_{H^1_D(\Omega)'}\right)+2\norm{R_0}_{H^1_D(\Omega)'}^2.
\end{align*}
Applying Gronwall's inequality yields
\begin{multline*}
	\max_{t\in[0,T]}\norm{u_0-v_0}_{\Omega}^2+\int_0^T\norm{\nabla (u_0-v_0)}_{\Omega}^2\,dt
	\leq \left(2\norm{u_0^0-v_0^0}_{\Omega}^2
	+\frac{\abs{z\beta}^2}{4}\norm{R_\Phi}_{*}^2+4\norm{R_0}_{*}^2\right)\\
	\exp\left(2C_G^{\frac{2}{1-\theta}}(1+C_{F,2,\Gamma_D}^2)^{\frac{\theta}{\theta-1}}\mu\norm{\nabla\widehat{\Phi}}_{X(q)}^{\frac{2}{1-\theta}}+C_{F,2,\Gamma_D}^2\frac{\abs{z}^4\beta^2}{8\lambda^4}\right).
\end{multline*}
We now summarize the inequalities from above in one theorem.
\begin{thm}\label{thm:abstract_general}
	Let Assumptions \ref{assumptions} and \ref{ass:perturbed} hold.
	Let $(u_0,\dots,u_n, \Phi)$ be a weak solution of \eqref{eqn:fullperturbed} and $(v_0,\dots,v_n,\Psi)$ be a weak solution of \eqref{equations}.
	The electric potentials $\Phi$ and $\Psi$ satisfy
	\begin{align*}
		\norm{\nabla (\Phi-\Psi)}_{\Omega} \leq C_{F,2,\Gamma_D}\frac{\abs{z}}{\lambda^2}\norm{u_0-v_0}_{\Omega}+\norm{R_\Phi}_{H^1_D(\Omega)'}.
	\end{align*}
	The difference $u_0-v_0$ can be bounded by	
	\begin{multline*}
		\max_{t\in[0,T]}\norm{u_0-v_0}_{\Omega}^2+\int_0^T\norm{\nabla (u_0-v_0)}_{\Omega}^2\,dt
		\leq \left(2\norm{u_0^0-v_0^0}_{\Omega}^2+\frac{\abs{z\beta}^2}{4}\norm{R_\Phi}_{*}^2+4\norm{R_0}_{*}^2\right)\\
		\exp\left(2C_G^{\frac{2}{1-\theta}}(1+C_{F,2,\Gamma_D}^2)^{\frac{\theta}{1-\theta}}\mu\norm{\nabla\widehat{\Phi}}_{X(q)}^{\frac{2}{1-\theta}}+C_{F,2,\Gamma_D}^2\frac{\abs{z}^4\beta^2}{8\lambda^4}\right).
	\end{multline*}
	The error $u_i-v_i$ for $i=1,\dots,n$ satisfies
	\begin{multline*}
		\max_{t\in[0,T]} \norm{u_i-v_i}_{\Omega}^2+\int_0^{T}\norm{\sqrt{v_0}\nabla(u_i-v_i)}_{\Omega}^2\,dt
		\leq\\
		\left(\norm{u_i^0-v_i^0}^2_{\Omega}
		+\frac{5}\gamma\norm{\nabla (v_0-u_0)}_{[0,T]\times\Omega}^2\norm{\nabla u_i}_{L^\infty(0,T;L^{\tilde{q}}(\Omega))}^2
		+\frac{5}\gamma\norm{F-G}_{[0,T]\times\Omega}^2
		+\frac{5}\gamma\norm{R_i}_{*}^2 \right)\\
		\exp\left(2C_G^{\frac{2}{1-\theta}}(1+C_{F,2,\Gamma_D}^2)^{\frac{\theta}{1-\theta}}\frac\mu{\gamma^{\frac{1+\theta}{1-\theta}}}\norm{F}_{X(q)}^{\frac{2}{1-\theta}}\right).
	\end{multline*}
	with $\tilde{q},q>2$ for $d=2$ and $\tilde{q},q\geq 3$ for $d=3$ and $\theta=\frac d2-\frac dp,\mu=\frac{1-\theta}{2}\left(\frac{1}{2(1+\theta)}\right)^\frac{\theta+1}{\theta-1}$.
\end{thm}
Let us comment on our stability framework and its relation to commonly used methods for stability and uniqueness of cross diffusion systems.
\begin{remark}\label{Whyl2}
	\textcite{gerstenmayer2018} showed a uniqueness result for the general model \eqref{eqn:general} using the Gajewski method.
	One uses the entropy functional
	\begin{align*}
		H[f] = \int_\Omega f(\log(f)-1)\,dx
	\end{align*}
	for $f\in H^1(\Omega)$.
	Following the route of Gajewski (see \cite{gajewski1994}), this leads to the metric
	\begin{align*}
		d(f,g) = \int_\Omega f\log(f)+g\log(g)-2\left(\frac{f+g}2\right)\log\left(\frac{f+g}2\right)\,dx.
	\end{align*}
	This metric can be seen as a symmetrization of the often used relative entropy. (see \cite[Remark 4]{chen2018})\\
	This entropy structure is often used to prove weak uniqueness and weak-strong uniqueness results for cross-diffusion systems see also \cite{burger2010,zamponi2015,berendsen2020,jungel2015,hopf2022}.
	Instead, we use the $L^2$ norm to establish an abstract stability framework.
	Using the $L^2$ metric in this context is advantageous.
	The $L^2$ metric is weaker in the sense of
	\begin{align*}
		\norm{f-g}_{L^2(\Omega)}^2\leq \norm{\sqrt{f}-\sqrt{g}}_{L^2(\Omega)}^2\leq d(f,g)
	\end{align*}
	for $f,g\in H^1(\Omega)$ with $0\leq f,g\leq 1$.
	This allows us to obtain an a posteriori error estimate.
	More precisely, in the Gajewski metric setting the equation for $u_i$ is tested with $\log\left(\frac{2u_i}{u_i+v_i}\right)$.
	It is unclear to us how to bound the residual tested with this function, since $\log\left(\frac{2u_i}{u_i+v_i}\right)\notin H^1(\Omega)$ for $u_i=0$ outside of a set of measure zero.
	Whether this problem can be overcome by the use of another numerical method or by reconstructing in a different way is an interesting question beyond the scope of this paper.\\
	One reason the Gajewski approach is used in the weak-strong uniqueness proof is that \textcite{gerstenmayer2018} aim to avoid the assumption that $\nabla u_0\in L^\infty(0,T;L^q(\Omega))$ for one solution.
	In the setting of a posteriori analysis, the role of $u_0$ is played by a continuous piecewise polynomial function and therefore $\nabla u_0\in L^\infty(0,T;L^q(\Omega))$ is satisfied.
	Furthermore, in the bound we only use the $L^2(0,T;L^q(\Omega))$ norm of $\nabla u_0$ and this is bounded if $\nabla u_0^0\in L^2(\Omega)$ and $q<\frac{2d}{d-2}$.\\
	The question remains, why is an $L^2$ stability result available for the model \eqref{equations}-\eqref{eqn:initial}?
	One part of the answer is that the $L^2$ metric is also a Gajewski metric, but with entropy functional
	\begin{align*}
		H[f] = \int_\Omega f^2\,dx.
	\end{align*}
	In this case the diffusion matrix is given by
	\begin{align*}
		(A(u))_{i,j} =
		\begin{cases}
			1-\sum_{\substack{j=1\\i\neq j}}^n u_j &\text{if } i=j\\
			u_i &\text{if } i\neq j.
		\end{cases}
	\end{align*}
	One can easily see that the matrix $A(u)$ has $n$ eigenvectors and spectrum $\{1,u_0\}$ and is therefore positive semidefinite.
\end{remark}
\section{A posteriori error estimator}\label{section:residual}
We now establish an a posteriori error estimate using the abstract error framework from Section \ref{section:abstract_error} and the reconstruction defined in Section \ref{section:scheme}.
The only thing left is to establish a computable upper bound, is to bound the residuals $R_i$ in the $L^2(0,T;H^1_D(\Omega)')$ norm.
To achieve this we use the properties of the Morley type reconstruction from Section \ref{subsec:Morley}.
We prove the first key property in Lemma \ref{orth_recon}.
The residual for the reduced model is given by
\begin{align*}
	\langle R_i,f\rangle
	= \int_\Omega \partial_t \widehat{u_i}\,f\,dx +\int_\Omega \left(\widehat{u_0}\nabla \widehat{u_i}-\widehat{u_i}\nabla \widehat{u_0}\right)\cdot\nabla f\,dx
\end{align*}
for all $f\in H^1_D(\Omega)$.
We first present the bound of the residual for the reduced model \eqref{eqn:reduced} in detail.
Together with the abstract stability framework and bounds on the difference $u_0-v_0$ from Section \ref{section:heat}, we can state the a posteriori error estimate for the difference $u_i-v_i$ ($i=1,\dots,n$).\\
The bound for the general model can be deduced in a similar fashion. 
We state the residual bound and a posteriori estimate for the general model in Section \ref{subsec:general_residual}.

\subsection{Bounding the $H^1_D(\Omega)'$ Norm of the residual for the reduced model}\label{Section:Residual_bound}
The next property of the reconstruction $\widehat{u_i}$ ($i=1,\dots,n$) is crucial to estimate the element residual in Theorem \ref{thm:Residual_bound}.
Since we do not reconstruct with respect to convection, we get an error term on the right hand side, that is not present in \cite[Lemma 5.2]{nicaise2006}.
\begin{lemma}\label{orth_recon}
	The reconstruction $\widehat{u_i}$ satisfies
	\begin{align*}
		\int_K \partial_t u_{i,h}^j dx -\int_K \operatorname{div}(\widehat{u_0}^j\nabla \widehat{u_i}^j-\widehat{u_i}^j\nabla \widehat{u_0}^j) dx=-\sum_{\sigma\in\mathcal{E}_K}\left(u_{i,\sigma}^jF_{\sigma}^K(u_0^j)-\int_\sigma \widehat{u_i}^j\nabla \widehat{u_0}^j\cdot n_{K,\sigma}\, dS\right)
	\end{align*}
	for all $K\in\mathcal{T}$ and $i=1,\dots,n$.
\end{lemma}
\begin{proof}
	We first use Gauss' theorem on every element $K\in\mathcal{T}$ to arrive at
	\begin{align*}
		\int_K \partial_t u_{i,h}^j dx -\int_K \operatorname{div}(\widehat{u_0}^j\nabla \widehat{u_i}^j-\widehat{u_i}^j\nabla \widehat{u_0}^j) dx
		&= \int_K \partial_t u_{i,h}^j dx -\sum_{\sigma\in\mathcal{E}_K}\int_\sigma \left(\widehat{u_0}^j\nabla \widehat{u_i}^j-\widehat{u_i}^j\nabla \widehat{u_0}^j\right)\cdot n_{K,\sigma} dS\\
		&= \int_K \partial_t u_{i,h}^j dx
		-\sum_{\sigma\in\mathcal{E}_K}\left(u_{0,\sigma}^jF^K_{\sigma}(u_i^j)-u_{i,\sigma}^jF_{\sigma}^K(u_0^j)\right)\\
		&-\sum_{\sigma\in\mathcal{E}_K}\left(u_{i,\sigma}^jF_{\sigma}^K(u_0^j)-\int_\sigma \widehat{u_i}^j\nabla \widehat{u_0}^j\cdot n_{K,\sigma}\, dS\right).
	\end{align*}
	The claim now follows from the definition of the numerical scheme.
\end{proof}
\begin{remark}
	Notice that we used $\partial_t u_{i,h}^j$ in Lemma \ref{orth_recon} with $u_i$ the piecewiese constant in space finite volume solution and not the reconstruction $\widehat{u_i}$.
	The reason for this is, that the finite volume solution solves the equation
	\begin{align*}
		m(K)\partial_t u_{i,h}^j = \sum_{\sigma\in\mathcal{E}_K} u_{0,\sigma}^j F_{\sigma}^K(u_i^j)-u_{i,\sigma}^j F_\sigma^K(u_0^j)
	\end{align*}
	for all $i=1,\dots,n$, $j=0,\dots,J$ and $K\in\mathcal{T}$.
	The reconstruction is done in such a way, that we treat $\partial_t u_{i,h}^j$ as a right hand side.
	Therefore, similar to \cite[Lemma 5.2]{nicaise2006} we have to use $\partial_t u_{i,h}^j$ instead of $\partial_t \widehat{u_i}^j$.
\end{remark}
We can now bound the $H^1_D(\Omega)'$ norm of $R_i$.
\begin{thm}\label{thm:Residual_bound}
	The residual $R_i$ is bounded in $H^1_D(\Omega)'$ by
	\begin{align*}
		\norm{R_i(t)}_{H^1_D(\Omega)'} &\leq R_{i,S}^j+R_{i,T}^j+R_{i,R}^j\quad\forall t\in[t_{j-1},t_j]
	\end{align*}
	where the spacial bound $R_{i,S}^j$, temporal bound $R_{i,T}^j$ and reconstruction bound $R_{i,R}^j$ are given by
	\begin{align*}
		R_{i,S}^j &= \sqrt{2}\left(\sum_{K\in\mathcal{T}}\left(h_K^2C_{P,2}^2\norm{\partial_t u_{i,h}^j-\operatorname{div}\left(\widehat{u_0}^j\nabla \widehat{u_i}^j-\widehat{u_i}^j\nabla \widehat{u_0}^j\right)}_K^2\right.\right.\\
		&\left.\left.+\frac{N_\partial}{4} C_{\text{app},2}^2h_K\sum_{\sigma\in\mathcal{E}_K\setminus\mathcal{E}^D}\norm{\llbracket \widehat{u_0}^j\nabla \widehat{u_i}^j-\widehat{u_i}^j\nabla \widehat{u_0}^j\rrbracket}_\sigma^2\right)\right)^\frac12\\
		R_{i,T}^j &= \norm{\widehat{u_0}^j\nabla \widehat{u_i}^j-\widehat{u_i}^j\nabla \widehat{u_0}^j-\widehat{u_0}^{j-1}\nabla \widehat{u_i}^{j-1}+\widehat{u_i}^{j-1}\nabla \widehat{u_0}^{j-1}}_\Omega\\
		R_{i,R}^j &=C_{F,2,\Gamma_D}\norm{\partial_t (\widehat{u_i}^j-u_{i,h}^j)}_{\Omega} +\frac{C_{\text{app},2}N_\partial^\frac12}{2}\left(\sum_{K\in\mathcal{E}}\sum_{\sigma\in\mathcal{E}_K\setminus\mathcal{E}^N}h_\sigma\norm{\widehat{u_i}^j\dc{\nabla \widehat{u_0}^j}\cdot n_{K,\sigma}-u_{i,\sigma}\frac{F_{\sigma}^K(u_0^j)}{\abs{\sigma}}}_{\sigma}^2\right)^\frac12.
	\end{align*}
\end{thm}
\begin{proof}
	Let $f\in H^1_D(\Omega)$.
	We split the residual into a temporal part $II$ and the rest $I$
	\begin{multline*}
		\langle R[u]_i,f\rangle
		= \underbrace{\langle \partial_t \widehat{u_i},f\rangle +\langle \widehat{u_0}^j\nabla \widehat{u_i}^j-\widehat{u_i}^j\nabla \widehat{u_0}^j,\nabla f\rangle}_{=:I}\\
		+\underbrace{\langle \widehat{u_0}\nabla \widehat{u_i}-\widehat{u_i}\nabla \widehat{u_0}-\widehat{u_0}^j\nabla \widehat{u_i}^j+\widehat{u_i}^j\nabla \widehat{u_0}^j,\nabla f\rangle}_{=:II}\\
	\end{multline*}
	For the first part $I$, we use Gauss' theorem on every triangle $K\in\mathcal{T}$ and Lemma \ref{orth_recon} to arrive at
	\begin{align*}
		I &=\int_\Omega \partial_t (\widehat{u_i}^j-u_{i,h}^j)f\,dx+\sum_{K\in\mathcal{T}} \int_K (\partial_t u_{i,h}^j - \operatorname{div}(\widehat{u_0}^j\nabla \widehat{u_i}^j-\widehat{u_i}^j\nabla \widehat{u_0}^j))f\, dx\\
		&+\sum_{\sigma\in\mathcal{E}_K\setminus\mathcal{E}^D}\int_\sigma (\widehat{u_0}\nabla \widehat{u_i} - \widehat{u_i}\nabla \widehat{u_0})\cdot n_{K,\sigma} f\, dS\\
		&=\underbrace{\int_\Omega \partial_t (\widehat{u_i}^j-u_{i,h}^j)f\,dx}_{=:I_1}+\underbrace{\sum_{K\in\mathcal{T}} \int_K (\partial_t u_{i,h}^j - \operatorname{div}(\widehat{u_0}^j\nabla \widehat{u_i}^j-\widehat{u_i}^j\nabla \widehat{u_0}^j))(f-\M_Kf) dx}_{=:I_2}\\
		&\underbrace{-\sum_{K\in\mathcal{T}}\sum_{\sigma\in\mathcal{E}_K}\left(u_{i,\sigma}^jF_{\sigma}^K(u_0^j)-\int_\sigma \widehat{u_i}^j\nabla \widehat{u_0}^j\cdot n_{K,\sigma}\,dS\right)\M_Kf +\sum_{\sigma\in\mathcal{E}_K\setminus\mathcal{E}^D}\int_\sigma\left(\widehat{u_0}^j\nabla \widehat{u_i}^j -\widehat{u_i}^j\nabla \widehat{u_0}^j\right)\cdot n_{K,\sigma} f\, dS}_{=:I_3} 
	\end{align*}
	For $I_1$, we use Hölder's inequality and Poincar\'e-Friedrichs inequality (Theorem \ref{thm:friedirchsineq}) to arrive at
	\begin{align}\label{eqn:bound_I_1}
		I_1\leq C_{F,2,\Gamma_D}\norm{\partial_t(\widehat{u_i}^j-u_{i,h}^j)}_\Omega\norm{\nabla f}_\Omega.
	\end{align}
	Notice, that similar to the proof of Theorem \ref{thm:H1_heat_bound} we estimated the $H^{-1}_D$-norm of $\partial_t(\widehat{u_i}^j-u_{i,h}^j)$ with the $L^2$-norm.
	Similarly, the $L^2$-norm converges linearly in $h$ and therefore we use the $L^2$-norm rather than the $H^{-1}_D$-norm.\\
	For $I_2$ we also use Hölder's inequality and Poincar\'e inequality on every triangle
	\begin{align}\label{eqn:bound_I_2}
		I_2 
		\leq C_{\text{P,2}}\sum_{K\in\mathcal{T}}h_K\norm{\partial_t u_{i,h}^j - \operatorname{div}\left(\widehat{u_0}^j\nabla \widehat{u_i}^j-\widehat{u_i}^j\nabla \widehat{u_0}^j\right)}_K\norm{\nabla f}_K.
	\end{align}
	For the last part $I_3$, we use the properties of the reconstruction to derive that
	\begin{align}\label{eqn:jump_i}
		\sum_{K\in\mathcal{T}}\sum_{\sigma\in\mathcal{E}_K\setminus\mathcal{E}^D}\int_\sigma\llbracket \widehat{u_0}^j\nabla \widehat{u_i}^j\rrbracket f\,dS
		=\sum_{K\in\mathcal{T}}\sum_{\sigma\in\mathcal{E}_K\setminus\mathcal{E}^D}\int_\sigma\llbracket \widehat{u_0}^j\nabla \widehat{u_i}^j\rrbracket (f-\M_K f)\,dS.
	\end{align}
	and
	\begin{align}\label{eqn:jump_av_equation}
		\int_\sigma \widehat{u_i}^j\nabla\widehat{u_0}^j\cdot n_{K,\sigma}(f-\M_Kf)\,dS
		= \int_\sigma \left(\frac12\jump{\widehat{u_i}^j\nabla\widehat{u_0}^j}+\widehat{u_i}^j\dc{\nabla \widehat{u_0}^j}\right)(f-\M_Kf)\,dS.
	\end{align}
	Furthermore, with the conservation of mass and $f\in H^1_D(\Omega)$ follows
	\begin{align}\label{eqn:conservation_residual}
		\sum_{K\in\mathcal{T}}\sum_{\sigma\in\mathcal{E}_K\setminus\mathcal{E}^N} \int_\sigma u_{i,\sigma}^j\frac{F_\sigma^K(u_0^j)}{\abs{\sigma}}f\,dS = 0
	\end{align}
	Using \eqref{eqn:jump_av_equation} and \eqref{eqn:jump_i} in for the second equality and \eqref{eqn:conservation_residual} in the last equality yields
	\begin{align*}
		I_3
		&= \sum_{K\in\mathcal{T}}\sum_{\sigma\in\mathcal{E}_K}\left(\int_\sigma \widehat{u_i}^j\nabla \widehat{u_0}^j\cdot n_{K,\sigma}(f-\M_Kf)+\left(\widehat{u_0}^j\nabla \widehat{u_i}^j\right)\cdot n_{K,\sigma} f\, dS-u_{i,\sigma}^jF_{\sigma}^K(u_0^j)\M_Kf\right) \\
		&= \sum_{K\in\mathcal{T}}\sum_{\sigma\in\mathcal{E}_K}\left(\int_\sigma \left(\frac12\jump{\widehat{u_i}^j\nabla \widehat{u_0}^j}+\widehat{u_i}^j\dc{\nabla \widehat{u_0}^j}\cdot n_{K,\sigma}\right)(f-\M_Kf)+\frac12\jump{\widehat{u_0}^j\nabla \widehat{u_i}^j} f\, dS-u_{i,\sigma}^jF_{\sigma}^K(u_0^j)\M_Kf\right) \\
		 &= \sum_{K\in\mathcal{T}}\left(\frac12\sum_{\sigma\in\mathcal{E}_K\setminus\mathcal{E}^D}\int_\sigma \llbracket \widehat{u_0}^j\nabla \widehat{u_i}^j-\widehat{u_i}^j\nabla \widehat{u_0}^j\rrbracket(f-\M_K f)\,dS\right.\\
		&\phantom{=}\left.+\sum_{\substack{\sigma\in\mathcal{E}_K\setminus\mathcal{E}^N}}\int_\sigma \left( \widehat{u_i}^j\dc{\nabla \widehat{u_0}^j}\cdot n_{K,\sigma}-u_{i,\sigma}^j\frac{F_{\sigma}^K(u_0^j)}{\abs{\sigma}}\right)(f-\M_Kf)\,dS\right).
	\end{align*}
	With Hölder and the trace inequality (Theorem \ref{pFaceInterpolation}) we arrive at
	\begin{multline}\label{eqn:bound_I_3_4}
		I_3\leq C_{\text{app},2}\frac12\sum_{K\in\mathcal{T}}h_K^\frac12\left(\sum_{\sigma\in\mathcal{E}_K\setminus\mathcal{E}^D}\norm{\llbracket \widehat{u_0}^j\nabla \widehat{u_i}^j-\widehat{u_i}^j\nabla \widehat{u_0}^j\rrbracket}_\sigma\right.\\
		\left.+\sum_{\sigma\in\mathcal{E}_K\setminus\mathcal{E}^N}\norm{u_{i,\sigma}\frac{F_{\sigma}^K(u_0^j)}{\abs{\sigma}}-\widehat{u_i}^j\left\{\!\!\!\left\{\nabla \widehat{u_0}^j\right\}\!\!\!\right\}\cdot n_{K,\sigma}}_\sigma\right)\norm{\nabla f}_K
	\end{multline}
	Combining \eqref{eqn:bound_I_1},\eqref{eqn:bound_I_2} and \eqref{eqn:bound_I_3_4} we can estimate with the discrete Hölder inequality
	\begin{align*}
		I\leq \left(R_{i,S}+R_{i,R}\right)\norm{\nabla f}_\Omega
	\end{align*}

	We can now proceed with the temporal part $II$ using Hölder's inequality
	\begin{multline*}
		\int_\Omega\left( \widehat{u_0}\nabla \widehat{u_i}-\widehat{u_i}\nabla \widehat{u_0}-\widehat{u_0}^j\nabla \widehat{u_i}^j+\widehat{u_i}^j\nabla \widehat{u_0}^j\right)\cdot\nabla f\,dx\\
		\leq \norm{\widehat{u_0}^j\nabla \widehat{u_i}^j-\widehat{u_i}^j\nabla \widehat{u_0}^j-\widehat{u_0}^{j-1}\nabla \widehat{u_i}^{j-1}+\widehat{u_i}^{j-1}\nabla \widehat{u_0}^{j-1}}_\Omega\norm{\nabla f}_\Omega
	\end{multline*}
\end{proof}
\subsection{Estimator}
We can now state a reliable a posteriori error bound for the difference $u_i-v_i$.
For this we use the abstract stability framework from Section \ref{section:abstract_error}, the a posteriori error bounds of the diffusion equation from Section \ref{section:heat} and the residual bound from Theorem \ref{thm:Residual_bound}.
\begin{thm}\label{thm:error_estimator}
	Let Assumption \ref{assumptions} and \ref{ass:uniform} hold.
	Let $(v_0,\dots,v_n)$ be a weak solution to the reduced model \eqref{eqn:reduced} similar to Theorem \ref{thm:existence} and $(\widehat{u_0},\dots,\widehat{u_n})$ the Morley type reconstruction of a finite volume solution discussed in Section \ref{section:scheme}.
	The difference $\widehat{u_i}-v_i$ for $i=1,\dots,n$ is bounded by
	\begin{multline*}
		\max_{t\in[0,T]}\norm{\widehat{u_i}-v_i}_{\Omega}^2+\norm{\sqrt{v_0}\nabla (\widehat{u_i}-v_i)}_{[0,T]\times\Omega}^2\\
		\leq \left(2\norm{\widehat{u_i}^0-v_i^0}_{\Omega}+\frac{12}\gamma\norm{\widehat{u_0}^0-v_0^0}_\Omega^2+\frac{12}\gamma\sum_{j=0}^J\tau_j\left(\left(\eta_{R,i}^j\right)^2+\left(\eta_{2}^j\right)^2\right)+\frac{12}{\gamma}\norm{\nabla \widehat{u_i}}_{[0,T]\times\Omega}^2\left(\eta_\infty^J\right)^2\right)\\
		\times\exp\left(2C_G^{\frac{2}{1-\theta}}(1+C_{F,2,\Gamma_D}^2)^{\frac{\theta}{1-\theta}}\frac{\mu}{\gamma^{\frac{1+\theta}{1-\theta}}}\norm{\nabla \widehat{u_0}}_{X(q)}^{\frac2{1-\theta}}\right).
	\end{multline*}
	with $\theta=\frac{d}{2}-\frac{d}p$ and $\mu=\frac{1-\theta}{2}\left(\frac{1}{2(1+\theta}\right)^{\frac{1+\theta}{1-\theta}}$
	\begin{align*}
		\eta_\infty^J&:=\norm{v_0^0-\widehat{u_0}^0}_{L^\infty(\Omega)}+\sum_{j=1}^J\eta_{T,\infty}^j+\sum_{j=1}^J\dot{\eta}_{S,q}^jC_{\text{Green},p}+\max_{0\leq j\leq J} \eta_{S,q}^jC_{\text{Green},p}\\
		\eta_2^j&:=\eta_{S,2}^j+\norm{\nabla\left(\widehat{u_0}^j-\widehat{u_0}^{j-1}\right)}_{\Omega}\\
		\eta_{R,i}^j &:= R_{i,S}^j+R_{i,T}^j+R_{i,R}^j,
	\end{align*}
	where $\eta^j_{T,\infty},\dot{\eta}_{S,q}^j$ was defined in Lemma \ref{lemma:ParabolicBound}, $\eta_{S,q}^j$ was defined in Lemma \ref{lemma:EllipticBound} and $R_{i,S}^j,R_{i,T}^j, R_{i,R}^j$ were defined in Theorem \ref{thm:Residual_bound}.
\end{thm}
\begin{proof}
	Theorem \ref{thm:abstract_error} yields
	\begin{multline*}
		\max_{t\in[0,T]} \norm{\widehat{u_i}-v_i}_\Omega^2+\int_0^{T}\norm{\sqrt{v_0}\nabla(\widehat{u_i}-v_i)}_\Omega^2\,dt
		\leq\\
		\left(2\norm{\widehat{u_i}^0-v_i^0}^2_{\Omega}
		+\frac{12}\gamma\norm{v_0-\widehat{u_0}}_{L^\infty([0,T]\times\Omega)}^2\norm{\nabla \widehat{u_i}}_{[0,T]\times\Omega}^2
		+\frac{12}\gamma\norm{\nabla(\widehat{u_0}-v_0)}_{[0,T]\times\Omega}^2
		+\frac{12}\gamma \norm{R_i}_{*}^2 \right)\\
		\exp\left(2C_G^{\frac{2}{1-\theta}}(1+C_{F,2,\Gamma_D}^2)^{\frac{\theta}{1-\theta}}\frac{\mu}{\gamma^{\frac{1+\theta}{1-\theta}}}\norm{\nabla \widehat{u_0}}_{X(q)}^{\frac{2}{1-\theta}}\right).
	\end{multline*}
	We now can use the estimates from Theorem \ref{thm:Residual_bound}, Theorem \ref{thm:H1_heat_bound} and Theorem \ref{thm:max_heat_bound} to infer the claim.
\end{proof}
\subsection{General Model}\label{subsec:general_residual}
Similar to the reduced model, we now only need bounds on the residuals $R_0,\dots,R_n$ and $R_\Phi$ in the $H^1_D(\Omega)'$ norm.
The bounds on $R_i$ ($i=0,\dots,n$) follow similarly to the bounds for the reduced model (see Theorem \ref{thm:Residual_bound}).
First we derive a bound for the residual $R_\Phi$.
Using integration by parts and the properties of the reconstruction we can bound the residual by
\begin{equation}\label{eqn:residual_bound_phi}
\begin{aligned}
	\langle R_\Phi,f\rangle
	&= \left( \nabla \widehat{\Phi},\nabla f\right)-\frac{z}{\lambda^2}\left( 1-\widehat{u_0},f\right)\\
	&= -\sum_{K\in\mathcal{T}}\int_K\left(\frac{z}{\lambda^2}(1-\widehat{u_0})+\Delta \widehat{\Phi}\right)f\,dx-\sum_{\sigma\in\mathcal{E}}\int_\sigma\llbracket\nabla \widehat{\Phi}\rrbracket f\,dS
\end{aligned}
\end{equation}
We use the same technique as in \cite[Corollary 5.4]{nicaise2005} to arrive at
\begin{align*}
	\langle R_\Phi,f\rangle
	&= -\sum_{K\in\mathcal{T}}\int_K\left(\frac{z}{\lambda^2}(1-u_{0,h})+\Delta \widehat{\Phi}\right)(f-\M_Kf)\,dx-\sum_{\sigma\in\mathcal{E}}\int_\sigma\llbracket\nabla \widehat{\Phi}\rrbracket (f-\M_Kf)\,dS+\int_\Omega\frac{z}{\lambda^2}(\widehat{u_0}-u_{0,h})f\,dx\\
	&	\leq \left(\sqrt{2}\left(\sum_{K\in\mathcal{T}}C_P^2h_K^2\norm{\frac{z}{\lambda^2}(1-u_{0,h})+\Delta \widehat{\Phi}}_K^2+\frac{C_{\text{app},2}^2h_KN_\partial}{4}\sum_{\sigma\in\mathcal{E}_K\setminus\mathcal{E}^D} \norm{\llbracket\nabla \widehat{\Phi}\rrbracket}_\sigma^2\right)^\frac12 \right.\\
	&\left.\quad+\frac{\abs{z}}{\lambda^2}C_{F,2,\Omega}\norm{\widehat{u_0}-u_{0,h}}_\Omega\right)\norm{\nabla f}_\Omega.
\end{align*}
We also can bound the residual $R_0$ using the same technique as in Section \ref{Section:Residual_bound}
\begin{align*}
	\norm{R_0(t)}_{H^1_D(\Omega)'}
	&\leq R_{0,T}^j+R_{0,S}^j+R_{0,R}^j =:\eta_{R,0}^j\quad\forall t\in[t_{j-1},t_j]
\end{align*}
with the temporal, spacial and residual bounds given by
\begin{equation}\label{eqn:residual_bound_0}
\begin{aligned}
	R_{0,T}^j &= \norm{\nabla \widehat{u_0}^j-\widehat{u_0}^j(1-\widehat{u_0}^j)\beta z\nabla\Phi^j-\nabla \widehat{u_0}^{j-1}+\widehat{u_0}^{j-1}(1-\widehat{u_0}^{j-1})\beta z\nabla\Phi^{j-1}}_\Omega\\
	R_{0,S}^j &= \sqrt{2}\left(\sum_{K\in\mathcal{T}}\left(C_{P,2}^2h_K^2\norm{\partial_t u_{0,h}^j-\operatorname{div}\left(\nabla \widehat{u_0}^j-\widehat{u_0}^j(1-\widehat{u_0}^j)\beta z\nabla \widehat{\Phi}^j\right)}_K^2\right.\right.\\
	&\left.\left.+\frac{N_\partial C_{\text{app},2}^2}{4}\sum_{\sigma\in\mathcal{E}\setminus\mathcal{E}^D}h_K\norm{\llbracket \nabla \widehat{u_0}^j-\widehat{u_0}^j(1-\widehat{u_0}^j)\nabla \widehat{\Phi}^j\rrbracket}_\sigma^2\right)\right)^\frac12\\
	R_{0,R}^j &=C_{F,2,\Gamma_N}\norm{\partial_t (\widehat{u_0}-u_{0,h})}_{\Omega}\\
	&+\frac{C_{\text{app},2}N_\partial^\frac12}{2}\left(\sum_{K\in\mathcal{T}}\sum_{\sigma\in\mathcal{E}\setminus\mathcal{E}^N}h_K\norm{\widehat{u_0}^j\beta z(1-\widehat{u_0}^j)\left\{\!\!\left\{\nabla \widehat{\Phi}^j\right\}\!\!\right\}\cdot n_{K,\sigma}-u_{0,\sigma}^j\beta z\left(\sum_{i=1}^nu_{i,\sigma}^j\right)\frac{F_{\sigma}^K(\Phi^j)}{\abs{\sigma}}}_{\sigma}^2\right)^\frac12.
\end{aligned}
\end{equation}
With \eqref{equations} we notice that $\sum_{i=1}^n u_{i,\sigma}^j$ is an approximation of $1-u_0$ on the edge $\sigma$.
The $R_{0,R}^j$ in \eqref{eqn:residual_bound_0} is also consistent with \eqref{eqn:scheme_0}.
Hence, we expect that the second term in $R_{0,R}^j$ is small.\\
The residuals $R_i$ can be bounded similar to Theorem \ref{thm:Residual_bound}
\begin{align*}
	\norm{R_i(t)}_{H^1_D(\Omega)'} &\leq R_{i,S}^j+R_{i,T}^j+R_{i,R}^j =:\eta_{R,i}^j\quad\forall t\in[t_{j-1},t_j]
\end{align*}
where the spacial bound $R_{i,S}^j$, temporal bound $R_{i,T}^j$ and reconstruction bound $R_{i,R}^j$ are given by
\begin{equation}\label{eqn:residual_bound_i}
	\begin{aligned}
		R_{i,S}^j &= \sqrt{2}\left(\sum_{K\in\mathcal{T}}\left(C_{P,2}^2h_K^2\norm{\partial_t u_{i,h}^j-\operatorname{div}\left(\widehat{u_0}^j\nabla \widehat{u_i}^j-\widehat{u_i}^j\nabla \widehat{u_0}^j+\widehat{u_i}^j\widehat{u_0}^j\beta z\nabla \widehat{\Phi}^j\right)}_K^2\right.\right.\\
		&+\left.\left.\frac{N_\partial C_{\text{app},2}^2}{4}\sum_{\sigma\in\mathcal{E}\setminus\mathcal{E}^D}h_K\norm{\llbracket \widehat{u_0}^j\nabla \widehat{u_i}^j-\widehat{u_i}^j\nabla \widehat{u_0}^j+\widehat{u_i}^j\widehat{u_0}^j\beta z\nabla \widehat{\Phi}^j\rrbracket}_\sigma^2\right)\right)^\frac12\\
		R_{i,T}^j &= \left\|\widehat{u_0}^j\nabla \widehat{u_i}^j-\widehat{u_i}^j\nabla \widehat{u_0}^j+\widehat{u_i}^j\widehat{u_0}^j\beta z\nabla \widehat{\Phi}^j\right.\\
		&\left.-\widehat{u_0}^{j-1}\nabla \widehat{u_i}^{j-1}+\widehat{u_i}^{j-1}\nabla \widehat{u_0}^{j-1}-\widehat{u_i}^{j-1}\widehat{u_0}^{j-1}\beta z\nabla \widehat{\Phi}^{j-1}\right\|_\Omega\\
		R_{i,R}^j &=C_{P,2,\Omega}\norm{\partial_t (\widehat{u_i}^j-u_{i,h}^j)}_{\Omega}\\
		&+C_{\text{app},2}N_\partial^\frac12\left(\sum_{\sigma\in\mathcal{E}\setminus\mathcal{E}^N}h_K\norm{\widehat{u_i}^j\left\{\!\!\!\left\{\nabla \widehat{u_0}^j-\beta z\widehat{u_0}^j\nabla\Phi^j\right\}\!\!\!\right\}\cdot n_{K,\sigma}-u_{i,\sigma}\frac{F_{\sigma}^K(u_0^j)-\beta z F_{\sigma}^K(\Phi^j)}{\abs{\sigma}}}_{\sigma}^2\right)^\frac12.
	\end{aligned}
\end{equation}
Notice that the residual bound has the same structure as the residual bound in Theorem \ref{thm:Residual_bound}.
The only difference is, that the convective part has changed from $-\widehat{u_i}\nabla \widehat{u_0}$ for the reduced model to $-\widehat{u_i}(\nabla \widehat{u_0}-\beta z\widehat{u_0}\nabla \Phi)$ and the numerical approximation to the convective part.
\begin{proof}[Proof of Theorem \ref{thm:main_estimator}]
	The claims follow from Theorem \ref{thm:abstract_general} and the bounds on the residual bounds from \eqref{eqn:residual_bound_phi}-\eqref{eqn:residual_bound_i}.
\end{proof}
\begin{remark}\label{remark:constants}
	All the constants appearing in the estimator in Theorem \ref{thm:main_estimator} except for $\gamma$ are computable for convex polygonal domains.
	The following bound on the Poincar\'e-Friedrichs constant $C_{F,2,\Gamma_D}$ for convex domains can be found in \cite{pauly2020}.
	\begin{align*}
		C_{P,2} \leq C_{F,2,\Gamma_D} \leq \frac{\operatorname{diam}(\Omega)}{\pi}.
	\end{align*}
	A bound on the Sobolev constant $C_S$ for a finite union of convex domains can be found in \cite{mizuguchi2017}.
	A bound for the constant in the Gagliardo-Nirenberg inequality used here can be found in Theorem \ref{thm:Nirenberg}.
\end{remark}
\section{Numerical Results}\label{section:numerics}
We now show the results of numerical experiments to validate the optimal scaling behavior of the error estimators from Theorem \ref{thm:error_estimator} and Theorem \ref{thm:main_estimator}.
The implementation is done in Python.
We define the estimated order of convergence (EOC) of two sequences $a(i)$ and $h(i)$ by
\begin{align*}
	\text{EOC}(a,h;i):=\frac{\log(a(i+1)/a(i))}{\log(h(i+1)/h(i))}.
\end{align*}
We also look at the effectivity index
\begin{align*}
	q_i &:=
	\begin{dcases}
		\frac{\sqrt{\max_{t\in[0,T]}\norm{v_0(t)-\widehat{u_0}(t)}_{\Omega}^2+\norm{\nabla(v_0-\widehat{u_0})}_{[0,T]\times\Omega}^2}}{\eta_2^J} &\text{for } i=0,\\
		\frac{\sqrt{\max_{t\in[0,T]}\norm{v_i(t)-\widehat{u_i}(t)}_{\Omega}^2+\norm{\sqrt{v_0}\nabla(v_i-\widehat{u_i})}_{[0,T]\times\Omega}^2}}{\eta_2^{i,J}} &\text{for } i=1,\dots,n.
	\end{dcases}\\
	 q_\Phi &:=\frac{\norm{\nabla (\widehat{\Phi}-\Psi)}_{[0,T]\times\Omega}}{\eta_\Phi^J}
\end{align*}
where $\eta_2^{i,J}$ and $\eta_2^J$ are the estimators defined in Theorem \ref{thm:error_estimator} for the reduced model and from Theorem \ref{thm:main_estimator}.
\subsection{The reduced model}
In what follows, we choose $\Omega=[0,1]^2$ with Dirichlet boundary $\Gamma_D:=\{0,1\}\times[0,1]$ and Neumann boundary $\Gamma_N:=[0,1]\times\{0,1\}$.
Furthermore, we set $T=1$ and consider the time intervall $[0,1]$.
We prescribe the solution of the reduced model \eqref{eqn:reduced} by
\begin{align*}
	v_1(t,x,y) &= 0.1+0.1x+tx(1-x)e^{-20x^2}\\
	v_2(t,x,y) &= 0.1-0.05x-0.5(1-\cos(t))x(1-x)\sin(x)\\
	v_3(t,x,y) &= 0.2+0.1x+tx(1-x)e^{-20(x-0.5)^2}.
\end{align*}
With initial and boundary condition chosen accordingly.
With this choice, we arrive at $\gamma=0.3$.
We here fix $q=42$ and $\tilde{q}=2.1$ and the other constants are then given by
\begin{align}\label{eqn:constants}
	\mu\leq 1.08,\quad
	C_G\leq 1.02,\quad
	C_S\leq 12.02
\end{align}
with Theorem \ref{thm:Nirenberg} and \cite[Table 2]{mizuguchi2017} and the Poincar\'e and Poincar\'e-Friedrichs constant given in Remark \ref{remark:constants}.
For the maximum norm estimator $\eta_\infty^J$ from Theorem \ref{thm:max_heat_bound} we set $p^*=1$ and $C_{\text{Green},1}=1$, since this constant is not accessible (see Remark \ref{remark:uniform}).
A sequence of approximate solutions obtained from the scheme and reconstruction described in Section \ref{section:scheme} using mesh width $h=2^{-i-1}$ and time step size $\tau=2^{-i-1}$ for $i=0,\dots,5$.\\
In Table \ref{reduced_eoc_table} we see a linear convergence of the estimators $\eta_2^J,\eta_\infty^J$ and $\eta_2^{1,J}$.
The estimators for the other species are similar to $\eta_2^{1,J}$ and are therefore omitted.
The effectivity indices indicate that the estimator $\eta_2^{1,J}$ overestimates the error by a large factor but the estimator scales with the same order as the true error.
\begin{table}[h]
\begin{tabular}{|l|l|l|l|l|l|l|l|l|}
	\hline
	i& $\eta_2^J$&EOC &$\eta_\infty^J$&EOC&$\eta_2^{1,J}$&EOC&$q_0$&$q_1$\\\hline
	0&1.458&1.1&18.14&0.98&118.1&1.2&0.16&0.00076\\
	1&0.6916&1.0&9.194&0.53&51.55&0.5&0.15&0.00084\\
	2&0.337&1.0&6.359&0.79&36.36&0.68&0.15&0.00055\\
	3&0.1655&1.0&3.669&0.96&22.67&0.91&0.15&0.00043\\
	4&0.0819&1.0&1.891&0.99&12.09&0.98&0.15&0.0004\\
	5&0.04072&-&0.9507&-&6.148&-&0.15&0.00039\\
	\hline
\end{tabular}
\caption{Estimators, EOC and effectivity index for the reduced model.}
\label{reduced_eoc_table}
\end{table}
\subsection{The general model}
In what follows, we choose $\Omega=[0,1]^2$ with Dirichlet boundary $\Gamma_D:=\{0,1\}\times[0,1]$ and Neumann boundary $\Gamma_N:=[0,1]\times\{0,1\}$.
We fix the solution of the general model \eqref{eqn:general} with constants $\beta=z=\lambda=1$
\begin{align*}
	v_1(t,x,y)&:=0.1+t^2\theta(y)x(1-x)\exp(-100(x-0.5)^2)\\
	v_2(t,x,y)&:=0.2x+0.1(1-x)-\frac12\sin(t)\theta(y))x(1-x)\cos(x)\\
	v_3(t,x,y)&:=0.2-0.1x-0.55tx(1-x)\theta(y)\exp(-100(x-0.5)^2)\\
	\Psi(t,x,y)&:=\frac{25}{3}(x-1)x(5x^2-2.6-2.4+t(x^2-x))\theta(y)
\end{align*}
with
\begin{align*}
	\theta(y) := \frac{y^4}3-2\frac{y^3}3+\frac{y^2}{3}+\frac{47}{48}.
\end{align*}
We choose $q=42$ and $\tilde{q}=2.1$ and $\gamma=0.57$ and the same constants as in \eqref{eqn:constants}.
A sequence of approximate solutions are obtained again by using the mesh width $h=2^{-i-1}$ and time step size $\tau=2^{-i-1}$ for $i=0,\dots,5$.
In Table \ref{table:eoc_general_model} we see a linear convergence of the estimators $\eta_2^J,\eta_2^{1,J}$ and $\eta_\Phi^{J}$.
The estimates for the other species are similar to $\eta_2^{1,J}$ and are therefore omitted.
We see that the estimators $\eta_2^J,\eta_2^{1,J}$ for the general model is much larger than the estimator for reduced model (see Table \ref{reduced_eoc_table}).
One reason for this is that due to the choice of example the norm appearing in the exponential term is much larger, i.e. $\norm{F}_{X(q)}^{\frac{2}{1-\theta}}\approx 4.7$ for the general model compared to $\norm{\nabla \widehat{u_0}}_{X(q)}^{\frac{2}{1-\theta}}\approx 0.4$ for the reduced model.
Furthermore, the maximum norm estimator for the solvent concentration in the reduced model enables us to use the $L^2([0,T]\times\Omega)$ norm of $\nabla \widehat{u_1}$ instead of the $L^\infty(0,T;L^q(\Omega))$ norm in the general model, also contributing to the smaller effectivity index in the second experiment.
For the estimator of the solvent convergence $\eta_2^J$ for the general model has an additional exponential term that is large for this example.
\begin{table}[h]
\begin{tabular}{|l|l|l|l|l|l|l|l|l|l|}
	\hline
	i& $\eta_2^J$&EOC &$\eta_2^{1,J}$&EOC&$\eta_\Phi^{J}$&EOC&$q_0$&$q_1$&$q_\Phi$\\\hline
	0&2.667e+07&-1.2&3.811e+11&-1.8&8.489e+06&-1.2&5.8e-09&8.7e-13&8.9e-08\\
	1&6.157e+07&0.46&1.292e+12&0.3&1.96e+07&0.46&1.4e-09&1.5e-13&1.9e-08\\
	2&4.484e+07&0.9&1.047e+12&0.86&1.427e+07&0.9&9e-10&8.3e-14&1.2e-08\\
	3&2.408e+07&0.98&5.784e+11&0.97&7.666e+06&0.98&8.1e-10&7.3e-14&1.1e-08\\
	4&1.218e+07&1.0&2.946e+11&1.0&3.878e+06&1.0&7.9e-10&7.1e-14&1.1e-08\\
	5&6.091e+06&-&1.475e+11&-&1.939e+06&-&7.9e-10&7e-14&1.1e-08\\
	\hline
\end{tabular}
\caption{Estimators, EOC and effectivity index for the general model.}
\label{table:eoc_general_model}
\end{table}\\
The code for the experiments in this section is provided in a git-repository at \url{https://git-ce.rwth-aachen.de/arne.berrens/a-posteriori-error-control-for-a-finite-volume-scheme-for-a-cross-diffusion-model-of-ion-transport}.
	\paragraph{Funding}
	The research of JG was supported by the Deutsche Forschungsgemeinschaft (DFG, German Research Foundation) - SPP 2410 Hyperbolic Balance Laws in Fluid Mechanics: Complexity, Scales, Randomness (CoScaRa) within the project 525877563 (A posteriori error estimators for statistical solutions of barotropic Navier-Stokes equations). JG also acknowledges support by the German Science Foundation (DFG) via grant TRR 154 (Mathematical modelling, simulation and optimization using the example of gas networks), sub-project C05 (Project 239904186).
	\printbibliography[heading=bibintoc]
\end{document}
