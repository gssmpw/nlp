\section{Related Work}
\subsection{Advances in Improving In-Context Learning (ICL)}

Recent advancements in in-context learning (ICL) focus on optimizing the selection and use of in-context examples. Several studies, such as those by \cite{Yin23}, have introduced refined methods for template selection, aiming to create more effective prompts. Other research efforts, including those by \cite{Rubin22, Wan23, Wan23a}, have developed techniques to enhance the choice of examples, ensuring they are relevant and informative. A notable contribution by \cite{Ye22} proposed a framework for evaluating examples based on their consistency, diversity, and frequency, enhancing the overall effectiveness of ICL. Further developments include methodologies like flipped learning \cite{Ye22}, which reorders the learning sequence to improve task comprehension, and noisy channel prompting \cite{Min22}, which helps align input context with the desired task outcome. Additionally, \cite{Xu22} introduced a method utilizing K-nearest neighbors for label assignment in multiple-choice ICL scenarios, while \cite{Yang24} proposed iterative context updates to refine model responses. 

\subsection{In-Context Vectors (ICV) and Related Techniques}

The concept of In-Context Vectors (ICV) aligns with recent approaches in ICL but offers distinct advantages. A concurrent study by \cite{Hendel23} describes a similar method involving the use of a "task vector" derived from the latent states of a specific model layer, which replaces these states during query processing. This method requires layer-specific modifications and relies on traditional accuracy metrics. In contrast, ICV enhances latent states across all layers, integrating new information without displacing the original states, making it particularly suitable for open-ended generative tasks.

\subsection{Activation Manipulation in Language Models}

Activation manipulation, also known as activation editing, has emerged as a technique for directing the outputs of language models towards specific goals. For example, \cite{Turner23} explored altering the activations of models like GPT-2-XL to modify sentiment or topic focus, while \cite{Zou23} introduced "representation engineering" to align model behavior with certain concepts. Other studies, such as \cite{Burns22}, have demonstrated that latent knowledge within the activation space can be linearly separated, enabling targeted adjustments. Techniques like those described by \cite{Mini23} utilized vectors derived from activations to alter behaviors in reinforcement learning settings, while \cite{Li22} explored how changing activations can counterfactually modify model outputs.

\subsection{Insights into the Mechanisms of In-Context Learning (ICL)}

The underlying mechanisms of ICL continue to be a subject of significant interest and exploration. Studies by \cite{Lu22, Shin22} have highlighted the crucial role of demonstration example selection and arrangement in influencing model performance. Theoretical frameworks, such as the one proposed by \cite{Xie21}, suggest that ICL mechanisms may function similarly to implicit Bayesian inference, providing a structured way to understand how models integrate new information. Further analysis by \cite{Wei23, Akyurek22} has shown parallels between ICL's learning processes and gradient descent methods, suggesting that ICL could act as a form of meta-optimization, although the exact internal workings in complex natural language tasks remain an area of ongoing research.