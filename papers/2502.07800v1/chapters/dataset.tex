% ADD THIS HEADER TO ALL NEW CHAPTER FILES FOR SUBFILES SUPPORT

% Allow independent compilation of this section for efficiency
\documentclass[../CLthesis.tex]{subfiles}

% Add the graphics path for subfiles support
\graphicspath{{\subfix{../images/}}}

% END OF SUBFILES HEADER

%%%%%%%%%%%%%%%%%%%%%%%%%%%%%%%%%%%%%%%%%%%%%%%%%%%%%%%%%%%%%%%%
% START OF DOCUMENT: Every chapter can be compiled separately
\begin{document}

\chapter{Dataset}%
\label{chap:dataset}

\section{Data Overview}
The project used neural and vocal data that were previously collected through three recording sessions from the same bird. During each session, neural and vocal data were recorded simultaneously, thus their start points were the same. Table~\ref{tab:recording_durations} shows the length of recorded vocal and neural data. Extra vocal recordings were removed from the end in this project.

\begin{table}[htbp]
   \centering
   \begin{tabular}{lcc}
       \toprule
       Recording Session & Neural Data Length  & Vocal Data Length\\
       \midrule
       Session 0 & 53.29\,min& 53.29\,min\\
       Session 1 & 89.91\,min& 90.06\,min\\
       Session 2 & 70.37\,min& 77.90\,min\\
       \bottomrule
   \end{tabular}
   \caption{Duration of recordings across different sessions}
   \label{tab:recording_durations}
\end{table}

\section{Neural Data}
The preprocessed neural data consists of an array of spike times and an array of spike cluster indices. The data was provided at a sampling rate of $20$\,kHz. The raw data was collected with Neuropixels $1.0$ \citep{Jun2017-uh} and the preprocessed spike data was obtained through spike-sorting with Kilosort \citep{Pachitariu2016-mi}. All recordings contain only non-directed songs. The data includes additional information: (1) depth, (2) channel on Neuropixels, (3) firing rate, (4) spike amplitude, and (5) cluster quality.

\subsection{Characteristics of Neurons}
Spike-sorting produces clusters of different qualities, which are marked with either good, multi-unit acitivities (MUA), or noise. In the data used in this project, all neuronal clusters were labeled as either good or MUA. The dataset consists of reliable neural recordings without noisy clusters, making it suitable for further analysis. Figure~\ref{fig:neuron_quality} shows the quality of clusters in dark grey and red, with circle diameter indicating the firing rate. LMAN neurons roughly distributed between 2300\,$\mu$m and 3000\,$\mu$m.

\begin{figure}[H]
   \centering
   \includegraphics[width=\textwidth]{images/neuron_quality.pdf}
   \caption{Quality of clusters}
   \label{fig:neuron_quality}
\end{figure}

\section{Vocal Data}
The vocal recordings were obtained from recording devices in the lab. The raw data was preprocessed by first downsampling from $20$\,kHz to $250$\,Hz, and then applying a STFT with a $16$\,ms offset ($64$ samples under $250$\,Hz). The data used for this project were prepared in spectrograms in $128$ frequency bins ranging from $[0\,\text{Hz}, -8000\,\text{Hz}]$ in $250$\,Hz. A separate annotation file provides the onset and duration for each vocalization cluster (referred to as syllable in this paper). Our analysis focused on valid vocalization patterns, labeled $2$--$8$. Other syllables were excluded from this investigation. Syllables $2$--$8$ are shown in Figure~\ref{fig:labeled_spectrogram}.

\begin{figure}[htbp]
   \centering
   \includegraphics[width=\textwidth]{images/labeledSpectrogram.pdf}
   \caption{Labeled spectrogram for all 7 syllables}
   \label{fig:labeled_spectrogram}
\end{figure}

\subsection{Distribution of Syllables}
Figure~\ref{fig:length_distribution} shows that there is considerable variation in syllable length. Figure~\ref{fig:syllable_distribution} shows syllables $2$--$7$ occur at similar frequencies, while syllable $8$ occurs rarely. On a sampling rate level, however, Figure~\ref{fig:syllable_distribution} shows that the data is imbalanced on a sampling time level, where label $3$, $5$, and $7$ together take up $80$\,\% of the sampling time data. Figure~\ref{fig:distributionFile} shows that the syllable distributions are similar across sessions.

\begin{figure}[htbp]
   \centering
   % First row with single image
   \begin{subfigure}[b]{0.85\textwidth}
       \centering
       \includegraphics[width=0.9\textwidth]{images/distribution_syllable_length.pdf}
       \caption{Distribution of syllable lengths}
       \label{fig:length_distribution}
   \end{subfigure}
   
   % Second row with two images
   \begin{subfigure}[b]{0.85\textwidth}
       \centering
       \includegraphics[width=\textwidth]{images/distribution_syllables.pdf}
       \caption{Total distribution of syllables}
       \label{fig:syllable_distribution}
   \end{subfigure}

   \begin{subfigure}[b]{0.85\textwidth}
       \centering
       \includegraphics[width=\textwidth]{images/distribution_per_file.pdf}
        \caption{Distribution of syllables per session}
       \label{fig:distributionFile}
   \end{subfigure}
    \caption{Distribution of annotated syllables}
   \label{fig:distribution_syllable}
\end{figure}

% \section{Scientific Writing Hints}
% \begin{itemize}
% \item What does the data look like? Give examples!
% \item Are there enough descriptive statistics (size, class distributions) for the reader to assess the properties/difficulty of the dataset?
% \item Do you clearly state what was done/developed personally by you? 
% \item Do you mention some cleaning or processing that you did?
% \end{itemize}

% self-defined macro: include bibliography even when compiling a single chapter
\subfilebibliography
\end{document}
