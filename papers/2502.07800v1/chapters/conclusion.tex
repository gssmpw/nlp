% ADD THIS HEADER TO ALL NEW CHAPTER FILES FOR SUBFILES SUPPORT

% Allow independent compilation of this section for efficiency
\documentclass[../CLthesis.tex]{subfiles}

% Add the graphics path for subfiles support
\graphicspath{{\subfix{../images/}}}

% END OF SUBFILES HEADER

%%%%%%%%%%%%%%%%%%%%%%%%%%%%%%%%%%%%%%%%%%%%%%%%%%%%%%%%%%%%%%%%
% START OF DOCUMENT: Every chapter can be compiled separately
\begin{document}
\chapter{Conclusion}%
\label{chap:conclusion}
\section{Achievements}
This project establishes several pilot frameworks in neural decoding and analysis. 
\subsection{Neural Activity Analysis and Classification}
We created a classification baseline using mean firing rates, achieving 77.3\,\% accuracy with SVM for syllable classification tasks. We combined XGBoost and SHAP analysis to reveal syllable-specific firing patterns and important neuron interactions. We also identified temporal relationships between LMAN burst and syllable production from the output of basic machine learning models, which aligns with neuroscience literature.

\subsection{Decoding Frameworks with New Architectures}
We developed a new framework to decode spike data, and demonstrated the potential of GPT-2 for neural decoding. We also investigated the decoding accuracy of simpler models using LSTM, CNN, Mamba-2, etc. EEGNet achieved the highest decoding accuracy (89.0\,\%). Mamba2 showed better performance when the data is sparse and long. We investigated the correlation of plateau point bin size of 1.5\,ms with the scale of AP.

\subsection{Generative Model with Contrastive Learning}
We implemented 15 models in CEBRA and achieved an accuracy as high as 75.7\,\% on syllable classification tasks. We confirmed the advantage that a contrastive learning framework offers. We also investigated the potential of VAE on all types of processed neural and vocal: segmented, trimmed, padded, warped, and in different durations. 

We used a combined contrastive-generative approach, achieving 91.0\,\% accuracy on syllable classification. The model offers the possibility to generate segmented vocal data from segmented neural data. In addition, CCA revealed strong correlations ($r=0.73$) between warped neural and vocal representations.


\section{Limitations}
This project had several limitations.

\subsection{Data Dependency and Generalization Challenges}
The decoding framework's reliance on annotated syllables creates a circular dependency, which limits the practical applications. The performance of the models on different birds was untested, and the transferability between different birds may be highly limited, as different birds have distinct syllables.
\subsection{Data Volume Constraints}
The data is insufficient for the current models to converge. A larger dataset would yield higher decoding accuracy of syllables or better reconstruction quality of neural data.
\subsection{Biological Plausibility}
The data came from the LMAN region, which primarily controls plasticity rather than production. Data from nuclei along the motor pathway would yield more accurate and interpretable results for decoding tasks.

\section{Future Direction}
The project establishes methodological foundations for future research, emphasizing the balance between computational performance, model interpretability, and biological plausibility through its achievements and limitations.

\subsection{Syllable-Based Approaches}
With data recorded from the motor pathway, the SHAP interaction values can be used to visualize how neurons integrate information and control the output syllables. 

Depending on the availability of data, a pre-trained model based on open source models like GPT-2 on multiple birds and syllables could be developed. With a pre-trained model, latent embeddings could serve as a target of interpretability analysis.

Mamba2 showed potential for decoding long and sparse spike trains. EEGNet and Mamba2 can be further investigated and developed in this direction to decode spike data in 20\,kHz or 30\,kHz with higher accuracies.

\subsection{Representation Learning}
Decoding neural activities with the CEBRA method using pretrained latent embeddings (like wav2vec) would improve interpretability on a finer scale. The CEBRA method could also be combined into the current joint VAE to better align neural representations to vocal representations in VAE. VAE framework itself could be further improved with statistical inference methods to better capture the patterns in the neural data.

\subsection{NeuroAI}
Recent advances in neuroscience and AI have given rise to a new field known as NeuroAI \citep{Benchetrit2023-pn, Conwell2023-ts, Francl2022-tm, Yamins2014-sn, Yamins2016-jf}. While current NeuroAI research focused primarily on visual perception, our project suggest promising direction of extending the work of NeuroAI to motor vocalization. Generative models can be built with song circuit inspired neural network architectures, while simultaneous recordings from the motor pathway and AFP could be used as inputs. Future work can perform ablation studies on layers representing LMAN, and evaluate the neural network by observing the variability in the reconstructed syllable.


% provides a self-supervised representational learning framework that maps between two modalities

% use lman data to guide ??? to supervise training??? 

% \section{Scientific Writing Hints}
% \begin{itemize}
% \item Does your study support the conclusions? 
% \item Does your conclusion summarize all your main findings and results?
% \item Do you present the main results in the form of concrete numbers? Do you avoid vague statements?

% \item Does the introduction and conclusion give a complete short version of your study? Do they correspond with the abstract?
% \end{itemize}

% \section{Scientific Writing Hints}
% \subsection{Results}
% \begin{itemize}
% \item Do you explain and formalize the main quantitative evaluation measures?
% \item Are the most important results qualitatively described?
% \item Are the results credible and valid?
% \item Are there  descriptive statistics of the results? Are they well presented?
% \item Do you show results from a baseline system or SOTA systems?
% \item Do you give enough examples? 
% \item Is there an error analysis? Is there an ablation study?
% \item Do you use tables and figures where appropriate?
% \end{itemize}
% \subsection{Discussion and Future Work}
% \begin{itemize}



% \item Does it answer the research question? 
% \item Are the results interpreted in relation to the goal of the study?
% \item Are the problems and limitations of the study critically assessed?
% \item Are the most important references made to related work and methods?
% \item Do you mention potential improvements for future work or next steps?
% \item Does your study support the conclusions? 

% \end{itemize}

\subfilebibliography
\end{document}
