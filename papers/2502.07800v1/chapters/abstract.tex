% ADD THIS HEADER TO ALL NEW CHAPTER FILES FOR SUBFILES SUPPORT

% Allow independent compilation of this section for efficiency
\documentclass[../CLthesis.tex]{subfiles}

% Add the graphics path for subfiles support
\graphicspath{{\subfix{../images/}}}

% END OF SUBFILES HEADER

%%%%%%%%%%%%%%%%%%%%%%%%%%%%%%%%%%%%%%%%%%%%%%%%%%%%%%%%%%%%%%%%
% START OF DOCUMENT: Every chapter can be compiled separately
\begin{document}
\chapter*{Abstract}
\thispagestyle{empty}
    Accurate decoding of neural spike trains and relating them to motor output is a challenging task due to the inherent sparsity and length in neural spikes and the complexity of brain circuits. This master project investigates experimental methods for decoding zebra finch motor outputs (in both discrete syllables and continuous spectrograms), from invasive neural recordings obtained from Neuropixels. 

    There are three major achievements: (1) XGBoost with SHAP analysis trained on spike rates revealed neuronal interaction patterns crucial for syllable classification. (2) Novel method (tokenizing neural data with GPT2) and architecture (Mamba2) demonstrated potential for decoding of syllables using spikes. (3) A combined contrastive learning-VAE framework successfully generated spectrograms from binned neural data.

    This work establishes a promising foundation for neural decoding of complex motor outputs and offers several novel methodological approaches for processing sparse neural data.
\end{document}