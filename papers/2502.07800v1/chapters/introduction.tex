% ADD THIS HEADER TO ALL NEW CHAPTER FILES FOR SUBFILES SUPPORT

% Allow independent compilation of this section for efficiency
\documentclass[../CLthesis.tex]{subfiles}

% Add the graphics path for subfiles support
\graphicspath{{\subfix{../images/}}}

% END OF SUBFILES HEADER

%%%%%%%%%%%%%%%%%%%%%%%%%%%%%%%%%%%%%%%%%%%%%%%%%%%%%%%%%%%%%%%%
% START OF DOCUMENT: Every chapter can be compiled separately
\begin{document}

\chapter{Introduction}%
\label{chap:intro}

Recent advances in neural recording technologies and speech analysis methods have created new possibilities to address challenges during analyzing the relationship between neural activity and behaviours. The challenges stem from the complex nature of spike trains -- sparse, long, and high dimensional -- as well as the need to align them with complex behavioural labels or external stimuli. For this project, we worked on bird vocalizations and the corresponding neural signals. We benefited from Neuropixels \citep{Jun2017-uh}, which allows for extremely high temporal and spatial resolution recording of neural spikes in animals, and from methods that enable effective extraction of syllable clusters from the bird vocalizations \citep{Lorenz2022-vm}. With these advances, we attempted to address some challenges in correlating spike train data with vocalization.

\section{Motivation}

This project is motivated by recent technological advances in Machine Learning (ML). First, Natural Language Processing (NLP) techniques have demonstrated remarkable success in text-processing. Transformer models excel at handling sentences, and have been adapted to handle sparse and long sequential data -- a characteristics shared with neural spike data. Second,  recent Mamba architectures \citep{Gu2023-vw, Dao2024-sp} efficiently process long sequences without the quadratic bottleneck which transformers have. These advances offer a perspective to address some key challenges in neural data analysis: the inherent sparsity of spike data and the need for long-range dependencies.

In addition, with emergent representation learning techniques, such as Variational Auto-Encoder (VAE) architecture and contrastive learning method, we would like to investigate correlations between neural data and behavioural data at a finer temporal scale. 
% show high human relevance

\section{Research Question}
Our project aimed to address three questions:
\begin{enumerate}
    \item Can we validate spike based classification method for syllables of Zebra Finch?
    \item How well can natural language processing methods and different deep learning architectures handle long, sparse, and high-dimensional spike data?
    \item Can we effectively map spike data to latent spaces for classification and reconstruction?
\end{enumerate}

\section{Methodological Framework}
To answer these questions, we developed a three-part methodological framework:

\subsection{Baseline Validation}
We first established performance benchmarks for syllable-based classification through classical ML methods: Support Vector Machine (SVM), Random Forest (RF), and XGBoost. In addition, we used SHAP values to analyze the interaction between neurons.

\subsection{Deep Learning Methods}
Then, we explored various neural data analysis methods. We used a tokenization method to analyze the data with GPT-2. We implemented several models, including Mamba-2 and EEGNet \citep{Lawhern2016-tf}, to deal with long range temporal dependencies over tokenized and time series data.

\subsection{Representational Learning}
We used CEBRA methods and VAE architecture to obtain interpretable latent representations from our neural data. We created a joint VAE with contrastive learning methods to map neural data into vocalizations. We evaluated all obtained representations on syllable classification tasks.

The code is publicly available at \href{https://github.com/askrbayern/neuro2voc_thesis}{https://github.com/askrbayern/neuro2voc\_thesis}.

\section{Thesis Structure}
The remainder of this thesis is divided into the following chapters:

\begin{itemize}
    \item Chapter~\ref{chap:related-work}: Related Work
    \item Chapter~\ref{chap:dataset}: Dataset
    \item Chapter~\ref{chap:methods}: Methodology
    \item Chapter~5--9: Experiments \newline
    Each chapter describes one key experiment, including:
    \begin{itemize}
        \item Experiment~\ref{exp:1} Machine Learning Benchmarks
        \item Experiment~\ref{exp:2} Token-based Deep Learning Methods 
        \item Experiment~\ref{exp:3} Temporal Deep Learning Methods
        \item Experiment~\ref{exp:4} CEBRA Latent Embeddings
        \item Experiment~\ref{exp:5} VAE Latent Representations
    \end{itemize}
   \item Chapter~\ref{chap:conclusion}:Conclusion
\end{itemize}

% \section{Research Objectives}
% Our research investigates the relationship between neural activity and vocal output in zebra finches. We focus on developing robust methods for syllable-level neural decoding. The project aims to design novel approaches for processing sparse, high-dimensional neural data. Through these methods, we analyze neural-vocal correlations at the syllable level to understand learning mechanisms.

% \section{Technical Challenges}
% Neural data analysis presents two fundamental challenges. First, NeuroPixels recordings produce high-dimensional data from hundreds of simultaneously recorded neurons. Second, spike trains are inherently sparse, with meaningful activity distributed across time.


% \begin{itemize}
%     \item Why is the  topic relevant? (Did you claim centrality of the topic?): We want to investigate the correlation between neural and 
%     \item What is the general (!) background/context?
%     we have neural spike recordings from neuropixels, and corinna lorenz developed systems that can divide the speech recordings into syllables. This provides an interesting ground for investigation when we combine neural and speech data.
%     The neural data is high dimensional, and very sparse in nature. We hope to use new approaches like GPT tokenization, mamba model, or new ways of doing latent embedding reconstruction to solve this issue.
%     \item What is the study about? (Did you establish a niche?)
%     This study is about to investigate the correlation between neural and speech data. One task is to predict the class by inputting the sequences - which is enabled by the class we have. and another task is trying to find the correlation and variance between neural and speech data. This refines the previous paper \cite{SinghAlvarado2021} from a 'motif' level on a 'syllable' level (a motif is composed of several syllables). 
%     \item What is the research gap?
%     The research gap is that the sparscity can not be easily solved and, on top of that, the high-dimensional spike data is also difficult to deal with.
%     \item What is the goal of the study? (How do you occupy the niche?)
%     The goal goal is to try to validaste the 'syllable' method for classification, and another is try to find new ways. (wait, what is the different between occupy the niche and establish the niche i should make a distinction here!)
%     \item How are the following sections organized?
%     following sections are divided into related work which is a literature review part, data, where I introduce the raw data and distribution of syllables. Then, I introduced the methodologies of some new models used in the methodology part. Then, due to the nature of this work, i introduced five experiments. Each experiment introduces the specific framwork, the data manipulation, the result, as well as the discussion. At the end there is another final discussion part.
% \end{itemize}

% A typical introduction contains the following sections 
% \begin{itemize}
% \item \textbf{Motivation}: Purpose and Scope: Introduce the main topic, objectives, and scope of the work. 
% \item \textbf{Background}: Provide context and background information to set the stage for the reader.
% \item \textbf{Research Questions}: State the key questions or hypotheses that the work aims to address.
% \item \textbf{Thesis Structure}: Provide a roadmap of the subsequent chapters.
% \end{itemize}


% \subsection{Literature References: Don't Panic!}
% \subsubsection{Checklist}
% \begin{itemize}
% \item Can ALL references from the text be found in the literature section?
% \item Do you mention ALL literature references in the text?
% \item Are your references consistently formatted? (pages, publisher, etc.)

% \item Do you have enough established scientifically  sources?  (not only web pages)
% \item Do your in-text citations make a difference between \href{https://www.scribbr.com/chicago-style/chicago-in-text-citation/}{textual/narrative and parenthetical style}?
% \end{itemize}

% We use the ACL bibliographic style, which supports URLs nicely!

% \subsection{Different Types of Citations}
% There are different ways to include a reference in your text:

% \paragraph{Parenthetical Citations}
% \begin{itemize}
%     \item \textbf{Command}: \texttt{\textbackslash cite\{key\}} or \texttt{\textbackslash citep\{key\}}
%     \item \textbf{Usage}: Use this when you want to reference a work within parentheses.
%     \item \textbf{Example}:
%     \begin{quote}
%         Text: "Natural language processing has seen significant advances in recent years \verb!\citep{Jurafsky2009}!." 
        
%         Output: "Natural language processing has seen significant advances in recent years \cite{Jurafsky2009}."
%     \end{quote}
% \end{itemize}

% \paragraph{Textual/Narrative Citations}
% \begin{itemize}
%     \item \textbf{Command}: \texttt{\textbackslash citet\{key\}}
%     \item \textbf{Usage}: Use this when you want to reference the author as part of the sentence.
%     \item \textbf{Example}:
%     \begin{quote}
%         Text: "\verb!\cite{Jurafsky2009}! demonstrated the effectiveness of transformer models."
        
%         Output: "Jurafsky and Martin (2009) demonstrated the effectiveness of transformer models."
%     \end{quote}
% \end{itemize}


% \paragraph{Author's Possessive Form}
% \begin{itemize}
%     \item \textbf{Command}: \texttt{\textbackslash citeposs\{key\}}
%     \item \textbf{Usage}: Use this for a possessive form of the citation.
%     \item \textbf{Example}:
%     \begin{quote}
%         Text: "\verb!\citeposs{Jurafsky2009}! approach significantly improved accuracy."
        
%         Output: "\citeposs{Jurafsky2009} approach significantly improved accuracy."
%     \end{quote}
% \end{itemize}

% \paragraph{Multiple Citations}
% \begin{itemize}
%     \item \textbf{Command}: \texttt{\textbackslash citep\{key1, key2, key3\}}
%     \item \textbf{Usage}: Use this to cite multiple references within the same parentheses.
%     \item \textbf{Example}:
%     \begin{quote}
%         Text: "Several studies have explored this area\newline \verb!\citep{Jurafsky2009, Koehn2005, noauthor_semantic_nodate}!."
        
%         Output: "Several studies have explored this area \citep{Jurafsky2009, Koehn2005, noauthor_semantic_nodate}."
%     \end{quote}
% \end{itemize}


% \paragraph{Including URLs in References}
% \begin{itemize}
%     \item \textbf{Command}: Use a BibTeX entry with a \texttt{url} field.
%     \item \textbf{Usage}: The ACL style supports URLs in references, so you can include a URL directly in the BibTeX entry.
%     \item \textbf{Example} (BibTeX entry):
%     \begin{verbatim}
%     @article{Smith2020,
%       author = {John Smith},
%       title = {An Overview of NLP},
%       journal = {Journal of Computational Linguistics},
%       year = {2020},
%       url = {http://example.com/nlp-overview}
%     }
%     \end{verbatim}
%     Output: The URL will be properly formatted and included in the references section.
% \end{itemize}

% To fix your bibtex file, use the \href{https://vilda.net/s/ryanize-bib/}{Ryanize tool}.


% \section{Motivation}
%  Excepteur sint occaecat cupidatat non proident, sunt in culpa qui officia deserunt mollit anim id
% est laborum\footnote{Lorem ipsum dolor sit amet, consectetur adipiscing elit, sed do eiusmod tempor incididunt ut labore et dolore
% magna aliqua. }


% \section{Research Questions}

% The research questions that shall be answered in this thesis, are:
% \begin{enumerate}
%     \item RQ One
%     \item RQ Two
%     \item RQ Three\footnote{Lorem ipsum dolor sit amet, consectetur adipiscing elit, sed do eiusmod tempor incididunt ut labore et dolore
% magna aliqua. }
% \end{enumerate}

% \section{Thesis Structure}
% The remainder of this thesis is thematically divided into the following chapters:
% Chapter 2 introduces \ldots
% Chapter 3 presents \ldots
% \footnote{Lorem ipsum dolor sit amet, consectetur adipiscing elit, sed do eiusmod tempor incididunt ut labore et dolore
% magna aliqua. }

% self-defined macro: include bibliography even when compiling a single chapter
\subfilebibliography
\end{document}
