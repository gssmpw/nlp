\section{Prompts Used in Our Experiments}\label{supsec:prompt}
This section provides the prompts used in our experiments.
Table~\ref{sup:prompt-pdfstyle} shows the prompt used to generate PDF-style text.
Table~\ref{sup:prompt-inst} presents the prompt used to generate instruction data.

\begin{table*}[t]
\begin{tcolorbox}[colback=gray!10, colframe=black, rounded corners]
You are an AI visual assistant, and you are seeing a single image. Generate a passage that resembles text commonly found in PDF documents and is relevant to the given image. The provided image is extracted from a PDF, but no additional context, such as the document’s text or structure, is available.\\
PDF-style text generally has the following characteristics:\\
1. No explicit captions or minimal captions: Instead of directly describing the image, related text may naturally integrate into the document’s content.\\
2. Indirect descriptions: The text does not explicitly reference the image but provides supporting information that the image complements.
\\\\
To keep the text concise, generate only 1 to 2 sentences per image, ensuring it aligns with common PDF writing styles. \\
You must respond in Japanese.
\end{tcolorbox}
\vspace{-4mm}
\caption{\textbf{Prompt for generating PDF-style text.} 
An image is provided to GPT-4o-mini along with this prompt.}
\label{sup:prompt-pdfstyle}
\end{table*}


\begin{table*}[t]
\begin{tcolorbox}[colback=gray!10, colframe=black, rounded corners]
You are an AI visual assistant, and you are seeing a single image. What you see are provided within several sentences, describing the same image you are looking at. Answer all questions as you are seeing the image.
\\\\
Design a conversation between you and a person asking about this photo. The answers should be in a tone that a visual AI assistant is seeing the image and answering the question. \\
Ask diverse questions and give corresponding answers.
\\\\
Include questions asking about the visual content of the image, including the object types, counting the objects, object actions, object locations, relative positions between objects, etc. Only include questions that have definite answers: \\
(1) one can see the content in the image that the question asks about and can answer confidently; \\
(2) one can determine confidently from the image that it is not in the image. \\
Do not ask any question that cannot be answered confidently.
\\\\
Also include complex questions that are relevant to the content in the image, for example, asking about background knowledge of the objects in the image, asking to discuss about events happening in the image, etc. Again, do not ask about uncertain details. \\
Provide detailed answers when answering complex questions. For example, give detailed examples or reasoning steps to make the content more convincing and well-organized.  You can include multiple paragraphs if necessary. 
\\\\ 
You must use Japanese all the time. \\
When creating a question, start with `質問:'.\\
When creating a response, start with `回答:'.\\
After finishing a question or response, always separate them with `\textbackslash n\textbackslash n'.
\end{tcolorbox}
\vspace{-4mm}
\caption{\textbf{Prompt to generate instruction data. }
An image and paired text are provided to GPT-4o-mini along with this prompt.}
\label{sup:prompt-inst}
\end{table*}




\section{Qualitative Analysis}\label{supsec:quali}
Figures~\ref{fig:heronqa1}, \ref{fig:heronqa2}, and \ref{fig:heronqa3} present qualitative analyses. Each figure includes an image, a question, the reference answer from GPT-4, the response from LLaVA1.5-Swallow trained up to stages 1 and 2, and the response from LLaVA1.5-Swallow further trained with stage 3 (CFT on PDF). The results show that performance improves after training up to stage 3, demonstrating the effectiveness of CFT using PDF data.

\begin{figure*}[t]
  \includegraphics[width=\linewidth]{images/heronQA1.jpg}
  \vspace{-4mm}
  \caption{\textbf{Qualitative analysis on Heron-Bench.} 
  Correct parts of the responses are highlighted in green, while incorrect parts are marked in red.}
  \label{fig:heronqa1}
\end{figure*}

\begin{figure*}[t]
  \includegraphics[width=\linewidth]{images/heronQA2.jpg}
  \vspace{-4mm}
  \caption{\textbf{Another example of qualitative analysis on Heron-Bench.}}
  \label{fig:heronqa2}
\end{figure*}

\begin{figure*}[t]
  \includegraphics[width=\linewidth]{images/heronQA3.jpg}
  \vspace{-4mm}
  \caption{\textbf{Further qualitative analysis on Heron-Bench.}}
  \label{fig:heronqa3}
\end{figure*}