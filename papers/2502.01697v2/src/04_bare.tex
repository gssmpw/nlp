\section{Base-Refine (\Sys{})}
\label{bare}

\begin{figure*}[ht!]
\vskip 0.2in
\begin{center}
\centerline{\includegraphics[width=.9\textwidth]{assets/sys2.pdf}}
\caption{Instruct-tuned models provide high-quality but low-diversity data, while base models provide low-quality but high-diversity data. \Sys{} independently generates a diverse initial set of data points with a base model and refines each entry individually with an instruct-tuned model to create a high-quality, high-diversity dataset. In this example of a real grade school math problem generation, the Llama-3.1-70B-Base model hallucinates in its answer to its own question. The refiner (Llama-3.1-70B-Instruct) recognizes this and disambiguates the question and corrects the reasoning.}
\label{system-diagram}
\end{center}
\vskip -0.2in
\end{figure*}


We leverage our insights to propose \textbf{\Sys{}}, a practical synthetic data generation method combining the diversity of base models with the quality of instruct models. \Sys{} uses a base model to generate an initial set of diverse but potentially lower quality data, after which an instruct-tuned model individually refines each example from the initial set -- as shown in the example in \cref{system-diagram}.

In the base step, the base model uses a prompt with minimal few-shot examples to generate. In the refine step, the instruct-tuned model is instructed to retain the concept of the base model generation while improving it according to specific criteria (e.g., realism, correctness). This retains the overall diversity of the base model generated set while exerting greater control over the quality of the final generation.

Importantly, few-shot examples are only required in the base step to ensure formatting (though they can also be included in the refine step), and only in small amounts. Thus, \textbf{\Sys{}} is especially useful in domains with very limited real-world data as little human effort is required to generate a diverse high-quality synthetic dataset. In our experiments, we limit ourselves to just three few-shot examples. In addition, we intentionally use very general prompts for \Sys{} to demonstrate its flexibility, underscoring the potential for even greater improvement with tailored prompts. Representative prompts are included in \cref{appendix_prompts}.