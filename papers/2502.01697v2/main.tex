\documentclass{article}

% Recommended, but optional, packages for figures and better typesetting:
\usepackage{microtype}
\usepackage{graphicx}
\usepackage{subfigure}
\usepackage{float}
\usepackage{booktabs} % for professional tables


\usepackage[authoryear]{natbib}

\usepackage{hyperref}


%
\setlength\unitlength{1mm}
\newcommand{\twodots}{\mathinner {\ldotp \ldotp}}
% bb font symbols
\newcommand{\Rho}{\mathrm{P}}
\newcommand{\Tau}{\mathrm{T}}

\newfont{\bbb}{msbm10 scaled 700}
\newcommand{\CCC}{\mbox{\bbb C}}

\newfont{\bb}{msbm10 scaled 1100}
\newcommand{\CC}{\mbox{\bb C}}
\newcommand{\PP}{\mbox{\bb P}}
\newcommand{\RR}{\mbox{\bb R}}
\newcommand{\QQ}{\mbox{\bb Q}}
\newcommand{\ZZ}{\mbox{\bb Z}}
\newcommand{\FF}{\mbox{\bb F}}
\newcommand{\GG}{\mbox{\bb G}}
\newcommand{\EE}{\mbox{\bb E}}
\newcommand{\NN}{\mbox{\bb N}}
\newcommand{\KK}{\mbox{\bb K}}
\newcommand{\HH}{\mbox{\bb H}}
\newcommand{\SSS}{\mbox{\bb S}}
\newcommand{\UU}{\mbox{\bb U}}
\newcommand{\VV}{\mbox{\bb V}}


\newcommand{\yy}{\mathbbm{y}}
\newcommand{\xx}{\mathbbm{x}}
\newcommand{\zz}{\mathbbm{z}}
\newcommand{\sss}{\mathbbm{s}}
\newcommand{\rr}{\mathbbm{r}}
\newcommand{\pp}{\mathbbm{p}}
\newcommand{\qq}{\mathbbm{q}}
\newcommand{\ww}{\mathbbm{w}}
\newcommand{\hh}{\mathbbm{h}}
\newcommand{\vvv}{\mathbbm{v}}

% Vectors

\newcommand{\av}{{\bf a}}
\newcommand{\bv}{{\bf b}}
\newcommand{\cv}{{\bf c}}
\newcommand{\dv}{{\bf d}}
\newcommand{\ev}{{\bf e}}
\newcommand{\fv}{{\bf f}}
\newcommand{\gv}{{\bf g}}
\newcommand{\hv}{{\bf h}}
\newcommand{\iv}{{\bf i}}
\newcommand{\jv}{{\bf j}}
\newcommand{\kv}{{\bf k}}
\newcommand{\lv}{{\bf l}}
\newcommand{\mv}{{\bf m}}
\newcommand{\nv}{{\bf n}}
\newcommand{\ov}{{\bf o}}
\newcommand{\pv}{{\bf p}}
\newcommand{\qv}{{\bf q}}
\newcommand{\rv}{{\bf r}}
\newcommand{\sv}{{\bf s}}
\newcommand{\tv}{{\bf t}}
\newcommand{\uv}{{\bf u}}
\newcommand{\wv}{{\bf w}}
\newcommand{\vv}{{\bf v}}
\newcommand{\xv}{{\bf x}}
\newcommand{\yv}{{\bf y}}
\newcommand{\zv}{{\bf z}}
\newcommand{\zerov}{{\bf 0}}
\newcommand{\onev}{{\bf 1}}

% Matrices

\newcommand{\Am}{{\bf A}}
\newcommand{\Bm}{{\bf B}}
\newcommand{\Cm}{{\bf C}}
\newcommand{\Dm}{{\bf D}}
\newcommand{\Em}{{\bf E}}
\newcommand{\Fm}{{\bf F}}
\newcommand{\Gm}{{\bf G}}
\newcommand{\Hm}{{\bf H}}
\newcommand{\Id}{{\bf I}}
\newcommand{\Jm}{{\bf J}}
\newcommand{\Km}{{\bf K}}
\newcommand{\Lm}{{\bf L}}
\newcommand{\Mm}{{\bf M}}
\newcommand{\Nm}{{\bf N}}
\newcommand{\Om}{{\bf O}}
\newcommand{\Pm}{{\bf P}}
\newcommand{\Qm}{{\bf Q}}
\newcommand{\Rm}{{\bf R}}
\newcommand{\Sm}{{\bf S}}
\newcommand{\Tm}{{\bf T}}
\newcommand{\Um}{{\bf U}}
\newcommand{\Wm}{{\bf W}}
\newcommand{\Vm}{{\bf V}}
\newcommand{\Xm}{{\bf X}}
\newcommand{\Ym}{{\bf Y}}
\newcommand{\Zm}{{\bf Z}}

% Calligraphic

\newcommand{\Ac}{{\cal A}}
\newcommand{\Bc}{{\cal B}}
\newcommand{\Cc}{{\cal C}}
\newcommand{\Dc}{{\cal D}}
\newcommand{\Ec}{{\cal E}}
\newcommand{\Fc}{{\cal F}}
\newcommand{\Gc}{{\cal G}}
\newcommand{\Hc}{{\cal H}}
\newcommand{\Ic}{{\cal I}}
\newcommand{\Jc}{{\cal J}}
\newcommand{\Kc}{{\cal K}}
\newcommand{\Lc}{{\cal L}}
\newcommand{\Mc}{{\cal M}}
\newcommand{\Nc}{{\cal N}}
\newcommand{\nc}{{\cal n}}
\newcommand{\Oc}{{\cal O}}
\newcommand{\Pc}{{\cal P}}
\newcommand{\Qc}{{\cal Q}}
\newcommand{\Rc}{{\cal R}}
\newcommand{\Sc}{{\cal S}}
\newcommand{\Tc}{{\cal T}}
\newcommand{\Uc}{{\cal U}}
\newcommand{\Wc}{{\cal W}}
\newcommand{\Vc}{{\cal V}}
\newcommand{\Xc}{{\cal X}}
\newcommand{\Yc}{{\cal Y}}
\newcommand{\Zc}{{\cal Z}}

% Bold greek letters

\newcommand{\alphav}{\hbox{\boldmath$\alpha$}}
\newcommand{\betav}{\hbox{\boldmath$\beta$}}
\newcommand{\gammav}{\hbox{\boldmath$\gamma$}}
\newcommand{\deltav}{\hbox{\boldmath$\delta$}}
\newcommand{\etav}{\hbox{\boldmath$\eta$}}
\newcommand{\lambdav}{\hbox{\boldmath$\lambda$}}
\newcommand{\epsilonv}{\hbox{\boldmath$\epsilon$}}
\newcommand{\nuv}{\hbox{\boldmath$\nu$}}
\newcommand{\muv}{\hbox{\boldmath$\mu$}}
\newcommand{\zetav}{\hbox{\boldmath$\zeta$}}
\newcommand{\phiv}{\hbox{\boldmath$\phi$}}
\newcommand{\psiv}{\hbox{\boldmath$\psi$}}
\newcommand{\thetav}{\hbox{\boldmath$\theta$}}
\newcommand{\tauv}{\hbox{\boldmath$\tau$}}
\newcommand{\omegav}{\hbox{\boldmath$\omega$}}
\newcommand{\xiv}{\hbox{\boldmath$\xi$}}
\newcommand{\sigmav}{\hbox{\boldmath$\sigma$}}
\newcommand{\piv}{\hbox{\boldmath$\pi$}}
\newcommand{\rhov}{\hbox{\boldmath$\rho$}}
\newcommand{\upsilonv}{\hbox{\boldmath$\upsilon$}}

\newcommand{\Gammam}{\hbox{\boldmath$\Gamma$}}
\newcommand{\Lambdam}{\hbox{\boldmath$\Lambda$}}
\newcommand{\Deltam}{\hbox{\boldmath$\Delta$}}
\newcommand{\Sigmam}{\hbox{\boldmath$\Sigma$}}
\newcommand{\Phim}{\hbox{\boldmath$\Phi$}}
\newcommand{\Pim}{\hbox{\boldmath$\Pi$}}
\newcommand{\Psim}{\hbox{\boldmath$\Psi$}}
\newcommand{\Thetam}{\hbox{\boldmath$\Theta$}}
\newcommand{\Omegam}{\hbox{\boldmath$\Omega$}}
\newcommand{\Xim}{\hbox{\boldmath$\Xi$}}


% Sans Serif small case

\newcommand{\Gsf}{{\sf G}}

\newcommand{\asf}{{\sf a}}
\newcommand{\bsf}{{\sf b}}
\newcommand{\csf}{{\sf c}}
\newcommand{\dsf}{{\sf d}}
\newcommand{\esf}{{\sf e}}
\newcommand{\fsf}{{\sf f}}
\newcommand{\gsf}{{\sf g}}
\newcommand{\hsf}{{\sf h}}
\newcommand{\isf}{{\sf i}}
\newcommand{\jsf}{{\sf j}}
\newcommand{\ksf}{{\sf k}}
\newcommand{\lsf}{{\sf l}}
\newcommand{\msf}{{\sf m}}
\newcommand{\nsf}{{\sf n}}
\newcommand{\osf}{{\sf o}}
\newcommand{\psf}{{\sf p}}
\newcommand{\qsf}{{\sf q}}
\newcommand{\rsf}{{\sf r}}
\newcommand{\ssf}{{\sf s}}
\newcommand{\tsf}{{\sf t}}
\newcommand{\usf}{{\sf u}}
\newcommand{\wsf}{{\sf w}}
\newcommand{\vsf}{{\sf v}}
\newcommand{\xsf}{{\sf x}}
\newcommand{\ysf}{{\sf y}}
\newcommand{\zsf}{{\sf z}}


% mixed symbols

\newcommand{\sinc}{{\hbox{sinc}}}
\newcommand{\diag}{{\hbox{diag}}}
\renewcommand{\det}{{\hbox{det}}}
\newcommand{\trace}{{\hbox{tr}}}
\newcommand{\sign}{{\hbox{sign}}}
\renewcommand{\arg}{{\hbox{arg}}}
\newcommand{\var}{{\hbox{var}}}
\newcommand{\cov}{{\hbox{cov}}}
\newcommand{\Ei}{{\rm E}_{\rm i}}
\renewcommand{\Re}{{\rm Re}}
\renewcommand{\Im}{{\rm Im}}
\newcommand{\eqdef}{\stackrel{\Delta}{=}}
\newcommand{\defines}{{\,\,\stackrel{\scriptscriptstyle \bigtriangleup}{=}\,\,}}
\newcommand{\<}{\left\langle}
\renewcommand{\>}{\right\rangle}
\newcommand{\herm}{{\sf H}}
\newcommand{\trasp}{{\sf T}}
\newcommand{\transp}{{\sf T}}
\renewcommand{\vec}{{\rm vec}}
\newcommand{\Psf}{{\sf P}}
\newcommand{\SINR}{{\sf SINR}}
\newcommand{\SNR}{{\sf SNR}}
\newcommand{\MMSE}{{\sf MMSE}}
\newcommand{\REF}{{\RED [REF]}}

% Markov chain
\usepackage{stmaryrd} % for \mkv 
\newcommand{\mkv}{-\!\!\!\!\minuso\!\!\!\!-}

% Colors

\newcommand{\RED}{\color[rgb]{1.00,0.10,0.10}}
\newcommand{\BLUE}{\color[rgb]{0,0,0.90}}
\newcommand{\GREEN}{\color[rgb]{0,0.80,0.20}}

%%%%%%%%%%%%%%%%%%%%%%%%%%%%%%%%%%%%%%%%%%
\usepackage{hyperref}
\hypersetup{
    bookmarks=true,         % show bookmarks bar?
    unicode=false,          % non-Latin characters in AcrobatÕs bookmarks
    pdftoolbar=true,        % show AcrobatÕs toolbar?
    pdfmenubar=true,        % show AcrobatÕs menu?
    pdffitwindow=false,     % window fit to page when opened
    pdfstartview={FitH},    % fits the width of the page to the window
%    pdftitle={My title},    % title
%    pdfauthor={Author},     % author
%    pdfsubject={Subject},   % subject of the document
%    pdfcreator={Creator},   % creator of the document
%    pdfproducer={Producer}, % producer of the document
%    pdfkeywords={keyword1} {key2} {key3}, % list of keywords
    pdfnewwindow=true,      % links in new window
    colorlinks=true,       % false: boxed links; true: colored links
    linkcolor=red,          % color of internal links (change box color with linkbordercolor)
    citecolor=green,        % color of links to bibliography
    filecolor=blue,      % color of file links
    urlcolor=blue           % color of external links
}
%%%%%%%%%%%%%%%%%%%%%%%%%%%%%%%%%%%%%%%%%%%



% Use the following line for the initial blind version submitted for review:
% \usepackage{icml2025}

% If accepted, instead use the following line for the camera-ready submission:
\usepackage[accepted]{icml2025}

% For theorems and such
\usepackage{amsmath}
\usepackage{amssymb}
\usepackage{mathtools}
\usepackage{amsthm}


\usepackage[most]{tcolorbox}
\usepackage{xcolor}

% if you use cleveref..
\usepackage[capitalize,noabbrev]{cleveref}

%%%%%%%%%%%%%%%%%%%%%%%%%%%%%%%%
% THEOREMS
%%%%%%%%%%%%%%%%%%%%%%%%%%%%%%%%
\theoremstyle{plain}
\newtheorem{theorem}{Theorem}[section]
\newtheorem{proposition}[theorem]{Proposition}
\newtheorem{lemma}[theorem]{Lemma}
\newtheorem{corollary}[theorem]{Corollary}
\theoremstyle{definition}
\newtheorem{definition}[theorem]{Definition}
\newtheorem{assumption}[theorem]{Assumption}
\theoremstyle{remark}
\newtheorem{remark}[theorem]{Remark}

% Todonotes is useful during development; simply uncomment the next line
%    and comment out the line below the next line to turn off comments
% \usepackage[disable,textsize=tiny]{todonotes}
% \usepackage[textsize=tiny]{todonotes}


% The \icmltitle you define below is probably too long as a header.
% Therefore, a short form for the running title is supplied here:
\icmltitlerunning{\Sys{}: Combining Base and Instruction-Tuned Language Models for Better Synthetic Data Generation}

\begin{document}
\twocolumn[
\icmltitle{\Sys{}: Combining Base and Instruction-Tuned Language Models for Better Synthetic Data Generation}

% It is OKAY to include author information, even for blind
% submissions: the style file will automatically remove it for you
% unless you've provided the [accepted] option to the icml2025
% package.

% List of affiliations: The first argument should be a (short)
% identifier you will use later to specify author affiliations
% Academic affiliations should list Department, University, City, Region, Country
% Industry affiliations should list Company, City, Region, Country

% You can specify symbols, otherwise they are numbered in order.
% Ideally, you should not use this facility. Affiliations will be numbered
% in order of appearance and this is the preferred way.
\icmlsetsymbol{equal}{*}

\begin{icmlauthorlist}
\icmlauthor{Alan Zhu}{equal,berkeley}
\icmlauthor{Parth Asawa}{equal,berkeley}
\icmlauthor{Jared Quincy Davis}{stanford,foundry}
\icmlauthor{Lingjiao Chen}{stanford,microsoft}
\icmlauthor{Boris Hanin}{princeton,foundry}
\icmlauthor{Ion Stoica}{berkeley}
\icmlauthor{Joseph E. Gonzalez}{berkeley}
\icmlauthor{Matei Zaharia}{berkeley}
\end{icmlauthorlist}

\icmlaffiliation{berkeley}{UC Berkeley}
\icmlaffiliation{stanford}{Stanford University}
\icmlaffiliation{princeton}{Princeton University}
\icmlaffiliation{microsoft}{Microsoft Research}
\icmlaffiliation{foundry}{Foundry}

\icmlcorrespondingauthor{Alan Zhu}{aczhu@berkeley.edu}
\icmlcorrespondingauthor{Parth Asawa}{pgasawa@berkeley.edu}

% You may provide any keywords that you
% find helpful for describing your paper; these are used to populate
% the "keywords" metadata in the PDF but will not be shown in the document
\icmlkeywords{Machine Learning, ICML}

\vskip 0.2in
]

% this must go after the closing bracket ] following \twocolumn[ ...

% This command actually creates the footnote in the first column
% listing the affiliations and the copyright notice.
% The command takes one argument, which is text to display at the start of the footnote.
% The \icmlEqualContribution command is standard text for equal contribution.
% Remove it (just {}) if you do not need this facility.

% \printAffiliationsAndNotice{}  % leave blank if no need to mention equal contribution
\printAffiliationsAndNotice{\icmlEqualContribution} % otherwise use the standard text.

\begin{abstract}
  In this work, we present a novel technique for GPU-accelerated Boolean satisfiability (SAT) sampling. Unlike conventional sampling algorithms that directly operate on conjunctive normal form (CNF), our method transforms the logical constraints of SAT problems by factoring their CNF representations into simplified multi-level, multi-output Boolean functions. It then leverages gradient-based optimization to guide the search for a diverse set of valid solutions. Our method operates directly on the circuit structure of refactored SAT instances, reinterpreting the SAT problem as a supervised multi-output regression task. This differentiable technique enables independent bit-wise operations on each tensor element, allowing parallel execution of learning processes. As a result, we achieve GPU-accelerated sampling with significant runtime improvements ranging from $33.6\times$ to $523.6\times$ over state-of-the-art heuristic samplers. We demonstrate the superior performance of our sampling method through an extensive evaluation on $60$ instances from a public domain benchmark suite utilized in previous studies. 


  
  % Generating a wide range of diverse solutions to logical constraints is crucial in software and hardware testing, verification, and synthesis. These solutions can serve as inputs to test specific functionalities of a software program or as random stimuli in hardware modules. In software verification, techniques like fuzz testing and symbolic execution use this approach to identify bugs and vulnerabilities. In hardware verification, stimulus generation is particularly vital, forming the basis of constrained-random verification. While generating multiple solutions improves coverage and increases the chances of finding bugs, high-throughput sampling remains challenging, especially with complex constraints and refined coverage criteria. In this work, we present a novel technique that enables GPU-accelerated sampling, resulting in high-throughput generation of satisfying solutions to Boolean satisfiability (SAT) problems. Unlike conventional sampling algorithms that directly operate on conjunctive normal form (CNF), our method refines the logical constraints of SAT problems by transforming their CNF into simplified multi-level Boolean expressions. It then leverages gradient-based optimization to guide the search for a diverse set of valid solutions.
  % Our method specifically takes advantage of the circuit structure of refined SAT instances by using GD to learn valid solutions, reinterpreting the SAT problem as a supervised multi-output regression task. This differentiable technique enables independent bit-wise operations on each tensor element, allowing parallel execution of learning processes. As a result, we achieve GPU-accelerated sampling with significant runtime improvements ranging from $10\times$ to $1000\times$ over state-of-the-art heuristic samplers. Specifically, we demonstrate the superior performance of our sampling method through an extensive evaluation on $60$ instances from a public domain benchmark suite utilized in previous studies.

\end{abstract}

\begin{IEEEkeywords}
Boolean Satisfiability, Gradient Descent, Multi-level Circuits, Verification, and Testing.
\end{IEEEkeywords}

\section{Introduction}\label{sec:Intro} 


Novel view synthesis offers a fundamental approach to visualizing complex scenes by generating new perspectives from existing imagery. 
This has many potential applications, including virtual reality, movie production and architectural visualization \cite{Tewari2022NeuRendSTAR}. 
An emerging alternative to the common RGB sensors are event cameras, which are  
 bio-inspired visual sensors recording events, i.e.~asynchronous per-pixel signals of changes in brightness or color intensity. 

Event streams have very high temporal resolution and are inherently sparse, as they only happen when changes in the scene are observed. 
Due to their working principle, event cameras bring several advantages, especially in challenging cases: they excel at handling high-speed motions 
and have a substantially higher dynamic range of the supported signal measurements than conventional RGB cameras. 
Moreover, they have lower power consumption and require varied storage volumes for captured data that are often smaller than those required for synchronous RGB cameras \cite{Millerdurai_3DV2024, Gallego2022}. 

The ability to handle high-speed motions is crucial in static scenes as well,  particularly with handheld moving cameras, as it helps avoid the common problem of motion blur. It is, therefore, not surprising that event-based novel view synthesis has gained attention, although color values are not directly observed.
Notably, because of the substantial difference between the formats, RGB- and event-based approaches require fundamentally different design choices. %

The first solutions to event-based novel view synthesis introduced in the literature demonstrate promising results \cite{eventnerf, enerf} and outperform non-event-based alternatives for novel view synthesis in many challenging scenarios. 
Among them, EventNeRF \cite{eventnerf} enables novel-view synthesis in the RGB space by assuming events associated with three color channels as inputs. 
Due to its NeRF-based architecture \cite{nerf}, it can handle single objects with complete observations from roughly equal distances to the camera. 
It furthermore has limitations in training and rendering speed: 
the MLP used to represent the scene requires long training time and can only handle very limited scene extents or otherwise rendering quality will deteriorate. 
Hence, the quality of synthesized novel views will degrade for larger scenes. %

We present Event-3DGS (E-3DGS), i.e.,~a new method for novel-view synthesis from event streams using 3D Gaussians~\cite{3dgs} 
demonstrating fast reconstruction and rendering as well as handling of unbounded scenes. 
The technical contributions of this paper are as follows: 
\begin{itemize}
\item With E-3DGS, we introduce the first approach for novel view synthesis from a color event camera that combines 3D Gaussians with event-based supervision. 
\item We present frustum-based initialization, adaptive event windows, isotropic 3D Gaussian regularization and 3D camera pose refinement, and demonstrate that high-quality results can be obtained. %

\item Finally, we introduce new synthetic and real event datasets for large scenes to the community to study novel view synthesis in this new problem setting. 
\end{itemize}
Our experiments demonstrate systematically superior results compared to EventNeRF \cite{eventnerf} and other baselines. 
The source code and dataset of E-3DGS are released\footnote{\url{https://4dqv.mpi-inf.mpg.de/E3DGS/}}. 





\section{Related Work}
\label{lit_review}

\begin{highlight}
{

Our research builds upon {\em (i)} Assessing Web Accessibility, {\em (ii)} End-User Accessibility Repair, and {\em (iii)} Developer Tools for Accessibility.

\subsection{Assessing Web Accessibility}
From the earliest attempts to set standards and guidelines, web accessibility has been shaped by a complex interplay of technical challenges, legal imperatives, and educational campaigns. Over the past 25 years, stakeholders have sought to improve digital inclusion by establishing foundational standards~\cite{chisholm2001web, caldwell2008web}, enforcing legal obligations~\cite{sierkowski2002achieving, yesilada2012understanding}, and promoting a broader culture of accessibility awareness among developers~\cite{sloan2006contextual, martin2022landscape, pandey2023blending}. 
Despite these longstanding efforts, systemic accessibility issues persist. According to the 2024 WebAIM Million report~\cite{webaim2024}, 95.9\% of the top one million home pages contained detectable WCAG violations, averaging nearly 57 errors per page. 
These errors take many forms: low color contrast makes the interface difficult for individuals with color deficiency or low vision to read text; missing alternative text leaves users relying on screen readers without crucial visual context; and unlabeled form inputs or empty links and buttons hinder people who navigate with assistive technologies from completing basic tasks. 
Together, these accessibility issues not only limit user access to critical online resources such as healthcare, education, and employment but also result in significant legal risks and lost opportunities for businesses to engage diverse audiences. Addressing these pervasive issues requires systematic methods to identify, measure, and prioritize accessibility barriers, which is the first step toward achieving meaningful improvements.

Prior research has introduced methods blending automation and human evaluation to assess web accessibility. Hybrid approaches like SAMBA combine automated tools with expert reviews to measure the severity and impact of barriers, enhancing evaluation reliability~\cite{brajnik2007samba}. Quantitative metrics, such as Failure Rate and Unified Web Evaluation Methodology, support large-scale monitoring and comparative analysis, enabling cost-effective insights~\cite{vigo2007quantitative, martins2024large}. However, automated tools alone often detect less than half of WCAG violations and generate false positives, emphasizing the need for human interpretation~\cite{freire2008evaluation, vigo2013benchmarking}. Recent progress with large pretrained models like Large Language Models (LLMs)~\cite{dubey2024llama,bai2023qwen} and Large Multimodal Models (LMMs)~\cite{liu2024visual, bai2023qwenvl} offers a promising step forward, automating complex checks like non-text content evaluation and link purposes, achieving higher detection rates than traditional tools~\cite{lopez2024turning, delnevo2024interaction}. Yet, these large models face challenges, including dependence on training data, limited contextual judgment, and the inability to simulate real user experiences. These limitations underscore the necessity of combining models with human oversight for reliable, user-centered evaluations~\cite{brajnik2007samba, vigo2013benchmarking, delnevo2024interaction}. 

Our work builds on these prior efforts and recent advancements by leveraging the capabilities of large pretrained models while addressing their limitations through a developer-centric approach. CodeA11y integrates LLM-powered accessibility assessments, tailored accessibility-aware system prompts, and a dedicated accessibility checker directly into GitHub Copilot---one of the most widely used coding assistants. Unlike standalone evaluation tools, CodeA11y actively supports developers throughout the coding process by reinforcing accessibility best practices, prompting critical manual validations, and embedding accessibility considerations into existing workflows.
% This pervasive shortfall reflects the difficulty of scaling traditional approaches---such as manual audits and automated tools---that either demand immense human effort or lack the nuanced understanding needed to capture real-world user experiences. 
%
% In response, a new wave of AI-driven methods, many powered by large language models (LLMs), is emerging to bridge these accessibility detection and assessment gaps. Early explorations, such as those by Morillo et al.~\cite{morillo2020system}, introduced AI-assisted recommendations capable of automatic corrections, illustrating how computational intelligence can tackle the repetitive, common errors that plague large swaths of the web. Building on this foundation, Huang et al.~\cite{huang2024access} proposed ACCESS, a prompt-engineering framework that streamlines the identification and remediation of accessibility violations, while López-Gil et al.~\cite{lopez2024turning} demonstrated how LLMs can help apply WCAG success criteria more consistently---reducing the reliance on manual effort. Beyond these direct interventions, recent work has also begun integrating user experiences more seamlessly into the evaluation process. For example, Huq et al.~\cite{huq2024automated} translate user transcripts and corresponding issues into actionable test reports, ensuring that accessibility improvements align more closely with authentic user needs.
% However, as these AI-driven solutions evolve, researchers caution against uncritical adoption. Othman et al.~\cite{othman2023fostering} highlight that while LLMs can accelerate remediation, they may also introduce biases or encourage over-reliance on automated processes. Similarly, Delnevo et al.~\cite{delnevo2024interaction} emphasize the importance of contextual understanding and adaptability, pointing to the current limitations of LLM-based systems in serving the full spectrum of user needs. 
% In contrast to this backdrop, our work introduces and evaluates CodeA11y, an LLM-augmented extension for GitHub Copilot that not only mitigates these challenges by providing more consistent guidance and manual validation prompts, but also aligns AI-driven assistance with developers’ workflows, ultimately contributing toward more sustainable propulsion for building accessible web.

% Broader implications of inaccessibility—legal compliance, ethical concerns, and user experience
% A Historical Review of Web Accessibility Using WAVE
% "I tend to view ads almost like a pestilence": On the Accessibility Implications of Mobile Ads for Blind Users

% In the research domain, several methods have been developed to assess and enhance web accessibility. These include incorporating feedback into developer tools~\cite{adesigner, takagi2003accessibility, bigham2010accessibility} and automating the creation of accessibility tests and reports for user interfaces~\cite{swearngin2024towards, taeb2024axnav}. 

% Prior work has also studied accessibility scanners as another avenue of AI to improve web development practices~\cite{}.
% However, a persistent challenge is that developers need to be aware of these tools to utilize them effectively. With recent advancements in LLMs, developers might now build accessible websites with less effort using AI assistants. However, the impact of these assistants on the accessibility of their generated code remains unclear. This study aims to investigate these effects.

\subsection{End-user Accessibility Repair}
In addition to detecting accessibility errors and measuring web accessibility, significant research has focused on fixing these problems.
Since end-users are often the first to notice accessibility problems and have a strong incentive to address them, systems have been developed to help them report or fix these problems.

Collaborative, or social accessibility~\cite{takagi2009collaborative,sato2010social}, enabled these end-user contributions to be scaled through crowd-sourcing.
AccessMonkey~\cite{bigham2007accessmonkey} and Accessibility Commons~\cite{kawanaka2008accessibility} were two examples of repositories that store accessibility-related scripts and metadata, respectively.
Other work has developed browser extensions that leverage crowd-sourced databases to automatically correct reading order, alt-text, color contrast, and interaction-related issues~\cite{sato2009s,huang2015can}.

One drawback of collaborative accessibility approaches is that they cannot fix problems for an ``unseen'' web page on-demand, so many projects aim to automatically detect and improve interfaces without the need for an external source of fixes.
A large body of research has focused on making specific web media (e.g., images~\cite{gleason2019making,guinness2018caption, twitterally, gleason2020making, lee2021image}, design~\cite{potluri2019ai,li2019editing, peng2022diffscriber, peng2023slide}, and videos~\cite{pavel2020rescribe,peng2021say,peng2021slidecho,huh2023avscript}) accessible through a combination of machine learning (ML) and user-provided fixes.
Other work has focused on applying more general fixes across all websites.

Opportunity accessibility addressed a common accessibility problem of most websites: by default, content is often hard to see for people with visual impairments, and many users, especially older adults, do not know how to adjust or enable content zooming~\cite{bigham2014making}.
To this end, a browser script (\texttt{oppaccess.js}) was developed that automatically adjusted the browser's content zoom to maximally enlarge content without introducing adverse side-effects (\textit{e.g.,} content overlap).
While \texttt{oppaccess.js} primarily targeted zoom-related accessibility, recent work aimed to enable larger types of changes, by using LLMs to modify the source code of web pages based on user questions or directives~\cite{li2023using}.

Several efforts have been focused on improving access to desktop and mobile applications, which present additional challenges due to the unavailability of app source code (\textit{e.g.,} HTML).
Prefab is an approach that allows graphical UIs to be modified at runtime by detecting existing UI widgets, then replacing them~\cite{dixon2010prefab}.
Interaction Proxies used these runtime modification strategies to ``repair'' Android apps by replacing inaccessible widgets with improved alternatives~\cite{zhang2017interaction, zhang2018robust}.
The widget detection strategies used by these systems previously relied on a combination of heuristics and system metadata (\textit{e.g.,} the view hierarchy), which are incomplete or missing in the accessible apps.
To this end, ML has been employed to better localize~\cite{chen2020object} and repair UI elements~\cite{chen2020unblind,zhang2021screen,wu2023webui,peng2025dreamstruct}.

In general, end-user solutions to repairing application accessibility are limited due to the lack of underlying code and knowledge of the semantics of the intended content.

\subsection{Developer Tools for Accessibility}
Ultimately, the best solution for ensuring an accessible experience lies with front-end developers. Many efforts have focused on building adequate tooling and support to help developers with ensuring that their UI code complies with accessibility standards.

Numerous automated accessibility testing tools have been created to help developers identify accessibility issues in their code: i) static analysis tools, such as IBM Equal Access Accessibility Checker~\cite{ibm2024toolkit} or Microsoft Accessibility Insights~\cite{accessibilityinsights2024}, scan the UI code's compliance with predefined rules derived from accessibility guidelines; and ii) dynamic or runtime accessibility scanners, such as Chrome Devtools~\cite{chromedevtools2024} or axe-Core Accessibility Engine~\cite{deque2024axe}, perform real-time testing on user interfaces to detect interaction issues not identifiable from the code structure. While these tools greatly reduce the manual effort required for accessibility testing, they are often criticized for their limited coverage. Thus, experts often recommend manually testing with assistive technologies to uncover more complex interaction issues. Prior studies have created accessibility crawlers that either assist in developer testing~\cite{swearngin2024towards,taeb2024axnav} or simulate how assistive technologies interact with UIs~\cite{10.1145/3411764.3445455, 10.1145/3551349.3556905, 10.1145/3544548.3580679}.

Similar to end-user accessibility repair, research has focused on generating fixes to remediate accessibility issues in the UI source code. Initial attempts developed heuristic-based algorithms for fixing specific issues, for instance, by replacing text or background color attributes~\cite{10.1145/3611643.3616329}. More recent work has suggested that the code-understanding capabilities of LLMs allow them to suggest more targeted fixes.
For example, a study demonstrated that prompting ChatGPT to fix identified WCAG compliance issues in source code could automatically resolve a significant number of them~\cite{othman2023fostering}. Researchers have sought to leverage this capability by employing a multi-agent LLM architecture to automatically identify and localize issues in source code and suggest potential code fixes~\cite{mehralian2024automated}.

While the approaches mentioned above focus on assessing UI accessibility of already-authored code (\textit{i.e.,} fixing existing code), there is potential for more proactive approaches.
For example, LLMs are often used by developers to generate UI source code from natural language descriptions or tab completions~\cite{chen2021evaluating,GitHubCopilot,lozhkov2024starcoder,hui2024qwen2,roziere2023code,zheng2023codegeex}, but LLMs frequently produce inaccessible code by default~\cite{10.1145/3677846.3677854,mowar2024tab}, leading to inaccessible output when used by developers without sufficient awareness of accessibility knowledge.
The primary focus of this paper is to design a more accessibility-aware coding assistant that both produces more accessible code without manual intervention (\textit{e.g.,} specific user prompting) and gradually enables developers to implement and improve accessibility of automatically-generated code through IDE UI modifications (\textit{e.g.}, reminder notifications).

}
\end{highlight}



% Work related to this paper includes {\em (i)} Web Accessibility and {\em (ii)} Developer Practices in AI-Assisted Programming.

% \ipstart{Web Accessibility: Practice, Evaluation, and Improvements} Substantial efforts have been made to set accessibility standards~\cite{chisholm2001web, caldwell2008web}, establish legal requirements~\cite{sierkowski2002achieving, yesilada2012understanding}, and promote education and advocacy among developers~\cite{sloan2006contextual, martin2022landscape, pandey2023blending}. In the research domain, several methods have been developed to assess and enhance web accessibility. These include incorporating feedback into developer tools~\cite{adesigner, takagi2003accessibility, bigham2010accessibility} and automating the creation of accessibility tests and reports for user interfaces~\cite{swearngin2024towards, taeb2024axnav}. 
% % Prior work has also studied accessibility scanners as another avenue of AI to improve web development practices~\cite{}.
% However, a persistent challenge is that developers need to be aware of these tools to utilize them effectively. With recent advancements in LLMs, developers might now build accessible websites with less effort using AI assistants. However, the impact of these assistants on the accessibility of their generated code remains unclear. This study aims to investigate these effects.

% \ipstart{Developer Practices in AI-Assisted Programming}
% Recent usability research on AI-assisted development has examined the interaction strategies of developers while using AI coding assistants~\cite{barke2023grounded}.
% They observed developers interacted with these assistants in two modes -- 1) \textit{acceleration mode}: associated with shorter completions and 2) \textit{exploration mode}: associated with long completions.
% Liang {\em et al.} \cite{liang2024large} found that developers are driven to use AI assistants to reduce their keystrokes, finish tasks faster, and recall the syntax of programming languages. On the other hand, developers' reason for rejecting autocomplete suggestions was the need for more consideration of appropriate software requirements. This is because primary research on code generation models has mainly focused on functional correctness while often sidelining non-functional requirements such as latency, maintainability, and security~\cite{singhal2024nofuneval}. Consequently, there have been increasing concerns about the security implications of AI-generated code~\cite{sandoval2023lost}. Similarly, this study focuses on the effectiveness and uptake of code suggestions among developers in mitigating accessibility-related vulnerabilities. 


% ============================= additional rw ============================================
% - Paulina Morillo, Diego Chicaiza-Herrera, and Diego Vallejo-Huanga. 2020. System of Recommendation and Automatic Correction of Web Accessibility Using Artificial Intelligence. In Advances in Usability and User Experience, Tareq Ahram and Christianne Falcão (Eds.). Springer International Publishing, Cham, 479–489
% - Juan-Miguel López-Gil and Juanan Pereira. 2024. Turning manual web accessibility success criteria into automatic: an LLM-based approach. Universal Access in the Information Society (2024). https://doi.org/10.1007/s10209-024-01108-z
% - s
% - Calista Huang, Alyssa Ma, Suchir Vyasamudri, Eugenie Puype, Sayem Kamal, Juan Belza Garcia, Salar Cheema, and Michael Lutz. 2024. ACCESS: Prompt Engineering for Automated Web Accessibility Violation Corrections. arXiv:2401.16450 [cs.HC] https://arxiv.org/abs/2401.16450
% - Syed Fatiul Huq, Mahan Tafreshipour, Kate Kalcevich, and Sam Malek. 2025. Automated Generation of Accessibility Test Reports from Recorded User Transcripts. In Proceedings of the 47th International Conference on Software Engineering (ICSE) (Ottawa, Ontario, Canada). IEEE. https://ics.uci.edu/~seal/publications/2025_ICSE_reca11.pdf To appear in IEEE Xplore
% - Achraf Othman, Amira Dhouib, and Aljazi Nasser Al Jabor. 2023. Fostering websites accessibility: A case study on the use of the Large Language Models ChatGPT for automatic remediation. In Proceedings of the 16th International Conference on PErvasive Technologies Related to Assistive Environments (Corfu, Greece) (PETRA ’23). Association for Computing Machinery, New York, NY, USA, 707–713. https://doi.org/10.1145/3594806.3596542
% - Zsuzsanna B. Palmer and Sushil K. Oswal. 0. Constructing Websites with Generative AI Tools: The Accessibility of Their Workflows and Products for Users With Disabilities. Journal of Business and Technical Communication 0, 0 (0), 10506519241280644. https://doi.org/10.1177/10506519241280644
% ============================= additional rw ============================================
\section{Motivation}
\label{sec:motivation}

% \yctcomment{3}{The reason we give the time spent for prev ref is that this graph shows our headroom as we only care about the time of prev ref (not the communication overhead of Hydra). Therefore, I suggest we change this section head to "Preventive refresh overhead of RowHammer....".}\agycomment{3}{This discussion should be over an email with Onur. It should not wait until the next iteration.}
\head{\yct{3}{Preventive refresh} overheads of RowHammer mitigation \om{1}{mechanisms}}
\srev{Prior works~\rowHammerDefenseScalingProblemsCitations{} already show that existing RowHammer mitigations \sql{R2.1}incur prohibitively high performance or area overheads as \gls{nrh} reduces \om{3}{with DRAM technology scaling}.} \agy{3}{This is because blocking a DRAM bank for performing a preventive refresh
% \agycomment{3}{Double-check whether preventive refresh is properly defined in Sec 2. If not, please define.}\yctcomment{3}{I defined and remove "RowHammer-preventive", we did not use anywhere that.}
might delay memory accesses and thus degrade system performance. We analyze} 
% We repeat this performance analysis with \param{five} state-of-the-art RowHammer mitigation \om{1}{mechanisms} by analyzing 
\yct{3}{the fraction of execution time \agy{3}{during which} a DRAM bank is \agy{3}{busy performing} preventive refreshes,}
% \agycomment{3}{the parenthesis did not say anything extra. I removed it}
% (i.e., the execution time spent on preventive refreshes)
% , including the more recent ones than the prior analyses. 
% We conduct these simulations 
following the methodology described in \secref{sec:eval_methodology} for \agy{3}{\param{five} state-of-the-art RowHammer mitigation \om{1}{mechanisms} and} \om{3}{six} different \gls{nrh} values \om{3}{on} \param{60} multi-programmed 4-core workload mixes.
% \agycomment{3}{I removed the randomly chosen explanation as it is already in the methodology in \secref{sec:eval_methodology}} 
% randomly chosen across benchmarks specified in \secref{sec:eval_methodology}.  
\figref{fig:performance_motivation} shows a line plot where each curve represents a different RowHammer mitigation \om{1}{mechanism}. The x-axis shows \om{7}{the} tested \gls{nrh} values and the y-axis shows
% the RowHammer mitigation \om{1}{mechanism}'s overhead in terms of 
% \agycomment{3}{I removed a part here. Does that look good?}
the fraction
% \agycomment{3}{this is fraction, figure says percentage. Why inconsistent? Make the figure fraction as well.}
of \om{3}{execution} time spent on preventive refreshes.
% \agycomment{3}{why does this have the \\om tag? isn't it your modification? Also, the fraction of execution time is repeated. It is not good.}
The shaded areas mark the minimum and maximum values across all workload mixes.
% \agycomment{3}{Onur has a comment here that is not addressed. Is this additional execution time due to preventive refreshes? Be clear}\yctcomment{3}{Thanks, added an explanation.} 
\yctcomment{7}{fixed colors}
\vspace{10pt}
\begin{figure}[ht]
\centering
\includegraphics[width=0.85\linewidth]{figures/fig3_RH_overhead.pdf}
\caption{\om{3}{Preventive refresh overhead of five RowHammer mitigation mechanisms as RowHammer vulnerability worsens}}
\label{fig:performance_motivation}
\end{figure}
% \agycomment{3}{mitigations, mitigation mechanisms, or mitigation techniques? Make your mind and be consistent}

% Graphene 64 100 ['0.78', '0.88', '0.97']
% Graphene 128 100 ['0.88', '0.95', '1.0']
% Graphene 256 100 ['0.94', '0.98', '1.0']
% Graphene 512 100 ['0.97', '0.99', '1.0']
% Graphene 1024 100 ['0.99', '1.0', '1.0']
% Graphene 2048 100 ['0.99', '1.0', '1.0']
% Hydra 64 100 ['0.13', '0.78', '0.97']
% Hydra 128 100 ['0.16', '0.84', '0.99']
% Hydra 256 100 ['0.24', '0.89', '1.0']
% Hydra 512 100 ['0.37', '0.94', '1.0']
% Hydra 1024 100 ['0.67', '0.97', '1.01']
% Hydra 2048 100 ['0.89', '0.99', '1.01']
% PARA 64 100 ['0.22', '0.49', '0.77']
% PARA 128 100 ['0.35', '0.62', '0.85']
% PARA 256 100 ['0.52', '0.75', '0.91']
% PARA 512 100 ['0.68', '0.85', '0.94']
% PARA 1024 100 ['0.8', '0.91', '0.98']
% PARA 2048 100 ['0.89', '0.95', '1.0']

\cqrev{
We make \param{four} observations from \figref{fig:performance_motivation}.
% \agycomment{3}{Onur has comments on the y-axis: Why do we focus on this metric and not overall performance overhead?
% Does it not also make sense to show overall performance overhead numbers here?
% Writing introduces overall performance overhead late.
% And the y axis metric of this figure is not precisely defined.} \yctcomment{3}{The reason why we use preventive refresh overhead rather than overall performance overhead is that we only reduce the time spent on prev refs (not Hydra's communication overhead). Basically this figure shows our headroom. I suggest we change the head of the section as "Preventive refresh overhead of RowHammer..."}
First, all tested RowHammer mitigation \om{1}{mechanisms} spend a larger fraction of their time performing preventive refreshes 
% the time spent for preventive refreshes issued by RowHammer mitigations significantly increases 
as \om{3}{\gls{nrh} reduces (i.e., DRAM chips become more vulnerable to read disturbance)}. 
% for all tested RowHammer mitigations. 
Second, RFM's, PRAC's, and PARA's \om{3}{overheads} reach up to 43.05\%, 19.19\%, and 10.95\%, respectively, in the worst-case across our tests.
\agy{3}{Third, \param{RFM} and \param{PARA} exhibit the highest time spent on preventive refreshes for all tested \gls{nrh} values larger than 32 (e.g., 19.30\% and 6.12\% of total execution time, on average across tested workload mixes at an \gls{nrh} of 64, respectively), while they introduce almost zero area overhead~\cite{kim2014flipping, jedec2020ddr5} (not shown in the figure).}
% \agycomment{3}{you dropped Onur's comment here. Be careful. I drafted and alternative. Please see the source code below for your version and see the diff.}
% \srev{Third, \param{RFM} and \param{PARA} introduce almost zero area \sql{R2.1}overhead~\cite{kim2014flipping, jedec2020ddr5} while exhibiting the highest time spent on preventive refreshes for all tested \gls{nrh} values larger than 32, reaching 19.30\% and 6.12\% of total execution time, on average across tested workload mixes at an \gls{nrh} of 64, respectively.}
%
Fourth, among the tested RowHammer mitigations, Graphene and Hydra \om{3}{spend the least} time \om{3}{on} preventive refreshes, \om{3}{e.g., only}
% \agycomment{3}{dont forget to check how it looks on the pdf}
\param{2.68\%} and \param{1.54\%} \om{3}{of total execution time} for an \gls{nrh} value of \param{32} on average across all tested workload mixes. Hydra \om{6}{wastes} the least \agy{3}{fraction of execution time} \om{6}{on} preventive refreshes\om{7}{,} but incurs significant \om{6}{system-level} slowdowns \om{7}{(not shown here; see~\secref{sec:eval_perf})} because it maintains its \om{3}{counter} metadata in DRAM and occupies \om{3}{the} memory channel to retrieve and update \om{3}{its metadata}~\cite{canpolat2024breakhammer, bostanci2024comet, olgun2024abacus}. 
%\param{Graphene} \agy{3}{spends} \om{7}{1.54\% of its execution} \agy{3}{time on performing preventive refreshes \yct{7}{at an \gls{nrh} value of \param{32}}\ieycomment{7}{at which threshold?}. Graphene achieves this}
\param{Graphene} \agy{3}{spends} \iey{6}{\agy{6}{a} higher fraction of execution time on preventive refreshes than Hydra, but incurs the lowest system-level slowdowns across all tested RowHammer mitigations (not shown here; see~\secref{sec:eval_perf}). Graphene achieves the lowest overall performance overhead}
% \om{3}{overall performance} overhead 
at the expense of increasingly large chip area overhead that reaches \iey{0}{\param{$10.38mm^2$} (\param{$4.45\%$} of the chip area of an Intel Xeon processor~\cite{wikichipcascade})
when configured with an \gls{nrh} of 32 for a \om{6}{dual-rank} system with 16 banks \om{6}{per} rank.}
With these \param{four} observations, we \yct{1}{demonstrate} that existing RowHammer mitigations incur significant \yct{3}{preventive refresh overheads} or area overheads as \gls{nrh} decreases. Therefore, we conclude that reducing the overheads of such mechanisms at low cost is critical.
}

% \agycomment{3}{we did not talk about energy in the previous part. It sort of comes out of no where}
% \yctcomment{3}{We do not use "RowHammer-preventive" anywhere, so I removed it.}
\head{Reducing the time \agy{0}{and energy} spent for preventive refreshes}\yctcomment{3}{Fig 3 provides the time cost with the nominal tRAS, here we will provide the cost of preventive refs with reduced tRAS.}
Prior works already show that DRAM timing constraints include large guardbands\om{6}{~\cite{lee2015adaptive, liu2013experimental, chang2016understanding, chang2017understanding, chang2017understandingphd, kim2018solar, yaglikci2022understanding, mathew2017using, lee2017design, chandrasekar2014exploiting, das2018vrldram}}\yctcomment{6}{added DATE2014, SIGMETRICS2017, VRL-DRAM}\yctcomment{7}{fixed refs}. To reduce the performance overhead of RowHammer mitigation \om{1}{mechanisms}, we explore the idea of reducing the time spent on \om{3}{each} preventive refresh.
To this end, we 
experimentally evaluate the effect of reducing \gls{tras}
\agy{3}{on the} i)~reliability of real DDR4 DRAM chips using the methodology described in \agy{3}{\secref{sec:methodology_dramchips}}, and ii)~\agy{3}{time and energy costs} of preventive refreshes.
% \agycomment{3}{was time cost not in Fig3? Onur asked this. Why do you not answer?} 
% \agycomment{3}{Why refer to Sec 4 while you can refer to 4.2?}
% reduce the timing constraint of the charge restoration process (\gls{tras}\om{3}{; see \secref{sec:methodology}}), the most time-consuming operation during a row refresh. We 
% \agycomment{3}{I chopped a redundant sentence here. Please check the source and we can discuss.} \yctcomment{3}{makes sense}
% \agycomment{3}{tRAS was supposed to be defined in the background. Why do we have a gls definition here? Check the pdf.} \yctcomment{fixed it}

% \yctcomment{3}{I will try to improve here further.}
% \agycomment{3}{Onur asked: why estimated? Where is the answer? We shouldn't say that this is estimated maybe or explain to Onur why it is estimated but not measured} 
\figref{fig:m_plot} shows the results of our motivational analysis for two representative DDR4 DRAM modules \om{1}{one from \agy{3}{Mfrs.~H and~S} \om{3}{(see \secref{sec:methodology_dramchips})}}. We reduce the charge restoration latency from right to left on the x-axis. $x=1.0$ marks the nominal \gls{tras} value \om{3}{(i.e., $33ns$)}. The y-axis shows the values of five different curves normalized to their respective values at the nominal \gls{tras} value, three of which are \om{3}{in} the top two subplots\iey{7}{,} and the remaining two are in the bottom two subplots. The enumerated five curves show \yct{3}{the following\om{6}{:}} 
1)~\om{3}{\emph{Prev\om{6}{entive} Refresh Latency}:} the time spent to perform a single preventive refresh \yct{3}{(solid \om{6}{black} line in top subplots)}. \yct{3}{DRAM standards~\dramStandardCitations{} do \emph{not} report how the preventive refresh latency is calculated. \om{6}{We} compute it as the sum of \gls{tras} and $t_{RP}$ timing parameters\om{6}{, since} a preventive refresh is functionally equivalent to opening and closing a DRAM row.}
% 1)~\om{3}{\emph{Prev. Refresh Latency}:} estimated time spent to perform a single preventive refresh, \om{3}{computed \agy{3}{as} the sum of \gls{tras} and $t_{RP}$ timing parameters}, 
2)~\om{3}{\emph{RowHammer Threshold}:} the minimum activation count needed to induce the first RowHammer bitflip, i.e., \gls{nrh}, observed on real DRAM chips~\yct{0}{when reduced charge restoration latency is used} \yct{3}{(dashed \om{6}{red} line in top subplots)}.
3)~\om{3}{\emph{Prev\om{6}{entive} Refresh Count}:} the number of preventive refreshes, \agy{3}{computed as 1/\gls{nrh} based on the insight that a preventive refresh is performed every \gls{nrh} activations \yct{3}{(dotted \om{6}{green} line in top subplots)}.}
4)~\om{3}{\emph{Total Time Cost}:} the time spent for all preventive refreshes as \agy{3}{the} product of \yct{7}{Preventive Refresh Count} and \yct{7}{Preventive Refresh Latency} \yct{3}{(solid purple line in bottom subplots)}.
% \agycomment{3}{There are two different dashed orange lines in the bottom-left subplot}\yctcomment{3}{fixed it}
5)~\om{3}{\emph{Total Energy Cost}:} energy consumption \om{6}{of} all preventive \agy{20}{refreshes} as the product of \yct{7}{Preventive Refresh Count} and the total time spent doing so \yct{3}{(dotted \om{6}{yellow} line in bottom subplots)}.
% \agycomment{3}{Onur asked: why estimated? How do you address it?}\yctcomment{3}{I think how it is computed explains why it is estimated.}
% \agycomment{3}{I did not check everything in this paragraph but found some silently dropped comments. Please double check and make sure we address all comments.}     
% \agycomment{3}{Let's make some of these curves dashed and dotted. Otherwise, it is very hard to match.} \yctcomment{3}{I changed it but tbh the previous figure was more beautiful.}\agycomment{3}{We cannot optimize for beauty only, though :(}\yctcomment{3}{art is dead}\agycomment{3}{science survives}

\begin{figure}[ht]
\centering
\includegraphics[width=0.92\linewidth]{figures/fig4_mot_analysis.pdf}
\caption{\srev{Effect of reducing charge restoration latency on \sql{R4}the time \om{3}{and energy spent on} preventive refreshes}}
\label{fig:m_plot}
\end{figure}

% \agycomment{3}{These curves }
% \agycomment{3}{should this not be reduces?} 
% \agycomment{3}{why it? you put the subject before i) already}
From \figref{fig:m_plot}, we make \param{four} observations. 
\yct{3}{First, \om{7}{Preventive Refresh Latency} reduces proportionally as charge restoration latency decreases.}
\yct{3}{Second, \om{6}{reducing the} charge restoration latency i)~\om{7}{reduces RowHammer Threshold} (\gls{nrh}) by less than \om{6}{only} $5\%$ \yct{6}{when} \gls{tras} is reduced by $64\%$ and $36\%$} for \om{1}{the \om{6}{two} \agy{3}{modules} from \agy{3}{Mfrs.~H and~S}}, respectively (marked with dashed blue lines), and ii)~\om{6}{reduces} \gls{nrh} \om{6}{much} more significantly if \gls{tras} is reduced \om{7}{more} than these values. 
% \agycomment{3}{is this while phrase really necessary?}
\yct{3}{Third, \om{6}{Total Time Cost} \om{6}{of preventive refreshes}
% time spent for all preventive refreshes (i.e., total time cost)
\agy{3}{i)~reduces} with reducing \gls{tras} until an inflection point (marked with a dark purple circle \om{6}{in \figref{fig:m_plot} (bottom)} at 36\% and 45\% of the nominal \gls{tras} value for modules from \agy{3}{Mfrs.~H and~S}, respectively)\om{6}{,} \agy{3}{and}
% ii)~reducing \gls{tras} beyond these inflection points the total time costs.
\agy{3}{ii)~increases beyond the respective inflection points.} 
% The reasoning behind this is that 
\agy{3}{This is because} the \om{7}{higher Preventive Refresh Count} \om{6}{at lower \gls{tras}} overwhelms the reduction in \om{7}{Preventive Refresh Latency}. Hence, the lowest \om{7}{Total Time Costs} are observed at these inflection points with a \om{7}{Total Time Cost} reduction of $43\%$ and $28\%$ for the \om{6}{two} \agy{3}{modules} from \agy{3}{Mfrs.~H and~S}, respectively.} 
\yct{3}{Fourth, \om{6}{Total Energy Cost \om{6}{of preventive refreshes} also has \om{6}{similar} inflection points}
% the energy consumption of preventive refreshes (i.e., total energy cost) 
% follows a similar curve to the curve of their total time costs, where the inflection points 
(marked with a dark \om{6}{yellow} circle \om{6}{in \figref{fig:m_plot} (bottom)})
% appear
at $36\%$ and $64\%$ of the nominal \gls{tras} value where the total energy cost can be reduced by $40\%$ and $19\%$ for the \agy{3}{modules} from \agy{3}{Mfrs.~H and~S}, respectively.}\yctcomment{7}{yes they are consistent with the figure}
Based on this motivational analysis, we conclude that reducing \agy{3}{charge} restoration latency is a promising approach to reduce the performance and energy overheads of RowHammer mitigations. 
% {However, to reliably reduce restoration latency, it is crucial to understand the effects of restoration latency on RowHammer vulnerability.}


\head{{Characterizing the Impact of {Charge} Restoration Latency}}
{To reliably reduce the {charge} restoration latency of preventive refreshes, it is critical to understand the limits of a DRAM chip. % and operate the DRAM chip accordingly.
\yct{3}{Therefore, it is \om{6}{important} to investigate i)~how RowHammer \om{6}{T}hreshold and the number of RowHammer bitflips
% \yctcomment{3}{we did not explain ber until here, should we explain it here? Actually the explanation is just Bit-Error-Rate} 
change with reduced charge restoration latency, ii)~whether \agy{3}{using reduced charge restoration latency for many consecutive \yct{3}{preventive} refreshes further worsens RowHammer threshold or number of RowHammer bitflips},
% it is safe to use the reduced charge restoration latency multiple times as multiple partial restorations might lead bitflips
and iii)~how the data retention time of a DRAM row changes with the reduced charge restoration latency.}
Although many prior works~\understandingRowHammerAllCitations{} study \om{7}{various aspects of} the \om{3}{read disturbance} characteristics of DRAM (e.g., \om{3}{access and data patterns}, voltage, temperature), \emph{no} prior work characterizes the effect of \agy{3}{charge} restoration latency on RowHammer.
}


\yct{7}{In this paper, we conduct a rigorous experimental characterization study to understand how reducing charge restoration latency affects RowHammer vulnerability in real DDR4 DRAM chips, along with its effects on system performance and energy efficiency.}

\section{Base-Refine (\Sys{})}
\label{bare}

\begin{figure*}[ht!]
\vskip 0.2in
\begin{center}
\centerline{\includegraphics[width=.9\textwidth]{assets/sys2.pdf}}
\caption{Instruct-tuned models provide high-quality but low-diversity data, while base models provide low-quality but high-diversity data. \Sys{} independently generates a diverse initial set of data points with a base model and refines each entry individually with an instruct-tuned model to create a high-quality, high-diversity dataset. In this example of a real grade school math problem generation, the Llama-3.1-70B-Base model hallucinates in its answer to its own question. The refiner (Llama-3.1-70B-Instruct) recognizes this and disambiguates the question and corrects the reasoning.}
\label{system-diagram}
\end{center}
\vskip -0.2in
\end{figure*}


We leverage our insights to propose \textbf{\Sys{}}, a practical synthetic data generation method combining the diversity of base models with the quality of instruct models. \Sys{} uses a base model to generate an initial set of diverse but potentially lower quality data, after which an instruct-tuned model individually refines each example from the initial set -- as shown in the example in \cref{system-diagram}.

In the base step, the base model uses a prompt with minimal few-shot examples to generate. In the refine step, the instruct-tuned model is instructed to retain the concept of the base model generation while improving it according to specific criteria (e.g., realism, correctness). This retains the overall diversity of the base model generated set while exerting greater control over the quality of the final generation.

Importantly, few-shot examples are only required in the base step to ensure formatting (though they can also be included in the refine step), and only in small amounts. Thus, \textbf{\Sys{}} is especially useful in domains with very limited real-world data as little human effort is required to generate a diverse high-quality synthetic dataset. In our experiments, we limit ourselves to just three few-shot examples. In addition, we intentionally use very general prompts for \Sys{} to demonstrate its flexibility, underscoring the potential for even greater improvement with tailored prompts. Representative prompts are included in \cref{appendix_prompts}.
\section{Results}
We first describe communication patterns within the full chronological context of the game in \textit{League of Legends (LoL)}, separated into four sections based on changing coordination dynamics. Based on this context, we identify core factors players assess to decide when to participate in communication with other teammates. Afterward, we discuss how communication shapes player perceptions toward their teammates, showing player's wariness towards players actively engaging in communication. 

\subsection{Communication Patterns in Context}

We discuss the communication patterns among teammates within the game. We organize the data into chronological phases of the game for a structured analysis of how the context shapes communication patterns. 

\subsubsection{Pre-game stage}
Before gameplay begins, team communication opens with \textit{team drafting}, where players are assigned roles (Top, Mid, Bot, Support, or Jungle) and take turns picking or banning champions. In Solo Ranked mode, roles are pre-assigned based on player preferences selected before queueing. Once teams are set, all players enter \textit{champion select} stage, alternating champion picks and banning up to five champions per team. During this stage, communication is limited to text chat. The usernames are anonymized (i.e., replacing the name with aliases) to prevent queue dodging by checking third-party stats sites such as OP.GG\footnote{https://www.op.gg/}, leaving the chat as the only option to inform individual strengths and preferences. 

Team composition in \textit{LoL} is crucial to the strategy and outcome of the game~\cite{ong2015player}, setting the basis for future interactions. Most participants acknowledged the importance of balanced and synergistic team composition, especially as players move into higher ranks where team coordination outweighs individual excellence. Yet, we observed a distinct lack of verbal communication between the members during this period across all ranks. Participants attributed the lack of willingness to initiate a conversation on the dangers of starting the game on a bad footing. They prioritized ``not creating friction'' during this stage as negative impressions can propagate throughout the game. Some participants attempted communication to reduce such friction, such as P14, who stated,``\textit{If I had the time, I wanted to say that I will be banning [this Champion], just in case a player on my team wanted to play them.}'' However, several participants viewed any communication during the pre-game phase with wariness, as dissatisfaction or conflict at this step portended negative interactions between players in the game (P3, P9, P15). Thus, even when participants expressed doubt about other teammates' unconventional or non-meta champion picks, they refrained from entering into discourse. This contrasts with findings by Kou and Gui~\cite{kou2014}, which showed players attempt to maintain a harmonious and constructive atmosphere through greetings and introductions.

Another emergent code of the reason for not engaging in communication in the pre-game stage stems from different purposes of playing the game (P1, P5, P13, P16, P17). Despite being in ranked mode, which is more prone to increased competitiveness and effort, participants showed differing goals and levels of interest in winning the game. Several players stated that they had previously exerted great mental load in coordinating synergistic plays, but stopped as they gave less importance to winning at all costs (``\textit{I don't really play to win. I play \textit{LoL} to relieve stress, so I don't engage in chat.}'', P5). These players saw verbal communication with the goal of coordination as an unnecessary or even cumbersome component of the pre-game stage.


\subsubsection{Structured phase}
In many MOBAs, including \textit{LoL}, the early stages of the game play out in a formulaic manner: players join their lanes (Top, Mid, and Bot/Support), defeat minions to gain gold, buy items towards certain ``builds'', kill or assist in early objectives (Jungle), and battle counterparts in their respective lanes. Participants at this stage expressed that most players possessed tacit knowledge of what must be done, such as knowing when to aid their Jungle to capture a jungle monster, choosing the opportune moments to leave their lanes, or positioning wards (i.e., a deployable unit which provides a vision of the surrounding area) at the ideal placements. The participants assumed each player knew their ``role'' to fulfill, often comparing it to ``doing their share'' (P1, P3, P7, P19). In line with this belief, players rarely initiated preemptive or proactive verbal communication for strategic or social purposes at the early stage. 

Pings, on the other hand, constantly permeated the game. At this stage, players used ping to provide information relevant to others from their position, such as letting others know if an enemy went missing from their lane. As the players are largely separated and independent from one another, pings (coupled with the minimap and scoreboard) served as the primary channel for maintaining context over the game rather than as warnings or direct guidance to the players. For other non-verbal gestures, while objective votes would occasionally appear, they were rarely answered. Instead, relevant players near the objective would place pings or move toward it to help out their teammates.

Participants viewed the structured phase as a routine, but uncertain period of the game where the pendulum could swing in either team's favor. Players --- especially Jungles who roam the board looking for opportunities to ambush the enemy team in lanes (``gank'') --- sometimes felt hesitant to make calls and demands at this stage since ``\textit{[they] could make a call, but if I fail, they'll start blaming my decisions down the line.}'' (P7) But at this stage, participants believed that they held personal agency over the final game outcome. P1 and P6 stated that they entered the game with the mindset that only they had to succeed regardless of the performance of their teammates. This belief was reflected in their chatting behavior, where players prioritized focusing on their circumstances over the team's (``\textit{I mute the chat so that I don't get swayed by the team, as I can win the game if I do well.}'', P9).


\subsubsection{Group engagement phase}
As the game enters its middle phase, it provides opportunities for more diverse decision-making. Players may swap lanes, seize or trade crucial objectives, and fight in large battles involving multiple champions. At this point, teams typically have a clear outlook on which players and team have the advantage, requiring more team-driven decisions to maintain or overcome their current standing. Thus, players used verbal communication to discuss more complicated tactics that could not be effectively conveyed through pings.

But more often than not, chat messages became judgment-based. As enemy engagement with larger groups occurred more frequently, the availability for chatting would come after death, which led to comments on past actions rather than future choices. Additionally, the respawn timer for deaths becomes longer as the game progresses, providing more time to observe other players than in earlier phases. This gave players more opportunities to express dissatisfaction specifically towards certain plays, such as placing Enemy Missing pings on the map where other teammates are located to bring attention to their questionable play.

This stage also gave much more exposure of each other to the allies as the team would gather at a single point, giving way to greater scrutiny by their teammates. Repeated or critical mistakes put participants on edge, as they braced for criticism from their teammates. They expressed relief or surprise when the chat remained silent or civil, with P8 stating ``\textit{I messed up there. No one is saying anything, thankfully.}''


\subsubsection{Point of no return}
Meanwhile, verbal communication flowed out when the game had a clear trajectory to the end. Previous research has shown that both toxic and non-toxic communication skyrockets near the end of the game~\cite{kwak2015linguistic} when the players have determined the game outcome with certainty. We saw that this phase opened up both positive and negative sides of communication for guaranteed win and loss, respectively. The winning team would compliment and cheer each other through chat messages and emotes, while the losing side devolved into arguments and calling out. The communication at this stage was driven by emotion, showing excitement or venting frustration.


\subsection{Communication Assessment Process}

We describe the factors that users mainly focus on to assess when or when not to involve themselves in communication with their teammates. 

\subsubsection{Calculating communication cost}
One of the most proximate factors behind when communication is performed is the limited action economy of the game. In \textit{LoL} and other MOBAs, players can rarely afford time to type out messages due to the fast-paced nature of the game. In time-sensitive scenarios, the time pressure makes communication particularly costly. It is therefore unsurprising that much of the communication occurs after major events (e.g., battles and objective hunting), as players are given more downtime while waiting for teammates or enemies to respawn or regroup.

For periods where players were still actively involved in gameplay, the players made conscious decisions on choosing which communication media to use based on the perceived action availability and the importance of communicating the message. Players relied on pings for non-critical indications, believing that the mutual understanding of the game would get the message across. However, many players recognized that pings were prone to be missed, ignored, or misinterpreted by their allies (P2, P9, P16, P17, P20). Subsequently, participants typed out information considered to be too important to the situation to be misunderstood or missed by other players even if it caused delays in their gameplay (P10, P11, P14). Simultaneously, the priority of importance constantly shifted --- we observed multiple times participants start to type, but stop to react to an ongoing play, only to never send out their message.

\subsubsection{Evaluating relevance and responsiveness}
When the brief window of communication opportunity is missed, players are unlikely to ever send out that information. In \textit{LoL}, situations can change within seconds and certain communication media cannot keep up with the changing state of the game. For example, almost all study participants did not participate in votes for objectives. Among the tens of objective votes initiated among all the games in this study, no objective vote saw more than three votes, frequently being left with no vote beyond the player who initiated the vote. Some players, when asked why they did not participate, stated that the votes they made often became irrelevant as the game state had changed during the time it took to vote (P2, P11). Other players also discussed how information conveyed through communication can get outdated fast (P1, P8, P9). 

\begin{quote}
I can't always follow through with what I say [in the chat] since the game is really dynamic. My teammates don't understand such situations, so I tend to not chat proactively. - P9
\end{quote}

Thus, some players instead preferred to react through direct action (P8, P10, P11, P16, P20). P10 stated, ``\textit{I think it's enough to show through action rather than [using objective voting]. I can look out for how the player reacts when I request something from them.}''

On the other hand, such action-based responses left the player to assess whether and how the communication was received. P10 stated that they tried to predict whether a player understood their ping direction by how they moved, but it was hard to interpret their intent: ``\textit{members sometimes seem to move towards me but then turn around, and sometimes they even ping back but don't come.}''. P16 discussed how they weren't sure whether the ping was received, but performed it anyway since it felt helpful.

Similarly, participation in surrender votes (or lack thereof) carried different intent by the player. During most of the games that ended in a loss, one or more surrender votes were called by the participant's team. However, only two surrender votes achieved four or more players' participation. However, the reasons why a player chose to not participate varied. Some had decided to wait and see how other teammates voted, which may have paradoxically led many members to not participate in the vote (P4, P9). Meanwhile, others didn't reply as they didn't think the vote was actually calling for a response: P13 stated, ``\textit{I didn't vote because they were just showing their anger. It's just a member venting through a surrender vote that they're not doing well.}''

\subsubsection{Balancing information access and psychological safety}
While recognizing that communication would be useful or even necessary in certain situations, participants also put their psychological safety first over information access. Some players, worn down by the normalization of toxic communication such as flaming, muted the chat (P1, P9).

Many participants expressed the sentiment of ``protecting [their] mentality'', describing how certain communication harmed their psychological well-being. This communication did not always refer to negative communication; P9 often muted players who gave commands as they did not want to be ``swept up'' by others' play-related judgments. This separation even extended to other more widely considered essential communication forms, such as pings. Even after acknowledging that pings were vital and useful to the game, P9 went as far as muting the ping of the support player in the same lane after they sent a barrage of Enemy Missing pings that signified aggression and criticism. 

Additionally, the abundance and high frequency of communication also strained the limited mental capacity of the players. Many players, when asked why they had not replied to an objective vote or other chat messages, stated that they simply did not notice them among other events happening (P1, P2, P3, P9, P12, P15, P18, P20, P21). The information overload caused stress and became distracting to players.

\subsubsection{Reducing potential friction}
As demonstrated in the pre-game stage of the game, players sometimes used communication to minimize friction between their teammates. Some participants sacrificed time to apologize to other players when they believed themselves to be at fault. When asked why, P12 replied, ``\textit{There are too many people who don't come to help gank if I don't apologize.}''. Similarly, P5 sacrificed time typing in an apology after a teammate had died despite still being in the middle of a fight as they didn't wish to give the other player a reason to start an attack.

However, some noted that silence is sometimes the best answer to a negative situation. P4, after dying to the enemy, put into chat ``Fighting!'' (roughly meaning, ``We can do it!''). They stated ``\textit{I don't know why I do it... it probably angers [my teammates] more.
}'' They also stated that ``\textit{for certain people, talking in the chat only spurs them more. You just have to let them be.}'' Other players shared similar sentiments that being quiet and dedicating focus to the game was a better choice (P1, P11, P14).

For female players, the fear of gender-based harassment shaped their communication patterns. While \textit{LoL} does not provide any demographic information of a player to other players, almost all female participants noted experiences of receiving derogatory remarks or doubts about their abilities based on other players' assumptions of their gender, a trend frequently seen in male-dominated online gaming cultures~\cite{fox2016women, norris2004gender, mclean2019female}. They noted that the players were able to correctly guess their gender when the participant's role and champion fit into the preconceived notions of what women ``tended to play'' (i.e., female-identifying support champions, such as Lulu) or their username ``seemed feminine'' (P18, P19, P20, P21, P22). This led to certain players adopting tactics that signaled male-like behavior, such as changing their speech style to be more gender-neutral or male-like (P19, P21) and changing their username to sound more gender-neutral. Cote describes similar instances of ``camouflaging gender'' as one of five main strategies for women coping with harassment~\cite{cote2017coping}. However, some players opted to keep playing their preferred character or maintaining their username even if it signaled their gender, such as P21 who expressed, ``\textit{I cherish and feel attached to my username, so I don’t want to change it just because of [harassment and inappropriate comments].}'' These players valued self-expression and identity even at the risk of increased risk to unpleasant communication experiences.


\subsubsection{Forming performance-based hierachy}
Naturally formed leadership has often been observed in other works on \textit{LoL} teams~\cite{kou2014}. Kim et al. showed that more hierarchy in in-game decision-making led to higher collective intelligence~\cite{kim2017}. While they used ``hierarchy'' to mean varying amounts of communication throughout the game, we observed that the hierarchy extends further to performance-based hierarchy, where teammates in more advantageous positions are given greater weight when communicating with other players. Players actively chose to refrain from suggesting strategic plans when they were ``holding down the team'', recognizing that they held less power and trust among the team members (P8, P10, P12, P14, P22). The player who was losing against the enemy team was viewed as having no ``right'' to lead the team, which was reserved for well-performing players.


\subsubsection{Enforcing norms and habits}
One of the most common answers to why players performed certain communication actions, especially non-verbal actions such as pings and emotes, was ``a force of habit'' (P6, P7, P8, P9, P10, P12, P17). Players formed learned practices of using communication channels at certain points by observing other players exhibit the same behaviors. This promoted, for example, replying to an emote sent by the teammate with their own or pinging readied skills and items to emphasize relevant information for other players throughout the game. 

On the other hand, this meant that players were averse to communication patterns outside of the norm --- participants stated that they had a hard time adapting to new forms of communication, seeing no immediate benefit or impact from using them (P1, P8, P14, P13, P15, P17). Most egregiously, the recently introduced objective pings were largely viewed to be awkward to use and unnecessary (P1, P4, P8, P12).


\subsection{Impact of Communication Assessment}
We describe how the communication patterns and assessment of the players impact the individual players' perspectives on team dynamics.

\subsubsection{Relationship between trust and communication frequency}
Most participants saw value in constant and well-informed communication but with an important distinction: verbal communication with strangers rarely ended well. Players largely recognized frequent verbal communication to burgeon conflict, regardless of the message within. Even when players understood the helpful intent behind positive messages from the players, they compared actively talking players to be possible bad actors who were likely to exhibit toxic behaviors when the game turned against them. (P1, P4, P8, P12, P14)

\begin{quote}
I need to make sure to not disturb Twisted Fate. I saw him start to flame. It's not because I don't want to hear more criticism. I know these types. The more I react and chat with them, the more deviant they will become. - P4  
\end{quote}

Similarly, P19 lamented that players used to socialize more in the chat during the pre-game phase to build a fun and prosocial environment, noting a memorable example of encouraging each other to do well on their academic exams, but noted that such prosocial behavior has become much rarer during the recent seasons. They noted that there are inevitably players ``who take it negatively'' and thus stopped proactively typing non-game related messages in the chat.

Ultimately, players desired assurance and trust of player commitment. The participants trusted actions more than words to prove that the player remained dedicated to the game. Both P10 and P17 pointed out that it was easy to tell who was still ``in the game'' and motivated to try their best and that ``staying on the keyboard'' likely meant that they weren't invested or focused on the game. Players viewed such commitment to be the most important aspect of a ``good'' teammate, sometimes even more than their skill or performance (P9, P14). It is interesting to note that unlike what previous literature may suggest~\cite{marlow2018}, players' averseness to talkative teammates had less to do with the cognitive overload or distraction caused by the frequent communication, but rather due to the threats of future team breakdown. This view in turn also affected how players decided to communicate or not, as they believed that players would not take their suggestions or comments in a positive light. 


\subsubsection{Perception of player commitment and fortitude}

Communication also acted as a mirror of their teammates' mental fortitude. A number of players mentioned how they valued a resilient mindset in their teammates playing the game, referring to players who remained committed to the game until the very end. They saw players who provoked or complained to teammates as ``having a weak mentality'' who had been altered by the bad outcomes of the game to act in an unhelpful manner towards the team through their communication. The communication actions of the teammate informed the participants of how steadfast their teammate remained in disadvantageous situations.  

\begin{quote}
It's not like I constantly reply in the chat or anything, but I pay attention [to the chat] to grasp the overall atmosphere of the team. If the team doesn't collaborate well then we lose, so I try to have a rough understanding of the mentality of the other players. - P13
\end{quote}

There were also instances of communication that helped players maintain a positive view of their teammates. For example, P11 mentioned near the beginning of the game, ``\textit{Looking at the chat, Varus player has strong mentality [for being so positive]. There were lots of points [in his support's] plays that he could have criticized.}'' Unfortunately, this view quickly soured when the Varus player devolved into criticism later in the late game phase where the Varus player started criticizing the support and other players. P11 then noted that the Varus player seemed to merely be ``bearing through the game''.

\section{Conclusion}
\label{sec:Conclusion}
In this paper, we proposed a complete real-time planning and control approach for continuous, reliable, and fast online generation of dynamically feasible Bernstein trajectories and control for FW aircrafts. The generated trajectories span kilometers, navigating through multiple waypoints. By leveraging differential flatness equations for coordinated flight, we ensure precise trajectory tracking. Our approach guarantees smooth transitions from simulation to real-world applications, enabling timely field deployment. The system also features a user-friendly mission planning interface. Continuous replanning  maintains the rajectory curvature 
$\kappa$ within limits, preventing abrupt roll changes.

Future works will include the ability to add  a higher-level kinodynamic path planner to optimize waypoint spatial allocation and improve replanning success, and enhancing the trajectory-tracking algorithm by refining the aerodynamic coefficient estimation. 

\section*{Acknowledgements}

We would like to thank Justin Wong, Peter West, Raj Ammanabrolu, Alex Dimakis, Alex Xu, and Andrew Qin for their many insightful discussions about this work.

Sky Computing Lab is supported by gifts from Accenture, AMD, Anyscale, Cisco, Google, IBM, Intel, Intesa Sanpaolo, Lambda, Microsoft, NVIDIA, Samsung SDS, SAP, and VMware.

This material is based upon work supported by the National Science Foundation Graduate Research Fellowship Program under Grant No. DGE 2146752. Any opinions, findings, and conclusions or recommendations expressed in this material are those of the author(s) and do not necessarily reflect the views of the National Science Foundation.

\bibliography{main}
\bibliographystyle{plainnat}

\newpage
\appendix
\onecolumn

\section{Additional Results}
\label{appendix_experiments}

\subsection{Downstream Evaluation - Additional Details}

\subsubsection{Fine-Tuning Task Hyperparameters}
\label{appendix_hyperparameters}

We list below the fine-tuning hyperparameters that were used in common for HotpotQA, PubMedQA, GSM8K, and LCB TOP. Learning rate was determined independently for each domain via learning rate sweeps (across orders of magnitude); each sweep gave the same optimal learning rate.

\begin{itemize}
    \setlength\itemsep{-0.5em}
    \item Learning Rate: 0.001
    \item LoRA $\alpha$: 16
    \item LoRA Rank: 8
    \item LoRA Dropout: 0.0
\end{itemize}

\subsubsection{Classification Task Setup}
\label{appendix_class_train}

The generated data is used to train a BERT-based classifier \cite{bert} for 2 epochs on Enron and 9 epochs on Newsgroups. The trained models are evaluated on a static test set with $n=500$ examples for each domain.

\subsection{Core Experiment Results - All Domains}
\label{appendix_core}

This appendix contains diversity, IR, and downstream performance results for all core experiments: generation with Llama 3.1 8B and 70B Base and Instruct models, \Sys{} with Llama 3.1 models of both families, and \Sys{} with the use of GPT-4o.

Note that HotpotQA RAFT and PubMedQA RAFT diversity results present here were not presented in \cref{diversity-table} as we believe the numbers are noisy and not fit for drawing conclusions, due to the use of $100$ different simulated retrieval contexts that generation was conditioned on (as required by RAFT). Not only does this introduce noise to the similarity calculation, but the strong instruction following capability of instruct models allow them to better leverage the inherent diversity in different prompts. However, for completeness, we report the values in the tables in this appendix.

\begin{table}[H]
\caption{Average pairwise embedding cosine similarity, IR, and downstream F1 results on a randomly selected static $n=500$ subset of Enron. A BERT model with a classification head was trained for $2$ epochs on the generated data $(n = 500)$. Only pairwise similarities for generations within the same class (spam or legitimate) were calculated.}
\label{enron-core}
\vskip 0.1in
\begin{center}
\begin{small}
\begin{sc}
\begin{tabular}{c|ccc}
\toprule
Generation Method & Average Embedding Similarity & IR & Downstream F1 \\
\midrule
Llama 3.1 8B Instruct              & 0.500 & 86.0\% & 0.753 \\
Llama 3.1 70B Instruct             & 0.450 & 85.0\% & 0.848 \\
Llama 3.1 8B Base                  & 0.368 & 63.5\% & 0.790 \\
Llama 3.1 70B Base                 & 0.350 & 74.5\% & 0.819 \\
\Sys{} Llama 3.1 8B                & 0.413 & 85.0\% & \textbf{0.872} \\
\Sys{} Llama 3.1 70B               & 0.406 & 82.0\% & 0.771 \\
\Sys{} GPT-4o + Llama 3.1 8B Base  & 0.379 & 84.5\% & \textbf{0.872} \\
\Sys{} GPT-4o + Llama 3.1 70B Base & 0.356 & 88.5\% & 0.846 \\
\bottomrule
\end{tabular}
\end{sc}
\end{small}
\end{center}
\vskip -0.1in
\end{table}

\begin{table}[H]
\caption{Average pairwise embedding cosine similarity, IR, and downstream accuracy results on on a randomly selected static $n=500$ subset of Newsgroups. A BERT model with a classification head was trained for $9$ epochs on the generated data $(n = 500)$.}
\label{newsgroups-core}
\vskip 0.1in
\begin{center}
\begin{small}
\begin{sc}
\begin{tabular}{c|ccc}
\toprule
Generation Method & Average Embedding Similarity & IR & Downstream Accuracy \\
\midrule
Llama 3.1 8B Instruct              & 0.271 & 85\% & 26\% \\
Llama 3.1 70B Instruct             & 0.246 & 82\% & 30\% \\
Llama 3.1 8B Base                  & 0.155 & 58\% & 41\% \\
Llama 3.1 70B Base                 & 0.162 & 78\% & 29\% \\
\Sys{} Llama 3.1 8B                & 0.162 & 91\% & 40\% \\
\Sys{} Llama 3.1 70B               & 0.134 & 93\% & \textbf{49\%} \\
\Sys{} GPT-4o + Llama 3.1 8B Base  & 0.131 & 81\% & 44\% \\
\Sys{} GPT-4o + Llama 3.1 70B Base & 0.285 & 87\% & 47\% \\
\bottomrule
\end{tabular}
\end{sc}
\end{small}
\end{center}
\vskip -0.1in
\end{table}

\begin{table}[H]
\caption{Average pairwise embedding cosine similarity, IR, and downstream accuracy results on a randomly selected static $n=100$ subset of HotpotQA RAFT. A Llama-3.1-8B-Instruct model was fine-tuned for 4 epochs on the generated data ($n=1000$). The baseline performance of the Llama-3.1-8B-Instruct model on the evaluation set prior to any fine-tuning is reported in the first row.}
\label{hotpot-core}
\vskip 0.1in
\begin{center}
\begin{small}
\begin{sc}
\begin{tabular}{c|ccc}
\toprule
Generation Method & Average Embedding Similarity & IR & Downstream Accuracy \\
\midrule
Baseline Performance               &   --  &  --  & 33\% \\
\midrule
Llama 3.1 8B Instruct              & 0.214 & 76\% & 49\% \\
Llama 3.1 70B Instruct             & 0.216 & 90\% & 55\% \\
Llama 3.1 8B Base                  & 0.221 & 62\% & 50\% \\
Llama 3.1 70B Base                 & 0.209 & 77\% & 53\% \\
\Sys{} Llama 3.1 8B                & 0.217 & 77\% & \textbf{58\%} \\
\Sys{} Llama 3.1 70B               & 0.210 & 88\% & 54\% \\
\Sys{} GPT-4o + Llama 3.1 8B Base  & 0.214 & 78\% & 57\% \\
\Sys{} GPT-4o + Llama 3.1 70B Base & 0.205 & 89\% & 56\% \\
\bottomrule
\end{tabular}
\end{sc}
\end{small}
\end{center}
\vskip -0.1in
\end{table}

\begin{table}[H]
\caption{Average pairwise embedding cosine similarity, IR, and downstream accuracy results on a randomly selected static $n=100$ subset of PubMedQA RAFT. A Llama-3.1-8B-Instruct model was fine-tuned for 4 epochs on the generated data ($n=1000$). The baseline performance of the Llama-3.1-8B-Instruct model on the evaluation set prior to any fine-tuning is reported in the first row.}
\label{pubmed-core}
\vskip 0.1in
\begin{center}
\begin{small}
\begin{sc}
\begin{tabular}{c|ccc}
\toprule
Generation Method & Average Embedding Similarity & IR & Downstream Accuracy \\
\midrule
Baseline Performance               &   --  &  --  & 53\% \\
\midrule
Llama 3.1 8B Instruct              & 0.376 & 63\% & 72\% \\
Llama 3.1 70B Instruct             & 0.373 & 73\% & 73\% \\
Llama 3.1 8B Base                  & 0.396 & 39\% & 68\% \\
Llama 3.1 70B Base                 & 0.603 & 31\% & 73\% \\
\Sys{} Llama 3.1 8B                & 0.377 & 47\% & 73\% \\
\Sys{} Llama 3.1 70B               & 0.367 & 72\% & \textbf{79\%} \\
\Sys{} GPT-4o + Llama 3.1 8B Base  & 0.385 & 57\% & 72\% \\
\Sys{} GPT-4o + Llama 3.1 70B Base & 0.474 & 40\% & 63\% \\
\bottomrule
\end{tabular}
\end{sc}
\end{small}
\end{center}
\vskip -0.1in
\end{table}

\begin{table}[H]
\caption{Average pairwise embedding cosine similarity, IR, and downstream accuracy results on a $n=500$ randomly selected static subset of GSM8K. A Llama-3.2-1B-Instruct model was fine-tuned for 4 epochs on the generated data ($n=1000$) instead of the 3.1 8B model due to its high no-training performance. The baseline performance of the Llama-3.2-1B-Instruct model on the evaluation set prior to any fine-tuning is reported in the first row. Llama-3.1-8B-Instruct generation results are not reported due to data generation derailing.}
\label{gsm8k-core}
\vskip 0.1in
\begin{center}
\begin{small}
\begin{sc} 
\begin{tabular}{c|ccc}
\toprule
Generation Method & Average Embedding Similarity & IR & Downstream Accuracy \\
\midrule
Baseline Performance               &   --  &  --  & 21.8\% \\
\midrule
Llama 3.1 8B Instruct              & N/A & N/A & N/A \\
Llama 3.1 70B Instruct             & 0.421 & 51\% & 17.8\% \\
Llama 3.1 8B Base                  & 0.310 & 23\% & 19.2\% \\
Llama 3.1 70B Base                 & 0.313 & 44\% & 22.4\% \\
\Sys{} Llama 3.1 8B                & 0.310 & 27\% & 24.8\% \\
\Sys{} Llama 3.1 70B               & 0.305 & 54\% & 29.8\% \\
\Sys{} GPT-4o + Llama 3.1 8B Base  & 0.295 & 60\% & 32.8\% \\
\Sys{} GPT-4o + Llama 3.1 70B Base & 0.302 & 64\% & \textbf{35.8\%} \\
\bottomrule
\end{tabular}
\end{sc}
\end{small}
\end{center}
\vskip -0.1in
\end{table}

\begin{table}[H]
\caption{Average pairwise embedding cosine similarity, IR, and downstream accuracy results on the full LCB TOP set. A Llama-3.1-8B-Instruct model was fine-tuned for 4 epochs on the generated data ($n=1000$). The baseline performance of the Llama-3.1-8B-Instruct model on the evaluation set prior to any fine-tuning is reported in the first row.}
\label{lcb-core}
\vskip 0.1in
\begin{center}
\begin{small}
\begin{sc}
\begin{tabular}{c|ccc}
\toprule
Generation Method & Average Embedding Similarity & IR & Downstream Accuracy \\
\midrule
Baseline Performance               &   --  &  --  & 18.6\% \\
\midrule
Llama 3.1 8B Instruct              & 0.416 & 51\% & 20.6\% \\
Llama 3.1 70B Instruct             & 0.389 & 49\% & 26.0\% \\
Llama 3.1 8B Base                  & 0.468 & 36\% & 24.9\% \\
Llama 3.1 70B Base                 & 0.477 & 57\% & 24.4\% \\
\Sys{} Llama 3.1 8B                & 0.462 & 47\% & \textbf{28.1}\% \\
\Sys{} Llama 3.1 70B               & 0.481 & 64\% & 25.6\% \\
\Sys{} GPT-4o + Llama 3.1 8B Base  & 0.459 & 72\% & 26.7\% \\
\Sys{} GPT-4o + Llama 3.1 70B Base & 0.471 & 68\% & 27.4\% \\
\bottomrule
\end{tabular}
\end{sc}
\end{small}
\end{center}
\vskip -0.1in
\end{table}


\subsection{Independent Sampling Temperature Ablations - HotpotQA, PubMedQA, and LCB TOP}
\label{appendix_temp}

This appendix contains diversity, IR, and downstream performance results for our temperature ablation experiments. We perform a temperature sweep for Llama-3.1-8B-Instruct generation with $t=0.5, 0.7, 1.0$. We find that while adjusting the temperature can improve downstream performance, in general the gains are small relative to gains by using \Sys{}.

\begin{table}[H]
\caption{Temperature ablations with independent sampling from Llama-3.1-8B-Instruct. Average pairwise embedding cosine similarity, IR, and downstream accuracy results on a randomly selected static $n=100$ subset of HotpotQA RAFT. A Llama-3.1-8B-Instruct model was fine-tuned for 4 epochs on the generated data ($n=1000$). The baseline performance of the Llama-3.1-8B-Instruct model on the evaluation set prior to any fine-tuning is reported in the first row.}
\label{hotpot-temp}
\vskip 0.1in
\begin{center}
\begin{small}
\begin{sc}
\begin{tabular}{c|ccc}
\toprule
Generation Method & Average Embedding Similarity & IR & Downstream Accuracy \\
\midrule
Baseline Performance              &   --  &  --  & 33\% \\
\midrule
Llama 3.1 8B Instruct ($t=1.0$)   & 0.214 & 76\% & 49\% \\
Llama 3.1 8B Instruct ($t=0.7$)   & 0.216 & 77\% & 50\% \\
Llama 3.1 8B Instruct ($t=0.5$)   & 0.220 & 83\% & 42\% \\
Llama 3.1 8B Base ($t=1.0$)       & 0.221 & 62\% & 50\% \\
\Sys{} Llama 3.1 8B (All $t=0.7$) & 0.217 & 77\% & \textbf{58\%} \\
\bottomrule
\end{tabular}
\end{sc}
\end{small}
\end{center}
\vskip -0.1in
\end{table}

\begin{table}[H]
\caption{Temperature ablations with independent sampling from Llama-3.1-8B-Instruct. Average pairwise embedding cosine similarity, IR, and downstream accuracy results on a randomly selected static $n=100$ subset of PubMedQA RAFT. A Llama-3.1-8B-Instruct model was fine-tuned for 4 epochs on the generated data ($n=1000$). The baseline performance of the Llama-3.1-8B-Instruct model on the evaluation set prior to any fine-tuning is reported in the first row.}
\label{pubmed-temp}
\vskip 0.1in
\begin{center}
\begin{small}
\begin{sc}
\begin{tabular}{c|ccc}
\toprule
Generation Method & Average Embedding Similarity & IR & Downstream Accuracy \\
\midrule
Baseline Performance              &   --  &  --  & 53\% \\
\midrule
Llama 3.1 8B Instruct ($t=1.0$)   & 0.376 & 62\% & 72\% \\
Llama 3.1 8B Instruct ($t=0.7$)   & 0.375 & 71\% & 72\% \\
Llama 3.1 8B Instruct ($t=0.5$)   & 0.377 & 73\% & \textbf{75\%} \\
Llama 3.1 8B Base ($t=1.0$)       & 0.396 & 39\% & 68\% \\
\Sys{} Llama 3.1 8B (All $t=0.7$) & 0.377 & 47\% & 73\% \\
\bottomrule
\end{tabular}
\end{sc}
\end{small}
\end{center}
\vskip -0.1in
\end{table}

\begin{table}[H]
\caption{Temperature ablations with independent sampling from Llama-3.1-8B-Instruct. Average pairwise embedding cosine similarity, IR, and downstream accuracy results on the full LCB TOP set. A Llama-3.1-8B-Instruct model was fine-tuned for 4 epochs on the generated data ($n=1000$). The baseline performance of the Llama-3.1-8B-Instruct model on the evaluation set prior to any fine-tuning is reported in the first row.}
\label{lcb-temp}
\vskip 0.1in
\begin{center}
\begin{small}
\begin{sc}
\begin{tabular}{c|ccc}
\toprule
Generation Method & Average Embedding Similarity & IR & Downstream Accuracy \\
\midrule
Baseline Performance              &   --  &  --  & 18.6\% \\
\midrule
Llama 3.1 8B Instruct ($t=1.0$)   & 0.365 & 33\% & 20.1\% \\
Llama 3.1 8B Instruct ($t=0.7$)   & 0.416 & 51\% & 20.6\% \\
Llama 3.1 8B Instruct ($t=0.5$)   & 0.450 & 53\% & 22.9\% \\
Llama 3.1 8B Base ($t=1.0$)       & 0.468 & 36\% & 24.9\% \\
\Sys{} Llama 3.1 8B (All $t=0.7$) & 0.462 & 47\% & \textbf{28.1\%} \\
\bottomrule
\end{tabular}
\end{sc}
\end{small}
\end{center}
\vskip -0.1in
\end{table}


\subsection{\Sys{} First Stage Ablations - GSM8K}
\label{appendix_refine}

This appendix contains diversity, IR, and downstream performance results for our ablation replacing the first stage of \Sys{} with an instruct-tuned model, specifically Llama-3.1-70B-Instruct. We refine using Llama-3.1-70B-Instruct and GPT-4o, and investigate the change in downstream performance compared to standard \Sys{} (using Llama-3.1-70B-Base in the first stage). Note that dataset diversity is unchanged compared to direct generation from Llama-3.1-70B-Instruct, that IR improves after refinement, and that downstream performance is consistently worse than standard \Sys{}.

\begin{table}[H]
\caption{Average pairwise embedding cosine similarity, IR, and downstream accuracy results on a $n=500$ randomly selected static subset of GSM8K. A Llama-3.2-1B-Instruct model was fine-tuned for 4 epochs on the generated data ($n=1000$). The baseline performance of the Llama-3.2-1B-Instruct model on the evaluation set prior to any fine-tuning is reported in the first row.}
\label{gsm8k-refine}
\vskip 0.1in
\begin{center}
\begin{small}
\begin{sc} 
\begin{tabular}{c|ccc}
\toprule
Generation Method & Average Embedding Similarity & IR & Downstream Accuracy \\
\midrule
Baseline Performance                   &   --  &  --  & 21.8\% \\
\midrule
Llama 3.1 70B Instruct                 & 0.421 & 51\% & 22.4\% \\
Llama 3.1 70B Instruct Self-Refine     & 0.422 & 63\% & 25.4\% \\
GPT-4o Refining Llama 3.1 70B Instruct & 0.421 & 70\% & 30.8\% \\
\midrule
\Sys{} Llama 3.1 70B                   & 0.305 & 54\% & 29.8\% \\
\Sys{} GPT-4o + Llama 3.1 70B Base     & 0.302 & 64\% & \textbf{35.8\%} \\
\bottomrule
\end{tabular}
\end{sc}
\end{small}
\end{center}
\vskip -0.1in
\end{table}


\section{GSM8K Prompt Examples}
\label{appendix_prompts}

In this appendix, we provide exact prompts used for the GSM8K domains, representative of those used throughout this work. Examples are formatted for inclusion in the prompts in the ``\{examples\}'' fields, with ``EXAMPLE START'' and ``EXAMPLE END'' delimiters for the base prompt. \Sys{} uses the standard Base Prompt in the base generation step.

\subsection*{Static Few-shot Examples}

\begin{tcolorbox}[colback=blue!5!white,colframe=blue!75!black,title=Example 1]
\textbf{Question:} Alice has 20 quarters. She wants to exchange them for nickels and so she goes to the bank. After getting back from the bank, she discovers that 20\% of the nickels are iron nickels worth \$3 each. What is the total value of her money now? \\
\textbf{Answer:} A quarter is worth five nickels because .25 / .05 = \(\ll.25/.05=5\gg 5\). She gets 100 nickels from the bank because 20 x 5 = \(\ll20*5=100\gg 100\). 20 of the nickels are iron nickels because 100 x .20 = \(\ll100*.20=20\gg 20\). 80 of the nickels are regular because 100 - 20 = \(\ll100-20=80\gg 80\). The iron nickels are worth \$60 because 20 x 3 = \(\ll20*3=60\gg 60\). The regular nickels are worth \$4 because 80 x .05 = \(\ll80*.05=4\gg 4\). Her money is now worth \$64 because 60 + 4 = \(\ll60+4=64\gg 64\). \texttt{\#\#\#\#} 64
\end{tcolorbox}

\begin{tcolorbox}[colback=blue!5!white,colframe=blue!75!black,title=Example 2]
\textbf{Question:} A church has 120 members. 40\% are adults. The rest are children. How many children more children are there than adults? \\
\textbf{Answer:} There are 48 adults because 120 x .4 = \(\ll120*.4=48\gg 48\). 60\% of members are children because 100 - 40 = \(\ll100-40=60\gg 60\). There are 72 children because 120 x .6 = \(\ll120*.6=72\gg 72\). There are 24 more children than adults because 72 - 48 = \(\ll72-48=24\gg 24\). \texttt{\#\#\#\#} 24
\end{tcolorbox}

\begin{tcolorbox}[colback=blue!5!white,colframe=blue!75!black,title=Example 3]
\textbf{Question:} Lisa is looking to attempt a World Record. She has decided to try and match Joey Chestnut's record of eating 75 full hotdogs, buns included, in 10 minutes. Halfway through the time Lisa has eaten 20 hotdogs. How many hotdogs will she have to eat per minute to at least tie Joey Chestnut's record? \\
\textbf{Answer:} Joey Chestnut ate 75 hotdogs to claim the record and Lisa has eaten 20 hot dogs so far, so she still needs to eat \(75-20=\ll75-20=55\gg 55\) hotdogs to tie Joey Chestnut. Lisa has a 10-minute time period to eat the hotdogs and half the time has already passed, which means Lisa has \(10/2=\ll10/2=5\gg 5\) minutes left until the competition is over. If she needs to eat 55 hotdogs to tie Joey Chestnut and there are 5 minutes left in the competition period, then she needs to eat \(55/5=\ll55/5=11\gg 11\) hot dogs per minute to have a chance of tying for a win. \texttt{\#\#\#\#} 11
\end{tcolorbox}


\begin{tcolorbox}[colback=blue!5!white,colframe=blue!60!white,title=Base Prompt]
Here are a few examples of grade school math word problems that require performing a sequence of elementary calculations using basic arithmetic operations. A bright middle school student should be able to solve each problem. The numerical answer is provided at the end of each example after \texttt{\#\#\#\#}.\\

\{examples\}\\

EXAMPLE START
\end{tcolorbox}

\begin{tcolorbox}[colback=orange!5!white,colframe=orange!75!black,title=Instruct Few-shot Prompt]
Provide an example of a grade school math word problem that requires performing a sequence of elementary calculations using basic arithmetic operations. A bright middle school student should be able to solve each problem. Problems require no concepts beyond the level of early Algebra. You must first specify the question, then provide the very concise reasoning and answer. Provide your example in the following format:\\

\texttt{Question: [question]} \\
\texttt{Answer: [answer]}\\

Provide only the question and answer in the given format. Note how the numerical answer is provided after \texttt{\#\#\#\#} after each brief reasoning for a question. Here are some examples:\\

\{examples\}\\

Now it's your turn. Start your response with the question.
\end{tcolorbox}

\begin{tcolorbox}[colback=green!5!white,colframe=green!75!black,title=Refine Prompt]
Improve the given grade school math word problem. Edit the problem or answer to be more similar in style to the examples, and disambiguate as necessary, in addition to correcting any errors. Do not change the theme of the problem. A bright middle school student should be able to solve each problem. Problems require no concepts beyond the level of early Algebra. Note how the numerical answer is provided after \texttt{\#\#\#\#} after each brief reasoning for a question. Provide your edited problem in the following format:\\

\texttt{Question: [question]} \\
\texttt{Answer: [answer]}\\

Provide only the question and answer in the given format. Here are some examples of categories and problems on those categories:\\

\{examples\}\\

Now it's your turn. Here is the question and answer for you to edit:\\
Question:\\
\{question\}\\
Answer:\\
\{answer\}\\

Provide only the improved question and answer in the given format. Do not include any commentary or notes. Start your response with the question.
\end{tcolorbox}

\begin{tcolorbox}[colback=purple!5!white,colframe=purple!75!black,title=Sequential Prompt]
Generate a new grade school math word problem that requires performing a sequence of elementary calculations using basic arithmetic operations. A bright middle school student should be able to solve each problem. Problems require no concepts beyond the level of early Algebra. Here are the previously generated examples:\\

\{examples\}\\

Your new problem should:
\begin{enumerate}
    \item Be different from the previous examples
    \item Follow the same format and style as prior problems
\end{enumerate}

Note how the numerical answer is provided after \texttt{\#\#\#\#} after each brief reasoning for a question. Provide only the question and answer in the given format here:

\texttt{Question: [question]} \\
\texttt{Answer: [answer]}\\

Start your response with the question.
\end{tcolorbox}

\begin{tcolorbox}[colback=brown!5!white,colframe=brown!75!black,title=In One Prompt]
Provide \texttt{\{num\}} examples of problems that might be grade school math word problems that require performing a sequence of elementary calculations using basic arithmetic operations. A bright middle school student should be able to solve each problem. Problems require no concepts beyond the level of early Algebra. You must first specify the question then provide the brief reasoning and answer. Note how the numerical answer is provided after \texttt{\#\#\#\#} after each brief reasoning for a question. Provide your examples in the following format:\\

\texttt{Question: [question]} \\
\texttt{Answer: [answer]}\\

Here are some examples:\\

\{examples\}\\

Now it's your turn. Generate \texttt{\{num\}} different problems following this format. Your question should be different in content from the examples. Make sure to only provide only the question and answer. Start each example with the question. Delimit the end of an example with the phrase "END OF EXAMPLE" (all caps) on a new line.
\end{tcolorbox}

\begin{tcolorbox}[colback=teal!5!white,colframe=teal!75!black,title=Persona Prompt]
\{persona\_description\}\\

Provide an example of a grade school math word problem that requires performing a sequence of elementary calculations using basic arithmetic operations. A bright middle school student should be able to solve each problem. Problems require no concepts beyond the level of early Algebra. You must first specify the question, then provide the very concise reasoning and answer. Provide your example in the following format:\\

\texttt{Question: [question]} \\
\texttt{Answer: [answer]}\\

Provide only the question and answer in the given format. Note how the numerical answer is provided after \texttt{\#\#\#\#} after each brief reasoning for a question. Here are some examples:\\

\{examples\}\\

Now it's your turn. Start your response with the question.
\end{tcolorbox}

\begin{tcolorbox}[colback=red!5!white,colframe=red!75!black,title=Indistinguishability Rate Prompt]
System Prompt:\\

You are an expert at evaluating question and answer pairs for grade school math word problems.\\

You will be shown \{$k$\} examples. Each example consists of some context, a question, and an answer. All but one of them is generated from a high quality AI while one of them is of low quality.\\

Your task is to identify which example (1, 2, ..., \{$k$\}) appears to be of low quality. Consider factors like:

\begin{enumerate}
    \item Differing natural language patterns and flow
    \item Differing question structure, clarity, and difficulty
    \item Context and specificity
    \item Any subtle artifacts or unnatural patterns
\end{enumerate}

Analyze each example carefully and explain your reasoning. End with 'Answer: [Question Number]' where Question Number is 1, 2, ..., \{$k$\}.

---

User Prompt:\\

Here are \{$k$\} examples. One of them is of low quality. Please identify which one:\\

\{questions\}\\

Analyze each example and explain which one you think is of low quality. End with 'Answer: [Question Number]'.
\end{tcolorbox}



\end{document}
