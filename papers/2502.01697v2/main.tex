\documentclass{article}

% Recommended, but optional, packages for figures and better typesetting:
\usepackage{microtype}
\usepackage{graphicx}
\usepackage{subfigure}
\usepackage{float}
\usepackage{booktabs} % for professional tables


\usepackage[authoryear]{natbib}

\usepackage{hyperref}

\newcommand{\thought}[1]{{\color[rgb]{0.2,0.39,0.66}(#1)}}
\newcommand{\todo}[1]{{\color[rgb]{1.0,0.0,0.0}(#1)}}
\newcommand{\hsh}[1]{{\color{green!50!black} Henrik: #1}}
\newcommand{\st}[1]{{\color{red!50!black} Sebastian: #1}}

\newcommand{\ulm}[1]{_{\scaleto{\mathrm{#1}}{3pt}}}
\newcommand\at[2]{\left.#1\right|_{#2}}











\newtheorem{assumption}{Assumption}

\DeclareMathOperator*{\argmax}{arg\,max}
\DeclareMathOperator*{\argmin}{arg\,min}

\newcommand{\swname}[1]{\texttt{#1}}
\newcommand{\ie}{i\/.\/e\/.,\/~}
\newcommand{\eg}{e\/.\/g\/.,\/~}
\newcommand{\cf}{cf\/.\/~}

\newcommand{\fig}{Fig\/.\/~}
\newcommand{\defn}{Def\/.\/~}
\newcommand{\sect}{Sec\/.\/~}
\newcommand{\tabl}{Tab\/.\/~}
\newcommand{\algo}{Algorithm~}
\newcommand{\theo}{Theorem~}

\newcommand{\bnnl}{3 hidden layers}
\newcommand{\bnnn}{50 neurons}
\newcommand{\bnna}{tanh activations}

\newcommand{\capt}[1]{\mdseries{\emph{#1}}}

\newcommand{\videolink}{at \url{https://youtu.be/_d7AqTRjz6g}}
\newcommand{\codelink}{\url{https://github.com/wheelbot/mini-wheelbot}}

\newcommand{\fakepar}[1]{\vspace{0mm}\noindent\textbf{#1.}}

\newcommand{\needref}{\textcolor{red}{[REF]}}

\newcommand{\plotfontsize}{9pt}


% Use the following line for the initial blind version submitted for review:
% \usepackage{icml2025}

% If accepted, instead use the following line for the camera-ready submission:
\usepackage[accepted]{icml2025}

% For theorems and such
\usepackage{amsmath}
\usepackage{amssymb}
\usepackage{mathtools}
\usepackage{amsthm}


\usepackage[most]{tcolorbox}
\usepackage{xcolor}

% if you use cleveref..
\usepackage[capitalize,noabbrev]{cleveref}

%%%%%%%%%%%%%%%%%%%%%%%%%%%%%%%%
% THEOREMS
%%%%%%%%%%%%%%%%%%%%%%%%%%%%%%%%
\theoremstyle{plain}
\newtheorem{theorem}{Theorem}[section]
\newtheorem{proposition}[theorem]{Proposition}
\newtheorem{lemma}[theorem]{Lemma}
\newtheorem{corollary}[theorem]{Corollary}
\theoremstyle{definition}
\newtheorem{definition}[theorem]{Definition}
\newtheorem{assumption}[theorem]{Assumption}
\theoremstyle{remark}
\newtheorem{remark}[theorem]{Remark}

% Todonotes is useful during development; simply uncomment the next line
%    and comment out the line below the next line to turn off comments
% \usepackage[disable,textsize=tiny]{todonotes}
% \usepackage[textsize=tiny]{todonotes}


% The \icmltitle you define below is probably too long as a header.
% Therefore, a short form for the running title is supplied here:
\icmltitlerunning{\Sys{}: Combining Base and Instruction-Tuned Language Models for Better Synthetic Data Generation}

\begin{document}
\twocolumn[
\icmltitle{\Sys{}: Combining Base and Instruction-Tuned Language Models for Better Synthetic Data Generation}

% It is OKAY to include author information, even for blind
% submissions: the style file will automatically remove it for you
% unless you've provided the [accepted] option to the icml2025
% package.

% List of affiliations: The first argument should be a (short)
% identifier you will use later to specify author affiliations
% Academic affiliations should list Department, University, City, Region, Country
% Industry affiliations should list Company, City, Region, Country

% You can specify symbols, otherwise they are numbered in order.
% Ideally, you should not use this facility. Affiliations will be numbered
% in order of appearance and this is the preferred way.
\icmlsetsymbol{equal}{*}

\begin{icmlauthorlist}
\icmlauthor{Alan Zhu}{equal,berkeley}
\icmlauthor{Parth Asawa}{equal,berkeley}
\icmlauthor{Jared Quincy Davis}{stanford,foundry}
\icmlauthor{Lingjiao Chen}{stanford,microsoft}
\icmlauthor{Boris Hanin}{princeton,foundry}
\icmlauthor{Ion Stoica}{berkeley}
\icmlauthor{Joseph E. Gonzalez}{berkeley}
\icmlauthor{Matei Zaharia}{berkeley}
\end{icmlauthorlist}

\icmlaffiliation{berkeley}{UC Berkeley}
\icmlaffiliation{stanford}{Stanford University}
\icmlaffiliation{princeton}{Princeton University}
\icmlaffiliation{microsoft}{Microsoft Research}
\icmlaffiliation{foundry}{Foundry}

\icmlcorrespondingauthor{Alan Zhu}{aczhu@berkeley.edu}
\icmlcorrespondingauthor{Parth Asawa}{pgasawa@berkeley.edu}

% You may provide any keywords that you
% find helpful for describing your paper; these are used to populate
% the "keywords" metadata in the PDF but will not be shown in the document
\icmlkeywords{Machine Learning, ICML}

\vskip 0.2in
]

% this must go after the closing bracket ] following \twocolumn[ ...

% This command actually creates the footnote in the first column
% listing the affiliations and the copyright notice.
% The command takes one argument, which is text to display at the start of the footnote.
% The \icmlEqualContribution command is standard text for equal contribution.
% Remove it (just {}) if you do not need this facility.

% \printAffiliationsAndNotice{}  % leave blank if no need to mention equal contribution
\printAffiliationsAndNotice{\icmlEqualContribution} % otherwise use the standard text.

End-to-end imitation learning offers a promising approach for training robot policies. However, generalizing to new settings—such as unseen scenes, tasks, and object instances—remains a significant challenge. Although large-scale robot demonstration datasets have shown potential for inducing generalization, they are resource-intensive to scale. In contrast, human video data is abundant and diverse, presenting an attractive alternative. Yet, these human-video datasets lack action labels, complicating their use in imitation learning. Existing methods attempt to extract grounded action representations (e.g., hand poses), but resulting policies struggle to bridge the embodiment gap between human and robot actions.
% our approach
We propose an alternative approach: leveraging language-based reasoning from human videos - essential for guiding robot actions - to train generalizable robot policies. Building on recent advances in reasoning-based policy architectures, we introduce Reasoning through Action-free Data (RAD). RAD learns from both robot demonstration data (with reasoning and action labels) and action-free human video data (with only reasoning labels). The robot data teaches the model to map reasoning to low-level actions, while the action-free data enhances reasoning capabilities. Additionally, we will release a new dataset of 3,377 human-hand demonstrations compatible with the Bridge V2 benchmark. This dataset includes chain-of-thought reasoning annotations and hand-tracking data to help facilitate future work on reasoning-driven robot learning.
% experiments
Our experiments demonstrate that RAD enables effective transfer across the embodiment gap, allowing robots to perform tasks seen only in action-free data. Furthermore, scaling up action-free reasoning data significantly improves policy performance and generalization to novel tasks. These results highlight the promise of reasoning-driven learning from action-free datasets for advancing generalizable robot control. 
% releasing dataset
Website: \href{https://rad-generalization.github.io}{here}.


\section{Introduction}
\label{sec:intro}


\ps{Challenges of technology scaling}

The growing demand for computing performance has always been met by increasing the number of transistors per chip, which is only possible due to CMOS technology scaling.
However, as we keep pushing the boundaries of technology scaling, we encounter multiple challenges.
Firstly, whenever we transition to a more advanced technology node, the non-recurring cost due to physical design, verification, software, mask sets, and prototyping almost doubles \cite{cost-tech-node}.
As a result, designing a chip in an advanced technology node is only economically viable if the chip is manufactured in vast quantities.
Secondly, many chip components such as I/O drivers, analog circuits, or \gls{srams} have reached their scaling limit.
This means that we cannot shrink these components further, even if we use a more advanced technology with a smaller feature size.
Thirdly, advanced technology nodes suffer from high defect rates, diminishing the yield and inflating the recurring cost.
To tackle these challenges, new chip-design paradigms have been developed.

\ps{Why 2.5D integration?}

One of these new paradigms is 2.5D integration, where multiple silicon dies called chiplets are integrated into the same package.
Once designed, a single chiplet can be reused in multiple 2.5D stacked chips, which increases the ratio of production volume to non-recurring cost.
Another advantage is that multiple chiplets - fabricated in different technologies - can be integrated into the same package.
This means that only components that can take full advantage of technology scaling are built in bleeding-edge technologies.
Components that have reached their scaling limit are fabricated in more mature and hence less costly technology nodes.
Furthermore, chiplets are smaller than monolithic chips.
Therefore, manufacturing chiplets results in less silicon area loss due to fabrication defects and hence a higher yield.
Due to these economic advantages, chip vendors such as AMD \cite{amd-chiplet} and NVIDIA \cite{chiplet-book} have adopted the 2.5D integration paradigm.  

\ps{Challenges of 2.5D integration}

An important challenge when designing 2.5D stacked chips is the construction of a low-latency and high-throughput \gls{ici}. 
To build an \gls{ici}, we connect different chiplets using \gls{d2d} links.
These links are fabricated in an organic package substrate, silicon bridge, or silicon interposer, and they are connected to the chiplets using \gls{c4} bumps or microbumps.
The number of bumps per chiplet is limited, and so is the bandwidth of \gls{d2d} links.
In addition to having lower bandwidth than links in monolithic chips, \gls{d2d} links also have higher latency.
This latency is caused by wire delay and by \gls{phys} that are necessary in both the sending and the receiving chiplet.
\gls{phys} are needed to convert between protocols, voltage levels, and frequencies, which are usually different between on-chiplet links and \gls{d2d} links.
Due to these limitations, the \gls{ici} can quickly become a bottleneck.

\ps{How we solve these challenges differently than the related work does.}

Existing approaches to maximize the performance of the \gls{ici} either optimize the placement of chiplets (with potentially heterogeneous shapes) for a predetermined \gls{ici} topology 
\cite{ho,liu,seemuth,eris,osmolovskyi,tap25d,chiou}, select one topology out of a set of candidates \cite{coskun-1, coskun-2}, or they optimize the \gls{ici} topology for a 2D grid of homogeneously shaped chiplets on an active interposer \cite{butterdonut, cluscross, kite}.
To the best of our knowledge, there is no prior work on \gls{ici} topologies for chips with heterogeneously shaped chiplets or with passive silicon interposers or silicon bridges.
To fill this gap, we propose \name, a novel optimization methodology to jointly optimize the chiplet placement and \gls{ici} topology of such architectures.
\ifnb
\else
\newpage
\fi

\ps{Details on \name~and the key idea}

The key idea is as follows: 
We optimize the chiplet placement without a predetermined topology.
For each placement generated by an optimization algorithm, we infer a placement-based \gls{ici} topology by connecting chiplets that are in close proximity in that specific placement.
We then compute the latency and throughput of this combination of placement and topology for different traffic types.
These latencies and throughputs together with the total chip area are used to compute a user-defined quality-score of the placement, which is returned to the optimization algorithm.
Based on this quality score, the algorithm can further optimize the placement.
By following this iterative process, we jointly optimize the chiplet placement and the \gls{ici} topology.

\ps{Short evaluation-summary}

We provide our open-source framework implementing the proposed placement and topology co-optimization methodology, which we evaluate using both synthetic traffic and traffic traces.
A 2D grid of chiplets with a mesh topology is used as a baseline since many proposals for 2.5D stacked chips \cite{dataflow_accel_dnn, cifher, simba, hecaton, dojo} use such an architecture.
We reduce the latency of synthetic L1-to-L2 and L2-to-memory traffic, the two most important traffic types for cache coherency traffic, by up to 28\% and 62\% respectively.
For real traffic traces, we reduce the average packet latency for almost all traces and architectures considered (reduced by an 8\% or 18\% on average depending on the configuration of \gls{phys} within a chiplet).


\section{Related work}


Recent advances in single-image animatable head avatar generation can be categorized into mainly 2D-based and 3D-based approaches. 

\paragraph{\bf Image to 2D Animatable Avatar.}
2D-based methods, leveraging the power of convolutional neural networks (CNNs)~\cite{DBLP:conf/cvpr/KarrasLAHLA20,DBLP:conf/cvpr/IsolaZZE17,DBLP:conf/nips/GoodfellowPMXWOCB14}, often employ generative adversarial networks (GANs)~\cite{DBLP:conf/cvpr/StyleGAN} for direct image synthesis. Early approaches~\cite{DBLP:conf/cvpr/WangDYSW23,DBLP:conf/cvpr/BurkovPGL20,DBLP:conf/iccv/ZakharovSBL19} focus on injecting expression and pose features into the generator network, often utilizing architectures like U-Net or StyleGAN~\cite{DBLP:conf/cvpr/StyleGAN}.
Some other 2D methods~\cite{DBLP:journals/corr/abs-2407-03168,DBLP:conf/cvpr/ZhangQZZW0CW023,DBLP:conf/cvpr/HongZS022,DBLP:conf/mm/DrobyshevCKILZ22,DBLP:conf/cvpr/BurkovPGL20,DBLP:conf/nips/SiarohinLT0S19} represent expressions and poses as warping fields applied to the source image. 
Benefiting from advances in image and video diffusion networks, more recent 2D-based works~\cite{DBLP:journals/corr/abs-2410-07718,DBLP:journals/corr/abs-2406-08801,DBLP:conf/eccv/TianWZB24} get improved results with diffusion techniques. 
However, these methods still face challenges related to long generation times and significant computational resource demands. Audio-driven 2D control methods~\cite{DBLP:conf/cvpr/ZhangCWZSGSW23,DBLP:journals/corr/abs-2211-12368,DBLP:conf/iccv/GuoCLLBZ21} are easy to use but cannot explicitly control facial expressions and poses. 2D-based techniques often struggle with large pose or expression variations due to the lack of an explicit 3D structure, sometimes producing unrealistic distortions or identity changes. While some 2D methods~\cite{SadTalker,StyleHEAT,Pirenderer,DBLP:conf/cvpr/WangM021,MegaPortraits} incorporate 3D Morphable Models (3DMMs)~\cite{DBLP:conf/fgr/GerigMBELSV18,DBLP:journals/tog/LiBBL017,DBLP:conf/avss/PaysanKARV09,DBLP:conf/siggraph/BlanzV99} to mitigate these issues, they typically cannot achieve free-viewpoint rendering. 

\vspace{-0.1in}

\begin{figure*}[h]
    \centering
    \includegraphics[width=0.9\linewidth]{images/framework.pdf}
    \caption{\textbf{Overall Framework.} Our framework utilizes learnable query features attached to FLAME vertices to perform cross-attention with the extracted multi-level image features. The extracted features are then decoded to reconstruct the Gaussian avatar in the canonical space, which can be animated utilizing standard linear blend skinning (LBS) and corrective blendshapes as the FLAME model did and rendered in real-time on various platforms.}
    \label{fig:framework}
\end{figure*}

\paragraph{\bf Image to 3D Animatable Avatar.}
3D-aware methods offer improved geometric consistency and free-viewpoint rendering capabilities. Early 3D approaches~\cite{DBLP:conf/eccv/KhakhulinSLZ22,DBLP:conf/cvpr/XuYCWDJT20} utilize 3DMMs for head avatar reconstruction. With the advent of Neural Radiance Fields (NeRFs)~\cite{DBLP:conf/eccv/MildenhallSTBRN20}, many recent methods~\cite{DBLP:conf/siggraph/YuFZWYBCSWSW23,DBLP:conf/cvpr/MaZQLZ23,DBLP:conf/cvpr/LiZWZ0CZWB023,GPAvatar,ye2024real3d,deng2024portrait4d,deng2024portrait4d2,DBLP:conf/eccv/KiMC24,DBLP:conf/cvpr/BaiFWZSYS23,PointAvatar,Nerfies,INSTA} have adopted this representation for higher fidelity, particularly in modeling fine details like hair. However, NeRF-based~\cite{DBLP:conf/cvpr/ZhangZLHLWGCL024,HAvatar,DBLP:conf/cvpr/BaiTHSTQMDDOPTB23,AD-NeRF,DBLP:journals/tog/GaoZXHGZ22,DBLP:journals/tog/ParkSHBBGMS21,DBLP:conf/cvpr/AtharXSSS22,DBLP:journals/corr/abs-2112-05637,DBLP:conf/iccv/TretschkTGZLT21,DBLP:conf/cvpr/GafniTZN21,DBLP:conf/eccv/KiMC24,DBLP:conf/cvpr/BaiFWZSYS23,PointAvatar,Nerfies,DBLP:conf/siggraph/YuFZWYBCSWSW23,DBLP:conf/cvpr/MaZQLZ23,DBLP:conf/cvpr/LiZWZ0CZWB023} approaches often require extensive training data, including multi-view or single-view videos, raising privacy concerns and limiting generalization to unseen identities. Some methods~\cite{DBLP:conf/cvpr/SunWWLZZL23,DBLP:conf/3dim/ZhuangMKS22,DBLP:journals/pami/SunWZHWL24,DBLP:journals/tvcg/TangZYZCMW24,DBLP:conf/iclr/XuZLZBFS23} bypass this data requirement by training generators with random noise and then inverting them for identity-specific reconstruction, but inversion accuracy remains a challenge. Test-time optimization offers another alternative, but its computational cost limits practical applications. Several recent works~\cite{goha2023,hidenerf2023,gpavatar2024,ye2024real3d,ma2024cvthead,deng2024portrait4d,deng2024portrait4d2,GGHead} have explored one-shot 3D head reconstruction to address the limitations of data requirements and computational cost. These methods employ various techniques, such as tri-plane features, deformation fields, point-based expression fields, and vertex-feature transformers. Despite these advancements, NeRF-based methods often struggle with real-time rendering. 
Recently, 3D Gaussian Splatting~\cite{GaussianSplatting} has emerged as a promising alternative, offering both high-quality results and fast rendering speeds. However, existing Gaussian Splatting methods~\cite{GaussianAvatar,DBLP:conf/cvpr/XuCL00ZL24} typically rely on video data for training for each person, limiting their ability to generalize to new identities. Instead, the most recent work, GAGAvatar~\cite{GAGAvatar}, proposes a one-shot 3D Gaussian-based head avatar generation method. However, it still relies heavily on complex 2D neural post-processing to achieve optimal animation outcomes, thus it is not a pure 3D solution and the extra neural network hinders its application on various platforms. In contrast, our work generates Gaussian heads that are immediately animatable and renderable without additional networks or post-processing steps, enabling seamless integration into existing rendering pipelines for real-time animation and rendering across a wide range of platforms, including mobile phones. 
\section{Motivation and Goal}

\param{TODO: Motivate}


\section{Base-Refine (\Sys{})}
\label{bare}

\begin{figure*}[ht!]
\vskip 0.2in
\begin{center}
\centerline{\includegraphics[width=.9\textwidth]{assets/sys2.pdf}}
\caption{Instruct-tuned models provide high-quality but low-diversity data, while base models provide low-quality but high-diversity data. \Sys{} independently generates a diverse initial set of data points with a base model and refines each entry individually with an instruct-tuned model to create a high-quality, high-diversity dataset. In this example of a real grade school math problem generation, the Llama-3.1-70B-Base model hallucinates in its answer to its own question. The refiner (Llama-3.1-70B-Instruct) recognizes this and disambiguates the question and corrects the reasoning.}
\label{system-diagram}
\end{center}
\vskip -0.2in
\end{figure*}


We leverage our insights to propose \textbf{\Sys{}}, a practical synthetic data generation method combining the diversity of base models with the quality of instruct models. \Sys{} uses a base model to generate an initial set of diverse but potentially lower quality data, after which an instruct-tuned model individually refines each example from the initial set -- as shown in the example in \cref{system-diagram}.

In the base step, the base model uses a prompt with minimal few-shot examples to generate. In the refine step, the instruct-tuned model is instructed to retain the concept of the base model generation while improving it according to specific criteria (e.g., realism, correctness). This retains the overall diversity of the base model generated set while exerting greater control over the quality of the final generation.

Importantly, few-shot examples are only required in the base step to ensure formatting (though they can also be included in the refine step), and only in small amounts. Thus, \textbf{\Sys{}} is especially useful in domains with very limited real-world data as little human effort is required to generate a diverse high-quality synthetic dataset. In our experiments, we limit ourselves to just three few-shot examples. In addition, we intentionally use very general prompts for \Sys{} to demonstrate its flexibility, underscoring the potential for even greater improvement with tailored prompts. Representative prompts are included in \cref{appendix_prompts}.
\begin{table*}[t]
    \centering
    \resizebox{\textwidth}{!}{
\begin{tabular}{l|rrllrrll}
\toprule
\textbf{Dataset} & \multicolumn{4}{c}{\textbf{GSM8K}} & \multicolumn{4}{c}{\textbf{MATH}} \\
\cmidrule(lr){1-1} \cmidrule(lr){2-5} \cmidrule(lr){6-9}
\textbf{Method} & Acc & Len & Rel. Acc & Rel. Len & Acc & Len & Rel. Acc & Rel. Len \\
\midrule
\multicolumn{9}{l}{\textit{Zero-Shot Prompting}} \\
\midrule
\hspace{12pt}Baseline & 78.06 & 241.87 & 100.00 \small{(0.00)} & 100.00 \small{(0.00)} & 46.40 & 480.37 & 100.00 \small{(0.00)} & 100.00 \small{(0.00)} \\
\hspace{12pt}Be Concise & 77.98 & 214.87 & 99.85 \small{(1.18)} & 88.46 \small{(10.37)} & 47.76 & 446.09 & 102.71 \small{(7.59)} & 92.66 \small{(7.46)} \\
\hspace{12pt}Hand Crafted 2 (ours) & 76.72 & 184.13 & 98.27 \small{(3.67)} & 77.10 \small{(22.27)} & 46.84 & 404.85 & 101.62 \small{(4.79)} & 85.26 \small{(15.97)} \\
\midrule
\multicolumn{9}{l}{\textit{FT - External Data}} \\
\midrule
\hspace{12pt}Direct Answer & 19.70 & 3.17 & 24.88 \small{(5.03)} & 1.36 \small{(0.40)} & 15.08 & 6.98 & 35.16 \small{(10.34)} & 1.44 \small{(0.73)} \\
\hspace{12pt}Human CoT & 65.73 & 127.85 & 83.82 \small{(7.28)} & 54.95 \small{(13.17)} & 33.88 & 243.54 & 75.61 \small{(13.56)} & 53.14 \small{(13.87)} \\
\hspace{12pt}GPT4o CoT & 76.36 & 156.24 & 97.65 \small{(3.63)} & 67.60 \small{(16.70)} & 40.44 & 399.80 & 90.52 \small{(15.07)} & 87.21 \small{(22.22)} \\
\midrule
\multicolumn{9}{l}{\textit{FT - Best-of-N Self-Generation}} \\
\midrule
\hspace{12pt}Naive BoN & 77.12 & 214.22 & 98.79 \small{(1.64)} & 87.17 \small{(8.79)} & 47.64 & 433.26 & 101.74 \small{(7.04)} & 89.89 \small{(3.99)} \\
\hspace{12pt}Rational Metareasoning & 76.15 & 207.49 & 97.21 \small{(5.74)} & 84.93 \small{(5.09)} & 47.56 & 432.56 & 103.02 \small{(6.56)} & 90.56 \small{(5.25)} \\
\midrule
\multicolumn{9}{l}{\textit{FT - Few-Shot Conditioned Self-Generation (ours)}} \\
\midrule
\hspace{12pt}FS-Human & 76.66 & 161.72 & 98.06 \small{(3.28)} & 67.96 \small{(16.62)} & 46.44 & 421.54 & 99.69 \small{(6.97)} & 87.78 \small{(5.98)} \\
\hspace{12pt}FS-GPT4o & 78.07 & 175.54 & 99.94 \small{(1.69)} & 73.15 \small{(13.49)} & 47.36 & 421.21 & 101.87 \small{(5.33)} & 87.58 \small{(6.60)} \\
\hspace{12pt}FS-Self & 77.27 & 190.03 & 98.86 \small{(2.51)} & 77.51 \small{(9.18)} & 48.00 & 426.67 & 102.67 \small{(5.24)} & 88.50 \small{(4.49)} \\
\midrule
\multicolumn{9}{l}{\textit{FT - Few-Shot Conditioned Best-of-N Self-Generation (ours)}} \\
\midrule
% GPT4o Best-of-16 (Naive) & 75.48 & 153.51 & 96.56 \small{(3.79)} & 64.12 \small{(16.35)} & 47.28 & 367.49 & 101.50 \small{(9.81)} & 76.96 \small{(11.42)} \\
\hspace{12pt}FS-GPT4o-BoN & 75.88 & 153.38 & 97.00 \small{(4.11)} & 64.25 \small{(16.66)} & 47.36 & 364.33 & 102.56 \small{(6.24)} & 76.30 \small{(10.56)} \\
\hspace{24pt}\raisebox{0.5ex}{$\llcorner$}\hspace{4pt}\textit{Budget-Matched} & 76.24 & 160.59 & 97.44 \small{(3.67)} & 67.15 \small{(16.41)} & 47.52 & 384.43 & 101.58 \small{(9.53)} & 80.43 \small{(9.04)} \\

\bottomrule
\end{tabular}
    }
    \caption{
        Evaluation of zero-shot prompting, fine-tuning (FT), and our proposed methods on GSM8K and MATH datasets, averaged across five main models.
        We report absolute accuracy (\%) and length (tokens), along with relative accuracy (\%) and length (\%) compared to baseline prompting. Values in parentheses show standard deviations.
    }
    \label{tab:main_results}
\end{table*}

\section{Results}
\label{results}

% \begin{table*}[t]
% \resizebox{\textwidth}{!}{
% \centering
% \small
% \begin{tabular}{m{2.5cm}|m{15cm}}
% \toprule
% \textbf{Method} & \textbf{Content} \\ %& \# \textbf{Tokens}\\
% \midrule
% \midrule[0.05em]
% \multicolumn{2}{c}{Question} \\ 
% \midrule
% - & A robe takes 2 bolts of blue fiber and half that much white fiber.  How many bolts in total does it take?  \\
% \midrule
% \midrule[0.05em]
% \multicolumn{2}{c}{Reasoning Paths} \\
% \midrule
% Zero-Shot & To find the total number of bolts needed, we need to calculate the amount of white fiber first, since it's half the amount of blue fiber. Step 1: Determine the amount of blue fiber needed. The robe takes 2 bolts of blue fiber. Step 2: Calculate the amount of white fiber needed. The robe takes half the amount of white fiber as blue fiber. Half of 2 bolts is $2/2$ = 1 bolt. Step 3: Add the amount of blue fiber and white fiber to find the total. Total bolts = blue fiber + white fiber. Total bolts = 2 bolts + 1 bolt. Total bolts = 3 bolts. The answer is \textbf{\textit{3}}. \\ %& 213 \\
% \midrule
% FS-GPT4o-BoN & The robe takes 2 bolts of blue fiber. It takes half that much white fiber, which is $2 / 2$ = 1 bolt. Add the blue and white fiber together: 2 + 1 = 3 bolts. The answer is \textbf{\textit{3}}. \\ %57 \\
% \bottomrule
% \end{tabular}
% }
% \caption{Example rationales generated from the original Llama-3.1-8B model and the model fine-tuned using our method. The question is randomly sampled from GSM8K. For the same question, one can observe that our rationale is much more concise, removing redundant parts that explain the question.}
% \label{tab:samples}
% \vspace{-0.1in}
% \end{table*}


\begin{table*}[t]
\resizebox{\textwidth}{!}{
\centering
\normalsize
\begin{tabular}{m{17.5cm}}
\toprule
\multicolumn{1}{c}{\textbf{Question}} \\ 
\midrule
A robe takes 2 bolts of blue fiber and half that much white fiber.  How many bolts in total does it take? \\
\midrule
\multicolumn{1}{c}{\textbf{Reasoning Paths}} \\
\midrule
\textbf{[Zero-Shot]} 
To find the total number of bolts needed, we need to calculate the amount of white fiber first, since it's half the amount of blue fiber. Step 1: Determine the amount of blue fiber needed. The robe takes 2 bolts of blue fiber. Step 2: Calculate the amount of white fiber needed. The robe takes half the amount of white fiber as blue fiber. Half of 2 bolts is $2/2$ = 1 bolt. Step 3: Add the amount of blue fiber and white fiber to find the total. Total bolts = blue fiber + white fiber. Total bolts = 2 bolts + 1 bolt. Total bolts = 3 bolts. The answer is 3. \\
\midrule
\textbf{[FS-GPT4o-BoN]} 
The robe takes 2 bolts of blue fiber. It takes half that much white fiber, which is $2 / 2$ = 1 bolt. Add the blue and white fiber together: 2 + 1 = 3 bolts. The answer is 3. \\
\bottomrule
\end{tabular}
}
\caption{Example rationales generated from the original Llama-3.1-8B model (\textbf{Zero-Shot}) and the model fine-tuned using our method (\textbf{FS-GPT4o-BoN}). The question is randomly sampled from GSM8K. For the same question, one can observe that our rationale is much more concise, removing redundant parts that explain the question.}
\label{tab:samples}
\vspace{-0.1in}
\end{table*}


\subsection{Main results}

Our main results, presented in \autoref{tab:main_results} and \autoref{fig:main_methods_comparison}, demonstrate the performance of our self-training methods against baseline approaches.
% We highlight key observations from these results below.

\paragraph{Naive BoN fine-tuning is effective but sample inefficient.}
Naive BoN fine-tuning effectively reduces output length without significantly degrading model performance. 
This also holds true for Qwen2.5-Math-1.5B and DeepSeekMath-7B (\autoref{tab:main_results_full_gsm8k} and \autoref{tab:main_results_full_math}), which failed to achieve length reduction through zero-shot prompting.
% However, while naive BoN does reduce output length, the reduction is limited to 12\%.
However, the length reduction from naive BoN with $N=16$ is limited to 12\% on average.
Furthermore, as illustrated in Figure~\ref{fig:bon_sample_efficiency}, achieving more compression with BoN becomes progressively less efficient.

\paragraph{Iterative baseline yields similar results as naive BoN fine-tuning.}
% We compare our single-step naive BoN approach with Rational Metareasoning \cite{de2024rational}, an iterative approach using expert iteration \cite{zelikman2022star}  which incorporates an additional \textit{utility reward} to balance efficiency and accuracy in BoN sampling.
Rational Metareasoning, an iterative baseline, yields similar relative length reduction and relative accuracy to BoN fine-tuning. 
This suggests that the utility reward proposed by \citet{de2024rational} may not effectively achieve both accuracy gains and token length reduction.

\begin{figure}[t] % "h" places the figure roughly here
    \centering
    \includegraphics[width=\columnwidth]{figures/main_methods_comparison.pdf} % Adjust width as needed
    \caption{Tradeoff between relative accuracy and length reduction for main methods. Results are averaged over GSM8K and MATH across five main models. Matching colors and shapes indicate the same FS prompt. FS conditioning without augmentation (†) are marked with lighter colors. 
    Relative length reduction refers to 100 - relative length (\%).}
    \label{fig:main_methods_comparison} % Label for referencing in text
\end{figure}
% \red{TODO - shorten this}

\paragraph{Few-shot conditioning outperforms BoN in length reduction.}
The results demonstrate that few-shot conditioning achieves a greater relative length reduction compared to naive BoN, including math-specialized models (\autoref{tab:main_results_full_gsm8k} and \autoref{tab:main_results_full_math}).
% This reduction is attributed to the fact that the fine-tuning datasets generated through few-shot conditioning contain shorter reasoning paths compared to those generated by naive BoN, as illustrated in \autoref{fig:bon_sample_efficiency}.  % too long
This is in line with the superior length reduction of few-shot conditioning, compared to naive BoN as shown in \autoref{fig:bon_sample_efficiency}.
Notably, self-training on generations conditioned on human-annotated examples (FS-Human) achieves an average relative length of 67.96\% on GSM8K, compared to 87.17\% with naive BoN.  % good to have some specific numbers in the text
% We further analyze the effect of fine-tuning on length reduction in \autoref{analysis}.



\paragraph{Self-training better preserves accuracy than training with external data.} 
\autoref{tab:main_results} shows fine-tuning with external data (\textit{FT-External Data}) leads to a significant reduction in relative length but causes a severe drop in relative accuracy. 
% \autoref{fig:main_methods_comparison} further highlights that while fine-tuning with GPT-4o CoT (FT-GPT4o) achieves slightly greater reduction in relative length than fine-tuning with self-generated data using few-shots from GPT-4o (FS-GPT4o), it results in substantially lower relative accuracy.  % a bit complicated / not concrete (conrete evidence = one where we beat external FT in both accuracy and reduction)
\autoref{fig:main_methods_comparison} further highlights the accuracy preservation of self-training: fine-tuning with external concise reasoning supervision from GPT-4o (FT-GPT4o) lies below the Pareto-curve of relative accuracy and relative length reduction, established by our self-training methods.
% NAMGYU - TODO add some commentary

\paragraph{Few-shot conditioned BoN achieves best length reduction while maintaining accuracy.}
% Few-shot conditioned BoN enables substantial length reduction compared to all other BoN and few-shot methods while maintaining relative accuracy.
FS-BoN elicits the largest length reduction among our self-training methods, while maintaining relative accuracy, on average.
Notably, for math-specialized models, FS-GPT4o-BoN achieves the greatest reduction among all methods, except those fine-tuned on external data which greatly sacrifice the accuracy (\autoref{tab:main_results_full_gsm8k} and \autoref{tab:main_results_full_math}). 
% This result reflects how applying BoN to few-shot conditioning further reduces the relative length of the training data while also increasing the proportion of correct samples.  % unnecessary

\paragraph{Augmentation boosts accuracy for few-shot conditioning.}
\autoref{fig:main_methods_comparison} compares few-shot conditioning, i.e., FS and FS-BoN, with and without augmentation (†). 
Augmentation improves accuracy by providing solutions for previously unsolvable hard questions as discussed in \autoref{sample_augmentation}. 
While augmentation may slightly affect reduction rates, they remain superior to naive BoN and RM.
% Similar effect is observed for augmentation in FS-BoN.
% Even when matching the budget (\textit{Budget-Matched}) with other fine-tuning methods using self-generated data in \autoref{tab:main_results}, it achieves the greatest length reduction among them with minimal accuracy degradation.
Even when matching the budget (\textit{Budget-Matched}) with other self-training methods in \autoref{tab:main_results}, it achieves the greatest length reduction among them with minimal accuracy degradation.
The effect of augmentation on training data length is analyzed in \autoref{appx_augmentation_length}.
% Furthermore, as shown in Figure \ref{fig:main_methods_comparison}, augmentation on few-shot conditioned BoN enhances accuracy similar to naive BoN and Meta-Reasoning while achieving greater length reduction.

\begin{figure}[t]
    \centering
    \includegraphics[width=\columnwidth]{figures/length_by_difficulty.pdf} % Adjust width as needed
    \caption{\textbf{Tokens are reduced adaptively according to question difficulty.} 
    Token reduction rate for each difficulty level on MATH, for 4 models individually and averaged.
    % Higher difficulty levels show lower reduction rates.
    Relative length reduction refers to 100 - relative length (\%).
    }
    \label{fig:length_difficulty} % Label for referencing in text
\end{figure}

\subsection{Analysis}
\label{analysis}
% This section analyzes length reduction: transfer from generation to fine-tuning, reduction by question difficulty, qualitative analysis, and consistency across model sizes. DeepSeekMath-7B is excluded from quantitative analysis due to cost.
% let's keep this short
In this section, we analyze the length reduction effects in depth.
We exclude DeepSeekMath-7B from quantiative analysis due to cost.


% \paragraph{Analysis on sample efficiency}
% As shown in \autoref{fig:bon_sample_efficiency}, best-of-n (BoN) sampling requires a substantial number of samples to be generated to achieve a level of reasoning length reduction comparable to that achievable through few-shot conditioning.
% In other words, it is infeasible to reach the reasoning length reduction performance of few-shot conditioning using BoN alone, without generating a prohibitively large number of samples.
% However, our experiments consistently demonstrate that combining few-shot conditioning with BoN sampling is more effective in reducing reasoning length than using either technique in isolation.
% Specifically, few-shot conditioning helps to guide the model towards generating more concise reasoning paths, while BoN sampling allows us to select the shortest and most accurate path from a diverse set of candidates.
% This synergistic effect results in a more efficient and effective approach to concise reasoning.


% \paragraph{FT can reduce generation length effectively.}
% As shown in \autoref{fig:ft_length_scatter}, after fine-tuning, the models tend to follow the length of the training data, suggesting that reasoning length reduction can be achieved through simple supervised fine-tuning on short reasoning samples.
% Note that test generation length is relatively longer than the training data length, as the models can generate lengthy incorrect answers, while the training data consists of correct answers.
% Correctly generated answers align more closely with training data length as shown in (Appendix~\ref{appx_length_scatter_correct}).

% \paragraph{Length reduction through generation and fine-tuning}
% Our method reduces reasoning length in two stages: generation and fine-tuning.
% First, as shown in \autoref{fig:ft_length_scatter}, 
% % generation length for training data varies depending on the method. 
% few-shot conditioning methods produce shorter outputs than naive BoN, with few-shot conditioned BoN achieving the shortest. 
% Second, fine-tuning with shorter rationales results in shorter model outputs, showing a strong correlation between test and training lengths\footnote{Test generation lengths are generally longer than training data lengths due to the possibility of lengthy incorrect answers during testing. Test outputs that are correct align more closely with training data lengths, as shown in Appendix~\ref{appx_length_scatter_correct}.}.
% Overall, FS-GPT4o-BoN effectively generates and trains for shorter reasoning paths.
% Additionally, unlike zero-shot methods, our approach significantly reduces token length in math-tuned models like Qwen2.5-Math-1.5B with FS-GPT4o-BoN, achieving 54.7\% relative length after fine-tuning. (See \autoref{tab:main_results_full_gsm8k} and \autoref{tab:main_results_full_math}).

\paragraph{Tokens are reduced adaptively according to question complexity.} 
The MATH dataset's difficulty levels range from 1 (basic algebra) to 5 (advanced calculus and complex mathematical reasoning).
As shown in \autoref{fig:length_difficulty}, our method adaptively reduces tokens based on question difficulty, with higher difficulty leading to less reduction.
% Most models achieve their peak reduction (around 20\%--40\%) at difficulty levels 1-2, where simple concepts allow for more concise explanations.
% The reduction rate gradually declines at levels 3-5, indicating our method's ability to preserve necessary details for complex problems automatically.
%  -> not precise. simple concepts allow for more concise explanations *in absolute terms*, but this does not necessarily mean that length reduction *relative to the default* should be high. E.g., if the model already uses very few tokens for easy questions, then relative reduction would be low.
The higher reduction (20\%--40\%) at easier difficulty levels (1--2) suggests that the original model outputs for these easier questions contained unnecessary tokens.
This reveals a gap in current models' ability to tailor their inference budget to problem complexity.
Our method effectively closes this gap by reducing redundancy, allowing for more precise token allocation based on question difficulty.

\begin{figure}[t] % "h" places the figure roughly here
    \centering
    \includegraphics[width=\columnwidth]{figures/scaling_methods_comparison.pdf} % Adjust width as needed
    \caption{Scaling study on baseline and few-shot conditioned self-training methods. Results are averaged over GSM8K and MATH for Llama 1B, 3B, and 8B.
    % Accuracy tends to be maintained, with greater length reduction using our FS-GPT4o(-BoN) method.
    Relative length reduction refers to 100 - relative length (\%).
    }
    \label{fig:scaling_methods_comparison} % Label for referencing in text
\end{figure}

\paragraph{Self-training maintains consistency across model scales.}
We conduct a scaling study on Llama-3.2-1B, 3B, and Llama-3.1-8B to examine consistency across different model sizes (\autoref{fig:scaling_methods_comparison}). 
Overall, token reduction increases as the model size increases, while the maintenance of accuracy does not show a strong correlation with model size. 
RM exhibits lower reduction rates compared to our few-shot conditioned self-training methods across all models and shows a decrease in accuracy with increasing model size. 
% The few-shot method also shows a similar trend in length reduction, but it achieves the best relative accuracy in the 3B model.
Our standalone few-shot conditioning method (FS-GPT4o) also shows a similar trend in length reduction, but better preserves accuracy.
Our joint FS-GPT4o-BoN method achieves the greatest reduction across all models, maintaining relative accuracy across different model sizes, especially in the largest 8B model.



\paragraph{Sample study}
\autoref{tab:samples} presents qualitative examples of reasoning paths generated by the model before and after fine-tuning with our method. 
The original reasoning exhibits verbosity, containing redundant processes such as question confirmation and repeated instructions. 
In contrast, the reasoning generated by our method includes only the necessary steps, significantly reducing the number of tokens while still arriving at the correct answer. 
% These examples demonstrate the effectiveness of our method in reducing token count. 
More examples are provided in the \autoref{appx_sample_studies}.

\begin{figure}[t]
    \centering
    \includegraphics[width=\columnwidth]{figures/both_length_scatter.pdf} % Adjust width as needed
    \caption{\textbf{Fine-tuning effectively transfers the length reduction to the model.} Correlation between the relative length of train data and test output averaged over GSM8K and MATH across 4 models. Training length includes only correct solutions. Solid points represent test lengths including all generated outputs, while lighter points indicate test lengths of correct solutions only.}
    \label{fig:ft_length_scatter} % Label for referencing in text
\end{figure}

\paragraph{Length reduction is transferred through fine-tuning.}
As shown in \autoref{fig:ft_length_scatter}, fine-tuning with shorter rationales results in shorter model outputs, showing a strong correlation between test and training lengths.
% Test generation lengths (solid datapoints) are generally longer than training data lengths due to the possibility of lengthy incorrect answers during testing.
% However, when comparing with test generation lengths that are correct (lighter datapoints), they align more closely with training data lengths.
We note that the length of test outputs (incorrect and correct) are longer than the length of training samples (only correct) on average.
This is mainly because incorrect paths are generally longer than correct ones.
We find a closer correspondence between train length and test length of correct samples only, indicated by the lighter datapoints.
This discrepancy suggests the need to terminate incorrect paths early to minimize redundant inference overhead.
We consider this for future work.

\section{Conclusion and future work}
In this study, we examined the ability of LLMs to produce self-generated counterfactual explanations (SCEs).
We design a prompt-based setup for evaluating the efficacy of \SCEs.
Our results show that LLMs consistently struggle with generating valid \SCEs. In many cases model prediction on a \SCE does not yield the same target prediction for which the model crafted the \SCE.
Surprisingly, we find that LLMs put significant emphasis on the context---the prediction on \SCE is significantly impacted by the presence of original prediction and instructions for generating the \SCE.
Based on this empirical evidence, we argue that LLMs are still far from being able to explain their own predictions counterfactually.
Our findings add to similar insights from recent studies on other forms of self-explanations~\cite{lanham2023measuring,tanneru2024quantifying}.



Our work opens several avenues for future work. Inspired by counterfactual data augmentation~\cite{sachdeva2023catfood}, one could include the counterfactual explanation capabilities a part of the LLM training process. This inclusion may enhance the counterfactual reasoning capabilities of the LLM. Follow ups should also explore the effect of prompt tuning, specifically, model-tailored prompts for generating \SCEs. These approaches might lead to better quality \SCEs.


We limited our investigation to open source models of upto 70B parameters. Extending our analysis to larger and more recent models, \eg, DeepSeek R1 671B, and closed source models like OpenAI o3 would be an interesting avenue for future work.

Finally, our experiments were limited to relatively simple tasks: classification and mathematics problems where the solution is an integer. This limitation was mainly due to the fact that it is difficult to automatically judge validity of answers for more open-ended language generation tasks like search and information retrieval. Scaling our analysis to such tasks would require significant human-annotation resources, and is an important direction for future investigations.

\section*{Acknowledgements}

We would like to thank Justin Wong, Peter West, Raj Ammanabrolu, Alex Dimakis, Alex Xu, and Andrew Qin for their many insightful discussions about this work.

Sky Computing Lab is supported by gifts from Accenture, AMD, Anyscale, Cisco, Google, IBM, Intel, Intesa Sanpaolo, Lambda, Microsoft, NVIDIA, Samsung SDS, SAP, and VMware.

This material is based upon work supported by the National Science Foundation Graduate Research Fellowship Program under Grant No. DGE 2146752. Any opinions, findings, and conclusions or recommendations expressed in this material are those of the author(s) and do not necessarily reflect the views of the National Science Foundation.

\bibliography{main}
\bibliographystyle{plainnat}

\newpage
\appendix
\onecolumn

\section{Additional Results}
\label{appendix_experiments}

\subsection{Downstream Evaluation - Additional Details}

\subsubsection{Fine-Tuning Task Hyperparameters}
\label{appendix_hyperparameters}

We list below the fine-tuning hyperparameters that were used in common for HotpotQA, PubMedQA, GSM8K, and LCB TOP. Learning rate was determined independently for each domain via learning rate sweeps (across orders of magnitude); each sweep gave the same optimal learning rate.

\begin{itemize}
    \setlength\itemsep{-0.5em}
    \item Learning Rate: 0.001
    \item LoRA $\alpha$: 16
    \item LoRA Rank: 8
    \item LoRA Dropout: 0.0
\end{itemize}

\subsubsection{Classification Task Setup}
\label{appendix_class_train}

The generated data is used to train a BERT-based classifier \cite{bert} for 2 epochs on Enron and 9 epochs on Newsgroups. The trained models are evaluated on a static test set with $n=500$ examples for each domain.

\subsection{Core Experiment Results - All Domains}
\label{appendix_core}

This appendix contains diversity, IR, and downstream performance results for all core experiments: generation with Llama 3.1 8B and 70B Base and Instruct models, \Sys{} with Llama 3.1 models of both families, and \Sys{} with the use of GPT-4o.

Note that HotpotQA RAFT and PubMedQA RAFT diversity results present here were not presented in \cref{diversity-table} as we believe the numbers are noisy and not fit for drawing conclusions, due to the use of $100$ different simulated retrieval contexts that generation was conditioned on (as required by RAFT). Not only does this introduce noise to the similarity calculation, but the strong instruction following capability of instruct models allow them to better leverage the inherent diversity in different prompts. However, for completeness, we report the values in the tables in this appendix.

\begin{table}[H]
\caption{Average pairwise embedding cosine similarity, IR, and downstream F1 results on a randomly selected static $n=500$ subset of Enron. A BERT model with a classification head was trained for $2$ epochs on the generated data $(n = 500)$. Only pairwise similarities for generations within the same class (spam or legitimate) were calculated.}
\label{enron-core}
\vskip 0.1in
\begin{center}
\begin{small}
\begin{sc}
\begin{tabular}{c|ccc}
\toprule
Generation Method & Average Embedding Similarity & IR & Downstream F1 \\
\midrule
Llama 3.1 8B Instruct              & 0.500 & 86.0\% & 0.753 \\
Llama 3.1 70B Instruct             & 0.450 & 85.0\% & 0.848 \\
Llama 3.1 8B Base                  & 0.368 & 63.5\% & 0.790 \\
Llama 3.1 70B Base                 & 0.350 & 74.5\% & 0.819 \\
\Sys{} Llama 3.1 8B                & 0.413 & 85.0\% & \textbf{0.872} \\
\Sys{} Llama 3.1 70B               & 0.406 & 82.0\% & 0.771 \\
\Sys{} GPT-4o + Llama 3.1 8B Base  & 0.379 & 84.5\% & \textbf{0.872} \\
\Sys{} GPT-4o + Llama 3.1 70B Base & 0.356 & 88.5\% & 0.846 \\
\bottomrule
\end{tabular}
\end{sc}
\end{small}
\end{center}
\vskip -0.1in
\end{table}

\begin{table}[H]
\caption{Average pairwise embedding cosine similarity, IR, and downstream accuracy results on on a randomly selected static $n=500$ subset of Newsgroups. A BERT model with a classification head was trained for $9$ epochs on the generated data $(n = 500)$.}
\label{newsgroups-core}
\vskip 0.1in
\begin{center}
\begin{small}
\begin{sc}
\begin{tabular}{c|ccc}
\toprule
Generation Method & Average Embedding Similarity & IR & Downstream Accuracy \\
\midrule
Llama 3.1 8B Instruct              & 0.271 & 85\% & 26\% \\
Llama 3.1 70B Instruct             & 0.246 & 82\% & 30\% \\
Llama 3.1 8B Base                  & 0.155 & 58\% & 41\% \\
Llama 3.1 70B Base                 & 0.162 & 78\% & 29\% \\
\Sys{} Llama 3.1 8B                & 0.162 & 91\% & 40\% \\
\Sys{} Llama 3.1 70B               & 0.134 & 93\% & \textbf{49\%} \\
\Sys{} GPT-4o + Llama 3.1 8B Base  & 0.131 & 81\% & 44\% \\
\Sys{} GPT-4o + Llama 3.1 70B Base & 0.285 & 87\% & 47\% \\
\bottomrule
\end{tabular}
\end{sc}
\end{small}
\end{center}
\vskip -0.1in
\end{table}

\begin{table}[H]
\caption{Average pairwise embedding cosine similarity, IR, and downstream accuracy results on a randomly selected static $n=100$ subset of HotpotQA RAFT. A Llama-3.1-8B-Instruct model was fine-tuned for 4 epochs on the generated data ($n=1000$). The baseline performance of the Llama-3.1-8B-Instruct model on the evaluation set prior to any fine-tuning is reported in the first row.}
\label{hotpot-core}
\vskip 0.1in
\begin{center}
\begin{small}
\begin{sc}
\begin{tabular}{c|ccc}
\toprule
Generation Method & Average Embedding Similarity & IR & Downstream Accuracy \\
\midrule
Baseline Performance               &   --  &  --  & 33\% \\
\midrule
Llama 3.1 8B Instruct              & 0.214 & 76\% & 49\% \\
Llama 3.1 70B Instruct             & 0.216 & 90\% & 55\% \\
Llama 3.1 8B Base                  & 0.221 & 62\% & 50\% \\
Llama 3.1 70B Base                 & 0.209 & 77\% & 53\% \\
\Sys{} Llama 3.1 8B                & 0.217 & 77\% & \textbf{58\%} \\
\Sys{} Llama 3.1 70B               & 0.210 & 88\% & 54\% \\
\Sys{} GPT-4o + Llama 3.1 8B Base  & 0.214 & 78\% & 57\% \\
\Sys{} GPT-4o + Llama 3.1 70B Base & 0.205 & 89\% & 56\% \\
\bottomrule
\end{tabular}
\end{sc}
\end{small}
\end{center}
\vskip -0.1in
\end{table}

\begin{table}[H]
\caption{Average pairwise embedding cosine similarity, IR, and downstream accuracy results on a randomly selected static $n=100$ subset of PubMedQA RAFT. A Llama-3.1-8B-Instruct model was fine-tuned for 4 epochs on the generated data ($n=1000$). The baseline performance of the Llama-3.1-8B-Instruct model on the evaluation set prior to any fine-tuning is reported in the first row.}
\label{pubmed-core}
\vskip 0.1in
\begin{center}
\begin{small}
\begin{sc}
\begin{tabular}{c|ccc}
\toprule
Generation Method & Average Embedding Similarity & IR & Downstream Accuracy \\
\midrule
Baseline Performance               &   --  &  --  & 53\% \\
\midrule
Llama 3.1 8B Instruct              & 0.376 & 63\% & 72\% \\
Llama 3.1 70B Instruct             & 0.373 & 73\% & 73\% \\
Llama 3.1 8B Base                  & 0.396 & 39\% & 68\% \\
Llama 3.1 70B Base                 & 0.603 & 31\% & 73\% \\
\Sys{} Llama 3.1 8B                & 0.377 & 47\% & 73\% \\
\Sys{} Llama 3.1 70B               & 0.367 & 72\% & \textbf{79\%} \\
\Sys{} GPT-4o + Llama 3.1 8B Base  & 0.385 & 57\% & 72\% \\
\Sys{} GPT-4o + Llama 3.1 70B Base & 0.474 & 40\% & 63\% \\
\bottomrule
\end{tabular}
\end{sc}
\end{small}
\end{center}
\vskip -0.1in
\end{table}

\begin{table}[H]
\caption{Average pairwise embedding cosine similarity, IR, and downstream accuracy results on a $n=500$ randomly selected static subset of GSM8K. A Llama-3.2-1B-Instruct model was fine-tuned for 4 epochs on the generated data ($n=1000$) instead of the 3.1 8B model due to its high no-training performance. The baseline performance of the Llama-3.2-1B-Instruct model on the evaluation set prior to any fine-tuning is reported in the first row. Llama-3.1-8B-Instruct generation results are not reported due to data generation derailing.}
\label{gsm8k-core}
\vskip 0.1in
\begin{center}
\begin{small}
\begin{sc} 
\begin{tabular}{c|ccc}
\toprule
Generation Method & Average Embedding Similarity & IR & Downstream Accuracy \\
\midrule
Baseline Performance               &   --  &  --  & 21.8\% \\
\midrule
Llama 3.1 8B Instruct              & N/A & N/A & N/A \\
Llama 3.1 70B Instruct             & 0.421 & 51\% & 17.8\% \\
Llama 3.1 8B Base                  & 0.310 & 23\% & 19.2\% \\
Llama 3.1 70B Base                 & 0.313 & 44\% & 22.4\% \\
\Sys{} Llama 3.1 8B                & 0.310 & 27\% & 24.8\% \\
\Sys{} Llama 3.1 70B               & 0.305 & 54\% & 29.8\% \\
\Sys{} GPT-4o + Llama 3.1 8B Base  & 0.295 & 60\% & 32.8\% \\
\Sys{} GPT-4o + Llama 3.1 70B Base & 0.302 & 64\% & \textbf{35.8\%} \\
\bottomrule
\end{tabular}
\end{sc}
\end{small}
\end{center}
\vskip -0.1in
\end{table}

\begin{table}[H]
\caption{Average pairwise embedding cosine similarity, IR, and downstream accuracy results on the full LCB TOP set. A Llama-3.1-8B-Instruct model was fine-tuned for 4 epochs on the generated data ($n=1000$). The baseline performance of the Llama-3.1-8B-Instruct model on the evaluation set prior to any fine-tuning is reported in the first row.}
\label{lcb-core}
\vskip 0.1in
\begin{center}
\begin{small}
\begin{sc}
\begin{tabular}{c|ccc}
\toprule
Generation Method & Average Embedding Similarity & IR & Downstream Accuracy \\
\midrule
Baseline Performance               &   --  &  --  & 18.6\% \\
\midrule
Llama 3.1 8B Instruct              & 0.416 & 51\% & 20.6\% \\
Llama 3.1 70B Instruct             & 0.389 & 49\% & 26.0\% \\
Llama 3.1 8B Base                  & 0.468 & 36\% & 24.9\% \\
Llama 3.1 70B Base                 & 0.477 & 57\% & 24.4\% \\
\Sys{} Llama 3.1 8B                & 0.462 & 47\% & \textbf{28.1}\% \\
\Sys{} Llama 3.1 70B               & 0.481 & 64\% & 25.6\% \\
\Sys{} GPT-4o + Llama 3.1 8B Base  & 0.459 & 72\% & 26.7\% \\
\Sys{} GPT-4o + Llama 3.1 70B Base & 0.471 & 68\% & 27.4\% \\
\bottomrule
\end{tabular}
\end{sc}
\end{small}
\end{center}
\vskip -0.1in
\end{table}


\subsection{Independent Sampling Temperature Ablations - HotpotQA, PubMedQA, and LCB TOP}
\label{appendix_temp}

This appendix contains diversity, IR, and downstream performance results for our temperature ablation experiments. We perform a temperature sweep for Llama-3.1-8B-Instruct generation with $t=0.5, 0.7, 1.0$. We find that while adjusting the temperature can improve downstream performance, in general the gains are small relative to gains by using \Sys{}.

\begin{table}[H]
\caption{Temperature ablations with independent sampling from Llama-3.1-8B-Instruct. Average pairwise embedding cosine similarity, IR, and downstream accuracy results on a randomly selected static $n=100$ subset of HotpotQA RAFT. A Llama-3.1-8B-Instruct model was fine-tuned for 4 epochs on the generated data ($n=1000$). The baseline performance of the Llama-3.1-8B-Instruct model on the evaluation set prior to any fine-tuning is reported in the first row.}
\label{hotpot-temp}
\vskip 0.1in
\begin{center}
\begin{small}
\begin{sc}
\begin{tabular}{c|ccc}
\toprule
Generation Method & Average Embedding Similarity & IR & Downstream Accuracy \\
\midrule
Baseline Performance              &   --  &  --  & 33\% \\
\midrule
Llama 3.1 8B Instruct ($t=1.0$)   & 0.214 & 76\% & 49\% \\
Llama 3.1 8B Instruct ($t=0.7$)   & 0.216 & 77\% & 50\% \\
Llama 3.1 8B Instruct ($t=0.5$)   & 0.220 & 83\% & 42\% \\
Llama 3.1 8B Base ($t=1.0$)       & 0.221 & 62\% & 50\% \\
\Sys{} Llama 3.1 8B (All $t=0.7$) & 0.217 & 77\% & \textbf{58\%} \\
\bottomrule
\end{tabular}
\end{sc}
\end{small}
\end{center}
\vskip -0.1in
\end{table}

\begin{table}[H]
\caption{Temperature ablations with independent sampling from Llama-3.1-8B-Instruct. Average pairwise embedding cosine similarity, IR, and downstream accuracy results on a randomly selected static $n=100$ subset of PubMedQA RAFT. A Llama-3.1-8B-Instruct model was fine-tuned for 4 epochs on the generated data ($n=1000$). The baseline performance of the Llama-3.1-8B-Instruct model on the evaluation set prior to any fine-tuning is reported in the first row.}
\label{pubmed-temp}
\vskip 0.1in
\begin{center}
\begin{small}
\begin{sc}
\begin{tabular}{c|ccc}
\toprule
Generation Method & Average Embedding Similarity & IR & Downstream Accuracy \\
\midrule
Baseline Performance              &   --  &  --  & 53\% \\
\midrule
Llama 3.1 8B Instruct ($t=1.0$)   & 0.376 & 62\% & 72\% \\
Llama 3.1 8B Instruct ($t=0.7$)   & 0.375 & 71\% & 72\% \\
Llama 3.1 8B Instruct ($t=0.5$)   & 0.377 & 73\% & \textbf{75\%} \\
Llama 3.1 8B Base ($t=1.0$)       & 0.396 & 39\% & 68\% \\
\Sys{} Llama 3.1 8B (All $t=0.7$) & 0.377 & 47\% & 73\% \\
\bottomrule
\end{tabular}
\end{sc}
\end{small}
\end{center}
\vskip -0.1in
\end{table}

\begin{table}[H]
\caption{Temperature ablations with independent sampling from Llama-3.1-8B-Instruct. Average pairwise embedding cosine similarity, IR, and downstream accuracy results on the full LCB TOP set. A Llama-3.1-8B-Instruct model was fine-tuned for 4 epochs on the generated data ($n=1000$). The baseline performance of the Llama-3.1-8B-Instruct model on the evaluation set prior to any fine-tuning is reported in the first row.}
\label{lcb-temp}
\vskip 0.1in
\begin{center}
\begin{small}
\begin{sc}
\begin{tabular}{c|ccc}
\toprule
Generation Method & Average Embedding Similarity & IR & Downstream Accuracy \\
\midrule
Baseline Performance              &   --  &  --  & 18.6\% \\
\midrule
Llama 3.1 8B Instruct ($t=1.0$)   & 0.365 & 33\% & 20.1\% \\
Llama 3.1 8B Instruct ($t=0.7$)   & 0.416 & 51\% & 20.6\% \\
Llama 3.1 8B Instruct ($t=0.5$)   & 0.450 & 53\% & 22.9\% \\
Llama 3.1 8B Base ($t=1.0$)       & 0.468 & 36\% & 24.9\% \\
\Sys{} Llama 3.1 8B (All $t=0.7$) & 0.462 & 47\% & \textbf{28.1\%} \\
\bottomrule
\end{tabular}
\end{sc}
\end{small}
\end{center}
\vskip -0.1in
\end{table}


\subsection{\Sys{} First Stage Ablations - GSM8K}
\label{appendix_refine}

This appendix contains diversity, IR, and downstream performance results for our ablation replacing the first stage of \Sys{} with an instruct-tuned model, specifically Llama-3.1-70B-Instruct. We refine using Llama-3.1-70B-Instruct and GPT-4o, and investigate the change in downstream performance compared to standard \Sys{} (using Llama-3.1-70B-Base in the first stage). Note that dataset diversity is unchanged compared to direct generation from Llama-3.1-70B-Instruct, that IR improves after refinement, and that downstream performance is consistently worse than standard \Sys{}.

\begin{table}[H]
\caption{Average pairwise embedding cosine similarity, IR, and downstream accuracy results on a $n=500$ randomly selected static subset of GSM8K. A Llama-3.2-1B-Instruct model was fine-tuned for 4 epochs on the generated data ($n=1000$). The baseline performance of the Llama-3.2-1B-Instruct model on the evaluation set prior to any fine-tuning is reported in the first row.}
\label{gsm8k-refine}
\vskip 0.1in
\begin{center}
\begin{small}
\begin{sc} 
\begin{tabular}{c|ccc}
\toprule
Generation Method & Average Embedding Similarity & IR & Downstream Accuracy \\
\midrule
Baseline Performance                   &   --  &  --  & 21.8\% \\
\midrule
Llama 3.1 70B Instruct                 & 0.421 & 51\% & 22.4\% \\
Llama 3.1 70B Instruct Self-Refine     & 0.422 & 63\% & 25.4\% \\
GPT-4o Refining Llama 3.1 70B Instruct & 0.421 & 70\% & 30.8\% \\
\midrule
\Sys{} Llama 3.1 70B                   & 0.305 & 54\% & 29.8\% \\
\Sys{} GPT-4o + Llama 3.1 70B Base     & 0.302 & 64\% & \textbf{35.8\%} \\
\bottomrule
\end{tabular}
\end{sc}
\end{small}
\end{center}
\vskip -0.1in
\end{table}


\section{GSM8K Prompt Examples}
\label{appendix_prompts}

In this appendix, we provide exact prompts used for the GSM8K domains, representative of those used throughout this work. Examples are formatted for inclusion in the prompts in the ``\{examples\}'' fields, with ``EXAMPLE START'' and ``EXAMPLE END'' delimiters for the base prompt. \Sys{} uses the standard Base Prompt in the base generation step.

\subsection*{Static Few-shot Examples}

\begin{tcolorbox}[colback=blue!5!white,colframe=blue!75!black,title=Example 1]
\textbf{Question:} Alice has 20 quarters. She wants to exchange them for nickels and so she goes to the bank. After getting back from the bank, she discovers that 20\% of the nickels are iron nickels worth \$3 each. What is the total value of her money now? \\
\textbf{Answer:} A quarter is worth five nickels because .25 / .05 = \(\ll.25/.05=5\gg 5\). She gets 100 nickels from the bank because 20 x 5 = \(\ll20*5=100\gg 100\). 20 of the nickels are iron nickels because 100 x .20 = \(\ll100*.20=20\gg 20\). 80 of the nickels are regular because 100 - 20 = \(\ll100-20=80\gg 80\). The iron nickels are worth \$60 because 20 x 3 = \(\ll20*3=60\gg 60\). The regular nickels are worth \$4 because 80 x .05 = \(\ll80*.05=4\gg 4\). Her money is now worth \$64 because 60 + 4 = \(\ll60+4=64\gg 64\). \texttt{\#\#\#\#} 64
\end{tcolorbox}

\begin{tcolorbox}[colback=blue!5!white,colframe=blue!75!black,title=Example 2]
\textbf{Question:} A church has 120 members. 40\% are adults. The rest are children. How many children more children are there than adults? \\
\textbf{Answer:} There are 48 adults because 120 x .4 = \(\ll120*.4=48\gg 48\). 60\% of members are children because 100 - 40 = \(\ll100-40=60\gg 60\). There are 72 children because 120 x .6 = \(\ll120*.6=72\gg 72\). There are 24 more children than adults because 72 - 48 = \(\ll72-48=24\gg 24\). \texttt{\#\#\#\#} 24
\end{tcolorbox}

\begin{tcolorbox}[colback=blue!5!white,colframe=blue!75!black,title=Example 3]
\textbf{Question:} Lisa is looking to attempt a World Record. She has decided to try and match Joey Chestnut's record of eating 75 full hotdogs, buns included, in 10 minutes. Halfway through the time Lisa has eaten 20 hotdogs. How many hotdogs will she have to eat per minute to at least tie Joey Chestnut's record? \\
\textbf{Answer:} Joey Chestnut ate 75 hotdogs to claim the record and Lisa has eaten 20 hot dogs so far, so she still needs to eat \(75-20=\ll75-20=55\gg 55\) hotdogs to tie Joey Chestnut. Lisa has a 10-minute time period to eat the hotdogs and half the time has already passed, which means Lisa has \(10/2=\ll10/2=5\gg 5\) minutes left until the competition is over. If she needs to eat 55 hotdogs to tie Joey Chestnut and there are 5 minutes left in the competition period, then she needs to eat \(55/5=\ll55/5=11\gg 11\) hot dogs per minute to have a chance of tying for a win. \texttt{\#\#\#\#} 11
\end{tcolorbox}


\begin{tcolorbox}[colback=blue!5!white,colframe=blue!60!white,title=Base Prompt]
Here are a few examples of grade school math word problems that require performing a sequence of elementary calculations using basic arithmetic operations. A bright middle school student should be able to solve each problem. The numerical answer is provided at the end of each example after \texttt{\#\#\#\#}.\\

\{examples\}\\

EXAMPLE START
\end{tcolorbox}

\input{src/boxes/gsm8k_instruct}
\begin{tcolorbox}[colback=green!5!white,colframe=green!75!black,title=Refine Prompt]
Improve the given grade school math word problem. Edit the problem or answer to be more similar in style to the examples, and disambiguate as necessary, in addition to correcting any errors. Do not change the theme of the problem. A bright middle school student should be able to solve each problem. Problems require no concepts beyond the level of early Algebra. Note how the numerical answer is provided after \texttt{\#\#\#\#} after each brief reasoning for a question. Provide your edited problem in the following format:\\

\texttt{Question: [question]} \\
\texttt{Answer: [answer]}\\

Provide only the question and answer in the given format. Here are some examples of categories and problems on those categories:\\

\{examples\}\\

Now it's your turn. Here is the question and answer for you to edit:\\
Question:\\
\{question\}\\
Answer:\\
\{answer\}\\

Provide only the improved question and answer in the given format. Do not include any commentary or notes. Start your response with the question.
\end{tcolorbox}

\begin{tcolorbox}[colback=purple!5!white,colframe=purple!75!black,title=Sequential Prompt]
Generate a new grade school math word problem that requires performing a sequence of elementary calculations using basic arithmetic operations. A bright middle school student should be able to solve each problem. Problems require no concepts beyond the level of early Algebra. Here are the previously generated examples:\\

\{examples\}\\

Your new problem should:
\begin{enumerate}
    \item Be different from the previous examples
    \item Follow the same format and style as prior problems
\end{enumerate}

Note how the numerical answer is provided after \texttt{\#\#\#\#} after each brief reasoning for a question. Provide only the question and answer in the given format here:

\texttt{Question: [question]} \\
\texttt{Answer: [answer]}\\

Start your response with the question.
\end{tcolorbox}

\begin{tcolorbox}[colback=brown!5!white,colframe=brown!75!black,title=In One Prompt]
Provide \texttt{\{num\}} examples of problems that might be grade school math word problems that require performing a sequence of elementary calculations using basic arithmetic operations. A bright middle school student should be able to solve each problem. Problems require no concepts beyond the level of early Algebra. You must first specify the question then provide the brief reasoning and answer. Note how the numerical answer is provided after \texttt{\#\#\#\#} after each brief reasoning for a question. Provide your examples in the following format:\\

\texttt{Question: [question]} \\
\texttt{Answer: [answer]}\\

Here are some examples:\\

\{examples\}\\

Now it's your turn. Generate \texttt{\{num\}} different problems following this format. Your question should be different in content from the examples. Make sure to only provide only the question and answer. Start each example with the question. Delimit the end of an example with the phrase "END OF EXAMPLE" (all caps) on a new line.
\end{tcolorbox}

\begin{tcolorbox}[colback=teal!5!white,colframe=teal!75!black,title=Persona Prompt]
\{persona\_description\}\\

Provide an example of a grade school math word problem that requires performing a sequence of elementary calculations using basic arithmetic operations. A bright middle school student should be able to solve each problem. Problems require no concepts beyond the level of early Algebra. You must first specify the question, then provide the very concise reasoning and answer. Provide your example in the following format:\\

\texttt{Question: [question]} \\
\texttt{Answer: [answer]}\\

Provide only the question and answer in the given format. Note how the numerical answer is provided after \texttt{\#\#\#\#} after each brief reasoning for a question. Here are some examples:\\

\{examples\}\\

Now it's your turn. Start your response with the question.
\end{tcolorbox}

\begin{tcolorbox}[colback=red!5!white,colframe=red!75!black,title=Indistinguishability Rate Prompt]
System Prompt:\\

You are an expert at evaluating question and answer pairs for grade school math word problems.\\

You will be shown \{$k$\} examples. Each example consists of some context, a question, and an answer. All but one of them is generated from a high quality AI while one of them is of low quality.\\

Your task is to identify which example (1, 2, ..., \{$k$\}) appears to be of low quality. Consider factors like:

\begin{enumerate}
    \item Differing natural language patterns and flow
    \item Differing question structure, clarity, and difficulty
    \item Context and specificity
    \item Any subtle artifacts or unnatural patterns
\end{enumerate}

Analyze each example carefully and explain your reasoning. End with 'Answer: [Question Number]' where Question Number is 1, 2, ..., \{$k$\}.

---

User Prompt:\\

Here are \{$k$\} examples. One of them is of low quality. Please identify which one:\\

\{questions\}\\

Analyze each example and explain which one you think is of low quality. End with 'Answer: [Question Number]'.
\end{tcolorbox}



\end{document}
