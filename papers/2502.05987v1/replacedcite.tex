\section{Related Work}
\textit{Card-based cryptography} is a research area studying cryptographic protocols that use a deck of physical cards. There are two main lines of research in this area.

\textbf{Secure Multi-Party Computation:} This area involves protocols that can securely compute functions with private input from multiple parties, without revealing the inputs to other parties. Card-based protocols to compute various Boolean functions, including a logical AND function ____, a logical XOR function ____, a \textit{majority function} ____, and an \textit{equality function} ____, have been developed. Nishida et al. ____ proved that any $n$-variable Boolean function can be computed using $2n+6$ cards. Shinagawa and Nuida ____ proved that any Boolean function can be computed using one shuffle. 

\textbf{Zero-Knowledge Proof:} A zero-knowledge proof is an interactive protocol between a prover and a verifier, which allows the prover to show that they know a solution of a specific problem without revealing the solution itself ____. Card-based zero-knowledge proof protocols for a wide range of problems have been developed, including computational problems such as graph isomorphism ____ and pancake sorting ____, pencil puzzles such as Sudoku ____ and Nonogram ____, and mobile games such as Ball Sort Puzzle ____.

Very recently, Shinagawa et al. ____ developed a card-based player simulation protocol for a card game Old Maid, which can simulate virtual players to play the game with real players. In particular, their protocol can remove pairs of cards having the same number from each virtual player's hand without revealing the rest of its hand. This result created a new possible line of research: simulating virtual players in card games.