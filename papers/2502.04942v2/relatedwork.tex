\section{Related Work}
Many studies have documented the important role Wikipedia plays in how people find and validate information. Wikipedia content appears in over 80\% of knowledge panels and top-linked content across three different search engines (Google, Bing, and DuckDuckGo) comprising a large portion of most user-facing knowledge graph assets \cite{mcmahon_substantial_2017,vincent_deeper_2021,vincent_examining_2018}. Given the tight integration between search engines and Wikipedia, these studies shed light on how Wikipedia impacts human decision-making and influences other knowledge classification systems \cite{lerner_knowledge_2018,c_thompson_user-generated_2024,formisano_counter-misinformation_2024}. Corporate-owned platforms also depend on Wikipedia data. Many websites use Wikipedia hyperlinks and content to increase visitation, engagement, and revenue \cite{gomezmartinez_wikipedia_2022,lerner_knowledge_2018,moyer_determining_2015,vincent_examining_2018}.  Unfortunately, this economic dependency appears nonreciprocal---while Wikipedia’s open licenses make it easy for corporate sites to capitalize on its content, it does not produce migratory benefits like more viewership or edits on Wikipedia itself \cite{vincent_examining_2018}. 

In addition to Wikipedia, audiences consult a range of online services for news including news websites, apps, and social media \cite{st_aubin_news_2024,vraga_news_2021}. People who include, but do not exclusively rely on, social media in their news diets tend to have higher news media knowledge \cite{schulz_role_2024}. Among these social media sites is Reddit, a social media platform where community members share, vote, and comment on content and foster community-based engagement on topics or themes---colloquially referred to as subreddits. Previous research has explored the relationship between news coverage and Reddit engagement. \citet{gozzi_collective_2020} demonstrated that COVID-19 news coverage drove users to comment on Reddit and search for information on Wikipedia, though this effect decreased over time---probably due to media saturation. Further research on Reddit has found that fact-checked information lasts longer when a post is deemed true \cite{bond_engagement_2023}, reinforcing that users rely on Reddit's discussions when interpreting news. However, other work has scrutinized the credibility of information posted to Reddit---without editorial oversight, users may share biased viewpoints and content from known misinformation sources \cite{chipidza_ideological_2022}. Tangled into these findings on social media and news consumption is the role Wikipedia might still play in this process. Studies have found that Reddit users regularly rely on Wikipedia hyperlinks to validate information---especially within the ``Today I Learned'' subreddit \cite{moyer_determining_2015, vincent_examining_2018}. Nonetheless, access to this previous dataset is no longer feasible---since 2023 Reddit’s API is no longer available for free public use. Researchers wishing to access Reddit data must now submit an application for access to the ``Reddit4Researchers'' beta program---a new approach to partnering with data scientists to balance between data accessibility and user protection \cite{perez_reddit_2024}.

Datasets like the one we present in this paper are part of a long line of research committed to making Wikipedia data accessible and available to other social scientists. Previous papers have built Wikipedia datasets to assess the quality of content on Wikipedia \cite{das_language-agnostic_2024}; study its hyperlink structure \cite{consonni_wikilinkgraphs_2019}; understand how people interact with ‘news events’ \cite{gildersleve_between_2023}; and document platform interdependencies \cite{meier_twikiltwitter_2022}. The organizational structure of the site, its size, and the fact that Wikipedia is open access facilitate these dataset creations \cite{mitrevski_wikihist_2020}. These studies are also pushing the boundaries of Anglocentrism, creating databases that leverage ``language-agnostic'' or multilingual techniques to identify linguistic gaps, explain the informational needs of marginalized populations, and analyze how ideas propagate across the languages \cite{das_language-agnostic_2024,valentim_tracking_2021,miquel-ribe_wikipedia_2019}. Creating these datasets is no simple task, Wikipedia is a massive corpus of densely interlinked content, not just a ``ready-made data source'' \cite{gildersleve_between_2023,valentim_tracking_2021}. 

% The goal of our dataset is to expand on the accessibility of Wikipedia data while providing more research on how information ripples in and out of Wikipedia from Reddit - another prominent site of news and information. By focusing on informational structure versus monetary structure we shed light on XYZ\_.