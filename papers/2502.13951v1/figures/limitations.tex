\begin{table}[htbp]
    \centering
    \begin{tabular}{ccc}
        \includegraphics[width=0.13\textwidth,height=0.13\textwidth]{images/limitations/example_1/zebra.jpg} &
        \includegraphics[width=0.13\textwidth,height=0.13\textwidth]{images/limitations/example_1/leopard.jpg} &
        \includegraphics[width=0.13\textwidth,height=0.13\textwidth]{images/limitations/example_1/zebra_leopard.jpg} \\
         ``Animal'' & ``Fur'' &  Result \\
        \includegraphics[width=0.13\textwidth,height=0.13\textwidth]{images/limitations/example_2/person.jpg} &
        \includegraphics[width=0.13\textwidth,height=0.13\textwidth]{images/limitations/example_2/dress.jpg} &
        \includegraphics[width=0.13\textwidth,height=0.13\textwidth]{images/limitations/example_2/person_dress.jpg} \\
        ``Person'' & ``Outfit'' &  Result
        
    \end{tabular}
    \captionof{figure}{Limitations. Demonstrating the affect of concept entanglement/disentanglement on our method. (Top) When attempting to compose leopard pattern with a zebra's body, the combination may produce giraffe-like features. (Bottom) When using descriptions that only specify outfit style, our method doesn't transfer the outfit color, demonstrating the gap between CLIP-disentanglement and the common intuition. This can be resolved by using more specific concept prompts.}
    \label{fig:limitations}
\end{table}
