\section{Method}
We begin by describing our method for the simple case of creating a composition of two images.
Given a reference image \( I_{\text{ref}} \) (typically one describing the background or scene layout) and a concept image \( I_c \), we would like to output a composition depicting the concept \( c \) from \( I_c \) while obtaining the rest of the attributes from \( I_{\text{ref}} \). 

At the core of our method lies the ability to isolate and extract the ``$c$ component'' from a CLIP image embedding. Motivated by recent findings on the existence of different semantic subspaces in CLIP, we aim to construct a projection matrix \( P_c \), which will be used to project CLIP image embeddings to obtain the encoding of the specific concept ``$c$''.

\paragraph{\textbf{Constructing The Projection Matrix}} To construct a projection matrix for a concept $c$, we first gather a set of texts $t_1, \dots, t_n$, each describing an instance of the concept, with the aim of conceptually spanning its domain. To do so, we query a large language model (LLM) and simply ask it to create texts that span the concept's attribute space.
Next, using the CLIP text encoder \( CLIP_t \), we obtain embeddings for the collected texts: \( CLIP_t(t_1), \dots, CLIP_t(t_n) \). To extract the most relevant directions of this subspace, we apply Singular Value Decomposition (SVD) to the matrix of text embeddings. Let the combined embedding matrix be represented as:
\begin{equation}
E = \left[CLIP_t(t_1), \dots, CLIP_t(t_n)\right]^T,
\end{equation}
where \( E \in \mathbb{R}^{n \times d} \), \( n \) is the number of texts and \( d \) the embedding dimension. The SVD of \( E \) can be expressed as:
\begin{equation}
E = U \Sigma V^T,
\end{equation}
where \( U \in \mathbb{R}^{n \times n} \), \( \Sigma \in \mathbb{R}^{n \times d} \), and \( V \in \mathbb{R}^{d \times d} \). The rows of \( V \), also referred to as the right singular vectors, represent directions in the embedding space. While it is a common practice to normalize the embeddings before constructing the matrix \( E \), we observe improved performance when working with the unnormalized embeddings which also preserve the natural variation in the data.

Finally, we select the top \( r \) singular vectors (corresponding to the \( r \) largest singular values) from \( V \). These vectors capture the most significant variations in the subspace corresponding to concept \( c \). The projection matrix \( P_c \in \mathbb{R}^{d \times d} \) is then computed as:
\begin{equation}
P_c = V_r^T V_r,
\end{equation}
where \( V_r \in \mathbb{R}^{r \times d} \) contains the top \( r \) singular vectors. The value of \( r \) is selected empirically, and depends on the nature of the concept. In practice, the same \( r \) can often be used for most concepts, but broader concepts like ``animals" commonly benefit from utilizing more directions than specific concepts like ``dog breeds".

\begin{figure*}[t]
\begin{center}
\includegraphics[width=.85\linewidth]{fig_overview_v3.pdf}
\end{center}
\caption{
FastAtlas Overview: In each frame, we compute charts spanning fully or partially visible triangles (a), determine texture space bounding boxes for the visible portions of the view-space projections of each chart, and tightly pack these boxes into atlases (b, here $2K \times 2K$). We simultaneously bijectively parameterize and shade the charts into their atlas boxes, obtaining high quality texture space shading (c), and use this shading to render the shaded frames (d).}
\label{fig:overview}
\label{fig:alg_overview}
\end{figure*}

\section{Overview}
\label{sec:overview}
Our work has two core contributions: a real-time, GPU-based algorithm for tight packing of general parameterized charts into compact atlases; and a real-time TSS method that
utilizes this packing.  

\paragraph*{FastAtlas Packing.}
FastAtlas runs entirely on the GPU as a series of compute shaders. It takes the bounding boxes of parameterized charts as input, and packs them into an atlas (Fig~\ref{fig:overview}b, Sec.~\ref{sec:pack}). As such, the only input it requires are the dimensions of the bounding boxes.
Its outputs are deterministic; identical input charts are packed into identical atlases. This is critical for TSS and similar applications, as it ensures that consecutive frames taken from the same camera view have the same shading. Even minute shading differences across such frames can cause sampling jitter, leading to undesirable flicker \cite{baker2012rock}. 
While prior methods such as \cite{mueller2018shading,hladky2019tessellated,hladky2021snakebinning,Neff2022MSA} cap the dimensions of the charts that can be packed as-is for a given atlas size, and scale down all charts that exceed these dimensions, we scale all charts by the same factor, and do so only when strictly necessary to achieve packing success (Figs~\ref{fig:atlas},~\ref{fig:sas_issues}). 

\paragraph*{TSS using FastAtlas.}
Our end-to-end TSS atlas generation method combines the packing method above with a novel approach for computing seamless per-frame charts. 
We define our charts as the connected components of the visible surfaces in each frame (Fig.~\ref{fig:overview}a), and efficiently compute them using a parallel union-find algorithm (Sec.~\ref{sec:visible}). Since the boundaries of these charts coincide with the contours of the rendered surface, they are {\em invisible} to the viewer. This approach 
eliminates the artifacts caused by shading discontinuities along visible seams (Fig.~\ref{fig:seams}). 

\begin{parWithWrapFigure}
\begin{wrapfigure}{l}{.27\columnwidth}%
\includegraphics[width=\linewidth]{fig_inset_view_plane.pdf}%
\end{wrapfigure}
We bijectively parametrize the {\em visible portions} of our charts by projecting them to view space (inset). This maps a constant number of texels to each pixel in the final rendered output, evenly distributing residual undersampling error across all image pixels. While conceptually straightforward, efficiently parameterizing charts containing partially visible triangles using viewspace projection is non-trivial, as the visible portions may no longer be triangular (e.g. green triangle in the inset); applying naive projection to triangles with vertices behind the camera may produce ill-posed results. Clipping triangles before projection is both computationally expensive and significantly complicates downstream operations. We avoid explicit clipping by observing that all that is required for atlas packing is the dimensions of, potentially conservative, bounding boxes of these projected visible portions. We compute such bounding boxes without explicit chart clipping by adapting a conservative screen coverage estimator \shortcite{Blinn:CalculatingScreenCoverage} (Sec.~\ref{sec:box}). We then pack the computed boxes using FastAtlas. 
\end{parWithWrapFigure}

Finally, we shade the visible portion of each chart into its corresponding atlas bounding box (Fig~\ref{fig:overview}c). 
The resulting texture is then used during rasterization as a standard texture map (Fig. ~\ref{fig:overview}d). 
Our framework is compatible with all existing approaches for texture space shading, including forward shading, raytraced illumination, or deferred shading in texture space \cite{baker:2016}. In the examples shown, we use the standard forward shading based rendering pipeline included in the G3D Innovation Engine \cite{G3D17}, a commercial grade renderer.


\paragraph{\textbf{Image Composition}} We aim to create a composite embedding that jointly encodes the concept \( c \) from \( I_c \) while preserving the remaining attributes of \( I_{\text{ref}} \). To achieve this, we simply replace the concept-space projection of \( I_{\text{ref}} \) with the projection of \( I_c \). More concretely, the composite embedding is given by:
\begin{equation}
\mathbf{e}_{\text{comp}} = \mathbf{e}_{\text{ref}} - P_c \mathbf{e}_{\text{ref}} + P_c \mathbf{e}_c,
\end{equation}
where \( \mathbf{e}_{\text{ref}} \) and \( \mathbf{e}_c \) are the CLIP embeddings of \( I_{\text{ref}} \) and \( I_c \), respectively. This composite embedding \( \mathbf{e}_{\text{comp}} \) is then passed to the IP-Adapter to generate the final composed image \( I_{\text{comp}} \), combining the attributes of \( I_{\text{ref}} \) with the concept instance extracted from \( I_c \).

\paragraph{\textbf{Generalization to Multiple Concepts}} The same approach can be extended to \( K \) concepts, \( \{c_1, c_2, \dots, c_K\} \), with corresponding projection matrices \( \{P_{c_1}, P_{c_2}, \dots, P_{c_K}\} \) and concept images \( \{I_{c_1}, I_{c_2}, \dots, I_{c_K}\} \). Here, the composed embedding is constructed by subtracting the concept-space projections of the reference embedding and adding the matching concept embedding from each source image:
\begin{equation}
\mathbf{e}_{\text{comp}} = \mathbf{e}_{\text{ref}} - \sum_{k=1}^K P_{c_k} \mathbf{e}_{\text{ref}} + \sum_{k=1}^K P_{c_k} \mathbf{e}_{c_k},
\end{equation}


where \( \mathbf{e}_{c_k} \) represents the embedding of the concept image \( I_{c_k} \). Note that we do not subtract the projection of each concept on the subspaces of the other concepts as we find empirically that this makes the compositions more sensitive to the choice of the number of singular vectors $r_c$ used for each concept.


    
    


\begin{figure}[t]
    \centering
    \hspace*{-6mm}
    \begin{tabular*}{\linewidth}{m{-4cm}m{2.3cm}m{2.3cm}m{2.3cm}}
    \quad & \hspace{3mm} \fontsize{8}{8}\selectfont Groundtruth & \hspace{6mm} \fontsize{8}{8}\selectfont Baseline & \hspace{7mm} \fontsize{8}{8}\selectfont \coolname{}\\
    \rotatebox{90}{\fontsize{7}{10}\selectfont SalsaNext~\cite{cortinhal2020salsanext}}&\includegraphics[width=0.16\textwidth]{pics/qualitative/salsanext/scene1_gt.png}&\includegraphics[width=0.16\textwidth]{pics/qualitative/salsanext/scene1_baseline.png}&\includegraphics[width=0.16\textwidth]{pics/qualitative/salsanext/scene1_flares.png}\\  
    \rotatebox{90}{\fontsize{7}{10}\selectfont FIDNet~\cite{zhao2021fidnet}}&\includegraphics[width=0.15\textwidth]{pics/qualitative/fidnet/scene1_gt.png}&\includegraphics[width=0.15\textwidth]{pics/qualitative/fidnet/scene1_baseline.png}&\includegraphics[width=0.15\textwidth]{pics/qualitative/fidnet/scene1_flares.png}\\
    \rotatebox{90}{\fontsize{7}{10}\selectfont CENet~\cite{cheng2022cenet}}&\includegraphics[width=0.16\textwidth]{pics/qualitative/cenet/scene1_gt.png}&\includegraphics[width=0.16\textwidth]{pics/qualitative/cenet/scene1_baseline.png}&\includegraphics[width=0.16\textwidth]{pics/qualitative/cenet/scene1_flares.png}\\  
    \rotatebox{90}{\fontsize{7}{10}\selectfont RangeViT~\cite{ando2023rangevit}}&\includegraphics[width=0.16\textwidth]{pics/qualitative/rangevit/scene1_gt.png}&\includegraphics[width=0.16\textwidth]{pics/qualitative/rangevit/scene1_baseline.png}&\includegraphics[width=0.16\textwidth]{pics/qualitative/rangevit/scene1_flares.png}\\ 
    \end{tabular*}
    
    \vspace{-0.3mm}
    \caption{\textbf{Qualitative results on SemanticKITTI}\cite{behley2019semantickitti} Points in \textcolor{incorrect}{red} and \textcolor{correct}{gray} represent incorrect and correct predictions, respectively. $^\star$More examples are provided in the supplementary material.}
    \vspace{-6mm}
    \label{fig:qualitative_results}
\end{figure}


\paragraph{\textbf{Implementation Details}} We implement our method on top of a pre-trained SDXL~\cite{podell2024sdxl} model using an IP-Adapter~\cite{ye2023ipadapter} encoder based on OpenCLIP-ViT-H-14~\cite{ilharco_gabriel_2021_5143773} (ip-adapter\_sdxl\_vit-h). To generate concept variation descriptions, we used GPT-4o~\cite{OpenAI2022ChatGPT} and asked for $150$ prompts. In cases that require higher subspace dimensions (\eg object insertion) we instead generated $500$ prompts.
