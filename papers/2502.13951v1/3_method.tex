\section{Method}
We begin by describing our method for the simple case of creating a composition of two images.
Given a reference image \( I_{\text{ref}} \) (typically one describing the background or scene layout) and a concept image \( I_c \), we would like to output a composition depicting the concept \( c \) from \( I_c \) while obtaining the rest of the attributes from \( I_{\text{ref}} \). 

At the core of our method lies the ability to isolate and extract the ``$c$ component'' from a CLIP image embedding. Motivated by recent findings on the existence of different semantic subspaces in CLIP, we aim to construct a projection matrix \( P_c \), which will be used to project CLIP image embeddings to obtain the encoding of the specific concept ``$c$''.

\paragraph{\textbf{Constructing The Projection Matrix}} To construct a projection matrix for a concept $c$, we first gather a set of texts $t_1, \dots, t_n$, each describing an instance of the concept, with the aim of conceptually spanning its domain. To do so, we query a large language model (LLM) and simply ask it to create texts that span the concept's attribute space.
Next, using the CLIP text encoder \( CLIP_t \), we obtain embeddings for the collected texts: \( CLIP_t(t_1), \dots, CLIP_t(t_n) \). To extract the most relevant directions of this subspace, we apply Singular Value Decomposition (SVD) to the matrix of text embeddings. Let the combined embedding matrix be represented as:
\begin{equation}
E = \left[CLIP_t(t_1), \dots, CLIP_t(t_n)\right]^T,
\end{equation}
where \( E \in \mathbb{R}^{n \times d} \), \( n \) is the number of texts and \( d \) the embedding dimension. The SVD of \( E \) can be expressed as:
\begin{equation}
E = U \Sigma V^T,
\end{equation}
where \( U \in \mathbb{R}^{n \times n} \), \( \Sigma \in \mathbb{R}^{n \times d} \), and \( V \in \mathbb{R}^{d \times d} \). The rows of \( V \), also referred to as the right singular vectors, represent directions in the embedding space. While it is a common practice to normalize the embeddings before constructing the matrix \( E \), we observe improved performance when working with the unnormalized embeddings which also preserve the natural variation in the data.

Finally, we select the top \( r \) singular vectors (corresponding to the \( r \) largest singular values) from \( V \). These vectors capture the most significant variations in the subspace corresponding to concept \( c \). The projection matrix \( P_c \in \mathbb{R}^{d \times d} \) is then computed as:
\begin{equation}
P_c = V_r^T V_r,
\end{equation}
where \( V_r \in \mathbb{R}^{r \times d} \) contains the top \( r \) singular vectors. The value of \( r \) is selected empirically, and depends on the nature of the concept. In practice, the same \( r \) can often be used for most concepts, but broader concepts like ``animals" commonly benefit from utilizing more directions than specific concepts like ``dog breeds".

\section{Overview}

\revision{In this section, we first explain the foundational concept of Hausdorff distance-based penetration depth algorithms, which are essential for understanding our method (Sec.~\ref{sec:preliminary}).
We then provide a brief overview of our proposed RT-based penetration depth algorithm (Sec.~\ref{subsec:algo_overview}).}



\section{Preliminaries }
\label{sec:Preliminaries}

% Before we introduce our method, we first overview the important basics of 3D dynamic human modeling with Gaussian splatting. Then, we discuss the diffusion-based 3d generation techniques, and how they can be applied to human modeling.
% \ZY{I stopp here. TBC.}
% \subsection{Dynamic human modeling with Gaussian splatting}
\subsection{3D Gaussian Splatting}
3D Gaussian splatting~\cite{kerbl3Dgaussians} is an explicit scene representation that allows high-quality real-time rendering. The given scene is represented by a set of static 3D Gaussians, which are parameterized as follows: Gaussian center $x\in {\mathbb{R}^3}$, color $c\in {\mathbb{R}^3}$, opacity $\alpha\in {\mathbb{R}}$, spatial rotation in the form of quaternion $q\in {\mathbb{R}^4}$, and scaling factor $s\in {\mathbb{R}^3}$. Given these properties, the rendering process is represented as:
\begin{equation}
  I = Splatting(x, c, s, \alpha, q, r),
  \label{eq:splattingGA}
\end{equation}
where $I$ is the rendered image, $r$ is a set of query rays crossing the scene, and $Splatting(\cdot)$ is a differentiable rendering process. We refer readers to Kerbl et al.'s paper~\cite{kerbl3Dgaussians} for the details of Gaussian splatting. 



% \ZY{I would suggest move this part to the method part.}
% GaissianAvatar is a dynamic human generation model based on Gaussian splitting. Given a sequence of RGB images, this method utilizes fitted SMPLs and sampled points on its surface to obtain a pose-dependent feature map by a pose encoder. The pose-dependent features and a geometry feature are fed in a Gaussian decoder, which is employed to establish a functional mapping from the underlying geometry of the human form to diverse attributes of 3D Gaussians on the canonical surfaces. The parameter prediction process is articulated as follows:
% \begin{equation}
%   (\Delta x,c,s)=G_{\theta}(S+P),
%   \label{eq:gaussiandecoder}
% \end{equation}
%  where $G_{\theta}$ represents the Gaussian decoder, and $(S+P)$ is the multiplication of geometry feature S and pose feature P. Instead of optimizing all attributes of Gaussian, this decoder predicts 3D positional offset $\Delta{x} \in {\mathbb{R}^3}$, color $c\in\mathbb{R}^3$, and 3D scaling factor $ s\in\mathbb{R}^3$. To enhance geometry reconstruction accuracy, the opacity $\alpha$ and 3D rotation $q$ are set to fixed values of $1$ and $(1,0,0,0)$ respectively.
 
%  To render the canonical avatar in observation space, we seamlessly combine the Linear Blend Skinning function with the Gaussian Splatting~\cite{kerbl3Dgaussians} rendering process: 
% \begin{equation}
%   I_{\theta}=Splatting(x_o,Q,d),
%   \label{eq:splatting}
% \end{equation}
% \begin{equation}
%   x_o = T_{lbs}(x_c,p,w),
%   \label{eq:LBS}
% \end{equation}
% where $I_{\theta}$ represents the final rendered image, and the canonical Gaussian position $x_c$ is the sum of the initial position $x$ and the predicted offset $\Delta x$. The LBS function $T_{lbs}$ applies the SMPL skeleton pose $p$ and blending weights $w$ to deform $x_c$ into observation space as $x_o$. $Q$ denotes the remaining attributes of the Gaussians. With the rendering process, they can now reposition these canonical 3D Gaussians into the observation space.



\subsection{Score Distillation Sampling}
Score Distillation Sampling (SDS)~\cite{poole2022dreamfusion} builds a bridge between diffusion models and 3D representations. In SDS, the noised input is denoised in one time-step, and the difference between added noise and predicted noise is considered SDS loss, expressed as:

% \begin{equation}
%   \mathcal{L}_{SDS}(I_{\Phi}) \triangleq E_{t,\epsilon}[w(t)(\epsilon_{\phi}(z_t,y,t)-\epsilon)\frac{\partial I_{\Phi}}{\partial\Phi}],
%   \label{eq:SDSObserv}
% \end{equation}
\begin{equation}
    \mathcal{L}_{\text{SDS}}(I_{\Phi}) \triangleq \mathbb{E}_{t,\epsilon} \left[ w(t) \left( \epsilon_{\phi}(z_t, y, t) - \epsilon \right) \frac{\partial I_{\Phi}}{\partial \Phi} \right],
  \label{eq:SDSObservGA}
\end{equation}
where the input $I_{\Phi}$ represents a rendered image from a 3D representation, such as 3D Gaussians, with optimizable parameters $\Phi$. $\epsilon_{\phi}$ corresponds to the predicted noise of diffusion networks, which is produced by incorporating the noise image $z_t$ as input and conditioning it with a text or image $y$ at timestep $t$. The noise image $z_t$ is derived by introducing noise $\epsilon$ into $I_{\Phi}$ at timestep $t$. The loss is weighted by the diffusion scheduler $w(t)$. 
% \vspace{-3mm}

\subsection{Overview of the RTPD Algorithm}\label{subsec:algo_overview}
Fig.~\ref{fig:Overview} presents an overview of our RTPD algorithm.
It is grounded in the Hausdorff distance-based penetration depth calculation method (Sec.~\ref{sec:preliminary}).
%, similar to that of Tang et al.~\shortcite{SIG09HIST}.
The process consists of two primary phases: penetration surface extraction and Hausdorff distance calculation.
We leverage the RTX platform's capabilities to accelerate both of these steps.

\begin{figure*}[t]
    \centering
    \includegraphics[width=0.8\textwidth]{Image/overview.pdf}
    \caption{The overview of RT-based penetration depth calculation algorithm overview}
    \label{fig:Overview}
\end{figure*}

The penetration surface extraction phase focuses on identifying the overlapped region between two objects.
\revision{The penetration surface is defined as a set of polygons from one object, where at least one of its vertices lies within the other object. 
Note that in our work, we focus on triangles rather than general polygons, as they are processed most efficiently on the RTX platform.}
To facilitate this extraction, we introduce a ray-tracing-based \revision{Point-in-Polyhedron} test (RT-PIP), significantly accelerated through the use of RT cores (Sec.~\ref{sec:RT-PIP}).
This test capitalizes on the ray-surface intersection capabilities of the RTX platform.
%
Initially, a Geometry Acceleration Structure (GAS) is generated for each object, as required by the RTX platform.
The RT-PIP module takes the GAS of one object (e.g., $GAS_{A}$) and the point set of the other object (e.g., $P_{B}$).
It outputs a set of points (e.g., $P_{\partial B}$) representing the penetration region, indicating their location inside the opposing object.
Subsequently, a penetration surface (e.g., $\partial B$) is constructed using this point set (e.g., $P_{\partial B}$) (Sec.~\ref{subsec:surfaceGen}).
%
The generated penetration surfaces (e.g., $\partial A$ and $\partial B$) are then forwarded to the next step. 

The Hausdorff distance calculation phase utilizes the ray-surface intersection test of the RTX platform (Sec.~\ref{sec:RT-Hausdorff}) to compute the Hausdorff distance between two objects.
We introduce a novel Ray-Tracing-based Hausdorff DISTance algorithm, RT-HDIST.
It begins by generating GAS for the two penetration surfaces, $P_{\partial A}$ and $P_{\partial B}$, derived from the preceding step.
RT-HDIST processes the GAS of a penetration surface (e.g., $GAS_{\partial A}$) alongside the point set of the other penetration surface (e.g., $P_{\partial B}$) to compute the penetration depth between them.
The algorithm operates bidirectionally, considering both directions ($\partial A \to \partial B$ and $\partial B \to \partial A$).
The final penetration depth between the two objects, A and B, is determined by selecting the larger value from these two directional computations.

%In the Hausdorff distance calculation step, we compute the Hausdorff distance between given two objects using a ray-surface-intersection test. (Sec.~\ref{sec:RT-Hausdorff}) Initially, we construct the GAS for both $\partial A$ and $\partial B$ to utilize the RT-core effectively. The RT-based Hausdorff distance algorithms then determine the Hausdorff distance by processing the GAS of one object (e.g. $GAS_{\partial A}$) and set of the vertices of the other (e.g. $P_{\partial B}$). Following the Hausdorff distance definition (Eq.~\ref{equation:hausdorff_definition}), we compute the Hausdorff distance to both directions ($\partial A \to \partial B$) and ($\partial B \to \partial A$). As a result, the bigger one is the final Hausdorff distance, and also it is the penetration depth between input object $A$ and $B$.


%the proposed RT-based penetration depth calculation pipeline.
%Our proposed methods adopt Tang's Hausdorff-based penetration depth methods~\cite{SIG09HIST}. The pipeline is divided into the penetration surface extraction step and the Hausdorff distance calculation between the penetration surface steps. However, since Tang's approach is not suitable for the RT platform in detail, we modified and applied it with appropriate methods.

%The penetration surface extraction step is extracting overlapped surfaces on other objects. To utilize the RT core, we use the ray-intersection-based PIP(Point-In-Polygon) algorithms instead of collision detection between two objects which Tang et al.~\cite{SIG09HIST} used. (Sec.~\ref{sec:RT-PIP})
%RT core-based PIP test uses a ray-surface intersection test. For purpose this, we generate the GAS(Geometry Acceleration Structure) for each object. RT core-based PIP test takes the GAS of one object (e.g. $GAS_{A}$) and a set of vertex of another one (e.g. $P_{B}$). Then this computes the penetrated vertex set of another one (e.g. $P_{\partial B}$). To calculate the Hausdorff distance, these vertex sets change to objects constructed by penetrated surface (e.g. $\partial B$). Finally, the two generated overlapped surface objects $\partial A$ and $\partial B$ are used in the Hausdorff distance calculation step.

\paragraph{\textbf{Image Composition}} We aim to create a composite embedding that jointly encodes the concept \( c \) from \( I_c \) while preserving the remaining attributes of \( I_{\text{ref}} \). To achieve this, we simply replace the concept-space projection of \( I_{\text{ref}} \) with the projection of \( I_c \). More concretely, the composite embedding is given by:
\begin{equation}
\mathbf{e}_{\text{comp}} = \mathbf{e}_{\text{ref}} - P_c \mathbf{e}_{\text{ref}} + P_c \mathbf{e}_c,
\end{equation}
where \( \mathbf{e}_{\text{ref}} \) and \( \mathbf{e}_c \) are the CLIP embeddings of \( I_{\text{ref}} \) and \( I_c \), respectively. This composite embedding \( \mathbf{e}_{\text{comp}} \) is then passed to the IP-Adapter to generate the final composed image \( I_{\text{comp}} \), combining the attributes of \( I_{\text{ref}} \) with the concept instance extracted from \( I_c \).

\paragraph{\textbf{Generalization to Multiple Concepts}} The same approach can be extended to \( K \) concepts, \( \{c_1, c_2, \dots, c_K\} \), with corresponding projection matrices \( \{P_{c_1}, P_{c_2}, \dots, P_{c_K}\} \) and concept images \( \{I_{c_1}, I_{c_2}, \dots, I_{c_K}\} \). Here, the composed embedding is constructed by subtracting the concept-space projections of the reference embedding and adding the matching concept embedding from each source image:
\begin{equation}
\mathbf{e}_{\text{comp}} = \mathbf{e}_{\text{ref}} - \sum_{k=1}^K P_{c_k} \mathbf{e}_{\text{ref}} + \sum_{k=1}^K P_{c_k} \mathbf{e}_{c_k},
\end{equation}


where \( \mathbf{e}_{c_k} \) represents the embedding of the concept image \( I_{c_k} \). Note that we do not subtract the projection of each concept on the subspaces of the other concepts as we find empirically that this makes the compositions more sensitive to the choice of the number of singular vectors $r_c$ used for each concept.


\begin{figure*}[!htb]
    \centering
    
    \begin{tikzpicture}
        \matrix (m1) [matrix of nodes,
            nodes={draw, minimum width=2cm, minimum height=1cm, inner sep=0pt, line width=1.5pt},
            row sep=0.45cm,
            column sep=0.443cm
        ] {
            \includegraphics[width=2.2cm,height=2.2cm]{images/qualitative_results/3_way/person_age_emotion/person.jpg} & \includegraphics[width=2.2cm,height=2.2cm]{images/qualitative_results/3_way/person_age_emotion/age.jpg} & \includegraphics[width=2.2cm,height=2.2cm]{images/qualitative_results/3_way/person_age_emotion/emotion.jpg} & \includegraphics[width=2.2cm,height=2.2cm]{images/qualitative_results/3_way/person_age_emotion/result_1.jpg} & \includegraphics[width=2.2cm,height=2.2cm]{images/qualitative_results/3_way/person_age_emotion/result_2.jpg} & \includegraphics[width=2.2cm,height=2.2cm]{images/qualitative_results/3_way/person_age_emotion/result_3.jpg} \\
            \includegraphics[width=2.2cm,height=2.2cm]{images/qualitative_results/3_way/person_clothes_color/person.jpg} & \includegraphics[width=2.2cm,height=2.2cm]{images/qualitative_results/3_way/person_clothes_color/outfit.jpg} & \includegraphics[width=2.2cm,height=2.2cm]{images/qualitative_results/3_way/person_clothes_color/outfit_color.jpg} & \includegraphics[width=2.2cm,height=2.2cm]{images/qualitative_results/3_way/person_clothes_color/result_1.jpg} & \includegraphics[width=2.2cm,height=2.2cm]{images/qualitative_results/3_way/person_clothes_color/result_2.jpg} & \includegraphics[width=2.2cm,height=2.2cm]{images/qualitative_results/3_way/person_clothes_color/result_3.jpg} \\
        };

        \path (m1-1-1) -- (m1-1-2) node[midway] {\Large $+$};
        \path (m1-1-2) -- (m1-1-3) node[midway] {\Large $+$};
        \path (m1-1-3) -- (m1-1-4) node[midway] {\Large $=$};

        \path (m1-2-1) -- (m1-2-2) node[midway] {\Large $+$};
        \path (m1-2-2) -- (m1-2-3) node[midway] {\Large $+$};
        \path (m1-2-3) -- (m1-2-4) node[midway] {\Large $=$};

        \node[below=-0.05cm of m1-1-1] {\small ``Person''};
        \node[below=-0.05cm of m1-1-2] {\small ``Age''};
        \node[below=-0.05cm of m1-1-3] {\small ``Expression''};
        \node[below=-0.05cm of m1-1-5] {\small Results};

        \node[below=-0.05cm of m1-2-1] {\small ``Person''};
        \node[below=-0.05cm of m1-2-2] {\small ``Outfit''};
        \node[below=-0.05cm of m1-2-3] {\small ``Color''};
        \node[below=-0.05cm of m1-2-5] {\small Results};
        
        \node[below=0.6cm of m1] (bottom-container) {
            \begin{tikzpicture}
                \matrix (m2) [
                    matrix of nodes,
                    nodes={draw, minimum width=2cm, minimum height=1cm, inner sep=0pt, line width=1.5pt},
                    row sep=0.3cm,
                    column sep=0.3cm
                ] {
                    \includegraphics[width=2.2cm,height=2.2cm]{images/qualitative_results/2_way/fruit_arrangement/scene.jpg} & \includegraphics[width=2.2cm,height=2.2cm]{images/qualitative_results/2_way/fruit_arrangement/fruit_1.jpg} & \includegraphics[width=2.2cm,height=2.2cm]{images/qualitative_results/2_way/fruit_arrangement/result_1.jpg} \\
                    \includegraphics[width=2.2cm,height=2.2cm]{images/qualitative_results/2_way/fruit_arrangement/scene.jpg} & \includegraphics[width=2.2cm,height=2.2cm]{images/qualitative_results/2_way/fruit_arrangement/fruit_2.jpg} & \includegraphics[width=2.2cm,height=2.2cm]{images/qualitative_results/2_way/fruit_arrangement/result_2.jpg} \\
                };

        
                \node[below=-0.05cm of m2-2-1] {\small ``Layout''};
                \node[below=-0.05cm of m2-2-2] {\small ``Fruit''};
                \node[below=-0.05cm of m2-2-3] {\small Result};

                \path (m2-1-1) -- (m2-1-2) node[midway, yshift=5.0pt] {\Large $+$};
                \path (m2-1-2) -- (m2-1-3) node[midway, yshift=5.0pt] {\Large $=$};
                \path (m2-2-1) -- (m2-2-2) node[midway, yshift=5.0pt] {\Large $+$};
                \path (m2-2-2) -- (m2-2-3) node[midway, yshift=5.0pt] {\Large $=$};
                
                \matrix (m3) [
                    matrix of nodes,
                    nodes={draw, minimum width=2cm, minimum height=1cm, inner sep=0pt, line width=1.5pt},
                    row sep=0.3cm,
                    column sep=0.3cm,
                    right=0.8cm of m2
                ] {
                    \includegraphics[width=2.2cm,height=2.2cm]{images/qualitative_results/2_way/patterns/object.jpg} & \includegraphics[width=2.2cm,height=2.2cm]{images/qualitative_results/2_way/patterns/pattern_1.jpg} & \includegraphics[width=2.2cm,height=2.2cm]{images/qualitative_results/2_way/patterns/result_1.jpg} \\
                    \includegraphics[width=2.2cm,height=2.2cm]{images/qualitative_results/2_way/patterns/object.jpg} & \includegraphics[width=2.2cm,height=2.2cm]{images/qualitative_results/2_way/patterns/pattern_2.jpg} & \includegraphics[width=2.2cm,height=2.2cm]{images/qualitative_results/2_way/patterns/result_2.jpg} \\
                };

                \node[below=-0.05cm of m3-2-1] {\small ``Object''};
                \node[below=-0.05cm of m3-2-2] {\small ``Pattern''};
                \node[below=-0.05cm of m3-2-3] {\small Result};

                \path (m3-1-1) -- (m3-1-2) node[midway, yshift=5.0pt] {\Large $+$};
                \path (m3-1-2) -- (m3-1-3) node[midway, yshift=5.0pt] {\Large $=$};
                \path (m3-2-1) -- (m3-2-2) node[midway, yshift=5.0pt] {\Large $+$};
                \path (m3-2-2) -- (m3-2-3) node[midway, yshift=5.0pt] {\Large $=$};
                
                \begin{pgfonlayer}{background}
                    \node[draw, rounded corners, inner sep=0.3cm, fit=(m2), line width=1pt] {};
                    \node[draw, rounded corners, inner sep=0.3cm, fit=(m3), line width=1pt] {};
                \end{pgfonlayer}
            \end{tikzpicture}
        };
        
        \begin{pgfonlayer}{background}
            \node[draw, rounded corners, inner sep=0.3cm, fit=(m1), line width=1pt] {};
        \end{pgfonlayer}
    \end{tikzpicture}
    \caption{Examples of visual concept compositions enabled by IP-Composer. Our method can seamlessly tackle texture-based tasks like colorization and pattern changes, but also convey layouts or modify object-level content. }
    \label{fig:our_results}
\end{figure*}


\paragraph{\textbf{Implementation Details}} We implement our method on top of a pre-trained SDXL~\cite{podell2024sdxl} model using an IP-Adapter~\cite{ye2023ipadapter} encoder based on OpenCLIP-ViT-H-14~\cite{ilharco_gabriel_2021_5143773} (ip-adapter\_sdxl\_vit-h). To generate concept variation descriptions, we used GPT-4o~\cite{OpenAI2022ChatGPT} and asked for $150$ prompts. In cases that require higher subspace dimensions (\eg object insertion) we instead generated $500$ prompts.
