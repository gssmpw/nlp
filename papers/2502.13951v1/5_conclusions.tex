\section{Limitation}
The use of 3D-printed PLA for structural components improves improving ease of assembly and reduces weight and cost, yet it causes deformation under heavy load, which can diminish end-effector precision. Using metal, such as aluminum, would remedy this problem. Additionally, \robot relies on integrated joint relative encoders, requiring manual initialization in a fixed joint configuration each time the system is powered on. Using absolute joint encoders could significantly improve accuracy and ease of use, although it would increase the overall cost. 

%Reliance on commercially available actuators simplifies integration but imposes constraints on control frequency and customization, further limiting the potential for tailored performance improvements.

% The 6 DoF configuration provides sufficient mobility for most tasks; however, certain bimanual operations could benefit from an additional degree of freedom to handle complex joint constraints more effectively. Furthermore, the limited torque density of commercially available proprioceptive actuators restricts the payload and torque output, making the system less suitability for handling heavier loads or high-torque applications. 

The 6 DoF configuration of the arm provides sufficient mobility for single-arm manipulation tasks, yet it shows a limitation in certain bimanual manipulation problems. Specifically, when \robot holds onto a rigid object with both hands, each arm loses 1 DoF because the hands are fixed to the object during grasping. This leads to an underactuated kinematic chain which has a limited mobility in 3D space. We can achieve more mobility by letting the object slip inside the grippers, yet this renders the grasp less robust and simulation difficult. Therefore, we anticipate that designing a lightweight 3 DoF wrist in place of the current 2 DoF wrist allows a more diverse repertoire of manipulation in bimanual tasks.

Finally, the limited torque density of commercially available proprioceptive actuators restricts the performance. Currently, all of our actuators feature a 1:10 gear ratio, so \robot can handle up to 2.5 kg of payload. To handle a heavier object and manipulate it with higher torque, we expect the actuator to have 1:20$\sim$30 gear ratio, but it is difficult to find an off-the-shelf product that meets our requirements. Customizing the actuator to increase the torque density while minimizing the weight will enable \robot to move faster and handle more diverse objects.

%These constraints highlight opportunities for improvement in future iterations, including alternative materials for enhanced rigidity, custom actuator designs for higher control precision and torque density, the adoption of absolute joint encoders, and optimized configurations to balance dexterity and weight.



\section{Limitations}

While our approach is typically more general than current training-based approaches, it still has limitations. One limitation arises from surprising entanglements in the CLIP and diffusion feature spaces. For example, when attempting to combine a zebra's body with a leopard fur pattern (\cref{fig:limitations} (top)), the diffusion model tends to produce animals with the head of a giraffe, even though no giraffe appears in either input image. We hypothesize that this may be related to the tendency of diffusion models to represent some concepts as a composition of more basic visual components~\citep{chefer2023hidden}, but leave further investigation to future work.

On the other hand, some concepts may be \textit{more} disentangled in CLIP-space than intuitively expected. For example, outfit types and colors are disentangled in CLIP-space, hence, an ``outfit'' subspace spanned with descriptions of different types of outfits (``dress'', ``tuxedo''...) will not preserve outfit colors (\cref{fig:limitations} (bottom)). However, this can be easily amended by also specifying colors in the spanning texts (``\textit{red} dress'', ``\textit{blue} tuxedo''...).



Finally, we note that IP-Adapter itself is limited in the level of detail captured from the input image. Hence, our approach will not be sufficient for capturing delicate details such as exact identities. Stronger encoders may achieve higher fidelity, but it is not clear that our embedding-space projections would generalize to more complex feature spaces.

\section{Conclusions}

We presented IP-Composer, a training-free method that allows a user to compose novel images from visual concepts derived through a set of input images. To do so, our approach uses a CLIP-based IP-Adapter, leveraging their joint disentangled subspace structure. Through this approach, we achieve comparable or better performance compared with existing training-based methods, and can more easily generalize to novel concepts derived solely from textual descriptions. 

We hope that our work can serve as an additional component of the creative toolbox, and open the way to additional composable-concept discovery methods. 

\section{Acknowledgment}
We would like to thank Ron Mokady and Yoad Tewel for providing feedback and helpful suggestions.

