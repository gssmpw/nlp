\section{Limitations} 

In this work, we compared the effectiveness and interplay of SFT and RL-based methods, under fixed data constraints. In particular, we chose offline methods like DPO and KTO as the baseline implementation of the RL method because it eliminates the need for reward modeling or iterative finetuning. This means that the process of development is limited to collecting an offline dataset and fientuning it - making it the most fair comparable to SFT in terms of implementation effort, compute costs and annotation efforts. Since this baseline RL method shows optimal performance over SFT, we hope that this motivates future work to study more complex RL-based methods and their interplay with SFT. In addition, we used GPT4o annotation for synthetic data generation, and also for evaluating Summarization and Helpfulness, which could include potential biases inherited from the model. 

In addition, we limited the size of the model to under 10 Billion parameters, to keep the finetuning cost low enough to ignore as compared to the data annotation costs. In addition, it would be extremely compute resource intensive to run thousands of finetuning runs with larger model sizes like 70B parameters. We hope that future work would study the scaling trends of RL-based methods against different model sizes, and also study the compute-data trade-off in-depth.


\section{Limitations}

While our approach is typically more general than current training-based approaches, it still has limitations. One limitation arises from surprising entanglements in the CLIP and diffusion feature spaces. For example, when attempting to combine a zebra's body with a leopard fur pattern (\cref{fig:limitations} (top)), the diffusion model tends to produce animals with the head of a giraffe, even though no giraffe appears in either input image. We hypothesize that this may be related to the tendency of diffusion models to represent some concepts as a composition of more basic visual components~\citep{chefer2023hidden}, but leave further investigation to future work.

On the other hand, some concepts may be \textit{more} disentangled in CLIP-space than intuitively expected. For example, outfit types and colors are disentangled in CLIP-space, hence, an ``outfit'' subspace spanned with descriptions of different types of outfits (``dress'', ``tuxedo''...) will not preserve outfit colors (\cref{fig:limitations} (bottom)). However, this can be easily amended by also specifying colors in the spanning texts (``\textit{red} dress'', ``\textit{blue} tuxedo''...).



Finally, we note that IP-Adapter itself is limited in the level of detail captured from the input image. Hence, our approach will not be sufficient for capturing delicate details such as exact identities. Stronger encoders may achieve higher fidelity, but it is not clear that our embedding-space projections would generalize to more complex feature spaces.

\section{Conclusions}

We presented IP-Composer, a training-free method that allows a user to compose novel images from visual concepts derived through a set of input images. To do so, our approach uses a CLIP-based IP-Adapter, leveraging their joint disentangled subspace structure. Through this approach, we achieve comparable or better performance compared with existing training-based methods, and can more easily generalize to novel concepts derived solely from textual descriptions. 

We hope that our work can serve as an additional component of the creative toolbox, and open the way to additional composable-concept discovery methods. 

\section{Acknowledgment}
We would like to thank Ron Mokady and Yoad Tewel for providing feedback and helpful suggestions.

