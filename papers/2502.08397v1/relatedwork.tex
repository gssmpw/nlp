\section{Literature review}
\label{sec:lit}

The most popular heuristic to optimize MSSC is the famous $k$-means algorithm \citep{macqueen1967some, lloyd1982least} ($\approx$ 3M references in Google Scholar 2025). It starts from an initial partition of $O$, so that each data point is assigned to a cluster. In the sequel, the centroids are computed and each point is then assigned to the cluster whose centroid is closest to it. If there are no changes in assignments, the heuristic stops. Otherwise, the
centroids are updated, and the process is repeated. The main drawback of $k$-means is that it often converges to locally optimal solutions that may be far from the global minimum. Moreover, it is highly sensitive to the choice of the initial cluster centers. As a result, considerable research has been devoted to improving initialization strategies (see, e.g.,\cite{arthur2006k,improvedkmeans2018,franti2019much}). 

Several heuristics and metaheuristics have been developed for the MSSC problem. These include simulated annealing \citep{lee2021simulated}, nonsmooth optimization \citep{bagirov2006new, BAGIROV201612}, tabu search \citep{ALSULTAN19951443}, variable neighborhood search \citep{HANSEN2001405,Orlov2018, carrizosa2013variable}, iterated local search \citep{likas2003global}, evolutionary algorithms \citep{MAULIK20001455,SARKAR1997975}), difference of convex functions programming \citep{tao2014new,BAGIROV201612,KARMITSA2017367,KARMITSA2018245}. The \(k\)-means algorithm is also used as a local search subroutine in various algorithms, such as the population-based metaheuristic proposed by \cite{gribel2019hg}.

The methods for solving the MSSC problem to global optimality are generally based on branch-and-bound (B\&B) and column generation (CG) algorithms. The earliest B\&B for the MSSC problem is attributed to \cite{koontz1975branch}, later refined by \cite{diehr1985evaluation}. This method focuses on partial solutions with fixed assignments for a subset of the data points, exploiting the observation that the optimal solution value for the full set is no less than the sum of the optimal solutions for the subsets. \cite{brusco2006repetitive} further developed this approach into the repetitive-branch-and-bound algorithm (RBBA), which solves sequential subproblems with increasing data points. RBBA is effective for synthetic datasets up to 240 points and real-world instances up to 60 data points. Despite being classified as a B\&B algorithm, Brusco's approach does not leverage lower bounds derived from relaxations of the MINLP model. Instead, it relies on bounds derived from the inherent properties of the MSSC problem. In contrast, a more traditional line of research employs B\&B algorithms where lower bounds are obtained through appropriate mathematical programming relaxations of the MINLP model. For example, the B\&B method by \cite{sherali2005global} employs the reformulation-linearization technique to derive lower bounds by transforming the nonlinear problem into a 0-1 mixed-integer program. This method claims to handle problems up to 1000 entities, though replication attempts have shown high computing times for real datasets with around 20 objects \citep{aloise2011evaluating}. More recent efforts by \cite{burgard2023mixed} have focused on mixed-integer programming techniques to improve solver performance, but these have not yet matched the leading exact methods for MSSC.

The first CG algorithm for MSSC problem has been proposed by \cite{du1999interior}. In this algorithm, the restricted master problem is solved using an interior point method while for the auxiliary problem a hyperbolic program with binary variables is used to find a column with negative reduced cost. To accelerate the resolution of the auxiliary problem, variable-neighborhood-search heuristics are used to obtain a good initial solution. Although this approach successfully solved medium-sized benchmark instances, including the popular Iris dataset with 150 entities, the resolution of the auxiliary problem proved to be a bottleneck due to the unconstrained quadratic 0-1 optimization problem. To overcome this issue, \cite{aloise2012improved} proposed an improved CG algorithm that uses a geometric-based approach to solve the auxiliary problem. Specifically, their approach involves solving a certain number of convex quadratic problems to obtain the solution of the auxiliary problem. When the points to be clustered are in the plane, the maximum number of convex problems to solve is polynomially bounded. Otherwise, the algorithm needs to find the cliques in a certain graph induced by the current solution of the master problem to solve the auxiliary problems. The efficiency of this algorithm depends on the sparsity of the graph, which increases as the number of clusters $K$ increases. Therefore, the algorithm proposed by \cite{aloise2012improved} is particularly efficient when the graph is sparse and $K$ is large. Their method was able to provide exact solutions for large-scale problems, including one instance of 2300 entities, but only when the ratio between $N$ and $K$ is small. More recently, \cite{sudoso2024column} combined the CG proposed by \cite{aloise2012improved} with dynamic constraint aggregation \citep{bouarab2017dynamic} to accelerate the resolution of the restricted master problem, which is known to suffer from high degeneracy.  When solving MSSC instances in the plane using branch-and-price, the authors show that this method can handle datasets up to about 6000 data points.

Finally, there is a large branch of literature towards the application of techniques from semidefinite programming (SDP). \cite{peng2005new} and \cite{peng2007approximating} showed the equivalence between the MINPL model and a 0-1 SDP reformulation. \cite{aloise2009branch} developed a branch-and-cut algorithm based on the linear programming (LP) relaxation of the 0-1 SDP model, solving instances up to 202 data points. \cite{piccialli2022sos} further advanced this with a branch-and-cut algorithm using SDP relaxation and polyhedral cuts, capable of solving real-world instances up to 4,000 data points. This algorithm currently represents the state-of-the-art exact solver for MSSC. Additionally, \cite{liberti2022side} examine the application of MINLP techniques to MSSC with side constraints, while the SDP-based exact algorithms proposed in \cite{piccialli2022semi, piccialli2023global} demonstrate state-of-the-art performance for constrained variants of MSSC.