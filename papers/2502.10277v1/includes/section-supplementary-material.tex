\section{AI System Pipeline and Training}
\label{sec:ai-system-pipeline}

In this study, our AI system followed a three-stage pipeline:
\begin{enumerate*}[label=(\alph*)]
    \item the finding localization module,
    \item the tooth index classification module, and
    \item the post-processing module.
\end{enumerate*}
The first two modules utilized convolutional neural networks (CNNs), while the post-processing module employed a probabilistic approach to integrate the outputs from the prior modules and produce the final AI assessment scores.

\subsection{Finding Localization Module}
The finding localization problem was formulated as an \emph{object detection} task in computer vision. 
We employed the YOLOv8 model~\citep{varghese2024yolov8}, a state-of-the-art CNN architecture for object detection.
YOLOv8 improved upon its predecessors with enhanced efficiency and accuracy, supporting high-resolution inputs (1024×1024) and complex segmentation tasks, making it particularly suitable for medical image analysis.
Specifically, we used the medium variant of the YOLOv8 model family.

The model was initialized with pre-trained weights from the Common Objects in Context (COCO) challenge~\citep{lin2014microsoft}.
Training was conducted on the Netherlands data of \num{4044} DPRs, split into 70\% for training and 30\% for validation.
The training spanned up to \num{50} epochs, using a batch size of 1 and an image size of 1024×1024 to retain high-resolution details essential for accurate localization.
A cosine decay learning rate schedule was applied, starting at \num{0.01} and decreasing gradually to stabilize performance.

To accelerate convergence, a warm-up phase of 3 epochs linearly increased the learning rate.
Momentum was set to 0.937 for a balance between stability and speed, while weight decay at 0.0005 prevented overfitting.
Data augmentation strategies, including random scaling (0.5), translation (0.1), and rotation (15 degrees), were applied to improve generalization.

During testing time (\ie, inference, or, case reading), the module generated a list of candidate objects. Each object instance $i$ included:
\begin{enumerate}
    \item an objectness score $\objness_i\in\zeroOne$;
    \item an finding probability vector $\objProb_i\in\zeroOne^C$, where $C=8$ is the number of finding types;
    \item a bounding box; and
    \item a probabilistic contour (mask) $\objMap_i\in\zeroOne^{H \times W}$, where $H$ and $W$ denote the image height and width. 
\end{enumerate}

The object list was sorted in descending order of objectness, and non-maximum suppression was applied to remove overlapping objects exceeding 50\% area overlap.
A maximum of \num{300} objects with objectness scores as low as \num{0.0001} were retained.

\subsection{Tooth Index Classification Module}
The tooth index classification module aimed to associate each image pixel with a specific FDI (World Dental Federation) index using a \emph{semantic segmentation} approach.
This module employed the DeepLabv3 model~\citep{chen2017rethinking}, which integrates a 101-layer backbone for feature extraction and an Atrous Spatial Pyramid Pooling (ASPP) module to capture multi-scale contextual information via atrous convolutions at varying rates.

We implemented the model using Detectron2~\citep{wu2019detectron2}, enabling efficient large-scale segmentation.
Input images were resized to 512×512, with dilated convolutions applied at rates of 6, 12, and 18 to expand the receptive field while preserving spatial resolution.
High-resolution low-level features from earlier backbone layers were projected via the DeepLabv3 head configuration for fine-grained segmentation.

Contours corresponding to the 32 full-dentition teeth were extracted and flattened onto maps matching the input image dimensions.
The model was trained for \num{100000} steps with a batch size of 3 images.
A linear learning rate warmup phase over the first \num{1000} iterations ensured stable optimization, after which the learning rate decayed via a cosine schedule to near-zero values.
A base learning rate of \num{0.001} was used, combined with momentum (0.9) and weight decay (\num{0.0001}) to reduce overfitting.
Gradient clipping at 1.0 prevented exploding gradients.
Each training iteration processed input images resized dynamically between 768×1024 and 1280×2048 to introduce variability and enhance generalization.

During inference, the module produced a dense probability map with 33 channels (32 tooth classes and 1 background class), denoted as $\semMap_j\in\zeroOne^{H \times W}$, where $j$ runs over $J=33$ classes, and $H$ and $W$ are the image dimensions.

\subsection{Post-Processing Module}
The post-processing module integrated outputs from the finding localization and tooth index classification modules.
Specifically, we obtained:
\begin{equation}
    \paral{\objness_i, \objProb_i, \objMap_i}_{i=1}^I \quad \text{and} \quad \paral{\semMap_j}_{j=1}^J.
\end{equation}

The correlation $\corr_{ij}$ between a finding object (instance) $i$ and a tooth index (semantic class) $j$ was defined as:
\begin{equation}
    \corr_{ij} \equiv \frac{\paraa{\objMap_i, \semMap_j}}{\sqrt{\paraa{\objMap_i, \objMap_i}}} \in \zeroOne,
\end{equation}
where $\paraa{\cdot, \cdot}$ denotes the inner product of probability maps, summing over the pixel-wise probability products over the whole image.

Following~\citet{ardila2019end}, we applied a \emph{soft-or} strategy to aggregate the contributions of all objects $i$ with detection scores $\objness_i$ and correlations $\corr_{ij}$.
The presence probability of finding $c$ in tooth index $j$, denoted $\para{\semProb_j}_c$, was calculated as:
\begin{equation}
    1 - \para{\semProb_j}_c \equiv \prod_i \param{ 1 - \func_c\para{\objness_i, \corr_{ij}; \paral{r}} }^{\para{\objProb_i}_c},
\end{equation}
where $\paral{r}$ is the set of all correlations, and $\func_c$ is a finding-dependent function defined as:
\begin{equation}
    \func_c\para{s_i, r_{ij}; \paral{r}} \equiv
    \begin{cases}
        \frac{r_{ij}}{\sum_{j\neq\text{background}} r_{ij}}, & \text{if } c = \text{missing}; \\
        s_i \cdot \frac{r_{ij}}{\sum_{j\neq\text{background}} r_{ij}}, & \text{for other findings}.
    \end{cases}
\end{equation}

The final output consisted of a set of probabilities:
\begin{equation}
    \paral{\semProb_j}_{j\neq\text{background}},
\end{equation}
where $\semProb_j \in \zeroOne^C$ provided the scores for $C = 8$ findings per tooth index $j$.
For 32 tooth indices, the system produced a total of 256 floating-point scores per DPR.

\newpage

\section{AI Performance -- Detailed Results}
\label{sec:performance-comparison-detailed}
% begin: generated-do-not-modify
% doc: https://docs.google.com/spreadsheets/d/1EjEpgC1aj12y2GlBNSyGbbFMRkUdaqQ5EuTgQGSnljo
\begin{table}[!h]
\centering\caption{  \textbf{Comprehensive Performance Comparison of AI and Human Readers.}   This table presents a horizontal comparison of various performance metrics for the AI system and human readers across multiple test datasets and prior studies.  The leftmost columns show results for the AI system on different test sets in this study, where the AI was optimized to maximize the $\textrm{F}_2$ score in a screening setting.  Note that Taiwan* represents a subset of the Taiwan test set.  The rightmost columns summarize performance metrics reported in prior works, evaluated using their respective datasets and AI systems.  Values in parentheses indicate the 95\% confidence intervals.  A ``--'' indicates that a finding was not included, the metric was not reported, or the metric was not applicable.}\label{tbl:ai-multinational}\tiny\rowcolors{1}{white}{table-bg}
\begin{tabular}{ p{0.167\linewidth}p{0.117\linewidth}p{0.083\linewidth}p{0.083\linewidth}p{0.083\linewidth}p{0.083\linewidth}|p{0.067\linewidth}p{0.067\linewidth} }
\toprule
Data Set & The Netherlands, test & Brazil & Taiwan & Taiwan* & Taiwan* & \citet{vinayahalingam2021automated} & \citet{bacsaran2022diagnostic} \\ \midrule
Reader & AI (Ours) & AI (Ours) & AI (Ours) & AI (Ours) & Human, mean & AI~\citep{vinayahalingam2021automated} & AI~\citep{bacsaran2022diagnostic} \\ \midrule
Missing &   &   &   &   &   &   &   \\
{~~~~}Positive/negative count & \num{6836} / \num{31596} & \num{5926} / \num{31610} & \num{1819} / \num{6213} & \num{866} / \num{2910} & \num{866} / \num{2910} & -- & -- \\
{~~~~}Metric, \% (95\% CI) &   &   &   &   &   &   &   \\
{~~~~~~~~}Sensitivity & 87.6 {(86.8--88.4)} & 93.6 {(92.9--94.2)} & 89.3 {(87.8--90.7)} & 92.0 {(90.0--93.7)} & 87.3 {(85.5--89.1)} & 96.5 & -- \\
{~~~~~~~~}Specificity & 95.5 {(95.3--95.8)} & 95.7 {(95.5--95.9)} & 95.1 {(94.5--95.6)} & 95.9 {(95.1--96.5)} & 99.1 {(98.9--99.4)} & -- & -- \\
{~~~~~~~~}Precision & 80.9 {(80.0--81.8)} & 80.3 {(79.3--81.2)} & 84.2 {(82.5--85.7)} & 86.9 {(84.6--88.9)} & 96.8 {(95.9--97.7)} & 97.5 & -- \\
{~~~~~~~~}$\textrm{F}_1$ score & 84.2 {(83.5--84.9)} & 86.4 {(85.7--87.1)} & 86.7 {(85.4--87.9)} & 89.4 {(87.7--91.1)} & 91.8 {(90.7--92.9)} & 97.0 & -- \\
{~~~~~~~~}$\textrm{F}_2$ score & 86.2 {(85.5--86.9)} & 90.6 {(90.0--91.2)} & 88.2 {(86.9--89.6)} & 91.0 {(89.3--92.7)} & 89.1 {(87.6--90.6)} & -- & -- \\
{~~~~~~~~}AUC-ROC & 94.4 {(94.0--94.9)} & 97.8 {(97.6--98.1)} & 95.7 {(95.1--96.4)} & 96.9 {(96.1--97.7)} & -- & -- & -- \\ \midrule

Implant &   &   &   &   &   &   &   \\
{~~~~}Positive/negative count & \num{619} / \num{6217} & \num{586} / \num{5340} & \num{191} / \num{1628} & \num{80} / \num{786} & \num{80} / \num{786} & -- & \num{26} / -- \\
{~~~~}Metric, \% (95\% CI) &   &   &   &   &   &   &   \\
{~~~~~~~~}Sensitivity & 98.2 {(96.8--99.0)} & 95.6 {(93.6--97.0)} & 99.0 {(96.3--99.7)} & 100.0 {(95.4--100.0)} & 95.0 {(92.3--97.7)} & 81.7 & 96.15 \\
{~~~~~~~~}Specificity & 97.6 {(97.2--98.0)} & 95.0 {(94.4--95.6)} & 98.3 {(97.5--98.8)} & 98.3 {(97.2--99.0)} & 99.7 {(99.4--99.9)} & -- & -- \\
{~~~~~~~~}Precision & 80.3 {(77.3--83.0)} & 67.8 {(64.5--70.9)} & 87.1 {(82.0--90.9)} & 86.0 {(77.5--91.6)} & 96.8 {(94.3--99.3)} & 80.0 & 92.59 \\
{~~~~~~~~}$\textrm{F}_1$ score & 88.4 {(86.5--90.2)} & 79.3 {(76.9--81.7)} & 92.6 {(89.8--95.5)} & 92.5 {(88.2--96.8)} & 95.9 {(94.0--97.8)} & 80.9 & 94.33 \\
{~~~~~~~~}$\textrm{F}_2$ score & 94.0 {(92.8--95.2)} & 88.3 {(86.6--90.1)} & 96.3 {(94.6--98.1)} & 96.9 {(94.8--98.9)} & 95.3 {(93.1--97.6)} & -- & -- \\
{~~~~~~~~}AUC-ROC & 99.5 {(99.4--99.6)} & 98.3 {(98.1--98.6)} & 99.7 {(99.6--99.9)} & 99.8 {(99.6--99.9)} & -- & -- & -- \\ \midrule
 
Residual root &   &   &   &   &   &   &   \\
{~~~~}Positive/negative count & \num{137} / \num{31459} & \num{198} / \num{31412} & \num{88} / \num{6125} & \num{38} / \num{2872} & \num{38} / \num{2872} & -- & \num{29} / -- \\
{~~~~}Metric, \% (95\% CI) &   &   &   &   &   &   &   \\
{~~~~~~~~}Sensitivity & 79.6 {(72.0--85.5)} & 56.1 {(49.1--62.8)} & 72.7 {(62.6--80.9)} & 71.1 {(55.2--83.0)} & 57.2 {(48.9--65.6)} & 73.3 & 82.14 \\
{~~~~~~~~}Specificity & 99.9 {(99.8--99.9)} & 99.5 {(99.4--99.6)} & 99.0 {(98.8--99.3)} & 99.2 {(98.8--99.4)} & 99.9 {(99.8--100.0)} & -- & -- \\
{~~~~~~~~}Precision & 76.2 {(68.6--82.5)} & 41.7 {(36.0--47.7)} & 52.0 {(43.3--60.7)} & 52.9 {(39.5--65.9)} & 90.0 {(79.7--100.4)} & 81.5 & 67.64 \\
{~~~~~~~~}$\textrm{F}_1$ score & 77.9 {(72.5--83.2)} & 47.8 {(42.2--53.5)} & 60.7 {(52.9--68.4)} & 60.7 {(48.7--72.7)} & 67.8 {(59.4--76.1)} & 77.2 & 74.19 \\
{~~~~~~~~}$\textrm{F}_2$ score & 78.9 {(73.0--84.7)} & 52.5 {(46.4--58.5)} & 67.4 {(59.4--75.3)} & 66.5 {(54.1--78.9)} & 60.8 {(52.5--69.1)} & -- & -- \\
{~~~~~~~~}AUC-ROC & 93.1 {(89.8--96.4)} & 92.8 {(90.3--95.3)} & 98.3 {(97.4--99.2)} & 97.6 {(95.7--99.5)} & -- & -- & -- \\ \midrule

Crown/bridge &   &   &   &   &   &   &   \\
{~~~~}Positive/negative count & \num{3038} / \num{28558} & \num{896} / \num{30714} & \num{851} / \num{5362} & \num{429} / \num{2481} & \num{429} / \num{2481} & -- & \num{236} / -- \\
{~~~~}Metric, \% (95\% CI) &   &   &   &   &   &   &   \\
{~~~~~~~~}Sensitivity & 92.7 {(91.7--93.5)} & 87.4 {(85.1--89.4)} & 97.2 {(95.8--98.1)} & 98.1 {(96.4--99.1)} & 96.4 {(95.2--97.6)} & 85.1 & 96.74 \\
{~~~~~~~~}Specificity & 98.7 {(98.6--98.8)} & 98.8 {(98.7--98.9)} & 99.4 {(99.1--99.6)} & 98.9 {(98.4--99.2)} & 99.7 {(99.6--99.8)} & -- & -- \\
{~~~~~~~~}Precision & 88.5 {(87.4--89.6)} & 68.3 {(65.5--70.9)} & 96.2 {(94.7--97.3)} & 93.8 {(91.1--95.7)} & 98.2 {(97.4--99.0)} & 87.2 & 86.30 \\
{~~~~~~~~}$\textrm{F}_1$ score & 90.6 {(89.8--91.4)} & 76.7 {(74.6--78.7)} & 96.7 {(95.7--97.6)} & 95.9 {(94.4--97.4)} & 97.3 {(96.5--98.0)} & 86.1 & 91.22 \\
{~~~~~~~~}$\textrm{F}_2$ score & 91.8 {(91.0--92.6)} & 82.8 {(80.8--84.7)} & 97.0 {(96.0--98.0)} & 97.2 {(96.0--98.5)} & 96.7 {(95.8--97.7)} & -- & -- \\
{~~~~~~~~}AUC-ROC & 98.7 {(98.5--98.9)} & 98.0 {(97.4--98.6)} & 99.8 {(99.7--99.9)} & 99.8 {(99.7--99.9)} & -- & -- & -- \\ \midrule

Root canal filling &   &   &   &   &   &   &   \\
{~~~~}Positive/negative count & \num{1867} / \num{29729} & \num{1316} / \num{30294} & \num{643} / \num{5570} & \num{326} / \num{2584} & \num{326} / \num{2584} & -- & \num{162} / -- \\
{~~~~}Metric, \% (95\% CI) &   &   &   &   &   &   &   \\
{~~~~~~~~}Sensitivity & 97.4 {(96.5--98.0)} & 94.1 {(92.7--95.3)} & 96.7 {(95.1--97.9)} & 97.2 {(94.8--98.5)} & 93.3 {(91.6--95.0)} & 88.2 & 86.70 \\
{~~~~~~~~}Specificity & 99.6 {(99.5--99.7)} & 99.4 {(99.3--99.5)} & 99.3 {(99.0--99.5)} & 99.3 {(98.9--99.6)} & 99.8 {(99.7--99.9)} & -- & -- \\
{~~~~~~~~}Precision & 93.6 {(92.4--94.6)} & 86.8 {(84.9--88.4)} & 93.8 {(91.7--95.4)} & 94.6 {(91.7--96.6)} & 98.1 {(97.2--99.0)} & 89.1 & 77.72 \\
{~~~~~~~~}$\textrm{F}_1$ score & 95.5 {(94.8--96.2)} & 90.3 {(89.1--91.5)} & 95.3 {(94.0--96.5)} & 95.9 {(94.2--97.6)} & 95.6 {(94.6--96.6)} & 88.6 & 81.96 \\
{~~~~~~~~}$\textrm{F}_2$ score & 96.6 {(95.9--97.3)} & 92.6 {(91.5--93.7)} & 96.1 {(94.9--97.4)} & 96.7 {(95.1--98.3)} & 94.2 {(92.8--95.6)} & -- & -- \\
{~~~~~~~~}AUC-ROC & 99.5 {(99.3--99.7)} & 99.2 {(99.0--99.5)} & 99.5 {(99.1--99.8)} & 99.4 {(98.8--100.0)} & -- & -- & -- \\ \midrule

Filling &   &   &   &   &   &   &   \\
{~~~~}Positive/negative count & \num{7412} / \num{24184} & \num{8127} / \num{23483} & \num{1370} / \num{4843} & \num{611} / \num{2299} & \num{611} / \num{2299} & -- & \num{519} / -- \\
{~~~~}Metric, \% (95\% CI) &   &   &   &   &   &   &   \\
{~~~~~~~~}Sensitivity & 92.2 {(91.6--92.8)} & 93.8 {(93.3--94.3)} & 87.6 {(85.7--89.2)} & 84.8 {(81.7--87.4)} & 74.4 {(71.8--77.0)} & 81.9 & 86.08 \\
{~~~~~~~~}Specificity & 93.9 {(93.6--94.2)} & 92.3 {(91.9--92.6)} & 88.0 {(87.1--88.9)} & 94.6 {(93.6--95.5)} & 98.4 {(98.1--98.7)} & -- & -- \\
{~~~~~~~~}Precision & 82.3 {(81.5--83.1)} & 80.8 {(80.0--81.6)} & 67.4 {(65.2--69.6)} & 80.7 {(77.5--83.6)} & 93.1 {(91.8--94.4)} & 84.1 & 87.61 \\
{~~~~~~~~}$\textrm{F}_1$ score & 87.0 {(86.4--87.6)} & 86.8 {(86.2--87.4)} & 76.2 {(74.5--77.9)} & 82.7 {(80.2--85.1)} & 82.1 {(80.3--83.9)} & 83.0 & 86.84 \\
{~~~~~~~~}$\textrm{F}_2$ score & 90.0 {(89.5--90.6)} & 90.9 {(90.4--91.4)} & 82.6 {(81.0--84.3)} & 83.9 {(81.3--86.5)} & 77.2 {(74.9--79.5)} & -- & -- \\
{~~~~~~~~}AUC-ROC & 97.3 {(97.0--97.5)} & 97.2 {(97.0--97.4)} & 94.8 {(94.1--95.5)} & 95.9 {(95.0--96.8)} & -- & -- & -- \\ \midrule

Caries &   &   &   &   &   &   &   \\
{~~~~}Positive/negative count & \num{1063} / \num{30533} & \num{1435} / \num{30175} & \num{317} / \num{5896} & \num{180} / \num{2730} & \num{180} / \num{2730} & -- & \num{256} / -- \\
{~~~~}Metric, \% (95\% CI) &   &   &   &   &   &   &   \\
{~~~~~~~~}Sensitivity & 46.3 {(43.3--49.3)} & 57.4 {(54.8--59.9)} & 52.7 {(47.2--58.1)} & 56.1 {(48.8--63.2)} & 40.6 {(35.3--45.8)} & -- & 30.26 \\
{~~~~~~~~}Specificity & 91.8 {(91.5--92.1)} & 84.7 {(84.3--85.1)} & 94.5 {(93.9--95.1)} & 95.3 {(94.5--96.0)} & 99.0 {(98.8--99.2)} & -- & -- \\
{~~~~~~~~}Precision & 16.5 {(15.2--17.9)} & 15.2 {(14.2--16.1)} & 34.1 {(30.0--38.4)} & 44.1 {(37.8--50.6)} & 77.4 {(72.3--82.4)} & -- & 50.96 \\
{~~~~~~~~}$\textrm{F}_1$ score & 24.3 {(22.5--26.1)} & 24.0 {(22.6--25.4)} & 41.4 {(37.1--45.7)} & 49.4 {(43.4--55.3)} & 50.7 {(45.3--56.1)} & -- & 37.98 \\
{~~~~~~~~}$\textrm{F}_2$ score & 34.0 {(31.8--36.2)} & 36.8 {(35.1--38.6)} & 47.5 {(42.8--52.2)} & 53.2 {(46.9--59.6)} & 43.9 {(38.6--49.2)} & -- & -- \\
{~~~~~~~~}AUC-ROC & 82.5 {(81.3--83.7)} & 79.8 {(78.6--81.0)} & 82.3 {(79.6--84.9)} & 83.0 {(79.3--86.6)} & -- & -- & -- \\ \midrule

Periapical radiolucency &   &   &   &   &   &   &   \\
{~~~~}Positive/negative count & \num{458} / \num{31138} & \num{377} / \num{31233} & \num{197} / \num{6016} & \num{138} / \num{2772} & \num{138} / \num{2772} & -- & -- \\
{~~~~}Metric, \% (95\% CI) &   &   &   &   &   &   &   \\
{~~~~~~~~}Sensitivity & 55.2 {(50.7--59.7)} & 53.8 {(48.8--58.8)} & 86.8 {(81.4--90.8)} & 93.5 {(88.1--96.5)} & 25.5 {(20.6--30.5)} & -- & -- \\
{~~~~~~~~}Specificity & 98.2 {(98.1--98.3)} & 97.2 {(97.1--97.4)} & 94.9 {(94.3--95.4)} & 97.0 {(96.3--97.6)} & 99.5 {(99.4--99.7)} & -- & -- \\
{~~~~~~~~}Precision & 31.2 {(28.1--34.4)} & 19.1 {(16.9--21.6)} & 35.8 {(31.6--40.2)} & 61.1 {(54.4--67.5)} & 73.1 {(64.1--82.1)} & -- & -- \\
{~~~~~~~~}$\textrm{F}_1$ score & 39.8 {(36.4--43.3)} & 28.2 {(25.2--31.3)} & 50.7 {(46.1--55.3)} & 73.9 {(68.7--79.1)} & 37.0 {(30.9--43.0)} & -- & -- \\
{~~~~~~~~}$\textrm{F}_2$ score & 47.8 {(44.1--51.6)} & 39.5 {(35.7--43.3)} & 67.5 {(63.2--71.8)} & 84.5 {(80.5--88.6)} & 29.1 {(23.7--34.5)} & -- & -- \\
{~~~~~~~~}AUC-ROC & 93.2 {(92.0--94.5)} & 88.1 {(85.9--90.3)} & 95.0 {(93.3--96.8)} & 97.4 {(95.9--99.0)} & -- & -- & -- \\ \bottomrule
\end{tabular}
\end{table}
% end: generated-do-not-modify

\newpage

\section{Consistency of Human Reader Performance}
\label{sec:consistency-of-human}
\begin{figure}[!t]
    \centering
    \includegraphics[width=1.0\textwidth, height=0.80\textheight, keepaspectratio]{./assets/fig-reader-agreement.pdf}
    \caption{
        \textbf{Inter-Reader Agreement Levels for Dental Finding Summaries in Taiwan.}
        This figure presents the agreement levels between pairs of readers as measured by Cohen's Kappa for various dental findings.
        Each blue error bar illustrates the Kappa agreement between each of the six possible reader pairings (G1/G2, G1/S1, G1/S2, G2/S1, G2/S2, and S1/S2) from four participating readers, grouped into general dentists (G1 and G2) and specialists (S1 and S2).
        The agreement levels are averaged for pairs within the same expertise group (generalists or specialists) and across different expertise, shown as red lines.
        The error bars represent the 95\% confidence intervals for the Kappa values.
        Significance testing using $t$-statistics revealed that differences in mean agreement levels for residual roots were statistically significant ($p = .039$), while other findings showed no significant differences.
    }
    \label{fig:reader-agreement}
\end{figure}

We inspected the mutual agreement among four readers (G1, G2, S1, and S2) involved in the study using Cohen's Kappa.
Kappa was calculated between every possible pair of human readers, and the overall average of agreements was strong~\citep{mchugh2012interrater} for the following findings: crown/bridges at 95.4\% (95\% CI: 92.9\% -- 97.8\%), implants at 92.3\% (95\% CI: 83.0\% -- 100.0\%), root canal fillings at 92.1\% (95\% CI: 86.8\% -- 97.4\%), and missing teeth at 90.1\% (95\% CI: 81.9\% -- 98.2\%).
Agreement for fillings was moderate, at 74.9\% (95\% CI: 69.8\% -- 80.0\%).
Residual roots and caries exhibited weak agreement, at 56.2\% (95\% CI: 35.6\% -- 76.8\%) and 46.5\% (95\% CI: 34.8\% -- 58.1\%), respectively.
The lowest mean agreement occurred for periapical radiolucencies, at 37.6\% (95\% CI: 14.7\% -- 60.6\%).

To further analyze the data, we categorized the readers into two expertise groups: general dentists (G1 and G2) and specialized dentists (S1 and S2).
Average agreement metrics were computed by taking the average Kappa value over reader pairs with specific criteria: \emph{same-expertise} pairs (G1-G2 and S1-S2) and \emph{different-expertise} pairs (G1-S1, G1-S2, G2-S1, and G2-S2), respectively.
Notably, for residual roots, a statistically significant difference in agreement was observed between the same-expertise and different-expertise average Kappa values, with a difference of 30.3\% (95\% CI: 1.6\% -- 58.9\%, $p = .039$).
No significant differences were found in agreement levels between same-expertise average and different-expertise average for other findings.