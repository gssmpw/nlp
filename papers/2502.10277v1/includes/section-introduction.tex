\section{Introduction}

Dental panoramic radiographs (DPRs), also known as orthopantomograms (OPG), provide an overview of the oral region and are widely used as a primary diagnostic tool by dentists and oral and maxillofacial (OMF) surgeons in routine clinical practice.
DPRs allow for the detection of diagnostic findings, such as dental caries and lesions, as well as the assessment of prior treatments, including root canal fillings and dental implants.
These radiographs are particularly valued due to their low radiation exposure, brief acquisition times, and cost-effectiveness, enabling early detection of oral diseases for improved prognostic outcomes~\citep{choi2011assessment} and enhanced assessment of treatment statuses.
However, accurately interpreting DPRs is challenging due to the overlapping structures of tissues and bones~\citep{fourcade2019deep}. 
Furthermore, clinicians often perform interpretations under significant time constraints, focusing predominantly on symptomatic areas.
Studies have consequently reported low inter-observer agreement even among experienced dentists~\citep{kweon2018panoramic}, which can lead to suboptimal diagnostic performance in clinical scenarios~\citep{plessas2019impact}.

With the advent of artificial intelligence (AI), significant advancements have been made in developing computer-aided detection (CAD) systems to interpret panoramic radiographs for a variety of dental findings essential for treatment planning and assessment.
While previous research predominantly concentrated on a limited array of findings, such as tooth counting~\citep{oktay2017tooth, koch2019accurate}, missing teeth detection~\citep{tuzoff2019tooth, kim2020automatic, xu2023robust}, caries~\citep{dayi2023novel, zhu2022cariesnet, oztekin2023explainable, lian2021deep}, and oral lesions~\citep{endres2020development, yang2020deep}, recent studies have begun to explore a broader range of dental issues~\citep{muresan2020teeth, vinayahalingam2021automated, rohrer2022segmentation, almalki2022deep, bacsaran2022diagnostic}.
Some even tackled a more complex and challenging setting, similar to this study, where identification of findings and associating them with numbered teeth are both required~\citep{gardiyanouglu2023automatic,van2024combining}.
Nevertheless, these studies often suffer from limitations due to their reliance on data collected at single clinical sites~\citep{muresan2020teeth, rohrer2022segmentation, bacsaran2022diagnostic, lian2021deep, endres2020development, yang2020deep}, overlooking the variability in patient demographics and the discrepancies in image acquisition techniques across different setups, which crucially impact AI performance, especially when they are tasked to operate across different sites.
Moreover, many of these studies do not concurrently evaluate the performance of human readers~\citep{muresan2020teeth, rohrer2022segmentation, almalki2022deep, bacsaran2022diagnostic}, missing opportunities to demonstrate how AI systems could enhance support for dentist.

In this study, we examined the interpretation of DPRs using large multinational data sets.
We compiled three data sets encompassing a set of 8 dental finding categories to achieve a comprehensive evaluation of performance metrics.
Furthermore, we developed a novel AI system to concurrently perform finding identification and localization, and examined its generalizability across data sets from various continents.
Importantly, we also investigated the performance of human readers in interpreting DPRs under realistic conditions and provided essential insights into the potential integration of AI into clinical workflows to assist dentists and OMF surgeons.