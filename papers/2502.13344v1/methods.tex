\section{Approach} \label{sec:methods} 
\subsection{Problem Definition}
Building on the problem illustrated in Figure \ref{fig:overview}, we aim to predict unobserved interactions in an inductive reasoning setup, as described in \citet{zhang2023emergingdruginteractionprediction}. These interactions can involve:
\begin{itemize}
\item A known entity (e.g., a well-studied drug or disease with some observed interactions) and an emerging entity (e.g., a drug or disease whose interactions of interest have not been observed), or
\item Two emerging entities.
\end{itemize}
The unobserved interactions may include drug-drug interactions or drug-disease interactions.


Formally, we define a knowledge graph $\mathcal{G} = \{(u, r, v) \mid u, v \in \mathcal{E}, r \in \mathcal{R} \}$ where $u, v \in \mathcal{E}$ represent biomedical entities (e.g., drugs, diseases, genes) and $r \in \mathcal{R}$ denotes a relation type. These relation types include known drug-drug and drug-disease interactions (observed interactions) and a broader biomedical network with relationships like drug-gene or gene-gene interactions.

Given two query entities \( u\) and \( v\), our goal is to infer their interaction type, which is framed as predicting the presence and type of relation $r$ between $u$ and $v$.
We define a computational model \(\phi(\mathcal{G})\) to predict these interactions under both zero-shot generative and supervised settings.
Specifically, we leverage LLMs for reasoning-based inference and GNNs for interaction prediction. 

\subsection{K-Paths Framework}
We introduce \textit{K-Paths}, a framework for predicting the interaction between entities \( u \) and \( v \). K-Paths comprises three key components:

\begin{enumerate}
    \item An \textit{Augmented KG}: This module constructs a knowledge graph $\mathcal{G}$ by integrating known drug-drug or drug-disease interactions with a broader biomedical KG. This integration incorporates additional entities such as genes and their known relationships (e.g., drug-gene or gene-gene interactions). 
    \item A \textit{Diverse Path Retrieval module}: This module employs a novel path retrieval algorithm to retrieve a diverse set of relevant reasoning paths connecting the query entities \( u \) and \( v \) from the augmented KG $\mathcal{G}$.
    \item A \textit{Path Integration module}: This module processes the retrieved query-specific paths for interaction prediction.
    For LLM-based reasoning and interaction inference, the paths are transformed into natural language and appended to the interaction query prompt.
    For GNN-based interaction prediction, the paths are reconstructed into query-specific subgraphs.
\end{enumerate}
The overall K-Paths framework is illustrated in Figure \ref{fig:pipeline-overview}.

\subsection{Augmented KG} \label{aug-network}
Following prior work~\citep{yu2021sumgnn, wang2024accurate, zhang2023emergingdruginteractionprediction}, we define the \textit{augmented KG} as our knowledge graph \(\mathcal{G}\), constructed by integrating:  
\begin{enumerate}
    \item Observed interactions (e.g., drug-drug and drug-disease) from the training set.
    \item Hetionet, a biomedical knowledge graph containing biological entities (e.g., genes, proteins, pathways) and their relationships (e.g., drug-gene, gene-gene, protein-protein interactions)~\citep{hetionet2017}.
\end{enumerate}


Since Hetionet is incomplete, incorporating known interactions from the training set enhances coverage and helps build a more comprehensive augmented KG. 
However, interactions involving emerging entities remain missing in this graph. To address this, we leverage existing relationships within the augmented KG to infer these unobserved interactions. Additionally, following~\citep{zhang2023emergingdruginteractionprediction}, we incorporate inverse relations to account for the directed nature of the augmented KG, ensuring bidirectional information flow.

The augmented KG \(\mathcal{G}\) serves as the structured knowledge graph for all subsequent tasks.



\begin{algorithm}[!ht]
\KwIn{A set of paths $P = \{p_1, p_2, \dots, p_K\}$, 
      where each path $p$ is described by \\
      $R(p) = (r_1, r_2, \dots, r_l)$ (sequence of relations) and \\
      $\mathcal{E}(p) = (e_1, e_2, \dots, e_m)$ (sequence of entities).}
\KwOut{A subset $P' \subseteq P$ with redundant paths removed.}

\BlankLine
Initialize $P' \leftarrow \emptyset$\;

\ForEach{path $p \in P$}{
  Let $R(p) = (r_1, r_2, \dots, r_l)$ be the relation sequence of $p$\;
  Let $l = |R(p)|$ (the length of the relation sequence)\;

  \If{\(\nexists\, q \in P'\) such that \(R(q) = R(p)\) 
      \textbf{and} \(|R(q)| = l\)}{
    $P' \leftarrow P' \cup \{p\}$\;
  }
}
\Return $P'$
\caption{Filtering algorithm to remove redundant retrieved paths.}
\label{alg:filtering}
\end{algorithm}


\subsection{Diverse Path Retrieval Module}
The path retrieval module is a key component of our framework.
It provides the downstream computational model with highly relevant yet manageable information from our augmented KG $\mathcal{G}$.

Our diverse path retrieval algorithm retrieves a set of $K$ shortest diverse paths between two entities, \( \{u, v\} \), from $\mathcal{G}$ using Yen's algorithm ~\cite {yen1971finding}.
We prioritize shortest paths for several reasons. 
Shorter paths capture stronger, more interpretable relationships, while longer paths introduce noise and uncertainty \cite{liben2003link, barabasi2004network}. 
Finally, empirical studies on biomedical KGs further show that meaningful interactions typically occur within a few hops\cite{zitnik2018modeling, himmelstein2016pharmacotherapydb}.

Yen's algorithm extends Dijkstra's algorithm \cite{dijkstra1959note} by iteratively computing shortest paths while temporarily excluding specific edges, thus generating progressively longer, loop-free alternatives.
The output from this process contains redundant paths. 
Therefore, we introduce a filtering algorithm (Algorithm \ref{alg:filtering}) to remove these redundancies.
Our filtering algorithm eliminates paths with duplicate relation sequences of the same path length.
This filtering step results in a diverse set of paths, which are passed to the next module in our framework. 

In cases where only one of \( \{u, v\} \) exists in $\mathcal{G}$, we retrieve its immediate neighbors and their connecting relationships, applying the same filtering algorithm.  
This provides valuable context even when a complete path between two entities cannot be established.  
If neither entity exists in $\mathcal{G}$, no information is retrieved. 


\subsection{Path Integration Module}
This module integrates the retrieved diverse paths to predict interactions between the query entities.
We explore two distinct approaches: LLM-based reasoning and GNN-based prediction.

\heading{LLM Reasoning} \label{llm-reasoning}
We convert the entities and relations from the diverse paths into natural language using predefined dictionaries.
These dictionaries map entities and relations to their respective types and textual representations.
To improve clarity, we append entity type descriptors in parentheses after each entity.

For example, as illustrated in Figure~\ref{fig:pipeline-overview}, consider the path:
\[
p_1 : u \rightarrow r_1 \rightarrow a_1 \rightarrow r_2 \rightarrow v
\]
representing the relationship between \textit{Quetiapine} and \textit{Bipolar disorder}. This path is transformed into the natural language description:

\begin{quote}
    \textit{Quetiapine binds HTR1A (gene) and HTR1A (gene) associates with Bipolar disorder}.
\end{quote}

Following \cite {zhang2023emergingdruginteractionprediction}, if the retrieved relation belongs to an inverse relation category, we convert it into passive voice. For instance, the relation:

\begin{quote}
    \textit{(Disease) downregulates (Gene)}
\end{quote}

is converted into:

\begin{quote}
    \textit{(Gene) is downregulated by (Disease)}.
\end{quote}

This explicit type mapping helps the LLM understand the semantic roles of each entity, even if it is unfamiliar with domain-specific entities like \textit{``HTR1A''}.
This conversion process is applied to all \textit{K} retrieved paths.
In the zero-shot setting, these textualized paths are appended to the original query, providing contextual information for the LLM to perform inference.
In the supervised setting, we can fine-tune the LLM using these textualized paths as training data.
Furthermore, representing the retrieved paths in natural language enhances the explainability of the LLM's predictions.

\heading{GNN Interaction Prediction} \label{gnn-learning}
For GNN-based prediction, instead of directly inputting the large augmented KG into the GNN, we use the diverse paths to construct smaller, query-specific subgraphs.
This approach significantly reduces the computational complexity and allows GNN to focus on the most relevant information.
During training, the GNN learns entity-specific representation by aggregating information from known drug-drug or drug-disease interactions and their corresponding query-specific subgraphs. 
These learned representations are then used to predict the type of interaction between entity pairs.  
During testing, we extend the learned representation space by incorporating new test nodes and their corresponding query-specific subgraphs.
This allows us to evaluate the model's ability to accurately predict unobserved interactions using this extended graph in a supervised inductive setting.

