Drug development and safety assessment have traditionally been time-intensive and costly, often spanning years and requiring billions of dollars \cite{paul2010improve, dimasi2003price}. 
However, recent increases in computational power and the amount of available clinical and biological data are transforming this landscape, enabling faster and more cost-effective approaches to discovering and validating safer drugs \cite{vamathevan2019applications, huang2024foundation, chandak2023building}.
For instance, deep learning models trained on genomic and chemical datasets now predict drug efficacy and toxicity with high accuracy, reducing the reliance on extensive laboratory experiments and accelerating the drug discovery process \cite{chaves2024txllm}. 
Yet, the sheer volume and heterogeneity of these datasets pose integration challenges, making it difficult to extract meaningful insights \cite{gligorijevic2015methods, davidson1995challenges}.

Knowledge Graphs (KGs) offer a structured solution by integrating complex biological relationships, linking diseases, drugs, and proteins into an interconnected framework \cite{hetionet2017, kuhn2007stitch, chandak2023building}, offering a rich resource for understanding and predicting mechanisms and interactions. 
Despite their potential, the scale and complexity of KGs can hinder the efficient retrieval of relevant information \cite{peng2023knowledge}.
While typical KGs contain tens of thousands of nodes and millions of edges, only a small subgraph is relevant for a given task \cite{yu2021sumgnn}, limiting their practical applications in critical areas like drug discovery where precise, targeted insights are essential.

\begin{figure}[t]
    \centering
    \includegraphics[width=\linewidth]{images/overview.pdf} 
    \caption{Schematic representation of the predictive framework for unobserved interactions. (a) Problem formulation: We are given a network of observed interactions among known entities (drugs or diseases) and a broader biomedical network containing additional relationships between various entities (e.g., drugs, diseases, genes, etc.). The task is to predict unobserved interactions between a known entity and an emerging entity or two emerging entities. (b) Augmented KG: The observed interactions and biomedical network are integrated to create a richer representation for the prediction task.}
    \label{fig:overview}
\end{figure}


Graph Neural Networks (GNNs) have shown promise in leveraging KG information for drug discovery, excelling in link prediction, and capturing biological relationships
\cite{Kipf2016SemiSupervisedCW, Schlichtkrull2017ModelingRD, zitnik2018modeling,huang2024foundation}.
However, training GNNs on large-scale KGs introduces substantial computational costs~\cite{abdallah2024task}. 
Additionally, GNNs struggle with inductive generalization, making it difficult to infer previously unobserved relationships for drugs or diseases \cite{hamilton2017inductive}, a drawback in drug discovery where new drugs and diseases continuously emerge.
Large Language Models (LLMs) offer a promising alternative with strong zero-shot generalization capabilities that enable adaptation to inductive tasks with minimal training \cite{kojima2022large, abdullahi2024learning, abdullahi2024retrieval}.
Recent work has also shown that LLMs can be further adapted to specialized domains with KGs to improve their factual accuracy and reduce hallucinations~\cite{abdullahi2024retrieval, abdullahi2024retrieval_jmir,edge2024localglobalgraphrag,wu2024medical}. 
However, effectively extracting meaningful insights from knowledge graphs using LLMs is an ongoing challenge in the community~\citep{fatemi2023talk, perozzi2024let}.
Recent work in drug discovery has proposed subgraph-based extraction methods to predict unobserved interactions from KGs, they often need to train an additional model and do not scale well beyond GNNs. 
Addressing these limitations is crucial for developing a scalable framework for drug discovery.

We introduce \textbf{``\sys''}, a novel retrieval framework designed to extract highly relevant entities and relationships from large biomedical KGs to aid in predicting unobserved drug-drug and drug-disease interactions. 
Unlike traditional KG retrieval methods, \sys generates structured, interpretable paths that can be directly used by LLMs, enabling efficient and accurate zero-shot reasoning.
\sys is particularly valuable in drug discovery, where identifying such interactions can facilitate drug repurposing and safer treatment opportunities.
Figure \ref{fig:overview} illustrates the problem \sys addresses: 
We are given a network of observed interactions among known drugs or diseases and a broader biomedical network containing additional relationships between various entities (e.g., drugs, diseases, genes, etc.). Our goal is to predict unobserved interactions either between a known entity and an emerging entity—an entity whose interaction of interest has not been observed—or between two emerging entities, where neither entity has an observed interaction of interest.
We focus on drug-disease and drug-drug interactions because they play a critical role in drug discovery, clinical decision-making, and patient safety.

As shown in Figure \ref{fig:pipeline-overview}, K-Paths operates in three steps. First, we construct an augmented KG by integrating observed interactions from the training set into the biomedical KG, enhancing its representation of interactions (as depicted in Figure \ref{fig:overview}). 
This enriched KG serves as the network for retrieving biologically meaningful connections.
Next, we design a diversity-aware adaptation of Yen's algorithm \cite{yen1971finding} to efficiently retrieve the $K$ shortest loopless paths connecting entities in an interaction query.  
While Yen's algorithm iteratively finds alternative shortest paths, \sys additionally prioritizes diverse and biologically meaningful
connections, ensuring broader coverage rather than redundant variations of the shortest route.
Finally, the retrieved diverse paths are transformed into natural language representations, enabling LLMs to reason over them and effectively predict interactions. 
Additionally, we show that these paths can be used to construct subgraphs, allowing GNN-based models to operate on a smaller, more focused graph and substantially reducing computational overhead.

To assess the impact of \sys, we evaluate its use in both zero-shot generative and supervised learning paradigms, focusing on drug repurposing and drug-drug interaction (DDI) analysis for emerging drugs and diseases.
Our experiments on benchmark datasets demonstrate that \sys significantly enhances LLM reasoning in a zero-shot setting by providing relevant relational insights. Specifically, on a drug repurposing dataset, \sys improves Llama 8.1B Instruct’s F1-score by 12.45 points, and on an interaction severity task, it achieves a gain of 13.42.
For Llama 70B Instruct, \sys provides comparable improvements, increasing the F1-score by 6.18 and 8.46 points, respectively, on the same tasks.
In a supervised setting, \sys reduces the KG size by 90\% and improves the training efficiency of EmerGNN \cite{zhang2023emergingdruginteractionprediction}, a state-of-the-art GNN model, without compromising model performance, showcasing its scalability.
More importantly, the retrieved paths provide interpretable rationales for predicted interactions, enhancing explainability and offering valuable biological insights.

