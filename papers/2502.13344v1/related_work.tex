\newcommand{\heading}[1]{\smallskip\noindent\textbf{#1}\enspace}


\begin{figure*}[t]
    \centering
    \includegraphics[width=\linewidth]{images/k_paths_framework.pdf} 
    \caption{K-Paths Overview. (1) Given a query about the effect of an entity ($u$) on another entity ($v$), (2) K-Paths extracts reasoning paths from an augmented KG connecting ($u$) and ($v$). (3) These paths are filtered for diversity and (4a) transformed into natural language descriptions for LLM inference. (4b) The retrieved paths can also be used to construct a subgraph, enabling GNNs to leverage more manageable information for training and prediction.}
    \label{fig:pipeline-overview}
\end{figure*}



\subsection{Biomedical knowledge graphs}
Biomedical knowledge graphs have been extensively used to study the complex relationships between drugs, diseases, and genes, leveraging these insights in drug discovery~\citep{zitnik2018modeling, huang2024foundation, yu2021sumgnn, wang2024accurate}. 
These large-scale knowledge graphs have been curated for specialized purposes such as studying drug-disease interactions~\citep{wishart2018drugbank,hetionet2017}, gene-drug interactions~\citep{WhirlCarrillo2021}, and protein-protein interactions~\citep{gao2023hierarchical,jha2022prediction}.
In this work, we focus on drug-disease interactions and drug-drug interactions because they play a critical role in drug discovery, clinical decision-making, and patient safety.

A key limitation of biomedical knowledge graphs is that they can be incomplete in several ways.
First, the knowledge graph may lack observed interactions between existing entities (e.g., drugs or diseases). 
Second, they may lack interactions between existing entities and emerging entities. 
We focus on predicting unobserved interactions among existing and emerging drugs or diseases, as well as interactions involving emerging drugs or diseases in large-scale biomedical knowledge graphs. 
These graphs are inherently incomplete, making such predictions particularly challenging.

\subsection{LLMs for drug discovery}
LLMs are becoming increasingly popular in many biomedical applications~\citep{chaves2024txllm,inoue2024drugagent,liang2023drugchat}.
Several successful applications have been used as conversational agents for drug discovery~\citep{chaves2024txllm}, repurposing~\citep{inoue2024drugagent}, and understanding drug molecules~\citep{liang2023drugchat}. 
A key factor in their success is in-context learning~\citep{brown2020language}, which allows them to process new information—such as examples and text—and reason effectively to generate accurate responses.
This ability is particularly valuable for multi-hop reasoning, where models must integrate information across multiple steps or knowledge sources, such as biomedical knowledge graph paths, to form coherent and reliable conclusions \cite{mavromatis2024gnnraggraphneuralretrieval}.
Recent research has demonstrated that LLMs can leverage multi-hop reasoning to navigate knowledge graph paths and generate faithful answers to complex factual queries~\cite{luo2024reasoninggraphsfaithfulinterpretable}.
However, applying this approach to biomedical knowledge graphs for knowledge completion tasks remains under-explored despite its importance for real-world applications like drug discovery.
In this work, we show that multi-hop reasoning over diverse paths from biomedical knowledge graphs significantly improves the zero-shot performance of off-the-shelf LLMs in predicting interactions between emerging drug entities.

\sys is related to a growing area of research at the intersection of retrieval augmented generation with graphs~\citep{xiong-etal-2024-benchmarking, edge2024localglobalgraphrag,luo2024reasoninggraphsfaithfulinterpretable}.
Recent works ~\citep{xiong-etal-2024-benchmarking, edge2024localglobalgraphrag, wu2024medical} constructs a knowledge graph from unstructured documents from the general domain with LLMs and reasons over it to produce the output. 
\sys uses a highly curated biomedical knowledge graph rather than constructing a knowledge graph on the fly to reason about drug entities and predict their interactions. 
Alternatively, studies like RoG~\citep{luo2024reasoninggraphsfaithfulinterpretable} use an off-the-shelf knowledge graph such as Freebase to retrieve paths between entities and predict their relationship. 
However, this approach requires training an additional planning module to retrieve the paths and reason over all the paths, which is computationally expensive. 
In contrast, \sys uses a heuristic approach to retrieve paths, eliminating the need for training and introducing a filtering step to diversify and reduce the number of paths needed for inference. 

\subsection{Graph neural networks for drug discovery}
For supervised setting, graph neural networks have achieved strong performance in modeling the structure of the biomedical knowledge graph and predicting undiscovered interactions between entities~\citep{zhang2023emergingdruginteractionprediction, wang2024accurate, lin2020kgnn,al2022prediction}.
Several variants of graph convolutional networks have been used to address this problem. 
Furthermore, recent work has used external knowledge graphs such as Hetionet to learn richer representations for the entities~\citep{hetionet2017, wang2024accurate}. 
However, modeling large-scale biomedical knowledge is computationally expensive.
To address this limitation, recent work introduces subgraph-based graph convolutional networks that extract fixed-sized subgraphs before passing them to the graph network.
However, these models are designed for transductive settings. Unlike inductive setting, which is the goal of this work, transductive setting aims at predicting interactions between known entities. This property limits their ability to generalize to emerging entities- which is not ideal for drug discovery.
In our work, we enhance the training knowledge graph using Hetionet and leverage \sys to retrieve diverse paths, enabling more efficient reasoning between existing and emerging entities while minimizing computational overhead.

Our work is closely related to EmerGNN~\citep{zhang2023emergingdruginteractionprediction} but differs in a few key ways. 
EmerGNN is a flow-based graph neural network that augments its knowledge graph with Hetionet and extracts paths from the knowledge graph to predict the interaction between emerging drugs. 
However, they train an additional model and use beam search to extract paths from the augmented knowledge graph.
In contrast, \sys eliminates the need for training an additional model and provides diverse paths for predicting interactions between emerging drugs.
Another limitation of EmerGNN is that their subgraph extraction is optimized for GNN models, limiting their applicability to other model architectures. 
In our work, our extracted paths are flexible and can be incorporated by both GNNs and LLMs, thus demonstrating their usability for both zero-shot and supervised learning.
Furthermore, in Results Section\ref{sec:results}, we show \sys can be integrated into EmerGNN and show improved performance compared to using the entire knowledge graph.
Finally, our method significantly improves zero-shot performance with LLMs without updating model parameters. 


