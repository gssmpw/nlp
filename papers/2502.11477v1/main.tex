\documentclass[10pt,twocolumn,letterpaper]{article}

\usepackage[pagenumbers]{cvpr} 

\newcommand{\CG}{\mathcal{G}\xspace}
\newcommand{\CV}{\mathcal{V}\xspace}
\newcommand{\CE}{\mathcal{E}\xspace}
\newcommand{\CA}{\mathcal{A}\xspace}
\newcommand{\CF}{\mathcal{F}\xspace}
\newcommand{\CR}{\mathcal{R}\xspace}
\newcommand{\CB}{\mathcal{B}\xspace}
\newcommand{\CX}{\mathcal{X}\xspace}
\newcommand{\CK}{\mathcal{K}\xspace}
\newcommand{\CM}{\mathcal{M}\xspace}
\newcommand{\CC}{\mathcal{C}\xspace}
\newcommand{\CL}{\mathcal{L}\xspace}
\newcommand{\CI}{\mathcal{I}\xspace}
\newcommand{\CQ}{\mathcal{Q}\xspace}
\newcommand{\CO}{\mathcal{O}\xspace}
\newcommand{\CP}{\mathcal{P}\xspace}
\newcommand{\CS}{\mathcal{S}\xspace}
\newcommand{\CT}{\mathcal{T}\xspace}
\newcommand{\CJ}{\mathcal{J}\xspace}
\usepackage[para]{footmisc}
\usepackage{subfig}
% \usepackage{subcaption}
% \usepackage{array}
% \usepackage{colortbl}


\def\viz{\emph{viz}\onedot}
\definecolor{cvprblue}{rgb}{0.21,0.49,0.74}
\usepackage[pagebackref,breaklinks,colorlinks,allcolors=cvprblue]{hyperref}

\usepackage{multirow}
\usepackage{graphicx}
\usepackage{subcaption}
\usepackage{algorithm}               
\usepackage{algpseudocode}
\usepackage{setspace}
\usepackage{enumitem}
\usepackage{color, colortbl}
\usepackage{amsmath}
\usepackage{amssymb}
\usepackage{mathtools}
\usepackage{natbib}

\title{Learning to Sample Effective and Diverse Prompts for Text-to-Image Generation}

\author{
{Taeyoung Yun$^{1}$\thanks{Work done during TY's visit to HKUST.} ~ Dinghuai Zhang$^{2}$ ~ Jinkyoo Park$^{1}$ ~ Ling Pan$^{3}$} 
\\
{$^{1}$ Korea Advanced Institute of Science and Technology $^{2}$ Microsoft Research} \\ 
$^{3}$ Hong Kong University of Science and Technology
}

\begin{document}
\maketitle

\begin{abstract}
Recent advances in text-to-image diffusion models have achieved impressive image generation capabilities. 
However, it remains challenging to control the generation process with desired properties (e.g., aesthetic quality, user intention), which can be expressed as black-box reward functions. 
In this paper, we focus on prompt adaptation, which refines the original prompt into model-preferred prompts to generate desired images. While prior work uses reinforcement learning (RL) to optimize prompts, we observe that applying RL often results in generating similar postfixes and deterministic behaviors.
To this end, we introduce \textbf{P}rompt \textbf{A}daptation with \textbf{G}FlowNets (\textbf{PAG}), a novel approach that frames prompt adaptation as a probabilistic inference problem. 
Our key insight is that leveraging Generative Flow Networks (GFlowNets) allows us to shift from reward maximization to sampling from an unnormalized density function, enabling both high-quality and diverse prompt generation.
However, we identify that a naive application of GFlowNets suffers from mode collapse and uncovers a previously overlooked phenomenon: the progressive loss of neural plasticity in the model, which is compounded by inefficient credit assignment in sequential prompt generation. To address this critical challenge, we develop a systematic approach in PAG with flow reactivation, reward-prioritized sampling, and reward decomposition for prompt adaptation.
Extensive experiments validate that PAG successfully learns to sample effective and diverse prompts for text-to-image generation. 
We also show that PAG exhibits strong robustness across various reward functions and transferability to different text-to-image models. 
\end{abstract}

\section{Introduction}
\begin{figure}[t]
    \centering
    \includegraphics[width=0.95\linewidth]{figures/main_figure_half.jpg}
    \caption{Comparison of adapted prompts and their corresponding images of Prompist~\cite{hao2024optimizing} (based on reward-maximizing RL) and our method, PAG. While Promptist leads to mode collapse in the prompt space and converges to similar outputs, PAG achieves high image quality while painting generation diversity.}
    \label{fig:main_figure_preview}
    \vspace{-18pt}
\end{figure}

Recent advances in diffusion models \citep{sohl2015deep, ho2020denoising}, combined with pre-trained text encoders \citep{raffel2020exploring, radford2021learning} have shown remarkable capability in generating creative and photorealistic images conditioned on novel prompts \citep{rombach2022high, ramesh2022hierarchical, saharia2022photorealistic}. However, generating images with desired properties (e.g., aesthetic quality, user intention) remains challenging, as these models are typically optimized for likelihood maximization over training data distributions~\citep{ho2020denoising}.

To further improve the generation process with desired properties, there has been a line of work focusing on fine-tuning diffusion models with human feedback~\citep{blacktraining, fan2024reinforcement,prabhudesai2023aligning, clarkdirectly, zhang2024improving}.
While these methods demonstrate promising results, they rely on access to model parameters, rendering them incompatible with several state-of-the-art closed-source models~\citep{ramesh2022hierarchical, saharia2022photorealistic}.
Moreover, as text-to-image diffusion models continue to grow exponentially in size, the computational cost of direct fine-tuning becomes prohibitively expensive. This challenge is further compounded by the need for model-specific retraining across different text-to-image diffusion models.

In contrast, prompt adaptation has emerged as a promising alternative~\citep{hao2024optimizing, kim2023multiprompter, wang2024discrete}, which aims to improve the initial prompt for generating images with desired properties. This approach eliminates the need for access to model parameters, enabling zero-shot transfer to different text-to-image diffusion models.
Notably, Promptist \citep{hao2024optimizing} employs reinforcement learning (RL) to fine-tune language models for prompt adaptation. While it shows promising results, our analysis reveals a critical limitation: its reward-maximizing principle tends to concentrate the policy on a narrow, high-reward region. This often leads to a deterministic policy that generates similar postfixes, as demonstrated in Figure~\ref{fig:main_figure_preview}. 
Such deterministic behavior can reduce the method to simple heuristics, significantly hindering its generalization capacity across different prompt types and text-to-image diffusion models. It underscores the critical need for approaches that can generate both effective and diverse adapted prompts for text-to-image generation.

In this paper, we introduce \textbf{P}rompt \textbf{A}daptation with \textbf{G}FlowNets (\textbf{PAG}), a novel approach that addresses these fundamental challenges by reformulating prompt adaptation as a probabilistic inference~\citep{bengio2021flow}. Our approach leverages Generative Flow Networks (GFlowNets, ~\citep{bengio2023gflownet}) to learn a generative policy that samples from an unnormalized reward distribution~\citep{bengio2021flow}, which is well-suited for this scenario.
While this approach shows initial promise, we observe that naively fine-tuning language models with GFlowNets for prompt adaptation suffers from a mode collapse issue. Our investigation uncovers a previously unrecognized phenomenon in GFlowNet fine-tuning that mirrors a key principle in neuroscience: gradual hardening of brain circuits~\citep{livingston1966brain, mateos2019impact}. In other words, GFlowNets agent experiences a progressive loss of neural plasticity as certain neural pathways become inactive, diminishing its capacity to learn from and adapt to diverse patterns. Furthermore, inefficient credit assignment across sequential prompt generation process leads to insufficient sample efficiency and ultimately exacerbates mode collapse issue.

To this end, we introduce our novel components in PAG to systematically address these critical challenges in prompt adaptation for text-to-image modeling. 
First, we propose a flow reactivation mechanism to revive dormant neural pathways in the GFlowNets agent, complemented by reward-prioritized sampling to effectively consolidate of high-quality experiences. These two components jointly achieve adaptation flexibility while maintaining training stability.
Building upon this foundation, we develop a progressive reward decomposition scheme in our framework to provide fine-grained learning signals at intermediate generation steps, which enables more precise credit assignment throughout the training procedure.

Our contributions can be summarized as below:
\begin{itemize}
    \item We introduce PAG, a novel framework that reformulates prompt adaptation as a probabilistic inference problem, leveraging GFlowNets to generate both effective and diverse adapted prompts.
    \item We identify and address a previously overlooked challenge in GFlowNets fine-tuning - the progressive loss of neural plasticity leading to mode collapse - through a systematic approach that maintains network expressivity and ensures precise credit assignment.
    \item Extensive experiments show that PAG generates both effective and diverse adapted prompts, exhibits robustness to different reward functions, and enables effective zero-shot transfer to various text-to-image diffusion models. 
\end{itemize}

\section{Preliminaries}
\subsection{Prompt Adaptation}
Let $p_{\theta}$ denote a pre-trained language model that generates improved prompt $\mathbf{y}$ conditioned on the initial prompt $\mathbf{x}$, i.e., $\mathbf{y}\sim p_{\theta}(\cdot\vert\mathbf{x})$. A text-to-image diffusion model $p_{\psi}$ then generates images given text inputs $\mathbf{y}$. Following~\citep{hao2024optimizing}, a reward function $r(\mathbf{x}, \mathbf{y})$ for prompt adaptation is defined as follows:
\begin{align}\label{eq:task_reward}
    &\mathbb{E}_{i_\mathbf{x}\sim p_{\psi}(\cdot\vert\mathbf{x}), i_\mathbf{y}\sim p_{\psi}(\cdot\vert\mathbf{y})}\left[r_{\text{aes}}(i_\mathbf{x}, i_\mathbf{y}) + r_{\text{rel}}(\mathbf{x}, i_\mathbf{y})\right], 
\end{align}
where $r_{\text{aes}}(i_{\mathbf{x}}, i_{\mathbf{y}})=g_{\text{aes}}(i_\mathbf{y}) - g_{\text{aes}}(i_\mathbf{x})$ measures the improvement in aesthetic quality of images using the LAION aesthetic predictor \citep{laion}, $g_{\text{aes}}$. The term $r_{\text{rel}}(\mathbf{x}, i_{\mathbf{y}})=\min(20.0\times g_{\text{CLIP}}(\mathbf{x}, i_{\mathbf{y}}) - 5.6, 0.0)$ quantifies the relevance between generate images from $\mathbf{y}$ and the initial prompt $\mathbf{x}$ using the CLIP similarity function, $g_{\text{CLIP}}$ \citep{radford2021learning}. Our objective is fine-tuning language model parameters $\theta$ to optimize the target reward function:
\begin{align}
\label{eq:objective}
\mathbb{E}_{\mathbf{x}\sim\mathcal{D}}\left[\mathbb{E}_{\mathbf{y}\sim p_{\theta}(\cdot\vert\mathbf{x})}\left[r(\mathbf{x}, \mathbf{y})\right]-\beta\cdot D_{\text{KL}}(p_{\theta}(\cdot\vert\mathbf{x})\Vert p_{\text{ref}}(\cdot\vert\mathbf{x}))\right],
\end{align}
where the $D_{\text{KL}}$ term enforces the generated prompts to be close to natural language that human can understand.

\subsection{Generative Flow Networks (GFlowNets)} \label{sec:bg}
GFlowNets are a family of probabilistic methods that sample compositional objects proportionally to an unnormalized distribution defined by a reward function~\citep{bengio2021flow,bengio2023gflownet}. 
Let $\mathcal{S}$ denote the state space and $\mathcal{A}$ the action space, forming nodes and edges in a directed acyclic graph. We define a unique initial state $s_0$ without incoming edges and set of terminal states $\mathcal{X}\subset\mathcal{S}$ without outgoing edges. A sequence from the initial state $s_0$ to the terminal state $x\in\mathcal{X}$ is called a trajectory, denoted as $\tau=(s_0\rightarrow s_1\rightarrow \cdots\rightarrow s_T=x)$, which is generated sequentially.
% 
Let $R:\mathcal{X}\rightarrow\mathbb{R}_{\geq0}$ be a non-negative reward function defined on terminal states. GFlowNets aim to train a stochastic policy $P_{F}$ that generates samples proportional to the reward, i.e., $P_{F}^{T}(x)\propto R(x)$, where $P_{F}^{T}(x)=\sum_{\tau_{\rightarrow x}}P_{F}(\tau)$ is the marginal likelihood of sampling trajectories that result in $x$ ($\tau_{\rightarrow x}$) from the forward policy:
\begin{align}
    P_F^{T}(x)=\sum_{\tau_{\rightarrow x}}\prod_{t=1}^{T}P_{F}(s_t\vert s_{t-1})\propto R(x).
\end{align}
In practice, GFlowNets can be trained by parameterizing the forward policy with a neural network, $P_{F}(s_t\vert s_{t-1};\theta)$, using various training objectives.

\vspace{5pt}
\noindent \textbf{Trajectory Balance (TB, \citep{malkin2022trajectory})} TB introduces an additional backward policy $P_{B}(s_{t-1}\vert s_t;\theta)$ that models the distribution of parents given a child state and a total flow $Z_{\theta}$ to approximate the partition function. Given a trajectory $\tau=(s_0\rightarrow\cdots\rightarrow s_T=x)$, TB aims to minimize the loss in Eq.~(\ref{eq:tb_loss}). If the loss becomes zero for all possible trajectories, it implies that $P_{F}^{T}(x)\propto R(x)$.
\begin{align}
    \mathcal{L}(\tau;\theta)=\left(\log\frac{Z_{\theta}\prod_{t=1}^{T}P_{F}(s_{t}\vert s_{t-1};\theta)}{R(x)\prod_{t=1}^{T}P_{B}(s_{t-1}\vert s_{t};\theta)}\right)^2.
    \label{eq:tb_loss}
\end{align}
\noindent \textbf{Detailed Balance (DB, \citep{bengio2023gflownet})} DB considers flow matching at the edge level instead of the trajectory level and introduces state flow function $F_{\theta}:\mathcal{S}\rightarrow\mathbb{R}_{\geq0}$, which approximates the total flow through state $s$. 
Given an intermediate transition $(s_{t-1}\rightarrow s_t)$, DB aims to minimize the loss in Eq.~(\ref{eq:db_loss}), with $F_{\theta}(s_t)$ replaced by the terminal reward $R(x)$ at terminal states for $t=T$ (i.e., $F_{\theta}(x)=R(x)$).
\begin{align}
    &\mathcal{L}(s_{t-1}, s_{t};\theta)=\left(\log\frac{F_{\theta}(s_{t-1})P_{F}(s_{t}\vert s_{t-1};\theta)}{F_{\theta}(s_{t})P_{B}(s_{t-1}\vert s_{t};\theta)}\right)^2.
    \label{eq:db_loss}
\end{align}
It is often challenging to train GFlowNets due to delayed credit assignment from terminal-only reward signals \citep{pan2023better, jang2024learning}. The forward-looking (FL) technique addresses this by extending rewards to all states and optimizing $\tilde{F_{\theta}}(s)$ in the sense that $F_{\theta}(s)=\tilde{F_{\theta}}(s)R(s)$ for all states, which is a reparameterization of flows in DB.
The resulting FL-DB objective is to minimize the following loss:
\begin{align}
\label{eq:fl-db-objective}
    &\mathcal{L}(s_{t-1}, s_{t};\theta)=\left(\log\frac{\tilde{F_{\theta}}(s_{t-1})P_{F}(s_{t}\vert s_{t-1};\theta)R(s_{t-1})}{\tilde{F_{\theta}}(s_{t})P_{B}(s_{t-1}\vert s_{t};\theta)R(s_t)}\right)^2.
\end{align}

\subsection{Dormant Neuron Phenomenon}
Recent studies have revealed that scaling deep RL networks faces challenges due to parameter under-utilization~\citep{kumarimplicit, lyleunderstanding, sokar2023dormant}. Following \citet{sokar2023dormant}, we quantify neural plasticity by tracking dormant neurons during the training progresses defined as below, where a neuron $i$ in layer $\ell$ is {${\tau}$-dormant} if its activation score is $s^{\ell}_{i}\leq\tau$. 

\vspace{5pt}
\noindent \textbf{Definition 2.3.1.} Given input dataset $\mathcal{D}$, let $h_i(\mathbf{x})$ be the activation of neuron $i$ in layer $\ell$ under input $\mathbf{x}\in\mathcal{D}$, then its activation score is defined as follows:
\begin{align}\label{eq:dormant}
    s^{\ell}_{i}=\frac{\mathbb{E}_{\mathbf{x}\in\mathcal{D}}\vert h^{\ell}_{i}(\mathbf{x})\vert}{\frac{1}{H^{\ell}}\sum_{k\in h}\mathbb{E}_{\mathbf{x}\in\mathcal{D}}\vert h^{\ell}_{k}(\mathbf{x})\vert}.
\end{align}
where $H^l$ is the number of hidden units in the $l$th layer.

\section{Prompt Adaptation with GFlowNets}
In this section, we present \textbf{P}rompt \textbf{A}daptation with \textbf{G}FlowNets (\textbf{PAG)}, a novel framework that reformulates prompt adaptation as probabilistic inference through GFlowNets-based language model (LM) fine-tuning. 
We begin by outlining how we can fine-tune LMs with GFlowNets to satisfy our objective. Next, we investigate the critical challenge of mode collapse in this framework. To systematically address this challenge, we present our key technical advances that enhance GFlowNets for prompt adaptation. \Cref{fig:overview} illustrates the overview of our method.

\subsection{Problem Formulation}\label{sec:mode_collapse}
Unlike previous RL-based approaches like Prompist~\cite{hao2024optimizing}, which maximize the reward proxy and often converge to deterministic behaviors, we formulate prompt adaption as a probabilistic inference problem~\citep{zhao2024probabilistic} based on GFlowNets (as introduced in Section~\ref{sec:bg}). This formulation enables us to learn a policy that samples high-quality prompts while preserving the coverage of reference policy. Specifically, the optimal policy for Eq. \eqref{eq:objective} can be analytically derived as follows:
\begin{align}\label{eq:reward}
p_{\theta}^{*}(\mathbf{y}\vert\mathbf{x})\propto p_{\text{ref}}(\mathbf{y}\vert\mathbf{x})\cdot\exp\left(\frac{1}{\beta}r(\mathbf{x}, \mathbf{y})\right)=:R(\mathbf{x}, \mathbf{y}).
\end{align}
This formulation reduces prompt adaptation to learning how to sample from the unnormalized density function defined by rewards $R$, aiming to match the underlying reward distribution rather than solely maximizing the reward. A GFlowNet with such reward specification will learn a stochastic policy that generates prompts proportional to their rewards, thereby ensuring both effectiveness and diversity. Please refer to \cref{app:diff_btw_rl_and_gfn} for more discussion between RL and GFlowNets.

\begin{figure*}[t]
    \centering
    \includegraphics[width=0.9\textwidth]{figures/overview_new.png}
    \caption{The high-level illustration of PAG. Given an initial prompt, LM generates adapted prompts by PAG. Then, we generate images from prompts and get a reward. Using observations, we fine-tune LM as a GFlowNet policy to generate prompts proportional to reward.}
    \label{fig:overview}
    \vspace{-12pt}
\end{figure*}
\begin{figure}[t]
    \centering
    \includegraphics[width=\linewidth]{figures/mode_collapse_new.png}
    \caption{Mode collapse issue in prompt adaptation with a naive application of GFlowNets. The proportion of dormant neurons steadily increases (left), while the diversity of generated prompts significantly decreases over training iterations (right).}
    \label{fig:summary_mode_collapse}
    \vspace{-12pt}
\end{figure}

\vspace{5pt}
\noindent\textbf{Mode Collapse Issue in GFlowNets Fine-tuning}
While GFlowNets are designed to sample from the reward distribution for discovering high-quality and diverse candidates, our empirical analysis reveals a significant challenge in prompt adaptation, specifically when fine-tuning language models with GFlowNets after the supervised fine-tuning stage: the policy consistently generates similar prompts despite the vast token space as training progresses.
As shown in \Cref{fig:summary_mode_collapse}, directly applying GFlowNets for fine-tuning language models leads to a substantial reduction in the diversity of prompts. This outcome contradicts the original objective of GFlowNets, suggesting underlying limitations in the training dynamics.

To systematically diagnose the mode collapse issue, we investigate the learning behavior and neural activation patterns in the flow model during fine-tuning. By tracking the dormant neuron ratio as defined in \cref{eq:dormant}, we uncover a key finding largely overlooked in previous research -- the proportion of dormant neurons consistently increases throughout training, as illustrated in the left part in Figure~\ref{fig:summary_mode_collapse}. 
This progressive loss of neural plasticity largely constrains the capacity of the model to learn from and adapt to diverse regions in the prompt space. As more neurons become inactive, the model becomes inflexible in exploring the prompt space, leading to the generation of highly similar prompts.

The mode collapse is further exacerbated by inherent learning challenges in the sequential prompt generation process, where GFlowNets are primarily guided by terminal rewards that are only available after completing entire adapted prompts. It leads to inefficient credit assignment as the model struggles to attribute these terminal rewards into individual tokens in the generation process. Without clear feedback on which intermediate choices contribute to successful outcomes, the model tends to conservatively exploit discovered high-reward patterns rather than exploring diverse alternatives, as evidenced by the increasingly similar prompt patterns shown in Figure~\ref{fig:summary_mode_collapse}. 
These two challenges create a self-reinforcing cycle: the loss of plasticity limits the ability of GFlowNets to learn from diverse samples, while inefficient credit assignment hinders effective exploration of diverse alternatives, ultimately leading to the generation of increasingly similar prompt patterns and exacerbates the mode collapse.

These insights motivate our development of novel training mechanisms tailored for prompt adaptation with GFlowNets, which we detail in the following sections. 

\subsection{Proposed Method}
Based on our observations shown in \Cref{fig:summary_mode_collapse}, we recognize that the mode collapse problem emerges when GFlowNets struggles to maintain expressive capacity while learning from solely on terminal reward in the sequential process.

To systematically address these challenges, we propose a comprehensive mechanism that encourages GFlowNets to retain expressivity targeting three complementary aspects: i) flow reactivation strategy that maintains the plasticity of neural network for exploring diverse prompt patterns, ii) reward-prioritized sampling that allows the model to focus on high-reward experiences, and iii) decomposed reward structure that provides fine-grained learning signals throughout the generation process with advanced training guidance, 
which helps the model explore diverse prompts and resist the tendency toward mode collapse. 
The overall training framework is summarized in Algorithm~\ref{alg:main}.

\vspace{5pt}
\noindent\textbf{Flow Reactivation}
It has been proven that periodic network reset is effective in RL for maintaining neural plasticity~\citep{nikishin2022primacy, zhangconfronting, dsample, liu2024neuroplastic}. Drawing inspiration from this approach, we introduce a targeted flow reactivation mechanism that reinitializes only the last layer of the flow function periodically every $M$ steps.
Note that we do not reset the parameters of the forward policy, which directly interacts with the environment for sampling prompts, as it could cause drastic changes that destabilize training.

\vspace{5pt}
\noindent\textbf{Reward-Prioritized Sampling}
However, relying solely on reset strategies~\citep{nikishin2022primacy} encounter significant challenges in prompt adaptation due to the vast token space, which requires the model to undergo extensive re-exploration to re-discover previously identified high-reward regions, and can therefore impact training efficiency.

To maintain the expressivity of flow function while preventing a significant drop in performance that leads to instability, we leverage reward-prioritized sampling during off-policy training of GFlowNets \citep{shen2023towards, kim2024adaptive}.
A key insight of our approach lies in maintaining consistent access to high-quality prompts through prioritized sampling, which serves as a warm-up for accelerating high-quality knowledge recovery. 
Specifically, we sample a batch of prompts from the replay buffer with probabilities proportional to their rewards as follows:
\begin{align}\label{eq:prt}
    (\mathbf{x, y})\sim P_{\mathcal{B}}(\mathbf{x}, \mathbf{y})=\frac{\exp\left(R(\mathbf{x}, \mathbf{y})\right)}{\sum_{(\mathbf{x}, \mathbf{y})\in\mathcal{B}}\exp\left(R(\mathbf{x}, \mathbf{y})\right)}.
\end{align}
This reward-prioritized sampling naturally complements the flow reactivation process. While reset maintains the expressivity of the model, the continuous exposure to high-reward prompts ensures efficient knowledge retention. Through this principled approach, we can prevent the GFlowNet policy from forgetting the previously discovered high-scoring regions and can quickly regain its ability to navigate towards promising regions in the token space.

\vspace{5pt}
\noindent\textbf{Reward Decomposition}
Beyond maintaining model plasticity and high-quality data retention, a critical challenge in the mode collapse problem in prompt adaption lies in the inefficient credit assignment problem~\citep{pan2023better, jang2024learning} as discussed in Section~\ref{sec:mode_collapse}.
To address this issue, we propose a progressive reward decomposition scheme with advanced learning objectives. By carefully analyzing our reward function, we find that the likelihood term $p_{\text{ref}}(\mathbf{y}\vert\mathbf{x})$ in the reward function naturally admits a step-wise decomposition.
In other words, we can precisely extend the domain of our reward function from the set of terminal states to all possible states:
\begin{align}
    R(\mathbf{x}, y_{0:t})=
    \begin{dcases}
        p_{\text{ref}}(y_{0:t}\vert \mathbf{x}) & \text{if }t\neq T \\
        p_{\text{ref}}(\mathbf{y}\vert \mathbf{x})\exp\left(\frac{1}{\beta}r(\mathbf{x}, \mathbf{y})\right) & \text{otherwise}
    \end{dcases}
\end{align}
This enables us to extend the FL-DB objective in~\cref{eq:fl-db-objective}  by incorporating local credit signals at each step by minimizing the loss $\mathcal{L}(\mathbf{x},\mathbf{y};\theta)=\sum_{t=0}^{T-1}\mathcal{L}(\mathbf{x},\mathbf{y}_{0:t+1};\theta)$, where 
\begin{align}\label{eq:fl_db}
    &\mathcal{L}(\mathbf{x},\mathbf{y}_{0:t+1};\theta)\\
    &=\big(\log\tilde{F_{\theta}}(y_t\vert\mathbf{x}, y_{0:t-1})+\log P_{F}(y_{t+1}\vert\mathbf{x}, y_{0:t};\theta)\nonumber \\
    &+\log R(\mathbf{x}, y_{0:t}) - \log\tilde{F_{\theta}}(y_{t+1}\vert\mathbf{x}, y_{0:t})-\log R(\mathbf{x}, y_{0:t+1})\big)^2,\nonumber
\end{align}
with $F_{\theta}(y_{t+1}\vert\mathbf{x}, y_{0:t})=\tilde{F}_{\theta}(y_{t+1}\vert\mathbf{x}, y_{0:t})R(\mathbf{x}, y_{0:t})$. This enables the model to effectively assess token-level decisions, leading to an increase of diversity and mitigating mode collapse, as demonstrated in Section~\ref{sec:main_res}.

\documentclass{MITstyle}

%\usepackage[table]{xcolor}
\usepackage{chngcntr}
\usepackage{hyperref}
\usepackage{microtype}

\title{A Lightweight and Extensible Cell Segmentation and Classification Model for Whole Slide Images}

\author{Nikita Shvetsov~$^{1, }$\footnote{Correspondence e-mail: nikita.shvetsov@uit.no}, Thomas K. Kilvaer~$^{2, 3}$, Masoud Tafavvoghi~$^{4}$, Anders Sildnes~$^{1}$, \\ Kajsa Møllersen~$^{4}$, Lill-Tove Rasmussen Busund~$^{5, 6}$, Lars Ailo Bongo~$^{1}$ \\
%
\vspace{1em} % Space between authors and afilliations
%
\normalfont{\small $^{1}$Department of Computer Science, UiT The Arctic University of Norway}\\
\normalfont{\small $^{2}$Department of Oncology, University Hospital of North Norway}\\
\normalfont{\small $^{3}$Department of Clinical Medicine, UiT The Arctic University of Norway}\\
\normalfont{\small $^{4}$Department of Community Medicine, UiT The Arctic University of Norway}\\
\normalfont{\small $^{5}$Department of Medical Biology, UiT The Arctic University of Norway} \\
\normalfont{\small $^{6}$Department of Clinical Pathology, University Hospital of North Norway} %\vspace{2em}
}

\begin{document}
\maketitle

\section*{Abstract}

% \begin{abstract}
% Developing clinically useful cell-level analysis tools in digital pathology remains challenging due to limitations in dataset granularity, inconsistent annotations, computational demands of advanced models, and difficulties in integrating new technologies into clinical workflows. To address these challenges, we propose a multi-faceted solution that enhances data quality, model performance, and usability to create a lightweight and extensible cell segmentation and classification model.

% First, we update data labels by employing a cross-relabeling process that refines the labels of two existing datasets, PanNuke and MoNuSAC, to create a new unified dataset with enhanced granularity, encompassing seven distinct cell types. Second, we leverage the H-Optimus foundation model as a fixed encoder to improve feature representation for simultaneous cell segmentation and classification tasks. Third, to address the computational demands of foundation models, we employ knowledge distillation to reduce model size and complexity while maintaining comparable performance. Finally, to facilitate integration into clinical workflows, we integrate the distilled model into the QuPath software, a widely used open-source platform in digital pathology.

% Our results demonstrate improvements in cell segmentation and classification performance using the H‑Optimus-based model compared to a CNN-based model. Specifically, the average $R^2$ improved from 0.575 to 0.871, and the average $PQ$ score improved from 0.450 to 0.492, indicating better alignment with actual cell counts and enhanced segmentation and classification quality. Furthermore, the distilled student model maintains performance comparable to the larger foundation model while reducing the parameter count by a factor of 48.
% Overall, by reducing computational complexity and integrating it into existing workflows, the proposed approach may significantly impact diagnostic processes, reduce the workload of pathologists, and contribute to improved patient outcomes. Though our approach shows potential enhancements in efficiency and usability of cell segmentation and classification models in digital pathology, extensive validation is needed to deploy these models in clinical practice.
% \end{abstract}

%%% shortened abstract
\begin{abstract}
Developing clinically useful cell-level analysis tools in digital pathology remains challenging due to limitations in dataset granularity, inconsistent annotations, high computational demands, and difficulties integrating new technologies into workflows. To address these issues, we propose a solution that enhances data quality, model performance, and usability by creating a lightweight, extensible cell segmentation and classification model. 

First, we update data labels through cross-relabeling to refine annotations of PanNuke and MoNuSAC, producing a unified dataset with seven distinct cell types. Second, we leverage the H-Optimus foundation model as a fixed encoder to improve feature representation for simultaneous segmentation and classification tasks. Third, to address foundation models' computational demands, we distill knowledge to reduce model size and complexity while maintaining comparable performance. Finally, we integrate the distilled model into QuPath, a widely used open-source digital pathology platform. 

Results demonstrate improved segmentation and classification performance using the H-Optimus-based model compared to a CNN-based model. Specifically, average $R^2$ improved from 0.575 to 0.871, and average $PQ$ score improved from 0.450 to 0.492, indicating better alignment with actual cell counts and enhanced segmentation quality. The distilled model maintains comparable performance while reducing parameter count by a factor of 48. By reducing computational complexity and integrating into workflows, this approach may significantly impact diagnostics, reduce pathologist workload, and improve outcomes. Although the method shows promise, extensive validation is necessary prior to clinical deployment.
\end{abstract}
\clearpage

\section{Introduction}
In digital pathology, accurate segmentation and classification of cells are crucial for many diagnostic, prognostic, and predictive analyses \cite{Jaber_Beziaeva_etal._2019,Lin_Pan_etal._2022,Park_Ock_etal._2022,Shen_Choi_etal._2024}. Nowadays, developments in computational pathology offer multiple solutions \cite{H._Qu_P._Wu_etal._2020,Javed_Mahmood_etal._2020} to utilize cell-level datasets to train machine learning models that solve these problems. The quality and specificity of training datasets are critical for robust and accurate models. Adhering to the principle of "garbage in, garbage out", it is essential to ensure that these datasets are extensively and accurately labeled with distinct classes that reflect the diverse biological characteristics of different cell types. Unfortunately, the number of open-source datasets comprising such high-quality annotations is limited. Existing cell segmentation datasets \cite{Gamper_Koohbanani_etal._2019,Graham_Vu_etal._2019,Verma_Kumar_etal._2021} may offer extensive annotations for certain cell types while providing more general labels for others. For example, in PanNuke, which is one of the largest open-source datasets comprising labeled cells, various types of morphologically and functionally different inflammatory cells like macrophages and lymphocytes are clustered in a broad "inflammatory" class. Consequently, these classes are frequently omitted from analyses or aggregated into broader meta-classes \cite{Gamper_Koohbanani_etal._2020} and likely interfere with other cell classes included in the dataset. This and similar inconsistencies in annotation granularity limit the ability of machine learning models to learn the comprehensive and nuanced features necessary for accurate cell segmentation and classification. To address these challenges, methods for refining and standardizing dataset annotations are essential to enhance the quality of training data.

A complementary approach to mitigate the absence of high-quality training data is the use of foundation models. Foundation models as encoders are defined as large-scale, versatile networks pre-trained on vast, diverse datasets using self-supervised learning, contrasting with convolutional neural network (CNN) pre-trained encoders that rely on supervised learning with labeled data. In practice, foundation models leverage enormous amounts of weakly or unlabeled data from millions of whole slide images (WSIs) and employ self-attention mechanisms to capture long-range dependencies and global context \cite{Chen_Ding_etal._2024,Saillard_Jenatton_etal._2024,Vorontsov_Bozkurt_etal._2024,Xu_Usuyama_etal._2024}. As a consequence, foundation models are able to produce transferable feature representations across different cell types and tissue environments. The feature representations can be leveraged by decoder networks to produce segmentation masks and pixel-level classifications. Because foundation models have comprehensive feature representations, they can be effectively fine-tuned using much smaller amounts of cell-level data compared to the large datasets needed to train models from scratch. Furthermore, foundation models incorporate adversarial training elements or contrastive learning \cite{Chen_Ding_etal._2024,Xu_Usuyama_etal._2024}, enhancing their resilience and adaptability by exposing them to challenging and varied scenarios during training. This may result in more generalizable models, often making them well-suited for diverse and complex tasks in digital pathology.

Despite the inherent advantages of foundation models, their deployment for practical use faces its own obstacles. In particular, they require substantial computational power, financial investments and rigorous testing to ensure reliability and efficacy for a given task \cite{Akkus_Dangott_etal._2022,Dragomir_Cocuz_etal._2022,Go_2022,Jafri_Farooqui_etal._2024}. Moreover, while foundation models enhance feature representation and performance, they depend on the quality of available annotations for decoder fine-tuning and, like any other model, cannot resolve existing inconsistencies or ambiguities in data labels. Therefore, there remains a critical need for solutions that address both data quality and practical deployment considerations.
Further, integrating new technologies into existing clinical workflows often encounters resistance, as it necessitates adjustments to established diagnostic processes. So, there is a need to develop solutions that could be integrated into current practices, minimizing the burden on medical professionals to adopt new tools \cite{King_Williams_etal._2023}.

Existing solutions \cite{Goldsborough_Philps_etal._2024,Hörst_Rempe_etal._2024}, while addressing some aspects of these challenges, fall short in providing a comprehensive approach. To address the data quality and clinical deployment issues, we propose a multi-faceted solution that encompasses data refinement, model optimization, and integration with existing pathology tools (\hyperref[fig:fig1]{Figure 1}). The outcome is a lightweight cell segmentation and classification model that can be integrated into digital pathology workflows for practical clinical use.

\begin{figure}[h!]
    \centering
    \includegraphics[width=\textwidth, height=0.82\textheight, keepaspectratio]{images/Figure_1.pdf}
    \caption{Overview of the proposed solution, including 1) Data refinement using cross-relabeling, 2) Teacher model development and fine tuning, 3) Student model optimization with knowledge distillation and 4) Student model and QuPath integration}
    \label{fig:fig1}
\end{figure}
\clearpage

Our approach begins with preparing the data for the fine-tuning and training of the machine learning models. We create a refined dataset, acquired via cross-relabeling two cell-level datasets, enhancing annotation specificity and consistency of the labeled data. Subsequently, we create a cell segmentation and classification model based on the foundation model. We leverage the foundation model as a fixed encoder and fine-tune a decoder using the refined dataset to improve generalization across diverse tissue- and cell types.
To ensure that the model remains lightweight and deployable in a possibly resource-constrained environment, we employ knowledge distillation to approximate the functionality of the foundation model. Finally, to facilitate the practical application of our model in digital pathology workflows, we integrate it with the QuPath \cite{Bankhead_Loughrey_etal._2017} application. Each methodological component contributes to the overarching goal of enhancing model performance, generalizability, and usability in clinical settings.

The primary contributions of this paper are:
\begin{enumerate}
    \item \textit{Data labels refinement through cross-relabeling:}
    
    We propose a new method for refining labels of cell-level datasets through cross-relabeling. This method employs classification models to re-label broad and ambiguous instances, resulting in a more diverse dataset. Our evaluation demonstrates that these classification models achieve high accuracy on test subsets, indicating the reliability of the method for label refinement.

    \item \textit{Enhanced model performance via foundation models:}
    
    We employ a foundation model as a feature extractor for the cell segmentation and classification task. In comparison with training a CNN model from scratch, the foundation model backbone only needs fine-tuning, which significantly reduces training time, computational resources and data requirements. We show that using a foundation model encoder leads to better performance in cell segmentation and classification networks than using a CNN-based encoder. This improvement may enable the model to generalize more effectively across various tissue types and imaging methods.
    
    \item \textit{Model optimization through knowledge distillation:}
    
    We show that a smaller student model trained using knowledge distillation on the refined dataset obtained via our cross-relabeling approach from a foundation model achieves comparable performance in cell segmentation and quantification tasks. As a result, this model is more suitable for deployment in environments without high-performance computing resources.
    
    \item \textit{Integration with QuPath:}
    
    We integrate the distilled cell segmentation and classification model into QuPath, a widely used open-source digital pathology platform, to accelerate clinical adaptation by enabling pathologists to more easily incorporate advanced computational tools into their existing workflows.
\end{enumerate}

Through these methodological steps, we aim to bridge the gap between advanced machine learning techniques and practical clinical applications, making accurate and efficient digital pathology accessible in a broader range of healthcare settings.

\section{Refining Existing Datasets Using Cross-Relabeling}
To address the limitations of sparse and ambiguous labeling of cell-level datasets, we propose a generalizable cross-relabeling strategy that can be applied to any dataset containing broadly categorized or imprecisely labeled cell types. This approach involves training and subsequently leveraging classification models to refine broad categories into more specific or biologically relevant classes.
When applied to cell-level data, the methodology includes extracting individual cell images from the dataset patches, preprocessing these images to standardize the size and accommodate partial cells, and then training deep learning classifiers capable of distinguishing between the finer cell subtypes within the coarser categories. 
To illustrate our approach, we focus on the PanNuke \cite{Gamper_Koohbanani_etal._2020, Gamper_Koohbanani_etal._2019} and MoNuSAC \cite{Verma_Kumar_etal._2021} datasets that we have used to train models for cell quantification in our previous works \cite{Shvetsov_Grønnesby_etal._2022,Shvetsov_Sildnes_etal._2024}. We find that for better cell differentiation we have to introduce more granular labels. PanNuke includes a broad classification of "inflammatory" cells, encompassing lymphocytes, macrophages, and neutrophils. Each cell type differs significantly in structure, function, and clinical relevance. Conversely, MoNuSAC uses the label "epithelial" for a class that comprises both benign epithelial cells and malignant neoplastic cells. This practice makes it challenging to differentiate between benign and malignant epithelial cells in the dataset, which is a critical distinction when identifying tumor areas within tissue samples. To address these issues, we implement a cross-relabeling strategy as shown in \hyperref[fig:fig2]{Figure 2}. The key components are two classification models: one is trained on singular cell images from PanNuke data to classify the epithelial meta-class into epithelial and neoplastic classes. The other is trained on MoNuSAC to refine the inflammatory class into lymphocytes, neutrophils, and macrophages.

\begin{figure}[h!]
    \centering
    \includegraphics[width=\textwidth]{images/Figure_2.pdf}
    \caption{Refined dataset generation via cross relabeling}
    \label{fig:fig2}
\end{figure}

The refining approach consists of three consecutive steps. The first is the preprocessing step, in which we extract individual cells from both datasets (\hyperref[fig:fig3]{Figure 3}). The specifics of PanNuke and MoNuSAC patch preparation before cell preprocessing are provided in \hyperref[chap:S1]{Appendix S1}.

\begin{figure}[h!]
    \centering
    \includegraphics[width=\textwidth]{images/Figure_3.pdf}
    \caption{Cell instances preprocessing including (1) cell map extraction, (2) bounding box delineation, (3) adjusting cell boxes and (4) cropping and resizing of cell images}
    \label{fig:fig3}
\end{figure}

During preprocessing, we extract cell type maps from the ground truth label mask and calculate bounding boxes around each cell instance. To accommodate partial cells at patch borders, a common issue in cropped patch images, we employ mirror padding and extend the field of view of the cell label by 15 pixels to capture adjacent cells. We then crop and resize the identified regions to $64 \times 64$ pixels using bicubic interpolation.

The preprocessed PanNuke dataset comprises 68,031 neoplastic and 23,207 epithelial cell images, while MoNuSAC comprises  33,104 lymphocytes, 1,252 neutrophils, and 1,695 macrophages, which we subsequently use in training cell classification models and classifying the cell image data \hyperref[fig:S2]{Appendix Figure S2 (1)}. 

The next step is to train two distinct ResNet50-based classifiers tailored to address the specific labeling challenges inherent in each dataset. We use ResNet50 for classification models due to its proven effectiveness for image classification tasks in histopathology \cite{pan2022reviewmachinelearningapproaches}, and its compatibility with small images. For the PanNuke dataset, we design the classifier, trained on MoNuSAC data, to disaggregate the heterogeneous "inflammatory" cell category into distinct subtypes: lymphocytes, macrophages, and neutrophils. Similarly, for the MoNuSAC dataset, the classifier is trained on PanNuke data and distinguishes between benign and malignant epithelial cells within the overarching "epithelial" label. By applying these targeted classifiers to their respective datasets, we assign more specific labels to individual cell instances, thus enabling us to create a unified dataset.
To ensure a balanced representation of classes, we train both models on datasets that had been equalized to match the size of the least represented class. Thus, we obtain datasets comprising 23,207 samples per class for PanNuke and 1,252 samples per class for MoNuSAC data. Next, we partition both of them into training (70\%), validation (20\%), and testing (10\%) subsets. To mitigate the risk of overfitting, we use a single dropout layer with a rate of p=0.5 in both models and data augmentation using randomized color perturbations, rotation, and horizontal and vertical flipping. We employ AdamW optimizer and the cross-entropy loss function for the training criterion.

To evaluate the two trained models, we measure the classification accuracy on the respective test subsets. The accuracies on the test subset for both classifiers are presented in \hyperref[tab:1]{Table 1}. The PanNuke model achieves an average accuracy of 93.57\%, with higher accuracy for neoplastic cells (96.06\%) compared to epithelial cells (86.26\%). The confusion matrix in Figure A3.1 shows that the model predominantly distinguishes accurately between epithelial and neoplastic tissues, with a substantial number of correct classifications and relatively few misclassifications. The MoNuSAC model demonstrates an average accuracy of 98.92\%, excelling in classifying lymphocytes (99.67\%) and macrophages (94.12\%), with lower performance for neutrophils (85.71\%). The confusion matrix in Figure A3.2 shows that the model identifies lymphocytes and performs reasonably well with macrophages and neutrophils.

\begin{table}[h!]
\renewcommand{\arraystretch}{1.5}
  \centering
  \caption{Cell classification results for PanNuke and MoNuSAC trained models (CI 95\%).}
  \label{tab:1}
  \begin{tabular}{|l|c|c|}
   \hline
   %\rowcolor{gray!30}
    Accuracy               & PanNuke model              & MoNuSAC model              \\
    \hline
    Average      & 0.936 (0.931--0.941)         & 0.989 (0.986--0.993)        \\
    \hline
    Neoplastic   & 0.961 (0.956--0.965)         & -                          \\
    \hline
    Epithelial   & 0.863 (0.849--0.877)         & -                          \\
    \hline
    Lymphocytes  & -                          & 0.997 (0.995--0.999)        \\
    \hline
    Neutrophils  & -                          & 0.857 (0.796--0.918)        \\
    \hline
    Macrophages  & -                          & 0.941 (0.906--0.976)        \\
    \hline
  \end{tabular}
\end{table}

Finally, during the last step, we use the model trained on PanNuke data for epithelial cells in MoNuSAC and the model trained on MoNuSAC for the inflammatory cells class in PanNuke. Specifically, we use classifier models to relabel epithelial cells in MoNuSAC and inflammatory cells in PanNuke data. Then we combine cells with refined labels and the rest of the cells in both datasets to create a refined dataset (\hyperref[fig:S2]{Appendix Figure S2 (2)}). The process of relabeling cells and visualizing them on a patch is shown in \hyperref[fig:fig4]{Figure 4}. The cell counts in the refined dataset are provided in \hyperref[tab:S4]{Appendix Table S4}.

\begin{figure}[h!]
    \centering
    \includegraphics[width=\textwidth, height=0.42\textheight, keepaspectratio]{images/Figure_4.pdf}
    \caption{Cell relabeling procedure for epithelial and inflammatory cell classes}
    \label{fig:fig4}
\end{figure}

%\hfill

Relabeling and combining datasets have been explored in a prior study \cite{Parulekar_Kanwat_etal._2023}, where consecutive fine-tuning on multiple datasets was employed to account for hierarchical class label structures. While the method presented in \cite{Parulekar_Kanwat_etal._2023} is intuitive, it often lacks consistency and requires multiple fine-tuning runs, which can be cumbersome and time-consuming. 
In contrast, cross-relabeling simplifies this process by using specialized classification models tailored to each dataset's specific labeling challenges. This approach provides better transparency and produces a unified dataset encompassing seven distinct cell types across multiple tissue samples, enhancing data diversity for further model training or fine-tuning.

Despite these improvements, cross-relabeling does not entirely resolve issues related to poor labeling quality or the amount of labeled data. Specifically, our results show lower accuracies persist for underrepresented classes, such as macrophages, which may stem from a limited sample availability and intrinsic challenges in distinguishing these cells based solely on H\&E staining. Furthermore, while our method enhances label specificity, it relies on the initial quality of the broad labels; thus, any fundamental inaccuracies in the original annotations can propagate through the relabeling process. Addressing the overall problem of limited data labels may require integrating additional data sources or utilizing complementary immunohistochemical staining methods.
Although the reported performance metrics are obtained from evaluations on the native test sets of each dataset, it is important to note that the primary application of these classifiers is to perform cross-relabeling, where a model trained on one dataset (e.g., PanNuke) is applied to another (e.g., MoNuSAC) and vice versa. We acknowledge that a more systematic evaluation of cross-dataset generalization is needed and could be performed in future work.

Overall, the refined dataset produced by our approach can enhance the supervised training or fine-tuning of cell segmentation and classification models, especially those that utilize pre-trained foundation models to improve feature extraction robustness. In addition, these models can detect nuanced classes that enable researchers to conduct more detailed analyses of biological processes in computational pathology.

\section{Foundation models for robust cell segmentation and classification}

Accurate cell segmentation and classification in digital pathology are hindered by limited labeled data and the fact that conventional CNNs are unable to capture global contextual information due to their local receptive field constraints \cite{Gheflati_Rivaz_2022,Yang_Marcus_etal.}. Traditional approaches in cell quantification have predominantly relied on CNN encoders, such as ResNet50, given their proven effectiveness in semantic segmentation tasks \cite{Deshmane_2023,Graham_Vu_etal._2019,Mukasheva_Koishiyeva_etal._2024,Stringer_Wang_etal._2021}. However, approaches that include fine-tuning of pretrained CNNs, data augmentation, and stain normalization to partially increase data variability and address staining differences often fail to achieve the necessary generalization and robustness across diverse tissue types and staining conditions \cite{G._Wang_W._Li_etal._2018,Gao_Bagci_etal._2018,Karim_El_Khoury_Martin_Fockedey_etal._2021}.

To overcome these challenges, we leverage an encoder-decoder network that uses a foundation model as the encoder and a CNN upsampling decoder (\hyperref[fig:fig5]{Figure 5}) for simultaneous cell segmentation and classification in 2D patches extracted from WSIs. Foundation models with transformer-based architectures are viable alternatives to CNN-based encoders \cite{Shamshad_Khan_etal._2023,Sourget_2023}. They enable the creation of more advanced architectures that can decode or transform learned features more effectively \cite{Chen_Duan_etal._2023,Cheng_Misra_etal._2022,Xie_Wang_etal._2021}.

\begin{figure}[h!]
    \centering
    \includegraphics[width=\textwidth]{images/Figure_5.pdf}
    \caption{UNETR-like model with foundational model as backbone}
    \label{fig:fig5}
\end{figure}

By utilizing a transformer-based encoder, we incorporate global contextual information into the feature extraction process, which is a key advantage of such architectures \cite{Chen_Lu_etal._2021}. This foundation model integration facilitates accurate pixel-wise segmentation and classification without the need for extensive encoder training, thereby potentially improving generalization across varied cellular structures and tissue types.
In our implementation, we employ a modified UNETR \cite{Hatamizadeh_Tang_etal._2021} architecture that combines a vision transformer (ViT) \cite{Dosovitskiy_Beyer_etal._2021} encoder with a CNN-based decoder. The encoder utilizes the pretrained H-Optimus foundation model, which contains 1.1 billion parameters and is trained on over 500,000 H\&E stained WSIs \cite{Saillard_Jenatton_etal._2024}. We extract outputs from four evenly spaced transformer blocks $Z_i$, where $i \in [1, 14, 26, 38]$, to serve as residual connections for the CNN decoder. We select these blocks based on our observation that features from non-adjacent levels of the encoder lead to better overall performance on the test subset.

The CNN decoder upsamples the feature representations, acquired from the transformer blocks, to generate an intermediate vector that is handled by two task-specific layers that generate cell segmentation and classification masks. The first task-specific layer is the ‘Cellpose head’,  which is used to delineate cell instances. The layer generates horizontal and vertical gradient maps to form vector fields that are refined through gradient tracking in a post-processing step using the Cellpose algorithm \cite{Stringer_Wang_etal._2021}, known for its efficacy in cell segmentation tasks and generalizability across multiple domains \cite{Pachitariu_Stringer_2022,Stringer_Pachitariu_2024}. The second task-specific layer is the "Cell type head", which assigns labels to individual pixels. In the post-processing step, we determine the output classification label of each segmented cell instance by majority voting over the labeled pixels that comprise the cell in the segmentation map.

To evaluate model performance and measure the impact of adding a foundation model as backbone, we compare it to a ResNet50-based model. ResNet50 is a widely used solution for encoders in segmentation architectures in the medical domain \cite{Deshmane_2023,Graham_Vu_etal._2019,Mukasheva_Koishiyeva_etal._2024,Stringer_Wang_etal._2021}. For the H-Optimus-based model, we utilize frozen weights for the encoder and only fine-tune the decoder to take advantage of the extensive pre-training of the foundation model. For the ResNet50-based model we start with ImageNet \cite{Deng_Dong_etal.} weights and train both encoder and decoder parts. Hyperparameters for the training step are set to be identical, where possible, for comparable evaluation. 
For this evaluation, we deliberately use the PanNuke dataset to provide a standardized and controlled comparison between the H‑Optimus and ResNet50-based models (\hyperref[fig:S2]{Appendix Figure S2 (3)}). Specifically, we use two of the default PanNuke dataset splits (66\%) for training and validation, and reserve the third split (33\%) for testing.

To address the challenge of cell class imbalance in the PanNuke dataset, which is a common characteristic in most cell-level H\&E patch datasets, both models’ training processes employ a weighted loss function comprising cross-entropy and focal loss \cite{Lin_Goyal_etal._2018}. The focal loss component is adjusted with coefficients derived from each cell class' instance frequency, emphasizing learning from underrepresented classes and enhancing the model's sensitivity to rare but significant cellular patterns. The cross-entropy loss is augmented with spectral decoupling regularization \cite{Pezeshki_Kaba_etal._2021,Pohjonen_Stürenberg_etal._2022} and spatially varying label smoothing \cite{Islam_Glocker_2021}, which potentially stabilizes training and improves generalization in case of complex tissue morphologies. For optimization, we employ the \textit{AdamW} \cite{Loshchilov_Hutter_2019} to counter unbalanced class scenarios, with cosine annealing learning rate scheduler.

We utilize the scikit-learn library \cite{Van_der_Walt_Schönberger_etal._2014} and HoVer-Net \cite{Graham_Vu_etal._2019} implementations of $R^2$ (the coefficient of determination) and $PQ$ (panoptic quality) to evaluate our experiments. Complete mathematical formulations and detailed explanations of these metrics are provided in \hyperref[chap:S5]{Appendix S5}. To compute confidence intervals, we use nonparametric bootstrapping, where after calculating the metric on the full sample, we generated 1000 bootstrap replicates by resampling with replacement and then determined the 95\% confidence intervals as the 2.5th and 97.5th percentiles of the resulting empirical distribution.

%\hfill

The model comparisons are summarized in \hyperref[tab:2]{Table 2}. The H‑Optimus-based model achieves higher $R^2$ across all cell classes compared to the ResNet50-based model, which means that its predictions are more closely aligned with the PanNuke cell counts, indicating a stronger correlation with the observed data. Notably, the improvement of $R^2_{dead}$ may be an indicator of better global contextual representations provided by the foundation model backbone. In terms of segmentation and classification quality combined, measured by the PQ score, the H‑Optimus-based model demonstrates notable improvements across most cell classes. Overall, the average $R^2$ improved from 0.575 to 0.871, while the average $PQ$ score improved from 0.450 to 0.492, demonstrating better performance of the H-Optimus-based model.

\begin{table}[h!]
\renewcommand{\arraystretch}{1.5}
  \centering
  \caption{Cell quantification metrics for baseline and proposed models (CI 95\%).}
  \label{tab:2}
  \begin{tabular}{|l|c|c|}
    \hline
    %\rowcolor{gray!30}
    Metric             & Resnet50-based            & H-optimus-based              \\
    \hline
    $R^2_{neoplastic}$    & 0.681 (0.576--0.769)       & \textbf{0.941 (0.917--0.960)} \\
    \hline
    $R^2_{inflammatory}$  & 0.863 (0.778--0.903)       & \textbf{0.949 (0.918--0.966)} \\
    \hline
    $R^2_{connective}$    & 0.600 (0.488--0.698)       & 0.609 (0.436--0.772)          \\
    \hline
    $R^2_{dead}$          & 0.097 (-11.389--0.669)     & 0.925 (0.404--0.982)          \\
    \hline
    $R^2_{epithelial}$    & 0.635 (0.490--0.747)       & \textbf{0.930 (0.886--0.964)} \\
    \hline
    $PQ_{neoplastic}$       & 0.517 (0.499--0.535)       & \textbf{0.589 (0.575--0.604)} \\
    \hline
    $PQ_{inflammatory}$     & 0.455 (0.429--0.482)       & \textbf{0.528 (0.507--0.549)} \\
    \hline
    $PQ_{connective}$       & 0.416 (0.400--0.431)       & \textbf{0.451 (0.436--0.465)} \\
    \hline
    $PQ_{dead}$             & 0.374 (0.342--0.408)       & 0.292 (0.209--0.365)          \\
    \hline
    $PQ_{epithelial}$       & 0.488 (0.460--0.519)       & \textbf{0.599 (0.579--0.618)} \\
    \hline
  \end{tabular}
\end{table}

Our results  show that integrating the H‑Optimus foundation model within the UNETR architecture enhances the model's ability to segment and classify cells across diverse tissues from PanNuke data. The pretrained transformer encoder provides robust feature representations, resulting in higher average $R^2$ and $PQ$ scores compared to the CNN-based model. This leads to more reliable cell quantification and more accurate downstream analysis. Additionally, the streamlined fine-tuning process reduces computational overhead and training time, making the model more adaptable for new data.

Despite these advancements, the foundation model-based approach does not fully resolve all challenges related to cell segmentation and classification. We observe lower metric scores for underrepresented classes in the training data. Furthermore, foundation models typically encompass billions of parameters, resulting in substantial computational and memory requirements. It therefore poses challenges for deployment in resource-constrained environments, limiting their practical applicability in certain clinical settings.

\section{Model optimization via Knowledge Distillation}

To address the limitations posed by the extensive size of foundation models, we implement knowledge distillation — a model compression technique that leverages the teacher-student paradigm \cite{Hinton_Vinyals_etal._2015}. By training a smaller, more efficient student model to replicate the output of a larger, pre-trained teacher model, we retain performance while significantly reducing the model's complexity and resource requirements (\hyperref[fig:fig6]{Figure 6}).

\begin{figure}[h!]
    \centering
    \includegraphics[width=\textwidth, height=0.45\textheight, keepaspectratio]{images/Figure_6.pdf}
    \caption{Knowledge distillation framework for training a student model using a pre-trained teacher}
    \label{fig:fig6}
\end{figure}

We employ knowledge distillation to compress the H‑Optimus-based teacher model into a more efficient student model. The teacher model is the modified UNETR architecture with the H‑Optimus foundation model described in the previous chapter. The student model is based on a UNet architecture augmented with residual connections and incorporates a smaller ViT encoder with 9 million parameters \cite{Steiner_Kolesnikov_etal._2022,Wightman_2019}. 

First, we fine-tune the teacher model using the refined dataset from the cross-relabeling procedure (Section 2). Initially we train the decoder of the teacher model while keeping the encoder weights frozen. We split the refined dataset into train (70\%), validation (20\%) and test (10\%) subsets (\hyperref[fig:S2]{Appendix Figure S2 (4)}). During fine-tuning, we use the train and validation subsets, while leaving the test subset for model evaluation. We set the training procedure and model hyperparameters to be identical to those that were used to demonstrate the utility of foundation models for the simultaneous cell segmentation and classification task.

Next, we perform knowledge distillation from teacher to student using the refined dataset used to fine-tune the teacher model. The student model is trained to replicate the teacher model's outputs. We utilize a specialized loss function that aligns the student's predicted probability distribution with the teacher's, incorporating the teacher's class probability distribution derived from the output. Following the methodology of Hinton et al. \cite{Hinton_Vinyals_etal._2015}, we experiment with various hyperparameter settings for the temperature ($T$) and the balancing coefficients ($\alpha$ and $\beta$) in the loss function. We vary $T$ from 1 to 20 and adjust $\alpha$ and $\beta$ to balance the distillation and student losses. Through iterative tuning and evaluation, we identify that setting $T=14$, $\alpha=0.3$, and $\beta=0.7$ yields a configuration that converges and closely approximates the teacher model's performance during training.

Finally, we assess the performance of both models using the $R^2$ and $PQ$ (defined in \hyperref[chap:S5]{Appendix S5}) on the test set of the refined dataset (\hyperref[tab:3]{Table 3}). We observe that the 95\% confidence intervals overlap for most cell types, so we cannot claim statistically significant performance differences between the teacher and student models. One exception appears in the neoplastic class. The teacher model produces an $R^2$ of 0.919, while the student model shows an $R^2$ of 0.852. In addition, the student model achieves higher $PQ$ values for the neoplastic and connective classes, though the confidence intervals show overlap.

\begin{table}[h!]
\renewcommand{\arraystretch}{1.5}
  \centering
  \caption{Cell quantification metrics for teacher and distilled student models (CI 95\%).}
  \label{tab:3}
  \begin{tabular}{|l|c|c|}
    \hline
    %\rowcolor{gray!30}
    Metric & Teacher & Student \\
    \hline
    $R^2_{neoplastic}$    & \textbf{0.919} (0.898--0.939) & 0.852 (0.800--0.891) \\
    \hline
    $R^2_{lymphocyte}$    & 0.969 (0.956--0.977)         & 0.969 (0.956--0.978) \\
    \hline
    $R^2_{connective}$    & 0.694 (0.548--0.809)         & 0.618 (0.469--0.741) \\
    \hline
    $R^2_{dead}$          & 0.755 (0.400--0.908)         & 0.424 (0.100--0.731) \\
    \hline
    $R^2_{epithelial}$    & 0.922 (0.870--0.958)         & 0.843 (0.738--0.917) \\
    \hline
    $R^2_{macrophage}$    & 0.384 (-0.369--0.724)        & 0.704 (0.352--0.859) \\
    \hline
    $R^2_{neutrofil}$     & 0.854 (0.578--0.929)         & 0.833 (0.502--0.925) \\
    \hline
    $PQ_{neoplastic}$       & 0.581 (0.569--0.593)         & 0.601 (0.588--0.613) \\
    \hline
    $PQ_{lymphocyte}$       & 0.536 (0.520--0.553)         & 0.563 (0.544--0.579) \\
    \hline
    $PQ_{connective}$       & 0.436 (0.421--0.451)         & 0.457 (0.441--0.474) \\
    \hline
    $PQ_{dead}$             & 0.272 (0.235--0.315)         & 0.279 (0.201--0.369) \\
    \hline
    $PQ_{epithelial}$       & 0.522 (0.500--0.545)         & 0.530 (0.506--0.555) \\
    \hline
    $PQ_{macrophage}$       & 0.524 (0.459--0.588)         & 0.474 (0.405--0.543) \\
    \hline
    $PQ_{neutrofil}$        & 0.541 (0.490--0.592)         & 0.565 (0.522--0.607) \\
    \hline
  \end{tabular}
\end{table}


We further decompose the $PQ$ metric into its $SQ$ and $DQ$ components (\hyperref[tab:S6]{Appendix Table S6}). Both models produce nearly identical $SQ$ values, which indicates that they predict instance boundaries with similar precision. Although the student model shows some improvement in $DQ$ scores for certain classes, the confidence intervals overlap and do not confirm a statistically significant difference.

We observe that the student and teacher models yield comparable detection performance despite the student model using a much smaller and simpler architecture. A model with fewer parameters reduces the risk of overfitting when training data are scarce relative to the model’s complexity \cite{Farias_Ludermir_etal._2022}. The knowledge distillation process also encourages the student model to focus on the most generalizable detection features learned from the teacher. These factors enable the student model to achieve similar detection performance across different cell types.

Additionally, considering the model sizes reported in \hyperref[tab:4]{Table 4}, the distilled model achieves a significant reduction compared to the teacher model, with a 48-fold decrease in parameter count and a 5.5-fold reduction in on-disk size. In inference mode, the teacher model requires 16 GB of VRAM for a batch size of 32, while the distilled model only needs 3 GB of VRAM for the same batch size. These reductions make the distilled model significantly more practical for fine-tuning and deployment in resource-constrained environments.

\begin{table}[h!]
\renewcommand{\arraystretch}{1.5}
  \centering
  \caption{Parameter counts and size of teacher and distilled model}
  \label{tab:4}
  \adjustbox{max width=\textwidth}{%
  \begin{tabular}{|l|c|c|c|}
    \hline
    %\rowcolor{gray!30}
    Metric & H-optimus-based (Teacher) & mobileViT-based (Student) & Magnitude of difference \\
    \hline
    Parameters count       & 1,158,917,906   & \textbf{24,093,393}   & \textbf{48x}  \\
    \hline
    Estimated Total Size (MB) & 87,912       & \textbf{15,935}    & \textbf{5.5x} \\
    \hline
  \end{tabular}%
}
\end{table}

%\hfill

With recent advancements in complex network architectures and the use of pretrained encoders to achieve state-of-the-art performance \cite{Baumann_Dislich_etal._2024,Hörst_Rempe_etal._2024} in cell segmentation and classification tasks, model size, computational complexity, and processing times have increased. This limits the scalability and accessibility of these models. As we demonstrate, this may be mitigated using knowledge distillation. Studies in the field of natural language processing have demonstrated the efficacy of knowledge distillation in retaining the capabilities of the teacher model while achieving significant reductions in size and complexity \cite{Huangpu_Gao_2024,Sun_Yu_etal.}. 

We demonstrate the feasibility of knowledge distillation in digital pathology, specifically for cell segmentation and classification tasks. Moreover, we achieve this performance while also significantly reducing the parameter count. In addressing the challenge of knowledge transfer, we found that distillation from a transformer-based model to a smaller transformer is more straightforward than attempting to map transformer features to CNN blocks. In our experiments, using a CNN-based network as a student results in worse cell quantification performance due to the structural constraints of CNN feature space dimensions. 

Although our primary approach relies on a transformer-based student model that performs well, it can be further optimized to incorporate advantages from CNN architectures. For example, employing alternative techniques such as using ViT adapters \cite{Chen_Duan_etal._2023} or $1 \times 1$ convolutions to adjust feature map sizes may be beneficial for harnessing CNN advantages like enhanced local feature extraction. Moreover, if additional performance improvements are desired, the process can be further enhanced by applying supplementary knowledge distillation techniques, such as self-distillation \cite{Zhang_Song_etal._2019} or online distillation \cite{Houyon_Cioppa_etal._2023}.

Despite these promising results, further validation on independent datasets is necessary to fully understand the model's limitations. Underrepresented classes may pose challenges when addressing complex cases. Pathologists need to validate these models to adopt them in clinical settings. While the distilled models are smaller and more deployable, a technological gap persists because pathologists traditionally rely on established methods for inspecting WSIs and diagnosing diseases. Addressing the complexities involved in deploying models for inference and supporting pathologists in adopting new tools is essential for integrating these models into clinical workflows.

\section{Model integration with QuPath}
Digital pathology tools with graphical user interfaces are essential for visualizing and analyzing WSIs. To make our student model useful in clinical pathology workflows, it needs to be integrated into a tool that enables inspecting regions, creating annotations, and providing quantitative analyses of biomarkers. Therefore, we integrate the trained student model from the previous chapter into the QuPath open‑source platform \cite{Bankhead_Loughrey_etal._2017}. QuPath provides the required annotation, visualization, and analysis tools to interpret complex histological data, including workflows for cell segmentation, classification, and quantification (\hyperref[fig:fig7]{Figure 7}). 

\begin{figure}[h!]
    \centering
    \includegraphics[width=\textwidth]{images/Figure_7.pdf}
    \caption{Visualization of model-generated cell quantification annotations (left) and the corresponding unannotated slide (right) in QuPath}
    \label{fig:fig7}
\end{figure}

To identify the regions in a WSI critical for prognosticating tumor development, such as specific tumor areas or border regions without overlapping healthy tissue, the pathologist uses QuPath to outline these regions. Then, the pathologist initiates a cell segmentation and classification script through the QuPath interface for the selected regions. The resulting annotations and quantified cell information are then directly overlaid onto the WSI in the QuPath interface. Additional design and implementation details are in \hyperref[chap:S7]{Appendix S7}. 

Two common approaches for integrating deep learning models into QuPath are Java‑based native QuPath extensions \cite{Goldsborough_Philps_etal._2024} and the execution of RESTful API requests to a model server coupled with handling the response via an extension, as demonstrated in the application of cell segmentation models applied to immunofluorescence images \cite{Sugawara_2023}. While the community is actively working on these integration strategies, there is currently no universal solution that fully addresses all integration and performance requirements.

Extensions may offer better integration with QuPath, allowing slightly improved performance and more widespread usage of the built-in QuPath models, but they lack the flexibility to customize models and modify their behavior. For example, the newest version of QuPath includes models such as StarDist \cite{Weigert_Schmidt} and InstanSeg \cite{Goldsborough_Philps_etal._2024} that can perform cell segmentation. Both models pose limitations when applied to simultaneous cell segmentation and classification. StarDist performs well only on convex, round shapes by design, whereas some neoplastic, inflammatory, and connective cells exhibit complex and non-convex shapes. InstanSeg provides only semantic segmentation without assigning classes to the segmented cells.

%\hfill

In contrast, our approach offers an alternative integration strategy. It utilizes the paquo library to directly interact with QuPath’s internal application programming interface from within Python. This enables data exchange and processing without the need for intermediate conversion steps and provides greater control over model customization, retraining, and the incorporation of custom processing steps.

The integration of our custom model with QuPath underscores its potential to significantly enhance the diagnostic process by reducing the time burden on pathologists and enabling them to focus on more complex interpretative tasks using familiar software. Leveraging a tool that is already well-established among pathologists increases the likelihood of its adoption into daily clinical workflows. The quantitative data generated through the automated workflow is critical for both clinical decision-making and research, facilitating more accurate biomarker analysis, enabling robust statistical evaluations, and supporting hypothesis generation and testing. Additionally, by streamlining cell segmentation and classification, the tool enhances the scalability and reproducibility of pathological assessments, ultimately contributing to improved diagnostic accuracy and patient outcomes.

\section{Conclusion and future work}

In this study, we address critical challenges in digital pathology and tackle the usability and deployment issues of the developed models in standard computing environments without the need for high-performance computing systems. Our multi-faceted approach encompasses data refinement through cross-relabeling, leveraging foundation models for robust cell segmentation and classification, optimizing model performance via knowledge distillation, and integrating the optimized model into the QuPath software for practical application. This approach is used to construct a capable, versatile, and adjustable model for cell segmentation and classification, with enhanced performance and usability.

\begin{sloppypar}
While our approach shows potential in the field of computational pathology, certain limitations persist. 
For example, our implementation currently exhibits lower performance in detecting macrophages. 
This serves as an instance of the broader challenge of accurately identifying complex cell types. In order to address this issue, extending our approach to incorporate additional data sources, exploring alternative modeling approaches, and integrating other imaging modalities such as immunohistochemical staining may help improve detection accuracy. Moreover, although the distilled model reduces computational demands, integrating advanced deep learning models into clinical practice requires addressing technological gaps and potential resistance to adopting new tools within established diagnostic processes.
\end{sloppypar}

Future work could focus on several key areas to refine the proposed approach and facilitate its adoption in clinical environments. Enhancing the cell-relabeling process with additional datasets \cite{Graham_Jahanifar_etal._2021} could improve the representation of underrepresented cell types and enhance overall model performance. Also, incorporating additional data sources, such as multi-modal imaging or complementary staining methods, may address limitations related to cell type differentiation and class imbalance. Exploring other foundation models \cite{Vorontsov_Bozkurt_etal._2024,Zimmermann_Vorontsov_etal._2024} or introducing additional modalities \cite{Ding_Wagner_etal._2024,Vaidya_Zhang_etal._2025} may provide alternative architectures better suited to specific tasks or offer improved efficiency. Implementing more complex knowledge distillation techniques \cite{Houyon_Cioppa_etal._2023,Zhang_Song_etal._2019} could further optimize the model's performance and adaptability. Additionally, deeper integration with QuPath or other digital pathology software could provide pathologists more control over cell quantification analysis directly within the QuPath interface, thereby increasing accessibility and usability. Such enhancements would not only refine model performance but also ensure greater adaptability and scalability within various clinical environments. Finally, extensive validation of the model by pathologists and benchmarking against independent datasets are essential steps toward establishing the model's reliability and fostering confidence in its clinical utility.

\section*{Acknowledgments} 
This work was funded in part by the Research Council of Norway grant no. 309439 SFI Visual Intelligence, and the North Norwegian Health Authority grant no. HNF1521-20.

\bibliographystyle{IEEEtran}
\begin{sloppypar}
\begin{thebibliography}{99}

\bibitem{chaplot2020neural} Chaplot, Devendra Singh, et al. "Neural topological slam for visual navigation." Proceedings of the IEEE/CVF conference on computer vision and pattern recognition. 2020.

\bibitem{maksymets2021thda} Maksymets, Oleksandr, et al. "Thda: Treasure hunt data augmentation for semantic navigation." Proceedings of the IEEE/CVF International Conference on Computer Vision. 2021.

\bibitem{mezghan2022memory} Mezghan, Lina, et al. "Memory-augmented reinforcement learning for image-goal navigation." 2022 IEEE/RSJ International Conference on Intelligent Robots and Systems (IROS). IEEE, 2022.

\bibitem{al2022zero} Al-Halah, Ziad, Santhosh Kumar Ramakrishnan, and Kristen Grauman. "Zero experience required: Plug \& play modular transfer learning for semantic visual navigation." Proceedings of the IEEE/CVF Conference on Computer Vision and Pattern Recognition. 2022.

\bibitem{ye2021auxiliary} Ye, Joel, et al. "Auxiliary tasks and exploration enable objectgoal navigation." Proceedings of the IEEE/CVF international conference on computer vision. 2021.

\bibitem{chaplot2020object} Chaplot, Devendra Singh, et al. "Object goal navigation using goal-oriented semantic exploration." Advances in Neural Information Processing Systems 33 (2020)

\bibitem{ramakrishnan2022poni} Ramakrishnan, Santhosh Kumar, et al. "Poni: Potential functions for objectgoal navigation with interaction-free learning." Proceedings of the IEEE/CVF Conference on Computer Vision and Pattern Recognition. 2022.

\bibitem{ramrakhya2022habitat} Ramrakhya, Ram, et al. "Habitat-web: Learning embodied object-search strategies from human demonstrations at scale." Proceedings of the IEEE/CVF Conference on Computer Vision and Pattern Recognition. 2022.

\bibitem{mousavian2019visual} Mousavian, Arsalan, et al. "Visual representations for semantic target driven navigation." 2019 International Conference on Robotics and Automation (ICRA). IEEE, 2019.

\bibitem{dhariwal2021diffusion} Dhariwal, Prafulla, and Alexander Nichol. "Diffusion models beat gans on image synthesis." Advances in neural information processing systems 34 (2021)

\bibitem{ho2022classifier} Ho, Jonathan, and Tim Salimans. "Classifier-free diffusion guidance." arXiv preprint arXiv:2207.12598 (2022).

\bibitem{nichol2021glide} Nichol, Alex, et al. "Glide: Towards photorealistic image generation and editing with text-guided diffusion models." arXiv preprint arXiv:2112.10741 (2021)

\bibitem{brooks2023instructpix2pix} Brooks, Tim, Aleksander Holynski, and Alexei A. Efros. "Instructpix2pix: Learning to follow image editing instructions." Proceedings of the IEEE/CVF Conference on Computer Vision and Pattern Recognition. 2023.

\bibitem{fu2023guiding} Fu, Tsu-Jui, et al. "Guiding instruction-based image editing via multimodal large language models." arXiv preprint arXiv:2309.17102 (2023).

\bibitem{geng2024instructdiffusion} Geng, Zigang, et al. "Instructdiffusion: A generalist modeling interface for vision tasks." Proceedings of the IEEE/CVF Conference on Computer Vision and Pattern Recognition. 2024.

\bibitem{zhou2024minedreamer} Zhou, Enshen, et al. "Minedreamer: Learning to follow instructions via chain-of-imagination for simulated-world control." arXiv preprint arXiv:2403.12037 (2024).

\bibitem{zhou2023esc} Zhou, Kaiwen, et al. "Esc: Exploration with soft commonsense constraints for zero-shot object navigation." International Conference on Machine Learning. PMLR, 2023.

\bibitem{yu2023l3mvn} Yu, Bangguo, Hamidreza Kasaei, and Ming Cao. "L3mvn: Leveraging large language models for visual target navigation." 2023 IEEE/RSJ International Conference on Intelligent Robots and Systems (IROS). IEEE, 2023.

\bibitem{gadre2023cows} Gadre, Samir Yitzhak, et al. "Cows on pasture: Baselines and benchmarks for language-driven zero-shot object navigation." Proceedings of the IEEE/CVF Conference on Computer Vision and Pattern Recognition. 2023.

\bibitem{shah2023navigation} Shah, Dhruv, et al. "Navigation with large language models: Semantic guesswork as a heuristic for planning." Conference on Robot Learning. PMLR, 2023.

\bibitem{cai2024bridging} Cai, Wenzhe, et al. "Bridging zero-shot object navigation and foundation models through pixel-guided navigation skill." 2024 IEEE International Conference on Robotics and Automation (ICRA). IEEE, 2024.

\bibitem{yu2023co} Yu, Bangguo, Hamidreza Kasaei, and Ming Cao. "Co-NavGPT: Multi-robot cooperative visual semantic navigation using large language models." arXiv preprint arXiv:2310.07937 (2023).

\bibitem{wu2024voronav} Wu, Pengying, et al. "Voronav: Voronoi-based zero-shot object navigation with large language model." arXiv preprint arXiv:2401.02695 (2024).

\bibitem{qin2023mp5} Qin, Yiran, et al. "Mp5: A multi-modal open-ended embodied system in minecraft via active perception." arXiv preprint arXiv:2312.07472 (2023).

\bibitem{du2024learning} Du, Yilun, et al. "Learning universal policies via text-guided video generation." Advances in Neural Information Processing Systems 36 (2024).

\bibitem{ajay2024compositional} Ajay, Anurag, et al. "Compositional foundation models for hierarchical planning." Advances in Neural Information Processing Systems 36 (2024).

\bibitem{liang2024skilldiffuser} Liang, Zhixuan, et al. "Skilldiffuser: Interpretable hierarchical planning via skill abstractions in diffusion-based task execution." Proceedings of the IEEE/CVF Conference on Computer Vision and Pattern Recognition. 2024.

\bibitem{heusel2017gans} Heusel, Martin, et al. "Gans trained by a two time-scale update rule converge to a local nash equilibrium." Advances in neural information processing systems 30 (2017).

\bibitem{zhang2018unreasonable} Zhang, Richard, et al. "The unreasonable effectiveness of deep features as a perceptual metric." Proceedings of the IEEE conference on computer vision and pattern recognition. 2018.

\bibitem{brown2020language} Brown, Tom B. "Language models are few-shot learners." arXiv preprint arXiv:2005.14165 (2020).

\bibitem{podell2023sdxl} Podell, Dustin, et al. "Sdxl: Improving latent diffusion models for high-resolution image synthesis." arXiv preprint arXiv:2307.01952 (2023).

\bibitem{brohan2022rt} Brohan, Anthony, et al. "Rt-1: Robotics transformer for real-world control at scale." arXiv preprint arXiv:2212.06817 (2022).

\bibitem{brohan2023rt} Brohan, Anthony, et al. "Rt-2: Vision-language-action models transfer web knowledge to robotic control." arXiv preprint arXiv:2307.15818 (2023).

\bibitem{li2024manipllm} Li, Xiaoqi, et al. "Manipllm: Embodied multimodal large language model for object-centric robotic manipulation." Proceedings of the IEEE/CVF Conference on Computer Vision and Pattern Recognition. 2024.

\bibitem{shah2023vint} Shah, Dhruv, et al. "ViNT: A foundation model for visual navigation." arXiv preprint arXiv:2306.14846 (2023).

\bibitem{liu2024visual} Liu, Haotian, et al. "Visual instruction tuning." Advances in neural information processing systems 36 (2024).

\bibitem{hu2021lora} Hu, Edward J., et al. "Lora: Low-rank adaptation of large language models." arXiv preprint arXiv:2106.09685 (2021).

\bibitem{qin2023supfusion} Qin, Yiran, et al. "SupFusion: Supervised LiDAR-camera fusion for 3D object detection." Proceedings of the IEEE/CVF International Conference on Computer Vision. 2023.

\bibitem{qin2024worldsimbench} Qin, Yiran, et al. "Worldsimbench: Towards video generation models as world simulators." arXiv preprint arXiv:2410.18072 (2024).

\bibitem{yu2025gamefactory} Yu, Jiwen, et al. "GameFactory: Creating New Games with Generative Interactive Videos." arXiv preprint arXiv:2501.08325 (2025).

\bibitem{zhou2024code} Zhou, Enshen, et al. "Code-as-Monitor: Constraint-aware Visual Programming for Reactive and Proactive Robotic Failure Detection." arXiv preprint arXiv:2412.04455 (2024).

\bibitem{zhang2024ad} Zhang, Zaibin, et al. "AD-H: Autonomous Driving with Hierarchical Agents." arXiv preprint arXiv:2406.03474 (2024).

\bibitem{wang2024toward} Wang, Chaoqun, et al. "Toward Accurate Camera-based 3D Object Detection via Cascade Depth Estimation and Calibration." arXiv preprint arXiv:2402.04883 (2024).

\bibitem{huang2024story3d} Huang, Yuzhou, et al. "Story3d-agent: Exploring 3d storytelling visualization with large language models." arXiv preprint arXiv:2408.11801 (2024).

\bibitem{savinov2018semi} Savinov, Nikolay, Alexey Dosovitskiy, and Vladlen Koltun. "Semi-parametric topological memory for navigation." arXiv preprint arXiv:1803.00653 (2018).

\bibitem{majumdar2022zson} Majumdar, Arjun, et al. "Zson: Zero-shot object-goal navigation using multimodal goal embeddings." Advances in Neural Information Processing Systems 35 (2022): 32340-32352.

\bibitem{yadav2023offline} Yadav, Karmesh, et al. "Offline visual representation learning for embodied navigation." Workshop on Reincarnating Reinforcement Learning at ICLR 2023. 2023.

\bibitem{yadav2023ovrl} Yadav, Karmesh, et al. "Ovrl-v2: A simple state-of-art baseline for imagenav and objectnav." arXiv preprint arXiv:2303.07798 (2023).

\bibitem{sun2024fgprompt} Sun, Xinyu, et al. "FGPrompt: fine-grained goal prompting for image-goal navigation." Advances in Neural Information Processing Systems 36 (2024).

\bibitem{zhu2017target} Zhu, Yuke, et al. "Target-driven visual navigation in indoor scenes using deep reinforcement learning." 2017 IEEE international conference on robotics and automation (ICRA). IEEE, 2017.

\bibitem{koh2024generating} Koh, Jing Yu, Daniel Fried, and Russ R. Salakhutdinov. "Generating images with multimodal language models." Advances in Neural Information Processing Systems 36 (2024).

\bibitem{krantz2022instance} Krantz, Jacob, et al. "Instance-specific image goal navigation: Training embodied agents to find object instances." arXiv preprint arXiv:2211.15876 (2022).

\bibitem{schulman2017proximal} Schulman, John, et al. "Proximal policy optimization algorithms." arXiv preprint arXiv:1707.06347 (2017).

\bibitem{anderson2018evaluation} Anderson, Peter, et al. "On evaluation of embodied navigation agents." arXiv preprint arXiv:1807.06757 (2018).

\bibitem{lin2024navcot} Lin, Bingqian, et al. "NavCoT: Boosting LLM-Based Vision-and-Language Navigation via Learning Disentangled Reasoning." arXiv preprint arXiv:2403.07376 (2024).

\bibitem{NavGPT} Zhou, Gengze, Yicong Hong, and Qi Wu. "Navgpt: Explicit reasoning in vision-and-language navigation with large language models." Proceedings of the AAAI Conference on Artificial Intelligence.

\bibitem{hahn2021no} Hahn, Meera, et al. "No rl, no simulation: Learning to navigate without navigating." Advances in Neural Information Processing Systems 34 (2021): 26661-26673.

\bibitem{li2025t2isafety} Li, Lijun, et al. "T2ISafety: Benchmark for Assessing Fairness, Toxicity, and Privacy in Image Generation." arXiv preprint arXiv:2501.12612 (2025).

\bibitem{an2024agfsync} An, Jingkun, et al. "AGFSync: Leveraging AI-Generated Feedback for Preference Optimization in Text-to-Image Generation." arXiv preprint arXiv:2403.13352 (2024).


\end{thebibliography}
\end{sloppypar}

\clearpage
\beginsupplement
\section*{Appendix}
\renewcommand{\thesubsection}{S\arabic{subsection}}

\subsection{\label{chap:S1}PanNuke and MoNuSAC preprocessing}
The PanNuke dataset comprises a set of 7,901 RGB patches, each with dimensions of $256 \times 256$ pixels, which we set as the standard patch size for our analysis. In contrast, the MoNuSAC dataset encompasses 294 images of heterogeneous dimensions. To standardize the MoNuSAC images with our experiments, we implement a standardization protocol. Specifically, for images exceeding the dimensions of $256 \times 256$ pixels, we segment them into equal-sized patches and apply mirror padding to the remaining portions to avoid information loss at the peripherals. Patches with dimensions less than $128 \times 128$ pixels are excluded from the dataset due to the insufficient resolution to capture relevant cellular details. For patches where either dimension falls between 128 and 256 pixels, we employ upsampling to achieve the standard patch size. As a result, we obtain a total of 2,823 RGB patches derived from the MoNuSAC dataset for subsequent analysis. For additional details on the MoNuSAC data preparation process, refer to the source code \cite{Shvetsov_2025a}.
\clearpage

\subsection{\label{chap:S2}Data usage for the methodology}

\counterwithin{figure}{subsection}
\renewcommand{\thefigure}{S\arabic{subsection}}

\begin{figure}[h!]
    \centering
    \includegraphics[width=\textwidth, height=0.85\textheight, keepaspectratio]{images/A2.pdf}
    \caption{Overview of the methodology for cross-labeling, dataset refinement, and model comparison. (1) Cross-relabeling - training and testing cell classification models, (2) Cross-relabeling - using cell classification models to create refined dataset, (3) Fine-tuning and training models for comparison, (4) Student knowledge distillation with refined dataset}
    \label{fig:S2}
\end{figure}
\clearpage

\subsection{\label{chap:S3}Confusion matrices for classification models}
\counterwithin{figure}{subsection}
\renewcommand{\thefigure}{S\arabic{subsection}.\arabic{figure}}

\begin{figure}[h!]
    \centering
    \includegraphics[width=\textwidth, height=0.4\textheight, keepaspectratio]{images/A3_1.pdf}
    \caption{Confusion matrix for PanNuke trained model}
    \label{fig:S3.1}
\end{figure}

\begin{figure}[h!]
    \centering
    \includegraphics[width=\textwidth, height=0.4\textheight, keepaspectratio]{images/A3_2.pdf}
    \caption{Confusion matrix for MoNuSAC trained model}
    \label{fig:S3.2}
\end{figure}

\clearpage

\subsection{\label{chap:S4}Datasets cell counts}

\counterwithin{table}{subsection}
\renewcommand{\thetable}{S\arabic{subsection}}

\begin{table}[h!]
\renewcommand{\arraystretch}{2.0}
\centering
\caption{\label{tab:S4}Cell counts for PanNuke, MoNuSAC and refined datasets. Numbers in parentheses indicate preprocessed cell counts for cell classifier models training and testing.}
%\adjustbox{max width=\textwidth}{%
\begin{tabular}{|l|c|c|c|}
\hline
%\rowcolor{gray!30}
Cell type & PanNuke & MoNuSAC & Refined \\
\hline
Neoplastic & 77,403 (68,031) & - & 105,451 \\
\hline
Epithelial & 26,572 (23,207) & - & 29,926 \\
\hline
Epithelial (benign and malignant) & - & 31,402 & - \\
\hline
Inflammatory & 32,276 & - & - \\
\hline
Lymphocytes & - & 37,045 (33,104) & 65,275 \\
\hline
Neutrophils & - & 1,355 (1,252) & 3,833 \\
\hline
Macrophage & - & 1,842 (1,695) & 3,410 \\
\hline
Dead & 2,908 & - & 2,908 \\
\hline
Connective & 50,585 & - & 50,585 \\
\hline
\end{tabular}
%
%}
\end{table}



\clearpage

\subsection{\label{chap:S5}Definition of validation metrics}
\counterwithin{equation}{subsection}
\renewcommand{\theequation}{\arabic{equation}}

\subsubsection{\label{chap:S5.1}R\textsuperscript{2}}
The coefficient of determination, denoted as $R^2$, is a statistical measure that represents the proportion of variance in the dependent variable that is predictable from the independent variables. In the context of cell quantification in pathology, $R^2$ is used to assess how well the predicted quantities of different cell types in a patch align with the actual quantities observed in the ground truth data, with higher values representing more accurate quantification. $R^2$ is defined as
\begin{equation*}
R^2 = 1 - \frac{\sum_{i=1}^n (y_i - \hat{y}_i)^2}{\sum_{i=1}^n (y_i - \bar{y})^2},
\end{equation*}
where $y_i$ represents the actual number of cells of a specific type in the $i$-th image, $\hat{y}_i$ represents the predicted number of cells of that type in the $i$-th image, $\bar{y}$ is the mean of the actual numbers across all images, and $n$ is the total number of images in the dataset.

The $R^2$ metric has a range of $(-\infty, 1]$. An $R^2$ of 1 indicates perfect prediction, where all predicted values exactly match the actual values. An $R^2$ of 0 suggests that the model explains none of the variability of the response data around its mean. If $R^2$ is negative, it indicates that the model performs worse than a model that simply predicts the mean of the actual values for all observations.

\subsubsection{\label{chap:S5.2}PQ}
Panoptic Quality ($PQ$) is a comprehensive metric used to evaluate the performance of segmentation models in tasks that require both instance segmentation and classification. $PQ$ provides a single score that encapsulates both the detection accuracy (i.e., how many objects were correctly identified) and the segmentation quality (i.e., how accurately the objects' boundaries were delineated). This metric is particularly useful in multiclass scenarios where each pixel is classified into distinct categories, such as different cell types in pathology images.

$PQ$ is calculated as the product of two terms: Detection Quality ($DQ$) and Segmentation Quality ($SQ$). It can be expressed as
\begin{equation*}
PQ = DQ \cdot SQ,
\end{equation*}
where
\begin{equation*}
DQ = \frac{TP}{TP + 0.5\, FP + 0.5\, FN},
\end{equation*}
\begin{equation*}
SQ = \frac{\sum_{(p, g) \in \mathcal{M}} IoU(p, g)}{TP}.
\end{equation*}
In these formulas, $TP$ denotes the number of correctly matched instances between ground truth and prediction, $FP$ denotes the predicted instances that have no corresponding ground truth, $FN$ denotes the ground truth instances that were not detected, $IoU(p, g)$ is the Intersection over Union for a pair of matched instances $p$ (prediction) and $g$ (ground truth), and $\mathcal{M}$ is the set of matched pairs.

The $PQ$ metric is calculated for each class and is averaged across classes to provide a global performance measure.

The $PQ$ score has a range of $[0, 1.0]$, where a higher score indicates better performance in both detecting and segmenting the instances correctly. A $PQ$ of 1 signifies perfect identification and segmentation of all instances, whereas a $PQ$ of 0 indicates that no instances were correctly identified and segmented.

\clearpage

\subsection{\label{chap:S6}Segmentation and Detection quality metrics for teacher and student models}

\begin{table}[h!]
\renewcommand{\arraystretch}{2.0}
\centering
\caption{Segmentation and detection quality for student and teacher models (CI 95\%)}
\label{tab:S6}
%\adjustbox{max width=\textwidth}{%
\begin{tabular}{|l|c|c|}
\hline
%\rowcolor{gray!30}
Metric & Teacher & Student \\
\hline
$SQ_{neoplastic}$ & 0.819 (0.815--0.823) & 0.824 (0.819--0.828) \\
\hline
$SQ_{lymphocyte}$ & 0.795 (0.788--0.802) & 0.790 (0.783--0.796) \\
\hline
$SQ_{connective}$ & 0.770 (0.762--0.776) & 0.780 (0.772--0.786) \\
\hline
$SQ_{dead}$ & 0.659 (0.623--0.688) & 0.657 (0.624--0.695) \\
\hline
$SQ_{epithelial}$ & 0.780 (0.770--0.790) & 0.788 (0.779--0.797) \\
\hline
$SQ_{macrophage}$ & 0.788 (0.760--0.810) & 0.757 (0.730--0.783) \\
\hline
$SQ_{neutrofil}$ & 0.782 (0.761--0.801) & 0.775 (0.759--0.792) \\
\hline
$DQ_{neoplastic}$ & 0.706 (0.692--0.719) & 0.727 (0.712--0.741) \\
\hline
$DQ_{lymphocyte}$ & 0.675 (0.656--0.698) & 0.713 (0.691--0.734) \\
\hline
$DQ_{connective}$ & 0.566 (0.546--0.584) & 0.583 (0.565--0.602) \\
\hline
$DQ_{dead}$ & 0.410 (0.361--0.465) & 0.435 (0.306--0.561) \\
\hline
$DQ_{epithelial}$ & 0.668 (0.639--0.694) & 0.673 (0.644--0.702) \\
\hline
$DQ_{macrophage}$ & 0.657 (0.583--0.727) & 0.615 (0.531--0.703) \\
\hline
$DQ_{neutrofil}$ & 0.691 (0.625--0.753) & 0.729 (0.679--0.778) \\
\hline
\end{tabular}
%
%}
\end{table}

\clearpage

\subsection{\label{chap:S7}QuPath integration method}
We adopt an integration strategy leveraging the paquo \cite{Bayer_AG} library, a Python package that enables direct interaction with QuPath’s internal API, thereby facilitating seamless data exchange without intermediate conversion steps. The data processing pipeline (\hyperref[fig:S7]{Appendix Figure S7}) begins with the acquisition of WSIs and their associated annotations from QuPath, which are represented as Shapely \cite{Gillies_Wel_etal._2024} polygons. Utilizing paquo, we directly read, create, and modify these annotations and detections within a QuPath project in the Python environment. Images are then cropped using these polygons and processed by cell segmentation and classification models employing standard vision processing toolkits such as OpenCV, pyvips, and PyTorch. Additionally, QuPath employs Groovy scripts to initiate a Python process that starts the entire pipeline from QuPath graphical interface: fetching polygons, extracting images from them, and running deep learning model inference on the cropped images. 
The results are returned to QuPath, leveraging paquo's Python bindings to manipulate QuPath data while minimizing the computational overhead typically associated with cross-environment communication.

\counterwithin{figure}{subsection}
\renewcommand{\thefigure}{S\arabic{subsection}}

\begin{figure}[h!]
    \centering
    \includegraphics[width=\textwidth]{images/A7.pdf}
    \caption{QuPath integration workflow using Python environment}
    \label{fig:S7}
\end{figure}

Compared to traditional workflows that involve exporting annotations as GeoJSON, classifying them in Python, and reimporting them into QuPath, our approach offers several advantages. We eliminate the need to switch between programming languages, providing a cohesive and streamlined development process entirely within QuPath software and removing the necessity to use other tools. Meanwhile, we avoid storing annotations as intermediate JSON files unless required for external use or archiving. By conducting the entire inference and post-processing workflow within the Python environment, we leverage the power and flexibility of Python libraries for image processing and machine learning. This approach also enables adjustments to any set of labels and models, thereby improving its applicability.

%\hfill

The distilled model and QuPath integration code are packaged into a Docker container, enabling streamlined execution with the Docker engine. Detailed integration code and deployment instructions can be found in the GitHub repository \cite{Shvetsov_2025b}.

Despite these benefits, we acknowledge that the paquo library is a proof‑of‑concept project in its early development stage and has not been tested across all versions of QuPath.

\clearpage

\subsection{\label{chap:S8}Data and code availability statement}
All datasets, models, and code used in this study are publicly available and can be obtained from the repositories listed below. 
The PanNuke \cite{Gamper_Koohbanani_etal._2019} and MoNuSAC \cite{Verma_Kumar_etal._2021} datasets are publicly accessible, and download information along with detailed descriptions can be found in their respective articles. Preprocessing scripts for PanNuke and MoNuSAC data, as well as individual cell extraction scripts, are available on GitHub \cite{Shvetsov_2025a}. The H-Optimus foundation model used in our experiments can be downloaded from the HuggingFace repository \cite{hoptimus2024}, and model information is available on GitHub \cite{Saillard_Jenatton_etal._2024}. In addition, the integration code for QuPath and the distilled model packaged in a Docker container are provided in the repository \cite{Shvetsov_2025b}, and paquo Python library is available from the authors GitHub repository \cite{Bayer_AG}.
\clearpage

\end{document}


\section{Experiments}
In this section, we conduct extensive experiments to validate the effectiveness of our method on prompt adaptation. For the policy model, we use a pretrained GPT-2 \citep{radford2019language} following the setup in~\citep{hao2024optimizing}. As a default setting, we employ Stable-Diffusion v1.4 \citep{rombach2022high} as the target text-to-image model and use DPM solver \citep{lu2022dpm} with 20 inference steps to accelerate the sampling process. Detailed information regarding the experimental settings can be found in \Cref{app:exp_details}.

\begin{figure*}[t]
    \centering
    \includegraphics[width=\textwidth]{figures/main_tradeoff.png}
    \caption{Reward and diversity of prompts generated by each method with different initial prompt datasets.}
    \label{fig:main_tradeoff}
    \vspace{-10pt}
\end{figure*}

\subsection{Experiment Setup}
\noindent\textbf{Dataset Preparation}
We strictly follow the setup of Promptist~\citep{hao2024optimizing} for dataset preparation. For training, we collect prompts from the Lexica website \citep{lexica}, DiffusionDB \citep{wang2023diffusiondb}, and COCO dataset \citep{chen2015microsoft}. For evaluation, we randomly sample $256$ prompts from the three types of training datasets. In addition, we introduce a challenging dataset using ChatGPT \citep{ouyang2022training} to
generate brief prompts that describe images around $5$ words. These brief prompts
naturally require diverse adaptations for better generation quality, representing a practical scenario for prompt adaptation.

\vspace{5pt}
\noindent\textbf{Baselines}
We consider several strong baselines to verify the efficacy of our method on the prompt adaptation task.
\begin{itemize}
    \item \textbf{Supervised Fine-tuning (SFT)}: A policy model fine-tuned by supervised learning on a set of prompt pairs of original user inputs and manually engineered prompts. 
    \item \textbf{Promptist}: A PPO-based approach \citep{schulman2017proximal} that directly trains the RL policy to maximize the reward function. 
    \item \textbf{Rule-Based}: Based on the observation that Promptist mostly generates similar postfixes with deterministic behaviors, we build a heuristic that appends the most frequently used postfixes in Promptist to the initial prompt.
    \item \textbf{GFlowNets}: We implement a vanilla GFlowNets method with the trajectory balance (TB) objective~\citep{malkin2022trajectory} to train the target policy. As our task is a conditional generation task, we use the VarGrad \citep{richter2020vargrad} version of TB loss, which is widely used for reducing variance \citep{zhangrobust, kim2024ant}.
    \item \textbf{DPO-Diff}: A gradient-based optimization method designed to discover effective negative prompts \citep{wang2024discrete}. As it is not directly comparable, we provide more discussion on DPO-Diff in Appendix~\ref{app:extend_main_results}. 
\end{itemize}

\vspace{5pt}
\noindent\textbf{Training and Evaluation}
Following Promptist~\citep{hao2024optimizing}, we initialize the policy with SFT policy before GFlowNet fine-tuning. To parametrize the flow function, we use a separate neural network that takes the last hidden embedding of the prompts as input and outputs a scalar value. We train both the policy and flow function for 10K steps with a batch size of 256. For learning rate, we use $1\times10^{-5}$ for the policy and $1\times10^{-4}$ for the flow function.

For evaluation, we generate $16$ prompts for each prompt via beam search with a length penalty of $1.0$. Then we generate $3$ images per prompt to compute the reward. To measure diversity, we compute the average pairwise cosine distance of 16 prompts for each initial prompt.

\subsection{Performance Comparison} \label{sec:main_res}
\Cref{fig:main_tradeoff} presents a comprehensive comparison of PAG against the baselines across different datasets. As depicted in the figure, while Promptist improves upon SFT through RL-based fine-tuning, it exhibits limited diversity due to the reward-maximization nature.
Vanilla GFlowNets mostly achieves significantly higher rewards than both SFT and Promptist, but still suffers from mode collapse. In contrast, PAG consistently surpasses all baselines in terms of both reward and diversity metrics, demonstrating its capability to generate both high-quality and diverse prompts across various input types.

\begin{figure*}[t]
    \centering
    \includegraphics[width=0.95\textwidth]{figures/main_figure.jpg}
    \caption{Images generated by optimized prompts using Stable Diffusion v1.4 (with the same seed to visualize the effect solely on prompts).
    Our method generates diverse and highly aesthetic images based on adapted prompts of high quality and diversity.}
    \label{fig:main_figure}
    \vspace{-12pt}
\end{figure*}

The challenging ChatGPT dataset, where initial prompts usually give us little information, further highlights the strengths of our approach. 
We observe that Promptist or vanilla GFlowNets lead to severe mode collapse and are outperformed by simple deterministic rule-based heuristics, while our method maintains robust performance.
We further visualize the generated prompts and corresponding images from each method in \Cref{fig:main_figure} (more figures can be found in Appendix~\ref{app:more_visualization}). As illustrated in the figures, PAG generates diverse and meaningful enhancements by not only appending postfixes but also introducing relevant adjectives and detailed descriptions, which is particularly valuable when the original prompt lacks sufficient information.

\begin{table}[t]
\centering
\caption{Reward and diversity of prompts generated by each method with different reward functions.}
\vspace{-5pt}
\resizebox{\linewidth}{!}{
\begin{tabular}{lcc|cc}
\toprule
\multirow{2}{*}{\textbf{Method}} & \multicolumn{2}{c}{ImageReward} & \multicolumn{2}{c}{HPSv2}  \\
\cmidrule{2-5}
& Reward & Diversity & Reward & Diversity \\
\midrule
SFT & 0.66 ± 0.01 & 0.14 ± 0.00 & 0.01 ± 0.00 & 0.14 ± 0.00 \\
Promptist & 0.70 ± 0.05 & 0.02 ± 0.00 & -0.05 ± 0.01 & 0.03 ± 0.00 \\
Rule-Based & 0.59 ± 0.02 & 0.12 ± 0.00 & 0.01 ± 0.00 & 0.12 ± 0.00 \\
GFlowNets & 0.81 ± 0.15 & 0.20 ± 0.00 & \textbf{0.02 ± 0.00} & 0.14 ± 0.00 \\
\midrule
\textbf{PAG (Ours)} & \textbf{0.83 ± 0.01} & \textbf{0.29 ± 0.00} & \textbf{0.02 ± 0.00} & \textbf{0.37 ± 0.00} \\
\bottomrule
\end{tabular}}
\label{tab:extend_reward_fn}
\vspace{-10pt}
\end{table}
\vspace{-.03in}
\subsection{Robustness across Different Reward Functions}
To evaluate the robustness of our framework in text-to-image modeling, we extend our experiments to include 
two additional widely-used reward functions in diffusion alignment: ImageReward \citep{xu2024imagereward} and HPSv2 \citep{wu2023human}. We maintain the same training procedure as our primary experiments while incorporating these different reward functions, and utilize COCO dataset for evaluation. Please refer to \Cref{app:exp_details} for more detailed experimental procedures. 

\Cref{tab:extend_reward_fn} summarizes the results of different methods, which demonstrates that PAG achieves state-of-the-art performance in terms of both reward and diversity metrics. This consistent performance across multiple reward functions validates the robustness of our approach. 

\subsection{Transferability to Different Text-to-Image Diffusion Models}
Since our method operates on prompt adaptation without modifying the underlying diffusion model parameters,
it has the potential to generalize across different text-to-image diffusion models in a zero-shot manner. 
To validate this, we evaluate the transferability of our method on various representative text-to-image diffusion models distinct from the target model used during training, including SD v1.5 \citep{rombach2022high}, SSD-1B \citep{gupta2024progressive}, SDXL-Turbo \citep{sauer2025adversarial}, and SD3 \citep{esser2024scaling} for evaluation. We use prompts generated by each method using SD v1.4 with initial prompts from the ChatGPT dataset.

\Cref{tab:transfer_t2i_models} summarizes the performance of various methods on different text-to-image diffusion models. As demonstrated in the table, PAG consistently generates high-rewarding images compared to other methods due to its ability to produce diverse prompts, which provides robustness on different text-to-image diffusion models. Surprisingly, we observe that there is a significant gap between our method and baselines in the SD3 model, showcasing that our method can be applied in practical settings. We also visualize the generated images across different text-to-image diffusion models in \cref{app:more_visualization}.

\begin{figure*}[t]
    \centering
    \includegraphics[width=0.95\textwidth]{figures/ddpo_comparison.jpg}
    \caption{Comparison with DDPO and PAG. We report the aesthetic score of images in bold.}
    \label{fig:ddpo_comparison}
    \vspace{-12pt}
\end{figure*}
\begin{table}[t]
\centering
\caption{We train the policy with SD v1.4 as a target model and evaluate the generated propmts with different text-to-image models in a zero-shot manner.}
\vspace{-5pt}
\resizebox{\linewidth}{!}{
\begin{tabular}{lcccc}
\toprule
\multirow{2}{*}{\textbf{Method}} & \multicolumn{4}{c}{Text-to-Image Diffusion Models} \\
\cmidrule{2-5}
& SD v1.5 & SSD-1B & SDXL-Turbo & SD3 \\
\midrule
SFT & 0.78 ± 0.05 & 0.53 ± 0.05 & 0.54 ± 0.05 & 0.77 ± 0.04 \\
Promptist & 0.80 ± 0.03 & 0.46 ± 0.05 & 0.47 ± 0.06 & 0.73 ± 0.03 \\
Rule-Based & 0.85 ± 0.02 & 0.54 ± 0.05 & 0.54 ± 0.05 & 0.76 ± 0.03 \\
GFlowNets & 0.62 ± 0.04 & 0.51 ± 0.02 & 0.49 ± 0.01 & 0.75 ± 0.06 \\
\midrule
\textbf{PAG (Ours)}  & \textbf{0.87 ± 0.02} & \textbf{0.61 ± 0.04} & \textbf{0.67 ± 0.02} & \textbf{0.95 ± 0.05} \\
\bottomrule
\end{tabular}}
\label{tab:transfer_t2i_models}
\vspace{-15pt}
\end{table}

\subsection{Comparison with Fine-tuning Text-to-Image Diffusion Models}
To verify the effectiveness of our framework, we also compare our method with approaches that directly fine-tune text-to-image diffusion models, and compare with DDPO \citep{blacktraining}, which is trained with the aesthetic score reward function on animal prompts. As shown in \Cref{fig:ddpo_comparison}, we find that PAG achieves competitive performance with DDPO in terms of aesthetic quality. Furthermore, we observe that generated samples from DDPO converge to similar styles, whereas PAG generates diverse and high-quality images, indicating that prompt adaptation can be a promising alternative and a complementary approach to directly fine-tuning diffusion models for generating images with desired properties.  
For more details on comparison between directly fine-tuning diffusion models and our method, please refer to \Cref{app:comp_diffusion}.

\subsection{Ablation Studies}
In this section, we conduct comprehensive ablation studies to investigate the effectiveness of the important components and analyze how our method systematically tackles the severe mode collapse problem. 

\Cref{tab:ablation} shows the effectiveness of flow reactivation (Reset), reward-prioritized sampling (PRT), and reward decomposition (FL) in COCO and ChatGPT datasets. As shown, each component contributes substantially to the overall performance. Notably, we observe that removing PRT leads to significant performance degradation, which indicates that we should carefully tackle the mode collapse issue from the angle of both network parameters and training samples. Moreover, reward decomposition improves both reward and diversity metrics, validating its effectiveness for better credit assignment and mitigating mode collapse.
Additional component analysis can be found in Appendix~\ref{app:extend_ablation}.

\begin{table*}
  [t]
  \centering
  \resizebox{\textwidth}{!}{%
  \begin{tabular}{cccccccccccc}
    \toprule \multicolumn{2}{c}{Components}                                                             & \multicolumn{5}{c}{Re-executability Rate (\%)} & \multicolumn{5}{c}{Readability (\#)} \\
    \cmidrule(lr){1-2} \cmidrule(lr){3-7} \cmidrule(lr){8-12}        \hspace{8pt}\labelemoji\hspace{8pt}                                                                & \hspace{8pt}\toolemoji\hspace{8pt}                                      & O0                                 & O1             & O2             & O3             & AVG            & O0             & O1             & O2             & O3             & AVG            \\
    \hline
    \rowcolor[rgb]{0.93,0.93,0.93}\multicolumn{12}{c}{\textbf{Initialize with LLM4Decompile-End-6.7B~\citep{llm4decompile}}}   \\
    \xmark                                                                                              & \xmark                                    & 69.51                              & 46.95          & 50.61          & 46.34          & 53.35          & 3.98 & 3.41 & 3.44 & 3.38 & 3.55 \\
    \cmark                                                                                              & \xmark                                    & 75.61                              & 50.61          & 50.00          & 50.00          & 56.55          & 4.01 & 3.44 & 3.39 & \textbf{3.49} & 3.58 \\
    \xmark                                                                                              & \cmark                                    & 83.54                     & \textbf{56.10}          & 51.22          & 50.61 & 60.37 & 4.05 & 3.51 & 3.51 & 3.42 & 3.62 \\
    \cmark                                                                                              & \cmark                                    & \textbf{85.37}                            & \textbf{56.10}                     & \textbf{51.83} & \textbf{52.43}          & \textbf{61.43} & \textbf{4.13} & \textbf{3.60} & \textbf{3.54} & \textbf{3.49} & \textbf{3.69} \\

    \rowcolor[rgb]{0.93,0.93,0.93}\multicolumn{12}{c}{\textbf{Initialize with Deepseek-Coder-6.7B-base~\citep{deepseekcoder}}} \\
    \xmark                                                                                              & \xmark                                    & 59.15                              & 35.98          & 39.02          & 37.80          & 42.99          & 3.71 & 3.05 & 3.16 & 3.05 & 3.24 \\
    \cmark                                                                                              & \xmark                                    & 66.46                              & 41.46          & 38.41          & 36.59          & 45.73          & 3.76 & 3.17 & \textbf{3.21} & 3.08 & 3.31 \\
    \xmark                                                                                              & \cmark                                    & 70.73                              & 39.63          & 39.02          & 40.24          & 47.41          & 3.90 & 3.17 & 3.08 & 3.11 & 3.31 \\
    \cmark                                                                                              & \cmark                                    & \textbf{79.88}                     & \textbf{45.73} & \textbf{43.90} & \textbf{42.68} & \textbf{53.05} & \textbf{3.96} & \textbf{3.21} & 3.18 & \textbf{3.19} & \textbf{3.38} \\
    \bottomrule
  \end{tabular}%
  }
  \caption{The ablation study of different methods across four optimization levels
  (O0, O1, O2, O3), as well as their average scores (AVG). The results in bold represent the optimal performance. The ~\labelemoji~ and ~\toolemoji~ means Relabedling and Function Call. \textbf{Bold} denotes the best performance.}
  \label{tab:ablation}
\end{table*}
\begin{figure}[t]
\begin{minipage}[t]{\linewidth}
    \begin{subfigure}[t]{0.48\linewidth}
        \centering
        \includegraphics[width=\textwidth]{figures/dormant.png}
    \end{subfigure}
    \begin{subfigure}[t]{0.48\linewidth}
        \centering
        \includegraphics[width=\textwidth]{figures/cos_dist.png}
    \end{subfigure}
    \caption{The effect of flow reactivation.}
    \label{fig:ablation}
\end{minipage}
\vspace{-15pt}
\end{figure}

Furthermore, we carefully analyze how flow reactivation mitigates the mode collapse problem by tracking the percentage of dormant neurons (based on Eq.~\ref{eq:dormant}) over the course of training. As depicted in \Cref{fig:ablation}, our flow reactivation mechanism exhibits a significantly lower dormant neuron rate, while the variant without this mechanism results in a high portion of dormant neurons during training. We also plot the cosine distance among generated prompts over training and observe that, without reactivation, the GFlowNet policy suffers from the mode collapse issue.
These findings validate its effectiveness in preventing our GFlowNets-based agent from converging to limited patterns by maintaining the expressivity of the flows, contributing to both performance and diversity improvements.

\section{Related Works}
\noindent\textbf{Aligning Diffusion Models}
There is a surge of interest in generating images with desired properties, which can be modeled as reward functions from human feedback \citep{ouyang2022training}. 
A widely recognized approach involves directly fine-tuning diffusion models with the reward function. \citet{blacktraining} and \citet{fan2024reinforcement} formulate the diffusion reverse process as a MDP and employ RL for fine-tuning diffusion models while \citet{clarkdirectly} and \citet{prabhudesai2023aligning} update model parameters through end-to-end backpropagation of the gradient of reward across denoising steps. 
While those methods have shown promising results, 
they require initiating fine-tuning from scratch for each different text-to-image diffusion model and necessitate access to the model parameters, which may be restricted due to confidential issues \citep{ramesh2022hierarchical, saharia2022photorealistic}. Unlike these methods, PAG enhances initial prompts into high-quality and diverse prompts by fine-tuning language models, allowing transferability across various text-to-image models. 

\vspace{5pt}
\noindent\textbf{Prompt Adaptation for Text-to-Image Models}
There have been some trials to generate high-quality images by adapting prompts instead of fine-tuning text-to-image models. A pioneering work of this approach is Promptist \citep{hao2024optimizing}, which formulates prompt adaptation as an RL problem.
A relevant recent work, DPO-Diff, \citep{wang2024discrete}, also tries to generate user-aligned images while optimizing negative prompts using a shortcut text gradient. We find that Promptist often results in deterministic policy, which can be easily replaced by heuristics. PAG utilizes GFlowNets to fine-tune language models for generating effective and diverse prompts. 

\vspace{5pt}
\noindent\textbf{GFlowNet Fine-Tuning}
GFlowNets are probabilistic methods that sample compositional objects proportional to unnormalized density through sequential decision-making~\citep{bengio2021flow, bengio2023gflownet} and energy-based modeling~\citep{zhang2022generative}, with applications in structure learning~\citep{deleu2022bayesian} and combinatorial optimization~\citep{zhangrobust,Zhang2023LetTF}, which can also be extended to continuous~\citep{lahlou2023cgfn} or stochastic scenarios~\citep{pan2023stochastic,zhang2023distributional}.
It has the potential for fine-tuning language models (LMs) for intractable posterior inference problems~\citep{huamortizing} and robust red-teaming~\citep{lee2024learning}.
Although there have been several attempts in improving exploration and training efficiency~\citep{pan2023generative, kimlocal, lau2024qgfn} and extending it to more general domains and learning paradigms, previous works typically train a GFlowNets policy from scratch~\citep{pan2024pre,he2024looking}, and largely overlooked the critical problem of plasticity loss during fine-tuning~\citep{zhang2024improving,liu2024efficientdiversitypreservingdiffusionalignment}.
Furthermore, most prior methods focus on unconditional generation and often suffer from mode collapse, requiring an additional post-supervised fine-tuning stage. 
In this work, we investigate the mode collapse problem in prompt adaptation and propose PAG, a novel approach to address this key challenge for diverse conditional prompt generation for text-to-image diffusion models.

\section{Conclusion}
In this paper, we propose a novel approach that systematically addresses mode collapse in prompt adaptation through GFlowNets-based probabilistic inference. We identify a critical plasticity loss problem in GFlowNets-based prompt adaptation when learning from rewards sequentially, and present PAG to maintain generation flexibility. Our extensive results show that PAG successfully learns to sample effective and diverse prompts for text-to-image generation. 

{
    \bibliographystyle{ieeenat_fullname}
    \bibliography{main}
}

\clearpage
\appendix

\noindent Our code is publicly available \href{https://github.com/dbsxodud-11/PAG.git}{here}.

\section{Illustrations of RL and GFlowNets}\label{app:diff_btw_rl_and_gfn}
In this section, we summarize the key difference between Reinforcement Learning (RL) and GFlowNets in more detail. GFlowNets~\citep{bengio2021flow} is designed to learn a stochastic policy that generates samples proportional to their rewards (i.e., $p(x)\propto R(x)$), while RL aims to learn a policy that maximizes the reward function as illustrated in \Cref{fig:rl_vs_gfn}. Learning a policy that samples proportional to the reward function leads to capturing multi-modal distribution and discovering high-quality and diverse candidates, which is particularly beneficial when the reward proxy is inaccurate \citep{bengio2021flow}, and their connections have been studied in \citep{tiapkin2024generative,deleu2024discrete,he2024rectifying}, which shares equivalences and similarities with entropy-regularized RL~\citep{nachum2017bridging} in tree-structured sequence generation and directed acyclic graph problems~\citep{huamortizing,lee2024learning}.
In prompt adaptation tasks, conventional RL-based approaches~\citep{hao2024optimizing} based on PPO~\citep{schulman2017proximal} that naively maximize a reward function can lead to reward overoptimization and hinder generalizability to different text-to-image diffusion models, while it is more desirable to generate not only effective and but also diverse prompts.     
\begin{figure}[h]
    \centering
    \includegraphics[width=0.9\linewidth]{figures/rl_vs_gfn.png}
    \vspace{-5pt}
    \caption{Comparison of the learning objective of reward-maximizing RL and reward-matching GFlowNets.}
    \label{fig:rl_vs_gfn}
    \vspace{-12pt}
\end{figure}

\section{Experiment Setting Details}\label{app:exp_details}
In this section, we present details of experiment settings.
\subsection{Data Preparation}
We strictly follow the procedure of Promptist \citep{hao2024optimizing} to prepare training and evaluation datasets. Additionally, we introduce a challenging dataset using ChatGPT \citep{ouyang2022training} interface to generate brief prompts that describe images. Specifically, we use the following prompt to query ChatGPT for short image descriptions:
\begin{itemize}
    \item \texttt{Generate N sentences describing photos /pictures/images with length around 5.}
\end{itemize}

\noindent Below are a few example prompts generated by ChatGPT:
\begin{itemize}
    \item \texttt{A bird sitting on a branch.}
    \item \texttt{A tree under a sky.}
    \item \texttt{A car drives on a road.}
    \item \texttt{A train moves through the city.}
    \item \texttt{A boat sails on a lake.}
\end{itemize}

\subsection{Baselines}
In this section, we provide more details on the baselines used for our experiments.

\paragraph{Supervised Fine-Tuning.} We fine-tune the pretrained GPT-2 policy model with supervised learning on a set of prompt pairs of original user inputs and manually engineered prompts provided by Promptist \citep{hao2024optimizing}. As a default, we use the pretrained weights of SFT publicly available\footnote{https://github.com/microsoft/LMOps/tree/main/promptist}.

\paragraph{Promptist.} We train the policy with a PPO-based approach where the policy is initialized with the supervised fine-tuned model. As a default, we use the pretrained weights of Promptist publicly available to ensure a fair comparison. To evaluate the generalizability of different reward functions, we train the policy with the same hyperparameter configurations.

\paragraph{Rule-Based.} Based on the observation that Promptist mostly generates similar postfixes for optimization, we build a rule-based method that appends the most frequently used postfixes in Promptist to the initial prompt. Below are a few example postfixes we used for evaluation:
\begin{itemize}
    \item \texttt{intricate, elegant, highly detailed, digital painting, artstation, concept art, sharp focus, illustration, by justin gerard and artgerm, 8 k.}
    \item \texttt{by greg rutkowski, digital art, realistic painting, fantasy, very detailed, trending on artstation.}
    \item \texttt{highly detailed, digital painting, artstation, concept art, sharp focus, illustration, art by greg rutkowski and alphonse mucha.}
\end{itemize}

\paragraph{GFlowNets.} We train a vanilla GFlowNets policy with TB \citep{malkin2022trajectory} objective. As our task is a conditional generation task, a naive implementation of TB should directly estimate a conditional partition function, $Z_{\theta}(\mathbf{x})$, which makes learning highly unstable \citep{kim2024ant}. For pratical implementation, we use VarGrad \citep{richter2020vargrad} objective to fine-tune the policy, which is widely used for reducing variance in GFlowNets literature \citep{zhangrobust, venkatraman2024amortizing}.

For each initial prompt $\mathbf{x}$ sampled in the minibatch, we generate $k=16$ on-policy samples $\mathbf{y}^{(1)},\cdots,\mathbf{y}^{(k)}$ with the forward policy. Each sample can be used to implicitly estimate $\log Z(\mathbf{x})$:
\begin{align*}
    \log \hat{Z}(\mathbf{x})^{(i)}=R(\mathbf{x},\mathbf{y}^{(i)})-\sum_{t=1}^{T}\log P_{F}(y_{t}^{(i)}\vert y_{0:t-1}^{(i)},\mathbf{x};\theta)
\end{align*}
Then we minimize the sample variance across the minibatch as follows:
\begin{align*}
    \mathcal{L}(\mathbf{x};\theta)=\frac{1}{k}\sum_{i=1}^{k}\left(
    \log\hat{Z}(\mathbf{x})^{(i)}-\frac{1}{k}\sum_{j=1}^{k}\log\hat{Z}(\mathbf{x})^{(j)}\right)^2
\end{align*}

\paragraph{DPO-Diff.} \citet{wang2024discrete} proposed a discrete prompt optimization for diffusion models (DPO-Diff), which is a gradient-based optimization method for discovering effective negative prompts to generate user-aligned images. It first generates a compact subspace comprised of only the most relevant words to user input with ChatGPT API, then uses shortcut text gradients to efficiently compute the text gradient for optimization. As the original reward function of DPO-Diff is a spherical clip loss, we replace the reward function the same as \cref{eq:task_reward}. As we consider a setting where the reward function is a black-box function, we use the evolutionary search module suggested by the paper for a fair comparison. Please refer to the paper for more details.

\subsection{Training and Evaluation}
For training, we initialize the policy with the SFT policy before the GFlowNet fine-tuning. To parametrize the flow function, we use a separate neural network that takes the last hidden embedding of the prompts as input and outputs a scalar value. We conducted all experiments with 4 NVIDIA A100 GPUs, and the training took approximately 24 hours.

For evaluation, we study two metrics: reward and diversity. To compute the reward, we generate $N=16$ prompts for each initial prompt via beam search with a length penalty of $1.0$. Then, we generate three images per prompt to compute the reward. We aggregate the max score among $N$ prompts and compute the average across initial prompts. 
\begin{align*}
    \text{Reward}(\mathcal{D}_{\text{eval}})&:=\frac{1}{\vert\mathcal{D}_{\text{eval}}\vert}\sum_{\mathbf{x}\in\mathcal{D}_{\text{eval}}}\max_{\mathbf{y}\sim p_{\theta}(\cdot\vert\mathbf{x})}\left(r(\mathbf{x},\mathbf{y})\right)
\end{align*}
To compute diversity, we embed the generated prompts using MiniLMv2 \citep{wang2020minilmv2} encoder, compute the average pairwise cosine distance between embeddings of the prompts, and compute the average across initial prompts.
For all evaluations, we conduct experiments with four random seeds and report the mean and standard deviation.

The input format for both training and evaluation is \texttt{[Initial Prompt] Rephrase:}, following Promptist \citep{hao2024optimizing}. The hyperparameters we used for modeling and training are listed in \Cref{tab:hyperparam}. We conduct several ablations studies on the impact of various hyperparameters in \Cref{app:extend_ablation}.
\begin{table}[!thp]
\caption{
\label{tab:hyperparam}
Hyperparameter configurations for the baseline methods evaluated in our experiments. These settings are used across multiple tasks to assess model performance in low-resource settings, as discussed in Sections~\ref{sec:intro} and Section~\ref{sec:exp}.
}
\resizebox{\textwidth}{!}{
\begin{tabular}{l|l|p{8cm}}
\toprule
\textbf{Baseline} & \textbf{Hyperparameter} & \textbf{Values} \\
\midrule
\multirow{2}{*}{BitFit~\cite{ben-zaken-etal-2022-bitfit}} & Bias Moudule & bias of Q,K and V from attention/bias of LayerNorm from attention outputs/bias of LayerNorm from hidden outputs \\ \cmidrule{2-3}
 & Learning Rate & 1e-4/ 5e-4 \\
\midrule
\multirow{2}{*}{RED~\cite{wu-etal-2024-advancing}} & Rank & 8 / 16 \\
\cmidrule{2-3} 
 & Learning Rate & 5e-5/ 2e-4 / 6e-2 \\
\midrule
REPE~\cite{zou2023representation} & method & Representation Reading / Representation Control \\
\midrule
\multirow{2}{*}{ReFT~\cite{wu2024reft}} & Prefix + suffix posotion & p7 + s7 / p11 + s11 \\
\cmidrule{2-3} 
 & Layers & all / 4,6,10,12,14,18,20,22/3,9,18,24 \\
\midrule
\multirow{2}{*}{LoFIT~\cite{yin2024lofit}} & number of attention heads & 32/64/128 \\
\cmidrule{2-3} 
 & Learning Rate & 5e-4 / 5e-3 \\
\bottomrule
\end{tabular}
}
\end{table}


\subsection{Robustness across Different Reward Functions}
To evaluate the robustness of our framework in terms of different reward functions, we use two widely-used reward functions in diffusion alignment: ImageReward \citep{xu2024imagereward} and HPSv2 \citep{wu2023human}. We provide a detailed description of each function below.
\paragraph{ImageReward} ImageReward is a general-purpose text-to-image preference reward model trained with pairs of prompts and images. To compute the score, ImageReward extracts image and text embeddings using BLIP \cite{li2022blip} encoder and combines them with cross attention, and uses MLP to generate a scale value for preference comparison.
\paragraph{HPSv2} HPSv2 is a scoring model that accurately predicts human preferences on generated images. To accurately predict the score, it fine-tunes the CLIP \citep{radford2021learning} with the HPDv2 dataset, a large-scale dataset that captures human preferences on images from various sources.

\subsection{Transferability to Different Text-to-Image Diffusion Models}
To validate the transferability of our framework to different text-to-image diffusion models, we prepare several widely-used text-to-image diffusion models: SD v1.5 \citep{rombach2022high}, SSD-1B \citep{gupta2024progressive}, SDXL-Turbo \citep{sauer2025adversarial}, and SD3 \citep{esser2024scaling}. As SDXL-Turbo and SD3 models do not align with DPM solver \citep{lu2022dpm}, we use a standard generation pipeline, which uses a PNDM scheduler \citep{liu2022pseudo} with 20 inference steps. Furthermore, as SDXL-Turbo does not use the guidance scale, we set the guidance scale to 0. for SDXL-Turbo, and 7.5 (default) for others. 

\begin{table*}[t]
\centering
\caption{Performance of prompt generated by each method. Aes score indicates aesthetic quality improvement compared to the image generated with the original input. Experiments are conducted with four random seeds, and mean and standard deviation are reported. \textbf{Bold} represent the best entry.}
\vspace{-5pt}
\resizebox{0.95\textwidth}{!}{
\begin{tabular}{lcc|cc|cc|cc}
\toprule
\multirow{3}{*}{\textbf{Method}} & 
\multicolumn{2}{c}{Lexica} & \multicolumn{2}{c}{DiffusionDB} &
\multicolumn{2}{c}{COCO} & \multicolumn{2}{c}{ChatGPT}  \\
\cmidrule{2-9}
% & Aes. & CLIP & Div. & Aes. & CLIP & Div.
% & Aes. & CLIP & Div. & Aes. & CLIP & Div. \\
& Reward & Diversity & Reward & Diversity 
& Reward & Diversity & Reward & Diversity \\
\midrule
Initial Prompt & -0.16 ± 0.03 & - 
               & -0.22 ± 0.02 & -
               & -0.35 ± 0.01 & -
               & -0.42 ± 0.03 & - \\
\midrule
SFT & 0.64 ± 0.02 & 0.13 ± 0.00
    & 0.58 ± 0.01 & 0.13 ± 0.00 
    & 0.54 ± 0.07 & 0.11 ± 0.02
    & 0.76 ± 0.03 & 0.19 ± 0.00 \\
Promptist & 0.76 ± 0.02 & 0.09 ± 0.00
    & 0.76 ± 0.03 & 0.10 ± 0.00
    & 0.65 ± 0.02 & 0.07 ± 0.00 
    & 0.76 ± 0.03 & 0.12 ± 0.00 \\
Rule-Based & 0.72 ± 0.02 & 0.26 ± 0.00
    & 0.69 ± 0.02 & 0.15 ± 0.00 
    & 0.75 ± 0.01 & 0.12 ± 0.00
    & 0.82 ± 0.03 & 0.17 ± 0.00 \\
GFlowNets & 0.96 ± 0.01 & 0.13 ± 0.00
    & 0.85 ± 0.03 & 0.13 ± 0.00
    & 0.73 ± 0.02 & 0.09 ± 0.00
    & 0.63 ± 0.03 & 0.10 ± 0.00 \\
DPO-Diff & 0.13 ± 0.02 & -
    & 0.28 ± 0.06 & -
    & -0.03 ± 0.06 & -
    & -0.17 ± 0.03 & - \\
\midrule
PAG (Ours) & \textbf{0.99 ± 0.05} &  \textbf{0.32 ± 0.00} 
    & \textbf{0.91 ± 0.04} & \textbf{0.33 ± 0.00}
    & \textbf{0.88 ± 0.02} &  \textbf{0.32 ± 0.00} 
    & \textbf{0.88 ± 0.04} & \textbf{0.32 ± 0.00} \\
\bottomrule
\end{tabular}}
\label{tab:main1}
\end{table*}
\begin{figure*}[t]
\begin{minipage}[t]{\textwidth}
    \begin{subfigure}[t]{0.24\textwidth}
        \centering
        \includegraphics[width=\textwidth]{figures/ablation_beta.png}
        \subcaption{Analysis on $\beta$}
        \label{fig:ablation_beta}
    \end{subfigure}
    \begin{subfigure}[t]{0.24\textwidth}
        \centering
        \includegraphics[width=\textwidth]{figures/ablation_M.png}
        \subcaption{Analysis on $M$}
        \label{fig:ablation_M}
    \end{subfigure}
    \begin{subfigure}[t]{0.24\textwidth}
        \centering
        \includegraphics[width=\textwidth]{figures/ablation_lr.png}
        \subcaption{Analysis on $\gamma$ for flow function}
        \label{fig:ablation_gamma}
    \end{subfigure}
    \begin{subfigure}[t]{0.24\textwidth}
        \centering
        \includegraphics[width=\textwidth]{figures/ablation_reset.png}
        \subcaption{Analysis on reset strategies}
        \label{fig:ablation_reset}
    \end{subfigure}
    \caption{Extended ablation studies on various components of PAG.}
    \label{fig:ablation_app}
\end{minipage}
\end{figure*}
\section{Extended Main Results}\label{app:extend_main_results}
In this section, we provide additional discussion and analysis of our main experimental results.
\subsection{Main Results}
We summarize the main experiment results in \Cref{tab:main1} including comparisons with DPO-Diff \citep{wang2024discrete}, a recent relevant work. Note that as DPO-Diff tries to optimize negative prompts and the initial prompt is always the same, it is meaningless to compute diversity between generated prompts.
\subsection{Discussion}
As shown in the table, we observe that DPO-Diff shows modest improvements in terms of the reward compared to other baselines. We find that while DPO-Diff can effectively improve the CLIP scores by optimizing negative prompts, it shows limited capability in improving the aesthetic score.   

\section{Extended Ablation Studies}\label{app:extend_ablation}
In this section, we include additional ablation studies that complement our main text due to space limitations.

\subsection{Analysis on $\beta$}
First, we analyze the effect of inverse temperature $\beta$ in \cref{eq:reward}, which controls the balance between the task reward term $r(\mathbf{x}, \mathbf{y})$ and the reference LM likelihood term $p_{\text{ref}}(\mathbf{y}\vert\mathbf{x})$. As a default setting, we set $\beta=0.05$. 

To analyze the effect of $\beta$, we fine-tune the GFlowNet policy with different $\beta$ values: $\{0.01, 0.05, 0.1, 0.2\}$. As shown in the \Cref{fig:ablation_beta}, there are no significant differences in terms of the performance with different $\beta$ values while using too small $\beta$, which leads to the policy focus on a high-reward region, suffers from mode collapse similar to naive GFlowNet and exhibits poor performance. 

\begin{figure*}[t]
    \centering
    \includegraphics[width=0.9\textwidth]{figures/dpok_comparison.jpg}
    \caption{Comparison with DPOK and PAG. We report the aesthetic score of images in bold.}
    \label{fig:dpok_comparison}
\end{figure*}

\subsection{Analysis on $M$}
We also conduct experiments by varying the flow function reset period ($M$) to analyze the effect of flow reactivation. If we reset the flow function too frequently, it is hard to capture high-rewarding multi-modal distribution, while rarely applying reset leads to the mode collapse issue. As a default setting, we use $M=2000$, which means that we reactivate the flow function four times over the whole training procedure.  

To analyze the effect of $M$, we fine-tune the GFlowNet policy with different $M$ values: $\{1000, 2000, 4000\}$. As shown in the \Cref{fig:ablation_M}, we find that too frequent reactivation ($M=1000$) does not capture high-reward regions and suffers from a significant drop in performance. While there is no big difference between $M=2000$ and $M=4000$, we empirically find that set $M=2000$ achieves the best performance in terms of both reward and diversity. This empirical finding is also aligned with the other papers, which utilize a reset strategy: \citet{nikishin2022primacy} also reset the last layer of the neural networks four times over the course of training.

\subsection{Analysis on learning rate of flow function}
Based on the observation that most actor-critic RL methods use slightly higher learning rates for the critic than the actor \citep{schulman2017proximal}, we use a higher learning rate ($1\times10^{-4}$) for the flow function training than the learning rate of the policy ($1\times10^{-5}$). Using a higher learning rate is also crucial for quickly recovering from the flow reactivation.

To analyze the effect of learning rate ($\gamma$) for the flow function, we fine-tune the GFlowNet policy with different $\gamma$ values: $\{1\times10^{-3}, 1\times10^{-4}, 1\times10^{-5}\}$. As shown in the \Cref{fig:ablation_gamma}, we find that using the same learning rate for the policy and the flow function leads to poor performance as expected. While there is no big difference between $\gamma=1\times10^{-3}$ and $\gamma=1\times10^{-4}$, we empirically find that set $\gamma=1\times10^{-4}$ achieves the best performance in terms of both reward and diversity. 

\subsection{Analysis on flow reactivation scheme}
To prevent a significant drop in performance and unstable training, we employ a targeted reset strategy that resets only the last layer of the flow function. To analyze the effect of our strategy, we conduct experiments with two additional reset strategies: (1) reset the whole layer of the flow function and (2) reset the policy. For resetting the policy, we randomly reset $0.01\%$ of neurons in the policy parameters.

\Cref{fig:ablation_reset} shows the performance of various reset strategies. As depicted in the figure, resetting the whole layer of the flow function does not recover policy towards high-scoring regions. Resetting the policy parameters also exhibits poor performance, as the policy directly affects the acquisition of online samples.

\subsection{Analysis on Diversity of Images}
We also compute the diversity between the final images sampled from text-to-image diffusion models conditioned on prompts generated by different methods. To compute diversity, we compute the average pairwise distance between feature vectors extracted by the pre-trained InceptionV3 model \citep{szegedy2016rethinking}. As shown in \Cref{tab:div_images}, PAG exhibits high diversity on images compared to baselines.
\begin{table}[h]
\centering
\caption{Diversity evaluation on images.}
\vspace{-10pt}
\resizebox{\linewidth}{!}{
\begin{tabular}{lcc|cc}
\toprule
\multirow{2}{*}{\textbf{Method}} & \multicolumn{2}{c}{COCO} & \multicolumn{2}{c}{ChatGPT}  \\
\cmidrule{2-5}
% & Lexica & DiffusionDB & COCO & ChatGPT \\
& Reward & Diversity & Reward & Diversity \\
\midrule
SFT & 0.54 ± 0.07 & 0.20 ± 0.01 & 0.76 ± 0.03 & 0.19 ± 0.01 \\
Promptist & 0.65 ± 0.02 & 0.19 ± 0.01 & 0.76 ± 0.03 & 0.17 ± 0.01 \\
Rule-Based & 0.75 ± 0.01 & 0.19 ± 0.00 & 0.82 ± 0.03 & 0.18 ± 0.01 \\
GFlowNets & 0.73 ± 0.02 & 0.18 ± 0.01 & 0.63 ± 0.03 & 0.16 ± 0.01 \\
\midrule
PAG (Ours) & \textbf{0.88 ± 0.02} & \textbf{0.21 ± 0.01} & \textbf{0.88 ± 0.04} & \textbf{0.20 ± 0.01}\\
\bottomrule
\end{tabular}}
\vspace{-15pt}
\label{tab:div_images}
\end{table}

\section{Comparison with Directly Fine-tuning Diffusion Models}\label{app:comp_diffusion}
In this section, we explain in detail the comparison with directly fine-tuning diffusion models to generate images with desired properties.
\subsection{Experiment Setup}
We strictly follow the experiment setup of DDPO\footnote{https://github.com/jannerm/ddpo} and DPOK\footnote{https://github.com/google-research/google-research/tree/master/dpok} for fine-tuning diffusion models. We fine-tune diffusion models with aesthetic quality as a reward function and use prompts from a list of 45 common animals. 

\subsection{Comparison with DPOK}
We also compare our method with DPOK \citep{fan2024reinforcement}, another representative method for fine-tuning diffusion models with black-box reward functions. As shown in \Cref{fig:dpok_comparison}, generated images from DPOK converge to similar styles, whereas PAG generates diverse and high-quality images.

\section{Additional Visualizations}\label{app:more_visualization}
In this section, we present additional visualizations to show the effectiveness of PAG for text-to-image generation as shown in Figures~\ref{fig:main_figure_type2}-\ref{fig:main_figure_type3} (besides Figure~\ref{fig:main_figure} in the main text). We also summarize the results for robustness across different reward functions and transferability to different text-to-image diffusion models as shown in Figure~\ref{fig:reward_fn}-\ref{fig:transfer}.
\begin{figure*}[t]
    \centering
    \includegraphics[width=0.9\textwidth]{figures/main_figure_type2.jpg}
    \caption{Additional images generated by optimized prompts using Stable Diffusion v1.4. We use the same seed to visualize the effect solely on prompt adaptation.}
    \label{fig:main_figure_type2}
\end{figure*}

\begin{figure*}[t]
    \centering
    \includegraphics[width=0.9\textwidth]{figures/main_figure_type3.jpg}
    \caption{Additional images generated by optimized prompts using Stable Diffusion v1.4. We use the same seed to visualize the effect solely on prompt adaptation.}
    \label{fig:main_figure_type3}
\end{figure*}

\begin{figure*}[t]
    \centering
    \includegraphics[width=0.9\textwidth]{figures/reward_fn.jpg}
    \caption{Images generated with prompts fine-tuned with different reward functions. We use the same seed to visualize the effect solely on prompt adaptation.}
    \label{fig:reward_fn}
\end{figure*}

\begin{figure*}[t]
    \centering
    \includegraphics[width=0.9\textwidth]{figures/transfer.jpg}
    \caption{Images generated with different text-to-image diffusion models. We use the same seed to visualize the effect solely on prompt adaptation.}
    \label{fig:transfer}
\end{figure*}

\end{document}

