\documentclass[10pt,twocolumn,letterpaper]{article}

\usepackage[pagenumbers]{cvpr} 

%
% --- inline annotations
%
\newcommand{\red}[1]{{\color{red}#1}}
\newcommand{\todo}[1]{{\color{red}#1}}
\newcommand{\TODO}[1]{\textbf{\color{red}[TODO: #1]}}
% --- disable by uncommenting  
% \renewcommand{\TODO}[1]{}
% \renewcommand{\todo}[1]{#1}



\newcommand{\VLM}{LVLM\xspace} 
\newcommand{\ours}{PeKit\xspace}
\newcommand{\yollava}{Yo’LLaVA\xspace}

\newcommand{\thisismy}{This-Is-My-Img\xspace}
\newcommand{\myparagraph}[1]{\noindent\textbf{#1}}
\newcommand{\vdoro}[1]{{\color[rgb]{0.4, 0.18, 0.78} {[V] #1}}}
% --- disable by uncommenting  
% \renewcommand{\TODO}[1]{}
% \renewcommand{\todo}[1]{#1}
\usepackage{slashbox}
% Vectors
\newcommand{\bB}{\mathcal{B}}
\newcommand{\bw}{\mathbf{w}}
\newcommand{\bs}{\mathbf{s}}
\newcommand{\bo}{\mathbf{o}}
\newcommand{\bn}{\mathbf{n}}
\newcommand{\bc}{\mathbf{c}}
\newcommand{\bp}{\mathbf{p}}
\newcommand{\bS}{\mathbf{S}}
\newcommand{\bk}{\mathbf{k}}
\newcommand{\bmu}{\boldsymbol{\mu}}
\newcommand{\bx}{\mathbf{x}}
\newcommand{\bg}{\mathbf{g}}
\newcommand{\be}{\mathbf{e}}
\newcommand{\bX}{\mathbf{X}}
\newcommand{\by}{\mathbf{y}}
\newcommand{\bv}{\mathbf{v}}
\newcommand{\bz}{\mathbf{z}}
\newcommand{\bq}{\mathbf{q}}
\newcommand{\bff}{\mathbf{f}}
\newcommand{\bu}{\mathbf{u}}
\newcommand{\bh}{\mathbf{h}}
\newcommand{\bb}{\mathbf{b}}

\newcommand{\rone}{\textcolor{green}{R1}}
\newcommand{\rtwo}{\textcolor{orange}{R2}}
\newcommand{\rthree}{\textcolor{red}{R3}}
\usepackage{amsmath}
%\usepackage{arydshln}
\DeclareMathOperator{\similarity}{sim}
\DeclareMathOperator{\AvgPool}{AvgPool}

\newcommand{\argmax}{\mathop{\mathrm{argmax}}}     


\def\viz{\emph{viz}\onedot}
\definecolor{cvprblue}{rgb}{0.21,0.49,0.74}
\usepackage[pagebackref,breaklinks,colorlinks,allcolors=cvprblue]{hyperref}

\usepackage{multirow}
\usepackage{graphicx}
\usepackage{subcaption}
\usepackage{algorithm}               
\usepackage{algpseudocode}
\usepackage{setspace}
\usepackage{enumitem}
\usepackage{color, colortbl}
\usepackage{amsmath}
\usepackage{amssymb}
\usepackage{mathtools}
\usepackage{natbib}

\title{Learning to Sample Effective and Diverse Prompts for Text-to-Image Generation}

\author{
{Taeyoung Yun$^{1}$\thanks{Work done during TY's visit to HKUST.} ~ Dinghuai Zhang$^{2}$ ~ Jinkyoo Park$^{1}$ ~ Ling Pan$^{3}$} 
\\
{$^{1}$ Korea Advanced Institute of Science and Technology $^{2}$ Microsoft Research} \\ 
$^{3}$ Hong Kong University of Science and Technology
}

\begin{document}
\maketitle

\begin{abstract}
Recent advances in text-to-image diffusion models have achieved impressive image generation capabilities. 
However, it remains challenging to control the generation process with desired properties (e.g., aesthetic quality, user intention), which can be expressed as black-box reward functions. 
In this paper, we focus on prompt adaptation, which refines the original prompt into model-preferred prompts to generate desired images. While prior work uses reinforcement learning (RL) to optimize prompts, we observe that applying RL often results in generating similar postfixes and deterministic behaviors.
To this end, we introduce \textbf{P}rompt \textbf{A}daptation with \textbf{G}FlowNets (\textbf{PAG}), a novel approach that frames prompt adaptation as a probabilistic inference problem. 
Our key insight is that leveraging Generative Flow Networks (GFlowNets) allows us to shift from reward maximization to sampling from an unnormalized density function, enabling both high-quality and diverse prompt generation.
However, we identify that a naive application of GFlowNets suffers from mode collapse and uncovers a previously overlooked phenomenon: the progressive loss of neural plasticity in the model, which is compounded by inefficient credit assignment in sequential prompt generation. To address this critical challenge, we develop a systematic approach in PAG with flow reactivation, reward-prioritized sampling, and reward decomposition for prompt adaptation.
Extensive experiments validate that PAG successfully learns to sample effective and diverse prompts for text-to-image generation. 
We also show that PAG exhibits strong robustness across various reward functions and transferability to different text-to-image models. 
\end{abstract}

\section{Introduction}
\begin{figure}[t]
    \centering
    \includegraphics[width=0.95\linewidth]{figures/main_figure_half.jpg}
    \caption{Comparison of adapted prompts and their corresponding images of Prompist~\cite{hao2024optimizing} (based on reward-maximizing RL) and our method, PAG. While Promptist leads to mode collapse in the prompt space and converges to similar outputs, PAG achieves high image quality while painting generation diversity.}
    \label{fig:main_figure_preview}
    \vspace{-18pt}
\end{figure}

Recent advances in diffusion models \citep{sohl2015deep, ho2020denoising}, combined with pre-trained text encoders \citep{raffel2020exploring, radford2021learning} have shown remarkable capability in generating creative and photorealistic images conditioned on novel prompts \citep{rombach2022high, ramesh2022hierarchical, saharia2022photorealistic}. However, generating images with desired properties (e.g., aesthetic quality, user intention) remains challenging, as these models are typically optimized for likelihood maximization over training data distributions~\citep{ho2020denoising}.

To further improve the generation process with desired properties, there has been a line of work focusing on fine-tuning diffusion models with human feedback~\citep{blacktraining, fan2024reinforcement,prabhudesai2023aligning, clarkdirectly, zhang2024improving}.
While these methods demonstrate promising results, they rely on access to model parameters, rendering them incompatible with several state-of-the-art closed-source models~\citep{ramesh2022hierarchical, saharia2022photorealistic}.
Moreover, as text-to-image diffusion models continue to grow exponentially in size, the computational cost of direct fine-tuning becomes prohibitively expensive. This challenge is further compounded by the need for model-specific retraining across different text-to-image diffusion models.

In contrast, prompt adaptation has emerged as a promising alternative~\citep{hao2024optimizing, kim2023multiprompter, wang2024discrete}, which aims to improve the initial prompt for generating images with desired properties. This approach eliminates the need for access to model parameters, enabling zero-shot transfer to different text-to-image diffusion models.
Notably, Promptist \citep{hao2024optimizing} employs reinforcement learning (RL) to fine-tune language models for prompt adaptation. While it shows promising results, our analysis reveals a critical limitation: its reward-maximizing principle tends to concentrate the policy on a narrow, high-reward region. This often leads to a deterministic policy that generates similar postfixes, as demonstrated in Figure~\ref{fig:main_figure_preview}. 
Such deterministic behavior can reduce the method to simple heuristics, significantly hindering its generalization capacity across different prompt types and text-to-image diffusion models. It underscores the critical need for approaches that can generate both effective and diverse adapted prompts for text-to-image generation.

In this paper, we introduce \textbf{P}rompt \textbf{A}daptation with \textbf{G}FlowNets (\textbf{PAG}), a novel approach that addresses these fundamental challenges by reformulating prompt adaptation as a probabilistic inference~\citep{bengio2021flow}. Our approach leverages Generative Flow Networks (GFlowNets, ~\citep{bengio2023gflownet}) to learn a generative policy that samples from an unnormalized reward distribution~\citep{bengio2021flow}, which is well-suited for this scenario.
While this approach shows initial promise, we observe that naively fine-tuning language models with GFlowNets for prompt adaptation suffers from a mode collapse issue. Our investigation uncovers a previously unrecognized phenomenon in GFlowNet fine-tuning that mirrors a key principle in neuroscience: gradual hardening of brain circuits~\citep{livingston1966brain, mateos2019impact}. In other words, GFlowNets agent experiences a progressive loss of neural plasticity as certain neural pathways become inactive, diminishing its capacity to learn from and adapt to diverse patterns. Furthermore, inefficient credit assignment across sequential prompt generation process leads to insufficient sample efficiency and ultimately exacerbates mode collapse issue.

To this end, we introduce our novel components in PAG to systematically address these critical challenges in prompt adaptation for text-to-image modeling. 
First, we propose a flow reactivation mechanism to revive dormant neural pathways in the GFlowNets agent, complemented by reward-prioritized sampling to effectively consolidate of high-quality experiences. These two components jointly achieve adaptation flexibility while maintaining training stability.
Building upon this foundation, we develop a progressive reward decomposition scheme in our framework to provide fine-grained learning signals at intermediate generation steps, which enables more precise credit assignment throughout the training procedure.

Our contributions can be summarized as below:
\begin{itemize}
    \item We introduce PAG, a novel framework that reformulates prompt adaptation as a probabilistic inference problem, leveraging GFlowNets to generate both effective and diverse adapted prompts.
    \item We identify and address a previously overlooked challenge in GFlowNets fine-tuning - the progressive loss of neural plasticity leading to mode collapse - through a systematic approach that maintains network expressivity and ensures precise credit assignment.
    \item Extensive experiments show that PAG generates both effective and diverse adapted prompts, exhibits robustness to different reward functions, and enables effective zero-shot transfer to various text-to-image diffusion models. 
\end{itemize}

\section{Preliminaries}
\subsection{Prompt Adaptation}
Let $p_{\theta}$ denote a pre-trained language model that generates improved prompt $\mathbf{y}$ conditioned on the initial prompt $\mathbf{x}$, i.e., $\mathbf{y}\sim p_{\theta}(\cdot\vert\mathbf{x})$. A text-to-image diffusion model $p_{\psi}$ then generates images given text inputs $\mathbf{y}$. Following~\citep{hao2024optimizing}, a reward function $r(\mathbf{x}, \mathbf{y})$ for prompt adaptation is defined as follows:
\begin{align}\label{eq:task_reward}
    &\mathbb{E}_{i_\mathbf{x}\sim p_{\psi}(\cdot\vert\mathbf{x}), i_\mathbf{y}\sim p_{\psi}(\cdot\vert\mathbf{y})}\left[r_{\text{aes}}(i_\mathbf{x}, i_\mathbf{y}) + r_{\text{rel}}(\mathbf{x}, i_\mathbf{y})\right], 
\end{align}
where $r_{\text{aes}}(i_{\mathbf{x}}, i_{\mathbf{y}})=g_{\text{aes}}(i_\mathbf{y}) - g_{\text{aes}}(i_\mathbf{x})$ measures the improvement in aesthetic quality of images using the LAION aesthetic predictor \citep{laion}, $g_{\text{aes}}$. The term $r_{\text{rel}}(\mathbf{x}, i_{\mathbf{y}})=\min(20.0\times g_{\text{CLIP}}(\mathbf{x}, i_{\mathbf{y}}) - 5.6, 0.0)$ quantifies the relevance between generate images from $\mathbf{y}$ and the initial prompt $\mathbf{x}$ using the CLIP similarity function, $g_{\text{CLIP}}$ \citep{radford2021learning}. Our objective is fine-tuning language model parameters $\theta$ to optimize the target reward function:
\begin{align}
\label{eq:objective}
\mathbb{E}_{\mathbf{x}\sim\mathcal{D}}\left[\mathbb{E}_{\mathbf{y}\sim p_{\theta}(\cdot\vert\mathbf{x})}\left[r(\mathbf{x}, \mathbf{y})\right]-\beta\cdot D_{\text{KL}}(p_{\theta}(\cdot\vert\mathbf{x})\Vert p_{\text{ref}}(\cdot\vert\mathbf{x}))\right],
\end{align}
where the $D_{\text{KL}}$ term enforces the generated prompts to be close to natural language that human can understand.

\subsection{Generative Flow Networks (GFlowNets)} \label{sec:bg}
GFlowNets are a family of probabilistic methods that sample compositional objects proportionally to an unnormalized distribution defined by a reward function~\citep{bengio2021flow,bengio2023gflownet}. 
Let $\mathcal{S}$ denote the state space and $\mathcal{A}$ the action space, forming nodes and edges in a directed acyclic graph. We define a unique initial state $s_0$ without incoming edges and set of terminal states $\mathcal{X}\subset\mathcal{S}$ without outgoing edges. A sequence from the initial state $s_0$ to the terminal state $x\in\mathcal{X}$ is called a trajectory, denoted as $\tau=(s_0\rightarrow s_1\rightarrow \cdots\rightarrow s_T=x)$, which is generated sequentially.
% 
Let $R:\mathcal{X}\rightarrow\mathbb{R}_{\geq0}$ be a non-negative reward function defined on terminal states. GFlowNets aim to train a stochastic policy $P_{F}$ that generates samples proportional to the reward, i.e., $P_{F}^{T}(x)\propto R(x)$, where $P_{F}^{T}(x)=\sum_{\tau_{\rightarrow x}}P_{F}(\tau)$ is the marginal likelihood of sampling trajectories that result in $x$ ($\tau_{\rightarrow x}$) from the forward policy:
\begin{align}
    P_F^{T}(x)=\sum_{\tau_{\rightarrow x}}\prod_{t=1}^{T}P_{F}(s_t\vert s_{t-1})\propto R(x).
\end{align}
In practice, GFlowNets can be trained by parameterizing the forward policy with a neural network, $P_{F}(s_t\vert s_{t-1};\theta)$, using various training objectives.

\vspace{5pt}
\noindent \textbf{Trajectory Balance (TB, \citep{malkin2022trajectory})} TB introduces an additional backward policy $P_{B}(s_{t-1}\vert s_t;\theta)$ that models the distribution of parents given a child state and a total flow $Z_{\theta}$ to approximate the partition function. Given a trajectory $\tau=(s_0\rightarrow\cdots\rightarrow s_T=x)$, TB aims to minimize the loss in Eq.~(\ref{eq:tb_loss}). If the loss becomes zero for all possible trajectories, it implies that $P_{F}^{T}(x)\propto R(x)$.
\begin{align}
    \mathcal{L}(\tau;\theta)=\left(\log\frac{Z_{\theta}\prod_{t=1}^{T}P_{F}(s_{t}\vert s_{t-1};\theta)}{R(x)\prod_{t=1}^{T}P_{B}(s_{t-1}\vert s_{t};\theta)}\right)^2.
    \label{eq:tb_loss}
\end{align}
\noindent \textbf{Detailed Balance (DB, \citep{bengio2023gflownet})} DB considers flow matching at the edge level instead of the trajectory level and introduces state flow function $F_{\theta}:\mathcal{S}\rightarrow\mathbb{R}_{\geq0}$, which approximates the total flow through state $s$. 
Given an intermediate transition $(s_{t-1}\rightarrow s_t)$, DB aims to minimize the loss in Eq.~(\ref{eq:db_loss}), with $F_{\theta}(s_t)$ replaced by the terminal reward $R(x)$ at terminal states for $t=T$ (i.e., $F_{\theta}(x)=R(x)$).
\begin{align}
    &\mathcal{L}(s_{t-1}, s_{t};\theta)=\left(\log\frac{F_{\theta}(s_{t-1})P_{F}(s_{t}\vert s_{t-1};\theta)}{F_{\theta}(s_{t})P_{B}(s_{t-1}\vert s_{t};\theta)}\right)^2.
    \label{eq:db_loss}
\end{align}
It is often challenging to train GFlowNets due to delayed credit assignment from terminal-only reward signals \citep{pan2023better, jang2024learning}. The forward-looking (FL) technique addresses this by extending rewards to all states and optimizing $\tilde{F_{\theta}}(s)$ in the sense that $F_{\theta}(s)=\tilde{F_{\theta}}(s)R(s)$ for all states, which is a reparameterization of flows in DB.
The resulting FL-DB objective is to minimize the following loss:
\begin{align}
\label{eq:fl-db-objective}
    &\mathcal{L}(s_{t-1}, s_{t};\theta)=\left(\log\frac{\tilde{F_{\theta}}(s_{t-1})P_{F}(s_{t}\vert s_{t-1};\theta)R(s_{t-1})}{\tilde{F_{\theta}}(s_{t})P_{B}(s_{t-1}\vert s_{t};\theta)R(s_t)}\right)^2.
\end{align}

\subsection{Dormant Neuron Phenomenon}
Recent studies have revealed that scaling deep RL networks faces challenges due to parameter under-utilization~\citep{kumarimplicit, lyleunderstanding, sokar2023dormant}. Following \citet{sokar2023dormant}, we quantify neural plasticity by tracking dormant neurons during the training progresses defined as below, where a neuron $i$ in layer $\ell$ is {${\tau}$-dormant} if its activation score is $s^{\ell}_{i}\leq\tau$. 

\vspace{5pt}
\noindent \textbf{Definition 2.3.1.} Given input dataset $\mathcal{D}$, let $h_i(\mathbf{x})$ be the activation of neuron $i$ in layer $\ell$ under input $\mathbf{x}\in\mathcal{D}$, then its activation score is defined as follows:
\begin{align}\label{eq:dormant}
    s^{\ell}_{i}=\frac{\mathbb{E}_{\mathbf{x}\in\mathcal{D}}\vert h^{\ell}_{i}(\mathbf{x})\vert}{\frac{1}{H^{\ell}}\sum_{k\in h}\mathbb{E}_{\mathbf{x}\in\mathcal{D}}\vert h^{\ell}_{k}(\mathbf{x})\vert}.
\end{align}
where $H^l$ is the number of hidden units in the $l$th layer.

\section{Prompt Adaptation with GFlowNets}
In this section, we present \textbf{P}rompt \textbf{A}daptation with \textbf{G}FlowNets (\textbf{PAG)}, a novel framework that reformulates prompt adaptation as probabilistic inference through GFlowNets-based language model (LM) fine-tuning. 
We begin by outlining how we can fine-tune LMs with GFlowNets to satisfy our objective. Next, we investigate the critical challenge of mode collapse in this framework. To systematically address this challenge, we present our key technical advances that enhance GFlowNets for prompt adaptation. \Cref{fig:overview} illustrates the overview of our method.

\subsection{Problem Formulation}\label{sec:mode_collapse}
Unlike previous RL-based approaches like Prompist~\cite{hao2024optimizing}, which maximize the reward proxy and often converge to deterministic behaviors, we formulate prompt adaption as a probabilistic inference problem~\citep{zhao2024probabilistic} based on GFlowNets (as introduced in Section~\ref{sec:bg}). This formulation enables us to learn a policy that samples high-quality prompts while preserving the coverage of reference policy. Specifically, the optimal policy for Eq. \eqref{eq:objective} can be analytically derived as follows:
\begin{align}\label{eq:reward}
p_{\theta}^{*}(\mathbf{y}\vert\mathbf{x})\propto p_{\text{ref}}(\mathbf{y}\vert\mathbf{x})\cdot\exp\left(\frac{1}{\beta}r(\mathbf{x}, \mathbf{y})\right)=:R(\mathbf{x}, \mathbf{y}).
\end{align}
This formulation reduces prompt adaptation to learning how to sample from the unnormalized density function defined by rewards $R$, aiming to match the underlying reward distribution rather than solely maximizing the reward. A GFlowNet with such reward specification will learn a stochastic policy that generates prompts proportional to their rewards, thereby ensuring both effectiveness and diversity. Please refer to \cref{app:diff_btw_rl_and_gfn} for more discussion between RL and GFlowNets.

\begin{figure*}[t]
    \centering
    \includegraphics[width=0.9\textwidth]{figures/overview_new.png}
    \caption{The high-level illustration of PAG. Given an initial prompt, LM generates adapted prompts by PAG. Then, we generate images from prompts and get a reward. Using observations, we fine-tune LM as a GFlowNet policy to generate prompts proportional to reward.}
    \label{fig:overview}
    \vspace{-12pt}
\end{figure*}
\begin{figure}[t]
    \centering
    \includegraphics[width=\linewidth]{figures/mode_collapse_new.png}
    \caption{Mode collapse issue in prompt adaptation with a naive application of GFlowNets. The proportion of dormant neurons steadily increases (left), while the diversity of generated prompts significantly decreases over training iterations (right).}
    \label{fig:summary_mode_collapse}
    \vspace{-12pt}
\end{figure}

\vspace{5pt}
\noindent\textbf{Mode Collapse Issue in GFlowNets Fine-tuning}
While GFlowNets are designed to sample from the reward distribution for discovering high-quality and diverse candidates, our empirical analysis reveals a significant challenge in prompt adaptation, specifically when fine-tuning language models with GFlowNets after the supervised fine-tuning stage: the policy consistently generates similar prompts despite the vast token space as training progresses.
As shown in \Cref{fig:summary_mode_collapse}, directly applying GFlowNets for fine-tuning language models leads to a substantial reduction in the diversity of prompts. This outcome contradicts the original objective of GFlowNets, suggesting underlying limitations in the training dynamics.

To systematically diagnose the mode collapse issue, we investigate the learning behavior and neural activation patterns in the flow model during fine-tuning. By tracking the dormant neuron ratio as defined in \cref{eq:dormant}, we uncover a key finding largely overlooked in previous research -- the proportion of dormant neurons consistently increases throughout training, as illustrated in the left part in Figure~\ref{fig:summary_mode_collapse}. 
This progressive loss of neural plasticity largely constrains the capacity of the model to learn from and adapt to diverse regions in the prompt space. As more neurons become inactive, the model becomes inflexible in exploring the prompt space, leading to the generation of highly similar prompts.

The mode collapse is further exacerbated by inherent learning challenges in the sequential prompt generation process, where GFlowNets are primarily guided by terminal rewards that are only available after completing entire adapted prompts. It leads to inefficient credit assignment as the model struggles to attribute these terminal rewards into individual tokens in the generation process. Without clear feedback on which intermediate choices contribute to successful outcomes, the model tends to conservatively exploit discovered high-reward patterns rather than exploring diverse alternatives, as evidenced by the increasingly similar prompt patterns shown in Figure~\ref{fig:summary_mode_collapse}. 
These two challenges create a self-reinforcing cycle: the loss of plasticity limits the ability of GFlowNets to learn from diverse samples, while inefficient credit assignment hinders effective exploration of diverse alternatives, ultimately leading to the generation of increasingly similar prompt patterns and exacerbates the mode collapse.

These insights motivate our development of novel training mechanisms tailored for prompt adaptation with GFlowNets, which we detail in the following sections. 

\subsection{Proposed Method}
Based on our observations shown in \Cref{fig:summary_mode_collapse}, we recognize that the mode collapse problem emerges when GFlowNets struggles to maintain expressive capacity while learning from solely on terminal reward in the sequential process.

To systematically address these challenges, we propose a comprehensive mechanism that encourages GFlowNets to retain expressivity targeting three complementary aspects: i) flow reactivation strategy that maintains the plasticity of neural network for exploring diverse prompt patterns, ii) reward-prioritized sampling that allows the model to focus on high-reward experiences, and iii) decomposed reward structure that provides fine-grained learning signals throughout the generation process with advanced training guidance, 
which helps the model explore diverse prompts and resist the tendency toward mode collapse. 
The overall training framework is summarized in Algorithm~\ref{alg:main}.

\vspace{5pt}
\noindent\textbf{Flow Reactivation}
It has been proven that periodic network reset is effective in RL for maintaining neural plasticity~\citep{nikishin2022primacy, zhangconfronting, dsample, liu2024neuroplastic}. Drawing inspiration from this approach, we introduce a targeted flow reactivation mechanism that reinitializes only the last layer of the flow function periodically every $M$ steps.
Note that we do not reset the parameters of the forward policy, which directly interacts with the environment for sampling prompts, as it could cause drastic changes that destabilize training.

\vspace{5pt}
\noindent\textbf{Reward-Prioritized Sampling}
However, relying solely on reset strategies~\citep{nikishin2022primacy} encounter significant challenges in prompt adaptation due to the vast token space, which requires the model to undergo extensive re-exploration to re-discover previously identified high-reward regions, and can therefore impact training efficiency.

To maintain the expressivity of flow function while preventing a significant drop in performance that leads to instability, we leverage reward-prioritized sampling during off-policy training of GFlowNets \citep{shen2023towards, kim2024adaptive}.
A key insight of our approach lies in maintaining consistent access to high-quality prompts through prioritized sampling, which serves as a warm-up for accelerating high-quality knowledge recovery. 
Specifically, we sample a batch of prompts from the replay buffer with probabilities proportional to their rewards as follows:
\begin{align}\label{eq:prt}
    (\mathbf{x, y})\sim P_{\mathcal{B}}(\mathbf{x}, \mathbf{y})=\frac{\exp\left(R(\mathbf{x}, \mathbf{y})\right)}{\sum_{(\mathbf{x}, \mathbf{y})\in\mathcal{B}}\exp\left(R(\mathbf{x}, \mathbf{y})\right)}.
\end{align}
This reward-prioritized sampling naturally complements the flow reactivation process. While reset maintains the expressivity of the model, the continuous exposure to high-reward prompts ensures efficient knowledge retention. Through this principled approach, we can prevent the GFlowNet policy from forgetting the previously discovered high-scoring regions and can quickly regain its ability to navigate towards promising regions in the token space.

\vspace{5pt}
\noindent\textbf{Reward Decomposition}
Beyond maintaining model plasticity and high-quality data retention, a critical challenge in the mode collapse problem in prompt adaption lies in the inefficient credit assignment problem~\citep{pan2023better, jang2024learning} as discussed in Section~\ref{sec:mode_collapse}.
To address this issue, we propose a progressive reward decomposition scheme with advanced learning objectives. By carefully analyzing our reward function, we find that the likelihood term $p_{\text{ref}}(\mathbf{y}\vert\mathbf{x})$ in the reward function naturally admits a step-wise decomposition.
In other words, we can precisely extend the domain of our reward function from the set of terminal states to all possible states:
\begin{align}
    R(\mathbf{x}, y_{0:t})=
    \begin{dcases}
        p_{\text{ref}}(y_{0:t}\vert \mathbf{x}) & \text{if }t\neq T \\
        p_{\text{ref}}(\mathbf{y}\vert \mathbf{x})\exp\left(\frac{1}{\beta}r(\mathbf{x}, \mathbf{y})\right) & \text{otherwise}
    \end{dcases}
\end{align}
This enables us to extend the FL-DB objective in~\cref{eq:fl-db-objective}  by incorporating local credit signals at each step by minimizing the loss $\mathcal{L}(\mathbf{x},\mathbf{y};\theta)=\sum_{t=0}^{T-1}\mathcal{L}(\mathbf{x},\mathbf{y}_{0:t+1};\theta)$, where 
\begin{align}\label{eq:fl_db}
    &\mathcal{L}(\mathbf{x},\mathbf{y}_{0:t+1};\theta)\\
    &=\big(\log\tilde{F_{\theta}}(y_t\vert\mathbf{x}, y_{0:t-1})+\log P_{F}(y_{t+1}\vert\mathbf{x}, y_{0:t};\theta)\nonumber \\
    &+\log R(\mathbf{x}, y_{0:t}) - \log\tilde{F_{\theta}}(y_{t+1}\vert\mathbf{x}, y_{0:t})-\log R(\mathbf{x}, y_{0:t+1})\big)^2,\nonumber
\end{align}
with $F_{\theta}(y_{t+1}\vert\mathbf{x}, y_{0:t})=\tilde{F}_{\theta}(y_{t+1}\vert\mathbf{x}, y_{0:t})R(\mathbf{x}, y_{0:t})$. This enables the model to effectively assess token-level decisions, leading to an increase of diversity and mitigating mode collapse, as demonstrated in Section~\ref{sec:main_res}.

% This must be in the first 5 lines to tell arXiv to use pdfLaTeX, which is strongly recommended.
\pdfoutput=1
% In particular, the hyperref package requires pdfLaTeX in order to break URLs across lines.

\documentclass[11pt]{article}

% Change "review" to "final" to generate the final (sometimes called camera-ready) version.
% Change to "preprint" to generate a non-anonymous version with page numbers.
\usepackage{acl}

% Standard package includes
\usepackage{times}
\usepackage{latexsym}

% Draw tables
\usepackage{booktabs}
\usepackage{multirow}
\usepackage{xcolor}
\usepackage{colortbl}
\usepackage{array} 
\usepackage{amsmath}

\newcolumntype{C}{>{\centering\arraybackslash}p{0.07\textwidth}}
% For proper rendering and hyphenation of words containing Latin characters (including in bib files)
\usepackage[T1]{fontenc}
% For Vietnamese characters
% \usepackage[T5]{fontenc}
% See https://www.latex-project.org/help/documentation/encguide.pdf for other character sets
% This assumes your files are encoded as UTF8
\usepackage[utf8]{inputenc}

% This is not strictly necessary, and may be commented out,
% but it will improve the layout of the manuscript,
% and will typically save some space.
\usepackage{microtype}
\DeclareMathOperator*{\argmax}{arg\,max}
% This is also not strictly necessary, and may be commented out.
% However, it will improve the aesthetics of text in
% the typewriter font.
\usepackage{inconsolata}

%Including images in your LaTeX document requires adding
%additional package(s)
\usepackage{graphicx}
% If the title and author information does not fit in the area allocated, uncomment the following
%
%\setlength\titlebox{<dim>}
%
% and set <dim> to something 5cm or larger.

\title{Wi-Chat: Large Language Model Powered Wi-Fi Sensing}

% Author information can be set in various styles:
% For several authors from the same institution:
% \author{Author 1 \and ... \and Author n \\
%         Address line \\ ... \\ Address line}
% if the names do not fit well on one line use
%         Author 1 \\ {\bf Author 2} \\ ... \\ {\bf Author n} \\
% For authors from different institutions:
% \author{Author 1 \\ Address line \\  ... \\ Address line
%         \And  ... \And
%         Author n \\ Address line \\ ... \\ Address line}
% To start a separate ``row'' of authors use \AND, as in
% \author{Author 1 \\ Address line \\  ... \\ Address line
%         \AND
%         Author 2 \\ Address line \\ ... \\ Address line \And
%         Author 3 \\ Address line \\ ... \\ Address line}

% \author{First Author \\
%   Affiliation / Address line 1 \\
%   Affiliation / Address line 2 \\
%   Affiliation / Address line 3 \\
%   \texttt{email@domain} \\\And
%   Second Author \\
%   Affiliation / Address line 1 \\
%   Affiliation / Address line 2 \\
%   Affiliation / Address line 3 \\
%   \texttt{email@domain} \\}
% \author{Haohan Yuan \qquad Haopeng Zhang\thanks{corresponding author} \\ 
%   ALOHA Lab, University of Hawaii at Manoa \\
%   % Affiliation / Address line 2 \\
%   % Affiliation / Address line 3 \\
%   \texttt{\{haohany,haopengz\}@hawaii.edu}}
  
\author{
{Haopeng Zhang$\dag$\thanks{These authors contributed equally to this work.}, Yili Ren$\ddagger$\footnotemark[1], Haohan Yuan$\dag$, Jingzhe Zhang$\ddagger$, Yitong Shen$\ddagger$} \\
ALOHA Lab, University of Hawaii at Manoa$\dag$, University of South Florida$\ddagger$ \\
\{haopengz, haohany\}@hawaii.edu\\
\{yiliren, jingzhe, shen202\}@usf.edu\\}



  
%\author{
%  \textbf{First Author\textsuperscript{1}},
%  \textbf{Second Author\textsuperscript{1,2}},
%  \textbf{Third T. Author\textsuperscript{1}},
%  \textbf{Fourth Author\textsuperscript{1}},
%\\
%  \textbf{Fifth Author\textsuperscript{1,2}},
%  \textbf{Sixth Author\textsuperscript{1}},
%  \textbf{Seventh Author\textsuperscript{1}},
%  \textbf{Eighth Author \textsuperscript{1,2,3,4}},
%\\
%  \textbf{Ninth Author\textsuperscript{1}},
%  \textbf{Tenth Author\textsuperscript{1}},
%  \textbf{Eleventh E. Author\textsuperscript{1,2,3,4,5}},
%  \textbf{Twelfth Author\textsuperscript{1}},
%\\
%  \textbf{Thirteenth Author\textsuperscript{3}},
%  \textbf{Fourteenth F. Author\textsuperscript{2,4}},
%  \textbf{Fifteenth Author\textsuperscript{1}},
%  \textbf{Sixteenth Author\textsuperscript{1}},
%\\
%  \textbf{Seventeenth S. Author\textsuperscript{4,5}},
%  \textbf{Eighteenth Author\textsuperscript{3,4}},
%  \textbf{Nineteenth N. Author\textsuperscript{2,5}},
%  \textbf{Twentieth Author\textsuperscript{1}}
%\\
%\\
%  \textsuperscript{1}Affiliation 1,
%  \textsuperscript{2}Affiliation 2,
%  \textsuperscript{3}Affiliation 3,
%  \textsuperscript{4}Affiliation 4,
%  \textsuperscript{5}Affiliation 5
%\\
%  \small{
%    \textbf{Correspondence:} \href{mailto:email@domain}{email@domain}
%  }
%}

\begin{document}
\maketitle
\begin{abstract}
Recent advancements in Large Language Models (LLMs) have demonstrated remarkable capabilities across diverse tasks. However, their potential to integrate physical model knowledge for real-world signal interpretation remains largely unexplored. In this work, we introduce Wi-Chat, the first LLM-powered Wi-Fi-based human activity recognition system. We demonstrate that LLMs can process raw Wi-Fi signals and infer human activities by incorporating Wi-Fi sensing principles into prompts. Our approach leverages physical model insights to guide LLMs in interpreting Channel State Information (CSI) data without traditional signal processing techniques. Through experiments on real-world Wi-Fi datasets, we show that LLMs exhibit strong reasoning capabilities, achieving zero-shot activity recognition. These findings highlight a new paradigm for Wi-Fi sensing, expanding LLM applications beyond conventional language tasks and enhancing the accessibility of wireless sensing for real-world deployments.
\end{abstract}

\section{Introduction}

In today’s rapidly evolving digital landscape, the transformative power of web technologies has redefined not only how services are delivered but also how complex tasks are approached. Web-based systems have become increasingly prevalent in risk control across various domains. This widespread adoption is due their accessibility, scalability, and ability to remotely connect various types of users. For example, these systems are used for process safety management in industry~\cite{kannan2016web}, safety risk early warning in urban construction~\cite{ding2013development}, and safe monitoring of infrastructural systems~\cite{repetto2018web}. Within these web-based risk management systems, the source search problem presents a huge challenge. Source search refers to the task of identifying the origin of a risky event, such as a gas leak and the emission point of toxic substances. This source search capability is crucial for effective risk management and decision-making.

Traditional approaches to implementing source search capabilities into the web systems often rely on solely algorithmic solutions~\cite{ristic2016study}. These methods, while relatively straightforward to implement, often struggle to achieve acceptable performances due to algorithmic local optima and complex unknown environments~\cite{zhao2020searching}. More recently, web crowdsourcing has emerged as a promising alternative for tackling the source search problem by incorporating human efforts in these web systems on-the-fly~\cite{zhao2024user}. This approach outsources the task of addressing issues encountered during the source search process to human workers, leveraging their capabilities to enhance system performance.

These solutions often employ a human-AI collaborative way~\cite{zhao2023leveraging} where algorithms handle exploration-exploitation and report the encountered problems while human workers resolve complex decision-making bottlenecks to help the algorithms getting rid of local deadlocks~\cite{zhao2022crowd}. Although effective, this paradigm suffers from two inherent limitations: increased operational costs from continuous human intervention, and slow response times of human workers due to sequential decision-making. These challenges motivate our investigation into developing autonomous systems that preserve human-like reasoning capabilities while reducing dependency on massive crowdsourced labor.

Furthermore, recent advancements in large language models (LLMs)~\cite{chang2024survey} and multi-modal LLMs (MLLMs)~\cite{huang2023chatgpt} have unveiled promising avenues for addressing these challenges. One clear opportunity involves the seamless integration of visual understanding and linguistic reasoning for robust decision-making in search tasks. However, whether large models-assisted source search is really effective and efficient for improving the current source search algorithms~\cite{ji2022source} remains unknown. \textit{To address the research gap, we are particularly interested in answering the following two research questions in this work:}

\textbf{\textit{RQ1: }}How can source search capabilities be integrated into web-based systems to support decision-making in time-sensitive risk management scenarios? 
% \sq{I mention ``time-sensitive'' here because I feel like we shall say something about the response time -- LLM has to be faster than humans}

\textbf{\textit{RQ2: }}How can MLLMs and LLMs enhance the effectiveness and efficiency of existing source search algorithms? 

% \textit{\textbf{RQ2:}} To what extent does the performance of large models-assisted search align with or approach the effectiveness of human-AI collaborative search? 

To answer the research questions, we propose a novel framework called Auto-\
S$^2$earch (\textbf{Auto}nomous \textbf{S}ource \textbf{Search}) and implement a prototype system that leverages advanced web technologies to simulate real-world conditions for zero-shot source search. Unlike traditional methods that rely on pre-defined heuristics or extensive human intervention, AutoS$^2$earch employs a carefully designed prompt that encapsulates human rationales, thereby guiding the MLLM to generate coherent and accurate scene descriptions from visual inputs about four directional choices. Based on these language-based descriptions, the LLM is enabled to determine the optimal directional choice through chain-of-thought (CoT) reasoning. Comprehensive empirical validation demonstrates that AutoS$^2$-\ 
earch achieves a success rate of 95–98\%, closely approaching the performance of human-AI collaborative search across 20 benchmark scenarios~\cite{zhao2023leveraging}. 

Our work indicates that the role of humans in future web crowdsourcing tasks may evolve from executors to validators or supervisors. Furthermore, incorporating explanations of LLM decisions into web-based system interfaces has the potential to help humans enhance task performance in risk control.






\section{Related Work}
\label{sec:relatedworks}

% \begin{table*}[t]
% \centering 
% \renewcommand\arraystretch{0.98}
% \fontsize{8}{10}\selectfont \setlength{\tabcolsep}{0.4em}
% \begin{tabular}{@{}lc|cc|cc|cc@{}}
% \toprule
% \textbf{Methods}           & \begin{tabular}[c]{@{}c@{}}\textbf{Training}\\ \textbf{Paradigm}\end{tabular} & \begin{tabular}[c]{@{}c@{}}\textbf{$\#$ PT Data}\\ \textbf{(Tokens)}\end{tabular} & \begin{tabular}[c]{@{}c@{}}\textbf{$\#$ IFT Data}\\ \textbf{(Samples)}\end{tabular} & \textbf{Code}  & \begin{tabular}[c]{@{}c@{}}\textbf{Natural}\\ \textbf{Language}\end{tabular} & \begin{tabular}[c]{@{}c@{}}\textbf{Action}\\ \textbf{Trajectories}\end{tabular} & \begin{tabular}[c]{@{}c@{}}\textbf{API}\\ \textbf{Documentation}\end{tabular}\\ \midrule 
% NexusRaven~\citep{srinivasan2023nexusraven} & IFT & - & - & \textcolor{green}{\CheckmarkBold} & \textcolor{green}{\CheckmarkBold} &\textcolor{red}{\XSolidBrush}&\textcolor{red}{\XSolidBrush}\\
% AgentInstruct~\citep{zeng2023agenttuning} & IFT & - & 2k & \textcolor{green}{\CheckmarkBold} & \textcolor{green}{\CheckmarkBold} &\textcolor{red}{\XSolidBrush}&\textcolor{red}{\XSolidBrush} \\
% AgentEvol~\citep{xi2024agentgym} & IFT & - & 14.5k & \textcolor{green}{\CheckmarkBold} & \textcolor{green}{\CheckmarkBold} &\textcolor{green}{\CheckmarkBold}&\textcolor{red}{\XSolidBrush} \\
% Gorilla~\citep{patil2023gorilla}& IFT & - & 16k & \textcolor{green}{\CheckmarkBold} & \textcolor{green}{\CheckmarkBold} &\textcolor{red}{\XSolidBrush}&\textcolor{green}{\CheckmarkBold}\\
% OpenFunctions-v2~\citep{patil2023gorilla} & IFT & - & 65k & \textcolor{green}{\CheckmarkBold} & \textcolor{green}{\CheckmarkBold} &\textcolor{red}{\XSolidBrush}&\textcolor{green}{\CheckmarkBold}\\
% LAM~\citep{zhang2024agentohana} & IFT & - & 42.6k & \textcolor{green}{\CheckmarkBold} & \textcolor{green}{\CheckmarkBold} &\textcolor{green}{\CheckmarkBold}&\textcolor{red}{\XSolidBrush} \\
% xLAM~\citep{liu2024apigen} & IFT & - & 60k & \textcolor{green}{\CheckmarkBold} & \textcolor{green}{\CheckmarkBold} &\textcolor{green}{\CheckmarkBold}&\textcolor{red}{\XSolidBrush} \\\midrule
% LEMUR~\citep{xu2024lemur} & PT & 90B & 300k & \textcolor{green}{\CheckmarkBold} & \textcolor{green}{\CheckmarkBold} &\textcolor{green}{\CheckmarkBold}&\textcolor{red}{\XSolidBrush}\\
% \rowcolor{teal!12} \method & PT & 103B & 95k & \textcolor{green}{\CheckmarkBold} & \textcolor{green}{\CheckmarkBold} & \textcolor{green}{\CheckmarkBold} & \textcolor{green}{\CheckmarkBold} \\
% \bottomrule
% \end{tabular}
% \caption{Summary of existing tuning- and pretraining-based LLM agents with their training sample sizes. "PT" and "IFT" denote "Pre-Training" and "Instruction Fine-Tuning", respectively. }
% \label{tab:related}
% \end{table*}

\begin{table*}[ht]
\begin{threeparttable}
\centering 
\renewcommand\arraystretch{0.98}
\fontsize{7}{9}\selectfont \setlength{\tabcolsep}{0.2em}
\begin{tabular}{@{}l|c|c|ccc|cc|cc|cccc@{}}
\toprule
\textbf{Methods} & \textbf{Datasets}           & \begin{tabular}[c]{@{}c@{}}\textbf{Training}\\ \textbf{Paradigm}\end{tabular} & \begin{tabular}[c]{@{}c@{}}\textbf{\# PT Data}\\ \textbf{(Tokens)}\end{tabular} & \begin{tabular}[c]{@{}c@{}}\textbf{\# IFT Data}\\ \textbf{(Samples)}\end{tabular} & \textbf{\# APIs} & \textbf{Code}  & \begin{tabular}[c]{@{}c@{}}\textbf{Nat.}\\ \textbf{Lang.}\end{tabular} & \begin{tabular}[c]{@{}c@{}}\textbf{Action}\\ \textbf{Traj.}\end{tabular} & \begin{tabular}[c]{@{}c@{}}\textbf{API}\\ \textbf{Doc.}\end{tabular} & \begin{tabular}[c]{@{}c@{}}\textbf{Func.}\\ \textbf{Call}\end{tabular} & \begin{tabular}[c]{@{}c@{}}\textbf{Multi.}\\ \textbf{Step}\end{tabular}  & \begin{tabular}[c]{@{}c@{}}\textbf{Plan}\\ \textbf{Refine}\end{tabular}  & \begin{tabular}[c]{@{}c@{}}\textbf{Multi.}\\ \textbf{Turn}\end{tabular}\\ \midrule 
\multicolumn{13}{l}{\emph{Instruction Finetuning-based LLM Agents for Intrinsic Reasoning}}  \\ \midrule
FireAct~\cite{chen2023fireact} & FireAct & IFT & - & 2.1K & 10 & \textcolor{red}{\XSolidBrush} &\textcolor{green}{\CheckmarkBold} &\textcolor{green}{\CheckmarkBold}  & \textcolor{red}{\XSolidBrush} &\textcolor{green}{\CheckmarkBold} & \textcolor{red}{\XSolidBrush} &\textcolor{green}{\CheckmarkBold} & \textcolor{red}{\XSolidBrush} \\
ToolAlpaca~\cite{tang2023toolalpaca} & ToolAlpaca & IFT & - & 4.0K & 400 & \textcolor{red}{\XSolidBrush} &\textcolor{green}{\CheckmarkBold} &\textcolor{green}{\CheckmarkBold} & \textcolor{red}{\XSolidBrush} &\textcolor{green}{\CheckmarkBold} & \textcolor{red}{\XSolidBrush}  &\textcolor{green}{\CheckmarkBold} & \textcolor{red}{\XSolidBrush}  \\
ToolLLaMA~\cite{qin2023toolllm} & ToolBench & IFT & - & 12.7K & 16,464 & \textcolor{red}{\XSolidBrush} &\textcolor{green}{\CheckmarkBold} &\textcolor{green}{\CheckmarkBold} &\textcolor{red}{\XSolidBrush} &\textcolor{green}{\CheckmarkBold}&\textcolor{green}{\CheckmarkBold}&\textcolor{green}{\CheckmarkBold} &\textcolor{green}{\CheckmarkBold}\\
AgentEvol~\citep{xi2024agentgym} & AgentTraj-L & IFT & - & 14.5K & 24 &\textcolor{red}{\XSolidBrush} & \textcolor{green}{\CheckmarkBold} &\textcolor{green}{\CheckmarkBold}&\textcolor{red}{\XSolidBrush} &\textcolor{green}{\CheckmarkBold}&\textcolor{red}{\XSolidBrush} &\textcolor{red}{\XSolidBrush} &\textcolor{green}{\CheckmarkBold}\\
Lumos~\cite{yin2024agent} & Lumos & IFT  & - & 20.0K & 16 &\textcolor{red}{\XSolidBrush} & \textcolor{green}{\CheckmarkBold} & \textcolor{green}{\CheckmarkBold} &\textcolor{red}{\XSolidBrush} & \textcolor{green}{\CheckmarkBold} & \textcolor{green}{\CheckmarkBold} &\textcolor{red}{\XSolidBrush} & \textcolor{green}{\CheckmarkBold}\\
Agent-FLAN~\cite{chen2024agent} & Agent-FLAN & IFT & - & 24.7K & 20 &\textcolor{red}{\XSolidBrush} & \textcolor{green}{\CheckmarkBold} & \textcolor{green}{\CheckmarkBold} &\textcolor{red}{\XSolidBrush} & \textcolor{green}{\CheckmarkBold}& \textcolor{green}{\CheckmarkBold}&\textcolor{red}{\XSolidBrush} & \textcolor{green}{\CheckmarkBold}\\
AgentTuning~\citep{zeng2023agenttuning} & AgentInstruct & IFT & - & 35.0K & - &\textcolor{red}{\XSolidBrush} & \textcolor{green}{\CheckmarkBold} & \textcolor{green}{\CheckmarkBold} &\textcolor{red}{\XSolidBrush} & \textcolor{green}{\CheckmarkBold} &\textcolor{red}{\XSolidBrush} &\textcolor{red}{\XSolidBrush} & \textcolor{green}{\CheckmarkBold}\\\midrule
\multicolumn{13}{l}{\emph{Instruction Finetuning-based LLM Agents for Function Calling}} \\\midrule
NexusRaven~\citep{srinivasan2023nexusraven} & NexusRaven & IFT & - & - & 116 & \textcolor{green}{\CheckmarkBold} & \textcolor{green}{\CheckmarkBold}  & \textcolor{green}{\CheckmarkBold} &\textcolor{red}{\XSolidBrush} & \textcolor{green}{\CheckmarkBold} &\textcolor{red}{\XSolidBrush} &\textcolor{red}{\XSolidBrush}&\textcolor{red}{\XSolidBrush}\\
Gorilla~\citep{patil2023gorilla} & Gorilla & IFT & - & 16.0K & 1,645 & \textcolor{green}{\CheckmarkBold} &\textcolor{red}{\XSolidBrush} &\textcolor{red}{\XSolidBrush}&\textcolor{green}{\CheckmarkBold} &\textcolor{green}{\CheckmarkBold} &\textcolor{red}{\XSolidBrush} &\textcolor{red}{\XSolidBrush} &\textcolor{red}{\XSolidBrush}\\
OpenFunctions-v2~\citep{patil2023gorilla} & OpenFunctions-v2 & IFT & - & 65.0K & - & \textcolor{green}{\CheckmarkBold} & \textcolor{green}{\CheckmarkBold} &\textcolor{red}{\XSolidBrush} &\textcolor{green}{\CheckmarkBold} &\textcolor{green}{\CheckmarkBold} &\textcolor{red}{\XSolidBrush} &\textcolor{red}{\XSolidBrush} &\textcolor{red}{\XSolidBrush}\\
API Pack~\cite{guo2024api} & API Pack & IFT & - & 1.1M & 11,213 &\textcolor{green}{\CheckmarkBold} &\textcolor{red}{\XSolidBrush} &\textcolor{green}{\CheckmarkBold} &\textcolor{red}{\XSolidBrush} &\textcolor{green}{\CheckmarkBold} &\textcolor{red}{\XSolidBrush}&\textcolor{red}{\XSolidBrush}&\textcolor{red}{\XSolidBrush}\\ 
LAM~\citep{zhang2024agentohana} & AgentOhana & IFT & - & 42.6K & - & \textcolor{green}{\CheckmarkBold} & \textcolor{green}{\CheckmarkBold} &\textcolor{green}{\CheckmarkBold}&\textcolor{red}{\XSolidBrush} &\textcolor{green}{\CheckmarkBold}&\textcolor{red}{\XSolidBrush}&\textcolor{green}{\CheckmarkBold}&\textcolor{green}{\CheckmarkBold}\\
xLAM~\citep{liu2024apigen} & APIGen & IFT & - & 60.0K & 3,673 & \textcolor{green}{\CheckmarkBold} & \textcolor{green}{\CheckmarkBold} &\textcolor{green}{\CheckmarkBold}&\textcolor{red}{\XSolidBrush} &\textcolor{green}{\CheckmarkBold}&\textcolor{red}{\XSolidBrush}&\textcolor{green}{\CheckmarkBold}&\textcolor{green}{\CheckmarkBold}\\\midrule
\multicolumn{13}{l}{\emph{Pretraining-based LLM Agents}}  \\\midrule
% LEMUR~\citep{xu2024lemur} & PT & 90B & 300.0K & - & \textcolor{green}{\CheckmarkBold} & \textcolor{green}{\CheckmarkBold} &\textcolor{green}{\CheckmarkBold}&\textcolor{red}{\XSolidBrush} & \textcolor{red}{\XSolidBrush} &\textcolor{green}{\CheckmarkBold} &\textcolor{red}{\XSolidBrush}&\textcolor{red}{\XSolidBrush}\\
\rowcolor{teal!12} \method & \dataset & PT & 103B & 95.0K  & 76,537  & \textcolor{green}{\CheckmarkBold} & \textcolor{green}{\CheckmarkBold} & \textcolor{green}{\CheckmarkBold} & \textcolor{green}{\CheckmarkBold} & \textcolor{green}{\CheckmarkBold} & \textcolor{green}{\CheckmarkBold} & \textcolor{green}{\CheckmarkBold} & \textcolor{green}{\CheckmarkBold}\\
\bottomrule
\end{tabular}
% \begin{tablenotes}
%     \item $^*$ In addition, the StarCoder-API can offer 4.77M more APIs.
% \end{tablenotes}
\caption{Summary of existing instruction finetuning-based LLM agents for intrinsic reasoning and function calling, along with their training resources and sample sizes. "PT" and "IFT" denote "Pre-Training" and "Instruction Fine-Tuning", respectively.}
\vspace{-2ex}
\label{tab:related}
\end{threeparttable}
\end{table*}

\noindent \textbf{Prompting-based LLM Agents.} Due to the lack of agent-specific pre-training corpus, existing LLM agents rely on either prompt engineering~\cite{hsieh2023tool,lu2024chameleon,yao2022react,wang2023voyager} or instruction fine-tuning~\cite{chen2023fireact,zeng2023agenttuning} to understand human instructions, decompose high-level tasks, generate grounded plans, and execute multi-step actions. 
However, prompting-based methods mainly depend on the capabilities of backbone LLMs (usually commercial LLMs), failing to introduce new knowledge and struggling to generalize to unseen tasks~\cite{sun2024adaplanner,zhuang2023toolchain}. 

\noindent \textbf{Instruction Finetuning-based LLM Agents.} Considering the extensive diversity of APIs and the complexity of multi-tool instructions, tool learning inherently presents greater challenges than natural language tasks, such as text generation~\cite{qin2023toolllm}.
Post-training techniques focus more on instruction following and aligning output with specific formats~\cite{patil2023gorilla,hao2024toolkengpt,qin2023toolllm,schick2024toolformer}, rather than fundamentally improving model knowledge or capabilities. 
Moreover, heavy fine-tuning can hinder generalization or even degrade performance in non-agent use cases, potentially suppressing the original base model capabilities~\cite{ghosh2024a}.

\noindent \textbf{Pretraining-based LLM Agents.} While pre-training serves as an essential alternative, prior works~\cite{nijkamp2023codegen,roziere2023code,xu2024lemur,patil2023gorilla} have primarily focused on improving task-specific capabilities (\eg, code generation) instead of general-domain LLM agents, due to single-source, uni-type, small-scale, and poor-quality pre-training data. 
Existing tool documentation data for agent training either lacks diverse real-world APIs~\cite{patil2023gorilla, tang2023toolalpaca} or is constrained to single-tool or single-round tool execution. 
Furthermore, trajectory data mostly imitate expert behavior or follow function-calling rules with inferior planning and reasoning, failing to fully elicit LLMs' capabilities and handle complex instructions~\cite{qin2023toolllm}. 
Given a wide range of candidate API functions, each comprising various function names and parameters available at every planning step, identifying globally optimal solutions and generalizing across tasks remains highly challenging.



\section{Preliminaries}
\label{Preliminaries}
\begin{figure*}[t]
    \centering
    \includegraphics[width=0.95\linewidth]{fig/HealthGPT_Framework.png}
    \caption{The \ourmethod{} architecture integrates hierarchical visual perception and H-LoRA, employing a task-specific hard router to select visual features and H-LoRA plugins, ultimately generating outputs with an autoregressive manner.}
    \label{fig:architecture}
\end{figure*}
\noindent\textbf{Large Vision-Language Models.} 
The input to a LVLM typically consists of an image $x^{\text{img}}$ and a discrete text sequence $x^{\text{txt}}$. The visual encoder $\mathcal{E}^{\text{img}}$ converts the input image $x^{\text{img}}$ into a sequence of visual tokens $\mathcal{V} = [v_i]_{i=1}^{N_v}$, while the text sequence $x^{\text{txt}}$ is mapped into a sequence of text tokens $\mathcal{T} = [t_i]_{i=1}^{N_t}$ using an embedding function $\mathcal{E}^{\text{txt}}$. The LLM $\mathcal{M_\text{LLM}}(\cdot|\theta)$ models the joint probability of the token sequence $\mathcal{U} = \{\mathcal{V},\mathcal{T}\}$, which is expressed as:
\begin{equation}
    P_\theta(R | \mathcal{U}) = \prod_{i=1}^{N_r} P_\theta(r_i | \{\mathcal{U}, r_{<i}\}),
\end{equation}
where $R = [r_i]_{i=1}^{N_r}$ is the text response sequence. The LVLM iteratively generates the next token $r_i$ based on $r_{<i}$. The optimization objective is to minimize the cross-entropy loss of the response $\mathcal{R}$.
% \begin{equation}
%     \mathcal{L}_{\text{VLM}} = \mathbb{E}_{R|\mathcal{U}}\left[-\log P_\theta(R | \mathcal{U})\right]
% \end{equation}
It is worth noting that most LVLMs adopt a design paradigm based on ViT, alignment adapters, and pre-trained LLMs\cite{liu2023llava,liu2024improved}, enabling quick adaptation to downstream tasks.


\noindent\textbf{VQGAN.}
VQGAN~\cite{esser2021taming} employs latent space compression and indexing mechanisms to effectively learn a complete discrete representation of images. VQGAN first maps the input image $x^{\text{img}}$ to a latent representation $z = \mathcal{E}(x)$ through a encoder $\mathcal{E}$. Then, the latent representation is quantized using a codebook $\mathcal{Z} = \{z_k\}_{k=1}^K$, generating a discrete index sequence $\mathcal{I} = [i_m]_{m=1}^N$, where $i_m \in \mathcal{Z}$ represents the quantized code index:
\begin{equation}
    \mathcal{I} = \text{Quantize}(z|\mathcal{Z}) = \arg\min_{z_k \in \mathcal{Z}} \| z - z_k \|_2.
\end{equation}
In our approach, the discrete index sequence $\mathcal{I}$ serves as a supervisory signal for the generation task, enabling the model to predict the index sequence $\hat{\mathcal{I}}$ from input conditions such as text or other modality signals.  
Finally, the predicted index sequence $\hat{\mathcal{I}}$ is upsampled by the VQGAN decoder $G$, generating the high-quality image $\hat{x}^\text{img} = G(\hat{\mathcal{I}})$.



\noindent\textbf{Low Rank Adaptation.} 
LoRA\cite{hu2021lora} effectively captures the characteristics of downstream tasks by introducing low-rank adapters. The core idea is to decompose the bypass weight matrix $\Delta W\in\mathbb{R}^{d^{\text{in}} \times d^{\text{out}}}$ into two low-rank matrices $ \{A \in \mathbb{R}^{d^{\text{in}} \times r}, B \in \mathbb{R}^{r \times d^{\text{out}}} \}$, where $ r \ll \min\{d^{\text{in}}, d^{\text{out}}\} $, significantly reducing learnable parameters. The output with the LoRA adapter for the input $x$ is then given by:
\begin{equation}
    h = x W_0 + \alpha x \Delta W/r = x W_0 + \alpha xAB/r,
\end{equation}
where matrix $ A $ is initialized with a Gaussian distribution, while the matrix $ B $ is initialized as a zero matrix. The scaling factor $ \alpha/r $ controls the impact of $ \Delta W $ on the model.

\section{HealthGPT}
\label{Method}


\subsection{Unified Autoregressive Generation.}  
% As shown in Figure~\ref{fig:architecture}, 
\ourmethod{} (Figure~\ref{fig:architecture}) utilizes a discrete token representation that covers both text and visual outputs, unifying visual comprehension and generation as an autoregressive task. 
For comprehension, $\mathcal{M}_\text{llm}$ receives the input joint sequence $\mathcal{U}$ and outputs a series of text token $\mathcal{R} = [r_1, r_2, \dots, r_{N_r}]$, where $r_i \in \mathcal{V}_{\text{txt}}$, and $\mathcal{V}_{\text{txt}}$ represents the LLM's vocabulary:
\begin{equation}
    P_\theta(\mathcal{R} \mid \mathcal{U}) = \prod_{i=1}^{N_r} P_\theta(r_i \mid \mathcal{U}, r_{<i}).
\end{equation}
For generation, $\mathcal{M}_\text{llm}$ first receives a special start token $\langle \text{START\_IMG} \rangle$, then generates a series of tokens corresponding to the VQGAN indices $\mathcal{I} = [i_1, i_2, \dots, i_{N_i}]$, where $i_j \in \mathcal{V}_{\text{vq}}$, and $\mathcal{V}_{\text{vq}}$ represents the index range of VQGAN. Upon completion of generation, the LLM outputs an end token $\langle \text{END\_IMG} \rangle$:
\begin{equation}
    P_\theta(\mathcal{I} \mid \mathcal{U}) = \prod_{j=1}^{N_i} P_\theta(i_j \mid \mathcal{U}, i_{<j}).
\end{equation}
Finally, the generated index sequence $\mathcal{I}$ is fed into the decoder $G$, which reconstructs the target image $\hat{x}^{\text{img}} = G(\mathcal{I})$.

\subsection{Hierarchical Visual Perception}  
Given the differences in visual perception between comprehension and generation tasks—where the former focuses on abstract semantics and the latter emphasizes complete semantics—we employ ViT to compress the image into discrete visual tokens at multiple hierarchical levels.
Specifically, the image is converted into a series of features $\{f_1, f_2, \dots, f_L\}$ as it passes through $L$ ViT blocks.

To address the needs of various tasks, the hidden states are divided into two types: (i) \textit{Concrete-grained features} $\mathcal{F}^{\text{Con}} = \{f_1, f_2, \dots, f_k\}, k < L$, derived from the shallower layers of ViT, containing sufficient global features, suitable for generation tasks; 
(ii) \textit{Abstract-grained features} $\mathcal{F}^{\text{Abs}} = \{f_{k+1}, f_{k+2}, \dots, f_L\}$, derived from the deeper layers of ViT, which contain abstract semantic information closer to the text space, suitable for comprehension tasks.

The task type $T$ (comprehension or generation) determines which set of features is selected as the input for the downstream large language model:
\begin{equation}
    \mathcal{F}^{\text{img}}_T =
    \begin{cases}
        \mathcal{F}^{\text{Con}}, & \text{if } T = \text{generation task} \\
        \mathcal{F}^{\text{Abs}}, & \text{if } T = \text{comprehension task}
    \end{cases}
\end{equation}
We integrate the image features $\mathcal{F}^{\text{img}}_T$ and text features $\mathcal{T}$ into a joint sequence through simple concatenation, which is then fed into the LLM $\mathcal{M}_{\text{llm}}$ for autoregressive generation.
% :
% \begin{equation}
%     \mathcal{R} = \mathcal{M}_{\text{llm}}(\mathcal{U}|\theta), \quad \mathcal{U} = [\mathcal{F}^{\text{img}}_T; \mathcal{T}]
% \end{equation}
\subsection{Heterogeneous Knowledge Adaptation}
We devise H-LoRA, which stores heterogeneous knowledge from comprehension and generation tasks in separate modules and dynamically routes to extract task-relevant knowledge from these modules. 
At the task level, for each task type $ T $, we dynamically assign a dedicated H-LoRA submodule $ \theta^T $, which is expressed as:
\begin{equation}
    \mathcal{R} = \mathcal{M}_\text{LLM}(\mathcal{U}|\theta, \theta^T), \quad \theta^T = \{A^T, B^T, \mathcal{R}^T_\text{outer}\}.
\end{equation}
At the feature level for a single task, H-LoRA integrates the idea of Mixture of Experts (MoE)~\cite{masoudnia2014mixture} and designs an efficient matrix merging and routing weight allocation mechanism, thus avoiding the significant computational delay introduced by matrix splitting in existing MoELoRA~\cite{luo2024moelora}. Specifically, we first merge the low-rank matrices (rank = r) of $ k $ LoRA experts into a unified matrix:
\begin{equation}
    \mathbf{A}^{\text{merged}}, \mathbf{B}^{\text{merged}} = \text{Concat}(\{A_i\}_1^k), \text{Concat}(\{B_i\}_1^k),
\end{equation}
where $ \mathbf{A}^{\text{merged}} \in \mathbb{R}^{d^\text{in} \times rk} $ and $ \mathbf{B}^{\text{merged}} \in \mathbb{R}^{rk \times d^\text{out}} $. The $k$-dimension routing layer generates expert weights $ \mathcal{W} \in \mathbb{R}^{\text{token\_num} \times k} $ based on the input hidden state $ x $, and these are expanded to $ \mathbb{R}^{\text{token\_num} \times rk} $ as follows:
\begin{equation}
    \mathcal{W}^\text{expanded} = \alpha k \mathcal{W} / r \otimes \mathbf{1}_r,
\end{equation}
where $ \otimes $ denotes the replication operation.
The overall output of H-LoRA is computed as:
\begin{equation}
    \mathcal{O}^\text{H-LoRA} = (x \mathbf{A}^{\text{merged}} \odot \mathcal{W}^\text{expanded}) \mathbf{B}^{\text{merged}},
\end{equation}
where $ \odot $ represents element-wise multiplication. Finally, the output of H-LoRA is added to the frozen pre-trained weights to produce the final output:
\begin{equation}
    \mathcal{O} = x W_0 + \mathcal{O}^\text{H-LoRA}.
\end{equation}
% In summary, H-LoRA is a task-based dynamic PEFT method that achieves high efficiency in single-task fine-tuning.

\subsection{Training Pipeline}

\begin{figure}[t]
    \centering
    \hspace{-4mm}
    \includegraphics[width=0.94\linewidth]{fig/data.pdf}
    \caption{Data statistics of \texttt{VL-Health}. }
    \label{fig:data}
\end{figure}
\noindent \textbf{1st Stage: Multi-modal Alignment.} 
In the first stage, we design separate visual adapters and H-LoRA submodules for medical unified tasks. For the medical comprehension task, we train abstract-grained visual adapters using high-quality image-text pairs to align visual embeddings with textual embeddings, thereby enabling the model to accurately describe medical visual content. During this process, the pre-trained LLM and its corresponding H-LoRA submodules remain frozen. In contrast, the medical generation task requires training concrete-grained adapters and H-LoRA submodules while keeping the LLM frozen. Meanwhile, we extend the textual vocabulary to include multimodal tokens, enabling the support of additional VQGAN vector quantization indices. The model trains on image-VQ pairs, endowing the pre-trained LLM with the capability for image reconstruction. This design ensures pixel-level consistency of pre- and post-LVLM. The processes establish the initial alignment between the LLM’s outputs and the visual inputs.

\noindent \textbf{2nd Stage: Heterogeneous H-LoRA Plugin Adaptation.}  
The submodules of H-LoRA share the word embedding layer and output head but may encounter issues such as bias and scale inconsistencies during training across different tasks. To ensure that the multiple H-LoRA plugins seamlessly interface with the LLMs and form a unified base, we fine-tune the word embedding layer and output head using a small amount of mixed data to maintain consistency in the model weights. Specifically, during this stage, all H-LoRA submodules for different tasks are kept frozen, with only the word embedding layer and output head being optimized. Through this stage, the model accumulates foundational knowledge for unified tasks by adapting H-LoRA plugins.

\begin{table*}[!t]
\centering
\caption{Comparison of \ourmethod{} with other LVLMs and unified multi-modal models on medical visual comprehension tasks. \textbf{Bold} and \underline{underlined} text indicates the best performance and second-best performance, respectively.}
\resizebox{\textwidth}{!}{
\begin{tabular}{c|lcc|cccccccc|c}
\toprule
\rowcolor[HTML]{E9F3FE} &  &  &  & \multicolumn{2}{c}{\textbf{VQA-RAD \textuparrow}} & \multicolumn{2}{c}{\textbf{SLAKE \textuparrow}} & \multicolumn{2}{c}{\textbf{PathVQA \textuparrow}} &  &  &  \\ 
\cline{5-10}
\rowcolor[HTML]{E9F3FE}\multirow{-2}{*}{\textbf{Type}} & \multirow{-2}{*}{\textbf{Model}} & \multirow{-2}{*}{\textbf{\# Params}} & \multirow{-2}{*}{\makecell{\textbf{Medical} \\ \textbf{LVLM}}} & \textbf{close} & \textbf{all} & \textbf{close} & \textbf{all} & \textbf{close} & \textbf{all} & \multirow{-2}{*}{\makecell{\textbf{MMMU} \\ \textbf{-Med}}\textuparrow} & \multirow{-2}{*}{\textbf{OMVQA}\textuparrow} & \multirow{-2}{*}{\textbf{Avg. \textuparrow}} \\ 
\midrule \midrule
\multirow{9}{*}{\textbf{Comp. Only}} 
& Med-Flamingo & 8.3B & \Large \ding{51} & 58.6 & 43.0 & 47.0 & 25.5 & 61.9 & 31.3 & 28.7 & 34.9 & 41.4 \\
& LLaVA-Med & 7B & \Large \ding{51} & 60.2 & 48.1 & 58.4 & 44.8 & 62.3 & 35.7 & 30.0 & 41.3 & 47.6 \\
& HuatuoGPT-Vision & 7B & \Large \ding{51} & 66.9 & 53.0 & 59.8 & 49.1 & 52.9 & 32.0 & 42.0 & 50.0 & 50.7 \\
& BLIP-2 & 6.7B & \Large \ding{55} & 43.4 & 36.8 & 41.6 & 35.3 & 48.5 & 28.8 & 27.3 & 26.9 & 36.1 \\
& LLaVA-v1.5 & 7B & \Large \ding{55} & 51.8 & 42.8 & 37.1 & 37.7 & 53.5 & 31.4 & 32.7 & 44.7 & 41.5 \\
& InstructBLIP & 7B & \Large \ding{55} & 61.0 & 44.8 & 66.8 & 43.3 & 56.0 & 32.3 & 25.3 & 29.0 & 44.8 \\
& Yi-VL & 6B & \Large \ding{55} & 52.6 & 42.1 & 52.4 & 38.4 & 54.9 & 30.9 & 38.0 & 50.2 & 44.9 \\
& InternVL2 & 8B & \Large \ding{55} & 64.9 & 49.0 & 66.6 & 50.1 & 60.0 & 31.9 & \underline{43.3} & 54.5 & 52.5\\
& Llama-3.2 & 11B & \Large \ding{55} & 68.9 & 45.5 & 72.4 & 52.1 & 62.8 & 33.6 & 39.3 & 63.2 & 54.7 \\
\midrule
\multirow{5}{*}{\textbf{Comp. \& Gen.}} 
& Show-o & 1.3B & \Large \ding{55} & 50.6 & 33.9 & 31.5 & 17.9 & 52.9 & 28.2 & 22.7 & 45.7 & 42.6 \\
& Unified-IO 2 & 7B & \Large \ding{55} & 46.2 & 32.6 & 35.9 & 21.9 & 52.5 & 27.0 & 25.3 & 33.0 & 33.8 \\
& Janus & 1.3B & \Large \ding{55} & 70.9 & 52.8 & 34.7 & 26.9 & 51.9 & 27.9 & 30.0 & 26.8 & 33.5 \\
& \cellcolor[HTML]{DAE0FB}HealthGPT-M3 & \cellcolor[HTML]{DAE0FB}3.8B & \cellcolor[HTML]{DAE0FB}\Large \ding{51} & \cellcolor[HTML]{DAE0FB}\underline{73.7} & \cellcolor[HTML]{DAE0FB}\underline{55.9} & \cellcolor[HTML]{DAE0FB}\underline{74.6} & \cellcolor[HTML]{DAE0FB}\underline{56.4} & \cellcolor[HTML]{DAE0FB}\underline{78.7} & \cellcolor[HTML]{DAE0FB}\underline{39.7} & \cellcolor[HTML]{DAE0FB}\underline{43.3} & \cellcolor[HTML]{DAE0FB}\underline{68.5} & \cellcolor[HTML]{DAE0FB}\underline{61.3} \\
& \cellcolor[HTML]{DAE0FB}HealthGPT-L14 & \cellcolor[HTML]{DAE0FB}14B & \cellcolor[HTML]{DAE0FB}\Large \ding{51} & \cellcolor[HTML]{DAE0FB}\textbf{77.7} & \cellcolor[HTML]{DAE0FB}\textbf{58.3} & \cellcolor[HTML]{DAE0FB}\textbf{76.4} & \cellcolor[HTML]{DAE0FB}\textbf{64.5} & \cellcolor[HTML]{DAE0FB}\textbf{85.9} & \cellcolor[HTML]{DAE0FB}\textbf{44.4} & \cellcolor[HTML]{DAE0FB}\textbf{49.2} & \cellcolor[HTML]{DAE0FB}\textbf{74.4} & \cellcolor[HTML]{DAE0FB}\textbf{66.4} \\
\bottomrule
\end{tabular}
}
\label{tab:results}
\end{table*}
\begin{table*}[ht]
    \centering
    \caption{The experimental results for the four modality conversion tasks.}
    \resizebox{\textwidth}{!}{
    \begin{tabular}{l|ccc|ccc|ccc|ccc}
        \toprule
        \rowcolor[HTML]{E9F3FE} & \multicolumn{3}{c}{\textbf{CT to MRI (Brain)}} & \multicolumn{3}{c}{\textbf{CT to MRI (Pelvis)}} & \multicolumn{3}{c}{\textbf{MRI to CT (Brain)}} & \multicolumn{3}{c}{\textbf{MRI to CT (Pelvis)}} \\
        \cline{2-13}
        \rowcolor[HTML]{E9F3FE}\multirow{-2}{*}{\textbf{Model}}& \textbf{SSIM $\uparrow$} & \textbf{PSNR $\uparrow$} & \textbf{MSE $\downarrow$} & \textbf{SSIM $\uparrow$} & \textbf{PSNR $\uparrow$} & \textbf{MSE $\downarrow$} & \textbf{SSIM $\uparrow$} & \textbf{PSNR $\uparrow$} & \textbf{MSE $\downarrow$} & \textbf{SSIM $\uparrow$} & \textbf{PSNR $\uparrow$} & \textbf{MSE $\downarrow$} \\
        \midrule \midrule
        pix2pix & 71.09 & 32.65 & 36.85 & 59.17 & 31.02 & 51.91 & 78.79 & 33.85 & 28.33 & 72.31 & 32.98 & 36.19 \\
        CycleGAN & 54.76 & 32.23 & 40.56 & 54.54 & 30.77 & 55.00 & 63.75 & 31.02 & 52.78 & 50.54 & 29.89 & 67.78 \\
        BBDM & {71.69} & {32.91} & {34.44} & 57.37 & 31.37 & 48.06 & \textbf{86.40} & 34.12 & 26.61 & {79.26} & 33.15 & 33.60 \\
        Vmanba & 69.54 & 32.67 & 36.42 & {63.01} & {31.47} & {46.99} & 79.63 & 34.12 & 26.49 & 77.45 & 33.53 & 31.85 \\
        DiffMa & 71.47 & 32.74 & 35.77 & 62.56 & 31.43 & 47.38 & 79.00 & {34.13} & {26.45} & 78.53 & {33.68} & {30.51} \\
        \rowcolor[HTML]{DAE0FB}HealthGPT-M3 & \underline{79.38} & \underline{33.03} & \underline{33.48} & \underline{71.81} & \underline{31.83} & \underline{43.45} & {85.06} & \textbf{34.40} & \textbf{25.49} & \underline{84.23} & \textbf{34.29} & \textbf{27.99} \\
        \rowcolor[HTML]{DAE0FB}HealthGPT-L14 & \textbf{79.73} & \textbf{33.10} & \textbf{32.96} & \textbf{71.92} & \textbf{31.87} & \textbf{43.09} & \underline{85.31} & \underline{34.29} & \underline{26.20} & \textbf{84.96} & \underline{34.14} & \underline{28.13} \\
        \bottomrule
    \end{tabular}
    }
    \label{tab:conversion}
\end{table*}

\noindent \textbf{3rd Stage: Visual Instruction Fine-Tuning.}  
In the third stage, we introduce additional task-specific data to further optimize the model and enhance its adaptability to downstream tasks such as medical visual comprehension (e.g., medical QA, medical dialogues, and report generation) or generation tasks (e.g., super-resolution, denoising, and modality conversion). Notably, by this stage, the word embedding layer and output head have been fine-tuned, only the H-LoRA modules and adapter modules need to be trained. This strategy significantly improves the model's adaptability and flexibility across different tasks.


\section{Experiment}
\label{s:experiment}

\subsection{Data Description}
We evaluate our method on FI~\cite{you2016building}, Twitter\_LDL~\cite{yang2017learning} and Artphoto~\cite{machajdik2010affective}.
FI is a public dataset built from Flickr and Instagram, with 23,308 images and eight emotion categories, namely \textit{amusement}, \textit{anger}, \textit{awe},  \textit{contentment}, \textit{disgust}, \textit{excitement},  \textit{fear}, and \textit{sadness}. 
% Since images in FI are all copyrighted by law, some images are corrupted now, so we remove these samples and retain 21,828 images.
% T4SA contains images from Twitter, which are classified into three categories: \textit{positive}, \textit{neutral}, and \textit{negative}. In this paper, we adopt the base version of B-T4SA, which contains 470,586 images and provides text descriptions of the corresponding tweets.
Twitter\_LDL contains 10,045 images from Twitter, with the same eight categories as the FI dataset.
% 。
For these two datasets, they are randomly split into 80\%
training and 20\% testing set.
Artphoto contains 806 artistic photos from the DeviantArt website, which we use to further evaluate the zero-shot capability of our model.
% on the small-scale dataset.
% We construct and publicly release the first image sentiment analysis dataset containing metadata.
% 。

% Based on these datasets, we are the first to construct and publicly release metadata-enhanced image sentiment analysis datasets. These datasets include scenes, tags, descriptions, and corresponding confidence scores, and are available at this link for future research purposes.


% 
\begin{table}[t]
\centering
% \begin{center}
\caption{Overall performance of different models on FI and Twitter\_LDL datasets.}
\label{tab:cap1}
% \resizebox{\linewidth}{!}
{
\begin{tabular}{l|c|c|c|c}
\hline
\multirow{2}{*}{\textbf{Model}} & \multicolumn{2}{c|}{\textbf{FI}}  & \multicolumn{2}{c}{\textbf{Twitter\_LDL}} \\ \cline{2-5} 
  & \textbf{Accuracy} & \textbf{F1} & \textbf{Accuracy} & \textbf{F1}  \\ \hline
% (\rownumber)~AlexNet~\cite{krizhevsky2017imagenet}  & 58.13\% & 56.35\%  & 56.24\%& 55.02\%  \\ 
% (\rownumber)~VGG16~\cite{simonyan2014very}  & 63.75\%& 63.08\%  & 59.34\%& 59.02\%  \\ 
(\rownumber)~ResNet101~\cite{he2016deep} & 66.16\%& 65.56\%  & 62.02\% & 61.34\%  \\ 
(\rownumber)~CDA~\cite{han2023boosting} & 66.71\%& 65.37\%  & 64.14\% & 62.85\%  \\ 
(\rownumber)~CECCN~\cite{ruan2024color} & 67.96\%& 66.74\%  & 64.59\%& 64.72\% \\ 
(\rownumber)~EmoVIT~\cite{xie2024emovit} & 68.09\%& 67.45\%  & 63.12\% & 61.97\%  \\ 
(\rownumber)~ComLDL~\cite{zhang2022compound} & 68.83\%& 67.28\%  & 65.29\% & 63.12\%  \\ 
(\rownumber)~WSDEN~\cite{li2023weakly} & 69.78\%& 69.61\%  & 67.04\% & 65.49\% \\ 
(\rownumber)~ECWA~\cite{deng2021emotion} & 70.87\%& 69.08\%  & 67.81\% & 66.87\%  \\ 
(\rownumber)~EECon~\cite{yang2023exploiting} & 71.13\%& 68.34\%  & 64.27\%& 63.16\%  \\ 
(\rownumber)~MAM~\cite{zhang2024affective} & 71.44\%  & 70.83\% & 67.18\%  & 65.01\%\\ 
(\rownumber)~TGCA-PVT~\cite{chen2024tgca}   & 73.05\%  & 71.46\% & 69.87\%  & 68.32\% \\ 
(\rownumber)~OEAN~\cite{zhang2024object}   & 73.40\%  & 72.63\% & 70.52\%  & 69.47\% \\ \hline
(\rownumber)~\shortname  & \textbf{79.48\%} & \textbf{79.22\%} & \textbf{74.12\%} & \textbf{73.09\%} \\ \hline
\end{tabular}
}
\vspace{-6mm}
% \end{center}
\end{table}
% 

\subsection{Experiment Setting}
% \subsubsection{Model Setting.}
% 
\textbf{Model Setting:}
For feature representation, we set $k=10$ to select object tags, and adopt clip-vit-base-patch32 as the pre-trained model for unified feature representation.
Moreover, we empirically set $(d_e, d_h, d_k, d_s) = (512, 128, 16, 64)$, and set the classification class $L$ to 8.

% 

\textbf{Training Setting:}
To initialize the model, we set all weights such as $\boldsymbol{W}$ following the truncated normal distribution, and use AdamW optimizer with the learning rate of $1 \times 10^{-4}$.
% warmup scheduler of cosine, warmup steps of 2000.
Furthermore, we set the batch size to 32 and the epoch of the training process to 200.
During the implementation, we utilize \textit{PyTorch} to build our entire model.
% , and our project codes are publicly available at https://github.com/zzmyrep/MESN.
% Our project codes as well as data are all publicly available on GitHub\footnote{https://github.com/zzmyrep/KBCEN}.
% Code is available at \href{https://github.com/zzmyrep/KBCEN}{https://github.com/zzmyrep/KBCEN}.

\textbf{Evaluation Metrics:}
Following~\cite{zhang2024affective, chen2024tgca, zhang2024object}, we adopt \textit{accuracy} and \textit{F1} as our evaluation metrics to measure the performance of different methods for image sentiment analysis. 



\subsection{Experiment Result}
% We compare our model against the following baselines: AlexNet~\cite{krizhevsky2017imagenet}, VGG16~\cite{simonyan2014very}, ResNet101~\cite{he2016deep}, CECCN~\cite{ruan2024color}, EmoVIT~\cite{xie2024emovit}, WSCNet~\cite{yang2018weakly}, ECWA~\cite{deng2021emotion}, EECon~\cite{yang2023exploiting}, MAM~\cite{zhang2024affective} and TGCA-PVT~\cite{chen2024tgca}, and the overall results are summarized in Table~\ref{tab:cap1}.
We compare our model against several baselines, and the overall results are summarized in Table~\ref{tab:cap1}.
We observe that our model achieves the best performance in both accuracy and F1 metrics, significantly outperforming the previous models. 
This superior performance is mainly attributed to our effective utilization of metadata to enhance image sentiment analysis, as well as the exceptional capability of the unified sentiment transformer framework we developed. These results strongly demonstrate that our proposed method can bring encouraging performance for image sentiment analysis.

\setcounter{magicrownumbers}{0} 
\begin{table}[t]
\begin{center}
\caption{Ablation study of~\shortname~on FI dataset.} 
% \vspace{1mm}
\label{tab:cap2}
\resizebox{.9\linewidth}{!}
{
\begin{tabular}{lcc}
  \hline
  \textbf{Model} & \textbf{Accuracy} & \textbf{F1} \\
  \hline
  (\rownumber)~Ours (w/o vision) & 65.72\% & 64.54\% \\
  (\rownumber)~Ours (w/o text description) & 74.05\% & 72.58\% \\
  (\rownumber)~Ours (w/o object tag) & 77.45\% & 76.84\% \\
  (\rownumber)~Ours (w/o scene tag) & 78.47\% & 78.21\% \\
  \hline
  (\rownumber)~Ours (w/o unified embedding) & 76.41\% & 76.23\% \\
  (\rownumber)~Ours (w/o adaptive learning) & 76.83\% & 76.56\% \\
  (\rownumber)~Ours (w/o cross-modal fusion) & 76.85\% & 76.49\% \\
  \hline
  (\rownumber)~Ours  & \textbf{79.48\%} & \textbf{79.22\%} \\
  \hline
\end{tabular}
}
\end{center}
\vspace{-5mm}
\end{table}


\begin{figure}[t]
\centering
% \vspace{-2mm}
\includegraphics[width=0.42\textwidth]{fig/2dvisual-linux4-paper2.pdf}
\caption{Visualization of feature distribution on eight categories before (left) and after (right) model processing.}
% 
\label{fig:visualization}
\vspace{-5mm}
\end{figure}

\subsection{Ablation Performance}
In this subsection, we conduct an ablation study to examine which component is really important for performance improvement. The results are reported in Table~\ref{tab:cap2}.

For information utilization, we observe a significant decline in model performance when visual features are removed. Additionally, the performance of \shortname~decreases when different metadata are removed separately, which means that text description, object tag, and scene tag are all critical for image sentiment analysis.
Recalling the model architecture, we separately remove transformer layers of the unified representation module, the adaptive learning module, and the cross-modal fusion module, replacing them with MLPs of the same parameter scale.
In this way, we can observe varying degrees of decline in model performance, indicating that these modules are indispensable for our model to achieve better performance.

\subsection{Visualization}
% 


% % 开始使用minipage进行左右排列
% \begin{minipage}[t]{0.45\textwidth}  % 子图1宽度为45%
%     \centering
%     \includegraphics[width=\textwidth]{2dvisual.pdf}  % 插入图片
%     \captionof{figure}{Visualization of feature distribution.}  % 使用captionof添加图片标题
%     \label{fig:visualization}
% \end{minipage}


% \begin{figure}[t]
% \centering
% \vspace{-2mm}
% \includegraphics[width=0.45\textwidth]{fig/2dvisual.pdf}
% \caption{Visualization of feature distribution.}
% \label{fig:visualization}
% % \vspace{-4mm}
% \end{figure}

% \begin{figure}[t]
% \centering
% \vspace{-2mm}
% \includegraphics[width=0.45\textwidth]{fig/2dvisual-linux3-paper.pdf}
% \caption{Visualization of feature distribution.}
% \label{fig:visualization}
% % \vspace{-4mm}
% \end{figure}



\begin{figure}[tbp]   
\vspace{-4mm}
  \centering            
  \subfloat[Depth of adaptive learning layers]   
  {
    \label{fig:subfig1}\includegraphics[width=0.22\textwidth]{fig/fig_sensitivity-a5}
  }
  \subfloat[Depth of fusion layers]
  {
    % \label{fig:subfig2}\includegraphics[width=0.22\textwidth]{fig/fig_sensitivity-b2}
    \label{fig:subfig2}\includegraphics[width=0.22\textwidth]{fig/fig_sensitivity-b2-num.pdf}
  }
  \caption{Sensitivity study of \shortname~on different depth. }   
  \label{fig:fig_sensitivity}  
\vspace{-2mm}
\end{figure}

% \begin{figure}[htbp]
% \centerline{\includegraphics{2dvisual.pdf}}
% \caption{Visualization of feature distribution.}
% \label{fig:visualization}
% \end{figure}

% In Fig.~\ref{fig:visualization}, we use t-SNE~\cite{van2008visualizing} to reduce the dimension of data features for visualization, Figure in left represents the metadata features before model processing, the features are obtained by embedding through the CLIP model, and figure in right shows the features of the data after model processing, it can be observed that after the model processing, the data with different label categories fall in different regions in the space, therefore, we can conclude that the Therefore, we can conclude that the model can effectively utilize the information contained in the metadata and use it to guide the model for classification.

In Fig.~\ref{fig:visualization}, we use t-SNE~\cite{van2008visualizing} to reduce the dimension of data features for visualization.
The left figure shows metadata features before being processed by our model (\textit{i.e.}, embedded by CLIP), while the right shows the distribution of features after being processed by our model.
We can observe that after the model processing, data with the same label are closer to each other, while others are farther away.
Therefore, it shows that the model can effectively utilize the information contained in the metadata and use it to guide the classification process.

\subsection{Sensitivity Analysis}
% 
In this subsection, we conduct a sensitivity analysis to figure out the effect of different depth settings of adaptive learning layers and fusion layers. 
% In this subsection, we conduct a sensitivity analysis to figure out the effect of different depth settings on the model. 
% Fig.~\ref{fig:fig_sensitivity} presents the effect of different depth settings of adaptive learning layers and fusion layers. 
Taking Fig.~\ref{fig:fig_sensitivity} (a) as an example, the model performance improves with increasing depth, reaching the best performance at a depth of 4.
% Taking Fig.~\ref{fig:fig_sensitivity} (a) as an example, the performance of \shortname~improves with the increase of depth at first, reaching the best performance at a depth of 4.
When the depth continues to increase, the accuracy decreases to varying degrees.
Similar results can be observed in Fig.~\ref{fig:fig_sensitivity} (b).
Therefore, we set their depths to 4 and 6 respectively to achieve the best results.

% Through our experiments, we can observe that the effect of modifying these hyperparameters on the results of the experiments is very weak, and the surface model is not sensitive to the hyperparameters.


\subsection{Zero-shot Capability}
% 

% (1)~GCH~\cite{2010Analyzing} & 21.78\% & (5)~RA-DLNet~\cite{2020A} & 34.01\% \\ \hline
% (2)~WSCNet~\cite{2019WSCNet}  & 30.25\% & (6)~CECCN~\cite{ruan2024color} & 43.83\% \\ \hline
% (3)~PCNN~\cite{2015Robust} & 31.68\%  & (7)~EmoVIT~\cite{xie2024emovit} & 44.90\% \\ \hline
% (4)~AR~\cite{2018Visual} & 32.67\% & (8)~Ours (Zero-shot) & 47.83\% \\ \hline


\begin{table}[t]
\centering
\caption{Zero-shot capability of \shortname.}
\label{tab:cap3}
\resizebox{1\linewidth}{!}
{
\begin{tabular}{lc|lc}
\hline
\textbf{Model} & \textbf{Accuracy} & \textbf{Model} & \textbf{Accuracy} \\ \hline
(1)~WSCNet~\cite{2019WSCNet}  & 30.25\% & (5)~MAM~\cite{zhang2024affective} & 39.56\%  \\ \hline
(2)~AR~\cite{2018Visual} & 32.67\% & (6)~CECCN~\cite{ruan2024color} & 43.83\% \\ \hline
(3)~RA-DLNet~\cite{2020A} & 34.01\%  & (7)~EmoVIT~\cite{xie2024emovit} & 44.90\% \\ \hline
(4)~CDA~\cite{han2023boosting} & 38.64\% & (8)~Ours (Zero-shot) & 47.83\% \\ \hline
\end{tabular}
}
\vspace{-5mm}
\end{table}

% We use the model trained on the FI dataset to test on the artphoto dataset to verify the model's generalization ability as well as robustness to other distributed datasets.
% We can observe that the MESN model shows strong competitiveness in terms of accuracy when compared to other trained models, which suggests that the model has a good generalization ability in the OOD task.

To validate the model's generalization ability and robustness to other distributed datasets, we directly test the model trained on the FI dataset, without training on Artphoto. 
% As observed in Table 3, compared to other models trained on Artphoto, we achieve highly competitive zero-shot performance, indicating that the model has good generalization ability in out-of-distribution tasks.
From Table~\ref{tab:cap3}, we can observe that compared with other models trained on Artphoto, we achieve competitive zero-shot performance, which shows that the model has good generalization ability in out-of-distribution tasks.


%%%%%%%%%%%%
%  E2E     %
%%%%%%%%%%%%


\section{Conclusion}
In this paper, we introduced Wi-Chat, the first LLM-powered Wi-Fi-based human activity recognition system that integrates the reasoning capabilities of large language models with the sensing potential of wireless signals. Our experimental results on a self-collected Wi-Fi CSI dataset demonstrate the promising potential of LLMs in enabling zero-shot Wi-Fi sensing. These findings suggest a new paradigm for human activity recognition that does not rely on extensive labeled data. We hope future research will build upon this direction, further exploring the applications of LLMs in signal processing domains such as IoT, mobile sensing, and radar-based systems.

\section*{Limitations}
While our work represents the first attempt to leverage LLMs for processing Wi-Fi signals, it is a preliminary study focused on a relatively simple task: Wi-Fi-based human activity recognition. This choice allows us to explore the feasibility of LLMs in wireless sensing but also comes with certain limitations.

Our approach primarily evaluates zero-shot performance, which, while promising, may still lag behind traditional supervised learning methods in highly complex or fine-grained recognition tasks. Besides, our study is limited to a controlled environment with a self-collected dataset, and the generalizability of LLMs to diverse real-world scenarios with varying Wi-Fi conditions, environmental interference, and device heterogeneity remains an open question.

Additionally, we have yet to explore the full potential of LLMs in more advanced Wi-Fi sensing applications, such as fine-grained gesture recognition, occupancy detection, and passive health monitoring. Future work should investigate the scalability of LLM-based approaches, their robustness to domain shifts, and their integration with multimodal sensing techniques in broader IoT applications.


% Bibliography entries for the entire Anthology, followed by custom entries
%\bibliography{anthology,custom}
% Custom bibliography entries only
\bibliography{main}
\newpage
\appendix

\section{Experiment prompts}
\label{sec:prompt}
The prompts used in the LLM experiments are shown in the following Table~\ref{tab:prompts}.

\definecolor{titlecolor}{rgb}{0.9, 0.5, 0.1}
\definecolor{anscolor}{rgb}{0.2, 0.5, 0.8}
\definecolor{labelcolor}{HTML}{48a07e}
\begin{table*}[h]
	\centering
	
 % \vspace{-0.2cm}
	
	\begin{center}
		\begin{tikzpicture}[
				chatbox_inner/.style={rectangle, rounded corners, opacity=0, text opacity=1, font=\sffamily\scriptsize, text width=5in, text height=9pt, inner xsep=6pt, inner ysep=6pt},
				chatbox_prompt_inner/.style={chatbox_inner, align=flush left, xshift=0pt, text height=11pt},
				chatbox_user_inner/.style={chatbox_inner, align=flush left, xshift=0pt},
				chatbox_gpt_inner/.style={chatbox_inner, align=flush left, xshift=0pt},
				chatbox/.style={chatbox_inner, draw=black!25, fill=gray!7, opacity=1, text opacity=0},
				chatbox_prompt/.style={chatbox, align=flush left, fill=gray!1.5, draw=black!30, text height=10pt},
				chatbox_user/.style={chatbox, align=flush left},
				chatbox_gpt/.style={chatbox, align=flush left},
				chatbox2/.style={chatbox_gpt, fill=green!25},
				chatbox3/.style={chatbox_gpt, fill=red!20, draw=black!20},
				chatbox4/.style={chatbox_gpt, fill=yellow!30},
				labelbox/.style={rectangle, rounded corners, draw=black!50, font=\sffamily\scriptsize\bfseries, fill=gray!5, inner sep=3pt},
			]
											
			\node[chatbox_user] (q1) {
				\textbf{System prompt}
				\newline
				\newline
				You are a helpful and precise assistant for segmenting and labeling sentences. We would like to request your help on curating a dataset for entity-level hallucination detection.
				\newline \newline
                We will give you a machine generated biography and a list of checked facts about the biography. Each fact consists of a sentence and a label (True/False). Please do the following process. First, breaking down the biography into words. Second, by referring to the provided list of facts, merging some broken down words in the previous step to form meaningful entities. For example, ``strategic thinking'' should be one entity instead of two. Third, according to the labels in the list of facts, labeling each entity as True or False. Specifically, for facts that share a similar sentence structure (\eg, \textit{``He was born on Mach 9, 1941.''} (\texttt{True}) and \textit{``He was born in Ramos Mejia.''} (\texttt{False})), please first assign labels to entities that differ across atomic facts. For example, first labeling ``Mach 9, 1941'' (\texttt{True}) and ``Ramos Mejia'' (\texttt{False}) in the above case. For those entities that are the same across atomic facts (\eg, ``was born'') or are neutral (\eg, ``he,'' ``in,'' and ``on''), please label them as \texttt{True}. For the cases that there is no atomic fact that shares the same sentence structure, please identify the most informative entities in the sentence and label them with the same label as the atomic fact while treating the rest of the entities as \texttt{True}. In the end, output the entities and labels in the following format:
                \begin{itemize}[nosep]
                    \item Entity 1 (Label 1)
                    \item Entity 2 (Label 2)
                    \item ...
                    \item Entity N (Label N)
                \end{itemize}
                % \newline \newline
                Here are two examples:
                \newline\newline
                \textbf{[Example 1]}
                \newline
                [The start of the biography]
                \newline
                \textcolor{titlecolor}{Marianne McAndrew is an American actress and singer, born on November 21, 1942, in Cleveland, Ohio. She began her acting career in the late 1960s, appearing in various television shows and films.}
                \newline
                [The end of the biography]
                \newline \newline
                [The start of the list of checked facts]
                \newline
                \textcolor{anscolor}{[Marianne McAndrew is an American. (False); Marianne McAndrew is an actress. (True); Marianne McAndrew is a singer. (False); Marianne McAndrew was born on November 21, 1942. (False); Marianne McAndrew was born in Cleveland, Ohio. (False); She began her acting career in the late 1960s. (True); She has appeared in various television shows. (True); She has appeared in various films. (True)]}
                \newline
                [The end of the list of checked facts]
                \newline \newline
                [The start of the ideal output]
                \newline
                \textcolor{labelcolor}{[Marianne McAndrew (True); is (True); an (True); American (False); actress (True); and (True); singer (False); , (True); born (True); on (True); November 21, 1942 (False); , (True); in (True); Cleveland, Ohio (False); . (True); She (True); began (True); her (True); acting career (True); in (True); the late 1960s (True); , (True); appearing (True); in (True); various (True); television shows (True); and (True); films (True); . (True)]}
                \newline
                [The end of the ideal output]
				\newline \newline
                \textbf{[Example 2]}
                \newline
                [The start of the biography]
                \newline
                \textcolor{titlecolor}{Doug Sheehan is an American actor who was born on April 27, 1949, in Santa Monica, California. He is best known for his roles in soap operas, including his portrayal of Joe Kelly on ``General Hospital'' and Ben Gibson on ``Knots Landing.''}
                \newline
                [The end of the biography]
                \newline \newline
                [The start of the list of checked facts]
                \newline
                \textcolor{anscolor}{[Doug Sheehan is an American. (True); Doug Sheehan is an actor. (True); Doug Sheehan was born on April 27, 1949. (True); Doug Sheehan was born in Santa Monica, California. (False); He is best known for his roles in soap operas. (True); He portrayed Joe Kelly. (True); Joe Kelly was in General Hospital. (True); General Hospital is a soap opera. (True); He portrayed Ben Gibson. (True); Ben Gibson was in Knots Landing. (True); Knots Landing is a soap opera. (True)]}
                \newline
                [The end of the list of checked facts]
                \newline \newline
                [The start of the ideal output]
                \newline
                \textcolor{labelcolor}{[Doug Sheehan (True); is (True); an (True); American (True); actor (True); who (True); was born (True); on (True); April 27, 1949 (True); in (True); Santa Monica, California (False); . (True); He (True); is (True); best known (True); for (True); his roles in soap operas (True); , (True); including (True); in (True); his portrayal (True); of (True); Joe Kelly (True); on (True); ``General Hospital'' (True); and (True); Ben Gibson (True); on (True); ``Knots Landing.'' (True)]}
                \newline
                [The end of the ideal output]
				\newline \newline
				\textbf{User prompt}
				\newline
				\newline
				[The start of the biography]
				\newline
				\textcolor{magenta}{\texttt{\{BIOGRAPHY\}}}
				\newline
				[The ebd of the biography]
				\newline \newline
				[The start of the list of checked facts]
				\newline
				\textcolor{magenta}{\texttt{\{LIST OF CHECKED FACTS\}}}
				\newline
				[The end of the list of checked facts]
			};
			\node[chatbox_user_inner] (q1_text) at (q1) {
				\textbf{System prompt}
				\newline
				\newline
				You are a helpful and precise assistant for segmenting and labeling sentences. We would like to request your help on curating a dataset for entity-level hallucination detection.
				\newline \newline
                We will give you a machine generated biography and a list of checked facts about the biography. Each fact consists of a sentence and a label (True/False). Please do the following process. First, breaking down the biography into words. Second, by referring to the provided list of facts, merging some broken down words in the previous step to form meaningful entities. For example, ``strategic thinking'' should be one entity instead of two. Third, according to the labels in the list of facts, labeling each entity as True or False. Specifically, for facts that share a similar sentence structure (\eg, \textit{``He was born on Mach 9, 1941.''} (\texttt{True}) and \textit{``He was born in Ramos Mejia.''} (\texttt{False})), please first assign labels to entities that differ across atomic facts. For example, first labeling ``Mach 9, 1941'' (\texttt{True}) and ``Ramos Mejia'' (\texttt{False}) in the above case. For those entities that are the same across atomic facts (\eg, ``was born'') or are neutral (\eg, ``he,'' ``in,'' and ``on''), please label them as \texttt{True}. For the cases that there is no atomic fact that shares the same sentence structure, please identify the most informative entities in the sentence and label them with the same label as the atomic fact while treating the rest of the entities as \texttt{True}. In the end, output the entities and labels in the following format:
                \begin{itemize}[nosep]
                    \item Entity 1 (Label 1)
                    \item Entity 2 (Label 2)
                    \item ...
                    \item Entity N (Label N)
                \end{itemize}
                % \newline \newline
                Here are two examples:
                \newline\newline
                \textbf{[Example 1]}
                \newline
                [The start of the biography]
                \newline
                \textcolor{titlecolor}{Marianne McAndrew is an American actress and singer, born on November 21, 1942, in Cleveland, Ohio. She began her acting career in the late 1960s, appearing in various television shows and films.}
                \newline
                [The end of the biography]
                \newline \newline
                [The start of the list of checked facts]
                \newline
                \textcolor{anscolor}{[Marianne McAndrew is an American. (False); Marianne McAndrew is an actress. (True); Marianne McAndrew is a singer. (False); Marianne McAndrew was born on November 21, 1942. (False); Marianne McAndrew was born in Cleveland, Ohio. (False); She began her acting career in the late 1960s. (True); She has appeared in various television shows. (True); She has appeared in various films. (True)]}
                \newline
                [The end of the list of checked facts]
                \newline \newline
                [The start of the ideal output]
                \newline
                \textcolor{labelcolor}{[Marianne McAndrew (True); is (True); an (True); American (False); actress (True); and (True); singer (False); , (True); born (True); on (True); November 21, 1942 (False); , (True); in (True); Cleveland, Ohio (False); . (True); She (True); began (True); her (True); acting career (True); in (True); the late 1960s (True); , (True); appearing (True); in (True); various (True); television shows (True); and (True); films (True); . (True)]}
                \newline
                [The end of the ideal output]
				\newline \newline
                \textbf{[Example 2]}
                \newline
                [The start of the biography]
                \newline
                \textcolor{titlecolor}{Doug Sheehan is an American actor who was born on April 27, 1949, in Santa Monica, California. He is best known for his roles in soap operas, including his portrayal of Joe Kelly on ``General Hospital'' and Ben Gibson on ``Knots Landing.''}
                \newline
                [The end of the biography]
                \newline \newline
                [The start of the list of checked facts]
                \newline
                \textcolor{anscolor}{[Doug Sheehan is an American. (True); Doug Sheehan is an actor. (True); Doug Sheehan was born on April 27, 1949. (True); Doug Sheehan was born in Santa Monica, California. (False); He is best known for his roles in soap operas. (True); He portrayed Joe Kelly. (True); Joe Kelly was in General Hospital. (True); General Hospital is a soap opera. (True); He portrayed Ben Gibson. (True); Ben Gibson was in Knots Landing. (True); Knots Landing is a soap opera. (True)]}
                \newline
                [The end of the list of checked facts]
                \newline \newline
                [The start of the ideal output]
                \newline
                \textcolor{labelcolor}{[Doug Sheehan (True); is (True); an (True); American (True); actor (True); who (True); was born (True); on (True); April 27, 1949 (True); in (True); Santa Monica, California (False); . (True); He (True); is (True); best known (True); for (True); his roles in soap operas (True); , (True); including (True); in (True); his portrayal (True); of (True); Joe Kelly (True); on (True); ``General Hospital'' (True); and (True); Ben Gibson (True); on (True); ``Knots Landing.'' (True)]}
                \newline
                [The end of the ideal output]
				\newline \newline
				\textbf{User prompt}
				\newline
				\newline
				[The start of the biography]
				\newline
				\textcolor{magenta}{\texttt{\{BIOGRAPHY\}}}
				\newline
				[The ebd of the biography]
				\newline \newline
				[The start of the list of checked facts]
				\newline
				\textcolor{magenta}{\texttt{\{LIST OF CHECKED FACTS\}}}
				\newline
				[The end of the list of checked facts]
			};
		\end{tikzpicture}
        \caption{GPT-4o prompt for labeling hallucinated entities.}\label{tb:gpt-4-prompt}
	\end{center}
\vspace{-0cm}
\end{table*}
% \section{Full Experiment Results}
% \begin{table*}[th]
    \centering
    \small
    \caption{Classification Results}
    \begin{tabular}{lcccc}
        \toprule
        \textbf{Method} & \textbf{Accuracy} & \textbf{Precision} & \textbf{Recall} & \textbf{F1-score} \\
        \midrule
        \multicolumn{5}{c}{\textbf{Zero Shot}} \\
                Zero-shot E-eyes & 0.26 & 0.26 & 0.27 & 0.26 \\
        Zero-shot CARM & 0.24 & 0.24 & 0.24 & 0.24 \\
                Zero-shot SVM & 0.27 & 0.28 & 0.28 & 0.27 \\
        Zero-shot CNN & 0.23 & 0.24 & 0.23 & 0.23 \\
        Zero-shot RNN & 0.26 & 0.26 & 0.26 & 0.26 \\
DeepSeek-0shot & 0.54 & 0.61 & 0.54 & 0.52 \\
DeepSeek-0shot-COT & 0.33 & 0.24 & 0.33 & 0.23 \\
DeepSeek-0shot-Knowledge & 0.45 & 0.46 & 0.45 & 0.44 \\
Gemma2-0shot & 0.35 & 0.22 & 0.38 & 0.27 \\
Gemma2-0shot-COT & 0.36 & 0.22 & 0.36 & 0.27 \\
Gemma2-0shot-Knowledge & 0.32 & 0.18 & 0.34 & 0.20 \\
GPT-4o-mini-0shot & 0.48 & 0.53 & 0.48 & 0.41 \\
GPT-4o-mini-0shot-COT & 0.33 & 0.50 & 0.33 & 0.38 \\
GPT-4o-mini-0shot-Knowledge & 0.49 & 0.31 & 0.49 & 0.36 \\
GPT-4o-0shot & 0.62 & 0.62 & 0.47 & 0.42 \\
GPT-4o-0shot-COT & 0.29 & 0.45 & 0.29 & 0.21 \\
GPT-4o-0shot-Knowledge & 0.44 & 0.52 & 0.44 & 0.39 \\
LLaMA-0shot & 0.32 & 0.25 & 0.32 & 0.24 \\
LLaMA-0shot-COT & 0.12 & 0.25 & 0.12 & 0.09 \\
LLaMA-0shot-Knowledge & 0.32 & 0.25 & 0.32 & 0.28 \\
Mistral-0shot & 0.19 & 0.23 & 0.19 & 0.10 \\
Mistral-0shot-Knowledge & 0.21 & 0.40 & 0.21 & 0.11 \\
        \midrule
        \multicolumn{5}{c}{\textbf{4 Shot}} \\
GPT-4o-mini-4shot & 0.58 & 0.59 & 0.58 & 0.53 \\
GPT-4o-mini-4shot-COT & 0.57 & 0.53 & 0.57 & 0.50 \\
GPT-4o-mini-4shot-Knowledge & 0.56 & 0.51 & 0.56 & 0.47 \\
GPT-4o-4shot & 0.77 & 0.84 & 0.77 & 0.73 \\
GPT-4o-4shot-COT & 0.63 & 0.76 & 0.63 & 0.53 \\
GPT-4o-4shot-Knowledge & 0.72 & 0.82 & 0.71 & 0.66 \\
LLaMA-4shot & 0.29 & 0.24 & 0.29 & 0.21 \\
LLaMA-4shot-COT & 0.20 & 0.30 & 0.20 & 0.13 \\
LLaMA-4shot-Knowledge & 0.15 & 0.23 & 0.13 & 0.13 \\
Mistral-4shot & 0.02 & 0.02 & 0.02 & 0.02 \\
Mistral-4shot-Knowledge & 0.21 & 0.27 & 0.21 & 0.20 \\
        \midrule
        
        \multicolumn{5}{c}{\textbf{Suprevised}} \\
        SVM & 0.94 & 0.92 & 0.91 & 0.91 \\
        CNN & 0.98 & 0.98 & 0.97 & 0.97 \\
        RNN & 0.99 & 0.99 & 0.99 & 0.99 \\
        % \midrule
        % \multicolumn{5}{c}{\textbf{Conventional Wi-Fi-based Human Activity Recognition Systems}} \\
        E-eyes & 1.00 & 1.00 & 1.00 & 1.00 \\
        CARM & 0.98 & 0.98 & 0.98 & 0.98 \\
\midrule
 \multicolumn{5}{c}{\textbf{Vision Models}} \\
           Zero-shot SVM & 0.26 & 0.25 & 0.25 & 0.25 \\
        Zero-shot CNN & 0.26 & 0.25 & 0.26 & 0.26 \\
        Zero-shot RNN & 0.28 & 0.28 & 0.29 & 0.28 \\
        SVM & 0.99 & 0.99 & 0.99 & 0.99 \\
        CNN & 0.98 & 0.99 & 0.98 & 0.98 \\
        RNN & 0.98 & 0.99 & 0.98 & 0.98 \\
GPT-4o-mini-Vision & 0.84 & 0.85 & 0.84 & 0.84 \\
GPT-4o-mini-Vision-COT & 0.90 & 0.91 & 0.90 & 0.90 \\
GPT-4o-Vision & 0.74 & 0.82 & 0.74 & 0.73 \\
GPT-4o-Vision-COT & 0.70 & 0.83 & 0.70 & 0.68 \\
LLaMA-Vision & 0.20 & 0.23 & 0.20 & 0.09 \\
LLaMA-Vision-Knowledge & 0.22 & 0.05 & 0.22 & 0.08 \\

        \bottomrule
    \end{tabular}
    \label{full}
\end{table*}




\end{document}


\section{Experiments}
In this section, we conduct extensive experiments to validate the effectiveness of our method on prompt adaptation. For the policy model, we use a pretrained GPT-2 \citep{radford2019language} following the setup in~\citep{hao2024optimizing}. As a default setting, we employ Stable-Diffusion v1.4 \citep{rombach2022high} as the target text-to-image model and use DPM solver \citep{lu2022dpm} with 20 inference steps to accelerate the sampling process. Detailed information regarding the experimental settings can be found in \Cref{app:exp_details}.

\begin{figure*}[t]
    \centering
    \includegraphics[width=\textwidth]{figures/main_tradeoff.png}
    \caption{Reward and diversity of prompts generated by each method with different initial prompt datasets.}
    \label{fig:main_tradeoff}
    \vspace{-10pt}
\end{figure*}

\subsection{Experiment Setup}
\noindent\textbf{Dataset Preparation}
We strictly follow the setup of Promptist~\citep{hao2024optimizing} for dataset preparation. For training, we collect prompts from the Lexica website \citep{lexica}, DiffusionDB \citep{wang2023diffusiondb}, and COCO dataset \citep{chen2015microsoft}. For evaluation, we randomly sample $256$ prompts from the three types of training datasets. In addition, we introduce a challenging dataset using ChatGPT \citep{ouyang2022training} to
generate brief prompts that describe images around $5$ words. These brief prompts
naturally require diverse adaptations for better generation quality, representing a practical scenario for prompt adaptation.

\vspace{5pt}
\noindent\textbf{Baselines}
We consider several strong baselines to verify the efficacy of our method on the prompt adaptation task.
\begin{itemize}
    \item \textbf{Supervised Fine-tuning (SFT)}: A policy model fine-tuned by supervised learning on a set of prompt pairs of original user inputs and manually engineered prompts. 
    \item \textbf{Promptist}: A PPO-based approach \citep{schulman2017proximal} that directly trains the RL policy to maximize the reward function. 
    \item \textbf{Rule-Based}: Based on the observation that Promptist mostly generates similar postfixes with deterministic behaviors, we build a heuristic that appends the most frequently used postfixes in Promptist to the initial prompt.
    \item \textbf{GFlowNets}: We implement a vanilla GFlowNets method with the trajectory balance (TB) objective~\citep{malkin2022trajectory} to train the target policy. As our task is a conditional generation task, we use the VarGrad \citep{richter2020vargrad} version of TB loss, which is widely used for reducing variance \citep{zhangrobust, kim2024ant}.
    \item \textbf{DPO-Diff}: A gradient-based optimization method designed to discover effective negative prompts \citep{wang2024discrete}. As it is not directly comparable, we provide more discussion on DPO-Diff in Appendix~\ref{app:extend_main_results}. 
\end{itemize}

\vspace{5pt}
\noindent\textbf{Training and Evaluation}
Following Promptist~\citep{hao2024optimizing}, we initialize the policy with SFT policy before GFlowNet fine-tuning. To parametrize the flow function, we use a separate neural network that takes the last hidden embedding of the prompts as input and outputs a scalar value. We train both the policy and flow function for 10K steps with a batch size of 256. For learning rate, we use $1\times10^{-5}$ for the policy and $1\times10^{-4}$ for the flow function.

For evaluation, we generate $16$ prompts for each prompt via beam search with a length penalty of $1.0$. Then we generate $3$ images per prompt to compute the reward. To measure diversity, we compute the average pairwise cosine distance of 16 prompts for each initial prompt.

\subsection{Performance Comparison} \label{sec:main_res}
\Cref{fig:main_tradeoff} presents a comprehensive comparison of PAG against the baselines across different datasets. As depicted in the figure, while Promptist improves upon SFT through RL-based fine-tuning, it exhibits limited diversity due to the reward-maximization nature.
Vanilla GFlowNets mostly achieves significantly higher rewards than both SFT and Promptist, but still suffers from mode collapse. In contrast, PAG consistently surpasses all baselines in terms of both reward and diversity metrics, demonstrating its capability to generate both high-quality and diverse prompts across various input types.

\begin{figure*}[t]
    \centering
    \includegraphics[width=0.95\textwidth]{figures/main_figure.jpg}
    \caption{Images generated by optimized prompts using Stable Diffusion v1.4 (with the same seed to visualize the effect solely on prompts).
    Our method generates diverse and highly aesthetic images based on adapted prompts of high quality and diversity.}
    \label{fig:main_figure}
    \vspace{-12pt}
\end{figure*}

The challenging ChatGPT dataset, where initial prompts usually give us little information, further highlights the strengths of our approach. 
We observe that Promptist or vanilla GFlowNets lead to severe mode collapse and are outperformed by simple deterministic rule-based heuristics, while our method maintains robust performance.
We further visualize the generated prompts and corresponding images from each method in \Cref{fig:main_figure} (more figures can be found in Appendix~\ref{app:more_visualization}). As illustrated in the figures, PAG generates diverse and meaningful enhancements by not only appending postfixes but also introducing relevant adjectives and detailed descriptions, which is particularly valuable when the original prompt lacks sufficient information.

\begin{table}[t]
\centering
\caption{Reward and diversity of prompts generated by each method with different reward functions.}
\vspace{-5pt}
\resizebox{\linewidth}{!}{
\begin{tabular}{lcc|cc}
\toprule
\multirow{2}{*}{\textbf{Method}} & \multicolumn{2}{c}{ImageReward} & \multicolumn{2}{c}{HPSv2}  \\
\cmidrule{2-5}
& Reward & Diversity & Reward & Diversity \\
\midrule
SFT & 0.66 ± 0.01 & 0.14 ± 0.00 & 0.01 ± 0.00 & 0.14 ± 0.00 \\
Promptist & 0.70 ± 0.05 & 0.02 ± 0.00 & -0.05 ± 0.01 & 0.03 ± 0.00 \\
Rule-Based & 0.59 ± 0.02 & 0.12 ± 0.00 & 0.01 ± 0.00 & 0.12 ± 0.00 \\
GFlowNets & 0.81 ± 0.15 & 0.20 ± 0.00 & \textbf{0.02 ± 0.00} & 0.14 ± 0.00 \\
\midrule
\textbf{PAG (Ours)} & \textbf{0.83 ± 0.01} & \textbf{0.29 ± 0.00} & \textbf{0.02 ± 0.00} & \textbf{0.37 ± 0.00} \\
\bottomrule
\end{tabular}}
\label{tab:extend_reward_fn}
\vspace{-10pt}
\end{table}
\vspace{-.03in}
\subsection{Robustness across Different Reward Functions}
To evaluate the robustness of our framework in text-to-image modeling, we extend our experiments to include 
two additional widely-used reward functions in diffusion alignment: ImageReward \citep{xu2024imagereward} and HPSv2 \citep{wu2023human}. We maintain the same training procedure as our primary experiments while incorporating these different reward functions, and utilize COCO dataset for evaluation. Please refer to \Cref{app:exp_details} for more detailed experimental procedures. 

\Cref{tab:extend_reward_fn} summarizes the results of different methods, which demonstrates that PAG achieves state-of-the-art performance in terms of both reward and diversity metrics. This consistent performance across multiple reward functions validates the robustness of our approach. 

\subsection{Transferability to Different Text-to-Image Diffusion Models}
Since our method operates on prompt adaptation without modifying the underlying diffusion model parameters,
it has the potential to generalize across different text-to-image diffusion models in a zero-shot manner. 
To validate this, we evaluate the transferability of our method on various representative text-to-image diffusion models distinct from the target model used during training, including SD v1.5 \citep{rombach2022high}, SSD-1B \citep{gupta2024progressive}, SDXL-Turbo \citep{sauer2025adversarial}, and SD3 \citep{esser2024scaling} for evaluation. We use prompts generated by each method using SD v1.4 with initial prompts from the ChatGPT dataset.

\Cref{tab:transfer_t2i_models} summarizes the performance of various methods on different text-to-image diffusion models. As demonstrated in the table, PAG consistently generates high-rewarding images compared to other methods due to its ability to produce diverse prompts, which provides robustness on different text-to-image diffusion models. Surprisingly, we observe that there is a significant gap between our method and baselines in the SD3 model, showcasing that our method can be applied in practical settings. We also visualize the generated images across different text-to-image diffusion models in \cref{app:more_visualization}.

\begin{figure*}[t]
    \centering
    \includegraphics[width=0.95\textwidth]{figures/ddpo_comparison.jpg}
    \caption{Comparison with DDPO and PAG. We report the aesthetic score of images in bold.}
    \label{fig:ddpo_comparison}
    \vspace{-12pt}
\end{figure*}
\begin{table}[t]
\centering
\caption{We train the policy with SD v1.4 as a target model and evaluate the generated propmts with different text-to-image models in a zero-shot manner.}
\vspace{-5pt}
\resizebox{\linewidth}{!}{
\begin{tabular}{lcccc}
\toprule
\multirow{2}{*}{\textbf{Method}} & \multicolumn{4}{c}{Text-to-Image Diffusion Models} \\
\cmidrule{2-5}
& SD v1.5 & SSD-1B & SDXL-Turbo & SD3 \\
\midrule
SFT & 0.78 ± 0.05 & 0.53 ± 0.05 & 0.54 ± 0.05 & 0.77 ± 0.04 \\
Promptist & 0.80 ± 0.03 & 0.46 ± 0.05 & 0.47 ± 0.06 & 0.73 ± 0.03 \\
Rule-Based & 0.85 ± 0.02 & 0.54 ± 0.05 & 0.54 ± 0.05 & 0.76 ± 0.03 \\
GFlowNets & 0.62 ± 0.04 & 0.51 ± 0.02 & 0.49 ± 0.01 & 0.75 ± 0.06 \\
\midrule
\textbf{PAG (Ours)}  & \textbf{0.87 ± 0.02} & \textbf{0.61 ± 0.04} & \textbf{0.67 ± 0.02} & \textbf{0.95 ± 0.05} \\
\bottomrule
\end{tabular}}
\label{tab:transfer_t2i_models}
\vspace{-15pt}
\end{table}

\subsection{Comparison with Fine-tuning Text-to-Image Diffusion Models}
To verify the effectiveness of our framework, we also compare our method with approaches that directly fine-tune text-to-image diffusion models, and compare with DDPO \citep{blacktraining}, which is trained with the aesthetic score reward function on animal prompts. As shown in \Cref{fig:ddpo_comparison}, we find that PAG achieves competitive performance with DDPO in terms of aesthetic quality. Furthermore, we observe that generated samples from DDPO converge to similar styles, whereas PAG generates diverse and high-quality images, indicating that prompt adaptation can be a promising alternative and a complementary approach to directly fine-tuning diffusion models for generating images with desired properties.  
For more details on comparison between directly fine-tuning diffusion models and our method, please refer to \Cref{app:comp_diffusion}.

\subsection{Ablation Studies}
In this section, we conduct comprehensive ablation studies to investigate the effectiveness of the important components and analyze how our method systematically tackles the severe mode collapse problem. 

\Cref{tab:ablation} shows the effectiveness of flow reactivation (Reset), reward-prioritized sampling (PRT), and reward decomposition (FL) in COCO and ChatGPT datasets. As shown, each component contributes substantially to the overall performance. Notably, we observe that removing PRT leads to significant performance degradation, which indicates that we should carefully tackle the mode collapse issue from the angle of both network parameters and training samples. Moreover, reward decomposition improves both reward and diversity metrics, validating its effectiveness for better credit assignment and mitigating mode collapse.
Additional component analysis can be found in Appendix~\ref{app:extend_ablation}.

% \begin{table}[!t]
% \centering
% \scalebox{0.68}{
%     \begin{tabular}{ll cccc}
%       \toprule
%       & \multicolumn{4}{c}{\textbf{Intellipro Dataset}}\\
%       & \multicolumn{2}{c}{Rank Resume} & \multicolumn{2}{c}{Rank Job} \\
%       \cmidrule(lr){2-3} \cmidrule(lr){4-5} 
%       \textbf{Method}
%       &  Recall@100 & nDCG@100 & Recall@10 & nDCG@10 \\
%       \midrule
%       \confitold{}
%       & 71.28 &34.79 &76.50 &52.57 
%       \\
%       \cmidrule{2-5}
%       \confitsimple{}
%     & 82.53 &48.17
%        & 85.58 &64.91
     
%        \\
%        +\RunnerUpMiningShort{}
%     &85.43 &50.99 &91.38 &71.34 
%       \\
%       +\HyReShort
%         &- & -
%        &-&-\\
       
%       \bottomrule

%     \end{tabular}
%   }
% \caption{Ablation studies using Jina-v2-base as the encoder. ``\confitsimple{}'' refers using a simplified encoder architecture. \framework{} trains \confitsimple{} with \RunnerUpMiningShort{} and \HyReShort{}.}
% \label{tbl:ablation}
% \end{table}
\begin{table*}[!t]
\centering
\scalebox{0.75}{
    \begin{tabular}{l cccc cccc}
      \toprule
      & \multicolumn{4}{c}{\textbf{Recruiting Dataset}}
      & \multicolumn{4}{c}{\textbf{AliYun Dataset}}\\
      & \multicolumn{2}{c}{Rank Resume} & \multicolumn{2}{c}{Rank Job} 
      & \multicolumn{2}{c}{Rank Resume} & \multicolumn{2}{c}{Rank Job}\\
      \cmidrule(lr){2-3} \cmidrule(lr){4-5} 
      \cmidrule(lr){6-7} \cmidrule(lr){8-9} 
      \textbf{Method}
      & Recall@100 & nDCG@100 & Recall@10 & nDCG@10
      & Recall@100 & nDCG@100 & Recall@10 & nDCG@10\\
      \midrule
      \confitold{}
      & 71.28 & 34.79 & 76.50 & 52.57 
      & 87.81 & 65.06 & 72.39 & 56.12
      \\
      \cmidrule{2-9}
      \confitsimple{}
      & 82.53 & 48.17 & 85.58 & 64.91
      & 94.90&78.40 & 78.70& 65.45
       \\
      +\HyReShort{}
       &85.28 & 49.50
       &90.25 & 70.22
       & 96.62&81.99 & \textbf{81.16}& 67.63
       \\
      +\RunnerUpMiningShort{}
       % & 85.14& 49.82
       % &90.75&72.51
       & \textbf{86.13}&\textbf{51.90} & \textbf{94.25}&\textbf{73.32}
       & \textbf{97.07}&\textbf{83.11} & 80.49& \textbf{68.02}
       \\
   %     +\RunnerUpMiningShort{}
   %    & 85.43 & 50.99 & 91.38 & 71.34 
   %    & 96.24 & 82.95 & 80.12 & 66.96
   %    \\
   %    +\HyReShort{} old
   %     &85.28 & 49.50
   %     &90.25 & 70.22
   %     & 96.62&81.99 & 81.16& 67.63
   %     \\
   % +\HyReShort{} 
   %     % & 85.14& 49.82
   %     % &90.75&72.51
   %     & 86.83&51.77 &92.00 &72.04
   %     & 97.07&83.11 & 80.49& 68.02
   %     \\
      \bottomrule

    \end{tabular}
  }
\caption{\framework{} ablation studies. ``\confitsimple{}'' refers using a simplified encoder architecture. \framework{} trains \confitsimple{} with \RunnerUpMiningShort{} and \HyReShort{}. We use Jina-v2-base as the encoder due to its better performance.
}
\label{tbl:ablation}
\end{table*}
\begin{figure}[t]
\begin{minipage}[t]{\linewidth}
    \begin{subfigure}[t]{0.48\linewidth}
        \centering
        \includegraphics[width=\textwidth]{figures/dormant.png}
    \end{subfigure}
    \begin{subfigure}[t]{0.48\linewidth}
        \centering
        \includegraphics[width=\textwidth]{figures/cos_dist.png}
    \end{subfigure}
    \caption{The effect of flow reactivation.}
    \label{fig:ablation}
\end{minipage}
\vspace{-15pt}
\end{figure}

Furthermore, we carefully analyze how flow reactivation mitigates the mode collapse problem by tracking the percentage of dormant neurons (based on Eq.~\ref{eq:dormant}) over the course of training. As depicted in \Cref{fig:ablation}, our flow reactivation mechanism exhibits a significantly lower dormant neuron rate, while the variant without this mechanism results in a high portion of dormant neurons during training. We also plot the cosine distance among generated prompts over training and observe that, without reactivation, the GFlowNet policy suffers from the mode collapse issue.
These findings validate its effectiveness in preventing our GFlowNets-based agent from converging to limited patterns by maintaining the expressivity of the flows, contributing to both performance and diversity improvements.

\section{Related Works}
\noindent\textbf{Aligning Diffusion Models}
There is a surge of interest in generating images with desired properties, which can be modeled as reward functions from human feedback \citep{ouyang2022training}. 
A widely recognized approach involves directly fine-tuning diffusion models with the reward function. \citet{blacktraining} and \citet{fan2024reinforcement} formulate the diffusion reverse process as a MDP and employ RL for fine-tuning diffusion models while \citet{clarkdirectly} and \citet{prabhudesai2023aligning} update model parameters through end-to-end backpropagation of the gradient of reward across denoising steps. 
While those methods have shown promising results, 
they require initiating fine-tuning from scratch for each different text-to-image diffusion model and necessitate access to the model parameters, which may be restricted due to confidential issues \citep{ramesh2022hierarchical, saharia2022photorealistic}. Unlike these methods, PAG enhances initial prompts into high-quality and diverse prompts by fine-tuning language models, allowing transferability across various text-to-image models. 

\vspace{5pt}
\noindent\textbf{Prompt Adaptation for Text-to-Image Models}
There have been some trials to generate high-quality images by adapting prompts instead of fine-tuning text-to-image models. A pioneering work of this approach is Promptist \citep{hao2024optimizing}, which formulates prompt adaptation as an RL problem.
A relevant recent work, DPO-Diff, \citep{wang2024discrete}, also tries to generate user-aligned images while optimizing negative prompts using a shortcut text gradient. We find that Promptist often results in deterministic policy, which can be easily replaced by heuristics. PAG utilizes GFlowNets to fine-tune language models for generating effective and diverse prompts. 

\vspace{5pt}
\noindent\textbf{GFlowNet Fine-Tuning}
GFlowNets are probabilistic methods that sample compositional objects proportional to unnormalized density through sequential decision-making~\citep{bengio2021flow, bengio2023gflownet} and energy-based modeling~\citep{zhang2022generative}, with applications in structure learning~\citep{deleu2022bayesian} and combinatorial optimization~\citep{zhangrobust,Zhang2023LetTF}, which can also be extended to continuous~\citep{lahlou2023cgfn} or stochastic scenarios~\citep{pan2023stochastic,zhang2023distributional}.
It has the potential for fine-tuning language models (LMs) for intractable posterior inference problems~\citep{huamortizing} and robust red-teaming~\citep{lee2024learning}.
Although there have been several attempts in improving exploration and training efficiency~\citep{pan2023generative, kimlocal, lau2024qgfn} and extending it to more general domains and learning paradigms, previous works typically train a GFlowNets policy from scratch~\citep{pan2024pre,he2024looking}, and largely overlooked the critical problem of plasticity loss during fine-tuning~\citep{zhang2024improving,liu2024efficientdiversitypreservingdiffusionalignment}.
Furthermore, most prior methods focus on unconditional generation and often suffer from mode collapse, requiring an additional post-supervised fine-tuning stage. 
In this work, we investigate the mode collapse problem in prompt adaptation and propose PAG, a novel approach to address this key challenge for diverse conditional prompt generation for text-to-image diffusion models.

\section{Conclusion}
In this paper, we propose a novel approach that systematically addresses mode collapse in prompt adaptation through GFlowNets-based probabilistic inference. We identify a critical plasticity loss problem in GFlowNets-based prompt adaptation when learning from rewards sequentially, and present PAG to maintain generation flexibility. Our extensive results show that PAG successfully learns to sample effective and diverse prompts for text-to-image generation. 

{
    \bibliographystyle{ieeenat_fullname}
    \bibliography{main}
}

\clearpage
\appendix

\noindent Our code is publicly available \href{https://github.com/dbsxodud-11/PAG.git}{here}.

\section{Illustrations of RL and GFlowNets}\label{app:diff_btw_rl_and_gfn}
In this section, we summarize the key difference between Reinforcement Learning (RL) and GFlowNets in more detail. GFlowNets~\citep{bengio2021flow} is designed to learn a stochastic policy that generates samples proportional to their rewards (i.e., $p(x)\propto R(x)$), while RL aims to learn a policy that maximizes the reward function as illustrated in \Cref{fig:rl_vs_gfn}. Learning a policy that samples proportional to the reward function leads to capturing multi-modal distribution and discovering high-quality and diverse candidates, which is particularly beneficial when the reward proxy is inaccurate \citep{bengio2021flow}, and their connections have been studied in \citep{tiapkin2024generative,deleu2024discrete,he2024rectifying}, which shares equivalences and similarities with entropy-regularized RL~\citep{nachum2017bridging} in tree-structured sequence generation and directed acyclic graph problems~\citep{huamortizing,lee2024learning}.
In prompt adaptation tasks, conventional RL-based approaches~\citep{hao2024optimizing} based on PPO~\citep{schulman2017proximal} that naively maximize a reward function can lead to reward overoptimization and hinder generalizability to different text-to-image diffusion models, while it is more desirable to generate not only effective and but also diverse prompts.     
\begin{figure}[h]
    \centering
    \includegraphics[width=0.9\linewidth]{figures/rl_vs_gfn.png}
    \vspace{-5pt}
    \caption{Comparison of the learning objective of reward-maximizing RL and reward-matching GFlowNets.}
    \label{fig:rl_vs_gfn}
    \vspace{-12pt}
\end{figure}

\section{Experiment Setting Details}\label{app:exp_details}
In this section, we present details of experiment settings.
\subsection{Data Preparation}
We strictly follow the procedure of Promptist \citep{hao2024optimizing} to prepare training and evaluation datasets. Additionally, we introduce a challenging dataset using ChatGPT \citep{ouyang2022training} interface to generate brief prompts that describe images. Specifically, we use the following prompt to query ChatGPT for short image descriptions:
\begin{itemize}
    \item \texttt{Generate N sentences describing photos /pictures/images with length around 5.}
\end{itemize}

\noindent Below are a few example prompts generated by ChatGPT:
\begin{itemize}
    \item \texttt{A bird sitting on a branch.}
    \item \texttt{A tree under a sky.}
    \item \texttt{A car drives on a road.}
    \item \texttt{A train moves through the city.}
    \item \texttt{A boat sails on a lake.}
\end{itemize}

\subsection{Baselines}
In this section, we provide more details on the baselines used for our experiments.

\paragraph{Supervised Fine-Tuning.} We fine-tune the pretrained GPT-2 policy model with supervised learning on a set of prompt pairs of original user inputs and manually engineered prompts provided by Promptist \citep{hao2024optimizing}. As a default, we use the pretrained weights of SFT publicly available\footnote{https://github.com/microsoft/LMOps/tree/main/promptist}.

\paragraph{Promptist.} We train the policy with a PPO-based approach where the policy is initialized with the supervised fine-tuned model. As a default, we use the pretrained weights of Promptist publicly available to ensure a fair comparison. To evaluate the generalizability of different reward functions, we train the policy with the same hyperparameter configurations.

\paragraph{Rule-Based.} Based on the observation that Promptist mostly generates similar postfixes for optimization, we build a rule-based method that appends the most frequently used postfixes in Promptist to the initial prompt. Below are a few example postfixes we used for evaluation:
\begin{itemize}
    \item \texttt{intricate, elegant, highly detailed, digital painting, artstation, concept art, sharp focus, illustration, by justin gerard and artgerm, 8 k.}
    \item \texttt{by greg rutkowski, digital art, realistic painting, fantasy, very detailed, trending on artstation.}
    \item \texttt{highly detailed, digital painting, artstation, concept art, sharp focus, illustration, art by greg rutkowski and alphonse mucha.}
\end{itemize}

\paragraph{GFlowNets.} We train a vanilla GFlowNets policy with TB \citep{malkin2022trajectory} objective. As our task is a conditional generation task, a naive implementation of TB should directly estimate a conditional partition function, $Z_{\theta}(\mathbf{x})$, which makes learning highly unstable \citep{kim2024ant}. For pratical implementation, we use VarGrad \citep{richter2020vargrad} objective to fine-tune the policy, which is widely used for reducing variance in GFlowNets literature \citep{zhangrobust, venkatraman2024amortizing}.

For each initial prompt $\mathbf{x}$ sampled in the minibatch, we generate $k=16$ on-policy samples $\mathbf{y}^{(1)},\cdots,\mathbf{y}^{(k)}$ with the forward policy. Each sample can be used to implicitly estimate $\log Z(\mathbf{x})$:
\begin{align*}
    \log \hat{Z}(\mathbf{x})^{(i)}=R(\mathbf{x},\mathbf{y}^{(i)})-\sum_{t=1}^{T}\log P_{F}(y_{t}^{(i)}\vert y_{0:t-1}^{(i)},\mathbf{x};\theta)
\end{align*}
Then we minimize the sample variance across the minibatch as follows:
\begin{align*}
    \mathcal{L}(\mathbf{x};\theta)=\frac{1}{k}\sum_{i=1}^{k}\left(
    \log\hat{Z}(\mathbf{x})^{(i)}-\frac{1}{k}\sum_{j=1}^{k}\log\hat{Z}(\mathbf{x})^{(j)}\right)^2
\end{align*}

\paragraph{DPO-Diff.} \citet{wang2024discrete} proposed a discrete prompt optimization for diffusion models (DPO-Diff), which is a gradient-based optimization method for discovering effective negative prompts to generate user-aligned images. It first generates a compact subspace comprised of only the most relevant words to user input with ChatGPT API, then uses shortcut text gradients to efficiently compute the text gradient for optimization. As the original reward function of DPO-Diff is a spherical clip loss, we replace the reward function the same as \cref{eq:task_reward}. As we consider a setting where the reward function is a black-box function, we use the evolutionary search module suggested by the paper for a fair comparison. Please refer to the paper for more details.

\subsection{Training and Evaluation}
For training, we initialize the policy with the SFT policy before the GFlowNet fine-tuning. To parametrize the flow function, we use a separate neural network that takes the last hidden embedding of the prompts as input and outputs a scalar value. We conducted all experiments with 4 NVIDIA A100 GPUs, and the training took approximately 24 hours.

For evaluation, we study two metrics: reward and diversity. To compute the reward, we generate $N=16$ prompts for each initial prompt via beam search with a length penalty of $1.0$. Then, we generate three images per prompt to compute the reward. We aggregate the max score among $N$ prompts and compute the average across initial prompts. 
\begin{align*}
    \text{Reward}(\mathcal{D}_{\text{eval}})&:=\frac{1}{\vert\mathcal{D}_{\text{eval}}\vert}\sum_{\mathbf{x}\in\mathcal{D}_{\text{eval}}}\max_{\mathbf{y}\sim p_{\theta}(\cdot\vert\mathbf{x})}\left(r(\mathbf{x},\mathbf{y})\right)
\end{align*}
To compute diversity, we embed the generated prompts using MiniLMv2 \citep{wang2020minilmv2} encoder, compute the average pairwise cosine distance between embeddings of the prompts, and compute the average across initial prompts.
For all evaluations, we conduct experiments with four random seeds and report the mean and standard deviation.

The input format for both training and evaluation is \texttt{[Initial Prompt] Rephrase:}, following Promptist \citep{hao2024optimizing}. The hyperparameters we used for modeling and training are listed in \Cref{tab:hyperparam}. We conduct several ablations studies on the impact of various hyperparameters in \Cref{app:extend_ablation}.
\begin{table}[!thp]
\caption{
\label{tab:hyperparam}
Hyperparameter configurations for the baseline methods evaluated in our experiments. These settings are used across multiple tasks to assess model performance in low-resource settings, as discussed in Sections~\ref{sec:intro} and Section~\ref{sec:exp}.
}
\resizebox{\textwidth}{!}{
\begin{tabular}{l|l|p{8cm}}
\toprule
\textbf{Baseline} & \textbf{Hyperparameter} & \textbf{Values} \\
\midrule
\multirow{2}{*}{BitFit~\cite{ben-zaken-etal-2022-bitfit}} & Bias Moudule & bias of Q,K and V from attention/bias of LayerNorm from attention outputs/bias of LayerNorm from hidden outputs \\ \cmidrule{2-3}
 & Learning Rate & 1e-4/ 5e-4 \\
\midrule
\multirow{2}{*}{RED~\cite{wu-etal-2024-advancing}} & Rank & 8 / 16 \\
\cmidrule{2-3} 
 & Learning Rate & 5e-5/ 2e-4 / 6e-2 \\
\midrule
REPE~\cite{zou2023representation} & method & Representation Reading / Representation Control \\
\midrule
\multirow{2}{*}{ReFT~\cite{wu2024reft}} & Prefix + suffix posotion & p7 + s7 / p11 + s11 \\
\cmidrule{2-3} 
 & Layers & all / 4,6,10,12,14,18,20,22/3,9,18,24 \\
\midrule
\multirow{2}{*}{LoFIT~\cite{yin2024lofit}} & number of attention heads & 32/64/128 \\
\cmidrule{2-3} 
 & Learning Rate & 5e-4 / 5e-3 \\
\bottomrule
\end{tabular}
}
\end{table}


\subsection{Robustness across Different Reward Functions}
To evaluate the robustness of our framework in terms of different reward functions, we use two widely-used reward functions in diffusion alignment: ImageReward \citep{xu2024imagereward} and HPSv2 \citep{wu2023human}. We provide a detailed description of each function below.
\paragraph{ImageReward} ImageReward is a general-purpose text-to-image preference reward model trained with pairs of prompts and images. To compute the score, ImageReward extracts image and text embeddings using BLIP \cite{li2022blip} encoder and combines them with cross attention, and uses MLP to generate a scale value for preference comparison.
\paragraph{HPSv2} HPSv2 is a scoring model that accurately predicts human preferences on generated images. To accurately predict the score, it fine-tunes the CLIP \citep{radford2021learning} with the HPDv2 dataset, a large-scale dataset that captures human preferences on images from various sources.

\subsection{Transferability to Different Text-to-Image Diffusion Models}
To validate the transferability of our framework to different text-to-image diffusion models, we prepare several widely-used text-to-image diffusion models: SD v1.5 \citep{rombach2022high}, SSD-1B \citep{gupta2024progressive}, SDXL-Turbo \citep{sauer2025adversarial}, and SD3 \citep{esser2024scaling}. As SDXL-Turbo and SD3 models do not align with DPM solver \citep{lu2022dpm}, we use a standard generation pipeline, which uses a PNDM scheduler \citep{liu2022pseudo} with 20 inference steps. Furthermore, as SDXL-Turbo does not use the guidance scale, we set the guidance scale to 0. for SDXL-Turbo, and 7.5 (default) for others. 

\begin{table*}[t]
\centering
\caption{Performance of prompt generated by each method. Aes score indicates aesthetic quality improvement compared to the image generated with the original input. Experiments are conducted with four random seeds, and mean and standard deviation are reported. \textbf{Bold} represent the best entry.}
\vspace{-5pt}
\resizebox{0.95\textwidth}{!}{
\begin{tabular}{lcc|cc|cc|cc}
\toprule
\multirow{3}{*}{\textbf{Method}} & 
\multicolumn{2}{c}{Lexica} & \multicolumn{2}{c}{DiffusionDB} &
\multicolumn{2}{c}{COCO} & \multicolumn{2}{c}{ChatGPT}  \\
\cmidrule{2-9}
% & Aes. & CLIP & Div. & Aes. & CLIP & Div.
% & Aes. & CLIP & Div. & Aes. & CLIP & Div. \\
& Reward & Diversity & Reward & Diversity 
& Reward & Diversity & Reward & Diversity \\
\midrule
Initial Prompt & -0.16 ± 0.03 & - 
               & -0.22 ± 0.02 & -
               & -0.35 ± 0.01 & -
               & -0.42 ± 0.03 & - \\
\midrule
SFT & 0.64 ± 0.02 & 0.13 ± 0.00
    & 0.58 ± 0.01 & 0.13 ± 0.00 
    & 0.54 ± 0.07 & 0.11 ± 0.02
    & 0.76 ± 0.03 & 0.19 ± 0.00 \\
Promptist & 0.76 ± 0.02 & 0.09 ± 0.00
    & 0.76 ± 0.03 & 0.10 ± 0.00
    & 0.65 ± 0.02 & 0.07 ± 0.00 
    & 0.76 ± 0.03 & 0.12 ± 0.00 \\
Rule-Based & 0.72 ± 0.02 & 0.26 ± 0.00
    & 0.69 ± 0.02 & 0.15 ± 0.00 
    & 0.75 ± 0.01 & 0.12 ± 0.00
    & 0.82 ± 0.03 & 0.17 ± 0.00 \\
GFlowNets & 0.96 ± 0.01 & 0.13 ± 0.00
    & 0.85 ± 0.03 & 0.13 ± 0.00
    & 0.73 ± 0.02 & 0.09 ± 0.00
    & 0.63 ± 0.03 & 0.10 ± 0.00 \\
DPO-Diff & 0.13 ± 0.02 & -
    & 0.28 ± 0.06 & -
    & -0.03 ± 0.06 & -
    & -0.17 ± 0.03 & - \\
\midrule
PAG (Ours) & \textbf{0.99 ± 0.05} &  \textbf{0.32 ± 0.00} 
    & \textbf{0.91 ± 0.04} & \textbf{0.33 ± 0.00}
    & \textbf{0.88 ± 0.02} &  \textbf{0.32 ± 0.00} 
    & \textbf{0.88 ± 0.04} & \textbf{0.32 ± 0.00} \\
\bottomrule
\end{tabular}}
\label{tab:main1}
\end{table*}
\begin{figure*}[t]
\begin{minipage}[t]{\textwidth}
    \begin{subfigure}[t]{0.24\textwidth}
        \centering
        \includegraphics[width=\textwidth]{figures/ablation_beta.png}
        \subcaption{Analysis on $\beta$}
        \label{fig:ablation_beta}
    \end{subfigure}
    \begin{subfigure}[t]{0.24\textwidth}
        \centering
        \includegraphics[width=\textwidth]{figures/ablation_M.png}
        \subcaption{Analysis on $M$}
        \label{fig:ablation_M}
    \end{subfigure}
    \begin{subfigure}[t]{0.24\textwidth}
        \centering
        \includegraphics[width=\textwidth]{figures/ablation_lr.png}
        \subcaption{Analysis on $\gamma$ for flow function}
        \label{fig:ablation_gamma}
    \end{subfigure}
    \begin{subfigure}[t]{0.24\textwidth}
        \centering
        \includegraphics[width=\textwidth]{figures/ablation_reset.png}
        \subcaption{Analysis on reset strategies}
        \label{fig:ablation_reset}
    \end{subfigure}
    \caption{Extended ablation studies on various components of PAG.}
    \label{fig:ablation_app}
\end{minipage}
\end{figure*}
\section{Extended Main Results}\label{app:extend_main_results}
In this section, we provide additional discussion and analysis of our main experimental results.
\subsection{Main Results}
We summarize the main experiment results in \Cref{tab:main1} including comparisons with DPO-Diff \citep{wang2024discrete}, a recent relevant work. Note that as DPO-Diff tries to optimize negative prompts and the initial prompt is always the same, it is meaningless to compute diversity between generated prompts.
\subsection{Discussion}
As shown in the table, we observe that DPO-Diff shows modest improvements in terms of the reward compared to other baselines. We find that while DPO-Diff can effectively improve the CLIP scores by optimizing negative prompts, it shows limited capability in improving the aesthetic score.   

\section{Extended Ablation Studies}\label{app:extend_ablation}
In this section, we include additional ablation studies that complement our main text due to space limitations.

\subsection{Analysis on $\beta$}
First, we analyze the effect of inverse temperature $\beta$ in \cref{eq:reward}, which controls the balance between the task reward term $r(\mathbf{x}, \mathbf{y})$ and the reference LM likelihood term $p_{\text{ref}}(\mathbf{y}\vert\mathbf{x})$. As a default setting, we set $\beta=0.05$. 

To analyze the effect of $\beta$, we fine-tune the GFlowNet policy with different $\beta$ values: $\{0.01, 0.05, 0.1, 0.2\}$. As shown in the \Cref{fig:ablation_beta}, there are no significant differences in terms of the performance with different $\beta$ values while using too small $\beta$, which leads to the policy focus on a high-reward region, suffers from mode collapse similar to naive GFlowNet and exhibits poor performance. 

\begin{figure*}[t]
    \centering
    \includegraphics[width=0.9\textwidth]{figures/dpok_comparison.jpg}
    \caption{Comparison with DPOK and PAG. We report the aesthetic score of images in bold.}
    \label{fig:dpok_comparison}
\end{figure*}

\subsection{Analysis on $M$}
We also conduct experiments by varying the flow function reset period ($M$) to analyze the effect of flow reactivation. If we reset the flow function too frequently, it is hard to capture high-rewarding multi-modal distribution, while rarely applying reset leads to the mode collapse issue. As a default setting, we use $M=2000$, which means that we reactivate the flow function four times over the whole training procedure.  

To analyze the effect of $M$, we fine-tune the GFlowNet policy with different $M$ values: $\{1000, 2000, 4000\}$. As shown in the \Cref{fig:ablation_M}, we find that too frequent reactivation ($M=1000$) does not capture high-reward regions and suffers from a significant drop in performance. While there is no big difference between $M=2000$ and $M=4000$, we empirically find that set $M=2000$ achieves the best performance in terms of both reward and diversity. This empirical finding is also aligned with the other papers, which utilize a reset strategy: \citet{nikishin2022primacy} also reset the last layer of the neural networks four times over the course of training.

\subsection{Analysis on learning rate of flow function}
Based on the observation that most actor-critic RL methods use slightly higher learning rates for the critic than the actor \citep{schulman2017proximal}, we use a higher learning rate ($1\times10^{-4}$) for the flow function training than the learning rate of the policy ($1\times10^{-5}$). Using a higher learning rate is also crucial for quickly recovering from the flow reactivation.

To analyze the effect of learning rate ($\gamma$) for the flow function, we fine-tune the GFlowNet policy with different $\gamma$ values: $\{1\times10^{-3}, 1\times10^{-4}, 1\times10^{-5}\}$. As shown in the \Cref{fig:ablation_gamma}, we find that using the same learning rate for the policy and the flow function leads to poor performance as expected. While there is no big difference between $\gamma=1\times10^{-3}$ and $\gamma=1\times10^{-4}$, we empirically find that set $\gamma=1\times10^{-4}$ achieves the best performance in terms of both reward and diversity. 

\subsection{Analysis on flow reactivation scheme}
To prevent a significant drop in performance and unstable training, we employ a targeted reset strategy that resets only the last layer of the flow function. To analyze the effect of our strategy, we conduct experiments with two additional reset strategies: (1) reset the whole layer of the flow function and (2) reset the policy. For resetting the policy, we randomly reset $0.01\%$ of neurons in the policy parameters.

\Cref{fig:ablation_reset} shows the performance of various reset strategies. As depicted in the figure, resetting the whole layer of the flow function does not recover policy towards high-scoring regions. Resetting the policy parameters also exhibits poor performance, as the policy directly affects the acquisition of online samples.

\subsection{Analysis on Diversity of Images}
We also compute the diversity between the final images sampled from text-to-image diffusion models conditioned on prompts generated by different methods. To compute diversity, we compute the average pairwise distance between feature vectors extracted by the pre-trained InceptionV3 model \citep{szegedy2016rethinking}. As shown in \Cref{tab:div_images}, PAG exhibits high diversity on images compared to baselines.
\begin{table}[h]
\centering
\caption{Diversity evaluation on images.}
\vspace{-10pt}
\resizebox{\linewidth}{!}{
\begin{tabular}{lcc|cc}
\toprule
\multirow{2}{*}{\textbf{Method}} & \multicolumn{2}{c}{COCO} & \multicolumn{2}{c}{ChatGPT}  \\
\cmidrule{2-5}
% & Lexica & DiffusionDB & COCO & ChatGPT \\
& Reward & Diversity & Reward & Diversity \\
\midrule
SFT & 0.54 ± 0.07 & 0.20 ± 0.01 & 0.76 ± 0.03 & 0.19 ± 0.01 \\
Promptist & 0.65 ± 0.02 & 0.19 ± 0.01 & 0.76 ± 0.03 & 0.17 ± 0.01 \\
Rule-Based & 0.75 ± 0.01 & 0.19 ± 0.00 & 0.82 ± 0.03 & 0.18 ± 0.01 \\
GFlowNets & 0.73 ± 0.02 & 0.18 ± 0.01 & 0.63 ± 0.03 & 0.16 ± 0.01 \\
\midrule
PAG (Ours) & \textbf{0.88 ± 0.02} & \textbf{0.21 ± 0.01} & \textbf{0.88 ± 0.04} & \textbf{0.20 ± 0.01}\\
\bottomrule
\end{tabular}}
\vspace{-15pt}
\label{tab:div_images}
\end{table}

\section{Comparison with Directly Fine-tuning Diffusion Models}\label{app:comp_diffusion}
In this section, we explain in detail the comparison with directly fine-tuning diffusion models to generate images with desired properties.
\subsection{Experiment Setup}
We strictly follow the experiment setup of DDPO\footnote{https://github.com/jannerm/ddpo} and DPOK\footnote{https://github.com/google-research/google-research/tree/master/dpok} for fine-tuning diffusion models. We fine-tune diffusion models with aesthetic quality as a reward function and use prompts from a list of 45 common animals. 

\subsection{Comparison with DPOK}
We also compare our method with DPOK \citep{fan2024reinforcement}, another representative method for fine-tuning diffusion models with black-box reward functions. As shown in \Cref{fig:dpok_comparison}, generated images from DPOK converge to similar styles, whereas PAG generates diverse and high-quality images.

\section{Additional Visualizations}\label{app:more_visualization}
In this section, we present additional visualizations to show the effectiveness of PAG for text-to-image generation as shown in Figures~\ref{fig:main_figure_type2}-\ref{fig:main_figure_type3} (besides Figure~\ref{fig:main_figure} in the main text). We also summarize the results for robustness across different reward functions and transferability to different text-to-image diffusion models as shown in Figure~\ref{fig:reward_fn}-\ref{fig:transfer}.
\begin{figure*}[t]
    \centering
    \includegraphics[width=0.9\textwidth]{figures/main_figure_type2.jpg}
    \caption{Additional images generated by optimized prompts using Stable Diffusion v1.4. We use the same seed to visualize the effect solely on prompt adaptation.}
    \label{fig:main_figure_type2}
\end{figure*}

\begin{figure*}[t]
    \centering
    \includegraphics[width=0.9\textwidth]{figures/main_figure_type3.jpg}
    \caption{Additional images generated by optimized prompts using Stable Diffusion v1.4. We use the same seed to visualize the effect solely on prompt adaptation.}
    \label{fig:main_figure_type3}
\end{figure*}

\begin{figure*}[t]
    \centering
    \includegraphics[width=0.9\textwidth]{figures/reward_fn.jpg}
    \caption{Images generated with prompts fine-tuned with different reward functions. We use the same seed to visualize the effect solely on prompt adaptation.}
    \label{fig:reward_fn}
\end{figure*}

\begin{figure*}[t]
    \centering
    \includegraphics[width=0.9\textwidth]{figures/transfer.jpg}
    \caption{Images generated with different text-to-image diffusion models. We use the same seed to visualize the effect solely on prompt adaptation.}
    \label{fig:transfer}
\end{figure*}

\end{document}

