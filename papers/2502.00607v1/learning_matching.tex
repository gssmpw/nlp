\documentclass[11pt]{article}
\usepackage[margin=1in]{geometry}
%\usepackage{times}
%\usepackage[compact]{titlesec}
\usepackage{amsfonts}
\usepackage{amsmath,amsthm,amssymb}
\usepackage{latexsym}
\usepackage{epic}
\usepackage{epsfig}
\usepackage{hyperref}
\usepackage{verbatim}
\usepackage[justification=centering]{caption}
\usepackage{enumitem}
\usepackage{color}
\allowdisplaybreaks[1]
\usepackage[numbers]{natbib}
\usepackage{algorithm}
\usepackage{algorithmic}
\usepackage{setspace}




\usepackage{tikz}
\usepackage{graphicx}
\usepackage{tikzscale}
\usepackage{wrapfig}
\usepackage{float}
\usetikzlibrary{arrows.meta,arrows}
\usetikzlibrary{decorations.pathreplacing}
\usetikzlibrary{positioning,chains,fit,shapes,calc}
\usepackage{wrapfig}
\usepackage{pgfplots}
\pgfplotsset{compat = newest}

\usepackage{caption}
\captionsetup{justification=raggedright,singlelinecheck=true}
\usepackage{subcaption}


\usepackage{xspace}




\newcommand{\thought}[1]{{\color[rgb]{0.2,0.39,0.66}(#1)}}
\newcommand{\todo}[1]{{\color[rgb]{1.0,0.0,0.0}(#1)}}
\newcommand{\hsh}[1]{{\color{green!50!black} Henrik: #1}}
\newcommand{\st}[1]{{\color{red!50!black} Sebastian: #1}}

\newcommand{\ulm}[1]{_{\scaleto{\mathrm{#1}}{3pt}}}
\newcommand\at[2]{\left.#1\right|_{#2}}











\newtheorem{assumption}{Assumption}

\DeclareMathOperator*{\argmax}{arg\,max}
\DeclareMathOperator*{\argmin}{arg\,min}

\newcommand{\swname}[1]{\texttt{#1}}
\newcommand{\ie}{i\/.\/e\/.,\/~}
\newcommand{\eg}{e\/.\/g\/.,\/~}
\newcommand{\cf}{cf\/.\/~}

\newcommand{\fig}{Fig\/.\/~}
\newcommand{\defn}{Def\/.\/~}
\newcommand{\sect}{Sec\/.\/~}
\newcommand{\tabl}{Tab\/.\/~}
\newcommand{\algo}{Algorithm~}
\newcommand{\theo}{Theorem~}

\newcommand{\bnnl}{3 hidden layers}
\newcommand{\bnnn}{50 neurons}
\newcommand{\bnna}{tanh activations}

\newcommand{\capt}[1]{\mdseries{\emph{#1}}}

\newcommand{\videolink}{at \url{https://youtu.be/_d7AqTRjz6g}}
\newcommand{\codelink}{\url{https://github.com/wheelbot/mini-wheelbot}}

\newcommand{\fakepar}[1]{\vspace{0mm}\noindent\textbf{#1.}}

\newcommand{\needref}{\textcolor{red}{[REF]}}

\newcommand{\plotfontsize}{9pt}


\newcommand{\tst}{\mathtt{test}}
\newcommand{\ag}{\mathtt{Ag}}
\newcommand{\bp}{\mathtt{Bp}}
\newcommand{\pac}{\mathtt{PAC}}
\newcommand{\tr}{\mathtt{Tr}}
\newcommand{\y}{\bvec{y}}
\newcommand{\x}{\bvec{x}}
\newcommand{\h}{\bvec{h}}
\newcommand{\dist}{\mathtt{dist}}
%\DeclareMathOperator{\tst}{test}
\newcommand{\vcd}{\mathtt{\mbox{VC-Dim}}}
\newcommand{\dsd}{\mathtt{\mbox{DS-Dim}}}


\title{PAC Learning is just Bipartite Matching\\ (Sort of) %\\ EARLY DRAFT (DO NOT DISTRIBUTE)
}

\author{
  Shaddin Dughmi\thanks{Supported by NSF Grant CCF-2009060} \\
Department of Computer Science\\
University of Southern California\\
{\tt shaddin@usc.edu}
}








\begin{document}

\maketitle

%\begin{abstract}
%\begin{abstract}
Retrieval-Augmented Generation (RAG) is often used with Large Language Models (LLMs) to infuse domain knowledge or user-specific information. In RAG, given a user query, a retriever extracts chunks of relevant text from a knowledge base. These chunks are sent to an LLM as part of the input prompt. Typically, any given chunk is repeatedly retrieved across user questions. However, currently, for every question, attention-layers in LLMs fully compute the key values (KVs) repeatedly for the input chunks, as state-of-the-art methods cannot reuse KV-caches when chunks appear at arbitrary locations with arbitrary contexts. Naive reuse leads to output quality degradation.  This leads to potentially redundant computations on expensive GPUs and increases latency. In this work, we propose \sys, a system for managing and reusing precomputed KVs corresponding to the text chunks (we call \textit{chunk-caches}) in RAG-based systems. We present how to identify \hl{\textit{chunk-caches} that are reusable}, how to efficiently perform a small fraction of recomputation to \textit{fix} the cache to maintain output quality, and how to efficiently store and evict \textit{chunk-caches} in the hardware for maximizing reuse while masking any overheads. With real production workloads as well as synthetic datasets, we show that \sys reduces redundant computation by \textbf{51\%} over SOTA prefix-caching and \textbf{75\%} over full recomputation.
\hl{Additionally, with continuous batching on a real production workload, we get a \textbf{1.6$\times$} speedup in throughput and a \textbf{2$\times$} reduction in end-to-end response latency over prefix-caching while maintaining quality, for both the \llama-3-8B and \llama-3-70B models. 
}
\end{abstract}





%\end{abstract}

\section{Introduction}
\label{sec:intro}

\begin{figure*}[tb]
    \centering
    \includegraphics[width=0.848\linewidth]{figs/circuitnn.pdf} 
    \caption{Illustration of differentiable CircuitNN. CircuitNN is designed based on differentiable NAND gates. After DAS is guided by PI and PO pairs of the truth table, CircuitNN can get the precise circuit architecture logic equivalent to the truth table.}
    \label{fig:circuitnn}
\end{figure*}

% 1. Describe the importance of logic synthesis
% 2. Existing Problems
% (a) Neural Architecture Search: Unstable, Predefined Setting, etc.
% (b) Circuit Generation: Probabilistic Model, Logic Equivalence

With the rapid advancement of technology, the scale of integrated circuits (ICs) has expanded exponentially. 
This expansion has introduced significant challenges in chip manufacturing, particularly concerning power and area metrics.
A primary objective in IC design is achieving the same circuit function with fewer transistors, thereby reducing power usage and area occupancy.

Logic synthesis~\cite{hachtel2005logicsynth}, a critical step in electronic design automation (EDA), transforms behavioral-level circuit designs into optimized gate-level circuits, ultimately yielding the final IC layout. 
The primary goal of logic synthesis is to identify the physical implementation with the fewest gates for a given circuit function. 
This task constitutes a challenging NP-hard combinatorial optimization problem. 
Current logic synthesis tools~\cite{brayton2010abc, wolf2013yosys} rely on human-designed heuristics, often leading to sub-optimal outcomes.

Differentiable architecture search (DAS) techniques~\cite{liu2018darts, chu2020darts} offer novel perspectives on addressing challenges in this problem.
Circuit functions can be represented through truth tables, which map binary inputs to their corresponding outputs. 
Truth tables provide a precise representation of input-output relationships, ensuring the design of functionally equivalent circuits.
Inspired by this, researchers~\cite{deepmind2024ai4sys, wang2024tnet} have begun exploring the application of DAS to synthesize circuits directly from truth tables.
Specifically, \citet{deepmind2024ai4sys} proposed CircuitNN, a framework that learns differentiable connection structures with logic gates, enabling the automatic generation of logic circuits from truth tables.
This approach significantly reduces the complexity of traditional circuit generation. 
Building on this, \citet{wang2024tnet} introduced T-Net, a triangle-shaped variant of CircuitNN, incorporating regularization techniques to enhance the efficiency of DAS.

Despite these advancements, several challenges remain. 
The computational complexity of DAS grows quadratically with the number of gates, posing scalability issues.
Although triangle-shaped architecture~\cite{wang2024tnet} partially mitigates this problem, redundancy persists. 
%Additionally, DAS is susceptible to converging to local optima, limiting the ability to search architectures that satisfy the given truth tables~\cite{liu2018darts}. 
%Furthermore, hyperparameters (network depth and layer width) require extensive searches, introducing complexity and prolonging the synthesis process. 
Additionally, DAS is susceptible to converging to local optima~\cite{liu2018darts} and hyperparameters (network depth and layer width) require extensive searches. 
The challenges arise from the vast search space in DAS. 
% Even with predefined settings for CircuitNN, finding a configuration that meets the truth table requires extensive trial and error during the DAS process. 
Intuitively, limiting the search space through predefined parameters (network depth, gates per layer, and connection probabilities) can significantly reduce the complexity.

Recent advances~\cite{openai2023gpt4, abramson2024alphafold3, esser2024sd3, li2024mar} in conditional generative models have demonstrated remarkable performance across language, vision, and graph generation tasks. 
Motivated by these developments, we propose a novel approach to circuit generation that generates preliminary circuit structures to guide DAS in generating refined circuits matching specified truth tables. 
Firstly, we introduce CircuitVQ, a tokenizer with a discrete codebook for circuit tokenization. 
Built upon our Circuit AutoEncoder framework~\cite{hou2022graphmae,li2023maskgae,wu2025mgvga}, CircuitVQ is trained through a circuit reconstruction task. 
Specifically, the CircuitVQ encoder encodes input circuits into discrete tokens using a learnable codebook, while the decoder reconstructs the circuit adjacency matrix based on these tokens.
Subsequently, the CircuitVQ encoder serves as a circuit tokenizer for CircuitAR pretraining, which employs a masked autoregressive modeling paradigm~\cite{chang2022maskgit, li2023mage}. 
In this process, the discrete codes function as supervision signals. 
After training, CircuitAR can generate discrete tokens progressively, which can be decoded into initial circuit structures by the decoder of the CircuitVQ. 
These prior insights can guide DAS in producing refined circuits that match the target truth tables precisely.

Our key contributions can be summarized as follows:
\begin{itemize}
\item We introduce CircuitVQ, a circuit tokenizer that facilitates graph autoregressive modeling for circuit generation, based on our Circuit AutoEncoder framework;
\item Develop CircuitAR, a model trained using masked autoregressive modeling, which generates initial circuit structures conditioned on given truth tables;
\item Propose a refinement framework that integrates differentiable architecture search to produce functionally equivalent circuits guided by target truth tables;
\item Comprehensive experiments demonstrating the scalability and capability emergence of our CircuitAR and the superior performance of the proposed circuit generation approach.
\end{itemize}

% Motivation
% (a) Diffusion (Vision, Graph), Autoregressive (Language, Vision)
% (b) Circuit Generation for Predefined Setting
% (c) Neural Architecture Search for Strict Logic Equivalence

% Contribution
% (a) Circuit Tokenizer (new transformer arch, training strategy)
% (b) CircuitAR (train and gen strategies, post-ar strategy)
% (c) Extensive Evaluation including BitD (Bit Distance) for Scalability

\section{Basic Background: Supervised Learning and the PAC Model}
\label{sec:background}

At this point almost everyone has heard of machine learning (ML). Anyone likely to stumble upon this article will have also heard of its most influential special case, supervised learning, and those theoretically inclined will also be familiar with the PAC model. Nonetheless, I will set the stage by  recapping the basics.

\subsection{Basics of Supervised Learning}%Let's set the stage in any case

\emph{Supervised Learning} is the task of ``coming up'' with a function $f: \X \to \Y$ to ``explain'' or ``fit'' a sequence of input/output examples   $(x_1,y_1), \ldots, (x_n,y_n)$, with $x_i \in \X$ and $y_i \in \Y$.  Here $\X$ is a \emph{data domain} consisting of \emph{datapoints} $x \in \X$, $\Y$ is a \emph{label set} consisting of \emph{labels} $y \in \Y$, and the sequence $(x_1,y_1),\ldots,(x_n,y_n)$ is the \emph{training data} consisting of \emph{labeled examples (a.k.a. samples)}~$(x_i,y_i)$.  I~will refer to the chosen function $f$ as a \emph{predictor}, and to $n$ as the \emph{sample size}. A \emph{learning algorithm} takes as input training data, and outputs (some representation of) a predictor $f \in \Y^\X$.\footnote{Note that this describes the usual \emph{batch}, a.k.a.~\emph{offline}, setting of supervised learning. I do not discuss other paradigms such as online or active learning in this article.} 



Success in supervised learning is defined as \emph{generalization} to  future examples: For a typical \emph{test example}  $(x_{\tst},y_{\tst})$, the predicted label $y'_{\tst}=f(x_{\tst})$ should ``equal'' $y_{\tst}$, perhaps approximately. We usually assume the test example is drawn from the same  ``source'' as the training data  --- commonly, i.i.d.~from the same distribution. The quality of the prediction is quantified by $\ell(y'_{\tst},y_{\tst})$, where $\ell:~\Y~\times~\Y \to \RR_{\geq 0}$ is a \emph{loss function} chosen as part of the problem definition. Common loss functions include the 0-1 loss $\ell_{0-1}(y',y) = [y' \neq y]$ for \emph{classification} problems,\footnote{The notation $[P]$ denotes $1$ when predicate $P$ is true, and denotes $0$ when $P$ is false.} as well as the absolute loss $|y'-y|$ or squared loss $(y'-y)^2$ for \emph{regression problems} featuring $\Y  \sse \RR$.

Nontrivial generalization properties are typically only possible if one assumes something about the data.\footnote{The need for such an assumption is formalized by the  \emph{no free lunch theorems} of supervised learning \cite{wolpert_connection_1992,wolpert_lack_1996,schaffer_conservation_1994}.} The Bayesian approach to  machine learning, common in many applications, assumes some parametric form for the distribution generating the data, and postulates a prior on the parameters. This is not the approach I will take in this article. Instead, I will focus on the frequentist --- and some would say ``worst-case'' or ``adversarial'' ---  approach that is common in the computational learning theory community, embodied by the PAC model. Here we assume that the (training and test) data can be explained, perhaps approximately, by a function in some ``simple enough to learn'' class of functions $\H \sse \Y^\X$, often called the \emph{hypotheses}. Equivalently, we  seek a predictor which explains the unseen data roughly  as well as the best hypothesis $h^* \in \H$, whether or not we assume that $h^*$ itself provides a perfect explanation.



 \paragraph{Common Algorithmic Templates.} Perhaps the best known general-purpose supervised learning algorithm is \emph{empirical risk minimization (ERM)}, which chooses as its predictor a hypothesis $f \in \H$ minimizing $\frac{1}{n} \sum_{i=1}^n \ell(f(x_i),y_i)$ --- a quantity called the \emph{training error}, \emph{empirical error}, or \emph{empirical risk} of $f$. %\footnote{When multiple hypotheses minimize the empirical risk, we assume ERM breaks ties arbitrarily.}
A common template for generalizing ERM involves adding a \emph{regularization term} $\psi(f)$ to the  objective function, typically chosen to measure some notion of ``hypothesis complexity.'' An algorithm instantiating this template is known as a \emph{structural risk minimizer (SRM)}, and chooses as its predictor the hypothesis $f \in \H$ minimizing the \emph{structural risk} $\frac{1}{n} \sum_{i=1}^n \ell(f(x_i),y_i) + \psi(f)$. Other well-known algorithms, such as gradient descent and its variations,  can frequently be interpreted as approximate implementations of ERM or SRM.


\paragraph{Proper vs Improper Learning.} A learning algorithm is said to be \emph{proper} if its predictor $f$ is always chosen from the hypothesis class, i.e., $f \in \H$, otherwise it is said to be \emph{improper}. ERM  is an example of a proper learning algorithm, as are SRM algorithms of the form described above.  In the \emph{proper regime} of learning, algorithms are required to be proper. This article will be concerned with the more flexible \emph{improper regime} (a.k.a \emph{representation-independent learning}), where no such constraint is placed on the learner. In other words, all we care about is predictive power at test time, rather than any insights derived from the functional form or representation of the predictor~itself.


\subsection{The PAC Model}
A standard mathematical setup for evaluation of supervised learning algorithms, at least in the theoretical computer science community, is Valiant's \emph{Probably Approximately Correct (PAC) model} of learning (see e.g.~\cite{kearns_introduction_1994,mohri_foundations_2018}). Here, we assume there is an unknown distribution $\D$ on $\X \times \Y$ from which training and test data are  drawn.  Specifically, the labeled datapoints of the training set  $(x_1,y_1), \ldots, (x_n,y_n)$, as well as the test data  $(x_\tst,y_\tst)$, are i.i.d.~from $\D$. Often it is assumed that $\D$ lies in some class of distributions of interest. The \emph{true expected loss}, or simply \emph{loss}, of a predictor $f: \X \to \Y$ is the expected loss it incurs on draws from $\D$, written $L_\D(f) = \Ex_{(x,y) \sim \D} \ell(f(x),y)$.


There are two main ``settings'' in PAC learning. The  \emph{realizable setting} only requires that the data be perfectly explained by some hypothesis in $\H$. More generally, the \emph{agnostic setting} makes no assumption relating the data to the hypotheses, but shifts the goalposts as necessary to allow nontrivial guarantees: the expected loss at test time is evaluated only ``relative'' to that of the best hypothesis $h^* \in \H$. There are other settings which make more nuanced assumptions, such as $\D$ being of a particular parametric form or its support living in some (unknown) lower-dimensional space, etc. I will mostly discuss the realizable and agnostic settings in this article, those being the simplest and most studied from a theoretical perspective. %TODO:We will briefly discuss other settings in Section ??

The PAC model demands high probability guarantees of learners, in the worst case over distributions of interest. Consider first the realizable setting, where $\D$ is such that $\min_{h \in \H} L_{\D}(h) = 0$. A PAC learner has \emph{error} $\epsilon=\epsilon(n)$ and \emph{confidence} $\delta=\delta(n)$ if, when training data consists of $n$ i.i.d~samples from a realizable distribution $\D$, it produces a predictor $f$  satisfying $L_\D(f) \leq \epsilon$ with probability at least $1-\delta$. In the agnostic setting, where $\D$ can be arbitrary, we require $L_\D(f) - \min_{h \in \H} L_\D(h) \leq \epsilon$ with probability $1-\delta$.

In both the realizable and agnostic settings, we look for PAC learners with small $\epsilon$ and $\delta$ as a function of the sample size $n$. An equivalent perspective looks at the sample complexity $m(\epsilon,\delta)$, which is the minimum sample size which guarantees error  at most $\epsilon$ with probability at least $1-\delta$. We say a problem is \emph{PAC learnable} if its PAC sample complexity is finite whenever $\epsilon,\delta > 0$.

For most PAC learning problems, learnability and sample complexity are characterized in terms of a  ``dimension'' of the hypothesis class. Most prominently this is the \emph{VC dimension} for binary classification, the \emph{fat shattering dimension} for agnostic regression, and the \emph{DS dimension} for multiclass classification (see \cite{anthony_neural_1999,daniely_optimal_2014,brukhim_characterization_2022}). Treatment of these is beyond the scope of this article. The unfamiliar reader need not worry, however,  as dimensions will feature only tangentially in our~discussion.




%\paragraph{Learning settings: Realizable, Agnostic, etc.} In learning theory, evaluating a supervised learning algorithm requires specifying a data model and an objective. We will leave the details of the data model flexible for now, to allow for both the PAC model and the adversarial transductive model. Nonetheless we will describe two variations, which we call ``settings'', which cut across different models. The  \emph{realizable setting}  requires only that the data be perfectly explained by some hypothesis $h \in \H$ --- i.e., there exists a hypothesis which is guaranteed to suffer a loss of $0$ on training and test data. The performance of the learning algorithm is its expected loss at test time for some ``worst case'' realizable instance. More generally, the \emph{agnostic setting} makes no assumption relating the data to the hypotheses, but shifts the goalposts as necessary to allow nontrivial guarantees: the expected loss at test time is evaluated only ``relative'' to that of the best hypothesis $h^* \in \H$, again for some ``worst case'' instance. There are other settings which make more nuanced assumptions about the data, such as it is drawn from a distribution of a particular parametric form, or that it lives in some (unknown) lower-dimensional space, etc. We will mostly discuss the realizable and agnostic settings, those being the simplest and most studied from a theoretical perspective.




%%% Local Variables:
%%% mode: latex
%%% TeX-master: "learning_matching"
%%% End:

\section{A Transductive Model of Learning, and why it's Good Enough}
\label{sec:trans}

I will now present the learning model that will be our main playground in this article, then try to convince you that we lose little by restricting our attention to it. This model was first employed by \citet{haussler_predicting_1994}, and has been quite busy in recent years, with its learners used as precursors to PAC learners (e.g. \cite{daniely_optimal_2014,brukhim_characterization_2022,aden-ali_optimal_2023,daskalakis_is_2024,asilis_regularization_2024,montasser_adversarially_2022}).
%
I will describe the model as conceptualized by \citet{daniely_optimal_2014} and further developed by  \citet{asilis_regularization_2024}, and refer to it simply as \emph{the transductive model}  in keeping with some of the recent literature.   % I will call this ``the single-prediction adversarial transductive model'', or simply  ``the transductive model'' as it is referred to in recent literature.
%The transductive model  has been quite busy in recent years, with its learners used as precursors to PAC learners with various guarantees (e.g. \cite{daniely_optimal_2014,brukhim_characterization_2022,aden-ali_optimal_2023,daskalakis_is_2024,asilis_regularization_2024,montasser_adversarially_2022}).
%
Truth be told, however,  the transductive approach to learning --- first explicitly articulated by \citet[Chapter 6.1]{vapnik_nature_1998} --- is much broader than just this particular model. Most notably, the transductive model I use in this article is concerned with making only a single prediction, and presumes data is chosen by a particular strong adversary. We will come back to transduction more generally later.   %\footnote{One could argue that a more appropriate moniker would be ``the single-prediction adversarial transductive model'', to place it within the broader transductive approach  first explicitly articulated by \citet[Chapter 6.1]{vapnik_nature_1998}.}

\subsection{The Transductive Model}
The (single-prediction, adversarial) transductive model is  a game of ``fill in the blank'' played between a learner and an adversary. For a given positive integer $n$, the game proceeds as follows:
\begin{enumerate}
\item The adversary chooses a \emph{labeled dataset} $(x_1,y_1), \ldots,(x_n,y_n)$, with $x_i \in \X$ and $y_i \in \Y$. \\(The adversary may  be constrained in their choice of dataset --- more on this shortly.)
\item One label $y_i$, chosen uniformly at random, is hidden. The remaining data \[(x_1,y_1),\ldots,(x_{i-1},y_{i-1}),(x_i,?),(x_{i+1},y_{i+1}),\ldots,(x_n,y_n),\] where ``?'' is a ``blank'' symbol obscuring $y_i$, is displayed to the learner. 
%\item One example $(x_i,y_i)$ is chosen uniformly at random. The remaining examples $T_{-i} =(x_1,y_1),\ldots,(x_{i-1},y_{i-1}),(x_{i+1},y_{i+1}),\ldots,(x_n,y_n)$ are displayed to the learner. 
\item The learner is prompted to make a prediction $y'_i$ for the label of $x_i$, i.e., to ``fill in the blank'', incurring loss $\ell(y'_i,y_i)$.
\end{enumerate}
%TODO: Figure with cats and blank and such

The adversary in the transductive model chooses the training and test data, but cannot control which is which: the identity of the test example $(x_i,y_i)$ is chosen uniformly at random. Then, the learner is fed training data   $\set{(x_j,y_j) : j \neq i}$ and prompted with the test datapoint $x_i$. In relation to  PAC learning, where training and test data are sourced i.i.d.~from some distribution, the transductive model can be viewed as ``hard-coding'' a uniform distribution over $n$ examples then drawing training and test data \emph{without replacement}.






We will distinguish two settings of the transductive model, \emph{realizable} and \emph{agnostic}, analogous to the same in the PAC model. The transductive model was originally considered in the realizable setting~\cite{haussler_predicting_1994,daniely_optimal_2014}, where the adversary is constrained to instances $(x_1,y_1), \ldots, (x_n,y_n)$ which are perfectly explained by a some hypothesis $h \in \H$, meaning $h(x_j) = y_j$ for all examples. The (realizable) \emph{transductive error}  of a learner on datasets of size $n$ is simply the expected loss it incurs in this game, i.e., for a worst-case realizable instance $(x_1,y_1), \ldots, (x_n,y_n)$ chosen by the adversary. Here, the expectation is over the random identity of the test example $(x_i,y_i)$ as well as any internal randomness to the learner. In the agnostic setting of this model, as articulated in~\cite{asilis_regularization_2024}, the adversary is completely unconstrained in their choice of instance $(x_1,y_1), \ldots, (x_n,y_n)$. We then shift the goalposts by subtracting the best-in-class loss on the same instance, as necessary to allow for nontrivial  learning: the \emph{agnostic transductive error} is the expected loss incurred by the learner minus $\min_{h \in \H} \frac{1}{n} \sum_{i=1}^n \ell(h(x_i),y_i)$.

Whether in the realizable or agnostic setting, we look for learners with small transductive error rate $\epsilon(n)$, as a function of the dataset size $n$. An equivalent perspective looks at the sample complexity $m(\epsilon)$, which is the minimum dataset size guaranteeing  error at most $\epsilon$. We say a problem is \emph{learnable} in the transductive model if its transductive sample complexity is finite for every $\epsilon > 0$.




\begin{figure}%[tbp]
  \centering
    \begin{subfigure}{0.3\textwidth}
        \centering
        \scalebox{0.8}{\input{figures/rectangles.tikz}}
        \caption{Labeled dataset consistent with an axis-aligned rectangle.}
        \label{fig:rectangles_a}
    \end{subfigure}
   \hfill % Optional: this inserts space between the two figures
    \begin{subfigure}{0.3\textwidth}
        \centering
        \scalebox{0.8}{\input{figures/rectangles_b.tikz}}
        \caption{When any label is omitted, both + and - are consistent with some rectangle.}
        \label{fig:rectangles_b}
      \end{subfigure}
      \hfill
    \begin{subfigure}{0.3\textwidth}
      \centering
      \scalebox{0.8}{\input{figures/rectangles_c.tikz}}
      \caption{The minimal axis-aligned rectangle is sensitive only to the four + points defining it.}
      \label{fig:rectangles_c}
    \end{subfigure}
    \caption{Realizable binary classification for axis aligned rectangles in the transductive model.}
    \label{fig:rectangles}
\end{figure}

\paragraph{Example: Axis-aligned Rectangles.} To illustrate the transductive model, consider realizable binary classification of axis-aligned rectangles as described in~\cite{haussler_predicting_1994} and illustrated in Figure~\ref{fig:rectangles}. The adversary selects $n$ datapoints in Euclidean space $\RR^d$, and labels them  in a manner consistent with some axis-aligned rectangle as in Figure~\ref{fig:rectangles_a}. A random label is then omitted, and the learner is prompted to predict this label. A learner which merely minimizes empirical risk --- i.e., arbitrarily chooses an axis-aligned rectangle which is consistent with observed labels, and fills in the missing label accordingly --- can make a mistake with probability $1$. Such a scenario is illustrated in Figure~\ref{fig:rectangles_b}. This stands in contrast to the PAC model, where empirical risk minimization guarantees a misclassification rate of $O(d/n)$ with high probability, since axis-aligned rectangles in $\RR^d$ have a VC dimension of $2d$. A more careful choice of learning rule guarantees an error probability of $\frac{2d}{n}$ also in the transductive model: choose the \emph{smallest} axis-aligned rectangle consistent with the observed labels, and fill in the missing label accordingly. This is illustrated in Figure~\ref{fig:rectangles_c}.



\paragraph{Analogy to Hat Puzzles.} The reader familiar with \emph{hat puzzles} (see e.g. \cite{krzywkowski_hat_2010,butler_hat_2009}) might notice a similarity to the transductive model. There are many variations of hat puzzles, the most relevant to us of which look something like this: There are $n$ players and a supply of hats of $k$ different colors. An adversary places a hat on the head of each of the players. Each player must then guess the color of their own hat by looking at all, or in some formulations some, of the hats of the other players. The players formulate a joint strategy in advance, and a typical goal is to maximize the number of correct guesses.

When the transductive model is applied to classification in the realizable setting, the  analogy to hat puzzles is clear: the players are $x_1,\ldots,x_n$, the hat colors are $y_1,\ldots,y_n$, and player $i$'s guessing task corresponds to the scenario in which $y_i$ is hidden. The learner  therefore corresponds to the joint strategy of the players, and transductive error is proportional to the number of players guessing incorrectly. 

There are important differences between hat puzzles and the transductive model, however. To my knowledge, most hat puzzles permit the adversary to assign hat colors either arbitrarily or uniformly at random. The transductive model, in contrast, is constrained to the hat assignments permitted by the hypothesis class (in the realizable setting, at least). Another important difference is that hat puzzles are commonly concerned with deterministic guessing strategies.  This is because the optimal randomized strategy against such a powerful adversary is the trivial one: each player randomly guesses their hat color. Obtaining a similar guarantee with a deterministic strategy is therefore the more interesting puzzle here. %trivial randomized strategy whereby each player randomly guesses their strategy of randomly guessing the hat color is optimal since the optimal randomized strategies are trivial given the power of the unconstrained adversary.

To my knowledge, the analogy between hat puzzles and transductive learning appears completely absent from the literature. Despite their differences, there might yet be fruitful connections between the two areas. One tantalizing tidbit is that the hypercube  perspective on hat puzzles articulated by \citet{butler_hat_2009} appears to be a special case of the \emph{one-inclusion graphs} that are almost synonymous with the transductive model --- more on these graphs in Section~\ref{sec:class}.



\paragraph{A Note on Transduction vs Induction, more generally.} The usual  conception of supervised learning is as a problem of \emph{inductive inference}: reasoning from particular observations (the training data) to a general rule (the chosen predictor), which can then be applied to answer specific questions at test time (labeling the test data). A (seemingly) more permissive approach is to care not for the rule itself, but for its evaluation at the particular points of interest; i.e.,  we reason from particular observations (the training data) to answer specific questions (labeling the test data). This is \emph{transductive learning} as first explicitly articulated by~\citet{vapnik_nature_1998}, though the essential ideas date back to \citet{vapnik_theory_1974} and \citet{vapnik_estimation_1982}. Notice that this general transductive approach does not presuppose a particular data model or objective, and can be considered in other setups such as the PAC model, online learning, etc.

When there is only a single test point, and learners are allowed to be improper, induction and transduction are essentially equivalent. Indeed,  the formal distinction between the two becomes simply a matter of \emph{currying}\footnote{Currying is a notion common in functional programming, and amounts to the observation that a function \linebreak $g: A \times B \to \C$ can equivalently be viewed as a function $g': A \to (B \to C)$, with $g'(a)$ itself a function from $B$ to $C$.}  --- see the related discussion in~\cite{montasser_transductive_2022}. This equivalence no longer holds in general when multiple test datapoints must be labeled, nor in the proper regime. Since we operate in the improper regime with a single test example, the distinction between transduction and induction becomes merely a matter of perspective for our purposes. Nonetheless, the  transductive model is best thought of as reasoning for a specific test example, hence the name.
% TODO: Say something about how transductive and improper are essentially synonyms as far as we're concerned. Transductive is a perspective that enables improperness by default.
%By catering the learned rule to a specific test example, transduction can be viewed as ``defaulting'' to improper learners.

A related perspective on the distinction between transduction and induction  is the following. By catering the learned predictor to a specific test example, transductive learners are ``improper by default'' when viewed across all possible test points. Indeed, by focusing only on prediction and freeing the learner completely from representing any general rule, the transductive perspective unleashes the full power of improper learning. Induction, in contrast, is most meaningful when encoding an a-priori bias towards rules of a particular form, as is the case in the proper regime.



\subsection{Relationship to the PAC Model}
\label{subsec:pac_trans}
Despite seeming  incomparable at first glance, the PAC and transductive models turn out to be closely related.\footnote{This is specific to the  improper (i.e., unrestricted) regime of PAC learning, which is our focus in this article. The transductive model, in allowing improperness by default, is more permissive than proper PAC learning~\cite{daniely_optimal_2014}.} Speaking qualitatively, learnability is equivalent in the two models, and this holds for both realizable and agnostic learning problems with bounded loss functions.

Quantitatively, we can say something quite strong in the realizable setting with bounded losses: The two models are equivalent up to  low-order factors in the error and the sample complexity. This follows from  efficient black-box reductions  between the two models, one in each direction. We therefore lose very little, and gain quite a lot through the connection to matching and otherwise, by studying the transductive model instead of the PAC model for realizable learning problems.

The exact quantitative relationship between the two models  is more unsettled in the agnostic setting: Whereas it appears that PAC learning is essentially  no harder  transductive learning (up to low-order terms, for most natural loss functions, through an efficient black-box reduction), it remains plausible that transductive sample complexity exceeds its PAC counterpart by a multiplicative factor on the order of $1/\epsilon$, even for classification problems.
%
This gap of $1/\epsilon$ in agnostic sample complexities is the main asterisk to the stance I take in this article of treating the PAC and transductive models as interchangeable. Given  the transductive model's  fruitful connections to basic graph-theoretic questions, as well as the possibility that future work could eliminate this gap, I hope the reader finds this qualification forgivable.

I will now recap both the realizable and agnostic states of affairs in more detail.



\subsubsection*{The Realizable Setting}
  In the realizable setting with losses in $[0,1]$, the transductive and PAC models are equivalent up to  low-order factors in the error and the sample complexity. This is somewhat folklore knowledge in the field, but the proofs are written down explicitly in~\cite{asilis_regularization_2024}.  I'll recap those proofs informally next.

For one direction of the equivalence, a realizable PAC learner guaranteeing error $\epsilon$ with probability $1-\delta$ on $n$ samples can be converted to a realizable transductive learner guaranteeing error $O(\epsilon + \delta)$ on datasets consisting of $n$ samples. This is done in pretty much the obvious way: the PAC learner is trained on uniform draws from the transductive dataset, and invoked to make a prediction for the test point. Since there is a constant probability the test point is not in the training data, the PAC learner solves a strictly harder problem with constant probability, from which the bound follows by Markov's inequality.

For the other direction, a realizable transductive learner guaranteeing error $\epsilon$ on datasets of size $n$ can be  converted to a realizable PAC learner which guarantees error $O(\epsilon)$ with probability $1-\delta$ when given $O(n \log \frac{1}{\delta})$ samples. This again is a pretty elementary reduction: the transductive learner guarantees expected loss $\epsilon$ on a PAC instance via a \emph{leave one out} argument --- essentially conditioning on the training and test data and invoking the principle of deferred decisions with regards to which is which. This can then be boosted to a high probability guarantee by repetition $O(\log \frac{1}{\delta})$ times. Markov's inequality and the union bound show that one of the predictors has loss $O(\epsilon)$ with probability $1-\delta/2$, and a small hold-out set of examples can be used to identify such a predictor with probability $1-\delta/2$.

For the large class of metric loss functions, a more sophisticated  reduction gets rid of the multiplicative blowup in sample complexity \cite{aden-ali_optimal_2023,dughmi_is_2024}. Specifically, the number of samples is reduced from $O\left(n \log \frac{1}{\delta}\right)$ to $n+O\left(\frac{1}{\epsilon} \log \frac{1}{\delta}\right)$. This is possible through a more economical repetition construction which reuses examples across multiple invocations of the transductive learner.


\subsection*{The Agnostic Setting}
In relating the transductive and PAC models in the agnostic setting, the best we can currently say for general bounded loss functions is the following:  The previously-described realizable reductions go through with a degraded bound on the number of samples, by a factor of at most $O(1/\epsilon)$. To see why this might be, note that both those reductions rely on Markov's inequality, which guarantees that a (random) predictor with expected loss $\epsilon$ has loss at most $2 \epsilon$ with probability at least $1/2$. Agnostic error discounts the loss of a predictor by the best-in-class loss, and in doing so no longer enjoys the same Markov guarantee. Indeed, when $\epsilon$ denotes this discounted expected loss, the probability guarantee provided by Markov's inequality degrades from $1/2$ to $O(\epsilon)$. Carrying this through either reduction increases the sample complexity bound by a factor of~$O(1/\epsilon)$.

This multiplicative blowup of $1/\epsilon$ in sample complexity seems somewhat more fundamental in one of the directions than the other.  For the large class of metric loss functions, a transductive learner guaranteeing agnostic error $\epsilon$ on datasets of size $n$ can be converted to a PAC learner guaranteeing agnostic error $O(\epsilon)$ with probability $1-\delta$ when given $n+O\left(\frac{1}{\epsilon^2} \log \frac{1}{\epsilon\delta}\right)$ samples. This is through an adaptation of the economical repetition construction of \citet{aden-ali_optimal_2023} to the agnostic setting by  \citet{dughmi_is_2024}.

This suggests that PAC learning is essentially no harder than transductive learning for most natural loss functions. Whether the converse is true remains very much in question,  as discussed and explored in \cite{dughmi_is_2024}. Specifically, it remains plausible that transductive sample complexity exceeds PAC sample complexity by a factor of up to $O(1/\epsilon)$, even for multiclass classification. Such a gap in sample complexity was ruled out for binary classification through a non-reduction approach in~\cite{dughmi_is_2024}, where we also conjecture that the gap can be  eliminated more generally.


%The equivalence is perhaps less tight for the agnostic setting, with the naive reductions suffering a blowup of $O\left(\frac{1}{\epsilon}\right)$ in sample complexity --- whether this is avoidable will be tackled  in  Section~\ref{sec:structural}. Nonetheless, learnability is equivalent for the PAC and transductive models, and the sample complexities are related through efficient reductions. We therefore often blur the distinction between the two models in this proposal.







%%% Local Variables:
%%% mode: latex
%%% TeX-master: "learning_matching"
%%% End:

\section{Classification, One-Inclusion Graphs, and Bipartite Matching}
\label{sec:class}

A \emph{classification problem} is a supervised learning problem with the 0-1 loss function $\ell_{0-1}(y',y)=[y' \neq y]$, which evaluates to $1$ when $y' \neq y$ and to $0$ when $y'=y$. It is important to note that this allows for an arbitrary domain $\X$, an arbitrary label set $\Y$, and an arbitrary hypothesis class $\H \sse \Y^\X$.  Classification therefore forms a rich and expressive class of problems,  garnering the lion's share of theoretical work on supervised learning (particularly in the PAC model), and featuring in essentially every application domain of machine learning. %. Classification problems are also prolific in practice, appearing in essentially every application domain of machine learning.
%
When there are only two labels, canonically $\Y= \set{0,1}$, we have a \emph{binary classification} problem.  More generally, \emph{multiclass classification} problems can have any number of labels, even infinitely many of any cardinality.

I will discuss classification problems in the transductive model, emboldened by its close relationship to the PAC model as described in Section~\ref{sec:trans}. I will start with a primer on one-inclusion graphs (OIGs), which encode transductive classification exactly  as a combinatorial optimization problem. I will then reinterpret this problem as bipartite matching, and discuss some recently-discovered implications of this interpretation.

It's worth noting that the transductive perspective is of perhaps limited utility for binary classification, where empirical risk minimization (ERM) has almost optimal PAC sample complexity~\cite{shalev-shwartz_understanding_2014}. It is in multiclass classification where transductive learners truly shine, in particular as the number of labels grows large. In fact, ERM can fail to learn entirely even for learnable multiclass problems~\cite{daniely_multiclass_2011}, as can SRM and any proper learner \cite{daniely_optimal_2014}.  To my knowledge, all known general-purpose learners for multiclass classification factor through the transductive model and  OIGs~\cite{daniely_optimal_2014, brukhim_characterization_2022,aden-ali_optimal_2023,asilis_regularization_2024}.





\subsection{One-Inclusion Graphs}
\label{sec:oig}
% Though transduction is most useful for multiclass problems, we start with binary classification for illustrative purposes.

The \emph{one-inclusion graph (OIG)} is a mathematical representation of classification in the transductive model of learning. OIGs were proposed by  \citet{haussler_predicting_1994} to model  realizable binary classification, and have since been  extended to realizable multiclass classification by \citet{rubinstein_shifting_2009}, and  to the agnostic setting by~\citet{asilis_regularization_2024}. As we will see shortly, OIGs represent these learning tasks exactly, with their \emph{orientations} in one-to-one correspondence with transductive learners. 

\paragraph{Realizable Binary Classification.} I will start by describing OIGs in their original form, restricted to realizable binary classification. Fix a domain $\X$ and a hypothesis class $\H \sse \set{0,1}^\X$. Recall that, in the  realizable setting of the transductive model, a collection of $n$  unlabeled datapoints $S=(x_1,\ldots,x_n) \in \X^n$ is chosen by an adversary, and provided to the learner at the outset. The adversary also selects a \emph{ground truth} hypothesis $h^* \in \H$,  then the learner is asked to predict the label $h^*(x_i)$ of a randomly-chosen test datapoint $x_i$ from the labels $h^*|_{S_{-i}}$ of the remaining $n-1$ datapoints.


\begin{wrapfigure}{R}{0.42\textwidth}
%\vspace{-0.3cm}
  \centering
  \input{figures/oig.tikz}
  \caption{Example OIG for realizable binary classification on three~datapoints.}
  \label{fig:binaryoig}
\end{wrapfigure}
The OIG is an undirected graph $G(\H,S)$  determined by the hypothesis class $\H$ and the unlabeled datapoints  $S$, both of which are known to the learner. The nodes correspond to the possible labelings $\H|_S$  of the datapoints $S$ by hypotheses in $\H$, of which there can be at most $2^n$.\footnote{A tighter bound is given by the \emph{growth function} $\Phi_d(n) = \sum_{i=0}^d \binom{n}{i}$, where $d$ is the VC dimension of $\H$.} %\footnote{For a binary hypothesis class of VC dimension $d$,  a tight upper bound on the number of possible labelings of $n$ datapoints, and hence the number of nodes of the OIG,  is given by the \emph{growth function} $\Phi_d(n) = \sum_{i=0}^d \binom{n}{i}$.} In other words, each node is a labeling $(y_1, \ldots, y_n) \in \set{-1,+1}^n$, with $y_i$ the label for $x_i$, consistent with some hypothesis $h \in \H$. 
  An edge is included between two labelings if they differ on exactly a single datapoint. Another perspective is that the OIG is simply the subgraph of the $n$-dimensional hypercube induced by $\H|_S$. An example OIG is depicted in Figure~\ref{fig:binaryoig}.


  
  Now consider a learner trying to predict the label of $x_i$ from labels $\set{y_j}_{j \neq i}$ provided for all the other datapoints. If there is a single node $(y_1, \ldots, y_i, \ldots, y_n)$  in $G$ that is consistent with the provided labels, then there is nothing to decide: $y_i$ must be the correct label for $x_i$. The interesting case is when there are two such nodes, one of which labels $x_i$ with a $0$ and the other with a~$1$. These two nodes, in differing only on $x_i$, must share an edge $e$ in the OIG. In such a situation, a learner predicting a label for $x_i$ chooses one of the two nodes to ``go with'', effectively \emph{orienting} $e$   towards the chosen prediction. In specifying a prediction for every such scenario, a learner orients all the edges of the OIG, turning it into a directed graph.  %In other words, the learner \emph{orients} the edge by imbuing it with a direction, turning it into a directed graph.
To summarize, learners are  in one-to-one correspondence with \emph{orientations} of the OIG.\footnote{More accurately, this is the case for deterministic learners. Randomized learners correspond to randomized orientations. This distinction is not especially consequential, so I blur it in this article.} 

  % outdegree
Given this graph-theoretic view of a transductive learner, we can also describe its error guarantee  in graph-theoretic terms: it is proportional to the maximum \emph{out-degree} of a node in the oriented graph. To see this, observe that each node  represents a possible ground truth labeling $\y=(y_1, \ldots, y_n)$ of the datapoints, and each edge represents an ambiguity faced by the learner looking to predict some $y_i$. When an edge is directed out of $\y$, this corresponds to a ``mistake'' by the learner when $\y$ happens to be the ground truth. Since the test datapoint~$x_i$ is chosen uniformly at random, and we evaluate error in the worst case over possible ground truths, we can conclude that the learner's error (i.e., worst-case misclassification rate) equals the maximum outdegree divided by $n$. Since each edge results in a mistake for one of its endpoints, one can think of transductive learning as seeking to ``spread the error around'' as evenly as possible among all possible ground truths.



% multiclass
\paragraph{Realizable Multiclass Classification.} One-inclusion graphs generalize naturally to multiclass classification, as articulated by \cite{rubinstein_shifting_2009}. For a set $\Y$ of labels, which may be finite or infinite, the nodes of the OIG are still the  ground truth labelings $\y=(y_1,\ldots,y_n) \in \Y^n$ consistent with some hypothesis. The edges, however, turn into hyper-edges, with a collection of nodes sharing a hyper-edge precisely when they all disagree on \emph{the same} $y_i$, and no other.  An OIG involving  three datapoints and three labels  is depicted in Figure~\ref{fig:oig}. %It will be sometimes convenient to identify a  hyper-edge with a \emph{partial labeling} $(y_1,\ldots, ? , \ldots y_n)$ with $y_j \in \Y$ for all but a single ``?'', where all nodes agreeing with the partial labeling on all but the ``?'' are included in the hyper-edge.
The hyper-edges again correspond to the ambiguities faced by the learner, where multiple predictions (up to $|\Y|$ many) are consistent with provided data. A learner can therefore be seen as orienting a hyper-edge towards the node corresponding to its prediction, and away from the others. As before, a ground truth incurs one ``mistake'' for every incident edge directed away from it.  Defining the out-degree of a node as the number of its incident edges directed away from it, the learner's error is again the maximum out-degree divided by $n$.

\definecolor{myblue}{RGB}{0,80,160}
\definecolor{mygreen}{RGB}{80,160,80}
\begin{figure}[tbp]
    \centering
    \begin{subfigure}[b]{0.45\textwidth}
        \centering
        \input{figures/hoig.tikz}
        \caption{Traditional OIG. Nodes depicted with circles,  hyper-edges with rectangles.}
        \label{fig:oig}
    \end{subfigure}
    \hfill % Optional: this inserts space between the two figures
    \begin{subfigure}[b]{0.45\textwidth}
        \centering
        \input{figures/bpoig.tikz}
        \caption{Bipartite OIG. Left side nodes with a single neighbor, such as $(0,0,?)$, omitted for clarity.}
        \label{fig:bpoig}
    \end{subfigure}
    \caption{Example OIG for realizable multiclass classification, in traditional and bipartite forms.}
    \label{fig:bothoig}
\end{figure}




%%% Local Variables:
%%% mode: latex
%%% TeX-master: "../learning_matching.tex"
%%% End:



\paragraph{Agnostic Multiclass Classification.} 
OIGs were recently further generalized to the agnostic setting of multiclass classification by  \citet{asilis_regularization_2024}, building on a closely-related definition employed by \citet{long_complexity_1999} for a different model of learning. This \emph{agnostic OIG}, denoted $G^\ag(\H,S)$, is as follows.  Since the ground truth labeling of the given datapoints $S=(x_1,\ldots,x_n)$ may now be completely arbitrary, we include all labelings $\Y^n$ as nodes of $G^\ag$.  As in the realizable case, a collection of nodes share a hype-edge precisely when they disagree on the same $y_i$ and no other.   Also as in the realizable case, a learner is an orientation of the hyper-edges, and the learner's error is related to out-degrees of all the nodes.

But what of the hypothesis class $\H$? Since we are in the agnostic setting, $\H$ serves merely as a benchmark for our learner. Therefore, in defining the OIG orientation problem we now ``discount'' the out-degree of each node  $\y=(y_1,\ldots,y_n)$ of $G^\ag$ by its \emph{Hamming distance} $\dist(y,\H)$ from $\H|_S$.\footnote{The Hamming distance between two vectors is the number of entries on which they differ. For the distance to a set of vectors, we minimize over that set appropriately.} %For error rate $\epsilon$, each labeling $\y$ on the right side of $G_\bp^\ag$ must be ``matched'' at least $(1-\epsilon)n - d(\y,\H)$ times, where $d(\y,\H)$ is the aforementioned discount for node $\y$.
This is quite natural, as the more the ground truth deviates from our benchmark class $\H$, the more we are ``off the hook'' for making accurate predictions. 



% computation, brute force, parsimonious need
\paragraph{A Note on Computational Complexity.} Some remarks are in order with regard to the computational complexity of working with the OIG. Even for binary classification with a finite VC dimension $d$, the OIG can have on the order of $n^d$ nodes. This is prohibitively large for all but the smallest values of $d$. In multiclass classification the situation can be even more dire. In fact, there are learnable problems with infinitely many labels and  OIGs of infinite size! It is therefore unsurprising that OIGs have usually not been thought of as a practical data structure for learning, but rather as abstractions for its mathematical study. In some sense, orienting the OIG can be viewed as ``brute-forcing'' the learning task: a representation of the algorithm's entire input/output behavior, at least with respect to labelings, is computed before even looking at the labels in the training data!  This stands in contrast to practical approaches  such as ERM and SRM (see Section~\ref{sec:background}), %, such as empirical risk minimization and structural risk minimization,
which represent the task parsimoniously and reason locally, input by input. Nonetheless, I will argue in this article that insights emanating from OIGs, and extensions of them, can  point the way to such tenable algorithms.



\subsection{The Matching Perspective}
\label{subsec:matching}

One inclusion graphs transform multiclass learning to what is essentially a combinatorial optimization problem on orientations, albeit one that is described implicitly. From here, a simple shift in perspective articulated by \cite{asilis_regularization_2024}, employing the classical connection between orientation and matching problems (see e.g. \cite{Schrijver_2003}), yields an equivalent bipartite matching problem. I will describe this next.

Let $G$ be an OIG associated with hypothesis class $\H \sse \Y^\X$ and unlabeled dataset $S=(x_1,\ldots,x_n)$. The transformation is essentially identical whether we let $G$ be the realizable or agnostic OIG, but I recommend the reader keep the realizable OIG in mind to keep things simple. On the right we have the  nodes of $G$, corresponding to possible ground-truth labelings $\y=(y_1,\ldots,y_n)$ of  $(x_1,\ldots,x_n)$ . On the left we have \emph{partial labelings} which occlude a single label from some ground truth. We represent these as a vector $(y_1, \ldots, ?, \ldots, y_n)$,  where $y_i \in \Y$ for all but a single ``?'' entry corresponding to the occluded label. An edge is included between a partial labeling on the left and a full labeling on the right if they agree on all but occluded label. 
In graph-theoretic terms,  the bipartite OIG $G_{\bp}$ is essentially  the \emph{edge-vertex incidence graph} of $G$, were $G$ to first be augmented with self-loops (i.e., singleton hyper-edges) so that every node has degree exactly~$n$. %We sometimes refer to $G_{\bp}$ as the \emph{bipartite OIG} for convenience.
A bipartite OIG for the realizable setting is depicted in Figure~\ref{fig:bpoig}.

The  left side nodes of $G_{\bp}$ --- the partial labelings --- represent the scenarios faced by the learner, where one label is omitted and the learner is asked to ``fill in the blank.'' Edges of $G_{\bp}$ encode the consistency relationship between the partial labelings on the left and the (full) labelings on the right. A learner, in having to predict each occluded label, can be viewed as an \emph{assignment} mapping each node on the left to one of its neighbors on the right. Note that each node on the right has degree exactly $n$ in $G_{\bp}$, one for each possible position of the ``?'',  whereas nodes on the left may have degree up to the number of labels $|\Y|$ (exactly $|\Y|$ in the agnostic setting). % with a corresponding learner in  Figure~\ref{fig:??}. %original OIG, corresponding bipartite OIG, highlighted 


Given this perspective on learners as assignments in the bipartite OIG, what of their error? It is easier to take a complementary perspective and describe the learner's \emph{accuracy}: the probability it correctly classifies the test point, in the worst case over ground truths. For a particular ground truth labeling $\y=(y_1,\ldots,y_n)$ on the right side of $G_{\bp}$, the learner faces $n$ equally likely scenarios, one for each left node $\tilde{\y} = (y_1, \ldots, ? , \ldots, y_n)$ incident on $\y$. The learner correctly classifies the test point precisely when it assigns $\tilde{\y}$ to  $\y$. Therefore, viewing a learner as a left to right assignment in $G_{\bp}$, its accuracy is its minimum degree to a node on the right, divided by $n$.  In the realizable setting, the error  is of course the complementary probability. In the agnostic setting, we discount each degree of a node $\y$ on the right by its Hamming distance $\dist(\y,\H)$ to the hypothesis class, then proceed identically to arrive at the agnostic error. 

We can now cast learning  as a  bipartite matching problem. To obtain error $\epsilon$, each node $\y$ on the right side of $G_{\bp}$ must be assigned --- we say \emph{matched} --- at least $(1-\epsilon) n$ times in the realizable case, and at least $(1-\epsilon) n - \dist(\y,\H)$ times in the agnostic case. This is what is referred to in combinatorial optimization as a  bipartite \emph{$b$-matching problem}, a generalization of bipartite matching where nodes can be prescribed a minimum required number of matches. This generalization is exclusively for convenience, as a bipartite $b$-matching problem can be equivalently posed as  a run-of-the-mill  maximum bipartite matching problem by appropriately cloning nodes that require multiple matches. It must be noted, however, that $G_{\bp}$ can like $G$ be an  infinite graph, and even when finite is typically prohibitively large for invoking matching algorithms directly.

What, then, does this matching perspective on classification  buy us? Quite a bit, as it turns out. I will describe some recent consequences  of this viewpoint next.


\subsection{Implication of Matching: A Compact Graph-Theoretic Characterization}
%We now describe a shift in perspective articulated in the PI's recent joint work \cite{asilis_regularization_2023}, one which reframes  realizable classification  as a bipartite matching problem. This new lens, while remarkably simple,  allows us to bring to bear tools from combinatorics, matching theory, and optimization to the analysis of learning. Furthermore, it sets the stage for generalizations of the OIG that go beyond realizable classification.
\label{sec:hall}



Characterizing the optimal sample complexity of  learning  is a chief concern of learning theory. Given the simplicity of the transductive model, one might hope for a crisp and interpretable expression for the sample complexity, or equivalently the error rate, of the optimal transductive learner. Such an expression would approximate the performance of the optimal PAC learner, as per the relationships described  in Section \ref{subsec:pac_trans}. The matching perspective yields such a graph-theoretic expression which captures the transductive  error rate ``on the nose'' \cite{asilis_regularization_2024}. Moreover, this expression refers only to  \emph{finite projections} of the learning problem.


%\begin{figure}
\begin{wrapfigure}{R}{0.41\textwidth}
%  \vspace{-0.5cm}
  \centering
  \scalebox{1}{\input{figures/infinite_matching.tikz}}
  \caption{Matching on infinite bipartite graphs absent degree constraints.}
  \label{fig:infinite_matching}
\end{wrapfigure}
%\end{figure}
The starting point here is Philip Hall's classic \emph{marriage theorem}~\cite{hall1987representatives}, which characterizes exactly when one side of a \emph{finite} bipartite graph can be matched to the other. Let $L$ and $R$ denote the left and right side nodes of the graph, respectively, and consider whether there is a matching which matches every node on the right. The original marriage theorem states that such a matching exists  if and only if every set $R' \sse R$ of right side nodes has at least $|R'|$ neighbors between them (on the left, naturally). Whereas this \emph{Hall condition} characterizes matching on finite bipartite graphs, it fails to extend to all infinite bipartite graphs, where it is necessary but not sufficient in general. This can be seen by way of a simple example, illustrated in Figure~\ref{fig:infinite_matching}.% as evidenced by a simple example.\footnote{Suppose $L=\set{\ell_1,\ell_2,\ldots}$ is indexed by the positive integers, $R=\set{r_0,r_1,r_2,\ldots}$ is indexed by the nonnegative integers, and the edges include $(\ell_i,r_i)$ and $(\ell_i,0)$ for all positive integers $i$. The Hall condition holds for all finite~$R' \sse R$. However, there is no matching for all of $R$: if $r_0$ is matched to some $\ell_i$, then $r_i$ can not be matched.  }


The ``troublemaker'' in Figure~\ref{fig:infinite_matching} is the infinite-degree node $0$ on the right. If we assume that the nodes we intend to match --- the right side nodes $R$, in our discussion --- have finite degrees, the marriage theorem can be recovered:  we can match  $R$ if and only if we can match every finite $R' \sse R$, which in turn is possible precisely when each finite $R' \sse R$ has at least $|R'|$ neighbors (the Hall condition). This holds for infinite graphs of arbitrary, even uncountable, cardinality, so long as the degrees on the right are all finite (not necessarily bounded); degrees on the left may be arbitrary, even uncountable. This \emph{infinite marriage theorem} was first established by Marshall Hall~\cite{jr_distinct_1948} by way of a quite sophisticated proof,\footnote{No relation to Philip Hall, coincidentally.}  though somewhat simpler proofs have since been discovered (e.g. \cite{halmos_marriage_1950,rado_note_1967}). The infinite marriage theorem also extends naturally to  $b$-matching by a simple cloning argument, under the same requirement of finite right-side degrees: If each node on the left can be matched at most once, whereas each node $r$ on the right requires at least $b_r$ matches, then this is possible if and only if it is possible for every finite $R' \sse R$, which in turn is possible if and only if each finite $R' \sse R$ has at least $b(R') = \sum_{r \in R'} b_r$ neighbors. In showing that an infinite graph admits a matching of a particular form precisely when the same holds for some of its finite subgraphs, M. Hall's infinite marriage theorem is what is referred to as a \emph{compactness} result.

This now brings us naturally to bipartite OIGs, which encode classification as a matching problem. In the realizable setting, a learner with error $\epsilon$ corresponds to a $b$-matching on $G_{\bp}$ with $b=(1-\epsilon)n$ for every right side side node $\y=(y_1, \ldots, y_n)$. The infinite marriage theorem then implies that this is possible precisely when every finite family $R'$ of full labelings on the right side of $G_{\bp}$ is incident on at least $(1-\epsilon)n \cdot |R'|$ partial labelings. This yields a precise expression of the optimal error as a function of the sample size $n$,  as the worst-case  ``matchability'' of finite subgraphs of a bipartite OIG. This graph-theoretic expression, stated precisely in \cite{asilis_regularization_2024}, is called the \emph{Hall complexity} of the hypothesis class. The expression is easily adapted to the agnostic setting by appropriately discounting the matching requirement of each node $\y$ in the agnostic OIG by $\dist(\y,\H)$. 

Much like in the case of matching, we now have a compactness result for classification. The Hall complexity expresses the optimal transductive error rate of a classification problem as the worst-case learning rate of its \emph{finite projections}: those problems which can be obtained by restricting attention to finite subsets of the domain $\X$ and finite subsets $\H' \sse \H$ of the hypothesis space. To see this, note that  each bipartite OIG is a function of a finite subdomain $x_1,\ldots,x_n$, and a finite subset of its right side nodes corresponds to finitely many hypotheses.  By exactly relating the matchability in the bipartite OIG to the same matchability in some of its finite subgraphs, we can conclude that the maximal obstructions to transductive classification are finite in nature. %Th compact characterization approximately extends to PAC learning, in black box fashion as described  in Section \ref{subsec:pac_trans}.


\paragraph{Bibliographic Remarks.} Simple and \emph{compact} expressions for the optimal error, or equivalently the sample complexity, have long been sought in learning theory. For realizable binary classification, the sample complexity is characterized in terms of the VC dimension both in the transductive \cite{haussler_predicting_1994,li_one-inclusion_2001} and PAC (see e.g. \cite{shalev-shwartz_understanding_2014}) models.    For multiclass classification, the \emph{DS dimension} of \citet{daniely_optimal_2014} yields an approximation of the optimal error in the realizable setting up to a constant in the exponent, as was recently shown by \citet{brukhim_characterization_2022}. Outside of  \emph{dimensions}, a quantity derived from the (classical) one-inclusion graph $G$ gives a constant factor approximation: the maximum average degree of a subgraph of $G$, divided by $n$, is with a factor of~$2$ of the optimal error in the realizable setting. The Hall complexity can be viewed as ``sharpening'' the maximum-average degree into an exact characterization, and extending it to the agnostic setting.

The most sought-after characterizations of learning are \emph{dimensions} such as the VC dimension, DS dimension, and others. Intuitively, a dimension measures the size of the ``largest obstruction'' to learning, with an infinite dimension implying unlearnability, and a finite dimension yielding a bound on the sample complexity by way of a simple expression. Compactness results of the sort provided by the Hall complexity are a weaker requirement than the existence of a dimension capturing optimal learning. I will elaborate on the relationship between compactness and dimensions in Section~\ref{sec:compactness}.  %I will note, however, that compactness is by no means ``trivial'': recent work of \citet{ben-david_learnability_2019} demonstrates a natural supervised learning problem --- in the proper setting not considered in this article ---  where compactness fails.\footnote{We note that \cite{ben-david_learnability_2019} do not pose their non-compact problem, which they term \emph{EMX learning}, as a supervised learning problem. However, it is not difficult to see that it can be rephrased as such.}



\subsection{Implication of Matching: An Algorithmic Template for Classification}
Another important concern of learning theory is identifying algorithmic approaches that are broadly successful for a large swath of learning problems, and moreover simple enough to describe, implement, and explore. Perhaps the best example of this is empirical risk minimization (ERM), as described in Section~\ref{sec:background}. ERM and closely related approaches such as structural risk minimization (SRM) pervade the practice of machine learning, and in some cases are also accompanied by theoretical guarantees.  Most notably, for binary classification problems in the PAC model, it has long been known that ERM characterizes learnability and moreover has almost-optimal sample complexity (see e.g. \cite{shalev-shwartz_understanding_2014}).

Such theoretical optimality guarantees  for ERM  tend to fall by the wayside as one moves beyond binary classification.  In fact, ERM can entirely fail to learn in some settings where learning is possible, as shown by \citet{shalev-shwartz_learnability_2010}.   This failure manifests   in multiclass classification as shown by \citet{daniely_multiclass_2011}, and in a slight generalization of binary classification as shown by \citet{alon_theory_2021}. Even in bounded agnostic linear regression, the sample complexity of ERM was recently shown to be quite suboptimal by \citet{vaskevicius_suboptimality_2023}.\footnote{That said, ERM characterizes learnability for bounded regression  in the coarse sense, disregarding sample complexity~\cite{alon_scale-sensitive_1997}.}  %Whereas a body of work has made significant progress in describing optimal learners for such general problems (e.g. \cite{daniely_optimal_2014,brukhim_characterization_2022}), the existence of an algorithmic template or principle which captures optimal learning remains open.

In settings where ERM fails to characterize optimal learning, the algorithms which do succeed in theory are often far removed from any parsimonious description or any heuristic employed in practice.  This is most apparent in multiclass classification, where the known near-optimal algorithms \cite{daniely_optimal_2014,brukhim_characterization_2022} involve explicitly-described orientations of the one-inclusion graph. Can the bipartite matching interpretation simplify these OIG-based algorithms? Bipartite matching tends to respond well to ``simple'' heuristics such as greedy algorithms, primal-dual approaches, and others --- can the same hold true for learning?


 \begin{figure}
  % \input{arXiv_version_2/figures/local_regularizer.tikz}
\centering
\input{figures/local_regularizer.tikz}
\caption{\emph{ %Hypotheses $h_1$ and $h_2$, depicted in yellow and blue respectively.
    A  local regularizer may favor the simplicity of $h_1$ on test points drawn from the right region of the domain, and the simplicity of $h_2$ on test points drawn from the left region.}}
\label{fig:local}
% \captionof{figure}{}
% This should be adjusted to not overlap once the manuscript is finished.
%\vspace{-6em}
%\caption{Hypotheses $h_0$ and $h_1$, depicted in blue and yellow respectively. A  local regularizer may favor the simplicity of $h_0$ on test points drawn from the right region of the domain, and the simplicity of $h_1$ on test points drawn from the left region.}
\end{figure}

The answer to both questions turns out to be yes, as shown recently by~\citet{asilis_regularization_2024}. Through primal-dual analysis of a certain convex program over matchings in the bipartite OIG, we derive a near-optimal learner for multiclass classification of a particular instructive form. In particular, this learner implements a variation of the SRM template from Section~\ref{sec:background}, relaxed on two fronts: (1) It employs  a  \emph{local} regularization function  $\psi(h,x_{\tst})$ that  measures predictor complexity differently depending on the test point, and  (2) it incorporates an unsupervised pre-training stage which learns this regularization function.

These relaxed SRMs --- derived from the connection to bipartite matching --- provide a  simpler template for optimal learning which is quite natural, and is reminiscent of approaches discovered to be useful in practice. Relaxing to local regularization is well-motivated, since different hypotheses may be simple in some areas of the domain and complex in others; this is illustrated in Figure~\ref{fig:local}. It is therefore not altogether surprising that this result is predated by applications of local regularization to image classification and restoration~\cite{wolf_local_2008,prost_learning_2021}. The second relaxation is in line with much of the recent empirical and theoretical evidence on the utility of an unsupervised learning stage. Indeed, unsupervised pre-training has seen widespread application to  computer vision, natural language processing, and speech recognition (see the discussion in \cite{ge_provable_2024}). 



%%%Local Variables:
%%% mode: latex
%%% TeX-master: "learning_matching"
%%% End:

\section{General Choice Models}
To generalize beyond uniform choice models, we consider two strategies to tackle general choice models.
The first considers sampling-based methods, which sample from different permutations to assess the quality of a menu. 
The second considers restrictions on the choice class, so that the utility for a particular menu can be expressed as a submodular function. 
Doing so allows us to optimize efficiently due to properties of submodular functions. 

\subsection{Sampling-based Methods}
We first consider algorithms which sample different permutations of $\pi$, and use this to find optimal menus. 
We first use concentration inequalities to show that we can efficiently approximate the utiltiy of a particular menu: 
\begin{lemma}
     Let $R(X_{1,1},X_{1,2},\cdots,X_{N,P}) = \mathbb{E}_{\pi}[\sum_{i=1}^{N} \sum_{j=1}^{P} \theta_{\pi_{i},j} Y_{\pi_{i},j}(\mathbf{Z}_{i})]$ be a utility function of the menus offered. Let $\mathrm{OPT} = \max\limits_{X_{1,1},X_{1,2},\ldots,X_{N,P}} \mathbb{E}_{\pi}[\sum_{i=1}^{N} \sum_{j=1}^{P} \theta_{\pi_{i},j} Y_{\pi_{i},j}(\mathbf{Z}_{i})]$. If $\theta_{i,j} \in [0,1] \forall i,j$, then $\mathrm{OPT} \leq \min(N,P)$
\end{lemma}

\begin{lemma}
    Let $R(X_{1,1},X_{1,2},\cdots,X_{N,P}) = \mathbb{E}_{\pi}[\sum_{i=1}^{N} \sum_{j=1}^{P} \theta_{\pi_{i},j} Y_{\pi_{i},j}(\mathbf{Z}_{i})]$ be a utility function of the menus offered.
    Consider a fixed menu $X_{1,1},X_{1,2},\cdots,X_{N,P}$. 
    Then after 
    \begin{equation}
        n = \frac{\ln(\frac{2}{\delta}) \min(N,P)^{2}}{\epsilon}
    \end{equation}
    samples, denoted $\pi^{1}, \pi^{2}, \ldots, \pi^{n}$, we have that, with probability $1-\delta$
    \begin{equation}
        |R(X_{1,1},X_{1,2},\cdots,X_{N,P}) - \frac{1}{n} \sum_{k=1}^{n} \sum_{i=1}^{N} \sum_{j=1}^{P} \theta_{\pi^{k}_{i},j} Y_{\pi^{k}_{i},j}(\mathbf{Z}_{i})| \leq \epsilon
    \end{equation}
\end{lemma}

From here, if our utility function is Lipschitz in the menus offered, then we can use a covering-style argument to find the optimal menu. 
We do this in two steps. 
First, we construct a set of menus which approximately cover the set of available menus, within some distance $r$. 
We use sampling-based techniques to approximate the utility for each of these menus. 
Second, we claim that the optimal menu is close to one of these menus, and because the utility is Lipschitz in the menu, we can bound the perforamnce gap between our best found menu and the optimal menu: 
\begin{theorem}
    Consider a utility function, so that $|R(X_{1,1},X_{1,2},\cdots,X_{N,P}) - R(X'_{1,1},X'_{1,2},\cdots,X'_{N,P})| \leq L \sum_{i=1}^{N} \sum_{j=1}^{P} |X_{i,j}-X'_{i,j}|$ for some $L \geq 0$. 
    Let $K(n,r)$ be the covering code for a binary string of length $n$ with hamming distance $r$; the covering code represents the minimum cardinality of a set $S$, so that all binary strings of length $n$ are at most distance $r$ away from some element of $S$. 
    Let $\mathrm{OPT} = \max\limits_{X_{1,1},X_{1,2},\ldots,X_{N,P}} \mathbb{E}_{\pi}[\sum_{i=1}^{N} \sum_{j=1}^{P} \theta_{\pi_{i},j} Y_{\pi_{i},j}(\mathbf{Z}_{i})]$
    Then with $\frac{K(NP,r) \ln(\frac{2}{\delta}) \min{(N,P)}^{2}}{\epsilon}$ evaluations of 
    \begin{equation}
        \sum_{i=1}^{N} \sum_{j=1}^{P} \theta_{\pi_{i},j} Y_{\pi_{i},j}(\mathbf{Z}_{i})
    \end{equation}
    using different combinations of $\pi$ and $X_{i,j}$, we can find a solution $\mathrm{ALG} = R(X_{1,1},X_{1,2},\cdots,X_{N,P})$, such that, with probability $1-\delta$, $\mathrm{ALG} \geq \mathrm{OPT} - rL$. 
\end{theorem}

\subsection{Submodular Utility}
When the utility function can be written as a submodular function, we can leverage techniques from submodular optimization to solve our problem. 
Moreover, we know that monotone submodular optimization can be approximated within $1-\frac{1}{e}$ in polynomial time~\cite{submodular_optimization} with cardinality constraints (and optimally without such constraints), while non-monotone optimization can be approximated within $\frac{2}{5}$~\cite{non_monotone_submodular}. 
We use this to bound the performance of polynomial time algorithms in this scenario: 
\begin{lemma}
    Let $R(X_{1,1},X_{1,2},\cdots,X_{N,P}) = \mathbb{E}_{\pi}[\sum_{i=1}^{N} \sum_{j=1}^{P} \theta_{\pi_{i},j} Y_{\pi_{i},j}(\mathbf{Z}_{i})]$ be a utility function of the menus offered.
    If $R$ is submodular in the variables $X_{i,j}$, and if we can evaluate $R(X_{1,1},X_{1,2},\cdots,X_{N,P})$ in polynomial time, then then we can find a greedy solution, $\mathrm{ALG}$, in polynomial time, such that $\mathrm{ALG} \geq \frac{2}{5} \mathrm{OPT}$, where $\mathrm{OPT} = \max\limits_{X_{1,1},X_{1,2},\ldots,X_{N,P}} \mathbb{E}_{\pi}[\sum_{i=1}^{N} \sum_{j=1}^{P} \theta_{\pi_{i},j} Y_{\pi_{i},j}(\mathbf{Z}_{i})]$
\end{lemma}

We note that such a solution is not a panacea, as we construct simple choice models which aren't submodular: 
\begin{lemma}
    There exists a selection of $Y_{i,j}(\mathbf{Z})$ so that $R(X_{1,1},X_{1,2},\cdots,X_{N,P}) = \mathbb{E}_{\pi}[\sum_{i=1}^{N} \sum_{j=1}^{P} \theta_{\pi_{i},j} Y_{\pi_{i},j}(\mathbf{Z}_{i})]$ is not submodular. 
\end{lemma}

Determining whether commonly used choice models result in submodular utility functions is still an open question:

\begin{conjecture}
    When $Y_{i,j}$ is the uniform choice model, $R(X_{1,1},X_{1,2},\cdots,X_{N,P})$ is submodular. 
\end{conjecture}

\begin{conjecture}
    When $Y_{i,j}$ is the MNL choice model, $R(X_{1,1},X_{1,2},\cdots,X_{N,P})$ is submodular. 
\end{conjecture}

\section*{Conclusion}
This paper aims to enhance our understanding of the computational complexity of computing various Shapley value variants. We found that for various ML models --- including decision trees, regression tree ensembles, weighted automata, and linear regression --- both local and global interventional and baseline SHAP can be computed in polynomial time under HMM modeled distributions. This extends popular algorithms, such as TreeSHAP, beyond their empirical distributional scope. We also establish strict complexity gaps between the various SHAP variants (baseline, interventional, and conditional) and prove the intractability of computing SHAP for tree ensembles and neural networks in simplified scenarios. Overall, we present SHAP as a versatile framework whose complexity depends on four key factors: \begin{inparaenum}[(i)] \item model type, \item SHAP variant, \item distribution modeling approach, \item and local vs. global explanations\end{inparaenum}. We believe this perspective provides deeper insight into the computational complexity of SHAP, paving the way for future work.




%We believe that our framework provides a more intricate understanding of SHAP computation complexity across different models, distributions, and variants, paving the way for further research.

Our work opens promising directions for future research. First, expanding our computational analysis to other SHAP-related metrics, such as asymmetric SHAP~\citep{frye20} and SAGE~\citep{covert2020understanding}, would be valuable. Additionally, we aim to explore more expressive distribution classes and relaxed assumptions beyond those in Section \ref{sec:tractable} while maintaining tractable SHAP computation. Finally, when exact computation is intractable (Section \ref{sec:intractable}), investigating the approximability of SHAP metrics through approximation and parameterized complexity theory~\citep{downey2012parameterized} is an important direction.

%Our work opens several promising avenues for future research on the computational properties of explainable AI methods, with a particular focus on SHAP. First, it would be interesting to broaden the computational analysis conducted in this work to include other popular SHAP-related metrics in the literature, such as asymmetric SHAP \cite{frye20} and SAGE \cite{covert2020understanding}. Also, in the future, we aim to explore more expressive distribution classes and relaxed distributional assumptions—extending beyond those examined in Section \ref{sec:tractable} —that still yield tractable SHAP computation. Finally, when exact computation proves intractable (Section \ref{sec:intractable}), it is worthwhile to theoretically investigate the question of the approximability of computing the SHAP metrics across various configurations, through the lens of approximation and parametrized complexity theory \cite{arora2009computational}.

%This paper aims to deepen our understanding of the computational complexity involved in obtaining different Shapley value variants. We found that for a variety of ML models, including decision trees, tree ensembles for regression, weighted automata, and linear regression models — computing both local and global interventional and baseline SHAP can be done in polynomial time when distributions are modeled by HMMs. This extends the distributional scope of popular algorithms like TreeSHAP, which is limited to empirical distributions. Additionally, we demonstrate a strict complexity gap between SHAP variants, showing that interventional and baseline SHAP can be strictly easier to compute than conditional SHAP. Despite these positive results, we uncovered intractability for various SHAP variants in neural networks and tree ensembles. Finally, we provided generalized complexity relations across SHAP variants. We believe that our framework offers a deeper understanding of the complexity involved in computing SHAP across various variants, models, distributions, as well as in both local and global computations, laying the groundwork for future research.

\section*{Acknowledgments}
Many of the ideas in this article came out of conversations and joint papers  with a stellar group of collaborators. In alphabetical order, these are: Julian Asilis, Siddartha Devic, Yusuf Kalayci, Vatsal Sharan, Shang-Hua Teng, and Grayson York. The analogy between the transductive model and hat puzzles came out of conversations with Siyu Zeng.


{%\small
\bibliography{learning,matching,misc,local_computation}
%\bibliographystyle{plainnat}
\bibliographystyle{abbrvnat}               %latex8
%\bibliographystyle{plain}
%\bibliographystyle{abbrv}               %latex8
}


\end{document}


%%% Local Variables: 
%%% mode: latex
%%% TeX-master: t
%%% End: 
