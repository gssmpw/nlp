\documentclass[11pt]{article}
\usepackage[margin=1in]{geometry}
%\usepackage{times}
%\usepackage[compact]{titlesec}
\usepackage{amsfonts}
\usepackage{amsmath,amsthm,amssymb}
\usepackage{latexsym}
\usepackage{epic}
\usepackage{epsfig}
\usepackage{hyperref}
\usepackage{verbatim}
\usepackage[justification=centering]{caption}
\usepackage{enumitem}
\usepackage{color}
\allowdisplaybreaks[1]
\usepackage[numbers]{natbib}
\usepackage{algorithm}
\usepackage{algorithmic}
\usepackage{setspace}




\usepackage{tikz}
\usepackage{graphicx}
\usepackage{tikzscale}
\usepackage{wrapfig}
\usepackage{float}
\usetikzlibrary{arrows.meta,arrows}
\usetikzlibrary{decorations.pathreplacing}
\usetikzlibrary{positioning,chains,fit,shapes,calc}
\usepackage{wrapfig}
\usepackage{pgfplots}
\pgfplotsset{compat = newest}

\usepackage{caption}
\captionsetup{justification=raggedright,singlelinecheck=true}
\usepackage{subcaption}


\usepackage{xspace}





%
\setlength\unitlength{1mm}
\newcommand{\twodots}{\mathinner {\ldotp \ldotp}}
% bb font symbols
\newcommand{\Rho}{\mathrm{P}}
\newcommand{\Tau}{\mathrm{T}}

\newfont{\bbb}{msbm10 scaled 700}
\newcommand{\CCC}{\mbox{\bbb C}}

\newfont{\bb}{msbm10 scaled 1100}
\newcommand{\CC}{\mbox{\bb C}}
\newcommand{\PP}{\mbox{\bb P}}
\newcommand{\RR}{\mbox{\bb R}}
\newcommand{\QQ}{\mbox{\bb Q}}
\newcommand{\ZZ}{\mbox{\bb Z}}
\newcommand{\FF}{\mbox{\bb F}}
\newcommand{\GG}{\mbox{\bb G}}
\newcommand{\EE}{\mbox{\bb E}}
\newcommand{\NN}{\mbox{\bb N}}
\newcommand{\KK}{\mbox{\bb K}}
\newcommand{\HH}{\mbox{\bb H}}
\newcommand{\SSS}{\mbox{\bb S}}
\newcommand{\UU}{\mbox{\bb U}}
\newcommand{\VV}{\mbox{\bb V}}


\newcommand{\yy}{\mathbbm{y}}
\newcommand{\xx}{\mathbbm{x}}
\newcommand{\zz}{\mathbbm{z}}
\newcommand{\sss}{\mathbbm{s}}
\newcommand{\rr}{\mathbbm{r}}
\newcommand{\pp}{\mathbbm{p}}
\newcommand{\qq}{\mathbbm{q}}
\newcommand{\ww}{\mathbbm{w}}
\newcommand{\hh}{\mathbbm{h}}
\newcommand{\vvv}{\mathbbm{v}}

% Vectors

\newcommand{\av}{{\bf a}}
\newcommand{\bv}{{\bf b}}
\newcommand{\cv}{{\bf c}}
\newcommand{\dv}{{\bf d}}
\newcommand{\ev}{{\bf e}}
\newcommand{\fv}{{\bf f}}
\newcommand{\gv}{{\bf g}}
\newcommand{\hv}{{\bf h}}
\newcommand{\iv}{{\bf i}}
\newcommand{\jv}{{\bf j}}
\newcommand{\kv}{{\bf k}}
\newcommand{\lv}{{\bf l}}
\newcommand{\mv}{{\bf m}}
\newcommand{\nv}{{\bf n}}
\newcommand{\ov}{{\bf o}}
\newcommand{\pv}{{\bf p}}
\newcommand{\qv}{{\bf q}}
\newcommand{\rv}{{\bf r}}
\newcommand{\sv}{{\bf s}}
\newcommand{\tv}{{\bf t}}
\newcommand{\uv}{{\bf u}}
\newcommand{\wv}{{\bf w}}
\newcommand{\vv}{{\bf v}}
\newcommand{\xv}{{\bf x}}
\newcommand{\yv}{{\bf y}}
\newcommand{\zv}{{\bf z}}
\newcommand{\zerov}{{\bf 0}}
\newcommand{\onev}{{\bf 1}}

% Matrices

\newcommand{\Am}{{\bf A}}
\newcommand{\Bm}{{\bf B}}
\newcommand{\Cm}{{\bf C}}
\newcommand{\Dm}{{\bf D}}
\newcommand{\Em}{{\bf E}}
\newcommand{\Fm}{{\bf F}}
\newcommand{\Gm}{{\bf G}}
\newcommand{\Hm}{{\bf H}}
\newcommand{\Id}{{\bf I}}
\newcommand{\Jm}{{\bf J}}
\newcommand{\Km}{{\bf K}}
\newcommand{\Lm}{{\bf L}}
\newcommand{\Mm}{{\bf M}}
\newcommand{\Nm}{{\bf N}}
\newcommand{\Om}{{\bf O}}
\newcommand{\Pm}{{\bf P}}
\newcommand{\Qm}{{\bf Q}}
\newcommand{\Rm}{{\bf R}}
\newcommand{\Sm}{{\bf S}}
\newcommand{\Tm}{{\bf T}}
\newcommand{\Um}{{\bf U}}
\newcommand{\Wm}{{\bf W}}
\newcommand{\Vm}{{\bf V}}
\newcommand{\Xm}{{\bf X}}
\newcommand{\Ym}{{\bf Y}}
\newcommand{\Zm}{{\bf Z}}

% Calligraphic

\newcommand{\Ac}{{\cal A}}
\newcommand{\Bc}{{\cal B}}
\newcommand{\Cc}{{\cal C}}
\newcommand{\Dc}{{\cal D}}
\newcommand{\Ec}{{\cal E}}
\newcommand{\Fc}{{\cal F}}
\newcommand{\Gc}{{\cal G}}
\newcommand{\Hc}{{\cal H}}
\newcommand{\Ic}{{\cal I}}
\newcommand{\Jc}{{\cal J}}
\newcommand{\Kc}{{\cal K}}
\newcommand{\Lc}{{\cal L}}
\newcommand{\Mc}{{\cal M}}
\newcommand{\Nc}{{\cal N}}
\newcommand{\nc}{{\cal n}}
\newcommand{\Oc}{{\cal O}}
\newcommand{\Pc}{{\cal P}}
\newcommand{\Qc}{{\cal Q}}
\newcommand{\Rc}{{\cal R}}
\newcommand{\Sc}{{\cal S}}
\newcommand{\Tc}{{\cal T}}
\newcommand{\Uc}{{\cal U}}
\newcommand{\Wc}{{\cal W}}
\newcommand{\Vc}{{\cal V}}
\newcommand{\Xc}{{\cal X}}
\newcommand{\Yc}{{\cal Y}}
\newcommand{\Zc}{{\cal Z}}

% Bold greek letters

\newcommand{\alphav}{\hbox{\boldmath$\alpha$}}
\newcommand{\betav}{\hbox{\boldmath$\beta$}}
\newcommand{\gammav}{\hbox{\boldmath$\gamma$}}
\newcommand{\deltav}{\hbox{\boldmath$\delta$}}
\newcommand{\etav}{\hbox{\boldmath$\eta$}}
\newcommand{\lambdav}{\hbox{\boldmath$\lambda$}}
\newcommand{\epsilonv}{\hbox{\boldmath$\epsilon$}}
\newcommand{\nuv}{\hbox{\boldmath$\nu$}}
\newcommand{\muv}{\hbox{\boldmath$\mu$}}
\newcommand{\zetav}{\hbox{\boldmath$\zeta$}}
\newcommand{\phiv}{\hbox{\boldmath$\phi$}}
\newcommand{\psiv}{\hbox{\boldmath$\psi$}}
\newcommand{\thetav}{\hbox{\boldmath$\theta$}}
\newcommand{\tauv}{\hbox{\boldmath$\tau$}}
\newcommand{\omegav}{\hbox{\boldmath$\omega$}}
\newcommand{\xiv}{\hbox{\boldmath$\xi$}}
\newcommand{\sigmav}{\hbox{\boldmath$\sigma$}}
\newcommand{\piv}{\hbox{\boldmath$\pi$}}
\newcommand{\rhov}{\hbox{\boldmath$\rho$}}
\newcommand{\upsilonv}{\hbox{\boldmath$\upsilon$}}

\newcommand{\Gammam}{\hbox{\boldmath$\Gamma$}}
\newcommand{\Lambdam}{\hbox{\boldmath$\Lambda$}}
\newcommand{\Deltam}{\hbox{\boldmath$\Delta$}}
\newcommand{\Sigmam}{\hbox{\boldmath$\Sigma$}}
\newcommand{\Phim}{\hbox{\boldmath$\Phi$}}
\newcommand{\Pim}{\hbox{\boldmath$\Pi$}}
\newcommand{\Psim}{\hbox{\boldmath$\Psi$}}
\newcommand{\Thetam}{\hbox{\boldmath$\Theta$}}
\newcommand{\Omegam}{\hbox{\boldmath$\Omega$}}
\newcommand{\Xim}{\hbox{\boldmath$\Xi$}}


% Sans Serif small case

\newcommand{\Gsf}{{\sf G}}

\newcommand{\asf}{{\sf a}}
\newcommand{\bsf}{{\sf b}}
\newcommand{\csf}{{\sf c}}
\newcommand{\dsf}{{\sf d}}
\newcommand{\esf}{{\sf e}}
\newcommand{\fsf}{{\sf f}}
\newcommand{\gsf}{{\sf g}}
\newcommand{\hsf}{{\sf h}}
\newcommand{\isf}{{\sf i}}
\newcommand{\jsf}{{\sf j}}
\newcommand{\ksf}{{\sf k}}
\newcommand{\lsf}{{\sf l}}
\newcommand{\msf}{{\sf m}}
\newcommand{\nsf}{{\sf n}}
\newcommand{\osf}{{\sf o}}
\newcommand{\psf}{{\sf p}}
\newcommand{\qsf}{{\sf q}}
\newcommand{\rsf}{{\sf r}}
\newcommand{\ssf}{{\sf s}}
\newcommand{\tsf}{{\sf t}}
\newcommand{\usf}{{\sf u}}
\newcommand{\wsf}{{\sf w}}
\newcommand{\vsf}{{\sf v}}
\newcommand{\xsf}{{\sf x}}
\newcommand{\ysf}{{\sf y}}
\newcommand{\zsf}{{\sf z}}


% mixed symbols

\newcommand{\sinc}{{\hbox{sinc}}}
\newcommand{\diag}{{\hbox{diag}}}
\renewcommand{\det}{{\hbox{det}}}
\newcommand{\trace}{{\hbox{tr}}}
\newcommand{\sign}{{\hbox{sign}}}
\renewcommand{\arg}{{\hbox{arg}}}
\newcommand{\var}{{\hbox{var}}}
\newcommand{\cov}{{\hbox{cov}}}
\newcommand{\Ei}{{\rm E}_{\rm i}}
\renewcommand{\Re}{{\rm Re}}
\renewcommand{\Im}{{\rm Im}}
\newcommand{\eqdef}{\stackrel{\Delta}{=}}
\newcommand{\defines}{{\,\,\stackrel{\scriptscriptstyle \bigtriangleup}{=}\,\,}}
\newcommand{\<}{\left\langle}
\renewcommand{\>}{\right\rangle}
\newcommand{\herm}{{\sf H}}
\newcommand{\trasp}{{\sf T}}
\newcommand{\transp}{{\sf T}}
\renewcommand{\vec}{{\rm vec}}
\newcommand{\Psf}{{\sf P}}
\newcommand{\SINR}{{\sf SINR}}
\newcommand{\SNR}{{\sf SNR}}
\newcommand{\MMSE}{{\sf MMSE}}
\newcommand{\REF}{{\RED [REF]}}

% Markov chain
\usepackage{stmaryrd} % for \mkv 
\newcommand{\mkv}{-\!\!\!\!\minuso\!\!\!\!-}

% Colors

\newcommand{\RED}{\color[rgb]{1.00,0.10,0.10}}
\newcommand{\BLUE}{\color[rgb]{0,0,0.90}}
\newcommand{\GREEN}{\color[rgb]{0,0.80,0.20}}

%%%%%%%%%%%%%%%%%%%%%%%%%%%%%%%%%%%%%%%%%%
\usepackage{hyperref}
\hypersetup{
    bookmarks=true,         % show bookmarks bar?
    unicode=false,          % non-Latin characters in AcrobatÕs bookmarks
    pdftoolbar=true,        % show AcrobatÕs toolbar?
    pdfmenubar=true,        % show AcrobatÕs menu?
    pdffitwindow=false,     % window fit to page when opened
    pdfstartview={FitH},    % fits the width of the page to the window
%    pdftitle={My title},    % title
%    pdfauthor={Author},     % author
%    pdfsubject={Subject},   % subject of the document
%    pdfcreator={Creator},   % creator of the document
%    pdfproducer={Producer}, % producer of the document
%    pdfkeywords={keyword1} {key2} {key3}, % list of keywords
    pdfnewwindow=true,      % links in new window
    colorlinks=true,       % false: boxed links; true: colored links
    linkcolor=red,          % color of internal links (change box color with linkbordercolor)
    citecolor=green,        % color of links to bibliography
    filecolor=blue,      % color of file links
    urlcolor=blue           % color of external links
}
%%%%%%%%%%%%%%%%%%%%%%%%%%%%%%%%%%%%%%%%%%%



\newcommand{\tst}{\mathtt{test}}
\newcommand{\ag}{\mathtt{Ag}}
\newcommand{\bp}{\mathtt{Bp}}
\newcommand{\pac}{\mathtt{PAC}}
\newcommand{\tr}{\mathtt{Tr}}
\newcommand{\y}{\bvec{y}}
\newcommand{\x}{\bvec{x}}
\newcommand{\h}{\bvec{h}}
\newcommand{\dist}{\mathtt{dist}}
%\DeclareMathOperator{\tst}{test}
\newcommand{\vcd}{\mathtt{\mbox{VC-Dim}}}
\newcommand{\dsd}{\mathtt{\mbox{DS-Dim}}}


\title{PAC Learning is just Bipartite Matching\\ (Sort of) %\\ EARLY DRAFT (DO NOT DISTRIBUTE)
}

\author{
  Shaddin Dughmi\thanks{Supported by NSF Grant CCF-2009060} \\
Department of Computer Science\\
University of Southern California\\
{\tt shaddin@usc.edu}
}








\begin{document}

\maketitle

%\begin{abstract}
%\begin{abstract}  
Test time scaling is currently one of the most active research areas that shows promise after training time scaling has reached its limits.
Deep-thinking (DT) models are a class of recurrent models that can perform easy-to-hard generalization by assigning more compute to harder test samples.
However, due to their inability to determine the complexity of a test sample, DT models have to use a large amount of computation for both easy and hard test samples.
Excessive test time computation is wasteful and can cause the ``overthinking'' problem where more test time computation leads to worse results.
In this paper, we introduce a test time training method for determining the optimal amount of computation needed for each sample during test time.
We also propose Conv-LiGRU, a novel recurrent architecture for efficient and robust visual reasoning. 
Extensive experiments demonstrate that Conv-LiGRU is more stable than DT, effectively mitigates the ``overthinking'' phenomenon, and achieves superior accuracy.
\end{abstract}  
%\end{abstract}

\section{Introduction}


\begin{figure}[t]
\centering
\includegraphics[width=0.6\columnwidth]{figures/evaluation_desiderata_V5.pdf}
\vspace{-0.5cm}
\caption{\systemName is a platform for conducting realistic evaluations of code LLMs, collecting human preferences of coding models with real users, real tasks, and in realistic environments, aimed at addressing the limitations of existing evaluations.
}
\label{fig:motivation}
\end{figure}

\begin{figure*}[t]
\centering
\includegraphics[width=\textwidth]{figures/system_design_v2.png}
\caption{We introduce \systemName, a VSCode extension to collect human preferences of code directly in a developer's IDE. \systemName enables developers to use code completions from various models. The system comprises a) the interface in the user's IDE which presents paired completions to users (left), b) a sampling strategy that picks model pairs to reduce latency (right, top), and c) a prompting scheme that allows diverse LLMs to perform code completions with high fidelity.
Users can select between the top completion (green box) using \texttt{tab} or the bottom completion (blue box) using \texttt{shift+tab}.}
\label{fig:overview}
\end{figure*}

As model capabilities improve, large language models (LLMs) are increasingly integrated into user environments and workflows.
For example, software developers code with AI in integrated developer environments (IDEs)~\citep{peng2023impact}, doctors rely on notes generated through ambient listening~\citep{oberst2024science}, and lawyers consider case evidence identified by electronic discovery systems~\citep{yang2024beyond}.
Increasing deployment of models in productivity tools demands evaluation that more closely reflects real-world circumstances~\citep{hutchinson2022evaluation, saxon2024benchmarks, kapoor2024ai}.
While newer benchmarks and live platforms incorporate human feedback to capture real-world usage, they almost exclusively focus on evaluating LLMs in chat conversations~\citep{zheng2023judging,dubois2023alpacafarm,chiang2024chatbot, kirk2024the}.
Model evaluation must move beyond chat-based interactions and into specialized user environments.



 

In this work, we focus on evaluating LLM-based coding assistants. 
Despite the popularity of these tools---millions of developers use Github Copilot~\citep{Copilot}---existing
evaluations of the coding capabilities of new models exhibit multiple limitations (Figure~\ref{fig:motivation}, bottom).
Traditional ML benchmarks evaluate LLM capabilities by measuring how well a model can complete static, interview-style coding tasks~\citep{chen2021evaluating,austin2021program,jain2024livecodebench, white2024livebench} and lack \emph{real users}. 
User studies recruit real users to evaluate the effectiveness of LLMs as coding assistants, but are often limited to simple programming tasks as opposed to \emph{real tasks}~\citep{vaithilingam2022expectation,ross2023programmer, mozannar2024realhumaneval}.
Recent efforts to collect human feedback such as Chatbot Arena~\citep{chiang2024chatbot} are still removed from a \emph{realistic environment}, resulting in users and data that deviate from typical software development processes.
We introduce \systemName to address these limitations (Figure~\ref{fig:motivation}, top), and we describe our three main contributions below.


\textbf{We deploy \systemName in-the-wild to collect human preferences on code.} 
\systemName is a Visual Studio Code extension, collecting preferences directly in a developer's IDE within their actual workflow (Figure~\ref{fig:overview}).
\systemName provides developers with code completions, akin to the type of support provided by Github Copilot~\citep{Copilot}. 
Over the past 3 months, \systemName has served over~\completions suggestions from 10 state-of-the-art LLMs, 
gathering \sampleCount~votes from \userCount~users.
To collect user preferences,
\systemName presents a novel interface that shows users paired code completions from two different LLMs, which are determined based on a sampling strategy that aims to 
mitigate latency while preserving coverage across model comparisons.
Additionally, we devise a prompting scheme that allows a diverse set of models to perform code completions with high fidelity.
See Section~\ref{sec:system} and Section~\ref{sec:deployment} for details about system design and deployment respectively.



\textbf{We construct a leaderboard of user preferences and find notable differences from existing static benchmarks and human preference leaderboards.}
In general, we observe that smaller models seem to overperform in static benchmarks compared to our leaderboard, while performance among larger models is mixed (Section~\ref{sec:leaderboard_calculation}).
We attribute these differences to the fact that \systemName is exposed to users and tasks that differ drastically from code evaluations in the past. 
Our data spans 103 programming languages and 24 natural languages as well as a variety of real-world applications and code structures, while static benchmarks tend to focus on a specific programming and natural language and task (e.g. coding competition problems).
Additionally, while all of \systemName interactions contain code contexts and the majority involve infilling tasks, a much smaller fraction of Chatbot Arena's coding tasks contain code context, with infilling tasks appearing even more rarely. 
We analyze our data in depth in Section~\ref{subsec:comparison}.



\textbf{We derive new insights into user preferences of code by analyzing \systemName's diverse and distinct data distribution.}
We compare user preferences across different stratifications of input data (e.g., common versus rare languages) and observe which affect observed preferences most (Section~\ref{sec:analysis}).
For example, while user preferences stay relatively consistent across various programming languages, they differ drastically between different task categories (e.g. frontend/backend versus algorithm design).
We also observe variations in user preference due to different features related to code structure 
(e.g., context length and completion patterns).
We open-source \systemName and release a curated subset of code contexts.
Altogether, our results highlight the necessity of model evaluation in realistic and domain-specific settings.





\section{Background}\label{sec:backgrnd}

\subsection{Cold Start Latency and Mitigation Techniques}

Traditional FaaS platforms mitigate cold starts through snapshotting, lightweight virtualization, and warm-state management. Snapshot-based methods like \textbf{REAP} and \textbf{Catalyzer} reduce initialization time by preloading or restoring container states but require significant memory and I/O resources, limiting scalability~\cite{dong_catalyzer_2020, ustiugov_benchmarking_2021}. Lightweight virtualization solutions, such as \textbf{Firecracker} microVMs, achieve fast startup times with strong isolation but depend on robust infrastructure, making them less adaptable to fluctuating workloads~\cite{agache_firecracker_2020}. Warm-state management techniques like \textbf{Faa\$T}~\cite{romero_faa_2021} and \textbf{Kraken}~\cite{vivek_kraken_2021} keep frequently invoked containers ready, balancing readiness and cost efficiency under predictable workloads but incurring overhead when demand is erratic~\cite{romero_faa_2021, vivek_kraken_2021}. While these methods perform well in resource-rich cloud environments, their resource intensity challenges applicability in edge settings.

\subsubsection{Edge FaaS Perspective}

In edge environments, cold start mitigation emphasizes lightweight designs, resource sharing, and hybrid task distribution. Lightweight execution environments like unikernels~\cite{edward_sock_2018} and \textbf{Firecracker}~\cite{agache_firecracker_2020}, as used by \textbf{TinyFaaS}~\cite{pfandzelter_tinyfaas_2020}, minimize resource usage and initialization delays but require careful orchestration to avoid resource contention. Function co-location, demonstrated by \textbf{Photons}~\cite{v_dukic_photons_2020}, reduces redundant initializations by sharing runtime resources among related functions, though this complicates isolation in multi-tenant setups~\cite{v_dukic_photons_2020}. Hybrid offloading frameworks like \textbf{GeoFaaS}~\cite{malekabbasi_geofaas_2024} balance edge-cloud workloads by offloading latency-tolerant tasks to the cloud and reserving edge resources for real-time operations, requiring reliable connectivity and efficient task management. These edge-specific strategies address cold starts effectively but introduce challenges in scalability and orchestration.

\subsection{Predictive Scaling and Caching Techniques}

Efficient resource allocation is vital for maintaining low latency and high availability in serverless platforms. Predictive scaling and caching techniques dynamically provision resources and reduce cold start latency by leveraging workload prediction and state retention.
Traditional FaaS platforms use predictive scaling and caching to optimize resources, employing techniques (OFC, FaasCache) to reduce cold starts. However, these methods rely on centralized orchestration and workload predictability, limiting their effectiveness in dynamic, resource-constrained edge environments.



\subsubsection{Edge FaaS Perspective}

Edge FaaS platforms adapt predictive scaling and caching techniques to constrain resources and heterogeneous environments. \textbf{EDGE-Cache}~\cite{kim_delay-aware_2022} uses traffic profiling to selectively retain high-priority functions, reducing memory overhead while maintaining readiness for frequent requests. Hybrid frameworks like \textbf{GeoFaaS}~\cite{malekabbasi_geofaas_2024} implement distributed caching to balance resources between edge and cloud nodes, enabling low-latency processing for critical tasks while offloading less critical workloads. Machine learning methods, such as clustering-based workload predictors~\cite{gao_machine_2020} and GRU-based models~\cite{guo_applying_2018}, enhance resource provisioning in edge systems by efficiently forecasting workload spikes. These innovations effectively address cold start challenges in edge environments, though their dependency on accurate predictions and robust orchestration poses scalability challenges.

\subsection{Decentralized Orchestration, Function Placement, and Scheduling}

Efficient orchestration in serverless platforms involves workload distribution, resource optimization, and performance assurance. While traditional FaaS platforms rely on centralized control, edge environments require decentralized and adaptive strategies to address unique challenges such as resource constraints and heterogeneous hardware.



\subsubsection{Edge FaaS Perspective}

Edge FaaS platforms adopt decentralized and adaptive orchestration frameworks to meet the demands of resource-constrained environments. Systems like \textbf{Wukong} distribute scheduling across edge nodes, enhancing data locality and scalability while reducing network latency. Lightweight frameworks such as \textbf{OpenWhisk Lite}~\cite{kravchenko_kpavelopenwhisk-light_2024} optimize resource allocation by decentralizing scheduling policies, minimizing cold starts and latency in edge setups~\cite{benjamin_wukong_2020}. Hybrid solutions like \textbf{OpenFaaS}~\cite{noauthor_openfaasfaas_2024} and \textbf{EdgeMatrix}~\cite{shen_edgematrix_2023} combine edge-cloud orchestration to balance resource utilization, retaining latency-sensitive functions at the edge while offloading non-critical workloads to the cloud. While these approaches improve flexibility, they face challenges in maintaining coordination and ensuring consistent performance across distributed nodes.


\section{A Transductive Model of Learning, and why it's Good Enough}
\label{sec:trans}

I will now present the learning model that will be our main playground in this article, then try to convince you that we lose little by restricting our attention to it. This model was first employed by \citet{haussler_predicting_1994}, and has been quite busy in recent years, with its learners used as precursors to PAC learners (e.g. \cite{daniely_optimal_2014,brukhim_characterization_2022,aden-ali_optimal_2023,daskalakis_is_2024,asilis_regularization_2024,montasser_adversarially_2022}).
%
I will describe the model as conceptualized by \citet{daniely_optimal_2014} and further developed by  \citet{asilis_regularization_2024}, and refer to it simply as \emph{the transductive model}  in keeping with some of the recent literature.   % I will call this ``the single-prediction adversarial transductive model'', or simply  ``the transductive model'' as it is referred to in recent literature.
%The transductive model  has been quite busy in recent years, with its learners used as precursors to PAC learners with various guarantees (e.g. \cite{daniely_optimal_2014,brukhim_characterization_2022,aden-ali_optimal_2023,daskalakis_is_2024,asilis_regularization_2024,montasser_adversarially_2022}).
%
Truth be told, however,  the transductive approach to learning --- first explicitly articulated by \citet[Chapter 6.1]{vapnik_nature_1998} --- is much broader than just this particular model. Most notably, the transductive model I use in this article is concerned with making only a single prediction, and presumes data is chosen by a particular strong adversary. We will come back to transduction more generally later.   %\footnote{One could argue that a more appropriate moniker would be ``the single-prediction adversarial transductive model'', to place it within the broader transductive approach  first explicitly articulated by \citet[Chapter 6.1]{vapnik_nature_1998}.}

\subsection{The Transductive Model}
The (single-prediction, adversarial) transductive model is  a game of ``fill in the blank'' played between a learner and an adversary. For a given positive integer $n$, the game proceeds as follows:
\begin{enumerate}
\item The adversary chooses a \emph{labeled dataset} $(x_1,y_1), \ldots,(x_n,y_n)$, with $x_i \in \X$ and $y_i \in \Y$. \\(The adversary may  be constrained in their choice of dataset --- more on this shortly.)
\item One label $y_i$, chosen uniformly at random, is hidden. The remaining data \[(x_1,y_1),\ldots,(x_{i-1},y_{i-1}),(x_i,?),(x_{i+1},y_{i+1}),\ldots,(x_n,y_n),\] where ``?'' is a ``blank'' symbol obscuring $y_i$, is displayed to the learner. 
%\item One example $(x_i,y_i)$ is chosen uniformly at random. The remaining examples $T_{-i} =(x_1,y_1),\ldots,(x_{i-1},y_{i-1}),(x_{i+1},y_{i+1}),\ldots,(x_n,y_n)$ are displayed to the learner. 
\item The learner is prompted to make a prediction $y'_i$ for the label of $x_i$, i.e., to ``fill in the blank'', incurring loss $\ell(y'_i,y_i)$.
\end{enumerate}
%TODO: Figure with cats and blank and such

The adversary in the transductive model chooses the training and test data, but cannot control which is which: the identity of the test example $(x_i,y_i)$ is chosen uniformly at random. Then, the learner is fed training data   $\set{(x_j,y_j) : j \neq i}$ and prompted with the test datapoint $x_i$. In relation to  PAC learning, where training and test data are sourced i.i.d.~from some distribution, the transductive model can be viewed as ``hard-coding'' a uniform distribution over $n$ examples then drawing training and test data \emph{without replacement}.






We will distinguish two settings of the transductive model, \emph{realizable} and \emph{agnostic}, analogous to the same in the PAC model. The transductive model was originally considered in the realizable setting~\cite{haussler_predicting_1994,daniely_optimal_2014}, where the adversary is constrained to instances $(x_1,y_1), \ldots, (x_n,y_n)$ which are perfectly explained by a some hypothesis $h \in \H$, meaning $h(x_j) = y_j$ for all examples. The (realizable) \emph{transductive error}  of a learner on datasets of size $n$ is simply the expected loss it incurs in this game, i.e., for a worst-case realizable instance $(x_1,y_1), \ldots, (x_n,y_n)$ chosen by the adversary. Here, the expectation is over the random identity of the test example $(x_i,y_i)$ as well as any internal randomness to the learner. In the agnostic setting of this model, as articulated in~\cite{asilis_regularization_2024}, the adversary is completely unconstrained in their choice of instance $(x_1,y_1), \ldots, (x_n,y_n)$. We then shift the goalposts by subtracting the best-in-class loss on the same instance, as necessary to allow for nontrivial  learning: the \emph{agnostic transductive error} is the expected loss incurred by the learner minus $\min_{h \in \H} \frac{1}{n} \sum_{i=1}^n \ell(h(x_i),y_i)$.

Whether in the realizable or agnostic setting, we look for learners with small transductive error rate $\epsilon(n)$, as a function of the dataset size $n$. An equivalent perspective looks at the sample complexity $m(\epsilon)$, which is the minimum dataset size guaranteeing  error at most $\epsilon$. We say a problem is \emph{learnable} in the transductive model if its transductive sample complexity is finite for every $\epsilon > 0$.




\begin{figure}%[tbp]
  \centering
    \begin{subfigure}{0.3\textwidth}
        \centering
        \scalebox{0.8}{\input{figures/rectangles.tikz}}
        \caption{Labeled dataset consistent with an axis-aligned rectangle.}
        \label{fig:rectangles_a}
    \end{subfigure}
   \hfill % Optional: this inserts space between the two figures
    \begin{subfigure}{0.3\textwidth}
        \centering
        \scalebox{0.8}{\input{figures/rectangles_b.tikz}}
        \caption{When any label is omitted, both + and - are consistent with some rectangle.}
        \label{fig:rectangles_b}
      \end{subfigure}
      \hfill
    \begin{subfigure}{0.3\textwidth}
      \centering
      \scalebox{0.8}{\input{figures/rectangles_c.tikz}}
      \caption{The minimal axis-aligned rectangle is sensitive only to the four + points defining it.}
      \label{fig:rectangles_c}
    \end{subfigure}
    \caption{Realizable binary classification for axis aligned rectangles in the transductive model.}
    \label{fig:rectangles}
\end{figure}

\paragraph{Example: Axis-aligned Rectangles.} To illustrate the transductive model, consider realizable binary classification of axis-aligned rectangles as described in~\cite{haussler_predicting_1994} and illustrated in Figure~\ref{fig:rectangles}. The adversary selects $n$ datapoints in Euclidean space $\RR^d$, and labels them  in a manner consistent with some axis-aligned rectangle as in Figure~\ref{fig:rectangles_a}. A random label is then omitted, and the learner is prompted to predict this label. A learner which merely minimizes empirical risk --- i.e., arbitrarily chooses an axis-aligned rectangle which is consistent with observed labels, and fills in the missing label accordingly --- can make a mistake with probability $1$. Such a scenario is illustrated in Figure~\ref{fig:rectangles_b}. This stands in contrast to the PAC model, where empirical risk minimization guarantees a misclassification rate of $O(d/n)$ with high probability, since axis-aligned rectangles in $\RR^d$ have a VC dimension of $2d$. A more careful choice of learning rule guarantees an error probability of $\frac{2d}{n}$ also in the transductive model: choose the \emph{smallest} axis-aligned rectangle consistent with the observed labels, and fill in the missing label accordingly. This is illustrated in Figure~\ref{fig:rectangles_c}.



\paragraph{Analogy to Hat Puzzles.} The reader familiar with \emph{hat puzzles} (see e.g. \cite{krzywkowski_hat_2010,butler_hat_2009}) might notice a similarity to the transductive model. There are many variations of hat puzzles, the most relevant to us of which look something like this: There are $n$ players and a supply of hats of $k$ different colors. An adversary places a hat on the head of each of the players. Each player must then guess the color of their own hat by looking at all, or in some formulations some, of the hats of the other players. The players formulate a joint strategy in advance, and a typical goal is to maximize the number of correct guesses.

When the transductive model is applied to classification in the realizable setting, the  analogy to hat puzzles is clear: the players are $x_1,\ldots,x_n$, the hat colors are $y_1,\ldots,y_n$, and player $i$'s guessing task corresponds to the scenario in which $y_i$ is hidden. The learner  therefore corresponds to the joint strategy of the players, and transductive error is proportional to the number of players guessing incorrectly. 

There are important differences between hat puzzles and the transductive model, however. To my knowledge, most hat puzzles permit the adversary to assign hat colors either arbitrarily or uniformly at random. The transductive model, in contrast, is constrained to the hat assignments permitted by the hypothesis class (in the realizable setting, at least). Another important difference is that hat puzzles are commonly concerned with deterministic guessing strategies.  This is because the optimal randomized strategy against such a powerful adversary is the trivial one: each player randomly guesses their hat color. Obtaining a similar guarantee with a deterministic strategy is therefore the more interesting puzzle here. %trivial randomized strategy whereby each player randomly guesses their strategy of randomly guessing the hat color is optimal since the optimal randomized strategies are trivial given the power of the unconstrained adversary.

To my knowledge, the analogy between hat puzzles and transductive learning appears completely absent from the literature. Despite their differences, there might yet be fruitful connections between the two areas. One tantalizing tidbit is that the hypercube  perspective on hat puzzles articulated by \citet{butler_hat_2009} appears to be a special case of the \emph{one-inclusion graphs} that are almost synonymous with the transductive model --- more on these graphs in Section~\ref{sec:class}.



\paragraph{A Note on Transduction vs Induction, more generally.} The usual  conception of supervised learning is as a problem of \emph{inductive inference}: reasoning from particular observations (the training data) to a general rule (the chosen predictor), which can then be applied to answer specific questions at test time (labeling the test data). A (seemingly) more permissive approach is to care not for the rule itself, but for its evaluation at the particular points of interest; i.e.,  we reason from particular observations (the training data) to answer specific questions (labeling the test data). This is \emph{transductive learning} as first explicitly articulated by~\citet{vapnik_nature_1998}, though the essential ideas date back to \citet{vapnik_theory_1974} and \citet{vapnik_estimation_1982}. Notice that this general transductive approach does not presuppose a particular data model or objective, and can be considered in other setups such as the PAC model, online learning, etc.

When there is only a single test point, and learners are allowed to be improper, induction and transduction are essentially equivalent. Indeed,  the formal distinction between the two becomes simply a matter of \emph{currying}\footnote{Currying is a notion common in functional programming, and amounts to the observation that a function \linebreak $g: A \times B \to \C$ can equivalently be viewed as a function $g': A \to (B \to C)$, with $g'(a)$ itself a function from $B$ to $C$.}  --- see the related discussion in~\cite{montasser_transductive_2022}. This equivalence no longer holds in general when multiple test datapoints must be labeled, nor in the proper regime. Since we operate in the improper regime with a single test example, the distinction between transduction and induction becomes merely a matter of perspective for our purposes. Nonetheless, the  transductive model is best thought of as reasoning for a specific test example, hence the name.
% TODO: Say something about how transductive and improper are essentially synonyms as far as we're concerned. Transductive is a perspective that enables improperness by default.
%By catering the learned rule to a specific test example, transduction can be viewed as ``defaulting'' to improper learners.

A related perspective on the distinction between transduction and induction  is the following. By catering the learned predictor to a specific test example, transductive learners are ``improper by default'' when viewed across all possible test points. Indeed, by focusing only on prediction and freeing the learner completely from representing any general rule, the transductive perspective unleashes the full power of improper learning. Induction, in contrast, is most meaningful when encoding an a-priori bias towards rules of a particular form, as is the case in the proper regime.



\subsection{Relationship to the PAC Model}
\label{subsec:pac_trans}
Despite seeming  incomparable at first glance, the PAC and transductive models turn out to be closely related.\footnote{This is specific to the  improper (i.e., unrestricted) regime of PAC learning, which is our focus in this article. The transductive model, in allowing improperness by default, is more permissive than proper PAC learning~\cite{daniely_optimal_2014}.} Speaking qualitatively, learnability is equivalent in the two models, and this holds for both realizable and agnostic learning problems with bounded loss functions.

Quantitatively, we can say something quite strong in the realizable setting with bounded losses: The two models are equivalent up to  low-order factors in the error and the sample complexity. This follows from  efficient black-box reductions  between the two models, one in each direction. We therefore lose very little, and gain quite a lot through the connection to matching and otherwise, by studying the transductive model instead of the PAC model for realizable learning problems.

The exact quantitative relationship between the two models  is more unsettled in the agnostic setting: Whereas it appears that PAC learning is essentially  no harder  transductive learning (up to low-order terms, for most natural loss functions, through an efficient black-box reduction), it remains plausible that transductive sample complexity exceeds its PAC counterpart by a multiplicative factor on the order of $1/\epsilon$, even for classification problems.
%
This gap of $1/\epsilon$ in agnostic sample complexities is the main asterisk to the stance I take in this article of treating the PAC and transductive models as interchangeable. Given  the transductive model's  fruitful connections to basic graph-theoretic questions, as well as the possibility that future work could eliminate this gap, I hope the reader finds this qualification forgivable.

I will now recap both the realizable and agnostic states of affairs in more detail.



\subsubsection*{The Realizable Setting}
  In the realizable setting with losses in $[0,1]$, the transductive and PAC models are equivalent up to  low-order factors in the error and the sample complexity. This is somewhat folklore knowledge in the field, but the proofs are written down explicitly in~\cite{asilis_regularization_2024}.  I'll recap those proofs informally next.

For one direction of the equivalence, a realizable PAC learner guaranteeing error $\epsilon$ with probability $1-\delta$ on $n$ samples can be converted to a realizable transductive learner guaranteeing error $O(\epsilon + \delta)$ on datasets consisting of $n$ samples. This is done in pretty much the obvious way: the PAC learner is trained on uniform draws from the transductive dataset, and invoked to make a prediction for the test point. Since there is a constant probability the test point is not in the training data, the PAC learner solves a strictly harder problem with constant probability, from which the bound follows by Markov's inequality.

For the other direction, a realizable transductive learner guaranteeing error $\epsilon$ on datasets of size $n$ can be  converted to a realizable PAC learner which guarantees error $O(\epsilon)$ with probability $1-\delta$ when given $O(n \log \frac{1}{\delta})$ samples. This again is a pretty elementary reduction: the transductive learner guarantees expected loss $\epsilon$ on a PAC instance via a \emph{leave one out} argument --- essentially conditioning on the training and test data and invoking the principle of deferred decisions with regards to which is which. This can then be boosted to a high probability guarantee by repetition $O(\log \frac{1}{\delta})$ times. Markov's inequality and the union bound show that one of the predictors has loss $O(\epsilon)$ with probability $1-\delta/2$, and a small hold-out set of examples can be used to identify such a predictor with probability $1-\delta/2$.

For the large class of metric loss functions, a more sophisticated  reduction gets rid of the multiplicative blowup in sample complexity \cite{aden-ali_optimal_2023,dughmi_is_2024}. Specifically, the number of samples is reduced from $O\left(n \log \frac{1}{\delta}\right)$ to $n+O\left(\frac{1}{\epsilon} \log \frac{1}{\delta}\right)$. This is possible through a more economical repetition construction which reuses examples across multiple invocations of the transductive learner.


\subsection*{The Agnostic Setting}
In relating the transductive and PAC models in the agnostic setting, the best we can currently say for general bounded loss functions is the following:  The previously-described realizable reductions go through with a degraded bound on the number of samples, by a factor of at most $O(1/\epsilon)$. To see why this might be, note that both those reductions rely on Markov's inequality, which guarantees that a (random) predictor with expected loss $\epsilon$ has loss at most $2 \epsilon$ with probability at least $1/2$. Agnostic error discounts the loss of a predictor by the best-in-class loss, and in doing so no longer enjoys the same Markov guarantee. Indeed, when $\epsilon$ denotes this discounted expected loss, the probability guarantee provided by Markov's inequality degrades from $1/2$ to $O(\epsilon)$. Carrying this through either reduction increases the sample complexity bound by a factor of~$O(1/\epsilon)$.

This multiplicative blowup of $1/\epsilon$ in sample complexity seems somewhat more fundamental in one of the directions than the other.  For the large class of metric loss functions, a transductive learner guaranteeing agnostic error $\epsilon$ on datasets of size $n$ can be converted to a PAC learner guaranteeing agnostic error $O(\epsilon)$ with probability $1-\delta$ when given $n+O\left(\frac{1}{\epsilon^2} \log \frac{1}{\epsilon\delta}\right)$ samples. This is through an adaptation of the economical repetition construction of \citet{aden-ali_optimal_2023} to the agnostic setting by  \citet{dughmi_is_2024}.

This suggests that PAC learning is essentially no harder than transductive learning for most natural loss functions. Whether the converse is true remains very much in question,  as discussed and explored in \cite{dughmi_is_2024}. Specifically, it remains plausible that transductive sample complexity exceeds PAC sample complexity by a factor of up to $O(1/\epsilon)$, even for multiclass classification. Such a gap in sample complexity was ruled out for binary classification through a non-reduction approach in~\cite{dughmi_is_2024}, where we also conjecture that the gap can be  eliminated more generally.


%The equivalence is perhaps less tight for the agnostic setting, with the naive reductions suffering a blowup of $O\left(\frac{1}{\epsilon}\right)$ in sample complexity --- whether this is avoidable will be tackled  in  Section~\ref{sec:structural}. Nonetheless, learnability is equivalent for the PAC and transductive models, and the sample complexities are related through efficient reductions. We therefore often blur the distinction between the two models in this proposal.







%%% Local Variables:
%%% mode: latex
%%% TeX-master: "learning_matching"
%%% End:

\section{Classification, One-Inclusion Graphs, and Bipartite Matching}
\label{sec:class}

A \emph{classification problem} is a supervised learning problem with the 0-1 loss function $\ell_{0-1}(y',y)=[y' \neq y]$, which evaluates to $1$ when $y' \neq y$ and to $0$ when $y'=y$. It is important to note that this allows for an arbitrary domain $\X$, an arbitrary label set $\Y$, and an arbitrary hypothesis class $\H \sse \Y^\X$.  Classification therefore forms a rich and expressive class of problems,  garnering the lion's share of theoretical work on supervised learning (particularly in the PAC model), and featuring in essentially every application domain of machine learning. %. Classification problems are also prolific in practice, appearing in essentially every application domain of machine learning.
%
When there are only two labels, canonically $\Y= \set{0,1}$, we have a \emph{binary classification} problem.  More generally, \emph{multiclass classification} problems can have any number of labels, even infinitely many of any cardinality.

I will discuss classification problems in the transductive model, emboldened by its close relationship to the PAC model as described in Section~\ref{sec:trans}. I will start with a primer on one-inclusion graphs (OIGs), which encode transductive classification exactly  as a combinatorial optimization problem. I will then reinterpret this problem as bipartite matching, and discuss some recently-discovered implications of this interpretation.

It's worth noting that the transductive perspective is of perhaps limited utility for binary classification, where empirical risk minimization (ERM) has almost optimal PAC sample complexity~\cite{shalev-shwartz_understanding_2014}. It is in multiclass classification where transductive learners truly shine, in particular as the number of labels grows large. In fact, ERM can fail to learn entirely even for learnable multiclass problems~\cite{daniely_multiclass_2011}, as can SRM and any proper learner \cite{daniely_optimal_2014}.  To my knowledge, all known general-purpose learners for multiclass classification factor through the transductive model and  OIGs~\cite{daniely_optimal_2014, brukhim_characterization_2022,aden-ali_optimal_2023,asilis_regularization_2024}.





\subsection{One-Inclusion Graphs}
\label{sec:oig}
% Though transduction is most useful for multiclass problems, we start with binary classification for illustrative purposes.

The \emph{one-inclusion graph (OIG)} is a mathematical representation of classification in the transductive model of learning. OIGs were proposed by  \citet{haussler_predicting_1994} to model  realizable binary classification, and have since been  extended to realizable multiclass classification by \citet{rubinstein_shifting_2009}, and  to the agnostic setting by~\citet{asilis_regularization_2024}. As we will see shortly, OIGs represent these learning tasks exactly, with their \emph{orientations} in one-to-one correspondence with transductive learners. 

\paragraph{Realizable Binary Classification.} I will start by describing OIGs in their original form, restricted to realizable binary classification. Fix a domain $\X$ and a hypothesis class $\H \sse \set{0,1}^\X$. Recall that, in the  realizable setting of the transductive model, a collection of $n$  unlabeled datapoints $S=(x_1,\ldots,x_n) \in \X^n$ is chosen by an adversary, and provided to the learner at the outset. The adversary also selects a \emph{ground truth} hypothesis $h^* \in \H$,  then the learner is asked to predict the label $h^*(x_i)$ of a randomly-chosen test datapoint $x_i$ from the labels $h^*|_{S_{-i}}$ of the remaining $n-1$ datapoints.


\begin{wrapfigure}{R}{0.42\textwidth}
%\vspace{-0.3cm}
  \centering
  \input{figures/oig.tikz}
  \caption{Example OIG for realizable binary classification on three~datapoints.}
  \label{fig:binaryoig}
\end{wrapfigure}
The OIG is an undirected graph $G(\H,S)$  determined by the hypothesis class $\H$ and the unlabeled datapoints  $S$, both of which are known to the learner. The nodes correspond to the possible labelings $\H|_S$  of the datapoints $S$ by hypotheses in $\H$, of which there can be at most $2^n$.\footnote{A tighter bound is given by the \emph{growth function} $\Phi_d(n) = \sum_{i=0}^d \binom{n}{i}$, where $d$ is the VC dimension of $\H$.} %\footnote{For a binary hypothesis class of VC dimension $d$,  a tight upper bound on the number of possible labelings of $n$ datapoints, and hence the number of nodes of the OIG,  is given by the \emph{growth function} $\Phi_d(n) = \sum_{i=0}^d \binom{n}{i}$.} In other words, each node is a labeling $(y_1, \ldots, y_n) \in \set{-1,+1}^n$, with $y_i$ the label for $x_i$, consistent with some hypothesis $h \in \H$. 
  An edge is included between two labelings if they differ on exactly a single datapoint. Another perspective is that the OIG is simply the subgraph of the $n$-dimensional hypercube induced by $\H|_S$. An example OIG is depicted in Figure~\ref{fig:binaryoig}.


  
  Now consider a learner trying to predict the label of $x_i$ from labels $\set{y_j}_{j \neq i}$ provided for all the other datapoints. If there is a single node $(y_1, \ldots, y_i, \ldots, y_n)$  in $G$ that is consistent with the provided labels, then there is nothing to decide: $y_i$ must be the correct label for $x_i$. The interesting case is when there are two such nodes, one of which labels $x_i$ with a $0$ and the other with a~$1$. These two nodes, in differing only on $x_i$, must share an edge $e$ in the OIG. In such a situation, a learner predicting a label for $x_i$ chooses one of the two nodes to ``go with'', effectively \emph{orienting} $e$   towards the chosen prediction. In specifying a prediction for every such scenario, a learner orients all the edges of the OIG, turning it into a directed graph.  %In other words, the learner \emph{orients} the edge by imbuing it with a direction, turning it into a directed graph.
To summarize, learners are  in one-to-one correspondence with \emph{orientations} of the OIG.\footnote{More accurately, this is the case for deterministic learners. Randomized learners correspond to randomized orientations. This distinction is not especially consequential, so I blur it in this article.} 

  % outdegree
Given this graph-theoretic view of a transductive learner, we can also describe its error guarantee  in graph-theoretic terms: it is proportional to the maximum \emph{out-degree} of a node in the oriented graph. To see this, observe that each node  represents a possible ground truth labeling $\y=(y_1, \ldots, y_n)$ of the datapoints, and each edge represents an ambiguity faced by the learner looking to predict some $y_i$. When an edge is directed out of $\y$, this corresponds to a ``mistake'' by the learner when $\y$ happens to be the ground truth. Since the test datapoint~$x_i$ is chosen uniformly at random, and we evaluate error in the worst case over possible ground truths, we can conclude that the learner's error (i.e., worst-case misclassification rate) equals the maximum outdegree divided by $n$. Since each edge results in a mistake for one of its endpoints, one can think of transductive learning as seeking to ``spread the error around'' as evenly as possible among all possible ground truths.



% multiclass
\paragraph{Realizable Multiclass Classification.} One-inclusion graphs generalize naturally to multiclass classification, as articulated by \cite{rubinstein_shifting_2009}. For a set $\Y$ of labels, which may be finite or infinite, the nodes of the OIG are still the  ground truth labelings $\y=(y_1,\ldots,y_n) \in \Y^n$ consistent with some hypothesis. The edges, however, turn into hyper-edges, with a collection of nodes sharing a hyper-edge precisely when they all disagree on \emph{the same} $y_i$, and no other.  An OIG involving  three datapoints and three labels  is depicted in Figure~\ref{fig:oig}. %It will be sometimes convenient to identify a  hyper-edge with a \emph{partial labeling} $(y_1,\ldots, ? , \ldots y_n)$ with $y_j \in \Y$ for all but a single ``?'', where all nodes agreeing with the partial labeling on all but the ``?'' are included in the hyper-edge.
The hyper-edges again correspond to the ambiguities faced by the learner, where multiple predictions (up to $|\Y|$ many) are consistent with provided data. A learner can therefore be seen as orienting a hyper-edge towards the node corresponding to its prediction, and away from the others. As before, a ground truth incurs one ``mistake'' for every incident edge directed away from it.  Defining the out-degree of a node as the number of its incident edges directed away from it, the learner's error is again the maximum out-degree divided by $n$.

\definecolor{myblue}{RGB}{0,80,160}
\definecolor{mygreen}{RGB}{80,160,80}
\begin{figure}[tbp]
    \centering
    \begin{subfigure}[b]{0.45\textwidth}
        \centering
        \input{figures/hoig.tikz}
        \caption{Traditional OIG. Nodes depicted with circles,  hyper-edges with rectangles.}
        \label{fig:oig}
    \end{subfigure}
    \hfill % Optional: this inserts space between the two figures
    \begin{subfigure}[b]{0.45\textwidth}
        \centering
        \input{figures/bpoig.tikz}
        \caption{Bipartite OIG. Left side nodes with a single neighbor, such as $(0,0,?)$, omitted for clarity.}
        \label{fig:bpoig}
    \end{subfigure}
    \caption{Example OIG for realizable multiclass classification, in traditional and bipartite forms.}
    \label{fig:bothoig}
\end{figure}




%%% Local Variables:
%%% mode: latex
%%% TeX-master: "../learning_matching.tex"
%%% End:



\paragraph{Agnostic Multiclass Classification.} 
OIGs were recently further generalized to the agnostic setting of multiclass classification by  \citet{asilis_regularization_2024}, building on a closely-related definition employed by \citet{long_complexity_1999} for a different model of learning. This \emph{agnostic OIG}, denoted $G^\ag(\H,S)$, is as follows.  Since the ground truth labeling of the given datapoints $S=(x_1,\ldots,x_n)$ may now be completely arbitrary, we include all labelings $\Y^n$ as nodes of $G^\ag$.  As in the realizable case, a collection of nodes share a hype-edge precisely when they disagree on the same $y_i$ and no other.   Also as in the realizable case, a learner is an orientation of the hyper-edges, and the learner's error is related to out-degrees of all the nodes.

But what of the hypothesis class $\H$? Since we are in the agnostic setting, $\H$ serves merely as a benchmark for our learner. Therefore, in defining the OIG orientation problem we now ``discount'' the out-degree of each node  $\y=(y_1,\ldots,y_n)$ of $G^\ag$ by its \emph{Hamming distance} $\dist(y,\H)$ from $\H|_S$.\footnote{The Hamming distance between two vectors is the number of entries on which they differ. For the distance to a set of vectors, we minimize over that set appropriately.} %For error rate $\epsilon$, each labeling $\y$ on the right side of $G_\bp^\ag$ must be ``matched'' at least $(1-\epsilon)n - d(\y,\H)$ times, where $d(\y,\H)$ is the aforementioned discount for node $\y$.
This is quite natural, as the more the ground truth deviates from our benchmark class $\H$, the more we are ``off the hook'' for making accurate predictions. 



% computation, brute force, parsimonious need
\paragraph{A Note on Computational Complexity.} Some remarks are in order with regard to the computational complexity of working with the OIG. Even for binary classification with a finite VC dimension $d$, the OIG can have on the order of $n^d$ nodes. This is prohibitively large for all but the smallest values of $d$. In multiclass classification the situation can be even more dire. In fact, there are learnable problems with infinitely many labels and  OIGs of infinite size! It is therefore unsurprising that OIGs have usually not been thought of as a practical data structure for learning, but rather as abstractions for its mathematical study. In some sense, orienting the OIG can be viewed as ``brute-forcing'' the learning task: a representation of the algorithm's entire input/output behavior, at least with respect to labelings, is computed before even looking at the labels in the training data!  This stands in contrast to practical approaches  such as ERM and SRM (see Section~\ref{sec:background}), %, such as empirical risk minimization and structural risk minimization,
which represent the task parsimoniously and reason locally, input by input. Nonetheless, I will argue in this article that insights emanating from OIGs, and extensions of them, can  point the way to such tenable algorithms.



\subsection{The Matching Perspective}
\label{subsec:matching}

One inclusion graphs transform multiclass learning to what is essentially a combinatorial optimization problem on orientations, albeit one that is described implicitly. From here, a simple shift in perspective articulated by \cite{asilis_regularization_2024}, employing the classical connection between orientation and matching problems (see e.g. \cite{Schrijver_2003}), yields an equivalent bipartite matching problem. I will describe this next.

Let $G$ be an OIG associated with hypothesis class $\H \sse \Y^\X$ and unlabeled dataset $S=(x_1,\ldots,x_n)$. The transformation is essentially identical whether we let $G$ be the realizable or agnostic OIG, but I recommend the reader keep the realizable OIG in mind to keep things simple. On the right we have the  nodes of $G$, corresponding to possible ground-truth labelings $\y=(y_1,\ldots,y_n)$ of  $(x_1,\ldots,x_n)$ . On the left we have \emph{partial labelings} which occlude a single label from some ground truth. We represent these as a vector $(y_1, \ldots, ?, \ldots, y_n)$,  where $y_i \in \Y$ for all but a single ``?'' entry corresponding to the occluded label. An edge is included between a partial labeling on the left and a full labeling on the right if they agree on all but occluded label. 
In graph-theoretic terms,  the bipartite OIG $G_{\bp}$ is essentially  the \emph{edge-vertex incidence graph} of $G$, were $G$ to first be augmented with self-loops (i.e., singleton hyper-edges) so that every node has degree exactly~$n$. %We sometimes refer to $G_{\bp}$ as the \emph{bipartite OIG} for convenience.
A bipartite OIG for the realizable setting is depicted in Figure~\ref{fig:bpoig}.

The  left side nodes of $G_{\bp}$ --- the partial labelings --- represent the scenarios faced by the learner, where one label is omitted and the learner is asked to ``fill in the blank.'' Edges of $G_{\bp}$ encode the consistency relationship between the partial labelings on the left and the (full) labelings on the right. A learner, in having to predict each occluded label, can be viewed as an \emph{assignment} mapping each node on the left to one of its neighbors on the right. Note that each node on the right has degree exactly $n$ in $G_{\bp}$, one for each possible position of the ``?'',  whereas nodes on the left may have degree up to the number of labels $|\Y|$ (exactly $|\Y|$ in the agnostic setting). % with a corresponding learner in  Figure~\ref{fig:??}. %original OIG, corresponding bipartite OIG, highlighted 


Given this perspective on learners as assignments in the bipartite OIG, what of their error? It is easier to take a complementary perspective and describe the learner's \emph{accuracy}: the probability it correctly classifies the test point, in the worst case over ground truths. For a particular ground truth labeling $\y=(y_1,\ldots,y_n)$ on the right side of $G_{\bp}$, the learner faces $n$ equally likely scenarios, one for each left node $\tilde{\y} = (y_1, \ldots, ? , \ldots, y_n)$ incident on $\y$. The learner correctly classifies the test point precisely when it assigns $\tilde{\y}$ to  $\y$. Therefore, viewing a learner as a left to right assignment in $G_{\bp}$, its accuracy is its minimum degree to a node on the right, divided by $n$.  In the realizable setting, the error  is of course the complementary probability. In the agnostic setting, we discount each degree of a node $\y$ on the right by its Hamming distance $\dist(\y,\H)$ to the hypothesis class, then proceed identically to arrive at the agnostic error. 

We can now cast learning  as a  bipartite matching problem. To obtain error $\epsilon$, each node $\y$ on the right side of $G_{\bp}$ must be assigned --- we say \emph{matched} --- at least $(1-\epsilon) n$ times in the realizable case, and at least $(1-\epsilon) n - \dist(\y,\H)$ times in the agnostic case. This is what is referred to in combinatorial optimization as a  bipartite \emph{$b$-matching problem}, a generalization of bipartite matching where nodes can be prescribed a minimum required number of matches. This generalization is exclusively for convenience, as a bipartite $b$-matching problem can be equivalently posed as  a run-of-the-mill  maximum bipartite matching problem by appropriately cloning nodes that require multiple matches. It must be noted, however, that $G_{\bp}$ can like $G$ be an  infinite graph, and even when finite is typically prohibitively large for invoking matching algorithms directly.

What, then, does this matching perspective on classification  buy us? Quite a bit, as it turns out. I will describe some recent consequences  of this viewpoint next.


\subsection{Implication of Matching: A Compact Graph-Theoretic Characterization}
%We now describe a shift in perspective articulated in the PI's recent joint work \cite{asilis_regularization_2023}, one which reframes  realizable classification  as a bipartite matching problem. This new lens, while remarkably simple,  allows us to bring to bear tools from combinatorics, matching theory, and optimization to the analysis of learning. Furthermore, it sets the stage for generalizations of the OIG that go beyond realizable classification.
\label{sec:hall}



Characterizing the optimal sample complexity of  learning  is a chief concern of learning theory. Given the simplicity of the transductive model, one might hope for a crisp and interpretable expression for the sample complexity, or equivalently the error rate, of the optimal transductive learner. Such an expression would approximate the performance of the optimal PAC learner, as per the relationships described  in Section \ref{subsec:pac_trans}. The matching perspective yields such a graph-theoretic expression which captures the transductive  error rate ``on the nose'' \cite{asilis_regularization_2024}. Moreover, this expression refers only to  \emph{finite projections} of the learning problem.


%\begin{figure}
\begin{wrapfigure}{R}{0.41\textwidth}
%  \vspace{-0.5cm}
  \centering
  \scalebox{1}{\input{figures/infinite_matching.tikz}}
  \caption{Matching on infinite bipartite graphs absent degree constraints.}
  \label{fig:infinite_matching}
\end{wrapfigure}
%\end{figure}
The starting point here is Philip Hall's classic \emph{marriage theorem}~\cite{hall1987representatives}, which characterizes exactly when one side of a \emph{finite} bipartite graph can be matched to the other. Let $L$ and $R$ denote the left and right side nodes of the graph, respectively, and consider whether there is a matching which matches every node on the right. The original marriage theorem states that such a matching exists  if and only if every set $R' \sse R$ of right side nodes has at least $|R'|$ neighbors between them (on the left, naturally). Whereas this \emph{Hall condition} characterizes matching on finite bipartite graphs, it fails to extend to all infinite bipartite graphs, where it is necessary but not sufficient in general. This can be seen by way of a simple example, illustrated in Figure~\ref{fig:infinite_matching}.% as evidenced by a simple example.\footnote{Suppose $L=\set{\ell_1,\ell_2,\ldots}$ is indexed by the positive integers, $R=\set{r_0,r_1,r_2,\ldots}$ is indexed by the nonnegative integers, and the edges include $(\ell_i,r_i)$ and $(\ell_i,0)$ for all positive integers $i$. The Hall condition holds for all finite~$R' \sse R$. However, there is no matching for all of $R$: if $r_0$ is matched to some $\ell_i$, then $r_i$ can not be matched.  }


The ``troublemaker'' in Figure~\ref{fig:infinite_matching} is the infinite-degree node $0$ on the right. If we assume that the nodes we intend to match --- the right side nodes $R$, in our discussion --- have finite degrees, the marriage theorem can be recovered:  we can match  $R$ if and only if we can match every finite $R' \sse R$, which in turn is possible precisely when each finite $R' \sse R$ has at least $|R'|$ neighbors (the Hall condition). This holds for infinite graphs of arbitrary, even uncountable, cardinality, so long as the degrees on the right are all finite (not necessarily bounded); degrees on the left may be arbitrary, even uncountable. This \emph{infinite marriage theorem} was first established by Marshall Hall~\cite{jr_distinct_1948} by way of a quite sophisticated proof,\footnote{No relation to Philip Hall, coincidentally.}  though somewhat simpler proofs have since been discovered (e.g. \cite{halmos_marriage_1950,rado_note_1967}). The infinite marriage theorem also extends naturally to  $b$-matching by a simple cloning argument, under the same requirement of finite right-side degrees: If each node on the left can be matched at most once, whereas each node $r$ on the right requires at least $b_r$ matches, then this is possible if and only if it is possible for every finite $R' \sse R$, which in turn is possible if and only if each finite $R' \sse R$ has at least $b(R') = \sum_{r \in R'} b_r$ neighbors. In showing that an infinite graph admits a matching of a particular form precisely when the same holds for some of its finite subgraphs, M. Hall's infinite marriage theorem is what is referred to as a \emph{compactness} result.

This now brings us naturally to bipartite OIGs, which encode classification as a matching problem. In the realizable setting, a learner with error $\epsilon$ corresponds to a $b$-matching on $G_{\bp}$ with $b=(1-\epsilon)n$ for every right side side node $\y=(y_1, \ldots, y_n)$. The infinite marriage theorem then implies that this is possible precisely when every finite family $R'$ of full labelings on the right side of $G_{\bp}$ is incident on at least $(1-\epsilon)n \cdot |R'|$ partial labelings. This yields a precise expression of the optimal error as a function of the sample size $n$,  as the worst-case  ``matchability'' of finite subgraphs of a bipartite OIG. This graph-theoretic expression, stated precisely in \cite{asilis_regularization_2024}, is called the \emph{Hall complexity} of the hypothesis class. The expression is easily adapted to the agnostic setting by appropriately discounting the matching requirement of each node $\y$ in the agnostic OIG by $\dist(\y,\H)$. 

Much like in the case of matching, we now have a compactness result for classification. The Hall complexity expresses the optimal transductive error rate of a classification problem as the worst-case learning rate of its \emph{finite projections}: those problems which can be obtained by restricting attention to finite subsets of the domain $\X$ and finite subsets $\H' \sse \H$ of the hypothesis space. To see this, note that  each bipartite OIG is a function of a finite subdomain $x_1,\ldots,x_n$, and a finite subset of its right side nodes corresponds to finitely many hypotheses.  By exactly relating the matchability in the bipartite OIG to the same matchability in some of its finite subgraphs, we can conclude that the maximal obstructions to transductive classification are finite in nature. %Th compact characterization approximately extends to PAC learning, in black box fashion as described  in Section \ref{subsec:pac_trans}.


\paragraph{Bibliographic Remarks.} Simple and \emph{compact} expressions for the optimal error, or equivalently the sample complexity, have long been sought in learning theory. For realizable binary classification, the sample complexity is characterized in terms of the VC dimension both in the transductive \cite{haussler_predicting_1994,li_one-inclusion_2001} and PAC (see e.g. \cite{shalev-shwartz_understanding_2014}) models.    For multiclass classification, the \emph{DS dimension} of \citet{daniely_optimal_2014} yields an approximation of the optimal error in the realizable setting up to a constant in the exponent, as was recently shown by \citet{brukhim_characterization_2022}. Outside of  \emph{dimensions}, a quantity derived from the (classical) one-inclusion graph $G$ gives a constant factor approximation: the maximum average degree of a subgraph of $G$, divided by $n$, is with a factor of~$2$ of the optimal error in the realizable setting. The Hall complexity can be viewed as ``sharpening'' the maximum-average degree into an exact characterization, and extending it to the agnostic setting.

The most sought-after characterizations of learning are \emph{dimensions} such as the VC dimension, DS dimension, and others. Intuitively, a dimension measures the size of the ``largest obstruction'' to learning, with an infinite dimension implying unlearnability, and a finite dimension yielding a bound on the sample complexity by way of a simple expression. Compactness results of the sort provided by the Hall complexity are a weaker requirement than the existence of a dimension capturing optimal learning. I will elaborate on the relationship between compactness and dimensions in Section~\ref{sec:compactness}.  %I will note, however, that compactness is by no means ``trivial'': recent work of \citet{ben-david_learnability_2019} demonstrates a natural supervised learning problem --- in the proper setting not considered in this article ---  where compactness fails.\footnote{We note that \cite{ben-david_learnability_2019} do not pose their non-compact problem, which they term \emph{EMX learning}, as a supervised learning problem. However, it is not difficult to see that it can be rephrased as such.}



\subsection{Implication of Matching: An Algorithmic Template for Classification}
Another important concern of learning theory is identifying algorithmic approaches that are broadly successful for a large swath of learning problems, and moreover simple enough to describe, implement, and explore. Perhaps the best example of this is empirical risk minimization (ERM), as described in Section~\ref{sec:background}. ERM and closely related approaches such as structural risk minimization (SRM) pervade the practice of machine learning, and in some cases are also accompanied by theoretical guarantees.  Most notably, for binary classification problems in the PAC model, it has long been known that ERM characterizes learnability and moreover has almost-optimal sample complexity (see e.g. \cite{shalev-shwartz_understanding_2014}).

Such theoretical optimality guarantees  for ERM  tend to fall by the wayside as one moves beyond binary classification.  In fact, ERM can entirely fail to learn in some settings where learning is possible, as shown by \citet{shalev-shwartz_learnability_2010}.   This failure manifests   in multiclass classification as shown by \citet{daniely_multiclass_2011}, and in a slight generalization of binary classification as shown by \citet{alon_theory_2021}. Even in bounded agnostic linear regression, the sample complexity of ERM was recently shown to be quite suboptimal by \citet{vaskevicius_suboptimality_2023}.\footnote{That said, ERM characterizes learnability for bounded regression  in the coarse sense, disregarding sample complexity~\cite{alon_scale-sensitive_1997}.}  %Whereas a body of work has made significant progress in describing optimal learners for such general problems (e.g. \cite{daniely_optimal_2014,brukhim_characterization_2022}), the existence of an algorithmic template or principle which captures optimal learning remains open.

In settings where ERM fails to characterize optimal learning, the algorithms which do succeed in theory are often far removed from any parsimonious description or any heuristic employed in practice.  This is most apparent in multiclass classification, where the known near-optimal algorithms \cite{daniely_optimal_2014,brukhim_characterization_2022} involve explicitly-described orientations of the one-inclusion graph. Can the bipartite matching interpretation simplify these OIG-based algorithms? Bipartite matching tends to respond well to ``simple'' heuristics such as greedy algorithms, primal-dual approaches, and others --- can the same hold true for learning?


 \begin{figure}
  % \input{arXiv_version_2/figures/local_regularizer.tikz}
\centering
\input{figures/local_regularizer.tikz}
\caption{\emph{ %Hypotheses $h_1$ and $h_2$, depicted in yellow and blue respectively.
    A  local regularizer may favor the simplicity of $h_1$ on test points drawn from the right region of the domain, and the simplicity of $h_2$ on test points drawn from the left region.}}
\label{fig:local}
% \captionof{figure}{}
% This should be adjusted to not overlap once the manuscript is finished.
%\vspace{-6em}
%\caption{Hypotheses $h_0$ and $h_1$, depicted in blue and yellow respectively. A  local regularizer may favor the simplicity of $h_0$ on test points drawn from the right region of the domain, and the simplicity of $h_1$ on test points drawn from the left region.}
\end{figure}

The answer to both questions turns out to be yes, as shown recently by~\citet{asilis_regularization_2024}. Through primal-dual analysis of a certain convex program over matchings in the bipartite OIG, we derive a near-optimal learner for multiclass classification of a particular instructive form. In particular, this learner implements a variation of the SRM template from Section~\ref{sec:background}, relaxed on two fronts: (1) It employs  a  \emph{local} regularization function  $\psi(h,x_{\tst})$ that  measures predictor complexity differently depending on the test point, and  (2) it incorporates an unsupervised pre-training stage which learns this regularization function.

These relaxed SRMs --- derived from the connection to bipartite matching --- provide a  simpler template for optimal learning which is quite natural, and is reminiscent of approaches discovered to be useful in practice. Relaxing to local regularization is well-motivated, since different hypotheses may be simple in some areas of the domain and complex in others; this is illustrated in Figure~\ref{fig:local}. It is therefore not altogether surprising that this result is predated by applications of local regularization to image classification and restoration~\cite{wolf_local_2008,prost_learning_2021}. The second relaxation is in line with much of the recent empirical and theoretical evidence on the utility of an unsupervised learning stage. Indeed, unsupervised pre-training has seen widespread application to  computer vision, natural language processing, and speech recognition (see the discussion in \cite{ge_provable_2024}). 



%%%Local Variables:
%%% mode: latex
%%% TeX-master: "learning_matching"
%%% End:

\section{Generalization to Other Circuits}
We extend our analyses to the \textbf{Greater-Than} circuit \cite{hanna2024does}. We find a similar pattern. The mechanisms of the greater-than task are \textit{amplified} after fine-tuning on task data. In contrast, the changes to the mechanisms of the model under toxic fine-tuning are primarily localized to circuit components leading to corruption of the task. Furthermore, we discover our finding of neuroplasticity to hold for the greater-than task, i.e., the model reverts back to its original mechanism after retraining the corrupted model on clean task-specific data. We detail our experiments on this task in \autoref{app:gt}.
\section{Conclusion}
In this work, we propose a simple yet effective approach, called SMILE, for graph few-shot learning with fewer tasks. Specifically, we introduce a novel dual-level mixup strategy, including within-task and across-task mixup, for enriching the diversity of nodes within each task and the diversity of tasks. Also, we incorporate the degree-based prior information to learn expressive node embeddings. Theoretically, we prove that SMILE effectively enhances the model's generalization performance. Empirically, we conduct extensive experiments on multiple benchmarks and the results suggest that SMILE significantly outperforms other baselines, including both in-domain and cross-domain few-shot settings.

\section*{Acknowledgments}
Many of the ideas in this article came out of conversations and joint papers  with a stellar group of collaborators. In alphabetical order, these are: Julian Asilis, Siddartha Devic, Yusuf Kalayci, Vatsal Sharan, Shang-Hua Teng, and Grayson York. The analogy between the transductive model and hat puzzles came out of conversations with Siyu Zeng.


{%\small
\bibliography{learning,matching,misc,local_computation}
%\bibliographystyle{plainnat}
\bibliographystyle{abbrvnat}               %latex8
%\bibliographystyle{plain}
%\bibliographystyle{abbrv}               %latex8
}


\end{document}


%%% Local Variables: 
%%% mode: latex
%%% TeX-master: t
%%% End: 
