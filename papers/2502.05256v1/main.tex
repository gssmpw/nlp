%%
%% This is file `sample-sigconf-authordraft.tex',
%% generated with the docstrip utility.
%%
%% The original source files were:
%%
%% samples.dtx  (with options: `all,proceedings,bibtex,authordraft')
%% 
%% IMPORTANT NOTICE:
%% 
%% For the copyright see the source file.
%% 
%% Any modified versions of this file must be renamed
%% with new filenames distinct from sample-sigconf-authordraft.tex.
%% 
%% For distribution of the original source see the terms
%% for copying and modification in the file samples.dtx.
%% 
%% This generated file may be distributed as long as the
%% original source files, as listed above, are part of the
%% same distribution. (The sources need not necessarily be
%% in the same archive or directory.)
%%
%%
%% Commands for TeXCount
%TC:macro \cite [option:text,text]
%TC:macro \citep [option:text,text]
%TC:macro \citet [option:text,text]
%TC:envir table 0 1
%TC:envir table* 0 1
%TC:envir tabular [ignore] word
%TC:envir displaymath 0 word
%TC:envir math 0 word
%TC:envir comment 0 0
%%
%%
%% The first command in your LaTeX source must be the \documentclass
%% command.
%%
%% For submission and review of your manuscript please change the
%% command to \documentclass[manuscript, screen, review]{acmart}.
%%
%% When submitting camera ready or to TAPS, please change the command
%% to \documentclass[sigconf]{acmart} or whichever template is reuired
%% for your publication.
%%
%%
% \documentclass[sigconf,authordraft]{acmart}
\documentclass[nonacm,sigconf]{acmart}
% \documentclass[manuscript,screen,review,anonymous]{acmart}

\usepackage{xspace}
\usepackage{pifont}
\usepackage[inline]{enumitem}
\usepackage{cleveref}
% \usepackage{subfig}
\usepackage{subcaption}

\DeclareMathOperator*{\argmax}{arg\,max}
\DeclareMathOperator*{\argmin}{arg\,min}
\newcommand{\ourmethod}{\texttt{BOUTT}}
\newcommand{\bx}{\mathbf{x}}
\newcommand{\bc}{\mathbf{c}}
\newcommand{\bz}{\mathbf{z}}
\newcommand{\bX}{\mathbf{X}}
\newcommand{\bZ}{\mathbf{Z}}
\newcommand{\bV}{\mathbf{V}}
\newcommand{\bfn}{\mathbf{f}}
\newcommand{\bu}{\mathbf{u}}
\newcommand{\by}{\mathbf{y}}
\newcommand{\calX}{\mathcal{X}}
\newcommand{\calR}{\mathcal{R}}
\newcommand{\calD}{\mathcal{D}}
\newcommand{\calZ}{\mathcal{Z}}
\newcommand{\calT}{\mathcal{T}}
\newcommand{\calGP}{\mathcal{GP}}



%%
%% \BibTeX command to typeset BibTeX logo in the docs
\AtBeginDocument{%
  \providecommand\BibTeX{{%
    Bib\TeX}}}

%% Rights management information.  This information is sent to you
%% when you complete the rights form.  These commands have SAMPLE
%% values in them; it is your responsibility as an author to replace
%% the commands and values with those provided to you when you
%% complete the rights form.
\setcopyright{acmlicensed}
\copyrightyear{2025}
\acmYear{2025}
\acmDOI{XXXXXXX.XXXXXXX}

%% These commands are for a PROCEEDINGS abstract or paper.
\acmConference[SIGMOD '25]{Proceedings of the 2025 International Conference on Management
of Data}{June 22--27,
  2025}{Berlin, Germany}
%%
%%  Uncomment \acmBooktitle if the title of the proceedings is different
%%  from ``Proceedings of ...''!
%%
%%\acmBooktitle{Woodstock '18: ACM Symposium on Neural Gaze Detection,
%%  June 03--05, 2018, Woodstock, NY}
\acmISBN{978-TBD}


%%
%% Submission ID.
%% Use this when submitting an article to a sponsored event. You'll
%% receive a unique submission ID from the organizers
%% of the event, and this ID should be used as the parameter to this command.
%%\acmSubmissionID{123-A56-BU3}

%%
%% For managing citations, it is recommended to use bibliography
%% files in BibTeX format.
%%
%% You can then either use BibTeX with the ACM-Reference-Format style,
%% or BibLaTeX with the acmnumeric or acmauthoryear sytles, that include
%% support for advanced citation of software artefact from the
%% biblatex-software package, also separately available on CTAN.
%%
%% Look at the sample-*-biblatex.tex files for templates showcasing
%% the biblatex styles.
%%

%%
%% The majority of ACM publications use numbered citations and
%% references.  The command \citestyle{authoryear} switches to the
%% "author year" style.
%%
%% If you are preparing content for an event
%% sponsored by ACM SIGGRAPH, you must use the "author year" style of
%% citations and references.
%% Uncommenting
%% the next command will enable that style.
%%\citestyle{acmauthoryear}


\newcommand{\sparagraph}[1]{\vspace{1mm}\noindent \textbf{#1}} 
\newcommand{\sysname}[0]{BayesQO\xspace}
\newcommand{\bo}[0]{BO\xspace}
\newcommand{\random}[0]{Random\xspace}

\newcommand{\circleOne}{\ding{182}\xspace}
\newcommand{\circleTwo}{\ding{183}\xspace}
\newcommand{\circleThree}{\ding{184}\xspace}
\newcommand{\circleFour}{\ding{185}\xspace}
\newcommand{\circleFive}{\ding{186}\xspace}
\newcommand{\circleSix}{\ding{187}\xspace}
\newcommand{\circleSeven}{\ding{188}\xspace}
\newcommand{\circleEight}{\ding{189}\xspace}
\newcommand{\circleNine}{\ding{190}\xspace}

\newcommand{\jrg}[1]{\textcolor{red}{[Jake: #1]}}
\newcommand{\nmaus}[1]{\textcolor{olive}{[Natalie: #1]}}
\newcommand{\haydn}[1]{\textcolor{magenta}{[Haydn: #1]}}
\newcommand{\jeff}[1]{\textcolor{blue}{[Jeff: #1]}}
\newcommand{\ryan}[1]{\textcolor{cyan}{[Ryan: #1]}}
\newcommand{\yimeng}[1]{\textcolor{green}{[Yimeng: #1]}}
\newcommand{\factcheck}[1]{\textcolor{orange}{[Fact Check: #1]}}

% \newcommand{\edit}[1]{\textcolor{blue}{#1}}
\newcommand{\edit}[1]{#1}



%%
%% end of the preamble, start of the body of the document source.
\begin{document}
\setlength{\textfloatsep}{0pt}
% \setlength{\floatsep}{4pt}
% \setlength{\intextsep}{0pt}

% \setlength{\abovedisplayskip}{4pt}
% \setlength{\belowdisplayskip}{4pt}
% \setlength{\abovedisplayshortskip}{4pt}
% \setlength{\belowdisplayshortskip}{4pt}

%%
%% The "title" command has an optional parameter,
%% allowing the author to define a "short title" to be used in page headers.
\title{Learned Offline Query Planning via Bayesian Optimization}

%%
%% The "author" command and its associated commands are used to define
%% the authors and their affiliations.
%% Of note is the shared affiliation of the first two authors, and the
%% "authornote" and "authornotemark" commands
%% used to denote shared contribution to the research.
\author{Jeffrey Tao}
\orcid{0000-0001-6407-1316}
\affiliation{
  \institution{University of Pennsylvania}
  %\city{Philadelphia}
  %\state{PA}
  %\country{USA}
}
\email{jefftao@seas.upenn.edu}

\author{Natalie Maus}
\orcid{0000-0002-6616-8506}
\affiliation{
  \institution{University of Pennsylvania}
  %\city{Philadelphia}
  %\state{PA}
  %\country{USA}
}
\email{nmaus@seas.upenn.edu}

\author{Haydn Jones}
\orcid{0000-0002-1006-4126}
\affiliation{
  \institution{University of Pennsylvania}
  %\city{Philadelphia}
  %\state{PA}
  %\country{USA}
}
\email{haydnj@seas.upenn.edu}

\author{Yimeng Zeng}
\orcid{0009-0001-9676-0893}
\affiliation{
  \institution{University of Pennsylvania}
  %\city{Philadelphia}
  %\state{PA}
  %\country{USA}
}
\email{yimengz@seas.upenn.edu}

\author{Jacob R. Gardner}
\orcid{0000-0003-1897-8384}
\affiliation{
  \institution{University of Pennsylvania}
  %\city{Philadelphia}
  %\state{PA}
  %\country{USA}
}
\email{jacobrg@seas.upenn.edu}

\author{Ryan Marcus}
\orcid{0000-0002-1279-1124}
\affiliation{
  \institution{University of Pennsylvania}
  %\city{Philadelphia}
  %\state{PA}
  %\country{USA}
}
\email{rcmarcus@seas.upenn.edu}


%%
%% By default, the full list of authors will be used in the page
%% headers. Often, this list is too long, and will overlap
%% other information printed in the page headers. This command allows
%% the author to define a more concise list
%% of authors' names for this purpose.
\renewcommand{\shortauthors}{Tao et al.}

%%
%% The abstract is a short summary of the work to be presented in the
%% article.
\begin{abstract}
Analytics database workloads often contain queries that are executed repeatedly. Existing optimization techniques generally prioritize keeping optimization cost low, normally well below the time it takes to execute a single instance of a query. If a given query is going to be executed thousands of times, could it be worth investing significantly more optimization time? In contrast to traditional online query optimizers, we propose an offline query optimizer that searches a wide variety of plans and incorporates query execution as a primitive. Our offline query optimizer combines variational auto-encoders with Bayesian optimization to find optimized plans for a given query. We compare our technique to the optimal plans possible with PostgreSQL and recent RL-based systems over several datasets, and show that our technique finds faster query plans.

%% \noindent\small{Artifacts: \url{https://anonymous.4open.science/r/bayesqo-artifacts-B3FB}}
\end{abstract}

%%
%% The code below is generated by the tool at http://dl.acm.org/ccs.cfm.
%% Please copy and paste the code instead of the example below.
%%
\begin{CCSXML}
\end{CCSXML}


%%
%% Keywords. The author(s) should pick words that accurately describe
%% the work being presented. Separate the keywords with commas.
\keywords{}
%% A "teaser" image appears between the author and affiliation
%% information and the body of the document, and typically spans the
%% page.

%\received{20 February 2007}
%\received[revised]{12 March 2009}
%\received[accepted]{5 June 2009}

%%
%% This command processes the author and affiliation and title
%% information and builds the first part of the formatted document.
\maketitle

\section{Introduction}


\begin{figure}[t]
\centering
\includegraphics[width=0.6\columnwidth]{figures/evaluation_desiderata_V5.pdf}
\vspace{-0.5cm}
\caption{\systemName is a platform for conducting realistic evaluations of code LLMs, collecting human preferences of coding models with real users, real tasks, and in realistic environments, aimed at addressing the limitations of existing evaluations.
}
\label{fig:motivation}
\end{figure}

\begin{figure*}[t]
\centering
\includegraphics[width=\textwidth]{figures/system_design_v2.png}
\caption{We introduce \systemName, a VSCode extension to collect human preferences of code directly in a developer's IDE. \systemName enables developers to use code completions from various models. The system comprises a) the interface in the user's IDE which presents paired completions to users (left), b) a sampling strategy that picks model pairs to reduce latency (right, top), and c) a prompting scheme that allows diverse LLMs to perform code completions with high fidelity.
Users can select between the top completion (green box) using \texttt{tab} or the bottom completion (blue box) using \texttt{shift+tab}.}
\label{fig:overview}
\end{figure*}

As model capabilities improve, large language models (LLMs) are increasingly integrated into user environments and workflows.
For example, software developers code with AI in integrated developer environments (IDEs)~\citep{peng2023impact}, doctors rely on notes generated through ambient listening~\citep{oberst2024science}, and lawyers consider case evidence identified by electronic discovery systems~\citep{yang2024beyond}.
Increasing deployment of models in productivity tools demands evaluation that more closely reflects real-world circumstances~\citep{hutchinson2022evaluation, saxon2024benchmarks, kapoor2024ai}.
While newer benchmarks and live platforms incorporate human feedback to capture real-world usage, they almost exclusively focus on evaluating LLMs in chat conversations~\citep{zheng2023judging,dubois2023alpacafarm,chiang2024chatbot, kirk2024the}.
Model evaluation must move beyond chat-based interactions and into specialized user environments.



 

In this work, we focus on evaluating LLM-based coding assistants. 
Despite the popularity of these tools---millions of developers use Github Copilot~\citep{Copilot}---existing
evaluations of the coding capabilities of new models exhibit multiple limitations (Figure~\ref{fig:motivation}, bottom).
Traditional ML benchmarks evaluate LLM capabilities by measuring how well a model can complete static, interview-style coding tasks~\citep{chen2021evaluating,austin2021program,jain2024livecodebench, white2024livebench} and lack \emph{real users}. 
User studies recruit real users to evaluate the effectiveness of LLMs as coding assistants, but are often limited to simple programming tasks as opposed to \emph{real tasks}~\citep{vaithilingam2022expectation,ross2023programmer, mozannar2024realhumaneval}.
Recent efforts to collect human feedback such as Chatbot Arena~\citep{chiang2024chatbot} are still removed from a \emph{realistic environment}, resulting in users and data that deviate from typical software development processes.
We introduce \systemName to address these limitations (Figure~\ref{fig:motivation}, top), and we describe our three main contributions below.


\textbf{We deploy \systemName in-the-wild to collect human preferences on code.} 
\systemName is a Visual Studio Code extension, collecting preferences directly in a developer's IDE within their actual workflow (Figure~\ref{fig:overview}).
\systemName provides developers with code completions, akin to the type of support provided by Github Copilot~\citep{Copilot}. 
Over the past 3 months, \systemName has served over~\completions suggestions from 10 state-of-the-art LLMs, 
gathering \sampleCount~votes from \userCount~users.
To collect user preferences,
\systemName presents a novel interface that shows users paired code completions from two different LLMs, which are determined based on a sampling strategy that aims to 
mitigate latency while preserving coverage across model comparisons.
Additionally, we devise a prompting scheme that allows a diverse set of models to perform code completions with high fidelity.
See Section~\ref{sec:system} and Section~\ref{sec:deployment} for details about system design and deployment respectively.



\textbf{We construct a leaderboard of user preferences and find notable differences from existing static benchmarks and human preference leaderboards.}
In general, we observe that smaller models seem to overperform in static benchmarks compared to our leaderboard, while performance among larger models is mixed (Section~\ref{sec:leaderboard_calculation}).
We attribute these differences to the fact that \systemName is exposed to users and tasks that differ drastically from code evaluations in the past. 
Our data spans 103 programming languages and 24 natural languages as well as a variety of real-world applications and code structures, while static benchmarks tend to focus on a specific programming and natural language and task (e.g. coding competition problems).
Additionally, while all of \systemName interactions contain code contexts and the majority involve infilling tasks, a much smaller fraction of Chatbot Arena's coding tasks contain code context, with infilling tasks appearing even more rarely. 
We analyze our data in depth in Section~\ref{subsec:comparison}.



\textbf{We derive new insights into user preferences of code by analyzing \systemName's diverse and distinct data distribution.}
We compare user preferences across different stratifications of input data (e.g., common versus rare languages) and observe which affect observed preferences most (Section~\ref{sec:analysis}).
For example, while user preferences stay relatively consistent across various programming languages, they differ drastically between different task categories (e.g. frontend/backend versus algorithm design).
We also observe variations in user preference due to different features related to code structure 
(e.g., context length and completion patterns).
We open-source \systemName and release a curated subset of code contexts.
Altogether, our results highlight the necessity of model evaluation in realistic and domain-specific settings.





\putsec{related}{Related Work}

\noindent \textbf{Efficient Radiance Field Rendering.}
%
The introduction of Neural Radiance Fields (NeRF)~\cite{mil:sri20} has
generated significant interest in efficient 3D scene representation and
rendering for radiance fields.
%
Over the past years, there has been a large amount of research aimed at
accelerating NeRFs through algorithmic or software
optimizations~\cite{mul:eva22,fri:yu22,che:fun23,sun:sun22}, and the
development of hardware
accelerators~\cite{lee:cho23,li:li23,son:wen23,mub:kan23,fen:liu24}.
%
The state-of-the-art method, 3D Gaussian splatting~\cite{ker:kop23}, has
further fueled interest in accelerating radiance field
rendering~\cite{rad:ste24,lee:lee24,nie:stu24,lee:rho24,ham:mel24} as it
employs rasterization primitives that can be rendered much faster than NeRFs.
%
However, previous research focused on software graphics rendering on
programmable cores or building dedicated hardware accelerators. In contrast,
\name{} investigates the potential of efficient radiance field rendering while
utilizing fixed-function units in graphics hardware.
%
To our knowledge, this is the first work that assesses the performance
implications of rendering Gaussian-based radiance fields on the hardware
graphics pipeline with software and hardware optimizations.

%%%%%%%%%%%%%%%%%%%%%%%%%%%%%%%%%%%%%%%%%%%%%%%%%%%%%%%%%%%%%%%%%%%%%%%%%%
\myparagraph{Enhancing Graphics Rendering Hardware.}
%
The performance advantage of executing graphics rendering on either
programmable shader cores or fixed-function units varies depending on the
rendering methods and hardware designs.
%
Previous studies have explored the performance implication of graphics hardware
design by developing simulation infrastructures for graphics
workloads~\cite{bar:gon06,gub:aam19,tin:sax23,arn:par13}.
%
Additionally, several studies have aimed to improve the performance of
special-purpose hardware such as ray tracing units in graphics
hardware~\cite{cho:now23,liu:cha21} and proposed hardware accelerators for
graphics applications~\cite{lu:hua17,ram:gri09}.
%
In contrast to these works, which primarily evaluate traditional graphics
workloads, our work focuses on improving the performance of volume rendering
workloads, such as Gaussian splatting, which require blending a huge number of
fragments per pixel.

%%%%%%%%%%%%%%%%%%%%%%%%%%%%%%%%%%%%%%%%%%%%%%%%%%%%%%%%%%%%%%%%%%%%%%%%%%
%
In the context of multi-sample anti-aliasing, prior work proposed reducing the
amount of redundant shading by merging fragments from adjacent triangles in a
mesh at the quad granularity~\cite{fat:bou10}.
%
While both our work and quad-fragment merging (QFM)~\cite{fat:bou10} aim to
reduce operations by merging quads, our proposed technique differs from QFM in
many aspects.
%
Our method aims to blend \emph{overlapping primitives} along the depth
direction and applies to quads from any primitive. In contrast, QFM merges quad
fragments from small (e.g., pixel-sized) triangles that \emph{share} an edge
(i.e., \emph{connected}, \emph{non-overlapping} triangles).
%
As such, QFM is not applicable to the scenes consisting of a number of
unconnected transparent triangles, such as those in 3D Gaussian splatting.
%
In addition, our method computes the \emph{exact} color for each pixel by
offloading blending operations from ROPs to shader units, whereas QFM
\emph{approximates} pixel colors by using the color from one triangle when
multiple triangles are merged into a single quad.


\vspace{-1mm}
\section{Zak-OTFS System Model}
\label{sec2}
Figure \ref{fig1} shows the block diagram of a Zak-OTFS transceiver.
\begin{figure*}
\centering    \includegraphics[width=0.95\linewidth]{Figures/continuous_zak_otfs_bd.eps}
\caption{Block diagram of Zak-OTFS transceiver.}
\label{fig1}      
\vspace{-4mm}
\end{figure*}
In Zak-OTFS, a pulse in the DD domain is the basic information carrier. A DD pulse is a quasi-periodic localized function defined by a delay period $\tau_{\mathrm{p}}$ and a Doppler period $\nu_{\mathrm{p}}=\frac{1}{\tau_{\mathrm{p}}}$. The fundamental period in the DD domain is defined as 
$\mathcal{D}_{0}= \{(\tau,\nu): 0\leq\tau<\tau_{\mathrm p}, 0\leq\nu<\nu_{\mathrm p}\}$,
where $\tau$ and $\nu$ represent the delay and Doppler variables, respectively. The fundamental period is discretized into $M$ bins on the delay axis and $N$ bins on the Doppler axis, as 
$\big\{(k\frac{\tau_{{\mathrm p}}}{M},l\frac{\nu_{{\mathrm p}}}{N}) | k=0,\ldots,M-1,l=0,\ldots,N-1\big\}$. The time domain Zak-OTFS frame is limited to a time duration $T=N\tau_{\mathrm p}$ and a bandwidth $B=M\nu_{\mathrm p}$. In each frame, $MN$ information symbols drawn from a modulation alphabet ${\mathbb A}$, $x[k,l]\in {\mathbb A}$, $k=0,\ldots,M-1$, $l=0,\ldots,N-1$, are multiplexed in the DD domain. The information symbol $x[k,l]$ is carried by DD domain pulse $x_{\mathrm{dd}}[k,l]$, which is a quasi-periodic function with period $M$ along the delay axis and period $N$ along the Doppler axis, i.e., for any $n,m\in\mathbb{Z}$,  
\begin{equation}
x_{\mathrm{dd}}[k+nM,l+mN]=x[k,l]e^{j2\pi n\frac{l}{N}}.
\end{equation}
These discrete DD domain signals $x_{\mathrm{dd}}[k,l]$s are supported on the information lattice 
$\Lambda_{\mathrm{dd}}=
\big\{\big(k\frac{\tau_{\mathrm p}}{M},l\frac{\nu_{\mathrm p}}{N}\big) | k,l\in \mathbb{Z}\big\}$.
The continuous DD domain information signal is given by
\vspace{-1mm}
\begin{equation}
x_{\mathrm{dd}}(\tau,\nu)=\sum_{k,l\in \mathbb{Z}} x_{\mathrm{dd}}[k,l] \delta\Big(\tau-\frac{k\tau_{\mathrm p}}{M}\Big)\delta\Big(\nu-\frac{l\nu_{\mathrm p}}{N}\Big),
\end{equation}
where $\delta(.)$ denotes the Dirac-delta impulse function. For any $n,m\in \mathbb{Z}$, we have
$x_{\mathrm{dd}}(\tau+n\tau_{\mathrm{p}},\nu+m\nu_{\mathrm{p}})=e^{j2\pi n\nu \tau_{\mathrm{p}}}x_{\mathrm{dd}}(\tau,\nu)$,
so that $x_{\mathrm{dd}}(\tau,\nu)$ is periodic with period $\nu_{\mathrm p}$ along the Doppler axis and quasi-periodic with period $\tau_{\mathrm p}$ along the delay axis.

The DD domain transmit signal $x_{\mathrm{dd}}^{w_{\mathrm{tx}}}(\tau,\nu)$ is given by the twisted convolution of the transmit pulse shaping filter $w_{\mathrm{tx}}(\tau,\nu)$ with $x_{\mathrm{dd}}(\tau,\nu)$ as $x_{\mathrm{dd}}^{w_{\mathrm{tx}}}(\tau,\nu) = w_{\mathrm{tx}}(\tau,\nu)*_{\sigma}x_{\mathrm{dd}}(\tau,\nu)$,
where $*_{\sigma}$ denotes the twisted convolution\footnote{Twisted convolution of two DD functions $a(\tau,\nu)$ and $b(\tau,\nu)$ is defined as 
$a(\tau,\nu) \ast_\sigma b(\tau,\nu) \overset{\Delta}{=} \iint a(\tau', \nu') b(\tau-\tau',\nu-\nu')e^{j2\pi\nu'(\tau-\tau')}d\tau'  d\nu'$.}. The transmitted time domain (TD) signal $s_{\mathrm{td}}(t)$ is the TD realization of $x_{\mathrm{dd}}^{w_{\mathrm{tx}}}(\tau,\nu)$, given by
$s_{\mathrm{td}}(t)=Z_{t}^{-1}\left(x_{\mathrm{dd}}^{w_{\mathrm{tx}}}(\tau,\nu)\right)$, where $Z_{t}^{-1}$ denotes the inverse time-Zak transform operation\footnote{Inverse time-Zak transform of a DD function $a(\tau,\nu)$ is defined as $Z_{t}^{-1}(a(\tau,\nu)) \overset{\Delta}{=} \sqrt{\tau_{\mathrm p}} \int_0^{\nu_{\mathrm p}} a(t,\nu) d\nu$.}. The transmit pulse shaping filter $w_{\mathrm{tx}}(\tau,\nu)$ 
limits the time and bandwidth of the transmitted signal $s_{\mathrm{td}}(t)$. The transmit signal $s_{\mathrm{td}}(t)$ passes through a doubly-selective channel to give the output signal $r_{\mathrm{td}}(t)$. The DD domain impulse response of the physical channel $h_{\mathrm{phy}}(\tau,\nu)$ is given by
\begin{equation}
h_{\mathrm{phy}}(\tau,\nu)=\sum_{i=1}^{P}h_{i}\delta(\tau-\tau_{i})\delta(\nu-\nu_{i}),
\end{equation}
where $P$ denotes the number of DD paths, and the $i$th path has gain $h_{i}$, delay shift $\tau_{i}$, and Doppler shift $\nu_{i}$. 

The received TD signal $y(t)$ at the receiver is given by $y(t)=r_{\mathrm{td}}(t)+n(t)$,
where $n(t)$ is AWGN with variance $N_{0}$, i.e., $\mathbb{E}[n(t)n(t+t')]=N_{0}\delta(t')$. The TD signal $y(t)$ is converted to the corresponding DD domain signal $y_{\mathrm{dd}}(\tau,\nu)$ by applying Zak transform\footnote{Zak transform of a continuous TD signal $a(t)$ is defined as
$Z_t\left(a(t)\right) \overset{\Delta}{=} \sqrt{\tau_p} \sum_{k \in \mathbb{Z}} a(\tau + k \tau_{\mathrm p}) e^{-j2\pi\nu k\tau_{\mathrm p}}$.}, i.e.,
\begin{eqnarray}
\hspace{-6mm}
y_{\mathrm{dd}}(\tau,\nu) = Z_{t}(y(t)) 
= r_{\mathrm{dd}}(\tau,\nu)+n_{\mathrm{dd}}(\tau,\nu),
\end{eqnarray}
where $r_{\mathrm{dd}}(\tau,\nu)=h_{\mathrm{phy}}(\tau,\nu)*_{\sigma}w_{\mathrm{tx}}(\tau,\nu)*_{\sigma}x_{\mathrm{dd}}(\tau,\nu)$ is the Zak transform of $r_{\mathrm{td}}(t)$, given by the twisted convolution cascade of $x_{\mathrm{dd}}(\tau,\nu)$, $w_{\mathrm{tx}}(\tau,\nu)$, and $h_{\mathrm{phy}}(\tau,\nu)$,  and $n_{\mathrm{dd}}(\tau,\nu)$ is the Zak transform of $n(t)$. The receiver filter $w_{\mathrm{rx}}(\tau,\nu)$ acts on $y_{\mathrm{dd}}(\tau,\nu)$ through twisted convolution to give the output 
\begin{eqnarray}
\hspace{-4mm} 
y_{\mathrm{dd}}^{w_{\mathrm{rx}}}(\tau,\nu) & \hspace{-2mm} = & \hspace{-2mm} w_{\mathrm{rx}}(\tau,\nu)*_{\sigma}y_{\mathrm{dd}}(\tau,\nu) \nonumber \\ 
& \hspace{-22mm} = & \hspace{-12mm} \underbrace{w_{\mathrm{rx}}(\tau,\nu)*_{\sigma}h_{\mathrm{phy}}(\tau,\nu)*_{\sigma}w_{\mathrm{tx}}(\tau,\nu)}_{\overset{\Delta}{=} \ h_{\mathrm{eff}}(\tau,\nu)}*_{\sigma}x_{\mathrm{dd}}(\tau,\nu) \nonumber \\ 
&\hspace{-22mm} & \hspace{-12mm} + \ \underbrace{w_{\mathrm{rx}}(\tau,\nu)*_{\sigma}n_{\mathrm{dd}}(\tau,\nu)}_{\overset{\Delta}{=} \ n_{\mathrm{dd}}^{w_{\mathrm{rx}}}(\tau,\nu)}, 
\label{cont1}
\end{eqnarray}
where $h_{\mathrm{eff}}(\tau,\nu)$ denotes the effective channel consisting of the twisted convolution cascade of $w_{\mathrm{tx}}(\tau,\nu),\ h_{\mathrm{phy}}(\tau,\nu)$, and $w_{\mathrm{rx}}(\tau,\nu)$, and $n_{\mathrm{dd}}^{w_{\mathrm{rx}}}(\tau,\nu)$ denotes the noise filtered through the Rx filter. The DD signal $y_{\mathrm{dd}}^{w_{\mathrm{rx}}}(\tau,\nu)$ is sampled on the information lattice, resulting in the discrete quasi-periodic DD domain received signal $y_{\mathrm{dd}}[k,l]$ as
\vspace{0mm}
\begin{equation}
y_{\mathrm{dd}}[k,l]=y_{\mathrm{dd}}^{w_{\mathrm{rx}}}\left(\tau=\frac{k\tau_{\mathrm p}}{M},\nu=\frac{l\nu_{\mathrm p}}{N}\right), \ \ k,l\in\mathbb{Z},
\end{equation} 
which is given by
$y_{\mathrm{dd}}[k,l]=h_{\mathrm{eff}}[k,l]*_{\sigma\text{d}}x_{\mathrm{dd}}[k,l]+n_{\mathrm{dd}}[k,l]$,
where $*_{\sigma\text{d}}$ is twisted convolution in discrete DD domain, i.e., 
$h_{\mathrm{eff}}[k,l]*_{\sigma\text{d}}x_{\mathrm{dd}}[k,l] = \sum_{k',l'\in\mathbb{Z}}h_{\mathrm{eff}}[k-k',l-l']x_{\mathrm{dd}}[k',l'] e^{j2\pi\frac{k'(l-l')}{MN}}$, where the effective channel filter $h_{\mathrm{eff}}[k,l]$ and filtered noise samples $n_{\mathrm{dd}}[k,l]$ are given by
\begin{align}
h_{\text{eff}}[k,l]=h_{\text{eff}}\left(\tau=\frac{k\tau_{p}}{M},\nu=\frac{l\nu_{p}}{N}\right), \label{discr2} \\ 
n_{\text{dd}}[k,l]=n_{\text{dd}}^{w_{\mathrm{rx}}}\left(\tau=\frac{k\tau_{p}}{M},\nu=\frac{l\nu_{p}}{N}\right).
\label{discr3}
\end{align}
Owing to the quasi-periodicity in the DD domain, it is sufficient to consider the received samples $y_{\mathrm{dd}}[k,l]$ within the fundamental period $\mathcal{D}_0$. Writing the $y_{\mathrm{dd}}[k,l]$ samples as a vector, the received signal model can be written in matrix-vector form as \cite{zak_otfs1},\cite{zak_otfs2}
\begin{equation}
\mathbf{y}=\mathbf{H_\text{eff}x}+\mathbf{n},
\label{sys_mod}
\end{equation}
where $\mathbf{x,y,n} \in\mathbb{C}^{MN\times 1}$, such that their $(kN+l+1)$th entries are given by $x_{kN+l+1}=x_{\mathrm{dd}}[k,l]$, $y_{kN+l+1}=y_{\mathrm{dd}}[k,l]$, $n_{kN+l+1}=n_{\mathrm{dd}}[k,l]$, and $\mathbf{H}_{\text{eff}}\in\mathbb{C}^{MN\times MN}$ is the effective channel matrix such that
\vspace{-1mm}
\begin{eqnarray}
\mathbf{H}_\text{eff}[k'N+l'+1,kN+l+1] & \hspace{-2mm} = & \hspace{-2mm} \sum_{m,n\in\mathbb{Z}}h_{\mathrm{eff}}[k'-k-nM, \nonumber \\
& \hspace{-45mm} & \hspace{-35mm} l'-l-mN]e^{j2\pi nl/N}e^{j2\pi\frac{(l'-l-mN)(k+nM)}{MN}},
\label{eqn_channel_matrix}
\vspace{-4mm}
\end{eqnarray}
where $k',k=0,\ldots,M-1$, $l',l=0,\ldots,N-1$. 

\vspace{-2mm}
\subsection{DD pulse shaping filters}
In the absence of pulse shaping, i.e., $w_\text{tx}(\tau,\nu)=\delta(\tau,\nu)$, the transmit signal has infinite time duration and bandwidth. Pulse shaping limits the time and bandwidth of transmission. We consider transmit DD pulse shaping filters of the form $w_\text{tx}(\tau,\nu)=w_1(\tau)w_2(\nu)$ \cite{zak_otfs2},\cite{zak_otfs6}. The time duration $T'$ of each frame is approximately related to the spread of $w_2({\nu})$ along the Doppler axis as $\frac{1}{T'}$. Likewise, the bandwidth $B'$ is approximately related to the spread of $w_1(\tau)$ along the delay axis as $\frac{1}{B'}$. That is, a larger bandwidth and time duration implies a smaller DD spread of $w_\text{tx}(\tau,\nu)$, and hence a smaller contribution to the spread of $h_\text{eff}(\tau,\nu)$. Sinc, RRC, and Gaussian filters have been considered in the Zak-OTFS literature and are described below. 

{\em Sinc filter:}
For sinc filter, $w_1({\tau})$ and $w_2({\nu})$ are given by
$w_1({\tau}) = \sqrt{B}\text{sinc}(B\tau)$ and $w_2({\nu}) = \sqrt{T}\text{sinc}(T\nu)$,
so that 
\begin{equation}
w_\text{tx}(\tau,\nu)
=\underbrace{\sqrt{B}\text{sinc}(B\tau)}_{w_1(\tau)} \underbrace{\sqrt{T}\text{sinc}(T\nu)}_{w_2(\nu)}. 
\label{eq:sinc1}
\end{equation}
For sinc filter, the frame duration $T'=T$ and frame bandwidth $B'=B$ (i.e., there is no time or bandwidth expansion), resulting in a spectral efficiency of $\frac{BT}{B'T'}=1$ symbol/dimension. 

{\em RRC filter:}
For RRC filter, $w_\text{tx}(\tau,\nu)$ is given by
\begin{equation}
w_\text{tx}(\tau,\nu) = \underbrace{\sqrt{B} \ \text{rrc}_{\beta_\tau}(B\tau)}_{w_1(\tau)} \ \underbrace{\sqrt{T} \ \text{rrc}_{\beta_\nu}(T\nu)}_{w_2(\nu)},
\label{eq:rrc1}
\end{equation}
where $0\leq \beta_{\tau},\beta_\nu \leq 1$ and
\begin{equation}
\text{rrc}_{\beta}(x)=\frac{\sin \left(\pi x(1-\beta) \right)+4\beta x\cos \left(\pi x(1+\beta) \right)}{\pi x \left(1-(4\beta x)^2 \right)}. 
\end{equation}
It can be seen that the choice of $\beta_\tau=\beta_\nu=0$ in the RRC filter (\ref{eq:rrc1}) specializes to the sinc filter. Also, for $\beta_\nu>0$ and $\beta_\tau>0$, there is time and bandwidth expansion such that  $T'=T(1+\beta_{\nu})$ and $B'=B(1+\beta_{\tau})$, resulting in a spectral efficiency of $\frac{BT}{B'T'}<1$ symbol/dimension. 

{\em Gaussian filter:}
For Gaussian filter, $w_\text{tx}(\tau,\nu)$ is given by \cite{zak_otfs7}
\begin{equation}
\hspace{-2mm}
w_\text{tx}(\tau,\nu) \hspace{-0.5mm} = \hspace{-0.5mm} \underbrace{\left(\frac{2\alpha_{\tau}B^2}{\pi}\right)^{\frac{1}{4}}e^{-\alpha_{\tau}B^{2}\tau^{2}}}_{w_1(\tau)} \ \underbrace{\left(\frac{2\alpha_{\nu}T^2}{\pi}\right)^{\frac{1}{4}}e^{-\alpha_{\nu}T^{2}\nu^{2}}}_{w_2(\nu)}\hspace{-1mm}. \hspace{-0mm}
\label{eq:gauss1}
\end{equation}
The Gaussian pulse can be configured by adjusting the parameters $\alpha_{\tau}$ and $\alpha_{\nu}$. Because of the infinite support in Gaussian pulse, a time duration $T'$ where 99$\%$ of the frame energy is localized in the time domain and a bandwidth $B'$ where 99$\%$ of the frame 
energy is localized in the frequency domain are considered \cite{zak_otfs7}. No time and bandwidth expansion (i.e., $T'=T$, $B'=B$) in the Gaussian pulse corresponds to setting 
$\alpha_{\tau}=\alpha_{\nu}=1.584$. 

\section{Bayesian Optimization for Query Plans}
\label{sec:technique}

Given our goal to perform offline optimization of queries while minimizing the cost of executing extra queries against the database, and given that we do not know how to efficiently explore the space of possible query plans, Bayesian optimization is a promising approach. Bayesian optimization (BO) enables optimization of expensive-to-evaluate, black-box functions while requiring relatively few evaluations of the expensive function. However, Bayesian optimization techniques operate over continuous, real-valued domains, whereas query plans are discrete tree structures.

Prior work on Latent Space Bayesian Optimization (LSBO)
% on drug discovery \cite{lolbo} 
has allowed Bayesian optimization to be applied over other discrete, combinatorial spaces by using a deep autoencoder model (DAE) to transform a discrete, structured search space $\mathcal{X}$ into a continuous, numerical one $\mathcal{Z}$ \cite{Weighted_Retraining,ladder,gomez2018automatic,Huawei,eissman2018bayesian,kajino2019molecular, lolbo}.
For example, \citet{lolbo} applied LSBO to problems in drug discovery using a DAE trained on string representations of molecules. 
% has applied BO in the latent space of a deep autoencoder model (DAE) trained on string representations of molecules. 
Inspired by this work, we define a string encoding format for query plans that can represent all possible query plans and in which each string decodes unambiguously to a particular query plan (\Cref{sec:string-format}). We train a variational autoencoder (VAE) on strings of this format using plans derived from a traditional query optimizer (\Cref{sec:technique-vae}). We perform BO in the latent space of this VAE (\Cref{sec:bo-background}), making novel contributions in the selection of timeouts (\Cref{sec:censored_observations}). Finally, we discuss different strategies for initializing the local BO process, as the quality of initialization points impacts BO performance (\Cref{sec:initialization_strategies}).

\subsection{Query Plan String Format}
\label{sec:string-format}
Our first step towards optimizing query plans with Bayesian optimization is to build a string language for query plans that we encode and decode into a vector space. We first explain the desired properties of this string language. Then, we explain the string language we designed to have these properties. While we believe our string languages makes some good tradeoffs, in theory any string language with the desired properties could be equally effective.

\sparagraph{Desiderata for string representations.} We identify two important attributes of our string language from prior work on molecular optimization. Maus et al.~\cite{lolbo} identify two essential properties that are key to success of a string language for Bayesian optimization (in the case of~\cite{lolbo}, the string language is called SELFIES \cite{selfies} and describes molecules). We translate those design constraints from the domain of~\cite{lolbo} to query optimization, and adopt these properties as requirements for our query plan language:
\begin{enumerate}[leftmargin=*]
    \item \textbf{Completeness.} Any valid query plan in $X$ must be representable as a string using the language. If this is not true, high-quality plans that are not representable will not be discoverable by our optimization algorithm.
    \item \textbf{Decoding validity.} Any sequence of characters in the language must correspond to a unique and valid query plan. If latent codes of the DAE decode to invalid query plans, optimization would need to be done under additional feasibility constraints, which adds unnecessary complexity to the optimization problem. Validity was a primary goal of the models in both \citet{JTVAE} and \citet{maus2022local}.
\end{enumerate}

\edit{The ideal string representation would also be \emph{injective}, meaning that each unique string maps to a unique plan~\cite{geom_card_est}. We were unable to find a string representation with all three of these properties, so we settle for a complete representation with decoding validity that is not injective (the same plan might be represented by multiple strings). Taking this tradeoff is motivated by prior work: Maus et al.~\cite{bayes_latent} showed that a string representation with only completeness and decoding validity (SELFIES~\cite{selfies}) outperformed an injective representation with completeness (SMILES~\cite{weininger_smiles}). We leave the investigation of alternative string representations to future work.}

\sparagraph{String language for query plans} We design an encoding format for binary-tree-structured query plan defining join orders and operators (henceforth, a ``join tree''). \edit{The join tree does not encode other aspects of the query such as selections, join and filter predictes, and aggregations. When the join tree is decoded to executable SQL, we translate the join tree to a hint string and prepend it to the original SQL text which contains these aspects of the query.} We observe that a join tree can be unambiguously reconstructed by knowing for each non-leaf node the left and right subtree that compose it, and that valid trees have left and right subtrees that are disjoint; we simply need an unambiguous way to identify these subtrees.

In our encoding format, each join subtree is expressed as a 3 symbol sequence (left, right, operator). Each physical join operator ($h$ash, $m$erge, $n$ested loops) is given a unique symbol. The fully specified query plan is simply the concatenation of these sequences.

The leaves of a join tree are always the tables being joined, so we define a unique symbol for each base table in the schema. However, we cannot define symbols for each possible join subtree, as doing so would be tantamount to defining a unique symbol for every possible query plan. Instead, we observe that after a table symbol is used once to specify a join subtree, it will never be used again, as any table will only ever appear as a leaf once. The same is true for each join subtree after it is specified to be the child of some larger subtree. As such, to the right of a particular join subtree sequence, the symbols composing the join instead refer to the larger subtree.

\edit{For multiple occurrences of the same table under different aliases within the same query, we rename such aliases to be numbered (e.g. \texttt{movies1}, \texttt{movies2}, ...), and we define a unique symbol in our language per table-number pair. This does require choosing a maximum number of possible aliases of a single table; in our experiments we select the maximum occurrences of a particular table within the benchmark queries.}

The leftmost occurrence of a table symbol in a plan string always references the base table itself, but subsequent occurrences represent the largest subtree that the table is part of. For example, in a join between three tables $A$, $B$, and $C$, $(A \bowtie_\text{hash} (B \bowtie_\text{merge} C))$, the valid encoding strings are $(B, C, \bowtie_\text{m}, A, B, \bowtie_\text{h})$ and $(B, C, \bowtie_\text{m}, A, C, \bowtie_\text{h})$.

To fulfill our second requirement, decoding validity, we use a simple trick. We maintain state about the partially-specified join tree as we decode the string from left to right. If the decoder encounters a symbol that is not syntactically valid (e.g. a table in place of a join operator) or semantically valid (e.g. a table that is not part of the join), it deterministically resolves it to a symbol that is valid by constructing a list of all valid symbols and using the invalid symbol's integer value as an index into the list.

\edit{We note that the choice of replacement symbol is arbitrary, but this scheme for ensuring decoding validity is preferable to more obvious ideas such as simply refusing and resampling when encountering invalid strings or decoding them to some default plan. Bayesian optimization will be performed over a vector space that decodes to potential plan strings. Rejecting strings would prevent the surrogate model learning about vast regions of the plan space. Decoding all invalid strings to some default plan would make vast regions of the space undifferentiated in performance. Our technique ensures that all strings decode to valid plans, that similar invalid strings are mapped to somewhat different valid query plans, and that these decodings are a valid function of the input string.}

\edit{
\sparagraph{Limitations} Our language does not represent subqueries and CTEs. When processing queries that contain such structures, they are left untouched, so the decoded query plan hint will not contain any reference to them or to tables only occurring within them.
}






\subsection{Encoding \& Decoding Query Plans} \label{sec:technique-vae}
While BO can be applied to various search spaces, it is most straightforward in a continuous, real-valued domain. This presents a challenge when optimizing query plans, which are inherently discrete tree structures. To address this, we construct a mapping between query plans and points in a continuous domain. Having defined our string representation in \Cref{sec:string-format}, we train a deep autoencoder (DAE) model on these encoded strings. This process generates a \emph{latent space} – a continuous, real-valued domain that serves as a proxy for the discrete space of query plans, enabling application of BO techniques. Intuitively, the goal of the DAE is to construct a latent space in which similar query plans are mapped to similar vectors. This way, a search routine that finds a particularly good plan can look at the ``neighbors'' of that plan in the latent space for similar plans. The notions of ``similar'' and ``neighbors'' are both highly approximate: no actual neighborhoods or similarity scores are computed, but instead this property is \emph{implicitly} created when training the DAE.

A DAE consists of an encoder $\Phi: X \to Z$ that maps from an input space $X$ to a latent space $Z$ (sometimes called a bottleneck~\cite{bottleneck}, as it is often lower dimensional than $X$ in order to force a degree of compression) and a decoder $\Gamma : Z \to X$ that maps from the latent space back to the input space. We use a type of DAE known as a variational autoencoder (VAE)~\cite{KingmaW13}, in which the encoder produces a distribution over latent points $\Phi(Z | X)$, and the decoder produces a distribution over $X$ given $Z$, $\Gamma(X | Z)$. The model is trained by maximizing the evidence lower bound (ELBO):
$$
\mathbb{E}_{\Phi(Z | X)}[\log \Gamma(X | Z) - \text{KL}(\Phi(Z | X) || p(Z) ]
$$

In training such a DAE, we construct a mapping $\Gamma : Z \to X$ that can produce string-encoded query plans given points in the latent space. The VAE regularization (the KL term, representing relative entropy) makes the search space smooth, facilitating more effective optimization. This is precisely what we need in order to perform BO: the surrogate model is defined over the latent space, and we evaluate the black-box function $f$ for points in the latent space by decoding the point to a string query plan through the DAE and executing them against the real database.

\sparagraph{Training data} In order to train the DAE, we compute a large set of encoded query plans ($\sim$1 million) from the database schema. Importantly, this process does not require any query execution, and can be done exclusively with only metadata from the DBMS. The idea is to create a suitably diverse set of ``reasonable'' query plans so that the DAE can learn a smooth probability distribution of the space of query plans. By ``reasonable'' here, we do not mean that these plans must all be optimal, just that they must be somewhat representative of the \emph{family} of optimal query plans---\edit{the purpose of this is to create a space of plans in which points that are close to each other have similar performance characteristics. However, the space still contains points for \emph{all possible} query plans.}
% that is, we hope that the optimal plan for any query is somewhat similar to at least one of the plans we generate as training data for the DAE.

% To generate this set of plans, we first create a set of random join queries by assembling a graph $G_{R} = (V, E)$ representing PK-FK relationships in the schema (the ``reference graph''). For a schema containing a set of tables
% $T = \left \{ t_1, t_2, ..., t_n \right \}$
% with each table $t_i$ having attributes
% $C^i = \left \{ c^i_1, c^i_2, ... c^i_m \right \}$
% including PK-FK relationships between foreign keys $c^i_f$ and primary keys $c^j_p$ denoted by
% $\text{Ref} \left( c^i_f, c^j_p \right )$,
% \begin{gather*}
%     V = T \\
%     E = \left \{ (t_i, t_j) \mid t_i, t_j \in T \wedge \exists c^i_f \in C^i, c^j_p \in C^j : \text{Ref}(c^i_f, c^j_p) \right \}
% \end{gather*}

% We then expand this graph to include up to $k$ aliases with copies of tables. For a set of aliases
% $K = \left \{ 1, 2, ... k \right \}$
% the ``alias-$k$ reference graph'' $G^k_R = (V^k, E^k)$,
% \begin{gather*}
%     V^k = \left \{ t^a_i \mid t_i \in V \wedge a \in K \right \} \\
%     E^k = \left \{ (t^a_i, t^b_j) \mid (t_i, t_j) \in E \wedge a, b \in K \right \}
% \end{gather*}
\edit{To generate this set of plans, we sample random PK-FK equijoin queries from the schema by constructing the ``alias-$k$ reference graph'' which contains $k$ nodes corresponding to each table and edges corresponding to all PK-FK references between tables. We choose $k$ equal to the highest number of aliases of the same table used in any query in the workload. From this alias-$k$ reference graph, we sample queries by selecting random connected subgraphs with varying numbers of vertices. Given a particular subgraph, we produce a query joining all table aliases with join predicates corresponding to all present edges.}
% From $G^k_R$, we sample queries within the schema by selecting random connected subgraphs of varying numbers of vertices. Given a particular subgraph $G^{k\prime}_R = (V^{k\prime}, E^{k\prime})$ we produce the query joining all table aliases in $V^{k\prime}$ with join predicates corresponding to all references in $E^{k\prime}$

For each sampled query, we plan the query using the existing default query optimizer (e.g. PostgreSQL), encode the plan in our string encoding format, and add it to the VAE training set. In order to expand the diversity of plans used to train the VAE, we additionally produce encoded plans using hints~\cite{url-pg_hints} to the default query optimizer (e.g. disable nested loops, disable sequential scans).

Our training data generation process makes two key design choices: (1) sampling random queries from the database schema, and (2) generating query plans with the database's default optimizer. The first decision ensures that we have coverage for a wide variety of input queries. Our goal is to train the DAE once per schema, and then reuse the DAE for every query over the schema. The second decision ensures that the query plans we get are somewhat reasonable. For example, the underlying database optimizer is unlikely to pick a plan full of cross joins, and is likely to take advantage of index structures if applicable.


\subsection{Background on Bayesian Optimization} \label{sec:bo-background}
Given a query plan language and a trained DAE to translate query plan strings into vectors in a latent space (and back), we can now optimize queries inside of the latent space using Bayesian optimization (BO). Intuitively, BO in our application works by learning the relationship between the DAE's latent space and actual query plan latency. BO learns this relationship by repeatedly testing points sampled from the latent space. If the BO algorithm can get a good estimation of the relationship between the latent space and query latency, then good plans can be found. This section gives important background on the BO technique we use in this paper. Then, in Section~\ref{sec:censored_observations}, we explain some of the small changes we made to traditional algorithms to address query optimization specifically.

% Bayesian optimization \cite{distill_bayesopt} is a method for optimizing black-box functions. For some multidimensional space of inputs $X$ and black-box evaluation function $f$, BO finds some $x \in X$ that minimizes $f(x)$. BO is typically deployed when $f$ is expensive to evaluate (e.g., $f$ requires executing a physical plan against a DBMS and measuring its runtime). Because of this constraint, BO techniques, unlike reinforcement learning techniques, are designed to be  \emph{sample-efficient}, requiring few evaluations of $f$.

% BO works by iteratively refining a \emph{surrogate model} of $f$: it
% \begin{enumerate*}
%     \item initializes the surrogate model with some prior distribution,
%     \item uses an \emph{acquisition function} to select points in the input space $x \in X$,
%     \item evaluates $f(x)$ and
%     \item updates the surrogate model with the point $(x, f(x))$
% \end{enumerate*}

% BO assumes that the true unknown function $f$ exhibits locality: observing a particular point $(x, f(x))$ suggests that points nearby $x' \approx x$ will have similar values of $f(x') \approx f(x)$. More evidence further refines the surrogate model, making it a better estimate of $f$. In our setting, $x$ is some representation of a query plan, and the unknown function $f$ returns the execution latency of that plan. %Evidence is obtained by actually executing the query plan represented by $x$ against a database.

% In order to efficiently find good values $x$ that minimize $f(x)$, BO must balance exploration (finding out more about relatively unknown regions of $X$) with exploitation (focusing on promising regions of $X$ already known to have the lowest values seen for $f(x)$). Selecting points to sample is the role of the \emph{acquisition function.} Our approach utilizes \emph{Thompson sampling}~\cite{thompson} as its acquisition function.

% \factcheck{Is everything from beginning of subsection to here correct?}
\sparagraph{Bayesian Optimization} This section provides a brief overview of Bayesian Optimization (BO). For readers unfamiliar with BO, we recommend the comprehensive book by \citet{garnett_bayesoptbook_2023}. Our methodology builds upon the approach developed by \citet{eriksson2019scalable}, \edit{with specific novel modifications tailored for optimizing query plans and execution latency in a DBMS.}

Bayesian Optimization is a method for optimizing black-box functions that are expensive to evaluate, aiming for \emph{sample efficiency}. Given an input space $\mathcal{X}$ and an unknown objective function $f: \mathcal{X} \rightarrow \mathbb{R}$, BO seeks to find an input $x^* \in \mathcal{X}$ that minimizes $f(x)$ in as few evaluations of $f$ as possible. This is particularly useful when each evaluation of $f(x)$ is costly---for example, when $f(x)$ involves executing a query plan in a DBMS to measure its runtime.

BO operates by constructing a probabilistic surrogate model of the objective function, which is iteratively refined as new data is acquired. The general optimization procedure follows these steps:

\begin{enumerate}[leftmargin=*]
\item \textbf{Initialization}: Build a surrogate model of the objective function $f$.
\item \textbf{Acquisition Function Optimization}: Use an acquisition function to select the next point $x_{\text{next}}$ to evaluate, balancing exploration and exploitation.
\item \textbf{Evaluation}: Compute the true function value $f(x_{\text{next}})$ by executing the query plan corresponding to $x_{\text{next}}$.
\item \textbf{Model Update}: Update the surrogate model with the new observation $(x_{\text{next}}, f(x_{\text{next}}))$, and repeat steps 2--4.
\end{enumerate}

To efficiently navigate the search space, BO leverages the surrogate model along with the acquisition function to select promising candidate plans while minimizing the number of expensive evaluations. In this work, we use \emph{Thompson Sampling}~\cite{thompson} as the acquisition function.

\sparagraph{Local BO} Standard BO methods can struggle with high-dimensional or discrete optimization problems, such as those encountered in query plan optimization, due to the curse of dimensionality and the combinatorial explosion of the search space. To address this, we incorporate methods from the local BO literature, specifically \emph{TuRBO}~\cite{eriksson2019scalable}. TuRBO maintains a hyper-rectangular ``trust region'' within the input space, which constrains the region from which points are sampled. By dynamically adjusting the size and location of these trust region based on the the optimization success / failure, TuRBO can balance global exploration with local exploitation, allowing for efficient optimization in high-dimensional spaces. \footnote{\edit{Though called ``local BO'', this is a \emph{global} optimization process that can produce results significantly different from the initialization points. Local BO methods are the most competitive methods in high-dimensional spaces, as established in Eriksson et al. ~\cite{eriksson2019scalable}.}}

\sparagraph{Right-Censored Observations}
During typical Bayesian optimization, when we make an observation at a point $\bx$, we obtain an associated objective value $f(\bx) = y$. 
For a right-censored observation, when we observe at the point $\bx$, we instead only learn that $y$ was greater than some threshold $\tau$. In our application, right-censored observations represent query timeouts: If a query $\bx$ is observed to execute for $\tau$ seconds before timing out, then we know that the true latency of $q$ is \emph{at least} $\tau$: $f(\bx) \geq \tau$. 

%Conventional BO techniques assume that evaluations of $f$ correspond to the true value of the observation with additive noise. 
In the query optimization setting, using right-censored observations is particularly important. Obtaining true values for arbitrary plans in the space of possible plans can be infeasible, as bad plans may take days or even weeks. Thus, it is more efficient if we can \emph{time out} plans that perform poorly and update the surrogate model with knowledge that the running time of $x$ is ``at least as bad as $y$''. Intuitively, for regions of $X$ that contain truly awful  plans, for the purposes of finding optimal plans, it is not necessary to know exactly how bad a particular plan is---it suffices to know that plans like it should be avoided. 

BO in the presence of censored observations was first explored by Hutter et al. ~\cite{DBLP:journals/corr/HutterHL13}, where an EM-like algorithm was used to impute the value of censored responses.
%They applied this to algorithm configuration, and terminated runs longer than a constant factor longer than the best running time observed so far.
They applied this method to algorithm configuration, terminating any runs that exceeded a constant factor of the shortest running time observed so far. Building on this, Eggensperger et al. ~\cite{eggensperger2020neural} trained a neural network surrogate on a likelihood based on the Tobit model to directly model right-censored observations:
%
\begin{equation}
\begin{aligned}
\label{eq: tobit}
p(\mathbf{y}|\mathbf{f}) &= \phi(\mathbf{z})^{1-\mathbf{I}}(1-\Phi(\mathbf{z}))^\mathbf{I} \\
\mathbf{z} &= \frac{\mathbf{f}-\mathbf{\mu}}{\mathbf{\sigma}^2}\\ 
I&= \begin{cases}
       0, & \text{if } \mathbf{y} \text{ is uncensored} \\
       1, & \text{if } \mathbf{y} \text{ is censored}
    \end{cases}
\end{aligned}
\end{equation}
%

\noindent where $\phi$ and $\Phi$ denote the Gaussian density and cumulative density function respectively. In~\cite{eggensperger2020neural}, timeout thresholds were chosen as a fixed percentile of existing observations.

\sparagraph{Approximate Gaussian Processes}  
Because the space of query plans is large, we anticipate needing to test a large number of query plans. As a result, we must select a surrogate model that (1) allows for \emph{probabilistic inference}, that is, gives a probability distribution at each point instead of a simple point estimate, and (2) can scale to a large number of observations. Thus, we select an \emph{approximate} Gaussian Process (GP) model.

Approximate GP models, such as the popular Scalable Variational Gaussian Process (SVGP)~\cite{svgp}, use inducing point methods in combination with variational inference to allow approximate GP inference on large data sets~\cite{hensman2013gaussian,titsias2009variational}. The standard evidence lower bound (ELBO) on the log-likelihood used to train a SVGP model is the following:
% 
\begin{align}
\label{eq: svgp}
\log p(\by) & \geq  \mathbb{E}_{q(\bfn)}[\log p(\by \mid \bfn)] - \textrm{KL}(q(\bu)\,||\,p(\bu)) 
\end{align}
%
% Because of the large number of total query plans we will execute over the course of optimization and the need for approximate inference due to the non-Gaussian likelihood \autoref{eq: tobit}, we adapt SVGP \citep{svgp} models to this purpose.



\subsubsection{Bayesian Optimization with Censored Observations} \label{sec:censored_observations}

While previous work on Bayesian optimization with censored observations (censored BO) did not use approximate SVGP~\cite{svgp} models, \edit{we contribute a straightforward extension} of SVGP~\cite{svgp} models to the censored BO setting. 
Starting from \Cref{eq: svgp} and using the Tobit likelihood given in~\Cref{eq: tobit}, we derive the new ELBO:
%
\begin{equation*}
\centering
% Im just doing this so it fits for now
\begin{aligned}
& \log p(\by) \geq  \mathbb{E}_{q(\bfn)}[\log p(\by \mid \bfn)] - \textrm{KL}(q(\bu)\,||\,p(\bu)) \\
			& = \mathbb{E}_{q(\bfn)}[\log \phi(\mathbf{Z})^{1-\mathbf{I}}(1-\Phi(\mathbf{Z}))^\mathbf{I}] - \textrm{KL}(q(\bu)\,||\,p(\bu)) \\
			& = \mathbb{E}_{q(\bfn)}[\log \phi(\mathbf{Z})^{1-\mathbf{I}} + \log(1-\Phi(\mathbf{Z}))^\mathbf{I}] - \textrm{KL}(q(\bu)\,||\,p(\bu)) \\
            & = \mathbb{E}_{q(\bfn)}[\log \phi(\mathbf{Z_u})] + \mathbb{E}_{q(\bfn)}[\log(1-\Phi(\mathbf{Z_c}))] - \textrm{KL}(q(\bu)\,||\,p(\bu)) 
\end{aligned}
\end{equation*}
%
Here, $\mathbf{Z}_{u}$ correspond to $\frac{\mathbf{f} - \mu}{\sigma^2}$ values for uncensored observations, and $\mathbf{Z}_{c}$ correspond to censored observations. The first term
%, $\mathbb{E}_{q(\bfn)}[\log \phi(\mathbf{Z_u})]$, 
can be computed analytically as in standard SVGP models. The second term, $\mathbb{E}_{q(\bfn)}[\log(1-\Phi(\mathbf{Z_c}))]$, can be computed using one dimensional numerical techniques like Gauss-Hermite quadrature.  

During optimization, we select a threshold $\tau$ for each executed query plan $\bx$, and cut off execution once the running time exceeds $\tau$. This results in right-censored observations. 
Selecting the timeout for any given observation is crucial: selecting too low of a timeout deprives BO of important knowledge about the space of plans, whereas selecting too high of a timeout wastes time executing bad plans. Previous work in BO uses a constant multiplier over the best observation seen so far~\cite{hutter2013_bocensored}, or a fixed percentile across all observations~\cite{eggensperger2020_censored}. Balsa~\cite{balsa} also uses a fixed multiplier  in order to bound the impact of executing bad plans. 
\edit{We use an uncertainty-based method for selecting timeouts that, compared to prior work, ensures that the surrogate model will be sufficiently confident that a particular point is suboptimal before timing out.}

% We cannot just treat the timeout as true values. Example: $P_1$ would've taken an hour, but times out after 5m. $P_2$ would've taken 30 minutes, but times out after 5m. $P_2$ is twice as good as $P_1$, but they look "the same" to the BO algo. Tricks like multipling the timeout by some constant require adhoc tuning (like balsa).

Before evaluating a new candidate query plan $x_t$ during step $t$ of optimization, we dynamically set a new timeout threshold $\tau_t$. 
We select thresholds so that, after conditioning on the right-censored observation $(\bx_{t}, \tau_{t})$, we are \textit{confident} that the best query plan observed so far, $\bx^{*}_{t}$, is still a better design than the candidate plan $\bx_{t}$. 
Because we do not want to waste additional running time evaluating $f(\bx_{t})$, we ideally want the \textit{smallest} such $\tau_{t}$. 

\paragraph{Selecting $\tau_{t}$.} The above discussion leads to the following optimization problem, where we find the smallest threshold $\tau$ so that our incumbent is confidently better than $\bx_{t}$ \textit{after conditioning on $(\bx_{t}, \tau)$}:
%
\begin{align*}
    \tau_{t}^{*} &= \argmin \tau \\
    & \textrm{s.t.}\;\; y^{*}_{t} \leq \mu'_{t}(\tau) - \kappa \sigma'_{t}(\tau)
\end{align*}
%


On \edit{its} surface, this optimization problem is challenging, as evaluating our constraint for a given $\tau$ involves updating the Gaussian process surrogate model with that value $\tau$ as the observed timeout. This is similar to other acquisition functions in the Bayesian optimization literature that use fantasization to do lookahead, e.g., knowledge gradient~\cite{frazier2009knowledge}.

Because we use variational GPs, there are several inexpensive strategies that we can use to evaluate the constraint. For example, \citet{maddox2021conditioning} recently proposed an efficient routine for online updating sparse variational GPs, both with conjugate and non-conjugate likelihoods. Alternatively, a few additional iterations of SGD can be used to update the model in a less sophisticated way.

Finally, we note that the value of $\mu'_{t}(\tau) - \kappa \sigma'_{t}(\tau)$ should generally be monotonic in $\tau$---fantasizing that $\bx_{t}$ cut off with a larger threshold should strictly increase the gap between our belief about $y_{t}$ and $y^{*}_{t}$. Therefore, given a routine to cheaply evaluate the constraint, the constrained minimization problem over $\tau$ can be solved e.g. with binary search.


\subsection{Initialization Strategies} \label{sec:initialization_strategies}
The initial step of BO for a given query typically involves selecting points within the latent space using the acquisition function. As the surrogate is initialized with a random prior, this amounts to selecting random points within the latent space. Theoretically, given sufficient time for BO execution, this approach would yield optimal results. However, to improve the practicality of BO within high dimensional spaces, it is helpful to initialize the process with a small number of precomputed $(x, f(x))$ pairs representing high-quality plans. We explore multiple methods of generating these initialization points.

% \jeff{cut? goes against story about information/time tradeoff} A common theme across all of these initialization techniques is that in the interest of saving computation time, not all proposed plans are executed to completion. Since our system can accommodate censored observations, as described in the previous section, execution of any particular plan is cut off at the runtime of the best plan seen so far. As the runtime of the best found plan improves, these strategies search through subsequent proposed plans more quickly as the timeout is lower.

% Regardless of the strategy we use to generate initial data points (plans), we execute the plans proposed by these strategies serially, seeking to minimize the total computation time spent. For faster wall clock time, it is possible to execute proposed plans in parallel batches, but computing the latency of the initialization points is usually not a significant computational cost (we verify this experimentally in Section~\ref{sec:eval}).

\sparagraph{Hinted plans (Bao)} We can leverage an existing traditional query optimizer that accepts hints, such as PostgreSQL, to generate the initialization points. We exhaust all of the combinations of join and scan hints (as in the hint sets used by Marcus et al.'s Bao~\cite{bao} optimizer) to produce 49 initialization points for each query. These 49 initialization points are \emph{guaranteed} to contain the best plan that could have possibly been chosen by Bao. We note that it is not important \emph{which} of the queries in the initialization set is optimal, merely that the initialization set contains some queries that represent a promising starting point for optimization. Thus, it is not necessary to ``prune'' hint sets from Bao, as is recommended in~\cite{bao}.

\sparagraph{The default optimizer plan} A simpler strategy would be to generate a single optimization point by using the DBMS' underlying optimizer. This approach has the advantage of simplicity, since the underlying DBMS almost surely has an optimizer. Unfortunately, we found that this approach does not work well in practice, mostly because initializing BO with a single initialization point seems to be suboptimal~\cite{lolbo}.

\sparagraph{LLM} Inspired by previous work demonstrating the effectiveness of large language models (LLMs) in optimizing program runtimes \cite{pie}, we explore the use of fine-tuned LLMs for generating initialization points. We collected trajectories from 606 \sysname runs, selecting the top-1 and top-5 query plans for each query to construct a fine-tuning dataset. Using this dataset, we fine-tuned GPT4o-mini for one epoch. For each new query, we use the fine-tuned model to sample 50 initialization points. This approach leverages the model's ability to learn patterns from previous optimization runs, potentially producing high-quality plans that outperform those generated by traditional query optimizers. Our evaluation (\cref{sec:eval-llm}) demonstrates that this LLM-based strategy can often produce the best query plan among all initialization strategies considered here.

\sparagraph{Extensibility} \sysname simply admits sets of initialization pairs $(x, f(x))$, so any strategy can be used to generate these pairs. As such, our approach can incorporate future improvements in traditional or learned query optimization techniques.

% tradeoffs around size of initialization set (bigger = more budget spent on executing the initial set, smaller = not enough information)

\subsection{Random plans} \label{sec:random}
Though not related to BO, we implement random plan search, which can be thought of as a completely exploration-based algorithm. The intuition behind this method is that joins are commutative but that cross-joins are generally bad for performance. This strategy samples random plans from the space of all plans that do not contain any cross joins.

% Given a particular query over a set of table aliases $T'$, we can construct the alias-$k$ reference graph $G_Q = (V_Q, E_Q)$ containing only the table aliases referenced in the query:

% \begin{gather*}
%     V_Q = T' \\
%     E_Q = \left \{ (t_1, t_2) \mid  t_1, t_2 \in T'; (t_1, t_2) \in E^k \right \}
% \end{gather*}

\edit{Given a particular query over a set of table aliases, we construct the subgraph of the schema's alias-$k$ reference graph containing only the table aliases referenced in the query.}
From this query graph, we can construct a random join tree by constructing a spanning tree. Whenever an edge is added to the spanning tree, we add the join between the two newly connected components to the join tree. Physical join operators are selected uniformly randomly.

One potential benefit of utilizing this strategy is that its viability implies that it may be possible to perform offline optimization in the absence of a traditional query optimizer. As we show in \Cref{sec:eval}, this strategy can be used on its own to perform offline optimization.









\section{Experiments}
\label{sec:experiments}
The experiments are designed to address two key research questions.
First, \textbf{RQ1} evaluates whether the average $L_2$-norm of the counterfactual perturbation vectors ($\overline{||\perturb||}$) decreases as the model overfits the data, thereby providing further empirical validation for our hypothesis.
Second, \textbf{RQ2} evaluates the ability of the proposed counterfactual regularized loss, as defined in (\ref{eq:regularized_loss2}), to mitigate overfitting when compared to existing regularization techniques.

% The experiments are designed to address three key research questions. First, \textbf{RQ1} investigates whether the mean perturbation vector norm decreases as the model overfits the data, aiming to further validate our intuition. Second, \textbf{RQ2} explores whether the mean perturbation vector norm can be effectively leveraged as a regularization term during training, offering insights into its potential role in mitigating overfitting. Finally, \textbf{RQ3} examines whether our counterfactual regularizer enables the model to achieve superior performance compared to existing regularization methods, thus highlighting its practical advantage.

\subsection{Experimental Setup}
\textbf{\textit{Datasets, Models, and Tasks.}}
The experiments are conducted on three datasets: \textit{Water Potability}~\cite{kadiwal2020waterpotability}, \textit{Phomene}~\cite{phomene}, and \textit{CIFAR-10}~\cite{krizhevsky2009learning}. For \textit{Water Potability} and \textit{Phomene}, we randomly select $80\%$ of the samples for the training set, and the remaining $20\%$ for the test set, \textit{CIFAR-10} comes already split. Furthermore, we consider the following models: Logistic Regression, Multi-Layer Perceptron (MLP) with 100 and 30 neurons on each hidden layer, and PreactResNet-18~\cite{he2016cvecvv} as a Convolutional Neural Network (CNN) architecture.
We focus on binary classification tasks and leave the extension to multiclass scenarios for future work. However, for datasets that are inherently multiclass, we transform the problem into a binary classification task by selecting two classes, aligning with our assumption.

\smallskip
\noindent\textbf{\textit{Evaluation Measures.}} To characterize the degree of overfitting, we use the test loss, as it serves as a reliable indicator of the model's generalization capability to unseen data. Additionally, we evaluate the predictive performance of each model using the test accuracy.

\smallskip
\noindent\textbf{\textit{Baselines.}} We compare CF-Reg with the following regularization techniques: L1 (``Lasso''), L2 (``Ridge''), and Dropout.

\smallskip
\noindent\textbf{\textit{Configurations.}}
For each model, we adopt specific configurations as follows.
\begin{itemize}
\item \textit{Logistic Regression:} To induce overfitting in the model, we artificially increase the dimensionality of the data beyond the number of training samples by applying a polynomial feature expansion. This approach ensures that the model has enough capacity to overfit the training data, allowing us to analyze the impact of our counterfactual regularizer. The degree of the polynomial is chosen as the smallest degree that makes the number of features greater than the number of data.
\item \textit{Neural Networks (MLP and CNN):} To take advantage of the closed-form solution for computing the optimal perturbation vector as defined in (\ref{eq:opt-delta}), we use a local linear approximation of the neural network models. Hence, given an instance $\inst_i$, we consider the (optimal) counterfactual not with respect to $\model$ but with respect to:
\begin{equation}
\label{eq:taylor}
    \model^{lin}(\inst) = \model(\inst_i) + \nabla_{\inst}\model(\inst_i)(\inst - \inst_i),
\end{equation}
where $\model^{lin}$ represents the first-order Taylor approximation of $\model$ at $\inst_i$.
Note that this step is unnecessary for Logistic Regression, as it is inherently a linear model.
\end{itemize}

\smallskip
\noindent \textbf{\textit{Implementation Details.}} We run all experiments on a machine equipped with an AMD Ryzen 9 7900 12-Core Processor and an NVIDIA GeForce RTX 4090 GPU. Our implementation is based on the PyTorch Lightning framework. We use stochastic gradient descent as the optimizer with a learning rate of $\eta = 0.001$ and no weight decay. We use a batch size of $128$. The training and test steps are conducted for $6000$ epochs on the \textit{Water Potability} and \textit{Phoneme} datasets, while for the \textit{CIFAR-10} dataset, they are performed for $200$ epochs.
Finally, the contribution $w_i^{\varepsilon}$ of each training point $\inst_i$ is uniformly set as $w_i^{\varepsilon} = 1~\forall i\in \{1,\ldots,m\}$.

The source code implementation for our experiments is available at the following GitHub repository: \url{https://anonymous.4open.science/r/COCE-80B4/README.md} 

\subsection{RQ1: Counterfactual Perturbation vs. Overfitting}
To address \textbf{RQ1}, we analyze the relationship between the test loss and the average $L_2$-norm of the counterfactual perturbation vectors ($\overline{||\perturb||}$) over training epochs.

In particular, Figure~\ref{fig:delta_loss_epochs} depicts the evolution of $\overline{||\perturb||}$ alongside the test loss for an MLP trained \textit{without} regularization on the \textit{Water Potability} dataset. 
\begin{figure}[ht]
    \centering
    \includegraphics[width=0.85\linewidth]{img/delta_loss_epochs.png}
    \caption{The average counterfactual perturbation vector $\overline{||\perturb||}$ (left $y$-axis) and the cross-entropy test loss (right $y$-axis) over training epochs ($x$-axis) for an MLP trained on the \textit{Water Potability} dataset \textit{without} regularization.}
    \label{fig:delta_loss_epochs}
\end{figure}

The plot shows a clear trend as the model starts to overfit the data (evidenced by an increase in test loss). 
Notably, $\overline{||\perturb||}$ begins to decrease, which aligns with the hypothesis that the average distance to the optimal counterfactual example gets smaller as the model's decision boundary becomes increasingly adherent to the training data.

It is worth noting that this trend is heavily influenced by the choice of the counterfactual generator model. In particular, the relationship between $\overline{||\perturb||}$ and the degree of overfitting may become even more pronounced when leveraging more accurate counterfactual generators. However, these models often come at the cost of higher computational complexity, and their exploration is left to future work.

Nonetheless, we expect that $\overline{||\perturb||}$ will eventually stabilize at a plateau, as the average $L_2$-norm of the optimal counterfactual perturbations cannot vanish to zero.

% Additionally, the choice of employing the score-based counterfactual explanation framework to generate counterfactuals was driven to promote computational efficiency.

% Future enhancements to the framework may involve adopting models capable of generating more precise counterfactuals. While such approaches may yield to performance improvements, they are likely to come at the cost of increased computational complexity.


\subsection{RQ2: Counterfactual Regularization Performance}
To answer \textbf{RQ2}, we evaluate the effectiveness of the proposed counterfactual regularization (CF-Reg) by comparing its performance against existing baselines: unregularized training loss (No-Reg), L1 regularization (L1-Reg), L2 regularization (L2-Reg), and Dropout.
Specifically, for each model and dataset combination, Table~\ref{tab:regularization_comparison} presents the mean value and standard deviation of test accuracy achieved by each method across 5 random initialization. 

The table illustrates that our regularization technique consistently delivers better results than existing methods across all evaluated scenarios, except for one case -- i.e., Logistic Regression on the \textit{Phomene} dataset. 
However, this setting exhibits an unusual pattern, as the highest model accuracy is achieved without any regularization. Even in this case, CF-Reg still surpasses other regularization baselines.

From the results above, we derive the following key insights. First, CF-Reg proves to be effective across various model types, ranging from simple linear models (Logistic Regression) to deep architectures like MLPs and CNNs, and across diverse datasets, including both tabular and image data. 
Second, CF-Reg's strong performance on the \textit{Water} dataset with Logistic Regression suggests that its benefits may be more pronounced when applied to simpler models. However, the unexpected outcome on the \textit{Phoneme} dataset calls for further investigation into this phenomenon.


\begin{table*}[h!]
    \centering
    \caption{Mean value and standard deviation of test accuracy across 5 random initializations for different model, dataset, and regularization method. The best results are highlighted in \textbf{bold}.}
    \label{tab:regularization_comparison}
    \begin{tabular}{|c|c|c|c|c|c|c|}
        \hline
        \textbf{Model} & \textbf{Dataset} & \textbf{No-Reg} & \textbf{L1-Reg} & \textbf{L2-Reg} & \textbf{Dropout} & \textbf{CF-Reg (ours)} \\ \hline
        Logistic Regression   & \textit{Water}   & $0.6595 \pm 0.0038$   & $0.6729 \pm 0.0056$   & $0.6756 \pm 0.0046$  & N/A    & $\mathbf{0.6918 \pm 0.0036}$                     \\ \hline
        MLP   & \textit{Water}   & $0.6756 \pm 0.0042$   & $0.6790 \pm 0.0058$   & $0.6790 \pm 0.0023$  & $0.6750 \pm 0.0036$    & $\mathbf{0.6802 \pm 0.0046}$                    \\ \hline
%        MLP   & \textit{Adult}   & $0.8404 \pm 0.0010$   & $\mathbf{0.8495 \pm 0.0007}$   & $0.8489 \pm 0.0014$  & $\mathbf{0.8495 \pm 0.0016}$     & $0.8449 \pm 0.0019$                    \\ \hline
        Logistic Regression   & \textit{Phomene}   & $\mathbf{0.8148 \pm 0.0020}$   & $0.8041 \pm 0.0028$   & $0.7835 \pm 0.0176$  & N/A    & $0.8098 \pm 0.0055$                     \\ \hline
        MLP   & \textit{Phomene}   & $0.8677 \pm 0.0033$   & $0.8374 \pm 0.0080$   & $0.8673 \pm 0.0045$  & $0.8672 \pm 0.0042$     & $\mathbf{0.8718 \pm 0.0040}$                    \\ \hline
        CNN   & \textit{CIFAR-10} & $0.6670 \pm 0.0233$   & $0.6229 \pm 0.0850$   & $0.7348 \pm 0.0365$   & N/A    & $\mathbf{0.7427 \pm 0.0571}$                     \\ \hline
    \end{tabular}
\end{table*}

\begin{table*}[htb!]
    \centering
    \caption{Hyperparameter configurations utilized for the generation of Table \ref{tab:regularization_comparison}. For our regularization the hyperparameters are reported as $\mathbf{\alpha/\beta}$.}
    \label{tab:performance_parameters}
    \begin{tabular}{|c|c|c|c|c|c|c|}
        \hline
        \textbf{Model} & \textbf{Dataset} & \textbf{No-Reg} & \textbf{L1-Reg} & \textbf{L2-Reg} & \textbf{Dropout} & \textbf{CF-Reg (ours)} \\ \hline
        Logistic Regression   & \textit{Water}   & N/A   & $0.0093$   & $0.6927$  & N/A    & $0.3791/1.0355$                     \\ \hline
        MLP   & \textit{Water}   & N/A   & $0.0007$   & $0.0022$  & $0.0002$    & $0.2567/1.9775$                    \\ \hline
        Logistic Regression   &
        \textit{Phomene}   & N/A   & $0.0097$   & $0.7979$  & N/A    & $0.0571/1.8516$                     \\ \hline
        MLP   & \textit{Phomene}   & N/A   & $0.0007$   & $4.24\cdot10^{-5}$  & $0.0015$    & $0.0516/2.2700$                    \\ \hline
       % MLP   & \textit{Adult}   & N/A   & $0.0018$   & $0.0018$  & $0.0601$     & $0.0764/2.2068$                    \\ \hline
        CNN   & \textit{CIFAR-10} & N/A   & $0.0050$   & $0.0864$ & N/A    & $0.3018/
        2.1502$                     \\ \hline
    \end{tabular}
\end{table*}

\begin{table*}[htb!]
    \centering
    \caption{Mean value and standard deviation of training time across 5 different runs. The reported time (in seconds) corresponds to the generation of each entry in Table \ref{tab:regularization_comparison}. Times are }
    \label{tab:times}
    \begin{tabular}{|c|c|c|c|c|c|c|}
        \hline
        \textbf{Model} & \textbf{Dataset} & \textbf{No-Reg} & \textbf{L1-Reg} & \textbf{L2-Reg} & \textbf{Dropout} & \textbf{CF-Reg (ours)} \\ \hline
        Logistic Regression   & \textit{Water}   & $222.98 \pm 1.07$   & $239.94 \pm 2.59$   & $241.60 \pm 1.88$  & N/A    & $251.50 \pm 1.93$                     \\ \hline
        MLP   & \textit{Water}   & $225.71 \pm 3.85$   & $250.13 \pm 4.44$   & $255.78 \pm 2.38$  & $237.83 \pm 3.45$    & $266.48 \pm 3.46$                    \\ \hline
        Logistic Regression   & \textit{Phomene}   & $266.39 \pm 0.82$ & $367.52 \pm 6.85$   & $361.69 \pm 4.04$  & N/A   & $310.48 \pm 0.76$                    \\ \hline
        MLP   &
        \textit{Phomene} & $335.62 \pm 1.77$   & $390.86 \pm 2.11$   & $393.96 \pm 1.95$ & $363.51 \pm 5.07$    & $403.14 \pm 1.92$                     \\ \hline
       % MLP   & \textit{Adult}   & N/A   & $0.0018$   & $0.0018$  & $0.0601$     & $0.0764/2.2068$                    \\ \hline
        CNN   & \textit{CIFAR-10} & $370.09 \pm 0.18$   & $395.71 \pm 0.55$   & $401.38 \pm 0.16$ & N/A    & $1287.8 \pm 0.26$                     \\ \hline
    \end{tabular}
\end{table*}

\subsection{Feasibility of our Method}
A crucial requirement for any regularization technique is that it should impose minimal impact on the overall training process.
In this respect, CF-Reg introduces an overhead that depends on the time required to find the optimal counterfactual example for each training instance. 
As such, the more sophisticated the counterfactual generator model probed during training the higher would be the time required. However, a more advanced counterfactual generator might provide a more effective regularization. We discuss this trade-off in more details in Section~\ref{sec:discussion}.

Table~\ref{tab:times} presents the average training time ($\pm$ standard deviation) for each model and dataset combination listed in Table~\ref{tab:regularization_comparison}.
We can observe that the higher accuracy achieved by CF-Reg using the score-based counterfactual generator comes with only minimal overhead. However, when applied to deep neural networks with many hidden layers, such as \textit{PreactResNet-18}, the forward derivative computation required for the linearization of the network introduces a more noticeable computational cost, explaining the longer training times in the table.

\subsection{Hyperparameter Sensitivity Analysis}
The proposed counterfactual regularization technique relies on two key hyperparameters: $\alpha$ and $\beta$. The former is intrinsic to the loss formulation defined in (\ref{eq:cf-train}), while the latter is closely tied to the choice of the score-based counterfactual explanation method used.

Figure~\ref{fig:test_alpha_beta} illustrates how the test accuracy of an MLP trained on the \textit{Water Potability} dataset changes for different combinations of $\alpha$ and $\beta$.

\begin{figure}[ht]
    \centering
    \includegraphics[width=0.85\linewidth]{img/test_acc_alpha_beta.png}
    \caption{The test accuracy of an MLP trained on the \textit{Water Potability} dataset, evaluated while varying the weight of our counterfactual regularizer ($\alpha$) for different values of $\beta$.}
    \label{fig:test_alpha_beta}
\end{figure}

We observe that, for a fixed $\beta$, increasing the weight of our counterfactual regularizer ($\alpha$) can slightly improve test accuracy until a sudden drop is noticed for $\alpha > 0.1$.
This behavior was expected, as the impact of our penalty, like any regularization term, can be disruptive if not properly controlled.

Moreover, this finding further demonstrates that our regularization method, CF-Reg, is inherently data-driven. Therefore, it requires specific fine-tuning based on the combination of the model and dataset at hand.
\section{Conclusion}
In this work, we propose a simple yet effective approach, called SMILE, for graph few-shot learning with fewer tasks. Specifically, we introduce a novel dual-level mixup strategy, including within-task and across-task mixup, for enriching the diversity of nodes within each task and the diversity of tasks. Also, we incorporate the degree-based prior information to learn expressive node embeddings. Theoretically, we prove that SMILE effectively enhances the model's generalization performance. Empirically, we conduct extensive experiments on multiple benchmarks and the results suggest that SMILE significantly outperforms other baselines, including both in-domain and cross-domain few-shot settings.

%%
%% The acknowledgments section is defined using the "acks" environment
%% (and NOT an unnumbered section). This ensures the proper
%% identification of the section in the article metadata, and the
%% consistent spelling of the heading.

%%
%% The next two lines define the bibliography style to be used, and
%% the bibliography file.
\bibliographystyle{ACM-Reference-Format}
\bibliography{references,ryan-cites-long,s3}


%%
%% If your work has an appendix, this is the place to put it.
% \appendix

% \section{Research Methods}

\end{document}
\endinput
%%
%% End of file `sample-sigconf-authordraft.tex'.
