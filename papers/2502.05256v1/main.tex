%%
%% This is file `sample-sigconf-authordraft.tex',
%% generated with the docstrip utility.
%%
%% The original source files were:
%%
%% samples.dtx  (with options: `all,proceedings,bibtex,authordraft')
%% 
%% IMPORTANT NOTICE:
%% 
%% For the copyright see the source file.
%% 
%% Any modified versions of this file must be renamed
%% with new filenames distinct from sample-sigconf-authordraft.tex.
%% 
%% For distribution of the original source see the terms
%% for copying and modification in the file samples.dtx.
%% 
%% This generated file may be distributed as long as the
%% original source files, as listed above, are part of the
%% same distribution. (The sources need not necessarily be
%% in the same archive or directory.)
%%
%%
%% Commands for TeXCount
%TC:macro \cite [option:text,text]
%TC:macro \citep [option:text,text]
%TC:macro \citet [option:text,text]
%TC:envir table 0 1
%TC:envir table* 0 1
%TC:envir tabular [ignore] word
%TC:envir displaymath 0 word
%TC:envir math 0 word
%TC:envir comment 0 0
%%
%%
%% The first command in your LaTeX source must be the \documentclass
%% command.
%%
%% For submission and review of your manuscript please change the
%% command to \documentclass[manuscript, screen, review]{acmart}.
%%
%% When submitting camera ready or to TAPS, please change the command
%% to \documentclass[sigconf]{acmart} or whichever template is reuired
%% for your publication.
%%
%%
% \documentclass[sigconf,authordraft]{acmart}
\documentclass[nonacm,sigconf]{acmart}
% \documentclass[manuscript,screen,review,anonymous]{acmart}

\usepackage{xspace}
\usepackage{pifont}
\usepackage[inline]{enumitem}
\usepackage{cleveref}
% \usepackage{subfig}
\usepackage{subcaption}

\DeclareMathOperator*{\argmax}{arg\,max}
\DeclareMathOperator*{\argmin}{arg\,min}
\newcommand{\ourmethod}{\texttt{BOUTT}}
\newcommand{\bx}{\mathbf{x}}
\newcommand{\bc}{\mathbf{c}}
\newcommand{\bz}{\mathbf{z}}
\newcommand{\bX}{\mathbf{X}}
\newcommand{\bZ}{\mathbf{Z}}
\newcommand{\bV}{\mathbf{V}}
\newcommand{\bfn}{\mathbf{f}}
\newcommand{\bu}{\mathbf{u}}
\newcommand{\by}{\mathbf{y}}
\newcommand{\calX}{\mathcal{X}}
\newcommand{\calR}{\mathcal{R}}
\newcommand{\calD}{\mathcal{D}}
\newcommand{\calZ}{\mathcal{Z}}
\newcommand{\calT}{\mathcal{T}}
\newcommand{\calGP}{\mathcal{GP}}



%%
%% \BibTeX command to typeset BibTeX logo in the docs
\AtBeginDocument{%
  \providecommand\BibTeX{{%
    Bib\TeX}}}

%% Rights management information.  This information is sent to you
%% when you complete the rights form.  These commands have SAMPLE
%% values in them; it is your responsibility as an author to replace
%% the commands and values with those provided to you when you
%% complete the rights form.
\setcopyright{acmlicensed}
\copyrightyear{2025}
\acmYear{2025}
\acmDOI{XXXXXXX.XXXXXXX}

%% These commands are for a PROCEEDINGS abstract or paper.
\acmConference[SIGMOD '25]{Proceedings of the 2025 International Conference on Management
of Data}{June 22--27,
  2025}{Berlin, Germany}
%%
%%  Uncomment \acmBooktitle if the title of the proceedings is different
%%  from ``Proceedings of ...''!
%%
%%\acmBooktitle{Woodstock '18: ACM Symposium on Neural Gaze Detection,
%%  June 03--05, 2018, Woodstock, NY}
\acmISBN{978-TBD}


%%
%% Submission ID.
%% Use this when submitting an article to a sponsored event. You'll
%% receive a unique submission ID from the organizers
%% of the event, and this ID should be used as the parameter to this command.
%%\acmSubmissionID{123-A56-BU3}

%%
%% For managing citations, it is recommended to use bibliography
%% files in BibTeX format.
%%
%% You can then either use BibTeX with the ACM-Reference-Format style,
%% or BibLaTeX with the acmnumeric or acmauthoryear sytles, that include
%% support for advanced citation of software artefact from the
%% biblatex-software package, also separately available on CTAN.
%%
%% Look at the sample-*-biblatex.tex files for templates showcasing
%% the biblatex styles.
%%

%%
%% The majority of ACM publications use numbered citations and
%% references.  The command \citestyle{authoryear} switches to the
%% "author year" style.
%%
%% If you are preparing content for an event
%% sponsored by ACM SIGGRAPH, you must use the "author year" style of
%% citations and references.
%% Uncommenting
%% the next command will enable that style.
%%\citestyle{acmauthoryear}


\newcommand{\sparagraph}[1]{\vspace{1mm}\noindent \textbf{#1}} 
\newcommand{\sysname}[0]{BayesQO\xspace}
\newcommand{\bo}[0]{BO\xspace}
\newcommand{\random}[0]{Random\xspace}

\newcommand{\circleOne}{\ding{182}\xspace}
\newcommand{\circleTwo}{\ding{183}\xspace}
\newcommand{\circleThree}{\ding{184}\xspace}
\newcommand{\circleFour}{\ding{185}\xspace}
\newcommand{\circleFive}{\ding{186}\xspace}
\newcommand{\circleSix}{\ding{187}\xspace}
\newcommand{\circleSeven}{\ding{188}\xspace}
\newcommand{\circleEight}{\ding{189}\xspace}
\newcommand{\circleNine}{\ding{190}\xspace}

\newcommand{\jrg}[1]{\textcolor{red}{[Jake: #1]}}
\newcommand{\nmaus}[1]{\textcolor{olive}{[Natalie: #1]}}
\newcommand{\haydn}[1]{\textcolor{magenta}{[Haydn: #1]}}
\newcommand{\jeff}[1]{\textcolor{blue}{[Jeff: #1]}}
\newcommand{\ryan}[1]{\textcolor{cyan}{[Ryan: #1]}}
\newcommand{\yimeng}[1]{\textcolor{green}{[Yimeng: #1]}}
\newcommand{\factcheck}[1]{\textcolor{orange}{[Fact Check: #1]}}

% \newcommand{\edit}[1]{\textcolor{blue}{#1}}
\newcommand{\edit}[1]{#1}



%%
%% end of the preamble, start of the body of the document source.
\begin{document}
\setlength{\textfloatsep}{0pt}
% \setlength{\floatsep}{4pt}
% \setlength{\intextsep}{0pt}

% \setlength{\abovedisplayskip}{4pt}
% \setlength{\belowdisplayskip}{4pt}
% \setlength{\abovedisplayshortskip}{4pt}
% \setlength{\belowdisplayshortskip}{4pt}

%%
%% The "title" command has an optional parameter,
%% allowing the author to define a "short title" to be used in page headers.
\title{Learned Offline Query Planning via Bayesian Optimization}

%%
%% The "author" command and its associated commands are used to define
%% the authors and their affiliations.
%% Of note is the shared affiliation of the first two authors, and the
%% "authornote" and "authornotemark" commands
%% used to denote shared contribution to the research.
\author{Jeffrey Tao}
\orcid{0000-0001-6407-1316}
\affiliation{
  \institution{University of Pennsylvania}
  %\city{Philadelphia}
  %\state{PA}
  %\country{USA}
}
\email{jefftao@seas.upenn.edu}

\author{Natalie Maus}
\orcid{0000-0002-6616-8506}
\affiliation{
  \institution{University of Pennsylvania}
  %\city{Philadelphia}
  %\state{PA}
  %\country{USA}
}
\email{nmaus@seas.upenn.edu}

\author{Haydn Jones}
\orcid{0000-0002-1006-4126}
\affiliation{
  \institution{University of Pennsylvania}
  %\city{Philadelphia}
  %\state{PA}
  %\country{USA}
}
\email{haydnj@seas.upenn.edu}

\author{Yimeng Zeng}
\orcid{0009-0001-9676-0893}
\affiliation{
  \institution{University of Pennsylvania}
  %\city{Philadelphia}
  %\state{PA}
  %\country{USA}
}
\email{yimengz@seas.upenn.edu}

\author{Jacob R. Gardner}
\orcid{0000-0003-1897-8384}
\affiliation{
  \institution{University of Pennsylvania}
  %\city{Philadelphia}
  %\state{PA}
  %\country{USA}
}
\email{jacobrg@seas.upenn.edu}

\author{Ryan Marcus}
\orcid{0000-0002-1279-1124}
\affiliation{
  \institution{University of Pennsylvania}
  %\city{Philadelphia}
  %\state{PA}
  %\country{USA}
}
\email{rcmarcus@seas.upenn.edu}


%%
%% By default, the full list of authors will be used in the page
%% headers. Often, this list is too long, and will overlap
%% other information printed in the page headers. This command allows
%% the author to define a more concise list
%% of authors' names for this purpose.
\renewcommand{\shortauthors}{Tao et al.}

%%
%% The abstract is a short summary of the work to be presented in the
%% article.
\begin{abstract}
Analytics database workloads often contain queries that are executed repeatedly. Existing optimization techniques generally prioritize keeping optimization cost low, normally well below the time it takes to execute a single instance of a query. If a given query is going to be executed thousands of times, could it be worth investing significantly more optimization time? In contrast to traditional online query optimizers, we propose an offline query optimizer that searches a wide variety of plans and incorporates query execution as a primitive. Our offline query optimizer combines variational auto-encoders with Bayesian optimization to find optimized plans for a given query. We compare our technique to the optimal plans possible with PostgreSQL and recent RL-based systems over several datasets, and show that our technique finds faster query plans.

%% \noindent\small{Artifacts: \url{https://anonymous.4open.science/r/bayesqo-artifacts-B3FB}}
\end{abstract}

%%
%% The code below is generated by the tool at http://dl.acm.org/ccs.cfm.
%% Please copy and paste the code instead of the example below.
%%
\begin{CCSXML}
\end{CCSXML}


%%
%% Keywords. The author(s) should pick words that accurately describe
%% the work being presented. Separate the keywords with commas.
\keywords{}
%% A "teaser" image appears between the author and affiliation
%% information and the body of the document, and typically spans the
%% page.

%\received{20 February 2007}
%\received[revised]{12 March 2009}
%\received[accepted]{5 June 2009}

%%
%% This command processes the author and affiliation and title
%% information and builds the first part of the formatted document.
\maketitle

\section{Introduction}
\label{sec:intro}

\begin{figure*}[tb]
    \centering
    \includegraphics[width=0.848\linewidth]{figs/circuitnn.pdf} 
    \caption{Illustration of differentiable CircuitNN. CircuitNN is designed based on differentiable NAND gates. After DAS is guided by PI and PO pairs of the truth table, CircuitNN can get the precise circuit architecture logic equivalent to the truth table.}
    \label{fig:circuitnn}
\end{figure*}

% 1. Describe the importance of logic synthesis
% 2. Existing Problems
% (a) Neural Architecture Search: Unstable, Predefined Setting, etc.
% (b) Circuit Generation: Probabilistic Model, Logic Equivalence

With the rapid advancement of technology, the scale of integrated circuits (ICs) has expanded exponentially. 
This expansion has introduced significant challenges in chip manufacturing, particularly concerning power and area metrics.
A primary objective in IC design is achieving the same circuit function with fewer transistors, thereby reducing power usage and area occupancy.

Logic synthesis~\cite{hachtel2005logicsynth}, a critical step in electronic design automation (EDA), transforms behavioral-level circuit designs into optimized gate-level circuits, ultimately yielding the final IC layout. 
The primary goal of logic synthesis is to identify the physical implementation with the fewest gates for a given circuit function. 
This task constitutes a challenging NP-hard combinatorial optimization problem. 
Current logic synthesis tools~\cite{brayton2010abc, wolf2013yosys} rely on human-designed heuristics, often leading to sub-optimal outcomes.

Differentiable architecture search (DAS) techniques~\cite{liu2018darts, chu2020darts} offer novel perspectives on addressing challenges in this problem.
Circuit functions can be represented through truth tables, which map binary inputs to their corresponding outputs. 
Truth tables provide a precise representation of input-output relationships, ensuring the design of functionally equivalent circuits.
Inspired by this, researchers~\cite{deepmind2024ai4sys, wang2024tnet} have begun exploring the application of DAS to synthesize circuits directly from truth tables.
Specifically, \citet{deepmind2024ai4sys} proposed CircuitNN, a framework that learns differentiable connection structures with logic gates, enabling the automatic generation of logic circuits from truth tables.
This approach significantly reduces the complexity of traditional circuit generation. 
Building on this, \citet{wang2024tnet} introduced T-Net, a triangle-shaped variant of CircuitNN, incorporating regularization techniques to enhance the efficiency of DAS.

Despite these advancements, several challenges remain. 
The computational complexity of DAS grows quadratically with the number of gates, posing scalability issues.
Although triangle-shaped architecture~\cite{wang2024tnet} partially mitigates this problem, redundancy persists. 
%Additionally, DAS is susceptible to converging to local optima, limiting the ability to search architectures that satisfy the given truth tables~\cite{liu2018darts}. 
%Furthermore, hyperparameters (network depth and layer width) require extensive searches, introducing complexity and prolonging the synthesis process. 
Additionally, DAS is susceptible to converging to local optima~\cite{liu2018darts} and hyperparameters (network depth and layer width) require extensive searches. 
The challenges arise from the vast search space in DAS. 
% Even with predefined settings for CircuitNN, finding a configuration that meets the truth table requires extensive trial and error during the DAS process. 
Intuitively, limiting the search space through predefined parameters (network depth, gates per layer, and connection probabilities) can significantly reduce the complexity.

Recent advances~\cite{openai2023gpt4, abramson2024alphafold3, esser2024sd3, li2024mar} in conditional generative models have demonstrated remarkable performance across language, vision, and graph generation tasks. 
Motivated by these developments, we propose a novel approach to circuit generation that generates preliminary circuit structures to guide DAS in generating refined circuits matching specified truth tables. 
Firstly, we introduce CircuitVQ, a tokenizer with a discrete codebook for circuit tokenization. 
Built upon our Circuit AutoEncoder framework~\cite{hou2022graphmae,li2023maskgae,wu2025mgvga}, CircuitVQ is trained through a circuit reconstruction task. 
Specifically, the CircuitVQ encoder encodes input circuits into discrete tokens using a learnable codebook, while the decoder reconstructs the circuit adjacency matrix based on these tokens.
Subsequently, the CircuitVQ encoder serves as a circuit tokenizer for CircuitAR pretraining, which employs a masked autoregressive modeling paradigm~\cite{chang2022maskgit, li2023mage}. 
In this process, the discrete codes function as supervision signals. 
After training, CircuitAR can generate discrete tokens progressively, which can be decoded into initial circuit structures by the decoder of the CircuitVQ. 
These prior insights can guide DAS in producing refined circuits that match the target truth tables precisely.

Our key contributions can be summarized as follows:
\begin{itemize}
\item We introduce CircuitVQ, a circuit tokenizer that facilitates graph autoregressive modeling for circuit generation, based on our Circuit AutoEncoder framework;
\item Develop CircuitAR, a model trained using masked autoregressive modeling, which generates initial circuit structures conditioned on given truth tables;
\item Propose a refinement framework that integrates differentiable architecture search to produce functionally equivalent circuits guided by target truth tables;
\item Comprehensive experiments demonstrating the scalability and capability emergence of our CircuitAR and the superior performance of the proposed circuit generation approach.
\end{itemize}

% Motivation
% (a) Diffusion (Vision, Graph), Autoregressive (Language, Vision)
% (b) Circuit Generation for Predefined Setting
% (c) Neural Architecture Search for Strict Logic Equivalence

% Contribution
% (a) Circuit Tokenizer (new transformer arch, training strategy)
% (b) CircuitAR (train and gen strategies, post-ar strategy)
% (c) Extensive Evaluation including BitD (Bit Distance) for Scalability


\section{Related Work} \label{sec:related}

% \textbf{Adversarial Attack}
\textbf{Attacks on SLAM.} 
%With the rise of machine learning, 
The robustness of computer vision systems is being actively investigated. With the emergence of adversarial images in the digital domain by adding optimized noise directly to images~\cite{szegedy2013intriguing,carlini2017towards}, researchers find that such attacks also exist physically in the real world \cite{eykholt2018robust,song2018physical,zhao2019seeing}. To fill the gap between attacks in the digital and physical worlds, recent studies have demonstrated that attacks on real-world computer vision systems are practical \cite{eykholt2018robust,li2019adversarial,man2020ghostimage,sharif2016accessorize,zhao2019seeing,zhou2018invisible}. However, attacks on traditional computer vision methods such as SLAM are relatively less explored. \cite{yoshida2022adversarial} proposes an attack against the scan matching algorithm in LiDAR-based SLAM, while most SLAMs in AR/VR devices rely on different sensors like RGB/depth cameras and IMUs. \cite{ikram2022perceptual} and \cite{chen2024adversary} mislead visual SLAM by poisoning the images with special patterns, and \cite{wang2021can} causes the camera to fail using infrared light. In our work, we demonstrate attacks on Visual-Inertial SLAM (VI-SLAM) by perturbing the IMU readings, rather than cameras, and showing its impact on XR user experience. 

\textbf{Acoustic Injection Attacks.} Among various physical attacks, acoustic injection attacks are attractive due to their low cost. Son~\etal~\cite{son2015rocking} were the first to introduce acoustic attacks on MEMS gyroscopes, demonstrating how these attacks could lead to sensor denial-of-service and result in drone crashes. WALNUT~\cite{trippel2017walnut} expanded on this by developing output biasing and control attacks that enable precise manipulation of MEMS accelerometer outputs using modulated sound waves. Wang et al.~\cite{wang2017sonic} demonstrated a sonic gun, showcasing the vulnerability of various smart devices (\eg drones and self-balancing vehicles) to acoustic attacks. Tu et al. \cite{tu2018injected} designed side-swing and switching attacks to alter the outputs of MEMS gyroscopes and accelerometers. Furthermore, Ji et al. \cite{ji2021poltergeist} fool the object detectors by applying acoustic attack to the image stabilizers commonly used in modern cameras. However, none of the existing works study the relationship between the acoustic injections and SLAM outputs on recent XR devices. 

% \zijian{Do we need one session about security in AR/VR?}
% \yicheng{TODO}
%\jiasi{cite the AIVR paper (UMass Amherst?) paper is we have not already. They add IMU perturbation but w/o SLAM, iirc} \yicheng{Cited}

\textbf{XR Security and Privacy.} 
%Security and privacy concerns in XR systems have gained significant attention. 
For single-user XR systems, researchers have demonstrated various side-channel attacks to extract sensitive information (\eg keystrokes) through video feeds~\cite{ling2019know}, head movements~\cite{nair2023unique, slocum2023going}, architectural hints~\cite{zhang2023its,shang2020arspy}, power usage~\cite{li2024dangers}, and EM side-channel leakages~\cite{al2021vr}. In multi-user XR systems, Su et al.~\cite{su2024remote} use avatar motion data to infer keystrokes in shared VR environments. Slocum et al.~\cite{slocum2024doesn} reveal vulnerabilities in the shared state frameworks of multi-user AR. Similarly, Lebeck et al.~\cite{lebeck2017securing} highlight risks like deceptive virtual objects and emphasize access control for managing shared physical and virtual spaces. Ruth et al.~\cite{ruth2019secure} further propose a secure multi-user AR framework focusing on content sharing and permissions.
Chandio et al.~\cite{chandio2024stealthy} %introduced a multi-modal spatiotemporal attack that 
simultaneously manipulated visual and inertial sensors to disrupt XR pose estimation. However, their study evaluated the attack using offline datasets and assumed the attacker's capability to manipulate IMU data streams through acoustic means, without real experiments. Ours is the first to demonstrate acoustic injection attacks on recent XR devices, like the Hololens 2, in the real world.
 


\vspace{-1mm}
\section{Zak-OTFS System Model}
\label{sec2}
Figure \ref{fig1} shows the block diagram of a Zak-OTFS transceiver.
\begin{figure*}
\centering    \includegraphics[width=0.95\linewidth]{Figures/continuous_zak_otfs_bd.eps}
\caption{Block diagram of Zak-OTFS transceiver.}
\label{fig1}      
\vspace{-4mm}
\end{figure*}
In Zak-OTFS, a pulse in the DD domain is the basic information carrier. A DD pulse is a quasi-periodic localized function defined by a delay period $\tau_{\mathrm{p}}$ and a Doppler period $\nu_{\mathrm{p}}=\frac{1}{\tau_{\mathrm{p}}}$. The fundamental period in the DD domain is defined as 
$\mathcal{D}_{0}= \{(\tau,\nu): 0\leq\tau<\tau_{\mathrm p}, 0\leq\nu<\nu_{\mathrm p}\}$,
where $\tau$ and $\nu$ represent the delay and Doppler variables, respectively. The fundamental period is discretized into $M$ bins on the delay axis and $N$ bins on the Doppler axis, as 
$\big\{(k\frac{\tau_{{\mathrm p}}}{M},l\frac{\nu_{{\mathrm p}}}{N}) | k=0,\ldots,M-1,l=0,\ldots,N-1\big\}$. The time domain Zak-OTFS frame is limited to a time duration $T=N\tau_{\mathrm p}$ and a bandwidth $B=M\nu_{\mathrm p}$. In each frame, $MN$ information symbols drawn from a modulation alphabet ${\mathbb A}$, $x[k,l]\in {\mathbb A}$, $k=0,\ldots,M-1$, $l=0,\ldots,N-1$, are multiplexed in the DD domain. The information symbol $x[k,l]$ is carried by DD domain pulse $x_{\mathrm{dd}}[k,l]$, which is a quasi-periodic function with period $M$ along the delay axis and period $N$ along the Doppler axis, i.e., for any $n,m\in\mathbb{Z}$,  
\begin{equation}
x_{\mathrm{dd}}[k+nM,l+mN]=x[k,l]e^{j2\pi n\frac{l}{N}}.
\end{equation}
These discrete DD domain signals $x_{\mathrm{dd}}[k,l]$s are supported on the information lattice 
$\Lambda_{\mathrm{dd}}=
\big\{\big(k\frac{\tau_{\mathrm p}}{M},l\frac{\nu_{\mathrm p}}{N}\big) | k,l\in \mathbb{Z}\big\}$.
The continuous DD domain information signal is given by
\vspace{-1mm}
\begin{equation}
x_{\mathrm{dd}}(\tau,\nu)=\sum_{k,l\in \mathbb{Z}} x_{\mathrm{dd}}[k,l] \delta\Big(\tau-\frac{k\tau_{\mathrm p}}{M}\Big)\delta\Big(\nu-\frac{l\nu_{\mathrm p}}{N}\Big),
\end{equation}
where $\delta(.)$ denotes the Dirac-delta impulse function. For any $n,m\in \mathbb{Z}$, we have
$x_{\mathrm{dd}}(\tau+n\tau_{\mathrm{p}},\nu+m\nu_{\mathrm{p}})=e^{j2\pi n\nu \tau_{\mathrm{p}}}x_{\mathrm{dd}}(\tau,\nu)$,
so that $x_{\mathrm{dd}}(\tau,\nu)$ is periodic with period $\nu_{\mathrm p}$ along the Doppler axis and quasi-periodic with period $\tau_{\mathrm p}$ along the delay axis.

The DD domain transmit signal $x_{\mathrm{dd}}^{w_{\mathrm{tx}}}(\tau,\nu)$ is given by the twisted convolution of the transmit pulse shaping filter $w_{\mathrm{tx}}(\tau,\nu)$ with $x_{\mathrm{dd}}(\tau,\nu)$ as $x_{\mathrm{dd}}^{w_{\mathrm{tx}}}(\tau,\nu) = w_{\mathrm{tx}}(\tau,\nu)*_{\sigma}x_{\mathrm{dd}}(\tau,\nu)$,
where $*_{\sigma}$ denotes the twisted convolution\footnote{Twisted convolution of two DD functions $a(\tau,\nu)$ and $b(\tau,\nu)$ is defined as 
$a(\tau,\nu) \ast_\sigma b(\tau,\nu) \overset{\Delta}{=} \iint a(\tau', \nu') b(\tau-\tau',\nu-\nu')e^{j2\pi\nu'(\tau-\tau')}d\tau'  d\nu'$.}. The transmitted time domain (TD) signal $s_{\mathrm{td}}(t)$ is the TD realization of $x_{\mathrm{dd}}^{w_{\mathrm{tx}}}(\tau,\nu)$, given by
$s_{\mathrm{td}}(t)=Z_{t}^{-1}\left(x_{\mathrm{dd}}^{w_{\mathrm{tx}}}(\tau,\nu)\right)$, where $Z_{t}^{-1}$ denotes the inverse time-Zak transform operation\footnote{Inverse time-Zak transform of a DD function $a(\tau,\nu)$ is defined as $Z_{t}^{-1}(a(\tau,\nu)) \overset{\Delta}{=} \sqrt{\tau_{\mathrm p}} \int_0^{\nu_{\mathrm p}} a(t,\nu) d\nu$.}. The transmit pulse shaping filter $w_{\mathrm{tx}}(\tau,\nu)$ 
limits the time and bandwidth of the transmitted signal $s_{\mathrm{td}}(t)$. The transmit signal $s_{\mathrm{td}}(t)$ passes through a doubly-selective channel to give the output signal $r_{\mathrm{td}}(t)$. The DD domain impulse response of the physical channel $h_{\mathrm{phy}}(\tau,\nu)$ is given by
\begin{equation}
h_{\mathrm{phy}}(\tau,\nu)=\sum_{i=1}^{P}h_{i}\delta(\tau-\tau_{i})\delta(\nu-\nu_{i}),
\end{equation}
where $P$ denotes the number of DD paths, and the $i$th path has gain $h_{i}$, delay shift $\tau_{i}$, and Doppler shift $\nu_{i}$. 

The received TD signal $y(t)$ at the receiver is given by $y(t)=r_{\mathrm{td}}(t)+n(t)$,
where $n(t)$ is AWGN with variance $N_{0}$, i.e., $\mathbb{E}[n(t)n(t+t')]=N_{0}\delta(t')$. The TD signal $y(t)$ is converted to the corresponding DD domain signal $y_{\mathrm{dd}}(\tau,\nu)$ by applying Zak transform\footnote{Zak transform of a continuous TD signal $a(t)$ is defined as
$Z_t\left(a(t)\right) \overset{\Delta}{=} \sqrt{\tau_p} \sum_{k \in \mathbb{Z}} a(\tau + k \tau_{\mathrm p}) e^{-j2\pi\nu k\tau_{\mathrm p}}$.}, i.e.,
\begin{eqnarray}
\hspace{-6mm}
y_{\mathrm{dd}}(\tau,\nu) = Z_{t}(y(t)) 
= r_{\mathrm{dd}}(\tau,\nu)+n_{\mathrm{dd}}(\tau,\nu),
\end{eqnarray}
where $r_{\mathrm{dd}}(\tau,\nu)=h_{\mathrm{phy}}(\tau,\nu)*_{\sigma}w_{\mathrm{tx}}(\tau,\nu)*_{\sigma}x_{\mathrm{dd}}(\tau,\nu)$ is the Zak transform of $r_{\mathrm{td}}(t)$, given by the twisted convolution cascade of $x_{\mathrm{dd}}(\tau,\nu)$, $w_{\mathrm{tx}}(\tau,\nu)$, and $h_{\mathrm{phy}}(\tau,\nu)$,  and $n_{\mathrm{dd}}(\tau,\nu)$ is the Zak transform of $n(t)$. The receiver filter $w_{\mathrm{rx}}(\tau,\nu)$ acts on $y_{\mathrm{dd}}(\tau,\nu)$ through twisted convolution to give the output 
\begin{eqnarray}
\hspace{-4mm} 
y_{\mathrm{dd}}^{w_{\mathrm{rx}}}(\tau,\nu) & \hspace{-2mm} = & \hspace{-2mm} w_{\mathrm{rx}}(\tau,\nu)*_{\sigma}y_{\mathrm{dd}}(\tau,\nu) \nonumber \\ 
& \hspace{-22mm} = & \hspace{-12mm} \underbrace{w_{\mathrm{rx}}(\tau,\nu)*_{\sigma}h_{\mathrm{phy}}(\tau,\nu)*_{\sigma}w_{\mathrm{tx}}(\tau,\nu)}_{\overset{\Delta}{=} \ h_{\mathrm{eff}}(\tau,\nu)}*_{\sigma}x_{\mathrm{dd}}(\tau,\nu) \nonumber \\ 
&\hspace{-22mm} & \hspace{-12mm} + \ \underbrace{w_{\mathrm{rx}}(\tau,\nu)*_{\sigma}n_{\mathrm{dd}}(\tau,\nu)}_{\overset{\Delta}{=} \ n_{\mathrm{dd}}^{w_{\mathrm{rx}}}(\tau,\nu)}, 
\label{cont1}
\end{eqnarray}
where $h_{\mathrm{eff}}(\tau,\nu)$ denotes the effective channel consisting of the twisted convolution cascade of $w_{\mathrm{tx}}(\tau,\nu),\ h_{\mathrm{phy}}(\tau,\nu)$, and $w_{\mathrm{rx}}(\tau,\nu)$, and $n_{\mathrm{dd}}^{w_{\mathrm{rx}}}(\tau,\nu)$ denotes the noise filtered through the Rx filter. The DD signal $y_{\mathrm{dd}}^{w_{\mathrm{rx}}}(\tau,\nu)$ is sampled on the information lattice, resulting in the discrete quasi-periodic DD domain received signal $y_{\mathrm{dd}}[k,l]$ as
\vspace{0mm}
\begin{equation}
y_{\mathrm{dd}}[k,l]=y_{\mathrm{dd}}^{w_{\mathrm{rx}}}\left(\tau=\frac{k\tau_{\mathrm p}}{M},\nu=\frac{l\nu_{\mathrm p}}{N}\right), \ \ k,l\in\mathbb{Z},
\end{equation} 
which is given by
$y_{\mathrm{dd}}[k,l]=h_{\mathrm{eff}}[k,l]*_{\sigma\text{d}}x_{\mathrm{dd}}[k,l]+n_{\mathrm{dd}}[k,l]$,
where $*_{\sigma\text{d}}$ is twisted convolution in discrete DD domain, i.e., 
$h_{\mathrm{eff}}[k,l]*_{\sigma\text{d}}x_{\mathrm{dd}}[k,l] = \sum_{k',l'\in\mathbb{Z}}h_{\mathrm{eff}}[k-k',l-l']x_{\mathrm{dd}}[k',l'] e^{j2\pi\frac{k'(l-l')}{MN}}$, where the effective channel filter $h_{\mathrm{eff}}[k,l]$ and filtered noise samples $n_{\mathrm{dd}}[k,l]$ are given by
\begin{align}
h_{\text{eff}}[k,l]=h_{\text{eff}}\left(\tau=\frac{k\tau_{p}}{M},\nu=\frac{l\nu_{p}}{N}\right), \label{discr2} \\ 
n_{\text{dd}}[k,l]=n_{\text{dd}}^{w_{\mathrm{rx}}}\left(\tau=\frac{k\tau_{p}}{M},\nu=\frac{l\nu_{p}}{N}\right).
\label{discr3}
\end{align}
Owing to the quasi-periodicity in the DD domain, it is sufficient to consider the received samples $y_{\mathrm{dd}}[k,l]$ within the fundamental period $\mathcal{D}_0$. Writing the $y_{\mathrm{dd}}[k,l]$ samples as a vector, the received signal model can be written in matrix-vector form as \cite{zak_otfs1},\cite{zak_otfs2}
\begin{equation}
\mathbf{y}=\mathbf{H_\text{eff}x}+\mathbf{n},
\label{sys_mod}
\end{equation}
where $\mathbf{x,y,n} \in\mathbb{C}^{MN\times 1}$, such that their $(kN+l+1)$th entries are given by $x_{kN+l+1}=x_{\mathrm{dd}}[k,l]$, $y_{kN+l+1}=y_{\mathrm{dd}}[k,l]$, $n_{kN+l+1}=n_{\mathrm{dd}}[k,l]$, and $\mathbf{H}_{\text{eff}}\in\mathbb{C}^{MN\times MN}$ is the effective channel matrix such that
\vspace{-1mm}
\begin{eqnarray}
\mathbf{H}_\text{eff}[k'N+l'+1,kN+l+1] & \hspace{-2mm} = & \hspace{-2mm} \sum_{m,n\in\mathbb{Z}}h_{\mathrm{eff}}[k'-k-nM, \nonumber \\
& \hspace{-45mm} & \hspace{-35mm} l'-l-mN]e^{j2\pi nl/N}e^{j2\pi\frac{(l'-l-mN)(k+nM)}{MN}},
\label{eqn_channel_matrix}
\vspace{-4mm}
\end{eqnarray}
where $k',k=0,\ldots,M-1$, $l',l=0,\ldots,N-1$. 

\vspace{-2mm}
\subsection{DD pulse shaping filters}
In the absence of pulse shaping, i.e., $w_\text{tx}(\tau,\nu)=\delta(\tau,\nu)$, the transmit signal has infinite time duration and bandwidth. Pulse shaping limits the time and bandwidth of transmission. We consider transmit DD pulse shaping filters of the form $w_\text{tx}(\tau,\nu)=w_1(\tau)w_2(\nu)$ \cite{zak_otfs2},\cite{zak_otfs6}. The time duration $T'$ of each frame is approximately related to the spread of $w_2({\nu})$ along the Doppler axis as $\frac{1}{T'}$. Likewise, the bandwidth $B'$ is approximately related to the spread of $w_1(\tau)$ along the delay axis as $\frac{1}{B'}$. That is, a larger bandwidth and time duration implies a smaller DD spread of $w_\text{tx}(\tau,\nu)$, and hence a smaller contribution to the spread of $h_\text{eff}(\tau,\nu)$. Sinc, RRC, and Gaussian filters have been considered in the Zak-OTFS literature and are described below. 

{\em Sinc filter:}
For sinc filter, $w_1({\tau})$ and $w_2({\nu})$ are given by
$w_1({\tau}) = \sqrt{B}\text{sinc}(B\tau)$ and $w_2({\nu}) = \sqrt{T}\text{sinc}(T\nu)$,
so that 
\begin{equation}
w_\text{tx}(\tau,\nu)
=\underbrace{\sqrt{B}\text{sinc}(B\tau)}_{w_1(\tau)} \underbrace{\sqrt{T}\text{sinc}(T\nu)}_{w_2(\nu)}. 
\label{eq:sinc1}
\end{equation}
For sinc filter, the frame duration $T'=T$ and frame bandwidth $B'=B$ (i.e., there is no time or bandwidth expansion), resulting in a spectral efficiency of $\frac{BT}{B'T'}=1$ symbol/dimension. 

{\em RRC filter:}
For RRC filter, $w_\text{tx}(\tau,\nu)$ is given by
\begin{equation}
w_\text{tx}(\tau,\nu) = \underbrace{\sqrt{B} \ \text{rrc}_{\beta_\tau}(B\tau)}_{w_1(\tau)} \ \underbrace{\sqrt{T} \ \text{rrc}_{\beta_\nu}(T\nu)}_{w_2(\nu)},
\label{eq:rrc1}
\end{equation}
where $0\leq \beta_{\tau},\beta_\nu \leq 1$ and
\begin{equation}
\text{rrc}_{\beta}(x)=\frac{\sin \left(\pi x(1-\beta) \right)+4\beta x\cos \left(\pi x(1+\beta) \right)}{\pi x \left(1-(4\beta x)^2 \right)}. 
\end{equation}
It can be seen that the choice of $\beta_\tau=\beta_\nu=0$ in the RRC filter (\ref{eq:rrc1}) specializes to the sinc filter. Also, for $\beta_\nu>0$ and $\beta_\tau>0$, there is time and bandwidth expansion such that  $T'=T(1+\beta_{\nu})$ and $B'=B(1+\beta_{\tau})$, resulting in a spectral efficiency of $\frac{BT}{B'T'}<1$ symbol/dimension. 

{\em Gaussian filter:}
For Gaussian filter, $w_\text{tx}(\tau,\nu)$ is given by \cite{zak_otfs7}
\begin{equation}
\hspace{-2mm}
w_\text{tx}(\tau,\nu) \hspace{-0.5mm} = \hspace{-0.5mm} \underbrace{\left(\frac{2\alpha_{\tau}B^2}{\pi}\right)^{\frac{1}{4}}e^{-\alpha_{\tau}B^{2}\tau^{2}}}_{w_1(\tau)} \ \underbrace{\left(\frac{2\alpha_{\nu}T^2}{\pi}\right)^{\frac{1}{4}}e^{-\alpha_{\nu}T^{2}\nu^{2}}}_{w_2(\nu)}\hspace{-1mm}. \hspace{-0mm}
\label{eq:gauss1}
\end{equation}
The Gaussian pulse can be configured by adjusting the parameters $\alpha_{\tau}$ and $\alpha_{\nu}$. Because of the infinite support in Gaussian pulse, a time duration $T'$ where 99$\%$ of the frame energy is localized in the time domain and a bandwidth $B'$ where 99$\%$ of the frame 
energy is localized in the frequency domain are considered \cite{zak_otfs7}. No time and bandwidth expansion (i.e., $T'=T$, $B'=B$) in the Gaussian pulse corresponds to setting 
$\alpha_{\tau}=\alpha_{\nu}=1.584$. 

\section{Bayesian Optimization for Query Plans}
\label{sec:technique}

Given our goal to perform offline optimization of queries while minimizing the cost of executing extra queries against the database, and given that we do not know how to efficiently explore the space of possible query plans, Bayesian optimization is a promising approach. Bayesian optimization (BO) enables optimization of expensive-to-evaluate, black-box functions while requiring relatively few evaluations of the expensive function. However, Bayesian optimization techniques operate over continuous, real-valued domains, whereas query plans are discrete tree structures.

Prior work on Latent Space Bayesian Optimization (LSBO)
% on drug discovery \cite{lolbo} 
has allowed Bayesian optimization to be applied over other discrete, combinatorial spaces by using a deep autoencoder model (DAE) to transform a discrete, structured search space $\mathcal{X}$ into a continuous, numerical one $\mathcal{Z}$ \cite{Weighted_Retraining,ladder,gomez2018automatic,Huawei,eissman2018bayesian,kajino2019molecular, lolbo}.
For example, \citet{lolbo} applied LSBO to problems in drug discovery using a DAE trained on string representations of molecules. 
% has applied BO in the latent space of a deep autoencoder model (DAE) trained on string representations of molecules. 
Inspired by this work, we define a string encoding format for query plans that can represent all possible query plans and in which each string decodes unambiguously to a particular query plan (\Cref{sec:string-format}). We train a variational autoencoder (VAE) on strings of this format using plans derived from a traditional query optimizer (\Cref{sec:technique-vae}). We perform BO in the latent space of this VAE (\Cref{sec:bo-background}), making novel contributions in the selection of timeouts (\Cref{sec:censored_observations}). Finally, we discuss different strategies for initializing the local BO process, as the quality of initialization points impacts BO performance (\Cref{sec:initialization_strategies}).

\subsection{Query Plan String Format}
\label{sec:string-format}
Our first step towards optimizing query plans with Bayesian optimization is to build a string language for query plans that we encode and decode into a vector space. We first explain the desired properties of this string language. Then, we explain the string language we designed to have these properties. While we believe our string languages makes some good tradeoffs, in theory any string language with the desired properties could be equally effective.

\sparagraph{Desiderata for string representations.} We identify two important attributes of our string language from prior work on molecular optimization. Maus et al.~\cite{lolbo} identify two essential properties that are key to success of a string language for Bayesian optimization (in the case of~\cite{lolbo}, the string language is called SELFIES \cite{selfies} and describes molecules). We translate those design constraints from the domain of~\cite{lolbo} to query optimization, and adopt these properties as requirements for our query plan language:
\begin{enumerate}[leftmargin=*]
    \item \textbf{Completeness.} Any valid query plan in $X$ must be representable as a string using the language. If this is not true, high-quality plans that are not representable will not be discoverable by our optimization algorithm.
    \item \textbf{Decoding validity.} Any sequence of characters in the language must correspond to a unique and valid query plan. If latent codes of the DAE decode to invalid query plans, optimization would need to be done under additional feasibility constraints, which adds unnecessary complexity to the optimization problem. Validity was a primary goal of the models in both \citet{JTVAE} and \citet{maus2022local}.
\end{enumerate}

\edit{The ideal string representation would also be \emph{injective}, meaning that each unique string maps to a unique plan~\cite{geom_card_est}. We were unable to find a string representation with all three of these properties, so we settle for a complete representation with decoding validity that is not injective (the same plan might be represented by multiple strings). Taking this tradeoff is motivated by prior work: Maus et al.~\cite{bayes_latent} showed that a string representation with only completeness and decoding validity (SELFIES~\cite{selfies}) outperformed an injective representation with completeness (SMILES~\cite{weininger_smiles}). We leave the investigation of alternative string representations to future work.}

\sparagraph{String language for query plans} We design an encoding format for binary-tree-structured query plan defining join orders and operators (henceforth, a ``join tree''). \edit{The join tree does not encode other aspects of the query such as selections, join and filter predictes, and aggregations. When the join tree is decoded to executable SQL, we translate the join tree to a hint string and prepend it to the original SQL text which contains these aspects of the query.} We observe that a join tree can be unambiguously reconstructed by knowing for each non-leaf node the left and right subtree that compose it, and that valid trees have left and right subtrees that are disjoint; we simply need an unambiguous way to identify these subtrees.

In our encoding format, each join subtree is expressed as a 3 symbol sequence (left, right, operator). Each physical join operator ($h$ash, $m$erge, $n$ested loops) is given a unique symbol. The fully specified query plan is simply the concatenation of these sequences.

The leaves of a join tree are always the tables being joined, so we define a unique symbol for each base table in the schema. However, we cannot define symbols for each possible join subtree, as doing so would be tantamount to defining a unique symbol for every possible query plan. Instead, we observe that after a table symbol is used once to specify a join subtree, it will never be used again, as any table will only ever appear as a leaf once. The same is true for each join subtree after it is specified to be the child of some larger subtree. As such, to the right of a particular join subtree sequence, the symbols composing the join instead refer to the larger subtree.

\edit{For multiple occurrences of the same table under different aliases within the same query, we rename such aliases to be numbered (e.g. \texttt{movies1}, \texttt{movies2}, ...), and we define a unique symbol in our language per table-number pair. This does require choosing a maximum number of possible aliases of a single table; in our experiments we select the maximum occurrences of a particular table within the benchmark queries.}

The leftmost occurrence of a table symbol in a plan string always references the base table itself, but subsequent occurrences represent the largest subtree that the table is part of. For example, in a join between three tables $A$, $B$, and $C$, $(A \bowtie_\text{hash} (B \bowtie_\text{merge} C))$, the valid encoding strings are $(B, C, \bowtie_\text{m}, A, B, \bowtie_\text{h})$ and $(B, C, \bowtie_\text{m}, A, C, \bowtie_\text{h})$.

To fulfill our second requirement, decoding validity, we use a simple trick. We maintain state about the partially-specified join tree as we decode the string from left to right. If the decoder encounters a symbol that is not syntactically valid (e.g. a table in place of a join operator) or semantically valid (e.g. a table that is not part of the join), it deterministically resolves it to a symbol that is valid by constructing a list of all valid symbols and using the invalid symbol's integer value as an index into the list.

\edit{We note that the choice of replacement symbol is arbitrary, but this scheme for ensuring decoding validity is preferable to more obvious ideas such as simply refusing and resampling when encountering invalid strings or decoding them to some default plan. Bayesian optimization will be performed over a vector space that decodes to potential plan strings. Rejecting strings would prevent the surrogate model learning about vast regions of the plan space. Decoding all invalid strings to some default plan would make vast regions of the space undifferentiated in performance. Our technique ensures that all strings decode to valid plans, that similar invalid strings are mapped to somewhat different valid query plans, and that these decodings are a valid function of the input string.}

\edit{
\sparagraph{Limitations} Our language does not represent subqueries and CTEs. When processing queries that contain such structures, they are left untouched, so the decoded query plan hint will not contain any reference to them or to tables only occurring within them.
}






\subsection{Encoding \& Decoding Query Plans} \label{sec:technique-vae}
While BO can be applied to various search spaces, it is most straightforward in a continuous, real-valued domain. This presents a challenge when optimizing query plans, which are inherently discrete tree structures. To address this, we construct a mapping between query plans and points in a continuous domain. Having defined our string representation in \Cref{sec:string-format}, we train a deep autoencoder (DAE) model on these encoded strings. This process generates a \emph{latent space} – a continuous, real-valued domain that serves as a proxy for the discrete space of query plans, enabling application of BO techniques. Intuitively, the goal of the DAE is to construct a latent space in which similar query plans are mapped to similar vectors. This way, a search routine that finds a particularly good plan can look at the ``neighbors'' of that plan in the latent space for similar plans. The notions of ``similar'' and ``neighbors'' are both highly approximate: no actual neighborhoods or similarity scores are computed, but instead this property is \emph{implicitly} created when training the DAE.

A DAE consists of an encoder $\Phi: X \to Z$ that maps from an input space $X$ to a latent space $Z$ (sometimes called a bottleneck~\cite{bottleneck}, as it is often lower dimensional than $X$ in order to force a degree of compression) and a decoder $\Gamma : Z \to X$ that maps from the latent space back to the input space. We use a type of DAE known as a variational autoencoder (VAE)~\cite{KingmaW13}, in which the encoder produces a distribution over latent points $\Phi(Z | X)$, and the decoder produces a distribution over $X$ given $Z$, $\Gamma(X | Z)$. The model is trained by maximizing the evidence lower bound (ELBO):
$$
\mathbb{E}_{\Phi(Z | X)}[\log \Gamma(X | Z) - \text{KL}(\Phi(Z | X) || p(Z) ]
$$

In training such a DAE, we construct a mapping $\Gamma : Z \to X$ that can produce string-encoded query plans given points in the latent space. The VAE regularization (the KL term, representing relative entropy) makes the search space smooth, facilitating more effective optimization. This is precisely what we need in order to perform BO: the surrogate model is defined over the latent space, and we evaluate the black-box function $f$ for points in the latent space by decoding the point to a string query plan through the DAE and executing them against the real database.

\sparagraph{Training data} In order to train the DAE, we compute a large set of encoded query plans ($\sim$1 million) from the database schema. Importantly, this process does not require any query execution, and can be done exclusively with only metadata from the DBMS. The idea is to create a suitably diverse set of ``reasonable'' query plans so that the DAE can learn a smooth probability distribution of the space of query plans. By ``reasonable'' here, we do not mean that these plans must all be optimal, just that they must be somewhat representative of the \emph{family} of optimal query plans---\edit{the purpose of this is to create a space of plans in which points that are close to each other have similar performance characteristics. However, the space still contains points for \emph{all possible} query plans.}
% that is, we hope that the optimal plan for any query is somewhat similar to at least one of the plans we generate as training data for the DAE.

% To generate this set of plans, we first create a set of random join queries by assembling a graph $G_{R} = (V, E)$ representing PK-FK relationships in the schema (the ``reference graph''). For a schema containing a set of tables
% $T = \left \{ t_1, t_2, ..., t_n \right \}$
% with each table $t_i$ having attributes
% $C^i = \left \{ c^i_1, c^i_2, ... c^i_m \right \}$
% including PK-FK relationships between foreign keys $c^i_f$ and primary keys $c^j_p$ denoted by
% $\text{Ref} \left( c^i_f, c^j_p \right )$,
% \begin{gather*}
%     V = T \\
%     E = \left \{ (t_i, t_j) \mid t_i, t_j \in T \wedge \exists c^i_f \in C^i, c^j_p \in C^j : \text{Ref}(c^i_f, c^j_p) \right \}
% \end{gather*}

% We then expand this graph to include up to $k$ aliases with copies of tables. For a set of aliases
% $K = \left \{ 1, 2, ... k \right \}$
% the ``alias-$k$ reference graph'' $G^k_R = (V^k, E^k)$,
% \begin{gather*}
%     V^k = \left \{ t^a_i \mid t_i \in V \wedge a \in K \right \} \\
%     E^k = \left \{ (t^a_i, t^b_j) \mid (t_i, t_j) \in E \wedge a, b \in K \right \}
% \end{gather*}
\edit{To generate this set of plans, we sample random PK-FK equijoin queries from the schema by constructing the ``alias-$k$ reference graph'' which contains $k$ nodes corresponding to each table and edges corresponding to all PK-FK references between tables. We choose $k$ equal to the highest number of aliases of the same table used in any query in the workload. From this alias-$k$ reference graph, we sample queries by selecting random connected subgraphs with varying numbers of vertices. Given a particular subgraph, we produce a query joining all table aliases with join predicates corresponding to all present edges.}
% From $G^k_R$, we sample queries within the schema by selecting random connected subgraphs of varying numbers of vertices. Given a particular subgraph $G^{k\prime}_R = (V^{k\prime}, E^{k\prime})$ we produce the query joining all table aliases in $V^{k\prime}$ with join predicates corresponding to all references in $E^{k\prime}$

For each sampled query, we plan the query using the existing default query optimizer (e.g. PostgreSQL), encode the plan in our string encoding format, and add it to the VAE training set. In order to expand the diversity of plans used to train the VAE, we additionally produce encoded plans using hints~\cite{url-pg_hints} to the default query optimizer (e.g. disable nested loops, disable sequential scans).

Our training data generation process makes two key design choices: (1) sampling random queries from the database schema, and (2) generating query plans with the database's default optimizer. The first decision ensures that we have coverage for a wide variety of input queries. Our goal is to train the DAE once per schema, and then reuse the DAE for every query over the schema. The second decision ensures that the query plans we get are somewhat reasonable. For example, the underlying database optimizer is unlikely to pick a plan full of cross joins, and is likely to take advantage of index structures if applicable.


\subsection{Background on Bayesian Optimization} \label{sec:bo-background}
Given a query plan language and a trained DAE to translate query plan strings into vectors in a latent space (and back), we can now optimize queries inside of the latent space using Bayesian optimization (BO). Intuitively, BO in our application works by learning the relationship between the DAE's latent space and actual query plan latency. BO learns this relationship by repeatedly testing points sampled from the latent space. If the BO algorithm can get a good estimation of the relationship between the latent space and query latency, then good plans can be found. This section gives important background on the BO technique we use in this paper. Then, in Section~\ref{sec:censored_observations}, we explain some of the small changes we made to traditional algorithms to address query optimization specifically.

% Bayesian optimization \cite{distill_bayesopt} is a method for optimizing black-box functions. For some multidimensional space of inputs $X$ and black-box evaluation function $f$, BO finds some $x \in X$ that minimizes $f(x)$. BO is typically deployed when $f$ is expensive to evaluate (e.g., $f$ requires executing a physical plan against a DBMS and measuring its runtime). Because of this constraint, BO techniques, unlike reinforcement learning techniques, are designed to be  \emph{sample-efficient}, requiring few evaluations of $f$.

% BO works by iteratively refining a \emph{surrogate model} of $f$: it
% \begin{enumerate*}
%     \item initializes the surrogate model with some prior distribution,
%     \item uses an \emph{acquisition function} to select points in the input space $x \in X$,
%     \item evaluates $f(x)$ and
%     \item updates the surrogate model with the point $(x, f(x))$
% \end{enumerate*}

% BO assumes that the true unknown function $f$ exhibits locality: observing a particular point $(x, f(x))$ suggests that points nearby $x' \approx x$ will have similar values of $f(x') \approx f(x)$. More evidence further refines the surrogate model, making it a better estimate of $f$. In our setting, $x$ is some representation of a query plan, and the unknown function $f$ returns the execution latency of that plan. %Evidence is obtained by actually executing the query plan represented by $x$ against a database.

% In order to efficiently find good values $x$ that minimize $f(x)$, BO must balance exploration (finding out more about relatively unknown regions of $X$) with exploitation (focusing on promising regions of $X$ already known to have the lowest values seen for $f(x)$). Selecting points to sample is the role of the \emph{acquisition function.} Our approach utilizes \emph{Thompson sampling}~\cite{thompson} as its acquisition function.

% \factcheck{Is everything from beginning of subsection to here correct?}
\sparagraph{Bayesian Optimization} This section provides a brief overview of Bayesian Optimization (BO). For readers unfamiliar with BO, we recommend the comprehensive book by \citet{garnett_bayesoptbook_2023}. Our methodology builds upon the approach developed by \citet{eriksson2019scalable}, \edit{with specific novel modifications tailored for optimizing query plans and execution latency in a DBMS.}

Bayesian Optimization is a method for optimizing black-box functions that are expensive to evaluate, aiming for \emph{sample efficiency}. Given an input space $\mathcal{X}$ and an unknown objective function $f: \mathcal{X} \rightarrow \mathbb{R}$, BO seeks to find an input $x^* \in \mathcal{X}$ that minimizes $f(x)$ in as few evaluations of $f$ as possible. This is particularly useful when each evaluation of $f(x)$ is costly---for example, when $f(x)$ involves executing a query plan in a DBMS to measure its runtime.

BO operates by constructing a probabilistic surrogate model of the objective function, which is iteratively refined as new data is acquired. The general optimization procedure follows these steps:

\begin{enumerate}[leftmargin=*]
\item \textbf{Initialization}: Build a surrogate model of the objective function $f$.
\item \textbf{Acquisition Function Optimization}: Use an acquisition function to select the next point $x_{\text{next}}$ to evaluate, balancing exploration and exploitation.
\item \textbf{Evaluation}: Compute the true function value $f(x_{\text{next}})$ by executing the query plan corresponding to $x_{\text{next}}$.
\item \textbf{Model Update}: Update the surrogate model with the new observation $(x_{\text{next}}, f(x_{\text{next}}))$, and repeat steps 2--4.
\end{enumerate}

To efficiently navigate the search space, BO leverages the surrogate model along with the acquisition function to select promising candidate plans while minimizing the number of expensive evaluations. In this work, we use \emph{Thompson Sampling}~\cite{thompson} as the acquisition function.

\sparagraph{Local BO} Standard BO methods can struggle with high-dimensional or discrete optimization problems, such as those encountered in query plan optimization, due to the curse of dimensionality and the combinatorial explosion of the search space. To address this, we incorporate methods from the local BO literature, specifically \emph{TuRBO}~\cite{eriksson2019scalable}. TuRBO maintains a hyper-rectangular ``trust region'' within the input space, which constrains the region from which points are sampled. By dynamically adjusting the size and location of these trust region based on the the optimization success / failure, TuRBO can balance global exploration with local exploitation, allowing for efficient optimization in high-dimensional spaces. \footnote{\edit{Though called ``local BO'', this is a \emph{global} optimization process that can produce results significantly different from the initialization points. Local BO methods are the most competitive methods in high-dimensional spaces, as established in Eriksson et al. ~\cite{eriksson2019scalable}.}}

\sparagraph{Right-Censored Observations}
During typical Bayesian optimization, when we make an observation at a point $\bx$, we obtain an associated objective value $f(\bx) = y$. 
For a right-censored observation, when we observe at the point $\bx$, we instead only learn that $y$ was greater than some threshold $\tau$. In our application, right-censored observations represent query timeouts: If a query $\bx$ is observed to execute for $\tau$ seconds before timing out, then we know that the true latency of $q$ is \emph{at least} $\tau$: $f(\bx) \geq \tau$. 

%Conventional BO techniques assume that evaluations of $f$ correspond to the true value of the observation with additive noise. 
In the query optimization setting, using right-censored observations is particularly important. Obtaining true values for arbitrary plans in the space of possible plans can be infeasible, as bad plans may take days or even weeks. Thus, it is more efficient if we can \emph{time out} plans that perform poorly and update the surrogate model with knowledge that the running time of $x$ is ``at least as bad as $y$''. Intuitively, for regions of $X$ that contain truly awful  plans, for the purposes of finding optimal plans, it is not necessary to know exactly how bad a particular plan is---it suffices to know that plans like it should be avoided. 

BO in the presence of censored observations was first explored by Hutter et al. ~\cite{DBLP:journals/corr/HutterHL13}, where an EM-like algorithm was used to impute the value of censored responses.
%They applied this to algorithm configuration, and terminated runs longer than a constant factor longer than the best running time observed so far.
They applied this method to algorithm configuration, terminating any runs that exceeded a constant factor of the shortest running time observed so far. Building on this, Eggensperger et al. ~\cite{eggensperger2020neural} trained a neural network surrogate on a likelihood based on the Tobit model to directly model right-censored observations:
%
\begin{equation}
\begin{aligned}
\label{eq: tobit}
p(\mathbf{y}|\mathbf{f}) &= \phi(\mathbf{z})^{1-\mathbf{I}}(1-\Phi(\mathbf{z}))^\mathbf{I} \\
\mathbf{z} &= \frac{\mathbf{f}-\mathbf{\mu}}{\mathbf{\sigma}^2}\\ 
I&= \begin{cases}
       0, & \text{if } \mathbf{y} \text{ is uncensored} \\
       1, & \text{if } \mathbf{y} \text{ is censored}
    \end{cases}
\end{aligned}
\end{equation}
%

\noindent where $\phi$ and $\Phi$ denote the Gaussian density and cumulative density function respectively. In~\cite{eggensperger2020neural}, timeout thresholds were chosen as a fixed percentile of existing observations.

\sparagraph{Approximate Gaussian Processes}  
Because the space of query plans is large, we anticipate needing to test a large number of query plans. As a result, we must select a surrogate model that (1) allows for \emph{probabilistic inference}, that is, gives a probability distribution at each point instead of a simple point estimate, and (2) can scale to a large number of observations. Thus, we select an \emph{approximate} Gaussian Process (GP) model.

Approximate GP models, such as the popular Scalable Variational Gaussian Process (SVGP)~\cite{svgp}, use inducing point methods in combination with variational inference to allow approximate GP inference on large data sets~\cite{hensman2013gaussian,titsias2009variational}. The standard evidence lower bound (ELBO) on the log-likelihood used to train a SVGP model is the following:
% 
\begin{align}
\label{eq: svgp}
\log p(\by) & \geq  \mathbb{E}_{q(\bfn)}[\log p(\by \mid \bfn)] - \textrm{KL}(q(\bu)\,||\,p(\bu)) 
\end{align}
%
% Because of the large number of total query plans we will execute over the course of optimization and the need for approximate inference due to the non-Gaussian likelihood \autoref{eq: tobit}, we adapt SVGP \citep{svgp} models to this purpose.



\subsubsection{Bayesian Optimization with Censored Observations} \label{sec:censored_observations}

While previous work on Bayesian optimization with censored observations (censored BO) did not use approximate SVGP~\cite{svgp} models, \edit{we contribute a straightforward extension} of SVGP~\cite{svgp} models to the censored BO setting. 
Starting from \Cref{eq: svgp} and using the Tobit likelihood given in~\Cref{eq: tobit}, we derive the new ELBO:
%
\begin{equation*}
\centering
% Im just doing this so it fits for now
\begin{aligned}
& \log p(\by) \geq  \mathbb{E}_{q(\bfn)}[\log p(\by \mid \bfn)] - \textrm{KL}(q(\bu)\,||\,p(\bu)) \\
			& = \mathbb{E}_{q(\bfn)}[\log \phi(\mathbf{Z})^{1-\mathbf{I}}(1-\Phi(\mathbf{Z}))^\mathbf{I}] - \textrm{KL}(q(\bu)\,||\,p(\bu)) \\
			& = \mathbb{E}_{q(\bfn)}[\log \phi(\mathbf{Z})^{1-\mathbf{I}} + \log(1-\Phi(\mathbf{Z}))^\mathbf{I}] - \textrm{KL}(q(\bu)\,||\,p(\bu)) \\
            & = \mathbb{E}_{q(\bfn)}[\log \phi(\mathbf{Z_u})] + \mathbb{E}_{q(\bfn)}[\log(1-\Phi(\mathbf{Z_c}))] - \textrm{KL}(q(\bu)\,||\,p(\bu)) 
\end{aligned}
\end{equation*}
%
Here, $\mathbf{Z}_{u}$ correspond to $\frac{\mathbf{f} - \mu}{\sigma^2}$ values for uncensored observations, and $\mathbf{Z}_{c}$ correspond to censored observations. The first term
%, $\mathbb{E}_{q(\bfn)}[\log \phi(\mathbf{Z_u})]$, 
can be computed analytically as in standard SVGP models. The second term, $\mathbb{E}_{q(\bfn)}[\log(1-\Phi(\mathbf{Z_c}))]$, can be computed using one dimensional numerical techniques like Gauss-Hermite quadrature.  

During optimization, we select a threshold $\tau$ for each executed query plan $\bx$, and cut off execution once the running time exceeds $\tau$. This results in right-censored observations. 
Selecting the timeout for any given observation is crucial: selecting too low of a timeout deprives BO of important knowledge about the space of plans, whereas selecting too high of a timeout wastes time executing bad plans. Previous work in BO uses a constant multiplier over the best observation seen so far~\cite{hutter2013_bocensored}, or a fixed percentile across all observations~\cite{eggensperger2020_censored}. Balsa~\cite{balsa} also uses a fixed multiplier  in order to bound the impact of executing bad plans. 
\edit{We use an uncertainty-based method for selecting timeouts that, compared to prior work, ensures that the surrogate model will be sufficiently confident that a particular point is suboptimal before timing out.}

% We cannot just treat the timeout as true values. Example: $P_1$ would've taken an hour, but times out after 5m. $P_2$ would've taken 30 minutes, but times out after 5m. $P_2$ is twice as good as $P_1$, but they look "the same" to the BO algo. Tricks like multipling the timeout by some constant require adhoc tuning (like balsa).

Before evaluating a new candidate query plan $x_t$ during step $t$ of optimization, we dynamically set a new timeout threshold $\tau_t$. 
We select thresholds so that, after conditioning on the right-censored observation $(\bx_{t}, \tau_{t})$, we are \textit{confident} that the best query plan observed so far, $\bx^{*}_{t}$, is still a better design than the candidate plan $\bx_{t}$. 
Because we do not want to waste additional running time evaluating $f(\bx_{t})$, we ideally want the \textit{smallest} such $\tau_{t}$. 

\paragraph{Selecting $\tau_{t}$.} The above discussion leads to the following optimization problem, where we find the smallest threshold $\tau$ so that our incumbent is confidently better than $\bx_{t}$ \textit{after conditioning on $(\bx_{t}, \tau)$}:
%
\begin{align*}
    \tau_{t}^{*} &= \argmin \tau \\
    & \textrm{s.t.}\;\; y^{*}_{t} \leq \mu'_{t}(\tau) - \kappa \sigma'_{t}(\tau)
\end{align*}
%


On \edit{its} surface, this optimization problem is challenging, as evaluating our constraint for a given $\tau$ involves updating the Gaussian process surrogate model with that value $\tau$ as the observed timeout. This is similar to other acquisition functions in the Bayesian optimization literature that use fantasization to do lookahead, e.g., knowledge gradient~\cite{frazier2009knowledge}.

Because we use variational GPs, there are several inexpensive strategies that we can use to evaluate the constraint. For example, \citet{maddox2021conditioning} recently proposed an efficient routine for online updating sparse variational GPs, both with conjugate and non-conjugate likelihoods. Alternatively, a few additional iterations of SGD can be used to update the model in a less sophisticated way.

Finally, we note that the value of $\mu'_{t}(\tau) - \kappa \sigma'_{t}(\tau)$ should generally be monotonic in $\tau$---fantasizing that $\bx_{t}$ cut off with a larger threshold should strictly increase the gap between our belief about $y_{t}$ and $y^{*}_{t}$. Therefore, given a routine to cheaply evaluate the constraint, the constrained minimization problem over $\tau$ can be solved e.g. with binary search.


\subsection{Initialization Strategies} \label{sec:initialization_strategies}
The initial step of BO for a given query typically involves selecting points within the latent space using the acquisition function. As the surrogate is initialized with a random prior, this amounts to selecting random points within the latent space. Theoretically, given sufficient time for BO execution, this approach would yield optimal results. However, to improve the practicality of BO within high dimensional spaces, it is helpful to initialize the process with a small number of precomputed $(x, f(x))$ pairs representing high-quality plans. We explore multiple methods of generating these initialization points.

% \jeff{cut? goes against story about information/time tradeoff} A common theme across all of these initialization techniques is that in the interest of saving computation time, not all proposed plans are executed to completion. Since our system can accommodate censored observations, as described in the previous section, execution of any particular plan is cut off at the runtime of the best plan seen so far. As the runtime of the best found plan improves, these strategies search through subsequent proposed plans more quickly as the timeout is lower.

% Regardless of the strategy we use to generate initial data points (plans), we execute the plans proposed by these strategies serially, seeking to minimize the total computation time spent. For faster wall clock time, it is possible to execute proposed plans in parallel batches, but computing the latency of the initialization points is usually not a significant computational cost (we verify this experimentally in Section~\ref{sec:eval}).

\sparagraph{Hinted plans (Bao)} We can leverage an existing traditional query optimizer that accepts hints, such as PostgreSQL, to generate the initialization points. We exhaust all of the combinations of join and scan hints (as in the hint sets used by Marcus et al.'s Bao~\cite{bao} optimizer) to produce 49 initialization points for each query. These 49 initialization points are \emph{guaranteed} to contain the best plan that could have possibly been chosen by Bao. We note that it is not important \emph{which} of the queries in the initialization set is optimal, merely that the initialization set contains some queries that represent a promising starting point for optimization. Thus, it is not necessary to ``prune'' hint sets from Bao, as is recommended in~\cite{bao}.

\sparagraph{The default optimizer plan} A simpler strategy would be to generate a single optimization point by using the DBMS' underlying optimizer. This approach has the advantage of simplicity, since the underlying DBMS almost surely has an optimizer. Unfortunately, we found that this approach does not work well in practice, mostly because initializing BO with a single initialization point seems to be suboptimal~\cite{lolbo}.

\sparagraph{LLM} Inspired by previous work demonstrating the effectiveness of large language models (LLMs) in optimizing program runtimes \cite{pie}, we explore the use of fine-tuned LLMs for generating initialization points. We collected trajectories from 606 \sysname runs, selecting the top-1 and top-5 query plans for each query to construct a fine-tuning dataset. Using this dataset, we fine-tuned GPT4o-mini for one epoch. For each new query, we use the fine-tuned model to sample 50 initialization points. This approach leverages the model's ability to learn patterns from previous optimization runs, potentially producing high-quality plans that outperform those generated by traditional query optimizers. Our evaluation (\cref{sec:eval-llm}) demonstrates that this LLM-based strategy can often produce the best query plan among all initialization strategies considered here.

\sparagraph{Extensibility} \sysname simply admits sets of initialization pairs $(x, f(x))$, so any strategy can be used to generate these pairs. As such, our approach can incorporate future improvements in traditional or learned query optimization techniques.

% tradeoffs around size of initialization set (bigger = more budget spent on executing the initial set, smaller = not enough information)

\subsection{Random plans} \label{sec:random}
Though not related to BO, we implement random plan search, which can be thought of as a completely exploration-based algorithm. The intuition behind this method is that joins are commutative but that cross-joins are generally bad for performance. This strategy samples random plans from the space of all plans that do not contain any cross joins.

% Given a particular query over a set of table aliases $T'$, we can construct the alias-$k$ reference graph $G_Q = (V_Q, E_Q)$ containing only the table aliases referenced in the query:

% \begin{gather*}
%     V_Q = T' \\
%     E_Q = \left \{ (t_1, t_2) \mid  t_1, t_2 \in T'; (t_1, t_2) \in E^k \right \}
% \end{gather*}

\edit{Given a particular query over a set of table aliases, we construct the subgraph of the schema's alias-$k$ reference graph containing only the table aliases referenced in the query.}
From this query graph, we can construct a random join tree by constructing a spanning tree. Whenever an edge is added to the spanning tree, we add the join between the two newly connected components to the join tree. Physical join operators are selected uniformly randomly.

One potential benefit of utilizing this strategy is that its viability implies that it may be possible to perform offline optimization in the absence of a traditional query optimizer. As we show in \Cref{sec:eval}, this strategy can be used on its own to perform offline optimization.









\section{Experiments: Planning outperforms Heuristics}
\label{sec:experiment}

We begin our empirical demonstrations by showcasing the effectiveness of our planning framework on both synthetic and real datasets. We focus on the simplest planning algorithm, 1-step lookaheads (Algorithm~\ref{alg:complete}), and show that even basic planning can hold great promise. 
We illustrate our framework using two uncertainty quantification modules---GPs and 
\ensembles/ \ensembleplus. 

Throughout this section, we focus on evaluating the mean squared error of 
a regression model $\model$,  and develop adaptive policies that minimize uncertainty on $g(f)$ defined in~\eqref{eqn:l2-g-f}.
When GPs provide a valid model of uncertainty, 
our experiments show that our planning framework significantly outperforms other baselines. 
We further demonstrate that our conceptual framework extends to deep learning-based uncertainty quantification methods such as  \ensembleplus while highlighting computational challenges that need to be resolved in order to scale our ideas. 
For simplicity, we assume a naive predictor, i.e., $\psi(\cdot) \equiv 0$. However, we emphasize that this problem is just as complex as if we were using a sophisticated model $\psi(.)$. The performance gap between the algorithms 
primarily depends
on the level  of uncertainty in our prior beliefs.

To evaluate the performance of our algorithm, we benchmark it against several baselines. 
%Active learning baselines use an acquisition function $\ac$ to select points that have the highest   function value: $X\opt_t \in \argmax_{X \in \xpoolj{t}} \ac({X})$ at every step $t$. These methods may also need an UQ module, which we simply use the same UQ module as in our algorithm, and it  outputs $V(X)$ that measures the the uncertainty of each point $X \in \xpoolj{t}$.
Our first set of baselines are from active learning~\citep{AggarwalKoGuHaPh14}:
\\ % \noindent\textbf{Active Learning Heuristics:} 
\textbf{(1)} 
\textsf{Uncertainty Sampling (Static):}  In this approach, we query the samples for which the model is least certain about. Specifically, we estimate the variance of the latent output $f(X)$ for each $X \in \xpool$ using the UQ module and select the top-$K$ points with the highest uncertainty. \\
\textbf{(2)} \textsf{Uncertainty Sampling (Sequential):} This is a greedy heuristic that sequentially selects the points with the highest uncertainty within a batch, while updating the posterior beliefs using pseudo labels from the current posterior state. Unlike \textsf{Uncertainty Sampling (Static)}, this method takes into account the information gained from each point within batch, and hence tries to diversify the selected points within a batch. 

 
We also compare our approach to the  \textbf{(3)} \textsf{Random Sampling}, which selects each batch uniformly at random from the pool. Additionally, we compare solving the planning problem using  \textsf{REINFORCE}-based policy gradients with   $\mathsf{Smoothed\text{-}Autodiff}$ policy gradients.\footnote{Our code repository is available at
  \url{https://github.com/namkoong-lab/adaptive-labeling}.}
%Detailed experimental setups are provided in Section \ref{sec:details-experiments}.

%We repeat all experiments with 10 random seeds.




\begin{figure}[t]
\centering
\begin{minipage}[b]{0.49\textwidth}
\centering
\includegraphics[width=\textwidth, height=5cm]{figures/original_scale/Var_of_l_2_loss.pdf}
\caption{(Synthetic data) Variance of mean squared loss evaluated through the posterior belief $\mu_t$ at each horizon $t$. This is the objective that policy gradient methods like \textsf{REINFORCE} and $\ouralgo$ optimizes. 1-step lookaheads are surprisingly effective even in long horizons.}
\label{fig:var-l2-sim}
\end{minipage}
\hfill
\begin{minipage}[b]{0.49\textwidth}
\centering \includegraphics[width=\textwidth, height=5cm]{figures/original_scale/Error_of_estimated_model_l_2_loss.pdf}
\caption{(Synthetic data) Error between MSE calculated based on collected data $\mc{D}^{0:T}$ vs. population oracle MSE over $\mc{D}_{\rm eval} \sim P_X$. Reducing uncertainty over posteriors directly leads to better OOD evaluations. 1-step lookaheads significantly outperform active learning heuristics in small horizons.}
\label{fig:mean-l2-sim}
\end{minipage}
%\caption{Simulated data for GPs}
%\label{fig:both_plots}
\end{figure}

\subsection{Planning with Gaussian processes}
\label{sec:experiment-plan-GP}
We now briefly describe the data generation process for the GP experiments,  deferring a more detailed discussion of the dataset generation to Section~\ref{sec:details-experiments}. 
We use both the synthetic data and the real data to test our methodology.
For the \emph{simulated data},  we construct a setting where the general population is distributed across \emph{51 non-overlapping clusters} while the initial labeled data $\dtrain$ just comes from one cluster. In contrast, both $\dpool \defeq (\xpool,\ypool),\deval \defeq (\xeval,\yeval)$ are generated   from all the clusters. 
We begin with a low-dimensional scenario, generating a one-dimensional regression setting using a GP. %Gaussian Process (GP).
Although the data-generating process is not known to the algorithms,  we assume that the GP hyperparameters are known to all the algorithms
to ensure fair comparisons. This can be viewed as a setting where our prior is well-specified, allowing us to isolate the effects
of different policy optimization approaches
 without any concerns about the misspecified priors. We select $10$ batches, each of size $K=5$ across $T = 10$ time horizons.

To examine the robustness of our method against the distributional assumptions made  in the simulated case, we then move to a real dataset where the correct prior is not known. We simulate selection bias from the eICU dataset~\citep{PollardJoRaCeMaBa18}, which contains real-world patient data with in-hospital mortality outcomes. 
We conduct a $k$-means clustering to generate 51 clusters and then select data from those clusters. We view this to be a credible replication of practice, as severe distribution shifts are common due to selection bias in clinical labels.  To convert the binary mortality labels into a regression setting, we train a  random forest classifier and fit a GP on predicted scores, which serves as the UQ module for all the algorithms. As before, the task is to select 10 batches, each consisting of 5 samples, across 10 time horizons.

 In Figures~\ref{fig:var-l2-sim} and~\ref{fig:mean-l2-sim}, we present results for the simulated data. 
Figure~\ref{fig:var-l2-sim} shows the variance of $\ell_2$ loss, and Figure~\ref{fig:mean-l2-sim} presents the error in the estimated $\ell_2$ loss using $\mu_t$ (relative to true $\ell_2$ loss, that is unknown to the algorithm). 
As we can see from these plots, our method one-step lookahead  gives substantial improvements  over active learning baselines and random sampling. In addition,
compared to the one-step lookahead planning approach using \textsf{REINFORCE}-based policy gradients, 
we observe that $\mathsf{Smoothed\text{-}Autodiff}$-based policy gradients provide significantly more robust performance over all horizons.

In Figures~\ref{fig:var-l2-real}~and~\ref{fig:mean-l2-real}, we observe similar findings on the eICU data. We see that planning policies (\textsf{REINFORCE} and $\mathsf{Smoothed\text{-}Autodiff}$) consistently outperform other heuristics by a large margin.  Active learning baselines perform poorly in these small-horizon batched problems and can sometimes be even worse than the random search baselines.  Overall, our results show the importance of careful planning in adaptive labeling for reliable model evaluation. 

We offer some intuition as to why one-step lookahead planning may outperform other heuristic algorithms. 
 First,  \textsf{Uncertainty sampling (Static)} while myopically selects the
 top-$K$ inputs with the highest uncertainty, it fails to consider 
the overlap in information content among the ``best” instances; see \citep{AggarwalKoGuHaPh14} for more details. 
In other words,  it might acquire points from the same region with high uncertainty while failing to induce diversity among the batch.
Although \textsf{Uncertainty Sampling (Sequential)} somewhat addresses the issue of information overlap, a significant drawback of 
this algorithm
is the disconnect between the objective we aim to optimize and the algorithm. For example, it might sample from a region with high uncertainty but very low density. 

\begin{figure}[t]
\centering
\begin{minipage}[b]{0.48\textwidth}
\centering
\includegraphics[width=\textwidth, height=5cm]{figures/original_scale/Var_of_l_2_loss_real.pdf}
\caption{(Real-world eICU data) Variance of mean squared loss evaluated through the posterior belief $\mu_t$ at each horizon $t$. Even 1-step lookaheads are extremely effective planners, and auto-differentiation-based pathwise policy gradients provide a reliable optimization algorithm based on low-variance gradient estimates.}
\label{fig:var-l2-real}
\end{minipage}
\hfill
\begin{minipage}[b]{0.48\textwidth}
\centering \includegraphics[width=\textwidth, height=5cm]{figures/original_scale/Error_of_estimated_model_l_2_loss_real.pdf}
\caption{(Real-world eICU data) Error between MSE calculated based on collected data $\mc{D}^{0:T}$ vs. population oracle MSE over $\mc{D}_{\rm eval} \sim P_X$. Reducing uncertainty over posteriors directly leads to better OOD evaluations. Our method significantly outperforms active learning-based heuristics, and random sampling.}
\label{fig:mean-l2-real}
\end{minipage}
%\caption{Real data for GPs}
\end{figure}
 
%\vspace{-1.5cm}
% \begin{wrapfigure}{r}{.32\columnwidth}
%   \vspace{-.5cm} 
%   \centering
% \includegraphics[scale=.29]{figures/Var of l2l_2 loss.pdf}
%   \vspace{-0.2cm}
%   \caption{Results of GP}
% \label{fig:var-l2-gp}
%   \vspace{-0.1cm}
% \end{wrapfigure}


% Attempts have been made  in the past to address these  drawbacks heuristically  (see \citep{AggarwalKoGuHaPh14}). We give a unified computational framework while approaching the problem in a more principled manner and solving it more optimally.




\subsection{Planning with  neural network-based uncertainty quantification methods ($\ensembleplus$)}


We now provide a proof-of-concept that shows the generalizability of our conceptual framework  to the deep learning-based UQ modules, specifically focusing on $\ensembleplus$ due to their previously observed superior performance~\citep{OsbandWenAsDwIbLuRo23}. Recall that implementing our framework with deep learning-based UQ modules  requires us to retrain the model across multiple possible random actions $\bm{a}(\theta)$ sampled from the current policy $\pi_\theta$.
This requires significant computational resources, in sharp contrast to the GPs where the posteriors are in closed form and can be readily updated and differentiated. 

Due to the computational constraints, we test $\ensembleplus$ on a toy setting to demonstrate the generalizability of our framework. We consider a setting where the general population consists of four clusters, while the initial labeled data only comes from one cluster. Again we generate data using GPs.  The task is to select a batch of 2 points in one horizon. We detail the $\ensembleplus$ architecture in Section \ref{sec:details-experiments}, and we assume prior uncertainty to be large (depends on the scaling of the prior generating functions). 
The results are summarized in the Table~\ref{tab:UQ_ensemble}.

% \begin{table}[H]
% \vspace{-10pt}
% \caption{Performance under \ensembleplus as UQ module}
%     \centering
%     \begin{tabular}{|m{3cm}|m{2.5cm}|m{2cm}|} 
%     \hline
%       Algorithm   & Variance of $\loss_2$ loss estimate & Error of $\loss_2$ loss estimate  \\ \hline Random Sampling 
%          & $1710.9 \pm 1352.1$ & $8.67\pm6.62$ 
%       \\ \hline \ouralgo & $1.30 \pm 0.68$ & $0.91\pm0.25$ \\ \hline
%     \end{tabular}
%     \label{tab:UQ_ensemble}
%     %\vspace{-10pt}
% \end{table}




\begin{table}[h]
\vspace{-10pt}
\caption{Performance under \ensembleplus as the UQ module}
\centering
\begin{tabular}{|l|l|l|}
\hline
Algorithm   & Variance of $\loss_2$ loss estimate & Error of $\loss_2$ loss estimate  \\
\hline
\textsf{Random sampling} & 7129.8 $\pm$ 1027.0 & 136.2 $\pm$ 8.28 \\ \hline
\textsf{Uncertainty sampling (Static)} & 10852 $\pm$ 0.0 & 162.156 $\pm$ 0.0 \\ \hline
\textsf{Uncertainty sampling (Sequential)} & 8585.5 $\pm$ 898.9 & 144 $\pm$ 6.93 \\ \hline
\textsf{REINFORCE} & 1697.1 $\pm$ 0.0 & 45.27 $\pm$ 0.0 \\ \hline
\ouralgo & 1697.1 $\pm$ 0.0 & 45.27 $\pm$ 0.0 \\ \hline
\end{tabular}
%\caption{Comparison of different algorithms based on variance   and   error in $\ell_2$ loss estimation with Ensemble $+$ as the UQ module. Our results demonstrate that {\ouralgo} and REINFORCE outperformthe other active learning based heuristics, confirming the benefits of our MDP formulation for the adaptive labeling problem, as also demonstrated in Section 4.\\
%\footnotesize{Experimental details: We use Gaussian Processes as our data generating process, GP parameters are the same as in Section D.3.  The task is to select a batch of 2 points along one horizon.The marginal distribution $p_X$ has 4 \textit{non-overlapping} clusters. Initial data comes from one cluster, while pool and evaluation points comes from all the clusters. We have $20$ initial labeled data points, $10$ pool points, and $252$ evaluation points.  Training procedures are similar to the one in Section D.3.} }
\label{tab:UQ_ensemble}
\end{table}



% We faced  issues in scaling up these experiments which will be our focus in the future. 





% \begin{itemize}
%     \item Posteriors should be consistent. Two dimensions: even with less training,  
%     \item the inference should be  fast enough
% \end{itemize}


% Potential research directions for uncertainty quantification

% In this section we consider a simple setting We consider a simpler setting and 


% For synthetic dataset generation, we use ...... For real datasets, we use ...... We compare our methodolgy to several baselines ()    This Section is structured as follows:
% \begin{itemize}
%     \item \textbf{GPs, square loss objective} (Section \ref{}): 
%     %the broad aim of the experiments  in this section is to isolate the performance of our methodology without any concerns for the inefficiencies induced due to a mis-specified prior or imperfect posterior inference. To accomplish this we generate synthetic datasets using GPs (detailed later). We use the well specified prior (GPs - with same hyperparameter setting) as our UQ module.   
%      As GPs provide differentaible posterior inference - any errors induced due to imperfect posterior updates are also isolated. We note that under this setting
%      \item In Section\ref{} we demonstrate why our methodology performs better than other baselines - by devising various synthetic experiments ()
%     \item  \textbf{UQ Benchmarking }(Section \ref{}): Before diving into the experiments using $\ensembleplus$ and ENNs,  we showcase our benchmarking experiments in Section \ref{}. We use real datasets We observe that ENNs perform better
%      \item \textbf{Ensemble $+$}, objective: recall, accuracy
%     \item \textbf{ENN}, objective: recall, accuracy
% \end{itemize}




% In Section {}, we test 
% \subsection{Experimental details}

% \begin{itemize}
%     \item UQ methodologies - GPs, ENNs
%     \item Objectives - Recall,  ATE
%     \item Datasets - ATE-synthetic datasets, Recall-synthetic, real datasets
%     \item Baselines - 
%     \begin{itemize}
%         \item Random sampling
%         \item Active learning - Uncertainty based sampling - In regression setting almost all of the 
%         \item Myopic greedy - Greedy Batch based sampling
%         \item Policy Gradient
%     \end{itemize}
    
% \end{itemize}

% \subsection{Experiments}
%     \begin{itemize}
%     \item GPs with square loss
%     \item Benchmarking ENN
%         \item ENNs with ATE
%         \item ENNs with Recall
%     \end{itemize}

% \subsection{Benefits over other algorithms - intuition and experiments}

%Active learning - Myopic greedy / Don't rely on the objective rather some entropy version.


%%% Local Variables:
%%% mode: latex
%%% TeX-master: "main"
%%% End:

\section*{Conclusion}
This paper aims to enhance our understanding of the computational complexity of computing various Shapley value variants. We found that for various ML models --- including decision trees, regression tree ensembles, weighted automata, and linear regression --- both local and global interventional and baseline SHAP can be computed in polynomial time under HMM modeled distributions. This extends popular algorithms, such as TreeSHAP, beyond their empirical distributional scope. We also establish strict complexity gaps between the various SHAP variants (baseline, interventional, and conditional) and prove the intractability of computing SHAP for tree ensembles and neural networks in simplified scenarios. Overall, we present SHAP as a versatile framework whose complexity depends on four key factors: \begin{inparaenum}[(i)] \item model type, \item SHAP variant, \item distribution modeling approach, \item and local vs. global explanations\end{inparaenum}. We believe this perspective provides deeper insight into the computational complexity of SHAP, paving the way for future work.




%We believe that our framework provides a more intricate understanding of SHAP computation complexity across different models, distributions, and variants, paving the way for further research.

Our work opens promising directions for future research. First, expanding our computational analysis to other SHAP-related metrics, such as asymmetric SHAP~\citep{frye20} and SAGE~\citep{covert2020understanding}, would be valuable. Additionally, we aim to explore more expressive distribution classes and relaxed assumptions beyond those in Section \ref{sec:tractable} while maintaining tractable SHAP computation. Finally, when exact computation is intractable (Section \ref{sec:intractable}), investigating the approximability of SHAP metrics through approximation and parameterized complexity theory~\citep{downey2012parameterized} is an important direction.

%Our work opens several promising avenues for future research on the computational properties of explainable AI methods, with a particular focus on SHAP. First, it would be interesting to broaden the computational analysis conducted in this work to include other popular SHAP-related metrics in the literature, such as asymmetric SHAP \cite{frye20} and SAGE \cite{covert2020understanding}. Also, in the future, we aim to explore more expressive distribution classes and relaxed distributional assumptions—extending beyond those examined in Section \ref{sec:tractable} —that still yield tractable SHAP computation. Finally, when exact computation proves intractable (Section \ref{sec:intractable}), it is worthwhile to theoretically investigate the question of the approximability of computing the SHAP metrics across various configurations, through the lens of approximation and parametrized complexity theory \cite{arora2009computational}.

%This paper aims to deepen our understanding of the computational complexity involved in obtaining different Shapley value variants. We found that for a variety of ML models, including decision trees, tree ensembles for regression, weighted automata, and linear regression models — computing both local and global interventional and baseline SHAP can be done in polynomial time when distributions are modeled by HMMs. This extends the distributional scope of popular algorithms like TreeSHAP, which is limited to empirical distributions. Additionally, we demonstrate a strict complexity gap between SHAP variants, showing that interventional and baseline SHAP can be strictly easier to compute than conditional SHAP. Despite these positive results, we uncovered intractability for various SHAP variants in neural networks and tree ensembles. Finally, we provided generalized complexity relations across SHAP variants. We believe that our framework offers a deeper understanding of the complexity involved in computing SHAP across various variants, models, distributions, as well as in both local and global computations, laying the groundwork for future research.

%%
%% The acknowledgments section is defined using the "acks" environment
%% (and NOT an unnumbered section). This ensures the proper
%% identification of the section in the article metadata, and the
%% consistent spelling of the heading.

%%
%% The next two lines define the bibliography style to be used, and
%% the bibliography file.
\bibliographystyle{ACM-Reference-Format}
\bibliography{references,ryan-cites-long,s3}


%%
%% If your work has an appendix, this is the place to put it.
% \appendix

% \section{Research Methods}

\end{document}
\endinput
%%
%% End of file `sample-sigconf-authordraft.tex'.
