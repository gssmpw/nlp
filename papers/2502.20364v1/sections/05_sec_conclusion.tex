
% This work introduces a framework that leverages advanced computational techniques in the legal domain, combining RAG with KGs and dimensionality reduction methods. Integrating \textbf{T-ELF}, vector databases, and metadata into a Neo4j-based KG offers a structured and queryable representation of constitutional provisions, statutes, and case law, enabling efficient retrieval, thematic clustering, and latent feature discovery.

% The dimensionality reduction method \textit{NMFk} identifies latent topics and optimally clusters legal documents, avoiding overfitting or underfitting. Hierarchical decomposition enhances the framework by capturing nested relationships within legal texts, enabling context-aware queries and analysis.

% However, gaps, such as missing citation networks between cases, limit the KG’s traceability. Addressing these areas could improve support for precedent-based reasoning and legislative analysis. Further, the New Mexico administrative code and judicial rules, which were not collected or analyzed, would allow further legal insight into a well-functioning state government if combined into the system. 

% This framework bridges traditional legal research and computational tools, offering a scalable, interpretable solution for navigating large legal datasets. Combining semantic reasoning, KGs, and advanced factorization techniques, it lays a strong foundation for data-driven legal research and decision-making.


This work introduced Smart-Slic, a generative AI framework tailored to the legal domain, leveraging RAG, vector stores, and a Neo4j-based knowledge graph constructed through \textit{NMFk}. Our approach uses \textbf{T-ELF} with metadata and chunking strategies to capture fine-grained H-clusters and minimize hallucinations to improve reliability. By bridging structured and unstructured data--spanning constitutional provisions, statutes, and case law--, our system supports advanced semantic reasoning and dimensionally reduced insight into the latent structure of legal texts.

Experimental results across multiple retrieval strategies show that chunking, combined with hierarchical \textit{NMFk}, improves accuracy, particularly for large, unstructured case law datasets. Short, highly structured documents, constitutional provisions, and statute sections benefit from minimal chunking, revealing the importance of aligning preprocessing approaches with data characteristics. We demonstrated the framework’s capability to derive interpretive legal topics and precisely answer domain-specific queries by harnessing topically segmented embeddings and explicit links within the knowledge graph.

Despite the positive results, challenges remain. Author attribution in networks is incomplete, limiting the knowledge graph’s potential for thorough precedent tracing and interlinking. Including additional datasets—such as administrative codes and judicial rules—would provide richer context and increase the system’s coverage of a functioning state government. Moreover, systematically reconciling informal post-decree agreements with formal judgments is needed to model a more acute legal flow.

This framework marks a step forward in computational law and legal AI, demonstrating a scalable, interpretable method for discovering, retrieving, and reasoning over complex legal corpora. Combining semantic embeddings, latent topic modeling, and knowledge graphs contributes to more robust, data-driven legal research pathways. Future directions include refining the citation extraction pipeline, expanding the collection to encompass broader legal instruments, and applying advanced LLM-driven reasoning for deeper precedent analysis and trend prediction.
