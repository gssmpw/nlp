Laws and their interpretations are limited in application to their respective jurisdictions. This work selected the State of New Mexico, using the available Supreme Court and Court of Appeals case law, the state constitution, and state statutes as the main data resource.

\subsection{Data Collection}
To compile a research-oriented corpus of legal documents, we utilized 
Justia's publicly accessible resources, in full compliance with their Terms of Service
and any applicable usage guidelines, including robots.txt. 
We developed a \emph{restricted} and \emph{responsible} crawler to 
 only access permitted documents---namely statutes, constitutional 
provisions, and case law in the public domain---and to avoid any sections or usage not authorized by Justia's policies. To minimize server load and respect rate limits, our scraper introduces time delays between requests,  monitors for HTTP status codes indicating rate-limiting or errors, and uses exponential backoff to handle potential network disruptions gracefully.

From the main landing page, a dynamic crawler identifies only those hyperlinks consistent with the site's navigation structure and relevant public legal content, converting them to absolute URLs and recursively traversing nested content where permitted. Our process extracts metadata such as chapter names, article titles, section numbers, full text for statutes, constitutional provisions, and judicial opinions for New Mexico's Court of Appeals and Supreme Court cases.

Throughout this procedure, we take multiple measures to maintain compliance and transparency:
\begin{itemize}
    \item \textbf{Logs} track processed URLs to prevent duplication, resume 
    interrupted crawls responsibly, and record request timing.
    \item \textbf{Rate-limiting measures} ensure we do not exceed usage 
    thresholds or impose an undue burden on Justia's servers.
    \item \textbf{No circumvention} of technical safeguards defined in\\ robots.txt and the TOS restrictions.
    \item \textbf{Non-commercial use only}: we do not republish, sell, or 
    otherwise exploit Justia's content and only use it for legitimate scholarly research.
\end{itemize}

These practices keep the data process ethical, legally sound, and consistent with Justia's rules for site usage and content retrieval.



\subsection{Dimension Reduction in Legal Texts}
Legal documents- constitutions, statutes, and case law- are traditionally organized into chapters, articles, and sections, but these structures do not always match the latent patterns revealed through factorization. Each document type is analyzed separately using non-negative tensor and matrix factorization to explore these hidden relationships. A TF-IDF matrix, \( \mathbf{X} \), is first constructed from the cleaned corpus. Constitutional provisions, statutory clauses, and case law paragraphs form the units of analysis for clustering.

In this study, we use the public repository called \textbf{Tensor Extraction of Latent Features} (\textbf{T-ELF})\footnote{\url{https://github.com/lanl/T-ELF}} \cite{TELF}, combined with automatic model selection, to decompose \( \mathbf{X} \) into coherent topic H-clusters. \textbf{T-ELF} efficiently identifies latent topics, grouping constitutional provisions around themes such as “separation of powers” and clustering statutes and case law based on regulatory objectives and recurring legal doctrines, respectively. 
% The optimal number of clusters, \( k \), is determined using silhouette scores and is accelerated through a binary search adaption \cite{barron2024binarybleedfastdistributed}.

\subsubsection{Application of Non-Negative Matrix Factorization to Legal Texts}

NMF approximates the matrix \( \mathbf{X} \) with two non-negative matrices, \( \mathbf{W} \) and \( \mathbf{H} \):

\[
\mathbf{X} \approx \mathbf{W} \cdot \mathbf{H},
\]

where \( \mathbf{W} \) describes how terms distribute over topics, and \( \mathbf{H} \) describes how these topics distribute across documents. Constitutional articles and sections reveal underlying governance or civil rights themes; statute clauses highlight regulatory objectives, and segmented judicial opinions expose common doctrines and legal reasoning patterns.

\subsubsection{Automatic Model Determination Using \textit{NMFk}}

A central challenge in applying NMF is selecting the best number of latent features (\( k \)). We use  \textit{NMFk}~\cite{alexandrov2020patent}, which combines clustering stability with reconstruction accuracy. Bootstrap resampling generates slightly perturbed versions of the original matrix, and repeated decompositions measure how consistently clusters form. Silhouette scores help ensure cohesive, well-separated topics, while reconstruction error verifies that the model effectively captures patterns in the original data.

By adapting this NMF approach to a hierarchical setting, legal texts can be organized into a tree-like structure. Constitutions may be segmented into articles and sections, statutes into chapters and clauses, and case law into layered precedents and sub-issues. This hierarchical perspective enhances the discovery of latent relationships at multiple levels of granularity, facilitating deeper analyses of large-scale legal corpora.

\begin{figure}[h!]
    \vspace{-.9em}
    \centering
    \includegraphics[width=.7\columnwidth]{figs/law_schema.pdf}
    \vspace{-1em}
    \caption{Knowledge Graph Schema, with the primary identifier in bold and attributes in brackets.}
    \label{fig:law_schema}
\end{figure}
\vspace{-1em}

\subsection{Knowledge Graph}


Features derived from \textbf{T-ELF} and document metadata are transformed into a series of head, entity, and tail relations, forming directional triplets integrated into a Neo4j KG \cite{neo4j2023}.

In the legal context, the KG incorporates metadata and latent features extracted from constitutions, statutes, and case law. The primary nodes in the graph represent legal documents, including constitutional provisions, statutory sections, and judicial opinions. These nodes are enriched with metadata such as titles, hierarchical identifiers (e.g., chapter and section numbers), jurisdiction, court names, decision dates, and topics derived from latent features. 

Edges in the KG establish relationships between nodes to represent the interconnected nature of legal documents. For instance:

\begin{itemize}
    \item \textbf{Constitutional Nodes:} Linked to statutory provisions or judicial interpretations that reference or rely on specific constitutional clauses.
    \item \textbf{Statutory Nodes:} Connected to cases interpreting the statute or related provisions within the same legislative framework.
    \item  \textbf{Case Law Nodes:} Interlinked based on shared topics, common legal principles, or hierarchical relationships in appellate decisions.
\end{itemize}

This graph structure enables the RAG system to query and retrieve legal documents based on semantic similarity and explicit relationships. For example, a query about "due process" might retrieve the relevant constitutional clause, cases that discuss its interpretation, and statutory provisions impacted by those rulings. By combining metadata and latent features, the KG supports advanced reasoning and logical traversal, enhancing the precision and depth of legal information retrieval.

Each part had decomposition topics containing keywords that connected to the documents. The bag of words (BOW) vocabulary was also extracted and inserted into the knowledge graph. These can be observed in the knowledge graph schema of Figure \ref{fig:law_schema}. 

\subsection{Vector Store}
A vector database was implemented to manage and index legal documents, improving the RAG process for legal research. Using Milvus~\cite{2021milvus}, the system stores vectorized representations of constitutions, statutes, and case law, treating each document type uniquely to ensure contextually relevant retrieval.

Constitutional provisions are segmented into paragraphs, each assigned a unique ID, and vectorized using OpenAI’s text-embedding-ada-002~\cite{openai_api} model for granular semantic searches. Statutes are divided by sections or clauses, with metadata like chapter numbers and section titles integrated for precise retrieval. Case law, due to its unstructured narrative format, is chunked into meaningful units, preserving logical flow and indexed with metadata such as case name and citation.

The RAG application queries the database to retrieve relevant fragments—constitutional paragraphs, statutory clauses, or case law sections—based on the query's focus. Retrieved text is processed by the LLM agents with custom prompts to construct responses, allowing the LLM to cite specific paragraphs or clauses, for traceability.

For additional context, the system leverages a connected knowledge graph to explore related statutes, judicial interpretations, or precedent cases.



\begin{figure}[ht]
    \vspace{-.9em}
    \centering
    \includegraphics[width=.8\columnwidth]{figs/court_cases_over_years.pdf}
    \vspace{-1.5em}
    \caption{New Mexico Supreme/Appeals case counts per year.}
    \vspace{-2em}
    \label{fig:court_cases}
\end{figure}
