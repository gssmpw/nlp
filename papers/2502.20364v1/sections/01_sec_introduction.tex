 The legal domain is uniquely complex, encompassing constitutions, statutes, court rules, regulations, ordinances, and case law. Each data type follows a different structural or organizational logic—constitutions and statutes often contain hierarchical elements, whereas case law typically consists of lengthy, unstructured opinions. As new legal texts are continually produced, practitioners and scholars require computational systems that move beyond simplistic keyword-based searches to deliver meaningful results. Traditional legal information retrieval methods, relying predominantly on Boolean logic \cite{westermann2021computer}  and lexical indexing (TF-IDF)\cite{bithel2021unsupervised}, frequently miss the subtle conceptual overlaps and deep contextual cues that characterize legal inquiries.

 In response, RAG has become a framework that unites the strengths of information retrieval and generative models. RAG systems retrieve relevant legal documents or data points and then employ a language model to synthesize that information into coherent, contextually grounded answers. When applied to the legal domain, RAG can help reduce issues experienced by LLMS, like hallucinations \cite{ji2023survey, lewis2020retrieval}. This is done by rooting responses in reliable texts, which then improves tasks such as case law retrieval, statutory interpretation, and document summarization. In order to fulfill these benefits, an underlying infrastructure must effectively represent, retrieve, and interpret large volumes of legal data.

This paper addresses these requirements by integrating three core technologies: \begin{itemize} 
    \item \textbf{VS:} By embedding legal texts into dense vector representations (from models like BERT \cite{devlin2019bert} or GPT \cite{Radford2018ImprovingLU, NEURIPS2020_1457c0d6}), our system encompasses semantic meanings beyond keyword matching. This helps users to locate relevant documents even when the exact search terms differ from the query’s language. 
    
    \item \textbf{ KG:} Legal concepts—such as statutes, precedential cases, and doctrines—are often interconnected through citations or shared legal principles. KGs formalize these relationships, enabling structured navigation of the domain, explicit linking of related legal authorities, and improves reasoning over case-to-case, case-to-statute, and statute-to-statute references. 
    \item \textbf{NMF:} While KGs excel at mapping explicit legal connections, many latent topics and patterns remain hidden in unstructured text. NMF uncovers these by factorizing word-embedding matrices into interpretable topics, aiding tasks like clustering related cases or identifying new legal trends 
    \end{itemize}

\noindent By combining the strengths of the thee components into a domain-specific RAG system, we aim to deliver insights and more reliable outputs for a wide spectrum of legal tasks. The integrated framework leverages the high-recall semantic matching of vector stores, the structured relationships codified in knowledge graphs, and the topic discovery capabilities of NMF. This combination not only improves retrieval quality but also strengthens the system’s ability to provide explainable reasoning over extremely large datasets.