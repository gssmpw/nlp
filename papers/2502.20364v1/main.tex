\PassOptionsToPackage{table,dvipsnames}{xcolor}
\documentclass[sigconf]{acmart}

\usepackage{enumitem}

\setlength{\marginparwidth}{2cm}
% \usepackage{outlines}
% \usepackage{todonotes}
% \usepackage{xcolor}
\usepackage[mathscr]{eucal}
\usepackage{bm}
\usepackage{algorithm}
\usepackage{subcaption} 
\usepackage{textcomp}
% \usepackage[table]{xcolor} % Enable xcolor for table coloring
\usepackage{array}         % For column width definitions

\NewDocumentCommand{\vect}{ O{} O{} m }{\mathbf{#3}\ifthenelse{\isempty{#1}}{}{^{(#1)}}\ifthenelse{\isempty{#2}}{}{_{#2}}}

\NewDocumentCommand{\mat}{ O{} O{} m }{\mathbf{#3}\ifthenelse{\isempty{#1}}{}{^{(#1)}}\ifthenelse{\isempty{#2}}{}{_{#2}}}

\NewDocumentCommand{\ten}{ O{} O{} m }{\pmb{\mathscr{#3}}\ifthenelse{\isempty{#1}}{}{^{(#1)}}\ifthenelse{\isempty{#2}}{}{_{#2}}}

% added for commenting within doc
\newcommand{\selma}[1]{{\color{violet}[\textsc{Selma:} #1]}}

%%
%% \BibTeX command to typeset BibTeX logo in the docs
\AtBeginDocument{%
  \providecommand\BibTeX{{%
    Bib\TeX}}}

%% Rights management information.  This information is sent to you
%% when you complete the rights form.  These commands have SAMPLE
%% values in them; it is your responsibility as an author to replace
%% the commands and values with those provided to you when you
%% complete the rights form.
\setcopyright{acmlicensed}
\copyrightyear{2025}
\acmYear{2025}
\acmDOI{XXXXXXX.XXXXXXX}

%% These commands are for a PROCEEDINGS abstract or paper.
\acmConference[ICAIL 2025]{The 20th International Conference on Artificial Intelligence and Law}{June 16--20,
  2024}{Chicago, IL, USA}
  
\acmISBN{978-1-4503-XXXX-X/18/06}


%%
%% Submission ID.
%% Use this when submitting an article to a sponsored event. You'll
%% receive a unique submission ID from the organizers
%% of the event, and this ID should be used as the parameter to this command.
%%\acmSubmissionID{123-A56-BU3}

%%
%% For managing citations, it is recommended to use bibliography
%% files in BibTeX format.
%%
%% You can then either use BibTeX with the ACM-Reference-Format style,
%% or BibLaTeX with the acmnumeric or acmauthoryear sytles, that include
%% support for advanced citation of software artefact from the
%% biblatex-software package, also separately available on CTAN.
%%
%% Look at the sample-*-biblatex.tex files for templates showcasing
%% the biblatex styles.
%%

%%
%% The majority of ACM publications use numbered citations and
%% references.  The command \citestyle{authoryear} switches to the
%% "author year" style.
%%
%% If you are preparing content for an event
%% sponsored by ACM SIGGRAPH, you must use the "author year" style of
%% citations and references.
%% Uncommenting
%% the next command will enable that style.
%%\citestyle{acmauthoryear}


%%
%% end of the preamble, start of the body of the document source.
\begin{document}

%%
%% The "title" command has an optional parameter,
%% allowing the author to define a "short title" to be used in page headers.
\title{Bridging Legal Knowledge and AI: \\Retrieval-Augmented Generation with Vector Stores, Knowledge Graphs, and Hierarchical Non-negative Matrix Factorization} 

% \author{Anonymous Author(s), Paper ID: 170}
 
\author{Ryan C. Barron}
\email{barron@lanl.gov}
\affiliation{
    \institution{Theoretical Division, \\ Los Alamos National Laboratory}
    \city{Los Alamos}
    \state{New Mexico}
    \country{USA}
}
\affiliation{
    \institution{CSEE, UMBC}
    \city{Baltimore}
    \state{Maryland}
    \country{USA}
}

\author{Maksim E. Eren}
\email{maksim@lanl.gov}
\affiliation{
    \institution{Advanced Research in Cyber Systems, Los Alamos National Laboratory}
    \city{Los Alamos}
    \state{New Mexico}
    \country{USA}
}

\author{Olga M. Serafimova}
\email{OMSerafimova@hollandhart.com}
\affiliation{
    \institution{Holland \& Hart LLP}
    \city{Santa Fe}
    \state{New Mexico}
    \country{USA}
}


\author{Cynthia Matuszek}
\email{cmat@umbc.edu}
\affiliation{
    \institution{CSEE, UMBC}
    \city{Baltimore}
    \state{Maryland}
    \country{USA}
}


\author{Boian S. Alexandrov}
\email{boian@lanl.gov}
\affiliation{
    \institution{Theoretical Division, \\Los Alamos National Laboratory}
    \city{Los Alamos}
    \state{New Mexico}
    \country{USA}
}

%%
%% By default, the full list of authors will be used in the page
%% headers. Often, this list is too long, and will overlap
%% other information printed in the page headers. This command allows
%% the author to define a more concise list
%% of authors' names for this purpose.
\renewcommand{\shortauthors}{Barron et al.}
\begin{abstract}
Agentic Generative AI, powered by Large Language Models (LLMs) and enhanced with Retrieval-Augmented Generation (RAG), Knowledge Graphs (KGs), and Vector Stores (VSs), represents a transformative technology applicable across specialized domains such as legal systems, research, recommender systems, cybersecurity, and global security, including proliferation research. This technology excels at inferring relationships within vast unstructured or semi-structured datasets.
The legal domain we focus on here comprises inherently complex data characterized by extensive, interrelated, and semi-structured knowledge systems with complex relations. It comprises constitutions, statutes, regulations, and case law. Extracting insights and navigating the intricate networks of legal documents and their relations is crucial for effective legal research and decision-making. Here, we introduce a generative AI system that integrates RAG, VS, and KG, constructed via Non-Negative Matrix Factorization (NMF), to enhance legal information retrieval and AI reasoning and minimize hallucinations. In the legal system, these technologies empower AI agents to identify and analyze complex connections among cases, statutes, and legal precedents, uncovering hidden relationships and predicting legal trends—challenging tasks that are essential for ensuring justice and improving operational efficiency. Our system employs web scraping techniques to systematically collect legal texts, such as statutes, constitutional provisions, and case law, from publicly accessible platforms like Justia. %\cite{justia}. 
It bridges the gap between traditional keyword-based searches and contextual understanding by leveraging advanced semantic representations, hierarchical relationships, and latent topic discovery.
This approach is demonstrated in legal document clustering, summarization, and cross-referencing tasks. The framework marks a significant step toward augmenting legal research with scalable, interpretable, and accurate retrieval methods for semi-structured data, advancing the intersection of computational law and artificial intelligence.

\end{abstract}

%%
%% The code below is generated by the tool at http://dl.acm.org/ccs.cfm.
%% Please copy and paste the code instead of the example below.
%%
% CCS Concept code for ACM
\begin{CCSXML}
<ccs2012>
   <concept>
       <concept_id>10010147.10010178.10010179</concept_id>
       <concept_desc>Computing methodologies~Natural language processing</concept_desc>
       <concept_significance>500</concept_significance>
       </concept>
   <concept>
       <concept_id>10010405.10010455.10010458</concept_id>
       <concept_desc>Applied computing~Law</concept_desc>
       <concept_significance>500</concept_significance>
       </concept>
   <concept>
       <concept_id>10010147.10010178.10010187.10010198</concept_id>
       <concept_desc>Computing methodologies~Reasoning about belief and knowledge</concept_desc>
       <concept_significance>300</concept_significance>
       </concept>
   <concept>
       <concept_id>10010147.10010178.10010187.10010195</concept_id>
       <concept_desc>Computing methodologies~Ontology engineering</concept_desc>
       <concept_significance>300</concept_significance>
       </concept>
   <concept>
       <concept_id>10002951.10003317.10003318.10003320</concept_id>
       <concept_desc>Information systems~Document topic models</concept_desc>
       <concept_significance>500</concept_significance>
       </concept>
   <concept>
       <concept_id>10002951.10003317.10003318.10003321</concept_id>
       <concept_desc>Information systems~Content analysis and feature selection</concept_desc>
       <concept_significance>500</concept_significance>
       </concept>
   <concept>
       <concept_id>10002951.10003317.10003318.10011147</concept_id>
       <concept_desc>Information systems~Ontologies</concept_desc>
       <concept_significance>500</concept_significance>
       </concept>
   <concept>
       <concept_id>10002951.10003317.10003318.10003324</concept_id>
       <concept_desc>Information systems~Document collection models</concept_desc>
       <concept_significance>500</concept_significance>
       </concept>
 </ccs2012>
\end{CCSXML}

\ccsdesc[500]{Computing methodologies~Natural language processing}
\ccsdesc[500]{Applied computing~Law}
\ccsdesc[300]{Computing methodologies~Reasoning about belief and knowledge}
\ccsdesc[300]{Computing methodologies~Ontology engineering}
\ccsdesc[500]{Information systems~Document topic models}
\ccsdesc[500]{Information systems~Content analysis and feature selection}
\ccsdesc[300]{Information systems~Ontologies}
\ccsdesc[300]{Information systems~Document collection models}

%%
%% Keywords. The author(s) should pick words that accurately describe
%% the work being presented. Separate the keywords with commas.
\keywords{law, legal knowledge,  nmf, topic labeling, llm, chain of thought, prompt tuning,information retrieval}

% \received{20 February 2007}
% \received[revised]{12 March 2009}
% \received[accepted]{5 June 2009}

%%
%% This command processes the author and affiliation and title
%% information and builds the first part of the formatted document.
\maketitle

%% Document sections
\section{Introduction}
\label{sec:introduction}
 The legal domain is uniquely complex, encompassing constitutions, statutes, court rules, regulations, ordinances, and case law. Each data type follows a different structural or organizational logic—constitutions and statutes often contain hierarchical elements, whereas case law typically consists of lengthy, unstructured opinions. As new legal texts are continually produced, practitioners and scholars require computational systems that move beyond simplistic keyword-based searches to deliver meaningful results. Traditional legal information retrieval methods, relying predominantly on Boolean logic \cite{westermann2021computer}  and lexical indexing (TF-IDF)\cite{bithel2021unsupervised}, frequently miss the subtle conceptual overlaps and deep contextual cues that characterize legal inquiries.

 In response, RAG has become a framework that unites the strengths of information retrieval and generative models. RAG systems retrieve relevant legal documents or data points and then employ a language model to synthesize that information into coherent, contextually grounded answers. When applied to the legal domain, RAG can help reduce issues experienced by LLMS, like hallucinations \cite{ji2023survey, lewis2020retrieval}. This is done by rooting responses in reliable texts, which then improves tasks such as case law retrieval, statutory interpretation, and document summarization. In order to fulfill these benefits, an underlying infrastructure must effectively represent, retrieve, and interpret large volumes of legal data.

This paper addresses these requirements by integrating three core technologies: \begin{itemize} 
    \item \textbf{VS:} By embedding legal texts into dense vector representations (from models like BERT \cite{devlin2019bert} or GPT \cite{Radford2018ImprovingLU, NEURIPS2020_1457c0d6}), our system encompasses semantic meanings beyond keyword matching. This helps users to locate relevant documents even when the exact search terms differ from the query’s language. 
    
    \item \textbf{ KG:} Legal concepts—such as statutes, precedential cases, and doctrines—are often interconnected through citations or shared legal principles. KGs formalize these relationships, enabling structured navigation of the domain, explicit linking of related legal authorities, and improves reasoning over case-to-case, case-to-statute, and statute-to-statute references. 
    \item \textbf{NMF:} While KGs excel at mapping explicit legal connections, many latent topics and patterns remain hidden in unstructured text. NMF uncovers these by factorizing word-embedding matrices into interpretable topics, aiding tasks like clustering related cases or identifying new legal trends 
    \end{itemize}

\noindent By combining the strengths of the thee components into a domain-specific RAG system, we aim to deliver insights and more reliable outputs for a wide spectrum of legal tasks. The integrated framework leverages the high-recall semantic matching of vector stores, the structured relationships codified in knowledge graphs, and the topic discovery capabilities of NMF. This combination not only improves retrieval quality but also strengthens the system’s ability to provide explainable reasoning over extremely large datasets.

\section{Relevant Work}
\label{sec:relevant_work}
This section reviews contributions across RAG domains, semantic search using vector embeddings, knowledge graph construction, NMF, and legal information systems.

\subsection{Non-Negative Matrix Factorization for Pattern Discovery}
NMF is a dimensionality reduction technique used to uncover latent patterns in data. \cite{lee2001algorithms} analyzed NMF as an interpretable method for extracting features and topics from large datasets, specifically highlighting its ability to identify meaningful and non-overlapping components. Building on this work, \cite{hoyer2004nonnegative} introduced sparseness constraints for better interpretability,  improving applications of NMF in real-world scenarios through more focused feature selection. In the legal domain, NMF has been valuable for analyzing complex textual data, such as case law and statutes, and assisting with topic discovery and clustering. For instance, \cite{Budahazy2021AnalysisOL} applied NMF to legal documents to extract latent topics and visualize relationships, demonstrating how NMF’s interpretable structure aids researchers in identifying underlying topics not readily apparent in raw text. They even applied NMF hierarchically to find fine-grained topics, however did not have a way to approximate the number of clusters automatically at each decomposition. More recently, \cite{Li2022GuidedSN} proposed a guided semi-supervised NMF approach for topic discovery in legal documents, using domain knowledge to steer factorization and ensure the extracted topics remain highly relevant. This semi-supervised extension bridges the gap between fully automated unsupervised techniques and expert-driven analysis.  NMF’s utility in legal contexts is particularly significant, providing an interpretable framework for analyzing large textual datasets.

\subsection{Retrieval-Augmented Generation}
RAG has emerged as a foundational approach for improving AI systems across various domains, including law. For instance, \cite{lewis2020retrieval} introduced a framework that dynamically retrieves relevant documents to ground generative outputs, achieving notable gains in accuracy. Building on this idea, \cite{guu2020realm} proposed a retrieval-augmented pretraining method that integrates external knowledge for improved downstream task performance. In contrast, \cite{izacard2021leveraging} demonstrated the effectiveness of retrieval in open-domain question answering. These advances lay the groundwork for applying RAG to the legal sector, where the method’s ability to ground LLMs in authoritative texts reduces hallucinations and increases accuracy in tasks such as case law retrieval, statutory reasoning, and judgment prediction. Notable examples include CBR-RAG, which incorporates Case-Based Reasoning to structure retrieval for legal QA \cite{10.1007/978-3-031-63646-2_29}, and LegalBench-RAG. This benchmark suite tailors evaluation metrics to the demands of legal information synthesis \cite{pipitone2024legalbenchragbenchmarkretrievalaugmentedgeneration}. Parallel work has demonstrated RAG’s capabilities in other domains, such as malware data analysis, by combining embeddings, KGs, and NMF \cite{barron2024domainspecificretrievalaugmentedgenerationusing}. Other works show how LLMs can dynamically decide when and what to retrieve to improve legal reasoning \cite{asai2023retrieve} and how multi-step legal judgment prediction can benefit from iterative retrieval and generation \cite{cui2023multistep}, further demonstrating th effectiveness of combining retrieval strategies with LLMs.


\subsection{Semantic Search with Vector Embeddings}
Semantic search operates on dense vector representations to find the deeper semantic relationships in texts, going beyond keyword matching to proper context retrieval \cite{karpukhin2020dense, guu2020realm}. This search is especially valuable in the legal domain, where queries often demand conceptual rather than surface-level understanding. Domain-specific pretraining has become increasingly important, as in LEGAL-BERT \cite{chalkidis2020legalbert}, outperforming general-purpose models by better capturing legal language nuances. Benchmarks like LeCaRD \cite{ma2021lecard} show the effectiveness of dense retrievers—such as SBERT—in legal case retrieval of Chinese law and the abilities of hybrid approaches that integrate both lexical and dense retrieval methods. Hierarchical transformer architectures \cite{xiao2021hierarchical} and long-context models like Longformer \cite{beltagy2020longformer} further address the challenges of lengthy legal documents. In contrast, citation-driven approaches such as SPECTER \cite{cohan2020specter} leverage metadata to improve retrieval.
Furthermore, work on neighborhood contrastive learning for scientific document representations (SciNCL) \cite{ostendorff2022neighborhood} demonstrates how controlled nearest neighbor sampling over citation graph embeddings can provide continuous similarity. This strategy could also inform citation-based retrieval improvements in legal domains. Challenges remain around scaling, explainability, and incorporating heterogeneous data sources in real-world legal workflows.

\subsection{Knowledge Graphs in Legal Research}
In the legal domain, researchers have used KGs for knowledge extraction \cite{Sovrano2020LegalKE}, constructed domain-specific graphs tailored to legal contexts \cite{Dong2021AKG}, and explored their potential for addressing trust, privacy, and transparency concerns \cite{a9b58e3cf6ea4306b01b44386cbf9c1f}. Recent studies have demonstrated the effectiveness of KGs in recommending similar legal cases \cite{dhani2024similarcasesrecommendationusing}, linking case law with statutes for improved retrieval \cite{zhou2024automaticknowledgegraphconstruction}, and using graphs to enhance legal case law search in Chinese legal systems \cite{BI2022109046}. Beyond retrieval, KGs have been incorporated into knowledge-aware machine reading systems for legal question answering \cite{long2021kalm}. Collectively, these works demonstrate KGs' capacity to represent intricate legal concepts, structure regulatory frameworks, and encourage reasoning over legal data.

\subsection{Legal Systems and Case Retrieval}
 
Legal information systems have evolved rapidly with the advent of neural architectures and hybrid retrieval pipelines, enabling precise tasks such as precedent retrieval, statute matching, and judgment prediction \cite{chalkidis2019neural, robaldo2020neural, chalkidis2021paragraph}. Benchmarks like LeCaRD \cite{ma2021lecard} and LexGLUE \cite{chalkidis2022lexglue}, along with LEGAL-BERT-based systems \cite{chalkidis2020legalbert}, have demonstrated the capability of these neural methods to improve accuracy in analyzing legal corpora. In particular, researchers have leveraged structured reasoning with transformers and graph representations to link statutes and precedents, as evidenced in the COLIEE competition \cite{rabelo2021coliee}. Nonetheless, several limitations remain: data scarcity and jurisdictional bias continue to restrict the generalizability of such models. At the same time, resource-intensive retrievers like BERT-based cross-encoders \cite{shao2022bert} have challenges scaling to large-scale legal databases. Earlier works in juris-informatics have highlighted the potential of automating legal reasoning and document analysis \cite{ashley2017artificial}, laying the groundwork for modern approaches that fuse knowledge graphs, transformers, vector stores, and agent-oriented RAG pipelines to deliver more explainable and efficient legal research workflows.



\section{Methods}
\label{sec:methods}
Laws and their interpretations are limited in application to their respective jurisdictions. This work selected the State of New Mexico, using the available Supreme Court and Court of Appeals case law, the state constitution, and state statutes as the main data resource.

\subsection{Data Collection}
To compile a research-oriented corpus of legal documents, we utilized 
Justia's publicly accessible resources, in full compliance with their Terms of Service
and any applicable usage guidelines, including robots.txt. 
We developed a \emph{restricted} and \emph{responsible} crawler to 
 only access permitted documents---namely statutes, constitutional 
provisions, and case law in the public domain---and to avoid any sections or usage not authorized by Justia's policies. To minimize server load and respect rate limits, our scraper introduces time delays between requests,  monitors for HTTP status codes indicating rate-limiting or errors, and uses exponential backoff to handle potential network disruptions gracefully.

From the main landing page, a dynamic crawler identifies only those hyperlinks consistent with the site's navigation structure and relevant public legal content, converting them to absolute URLs and recursively traversing nested content where permitted. Our process extracts metadata such as chapter names, article titles, section numbers, full text for statutes, constitutional provisions, and judicial opinions for New Mexico's Court of Appeals and Supreme Court cases.

Throughout this procedure, we take multiple measures to maintain compliance and transparency:
\begin{itemize}
    \item \textbf{Logs} track processed URLs to prevent duplication, resume 
    interrupted crawls responsibly, and record request timing.
    \item \textbf{Rate-limiting measures} ensure we do not exceed usage 
    thresholds or impose an undue burden on Justia's servers.
    \item \textbf{No circumvention} of technical safeguards defined in\\ robots.txt and the TOS restrictions.
    \item \textbf{Non-commercial use only}: we do not republish, sell, or 
    otherwise exploit Justia's content and only use it for legitimate scholarly research.
\end{itemize}

These practices keep the data process ethical, legally sound, and consistent with Justia's rules for site usage and content retrieval.



\subsection{Dimension Reduction in Legal Texts}
Legal documents- constitutions, statutes, and case law- are traditionally organized into chapters, articles, and sections, but these structures do not always match the latent patterns revealed through factorization. Each document type is analyzed separately using non-negative tensor and matrix factorization to explore these hidden relationships. A TF-IDF matrix, \( \mathbf{X} \), is first constructed from the cleaned corpus. Constitutional provisions, statutory clauses, and case law paragraphs form the units of analysis for clustering.

In this study, we use the public repository called \textbf{Tensor Extraction of Latent Features} (\textbf{T-ELF})\footnote{\url{https://github.com/lanl/T-ELF}} \cite{TELF}, combined with automatic model selection, to decompose \( \mathbf{X} \) into coherent topic H-clusters. \textbf{T-ELF} efficiently identifies latent topics, grouping constitutional provisions around themes such as “separation of powers” and clustering statutes and case law based on regulatory objectives and recurring legal doctrines, respectively. 
% The optimal number of clusters, \( k \), is determined using silhouette scores and is accelerated through a binary search adaption \cite{barron2024binarybleedfastdistributed}.

\subsubsection{Application of Non-Negative Matrix Factorization to Legal Texts}

NMF approximates the matrix \( \mathbf{X} \) with two non-negative matrices, \( \mathbf{W} \) and \( \mathbf{H} \):

\[
\mathbf{X} \approx \mathbf{W} \cdot \mathbf{H},
\]

where \( \mathbf{W} \) describes how terms distribute over topics, and \( \mathbf{H} \) describes how these topics distribute across documents. Constitutional articles and sections reveal underlying governance or civil rights themes; statute clauses highlight regulatory objectives, and segmented judicial opinions expose common doctrines and legal reasoning patterns.

\subsubsection{Automatic Model Determination Using \textit{NMFk}}

A central challenge in applying NMF is selecting the best number of latent features (\( k \)). We use  \textit{NMFk}~\cite{alexandrov2020patent}, which combines clustering stability with reconstruction accuracy. Bootstrap resampling generates slightly perturbed versions of the original matrix, and repeated decompositions measure how consistently clusters form. Silhouette scores help ensure cohesive, well-separated topics, while reconstruction error verifies that the model effectively captures patterns in the original data.

By adapting this NMF approach to a hierarchical setting, legal texts can be organized into a tree-like structure. Constitutions may be segmented into articles and sections, statutes into chapters and clauses, and case law into layered precedents and sub-issues. This hierarchical perspective enhances the discovery of latent relationships at multiple levels of granularity, facilitating deeper analyses of large-scale legal corpora.

\begin{figure}[h!]
    \vspace{-.9em}
    \centering
    \includegraphics[width=.7\columnwidth]{figs/law_schema.pdf}
    \vspace{-1em}
    \caption{Knowledge Graph Schema, with the primary identifier in bold and attributes in brackets.}
    \label{fig:law_schema}
\end{figure}
\vspace{-1em}

\subsection{Knowledge Graph}


Features derived from \textbf{T-ELF} and document metadata are transformed into a series of head, entity, and tail relations, forming directional triplets integrated into a Neo4j KG \cite{neo4j2023}.

In the legal context, the KG incorporates metadata and latent features extracted from constitutions, statutes, and case law. The primary nodes in the graph represent legal documents, including constitutional provisions, statutory sections, and judicial opinions. These nodes are enriched with metadata such as titles, hierarchical identifiers (e.g., chapter and section numbers), jurisdiction, court names, decision dates, and topics derived from latent features. 

Edges in the KG establish relationships between nodes to represent the interconnected nature of legal documents. For instance:

\begin{itemize}
    \item \textbf{Constitutional Nodes:} Linked to statutory provisions or judicial interpretations that reference or rely on specific constitutional clauses.
    \item \textbf{Statutory Nodes:} Connected to cases interpreting the statute or related provisions within the same legislative framework.
    \item  \textbf{Case Law Nodes:} Interlinked based on shared topics, common legal principles, or hierarchical relationships in appellate decisions.
\end{itemize}

This graph structure enables the RAG system to query and retrieve legal documents based on semantic similarity and explicit relationships. For example, a query about "due process" might retrieve the relevant constitutional clause, cases that discuss its interpretation, and statutory provisions impacted by those rulings. By combining metadata and latent features, the KG supports advanced reasoning and logical traversal, enhancing the precision and depth of legal information retrieval.

Each part had decomposition topics containing keywords that connected to the documents. The bag of words (BOW) vocabulary was also extracted and inserted into the knowledge graph. These can be observed in the knowledge graph schema of Figure \ref{fig:law_schema}. 

\subsection{Vector Store}
A vector database was implemented to manage and index legal documents, improving the RAG process for legal research. Using Milvus~\cite{2021milvus}, the system stores vectorized representations of constitutions, statutes, and case law, treating each document type uniquely to ensure contextually relevant retrieval.

Constitutional provisions are segmented into paragraphs, each assigned a unique ID, and vectorized using OpenAI’s text-embedding-ada-002~\cite{openai_api} model for granular semantic searches. Statutes are divided by sections or clauses, with metadata like chapter numbers and section titles integrated for precise retrieval. Case law, due to its unstructured narrative format, is chunked into meaningful units, preserving logical flow and indexed with metadata such as case name and citation.

The RAG application queries the database to retrieve relevant fragments—constitutional paragraphs, statutory clauses, or case law sections—based on the query's focus. Retrieved text is processed by the LLM agents with custom prompts to construct responses, allowing the LLM to cite specific paragraphs or clauses, for traceability.

For additional context, the system leverages a connected knowledge graph to explore related statutes, judicial interpretations, or precedent cases.



\begin{figure}[ht]
    \vspace{-.9em}
    \centering
    \includegraphics[width=.8\columnwidth]{figs/court_cases_over_years.pdf}
    \vspace{-1.5em}
    \caption{New Mexico Supreme/Appeals case counts per year.}
    \vspace{-2em}
    \label{fig:court_cases}
\end{figure}


\section{Results}
\label{sec:results}
This section presents the resulting atomic parts of the legal texts and their hierarchical decomposition, offering a detailed breakdown of the sections and cases within each document type. The following results illustrate the scope and depth of the collected data, providing a foundation for further exploration of trends and patterns across the legal corpus.

\subsection{Dataset}



After collecting and structuring the data, the three types of legal documents had the following atomic parts of data, either as sections in the constitution and statutes or cases from case law:

\begin{itemize}
    \item \textbf{Constitution:} 265 sections
    \item \textbf{Statutes:} 28,251 sections
    \item \textbf{Case Law:}
        \begin{itemize}
            \item \textbf{\textit{ Court of Appeals:}} 10,072 cases
            \item \textbf{\textit{Supreme Court:}} 5,727 cases
        \end{itemize}
    
\end{itemize}

Figure~\ref{fig:court_cases} shows the trends in the 5,727 Supreme Court and 10,072 Court of Appeals cases over the years,  as they were available from Justia \cite{justia}, which also includes the expansions and creations of the courts themselves.

\begin{figure*}[ht]
    \centering
    % Single Row
    \begin{subfigure}[t]{0.24\textwidth}
        \centering
        \includegraphics[width=\textwidth]{figs/sunburst_chart_constitution.pdf}
        \caption{10 latent topics from \\ \textbf{Constitutional Provisions}}
        \label{fig:decomp_const}
    \end{subfigure}
    \hfill
    \begin{subfigure}[t]{0.24\textwidth}
        \centering
        \includegraphics[width=\textwidth]{figs/sunburst_chart_statutes.pdf}
        \caption{985 latent topics from \\ \textbf{Statutory Sections}}
        \label{fig:decomp_stat}
    \end{subfigure}
    \hfill
    \begin{subfigure}[t]{0.24\textwidth}
        \centering
        \includegraphics[width=\textwidth]{figs/sunburst_chart_appeals.pdf}
        \caption{420 latent topics from \\ \textbf{Court of Appeals Cases}}
        \label{fig:decomp_sup}
    \end{subfigure}
    \hfill
    \begin{subfigure}[t]{0.24\textwidth}
        \centering
        \includegraphics[width=\textwidth]{figs/sunburst_chart_supreme.pdf}
        \caption{248 latent leaf topics from \textbf{Supreme Court Cases}}
        \label{fig:decomp_app}
    \end{subfigure}
    \vspace{-.5em}
    \caption{Legal Documents from New Mexico hierarchically decomposed. The Constitution only had enough documents to decompose the first depths, whereas the other three sources continued to the terminal depth of 2 (a hyper-parameter of decomposition). Each H-cluster has a natural language label, where depth-0 from each can be seen in Tables \ref{tab:Constitution}, \ref{tab:statutes}, \ref{tab:supreme}, and \ref{tab:appeals}. }
    \label{fig:decomp_all}
    \vspace{-1.5em}
\end{figure*}


\subsection{Decomposition}


The four component data types were decomposed hierarchically with \textit{NMFk}. The vocabulary for building the TF-IDF matrix was collected using specific parameters for each part. For the Constitution vocabulary, the minimum token document frequency (DF) was set to 5 documents, with a maximum token DF of 80\% of the corpus, resulting in a final size of 416 tokens. For the Statutes vocabulary, the minimum token DF was set to 30 documents, with a maximum token DF of 70\% of the corpus, yielding 7,508 tokens. The Court of Appeals vocabulary used a minimum token DF of 50 documents (cases) and a maximum token DF of 70\% of the corpus, resulting in 10,189 tokens. Last, the Supreme vocabulary employed a minimum token DF of 50 papers and a maximum token DF of 70\% of the corpus, with a final size of 8,425 tokens.
The maximum depth was set to 2, of which only the constitution sections did not reach due to the limited number of sections. The limiting factor of further decompositions was 100 documents, so if the preceding cluster had $\geq 100$, the cluster would decompose and stop otherwise. As seen in Figure \ref{fig:decomp_const}, the largest H-cluster is from cluster 4 in dark blue with 49 sections, with the fewest documents in yellow with six sections in H-cluster 0. The other three decompositions can be examined in the larger Figure \ref{fig:decomp_all}, where the court cases, there are 10 leaf H-clusters in the constitution decomposition, 985 leaf H-clusters in the statutes decomposition,  420 leaf H-clusters in the Court of Appeals cases, and 248 leaf H-clusters in the Supreme Court cases. From the methods in \cite{topic_tag}, each H-cluster throughout the decomposition hierarchies has LLM-generated labels for ease of reference and quick insight. Labels for the first decomposition depth for the constitution can be observed in Table \ref{tab:Constitution},  the statutes in Table \ref{tab:statutes}, the Supreme court in Table \ref{tab:supreme}, and the appeals in Table \ref{tab:appeals}. The depth-0 H-clustering corresponds to the first ring radial from the center totals in Figure \ref{fig:decomp_all}.

 
\definecolor{darkgreen}{rgb}{0.0, 0.6 , 0.0}
\definecolor{lightgreen}{rgb}{0.0, 0.9, 0.0}
\definecolor{goldyellow}{rgb}{1.0, 0.84, 0.0}
\definecolor{darkorange}{rgb}{1.0, 0.55, 0.0}
\definecolor{lightblue}{rgb}{0.0, 0.9, 1.0}
\definecolor{darkpurple}{rgb}{0.4, 0.0, 0.4}

\begin{table}[!htbp]
    \scriptsize  
    % \vspace{-1.5em}
    \caption{NM Constitutional Depth-0 H-Clusters}
    \vspace{-1em}
    \label{tab:Constitution}
    \renewcommand{\arraystretch}{0.9} 
    \rowcolors{2}{gray!25}{white}
    \begin{tabular}{@{}p{0.02\columnwidth}p{0.89\columnwidth}@{}}
        \rowcolor{darkgreen} 
        \textbf{\#} & \textbf{Label} \\
        \toprule
        0 & Irrigation and Water Resource Management Principles \\
        1 & Regulation of Private Sector Influence on Public Schools and Education Services \\
        2 & Branches of Government Structure and Functionality \\
        3 & Education Funding for New Mexico's Educational Institutions \\
        4 & Legislative Proceedings and Lawmaking Activities Enacted During Sessions \\
        5 & Territorial Transition: Constitutional Ratification and Statehood Provisions\\
        6 & Governance, Land, and Taxation Framework \\
        7 & Municipal and County Financial Obligations and Liabilities Management \\
        8 & Judicial Power and Jurisdictional Frameworks \\
        9 & Public Service Election Governance Structure and Processes \\
        \bottomrule
    \end{tabular}
\end{table}

\begin{table}[!htbp]
    \scriptsize 
    % \vspace{-1.5em}
    \caption{NM Statutory Depth-0 H-Clusters}
    \vspace{-1em}
    \label{tab:statutes}
    \renewcommand{\arraystretch}{0.9}  
    \rowcolors{2}{gray!25}{white}
    \begin{tabular}{@{}p{0.02\columnwidth}p{0.89\columnwidth}@{}}
        \rowcolor{lightgreen} 
        \textbf{\#} & \textbf{Label} \\
        \toprule
        0  & Municipal Court Civil Cases Involving Children's Rights \\
        1  & Public Education Infrastructure Management Systems \\
        % 2  &  FIX - New Mexican Subdivision Legislation Amendment Act of 2021 \\

        2 &  Criminal Codes, Local Governance Boundaries, Licensing Rules\\
        3  & Comprehensive Emergency Health and Human Services Response Framework \\
        4  & Taxation and Revenue Collection Oversight \\
        % 5  & FIX - Election Governance and Administrative Matters \\ %for Members \\
        5 & Military Decorations, Licensing Procedures, Governance Boards \\
        6  & Regulatory Insurance Contract Law and Policy Analysis \\
        7  & Public Obligations Financing and Project Bonds Issuance \\
        8  & State Government Investment Grants for Education and Economic Development \\
        9  & Irrigation and Water Rights Regulations \\
        % 10 & FIX - Legislative Repeals and Enactments \\
        % Regarding Congressional Provisions \\
        10 & Military Honors, Discrimination Penalties, and Trust Authority \\
        11 & Corporation Governance Framework and Regulatory Compliance \\
        \bottomrule
    \end{tabular}
\end{table}

\begin{table}[!htbp]
    \scriptsize  
    % \vspace{-1.5em}
    \caption{NM Appeals Court Case Law Depth-0 H-Clusters}
    \vspace{-1em}
    \label{tab:appeals}
    \renewcommand{\arraystretch}{0.9}  
    \rowcolors{2}{gray!25}{white}
    \begin{tabular}{@{}p{0.02\columnwidth}p{0.89\columnwidth}@{}}
        \rowcolor{goldyellow} 
        \textbf{\#} & \textbf{Label} \\
        \toprule
        0  & Real Property Rights and Interests \\
        1  & Civil Liability and Injury Issues Arising from Healthcare Services \\
        2  & Parental Rights and Custody Proceedings Involving Disputed Parental Fitness \\
        3  & Motor Vehicle Insurance and Liability Claims Processing Procedures \\
        4  & Mortgage Foreclosure and Secured Lending Frameworks \\
        % 5  & FIX - Jurisdictional Determinations and Dispute Resolution Outcomes \\

        5 & Arraignments, Tribal Jurisdiction, Divorce, Bond Conditions, Motor Licensing \\ 
        6  & Courtroom Advocacy and Representation Strategies \\
        7  & Petitioner's Right to Parental Decision Making \\
        8  & Fourth Amendment Protections Against Unlawful Searches and Seizures \\
        9  & Taxation of Gross Receipts and Sales Transactions in a Business Context \\
        10 & Key Legal Concepts Related to Sexual Offenses \\
        11 & Child Protective Services Laws and Regulations \\
        12 & Work-Related Injury and Disability Compensation Process \\
        13 & Healthcare Contract Dispute Resolution Processes \\
        14 & Marital Property and Financial Disposition \\
        15 & Legal Proceedings Involving Jury Determination of Guilt \\
        16 & Probation Supervision and Monitoring Oversight Process \\
        % 17 & FIX - Delays of Criminal Trials Due to Pretrial Proceedings \\
        17 & Juvenile Protection, Miranda Rights, Evidence Collection \\
        % 18 & FIX - Court Proceedings and Decisions Regarding Agency Dispositions \\ %and Convictions \\
        18 & Firearms, Substances, DWI, and Sentencing \\
        19 & Sentencing Enhancements for Serious Repeat Offenders \\
        20 & Traffic Stop Under Suspicion with Mandatory Blood Alcohol Testing \\
        21 & Criminal Proceedings Trial Litigation Documentation and Record-Keeping Procedures \\
        22 & Workers' Rights and Insurance Benefits \\
        23 & Children's Welfare and Family Reunification Efforts \\
        24 & Public Municipal Legal Frameworks and Governance Structure \\
        % 25 & FIX - Official NMRA Decisions and Rulings Regarding State-Specific Matters \\
        25 & Business Disputes, Wrongful Injury, Taxation Appeals, Workers’ Compensation \\ 
        % 26 & FIX - Proposed Electronic Filing Requirements \\ % Memorandum \\
        26 & Administrative Disputes,  Native American Legal Memorandums \\
        \bottomrule
    \end{tabular}
\end{table}

\begin{table}[!htbp]
    \scriptsize  
    % \vspace{-1.5em}
    \caption{NM Supreme Court Case Law Depth-0 H-Clusters}
    \vspace{-1em}
    \label{tab:supreme}
    \renewcommand{\arraystretch}{0.9} 
    \rowcolors{2}{gray!25}{white}
    \begin{tabular}{@{}p{0.02\columnwidth}p{0.89\columnwidth}@{}}
        \rowcolor{darkorange} 
        \textbf{\#} & \textbf{Label} \\
        \toprule
        0  & Arbitration of Contract Disputes and Judicial Decision-Making in Motor Vehicle Cases \\
        1  & Revenue and Taxation Frameworks in Governance and Administration \\
        % 2  & FIX - Mineral Rights Leases \\
        2  & Mineral Rights Leases \\
        3  & Native American Self-Governance and Tribal Jurisdictional Frameworks \\
        4  & Municipal Zoning Ordinances and Regulations of Local Governance Areas \\
        5  & Damages Award for Wrongful Conduct Against Business Partners \\
        % 6  & FIX - Sentencing Framework for Criminal Offenders \\ %Federal Offenders \\
        6 & Jurisdictional Appeals, Juvenile Sentencing, Felony Enhancements, Parole Terms \\
        7  & Post-Divorce Asset Distribution and Management Strategies \\
        8  & Constitutional Challenges to Public Education Governance \\
        9  & Secured Financial Instruments and Property Rights \\
        10 & Water Resource Allocation and Management \\
        11 & Electric Utility Rate Regulation Oversight Authority \\
        12 & Law Enforcement Procedures and Rights Protection under Fourth Amendment Protections \\
        13 & State Legislative Review Process Decisions \\
        14 & Denial of Petition for Habeas Corpus \\ % Review \\
        15 & Workers' Compensation Process for Work-Related Injuries and Disabilities \\
        16 & Real Estate Boundary Disputes and Conveyance Matters \\
        17 & Parental Rights and Legal Guardianship Proceedings \\
        18 & Appellant's Argument Against the Ruling of a Trial Court \\
        19 & Document Terms and Conditions Regarding Business Transactions \\
        20 & Mutual Insurance Policies for Vehicle and Individual Coverage \\
        21 & Accident resulting from driver error on public roadways leads to legal consequences \\
        22 & Court proceedings involving criminal trials and testimonial evidence \\
        23 & Civil Litigation Proceedings and Trials \\
        \bottomrule
    \end{tabular}
    % \vspace{-2em}
\end{table}


\begin{table}[ht]

\centering
\caption{Neo4j Node and Edge Overview}
    \vspace{-1em}
\label{tab:neo4j_counts}
\definecolor{headerblue}{HTML}{CFE2F3} 
\definecolor{lightgray}{gray}{0.92}     
\rowcolors{2}{lightgray}{white}
\begin{tabular}{@{}lrrr@{}}
  \rowcolor{headerblue}
  \toprule
  \textbf{Node Type}      & \textbf{Nodes}   & \textbf{Out Edges}   & \textbf{Legal Cites} \\
  \midrule
  \textit{NMFk} Topics             & 2,469            & 92,634               & --                   \\
  \textit{NMFk} Keyword            & 11,076            & 8,281,843                   & --                   \\
  BOW Vocabulary          & 132,423          & --                   & --                   \\
  Constitution     & 265              & 9,067                & 41                   \\
  Statute          & 28,251           & 1,930,707            & 81,353               \\
  Supreme Court Case      & 5,727            & 2,437,161            & 76,478               \\
   Court of Appeals Case       & 10,072           & 4,176,288            & 131,230              \\
  \midrule
  \textbf{Total Unique}   & \textbf{190,283} & \textbf{16,927,700}  & \textbf{289,102}     \\
  \bottomrule
\end{tabular}
\end{table}

\begin{figure}[ht]
    % \vspace{-.9em}
    \centering
    \includegraphics[width=\columnwidth]{figs/estopple.png}
    \vspace{-2em}
    \caption{Examination of `Estoppel' relating to being a keyword in topics, vs bag of word vocabulary }
    \vspace{-1em}
    \label{fig:estoppel}
\end{figure}
\subsection{Knowledge Graph}
The four data parts, 265 constitutional provisions, 28,251 Statute sections, 5,727  Supreme Court cases, and 10,072 Court of Appeals cases, were inserted into the neo4j \cite{neo4j2023} knowledge graph. 
The graph's number of nodes and edges can be seen in Table \ref{tab:neo4j_counts}, where edge counts are where the triplet's tail originates with the row's node. The legal citations were collected by iterating the text of each case or section into chat-gpt-3.5-turbo with the following prompt: "You are an expert legal document analyzer. Your job is to find all references to the Constitution, Case Law, or Statutes in the text." The result was that the LLM acted like a named entity extractor, such that any citations in the text were pulled out in an enumerated list. The citations mainly included the cases, statutes, and constitution of New Mexico but also had references to the United States Constitution, New Mexico Administrative Code, and New Mexico Rules Annotated (NMRA). The NMRA had many references to Uniform Jury instructions and the rules of criminal and civil proceedings.

In figure \ref{fig:estoppel}, the \textit{NMFk} topic cluster keyword and a bag of word vocabulary were both queried for `estoppel.' The \textit{NMFk} keyword is Dark blue, the topics are light blue, and the BOW node is brown. Three documents occur for this keyword: the statutes are green nodes, the Court of Appeals cases are yellow, and the Supreme Court cases are orange. The constitution neither clustered over the term nor mentions it, which is not represented in Figure \ref{fig:estoppel}. There is a partition in the nodes at the brown bag of words node since all nodes with the \textit{NMFk} keyword have the bag of word node connection. Still, not all documents mentioning `estoppel' were clustered with the word, which is to say other terms and concepts from those terms had more importance for the documents on the left side of the image that `estoppel'. There are 14 topics, 441 Court of Appeals cases,  276 Supreme Court Cases, and 131 Statutes in Figure \ref{fig:estoppel}. These topics have `estoppel' in their top keywords, whereas if every topic that contained `estoppel' in its words were called, there would be 341 topics. Of the 14 topics, one was connected to the Statutes, \scriptsize ``Collection and recovery of liabilities made to board members with errors and omissions." 
\normalsize Three of the 14 topics were connected to supreme court cases: 
\scriptsize ``Employment Rights and Property Disputes in New Mexico Municipal Affairs", ``Public Corporation Property Taxation Matters and Disputes with Licensing Authorities", ``Drilling and Gas Agreement Terms Regarding Oil Wells."
\normalsize Finally, the remain 10 of the 14 topics with `estoppel' in its top words were connected to cases from the Court of Appeals: \scriptsize "Litigation outcomes and jurisdictional limitations", "Administrative License Revocation Proceedings by Division Officers", "Legal Proceedings and Litigation Issues in a Medical Context", "Corporate Governance and Financial Management Matters", "Dispute Resolution Process for Agricultural Property Transactions", "Motor Vehicle Administrative License Actions", "Insanity Defense Expert Witness Testimony", "Malpractice claim within time constraints", "Employer Liability for Federal Disability Claims Against Administration Agencies", "Criminal offenses and doctrine involve multiple types of larceny charges", "Collection and recovery of liabilities made to board members with errors and omissions", "Public Corporation Property Taxation Matters and Disputes with Licensing Authorities", "Employment Rights and Property Disputes in New Mexico Municipal Affairs", "Drilling and Gas Agreement Terms Regarding Oil Wells."
\normalsize


\begin{figure}[ht]
    % \vspace{-.9em}
    \centering
    \includegraphics[width=\columnwidth]{figs/attempted_accuracy_comparison.pdf}
    \vspace{-2em}
    \caption{25 questions queried across five different LLM channels. Attempts are the percentage of responses tried to answer the question, and accuracy is the percentage of those answered with correct responses. }
    \vspace{-1em}
    \label{fig:attempted_accuracy_comparison}
\end{figure}


\subsection{Query Cross Comparison}
\definecolor{paper_purple}{rgb}{0.5, 0.0, 0.5} % Purple
\definecolor{paper_indigo}{rgb}{0.3, 0.0, 0.7} % Indigo
\definecolor{paper_blue}{rgb}{0.0, 0.0, 0.8} % Pure blue
\definecolor{paper_seagreen}{rgb}{0.0, 0.5, 0.3} % Sea green
\definecolor{paper_darkgreen}{rgb}{0.0, 0.39, 0.0} % Dark green


Questions were formulated to query information about legal concepts from LLM channels. There were five questions from each data source: the constitutional provisions, the Statutes, the Court of Appeals, and the Supreme Court. Then, five additional quantity questions were added. The results can be seen in Figure \ref{fig:attempted_accuracy_comparison}, where the five different models were  OpenAi's GPT-4 \cite{openai2023gpt4} over the API, Google's Gemini \cite{google2023gemini} over a web chat, Nvidia's nemotron:70b-instruct with 8-bit quantization \cite{nvidia2023nemotron}, OpenAi's GPT-4 \cite{openai2023gpt4}  over a web chat, and finally our system for this work, Smart-SLIC. Questions related to the Constitution appear generally answerable by the language models, likely due to its structured and concise wording. 

Answering legal questions becomes increasingly challenging with larger datasets, especially case law, which was the most difficult. Unlike statutes, where hierarchical structures are more straightforward for LLMs to train and summarize, case law is unstructured, lengthy texts that are more challenging to summarize. Constitutional questions were relatively simple, though many models mentioned articles rather than sections, citing frequent changes in how sections are enumerated. Our evaluation procedure assigns zero if the model refuses to answer and if any attempt to respond was made. Accuracy then ranges from 0 to 3: a correct or nearly correct answer receives 3, primarily correct but partially misunderstood responses receive 2, mostly incorrect yet partially true answers receive 1, and responses entirely wrong or hallucinations get 0. Some models were given points when they produced close numeric estimates, even if derived through flawed “database lookups.” In contrast, our system provides traceable reasoning by referencing the KG and analyzing the decomposed hierarchical structure of the texts themselves.

The following are comparisons between each of the models with formatting and truncations (as ...) so the text is more readable:

\textit{(a) Comparison 1 - Quantities: }
\begin{itemize}
    \footnotesize 
    \item \textbf{Question:} ``How many New Mexico Supreme Court cases mention `Habeas Corpus'?''
    \item \textbf{GPT-4,API:} %\color{paper_purple}
    \textit{I don't have the ability to provide real-time or exact counts of Supreme Court cases mentioning "habeas corpus" up to the current date...}
    % \color{black}

    \begin{flushright}
        \vspace{-0.25cm}
        \rule{7.5cm}{0.01mm} 
        \vspace{-0.08cm}
    \end{flushright}
        
    \item \textbf{GPT-4o,ChatGPT:} %\color{paper_indigo}
    \textit{Determining the exact number of New Mexico Supreme Court cases that mention 'habeas corpus' is challenging due to the extensive history of the court's decisions and the limitations of available databases...}
    %\color{black}
    
    \begin{flushright}
        \vspace{-0.25cm}
        \rule{7.5 cm}{0.01mm} 
        \vspace{-0.08cm}
    \end{flushright}
        
    \item \textbf{Smart-Slic:} % \color{paper_blue}
    \textit{There are 215 Supreme Court cases that mention 'Habeas Corpus.'} %\color{black}

    \begin{flushright}
        \vspace{-0.25cm}
        \rule{7.5cm}{0.01mm} 
        \vspace{-0.08cm}
    \end{flushright}
        
    \item \textbf{Gemini:} %\color{paper_seagreen}
    \textit{The exact number of Supreme Court cases that mention "Habeas Corpus" is difficult to determine precisely. Here's why: \textbf{- Vast Number of Cases:} ... \textbf{- Varied Mention:} ... \textbf{- No Centralized Database:} ... }
    % \color{black}

    \begin{flushright}
        \vspace{-0.25cm}
        \rule{7.5cm}{0.01mm} 
        \vspace{-0.08cm}
    \end{flushright}
    
    \item \textbf{nemotron:70b-instruct-q8\_0:} %\color{paper_darkgreen} 
    \textit{I'll outline my research methodology and then share the results.
    Research Methodology: \textbf{1. Database Selection:} ... \textbf{2. Search Parameters:} ... \textbf{3. Search Approach:} ... After conducting thorough searches across the selected databases:\textbf{ Approximate Number of Cases Mentioning 'Habeas Corpus':} Over 1,300 ... Breakdown ...: \textbf{* Pre-1950:} ... 450 ... \textbf{* 1950-2000:} ... 550 ... \textbf{* 2001-Present:} ... 300 ...}
    % \color{black}
\end{itemize}


The responses highlight significant differences in model abilities. GPT-4 API and ChatGPT provided an understanding of their limits to offer real-time or exact counts due to limitations such as access to historical databases or the complexity of case records. Gemini followed a similar pattern, explaining why determining a precise number is challenging. In contrast, our system directly reported a specific count, stating 215 cases mentioning 'Habeas Corpus,' showing its ability to process and retrieve meaningful data. Meanwhile, nemotron:70b-instruct-q8\_0 provided a detailed method to find the correct answer in an open context. It even went into breaking the count into approximate ranges across different periods, suggesting over 1,300 cases in total.  However, these counts from Nemotron are part of an extreme hallucination since it was an offline model without the ability to make any lookups. 

\textit{(b) Comparison 2 - Legal Citations: }
\begin{itemize}
    \footnotesize 
    \item \textbf{Question:} ``What are common citations among New Mexico Court of Appeals cases that mention `malpractice'?''
    
    \item \textbf{GPT-4,API:} %\color{paper_purple}
     \textit{ I don't have direct access to specific databases or the ability to browse the internet.}
    % \color{black}

    \begin{flushright}
        \vspace{-0.25cm}
        \rule{7.5cm}{0.01mm} 
        \vspace{-0.08cm}
    \end{flushright}

    \item \textbf{GPT-4o,ChatGPT:} %\color{paper_indigo}
    \textit{ In New Mexico Court of Appeals cases addressing malpractice, several key legal principles and precedents are frequently cited: 
    \textbf{Elements of Legal Malpractice:} ... Mireles v. Broderick, 1992 ...
    \textbf{Statute of Limitations:} ... Sharts v. Natelson, 1994 ...
    \textbf{Expert Testimony Requirement:} ... Rummell v. Edgemont Realty Partners ... 1993 ...
    \textbf{Causation and "Case Within a Case" Doctrine:} ... Garcia v. Rodey, Dickason, Sloan, Akin \& Robb ... 1988 ...
    \textbf{Damages:} ... Sanders, Bruin, Coll \& Worley ... 2010 ...
    }
    % \color{black}

    \begin{flushright}
        \vspace{-0.25cm}
        \rule{7.5cm}{0.01mm} 
        \vspace{-0.08cm}
    \end{flushright}
    
    \item \textbf{Smart-Slic:} %\color{paper_blue}  
    % \textit{\textbf{1.} Rule 12-405 NMRA with 61 cases  \textbf{2.} NMSA 41-5-1 with 36 cases  \textbf{3.} NMSA 41-5-13 with 35 cases  \textbf{4.} NMSA 41-4-1 with 23 cases  \textbf{5.} CERVANTES\_v\_FORBIS\_1964 with 18 cases}
    \textit{The common citations among New Mexico Court of Appeals cases that mention 'malpractice' include NMSA 41-5-1 with 36 cases, NMSA 41-5-13  with 35 cases, NMSA 41-4-1 with 23 cases, CERVANTES v FORBIS 1964 with 18 cases, NMSA 37-1-8  with 15 cases, NMSA 41-5-3  and NMSA 41-5-15  each with 14 cases, NMSA 41-5-6  with 13 cases, GOODMAN v BROCK 1972  with 12 cases, and NMSA 41-5-22 with 10 cases.}
    % \color{black}

    \begin{flushright}
        \vspace{-0.25cm}
        \rule{7.5cm}{0.01mm} 
        \vspace{-0.08cm}
    \end{flushright}

    \item \textbf{Gemini:} %\color{paper_seagreen} 
    \textit{ \textbf{Common citations in New Mexico Court of Appeals cases mentioning "malpractice" often include:} - Leyba v. Whitley...1995... - Rancho del v. Weisfeld...1995...
    Bassett v. Sheehan, Sheehan \& Stelzner...2008...
    Akutagawa v. Laflin, Pick \& Heer...2005... }
    % \color{black}

    \begin{flushright}
        \vspace{-0.25cm}
        \rule{7.5cm}{0.01mm} 
        \vspace{-0.08cm}
    \end{flushright}

    \item \textbf{nemotron:70b-instruct-q8\_0:} %\color{paper_darkgreen} 
    \textit{Steps to Find Common Citations: \textbf{1. Access Legal Databases:} ... \textbf{2. Construct Your Search Query:} ... \textbf{3. Filter Results (if possible):} ... 
    \textbf{4. Analyze Top Cases for Citations:} ... 
    Please note, the following are hypothetical examples... 
    \textbf{1. Cases: - Hypothetical Example 1:} Smith v. Jones, 2010 ... 
    \textbf{- Hypothetical Example 2:} Doe v. Healthcare Inc., 2015 ...
    \textbf{2. Statutes or Regulations:} - ... NMSA 1978, Section 41-5-1 ... - NMSA 1978, § 41-4-11 to 41-4-17 ...   \textbf{3. Court Rules:} - ... NMRCvP, Rule 1-026 ... - ... NMRE, Rule 702 ...}
    % \color{black}
\end{itemize}

GPT-4 API again opted not to answer, stating that it did not have access to specific databases or the ability to browse legal records. GPT-4o (ChatGPT), on the other hand, offered an insightful summary of key legal principles and precedents frequently cited in malpractice cases, naming specific examples like Mireles v. Broderick (1992) and Garcia v. Rodey (1988). This response demonstrated domain awareness but was missing particular detail.

In contrast, our system delivered a detailed list of the most common citations, including exact counts for specific statutes and cases, such as NMSA 41-5-1 cited in 36 cases and CERVANTES v FORBIS (1964) cited in 18 cases. Our system's response is specific and direct, allowing for faster and more reasoned legal research. Meanwhile, Gemini offered a mix of case-specific citations, similar to GPT-4o, but with limited depth and no accompanying counts, which made the response less precise.

Finally, nemotron:70b-instruct-q8\_0 outlined a hypothetical strategy to find citations, including steps like accessing legal databases and filtering results. However, the examples provided, such as "Smith v. Jones, 2010," were explicitly hypothetical, undermining the credibility of the response. Additionally, nemotron's inclusion of actual statutes like NMSA 41-5-1 blurred the line between hypothetical examples and genuine analysis, leading to potential confusion.

This comparison highlights the strengths and weaknesses of each system: some models are better at summarizing legal principles, such as GPT-4o and Gemini, while our System shows the capability for precise data retrieval and analysis. Nemotron, despite its detailed methodology, suffered most from hallucinations. 
% \subsection{Vector Store}
% Each section and case from the four parts are vectorized into Milvus \cite{2021milvus}. Each data set has two embeddings and a subsequent index: the raw uncleaned title and full text of the document and a layman's term translation of the law. The layman term summarization was made using llama3.2:3b-instruct-fp16. To make the layman's terms, the Constitution took 20 minutes, the statutes took 14 hours 2 minutes, the Supreme Court cases took 8 hours 42 minutes, and the Court of Appeals took 13 hours 33 minutes. The benefit of the second index, in layman's terms, allows natural language questions to more closely approximate the correct embedding of the legal section without requiring more formal legal terms in question formulation, with the benefit of the raw text index for support on queries more directly matching the natural law language.


\begin{figure}[h]
    % \vspace{-.9em}
    \centering
    \includegraphics[width=\columnwidth]{figs/mrr_retreival.pdf}
    \vspace{-3em}
    \caption{MRR and top 10 hit rate percentage per data category over several embedding methods.}
    \vspace{-2em}
    
    \label{fig:mrr_retrieval}
\end{figure}
\subsection{Question Answering}
To evaluate the effectiveness of our QA SLIC-SMART system, we conducted a series of experiments focusing on various performance metrics. Figure \ref{fig:mrr_retrieval} shows various retrieval methods from the embedding space. The four data sources are treated separately for comparison. Each data source uses four separate methods to embed the data. The first is the entire corpus of each, where all documents are put into a single vector store. Where the text was short and structured, this had top 10 retrieval percentages more significant than 70 to 80\%, such as the statutes and constitutional provisions. The cases had less structure and longer text, making the plain embedding challenging.  Next is the chunking method, where the entire dataset is again placed into a single dataset but chunked at 3000 characters with 500 overlapping. The top row, with more concise and structured concepts, decreased slightly in performance, whereas the cases significantly improved. This is likely because the concepts are broken down across multiple texts instead of being in one coherent text to query. Still, with the decrease, the structured texts outperformed the cases by more than 20\%. Third, the \textit{NMFk} hierarchical cases are incorporated into the vector store process, where each topic has an individual vector store. The topics per query were controlled such that the query was searched in the vector store corresponding to the question origin. Topic identification in the vector store is an outstanding problem for future work. When the \textit{NMFk} hierarchical topics were introduced, the performance significantly improved across all data sources, with the most significant gains in the longer case texts. When the chunking process was incrementally added on top of the \textit{NMFk} topics, the cases still improved, again likely to break up much larger texts into more meaningful segments of the texts. In contrast, the structured texts in the constitutional provisions and the statutes slightly decreased performance, likely attributed to the splitting of concepts in the questions across several chunks. Still, \textit{NMFk} topics with chunking and correct topic identification outperform the other methods overall. 

\subsubsection{Case Studies}
To illustrate the practical applications and effectiveness of the QA Smart-Slic system, we present four detailed case studies:

\textit{(a) Case Study 1 - Constitutional Analysis:}
\begin{itemize}
\footnotesize 
    \item \textbf{Question Source:} Article IV, Section 22 \cite{nm_const_art4_sec22}
    \item \textbf{Question:} ``What happens to a bill if the governor neither returns it within the specified three-day window (Sundays excepted) nor signs it?''
    \item \textbf{Response:}  %\color{paper_blue}
    \textit{According to Article IV - Legislative Department -- § 22 "Governor's approval or veto of bills" of the provided text, if a bill is presented to the governor and he neither returns it within the specified three-day window (Sundays excepted) nor signs it, then the bill becomes a law automatically, whether signed by him or not, unless the legislature by adjournment prevents such return.}
    % \color{black}
\end{itemize}

\textit{(b) Case Study 2 - Statutory Analysis:}
\begin{itemize}
\footnotesize 
    \item \textbf{Question Source:} Chapter 50, Article 17, Section 3 \cite{healthy_workplaces_earned_sick_leave}
    \item \textbf{Question:} ``Under the Healthy Workplaces Act, how does an employee accrue and utilize earned sick leave, and what provisions exist for its carryover, use for family or personal health, and reinstatement after separation from employment?''
    \item \textbf{Response:} %\color{paper_blue}
    \textit{ \textbf{Accrue}: Employees accrue a minimum of one hour of earned sick leave for every thirty hours worked. Employers may elect to provide the full sixty-four hours of earned sick leave for the year on January 1 or pro-rata for mid-year hires. \\
    \textbf{Use}: Earned sick leave can be used for the employee’s or their family members' illness, injury, health condition, medical care, or preventive medical care. It can also be used for issues related to domestic abuse, sexual assault, or stalking. \\
    \textbf{Carryover}: Unused earned sick leave carries over year to year, although use may be capped at 64 hours per twelve-month period, unless the employer sets a higher cap. \\
    \textbf{Reinstatement}: If an employee is rehired within twelve months of separation by the same employer, previously accrued and unused earned sick leave must be reinstated. Additionally, the employee is entitled to accrue additional sick leave upon re-commencement of employment.}
    % \color{black}
\end{itemize}

\textit{(c) Case Study 3 - Court of Appeals Case Law Analysis:}
\begin{itemize}
\footnotesize 
    \item \textbf{Question Source:} KRM, Inc. v. Caviness, 1996 \cite{krm_v_caviness_1996}
    \item \textbf{Question:} ``In the case KRM, Inc. v. Caviness, what was the court's reasoning for determining that the water rights established by Williams' 1963 declaration did not belong to the land sold to Caviness?''
    \item \textbf{Response:} %\color{paper_blue}
    \textit{The court determined that the water rights established by Williams' 1963 declaration were not appurtenant to the land sold to Caviness because there was no evidence that the water had been used for irrigation on that land. Since the appropriated water was never used for irrigation purposes on the property, it did not qualify as appurtenant water rights that would automatically transfer with the land. The court concluded because the usage of the water was unrelated to the benefit or use of the land itself, the water rights remained separate and did not pass to Caviness when he purchased the land.}%\color{black}
\end{itemize}

\textit{(d) Case Study 4 - Supreme Court Case Law Analysis:}
\begin{itemize}
\footnotesize 
    \item \textbf{Question Source:} Smith v. South, 1955 \cite{smith_v_south_1955}
    \item \textbf{Question:} ``In Smith v. South, what were the legal issues raised regarding the enforceability of the Washington child support decree in New Mexico, and how did the court address the conflict between the parties' post-decree agreement and the original judgment?''
    \item \textbf{Response:} %\color{paper_blue}"
    \textit{The legal issues raised in Smith v. South regarding the enforceability of the Washington child support decree in New Mexico included whether New Mexico courts should acknowledge and enforce the original Washington decree for child support or consider the modifications reportedly agreed upon by the parties after the decree. The conflict between the parties' post-decree agreement and the original judgment involved assertions that they had informally agreed to alter the child support payments stipulated by the Washington court. \\    
    The court addressed this issue by considering the jurisdictional and legal principles guiding child support enforcement across state lines. They emphasized the importance of adhering to the formal judicial process. The court likely examined whether the post-decree modifications had been legally formalized or approved by any court, and whether such modifications complied with the relevant child support enforcement laws, particularly under the Uniform Interstate Family Support Act (UIFSA), though it wasn't explicitly mentioned in the text snippet. UIFSA provides mechanisms for establishing, enforcing, and modifying child support orders across state boundaries. \\
    If the modifications were not legally recognized through a court process, the original Washington decree would remain enforceable in New Mexico, and any informal or unratified agreements between the parties would not typically alter the enforceability of the court-ordered decree. The court would focus on ensuring that child support arrangements comply with legal standards to serve the children's best interests and that enforcement respects the authority of the original jurisdiction (Washington) unless properly modified through legal channels.}%\color{black}
\end{itemize}
 



\section{Conclusion}
\label{sec:conclusion}

% This work introduces a framework that leverages advanced computational techniques in the legal domain, combining RAG with KGs and dimensionality reduction methods. Integrating \textbf{T-ELF}, vector databases, and metadata into a Neo4j-based KG offers a structured and queryable representation of constitutional provisions, statutes, and case law, enabling efficient retrieval, thematic clustering, and latent feature discovery.

% The dimensionality reduction method \textit{NMFk} identifies latent topics and optimally clusters legal documents, avoiding overfitting or underfitting. Hierarchical decomposition enhances the framework by capturing nested relationships within legal texts, enabling context-aware queries and analysis.

% However, gaps, such as missing citation networks between cases, limit the KG’s traceability. Addressing these areas could improve support for precedent-based reasoning and legislative analysis. Further, the New Mexico administrative code and judicial rules, which were not collected or analyzed, would allow further legal insight into a well-functioning state government if combined into the system. 

% This framework bridges traditional legal research and computational tools, offering a scalable, interpretable solution for navigating large legal datasets. Combining semantic reasoning, KGs, and advanced factorization techniques, it lays a strong foundation for data-driven legal research and decision-making.


This work introduced Smart-Slic, a generative AI framework tailored to the legal domain, leveraging RAG, vector stores, and a Neo4j-based knowledge graph constructed through \textit{NMFk}. Our approach uses \textbf{T-ELF} with metadata and chunking strategies to capture fine-grained H-clusters and minimize hallucinations to improve reliability. By bridging structured and unstructured data--spanning constitutional provisions, statutes, and case law--, our system supports advanced semantic reasoning and dimensionally reduced insight into the latent structure of legal texts.

Experimental results across multiple retrieval strategies show that chunking, combined with hierarchical \textit{NMFk}, improves accuracy, particularly for large, unstructured case law datasets. Short, highly structured documents, constitutional provisions, and statute sections benefit from minimal chunking, revealing the importance of aligning preprocessing approaches with data characteristics. We demonstrated the framework’s capability to derive interpretive legal topics and precisely answer domain-specific queries by harnessing topically segmented embeddings and explicit links within the knowledge graph.

Despite the positive results, challenges remain. Author attribution in networks is incomplete, limiting the knowledge graph’s potential for thorough precedent tracing and interlinking. Including additional datasets—such as administrative codes and judicial rules—would provide richer context and increase the system’s coverage of a functioning state government. Moreover, systematically reconciling informal post-decree agreements with formal judgments is needed to model a more acute legal flow.

This framework marks a step forward in computational law and legal AI, demonstrating a scalable, interpretable method for discovering, retrieving, and reasoning over complex legal corpora. Combining semantic embeddings, latent topic modeling, and knowledge graphs contributes to more robust, data-driven legal research pathways. Future directions include refining the citation extraction pipeline, expanding the collection to encompass broader legal instruments, and applying advanced LLM-driven reasoning for deeper precedent analysis and trend prediction.




\begin{acks}
This research was funded by the U.S. Department of Energy National Nuclear Security Administration’s Office of Defense Nuclear Nonproliferation Research and Development
(DNN R\&D), supported by the U.S. DOE NNSA under Contract No. 89233218CNA000001, as well as by the LANL Institutional Computing Program.
\end{acks}



%%
%% The next two lines define the bibliography style to be used, and
%% the bibliography file.
\bibliographystyle{ACM-Reference-Format}
\bibliography{references.bib}


\end{document}
\endinput
%%
%% End of file `sample-sigconf.tex'.
