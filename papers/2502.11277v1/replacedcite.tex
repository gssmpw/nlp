\section{Related Work}
We first review the literature concerning general behavior models around mental health care-seeking and then center on the practices of marginalized populations. We then elaborate on the ecological view of mental health care-seeking which serves as an overarching methodology approach that supports our analysis. 

\subsection{Seeking Care for Mental Health Concerns} 
Care-seeking is also referred to as help-seeking, denoting people's behavior of actively obtaining assistance from various sources ____. The assistance includes suggestions, information, and treatment that serve pragmatic purposes and also social and emotional support and understanding ____. 

One of the earliest models focused on formal help-seeking in the medical system. \textcite{huxley_mental_1996} outlined pathways that a patient may need to navigate through to get care. The pathway consists of five layers. Potential patients entered the pathway from general practitioners, via whom they were referred to primary care, then secondary psychiatric care, and ended with admission to the hospital. The following studies identified other resources that act as referrals within ____ and outside ____ of the medical system, such as native or religious healers, nurses, psychiatric services, and police institutions.
The model does not sufficiently consider external factors such as socio-economic status, cultural influences, or the role of informal support networks, which can significantly impact the care-seeking journey. 

Rickwood’s Stages of help-seeking framework ____ describes patients' subjective experiences of the help-seeking process. The model delineates four key steps: (1) recognizing symptoms and realizing the need for assistance, (2) expressing symptoms and signaling the need for support, (3) identifying available and accessible sources of help, and (4) the final step contingent on the individual's willingness to disclose difficulties to the chosen source. The stages are presented as static categories, potentially overlooking the dynamic nature of help-seeking over time. 

Seeking help for mental health is not a quick and simple decision; rather, it is a complex and dynamic process dubbed with stigmatization and solitary feelings ____. In the process, people constantly negotiate with themselves "if seeking help is necessary" and "when and how to seek help". \textcite{biddle_explaining_2007} delineated this struggle through the Cycle of Avoidance (COA) model, depicting the tensions involved in making sense of, accepting, and avoiding mental distress. They identified several key actions that postpone help-seeking, including normalizing symptoms, proposing alternative explanations, accommodating increasingly severe distress, pushing the threshold between “normal” and “real” distress, and delaying help-seeking. Other studies also pinpointed similar cognitive strategies such as utilizing self-resilience, denying the effects of professional help, and problematizing help-seeking ____. 

\one{While these models effectively capture the complexities of mental health care-seeking behaviors and stages, they do not examine how individuals interact with specific health resources and technologies. Moreover, these models are primarily based on general population studies, leaving unclear how marginalized identities shape care-seeking practices and resource engagement.}

%These empirical studies and theoretical models mainly come from the medical and psychiatry fields, and thus focus on characterizing the behaviors of and barriers to young adults' help-seeking. Thus, this study tries to integrate the psychological model into the technology used in mental health help-seeking. 

\subsection{Marginalized Populations' Care-Seeking Behaviors}
\one{HCI, CSCW, and ICT4D scholars have increasingly focused their attention on marginality and the empowerment of society's most vulnerable populations ____. Marginalization manifests in multiple dimensions, encompassing economic hardship, gender discrimination, ethnic exclusion, geographical isolation, educational disparities, and political disenfranchisement ____. These structural inequalities profoundly shape individuals' aspirations and life trajectories, with intersecting identities of race, class, gender, and disability ____ creating complex barriers that diminish both the intention and capacity to seek care ____.}

\one{The reality of limited resources presents immediate and tangible barriers for marginalized populations seeking healthcare services. Evidence demonstrates that individuals in remote rural areas face reduced access to essential services, such as helplines ____, and utilize significantly fewer health services ____. This perception of resource scarcity may also contribute to a deeper pattern of symptom normalization and treatment avoidance ____. For example, \textcite{pendse_marginalization_2023} found that online mental health narratives from people in low-resource areas tend to emphasize somatic expressions of distress through body-focused language and have their experiences stigmatized and invalidated. Such evidence suggests that the impact of marginality extends beyond mere resource limitations, manifesting in more nuanced ways in people's identity perception and throughout the care-seeking process.}

\one{HCI research has documented the complex interactions between individuals with marginalized identities and mental health technologies ____, revealing significant cultural and social barriers to care. For instance, \textcite{bhattacharjee_whats_2023} illuminated how Indian users express profound skepticism toward talk therapy rooted in Western psychological paradigms, highlighting fundamental cultural disconnects in mental health interventions. Studies also demonstrated how perceived discrimination based on marginalized identities actively deters individuals from seeking clinical care ____. These negative experiences create lasting psychological barriers to healthcare access ____, initiating self-reinforcing cycles that further alienate individuals from mental health resources. The impact of these barriers extends into digital spaces, as \textcite{feuston_everyday_2019} documented how social media algorithms and platform affordances shape narratives around mental illness. These technological mechanisms, influenced by community norms that equate mental illness with deviance, inadvertently amplify and perpetuate stigma in online spaces, creating additional obstacles to care-seeking behaviors.}

\one{While existing research has examined marginalized populations' interactions with specific technologies, there remains a critical gap in understanding how individuals with marginalized identities navigate their broader care-seeking journey and engage with diverse available resources ____. This comprehensive understanding is crucial for developing culturally sensitive designs that align with ecological systems ____, cultural values ____, and community preferences ____.}

%Our study addresses this gap by investigating how marginalized populations negotiate the complex interplay between their personal identity, disadvantaged social position, and the healthcare system in their pursuit of care. This research aims to inform more inclusive and effective healthcare interventions that acknowledge and accommodate the unique challenges faced by marginalized communities.
%____
%People's real-world health resource utilization  ____ are enacted in interacting with complicated situations. 
\subsection{Ecological Approach for Technology-Mediated Care-Seeking}
\one{Recent HCI scholarship has expanded beyond studying users' interactions with isolated personal information systems to examine them within complex social ecologies ____. For example, \textcite{wong_mental_2023} examined how workplace environments shape and limit engineers' engagement with mental health resources.  \textcite{theofanopoulou_exploring_2022} studied tangible toys as mediators for parent-child emotional communication in family settings.
Bronfenbrenner's social systems framework ____ has been applied in such HCI research to examine the layered context of health resource interactions. Researchers have used its components—microsystem, macrosystem, and chronosystem—to study social influences on technology-mediated mental health management ____.}

\one{This ecological perspective has particular relevance for marginalized populations who face distinct socio-economic considerations ____. For instance, \textcite{tachtler_unaccompanied_2021} demonstrated how macro-level factors, particularly resettlement policies, constrain unaccompanied migrant youths' engagement with mental health mobile applications. Our study extends this ecological approach to examine how marginalized young adults navigate mental health resources within their lived contexts. }

We draw on Social Ecological Theory ____ to examine young adults' mental health navigation. The framework outlined by \textcite{mcleroy_ecological_1988} identified five levels affecting health behavior: individual characteristics, social relationships, organizational factors, community characteristics, and societal factors including physical, social, and political environments ____. This framework has been used to study individuals' interactions with diverse health resources and technologies ____. Among its variants, we use the version by ____ as a pragmatic framework, which is detailed in 3.3 to analyze our participants' resource interactions.

\one{Building on this ecological perspective, our study investigates marginalized young adults' care-seeking practices across digital platforms, institutional services, and informal support networks. By examining how they choose, access, and perceive these resources within their socioeconomic constraints, we aim to identify barriers and enablers in the current care ecosystem that can inform the design of mental health technologies and services for marginalized young adults ____.}

%Barriers
%Despite the ample resources, young adults are still facing multifaceted and intertwined obstacles in mental health help-seeking. Avoiding and delaying help-seeking are two common psychological deterrents at the individual level ____. Factors such as a lack of trust in healthcare systems ____, a tendency towards self-reliance____, and limited mental health literacy ____ all contribute to the reluctance to reach out for support. Such barriers are deeply rooted in broader contexts at the social ____, cultural ____, and structural ____ levels. Existing research has emphasized the enduring issue of stigmatization, highlighting how pervasive social norms deter people from openly discussing struggles and seeking help for mental health concerns ____. %Structural barriers, such as accessibility and availability of mental health resources, concerns about service suitability, and therapist-caused harms, emerge as substantial hindrances ____. %Underserved populations such as race minority ____ have additionally limited options for care.

%To address the barriers to help-seeking and support efficient resource utilization, this study aimed to provide a conceptual framework to categorize the diverse resources and systematically understand young adults' resource use practices. 
%This framework should be based on the lived experiences of young adults.


%First contact with health care has significant influences on the following help-seeking intentions. Perceived helpfulness of websites for mental health information ________

%Further, while they offered a structural perspective to examine the streams of retainment, loss, and transformation of potential patients, they failed to uncover the interactions and experiences of patients in contact with each access point, which could shed light on the barriers and facilitators at each center. 

%\textbf{Aims: There are psychiatric models of mental health help-seeking and studies testing the efficacy of single and standalone technologies, Our study aimed to merge the gap between these studies and integrate the role of technology into the behavioral models of help-seeking. }