\documentclass[acmsmall]{acmart}
%\documentclass[manuscript,review,anonymous]{acmart} 
%acmsmall,
\usepackage{longtable}
\usepackage{colortbl}
\usepackage{multirow}
\usepackage{subcaption}
\usepackage{csquotes}
\usepackage{color}
\usepackage{tabularray}

% Still need to single quote `like this' or \enquote*{like this}
\renewcommand{\mkblockquote}{\small\itshape}
\NewDocumentCommand{\pquote}{+O{} +m}{\blockquote[#1]{\textit{#2}}}

\newcommand{\elide}[1]{\textelp{}} % produces ... instead of argument
\NewDocumentCommand{\textrep}{+m +m}{\textins{#2}} 

\newcommand{\one}[1]{\textcolor{black}{#1}}
% Usage: \textrep{a long quote}{this.} => [this.]

% \textins{text} = [text]
% \textins*{T}ext = [T]ext
% text\textdel{s} = text[]

\newcommand{\textcite}[1]{\citeauthor{#1}~\cite{#1}}

\begin{document}

\title{"When I lost it, they dragged me out": How Care Encounters Empower Marginalized Young Adults' Mental Health Care-Seeking}

\author{Jiaying Liu}
\affiliation{%
  \institution{The University of Texas at Austin}
  \country{USA}}
\email{jiayingliu@utexas.edu}

\author{Yan Zhang}
\affiliation{%
  \institution{The University of Texas at Austin}
  \city{Austin}
  \country{USA}
}
\email{yanz@utexas.edu}
\renewcommand{\shortauthors}{Jiaying Liu \& Yan Zhang}

\begin{abstract}
CSCW research has long explored ways to enhance the well-being of marginalized populations. This study examines how young adults with diverse marginalized identities navigate their mental health care-seeking journeys through in-depth interviews and visual elicitation methods. Our research with 18 U.S. participants reveals predominantly \textit{passive} behavioral patterns shaped by participants' lived experiences of marginalization.
A key finding centers on \textit{"care encounters"} - serendipitous interactions with mental health resources that occur when individuals are not actively seeking support. These encounters emerged as critical turning points, catalyzing shifts from passive to proactive care-seeking behaviors. The transformative impact of these encounters operated through three primary mechanisms: tangible assistance, supportive discourse, and social connection building. Our analysis illuminates the infrastructural conditions that enable such transformative encounters and their effectiveness in connecting marginalized young adults with mental health care.
This research advances our understanding of how marginalization influences care-seeking behaviors while providing concrete design implications for socio-technical interventions. Our findings suggest strategies for intentionally engineering care encounters that can help overcome passive behavioral patterns and strengthen marginalized young adults' engagement with mental health resources.
\end{abstract}

%Guided by Social Ecological Theory, we proposed a Socio-technical Ecosystem Framework for mental health care, consisting of four levels of resources, including technological-, interpersonal-, community-, and societal level resources.  
% \ccsdesc[500]{Computer systems organization~Embedded systems}
% \ccsdesc[300]{Computer systems organization~Redundancy}
% \ccsdesc{Computer systems organization~Robotics}
% \ccsdesc[100]{Networks~Network reliability}

\keywords{Help-Seeking, Socio-technical ecosystem, Qualitative, Pathway to Care, Community Health, Social Ecological Theory}

\maketitle

\section{Introduction}
The mental health crisis continues to escalate both in the United States and globally. According to the National Institute of Mental Health (2023) \cite{national_institute_of_mental_health_mental_2023}, approximately 57.8 million U.S. adults lived with a mental illness in 2021. Young adults aged 18-25 demonstrate the highest prevalence of mental health concerns (33.7\%), yet paradoxically exhibit the lowest rate of seeking outpatient mental health treatment. This disparity becomes even more pronounced among young adults with marginalized identities, particularly racial minorities and LGBTQ+ individuals, who demonstrate significantly lower care-seeking behaviors \cite{hunt_mental_2010}.

\one{A substantial body of HCI and CSCW has addressed mental health support through various digital interventions, such as self-tracking tools \cite{wiljer_effects_2020, alqahtani_co-designing_2021, petelka_being_2020} and peer support platforms \cite{wong_postsecondary_2021, toscos_college_2018, zhang_online_2018}, targeting the general population. However, researchers have increasingly recognized that certain technological solutions may prove inadequate for marginalized populations, who often face limited access to technological resources \cite{hickey_smart_2021, miller_awedyssey_2023}. Moreover, these populations encounter multifaceted structural barriers in adopting such tools due to their unique cultural, social, and economic circumstances. For instance, \textcite{tachtler_unaccompanied_2021} identified specific micro and macro-level barriers that impede unaccompanied immigrant children's access to and utilization of mental health applications.}

\one{Existing research has documented significantly low technology adoption rates among marginalized young adults seeking mental health support \cite{gulliver_perceived_2010}. To design ecologically valid socio-technical interventions, it is crucial to understand the lived experiences shaping their care-seeking journeys. While prior work has examined various barriers of marginalized communities to access mental health care \cite{shahid_asian_2021, soubutts_challenges_2024}, there remains a critical gap in understanding how perceived marginalized identity influences care-seeking practices. This research addresses this gap by investigating two research questions:}

\one{
\begin{itemize}
    \item How do marginalized young adults navigate their care-seeking journey?
    \item How does the socio-tech ecosystem of resources facilitate or hinder care-seeking?
\end{itemize}
}

\one{We interviewed 18 U.S. young adults with diverse marginalized identities such as LGBTQ+, racial minorities, and homelessness. Our analysis reveals patterns of passiveness in participants' care-seeking behaviors, demonstrating how their marginalized positions and identities constrain access to care and shape their perceptions of mental health resources. These limitations not only restrict their care aspirations but also create isolation from support systems. }

\one{Importantly, we found that passive care-seeking practices can transform into proactive patterns through "care encounters" - serendipitous interactions that provide marginalized young adults with tangible assistance, supportive discourse, and social connection. Our analysis examines how institutional and infrastructural designs facilitate these transformative encounters. However, we also identified two critical barriers within the existing socio-technical ecosystem that can impede transitions toward proactive care-seeking: challenges in establishing trust and limitations in empowering individual agency.}

\one{
\begin{itemize}
\item We reveal through in-depth qualitative analysis how marginalization shapes and constrains young adults' mental health care-seeking journeys.
\item We identify and examine how "care encounters" transform care-seeking behaviors, illuminating the infrastructural conditions that enable these pivotal interactions.
\item We propose design strategies for engineering effective care encounters within the socio-technical ecosystem, addressing multiple levels of mental health support.
%: from individual technologies to community resources and institutional systems.
\end{itemize}
}
%\textbf{Terminology}. We acknowledge the limitations of the term of marginalized populations. We refer marginalized as a situation rather with the respect to their agency. 

\section{Related Work}
We first review the literature concerning general behavior models around mental health care-seeking and then center on the practices of marginalized populations. We then elaborate on the ecological view of mental health care-seeking which serves as an overarching methodology approach that supports our analysis. 

\subsection{Seeking Care for Mental Health Concerns} 
Care-seeking is also referred to as help-seeking, denoting people's behavior of actively obtaining assistance from various sources \cite{rickwood_conceptual_2012}. The assistance includes suggestions, information, and treatment that serve pragmatic purposes and also social and emotional support and understanding \cite{lachmar_mydepressionlookslike_2017}. 

One of the earliest models focused on formal help-seeking in the medical system. \textcite{huxley_mental_1996} outlined pathways that a patient may need to navigate through to get care. The pathway consists of five layers. Potential patients entered the pathway from general practitioners, via whom they were referred to primary care, then secondary psychiatric care, and ended with admission to the hospital. The following studies identified other resources that act as referrals within \cite{gater_pathways_1991} and outside \cite{bhui_mental_2002} of the medical system, such as native or religious healers, nurses, psychiatric services, and police institutions.
The model does not sufficiently consider external factors such as socio-economic status, cultural influences, or the role of informal support networks, which can significantly impact the care-seeking journey. 

Rickwood’s Stages of help-seeking framework \cite{rickwood_young_2005} describes patients' subjective experiences of the help-seeking process. The model delineates four key steps: (1) recognizing symptoms and realizing the need for assistance, (2) expressing symptoms and signaling the need for support, (3) identifying available and accessible sources of help, and (4) the final step contingent on the individual's willingness to disclose difficulties to the chosen source. The stages are presented as static categories, potentially overlooking the dynamic nature of help-seeking over time. 

Seeking help for mental health is not a quick and simple decision; rather, it is a complex and dynamic process dubbed with stigmatization and solitary feelings \cite{lannin_does_2016}. In the process, people constantly negotiate with themselves "if seeking help is necessary" and "when and how to seek help". \textcite{biddle_explaining_2007} delineated this struggle through the Cycle of Avoidance (COA) model, depicting the tensions involved in making sense of, accepting, and avoiding mental distress. They identified several key actions that postpone help-seeking, including normalizing symptoms, proposing alternative explanations, accommodating increasingly severe distress, pushing the threshold between “normal” and “real” distress, and delaying help-seeking. Other studies also pinpointed similar cognitive strategies such as utilizing self-resilience, denying the effects of professional help, and problematizing help-seeking \cite{martinez-hernaez_non-professional-help-seeking_2014, abavi_exploration_2020}. 

\one{While these models effectively capture the complexities of mental health care-seeking behaviors and stages, they do not examine how individuals interact with specific health resources and technologies. Moreover, these models are primarily based on general population studies, leaving unclear how marginalized identities shape care-seeking practices and resource engagement.}

%These empirical studies and theoretical models mainly come from the medical and psychiatry fields, and thus focus on characterizing the behaviors of and barriers to young adults' help-seeking. Thus, this study tries to integrate the psychological model into the technology used in mental health help-seeking. 

\subsection{Marginalized Populations' Care-Seeking Behaviors}
\one{HCI, CSCW, and ICT4D scholars have increasingly focused their attention on marginality and the empowerment of society's most vulnerable populations \cite{pearce_socially-oriented_2020, wyche_learning_2012, devito_social_2019, pendse_mental_2019}. Marginalization manifests in multiple dimensions, encompassing economic hardship, gender discrimination, ethnic exclusion, geographical isolation, educational disparities, and political disenfranchisement \cite{pal_marginality_2013}. These structural inequalities profoundly shape individuals' aspirations and life trajectories, with intersecting identities of race, class, gender, and disability \cite{das_studying_2023} creating complex barriers that diminish both the intention and capacity to seek care \cite{crenshaw_mapping_2013}.}

\one{The reality of limited resources presents immediate and tangible barriers for marginalized populations seeking healthcare services. Evidence demonstrates that individuals in remote rural areas face reduced access to essential services, such as helplines \cite{pendse_can_2021}, and utilize significantly fewer health services \cite{williams_analysis_2021}. This perception of resource scarcity may also contribute to a deeper pattern of symptom normalization and treatment avoidance \cite{shahid_asian_2021}. For example, \textcite{pendse_marginalization_2023} found that online mental health narratives from people in low-resource areas tend to emphasize somatic expressions of distress through body-focused language and have their experiences stigmatized and invalidated. Such evidence suggests that the impact of marginality extends beyond mere resource limitations, manifesting in more nuanced ways in people's identity perception and throughout the care-seeking process.}

\one{HCI research has documented the complex interactions between individuals with marginalized identities and mental health technologies \cite{oguamanam2023intersectional}, revealing significant cultural and social barriers to care. For instance, \textcite{bhattacharjee_whats_2023} illuminated how Indian users express profound skepticism toward talk therapy rooted in Western psychological paradigms, highlighting fundamental cultural disconnects in mental health interventions. Studies also demonstrated how perceived discrimination based on marginalized identities actively deters individuals from seeking clinical care \cite{lu_barriers_2021, gulliver_perceived_2010}. These negative experiences create lasting psychological barriers to healthcare access \cite{liu_exploring_2024}, initiating self-reinforcing cycles that further alienate individuals from mental health resources. The impact of these barriers extends into digital spaces, as \textcite{feuston_everyday_2019} documented how social media algorithms and platform affordances shape narratives around mental illness. These technological mechanisms, influenced by community norms that equate mental illness with deviance, inadvertently amplify and perpetuate stigma in online spaces, creating additional obstacles to care-seeking behaviors.}

\one{While existing research has examined marginalized populations' interactions with specific technologies, there remains a critical gap in understanding how individuals with marginalized identities navigate their broader care-seeking journey and engage with diverse available resources \cite{robards_how_2018}. This comprehensive understanding is crucial for developing culturally sensitive designs that align with ecological systems \cite{burgess_i_2019}, cultural values \cite{li_sunforum_2016}, and community preferences \cite{das_studying_2023}.}

%Our study addresses this gap by investigating how marginalized populations negotiate the complex interplay between their personal identity, disadvantaged social position, and the healthcare system in their pursuit of care. This research aims to inform more inclusive and effective healthcare interventions that acknowledge and accommodate the unique challenges faced by marginalized communities.
%\cite{Seeking in Cycles: How Users Leverage Personal Information Ecosystems to Find Mental Health Information}
%People's real-world health resource utilization  \cite{andersen_behavioral_1968,andersen_revisiting_1995,babitsch_re-revisiting_2012} are enacted in interacting with complicated situations. 
\subsection{Ecological Approach for Technology-Mediated Care-Seeking}
\one{Recent HCI scholarship has expanded beyond studying users' interactions with isolated personal information systems to examine them within complex social ecologies \cite{murnane_personal_2018, siddiqui_exploring_2023, ongwere_challenges_2022}. For example, \textcite{wong_mental_2023} examined how workplace environments shape and limit engineers' engagement with mental health resources.  \textcite{theofanopoulou_exploring_2022} studied tangible toys as mediators for parent-child emotional communication in family settings.
Bronfenbrenner's social systems framework \cite{bronfenbrenner1979ecology} has been applied in such HCI research to examine the layered context of health resource interactions. Researchers have used its components—microsystem, macrosystem, and chronosystem—to study social influences on technology-mediated mental health management \cite{burgess_i_2019, tachtler_unaccompanied_2021, murnane_personal_2018}.}

\one{This ecological perspective has particular relevance for marginalized populations who face distinct socio-economic considerations \cite{kaziunas_precarious_2019}. For instance, \textcite{tachtler_unaccompanied_2021} demonstrated how macro-level factors, particularly resettlement policies, constrain unaccompanied migrant youths' engagement with mental health mobile applications. Our study extends this ecological approach to examine how marginalized young adults navigate mental health resources within their lived contexts. }

We draw on Social Ecological Theory \cite{stokols_translating_1996} to examine young adults' mental health navigation. The framework outlined by \textcite{mcleroy_ecological_1988} identified five levels affecting health behavior: individual characteristics, social relationships, organizational factors, community characteristics, and societal factors including physical, social, and political environments \cite{mccloskey_principles_2011}. This framework has been used to study individuals' interactions with diverse health resources and technologies \cite{sallis_ecological_2008, trace_information_2023}. Among its variants, we use the version by \cite{centers_for_disease_social-ecological_2022} as a pragmatic framework, which is detailed in 3.3 to analyze our participants' resource interactions.

\one{Building on this ecological perspective, our study investigates marginalized young adults' care-seeking practices across digital platforms, institutional services, and informal support networks. By examining how they choose, access, and perceive these resources within their socioeconomic constraints, we aim to identify barriers and enablers in the current care ecosystem that can inform the design of mental health technologies and services for marginalized young adults \cite{c_feasibility_2022, wong_postsecondary_2021}.}

%Barriers
%Despite the ample resources, young adults are still facing multifaceted and intertwined obstacles in mental health help-seeking. Avoiding and delaying help-seeking are two common psychological deterrents at the individual level \cite{gulliver_perceived_2010}. Factors such as a lack of trust in healthcare systems \cite{macdonald_pathways_2018}, a tendency towards self-reliance\cite{chan_university_2016}, and limited mental health literacy \cite{mahmoodi_mental_2022} all contribute to the reluctance to reach out for support. Such barriers are deeply rooted in broader contexts at the social \cite{j_people_2018}, cultural \cite{lynch_young_2018,han_mental_2015}, and structural \cite{ellinghaus_im_2021} levels. Existing research has emphasized the enduring issue of stigmatization, highlighting how pervasive social norms deter people from openly discussing struggles and seeking help for mental health concerns \cite{andalibi_self-disclosure_2017}. %Structural barriers, such as accessibility and availability of mental health resources, concerns about service suitability, and therapist-caused harms, emerge as substantial hindrances \cite{bogia_institutional_2018}. %Underserved populations such as race minority \cite{lu_barriers_2021} have additionally limited options for care.

%To address the barriers to help-seeking and support efficient resource utilization, this study aimed to provide a conceptual framework to categorize the diverse resources and systematically understand young adults' resource use practices. 
%This framework should be based on the lived experiences of young adults.


%First contact with health care has significant influences on the following help-seeking intentions. Perceived helpfulness of websites for mental health information \cite{oh_perceived_2009}\cite{gonsalves_design_2019}

%Further, while they offered a structural perspective to examine the streams of retainment, loss, and transformation of potential patients, they failed to uncover the interactions and experiences of patients in contact with each access point, which could shed light on the barriers and facilitators at each center. 

%\textbf{Aims: There are psychiatric models of mental health help-seeking and studies testing the efficacy of single and standalone technologies, Our study aimed to merge the gap between these studies and integrate the role of technology into the behavioral models of help-seeking. }

\section{Methods}
We adopted in-depth interviews to understand the lived experiences of young adults’ mental health help-seeking practices.
\subsection{Recruitment and participants}
We disseminated recruitment messages on the listserv of a southern university in America as previous studies suggested that university students face high and multifaceted pressure \cite{hunt_mental_2010}. \one{To recruit young adults from diverse backgrounds, we sourced participants through Reddit, aiming to include individuals from marginalized communities who are often underrepresented in university-based studies. Following Reddit's community guidelines, we posted recruitment messages in 12 regional marketplace and job-focused subreddits with large memberships across the United States, including r/SanAntonioJobs and r/NewOrleansMarketplace.}

Interested participants ages 18-25 were invited to finish a screening questionnaire where we asked for their demographic information, such as age, gender, education, marginalized identities (e.g., first-generation college students, low-income, LGBTQ+, race minority), mental illness diagnosis, help-seeking practices (e.g., previously used resources and technologies such as family, friends, social media, and apps), and an open-ended question about the most recent/impressive mental health help-seeking experience. We also include the PHQ-4 \cite{kroenke_ultra-brief_2009}, a validated screening tool for depression and anxiety to get a rough estimate of their current mental health status. We intentionally chose participants to maximize the sample demographic diversity and mental health status.
\one{Following the theoretical sampling strategy \cite{charmaz2000grounded}, our recruitment process proceeded in parallel with data analysis to include participants with diverse help-seeking practices, including varied preferred sources and both satisfying and dissatisfying experiences. For instance, we noticed the prominent role of resources actively reaching out to marginalized young adults based on the first twelve interviews. To collect more cases and further understand the care-outreaching, the team carefully selected subsequent participants who reported relevant experiences in the screening survey and invited them to join the interview.}
%It consists of four questions asking about the frequency of feelings of "nervous, anxious or on edge", "not being able to stop or control worrying," "little interest or pleasure in doing things," and "feeling down, depressed, or hopeless" over the last two weeks. 

As shown in Table \ref{tab:participants}, 18 participants were interviewed (7 females, 9 males, and 2 trans/non-binary; 6 White, 5 Black, 4 Asian, and 3 Hispanic). Half of the participants were still in university at the time of the interviews, 6 had graduated from colleges, and 3 had no college education. Thirteen participants had formal diagnoses of mental illness, mainly depression and anxiety. Their anxiety and depression status indicated by PHQ-4 ranged from mild (3-5), moderate (6-8) to severe (9-12). 

\begin{table}[h]
\caption{Participants Information}
\label{tab:participants}
\resizebox{\columnwidth}{!}{%
\begin{tabular}{|l|l|l|l|l|l|l|l|l|}
\hline
\textbf{ID} &
  \textbf{Age} &
  \textbf{Sex} &
  \textbf{Education} &
  \textbf{Race} &
  \textbf{Diagnosis} &
  \textbf{PHQ4} &
  \textbf{Resources Used in Mental Health Help-seeking} \\ \hline
P01 &
  20 &
  F &
  Some college &
  Black &
  depression &
  \cellcolor[HTML]{8AC97D}5 &
  professional help, search engine, friend, family \\ \hline
P02 &
  24 &
  M &
  High school &
  Black &
  depression &
  \cellcolor[HTML]{FFEB84}8 &
  social media, online   communities, friends, professional help \\ \hline
P03 &
  20 &
  F &
  Some college &
  Asian &
  no &
  \cellcolor[HTML]{63BE7B}4 &
  social media, search engine, friends, family, mobile apps \\ \hline
P04 &
  23 &
  M &
  Some college &
  White &
  Anxiety &
  \cellcolor[HTML]{F8696B}12 &
  \begin{tabular}[c]{@{}l@{}}Family, friends, professionals, online communities, services in the local \\ community\end{tabular} \\ \hline
P05 &
  22 &
  M &
  8 through 11 years &
  Black &
  anxiety &
  \cellcolor[HTML]{F8696B}12 &
  family, friend, professional, social media, online communities, telehealth \\ \hline
P06 &
  22 &
  F &
  Some college &
  Asian &
  depression &
  \cellcolor[HTML]{FFEB84}8 &
  telehealth, professional help \\ \hline
P07 &
  21 &
  F &
  Some college &
  Hispanic &
  \begin{tabular}[c]{@{}l@{}}Depression, \\ Anxiety\end{tabular} &
  \cellcolor[HTML]{63BE7B}4 &
  Friends, professors, people with   similar experiences \\ \hline
P08 &
  23 &
  F &
  College graduate &
  Asian &
  Depression &
  \cellcolor[HTML]{D8DF81}7 &
  \begin{tabular}[c]{@{}l@{}}Family, friends, people with similar experiences, Social media, \\ online mental health forums or communities, Search engines, \\ Teletherapy services\end{tabular} \\ \hline
P09 &
  24 &
  M &
  College graduate &
  White &
   &
  \cellcolor[HTML]{8AC97D}5 &
  Family, friends, Reddit \\ \hline
P10 &
  24 &
  M &
  College graduate &
  Asian &
  no &
  \cellcolor[HTML]{63BE7B}4 &
  Hotline, family \\ \hline
P11 &
  22 &
  M &
  Some college &
  White &
  Depression &
  \cellcolor[HTML]{FCAA78}10 &
  Family, professionals, social media, search engine, teletherapy \\ \hline
P12 &
  20 &
  F &
  Some college &
  White &
  Depression &
  \cellcolor[HTML]{F8696B}12 &
  \begin{tabular}[c]{@{}l@{}}Family, professionals, people with similar experiences, social media, \\ online communities, teletherapy, hotlines, search engines\end{tabular} \\ \hline
P13 &
  22 &
  M &
  Some college &
  White &
  Depression &
  \cellcolor[HTML]{FECB7E}9 &
  Family, friends, search engines, mobile applications \\ \hline
P14 &
  21 &
  F &
  College graduate &
  Hispanic &
  Depression &
  \cellcolor[HTML]{8AC97D}5 &
  Professionals, online communities \\ \hline
P15 &
  25 &
  \begin{tabular}[c]{@{}l@{}}Trans\end{tabular} &
  Some college &
  Hispanic &
  \begin{tabular}[c]{@{}l@{}}Depression, PTSD,\\anxiety, ADHD\end{tabular} &
  \cellcolor[HTML]{FA8A72}11 &
  Family, professionals, social media, search engine \\ \hline
P16 &
  24 &
  Trans &
  College graduate &
  Black &
  Depression &
  \cellcolor[HTML]{FFEB84}8 &
  friends, professional, stranger \\ \hline
P17 &
  21 &
  M &
  High school &
  Black & no
   & 
  \cellcolor[HTML]{B1D47F}6 &
  \begin{tabular}[c]{@{}l@{}}Family, friends, professionals, social media, search engine, \\ teletherapy, services in the local community\end{tabular} \\ \hline
P18 &
  22 &
  M &
  College graduate &
  White &
  no &
  \cellcolor[HTML]{FCAA78}10 &
  Family, friends, social media, mobile apps, services in local communities \\ \hline

\end{tabular}%
}
\footnotesize Notes: PTSD: Post-traumatic stress disorder; ADHD: Attention-deficit/hyperactivity disorder
\end{table}

\subsection{Interview Procedure}
The interviews averaged a duration of 1 hour and 20 minutes and each interviewee was compensated with a \$30 Amazon gift card. The interviews took place virtually via Zoom from January to September 2023. Before each interview, we introduced the scope of this study and answered any questions concerning the informed consent form. We also assured the participants that the interviews would be anonymous, confidential, and non-judgmental, encouraging them to be as open as they wanted.

We began the interviews by asking participants' current jobs, residential areas, and other background information. Then, the interviewer asked participants to describe their current mental health status. Example questions included, “Could you tell me about your mental health concerns?” “When did you first notice that?” and “How did it evolve over time?” 

We utilized \textbf{visual elicitation techniques} for a richer exploration of their (non-)help-seeking journeys \cite{chen_timeline_2018}. When participants started to dive into specific story-telling, we asked them to draw their experiences of when, where, and how they sought help. Participants were encouraged to take some time to recall their experiences and present their stories in any visualization format. Our participants drew their help-seeking experiences from different perspectives, shown in Fig \ref{fig:examplemaps}. The interviewer then asked follow-up questions based on the drawings to elicit details, for example, “What did you do to cope with [a specific event] in the drawing?” “Did you talk with any people (e.g., family, friends) or use any technology (e.g., social media groups, mobile apps, hotlines)?” and "Why did you decide to do that?" In this process, the interviewer paid special attention to how they perceived and used different resources to seek help. The interviewer also encouraged participants to continually add contexts to the drawings as they narrated. Participants were instructed to focus the camera on the drawing to share with the interviewer during this process. 

After the interview, we asked participants to reflect on their experiences and evaluate their satisfaction with each mentioned resource using a five-point scale (1- very unsatisfied; 5- very satisfied). We then invited them to share the most satisfying and unsatisfying experiences, the challenges they encountered during help-seeking, and the ideal help they wished to get.
% reminded them to keep developing the drawing prompts like "they are great visuals to understand your story. Can you keep drawing as you tell me?" 
%After going through the help-seeking experiences on the journey map, we screen shared a checklist that aggregated resources listed in the U.S. Surgeon General’s Advisory \cite{office_of_the_surgeon_general_osg_protecting_2021} to remind participants of their interactions with other resources. 

\begin{figure}[h]

% \begin{subfigure}{0.3\textwidth}
% \centering
% \includegraphics[width=\linewidth]{images/Resources.png}
% \centering
% \caption{Visual prompts during interviews}
% \label{fig:visualprompts}
% \end{subfigure}
% \begin{subfigure}{0.6\textwidth}
\includegraphics[width=.8\linewidth]{images/examplemaps.jpg}
\caption{Examples of Journey Maps Drawn by Participants. P02 and P07 presented their help-seeking timeline, P09 demonstrated his resources network, and P10 quantified his satisfaction with each resource using self-defined mathematic formulas.}
\footnotesize
\Description{
 Examples of Journey Maps Drawn by Participants. P02 and P07 presented their help-seeking timeline, P09 demonstrated his resources network, and P10 quantified his satisfaction with each resource using self-defined mathematic formulas.}
\label{fig:examplemaps}
% \end{subfigure}

% \caption{Interview materials}
% \label{fig:interviewmaterials}
\end{figure}

\textbf{Ethical concerns}
This study was approved by the University Institutional Review Board (IRB). To protect the safety of our participants, in the informed consent form and at the beginning of interviews, we reminded the participants that they should feel free to take breaks during the interview and that they could exit the interview at any time. During the interview process, we kept sensitive to participants’ emotional changes when difficult experiences were disclosed and checked whether they would like to continue when negative emotions were observed \cite{draucker_developing_2009}. After the interview, we provided a list of mental health resources to the participants for future use, which was considered helpful and appreciated by many participants. 

\subsection{Data Analysis}
All the interviews were conducted by the first author and were audio recorded and transcribed. The interviewer wrote debriefs immediately after each interview. We used iterative open coding and axial coding methods \cite{corbin_basics_2014} to systematically analyze the transcripts. The analysis was assisted by using NVivo, a qualitative content analysis software. This iterative process continued until the twelfth interview, at which point the codes reached a stable state, forming the foundation of our initial open coding schema. The open coding schema includes codes about resource type (e.g., technologies, family, professionals), support type (e.g., informational support, distraction), and challenges (e.g., high-cost, hard to talk). 

We followed the Social Ecological Theory \cite{stokols_translating_1996} 
%to conduct \enquote{ecological analyses characterize environmental settings as having multiple physical, social, and cultural dimensions that can influence a variety of health outcomes.} 
and categorized these resources into four levels: technological, interpersonal, community, and societal levels. The technological level emerged as a new category through our data analysis, while the other three levels were previously discussed in existing literature \cite{stokols_translating_1996,thompson_social_1990,seligman_depression_1975}.

Using this framework, we began axial coding to reorganize codes and generate themes and sub-themes related to participants' practices and challenges in seeking help. The two authors convened weekly meetings to review new interview debriefs, reflect on emerging themes in coding, compare codes with existing literature, and document these reflections in notes and memos. Through discussions, we purposely selected the next interviewees to further test and refine the framework. For instance, since societal resources were less frequently mentioned by earlier participants, we deliberately chose participants who utilized societal resources, focusing on how they defined, accessed, and perceived various societal services in comparison to resources at other levels. The initial open code "support type" evolved into the theme "support mechanisms" with three subthemes in axial coding. In addition, we systematically mapped resources used by participants harnessing the Socio-technical Ecosystem Framework, identifying two types of support systems, various pathways, and barriers in help-seeking. 

\one{\subsection{Positionality} 
Our research on mental health help-seeking behaviors and resource utilization is part of a broader effort to support young adults from diverse backgrounds through innovative technological solutions and social support. This work is informed by the lead author's lived experience with mental illness and involvement in peer support groups for young adults facing mental health challenges.}

\subsection{Limitations}
\one{Our study only included 18 participants who were young adults based in the U.S. who were willing to talk about their mental health concerns and help-seeking processes.} They might also have less self-stigma. Thus, we acknowledge that their help-seeking behaviors could not represent all young adults, particularly those who did not seek help or were unwilling to talk about their experiences. Future research can examine the non-help-seeking behaviors of young adults.

\section{Findings}

To understand how marginalized young adults navigate their care-seeking journey (RQ1), we mapped participants' interactions with healthcare resources, revealing complex relationships between marginalized identity, mental health understanding, and care-seeking behaviors. In examining how resources facilitate or hinder care-seeking (RQ2), we identified pivotal "care encounters" that can catalyze more proactive health-seeking behaviors. We also analyze the challenges within these encounters that may hinder participants' care-seeking.

\subsection{Participants' Resource Use Practices}

Our participants represent a diverse range of backgrounds and varying self-identified marginality. Seven participants identified as first-generation college students, eight identified as racial or gender minorities, seven reported coming from low-income backgrounds, and three reported other forms of marginality, such as disabilities or recent immigration. Notably, five participants reported experiencing intersectional marginality.

All of our participants' care-seeking journey spanned more than a year and they shared both positive and negative experiences with different resources. In Figure \ref{fig:supportsystem}, we map their satisfaction with each resource they have interacted with based on their responses to the interview question: \enquote{To what extent are you satisfied with [a specific resource]?}

\begin{figure}[h]
    \centering
    \includegraphics[width=\linewidth]{images/supportsystemMarginal.png}
    \caption{Participants' Interactions with Resources across Mental Health Care-Seeking Journey. Single Level Care denotes that participants have "satisfied" or "very satisfied" experiences with resources at one level; Multi-Level Care means that they have "satisfied" or "very satisfied" experiences with more than one type of resources.}
    \label{fig:supportsystem}
\end{figure}

As mentioned in Section 3.3, we followed the Social Ecological Theory \cite{stokols_translating_1996} and categorized these resources into four levels.
The \textit{technological level} includes search engines, online forums, video-sharing platforms, and mobile apps. These resources were frequently mentioned by participants and most of the use experiences were satisfying, but it should be noted that these resources had relatively small effects on participants. 
The \textit{interpersonal level} consists of participants' interpersonal ties. Except for P01, all other participants at least tried to seek help from family, friends, relatives, or partners, although six of them got neutral or unsatisfying feedback. 
The \textit{community level} contains all resources provided by entities that participants are members of, such as pre-college education institutions, universities, companies, and local neighborhoods. 11/18 participants tried community-level resources and six participants had varying degrees of satisfaction. 
The \textit{societal level} contains non-profit institutions and professional care including professional help from medical systems, the National Lifeline, and governmental institutions. Only 8 participants utilized these resources. 

%The aggregation of experiences presented in Figure \ref{fig:supportsystem} provides a static view and does not capture the temporal dynamics of participants' care-seeking journeys. 
As shown in Figure \ref{fig:supportsystem}, seven participants only received satisfying support from a single level of resources (i.e. single-level care). Among them, three  (P09, P13, and P15) primarily relied on technological solutions, two (P10 and P18) mainly sought emotional support from family members, and one (P08) only used resources from her community, the university mental health services. Notably, one participant (P01) did not report any significant experiences with care resources. The remaining eleven participants were able to navigate and obtain care from multiple levels of resources (i.e., multi-level care). We elaborate on participants' navigation of the socio-technical ecosystem in the following sections.

\subsection{Passive Behavioral Pattern of Marginalized Young Adults' Care-Seeking}
Our analysis reveals participants' approach in dealing with mental health concerns can be characterized as being passive. These individuals, shaped by their unique experiences and backgrounds, frequently engage in behaviors that delay or avoid the active pursuit of mental health support, that seemed to be rooted in the complex interplay of cultural, personal, and societal factors. 

\subsubsection{Reality: Social Isolation}

For many participants, marginalization extends beyond mental health concerns to encompass broader social isolation where participants don't have access to various types of care resources. Some participants face extreme situations, as illustrated by P02:
\pquote{when I was 17 years old, my parents separated and I was I had no place to live. I was forced to live with my uncle, who was very abusive to me. It was at that point I developed some mental concerns because of the abuse now I'm being detained being spoken many times being denied food. It's just been so tough for me, you know? I just felt very alone. It was the lowest point of my life. It's just terrible. It's just awful. I was just like, The Dark Side of the Moon you know? (P02)} 
Others struggle with the unavailability of family support. P03 shared: \pquote{It's really busy with school and with family. My mom is struggling with back issues. My grandma, she has Psoriasis, so she has to get shots. My dad had heart surgery last summer. And then the only one that is currently able to help is my eldest sister there, who is now a single mom starting to date a new guy and it's all a struggle. So I try to be at least talking to them. (P03)}

Social disparities they are experiencing further exacerbate these challenges, as expressed by P18, \pquote{I have a lot of friends that are rich. And they come from a rich family. And it go around the go with yourself. But I can join them because I always feel I don't belong... discrimination and some other things that might. Of the ceiling. Of the So if one's information, get, so, it's kind of one, I think, they'll use it again somewhere. (P18)}

\subsubsection{Perception: Low aspirations}

A prevalent lack of aspiration for care was observed among many participants, including devaluing the importance of mental health and disbelief in professional help. Cultural influences were one force that perpetuates such internalizations, as one participant shared that \pquote{Mental health isn't really a thing in Asian communities. It's not really like shown by parents. I'm sure that there was one time when I was younger that I was sad and then my mom or dad was just, deal with it by yourself. And then that's what I do. I deal with it internally. (P01)}

The low aspiration also frequently stemmed from a deep-seated disbelief in institutional-level care provisions or negative experiences with organizations. P02 expressed skepticism about governmental resources:
\pquote{because you don't see these things are gonna happen. This is from the government, so the question is that these resources are very limited and you don't think you probably can get this. (P02)}
The inaccessibility of information further exacerbates this issue, as noted by P15, \pquote{they [institutions] did not make the information accessible in a way that you can really get the support.}

\subsubsection{Behavior: Waiting and concealing}

Marginalized young adults often conceal their mental health concerns and adopt a passive stance toward seeking help. They frequently wait for what they perceive as the "right opportunity" to address their mental health issues. This ambiguous pattern is common, as exemplified by P10 who expressed that he would treat his occasional depressive episodes more seriously "later" but did not have a concrete plan, 
\pquote{I mean, I might tell my parents when it's the right time.}
Concealing mental health situations might stem from a view that mental health struggles are personal challenges to be overcome independently, as P03 articulated:
\pquote{I think it really also is important to have sort of a self-serving with mental health because you're with yourself for the rest of your life, right? Yes, it's important to have a support system externally, but internally is more important. (P03)}

%Although passive, such behavior also suggest they are still hopeful for the care that not yet come.
%Some have more positive \pquote{I haven't found a therapist that's a good fit, but I'm hopeful. (P15)}

\subsection{The "Care Encounters" as Moments of Change}
While participants' care-seeking behaviors may appear passive, they reflect a nuanced approach of receptive anticipation rather than active avoidance of care resources. Our analysis revealed "care encounters" - serendipitous moments when participants discovered or got exposed to previously unimagined resources. We identified three distinct types of encounters that transformed participants' relationships with care: those providing tangible assistance, those offering supportive narratives, and those facilitating social connections. These encounters helped marginalized young adults expand their aspirations for care and overcome their initial resource limitations. We further examined the affordances of the resource ecosystem that facilitated these transformative encounters.

%We found that in participants' interactions with the care resource ecosystem, there are opportunities that the ecosystem facilitate the aspirations, create open discourses about mental health, and foster supportive social groups that  We elaborate institutional agency and/or individual's agency interact in this process and identify the mechanisms.

\subsubsection{Encountering Tangible Assistance}
The most direct and impactful encounters that reshape participants' perceptions and behavior of care-seeeking occur through tangible assistance, such as free therapy sessions, performance accommodations, and hospitalization services. We identified two circumstances that facilitate participants' engagement with tangible assistance: accessibility-driven encounters, where designated resource availability motivates engagement, and emergency-driven encounters, which occur during crisis situations.

\paragraph{Designated Accessibility}
Beyond mere availability, designated accessibility refers to how care resources thoughtfully woven into the fabric of participants' daily lives became more inviting and attainable. P08's experience exemplifies this transition in perception and behavior. During the isolating period of the COVID-19 pandemic, a series of email reminders about depression awareness became her unexpected lifeline:
\pquote{I think it was like halfway through Covid, I received consecutive emails day by day, saying to focus on your mental health. That was when I felt all the stressors were hitting on me all at once and I wondered why didn't I have a break from school? Let me, just, attend one of these Zoom Meetings, and see what they are about. Let me just check these resources out. It was during that time that I realized that I probably need to get some help because these are the symptoms.}
Following these initial encounters, P08 utilized the free therapy sessions introduced in the workshops and emails—a pathway also taken by four other participants. 

For instance, P06 benefited from a reduced class load; P07 was offered extended tuition payment options; and P08 maintained access to free therapy sessions even after graduation. These accommodations strengthened the incentive for care-seeking, demonstrating that inclusiveness extends beyond vague promises. Reflecting on this experience, P08 expressed surprise at the abundance of available resources: \pquote{\elide{I think I'm more familiar with the [university] resources, and }I wasn't previously aware that there are so many resources out there possible to help you, and was just waiting for, like reach out and stuff, until I was told I was offered it.}

%For instance, P12's experience illustrates the critical importance of availability in moments of crisis. He turned to the national lifeline when no other ones seems possible, \pquote{I just wanted to talk to someone that would listen to me... I think I got what I needed out of it, like just talking to someone because it was able to calm me down quite a bit.} 

\paragraph{Reactive interventions in emergencies}
Reactive interventions which refer to situation-triggered responses to acute mental health needs, play a crucial role in challenging and changing participants' biases or negative attitudes toward care-seeking. These interventions are particularly vital during emergencies such as suicidal ideation or severe mood episodes.
P07's experience with the university crisis line exemplifies the comprehensive approach taken by community resources. Upon learning about her self-harming tendencies, the staff mobilized a multi-faceted response: \pquote{I stayed in the hospital for two days...they diagnosed me with a major depressive disorder and generalized anxiety disorder, and then that's when they gave me medications to start to treat both of those.} This intervention not only provided immediate care but also set P07 on a path of ongoing treatment and support. It demonstrates how active interventions can bridge the gap between the awareness of resources and the actual engagement with mental health services, especially in critical situations.

Besides the community-provided mental health services, P16's story illustrates the proactive role of interpersonal relationships during a crisis. After both of her parents passed away, P16 isolated herself at home for two months, unaware that she might be experiencing depression. Her friends' persistent check-ins culminated in a crucial intervention:
\pquote{This was too much for me. They came in when I lost it during that time. They took me to the hospital, checked with the hospital, and showed that I really needed this [professional help]. Then they decided that I needed to change the environment and of course, I listened to them. They are my support network and I tell everyone to find your network. This is what truly supports you no matter what bad happens.}

\subsubsection{Encounter Supportive Discourses}
Supportive discourses can open conversations and messaging that normalize mental health discussions and validate care-seeking behaviors. Participants frequently mentioned encountering these conversations in spaces that created supportive environments for mental health narratives, which, often gradually, mitigate their perceived stigmas of mental health and facilitate their further engagement with mental health resources.

\paragraph{Inclusive Atmosphere}
As mentioned in Section 4.1, many participants faced stigmas due to their familial, cultural, or religious backgrounds. However, some eventually encountered inclusive environments and spaces that actively welcomed conversations around mental health experiences.

The university campus was frequently mentioned as a microcosm of this awareness-raising ecosystem. For instance, P10 and P13's observations of omnipresent hotline signs in hallways, offices, and dining halls paint a picture of an environment saturated with care-seeking encouragement. Additionally, participants reported finding such spaces across various channels, from digital platforms like emails (P08, P16) and social media (P11) to physical spaces hosting awareness events (P15, P06). These messages create a constant, gentle nudge toward mental health consciousness, normalizing conversations around mental health and subtly reinforcing the availability of support.
These inclusive encounters also facilitated discussions that were previously deemed impossible. P07, unable to grieve her grandfather's death within her family setting, finally found an opportunity to process her grief in a university event that had been held off since middle school.
%The process of encountering and embracing these inclusive spaces often took time, with participants' attitudes changing after multiple exposures. 
%P07's journey with medication exemplifies this process. Repeated encounters with recommendations from online searches gradually shifted her perspective, leading her to consider medication as a potential solution for her mental health concerns. 

%The role of family and friends in identifying early-stage symptoms cannot be overstated. Their daily interactions provide a unique vantage point for recognizing subtle changes in behavior or mood. For five participants in our study, this intimate awareness led to early interventions. The experiences of P07 and P16, who were taken to psychiatrists by their parents during elementary school, highlight the critical role of familial support in initiating professional care.

\paragraph{Intimate Conversations}
One major reason behind the waiting and concealing behaviors among marginalized young adults, mentioned by six participants in our study, was the challenge of initiating open conversations about mental health with family and friends. We observed a delicate tension between their desire for support and the difficulty in articulating their needs. P13's reflection captures this struggle: \pquote{I mean just it's hard to talk about it...I'm not good at talking to people about the hard stuff.}
In such situations, participants found breakthrough moments when family members and friends took the initiative to start these deeper conversations. These proactive gestures often opened doors to meaningful support. P14 shared how her mother's willingness to discuss her depression not only aided her recovery journey but also strengthened their relationship: \pquote{She was like, I was really scared you would do something... I feel like if she ever saw me in such a sad place... that she can understand, she knew about my situation, at least the general situations.}
%However, participants sometimes strategically chose partial disclosure of their true state. P14's perspective encapsulates this nuanced approach:\pquote{Obviously they [my friends] are gonna support me,}Yet, in the same breath, P14 admits, \pquote{actually I don't want to tell [them about] my mental distress.} This approach allows individuals to modulate their level of openness, sharing only what they feel comfortable revealing in any given situation. It's a delicate balance between seeking support and maintaining personal boundaries.

\subsubsection{Encountering Social Connections}
Another crucial type of encounter involves people who truly empower care-seeking. These relationships form a vital layer of care, providing comfort and understanding in navigating mental health challenges.

\paragraph{New Friends from In-person Events}
For many marginalized young adults, pivotal moments arise when they connect with peers, friends, or even strangers who offer transformative support, both immediate and long-lasting. P12 found solace in a therapist whose care evolved into a nurturing paternal connection: \pquote{I feel connected with him. He talks to me like a daughter because he has a daughter like me. I enjoyed going to play with his family and his children.} P15's journey was transformed by a sponsor he met at a peer-support group who became like family, extending both emotional and financial support for treatment. Sometimes, the most profound connections emerge from chance encounters, as P16 discovered during a moment of crisis outside a shopping mall: \pquote{She's a stranger, total stranger... I was just getting out of the shopping mall. That's when I broke down. She was like, oh, what's wrong? Can you sit down?
I cried a lot, cried, cried and talking and talking and talking and she was there she listened... I felt really very free with her because she was she was a She give me, that, atmosphere for listening.}


\paragraph{Online groups}
The development of supportive relationships extends beyond immediate circles to online communities, where individuals find connections based on shared experiences. The digital landscape further expands these information-sharing possibilities. Online communities provide platforms for reading others' stories, commenting on posts, and sharing personal experiences. For some, like P05 and P18, these "online friends" even surpass "real friends" in terms of support and understanding. P02's engagement with Facebook groups, where she formed meaningful connections with two online friends, exemplifies this phenomenon. These digital spaces become forums for direct exchange of information about symptoms, potential causes, and coping strategies.
The value of these online communities is particularly evident in P12's reflection:
\pquote{I will say this online group is safer because the online friends have seen their issues but, the majority of my physical friends don't experience what I'm experiencing.}
This sentiment underscores the unique comfort found in communities of shared experience, where understanding comes from firsthand knowledge of similar struggles.

\subsection{Challenges in the care encounters as transformative moments}
4.2 illustrates cases of care encounters that successfully lead to attitudinal and behavioral changes towards more proactive care-seeking, however, we also observe challenges in care encounters that fail to transform marginalized young adults' behaviors. We first summarize the pragmatic barriers at each level in Table \ref{tab:challenges} and demonstrate the fundamental challenges regarding trust forging and agency raising.
\label{sec: challenges}

\section{Problem Statement} \label{sec: statement}


\subsection{Deploying GNN locally Causes Vulnerabilities} \label{ps: gnn inference}
Deploying GNNs on local devices requires access to graph data in addition to the trained model, which introduces unique security and privacy challenges. 
Similar to DNN deployment, the IP of the well-performed local model, including its trained weights and biases, is valuable asset that must be protected against model extraction attacks.
Beyond the model IP, local GNN inference raises additional privacy concerns due to the nature of GNN architecture. 
Specifically, during the message-passing phase of GNN inference, target nodes aggregate information from neighboring nodes to update their embeddings. 
This process involves accessing sensitive edge data, such as user-product interactions in recommender systems.
In our work, we will address the GNN IP infringement and edge data breach vulnerability during GNN deployment.

\subsection{Edge Privacy is Valuable} \label{motivation: edge importance}

\begin{figure}[t]
    \centering
    \includegraphics[width=0.95\linewidth]{imgs/scenario.pdf}
    \caption{\textbf{Motivation Example:} Alice (victim) builds a graph of products and trains a GNN RS. She deploys both edge data and RS on local devices. Bob (attacker) accesses this device and steals the edge data and model parameters.}
    \label{fig: problem-statement}
\end{figure}

Membership inference attack is the most common data privacy threat to machine learning models~\cite{shokri2017membership}, where the goal is to determine whether a given data point belongs to the training set. 
However, in the context of GNNs, edge data raises additional privacy concerns. 
Link stealing attacks~\cite{he2021stealing, ding2023vertexserum} aim to infer the connectivity between any pair of given nodes. 
In this work, we focus on the adjacency information (edges), while considering the node features as public.
A real-world example is illustrated in Fig.~\ref{fig: problem-statement}, where Alice (victim) deploys a recommender system (RS) on local edge devices. 
In such a product graph, the node features are public attributes of the products—such as price, user reviews, or categories—that are available to any user. 
However, the internal relationships between products require intensive learning from user behavior data, which is valuable IP for the model vendor. 
Therefore, safeguarding the node connectivity information during GNN local inference is of great importance.


\subsection{TEE Has Memory Restrictions} \label{ps: TEE}
The introduction of TEE greatly enhances data security and privacy with secure compartments. 
However, TEE platforms face significant memory limitations, a critical constraint for secure computation. 
For instance, for Intel SGX trusted enclaves, the physical reserved memory (PRM) is limited to 128MB, with 96 MB of it allocated to the Enclave Page Cache (EPC)~\cite{intel2017sgx}. 
Excessive memory allocation will lead to frequent page swapping between the unprotected main memory and the protected enclave, which can cause high overhead and additional encryption/decryption to ensure data integrity~\cite{costan2016intel}.
This memory constraint poses a significant challenge for deploying GNN models and the entire graph (including node features and adjacency information) within the secure enclave, which often far exceed the PRM limitation of enclaves.



\subsubsection{Trust in Resources}
Our analysis reveals complex barriers to trust-building between marginalized populations and care resources.

Marginalized populations' existing distrust in care resources can be further reinforced by negative experiences. When participants used web searches to find mental health resources, poorly designed interfaces often undermined their initial attempts to build trust in these services. For instance, P15, seeking free therapeutic services due to financial constraints, encountered a labyrinthine web of poorly designed websites. She noted, \pquote{every time I've looked it up, it seemed inaccessible. Not every website is very user-friendly. You have to go digging and digging to find what you need to find. And when I'm frustrated and feeling like I need help, that's the last thing I want to do.} Her subsequent analysis revealed a deeper systemic issue: \pquote{Honestly, I've studied how Google search comes up. Results from Google are not mainly from your local clinics. It's people that pay more money for their website to show up first...But they just look like businesses.} 

Beyond accessibility challenges, trust deficits manifest in concerns about efficacy and cultural compatibility. Participants, particularly those with intersectional marginalized identities, expressed strong preferences for providers who share their lived experiences. P01's experience exemplifies this challenge: \pquote{I tried to find a therapist who is also a black woman, but I did not see any on the website.} This scarcity of representation in care providers reinforces existing barriers to trust and engagement, highlighting how identity-based needs remain unmet in current care infrastructures.

The online mental health communities, while intended to facilitate support, can inadvertently amplify feelings of marginalization when users encounter content misaligned with their experiences. P05's poignant account illustrates this disconnect: \pquote{I'm sorry, mine [situation] was going up for two or three weeks. Most of them are like coincidences. So I feel like I was tragic and traumatized because I was being attacked and my friend was beaten to death.} When confronted with posts about relatively minor stressors while processing severe trauma, participants like P05 reported feeling further alienated rather than supported. This experiential mismatch not only undermined their engagement with online communities but also eroded their trust in available mental health resources more broadly. 

\subsubsection{Agency Empowerment}
Our analysis reveals how different types of mental health resources vary in their ability to facilitate and sustain agency among marginalized young adults in their care-seeking journey. This variation manifests across societal, community, and technological resource levels, each presenting distinct challenges and opportunities for agency development.

Societal-level resources, while offering professional help, often place a substantial burden on individuals' agency. Only six participants successfully navigated these resources, with many citing overwhelming requirements for access. As P08 explained: \pquote{Because I felt I could not handle it alone anymore and they were the closest and had anonymity.} This suggests that societal resources often become a last resort, accessed only when individuals feel compelled by severe circumstances rather than through proactive choice, indicating a reactive rather than empowering approach to agency.

Community-level resources present a complex dynamic where agency support varies significantly across institutions. While some community entities actively facilitate care-seeking, others demonstrate limited investment in mental health support. For instance, P13, a sports official, described the absence of workplace mental health resources: \pquote{There's no training on how to handle the stress... we're just expected to deal with it.} Even within similar institutional contexts, approaches to agency empowerment differ markedly. P05's university demonstrated minimal proactive engagement, while P10's institution maintained sustained mental health initiatives through frequent mental health awareness events. 
%The temporal dimension of support emerged as a critical factor in sustained agency development. While community-level resources often provide proactive outreach, their temporary nature and position outside the professional medical system limits their ability to foster long-term agency in care-seeking. This temporal limitation can interrupt the development of sustained care-seeking behaviors, particularly for marginalized individuals who may require consistent support to build and maintain agency.

Technological resources, despite their widespread use among participants, primarily served an auxiliary role in agency development. While participants frequently initiated their help-seeking journey through technological platforms, using them as extensions of self-coping through information gathering and peer connection, these resources often proved insufficient as standalone support. Notably, four participants who initially relied exclusively on technological resources ultimately transitioned to other forms of support, suggesting technology's limitations in sustaining long-term agency in care-seeking. This pattern indicates that while technology can facilitate initial steps toward agency, it may not provide the comprehensive support needed for sustained empowerment in mental health care-seeking.



%The "official procedure" involved in seeking professional assistance also contributes to stigmatization, as articulated by P13, who described the difficulties , \pquote{And underneath the bridge over here is where all the mental health and all that building is. They have an office in here. But to get in there, you have to use this entrance, and this is the entrance, and these are the windows of our building.}

%\pquote{I just kinda, like, wanna keep it private. But at the same time, I know that people use it as a safe forum to discuss their issues and get comments and feedback. But I don't think it occurred to me at all to post my question. I don't feel comfortable releasing it to the public} (P08)

\section{Discussion}
Our study contributes to CSCW health literature by revealing how mental health care-seeking practices among young adults with diverse marginalized identities are shaped by their identity perceptions and lived realities. While many participants initially showed passive care-seeking patterns, we identified potential pathways toward proactive health engagement through transformative "care encounters" that helped them transcend limited resource endowments. These encounters served as catalysts for change, though we also documented instances where they failed to initiate behavioral changes. Our findings inform future technology design efforts aimed at empowering vulnerable populations by moving beyond access-focused solutions to address the complex interplay between identity and healthcare engagement. 

\subsection{Seeking care from the margins: From passive to proactive}
This study follows CSCW scholars' mission to care for people on the margin \cite{chordia_social_2024} and specifically focuses on how marginalized identities influence the way people seek care. %Thus, this study center on how the marginazlied identities and reveal how the reality and perception of marginalization mutually reinforce each other, culminating in the passive behavior pattern characterized by waiting and self-concealment from potential help resources. 

\subsubsection{Marginalized Identity and Passive Care-Seeking Patterns}
Prior research has established statistical correlations between marginalized identities (e.g., racial and cultural minorities) and low health resource utilization rates \cite{gulliver_perceived_2010}. While some studies attribute this reduced resource use primarily to material constraints within these communities, \textcite{pendse_marginalization_2023} revealed how resource scarcity can fundamentally reshape individuals' somatic understanding of mental health, suggesting that marginalization's impact extends beyond mere resource constraints.

Our findings advance this understanding by illuminating how individuals' internalized perceptions of marginalization can suppress care-seeking aspirations, creating barriers that persist even when resources become available. For instance, multiple participants remained inert towards university mental health services despite their accessibility. This internalized skepticism was exemplified by P12's dismissive attitude: \pquote{it's something they [the university] have to do... I don't think they will be helpful}. Such responses reveal how marginalized individuals may interpret institutional support as merely performative rather than genuinely beneficial, contributing to their reluctance to engage with available resources.

In addition to previous research examining how cultural background and racial identities influence care-seeking behaviors \cite{shahid_asian_2021, soubutts_challenges_2024}, our analysis reveals additional dimensions of marginalization that critically impact mental health resource utilization. Participants with unconcealed LGBTQ+ identity, experiences of homelessness, or unemployment often faced compound barriers due to their disconnection from traditional support networks and community resources typically tied to stable affiliations. For example, P15, who identified as LGBTQ+, low-income, and disabled, described feeling "helpless" from care resources, noting how their intersecting identities created multiple layers of disconnection from support systems.

Importantly, our analysis distinguishes these passive care-seeking behaviors, rooted in experiences of marginalization, from active resistance stemming from cultural or religious beliefs \cite{bhattacharjee_whats_2023}. While appearing outwardly resistant, often harbor latent hope for potential change and care access. For instance, P06's experience illustrates this underlying receptivity: despite initially expressing reluctance toward seeking care, she scheduled a therapy session upon learning about free university services. This pattern of dormant hope manifesting when opportunities arise was consistent across multiple participants, suggesting that passive waiting often masks an openness to care when barriers are reduced or removed. This insight offers crucial implications for how mental health resources and outreach efforts might be better structured to engage marginalized populations which we will discuss in 5.2.
%While previous studies have documented the importance of general accessibility metrics such as operating hours, our analysis reveals that availability manifests both in participants' specific situational contexts and through active outreach from care resources. This dynamic interplay between situational accessibility and proactive resource intervention shapes how participants encounter and subsequently engage with potential sources of support.

%This transformation underscores the importance of accessible, culturally competent care that acknowledges and addresses the unique challenges faced by marginalized populations.

\subsubsection{"Care Encounters" as Catalysts of Aspiration Changes}
Our findings reveal how serendipitous "care encounters" serve as pivotal moments in reshaping marginalized young adults' care-seeking trajectories. These encounters manifest through three distinct mechanisms: tangible assistance (e.g., free therapy sessions, academic accommodations, assisted hospitalization), supportive discourse (e.g., normalized conversations about mental health in both community contexts and intimate interpersonal exchanges), and social connection building that fosters meaningful relationships. While previous research has documented attitudinal changes towards care-seeking in digital spaces \cite{10.1145/3290605.3300294}, our study illuminates broader transformation patterns across both online and offline contexts.

Our findings identify that these encounters operate along two temporal dimensions. In acute situations, encounters can catalyze immediate transformation - notably exemplified by P7 and P12's experiences with emergency interventions that, contrary to previous literature's emphasis on negative impacts, served as positive turning points. More commonly, transformation unfolds gradually through accumulated positive interactions that incrementally expand participants' understanding of available resources and possibilities for care, as exemplified by P07's growing comfort in sharing her grief with her grandfather.

Drawing from Park's \cite{pal_marginality_2013} work on aspiration, our findings demonstrate how care encounters can reshape aspirations for mental healthcare, particularly paramount given marginalized populations typically express limited future aspirations due to structural constraints. For instance, P16's journey from crisis to advocacy illustrates this transformation: \pquote{Now I am a big advocate of mental health and I tell everyone about my story. My friends, they are my support system and I tell people to find your support system.} Her evolution from experiencing forced hospitalization to actively building a supportive network demonstrates a fundamental shift toward proactive care-seeking. These findings suggest the importance of aspiration-centered design approaches \cite{freeman_aspirational_2017, pendse_mental_2019, kumar_aspirations-based_2019} in creating transformative care experiences for marginalized young adults.


%on online platforms or \cite{zhang_online_2018, metts_perceptions_2022}, and tracking apps \cite{costello_predictive_2020}.
%<Perceptions of Helpful and Unhelpful Responses to Disclosures of Suicidality in a Sample of Mobile App Users> 
%have highlighted the importance of resources aligned with the socio-economic situations and experiences of help providers to facilitate help-seeking \cite{gould_technology_2020}

% \cite{lattie_designing_2020} contextualized the needs of college students in their social ecosystem and social support networks and co-designed technologies that can be fitted into their everyday lives.  \textcite{le_exploring_2021} proposed a multifaceted approach incorporating mobile applications, individual interventions, and naturalistic conversations to mitigate college students' everyday anxiety. 
% Notably, a recent workshop emphasized an ecological perspective to enhance the accessibility of mental health services \cite{ongwere_challenges_2022}.
% <Conceptual framework for personal recovery in mental health: systematic review and narrative synthesis> The roles of pathways and resources \cite{leamy_conceptual_2011}.

\subsection{Empower Marginalized Young Adults via the Socio-Technical Care Ecosystem}
While many participants described their care encounters as "coincidence" or "luck," our analysis reveals these experiences are systematically enabled by infrastructural designs that structure and facilitate the possibility of such encounters \cite{pendse_like_2020}. We discuss the affordances of various resources in the socio-technical care ecosystem and identify critical design implications for improving mental health support systems.

\subsubsection{Socio-technical Care Ecosystem Behind "Care Encounters"}
Following the ecological approach to care-seeking \cite{burgess_technology_2021, murnane_personal_2018, siddiqui_exploring_2023, ongwere_challenges_2022}, we adopted Social Ecological Theory \cite{stokols_translating_1996} to examine participants' encounters with four types of resources: societal, community, interpersonal, and technological. Our analysis reveals that the community sources emerge as principal contexts facilitating proactive care-seeking, while societal and technological resources often struggle to achieve similar engagement levels.

The ecological efficacy of community resources proves particularly powerful in empowering young adults with marginalized identities. Our findings highlight how university-based care outreach efforts foster empowerment through two interrelated channels: resource delivery providing accessible mental health support tools and information, while conversation invitations create safe spaces for open dialogue about mental health concerns. This dual approach proves particularly effective in cultivating supportive microenvironments that encourage care-seeking behaviors among marginalized young adults.

Community resources also serve as crucial bridges to societal resources while expanding marginalized young adults' interpersonal networks. Several participants, after positive experiences with university mental health services, decided to continue therapy post-graduation. For instance, P08's university extended her free therapy sessions while she searched for post-graduation care options. Communities can host events that foster interpersonal relationships, crucial for marginalized young adults who often experience exclusion from their families and lack friends comfortable with mental health discussions.
University-managed peer support groups emerged as critical spaces for connection. In one notable case, P17 encountered an alumnus who sponsored all treatment fees, including therapist visits and medication. Such local support groups were perceived by participants (e.g., P07 and P17) as more trustworthy and engaging compared to online peer support groups, enhancing the potential for positive care experiences \cite{gould_technology_2020}.
On the downside, it is crucial for future study to investigate support for individuals without readily available community resources. The four non-university-affiliated participants in our study—including individuals experiencing homelessness, unemployment, and workplace isolation—underscore the need for support systems beyond university contexts.

Our analysis reveals limitations in technological-level resources' efficacy in initiating active interventions and motivating care-seeking. P16's experience with online portals demonstrates how poorly designed technological interfaces can reinforce feelings of exclusion and diminish care aspirations. The limited engagement with online resources was also observed in previous studies \cite{pretorius_searching_2020}, suggests the disconnection of these tools from young adults' daily lives may explain their relatively low ecological validity \cite{mohr_three_2017}.
These findings emphasize the need for a more integrated approach to mental health support that combines technological solutions with community-based resources while recognizing the crucial role of interpersonal connections in facilitating care encounters. 

\subsubsection{Empowering through Technologies}
Building on research that integrates mental health technologies within young adults' social ecosystems \cite{lattie_designing_2020, le_exploring_2021, stefanidi_children_2023}, we discuss several key implications that facilitate future researchers and designers to embed and integrate technologies within marginalized young adults' social, cultural, and material environments \cite{gould_technology_2020}.

\paragraph{Leveraging Low-Technology Solutions for Effective Outreach}
While numerous studies demonstrate engagement patterns with mental health self-management applications \cite{wiljer_effects_2020, alqahtani_co-designing_2021}, our research revealed notably limited adoption of these tools among young adults with marginalized identities, with only two out of twelve participants reporting attempted usage. This low adoption rate likely stems from insufficient trust and motivation \cite{burke_qualitative_2022}. In contrast, low-technology interventions, particularly text messages, and notifications \cite{gitlow_how_2019}, have shown promising results with higher acceptability and seamless integration into daily routines \cite{gould_technology_2020}. Our findings identified similar low-technology outreaching mechanisms as four participants specifically highlighted the effectiveness of university outreach through traditional channels—emails, posters, and social media—in facilitating service utilization. Moreover, three participants discovered the National Crisis Line through prominent search engine results, underscoring the impact of strategic information placement across commonly accessed platforms. These findings suggest that compared to sophisticated mobile applications attempting to address all mental health needs comprehensively, marginalized young adults may respond more favorably to low-tech communication strategies. Such approaches can serve as effective nudges, anchoring critical information and fostering gradual attitudinal shifts toward help-seeking.

\paragraph{Facilitating Technology-Mediated Disclosure in Complicated Interpersonal Relationships}
We found five participants struggle with complicated family dynamics that inhibit trust and open discussion of mental health concerns. These individuals often face cultural barriers that create hesitation in disclosing and addressing their mental health needs \cite{chung_medical_2020}. Building intimacy and trust through sustained engagement emerges as crucial \cite{dalsgaard_mediated_2006, shin_designing_2021}, particularly when traditional family support structures are strained. Technology presents unique opportunities to bridge these interpersonal gaps, functioning both as an accessibility enabler and a relationship strengthener \cite{palmier-claus_integrating_2013}. The emergence of Large Language Models (LLMs) with enhanced emotion detection capabilities \cite{wang_reprompt_2023} has enabled the development of mediating tools, such as empathetic chatbots \cite{jo_understanding_2023}, which show promise in facilitating difficult conversations. This technological approach resonates particularly well with young adults \cite{koulouri_chatbots_2022}, with mental health professionals confirming its efficacy in raising awareness and managing negative emotions within complex family dynamics \cite{lee_i_2020}. Notably, our study identified alternative platforms, such as gaming environments, as especially effective in creating safe spaces for emotional expression. Four male participants demonstrated this by utilizing gaming platforms to maintain friendships and share personal thoughts away from family tensions \cite{wong-villacres_technology-mediated_2011}. These technology-mediated environments offer alternative pathways for building trust and maintaining connections when traditional face-to-face family interactions prove challenging or emotionally fraught \cite{gould_technology_2020}.

\paragraph{Build Trust through Identity and Experience-Based Interactions}
Our findings reveal that current platforms inadequately address personalization needs, particularly for marginalized users seeking providers with shared lived experiences. For instance, P05's difficulty relating to posts from more resourced individuals highlights the importance of socioeconomic status alignment in care relationships \cite{liang_embracing_2021}. These insights suggest new directions for recommendation systems and online peer support group design that prioritize identity-based and experience-based connections, as users demonstrate strong engagement with narratives that mirror their personal circumstances \cite{ liu_consumer_2023}. Future research can investigate how to effectively facilitate digital mental health storytelling, given its documented therapeutic benefits \cite{kim2020ever}, especially among young adults \cite{de2016digital}.

%This aligns with recent research showing how peer support significantly enhances engagement with mobile mental health apps \cite{wong_postsecondary_2021, jonathan_smartphone-based_2021},

% \paragraph{Support Long-term Care-Seeking Journey}
% Eight participants' help-seeking journeys began through active intervention from interpersonal and community resources. This suggests an opportunity for technologies capable of monitoring early signs of mental health challenges to proactively engage independently living young adults. Future development should prioritize user involvement in early design stages \cite{alqahtani_co-designing_2021} and integrate innovative approaches such as asynchronous remote support \cite{bhattacharya_designing_2021} and multimodal communication \cite{h_barriers_2021} to enhance engagement with professional treatment.

\section{Conclusion}
Through interviews with 18 young adults with diverse marginalized identities, our study reveals how passive care-seeking behaviors are shaped by experiences of marginalization. We identify "care encounters" as critical turning points that can transform passive patterns into proactive engagement through tangible assistance, supportive discourse, and social connections. These findings inform the design of interventions that can intentionally facilitate such transformative encounters, ultimately improving marginalized young adults' access to mental health care.

\section{Acknowledgements}
We extend our sincere gratitude to the participants for their brave and heartful sharing of their experiences and feelings, which has greatly inspired the authors and made this paper possible. We also thank the reviewers for their valuable insights, which helped refine our findings and clarify our contributions.
%Dr. Elliot Hauser


\bibliographystyle{ACM-Reference-Format}
\bibliography{references, extra}

\appendix
\end{document}
\endinput
