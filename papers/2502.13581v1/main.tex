%%%%%%%% ICML 2025 EXAMPLE LATEX SUBMISSION FILE %%%%%%%%%%%%%%%%%

\documentclass{article}

% Recommended, but optional, packages for figures and better typesetting:
\usepackage{microtype}
\usepackage{graphicx}
\usepackage{subfigure}
\usepackage{booktabs} % for professional tables

% hyperref makes hyperlinks in the resulting PDF.
% If your build breaks (sometimes temporarily if a hyperlink spans a page)
% please comment out the following usepackage line and replace
% \usepackage{icml2025} with \usepackage[nohyperref]{icml2025} above.
\usepackage{xcolor}
\usepackage{hyperref}


% Attempt to make hyperref and algorithmic work together better:
\newcommand{\theHalgorithm}{\arabic{algorithm}}

% Use the following line for the initial blind version submitted for review:
% \usepackage{icml2025}

% If accepted, instead use the following line for the camera-ready submission:
\usepackage[accepted]{icml2025}

% For theorems and such
\usepackage{amsmath}
\usepackage{amssymb}
\usepackage{mathtools}
\usepackage{amsthm}
\usepackage{balance}

\usepackage{algorithm}
\usepackage{algorithmic}
% \usepackage{crenames}

% if you use cleveref..
\usepackage[capitalize,noabbrev]{cleveref}

%%%%%%%%%%%%%%%%%%%%%%%%%%%%%%%%
% THEOREMS
%%%%%%%%%%%%%%%%%%%%%%%%%%%%%%%%
\theoremstyle{plain}
\newtheorem{theorem}{Theorem}[section]
\newtheorem{proposition}[theorem]{Proposition}
\newtheorem{lemma}[theorem]{Lemma}
\newtheorem{corollary}[theorem]{Corollary}
\theoremstyle{definition}
\newtheorem{definition}[theorem]{Definition}
\newtheorem{assumption}[theorem]{Assumption}
\theoremstyle{remark}
\newtheorem{remark}[theorem]{Remark}

% Todonotes is useful during development; simply uncomment the next line
%    and comment out the line below the next line to turn off comments
%\usepackage[disable,textsize=tiny]{todonotes}
\usepackage[textsize=tiny]{todonotes}

\usepackage{xspace}
\usepackage{bm}
\usepackage{multirow}
\usepackage{multicol}
\usepackage{balance}
\usepackage{bbding}
\newcommand{\ie}{\emph{i.e.,}\xspace}
\newcommand{\eg}{\emph{e.g.,}\xspace}
\newcommand{\aka}{\emph{a.k.a.,}\xspace}
\newcommand{\etal}{\emph{et al.}\xspace}
\newcommand{\etc}{\emph{etc.}\xspace}
\newcommand{\paratitle}[1]{\vspace{1.5ex}\noindent\textbf{#1}}
\newcommand{\wrt}{w.r.t.\xspace}
\newcommand{\ignore}[1]{}

\DeclareMathOperator*{\argmax}{arg\,max}

\newcommand{\tba}{\textcolor{red}{xxx }}
\newcommand{\outd}{\textcolor{red}{[Outdated]}~}
\newcommand{\tabincell}[2]{\begin{tabular}{@{}#1@{}}#2\end{tabular}}


\definecolor{myblue}{RGB}{212, 225, 245}
\definecolor{mygreen}{RGB}{213, 232, 212}
\definecolor{myyellow}{RGB}{255, 242, 204}
\definecolor{myred}{RGB}{248, 206, 204}
\definecolor{codecomment}{RGB}{63,127,127}
% \renewcommand{\algorithmiccomment}[1]{{\color{myblue}\%\ #1}}
% \renewcommand{\algorithmiccomment}[1]{\hskip2em \textcolor{codecomment}{\%\ #1}}
% \renewcommand{\algorithmiccomment}[1]{\{#1\}}
\renewcommand{\algorithmiccomment}[1]{\textcolor{codecomment}{\#\ #1}}
% \algnewcommand{\TO}{\textbf{to}}
% \algnewcommand{\RETURN}[1]{\STATE \textbf{return} #1}


\definecolor{codegreen}{rgb}{0,0.3,0.6}
\definecolor{codegray}{rgb}{0.5,0.5,0.5}

\usepackage{newfloat}
\usepackage{listings}

\lstset{%
    basicstyle={\footnotesize\ttfamily},% footnotesize acceptable for monospace
    numbers=left,numberstyle=\footnotesize,xleftmargin=2em,% show line numbers, remove this entire line if you don't want the numbers.
    aboveskip=0pt,belowskip=0pt,%
    showstringspaces=false,tabsize=2,breaklines=true,
    commentstyle=\color{codegreen},
    keywordstyle=\color{magenta},
    numberstyle=\tiny\color{codegray},
    frame = single
}
\floatstyle{ruled}
\newfloat{listing}{tb}{lst}{}
\floatname{listing}{Listing}



% The \icmltitle you define below is probably too long as a header.
% Therefore, a short form for the running title is supplied here:
\icmltitlerunning{ActionPiece: Contextually Tokenizing Action Sequences for Generative Recommendation}

\begin{document}

\twocolumn[
\icmltitle{ActionPiece: Contextually Tokenizing Action Sequences for\\ Generative Recommendation}

% It is OKAY to include author information, even for blind
% submissions: the style file will automatically remove it for you
% unless you've provided the [accepted] option to the icml2025
% package.

% List of affiliations: The first argument should be a (short)
% identifier you will use later to specify author affiliations
% Academic affiliations should list Department, University, City, Region, Country
% Industry affiliations should list Company, City, Region, Country

% You can specify symbols, otherwise they are numbered in order.
% Ideally, you should not use this facility. Affiliations will be numbered
% in order of appearance and this is the preferred way.
% \icmlsetsymbol{equal}{*}
\icmlsetsymbol{intern}{*}

\begin{icmlauthorlist}
\icmlauthor{Yupeng Hou}{ucsd,intern}
\icmlauthor{Jianmo Ni}{gdm}
\icmlauthor{Zhankui He}{gdm}
\icmlauthor{Noveen Sachdeva}{gdm}
\icmlauthor{Wang-Cheng Kang}{gdm}\\ 
\icmlauthor{Ed H. Chi}{gdm}
\icmlauthor{Julian McAuley}{ucsd}
\icmlauthor{Derek Zhiyuan Cheng}{gdm}
\end{icmlauthorlist}

\icmlaffiliation{ucsd}{University of California, San Diego}
\icmlaffiliation{gdm}{Google DeepMind}

\icmlcorrespondingauthor{Yupeng Hou}{\texttt{yphou@ucsd.edu}}

% You may provide any keywords that you
% find helpful for describing your paper; these are used to populate
% the "keywords" metadata in the PDF but will not be shown in the document
\icmlkeywords{Generative Recommendation, Action Tokenization}

\vskip 0.3in
]

% this must go after the closing bracket ] following \twocolumn[ ...

% This command actually creates the footnote in the first column
% listing the affiliations and the copyright notice.
% The command takes one argument, which is text to display at the start of the footnote.
% The \icmlEqualContribution command is standard text for equal contribution.
% Remove it (just {}) if you do not need this facility.

\printAffiliationsAndNotice{\textsuperscript{*}Work done as a student researcher at Google DeepMind. }  % leave blank if no need to mention equal contribution
% \printAffiliationsAndNotice{\icmlEqualContribution} % otherwise use the standard text.

% TL;DR We propose ActionPiece, the first context-aware action sequence tokenizer for generative recommendation. Unlike traditional action tokenizers, ActionPiece can tokenize the same action into different tokens based on the surrounding context in different sequences.

\begin{abstract}
Generative recommendation (GR) is an emerging paradigm where user actions
% , represented by interacted items, 
are tokenized into discrete token patterns and autoregressively generated as predictions.
% However, existing GR models tokenize each action independently, without considering the contextual relationships between other actions in a sequence. 
However, existing GR models tokenize each action independently, assigning the same fixed tokens to identical actions across all sequences without considering contextual relationships.
This lack of context-awareness can lead to suboptimal performance, as the same action may hold different meanings depending on its surrounding context.
To address this issue, we propose ActionPiece to explicitly incorporate context when tokenizing action sequences. In ActionPiece, each action is represented as a \emph{set} of item features, which serve as the initial tokens.
% Our tokenizer then iteratively merges frequently co-occurred tokens, either within a single feature set or between adjacent sets in the action sequence corpora, into new tokens.
% The tokenizer is then trained on the action sequence corpora to memorize feature patterns as new tokens, based on their frequency both within individual sets and across adjacent sets.
Given the action sequence corpora, we construct the vocabulary by merging feature patterns as new tokens, based on their co-occurrence frequency both within individual sets and across adjacent sets.
% These new tokens act as shortcuts, helping the GR models memorize important feature patterns.
% By controlling the vocabulary size, ActionPiece allows for a balance between generalization and memorization.
Considering the unordered nature of feature sets, we further introduce set permutation regularization, which produces multiple segmentations of action sequences with the same semantics.
% during both training and inference
% to enhance the utilization of ActionPiece tokens. 
Experiments on public datasets demonstrate that ActionPiece consistently outperforms existing action tokenization methods, improving NDCG@$10$ by $6.00\%$ to $12.82\%$.
% By adjusting the vocabulary size, ActionPiece is also shown to be able to balance memory efficiency and recommendation performance~(\Cref{fig:intro}).
% Code is available: \url{https://anonymous.4open.science/r/ActionPiece/}.
\end{abstract}

\section{Introduction}
\label{sec:intro}

Foundational models (FMs)~\cite{zhang2024data, zhou2023comprehensive} have shown remarkable progress in the healthcare domain, enabling professional-like assessment of disease diagnosis, treatment decision-making, and monitoring~\cite{zhang2023text, wang2022medclip, lu2023mi-zero}. 
Examples include LLaVA-Med~\cite{li2023llava}, Med-PaLM Multimodal~\cite{tu2024towards}, and Med-Flamingo~\cite{moor2023med}, have demonstrated their capacity on question answering, medical image analysis, and report generation.
These studies follow a predominant top-down model development strategy that requires upstream developers to collect data and train models for downstream tasks. 
Consequently, the developed model capabilities are heavily dependent on the training data, limiting their generalization performance in diverse clinical scenarios. 
For instance, Med-Gemini~\cite{yang2024advancing} reveals promising general capabilities in report generation while it lags behind state-of-the-art (SoTA) models on classification tasks, especially for out-of-domain applications. 
This indicates that while the generalizability of the foundation model is promising, more solutions are expected to meet the various specialized clinical needs.

To address these challenges, multi-center data centralization becomes essential to enhance model capacity and robustness across varied clinical scenarios~\cite{rajpurkar2022ai}. 
Centralizing distributed data can significantly improve model training and inference performance.
However, the process of medical data storage, transfer, and aggregation among centers requires extra efforts to ensure data security and system interoperability~\cite{bradford2020international}.
Moreover, a growing concern for patient privacy makes large-scale multi-center data sharing particularly challenging. 
While efforts like federated learning~\cite{wen2023survey, li2020review} can achieve good model performance on local data, the need for synchronized system coordination presents significant challenges, as clients are unable to update asynchronously. This limitation greatly restricts the practical capability of such approaches.
As a result, without a flexible collaboration, medical community still struggles to fully utilize the isolated data and local computation resources for comprehensive medical AI model development. 
To address this dilemma, open-source platforms encourage public data sharing and knowledge integration~\cite{markiewicz2021openneuro, zenodo}.
However, these platforms focus solely on raw data sharing while seldom providing collaborative model training or cooperation between different institutions.
Recently, collaborative learning has emerged as a viable approach for enhancing multi-model robustness~\cite{boulemtafes2020review}. 
For instance, software-like model development~\cite{raffel2023building} mimics software engineering practices by introducing structured workflows, enabling merging, version control, and continuous model integration.
Under this design, model ability can be strengthened with incremental knowledge updates similar to the version updating in software development. 

Although collaborative learning provides a multi-model collaboration, two key challenges remain in the leakage of raw data during collaboration~\cite{huang2023lorahub} and the synchronization of multiple collaborators~\cite{mcmahan2017communication} in the medical AI community. It is still challenging to integrate decentralized, privacy-sensitive data across institutions, leading to under-utilized insights and fragmented knowledge sharing~\cite{kaissis2020secure, rajpurkar2022ai, abdullah2021ethics}.
 To address these challenges, inspired by the collaborative software development, we propose \textbf{Med}ical \textbf{Fo}undation Models Me\textbf{rg}ing (\textbf{MedForge}), a cooperative workflow enabling continuously community-driven foundation model (FM) development.
MedForge enables a lightweight manner for individual centers to share their knowledge among multiple centers, minimizing the burden of data transmission and integration while enhancing model robustness.
Meanwhile, MedForge facilitates asynchronous and flexible collaboration, allowing individual centers to continuously update and improve medical FMs without the need for real-time synchronization.
Similar to open-source software development, MedForge incrementally updates medical knowledge and follows a sustainable model development scheme. 
This key design emphasizes a bottom-up construction of a multi-task medical FM, allowing downstream users to collaboratively build, refine, and update the upstream model according to their local resources. Our major contributions of MedForge are as below: 
\begin{enumerate}
    \item[$\bullet$] We introduce a collaborative workflow to promote the merging scheme of open-source software development. Our proposed MedForge allows distributed clinical centers to asynchronously contribute to comprehensive medical model construction while reducing transmitting costs among centers and avoiding the leakage of raw data, thus enhancing the utilization of private resources in the healthcare system. 
    \item[$\bullet$] We propose two effective knowledge-merging strategies for the asynchronous branch contribution. The MedForge-Fusion strategy updates the plugin module parameters of the main model during the merging phase, whereas the MedForge-Mixture strategy integrates the output of the plugin module by memorizing each contributor's coefficient. These strategies make MedForge more flexible and versatile. MedForge-Fusion is friendly to implement, while the MedForge-Mixture offers better performance and robustness.
    \item[$\bullet$]  We comprehensively evaluate model merging strategies to accumulate medical knowledge among multiple branch plugin modules. MedForge yields superior performance on medical classification tasks compared to other collaborative baselines across multiple datasets. We demonstrate the robustness of MedForge by shuffling the task order and evaluating various configurations of plugin modules and dataset distillation methods.
\end{enumerate}



\section{Background}

\paragraph{Information Extraction} Information extraction (IE) is one of the most fundamental applications in natural language processing. IE systems take the user's requirement (e.g., defined by a label text, a question, or an instruction) and extract spans of several tokens from input texts. The two most frequent categories of IE targets are entity and relation, which structure many IE tasks, such as named entity recognition~\cite{conll2003}, relation extraction~\cite{conll2004}, event extraction~\citep{ace2005multilingual}, and others~\citep{srl-task,DBLP:conf/semeval/PontikiGPPAM14,aste-task}. A crucial challenge to modern IE systems is the growing number of IE targets (e.g., various label names) in the open world, which are scarce in annotation and require IE systems for quick transfer learning. Thus, many works have collected massive automated IE annotations to pre-train IE models~\cite{fewnerd,multinerd,TadNER,NuNER,metaie}, which shows benefits in transferring to low-resource IE targets.

\paragraph{Large Language Model} The biggest game-changer for natural language processing in all domains is the large language model (LLM)~\citep{tulu,llama-2,achiam2023gpt4,olmo,dubey2024llama3,team2024gemma}. Learning on trillions of tokens for pre-training and post-training, LLMs have shown surprisingly strong performance on all kinds of tasks~\citep{achiam2023gpt4}. Next token prediction, the paradigm behind the success of LLMs, supports exploiting every token in raw texts as the annotation to strengthen the model's capability. Consequently, many IE researchers have turned toward LLMs~\citep{llm4clinicalie,gpt-ner,llm4ie} to use them as strategic information extractors with planning~\citep{LLM_Plan,LLM_NestNER} and chain-of-thoughts~\citep{chain_of_thoughts,cot_re}.

\paragraph{Pre-training Paradigm: IE v.s. LLM} The rise of LLMs has challenged the meaningfulness of IE pre-training with an overwhelmingly larger number of annotations. The lagging of IE pre-training can be attributed to the relatively high format requirement for IE annotation like labels in Wikipedia links. This paper shows IE pre-training can take a free ride on LLM's NTP paradigm to unleash the power of massive pre-training.
\vspace{-5pt}
\section{Method}
\label{sec:method}
\begin{figure*}[t]
\begin{center}
\includegraphics[width=.85\linewidth]{fig_overview_v3.pdf}
\end{center}
\caption{
FastAtlas Overview: In each frame, we compute charts spanning fully or partially visible triangles (a), determine texture space bounding boxes for the visible portions of the view-space projections of each chart, and tightly pack these boxes into atlases (b, here $2K \times 2K$). We simultaneously bijectively parameterize and shade the charts into their atlas boxes, obtaining high quality texture space shading (c), and use this shading to render the shaded frames (d).}
\label{fig:overview}
\label{fig:alg_overview}
\end{figure*}

\section{Overview}
\label{sec:overview}
Our work has two core contributions: a real-time, GPU-based algorithm for tight packing of general parameterized charts into compact atlases; and a real-time TSS method that
utilizes this packing.  

\paragraph*{FastAtlas Packing.}
FastAtlas runs entirely on the GPU as a series of compute shaders. It takes the bounding boxes of parameterized charts as input, and packs them into an atlas (Fig~\ref{fig:overview}b, Sec.~\ref{sec:pack}). As such, the only input it requires are the dimensions of the bounding boxes.
Its outputs are deterministic; identical input charts are packed into identical atlases. This is critical for TSS and similar applications, as it ensures that consecutive frames taken from the same camera view have the same shading. Even minute shading differences across such frames can cause sampling jitter, leading to undesirable flicker \cite{baker2012rock}. 
While prior methods such as \cite{mueller2018shading,hladky2019tessellated,hladky2021snakebinning,Neff2022MSA} cap the dimensions of the charts that can be packed as-is for a given atlas size, and scale down all charts that exceed these dimensions, we scale all charts by the same factor, and do so only when strictly necessary to achieve packing success (Figs~\ref{fig:atlas},~\ref{fig:sas_issues}). 

\paragraph*{TSS using FastAtlas.}
Our end-to-end TSS atlas generation method combines the packing method above with a novel approach for computing seamless per-frame charts. 
We define our charts as the connected components of the visible surfaces in each frame (Fig.~\ref{fig:overview}a), and efficiently compute them using a parallel union-find algorithm (Sec.~\ref{sec:visible}). Since the boundaries of these charts coincide with the contours of the rendered surface, they are {\em invisible} to the viewer. This approach 
eliminates the artifacts caused by shading discontinuities along visible seams (Fig.~\ref{fig:seams}). 

\begin{parWithWrapFigure}
\begin{wrapfigure}{l}{.27\columnwidth}%
\includegraphics[width=\linewidth]{fig_inset_view_plane.pdf}%
\end{wrapfigure}
We bijectively parametrize the {\em visible portions} of our charts by projecting them to view space (inset). This maps a constant number of texels to each pixel in the final rendered output, evenly distributing residual undersampling error across all image pixels. While conceptually straightforward, efficiently parameterizing charts containing partially visible triangles using viewspace projection is non-trivial, as the visible portions may no longer be triangular (e.g. green triangle in the inset); applying naive projection to triangles with vertices behind the camera may produce ill-posed results. Clipping triangles before projection is both computationally expensive and significantly complicates downstream operations. We avoid explicit clipping by observing that all that is required for atlas packing is the dimensions of, potentially conservative, bounding boxes of these projected visible portions. We compute such bounding boxes without explicit chart clipping by adapting a conservative screen coverage estimator \shortcite{Blinn:CalculatingScreenCoverage} (Sec.~\ref{sec:box}). We then pack the computed boxes using FastAtlas. 
\end{parWithWrapFigure}

Finally, we shade the visible portion of each chart into its corresponding atlas bounding box (Fig~\ref{fig:overview}c). 
The resulting texture is then used during rasterization as a standard texture map (Fig. ~\ref{fig:overview}d). 
Our framework is compatible with all existing approaches for texture space shading, including forward shading, raytraced illumination, or deferred shading in texture space \cite{baker:2016}. In the examples shown, we use the standard forward shading based rendering pipeline included in the G3D Innovation Engine \cite{G3D17}, a commercial grade renderer.


Our goal is to increase the robustness of T2I models, particularly with rare or unseen concepts, which they struggle to generate. To do so, we investigate a retrieval-augmented generation approach, through which we dynamically select images that can provide the model with missing visual cues. Importantly, we focus on models that were not trained for RAG, and show that existing image conditioning tools can be leveraged to support RAG post-hoc.
As depicted in \cref{fig:overview}, given a text prompt and a T2I generative model, we start by generating an image with the given prompt. Then, we query a VLM with the image, and ask it to decide if the image matches the prompt. If it does not, we aim to retrieve images representing the concepts that are missing from the image, and provide them as additional context to the model to guide it toward better alignment with the prompt.
In the following sections, we describe our method by answering key questions:
(1) How do we know which images to retrieve? 
(2) How can we retrieve the required images? 
and (3) How can we use the retrieved images for unknown concept generation?
By answering these questions, we achieve our goal of generating new concepts that the model struggles to generate on its own.

\vspace{-3pt}
\subsection{Which images to retrieve?}
The amount of images we can pass to a model is limited, hence we need to decide which images to pass as references to guide the generation of a base model. As T2I models are already capable of generating many concepts successfully, an efficient strategy would be passing only concepts they struggle to generate as references, and not all the concepts in a prompt.
To find the challenging concepts,
we utilize a VLM and apply a step-by-step method, as depicted in the bottom part of \cref{fig:overview}. First, we generate an initial image with a T2I model. Then, we provide the VLM with the initial prompt and image, and ask it if they match. If not, we ask the VLM to identify missing concepts and
focus on content and style, since these are easy to convey through visual cues.
As demonstrated in \cref{tab:ablations}, empirical experiments show that image retrieval from detailed image captions yields better results than retrieval from brief, generic concept descriptions.
Therefore, after identifying the missing concepts, we ask the VLM to suggest detailed image captions for images that describe each of the concepts. 

\vspace{-4pt}
\subsubsection{Error Handling}
\label{subsec:err_hand}

The VLM may sometimes fail to identify the missing concepts in an image, and will respond that it is ``unable to respond''. In these rare cases, we allow up to 3 query repetitions, while increasing the query temperature in each repetition. Increasing the temperature allows for more diverse responses by encouraging the model to sample less probable words.
In most cases, using our suggested step-by-step method yields better results than retrieving images directly from the given prompt (see 
\cref{subsec:ablations}).
However, if the VLM still fails to identify the missing concepts after multiple attempts, we fall back to retrieving images directly from the prompt, as it usually means the VLM does not know what is the meaning of the prompt.

The used prompts can be found in \cref{app:prompts}.
Next, we turn to retrieve images based on the acquired image captions.

\vspace{-3pt}
\subsection{How to retrieve the required images?}

Given $n$ image captions, our goal is to retrieve the images that are most similar to these captions from a dataset. 
To retrieve images matching a given image caption, we compare the caption to all the images in the dataset using a text-image similarity metric and retrieve the top $k$ most similar images.
Text-to-image retrieval is an active research field~\cite{radford2021learning, zhai2023sigmoid, ray2024cola, vendrowinquire}, where no single method is perfect.
Retrieval is especially hard when the dataset does not contain an exact match to the query \cite{biswas2024efficient} or when the task is fine-grained retrieval, that depends on subtle details~\cite{wei2022fine}.
Hence, a common retrieval workflow is to first retrieve image candidates using pre-computed embeddings, and then re-rank the retrieved candidates using a different, often more expensive but accurate, method \cite{vendrowinquire}.
Following this workflow, we experimented with cosine similarity over different embeddings, and with multiple re-ranking methods of reference candidates.
Although re-ranking sometimes yields better results compared to simply using cosine similarity between CLIP~\cite{radford2021learning} embeddings, the difference was not significant in most of our experiments. Therefore, for simplicity, we use cosine similarity between CLIP embeddings as our similarity metric (see \cref{tab:sim_metrics}, \cref{subsec:ablations} for more details about our experiments with different similarity metrics).

\vspace{-3pt}
\subsection{How to use the retrieved images?}
Putting it all together, after retrieving relevant images, all that is left to do is to use them as context so they are beneficial for the model.
We experimented with two types of models; models that are trained to receive images as input in addition to text and have ICL capabilities (e.g., OmniGen~\cite{xiao2024omnigen}), and T2I models augmented with an image encoder in post-training (e.g., SDXL~\cite{podellsdxl} with IP-adapter~\cite{ye2023ip}).
As the first model type has ICL capabilities, we can supply the retrieved images as examples that it can learn from, by adjusting the original prompt.
Although the second model type lacks true ICL capabilities, it offers image-based control functionalities, which we can leverage for applying RAG over it with our method.
Hence, for both model types, we augment the input prompt to contain a reference of the retrieved images as examples.
Formally, given a prompt $p$, $n$ concepts, and $k$ compatible images for each concept, we use the following template to create a new prompt:
``According to these examples of 
$\mathord{<}c_1\mathord{>:<}img_{1,1}\mathord{>}, ... , \mathord{<}img_{1,k}\mathord{>}, ... , \mathord{<}c_n\mathord{>:<}img_{n,1}\mathord{>}, ... , $
$\mathord{<}img_{n,k}\mathord{>}$,
generate $\mathord{<}p\mathord{>}$'', 
where $c_i$ for $i\in{[1,n]}$ is a compatible image caption of the image $\mathord{<}img_{i,j}\mathord{>},  j\in{[1,k]}$. 

This prompt allows models to learn missing concepts from the images, guiding them to generate the required result. 

\textbf{Personalized Generation}: 
For models that support multiple input images, we can apply our method for personalized generation as well, to generate rare concept combinations with personal concepts. In this case, we use one image for personal content, and 1+ other reference images for missing concepts. For example, given an image of a specific cat, we can generate diverse images of it, ranging from a mug featuring the cat to a lego of it or atypical situations like the cat writing code or teaching a classroom of dogs (\cref{fig:personalization}).
\vspace{-2pt}
\begin{figure}[htp]
  \centering
   \includegraphics[width=\linewidth]{Assets/personalization.pdf}
   \caption{\textbf{Personalized generation example.}
   \emph{ImageRAG} can work in parallel with personalization methods and enhance their capabilities. For example, although OmniGen can generate images of a subject based on an image, it struggles to generate some concepts. Using references retrieved by our method, it can generate the required result.
}
   \label{fig:personalization}\vspace{-10pt}
\end{figure}
\begin{table*}[t!]
\centering
% \vspace{5pt}
\begin{small}
\begin{tabular}{l|c|c|c|c|c|c|c}
\toprule
\textbf{Method} & \textbf{Type} & \textbf{ToMi} & \textbf{BigToM} & \textbf{MMToM-QA} & \textbf{MuMA-ToM} & \textbf{Hi-ToM} & \textbf{All} \\
\midrule
SymbolicToM & Specific & \textbf{98.60} & - &  - & - & - & - \\
TimeToM & Specific & 87.80 & - &   - & - & - & - \\
% \textbf{96.00$^*$}
PercepToM & Specific & 82.90 & - & - & - & - & - \\
BIP-ALM & Specific & - & - & 76.70 & 33.90 & - & - \\
LIMP & Specific & - & - & - & 76.60 & - & - \\
\ours w/ Model Spec. & Specific & 88.80 & \textbf{86.75} & \textbf{79.83} & \textbf{84.00} & \textbf{74.00} & \textbf{82.68} \\
\midrule
Llama 3.1 70B & General & 72.00 & 77.83 & 43.83 & 55.78 & 35.00 & 47.41 \\
Gemini 2.0 Flash & General & 66.70 & 82.00 & 48.00 & 55.33 & 52.50 & 60.91\\
Gemini 2.0 Pro & General & 71.90 & 86.33 & 50.84 &  62.22 & 57.50 & 65.76 \\ 
GPT-4o & General & 77.00 & 82.42 & 44.00 & 63.55 & 50.00 & 63.39 \\
SimToM & General & 79.90 & 77.50 & 51.00 & 47.63 & 71.00 & 65.41\\ 
\ours & General & \textbf{88.30} & \textbf{86.92} & \textbf{75.50} & \textbf{81.44} & \textbf{72.50} & \textbf{80.93} \\
\bottomrule
\end{tabular}
\end{small}
\caption{Results of \ours and baselines on all benchmarks. There are two groups of methods: methods that require domain-specific knowledge (e.g., AutoToM w/ Model Spec.) or implementations (e.g., SymbolicToM) and methods that can be generally applied to any domain. ``-'' indicates that the domain-specific method is not applicable to the benchmark. The best results for each method type are highlighted in bold.}
\label{tab:results}
\vspace{-10pt}
\end{table*}



\section{Experiments}
\subsection{Experimental Settings}



We evaluated our method on multiple Theory of Mind benchmarks, including ToMi \citep{le2019revisiting}, BigToM \citep{gandhi2024understanding}, MMToM-QA \cite{jin2024mmtom}, MuMA-ToM \citep{shi2024muma}, and Hi-ToM \cite{he2023hi}. The diversity and complexity of these benchmarks pose significant reasoning challenges. For instance, MMToM-QA and MuMA-ToM incorporate both visual and textual input, while MuMA-ToM and Hi-ToM require higher-order inference. Additionally, MMToM-QA features exceptionally long contexts, and BigToM presents open-ended scenarios.



Besides the full \ours method, we additionally evaluated \ours given manually specified models (AutoToM w/ Model Spec.). 

We compared \ours against state-of-the-art baselines:
    \textbf{LLMs:} Llama 3.1 70B \citep{dubey2024llama}, Gemini 2.0 Flash, Gemini 2.0 Pro \cite{team2023gemini} and GPT-4o \cite{achiam2023gpt};
    
     \textbf{ToM prompting for LLMs:} SymbolicToM \cite{sclar2023minding}, SimToM \cite{wilf2023think}, TimeToM \cite{hou2024timetom}, and PercepToM \citep{jung2024perceptions};
 
  \textbf{Model-based inference:} BIP-ALM \cite{jin2024mmtom} and LIMP \cite{shi2024muma}.


For multimodal benchmarks, MMToM-QA and MuMA-ToM, we adopt the information fusion methods proposed by \citet{jin2024mmtom} and \citet{shi2024muma} to fuse information from visual and text inputs respectively. The fused information is in text form. We ensure that all methods use the same fused information as their input.


We use GPT-4o as the LLM backend for \ours and all ToM prompting and model-based inference baselines to ensure a fair comparison—except for TimeToM, which relies on GPT-4 and is not open-sourced.


\subsection{Results}
The main results are summarized in Table~\ref{tab:results}. Unlike \ours, many recent ToM baselines can only be applied to specific benchmarks. Among general methods, \ours achieves state-of-the-art results across all benchmarks. In particular, it outperforms its LLM backend, GPT-4o, by a large margin. This is because Bayesian inverse planning is more robust for inferring mental states given long contexts with complex environments and agent behavior. It is also more adept at recursive reasoning which is key to higher-order inference. Notably, \ours performs comparably to manually specified models, showing that automatic model discovery without domain knowledge is as effective as human-provided models. We provide additional results and qualitative examples in Appendix~\ref{sec:more_results}.


\subsection{Ablated Study}



\begin{figure}[t!]
  \centering
  \includegraphics[width=0.8\linewidth]{figures/comparison.pdf}
    \vspace{-10pt}
  \caption{Averaged performance and compute of the full \ours method (star) and the ablated methods (circles) on all benchmarks.}
  \label{fig:ablation}
  \vspace{-10pt}
\end{figure}


We evaluated the following variants of \ours for an ablation study: no hypothesis reduction (\textbf{w/o hypo. reduction}); always using POMDP (\textbf{w/ POMDP}); always using the initial model proposal without variable adjustment (\textbf{w/o variable adj.}); only considering the last timestep (\textbf{w/ last timestep}); and considering all timesteps without timestep adjustment (\textbf{w/ all timesteps}).

The results in Figure~\ref{fig:ablation} show that the full \ours method constructs a suitable BToM model, enabling rich ToM inferences while reducing compute. We analyze key model components below:

\textbf{Hypothesis reduction.}
Compared to the full method, \ours w/o hypo. reduction has a similar accuracy but consumes 53\% more tokens on average, demonstrating that hypothesis reduction optimizes efficiency without sacrificing performance.

\textbf{Variable adjustment.}
\ours dynamically identifies relevant variables for ToM inference, generalizing domain-specific BIP approaches to open-ended scenarios. Compared to its variant without variable adjustment, \ours improves performance with minimal additional compute. The variant that always uses POMDP performs well in scenarios aligned with the POMDP assumption (e.g., MMToM-QA) but generalizes poorly elsewhere and incurs much higher computational costs. %, leading to an 8.5% performance deficit.

\textbf{Timestep adjustment.}
By selecting relevant steps for inference, timestep adjustment enhances performance by focusing on essential information. In contrast, the variant using only the last timestep misses crucial details, significantly lowering performance. The variant incorporating all timesteps suffers from higher computational costs and reduced accuracy due to conditioning on unnecessary, potentially distracting information.



Full ablation results are provided in Appendix~\ref{sec:more_results_ablation}.


\section{Conclusion}

In this paper, we introduce ActionPiece, the first context-aware action sequence tokenizer for generative recommendation. By considering the surrounding context, the same action can be tokenized into different tokens in different sequences. We formulate generative recommendation as a task on sequences of feature sets and merge important feature patterns into tokens. During vocabulary construction, we propose assigning weights to token pairs based on their structures, such as those within a single set or across adjacent sets. To enable efficient vocabulary construction, we use double-ended linked lists to maintain the corpus and introduce intermediate nodes to store tokens that combine features across adjacent sets. Additionally, we propose set permutation regularization, which segments a single action sequence into multiple token sequences with the same semantics. These segments serve as natural augmentations for training and as ensemble instances for inference.

In the future, we plan to align user actions with other modalities by constructing instructions that combine ActionPiece tokens and other types of tokens.
We also aim to extend the proposed tokenizer to other tasks that can be framed as set sequence modeling problems, including audio modeling, sequential decision-making, and time series forecasting.

% Acknowledgements should only appear in the accepted version.
% \section*{Acknowledgements}

% \newpage

% \section*{Impact Statement}

% This paper introduces ActionPiece, a context-aware action sequence tokenizer to enhance generative recommendation. This work aims to advance personalized recommender systems, enabling more accurate understanding of user preferences. We argue that this work is not directly correlated to certain society or ethical concerns. The impact of this work is primarily tied to the broader implications of recommender systems in various domains, such as e-commerce, entertainment, and social platforms.

% Authors are \textbf{required} to include a statement of the potential 
% broader impact of their work, including its ethical aspects and future 
% societal consequences. This statement should be in an unnumbered 
% section at the end of the paper (co-located with Acknowledgements -- 
% the two may appear in either order, but both must be before References), 
% and does not count toward the paper page limit. In many cases, where 
% the ethical impacts and expected societal implications are those that 
% are well established when advancing the field of Machine Learning, 
% substantial discussion is not required, and a simple statement such 
% as the following will suffice:

% ``This paper presents work whose goal is to advance the field of 
% Machine Learning. There are many potential societal consequences 
% of our work, none which we feel must be specifically highlighted here.''

% The above statement can be used verbatim in such cases, but we 
% encourage authors to think about whether there is content which does 
% warrant further discussion, as this statement will be apparent if the 
% paper is later flagged for ethics review.


% In the unusual situation where you want a paper to appear in the
% references without citing it in the main text, use \nocite
% \nocite{langley00}

\bibliographystyle{icml2025}
\balance
\bibliography{ref}



%%%%%%%%%%%%%%%%%%%%%%%%%%%%%%%%%%%%%%%%%%%%%%%%%%%%%%%%%%%%%%%%%%%%%%%%%%%%%%%s
%%%%%%%%%%%%%%%%%%%%%%%%%%%%%%%%%%%%%%%%%%%%%%%%%%%%%%%%%%%%%%%%%%%%%%%%%%%%%%%
% APPENDIX
%%%%%%%%%%%%%%%%%%%%%%%%%%%%%%%%%%%%%%%%%%%%%%%%%%%%%%%%%%%%%%%%%%%%%%%%%%%%%%%
%%%%%%%%%%%%%%%%%%%%%%%%%%%%%%%%%%%%%%%%%%%%%%%%%%%%%%%%%%%%%%%%%%%%%%%%%%%%%%%
\clearpage
\appendix
% \section{You \emph{can} have an appendix here.}

\onecolumn

% \twocolumn[
\begin{center}
    {\Large \textbf{Appendices}}
\end{center}
% \vspace{0.4cm}
% ]


% \section{Additional Related Work}

% \paratitle{Sequential recommendation.}

% \paratitle{Tokenization for non-text data.}

\begin{table*}[!t] %
    \small
	\caption{Notations and explanations.}
	\label{tab:notation}
	\vskip 0.1in
	\resizebox{\columnwidth}{!}{
	\begin{tabular}{cl}
		\toprule
		\textbf{Notation} & \textbf{Explaination}\\
		\midrule
		$i$, $i_1$, $i_j$ & item, item identifier, item ID\\
		$t$ & the number of actions in the input action sequence; the timestamp when the model makes a prediction\\
		$i_{t+1}$ & the ground-truth next item \\
		$\hat{i}_{t+1}$ & the predicted next item \\
		$S = \{i_1,i_2,\ldots,i_t\}$ & the action sequence where each action is represented with the interacted item ID \\
		$\mathcal{A}$, $\mathcal{A}_1$, $\mathcal{A}_j$ & a set of item features or tokens \\
		$m = |\mathcal{A}_j|$ & the number of features associated with each item \\
		$f_{j,k}$ & the $k$-th feature of item $i_j$ \\
		$\mathcal{F}_k$ & the collection of all possible choices for the $k$-th feature \\
		$S' = \{\mathcal{A}_1, \mathcal{A}_2, \ldots, \mathcal{A}_t\}$ & the action sequence where each action is represented with a set of item features \\
		$c$, $c_1$, $c_j$ & input \& generated tokens \\
		$l$ & the number of tokens in the token sequence \\
		$C = \{c_1, c_2, \ldots, c_l\}$ & the token sequence tokenized from the input action sequence $S'$ \\
		$\{c_{l+1}, \ldots, c_q\}$ & the tokens generated by the GR model \\
		$\mathcal{V}$ & vocabulary of ActionPiece tokenizer \\
		$\mathcal{R}$ & merge rules of ActionPiece tokenizer \\
		$\{(c_u, c_v) \to c_{\text{new}}\}$ & one merge rule indicating two adjacent tokens $c_u$ and $c_v$ can be replaced by a token $c_{\text{new}}$ \\
		$Q = |\mathcal{V}|$ & size of ActionPiece vocabulary \\
		$P(c, c')$ & probability that tokens $c$ and $c'$ are adjacent when flattening a sequence of sets into a token sequence \\
		$N$ & the number of action sequences in the training corpus \\
		$L$ & the average length of action sequences in the training corpus \\
		$H$ & Maximal heap size, $O(NLm)$\\
% 		$D$ & Average time a token appears in the training corpus, $O(\frac{NLm}{Q})$\\
		$q$ & The number of segmentations produced using set permutation regularization during inference\\
		\bottomrule
	\end{tabular}
	}
\end{table*}

\section{Notations}

We summarize the notations used in this paper in~\Cref{tab:notation}.

% Belows are unpolished by LLMs

\section{Algorithmic Details}

In this section, we provide detailed algorithms for vocabulary construction and segmentation.

\subsection{Vocabulary Construction Algorithm}


\begin{algorithm}[!t]
  \caption{ActionPiece Vocabulary Construction -- Count (\Cref{fig:weight})}
  \label{alg:vocab_construction_count}
\begin{algorithmic}[1]
\INPUT Action sequence corpus $\mathcal{S}'$, current vocabulary $\mathcal{V}$
   \OUTPUT Accumulated weighted token co-occurrences\ \ $\text{count}(\cdot, \cdot)$
   \FOR{$i \gets 0$ \textbf{to} $|\mathcal{V}|, j \gets 0$ \textbf{to} $|\mathcal{V}|$}
    \STATE $\text{count}(c_i, c_j) \gets 0$
   \ENDFOR
    \FORALL{sequence $S' \in \mathcal{S}'$}
    \STATE $t \gets \text{length}(S')$ \COMMENT{number of action nodes in sequence}
    \FOR{$k \gets 0$ \textbf{to} $t-1$}
        \STATE $\mathcal{A}_k \gets S'[k]$ \COMMENT{current action node}
        \STATE \COMMENT{Process all unordered token pairs within $\mathcal{A}_k$}
        \FORALL{$c_i, c_j \in \mathcal{A}_k, i\neq j$}
                \STATE $\text{count}(c_i, c_j) \gets \text{count}(c_i, c_j) + 2 / |\mathcal{A}_k|$ \COMMENT{weight of tokens within a single set (\Cref{eq:p_one_set})}
                \STATE $\text{count}(c_j, c_i) \gets \text{count}(c_j, c_i) + 2 / |\mathcal{A}_k|$ \COMMENT{symmetric update}
        \ENDFOR
        
        \COMMENT{Process all ordered token pairs between $A_k$ and $A_{k+1}$}
        \IF{$k < t-1$}
            \STATE $\mathcal{A}_{k+1} \gets S'[k+1]$
            \FORALL{$c_i \in \mathcal{A}_k, c_j \in \mathcal{A}_{k+1}$}
                \STATE $\text{count}(c_i, c_j) \gets \text{count}(c_i, c_j) + 1 / (|\mathcal{A}_k| \times |\mathcal{A}_{k+1}|)$ \COMMENT{weight of tokens from two adjacent sets (\Cref{eq:p_two_sets})}
            \ENDFOR
        \ENDIF
    \ENDFOR
\ENDFOR
\item[\textbf{return} $\text{count}(\cdot, \cdot)$]
\end{algorithmic}
\end{algorithm}

\begin{algorithm}[!t]
  \caption{ActionPiece Vocabulary Construction -- Update (\Cref{fig:update})}
  \label{alg:vocab_construction_update}
\begin{algorithmic}[1]
\INPUT Action sequence corpus $\mathcal{S}'$ before updating, current merge rule $\{(c_u, c_v) \to c_{\text{new}}\}$
\OUTPUT Updated action sequence corpus $\mathcal{S}'$
   \FORALL{sequence $S' \in \mathcal{S}'$}
        \STATE $t \gets \text{length}(S')$
        \FOR{$k \gets 0$ \textbf{to} $t-1$}
        \STATE $\mathcal{A}_k \gets S'[k]$
        \STATE \COMMENT{Merge tokens in one action node}
        \IF{$c_u \in \mathcal{A}_k$ \AND $c_v \in \mathcal{A}_k$}
          \STATE Replace $c_u$ and $c_v$ in $\mathcal{A}_k$ with $c_{\text{new}}$
        \ENDIF
        
        \STATE \COMMENT{Merge tokens from two adjacent nodes}
        \IF{$k < t-1$}
          \STATE $\mathcal{A}_{k+1} \gets S'[k+1]$
          \IF{$c_u \in \mathcal{A}_k$ \AND $c_v \in \mathcal{A}_{k+1}$}
            \IF{$\mathcal{A}_k, \mathcal{A}_{k+1}$ are both action nodes}
                \STATE Create intermediate node $M$ between $\mathcal{A}_k$ and $\mathcal{A}_{k+1}$
                \STATE $M \gets \{c_{\text{new}}\}$ \COMMENT{linked list: $\mathcal{A}_k \to M \to \mathcal{A}_{k+1}$}
                \STATE $\mathcal{A}_k \gets \mathcal{A}_k \setminus c_u$
                \STATE $\mathcal{A}_{k+1} \gets \mathcal{A}_{k+1} \setminus c_v$
            \ELSIF{$\mathcal{A}_k$ is intermediate node}
                \STATE $\mathcal{A}_k \gets \{c_{\text{new}}\}$
                \STATE $\mathcal{A}_{k+1} \gets \mathcal{A}_{k+1} \setminus c_v$
            \ELSIF{$\mathcal{A}_{k+1}$ is intermediate node}
                \STATE $\mathcal{A}_k \gets \mathcal{A}_k \setminus c_u$
                \STATE $\mathcal{A}_{k+1} \gets \{c_{\text{new}}\}$
            \ENDIF
          \ENDIF
        \ENDIF
        \ENDFOR
    \ENDFOR
\item[\textbf{return} $\mathcal{S}'$]
\end{algorithmic}
\end{algorithm}

The overall procedure for vocabulary construction is illustrated in~\Cref{alg:vocab_construction_overall}. As described in~\Cref{subsubsec:vocab_construct}, this process involves iterative \textbf{Count}~(\Cref{alg:vocab_construction_count}) and \textbf{Update}~(\Cref{alg:vocab_construction_update}) operations.


\subsection{Segmentation with Set Permutation Regularization Algorithm}

\begin{algorithm}[!t]
\caption{Segmentation via Set Permutation Regularization (SPR) (\Cref{subsubsec:segmentation})}
\label{alg:spr}
\begin{algorithmic}[1]
\INPUT Action sequence $S$, merge rules $\mathcal{R}$
\OUTPUT Segmented token sequences $C$
    \STATE $C \gets [\;]$ \COMMENT{initialize permuted initial token sequence}
    \FORALL{token set $\mathcal{A}_i \in S$}
        \STATE Generate random permutation of $\mathcal{A}_i$ as $[c_1, c_2, \dots, c_{|\mathcal{A}_i|}]$
        \STATE Extend $C$ with $[c_1, c_2, \dots, c_{|\mathcal{A}_i|}]$ \COMMENT{concatenate permutations}
    \ENDFOR
    \STATE
    \STATE \COMMENT{Apply BPE~\cite{sennrich2016bpe} segmentation on permuted sequence}
    \REPEAT
        \STATE $\mathcal{R}' \gets \emptyset$ \COMMENT{candidate merge rules}
        \FOR{$i \gets 0$ \textbf{to} $|C| - 1$}
            \IF{$\{(c_i, c_{i+1}) \to c'\} \in \mathcal{R}$}
                \STATE $\mathcal{R}' \gets \mathcal{R}' \cup \{(c_i, c_{i+1}) \to c'\}$
            \ENDIF
        \ENDFOR
        \STATE Select $\{(c_k, c_{k+1}) \to c'\} \in \mathcal{R}'$ with the smallest index among all merge rules $\mathcal{R}$
        \STATE $C\gets [c_1, \dots, c_{k-1}, c', c_{k+2}, \dots]$ \COMMENT{replace $(c_k, c_{k+1})$ with a new token $c'$}
    \UNTIL{$\mathcal{R}'$ is $\emptyset$}
\item[\textbf{return} $C$]
\end{algorithmic}
\end{algorithm}

The detailed algorithm for segmenting action sequences into token sequences using set permutation regularization (SPR) is shown in~\Cref{alg:spr}. In practice, we often run~\Cref{alg:spr} multiple times to augment the training corpus or ensemble recommendation outputs, as described in~\Cref{subsubsec:training,subsubsec:inference}.


\section{Efficient Vocabulary Construction Implementation}\label{sec:vocab_construct_algo}

\begin{figure}[!t]
\lstinputlisting[language=Python]{fig/train_step.py}
\caption{Pseudocode for a single iteration of the efficient vocabulary construction algorithm, illustrating how a max-heap with lazy updates is used to track and merge frequent token pairs.} 
\label{fig:torch_implementation}
\end{figure}

To efficiently construct the ActionPiece vocabulary, we propose using data structures such as heaps with a lazy update trick, linked lists, and inverted indices to speed up each iteration of the construction process. The key idea is to avoid recalculating token co-occurrences in every iteration and instead update the data structures. The pseudocode is shown in~\Cref{fig:torch_implementation}.

\subsection{Data Structures}

The data structures used in the proposed algorithm are carefully designed to optimize the efficiency of vocabulary construction. Here is a detailed discussion of their roles and implementations:
\begin{itemize}
    \item \textbf{Linked list:}  
    Each action sequence in the training corpus is stored as a linked list. This allows efficient local updates during token merging. When a token pair $(c_u, c_v)$ is replaced by a new token $c_{new}$, only the affected nodes and their neighbors in the linked list need to be modified (as shown in~\Cref{alg:vocab_construction_update,fig:update}).
    \item \textbf{Heap with lazy update trick:}  
    A max-heap prioritizes token pairs by their co-occurrences. Instead of recalculating the heap entirely in each iteration, a ``lazy update'' strategy is employed: outdated entries (with mismatched co-occurrence counts) are retained but skipped during extraction. In the pseudocode, the loop checks if the top element is outdated via \texttt{is\_outdated}. Invalid entries are discarded, and only valid ones are processed. Updated co-occurrences are pushed as new entries (with negative counts for max-heap emulation).
    \item \textbf{Inverted indices:}  
    The \texttt{pair2head} dictionary maps token pairs to the sequences containing them. When a pair $(c_u, c_v)$ is merged, the algorithm directly retrieves affected sequence IDs via \texttt{pair2head[(c\_u, c\_v)]}, avoiding a full corpus scan. After merging, the inverted indices are incrementally updated: new token pairs (\eg $(c_{prev}, c_{new})$ and $(c_{new}, c_{next})$) are added to \texttt{pair2head}, while obsolete pairs are removed. This enables targeted updates and ensures subsequent iterations efficiently access relevant sequences.
\end{itemize}
These structures collectively reduce time complexity by focusing computation on dynamically changing parts of the corpus and avoiding redundant global operations. The linked list enables localized edits, the heap minimizes priority recalculation, and the inverted indices eliminate brute-force searches, making the algorithm scalable to large corpora.

\subsection{Time Complexity}

The time complexity of the efficient vocabulary construction algorithm can be analyzed through two main components: \textbf{initialization} and \textbf{iterative merging}.

\begin{itemize}  
   \item \textbf{Initialization phase}  
   involves building the initial max-heap to track co-occurrence frequencies. Given \(N\) input sequences (each with an average length of \(L\)), we count co-occurrences for all \(O(m^2)\) token pairs within each set of size \(m\). This requires \(O(NLm^2)\) time.  

   \item \textbf{Iterative merging phase}
   dynamically processes the involved sequences. The total number of such sequences across all iterations is approximately  
   \[
   O\left(\frac{N}{|\mathcal{V}_0|}\right) + O\left(\frac{N}{|\mathcal{V}_0| + 1}\right) + \dots + O\left(\frac{N}{Q}\right) \simeq O(\log{Q}N).
   \]  
   For each sequence, updating the linked list requires \(O(Lm)\) time, counting co-occurrences takes \(O(Lm^2)\) time, and inserting co-occurrences into the max-heap requires at most \(O(Lm^2\log{H})\) time. Here, \(H\) represents the heap size, which is at most \(O(NLm)\). Thus, the overall time complexity for iterative merging is  
   \[
   O(\log{Q}N(Lm + Lm^2 + Lm^2\log{H})) = O(\log{Q}\log{H} \cdot NLm^2).
   \]  
\end{itemize}  

Therefore, the overall time complexity of our proposed vocabulary construction algorithm is \(O(\log{Q}\log{H} \cdot NLm^2)\), where the iterative merging phase dominates. This complexity is significantly better than the naive vocabulary construction complexity of \(O(QNLm^2)\).  

\section{Discussion: Comparison Between ActionPiece and BPE}

\begin{table}[!t]
\small
\centering
\caption{Comparison between ActionPiece and BPE.}
\vskip 0.1in
\label{tab:comparison}
\begin{tabular}{lll}
\toprule
\textbf{Aspect} & \textbf{BPE} & \textbf{ActionPiece} \\
\midrule
\textbf{Data Type} & text sequences & action (unordered feature set) sequences \\
\textbf{Token} & a byte sequence & a feature set \\
\textbf{Initial Vocabulary} & single bytes & single-feature sets \\
\textbf{Merging Unit} & adjacent byte pairs & feature pairs within one set or between adjacent sets \\
% \textbf{Context Handling} & Local character adjacency & Explicit cross-action context modeling \\
\textbf{Co-occurrence Weighting} & raw frequency counting & probabilistic weighting~(\Cref{fig:weight}) \\
\textbf{Segmentation Strategy} & greedy fixed-order merging & set permutation regularization~(\Cref{alg:spr}) \\
\textbf{Intermediate Structures} & N/A & intermediate nodes for cross-action merges \\
% \textbf{Time Complexity} & O(NL) with full recomputing & O(logQ logH·NLm²) with incremental updates \\
% \textbf{Set Unorderedness} & N/A (ordered text) & Explicit set permutation invariance \\
\bottomrule
\end{tabular}
\end{table}

While ActionPiece follows a similar algorithmic framework as BPE, its design is fundamentally different because it is tailored for tokenizing action sequences. To clarify, we summarize the key differences in~\Cref{tab:comparison}.

\section{Datasets}\label{app:datasets}

\begin{table*}[!t] %
    \small
    \centering
	\caption{Statistics of the processed datasets. ``Avg.~$t$'' denotes the average number of actions in an action sequence.}
	\label{tab:dataset}
	\vskip 0.1in
% 	\resizebox{\columnwidth}{!}{
	\begin{tabular}{c@{\hspace{0.5in}}r@{\hspace{0.5in}}r@{\hspace{0.5in}}r@{\hspace{0.5in}}r}
		\toprule
		\textbf{Datasets} & \textbf{\#Users} & \textbf{\#Items} & \textbf{\#Actions} & \textbf{Avg.~$t$}\\
		\midrule
		\textbf{Sports}  & 18,357            & 35,598           & 260,739            & 8.32 \\
		\textbf{Beauty}  & 22,363            & 12,101           & 176,139            & 8.87 \\
% 		\textbf{Toys}    & 19,412            & 11,924           & 148,185            & 8.63 \\
		\textbf{CDs}     & 75,258            & 64,443           & 1,022,334          & 14.58 \\
% 		\textbf{ML-1M} \\
		\bottomrule
	\end{tabular}
% 	}
    % \vskip -0.3in
\end{table*}

\textbf{Categories.} Among all the datasets, ``Sports'' and ``Beauty'' are two widely used benchmarks for evaluating generative recommendation models~\cite{rajput2023tiger,jin2024lmindexer,hua2023p5cid}. We conduct experiments on these benchmarks to ensure fair comparisons with existing results. Additionally, we introduce ``CDs'', which contains about $4\times$ more interactions than ``Sports'', making it a larger dataset for evaluating the scalability of GR models. For ``CDs'', we apply the same data processing strategy as the public benchmarks. The statistics of the processed datasets are shown in~\Cref{tab:dataset}.

\textbf{Sequence truncation length.} Following \citet{rajput2023tiger}, we filter out users with fewer than $5$ reviews and truncate action sequences to a maximum length of $20$ for ``Generative'' methods, including ActionPiece. For ``ID-based'' and ``Feature + ID'' baselines, we set the maximum length to $50$, as suggested in their original papers.

\textbf{Item text features.} Following~\citet{rajput2023tiger,zheng2024lcrec,sheng2025alpharec}, the first step for feature engineering is to combine multiple raw text features into a single sentence for each item. Then, we use a pretrained sentence embedding model to encode this sentence into a vector representation. In all our implementations, we concatenate \emph{title}, \emph{price}, \emph{brand}, \emph{feature}, \emph{categories}, and \emph{description}, and use \texttt{sentence-t5-base}~\cite{ni2022sentencet5} as the sentence embedding model.
\begin{itemize}
\item The encoded sentence embeddings of $768$ dimension are directly used as textual item representations for UniSRec.
\item We quantize the sentence embeddings using residual quantization (RQ)~\cite{rajput2023tiger,zeghidour2021rqvae,zheng2024enhancing} into three codes, each with $256$ candidates. To prevent conflicts, we add an extra identification code. These four codes together serve as the RQ-based semantic IDs for TIGER and SPM-SID.
\item For other baselines that require item features, such as FDSA, S$^3$-Rec, VQ-Rec, HSTU, and our method, we follow~\citet{hou2023vqrec} and quantize the sentence embeddings using optimized product quantization (OPQ)~\cite{ge2013opq}. Except for VQ-Rec, where the sentence embeddings are quantized into $32$ codes as suggested in the original paper, we quantize the sentence embeddings into $4$ codes for all other methods to ensure a fair comparison. The codebook size is $256$ for each digit of code. For generative methods HSTU and ActionPiece, we also include an additional identification code to prevent conflicts. Note that, unlike RQ-based semantic IDs, features produced by product/vector quantization do not require a specific order.
\end{itemize}

\section{Baselines}\label{appendix:baselines}

We compare ActionPiece with the following representative baselines:

\subsection{ID-Based Sequential Recommendation Methods}

\begin{itemize}
    \item \textbf{SASRec}~\cite{kang2018sasrec} represents each item using its unique item ID. It encodes item ID sequences with a self-attentive Transformer decoder. The model is trained by optimizing a binary cross-entropy objective.
    \item \textbf{BERT4Rec}~\cite{sun2019bert4rec} also represents each item using its unique item ID. Unlike SASRec, it encodes sequences of item IDs with a bidirectional Transformer encoder. The model is trained using a masked prediction objective.
\end{itemize}

\subsection{Feature-Enhanced Sequential Recommendation Methods}

\begin{itemize}
    \item \textbf{FDSA}~\cite{zhang2019fdsa} integrates item feature embeddings with vanilla attention layers to obtain feature representations. It then processes item ID sequences and feature sequences separately through self-attention blocks.
    \item \textbf{S$^3$-Rec}~\cite{zhou2020s3} first employs self-supervised pre-training to capture the correlations between item features and item IDs. Then the checkpoints are loaded and fine-tuned for next-item prediction, using only item IDs.
    \item \textbf{VQ-Rec}~\cite{hou2023vqrec} encodes text features into dense vectors using pre-trained language models. It then applies product quantization to convert these dense vectors into semantic IDs. The semantic ID embeddings are pooled together to represent each item. Since the experiments are not performed in a transfer learning setting, we omit the two-stage training strategy outlined in the original paper. Instead, we reuse the model architecture and train it from scratch using an in-batch contrastive loss with a batch size of $256$.
\end{itemize}

\subsection{Generative Recommendation Methods}

Each generative recommendation baseline corresponds to an action tokenization method described in~\Cref{tab:act_tokenization}.

\begin{itemize}
    \item \textbf{P5-CID}~\cite{hua2023p5cid} is an extension of P5~\cite{geng2022p5}, which formulates recommendation tasks in a text-to-text format. Building on P5, the authors explored several tokenization methods to index items for better recommendations. In this study, we use P5-CID as a representative hierarchical clustering-based action tokenization method. It organizes the eigenvectors of the Laplacian matrix of user-item interactions into a hierarchy and assigns cluster IDs at each level as item indices. When implementing this baseline method, we adopt the same model backbone as ActionPiece (encoder-decoder Transformers trained from scratch) and use the indices  produced by P5-CID.
    \item \textbf{TIGER}~\cite{rajput2023tiger} encodes text features similarly to VQ-Rec but quantizes them into semantic IDs using RQ-VAE. The model is then trained to autoregressively predict the next semantic ID and employs beam search for inference. We use a beam size of $50$ in beam search to generate the top-$K$ recommendations.
    \item \textbf{LMIndexer}~\cite{jin2024lmindexer} takes text as input and predicts semantic IDs. The text description of each item is first tokenized using a text tokenizer. The resulting text tokens are then concatenated to form input action sequences. The model is trained with self-supervised objectives to learn the semantic IDs of target items. The reported results in~\Cref{tab:performance} are taken from the original paper. We do not report the results of LMIndexer on the large dataset ``CDs'' because it does not converge under similar computing budget as the other methods.
    \item \textbf{HSTU}~\cite{zhai2024hstu} discretizes raw item features into tokens, treating them as input tokens for generative recommendation. The authors also propose a lightweight Transformer layer that improves both performance and efficiency. For action tokenization, we use the same item features as our method and arrange them in a specific order to form the tokenized tokens of each item.
    \item \textbf{SPM-SID}~\cite{singh2024spmsid} first tokenizes each item into semantic IDs. It then uses the SentencePiece model (SPM)~\cite{kudo2018sentencepiece} to merge important semantic ID patterns within each item into new tokens in the vocabulary. While the original paper introduces this method for ranking models, we adapt it for the generative recommendation task. Specifically, we concatenate the SPM tokens as inputs, feed them into the T5 model, and autoregressively generate SPM tokens as recommendations.
\end{itemize}

\begin{table}[!t]
\centering
\small
\caption{Hyperparameter settings of ActionPiece for each dataset.}
\label{tab:reproduction}
\vskip 0.1in
\begin{tabular}{l@{\hspace{0.5in}}c@{\hspace{0.5in}}c@{\hspace{0.5in}}c}
\toprule
\textbf{Hyperparameter} & \textbf{Sports} & \textbf{Beauty} & \textbf{CDs} \\
\midrule
learning\_rate & 0.005 & 0.001 & 0.001 \\
warmup\_steps & 10,000 & 10,000 & 10,000 \\
dropout\_rate & 0.1 & 0.1 & 0.1 \\
weight\_decay & 0.15 & 0.15 & 0.07 \\
vocabulary\_size & 40,000 & 40,000 & 40,000 \\
n\_inference\_segments & 5 & 5 & 5 \\
beam\_size & 50 & 50 & 50 \\
num\_layers & 4 & 4 & 4 \\
d\_model & 128 & 128 & 256 \\
d\_ff & 1,024 & 1,024 & 2,048 \\
num\_heads & 6 & 6 & 6 \\
d\_kv & 64 & 64 & 64 \\
optimizer & adamw & adamw & adamw \\
lr\_scheduler & cosine & cosine & cosine \\
train\_batch\_size & 256 & 256 & 256 \\
max\_epochs & 200 & 200 & 200 \\
early\_stop\_patience & 20 & 20 & 20 \\
\bottomrule
\end{tabular}
\end{table}

\section{Implementation Details}\label{appendix:implementation}

\textbf{Baselines.} The results of BERT4Rec, SASRec, FDSA, S$^3$-Rec, TIGER, and LMIndexer on the ``Sports'' and ``Beauty'' benchmarks are taken directly from existing papers~\cite{zhou2020s3,rajput2023tiger,jin2024lmindexer}. For other results, we carefully implement the baselines and tune hyperparameters according to the suggestions in their original papers. We implement BERT4Rec, SASRec, FDSA, and S$^3$-Rec using the open-source recommendation 
library RecBole~\cite{zhao2021recbole}. For other methods, we implement them ourselves with HuggingFace Transformers~\cite{wolf2020transformers} and PyTorch~\cite{paszke2019pytorch}. We use FAISS~\cite{douze2024faiss} to quantize sentence representations.
% By adjusting embedding dimensions, linear layer shapes, \emph{etc.}, we aim to keep the model parameters of all compared methods similar. This ensures a fair comparison from the perspective of scaling laws~\cite{kaplan2020scaling}.

\textbf{ActionPiece.} We use an encoder-decoder Transformer architecture similar to T5~\cite{raffel2020t5}. We use four layers for both the encoder and decoder. The multi-head attention module has six heads, each with a dimension of $64$. For the public benchmarks ``Sports'' and ``Beauty'', we follow~\citet{rajput2023tiger} and set the token embedding dimension to $128$ and the intermediate feed-forward layer dimension to $1024$. This results in a total of 4.46M non-embedding parameters. For the larger ``CDs’’ dataset, we use a token embedding dimension of $256$ and an intermediate feed-forward layer dimension of $2048$, leading to 13.11M non-embedding parameters. For model inference, we use beam search with a beam size of $50$. Note that the baselines P5-CID, TIGER, and SPM-SID use the same model architecture, differing only in their action tokenization methods. For ActionPiece-specific hyperparameters, we set the number of segmentations produced using set permutation regularization during inference to $q = 5$. We tune the vocabulary size in $\{5k, 10k, 20k, 30k, 40k\}$.

\textbf{Training.} We train the GR models from scratch for up to $200$ epochs, using early stopping if the model does not achieve a better NDCG@$10$ on the validation set for $20$ consecutive epochs. The training batch size is set to $256$. The learning rate is selected from $\{1\times 10^{-3}, 3\times 10^{-3}, 5\times 10^{-3}\}$ with a warmup step of $10{,}000$. We use a dropout rate of $0.1$ and tune the weight decay from $\{0.07, 0.1, 0.15, 0.2\}$. For all methods implemented by us, we conduct five repeated experiments using random seeds $\{2024, 2025, 2026, 2027, 2028\}$. The model checkpoints with the best average NDCG@$10$ on the validation set are selected for evaluation on the test set, and we report these results. Each model is trained on a single $40$G NVIDIA A100 GPU.


\section{Reproduction}

% \textbf{We will release the code to reproduce our results after the review phase.} 
To improve reproducibility, we provide the algorithms of vocabulary construction and segmentation processes in~\Cref{alg:vocab_construction_overall,alg:vocab_construction_count,alg:vocab_construction_update,alg:spr}. We also provide the pseudocode for the efficient vocabulary construction implementation in~\Cref{fig:torch_implementation}. In addition, we provide the best hyperparameters of ActionPiece for all experimental datasets in~\Cref{tab:reproduction}.





% You can have as much text here as you want. The main body must be at most $8$ pages long.
% For the final version, one more page can be added.
% If you want, you can use an appendix like this one.  

% The $\mathtt{\backslash onecolumn}$ command above can be kept in place if you prefer a one-column appendix, or can be removed if you prefer a two-column appendix.  Apart from this possible change, the style (font size, spacing, margins, page numbering, etc.) should be kept the same as the main body.
%%%%%%%%%%%%%%%%%%%%%%%%%%%%%%%%%%%%%%%%%%%%%%%%%%%%%%%%%%%%%%%%%%%%%%%%%%%%%%%
%%%%%%%%%%%%%%%%%%%%%%%%%%%%%%%%%%%%%%%%%%%%%%%%%%%%%%%%%%%%%%%%%%%%%%%%%%%%%%%


\end{document}


% This document was modified from the file originally made available by
% Pat Langley and Andrea Danyluk for ICML-2K. This version was created
% by Iain Murray in 2018, and modified by Alexandre Bouchard in
% 2019 and 2021 and by Csaba Szepesvari, Gang Niu and Sivan Sabato in 2022.
% Modified again in 2023 and 2024 by Sivan Sabato and Jonathan Scarlett.
% Previous contributors include Dan Roy, Lise Getoor and Tobias
% Scheffer, which was slightly modified from the 2010 version by
% Thorsten Joachims & Johannes Fuernkranz, slightly modified from the
% 2009 version by Kiri Wagstaff and Sam Roweis's 2008 version, which is
% slightly modified from Prasad Tadepalli's 2007 version which is a
% lightly changed version of the previous year's version by Andrew
% Moore, which was in turn edited from those of Kristian Kersting and
% Codrina Lauth. Alex Smola contributed to the algorithmic style files.
