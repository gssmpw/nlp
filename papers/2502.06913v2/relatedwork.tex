\section{Related Work}
\vspace{-0.3em}
\textbf{Mutational Effect Prediction.}
The prediction of mutation effects on single proteins has been well studied, which mainly mines co-evolutionary information from protein sequences by Multiple Sequence Alignments (MSAs)~\citep{frazer2021disease,luo2021ecnet} or Protein Language Models (PLMs)~\citep{meier2021language,notin2022tranception}. However, predicting \emph{the change in binding free energy} ($\Delta\Delta G$) of protein complexes upon mutations is more challenging because it involves complex interactions between proteins. Computational methods for $\Delta\Delta G$ prediction have undergone a paradigm shift from biophysics-based and statistics-based techniques~\citep{schymkowitz2005foldx,park2016simultaneous} to Deep Learning (DL) techniques, among which pre-training-based approaches are the most popular solutions. RDE~\citep{luo2023rotamer} pre-trains by using a normalizing flow model to estimate the density of sidechain conformations (rotamers). Similarly, DiffAffinity~\citep{liu2023predicting} also models the side-chain distribution, but with a conditional diffusion model. Besides, \cite{mo2024multi} proposes a multi-level pre-training framework, ProMIM, to fully capture all three levels of protein-protein interactions. Recently, Prompt-DDG~\citep{wu2024learning} proposes a microenvironment-aware hierarchical codebook that generates prompts for better $\Delta\Delta G$ prediction.

\textbf{Antibody Optimization.} Early approaches for antibody design are mostly energy-based~\citep{adolf2018rosettaantibodydesign,lapidoth2015abdesign}, and it is recently extended to deep generative models, including RefineGNN~\citep{jin2021iterative}, MEAN~\citep{kong2022conditional}, RAAD~\citep{wu2024relation}, DiffAb~\citep{luo2022antigen}, etc. These models train a conditional antibody generator and screen out a number of high-quality antibodies using a $\Delta\Delta G$ predictor. These high-quality antibodies will be used as training data to further fine-tune the antibody generator for directed antibody optimization. In this paper, we rethink the role of $\Delta\Delta G$ prediction for antibody optimization, demonstrating that a simple yet effective $\Delta\Delta G$ predictor can directly serve as a good unsupervised antibody optimizer and explainer, without requiring additional functional annotations or deep generative models.


\vspace{-0.5em}