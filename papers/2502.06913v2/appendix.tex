\renewcommand\thefigure{A\arabic{figure}}
\renewcommand\thetable{A\arabic{table}}
\setcounter{table}{0}
\setcounter{figure}{0}
\renewcommand{\dblfloatpagefraction}{.95}


% Camera-Ready的时候需要补充上解释和设计相关模型的超参数设置。
\subsection*{A. Pseudo Code}
The pseudo-code of the proposed Uni-Anti framework is summarized in Algorithm~\ref{algo:1}.

\subfile{algorithms.tex}


\subsection*{B. Hyperparameters and Implementation Details}
Experiments are conducted based on Pytorch 1.8.0 on a hardware platform with Intel(R) Xeon(R) Gold 6240R @ 2.40GHz CPU and NVIDIA A100 GPU. The hyperparameters are as follows: learning rate $lr=0.0003$, batch size $B=32$, pre-training iterations $E_{\text{Aug}}=5,000$, $\Delta\Delta G$ iterations $E_{\Delta\Delta G}=15,000$, hidden dimension $F=128$, number of Transformer layers $L=4$, number of attention heads $K=4$ (by default), loss weight $\beta=0.1$, and momentum rate $\alpha=0.9$. Besides, we crop sequences or structures into patches containing 32 residues by first choosing a seed residue, and then selecting its 31 nearest neighbors based on the sequential distances or the C$_\beta$-C$_\beta$ distances. 


\subsection*{C. Antibody Optimization on SAbDab}
To further demonstrate Uni-Anti's effectiveness in antibody optimization in addition to anti-SARS-CoV-2 antibody, we further optimize CDR-H3 of 500 antigen-antibody complexes from the SAbDab dataset~\citep{dunbar2014sabdab} and report the average (optimal) $\Delta\Delta G$ scores of various baselines in Table.~\ref{tab:A1}. For RefineGNN, MEAN, dyMEAN, we incorporate Iterative Target Augmentation (ITA)~\citep{yang2020improving} into the optimization process to fine-tune the generators. For DiffAb, we directly generated 10,000 candidate samples and then selected the best one. For all baselines, we feed their optimized sequence together with wild-type complex structures into Light-DDG to predict $\Delta\Delta G$s. The results in Table.~\ref{tab:A1} show that Uni-Anti achieves superior results with a notably lower average $\Delta\Delta G$ score, demonstrating its potential advantages in terms of antibody optimization.

\subfile{table_A1.tex}