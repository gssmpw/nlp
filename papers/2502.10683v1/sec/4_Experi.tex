\section{Experiments and Results}
\begin{table*}[h]
    \centering
    \addtolength{\tabcolsep}{-0.2pt}
    % \resizebox{\textwidth}{!}{%
    \begin{tabular}{lcccccccccccc}
        \toprule
        Models & Backbone & Epochs & AP & $AP_{50}$ & $AP_{75}$ & $AP_S$ & $AP_M$ & $AP_L$ & GFLOPs & Params \\
        \midrule
        \multirow{4}{*}{DINO} & ResNet-101(T) & 36 & 66.9 & 91.3 & 77.5 & 60.3 & 66.2 & 73.4 & 197 & 66M \\
         & ResNet-50(S) & 12 & 57.7 & 86.4 & 64.7 & 46.1 & 57.6 & 68.0 & 155 & 47M \\
         % & KD-DETR & 12 & 62.1 & 88.7 & 72.3 & 55.2 & 61.8 & 67.9 & 155 & 47M \\
         & Ours & 12 & 63.7 & 89.7 & 72.6 & 57.3 & 63.2 & 69.8 & 155 & 47M \\
         & Gains & - & \textbf{+6.0} & \textbf{+3.3} & \textbf{+7.9} & \textbf{+11.2} & \textbf{+5.6} & \textbf{+1.8} & - & - \\
        \midrule
        \multirow{4}{*}{DINO} & ResNet-50(T) & 36 & 66.2 & 91.3 & 75.5 & 59.5 & 66.0 & 72.6 & 155 & 47M \\
         & ResNet-18(S) & 12 & 56.1 & 85.1 & 62.9 & 45.5 & 55.7 & 65.4 & 128 & 31M\\
         % & KD-DETR & 12 & 60.9 & 88.2 & 68.6 & 53.4 & 60.7 & 67.9 & 128 & 31M \\
         & Ours & 12 & 62.5 & 89.2 & 72.0 & 55.7 & 62.3 & 67.8 & 128 & 31M \\
         & Gains & - & \textbf{+6.4} & \textbf{+4.1} & \textbf{+9.1} & \textbf{+10.2} & \textbf{+6.6} & \textbf{+2.4} & - & - \\
        \midrule
        \multirow{4}{*}{DAB-DETR} & ResNet-101(T) & 50 & 50.9 & 81.6 & 54.1 & 37.7 & 50.8 & 61.9 & 98 & 63M \\
         & ResNet-50(S) & 50 & 47.5 & 80.1 & 48.6 & 30.8 & 47.6 & 62.3 & 56 & 44M \\
         % & KD-DETR & 50 & 48.5 & 81.6 & 50.4 & 34.9 & 48.5 & 60.6 & 56 & 44M \\
         & Ours & 50 & 50.1 & 81.8 & 54.3 & 36.0 & 50.6 & 60.9 & 56 & 44M \\
         & Gains & - & \textbf{+2.6} & \textbf{+1.7} & \textbf{+5.7} & \textbf{+5.2} & \textbf{+3.0} & \textbf{-1.4} & - & - \\
        \midrule
        \multirow{4}{*}{DAB-DETR} & ResNet-50(T) & 50 & 47.5 & 80.1 & 48.6 & 30.8 & 47.6 & 62.3 & 56 & 44M \\
         & ResNet-18(S) & 50 & 39.8 & 74.6 & 38.4 & 21.3 & 39.5 & 57.0 & 31 & 31M\\
         % & KD-DETR & 50 & 44.5 & 77.3 & 45.2 & 29.4 & 43.4 & 60.0 & 31 & 31M \\
         & Ours & 50 & 45.2 & 78.1 & 46.2 & 30.1 & 44.9 & 60.3 & 31 & 31M \\
         & Gains & - & \textbf{+5.4} & \textbf{+3.5} & \textbf{+7.8} & \textbf{+8.8} & \textbf{+5.4} & \textbf{+3.3} & - & - \\
        \midrule
        \multirow{4}{*}{Deformable-DETR} & ResNet-101(T) & 50 & 61.1 & 88.9 & 70.1 & 51.7 & 60.9 & 70.3 & 148 & 59M \\
         & ResNet-50(S) & 50 & 58.7 & 87.6 & 67.3 & 51.3 & 58.3 & 67.7 & 106 & 40M \\
         % & KD-DETR & 50 & 59.9 & 88.1 & 68.7 & 53.3 & 58.8 & 68.1 & 106 & 40M \\
         & Ours & 50 & 60.9 & 88.5 & 68.1 & 53.0 & 59.2 & 68.6 & 106 & 40M \\
         & Gains & - & \textbf{+2.2} & \textbf{+0.9} & \textbf{+0.8} & \textbf{+1.7} & \textbf{+0.9} & \textbf{+0.9} & - & - \\
        \midrule
        \multirow{4}{*}{Deformable-DETR} & ResNet-50(T) & 50 & 58.7 & 89.8 & 66.8 & 51.3 & 58.3 & 69.0 & 106 & 40M \\
         & ResNet-18(S) & 50 & 56.9 & 88.5 & 63.0 & 48.8 & 57.1 & 66.1 & 79 & 24M\\
         % & KD-DETR & 50 & 58.5 & 88.9 & 65.6 & 52.2 & 58.3 & 65.6 & 79 & 24M \\
         & Ours & 50 & 59.8 & 89.1 & 68.8 & 54.8 & 60.9 & 68.2 & 79 & 24M \\
         & Gains & - & \textbf{+2.9} & \textbf{+0.6} & \textbf{+5.8} & \textbf{+6.0} & \textbf{+3.8} & \textbf{+2.1} & - & - \\
        \bottomrule
    \end{tabular}
    \caption{Distillation results of our CLoCKDistill method compared to the baseline across different DETR detectors on the KITTI dataset. T: Teacher, S: Student. Input image dimensions: (402, 1333). All models converged within the indicated epochs.}
    \label{tab:KITTI}
    % }
    \vspace{0.2in}
\end{table*}





\section{COCO Problem}




On round $t,$ the online policy first chooses an admissible action $x_t \in \mathcal{X}\subset \bbR^d,$ 
 and then the adversary chooses a convex cost function $f_t: \mathcal{X} \to \mathbb{R}$ and a constraint of the form $g_{t}(x) \leq 0,$ where $g_{t}: \mathcal{X} \to \mathbb{R}$ is a convex function. Once the action $x_t$ has been chosen, we let $\nabla f_t(x_t)$ and full function $g_t$ or the set $\{x: g_t(x)\le 0\}$ to be revealed, as is standard in the literature.
 We now state the  standard assumptions made  in the  literature while studying the COCO problem \cite{guo2022online, yi2021regret, neely2017online, Sinha2024}.
\begin{assumption}[Convexity] \label{cvx}
$\mathcal{X} \subset \bbR^d$ is the admissible set that is closed, convex and has a finite Euclidean diameter $D$.  The cost function $f_t: \mathcal{X} \mapsto \mathbb{R}$ and the constraint function $g_{t}: \mathcal{X} \mapsto \mathbb{R}$ are convex for all $t\geq 1$.  
 %Moreover, $D$ is known ahead of time.
\end{assumption}
%\vspace{-0.18in}
\begin{assumption}[Lipschitzness] \label{bddness}
 %We have $\textrm{diam}(\mathcal{X}) \leq D, ||\nabla f_t(x)||_2 \leq G/2, \textrm{and}~ ||\nabla g_t(x))||_2 \leq G/2,~\forall t, \forall x\in \mathcal{X}$ for some finite constants $D$ and $G.$ If the functions are not necessarily differentiable, we require that the maximum magnitude of the subgradients be bounded accordingly.  Each
All cost functions $\{f_t\}_{t\geq 1}$ and the constraint functions $\{g_{t}\}_{ t\geq 1}$'s are $G$-Lipschitz, i.e., for any $x, y \in \mathcal{X},$ we have 
 \begin{eqnarray*}
 	|f_t(x)-f_t(y)| \leq G||x-y||,~
 	|g_{t}(x)-g_{t}(y)| \leq G||x-y||, ~\forall t\geq 1.
 \end{eqnarray*}
	\end{assumption}
 \begin{assumption}[Feasibility] \label{feas-constr}
With ${\cal G}_t = \{x\in \cX : g_t(x)\le 0\}$, we assume that  $\mathcal{X}^\star = \cap_{t=1}^T G_t  \neq \varnothing $.	
Any action $x^\star \in \cX^\star$ is defined to be feasible. %There exists some feasible action $x^\star \in \mathcal{X} $ s.t. $g_{t}(x^\star) \leq 0, \ \forall t \ \in T.$ The set of all feasible actions, denoted by $\mathcal{X}^\star,$ is called the feasible set. The feasibility assumption implies that $\mathcal{X}^\star \neq \emptyset.$
\end{assumption}
The feasibility assumption distinguishes the cost functions from the constraint functions and is common across all previous literature on COCO \cite{guo2022online, neely2017online, yu2016low,yuan2018online,yi2023distributed, georgios-cautious,Sinha2024}. 

%In light of Assumption \ref{feas-constr}, including multiple constraints at each time is straightforward because of the existence of a common feasible point for all $g_t, t=1, \dots, T$.

For any real number $z$, we define $(z)^+ \equiv \max(0,z).$ Since $g_{t}$'s are revealed after the action $x_t$ is chosen, any online policy need not necessarily take feasible actions on each round. 
 Thus in addition to the static\footnote{ The static-ness refers to the fixed benchmark using only one action $x^\star$ throughout the horizon of length $T$}  regret defined below
%\vspace{-0.1in}
\begin{eqnarray} \label{regret-def}
	\textrm{Regret}_{[1:T]} \equiv \sup_{\{f_t\}_{t=1}^T} \sup_{x^\star \in \mathcal{X}^\star} \textrm{Regret}_{[1:T]}(x^\star), \end{eqnarray}
	where $\textrm{Regret}_{[1:T]}(x^\star) \equiv \sum_{t=1}^T f_t(x_t) - \sum_{t=1}^T f_t(x^\star)$, 
an additional obvious metric of interest is  the total cumulative constraint violation (CCV) defined as 
 %\vspace{-0.1in}
  \begin{eqnarray} \label{gen-oco-goal}
 	\textrm{CCV}_{[1:T]}  = \sum_{t=1}^T (g_{t}(x_t))^+. 
	\end{eqnarray}
	 Under the standard assumption (Assumption \ref{feas-constr}) that $\mathcal{X}^\star$ is not empty, the goal is to design an online policy to simultaneously achieve a small regret \eqref{intro-regret-def} with $x^\star \in \mathcal{X}^\star$ and a small CCV \eqref{intro-gen-oco-goal}. We refer to this problem as the constrained OCO (COCO). 
	 
For simplicity, we define set 
\begin{equation}\label{defn:S}
 \ S_t = \cap_{\tau=1}^t {\cal G}_\tau,
\end{equation}
where $G_t$ is as defined in Assumption \ref{feas-constr}.
All ${\cal G}_t$'s are convex and consequently, all $S_t$'s are convex and are nested, i.e. $S_t\subseteq S_{t-1}$. Moreover, because of Assumption \ref{feas-constr},  each $S_t$ is non-empty and in particular $\cX^\star\in S_t$ for all $t$. After action $x_t$ has been chosen, set $S_t$ controls the constraint violation, which can be used to write an upper bound on the $\textrm{CCV}_{[1:T]}$ as follows.

\begin{definition}
For a convex set $\chi$ and a point $x\notin \chi$, 
$$\text{dist}(x,\chi) = \min_{y\in \chi} || x-y||.$$
\end{definition}

Thus, the constraint violation at time $t$, 
\begin{equation}\label{eq:distviolationrelation}
(g_t(x_t))^+ \le G\text{dist}(x_t,S_t), \ \text{and} \  \textrm{CCV}_{[1:T]}  \le G\sum_{t=1}^T \text{dist}(x_t,S_t),
\end{equation}
where $G$ is the common Lipschitz constants for all $g_t$'s.


	 
	 
% \begin{figure*}[h]
% \begin{center}
%     \begin{subfigure}[b]{0.19\textwidth}
%     \includegraphics[]{AnonymousSubmission/Images/000000000154_gt.jpg}
%     \caption{(a)}
%     \end{subfigure}
%     \begin{subfigure}[b]{0.19\textwidth}
%     \includegraphics[]{AnonymousSubmission/Images/teacher_000000000154.jpg}
%     \caption{(b)}
%     \end{subfigure}
% \includegraphics[width=0.19\linewidth]{AnonymousSubmission/Images/student_000000000154.jpg}
% \includegraphics[width=0.19\linewidth]{AnonymousSubmission/Images/kddetr_000000000154.jpg}
% \includegraphics[width=0.19\linewidth]{AnonymousSubmission/Images/ours_000000000154.jpg}
% \end{center}
% \caption{
% Attention maps from (a) Image w/ bounding boxes, (b) the teacher model, (c) the student model (d) KD-DETR distilled \cite{wang2022distilling} (e) our distilled model. Our distilled model allocates more focus to relevant objects, where red denotes the highest attention and blue the lowest.}
% \label{fig:activate vis}
% \end{figure*}
\begin{figure*}
\begin{center}
    \begin{subfigure}[b]{0.19\textwidth}
        \includegraphics[width=\textwidth]{sec/Images/000000000154.jpg}
        \caption{Original Image}\label{fig:ori_zebra}
    \end{subfigure}
    \begin{subfigure}[b]{0.19\textwidth}
        \includegraphics[width=\textwidth]{sec/Images/teacher_000000000154.jpg}
        \caption{Teacher}\label{fig:teacherattn_zebra}
    \end{subfigure}
    \begin{subfigure}[b]{0.19\textwidth}
        \includegraphics[width=\textwidth]{sec/Images/student_000000000154.jpg}
        \caption{Student baseline}\label{fig:studentattn_zebra}
    \end{subfigure}
    \begin{subfigure}[b]{0.19\textwidth}
        \includegraphics[width=\textwidth]{sec/Images/kddetr_000000000154.jpg}
        \caption{KD-DETR}\label{fig:kddetrattn_zebra}
    \end{subfigure}
    \begin{subfigure}[b]{0.19\textwidth}
        \includegraphics[width=\textwidth]{sec/Images/ours_000000000154.jpg}
        \caption{Ours}\label{fig:ourattn_zebra}
    \end{subfigure}

    % \begin{subfigure}[b]{0.19\textwidth}
    %     \includegraphics[width=\textwidth]{sec/Images/000000000143.jpg}
    %     \caption{Original Image}\label{fig:ori_birds}
    % \end{subfigure}
    % \begin{subfigure}[b]{0.19\textwidth}
    %     \includegraphics[width=\textwidth]{sec/Images/teacher_000000000143.jpg}
    %     \caption{Teacher}\label{fig:teacherattn_birds}
    % \end{subfigure}
    % \begin{subfigure}[b]{0.19\textwidth}
    %     \includegraphics[width=\textwidth]{sec/Images/student_000000000143.jpg}
    %     \caption{Student baseline}\label{fig:studentattn_birds}
    % \end{subfigure}
    % \begin{subfigure}[b]{0.19\textwidth}
    %     \includegraphics[width=\textwidth]{sec/Images/kddetr_000000000143.jpg}
    %     \caption{KD-DETR}\label{fig:kddetrattn_birds}
    % \end{subfigure}
    % \begin{subfigure}[b]{0.19\textwidth}
    %     \includegraphics[width=\textwidth]{sec/Images/ours_000000000143.jpg}
    %     \caption{Ours}\label{fig:ourattn_birds}
    % \end{subfigure}
    
\end{center}
\caption{Attention maps of (a) an example COCO image from the (b) teacher, (c) student baseline, (d) KD-DETR distilled \cite{wang2024knowledge}, and (e) our distilled models. Our distilled model allocates more focus to relevant objects and exhibits more clearly defined boundaries. Red indicates the highest attention and blue the lowest.}
\label{fig:activate vis}
\end{figure*}


\subsection{Datasets}
We demonstrate our method's efficacy using two datasets:

\noindent\textbf{KITTI} \cite{Geiger2012CVPR} offers a widely used 2D object detection benchmark for autonomous driving and computer vision research. It specifically represents scenarios that demand low latency and constrained computational resources. The dataset includes seven types of road objects and contains 7,481 annotated images, which are split into a training set and a validation set using an 8:2 ratio. We follow the convention in \cite{lan2024gradient} for pre-processing.
%Specifically, \textit{car, van, truck, tram} are regrouped as Car, \textit{pedestrian, person} as Pedestrian, and \textit{cyclist} as Cyclist.

\noindent\textbf{MS COCO} is a dataset that includes 80 categories spanning a wide range of objects, with 117K training images and 5K validation images.
We adopt COCO-style evaluation metrics, e.g., mean Average Precision (mAP), for both datasets. 
% delete when needed
The validation process for each dataset closely follows these standard metrics to ensure consistency and comparability.

\subsection{Implementation Details}
We evaluate our approach on three state-of-the-art DETR detectors: DAB-DETR \cite{liu2022dab}, Deformable-DETR \cite{zhu2020deformable}, and DINO \cite{zhang2023dino}. All experiments are conducted using the MMDetection framework \cite{mmdetection} on Nvidia A100 GPUs, adhering to the original hyperparameter settings and optimizer configurations. ResNets \cite{he2016deep} are used as backbones for both the teacher and student models. The distillation loss coefficients are set as follows: $\alpha = 5\times 10^{-5}, \beta = 1\times 10^{-7}, \lambda_{kl} = 1$, $\lambda_{L1} = 5$, and $\lambda_{GIoU} = 2$. The number of distillation points is set to 300 for DAB-DETR and Deformable DETR, and 900 for DINO. The number of copies for target-aware queries is set to 3 for all experiments. 

\subsection{Quantitative Results}
Table \ref{tab:KITTI} presents the results on KITTI, highlighting the effectiveness of our distillation method across various DETR models and backbones. Notably, our distillation approach boosts the performance of DINO with ResNet-18 and ResNet-50 backbones by 6.4\% and 6.0\% mAP, respectively. %The student distilled by our method surpasses the state-of-the-art KD-DETR \cite{wang2024knowledge} by 1.6\% for both ResNet-50 and ResNet-18 backbones. 
For DAB-DETR, our method improves the performance of the detectors using ResNet-18 and ResNet-50 backbones by 5.4\% and 2.6\% mAP, respectively. Furthermore, in the case of Deformable DETR, our approach achieves mAP improvements of 2.9\% and 2.2\% for the ResNet-18 and ResNet-50 backbones, respectively.

%We also beat the baseline and competing methods by clear margins on COCO (Tab. \ref{tab:COCO}). For instance, our method boosts DAB-DETR performance by 2.8\% mAP when using the ResNet-18 backbone. This outperforms the cutting-edge distillation method \cite{wang2024knowledge} by 1.1\% mAP.

We also beat the baselines by clear margins on the COCO dataset (Table \ref{tab:COCO}). For instance, our method boosts DINO performance by 2.9\% mAP when using the ResNet-18 backbone. %This outperforms the cutting-edge distillation method \cite{wang2024knowledge} by 1.1\% mAP.

In Table~\ref{tab:kd methods}, we present a comparison of our method with state-of-the-art knowledge distillation methods for object detectors, including FitNet \cite{romero2014fitnets}, FGD \cite{yang2022focal}, MGD \cite{yang2022masked}, and KD-DETR \cite{wang2024knowledge}. Notably, KD-DETR \cite{wang2024knowledge} (CVPR 2024) represents the latest advancement specifically tailored for DETR distillation. According to the results, while KD-DETR demonstrates superior performance over conventional KD methods, the margin over MGD is not significant. Our method enlarges the performance gap between DETR-oriented and conventional KD methods by taking advantage of transformer-specific global context and strategically incorporating ground truth information into both feature and logit distillation. 
% \begin{table*}[h]
% \centering
% \begin{tabular}{|lccccccc|}
% \toprule
% Method & Epoch & AP & $AP_{50}$ & $AP_{75}$ & $AP_s$ & $AP_m$ & $AP_l$ \\ \midrule
% DINO-R50 & 36 & 66.2 & 91.3 & 75.5 & 59.5 & 66.0 & 72.6 \\ 
% DINO-R18 & 12 & 56.1 & 85.1 & 62.9 & 45.5 & 55.7 & 65.4 \\ 
% FGD & 12 & 57.1 & 85.3 & 64.5 & 49.1 & 56.9 & 63.1 \\ 
% MGD & 12 & 60.1 & 87.7 & 67.5 & 53.7 & 53.7 & 59.6 \\ 
% FitNet & 12 & 57.4 & 85.2 & 63.8 & 48.5 & 57.2& 64.9 \\
% KD-DETR & 12 & 60.9 & 88.2 & 68.6 & 53.4 & 60.7 & 67.9 \\ 
% Ours & 12 & \textbf{62.5} & 89.2 & 72.0 & 55.7 & 62.3 & 67.8 \\ \bottomrule
% \end{tabular}
% \caption{Comparison with different state-of-the-art KD methods on DINO.}
% \label{tab:kd methods}
% \end{table*}


\begin{table}[h]
\centering
\begin{tabular}{c|cccc}
\toprule
Method & Epoch & AP & $AP_{50}$ & $AP_{75}$ \\ \midrule
DINO ResNet-50(T) & 36 & 66.2 & 91.3 & 75.5  \\ 
DINO ResNet-18(S) & 12 & 56.1 & 85.1 & 62.9  \\ 
FitNet & 12 & 57.4 & 85.2 & 63.8 \\
FGD & 12 & 57.1 & 85.3 & 64.5 \\ 
MGD & 12 & 60.1 & 87.7 & 67.5 \\ 
KD-DETR & 12 & 60.9 & 88.2 & 68.6  \\ \midrule
\textbf{Ours} & 12 & \textbf{62.5} & \textbf{89.2} & \textbf{72.0} \\ \bottomrule
\end{tabular}
\caption{Comparison of our method with state-of-the-art KD methods. T: Teacher, S: Student. Detector: DINO, Dataset: KITTI. All models converged within the indicated epochs.}
\label{tab:kd methods}
\end{table}

\subsection{Qualitative Results}
The attention maps of different approaches in Figure~\ref{fig:activate vis} also demonstrate our CLoCKDistill's effectiveness. In Figure~\ref{fig:teacherattn_zebra}, the teacher model's attention map reveals wide dispersion, with attention spread across both the foreground objects (zebras) and the background. This broad focus reflects the teacher model's strong representational capacity due to its larger size; however, it also leads to substantial attention on irrelevant background areas, which can complicate or interfere with accurate detection.
Unlike the over-parameterized teacher model capturing many unnecessary and irrelevant features, the capacity-limited student baseline learns to focus most of its attention on the foreground objects while trying to achieve a satisfactory detection performance on its own (as shown in Figure~\ref{fig:studentattn_zebra}). However, this focus is confined to specific boundary areas, and the student baseline exhibits variability in attention across zebras of different scales, highlighting the model's sensitivity to scale changes.
When learning from the teacher via KD-DETR (Figure~\ref{fig:kddetrattn_zebra}), the student model inherits much of the teacher’s irrelevant focus on distracting background features, which impedes its ability to concentrate effectively on objects of interest.
In contrast, our CLoCKDistill model (Figure~\ref{fig:ourattn_zebra}) achieves a more targeted focus, prioritizing relevant objects. Additionally, the attention boundaries are more sharply defined, indicating that our distilled model has developed a clearer and more precise understanding of the objects.


\subsection{Ablation Study}
In this section, we conduct ablation studies on the key components of our CLoCKDistill method and the number of transformer encoder-decoder layers.
%The experiments are on the KITTI dataset using DINO, with ResNet-50 as the teacher and ResNet-18 as the student backbone.
\subsubsection{CLoCKDistill components}
Table \ref{tab:Ablation} presents the influence of the main components of our CLoCKDistill, i.e., distilling memory, masking memory with location info, and target-aware queries. As we can see, memory distillation alone boosts the student model's performance by 3.8\% mAP. Further masking memory with location info enhances performance by 5.5\% mAP. By integrating all components, including target-aware queries, we achieve a 6.4\% mAP improvement for the DINO detector with a ResNet-18 backbone.
The map of strategy to its corresponding abbreviated tag is listed in Table\ref{table:DA_map}.
\begin{table}[h]
  \centering
    \begin{tabular}{{p{0.75\linewidth} p{0.25\linewidth} }}
    \hline
    Strategy & Tag\\
    \hline
    Acknowledge (Backchannel) & Ack\\
    Statement-non-opinion & StaNo\\
    Statement-opinion & Sta\\
    Affirmation and Reassurance & Agr\\
    Appreciation & App\\
    Conventional-closing & ConC\\
    Hedge & H\\
    Other & Oth\\
    Quotation & Quo\\
    Action-directive & AcD\\
    Collaborative Completion & CoC\\
    Restatement or Paraphrasing & Rep\\
    Offers Options Commits &Off\\
    Self-talk & Sel\\
    Apology & Apo\\
    Reflection of Feelings & RoF\\
    Yes-No-Question & YNQ\\
    Wh-Question & WhQ\\
    Declarative Yes-No-Question & DYNQ\\
    Open-Question & OpQ\\
    Or-Clause & OrC\\
    Conventional-opening & CoO\\
    Self-disclosure & Sd\\
    Providing Suggestions & PS\\
    Information & I\\
    \hline
    \end{tabular}
  \caption{Dialogue Act to its corresponding tag.}
  \label{table:DA_map}
\end{table}

There are two kinds of strategies: backward-looking and forward-looking. Backward-looking strategies reflect how the current utterance relates to the previous discourse. Forward-looking strategies reflect the current utterance constrains the future beliefs and actions of the participants and affects the discourse. Table \ref{tab:backward-looking DA} and Table \ref{tab:forward-looking DA} provide definitions and examples of each.
% The backward-looking strategies, definitions, and examples are shown in Table \ref{tab:backward-looking DA}. The forward-looking strategies, definitions, and examples are shown in Table \ref{tab:forward-looking DA}. 

\begin{table*}[h]
  \centering
   \resizebox{0.96\textwidth}{!}{
      \begin{tabular}{{p{0.1\textwidth} p{0.6\textwidth} p{0.3\textwidth}}}
        \hline
        \textbf{Strategy}           & \textbf{Definiton} & \textbf{Example} \\
        \hline
        StaNo
        & A factual statement or descriptive utterance that does not include an opinion. 
        & Me, I'm in the legal department.   \\
        Ack
        & A brief utterance that signals understanding, agreement, or active listening.
        & Uh-huh.\\
        Sta
        & A statement that conveys a personal belief, judgment, or opinion.
        & I think it's great\\
        Agr
        & Affirm the help seeker's strengths, motivation, and capabilities and provide reassurance and encouragement.
        & That's exactly it.\\
        App
        & An expression of gratitude, admiration, or acknowledgment of another’s effort or input. 
        & I can imagine.\\
        ConC
        & A formal or socially standard utterance signaling the end of a conversation. 
        & Well, it's been nice talking to you.\\
        H
        & An expression that introduces uncertainty or qualification to a statement, often to soften its impact. 
        & I don't know if I'm making any sense or not.\\
        Oth
        & Exchange pleasantries and use other support strategies that do not fall into the above categories. 
        & Well give me a break, you know.\\
        Quo
        & A direct or indirect repetition of someone else’s words. 
        & Albert Einstein once said, “Imagination is more important than knowledge.”\\
        AcD
        & A command, request, or suggestion directing someone to take action.
        & Why don't you go first\\
        CoC
        &A continuation or completion of someone else’s utterance in a collaborative manner. 
        & If we want to make it to the top of the mountain before sunset, we should…\\
        Rep
        & A simple, more concise rephrasing of the help-seeker's statements that could help them see their situation more clearly. 
        & It sounds like you’re saying that you’re struggling to stay on top of your work, and it’s leaving you feeling overwhelmed.\\
        Off
        & A statement proposing choices, making a commitment, or offering to do something. 
        & I'll have to check that out\\
        Sel
        & An utterance directed at oneself, often reflecting internal thought processes or problem-solving.
        & What's the word I'm looking for\\
        Apo
        & An expression of regret or asking for forgiveness. 
        & I'm sorry.\\
        RoF
        & Articulate and describe the help-seeker's feelings.
        & It sounds like you’re feeling really frustrated and drained because your efforts don’t seem to be paying off.\\
        \hline
      \end{tabular}
  }
  \caption{Backward-looking strategies, definition, and example. }
  \label{tab:backward-looking DA}
\end{table*}
\clearpage
\newpage
\begin{table*}[htbp!]
  \centering
   \resizebox{0.95\textwidth}{!}{
      \begin{tabular}{{p{0.1\textwidth} p{0.5\textwidth} p{0.4\textwidth}}}
        \hline
        \textbf{Strategy} & \textbf{Definiton} & \textbf{Example} \\
        \hline
        YNQ
        & A question expecting a binary (yes/no) response. 
        & Do you have to have any special training?\\
        WhQ
        & A question beginning with a wh-word (e.g., what, who, where), seeking specific information. 
        & Well, how old are you?\\
        DYNQ
        & A statement posed as a question, expecting a yes/no answer.
        & So you can afford to get a house?\\
        OpQ
        & A broad question inviting a wide range of responses, often conversational.
        & How about you?\\
        OrC
        & A question offering explicit alternatives, often in the form of “or.”
        & or is it more of a company?\\
        CoO
        & A socially standard utterance used to initiate a conversation.
        & How are you?\\
        Sd
        & Divulge similar experiences that you have had or emotions that you share with the help-seeker to express your empathy.
        & I completely understand how you feel. I remember feeling the same way before my first big presentation at work. I was so anxious, but I found that practicing a few extra times really helped calm my nerves.\\
        PS
        & Provide suggestions about how to change, but be careful to not overstep and tell them what to do.
        & You can keep a note to stop your idea from going. \\
        I
        & Provide useful information to the help-seeker, for example with data, facts, opinions, resources, or by answering questions.
        & Taking silver line from Washington D.C. to Dulles Intel Airport costs about 1 hour.\\
        \hline
      \end{tabular}
  }
  \caption{Forward-looking strategies, definition, and example. }
  \label{tab:forward-looking DA}
\end{table*}
\clearpage
\newpage




\subsubsection{Number of transformer layers}
\begin{comment}
\begin{table}[h]
    \centering
    \addtolength{\tabcolsep}{-0.1pt}
    \begin{tabular}{c|ccc|cc}
        \toprule
        Enc/Dec & AP & $AP_{50}$ & $AP_{75}$ & FPS & Params \\
        \midrule
        6/6 & 56.1 & 85.1 & 62.9 & 27.5 & 31M\\
        Ours & 62.5(+6.4) & 89.2 & 72.0 & 27.5  & 31M\\
        3/6 & 56.1 & 84.8 & 63.4 & 34.7 & 27M\\
        Ours & 60.4(+4.3) & 89.1 & 68.2 & 34.7 & 27M\\
        6/3 & 54.7 & 81.7 & 61.4 & 33.6 & 26M\\
        Ours & 61.7(+5.0) & 88.2 & 70.9 & 33.6 & 26M\\
        3/3 & 52.5 & 78.7 & 59.7 & 44.8 & 22M\\
        Ours & 59.5(+7.0) & 87.2 & 67.2 & 44.8 & 22M \\
        \bottomrule
    \end{tabular}
    \caption{Distillation performance with varying numbers of transformer layers in DINO on the KITTI dataset. Teacher backbone: ResNet-50, student backbone: ResNet-18.}
    \label{tab:enc_dec}
\end{table}
\end{comment}
\begin{table}
    \centering
    \addtolength{\tabcolsep}{-0.1pt}
    \begin{tabular}{c|c|ccc|c}
        \toprule
        Model & Enc/Dec & AP & $AP_{50}$ & $AP_{75}$ & FPS \\
        \midrule
        StuBase & 6/6 & 56.1 & 85.1 & 62.9 & \multirow{2}{*}{27.5}\\
        Ours & 6/6 & 62.5 \textbf{(\textuparrow 6.4)} & 89.2 & 72.0 & \\
        \midrule
        StuBase & 3/6 & 56.1 & 84.8 & 63.4 & \multirow{2}{*}{34.7}\\
        Ours & 3/6 & 60.4 \textbf{(\textuparrow 4.3)} & 89.1 & 68.2 & \\
        \midrule
        StuBase & 6/3 & 54.7 & 81.7 & 61.4 & \multirow{2}{*}{33.6}\\
        Ours & 6/3 & 61.7 \textbf{(\textuparrow 5.0)} & 88.2 & 70.9 &\\
        \midrule
        StuBase & 3/3 & 52.5 & 78.7 & 59.7 & \multirow{2}{*}{44.8}\\
        Ours & 3/3 & 59.5 \textbf{(\textuparrow 7.0)} & 87.2 & 67.2 & \\
        \bottomrule
    \end{tabular}
    \caption{Distillation performance with varying numbers of transformer encoder and/or decoder layers in the DINO detector on the KITTI dataset. StuBase indicates the student baseline with a ResNet-18 backbone. The teacher has a ResNet-50 backbone with 6 encoders and 6 decoders, with an FPS of 19.8. FPS stands for frames per second, which is measured on a single Nvidia A100 GPU.}
    \label{tab:enc_dec}
\end{table}
The transformer detection head requires considerable computation, so reducing the number of encoder and decoder layers can improve model efficiency. To explore this, we conducted experiments with different reduced layer configurations. To address the layer mismatch between the student and teacher models, we applied our distillation method only to the final encoder and decoder layers.

Table~\ref{tab:enc_dec} shows the impact of varying the number of transformer encoder and decoder layers. As expected, reducing the number of encoder/decoder layers decreases the model performance. The student baseline is more sensitive to reductions in decoder layers, while our distilled model is more affected by reductions in encoder layers.
Notably, with only half of the encoder and decoder layers, our distilled model still surpasses the full-scale baseline by 3.4\% in mAP and achieves a 1.6x increase in frames per second (FPS). 
