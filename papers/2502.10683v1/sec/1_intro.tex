\section{Introduction}
\label{sec:intro}
Early deep learning detection approaches predominantly rely on convolutional neural networks (CNNs) to process regional features, involving extensive tuning of hand-crafted components such as anchor sizes and aspect ratios \cite{ren2015faster, lin2017focal}. Inspired by the success of transformers in NLP and visual classification tasks, DEtection TRansformer (DETR) \cite{carion2020end} has been introduced for visual detection. DETR treats object detection as a set prediction problem, eliminating the need for hand-crafted anchor design \cite{ren2015faster} and non-maximum suppression (NMS) \cite{hosang2017learning}. Despite achieving state-of-the-art performance, transformer-based detectors \cite{carion2020end, zhu2020deformable} suffer from high computational costs, making them challenging to deploy in real-time applications (e.g., autonomous driving perception).
Knowledge distillation (KD) \cite{hinton2015distilling}, a technique where a smaller model is trained to mimic a larger model's behavior, has shown to be an effective compression method for convolution-based detectors \cite{zhang2020improve, li2017mimicking, wang2019distilling, dai2021general, chen2017learning, dai2021general, zheng2022localization}. However, the exploration of knowledge distillation in DETRs is still in its early stages, with several critical issues remaining.

\paragraph{Issues in feature-level distillation} Most feature-level distillation methods, e.g., \cite{chang2023detrdistill, yang2022focal, guo2021distilling}, focus on the backbone\footnote{In this paper, the term backbone is used loosely to encompass both the traditional backbone network and the neck layers (e.g., FPN).} features that capture only local context from CNNs.
Additionally, previous methods \cite{li2017mimicking, zhang2020improve} often rely on the teacher model in all locations, which can sometimes mislead the student model. Also, these works typically fail to balance the contributions from the foreground and background as well as from objects of different sizes. \citet{ijcai2024p74} focus exclusively on decoder distillation, overlooking the crucial context information embedded in the encoder features. 
%\citet{chang2023detrdistill} utilizes the teacher’s queries to interact with convolution-based FPN features. 
%However, those approaches all fail to leverage the richer global context available in the encoder outputs.
To address these challenges, we propose using location-and-context-aware DETR memory (i.e., transformer encoder output) as a more appropriate alternative for distillation. The encoder processes the sequence of token embeddings, including the positional encoding, through multiple layers of self-attention and feed-forward neural networks. Unlike local CNN features that most previous works use, the memory in DETR captures the transformer-specific global context provided by self-attention, allowing for better reasoning about long-range relationships within the entire image, which is crucial for object detection \cite{hu2018relation}.
Table~\ref{tab:Preliminary} demonstrates the superiority of distilling the memory over distilling backbone features or using a combination of both.

\documentclass[11pt]{article}

\usepackage{amsmath}
\usepackage{amssymb}
\usepackage{amsthm}
\usepackage{graphicx}
\usepackage{enumerate}
\usepackage{theoremref}
%you can add more packages using the same code above
\usepackage{spconf}

\usepackage{caption}
\usepackage{bm}
%\usepackage{natbib}
\usepackage[dvipsnames]{xcolor}
\usepackage{caption}
\usepackage[font=small,labelfont=bf]{caption}
\usepackage{subcaption}
\usepackage{algorithmic}
\usepackage{algorithm}
\usepackage[hidelinks]{hyperref}
\usepackage{cleveref}
\usepackage{todonotes}
%\usepackage{subfigure}
\usepackage{subcaption}
\usepackage{titlesec}
\usepackage{setspace}

%------------------
 
\newtheorem{theorem}{Theorem}[section]
\newtheorem{proposition}[theorem]{Proposition}
\newtheorem{lemma}[theorem]{Lemma}
\newtheorem{corollary}[theorem]{Corollary}

\newtheorem{question}[theorem]{Question}
\newtheorem{remark}[theorem]{Remark}
\newtheorem{conjecture}{Conjecture}
\theoremstyle{definition}
\newtheorem{definition}{Definition}
\newtheorem*{example}{Example}

\graphicspath{{fig/}}

\def\fro{\textnormal{F}}
\def\circ{\textnormal{circ}}
\def\diag{\textnormal{diag}}
\def\ifft{\textnormal{ifft}}
\def\fft{\textnormal{fft}}
\def\reshape{\textnormal{resh}}
\newcommand{\lx}[1]{\textcolor{red}{#1 }}
 
%------------------

%Everything before begin document is called the pre-amble and sets out how the document will look
%It is recommended you don't touch the pre-amble until you are familiar with LateX
%\titlespacing*{\subsection}{0pt}{10pt}{5pt}

\setstretch{0.9}
\begin{document}

\title{Three-dimensional  Signal Processing: A New Approach in Dynamical Sampling via Tensor Products}%{Dynamical Sampling in Multidimensional Signal Recovery}
\name{Yisen Wang$^{1}$, HanQin Cai$^{2}$,  Longxiu Huang$^{1}$}

\address{$^{1}$ Michigan State University\\$^{2}$ University of Central Florida
}
% \name{Xia Li, Longxiu Huang, Deanna Needell}
% \address{University of California, Los Angeles}
\date{}
\maketitle

\begin{abstract}
The dynamical sampling problem is centered around reconstructing signals that evolve over time according to a dynamical process, from spatial-temporal samples that may be noisy. This topic has been thoroughly explored for one-dimensional signals. Multidimensional signal recovery has also been studied, but primarily in scenarios where the driving operator is a convolution operator. In this work, we shift our focus to the dynamical sampling problem in the context of three-dimensional signal recovery, where the evolution system can be characterized by tensor products. Specifically, we provide a necessary condition for the sampling set that ensures successful recovery of the three-dimensional signal. Furthermore, we reformulate the reconstruction problem as an optimization task, which can be solved efficiently.  To demonstrate the effectiveness of our approach, we include some straightforward numerical simulations that showcase the reconstruction performance.
\end{abstract}


 
\section{Introduction}
The concept of dynamical sampling, first introduced in the works of \cite{lu2009spatial}, %lu2011localization, ranieri2011sampling, hormati2009distributed}, 
addresses the challenge of compensating for spatial sampling deficiencies by leveraging the temporal evolution of data during recovery \cite{aldroubi2013dynamical}. This method utilizes the time-dependent nature of signal evolution, driven by external forces, to enhance the quality of the collected samples, % \cite{huang2021robust}, 
setting it apart from traditional static sampling.

Dynamical sampling allows for efficient data acquisition by sampling only a subset of data points in time and space. This is particularly useful in systems where collecting data from every point in space or at every moment in time is either costly or impractical. For instance, in large sensor networks or medical imaging, dynamical sampling can reduce the number of measurements needed while still allowing accurate reconstruction of the full signal.

Dynamical sampling has been extensively explored for one-dimensional signals, with studies like \cite{aldroubi2017dynamical, aldroubi2019frames, aldroubi2017iterative,huang2024robust}.  Particularly, for the scenario where the evolution operator is represented by a matrix \(A \in \mathbb{C}^{d \times d}\) and the signal, \(f \in \mathbb{C}^d\), is to be recovered,  \cite{aldroubi2017dynamical} provides necessary and/or sufficient conditions on the sampling set of indices \(\Omega \subseteq \{1, 2, \ldots, d\}\) and  the numbers $\{\ell_i\}_{i\in\Omega}$ such that $f\in\mathbb{C}^d$ can be recovered from the samples $\{A^jf(i):i\in\Omega,j=0,\cdots,\ell_i-1\}$. 

Although one-dimensional signals have been extensively studied, research on multi-dimensional signals remains relatively limited.  However, in industrial applications, the observed time-varying signals often involve multiple variables, highlighting the critical importance of studying multi-dimensional dynamical sampling. For example, in sensor networks used for environmental monitoring or industrial processes, data such as temperature, pressure, and humidity are collected over time across various spatial locations, forming a three-dimensional tensor \cite{vairamani2013}. Each dimension can represent spatial coordinates and time, highlighting the complexity of the data. To date, the primary research in this area has focused on signals evolving under convolution-driven operators in multi-dimensional settings \cite{aceska2015multidimensional}. This underscores a significant gap in the literature and highlights the need for further investigation into multi-dimensional dynamical sampling. %Dynamical sampling provides an efficient, scalable, and flexible framework for acquiring and processing signals in systems that evolve over time. Its ability to adapt to multi-dimensional data and time-varying processes makes it an essential tool in many modern applications, from industrial monitoring to medical imaging and beyond.

%Our work mainly focuses on reconstructing the initial signal. In signal recovery or reconstruction problems, particularly in dynamical sampling, the initial signal plays a key role. When trying to recover a full signal from sparse measurements, knowledge of the initial signal can dramatically improve the accuracy of reconstruction. This is because the initial signal often contains critical information about the structure and features of the underlying signal, which helps to "guide" the recovery process.




In this work, we explore the dynamical sampling problem where the initial signal \( \mathcal{F} \) is in \( \mathbb{C}^{m \times p \times n} \), and evolves over time driven by the t-product of the tensor \( \mathcal{A} \in \mathbb{C}^{m \times m \times n} \).  Specifically,  the signal at time \( t \) is transformed according to:
\begin{equation}\label{eqn:evolution rule}
    \mathcal{F}_t = \mathcal{A}^t \ast \mathcal{F},
\end{equation}
where \( \ast \) denotes the t-product between two tensors \cite{kilmer2011factorization}, and \( \mathcal{A}^t \) represents the \( t \)-th power of \( \mathcal{A} \) under the t-product. We consider the spatio-temporal sampling data represented by the set \( \Psi = \{\mathcal{F}_t(i,j,k) : (i,j,k) \in \Omega\subseteq[m]\times[p]\times[n], t \in \{0\}\cup[T-1]\} \) with $[m]=\{1,2,\cdots, m\}$. The objective of this study is to identify the conditions on \( \Omega \) and \( T \) necessary to guarantee the reconstruction of the initial signal \( \mathcal{F} \) from \( \Psi \), and to formulate the reconstruction of \( \mathcal{F}\) as an optimization problem that can be efficiently solved.
 
    \subsection{Contributions}
Our main contributions are as follows:
\begin{itemize}
    \item We have established a necessary condition on $\Omega$ for the successful recovery of the initial three-dimensional signal from the given samples.
    \item We have transformed the reconstruction of the three-dimensional signal $\mathcal{F}$ from spatio-temporal samples into $p$ independent optimization problems.
    \item We have conducted several experiments to demonstrate the effectiveness of our method, determine the optimal total sampling time 
$T$, and verify our conjecture.
\end{itemize}






\section{Preliminaries}
In this section, we introduce the mathematical notations and concepts required for our study, with a focus on the t-product and other tensor operations.
\begin{definition}[t-product] The t-product of tensors $\mathcal{T}_1\in\mathbb{C}^{m\times p\times n}$ and  $\mathcal{T}_2\in\mathbb{R}^{p\times q\times n}$ is denoted by $\mathcal{T}_1\ast\mathcal{T}_2=:\mathcal{T}\in\mathbb{C}^{m\times q\times n}$  and can be defined by the following steps:\\
    \begin{itemize}
    
\item  $\widehat {\mathcal{T}_1 }= \fft(\mathcal{T}_1,[], 3)$, $\widehat {\mathcal{T}_2 }= \fft(\mathcal{T}_2,[], 3)$ 
\item $\widehat {\mathcal{T}} (:,:,k)$=$\widehat {\mathcal{T}_1} (:,:,k)\widehat {\mathcal{T}_2} (:,:,k)$
\item $\mathcal{T}:=\mathcal{T}_1\ast \mathcal{T}_2=\ifft(\widehat{\mathcal{T}},[],3)$.
\end{itemize}
\end{definition}
Apart from t-product, we also involve other products between the tensors for the signal recovery.
\begin{definition}\label{def:others}
Other products between tensor:
\begin{itemize}
    \item  Element-wise tensor product $\odot$: $\mathcal{T}=\mathcal{T}_1\odot \mathcal{T}_2$ for $\mathcal{T}, \mathcal{T}_1, \mathcal{T}_2 \in \mathbb{C}^{m\times p\times n}$, with $[\mathcal{T}]_{i,j,k}=[\mathcal{T}_1]_{i,j,k}[\mathcal{T}_2]_{i,j,k}$.
    \item  Tube-wise circular convolution $\circledast$: $\mathcal{T}=\mathcal{T}_1\circledast \mathcal{T}_2$ for $\mathcal{T}, \mathcal{T}_1, \mathcal{T}_2 \in \mathbb{C}^{m\times p\times n}$, with $[\mathcal{T}]_{i,j,:}=[\mathcal{T}_1]_{i,j,:}\ast [\mathcal{T}_2]_{i,j,:}$.
    \item  Frontal-slice-wise product $\bigtriangleup$: $\mathcal{T}=\mathcal{T}_1\bigtriangleup \mathcal{T}_2$ for $\mathcal{T}_1 \in \mathbb{C}^{m\times n\times p}$, $\mathcal{T}_2 \in \mathbb{C}^{n\times s\times p}$, $\mathcal{T} \in \mathbb{C}^{m\times s\times p}$, with $[\mathcal{T}]_{:,:,k}=[\mathcal{T}_1]_{:,:,k}[\mathcal{T}_2]_{:,:,k}$.
\end{itemize}
\end{definition}

\section{Main results}
\subsection{Necessary condition}
We have initially focused on the sampling set \(\Omega\) structured in a lattice form, specifically \(\Omega = I \times J \times [n]\), where \(I \subseteq [m]\) and \(J \subseteq [p]\). In this configuration, we present the following result:

\begin{theorem}
   Suppose \( \mathcal{F} \in \mathbb{C}^{m \times p \times n} \) and \( \mathcal{A} \in \mathbb{C}^{m \times m \times n} \). And suppose that the signal at time \( t \) follows the transformation specified in \eqref{eqn:evolution rule}. Then, the recovery of \( \mathcal{F} \) from \( \Psi \), with \( \Omega = I \times J \times [n] \) where \( I \subseteq [m] \) and \( J \subseteq [p] \), is not possible if \( J \neq [p] \).
\end{theorem}
\begin{proof}
    Given that \(\Omega = I \times J \times [n]\), the samples at time \(t\) can be represented as:
\begin{equation}\label{eqn:samplesatt}
    \mathcal{Y}_t = [\mathcal{I}_m]_{I,:,:} \ast \mathcal{F}_t \ast [\mathcal{I}_p]_{:,J,:}
\end{equation}
where \(\mathcal{I}_m\) is the \(m \times m \times n\) identity tensor \cite{kilmer2011factorization}. Utilizing the properties of the t-product and applying the discrete Fourier transformation on the third dimension of both sides of \eqref{eqn:samplesatt}, we obtain:
\begin{equation}\label{eqn:samplesInfrequency}
    [\widehat{\mathcal{Y}}_t]_{:,:,k} = [\mathbb{I}_m]_{I,:} \widehat{\mathcal{A}}_{:,:,k}^t [\widehat{\mathcal{F}}]_{:,J,k}
\end{equation}
where $\mathbb{I}_m$ stands for the $m\times m$ identity matrix. 
Given that the Fourier transformation is a unitary transformation, reconstructing \(\mathcal{F}\) from \(\Psi\) is equivalent to reconstructing \(\widehat{\mathcal{F}}\) from \(\widehat{\Psi} = \{\widehat{\mathcal{Y}}_t : t \in 0 \cup [T-1]\}\).

From \eqref{eqn:samplesInfrequency}, it is evident that the reconstructions of \([\widehat{\mathcal{F}}]_{:,j,k}\) are independent for different \(j \in J\). This implies that if \(J \neq [p]\), there will be some \(j \in [p] \setminus J\) for which \([\widehat{\mathcal{F}}]_{:,j,k}\) cannot be reconstructed from \(\widehat{\Psi}\). Consequently, \(\widehat{\mathcal{F}}\) cannot be fully reconstructed from \(\widehat{\Psi}\). The result of this theorem is thus established.
\end{proof}
Based on this result, we propose the following conjecture, which we intend to explore in our future work:
\begin {conjecture}\label{conj1}
A necessary condition for the successful recovery of \( \mathcal{F} \) from \( \Psi \) is that \( \bigcup_{(i,j,k)\in\Omega} \{j\} = [p] \).
\end {conjecture}
Although we do not provide a formal theoretical proof for this conjecture here, we conducted simulations to test our hypothesis. The results indicate that losing any vertical index from the second dimension leads to a failure in recovery, thereby supporting the conjecture.
\subsection{Method development for signal recovery}\label{sec:alg}
We now focus on reconstructing $\mathcal{F}$ from $\Psi$. If the samples sufficiently guarantee the reconstruction of $\mathcal{F}$, then the following optimization problem will yield a unique solution:
\begin{equation}\label{eqn:DS2opt}
\min_{\mathcal{X}}\sum_{t=0}^{T-1}\sum_{(i,j,k)\in\Omega}\|[\mathcal{A}^t\ast \mathcal{X}]_{i,j,k}-[\mathcal{F}_t]_{i,j,k}\|_{\fro}^2.
\end{equation}
For clarity, we introduce the following definitions:
Let $\mathcal{P}_{\Omega}(\cdot)$ denote the projection of a tensor onto the observed set $\Omega$ such that
\[
[\mathcal{P}_{\Omega}(\mathcal{T})]_{i,j,k}=\begin{cases}
    \mathcal{T}_{i,j,k}, & \text{if } (i,j,k)\in\Omega\\
    0, & \text{otherwise}
\end{cases}.
\]
According to \Cref{def:others}, we can reformulate \eqref{eqn:DS2opt} as:
\begin{equation}\label{eqn:optProj}
\min_{\mathcal{X}}\sum_{t=0}^{T-1} \|\mathcal{P}_{\Omega}(\mathcal{A}^t\ast \mathcal{X})-\mathcal{P}_{\Omega}(\mathcal{F}_t)\|_{\fro}^2.
\end{equation}
Consider that
\begin{equation}
\begin{aligned}
\mathcal{P}_{\Omega}(\mathcal{F}_t) &= \mathcal{P}_{\Omega}\odot\mathcal{F}_t \\
&= \ifft(\widehat{\mathcal{P}_{\Omega}}\circledast\widehat{\mathcal{F}_t},[],3)/n.
\end{aligned}
\end{equation}
Drawing on the properties of the t-product and inspired by \cite{liu2019low}, we can convert the least squares minimization problem \eqref{eqn:optProj} into a frequency domain version:
 \begin{equation}\label{eqn:optfrequency}
    \min_{\widehat{\mathcal{X}}\in\mathbb{C}^{m\times p\times n}}\sum_{t=0}^{T-1} \|\widehat{\mathcal{P}_{\Omega}} \circledast(\widehat{\mathcal{A}}^t\bigtriangleup \widehat{\mathcal{X}})/n-\widehat{\mathcal{P}_{\Omega}(\mathcal{F}_t)}\|_{\fro}^2. 
 \end{equation}
This problem can be decomposed into $p$ separate subproblems, one for each $j\in[p]$, where we solve:
\begin{equation}\label{eqn:optfrequency-sub}
\min_{[\widehat{\mathcal{X}}]_{:,j,:} }\sum_{t=0}^{T-1} \|[\widehat{\mathcal{P}_{\Omega}}]_{:,j,:} \circledast(\widehat{\mathcal{A}}^t\bigtriangleup [\widehat{\mathcal{X}}]_{:,j,:})/n-[\widehat{\mathcal{P}_{\Omega}(\mathcal{F}_t)}]_{:,j,:}\|_{\fro}^2.
\end{equation}
The goal is to achieve a minimal value of zero for each subproblem. To facilitate this, we construct the following system for each $t$ and $j$:
\begin{equation}\label{eq:optfre-eq}
A_3(j)A_1(t)x(j)=b(j,t)
\end{equation}
where 
\begin{equation}
x(j)=\begin{bmatrix}[\widehat{\mathcal{X}}]_{:,j,1};\cdots;[\widehat{\mathcal{X}}]_{:,j,n}\end{bmatrix}\in\mathbb{C}^{mn\times 1},
\end{equation}
\begin{equation}
b(j,t)=\begin{bmatrix}[\widehat{\mathcal{P}}_{\Omega}(\mathcal{F}_t)]_{:,j,1};\cdots;[\widehat{\mathcal{P}}_{\Omega}(\mathcal{F}_t)]_{:,j,n}\end{bmatrix}\in\mathbb{C}^{mn\times 1},
\end{equation}
and $A_1(t)$ and $A_3(j)$ are defined as:
\begin{equation}
A_1(t)=\begin{bmatrix}
    [\widehat{\mathcal{A}}]_{:,:,1}^t&&\\
    &\ddots&\\
    &&[\widehat{\mathcal{A}}]_{:,:,n}^t
\end{bmatrix}\in\mathbb{C}^{mn\times mn},
\end{equation}
\resizebox{0.5\textwidth}{!}{$
A_3(j)=\begin{bmatrix}
    \diag([\mathcal{A}_2(j)]_{1,1,:})& \diag([\mathcal{A}_2(j)]_{1,2,:})&\cdots& \diag([\mathcal{A}_2(j)]_{1,n,:})\\
    \diag([\mathcal{A}_2(j)]_{2,1,:})& \diag([\mathcal{A}_2(j)]_{2,2,:})&\cdots& \diag([\mathcal{A}_2(j)]_{2,n,:})\\
    \vdots&\vdots&\ddots&\vdots\\
    \diag([\mathcal{A}_2(j)]_{n,1,:})& \diag([\mathcal{A}_2(j)]_{n,2,:})&\cdots& \diag([\mathcal{A}_2(j)]_{n,n,:})
\end{bmatrix}$}
with $[\mathcal{A}_2(j)]_{:,:,\ell}=\circ([\widehat{\mathcal{P}_{\Omega}}]_{\ell,j,:})$. Solving this system will enable us to recover $x(j)$. Once all $x(j)$ values are obtained, they are combined and reshaped into a tensor $\widehat{\mathcal{X}}_{\textnormal{app}}$ of size $m\times n\times p$. The estimation of $\mathcal{F}$ is then set as $\mathcal{X}=\ifft(\widehat{\mathcal{X}}_{\textnormal{app}},[],3)$.

  
\subsection{Experiments}

To evaluate the performance of our proposed method, we conducted several simulations aimed at recovering the initial signal. The experiments are structured in three parts: the first part evaluates the overall recovery performance and the point-wise recovery of our algorithm, the second part focuses on determining the optimal value of the parameter $T$, which is crucial for effective signal recovery, the third part verifies the conjecture, demonstrating that to fully recover the signal, the union of the second dimension in our dataset must equal to $p$, i.e., the second dimension of the initial signal.
\subsubsection{Datasets}
To generate the synthetic datasets, we first create a random tensor of size \(20 \times 15 \times 5\) as the initial signal \(\mathcal{F}\) and another tensor of size \(20 \times 20 \times 5\) as the driven operator \(\mathcal{A}\). Using these, we produce signals at different times \(t\) by applying the operation \(\mathcal{F}_{t} = \mathcal{A}^t * \mathcal{F}\).  
Next, we generate another tensor \(\mathcal{P}_{\Omega} \in \{0,1\}^{20 \times 15 \times 5}\) of the same size as \(\mathcal{F}\) to denote the sampled locations. The entries of \(\mathcal{P}_{\Omega}\) are generated using a Bernoulli distribution, where an entry of \(1\) indicates the presence of a sample and \(0\) indicates its absence. The probability of \(1\) in \(\mathcal{P}_{\Omega}\) is set to the sampling rate \(\alpha\). 
Finally, we extract the spatio-temporal samples by generating \(\mathcal{P}_{\Omega}(\mathcal{F}_t)\) at various time points \(t \in \{0\} \cup [T-1]\), capturing samples across different locations and times.


 
\subsubsection{Recovery accuracy}
In the first set of experiments, we evaluated recovery accuracy. We set the maximum sampling time \( T \) to 5 and repeated the experiment 10 times to assess the stability of the method. As shown in \Cref{fig:recovery performance}, when the sampling rate reached 40\%, the relative error decreased to approximately \( 10^{-12} \), demonstrating the effectiveness of our approach.
\begin{figure}[ht]
    \centering
    \includegraphics[width=0.68\linewidth]{fig/newer_2.png}
    \caption{Relative error v.s. sampling rate $\alpha$: we repeated this experiment 10 times, calculating both the mean value and the standard deviation of the results. As shown by the shadow, the standard deviation is relatively small, indicating that our method consistently performs well across different trials.}
    \label{fig:recovery performance}
\end{figure}


 
\begin{figure}[ht]
    \centering
    \includegraphics[width=0.68\linewidth]{fig/tfig2.png}
    \caption{This figure shows the  point-wise gap between  
        the reconstructed signal and the ground truth signal, there are $20\times 15\times 5=1500$ sample points in total, we compared each point from the constructed tensor with the ground truth tensor.}
    \label{fig:point-wise recovery}
\end{figure}
 It is important to note that the product of \( T \) and the sampling rate \(\alpha\) provides a measure that reflects the overall sampling size.   If this product is less than 1, recovery is likely to fail, as the sample set lacks sufficient information to reconstruct the initial signal. Additionally, as observed in \Cref{fig:point-wise recovery}, a sampling rate of 40\% enables successful recovery of all points, further underscoring the robustness of our method.
 
 
\subsubsection{Optimal maximum sampling time T}
The parameter \( T \) is a critical hyperparameter, especially when samples are affected by noise. To investigate the effect of \( T \) on recovery performance, we conducted a series of experiments, varying \( T \) from 1 to 15 while keeping the sampling rate $\alpha$ fixed at 40\%. Additive Gaussian noises with mean 0 and variance \(\sigma^2\) were applied to the samples, i.e., \(\varepsilon \sim \mathcal{N}(0, \sigma^2)\).

\begin{figure*}[th]
    \centering
    \begin{subfigure}{0.32\textwidth}
        \centering
        \includegraphics[width=\textwidth]{fig/Opttt.png} 
        \caption{$\alpha=0.2$}
    \end{subfigure}
    \begin{subfigure}{0.32\textwidth}
        \centering
        \includegraphics[width=\textwidth]{fig/npc.png} 
        \caption{$\alpha=0.5$}
    \end{subfigure}
    
    
    \caption{Relative error v.s. maximum sampling time  $T$   under different noise levels and sampling rates}
    \label{fig:Optimal T}
\end{figure*}
As shown in \Cref{fig:Optimal T}, increasing \( T \) does not always enhance recovery performance; rather, an optimal value of \( T \) exists. For \( T > 10 \), we calculate the condition number \(\kappa(j)\) of the matrix 
{\footnotesize
\[
\begin{bmatrix}
    (A_3(j)A_1(0))^\top & (A_3(j)A_1(1))^\top & \cdots & (A_3(j)A_1(T))^\top
\end{bmatrix}^\top
\]}
and define \( K = \max_{j} \kappa(j) \) as the condition number of the entire system. We observed that \( K \) became excessively large (on the order of \(10^{11}\)), which caused instability in the linear system \eqref{eq:optfre-eq} and led to a significant increase in relative error, as shown in \Cref{fig:condition number}. This indicates that selecting an appropriate value for \( T \) is crucial for achieving stable and accurate recovery.



 \begin{figure}[ht]
    \centering
     \includegraphics[width=0.6\linewidth]{fig/cond.png}
     \caption{Condition number for different T}
     \label{fig:condition number}
 \end{figure} 
 \begin{figure}[ht]
    \centering
    \includegraphics[width=0.6\linewidth]{fig/Vcoo_2.png }
    \caption{Undersampled second dimension}
    \label{fig:verify conj1}

    \vspace{0.5cm} 

    \includegraphics[width=0.6\linewidth]{fig/ffp.png} 
    \caption{Undersampled first and third dimension}
    \label{fig:verify conje}
\end{figure}

  %\begin{table}[h!]
% %\centering
% %\begin{tabular}{|c|c|c|c|c|c|c|c|c|c|c|c|c|c|}
% %\hline
%  % & T=2 & T=3 &T=4 & T=5& T=6 & T=7 & T=8 &T=9 & T=10 &T=11 & T=12 &T=13 &T=14 &T=15 \\ \hline
% %Condition number   &$2.2 \times 10^16$    & $409$  & $2638$ & $1.4 \time 10^7$ & $5.5 \times 10^8$  & $1.2 \times 10^9$ & $1.6 \times 10^10$ & $1.8 \times 10^12$ & $6.5 \times 10^12$& $8.6 \times 10^13$ & $6.8 \times 10^14$\\ \hline

% % \begin{table}[h!]
% % \centering
% % \begin{tabular}{|c|c|c|c|c|c|c|c|c|c|c|c|c|c|}
% % \hline
% %         & Column 2 & Column 3 & Column 4 & Column 5 & Column 6 & Column 7 & Column 8 & Column 9 & Column 10 & Column 11 & Column 12 & Column 13 & Column 14 \\ \hline
% % Data 1  & Data 2   & Data 3   & Data 4   & Data 5   & Data 6   & Data 7   & Data 8   & Data 9   & Data 10   & Data 11   & Data 12   & Data 13   & Data 14   \\ \hline

% % \caption{2x14 Table with Blank First Cell}
% % \label{tab:2x14_blank_first}
% % \end{tabular}
% % \end{table}





\subsubsection{Verifying the conjecture}
\label{sec 234}
In this section, we conducted simulations to evaluate \Cref{conj1}. The initial signal is a tensor of size \( 20 \times 15 \times 5 \). To test the conjecture, we first set the sampling rate to \( \alpha = 1 \), generating the sampling location set \( \Omega = \bigcup_{j=1}^{15} \Omega_{j} \), where \( \Omega_{j} \) represents the locations corresponding to the second index equal to \( j \). In each experiment, one \( \Omega_{j} \) was excluded by varying \( j \) from 1 to 15, resulting in the sampling location set \( \Omega^{j} = \Omega \setminus \Omega_{j} \). This configuration led to a sampling rate of 93\%. However, as shown in \Cref{fig:verify conj1}, the relative error remained above 0.2, indicating that the initial signal could not be fully recovered.

To further explore this, we generated a new sampling location set with \( \alpha = 0.5 \), and similarly adjusted the set by excluding all locations where the first (or third) index equals \( j \), with \( j \) varying from 1 to 20 (or from 1 to 5 for the third index). As shown in \Cref{fig:verify conje}, excluding locations where the first or third index takes these values did not affect the recovery, and the relative error reduced to \( 10^{-11} \).

\small




\bibliographystyle{plain}%{plainnat}
\bibliography{ref}
\end{document}

Also, while the DETR self-attention mechanism effectively captures complex dependencies between tokens, it does not inherently alter the sequence order.
As illustrated in the attention maps of Figure \ref{fig:activate region}, both convolutional features and those derived from self-attention maintain a strong indication of spatial locations, with active regions consistently focusing on the foreground.
\begin{figure}[t]
\begin{center}
    \begin{subfigure}[b]{0.31\linewidth}
    \includegraphics[width=\textwidth]{sec/Images/ground_truth_image_1.jpg}
    \end{subfigure}
    \begin{subfigure}[b]{0.31\linewidth}
    \includegraphics[width=\textwidth]{sec/Images/ATSS_image_1.jpeg}
    \end{subfigure}
    \begin{subfigure}[b]{0.31\linewidth}
    \includegraphics[width=\textwidth]{sec/Images/DETR_image_1.jpeg}
    \end{subfigure}

    \begin{subfigure}[b]{0.31\linewidth}
        \includegraphics[width=\textwidth]{sec/Images/ground_truth_image_2.png}
        \caption{Image w/ bboxs}\label{fig:ori}
    \end{subfigure}
    \begin{subfigure}[b]{0.31\linewidth}
        \includegraphics[width=\textwidth]{sec/Images/ATSS_image_2.jpg}
        \caption{CNN-based}\label{fig:atss}
    \end{subfigure}
    \begin{subfigure}[b]{0.31\linewidth}
        \includegraphics[width=\textwidth]{sec/Images/DETR_image_2.jpg}
        \caption{self-attention}\label{fig:detr}
    \end{subfigure}
\end{center}
\caption{
Active regions of CNN-based and DETR-based detectors: (a) Image w/ bounding boxes, (b) ATSS \cite{zhang2020bridging} and (c) Deformable DETR \cite{zhu2020deformable}.}
\label{fig:activate region}
\end{figure}
Motivated by this observation, we propose transforming the ground truth location information to emphasize relevant, contextually-enriched features in the flattened DETR memory during distillation.
%and mitigates inheriting their errors. 
%Notably, we are the first to utilize location-and-context-aware memory for distillation.

\paragraph{Issues in distilling DETR logits}
DETR logit-level distillation faces stability issues due to mismatched distillation points between the teacher and student models. Unlike convolution-based detectors with a fixed spatial arrangement of anchors, DETR's unordered box predictions lack a natural one-to-one correspondence, complicating the alignment between the teacher and student outputs.
\citet{chang2023detrdistill} and \citet{ijcai2024p74} rely on Hungarian matching for teacher-student logit alignment, but this approach results in many meaningless/background predictions being matched unstably.
While \citet{wang2024knowledge} propose consistent distillation points to guide the process, these points are generated randomly or solely based on the fallible teacher model, ignoring the precise location information readily available from the ground truth during distillation. To overcome these challenges, we propose leveraging both category and location information from the ground truth to generate consistent target-aware queries. These queries are designed to be unlearnable and are fed into both the student and teacher models, ensuring that the distillation process precisely and consistently identifies the most informative (and global-context-aware) feature areas for effective knowledge transfer.

In summary, to address the unsolved challenges in DETR distillation at both the feature and logit levels, we propose Consistent Location-and-Context-aware Knowledge Distillation (CLoCKDistill). Our method takes advantage of transformer-specific global context and incorporates ground truth information into both the encoder memory and decoder query for consistent and location-informed distillation. Our contributions are threefold:

\begin{itemize}
    \item We propose global-context-aware feature distillation by utilizing the contextually rich representations in the DETR memory (encoder output). Unlike CNN-generated backbone features, DETR memory is more refined and captures long-range dependencies through self-attention.
    \item Furthermore, we take advantage of the implicit token ordering in DETR memory and utilize ground truth location data to direct feature distillation attention toward more relevant features.
    \item Finally, for logit distillation, we design target-aware decoder queries. These ground-truth-informed queries provide consistent and precise spatial guidance to both the teacher and student models on where to focus in memory when generating logits for distillation.
\end{itemize}










% Please follow the steps outlined below when submitting your manuscript to the IEEE Computer Society Press.
% This style guide now has several important modifications (for example, you are no longer warned against the use of sticky tape to attach your artwork to the paper), so all authors should read this new version.

% %-------------------------------------------------------------------------
% \subsection{Language}

% All manuscripts must be in English.

% \subsection{Dual submission}

% Please refer to the author guidelines on the \confName\ \confYear\ web page for a
% discussion of the policy on dual submissions.

% \subsection{Paper length}
% Papers, excluding the references section, must be no longer than eight pages in length.
% The references section will not be included in the page count, and there is no limit on the length of the references section.
% For example, a paper of eight pages with two pages of references would have a total length of 10 pages.
% {\bf There will be no extra page charges for \confName\ \confYear.}

% Overlength papers will simply not be reviewed.
% This includes papers where the margins and formatting are deemed to have been significantly altered from those laid down by this style guide.
% Note that this \LaTeX\ guide already sets figure captions and references in a smaller font.
% The reason such papers will not be reviewed is that there is no provision for supervised revisions of manuscripts.
% The reviewing process cannot determine the suitability of the paper for presentation in eight pages if it is reviewed in eleven.

% %-------------------------------------------------------------------------
% \subsection{The ruler}
% The \LaTeX\ style defines a printed ruler which should be present in the version submitted for review.
% The ruler is provided in order that reviewers may comment on particular lines in the paper without circumlocution.
% If you are preparing a document using a non-\LaTeX\ document preparation system, please arrange for an equivalent ruler to appear on the final output pages.
% The presence or absence of the ruler should not change the appearance of any other content on the page.
% The camera-ready copy should not contain a ruler.
% (\LaTeX\ users may use options of \texttt{cvpr.sty} to switch between different versions.)

% Reviewers:
% note that the ruler measurements do not align well with lines in the paper --- this turns out to be very difficult to do well when the paper contains many figures and equations, and, when done, looks ugly.
% Just use fractional references (\eg, this line is $087.5$), although in most cases one would expect that the approximate location will be adequate.


% \subsection{Paper ID}
% Make sure that the Paper ID from the submission system is visible in the version submitted for review (replacing the ``*****'' you see in this document).
% If you are using the \LaTeX\ template, \textbf{make sure to update paper ID in the appropriate place in the tex file}.


% \subsection{Mathematics}

% Please number all of your sections and displayed equations as in these examples:
% \begin{equation}
%   E = m\cdot c^2
%   \label{eq:important}
% \end{equation}
% and
% \begin{equation}
%   v = a\cdot t.
%   \label{eq:also-important}
% \end{equation}
% It is important for readers to be able to refer to any particular equation.
% Just because you did not refer to it in the text does not mean some future reader might not need to refer to it.
% It is cumbersome to have to use circumlocutions like ``the equation second from the top of page 3 column 1''.
% (Note that the ruler will not be present in the final copy, so is not an alternative to equation numbers).
% All authors will benefit from reading Mermin's description of how to write mathematics:
% \url{http://www.pamitc.org/documents/mermin.pdf}.

% \subsection{Blind review}

% Many authors misunderstand the concept of anonymizing for blind review.
% Blind review does not mean that one must remove citations to one's own work---in fact it is often impossible to review a paper unless the previous citations are known and available.

% Blind review means that you do not use the words ``my'' or ``our'' when citing previous work.
% That is all.
% (But see below for tech reports.)

% Saying ``this builds on the work of Lucy Smith [1]'' does not say that you are Lucy Smith;
% it says that you are building on her work.
% If you are Smith and Jones, do not say ``as we show in [7]'', say ``as Smith and Jones show in [7]'' and at the end of the paper, include reference 7 as you would any other cited work.

% An example of a bad paper just asking to be rejected:
% \begin{quote}
% \begin{center}
%     An analysis of the frobnicatable foo filter.
% \end{center}

%    In this paper we present a performance analysis of our previous paper [1], and show it to be inferior to all previously known methods.
%    Why the previous paper was accepted without this analysis is beyond me.

%    [1] Removed for blind review
% \end{quote}


% An example of an acceptable paper:
% \begin{quote}
% \begin{center}
%      An analysis of the frobnicatable foo filter.
% \end{center}

%    In this paper we present a performance analysis of the  paper of Smith \etal [1], and show it to be inferior to all previously known methods.
%    Why the previous paper was accepted without this analysis is beyond me.

%    [1] Smith, L and Jones, C. ``The frobnicatable foo filter, a fundamental contribution to human knowledge''. Nature 381(12), 1-213.
% \end{quote}

% If you are making a submission to another conference at the same time, which covers similar or overlapping material, you may need to refer to that submission in order to explain the differences, just as you would if you had previously published related work.
% In such cases, include the anonymized parallel submission~\cite{Authors14} as supplemental material and cite it as
% \begin{quote}
% [1] Authors. ``The frobnicatable foo filter'', F\&G 2014 Submission ID 324, Supplied as supplemental material {\tt fg324.pdf}.
% \end{quote}

% Finally, you may feel you need to tell the reader that more details can be found elsewhere, and refer them to a technical report.
% For conference submissions, the paper must stand on its own, and not {\em require} the reviewer to go to a tech report for further details.
% Thus, you may say in the body of the paper ``further details may be found in~\cite{Authors14b}''.
% Then submit the tech report as supplemental material.
% Again, you may not assume the reviewers will read this material.

% Sometimes your paper is about a problem which you tested using a tool that is widely known to be restricted to a single institution.
% For example, let's say it's 1969, you have solved a key problem on the Apollo lander, and you believe that the 1970 audience would like to hear about your
% solution.
% The work is a development of your celebrated 1968 paper entitled ``Zero-g frobnication: How being the only people in the world with access to the Apollo lander source code makes us a wow at parties'', by Zeus \etal.

% You can handle this paper like any other.
% Do not write ``We show how to improve our previous work [Anonymous, 1968].
% This time we tested the algorithm on a lunar lander [name of lander removed for blind review]''.
% That would be silly, and would immediately identify the authors.
% Instead write the following:
% \begin{quotation}
% \noindent
%    We describe a system for zero-g frobnication.
%    This system is new because it handles the following cases:
%    A, B.  Previous systems [Zeus et al. 1968] did not  handle case B properly.
%    Ours handles it by including a foo term in the bar integral.

%    ...

%    The proposed system was integrated with the Apollo lunar lander, and went all the way to the moon, don't you know.
%    It displayed the following behaviours, which show how well we solved cases A and B: ...
% \end{quotation}
% As you can see, the above text follows standard scientific convention, reads better than the first version, and does not explicitly name you as the authors.
% A reviewer might think it likely that the new paper was written by Zeus \etal, but cannot make any decision based on that guess.
% He or she would have to be sure that no other authors could have been contracted to solve problem B.
% \medskip

% \noindent
% FAQ\medskip\\
% {\bf Q:} Are acknowledgements OK?\\
% {\bf A:} No.  Leave them for the final copy.\medskip\\
% {\bf Q:} How do I cite my results reported in open challenges?
% {\bf A:} To conform with the double-blind review policy, you can report results of other challenge participants together with your results in your paper.
% For your results, however, you should not identify yourself and should not mention your participation in the challenge.
% Instead present your results referring to the method proposed in your paper and draw conclusions based on the experimental comparison to other results.\medskip\\

% \begin{figure}[t]
%   \centering
%   \fbox{\rule{0pt}{2in} \rule{0.9\linewidth}{0pt}}
%    %\includegraphics[width=0.8\linewidth]{egfigure.eps}

%    \caption{Example of caption.
%    It is set in Roman so that mathematics (always set in Roman: $B \sin A = A \sin B$) may be included without an ugly clash.}
%    \label{fig:onecol}
% \end{figure}

% \subsection{Miscellaneous}

% \noindent
% Compare the following:\\
% \begin{tabular}{ll}
%  \verb'$conf_a$' &  $conf_a$ \\
%  \verb'$\mathit{conf}_a$' & $\mathit{conf}_a$
% \end{tabular}\\
% See The \TeX book, p165.

% The space after \eg, meaning ``for example'', should not be a sentence-ending space.
% So \eg is correct, {\em e.g.} is not.
% The provided \verb'\eg' macro takes care of this.

% When citing a multi-author paper, you may save space by using ``et alia'', shortened to ``\etal'' (not ``{\em et.\ al.}'' as ``{\em et}'' is a complete word).
% If you use the \verb'\etal' macro provided, then you need not worry about double periods when used at the end of a sentence as in Alpher \etal.
% However, use it only when there are three or more authors.
% Thus, the following is correct:
%    ``Frobnication has been trendy lately.
%    It was introduced by Alpher~\cite{Alpher02}, and subsequently developed by
%    Alpher and Fotheringham-Smythe~\cite{Alpher03}, and Alpher \etal~\cite{Alpher04}.''

% This is incorrect: ``... subsequently developed by Alpher \etal~\cite{Alpher03} ...'' because reference~\cite{Alpher03} has just two authors.

% \begin{figure*}
%   \centering
%   \begin{subfigure}{0.68\linewidth}
%     \fbox{\rule{0pt}{2in} \rule{.9\linewidth}{0pt}}
%     \caption{An example of a subfigure.}
%     \label{fig:short-a}
%   \end{subfigure}
%   \hfill
%   \begin{subfigure}{0.28\linewidth}
%     \fbox{\rule{0pt}{2in} \rule{.9\linewidth}{0pt}}
%     \caption{Another example of a subfigure.}
%     \label{fig:short-b}
%   \end{subfigure}
%   \caption{Example of a short caption, which should be centered.}
%   \label{fig:short}
% \end{figure*}
