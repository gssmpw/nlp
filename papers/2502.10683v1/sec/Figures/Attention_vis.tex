% \begin{figure*}[h]
% \begin{center}
%     \begin{subfigure}[b]{0.19\textwidth}
%     \includegraphics[]{AnonymousSubmission/Images/000000000154_gt.jpg}
%     \caption{(a)}
%     \end{subfigure}
%     \begin{subfigure}[b]{0.19\textwidth}
%     \includegraphics[]{AnonymousSubmission/Images/teacher_000000000154.jpg}
%     \caption{(b)}
%     \end{subfigure}
% \includegraphics[width=0.19\linewidth]{AnonymousSubmission/Images/student_000000000154.jpg}
% \includegraphics[width=0.19\linewidth]{AnonymousSubmission/Images/kddetr_000000000154.jpg}
% \includegraphics[width=0.19\linewidth]{AnonymousSubmission/Images/ours_000000000154.jpg}
% \end{center}
% \caption{
% Attention maps from (a) Image w/ bounding boxes, (b) the teacher model, (c) the student model (d) KD-DETR distilled \cite{wang2022distilling} (e) our distilled model. Our distilled model allocates more focus to relevant objects, where red denotes the highest attention and blue the lowest.}
% \label{fig:activate vis}
% \end{figure*}
\begin{figure*}
\begin{center}
    \begin{subfigure}[b]{0.19\textwidth}
        \includegraphics[width=\textwidth]{sec/Images/000000000154.jpg}
        \caption{Original Image}\label{fig:ori_zebra}
    \end{subfigure}
    \begin{subfigure}[b]{0.19\textwidth}
        \includegraphics[width=\textwidth]{sec/Images/teacher_000000000154.jpg}
        \caption{Teacher}\label{fig:teacherattn_zebra}
    \end{subfigure}
    \begin{subfigure}[b]{0.19\textwidth}
        \includegraphics[width=\textwidth]{sec/Images/student_000000000154.jpg}
        \caption{Student baseline}\label{fig:studentattn_zebra}
    \end{subfigure}
    \begin{subfigure}[b]{0.19\textwidth}
        \includegraphics[width=\textwidth]{sec/Images/kddetr_000000000154.jpg}
        \caption{KD-DETR}\label{fig:kddetrattn_zebra}
    \end{subfigure}
    \begin{subfigure}[b]{0.19\textwidth}
        \includegraphics[width=\textwidth]{sec/Images/ours_000000000154.jpg}
        \caption{Ours}\label{fig:ourattn_zebra}
    \end{subfigure}

    % \begin{subfigure}[b]{0.19\textwidth}
    %     \includegraphics[width=\textwidth]{sec/Images/000000000143.jpg}
    %     \caption{Original Image}\label{fig:ori_birds}
    % \end{subfigure}
    % \begin{subfigure}[b]{0.19\textwidth}
    %     \includegraphics[width=\textwidth]{sec/Images/teacher_000000000143.jpg}
    %     \caption{Teacher}\label{fig:teacherattn_birds}
    % \end{subfigure}
    % \begin{subfigure}[b]{0.19\textwidth}
    %     \includegraphics[width=\textwidth]{sec/Images/student_000000000143.jpg}
    %     \caption{Student baseline}\label{fig:studentattn_birds}
    % \end{subfigure}
    % \begin{subfigure}[b]{0.19\textwidth}
    %     \includegraphics[width=\textwidth]{sec/Images/kddetr_000000000143.jpg}
    %     \caption{KD-DETR}\label{fig:kddetrattn_birds}
    % \end{subfigure}
    % \begin{subfigure}[b]{0.19\textwidth}
    %     \includegraphics[width=\textwidth]{sec/Images/ours_000000000143.jpg}
    %     \caption{Ours}\label{fig:ourattn_birds}
    % \end{subfigure}
    
\end{center}
\caption{Attention maps of (a) an example COCO image from the (b) teacher, (c) student baseline, (d) KD-DETR distilled \cite{wang2024knowledge}, and (e) our distilled models. Our distilled model allocates more focus to relevant objects and exhibits more clearly defined boundaries. Red indicates the highest attention and blue the lowest.}
\label{fig:activate vis}
\end{figure*}
