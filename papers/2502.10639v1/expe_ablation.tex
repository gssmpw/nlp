\vspace*{-5mm}
\subsection{Fast search with minimized space overhead}
\label{sect:eval0space}

\begin{table*}[htbp]
	\centering
		\resizebox{1.9\columnwidth}{!}{
		\begin{tabular}{ r r l |l l ll l |llll l|r}
			\hline\hline
			 & & & \multicolumn{2}{c}{ 
{\bf MSMARCO Dev} }& {\bf DL19}& {\bf DL20} & {\bf BEIR} 
& \multicolumn{2}{c} { {\bf MSMARCO Dev} }& {\bf DL19}& {\bf DL20} & {\bf BEIR} 
& {\bf  Latency}    \\
			& &\% D& {MRR@10}& R@1k& {NDCG@10}&  {NDCG@10}& {NDCG@10}& {MRR@10}& {R@1k}& {NDCG@10}&{NDCG@10}& {NDCG@10}& ms \\
    \hline
 1. & S={\bf SPLADE-HT1} & --& 0.396& 0.980 & 0.732   & 0.721    & 0.500 & -- & --& --& --& --&31.2  \\
 & & & \multicolumn{5}{c|}{D=\textbf{SimLM}}& \multicolumn{5}{c|}{D=\textbf{RetroMAE}}&\\
   \hline
   &&& \multicolumn{10}{c}{\bf{Uncompressed flat setting. Embedding space  27.2GB}}\\
     \hline
          2. &   D  & 100 & 0.411$^\dag$ & 0.985$^\dag$& 0.714 &  0.697  & 0.429 & 0.416$^\dag$& 0.988& 0.720&0.703& 0.482 & 1674.1 \\
            3. &   $\blacktriangle$ S $+$ D   &100 & 0.424 & 0.989 & 0.740 & 0.726  & 0.518 & 0.425& 0.988& 0.740&0731& 0.520~& $+$1674.1 \\
           4. &    \bf S $+$ CluSD &0.3& 0.425 & 0.987 & 0.744 & 0.724  & 0.516 & 0.426& 0.987& 0.744&0.734 & 0.518 & $+$13.2 \\
              \hline
            &&& \multicolumn{10}{c}{\bf{OPQ $m=128$. Embedding space  1.1GB}}\\
              \hline
            5. &   $\blacktriangle$  S $+$ D-OPQ &100 & 0.420 & 0.988 & 0.735 &  0.719  & 0.508 & 0.416 & 0.988 & 0.737 & 0.732 & 0.515& $+$568.9 \\
            6. &  S $+$ D-IVFOPQ &10& 0.414$^\dag$& 0.987 & 0.726 & 0.734  & 0.505 & 0.404$^\dag$& 0.987& 0.713&0.722& 0.513 &  $+$95.2 \\
           7. &   &5& 0.407$^\dag$& 0.987 & 0.717 & 0.733  & 0.499  & 0.394$^\dag$& 0.987& 0.687&0.817&0.507&$+$ 48.8 \\
           8. &   &2& 0.392$^\dag$& 0.986$^\dag$& 0.691 & 0.738  & 0.467 & 0.374$^\dag$& 0.986$^\dag$& 0.656&0.815& 0.499&$+$21.3 \\
           9. &   \textbf{S $+$ CluSD}  &0.3& 0.423 & 0.987 & 0.739& 0.725 &0.514 & 0.417& 0.986& 0.742&0.735& 0.514  & $+$11.4  \\
             \hline
           & & & \multicolumn{10}{c}{\bf{OPQ $m=64$. Embedding space 0.6GB}}\\
            \hline
           10. & $\blacktriangle$  S $+$ D-OPQ  &100 & 0.409 & 0.986 & 0.718 &  0.721  &0.501 & 0.402 & 0.986 & 0.717 &  0.719 & 0.508 &$+$ 290.4 \\
           11. &   S $+$ D-IVFOPQ &10& 0.405 & 0.986 & 0.716 & 0.734 & 0.502 & 0.393$^\dag$& 0.986& 0.676&0.730& 0.505& $+$44.4 \\
           12. &   &5& 0.397$^\dag$& 0.986 & 0.704 & 0.733  & 0.497 & 0.384$^\dag$& 0.985& 0.659&0.717& 0.500& $+$23.8 \\ 
            13. &  &2& 0.383$^\dag$& 0.985 & 0.680 & 0.738 & 0.487 & 0.368$^\dag$& 0.985 & 0.643&0.707& 0.493 & $+$11.2 \\
            14. &  \textbf{S $+$ CluSD}  &0.3& 0.414 & 0.986 & 0.743& 0.726 & 0.511 & 0.403 & 0.987& 0.729&0.724 & 0.506 & $+$9.6 \\
              \hline\hline
		\end{tabular}
		}
	\caption{
Sparse retrieval argumentation with minimum extra space overhead.
%no  with compressed or uncompressed dense emebddings S=SPLADE-HT1, D=SimLM or RetroMAE. 
%Model relevance and latency under compressed or uncompressed dense emebddings S=SPLADE-HT1, D=SimLM or RetroMAE. 
%Row  ``S $+$ D'' marked with $\blacktriangle$ indicates the oracle setting of sparse model interpolated with full dense retrieval. 
For MSMARCO Dev set, $^\dag$ is tagged when statistically significant drop is observed from the oracle $\blacktriangle$ at 5\% confidence level. 
}
\vspace*{-5mm}
	\label{tab:mainspace}
\end{table*}
	%\caption{Model relevance and latency under different configurations with storage constraints. S=SPLADE-HT1, D=SimLM or RetroMAE. The row  ``S $+$ D'' marked with $\blacktriangle$ indicates the oracle setting of sparse model interpolated with full flat dense search model. For MSMARCO Dev set, $^\dag$ is marked when statistically significant drop is observed from the oracle $\blacktriangle$ at 5\% confidence level.



Table~\ref{tab:mainspace} demonstrates  the effectiveness  of CluSD in searching MS MARCO and BEIR datasets under
two quantization  configurations and the uncompressed setting. 
IVFOPQ is compared against CluSD as it selects only top clusters and does not incur extra space overhead. 
%is compared, which use some percentage of top dense clusters sorted by query to centroid clusters.
The sparse model used is SPLADE-HT1 while the dense models are SimLM and RetroMAE.
%The top portion of this table reports the performance when compression is not used, which captures a peak possible performance.
Column marked with  ``\%D'' means the percentage of dense embeddings  evaluated. 
Last column on ``Latency'' is the average single-query time using SimLM for MS MARCO Dev set.  We only report SimLM latency because the latency of SimLM and RetroMAE are similar. 
When 	prefix `+' is marked for a latency entry, it indicates  the extra milliseconds is spent 
for full or selective dense retrieval without counting   sparse retrieval time.

Tables~\ref{tab:mainspace} and~\ref{tab:maintime} use the same CluSD setting with $n=32$, which results in  selection of
22.3 clusters on average per query. 
The mean latency and 99 percentile latency of CluSD are 13.2 ms and 18.1 ms for searching uncompressed flat index, respectively; 
11.4 ms and 14.6ms for searching OPQ m=128 index;  9.6 ms and 11.1ms for searching OPQ m=128 index.

\comments{
%In realistic settings, a flat dense index is usually not affordable when corpus size is not small. Certain compression technique need to be used. 
The top portion of this table reports the performance of  SimLM dense model marked as ``D'' in Row 2.
In Row 1, ``S'' denotes sparse retrieval using  a variation of SPLADE with hard thresholding called HT1~\cite{2023SIGIR-Qiao}.
Sparse results merged with full  dense retrieval  results  are marked in Row 3, ``S $+$ D''.  
Our pipeline is reported in Row 4,``S $+$ CluSD''. 
The second and third portions of this table reports
the search performance   with  two compression configurations using OPQ implemented in FAISS:
1) ``m128'' that compresses dense embeddings from about 27GB into 1.1GB with 128 quantization codebooks.
2) ``m64'' that  yields 0.6GB compressed embeddings with 64 codebooks. 
%Setting where the storage for dense index is 1.1GB and 0.6GB. 
Under these two configurations, we report the Sparse model interpolated with dense retrieval under
IVFOPQ or OPQ using the default FAISS setting.


%In the first section of the table, we include the full dense retrieval SimLM (denoted as D in the table), 
%and full sparse retrieval SPLADE-HT1 (denoted as S in the table), 
%and 
%We then report dense retrieval with two compression configurations, using OPQ 
%with m=128 and m=64 respectively, resulting in an index 
%with 1.1GB and 0.6GB. These compressed configuration result in a 96\% and 97.8\% reduction in storage compared to flat index.
% and compare them against our pipeline.
}
 This table shows that
selective dense  to augment sparse retrieval by top IVF clusters sorted by query-centroid distances 
can be fast below  100ms  when the percentage of clusters searched  drops from 100\% to 10\%. But the relevance drop is significant. 
Moreover, CluSD is as  fast  as IVFOPQ  top 2\% search at m=64 and 2x fast at m=128 while
its relevance is much higher.
CluSD  performs  closely as the oracle at each compression setting, which  
indicates CluSD effectively minimizes visitation of unnecessary  dense clusters. 

 



\comments{
 with full dense retrieval can reach a high relevance in all datasets and thus are marked as an oracle sign with or without compression.   
1. Almost no performance degradation when do S+CluSD.  Sometimes S+CluSD achieves slightly higher performance than S+D. It could be due to the fact that when we limit the search scope a lot of noisy documents are not considered by the dense model.

2. Under m=128 setting the performance degradation is acceptable especially on DL19 and 20. 

3. SimLM performs better under compression. This could be due to the fact that SimLM trained embedding is normalized while RetroMAE is not.

4. If we compare S+D-IVFOPQ with S+CluSD, given the same IVFPQ index, S+D-IVFOPQ represents the case where we use default order, even if we use 50x more clusters, the performance is not as good as CluSD. 
}


% table with skipping, 40ms for dense
\comment{
\begin{table}[htbp]
	\centering
		\resizebox{1.05\columnwidth}{!}{
		\begin{tabular}{ r l l l |r r  |r r  |r}
			\hline\hline
			 &\% & \multicolumn{2}{c|}{{MSMARCO Dev}}& \multicolumn{2}{c|}{DL19}& 
\multicolumn{2}{c|}{DL20}& Latency    \\
			 &n& {MRR@10}& {R@1k}& {N@10}& {R@1k}&  {N@10}& {R@1k}& ms \\
			\hline
            D  &--& 0.411$^\dag$ & 0.985 & 0.714 & 0.767 &  0.697 & 0.773 & 1674.1  \\
          S  &--& 0.396$^\dag$ & 0.980 & 0.732   &  0.811 & 0.721    &  0.815  &   31.2  \\
              $\blacktriangle$ S $+$ D   &--& 0.424 & 0.989 & 0.740 & 0.822 & 0.726 & 0.824 & 1705.0 \\
              \bf S $+$ CluSD  &0.2& 0.425 & 0.987 & 0.744 & 0.823 & 0.724 & 0.819 & 14.4 \\
              \hline
            \multicolumn{9}{c}{\bf{OPQ M=128 setting, Storage 1.1GB}}\\
            \hline
             S $+$ D-IVFOPQ&10& 0.414 & 0.987 & 0.726 & 0.820 & 0.734 & 0.818 & 95.2 \\
             &5& 0.407 & 0.987 & 0.717 & 0.820 & 0.733 & 0.817 & 48.8 \\
             &2& 0.392 & 0.869 & 0.691 & 0.815 & 0.738 & 0.815 & 21.3 \\
             S $+$ OPQ &--& 0.420 & 0.988 & 0.735 & 0.816 & 0.719 & 0.821 &  568.9 \\
             HNSW+OPQ && &  &  & & & &   \\
             \textbf{S $+$ CluSD}  &0.2& 0.422 & 0.986 & 0.739& 0.822& 0.725& 0.818& 14.2  \\
             \hline
          \multicolumn{9}{c}{\bf{OPQ M=64 setting, 0.6GB}}\\
          \hline
             S $+$ IVFOPQ&10& 0.405 & 0.986 & 0.716 & 0.814 & 0.734 & 0.818 & 44.4 \\
             &5& 0.397 & 0.986 & 0.704 & 0.814 & 0.733 & 0.817 & 23.8 \\ 
             &2& 0.383 & 0.985 & 0.680 & 0.813 & 0.738 & 0.815 & 11.2 \\
             S $+$ OPQ  &--& 0.409 & 0.986 & 0.718 & 0.815 & 0.721 & 0.821 & 290.4 \\
             HNSW+OPQ && &  &  & & & &   \\
             \textbf{S $+$ CluSD}  &0.2& 0.411 & 0.985 & 0.743& 0.823& 0.726& 0.816& 11.5 \\
              \hline\hline
		\end{tabular}
		}
	\caption{Model relevance and latency under different configurations with storage constraints. S=SPLADE-HT1, D=SimLM. The row  ``S $+$ D'' marked with $\blacktriangle$ indicates the oracle setting of sparse model interpolated with full flat dense search model. For MSMARCO Dev set, $^\dag$ is marked when statistically significant drop is observed from the oracle $\blacktriangle$ at 5\% confidence level.
}
	\label{tab:mainspace}
\end{table}

}



\subsection{Search with no memory space constraints}
\label{sect:evaltimebudget}
Table~\ref{tab:maintime} compares  CluSD with  several baselines under two time budgets when memory space restriction is removed.
Uncompressed dense embeddings  are used to exclude the relevance impact by quantization.
The last two columns report the total latency time in milliseconds including sparse retrieval when needed,
and data space cost. Extra memory space on the top of sparse retrieval is marked with prefix `+'. 
``MRR'' means  MRR@10 and ``NDCG'' means  NDCG@10.
The extra space for quantized inter-cluster distances for CluSD takes about 32MB. 

%As discussed in Section~\ref{sect:design}, the overall latency budget is set as 50ms and 25ms respectively. 
We allow some time for budget relaxation since it is difficult to tune parameters of all baselines to meet the constraints exactly. 
For IVF, we use 1.5$\%$ and 3$\%$ top clusters for ``S+IVF'' and ``IVF''.
%  but we allow some variation for some baselines because it is difficult to find parameters that exactly control the latency under this budget. 
%In this case, we choose the dense configuration such that the overall latency is within 70 or 30 milliseconds. 
For HNSW under 50ms time budget, we set its expansion factor parameter (ef) as 1024, 
2048 for S$+$D-HNSW and D-HNSW. 
Under 30ms time constraint, the expansion factor (ef) is  512 and  1024 respectively. 
%We include option ``S+Re-rank" which uses the top 1,000 sparse retrieval results and only compute dense scores of these top results, and merge the score with an interpolation.
%For HNSW, we fix the number of neighbors to be 128 and vary the expansion factor(ef) to meet the latency constraint.
For LADR, we use its default setting with seed = 200, number of neighbor = 128 and use the exploration depth as  50 to meet the latency budget.
%Table~\ref{tab:maintime}.
\comments{
Under time budget 50ms, 
model  SPLADE-HT1 is used and it has a latency of 31.2 ms in sparse retrieval.  
Under  time budget 25ms, model SPLADE-effi-HT3 is used and it is more aggressive in  efficiency 
with a 12.4ms latency at the cost of a relevance degradation. 

%using the dense model only, We also include sparse retriever 
%IVFOPQ, IVF, OPQ and HNSW using the dense model only, We also include sparse retriever 
%with reranking top 1000 results using the dense model. We also include the results of document evaluation using LADR algorithm. 
Note that when ``S$+$'' is in the row, it means we use the sparse model and the reported results are interpolated. 

}
%[DISCUSS TABLE 4 RESULTS.]
The takeaways from this table are:

\textbullet\ While IVF does not add extra space cost,
the relevance  of IVF top cluster search under the time budget performs  worse than others.
Simple re-ranking ``D-rerank'' achieves good MRR@10 but under-performs on recall.
When the recall of a sparse retriever is relatively low, it cannot boost further as seen in the case of $m=64$.

\textbullet\ HNSW and LADR perform comparably, at the cost of extra online memory overhead  and extra complexity to compute offline in advance.
HNSW requires extra 9.2GB and LADR requires extra 4.3GB to store graph data in memory. 
When we  try to reduce their graph size by decreasing  the number of neighbors for HNSW and LADR, their  MRR@10 and/or  recall@1000 numbers drop visibly. For example, 
using the default 32 neighbors for the HNSW graph results in 2.4GB graph space, however  MRR@10 drops from 0.407 to 0.4 under the 25ms time budget. 

\textbullet\ CluSD incurs negligible extra space cost other than hosting embeddings, compared to HNSW and LADR.
For $m=128$,  CluSD has a higher MRR@10 while its DL19 and DL20 numbers are higher than HNSW, but comparable as LADR.
CluSD has more  advantages in relevance when $m=64$. 

\textbullet\ The time budget setting is discussed in Section~\ref{sect:design}.
Even when we relax the latency requirements for the baselines, CluSD still maintains its advantages in different datasets. 
For instance, if we double the time budget to around  110ms
for 	``S+D-HNSW'' after  SPLADE-HT1 retrieval, its MRR@10 for MS MARCO Dev is 0.422, which is still below S$+$CluSD with 46.3ms.




 

\begin{table}[htbp]
	\centering
 \resizebox{0.9\columnwidth}{!}{
		\begin{tabular}{r |llll|rr}
			\hline\hline
			& \multicolumn{2}{|c}{\bf{MSMARCO Dev}}& \bf{DL19}& 
\bf{DL20}& \bf{Latency}& \bf{Space}\\
			 & {MRR}& {R@1k}& \smaller{NDCG} &   \smaller{NDCG} & Total& GB\\
			\hline
        \multicolumn{7}{c}{\bf{Time Budget = 50 ms, S=SPLADE-HT1}}\\
        \hline
            D=SimLM & 0.411 & 0.985$^\dag$& 0.714 &  0.697 & 1674.1 & 27.2 \\
          S& 0.396$^\dag$& 0.980$^\dag$& 0.732   & 0.721    &   31.2 & 3.9 \\
           S $+$ D-rerank &  0.421 & 0.980$^\dag$& 0.745 & 0.728 & 34.6 & 31.1 \\
             $\blacktriangle$  S $+$ D  & 0.424 & 0.989 & 0.740 & 0.726 & 1705.0 & 31.1 \\
              \hline
              % D-IVF & 0.398$^\dag$& 0.935$^\dag$& 0.690 & 0.693 & 81.4 & $+$0.0\\
            D-IVF & 0.389$^\dag$& 0.907$^\dag$& 0.671 & 0.676 & 49.5 & $+\sim$0\\
            S $+$ D-IVF & 0.393$^\dag$& 0.987  & 0.678 & 0.715 & 56.5 & $+\sim$0\\
            %D-OPQ& 0.314$^\dag$& 0.957$^\dag$& 0.619 & 0.608 & 109.1 & 0.3 \\
           % D-IVFOPQ & 0.386$^\dag$& 0.929$^\dag$& 0.680 & 0.697 & 48.8 & 1.2\\
             D-HNSW & 0.409$^\dag$& 0.978$^\dag$& 0.669 & 0.695 & 54.4 & $+$9.2\\
            %S $+$ rerank-D &  0.412 & 0.980    &0.715  & 0.811 & 0.698 & 0.815 & $+$2.3 \\
            S $+$ D-LADR & 0.422& 0.984$^\dag$& 0.743&0.728& 43.6& $+$4.3\\
%           7.1 &  S $+$ D-LADR & 0.423 & 0.987$^\dag$& 0.745 &0.728& 53.0 & $+$4.3\\
           % S $+$ D-IVF & 0.398$^\dag$& 0.987  & 0.674 & 0.720 & 64.7 & $+$0.0\\
           %S $+$ D-IVFOPQ & 0.392$^\dag$& 0.987 & 0.656 & 0.700 &  52.5 & 4.0\\
           S $+$ D-HNSW & 0.420 & 0.987 & 0.718 & 0.723 & 59.6 & $+$9.2\\
            \bf S $+$ CluSD & 0.426 & 0.987 & 0.744 & 0.724 &46.3& $+$0.03\\
       \hline       
            \multicolumn{7}{c}{\bf{Time Budget = 25 ms, S=SPLADE-effi-HT3}}\\
            \hline
            S& 0.380$^\dag$& 0.944$^\dag$& 0.721 &  0.726 & 12.4 & 3.0 \\
             S + D-rerank & 0.406 & 0.944$^\dag$& 0.728 & 0.738 & 15.1 & 30.2 \\
            $\blacktriangle$  S + D & 0.413 & 0.984 & 0.721 &0.729 & 1654.2& 30.2  \\
            \hline
            D-HNSW & 0.406 & 0.971$^\dag$& 0.672 & 0.694 & 28.4 & $+$9.2\\
             S $+$ D-HNSW & 0.407 & 0.984 & 0.706 & 0.724 & 28.7 & $+$9.2\\
            S $+$ D-LADR & 0.404$^\dag$& 0.984 & 0.728 &  0.736 & 24.2 & $+$4.3\\
%           16.1 & S $+$ D-LADR & 0.409 & 0.987 & 0.744 &  0.728  & 41.1 & $+$4.3\\
            \bf S $+$ CluSD & 0.412 & 0.984 & 0.733 & 0.728 & 25.4 & $+$0.03\\
			\hline\hline
		\end{tabular}  }
	\caption{Search under time budget but  no memory constraints 
}
  \vspace*{-10mm}
	\label{tab:maintime}
\end{table}
%For pipelines with a time budget, we report the additional overhead in the Storage (GB) column. 




