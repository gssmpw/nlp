%\section{Limitations and Future Work}
\vspace*{-4mm}
\section{Concluding Remarks}

CluSD is a lightweight cluster-based partial dense retrieval with
fast CPU response time and competitive relevance.
% without the need of an expensive document-wise proximity graph.
Compared to CDFS~\cite{2024SIGIR-CDFS-Yang},
CluSD does not make any strong statistical assumptions about the sparse ranking result distribution.
The evaluation results show that  CluSD can effectively select top dense
clusters with a performance  reasonably competitive to CDFS in relevance and latency
even CluSD incurs small LSTM computing overhead.  
%In some tested cases, it can select less dense clusters, resulting in a shorter retrieval time. 
More studies and tuning could be conducted
in the future in evaluating CluSD and CDFS.
     
%For example, CluSD is 23.3\% faster than CDFS with on-disk  RepLLaMA MS MARCO data while their MRR@10 numbers are close.
%The main contribution of this work is  to  show  how 
%a lightweight cluster-based partial dense retrieval 
%with competitive  relevance can be made possible
%to deliver fast response time on affordable computing platforms, without the need of an expensive  document-wise  proximity graph.
%CluSD carries with minimal space overhead compared to the orginal dense retrieval
%and incurrs a small amount of CPU latency time due to the limited dense computation.

%Guided by sparse retrieval, 
The evaluation shows CluSD significantly outperforms cluster-based partial dense retrieval with IVF in relevance.
% under the same time budget.
Under the same time budget, CluSD with negligible extra space overhead 
delivers better or similar relevance than HNSW and LADR that rely upon
a document-level proximity graph.
Further, CluSD is up-to 12.1x faster than them
% by avoiding fine-grained document-level disk I/O 
when embeddings 
do not fit in memory.
When compared to on-disk ANN methods DiskANN and SPANN~\cite{NEURIPS2019_DiskANN,2023Web-Filtered-DiskANN},
CluSD is 2.2x and 4.97x faster respectively, while achieving better relevance
% for text document search 
by leveraging sparse retrieval.
Compared to graph navigation approaches, 
CluSD does not need  a  document-level proximity graph,  and  it conducts faster block I/O when fetching clustered document embeddings  
from disk.  

\comments{
CluSD is also stronger and more versatile than reranking. 
Simple reranking does surprisingly well if sparse results are strong. But as stated in~\cite{2022CIKM-MacAvaneyGraphReRank,2023SIGIR-LADR},
when the sparse retriever doesn’t achieve high relevance, reranking is too 
restrictive and recall@1K does not increase as shown in Table 5. 
CluSD’s advantage becomes more significant in zero-shot retrieval as shown in Table 6 with 13 BEIR datasets. CluSD’s average NDCG number is 
7.2\% higher than reranking with RetroMAE. With SimLM, CluSD is 4\% higher than reranking.
In addition, when embeddings are not available in memory, reranking is 2.85x slower in mean latency.
%becomes very slow with a large number of document-level random I/O operations. 

CluSD works well  with LLaMA-2 based RepLLaMA dense  model for the tested datasets.
More such models  can be developed in the future using large language models of high dimensionality,
and our study sheds a light on how to take advantages of such dense models
on low-cost CPU servers, especially when their embeddings cannot fit into memory and are hosted on disk.

}
\comments{
Our evaluation has not used ColBERT with a multi-vector representation because 
ColBERTv2's MRR@10~\cite{Santhanam2021ColBERTv2}  for MS MARCO is 0.397, which is lower than that the single-vector dense models studied in this paper. 
CluSD techniques are  orthogonal to ColBERTv2 and could be used with ColBERTv2 after small modifications.
}

{\bf Acknowledgments}.
We thank anonymous referees for their valuable comments.
This work is supported in part by U.S. NSF IIS-2225942  and has used the computing resource of the ACCESS program supported by NSF.
Any opinions, findings, conclusions or recommendations expressed in this material
are those of the authors and do not necessarily reflect the views of the U.S. NSF.
%\pagebreak
