
\comments{
{\bf Abstract for Oct 5 abstract submission:}
Recent studies show that sparse retrieval with a learned representation
can be greatly accelerated by BM25-guided index traversal with a tradeoff  in its relevance effectiveness.
This paper proposes the relevance argumentation of fast sparse  retrieval with selective dense retrieval 
while only adds   a modest amount of extra computing cost on  a CPU-only platform.
This scheme takes an advantage of an advanced  dense retrieval model and 
boosts sparse retrieval with limited  embedding  computation.  
This paper provides a detailed evaluatin with MS MARCO and BEIR  datasets  
on the usefulness of the proposed scheme  in improving the recall and relevance metric of sparse retrieval.
The emprial findings suggest that it  can outperform other baselines for a balanced effectiveness and efficiency. 

{\bf Abstract Version 1  (weakness too much like LADR):}
Recent studies show that sparse retrieval with a learned representation
can be greatly accelerated by BM25-guided index traversal with a tradeoff  in its relevance effectiveness.
This paper proposes the relevance argumentation of fast sparse  retrieval with cluster-based selective dense retrieval called CSDR 
while only imposes   a modest amount of extra computing cost on  a CPU-only platform.
CSDR takes an advantage of clustering under an advanced  dense retrieval model and
detects clusters of documents that can boost sparse retrieval with  limited  embedding  computation.
This paper provides a detailed evaluatin with MS MARCO and BEIR  datasets  
on the usefulness of CSDR in improving the recall and relevance metric of sparse retrieval.
The emprial findings suggest that CSRF  can outperform other baselines for a balanced effectiveness and efficiency. 

%make a determination if  a query can  benefit from additional dense retrieval results,  

{\bf Version 2  (weakness too much on fusion)}
Dense and sparse retrievers are two main categories of the
first-stage search techniques for text  documents, and both of them have benefited 
from pretrained language models to provide effective semantic text matching between a query and a document.
Previous work has shown that  ranking of documents from dense and sparse retrievers could be combined for higher relevance effectiveness. 
The downside is that dense retrieval is time consuming on a CPU-only server platform.
This paper considers the optimization of dense and sparse retrieval fusion and 
 proposes a cluster-based selective dense retrieval scheme called CSDR which only requires triggering of dense retrieval 
computation for a relatively small subset of documents after  fast sparse retrieval. 
CSDR takes an advantage of clustering under an advanced  dense retrieval model and
detects clusters of documents that can boost sparse retrieval with  embedding  computation.
This paper provides a detailed evaluation with MS MARCO and BEIR  datasets  
on the usefulness of CSDR in improving the effectiveness of sparse retrieval with limited dense  retrieval
traversal.  Our empirical findings suggest that CSDR can outperform other baselines on a single CPU server 
without GPUs, and achieve a strong tradeoff of effectiveness and efficiency compared to an exhaustive dense retrieval search.
}
%{\bf Version 3  (combine 2 and 3)}
%Dense and sparse retrievers are two main categories of the
%first-stage search techniques for text  documents, and both of them have benefited 
%from pretrained language models to provide effective semantic text matching between a query and a document.
%Previous work has shown that  ranking of documents from dense and sparse retrievers could be combined for higher relevance effectiveness. 
%The downside is that dense retrieval is time consuming on a CPU-only server platform.

%Sparse retrieval and dense retrieval with a learned representation space
%can be greatly accelerated with tradeoffs  in its relevance effectiveness.
%suggest  that both categories  of retrievers tend to capture  different relevant signals.

%Sparse and dense  sparse retrievers are two main categories of the
%first-stage search techniques for text  documents, and both of them can be accelerated with various relevance tradeoffs. 
%This paper studies how to fuse them but avoid  excessive redundant computation.

Recent studies show that sparse retrieval with a learned representation
can be greatly accelerated but there are  tradeoffs  in its relevance effectiveness.
This paper proposes a relevance augmentation scheme of fast sparse  retrieval with cluster-based selective dense retrieval called CluSD
on  a CPU-only platform.
CluSD takes an advantage of clustering under an advanced  dense retrieval model and
detects clusters of documents that can boost sparse retrieval with  limited  extra memory space overhead and embedding selection and computing time. 
This paper provides a detailed evaluation of CluSD with MS MARCO and BEIR  datasets, 
and demonstrates its effectiveness in improving the relevance metric of sparse retrieval by scoring  selected clusters 
at a low time and space cost.

%  This paper proposes the relevance argumentation of fast sparse  retrieval with cluster-based selective dense retrieval called CluSD 
%  This paper proposes the relevance augmentation of fast sparse  retrieval with cluster-based selective dense retrieval called CluSD,
%  while only imposing   a modest amount of extra cost on  a CPU-only platform.
%  CluSD takes an advantage of clustering under an advanced  dense retrieval model and
%  detects clusters of documents that can boost sparse retrieval with  limited  extra memory space overhead and embedding  computation. 
%  This paper provides a detailed evaluatin of CluSD when searching  in-domain MS MARCO passages and out-of-domain BEIR  datasets 
%  This paper provides a detailed evaluation of CluSD when searching  in-domain MS MARCO passages and out-of-domain BEIR  datasets 
%  and demonstrates its effectivness in improving the relevance metric of sparse retrieval by scoring  selected clusters with dense embeddings
%  at a low time and space cost.
 



%The emprial findings suggest that CSRF  can outperform other baselines for a balanced effectiveness and efficiency. 
