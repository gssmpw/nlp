


\comment{
{\bf Old abstract:}
Recent studies show that neural sparse retrieval for searching  text documents with a learned neural representation
can be greatly accelerated but there are  tradeoffs  in its relevance effectiveness.
This paper proposes a relevance augmentation scheme of fast sparse  document retrieval with cluster-based selective dense retrieval called CluSD
on  a CPU-only platform.
CluSD takes an advantage of clustering under an advanced  dense retrieval model and
detects clusters of neural embeddings  that can boost sparse retrieval with  limited  extra memory space overhead and embedding selection and computing time. 
This paper provides a detailed evaluation of CluSD for searching  MS MARCO and BEIR  datasets, 
and demonstrates its effectiveness in improving the relevance metric of sparse retrieval by scoring  selected dense embedding clusters 
at a low time and space cost.


{\bf New abstract:}
 both of which have benefited from pretrained language models.
}
%Dense retrieval and sparse lexical retrieval are two main categories of the first-stage search techniques for 
%text  documents.  
Dense retrieval is attractive  in semantic  matching between queries and
text  documents, though it can be time-consuming on a CPU-only platform.
The cost of dense retrieval can be mitigated through approximate nearest neighbor search techniques, including IVF clustering, quantization, and/or 
proximity graph navigation. However, the reduction in latency by using these techniques comes with a tradeoff  in relevance effectiveness. 
This paper proposes a cluster-based selective dense retrieval method called CluSD with an augmentation from sparse lexical retrieval 
at a low space and CPU time cost. 
CluSD takes a lightweight cluster-based approach and exploits the overlap of sparse retrieval results and embedding clusters
in a two-stage selection process with an LSTM model to quickly identify relevant clusters  while incurring limited  extra memory space overhead. 
CluSD  triggers partial dense retrieval, performs cluster-based block disk I/O if needed, and combines dense scores  with  sparse results to enhance relevance. 
%advantage of clustering of embeddings from an advanced dense retrieval model, 
This paper provides a comprehensive evaluation of CluSD on in-memory and on-disk search of the MS MARCO and BEIR  datasets.
%(the MS MARCO dataset) and out-of-domain (BEIR  datasets) search tasks
%and demonstrates its effectiveness in relevance and efficiency.


