%\subsection{CluSD with other quantization models}
%\label{sect:evalcompress}



{\bf CluSD with advanced quantization models}.
%Tables~\ref{tab:mainspace}, \ref{tab:beir},
%and \ref{tab:RepLLaMA}  use the OPQ quantization method~\cite{Ge_2013_CVPR}  implemented in FAISS.
Table~\ref{tab:compress} shows the use of two other recent quantization methods 
DistillVQ~\cite{Xiao2022Distill-VQ} and JPQ~\cite{2021CIKM-JPQ-Zhan} 
with in-memory CluSD for MS MARCO Passage Dev set. 
%Latency column includes sparse retrieval time for last two rows where fusion is conducted. 
%We use the OPQ implementation from FAISS with results in Table \ref{tab:sparse}.
For JPQ, we use its released model and compression checkpoints and follow its default setting with $m=96$. 
For DistillVQ, we use its released code to train compression on the RetroMAE model using its default  value $m=128$.
%This DistillVQ setting  yields 1.38GB storage space for RetroMAE embeddings while JPQ yields  0.95GB for 
%its dense model. 
%Table ~\ref{tab:compress} shows that 
%under the same quantization (DistillVQ or JPQ), CluSD significantly outperforms FAISS IVF in relevance and latency 
%before and after fusion.  This is consistent with our findings with OPQ.
%Although DistillVQ outperforms OPQ for standalone dense retrieval,
%relevance of S+CluSD with OPQ for RetroMAE shown in  Table~\ref{tab:mainspace} is about the same  as S+CluSD with DistillVQ (same 0.417 MRR@10;  0.986 vs. 0.987 R@1K).  
While switching to a different quantization affects compression and  relevance, 
this result shows that CluSD is still effective and outperforms other baselines.
% under the same quantization. 
%We do not directly evaluate with RepCONC~\cite{2022WSDM-Zhan-RepCONC}
%because ~\cite{Xiao2022Distill-VQ} shows that it has similar relevance to JPQ
%and is outperformed by DistillVQ.
% \vspace*{-5mm}









\comments{
\begin{verbatim}
DistillVQ	MRR@10	Recall@1K	Latency (ms)
IVF	0.365	0.899	20.8
CluSD	0.393	0.977	9.7
SPLADE+ IVF	0.392	0.987	52.0
SPLADE+CluSD	0.417	0.987	40.9

JPQ	MRR@10	Recall@1K	Latency (ms)
IVF	0.326	0.917	21.3
CluSD	0.341	0.969	11.2
SPLADE+IVF	0.379	0.985	52.5
SPLADE+CluSD	0.392	0.985	42.4
\end{verbatim}
}


\begin{table}[htbp]
        \centering
        \caption{ Use of CluSD with   DistillVQ and JPQ quantization}
                %\resizebox{0.75\columnwidth} {%
                \begin{tabular}{ r |l l l |ll l}
                        \hline\hline
                Quantization & \multicolumn{3}{c|}{\bf{DistillVQ}} & \multicolumn{3}{c}{\bf{JPQ}}   \\
%\cline{1-6}
                        \hline
S=SPLADE-HT1                        & {MRR@10} & {R@1K}  & {Latency } & {MRR@10} & {R@1K}    & {Latency}\\
                        \hline
                        IVF (2\%)  & 0.365$^\dag$ &	0.899$^\dag$ &	20.8 ms
& 0.326$^\dag$	&0.917$^\dag$ &	21.3 ms\\
%CluSD &0.393&	0.977&	9.7ms &0.341&	0.969&	11.2ms\\
S+ IVF (2\%)	&0.392$^\dag$ &	0.987&	52.0 ms
& 0.379$^\dag$ & 	0.985& 	52.5 ms\\
$\blacktriangle$ S+CluSD&	 0.417 &	0.987 &	40.9 ms
 &0.392 &	0.985 &	42.4 ms\\
\hline           
\hline           
                \end{tabular}%
                %}
\vspace*{-5mm}  
        \label{tab:compress}
\end{table}     

