%Dense and sparse retrievers are two main categories of the
%first-stage search techniques for text  documents, and both of them have benefited 
%from pretrained language models to provide effective semantic text matching between a query and a document.
Previous work has shown that  ranking of documents from dense and sparse retrievers could be combined for higher relevance effectiveness. 
The downside is that dense retrieval is time consuming on a CPU-only server platform.
This paper considers the optimization of dense and sparse retrieval fusion and 
 proposes a cluster-based selective dense retrieval scheme called CluSD which only requires triggering of dense retrieval 
computation for a relatively small subset of documents after  fast sparse retrieval. 
CluSD takes an advantage of clustering under an advanced  dense retrieval model and
detects clusters of documents that can boost sparse retrieval with  embedding  computation.
This paper provides a detailed evaluation with MS MARCO and BEIR  datasets  
on the usefulness of CluSD in improving the effectiveness of sparse retrieval with limited dense  retrieval
traversal.  Our empirical findings suggest that CluSD can outperform other baselines on a single CPU server 
without GPUs, and achieve a strong tradeoff of effectiveness and efficiency compared to an exhaustive dense retrieval search.

