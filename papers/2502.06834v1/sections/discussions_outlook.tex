\section{Conclusion and Future Work}
\label{sec:discussions_future}

One big drawback of the traditional ads ranking approaches that rely solely on supervised learning is their inherent inability to generalize to the ad candidates that have not yet been delivered to users for ad impressions. Semi-supervised learning can be used to mitigate this drawback, and our proposed UKDSL framework enables the application of semi-supervised learning in an industrial-grade ads ranking systems at scale, allowing developers to deploy semi-supervised learning to all model types across all stages with minimal additional cost and effort. Our experiment results show that the UKDSL framework helps overcome the inherent mis-calibration issue, thus improving model calibrations, but also improves their performance on both labeled and unlabeled data.
UKDSL has been launched to major industrial-scale ads recommendation models across different ranking stages and traffic. This indicates that it can be generalized to diverse user demographics and content types, considering the scale and reach of the deployed ads platform.

By sharing our success stories in leveraging unlabeled data in an industrial-scale ads ranking systems, we hope to motivate more future work in this important and impactful area of research. Particularly, the use of foundation models for distillation on earlier stage models is an interesting and currently under-explored direction. Having the same foundation model as the teacher for all ranking stages can not only potentially improve their individual model performance, but also further improve the cross-stage prediction consistency. Other areas worth further exploring include developing more efficient foundation models, improving knowledge transfer efficiency, and improved sampling methods for learning from unlabeled data more efficiently. 