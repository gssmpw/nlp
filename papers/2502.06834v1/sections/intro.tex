\section{Introduction}
\begin{figure}
\centering
\includegraphics[width=7cm]{figs/fig1.png}
  \caption{a) a block-diagram of a multi-stage ranking system. b) demonstration of unlabeled and labeled data used in industrial systems.}
  \label{fig:teaser}
\end{figure}
When users surf on applications with large amounts of users (e.g., social media, streaming services, online shops, etc), they would like to see ads that they find relevant to their interests and needs. To achieve this goal, ads ranking systems are built to find a handful of best matched ads from the candidate pool of large number of ads within seconds.

To balance the accuracy and efficiency under such tight time and large scale constraints, industrial ads ranking systems use a cascade multi-stage ranking design~\cite{wang2023towards}. Each stage is responsible for scanning through the candidates provided by the previous stage, and provides the top candidates for the next stage (see Figure 1). Typically, the models used in earlier stages (e.g., retrieval stage) are simpler, more efficient but less accurate (e.g., two tower network~\cite{scalingIG23}), whereas the models used in later stages are more complex, accurate, but slower. After the top ads were delivered to the users, feedback from the users (e.g., click) on those delivered ads (i.e. impression data) can be collected as labels. 

Traditionally, the models from all stages are trained with the impression data in a supervised fashion. However, the impression data is only a very small subset of the data that the early stage models (i.e., retrieval and pre-ranking stage models) see and make predictions about. Hence, there are large discrepancies between training and serving data for the early stage models. Such discrepancies are often a cause for online performance degradations due to lack of generalization.

In this work, we leverage a Unified framework for Knowledge-Distillation and Semi-supervised Learning (UKDSL) to improve models by closing the gap between the training and serving data. We evaluate the UKDSL framework on industry-scale ads ranking systems across different stages (retrieval and pre-ranking stages), different types (click-through rate models and conversion rate models).
In summary, the contributions of our paper are as follows:
\newline
\begin{itemize}
    \item \textbf{Inherent drawbacks of multi-stage ranking systems}: We provide theoretical analyses revealing  two important inherent problems in multi-stage ranking models. Through formal analysis and numerical simulations, we shed light on causes of these drawbacks.
    \item \textbf{\textbf{U}nified framework for Knowledge-Distillation and Semi-Supervised Learning}: We present \textbf{UKDSL}, a \textbf{U}nified framework for \textbf{K}nowledge-\textbf{D}istillation and \textbf{S}emi-supervised \textbf{L}earning in industrial ads delivery systems. We provide strong empirical evidence of the effectiveness of this framework in improving both offline and online ads rankings performance. We present empirical evidence on applying the UKDSL framework on industry-scale ads ranking systems, showing this method can bring significant improvement to the ads ranking systems.    
\end{itemize}
\vspace{6pt}
The remainder of this paper is as follows. In Section~\ref{sec:related} we present related work and existing methods from the literature.
Section~\ref{sec:problem} will present the problem statement, definitions, and theoretical results.
We detail our UKDSL framework in Section~\ref{sec:method} and present our empirical evidence.
And finally, Section~\ref{sec:discussions_future} concludes this work and portray possible future directions for the research community.