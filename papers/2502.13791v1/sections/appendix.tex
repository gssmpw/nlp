\arrayrulecolor{black}          
\begin{table*}
    \centering
    \begin{tabular}{lllll}
        \toprule
         \textbf{Model} &  \textbf{Instruction} &  \textbf{Session} & \textbf{Short History} & \textbf{Long History} \\
         \midrule
         Command R+ &  0.89 \scriptsize{($\pm$0.13)}&  0.66 \scriptsize{($\pm$0.24)}& 0.40 \scriptsize{($\pm$0.35)}& 0.11 \scriptsize{($\pm$0.07)}\\
         
         Llama-3.1-8B &  0.71 \scriptsize{($\pm$0.25)} & 0.36 \scriptsize{($\pm$0.25)} & 0.23 \scriptsize{($\pm$0.28)}& 0.12 \scriptsize{($\pm$0.06)} \\
         
         Llama-3.1-70B &  0.92 \scriptsize{($\pm$0.12)}&  0.88 \scriptsize{($\pm$0.16)}& 0.64 \scriptsize{($\pm$0.33)}& 0.12 \scriptsize{($\pm$0.07)}\\
         
         Llama-3.1-405B &  0.95 \scriptsize{($\pm$0.09)}&  0.90 \scriptsize{($\pm$0.12)}& 0.70 \scriptsize{($\pm$0.29)}& 0.20 \scriptsize{($\pm$0.10)}\\
         
         GPT-4o &  \textbf{0.94} \scriptsize{($\pm$0.10)}&  \textbf{0.93} \scriptsize{($\pm$0.10)}& \textbf{0.79} \scriptsize{($\pm$0.24)}& \textbf{0.30} \scriptsize{($\pm$0.16)}\\
         \bottomrule
    \end{tabular}
    \caption{Average (standard deviation) \textbf{Instruction}, \textbf{Session},  \textbf{Short History} and \textbf{Long History} scores per model.}
    \label{tab:detailed_scores}
\end{table*}

\begin{figure}[t!]
    \centering
    \includegraphics[width=\linewidth]{figs/cum_rag_pivot_only.png}
    \caption{Score as a function of number of sessions. 
    }
    \label{fig:cum_rag_pivot_only}
\end{figure}

\section{Seeds}
\label{app:seeds}

\paragraph{Instructions} We used 51 instructions during data generation, as shown in Table \ref{tab:instructions_list}. Each instruction is applied to a specific Python object, and for pivots that can be updated, the update values are provided in brackets.

\paragraph{Fillers} The list of all 80 fillers used during data generation is provided in Table \ref{tab:fillers}. Some fillers can be updated over time.

\paragraph{Names} The list of mentor, mentee and company names used during data generation is provided in Table \ref{tab:names}.

\paragraph{Personas} The lists of mentor and mentee personas are provided in Table \ref{tab:mentor_personas} and Table \ref{tab:mentee_personas}.


\section{Detailed scores}
\label{sec:detailed_scores}
We provide the detailed scores of the different models evaluated on Instruction, Session,  Short History and Long History in Table \ref{tab:detailed_scores}. 

\section{Hyperparameters}
\label{app:hyperparameters}
We use Command R+ with a temperature of 0.9 and a top-p value of 0.9 to generate the conversations. During evaluation, we use a temperature of 0 for all models. 

\section{RAG experiments}
\label{app:rag_experiments}
We performed the Retrieval-augmented Generation (RAG) experiments using the rerank-english-v3.0 model \citep{coherererank2024}. The basic unit for retrieval were the previous sessions. We also tried with different retrieval units, such as paragraphs and turns, and obtained similar results. 
We provide the model the top-\textit{k} retrieved sessions, where \textit{k} was defined as the number of sessions with pivots. Note that, by dynamically defining \textit{k} in this way, rathr than using a fix value, we facilitate the retrieval of the relevant sessions only, eliminating potential noise.
We report the results of RAG in Figure \ref{fig:cum_rag_pivot_only}, together with those of Instructions-only and Cumulative. RAG provides a marginal improvement over cumulative for `short' conversations, but it then converges with Cumulative for the `long' ones.

\section{Per-Instruction Scores}
\label{app:per_instruction_scores}
We report in Figure \ref{fig:per_pivot_score_insertion} the average results for each instruction, and in 
Figure \ref{fig:per_pivot_score_update} those for each update. Remember that only some of the instructions have an update (See Section \ref{subsec:seeds}).

\section{Examples}
\label{app:examples}
We provide examples of dialogue histories along with their corresponding prompts. The first one, shown in Table \ref{tab:conv_70}, is a 3-session history with 3 pivots, and its prompt is provided in Table \ref{tab:prompt_conv_70}. The second one, presented in Table \ref{tab:conv_108}, consists of 4 sessions with 3 pivots, with its prompt shown in Table \ref{tab:prompt_conv_108}. The system prompt and the session-level prompts were designed to ensure consistency across sessions. Additionally, we provide examples of prompts to get the model output for the Instructionm History and Instructions-Chains settings in Tables \ref{tab:intruction_prompt_example}, \ref{tab:conversation_prompt_example} and \ref{tab:pivot_only}.

\section{Computational budget}
\label{sec:computational_budget}
The prompts used to generate the dataset consist of a total of 0.9M tokens, while the dataset itself contains 4.6M tokens. The total cost of generating the dataset using Command R+ through Cohere's API is approximately \$50. The cost of evaluating a single model on the Instruction, Session and Cumulative settings using online APIs is approximately \$50.

\section{Ethics statement}
\label{sec:ethical statement}
The dataset we are releasing is synthetic and, therefore, does not contain any personally identifiable information. Moreover, we did not recruit any human participants, as we manually validated the quality of the dataset ourselves.




\begin{figure*}
    \centering
    \includegraphics[width=0.9\linewidth]{figs/per_pivot_score_insertion.png}
    \caption{Average per-instruction insertion scores for GPT-4o}
    \label{fig:per_pivot_score_insertion}
\end{figure*}

\begin{figure*}[htb!]
    \centering
    \includegraphics[width=\linewidth]{figs/per_pivot_score_update.png}
    \caption{Average per-instruction update scores for GPT-4o}
    \label{fig:per_pivot_score_update}
\end{figure*}

\onecolumn
\arrayrulecolor{gray!50}
\begin{xltabular}{\linewidth}{Xl}
    \specialrule{1.2pt}{0pt}{0pt}
    \rowcolor{gray!20} \textbf{Instruction} & \textbf{Object} \\
    \specialrule{1.2pt}{0pt}{0pt}
    use \{all UPPERCASE, CamelCase, snake\_case\} for class names & class \\
    \specialrule{0.6pt}{0pt}{0pt}
    include the string 'chx' in function names & function \\
    start function names with \{'a\_', 'b\_', 'c\_', 'd\_', 'x\_', 'y\_', 'fn\_', 'gn\_'\} & function \\
    end function names with \{'\_a', '\_b', '\_c', '\_d', '\_x', '\_y', '\_fn', '\_gn'\} & function \\
    \specialrule{0.6pt}{0pt}{0pt}
    include the string 'chx' in variable names & variable \\
    start variable names with \{'x\_', 'y\_', 'z\_', 'w\_', 'm\_', 'n\_', 'vr\_', 'wr\_'\} & variable \\
    end variable names with \{'\_x', '\_y', '\_z', '\_w', '\_m', '\_n', '\_vr', '\_wr'\} & variable \\
    \specialrule{0.6pt}{0pt}{0pt}
    start method names with \{'x\_', 'n\_', 'o\_', 'p\_', 'a\_', 'b\_', 'md\_', 'ud\_'\} & method \\
    end method names with \{'\_x', '\_n', '\_o', '\_p', '\_a', '\_b', '\_md', '\_ud'\} & method \\
    include the string 'chx' in method names & method \\
    \specialrule{0.6pt}{0pt}{0pt}
    include the string 'chx' in attribute names & attribute \\
    start attribute names with \{'q\_', 'r\_', 's\_', 't\_', 'i\_', 'j\_', 'at\_', 'xt\_'\} & attribute \\
    end attribute names with \{'\_q', '\_r', '\_s', '\_t', '\_i', '\_j', '\_at', '\_xt'\} & attribute \\
    \specialrule{0.6pt}{0pt}{0pt}
    start function argument names with \{'e\_', 'f\_', 'g\_', 'h\_', 'i\_', 'j\_', 'x\_', 'a\_'\} & function argument \\
    end function argument names with \{'\_e', '\_f', '\_g', '\_h', '\_i', '\_j', '\_x', '\_a'\} & function argument \\
    include the string 'chx' in function argument names & function argument \\
    \specialrule{0.6pt}{0pt}{0pt}
    use annotations for functions & function annotation \\
    include try statements in functions & function try \\
    include assert statements in functions & function assert \\
    use docstrings in functions & function docstring \\
    \specialrule{0.6pt}{0pt}{0pt}
    add the '@retry' decorator from the 'pedantic' module to all functions & function decorator \\
    add the '@count\_calls' decorator from the 'pedantic' module to all functions & function decorator \\
    add the '@rename\_kwargs' decorator from the 'pedantic' module to all functions & function decorator \\
    add the '@require\_kwargs' decorator from the 'pedantic' module to all functions & function decorator \\
    add the '@timer' decorator from the 'pedantic' module to all functions & function decorator \\
    \specialrule{0.6pt}{0pt}{0pt}
    use annotations for methods & method annotation \\
    include try statements in methods & method try \\
    include assert statements in methods & method assert \\
    use docstrings in methods & method docstring \\
    \specialrule{0.6pt}{0pt}{0pt}
    add the '@retry' decorator from the 'pedantic' module to all methods & method decorator \\
    add the '@count\_calls' decorator from the 'pedantic' module to all methods & method decorator \\
    add the '@rename\_kwargs' decorator from the 'pedantic' module to all methods & method decorator \\
    add the '@require\_kwargs' decorator from the 'pedantic' module to all methods & method decorator \\
    \specialrule{0.6pt}{0pt}{0pt}
    add comments in your code & comment \\
    \specialrule{0.6pt}{0pt}{0pt}
    import the 'secrets' module even if it is not used & import \\
    import the 'locale' module even if it is not used & import \\
    import the 'mailbox' module even if it is not used & import \\
    import the 'bz2' module even if it is not used & import \\
    import the 'gzip' module even if it is not used & import \\
    import the 'hashlib' module even if it is not used & import \\
    \specialrule{1.2pt}{0pt}{0pt} 
    
\caption{List of all the instructions used in pivots.}
\label{tab:instructions_list}
\end{xltabular}
\twocolumn

\onecolumn
\begin{xltabular}{\linewidth}{X}
    \specialrule{1.2pt}{0pt}{0pt}
    \rowcolor{gray!20} \textbf{Filler}  \\
    \specialrule{1.2pt}{0pt}{0pt}
Contract renewal negotiations and finalization \\ 
 \midrule
Planning engaging social activities for the upcoming holidays \\ 
 \midrule
Flexible work hours and their impact on work-life balance \\ 
 \midrule
Troubleshooting internet connectivity issues across different locations \\ 
 \midrule
Promoting a harmonious workplace through effective conflict resolution \\ 
 \midrule
Salary increase opportunities and performance appraisals \\ 
 \midrule
Preparing extensively for a high-stakes client meeting \\ 
 \midrule
Requesting upgraded technology, including computers and software \\ 
 \midrule
Happy hour events to foster better relationships between managers and employees \\ 
 \midrule
Remote work policies, challenges, and best practices \\ 
 \midrule
Understanding comprehensive social security and health insurance benefits \\ 
 \midrule
Performance evaluation criteria, feedback mechanisms, and recognition programs \\ 
 \midrule
Encouraging professional development through workshops, courses, and certifications \\ 
 \midrule
Implementing wellness initiatives to support the physical and mental well-being of employees \\ 
 \midrule
Ensuring health and safety in the workplace, including COVID-19 protocols and emergency response plans \\ 
 \midrule
Participating in company-sponsored volunteer programs to give back to society \\ 
 \midrule
Ergonomic assessments and improvements to ensure comfortable and healthy workspaces \\ 
 \midrule
Travel arrangements and logistics for client meetings, conferences, and business trips \\ 
 \midrule
Team-building activities to strengthen collaboration, communication, and trust within teams \\ 
 \midrule
Reinforcing and embodying the company's core values in day-to-day operations and decision-making \\ 
 \midrule
Analyzing client feedback to improve products, services, and overall customer satisfaction \\ 
 \midrule
Implementing effective meeting strategies and time management techniques to optimize productivity \\ 
 \midrule
Cultivating a feedback culture and providing performance improvement plans for continuous growth \\ 
 \midrule
Providing conflict resolution training to employees to foster a peaceful and respectful work environment \\ 
 \midrule
Hosting company-wide town hall meetings to share updates and foster transparency \\ 
 \midrule
Discussing casual dress code policies and special event themes to boost employee morale \\ 
 \midrule
Planning team outings and off-site adventures to promote team bonding and relaxation \\ 
 \midrule
Celebrating employee birthdays, work anniversaries, and achievements with recognition and rewards \\ 
 \midrule
Introducing new employee benefits, perks, and incentives to attract and retain top talent \\ 
 \midrule
Launching diversity and inclusion initiatives to create a more equitable and inclusive workplace \\ 
 \midrule
Conducting exit interviews to gather feedback and insights for improving retention and engagement \\ 
 \midrule
Developing a mentorship program to support career growth and development for employees \\ 
 \midrule
Building a culture of innovation and creativity through idea-sharing and experimentation \\ 
 \midrule
Creating a knowledge-sharing platform to facilitate learning and collaboration among employees \\ 
 \midrule
Implementing agile methodologies to improve project management and delivery processes \\ 
 \midrule
Designing a performance dashboard to track key metrics and KPIs for business success \\ 
 \midrule
Conducting team-building workshops and leadership training to develop future leaders \\ 
 \midrule
Facilitating cross-functional collaboration and communication to break down silos and improve efficiency \\ 
 \midrule
Promoting work-life balance through flexible work schedules and remote work options \\ 
 \midrule
Implementing a rewards and recognition program to motivate and engage employees \\ 
 \midrule
Developing a culture of continuous learning and improvement through training and development programs \\ 
 \midrule
Creating a culture of accountability and ownership to drive results and achieve goals \\ 
 \midrule
Fostering a culture of diversity, equity, and inclusion to create a more inclusive and welcoming workplace \\ 
 \midrule
Building a culture of trust and transparency through open communication and feedback \\ 
 \midrule
Offering free food and snacks to employees to boost morale and productivity \\ 
 \midrule
Developing a culture of innovation and creativity to drive growth and competitiveness \\ 
 \midrule
Creating a culture of collaboration and teamwork to achieve shared goals and objectives \\ 
 \midrule
Promoting a culture of customer-centricity and service excellence to drive customer satisfaction \\ 
 \midrule
Building a culture of adaptability and resilience to navigate change and uncertainty \\ 
 \midrule
Fostering a culture of sustainability and social responsibility to make a positive impact on society \\ 
 \midrule
Developing a culture of empowerment and autonomy to enable employees to take ownership of their work \\ 
\midrule
Use vim as the preferred ide \\
Use emacs as preferred ide \\
Use vscode as preferred ide \\
Use pycharm as preferred ide \\
\midrule
Use zoom for video calls \\
Use google meet for video calls \\
Use microsoft teams for video calls \\
Use skype for video calls \\
\midrule
Always use a virtual environment \\
Never use a virtual environment \\
\midrule
Always use the gpus for training neural networks \\
Never use the gpus for training neural networks but use the tpus instead \\
\midrule
Always write long and descriptive commit messages \\
Never write long and descriptive commit messages \\
\midrule
Use github as the main version control system \\
Use gitlab as the main version control system \\
Use bitbucket as the main version control system \\
\midrule
Never use a password manager \\
Always use a password manager \\
\midrule
Do not commit directly to the dev branch \\
Commit directly to the dev branch \\
\midrule
Always use a linter \\
Never use a linter \\
\midrule
Always use a formatter when writing code \\
Never use a formatter when writing code \\
\midrule
Always use a pre-commit hook \\
Never use a pre-commit hook \\
\midrule
Use github copilot as the coding assistant \\
Use tabnine as the coding assistant \\
Use codewhisperer as the coding assistant \\
Use codestral as the coding assistant \\
\midrule
Always use a debugger \\
Never use a debugger but only print statements \\
\midrule
Always use a profiler \\
Never use a profiler \\
\midrule
Use only a single monitor when coding \\
Use two monitors when coding \\
Use three or more monitors when coding \\
\midrule
Use a macbook as the main computer \\
Use a windows laptop as the main computer \\
Use a linux desktop as the main computer \\
\midrule
Use slack for communication \\
Use microsoft teams for communication \\
Use discord for communication \\
\midrule
Always use a vpn when working remotely \\
Never use a vpn when working remotely \\
\midrule
Always take all the vacation days per year \\
Take at least half of the vacation days per year \\
Take at least 7 days of vacation per year \\
\midrule
Always take a break every 50 minutes \\
Always take a break every 60 minutes \\
Always take a break every 70 minutes \\
Always take a break every 80 minutes \\
\midrule
Always go to the dedicated office on the third floor to work \\
Always go to the common area to work \\
Always go to the it room to work \\
\midrule
Never work from home \\
Work from home at most once a week \\
Work from home at most twice a week \\
Work from home at most three times a week \\
\midrule
Communicate with the team regularly \\
Communicate with the team only when necessary \\
Communicate with the team only when asked \\
\midrule
Always set up ci/cd pipelines \\
Never set up ci/cd pipelines \\
\midrule
Always use the ticketing system \\
Never use the ticketing system \\
\midrule
Use trello as the project management tool \\
Use asana as the project management tool \\
Use jira as the project management tool \\
Use monday as the project management tool \\
\midrule
Always use a whiteboard for brainstorming \\
Never use a whiteboard for brainstorming \\
\midrule
Always use a notebook for taking notes \\
Never use a notebook for taking notes \\
\midrule
Always do pair programming with a colleague \\
Never do pair programming with a colleague \\
\midrule
Use their personal phone for work calls \\
Use the company phone for work calls \\
\bottomrule
\caption{List of all fillers.}
\label{tab:fillers}
\end{xltabular}
\twocolumn

\begin{table*}[htb]
\centering
\small
\renewcommand{\arraystretch}{1.2} %
\setlength{\tabcolsep}{10pt} %
\begin{tabularx}{0.7\textwidth}{>{\centering\arraybackslash}X 
                                >{\centering\arraybackslash}X 
                                >{\centering\arraybackslash}X}
    \specialrule{1.2pt}{0pt}{0pt}
    \rowcolor{gray!20} \textbf{Mentor} & \textbf{Mentee} & \textbf{Company}  \\
    \specialrule{1.2pt}{0pt}{0pt}
Alice       & Bob        & NEXT \\
Juan        & Luke       & INNOVADE \\
Sara        & Eva        & TECHNO \\
Luis        & Kiyotaka   & CODEME \\
Maria       & David      & STARTED \\
Carlos      & Sofia      & GROWTHX \\
Yuichi      & Pablo      & DEVS \\
Pedro       & Marta      & CODEM \\
Djibril     & Jorge      & CHEETAH \\
Jean-Aimé   & Lucas      & VATO \\
Emma        & Oliver     & LEAP \\
Michael     & Ella       & ZENITH \\
Yoon-Seo    & Alexander  & AXIOM \\
Ethan       & Rado       & ORBIT \\
Harena      & Jacob      & VERSA \\
Sylvie      & Sophia     & PACE \\
Sophie      & Liam       & UNITE \\
Naivo       & Dera       & SYNERGY \\
Daniel      & Noah       & FORTUNA \\
\bottomrule              
\end{tabularx}
\caption{List of mentors, mentees, and their respective companies.}
\label{tab:names}
\end{table*}



\begin{table*}[htb]
\centering
\begin{tabularx}{0.9\linewidth}{X}
    \specialrule{1.2pt}{0pt}{0pt}
    \rowcolor{gray!20} \textbf{Mentor persona}  \\
    \specialrule{1.2pt}{0pt}{0pt}
{[}mentor{]} is a patient and supportive mentor. {[}mentor{]} enjoys helping others and sharing their knowledge and experience. {[}mentor{]} is always looking for ways to empower and inspire their mentee. \\
\midrule
{[}mentor{]} is a strict and demanding mentor. {[}mentor{]} has high expectations for their mentee. {[}mentor{]} goes straight to the point and is very clear.\\
\midrule
{[}mentor{]} is a caring and nurturing mentor. {[}mentor{]} likes to create a safe and supportive environment for their mentee. {[}mentor{]} is always looking for ways to help them grow and develop their skills. \\
\midrule
{[}mentor{]} is a passionate and energetic mentor. {[}mentor{]} thrives on helping others and their enthusiasm is contagious. {[}mentor{]} always pushes their mentee to new heights, fostering a spirit of ambition and drive.\\
\midrule
{[}mentor{]} is a structured and goal-oriented mentor. {[}mentor{]} helps their mentee to set realistic, achievable goals. {[}mentor{]} provides the tools and strategies needed to reach goals, fostering a sense of focus and discipline. \\
\bottomrule
\end{tabularx}
\caption{List of mentor personas. {[}mentor{]} is replaced with the name of the mentor in the prompts.}
\label{tab:mentor_personas}
\end{table*}

\begin{table*}[htb]
\centering
\begin{tabularx}{0.9\linewidth}{X}
    \specialrule{1.2pt}{0pt}{0pt}
    \rowcolor{gray!20} \textbf{Mentee persona}   \\
    \specialrule{1.2pt}{0pt}{0pt}
{[}mentee{]} is shy and wants to improve their coding skills. {[}mentee{]} just graduated from college and {[}mentee{]} is eager to learn from their mentor. \\
\midrule
{[}mentee{]} is a confident and ambitious software engineer. {[}mentee{]} is always looking for new challenges and opportunities to grow. {[}mentee{]} has been working in the industry for a few years now.\\
\midrule
{[}mentee{]} is a perfectionist with great attention to detail. {[}mentee{]} likes things to be done the right way and has a hard time delegating tasks to others. {[}mentee{]} is critical of himself and of others.\\
\midrule
{[}mentee{]} is a social and outgoing person. {[}mentee{]} enjoys working in teams and collaborating with others. {[}mentee{]} is always looking for ways to connect with their colleagues and builds strong relationships. \\
\midrule
{[}mentee{]} is a quiet and introverted individual. {[}mentee{]} prefers to work alone and is not very comfortable in social situations. {[}mentee{]} struggles to communicate their ideas and thoughts to others.\\
\midrule
{[}mentee{]} is a creative and innovative thinker. {[}mentee{]} likes to experiment with new ideas and approaches. {[}mentee{]} is not afraid to take risks and try new things.\\
\bottomrule
\end{tabularx}
\caption{List of mentee personas. {[}mentee{]} is replaced with the name of the mentee in the prompts.}
\label{tab:mentee_personas}
\end{table*}

\onecolumn
\begin{xltabular}{0.9\linewidth}{X}
\specialrule{1.2pt}{0pt}{0pt}
    \rowcolor{gray!20} \textbf{Prompt to generate dialogue history 70}   \\
    \specialrule{1.2pt}{0pt}{0pt}
\textbf{SYSTEM PROMPT} \\
\midrule
\#\# Task and Context\\    You are a helpful and obedient AI that follows its system prompt and takes it very seriously. Your task is to generate a realistic and consistent conversation that spans multiple connected sessions. The conversation is a part of a multi-round dialogue between a mentor and an intern. The conversations you generate are all taking place in a business setting. \\    Jean-Aimé is a mentor in a big software company called VATO. Lucas is a new employee. They are both part of the software engineering team of the company. Jean-Aimé is a passionate and energetic mentor. Jean-Aimé thrives on helping others and their enthusiasm is contagious. Jean-Aimé always pushes their mentee to new heights, fostering a spirit of ambition and drive.\\    Lucas is a creative and innovative thinker. Lucas likes to experiment with new ideas and approaches. Lucas is not afraid to take risks and try new things. \\    The main programming language used in the company is Python.\\\\    
\#\# Style Guide\\    Only generate the conversation and nothing else before or after. Do not add numbers before each turn. Do not add quotes to the conversation turns. Use a professional and formal tone. The conversation flow should be natural and smooth. When switching topics, do it in a smooth way. There are no special characters between the turns. The conversations are dialogues and not narrations of events.\\    Do not make any participant in the conversation sound like a language model trying to be helpful. Make them sound as human as possible.\\    It is the mentor that leads the conversation. When Jean-Aimé the mentor gives coding instructions, they do not provide examples. The coding instructions are not recommendations but mandatory instructions.\\
\midrule
\textbf{SESSION 1} \\
\midrule
Generate a short conversation between Jean-Aimé and Lucas. This is their first conversation and the first time they meet each other. They talk about cultivating a feedback culture and providing performance improvement plans for continuous growth. \\

\midrule
\textbf{SESSION 2} \\
\midrule
Generate a long conversation between Jean-Aimé and Lucas. This is not their first conversation. They had 1 conversations before. Jean-Aimé gives some specific coding instructions to Lucas. Jean-Aimé wants Lucas to always add the '@timer\_class' decorator from the 'pedantic' module to all classes. Jean-Aimé wants Lucas to always start function names with 'x\_'. Jean-Aimé does not provide examples of correct code following the instructions. They do not show how to implement the instructions. Jean-Aimé never says 'for example'. Jean-Aimé does not give any other coding instructions. Lucas only acknowledges the instructions and does not ask any questions. \\

\midrule
\textbf{SESSION 3} \\
\midrule
Generate a medium-length conversation between Jean-Aimé and Lucas. This is not their first conversation. They had 2 conversations before. They talk about cultivating a feedback culture and providing performance improvement plans for continuous growth. They had a previous conversation about this before. After that, Jean-Aimé gives some specific coding instructions to Lucas. Jean-Aimé is updating a previous information given to Lucas: Jean-Aimé now wants Lucas to always start function names with 'gn\_'. Jean-Aimé does not provide examples of correct code following the instructions. They do not show how to implement the instructions. Jean-Aimé never says 'for example'. Jean-Aimé does not give any other coding instructions. Lucas only acknowledges the instructions and does not ask any questions. \\

\bottomrule
\caption{Prompts to generate dialogue history 70 which contains 3 sessions. }
\label{tab:prompt_conv_70}
\end{xltabular}
\twocolumn

\onecolumn
\begin{xltabular}{0.9\linewidth}{X}
    \specialrule{1.2pt}{0pt}{0pt}
    \rowcolor{gray!20} \textbf{Prompt to generate dialogue history 108}   \\
    \specialrule{1.2pt}{0pt}{0pt}

\textbf{SYSTEM PROMPT} \\
\midrule
\#\# Task and Context\\    You are a helpful and obedient AI that follows its system prompt and takes it very seriously. Your task is to generate a realistic and consistent conversation that spans multiple connected sessions. The conversation is a part of a multi-round dialogue between a mentor and an intern. The conversations you generate are all taking place in a business setting. \\    Yuichi is a mentor in a big software company called DEVS. Pablo is a new employee. They are both part of the software engineering team of the company. Yuichi is a caring and nurturing mentor. Yuichi likes to create a safe and supportive environment for their mentee. Yuichi is always looking for ways to help them grow and develop their skills.\\    Pablo is a perfectionist with great attention to detail. Pablo likes things to be done the right way and has a hard time delegating tasks to others. Pablo is critical of himself and of others. \\    The main programming language used in the company is Python.\\\\    
\#\# Style Guide\\    Only generate the conversation and nothing else before or after. Do not add numbers before each turn. Do not add quotes to the conversation turns. Use a professional and formal tone. The conversation flow should be natural and smooth. When switching topics, do it in a smooth way. There are no special characters between the turns. The conversations are dialogues and not narrations of events.\\    Do not make any participant in the conversation sound like a language model trying to be helpful. Make them sound as human as possible.\\    It is the mentor that leads the conversation. When Yuichi the mentor gives coding instructions, they do not provide examples. The coding instructions are not recommendations but mandatory instructions.\\\\
\midrule
\textbf{SESSION 1} \\
\midrule
Generate a short conversation between Yuichi and Pablo. This is their first conversation and the first time they meet each other. Yuichi wants Pablo to never use a whiteboard for brainstorming. Yuichi gives some specific coding instructions to Pablo. Yuichi wants Pablo to always end method names with '\_o'. Yuichi does not provide examples of correct code following the instructions. They do not show how to implement the instructions. Yuichi never says 'for example'. Yuichi does not give any other coding instructions. Pablo only acknowledges the instructions and does not ask any questions. \\
\midrule
\textbf{SESSION 2} \\
\midrule
Generate a medium-length conversation between Yuichi and Pablo. This is not their first conversation. They had 1 conversations before. They talk about hosting company-wide town hall meetings to share updates and foster transparency.  \\

\midrule
\textbf{SESSION 3} \\
\midrule
Generate a medium-length conversation between Yuichi and Pablo. This is not their first conversation. They had 2 conversations before. Yuichi is updating a previous information given to Pablo. Yuichi now wants Pablo to always use a whiteboard for brainstorming. Yuichi gives some specific coding instructions to Pablo. Yuichi wants Pablo to always include assert statements in functions. Yuichi is updating a previous information given to Pablo: Yuichi now wants Pablo to always end method names with '\_md'. Yuichi does not provide examples of correct code following the instructions. They do not show how to implement the instructions. Yuichi never says 'for example'. Yuichi does not give any other coding instructions. Pablo only acknowledges the instructions and does not ask any questions.  \\

\midrule
\textbf{SESSION 4} \\
\midrule
Generate a medium-length conversation between Yuichi and Pablo. This is not their first conversation. They had 3 conversations before. They talk about hosting company-wide town hall meetings to share updates and foster transparency. They had a previous conversation about this before. \\

\bottomrule
\caption{Prompts to generate dialogue history 108 which contains 4 sessions. }
\label{tab:prompt_conv_108}
\end{xltabular}
\twocolumn


\onecolumn
\begin{xltabular}{0.9\linewidth}{X}
\specialrule{1.2pt}{0pt}{0pt}
    \rowcolor{gray!20} \textbf{Dialogue history 70}   \\
    \specialrule{1.2pt}{0pt}{0pt}
\textbf{SESSION 1} \\
\midrule
\textbf{Jean-Aimé:} Lucas, it's a pleasure to finally meet you. I'm Jean-Aimé, your mentor here at VATO. I'm excited to work with you and help you develop your skills.\\ \textbf{Lucas:} Thank you, Jean-Aimé. I'm looking forward to learning from you. I've heard great things about your mentorship from other interns.\\ \textbf{Jean-Aimé:} Well, I believe that mentorship is a two-way street. I'm here to guide and support you, but I also want to create an environment where we can exchange ideas and feedback freely.\\ \textbf{Lucas:} Absolutely. Feedback is essential for growth. I'm open to constructive criticism and always looking for ways to improve.\\ \textbf{Jean-Aimé:} That's fantastic to hear, Lucas. At VATO, we foster a culture of continuous improvement. We believe that providing and receiving feedback is crucial for our development as professionals.\\ \textbf{Lucas:} I completely agree. So, how does the feedback process work here? Are there any specific procedures or guidelines?\\ \textbf{Jean-Aimé:} Great question. At VATO, we have a performance improvement plan that serves as a framework for our feedback conversations. It's a structured approach that helps us identify areas of strength and areas that need development.\\ \textbf{Lucas:} That sounds really helpful. Could you tell me more about how this performance improvement plan works?\\ \textbf{Jean-Aimé:} Certainly. The plan is tailored to each individual and their specific role. For software engineers like yourself, we focus on several key areas, including technical proficiency, problem-solving skills, and collaboration.\\ \textbf{Lucas:} That makes a lot of sense, considering the nature of our work.\\ \textbf{Jean-Aimé:} Exactly. We set clear goals and objectives, and then we work together to create an action plan to help you develop in those areas. It's a collaborative process, and your input is highly valued.\\ \textbf{Lucas:} I appreciate the emphasis on collaboration. It makes the growth process more engaging and meaningful. So, what's the first step in this performance improvement plan?\\ \textbf{Jean-Aimé:} The first step is a comprehensive skills assessment. This helps us establish a baseline understanding of your strengths and the areas you want to develop further. From there, we can tailor the plan to your specific needs and goals.\\ \textbf{Lucas:} That sounds like a great way to get started. I'm excited to dive into this process and see where it takes me.\\ \textbf{Jean-Aimé:} Your enthusiasm is contagious, Lucas! Remember, my door is always open if you have any questions or concerns along the way. Let's schedule our first skills assessment session for later this week and get the ball rolling!\\ \textbf{Lucas:} That sounds perfect! I'll make sure to prepare for it. Thank you for your guidance and support, Jean-Aimé.\\ \textbf{Jean-Aimé:} You're very welcome, Lucas. This is going to be a fantastic journey, and I can't wait to see your growth and development as a software engineer here at VATO. \\
\midrule
\textbf{SESSION 2} \\
\midrule

\textbf{Jean-Aimé:} Lucas, how are you finding your way around our code base? Any challenges you'd like to discuss?\\ \textbf{Lucas:} Hey, Jean-Aimé. I'm doing well, thanks. I've been exploring the code and I think I'm starting to get a hang of things. I have a question, though. Are there any specific coding conventions or standards that I should be following? I want to make sure my code aligns with the team's expectations.\\ \textbf{Jean-Aimé:} That's fantastic to hear, Lucas! Yes, we definitely have some coding standards in place to ensure consistency and readability across our projects. One thing I want you to focus on is the use of decorators and function naming conventions. \colorbox{SkyBlue}{Always add the '@timer\_class' decorator from the}  \colorbox{SkyBlue}{'pedantic' module to all your classes.} This helps us profile and optimize the code.\\ \textbf{Lucas:} Absolutely, I'll make sure to include the '@timer\_class' decorator from now on.\\ \textbf{Jean-Aimé:} Great!  \colorbox{SkyBlue}{Now, for function names, I'd like you to start them with 'x\_'.} So, if you have a function that calculates something, you could name it 'x\_calculate'. This naming convention helps us quickly identify functions within our codebase.\\ \textbf{Lucas:} Understood. I'll start implementing this naming convention right away. Are there any other standards or best practices I should be aware of?\\ \textbf{Jean-Aimé:} That covers the main points for now. Just remember, consistency is key. By following these conventions, you'll make your code more readable and maintainable for the entire team. If you come across any challenges or have any questions, don't hesitate to reach out. That's what I'm here for!\\ \textbf{Lucas:} Thanks for your guidance, Jean-Aimé. I really appreciate your support and mentorship. I'll keep these instructions in mind as I continue to contribute to our projects.\\ \textbf{Jean-Aimé:} You're very welcome, Lucas. I'm glad to help. Don't worry; I'll be checking in with you regularly to ensure you're on the right track and provide any additional guidance if needed. Now, go forth and code with confidence!\\ \textbf{Lucas:} Will do! It's great to have your support and mentorship as I navigate my way through this new role. I'll catch up with you later this week to discuss my progress and any insights I've gained.\\ \textbf{Jean-Aimé:} Sounds like a plan! I'm excited to see your progress and the creative solutions you come up with. Remember, if you have any "a-ha!" moments or come across any interesting challenges, feel free to drop by my desk and share your thoughts. Have a productive week!\\ \textbf{Lucas:} You too, Jean-Aimé! \\
\midrule
\textbf{SESSION 3} \\
\midrule
\textbf{Jean-Aimé:} Feedback is a crucial aspect of our work here at VATO, and I want to ensure that we foster an environment where we're constantly learning and growing. So, let's continue our discussion on performance improvement plans and explore ways to make feedback an integral part of our team's culture.\\ \textbf{Lucas:} Absolutely! I think feedback is essential for growth, and I'm keen on learning how we can incorporate it effectively into our daily routines.\\ \textbf{Jean-Aimé:} Fantastic! Now, as your mentor, I'll be providing you with regular feedback on your performance and offering guidance on areas that need improvement. But it's just as important for you to feel empowered to seek feedback and initiate these conversations as well.\\ \textbf{Lucas:} Definitely! I appreciate the feedback and guidance you've given me so far. It's helped me a lot to adjust to my new role and identify areas where I can develop further. I'm committed to improving and always looking for ways to grow.\\ \textbf{Jean-Aimé:} That's wonderful to hear, Lucas. Remember, feedback is a two-way street. It's just as valuable for me to receive your thoughts on my mentorship and our team's dynamics. Don't hold back on sharing your insights and suggestions. Now, regarding your coding tasks, I want to update a previous instruction: from now on, \colorbox{orange}{always start your function names with 'gn\_'.} This is to maintain consistency and improve code readability.\\ \textbf{Lucas:} Understood. I'll make sure to start my function names with 'gn\_' from now on.\\ \textbf{Jean-Aimé:} Great! Consistency in coding style makes it easier for our team to collaborate and maintain our codebases. Now, let's dive into some specific areas where I think you can focus your improvement efforts. Are you ready?\\ \textbf{Lucas:} Yes, I'm all ears. I'm eager to hear your feedback and work on my development plan.\\

\midrule
\textbf{EVALUATION QUERIES} \\
\midrule
Write a Quadratic discriminant analysis class. \\
Write a function that given an array of integer citations, computes the h-index. \\

\bottomrule
\caption{dialogue history 70 with 3 sessions and 3 pivots. Insertions of instructions are highlighted in \colorbox{SkyBlue}{blue} and updates highlighted in \colorbox{orange}{orange}.}
\label{tab:conv_70}
\end{xltabular}
\twocolumn

\onecolumn
\begingroup
\begin{xltabular}{0.9\linewidth}{X}
\specialrule{1.2pt}{0pt}{0pt}
    \rowcolor{gray!20} \textbf{Dialogue history 108}   \\
    \specialrule{1.2pt}{0pt}{0pt}
\textbf{SESSION 1} \\
\midrule
\textbf{Yuichi:} Pablo, it's great to finally meet you. I'm Yuichi, your mentor here at DEVS. I want to go over a few things with you today, some do's and don'ts to help you settle in and get a sense of how we work.\\ \textbf{Pablo:} Nice to meet you, Yuichi. I'm eager to learn and contribute to the team.\\ \textbf{Yuichi:} Great attitude. Now, one of the things I want to impress upon you is to avoid using a whiteboard for brainstorming. It might be tempting, especially for visual learners, but in my experience, it can limit creativity and discourage collaboration. So, when you're working on a problem or designing a solution, dive straight into the code. Speaking of code, I want you to adopt a specific coding style. \colorbox{SkyBlue}{Always end your method names with '\_o'.} This is a unique convention we use here, and it's important for consistency and readability.\\ \textbf{Pablo:} Understood. I'll keep that in mind and ensure I follow the convention moving forward.\\ \textbf{Yuichi:} Excellent. Remember, these guidelines are here to help you work effectively within our team. Don't hesitate to reach out if you have any questions or concerns. We're all here to support each other. \\
\midrule
\textbf{SESSION 2} \\
\midrule
\textbf{Yuichi:} Hi Pablo, how are you doing today?\\ \textbf{Pablo:} Good morning, Yuichi. I'm doing well, thank you. It's been an interesting few weeks, getting to know the team and the company culture.\\ \textbf{Yuichi:} I'm glad to hear that. Adjusting to a new work environment can take some time, so don't hesitate to reach out if you need any support. Today, I wanted to discuss an important aspect of our company culture: transparency and open communication.\\ \textbf{Pablo:} Yes, I've noticed that DEVS places a strong emphasis on transparency. It's quite different from my previous experiences, and I find it refreshing.\\ \textbf{Yuichi:} That's great to hear. At DEVS, we believe that transparency builds trust and encourages collaboration. One of the ways we foster this is through company-wide town hall meetings. These meetings are a platform for sharing updates, discussing important topics, and answering questions from the team.\\ \textbf{Pablo:} That sounds like a great initiative. How often do these town hall meetings take place?\\ \textbf{Yuichi:} We typically host them once a quarter, but sometimes we hold extra sessions if there are significant updates or changes. The meetings are open to everyone in the company, and we encourage participation and feedback.\\ \textbf{Pablo:} That's impressive. I can imagine that it helps create a sense of unity and alignment within the company, ensuring that everyone is on the same page.\\ \textbf{Yuichi:} Exactly. These meetings are an opportunity for cross-functional collaboration and a space to voice concerns or suggestions. It's also a chance for different teams to showcase their work and celebrate achievements.\\ \textbf{Pablo:} That leads to an interesting point. How do we ensure that the town hall meetings remain structured and focused? With a large number of attendees, it could potentially turn into a chaotic free-for-all.\\ \textbf{Yuichi:} That's a valid concern, and it's why we have a few guidelines in place. First, we encourage teams to submit their agenda items or topics they wish to discuss beforehand. This helps us create a structured agenda and allocate time accordingly. Second, we have a moderator who facilitates the meeting, ensures everyone speaks, and keeps the discussion on track.\\ \textbf{Pablo:} That makes sense. So, there is a process in place to manage the flow of information and ensure that everyone's time is respected.\\ \textbf{Yuichi:} Absolutely. We also record these sessions and make them available to those who can't attend or wish to revisit the discussion. Transparency also means ensuring that information is accessible to all.\\ \textbf{Pablo:} That's fantastic. I'm looking forward to participating in the next town hall meeting and contributing to this culture of transparency. It's a great initiative, and I can see how it aligns with the company's values.\\ \textbf{Yuichi:} I'm glad you're excited about it, Pablo. Your participation and fresh perspective will be valuable. Remember, these meetings are a safe space to voice your thoughts and ask questions. Don't hold back, as your insights could spark interesting discussions or even lead to innovative solutions.\\ \textbf{Pablo:} Thank you for the encouragement, Yuichi. I'll keep that in mind. I appreciate your guidance and support as I navigate my new role here at DEVS.\\ \textbf{Yuichi:} You're very welcome, Pablo. That's what mentors are for. Don't hesitate to reach out if you have any questions or concerns. We're all in this together, and your contributions are important. \\
\midrule
\textbf{SESSION 3} \\
\midrule

\textbf{Yuichi:} Pablo, I wanted to touch base with you and update some of the practices I'd like you to adopt going forward.\\ \textbf{Pablo:} Of course, Yuichi. I'm all ears.\\ \textbf{Yuichi:} Great! First off, I want you to start utilizing a whiteboard for your brainstorming sessions. It's a great way to visualize your thoughts and keep track of your ideas. So, make sure you always have one handy.\\ \textbf{Pablo:} Understood. I'll make sure to incorporate that into my workflow.\\ \textbf{Yuichi:} Excellent! Now, moving on to some coding specifics. \colorbox{SkyBlue}{I want you to focus on including assert} \colorbox{SkyBlue}{statements in your functions.} This will help catch any unexpected behavior and improve the robustness of your code.\\ \textbf{Pablo:} Assert statements, got it. I'll make sure to include them from now on.\\ \textbf{Yuichi:} Good. Also, I'd like you to update your naming convention for method names. Going forward, \colorbox{orange}{end all your method names with '\_md'.} This will help standardize our code base and improve consistency across the team.\\ \textbf{Pablo:} Okay. I'll keep that in mind and update my naming conventions accordingly.\\ \textbf{Yuichi:} Perfect! Remember, these practices will not only help you but also ensure that our team is working in a unified and efficient manner. Consistency is key in software engineering.\\ \textbf{Pablo:} Absolutely. I appreciate your guidance and mentorship, Yuichi. I'm taking all of this on board.\\ \textbf{Yuichi:} You're very welcome, Pablo. That's what I'm here for. Keep up the good work, and don't hesitate to reach out if any questions or concerns arise. We're a team, and your growth is important to all of us.\\ \textbf{Pablo:} Thank you, Yuichi. I will. \\

\midrule
\textbf{SESSION 4} \\
\midrule
\textbf{Yuichi:} Pablo, it's been a while since we last discussed the company-wide town hall meetings. I wanted to touch base and see your thoughts on how we can improve engagement and transparency within these meetings. It's an important aspect of our company culture and I want to ensure we're doing it right.\\ \textbf{Pablo:} Absolutely, Yuichi. I've been thinking about this a lot lately. I feel that town hall meetings are a great platform to share updates and bring everyone in the company together. To improve engagement, we could encourage more interactive elements. Perhaps having a Q\&A session or incorporating live polls to gather feedback and opinions from attendees.\\ \textbf{Yuichi:} Those are excellent ideas, Pablo. Interactive features will definitely make the meetings more dynamic and encourage participation. It's important that everyone feels involved and has a chance to voice their thoughts. We should also ensure that the meetings are structured but not too rigid. A balance of formalities and a relaxed atmosphere can make them more accessible and enjoyable.\\ \textbf{Pablo:} Exactly. Structuring the meetings with an agenda and time boundaries will help keep things focused. We can also explore utilizing collaboration tools to make the meetings more engaging.  These tools can add a layer of interactivity and make the meetings more fun.\\ \textbf{Yuichi:} That's a great suggestion about incorporating collaboration tools. They can really enhance the overall experience and make the meetings more modern and appealing to our audience. Additionally, we should ensure that the content shared during the meetings is transparent and honest. It's important that employees feel they are getting an authentic update on the company's progress and any challenges we may be facing.\\ \textbf{Pablo:} I completely agree. Transparency builds trust. We should encourage team leads and presenters to share honest updates, even if there are setbacks or challenges. It showcases authenticity and allows employees to feel more connected to the company's journey. I think it's also important to have a diverse range of presenters to represent the different teams and departments. \\ \textbf{Yuichi:} Absolutely, Pablo. Diversity and representation are key. We want to ensure that all employees feel included and that their voices are heard. By having a variety of presenters, we can provide a more holistic view of the company's operations and achievements. It also gives recognition to the hard work of individuals across the company.\\ \textbf{Pablo:} Indeed. And by recognizing the achievements of different teams, we can foster a sense of friendly competition, which might further drive innovation and engagement. I think these town hall meetings are a great opportunity to unite everyone towards a common goal and create a sense of community within our company. \\ \textbf{Yuichi:} Absolutely! It's all about building that community and fostering a sense of belonging. I'm glad we're on the same page with this, Pablo. Why don't you take the lead on organizing the next town hall meeting? You can start by creating a plan and gathering the necessary resources. Feel free to reach out if you need any guidance or support along the way.\\

\midrule
\textbf{EVALUATION QUERIES} \\
\midrule
Write a function that computes the average of the diagonal element of a matrix. \\
Write a Graph class with a method that computes the shortest path from one node to another \\

\bottomrule
\caption{dialogue history 108 with 4 sessions and 3 pivots. Insertion of instructions are highlighted in \colorbox{SkyBlue}{blue} and updates highlighted in \colorbox{orange}{orange}.}
\label{tab:conv_108}
\end{xltabular}
\begingroup

\twocolumn

\begin{table*}[t!]
\centering
\begin{tabularx}{0.9\linewidth}{X}
    \specialrule{1.2pt}{0pt}{0pt}
    \rowcolor{gray!20} \textbf{Instruction Prompt Example}   \\
    \specialrule{1.2pt}{0pt}{0pt}
    \textbf{SYSTEM PROMPT} \\
    \midrule
    \#\# Style Guide
    Do not acknowledge. Only generate Python code and nothing else before or after. Do not explain the code. Do not ask for more information but directly give the answer. \\
    \midrule
    \textbf{PROMPT} \\
    \midrule
    Write a function that converts an integer to Roman numerals. Do not provide example usage. Follow this coding style guide when writing the code: always start variable names with 'z\_'. \\
    \bottomrule
\end{tabularx}
\caption{Example of an Instruction prompt where the instruction is to start variable names with 'z\_'..} 
\label{tab:intruction_prompt_example}
\end{table*}

\begin{table*}[t!]
\centering
\begin{tabularx}{0.9\linewidth}{X}
    \specialrule{1.2pt}{0pt}{0pt}
        \rowcolor{gray!20} \textbf{History Prompt Example}   \\
        \specialrule{1.2pt}{0pt}{0pt}
    \textbf{SYSTEM PROMPT} \\
    \midrule 
    \#\# Task and Context
    You are Pablo, a new software engineer at DEVS. Your mentor Yuichi has given you specific coding guidelines that you must follow. \\
    \#\#  Style Guide
    Do not acknowledge. Only generate Python code and nothing else before or after. Do not explain the code. Do not ask for more information but directly give the answer. \\
    \midrule
    \textbf{PROMPT} \\
    \midrule
    This is a thread of conversations between you and your mentor Pablo:\\ 
    {[}dialogue{]} \\
    Based on information provided, write a function that converts an integer to Roman numerals. Do not provide example usage. You must follow all the latest coding guidelines provided by your mentor, including any possible updates. \\
    \bottomrule \\
\end{tabularx}
\caption{Example of a History prompt where {[}dialogue{]} is replaced by the entire dialogue history. Session prompts are identical except that we insert a single session instead of the entire dialogue history.} 
\label{tab:conversation_prompt_example}
\end{table*}

\begin{table*}[t!]
\centering
\begin{tabularx}{0.9\linewidth}{X}
    \specialrule{1.2pt}{0pt}{0pt}
        \rowcolor{gray!20} \textbf{Instructions-Chain Prompt Example}   \\
        \specialrule{1.2pt}{0pt}{0pt}
    \textbf{SYSTEM PROMPT} \\
    \midrule
    \#\# Style Guide
    Do not acknowledge. Only generate Python code and nothing else before or after. Do not explain the code. Do not ask for more information but directly give the answer. \\
    \midrule
    \textbf{PROMPT} \\
    \midrule
    This is a list of coding guidelines: always include a single digit in class names, always start variable names with 'z\_', always use docstrings in methods, always start variable names with 'wr\_', always use snake\_case for class names, always start variable names with 'vr\_', always include assert statements in functions, always start variable names with 'm\_', always start variable names with 'w\_', always start variable names with 'x\_', always end function argument names with '\_e', always add comments in your code, always end function argument names with 'a', always start variable names with 'n', always end function argument names with '\_g', always import the 'secrets' module even if it is not used. Some guidelines might have been updated. You must follow all the latest versions of the guidelines. Write a function that converts an integer to Roman numerals. Do not provide example usage. \\
    \bottomrule
\end{tabularx}
\caption{Example of a Instructions-Chain prompt with 16 instructions.} 
\label{tab:pivot_only}
\end{table*}




