%\documentclass[anon,12pt]{colt2025} % Anonymized submission
%\documentclass[final,12pt]{colt2025} % Include author names

\documentclass[letterpaper,11pt,]{article}

\usepackage[utf8]{inputenc}


\usepackage{amsmath, amssymb}
\usepackage{algpseudocode}
\usepackage{etoolbox}


%%% common
\usepackage{etex}
% fix whitespace problems with default verbatim environments
\usepackage{verbatim}
\usepackage{xspace,enumerate}
\usepackage[dvipsnames]{xcolor}
\usepackage[T1]{fontenc}
\usepackage[full]{textcomp}
\usepackage[american]{babel}
\usepackage{mathtools}
% fix for "too many math alphabets" problem
\newcommand\hmmax{0} % default 3
% load amsthm here already
\usepackage{amsthm}
%%% setup/geometry-paper
\usepackage[
letterpaper,
top=1in,
bottom=1in,
left=1in,
right=1in]{geometry}
%%% font/newpx
\usepackage{newpxtext} % T1, lining figures in math, osf in text
\usepackage{textcomp} % required for special glyphs
\usepackage[varg,bigdelims]{newpxmath}
\usepackage[scr=rsfso]{mathalfa}% \mathscr is fancier than \mathcal
\usepackage{bm} % load after all math to give access to bold math
\linespread{1.1}% Give Palatino more leading (space between lines)
\let\mathbb\varmathbb
\usepackage{microtype}
\usepackage[pagebackref,colorlinks=true,urlcolor=blue,linkcolor=blue,citecolor=OliveGreen]{hyperref}
%%% cleveref
\usepackage[capitalise,nameinlink]{cleveref}

\let\theoremref\cref
\let\definitionref\cref
\let\lemmaref\cref
\let\sectionref\cref

\crefname{lemma}{Lemma}{Lemmas}
\crefname{fact}{Fact}{Facts}
\crefname{theorem}{Theorem}{Theorems}
\crefname{corollary}{Corollary}{Corollaries}
\crefname{claim}{Claim}{Claims}
\crefname{example}{Example}{Examples}
\crefname{algorithm}{Algorithm}{Algorithms}
\crefname{problem}{Problem}{Problems}
\crefname{definition}{Definition}{Definitions}
\crefname{exercise}{Exercise}{Exercises}
\crefname{condition}{Condition}{Conditions}
\crefname{property}{Property}{Properties}


% \usepackage{paralist}
% %%% setup/frenchspacing
% % no additional space after periods
% \frenchspacing
% %%% setup/leftrightspacing
% \let\originalleft\left
% \let\originalright\right
% \renewcommand{\left}{\mathopen{}\mathclose\bgroup\originalleft}
% \renewcommand{\right}{\aftergroup\egroup\originalright}
% %%% setup/custompackages
% \usepackage{turnstile}
% \usepackage{mdframed}
% \usepackage{tikz}
% \usetikzlibrary{positioning}
% \usepackage{caption}
\usepackage{newfloat}
\usepackage{array}
\usepackage{subfig}
\usepackage{bbm}
%%% setup/paragraphperiod
\usepackage{xparse}
\usepackage{amsthm} % for \@addpunct

\newtheorem{exercise}{Exercise}[section]
\newtheorem{theorem}{Theorem}[section]
\newtheorem*{theorem*}{Theorem}
\newtheorem{lemma}[theorem]{Lemma}
\newtheorem*{lemma*}{Lemma}
\newtheorem{fact}[theorem]{Fact}
\newtheorem*{fact*}{Fact}
\newtheorem{proposition}[theorem]{Proposition}
\newtheorem*{proposition*}{Proposition}
\newtheorem{corollary}[theorem]{Corollary}
\newtheorem*{corollary*}{Corollary}
\newtheorem{hypothesis}[theorem]{Hypothesis}
\newtheorem*{hypothesis*}{Hypothesis}
\newtheorem{conjecture}[theorem]{Conjecture}
\newtheorem*{conjecture*}{Conjecture}
\theoremstyle{definition}
\newtheorem{definition}[theorem]{Definition}
\newtheorem*{definition*}{Definition}
\newtheorem{construction}[theorem]{Construction}
\newtheorem*{construction*}{Construction}
\newtheorem{example}[theorem]{Example}
\newtheorem*{example*}{Example}
\newtheorem{question}[theorem]{Question}
\newtheorem*{question*}{Question}
\newtheorem{algorithm}[theorem]{Algorithm}
\newtheorem*{algorithm*}{Algorithm}
\newtheorem{assumption}[theorem]{Assumption}
\newtheorem*{assumption*}{Assumption}
\newtheorem{problem}[theorem]{Problem}
\newtheorem*{problem*}{Problem}
\newtheorem{openquestion}[theorem]{Open Question}
\newtheorem*{openquestion*}{Open Question}
\theoremstyle{remark}
\newtheorem{claim}[theorem]{Claim}
\newtheorem*{claim*}{Claim}
\newtheorem{remark}[theorem]{Remark}
\newtheorem*{remark*}{Remark}
\newtheorem{observation}[theorem]{Observation}
\newtheorem*{observation*}{Observation}
%to remove above for colt


\usepackage{times}

% custom macros
\usepackage{mathtools}

\usepackage{bm} % load after all math to give access to bold math

%%% setup/leftrightspacing
\let\originalleft\left
\let\originalright\right
\renewcommand{\left}{\mathopen{}\mathclose\bgroup\originalleft}
\renewcommand{\right}{\aftergroup\egroup\originalright}

% %%% setup amsthm
% \newtheorem{exercise}{Exercise}[section]
% \newtheorem*{lemma*}{Lemma}
% \newtheorem*{fact*}{Fact}
% \newtheorem*{hypothesis*}{Hypothesis}
% \newtheorem*{definition*}{Definition}
% \newtheorem*{construction*}{Construction}
% %\newtheorem{remark}[theorem]{Remark}
% \newtheorem*{remark*}{Remark}
%\newtheorem{fact}[theorem]{Fact}
%%% setup/custompackages
\usepackage{turnstile}
\usepackage{mdframed}
\usepackage{tikz}
\usetikzlibrary{positioning}
\usepackage{caption}
%\DeclareCaptionType{Algorithm}
\usepackage{newfloat}
\usepackage{array}
%\usepackage{subfig}
\usepackage{bbm}
%%% setup/paragraphperiod
\usepackage{xparse}
\makeatletter
\let\latexparagraph\paragraph
\RenewDocumentCommand{\paragraph}{som}{%
	\IfBooleanTF{#1}
	{\latexparagraph*{#3}}
	{\IfNoValueTF{#2}
		{\latexparagraph{\maybe@addperiod{#3}}}
		{\latexparagraph[#2]{\maybe@addperiod{#3}}}%
	}%
}
\newcommand{\maybe@addperiod}[1]{%
	#1\@addpunct{.}%
}
\makeatother

% MACROS

%%% macro/authornotes
\newcommand{\Authornote}[2]{{\sffamily\small\color{red}{[#1: #2]}}}
\newcommand{\Authornotecolored}[3]{{\sffamily\small\color{#1}{[#2: #3]}}}
\newcommand{\Authorcomment}[2]{{\sffamily\small\color{gray}{[#1: #2]}}}
\newcommand{\Authorfnote}[2]{\footnote{\color{red}{#1: #2}}}
%%% macro/authornotes/authors
\newcommand{\Jnote}{\Authornote{J}}
\newcommand{\Jcomment}{\Authorcomment{J}}
\newcommand{\Jfnote}{\Authorfnote{J}}
\newcommand{\Tnote}{\Authornote{T}}
\newcommand{\Tcomment}{\Authorcomment{T}}
\newcommand{\Tfnote}{\Authorfnote{T}}
\newcommand{\Rnote}{\Authornote{R}}
\newcommand{\Rcomment}{\Authorcomment{R}}
\newcommand{\Rfnote}{\Authorfnote{R}}
\newcommand{\Dnote}{\Authornote{D}}
\newcommand{\Dcomment}{\Authorcomment{D}}
\newcommand{\Dfnote}{\Authorfnote{D}}
\newcommand{\Gnote}{\Authornote{G}}
\newcommand{\Gcomment}{\Authorcomment{G}}
\newcommand{\Gfnote}{\Authorfnote{G}}
%%% boxedminipage
\usepackage{boxedminipage}
% example:
% \center \noindent\begin{boxedminipage}{1.0\linewidth}}
% content
% \end{boxedminipage}
% \noindent
%%% parentheses
% various bracket-like commands
% round parentheses
\newcommand{\paren}[1]{(#1)}
\newcommand{\Paren}[1]{\left(#1\right)}
\newcommand{\bigparen}[1]{\big(#1\big)}
\newcommand{\Bigparen}[1]{\Big(#1\Big)}
% square brackets
\newcommand{\brac}[1]{[#1]}
\newcommand{\Brac}[1]{\left[#1\right]}
\newcommand{\bigbrac}[1]{\big[#1\big]}
\newcommand{\Bigbrac}[1]{\Big[#1\Big]}
\newcommand{\Biggbrac}[1]{\Bigg[#1\Bigg]}
% absolute value
\newcommand{\abs}[1]{\lvert#1\rvert}
\newcommand{\Abs}[1]{\left\lvert#1\right\rvert}
\newcommand{\bigabs}[1]{\big\lvert#1\big\rvert}
\newcommand{\Bigabs}[1]{\Big\lvert#1\Big\rvert}
% cardinality
\newcommand{\card}[1]{\lvert#1\rvert}
\newcommand{\Card}[1]{\left\lvert#1\right\rvert}
\newcommand{\bigcard}[1]{\big\lvert#1\big\rvert}
\newcommand{\Bigcard}[1]{\Big\lvert#1\Big\rvert}
% set
\newcommand{\set}[1]{\{#1\}}
\newcommand{\Set}[1]{\left\{#1\right\}}
\newcommand{\bigset}[1]{\big\{#1\big\}}
\newcommand{\Bigset}[1]{\Big\{#1\Big\}}
% norm
\newcommand{\norm}[1]{\lVert#1\rVert}
\newcommand{\Norm}[1]{\left\lVert#1\right\rVert}
\newcommand{\bignorm}[1]{\big\lVert#1\big\rVert}
\newcommand{\Bignorm}[1]{\Big\lVert#1\Big\rVert}
\newcommand{\Biggnorm}[1]{\Bigg\lVert#1\Bigg\rVert}
% 2-norm
\newcommand{\normt}[1]{\norm{#1}_2}
\newcommand{\Normt}[1]{\Norm{#1}_2}
\newcommand{\bignormt}[1]{\bignorm{#1}_2}
\newcommand{\Bignormt}[1]{\Bignorm{#1}_2}
% 2-norm squared
\newcommand{\snormt}[1]{\norm{#1}^2_2}
\newcommand{\Snormt}[1]{\Norm{#1}^2_2}
\newcommand{\bigsnormt}[1]{\bignorm{#1}^2_2}
\newcommand{\Bigsnormt}[1]{\Bignorm{#1}^2_2}
% norm squared
\newcommand{\snorm}[1]{\norm{#1}^2}
\newcommand{\Snorm}[1]{\Norm{#1}^2}
\newcommand{\bigsnorm}[1]{\bignorm{#1}^2}
\newcommand{\Bigsnorm}[1]{\Bignorm{#1}^2}
% 1-norm
\newcommand{\normo}[1]{\norm{#1}_1}
\newcommand{\Normo}[1]{\Norm{#1}_1}
\newcommand{\bignormo}[1]{\bignorm{#1}_1}
\newcommand{\Bignormo}[1]{\Bignorm{#1}_1}
% infty-norm
\newcommand{\normi}[1]{\norm{#1}_\infty}
\newcommand{\Normi}[1]{\Norm{#1}_\infty}
\newcommand{\bignormi}[1]{\bignorm{#1}_\infty}
\newcommand{\Bignormi}[1]{\Bignorm{#1}_\infty}
%k-sparse-norm
\newcommand{\normk}[1]{\norm{#1}_{k\text{-sparse}}}
\newcommand{\Normk}[1]{\Norm{#1}_{k\text{-sparse}}}
\newcommand{\bignormk}[1]{\bignorm{#1}_{k\text{-sparse}}}
\newcommand{\Bignormk}[1]{\Bignorm{#1}_{k\text{-sparse}}}
% inner product
\newcommand{\iprod}[1]{\langle#1\rangle}
\newcommand{\Iprod}[1]{\left\langle#1\right\rangle}
\newcommand{\bigiprod}[1]{\big\langle#1\big\rangle}
\newcommand{\Bigiprod}[1]{\Big\langle#1\Big\rangle}
%%% probability
% expectation, probability, variance
\newcommand{\Esymb}{\mathbb{E}}
\newcommand{\Psymb}{\mathbb{P}}
\newcommand{\Vsymb}{\mathbb{V}}
\newcommand{\Covsymb}{\mathrm{Cov}}
\DeclareMathOperator*{\E}{\Esymb}
\DeclareMathOperator*{\Var}{\Vsymb}
\DeclareMathOperator*{\ProbOp}{\Psymb}
\DeclareMathOperator*{\Cov}{\Covsymb}
\renewcommand{\Pr}{\ProbOp}
%%% middle
%\newcommand{\given}{\;\middle\vert\;}
\newcommand{\given}{\mathrel{}\middle\vert\mathrel{}}
%\newcommand{\given}{\mathrel{}\middle|\mathrel{}}
% {{{ miscmacros }}}
% middle delimiter in the definition of a set
\newcommand{\suchthat}{\;\middle\vert\;}
% tensor product
\newcommand{\tensor}{\otimes}
% add explanations to math displays
\newcommand{\where}{\text{where}}
\newcommand{\textparen}[1]{\text{(#1)}}
\newcommand{\using}[1]{\textparen{using #1}}
\newcommand{\smallusing}[1]{\text{(\small using #1)}}
\newcommand{\by}[1]{\textparen{by #1}}
% spectral order (Loewner order)
\newcommand{\sge}{\succeq}
\newcommand{\sle}{\preceq}
% smallest and largest eigenvalue
\newcommand{\lmin}{\lambda_{\min}}
\newcommand{\lmax}{\lambda_{\max}}
\newcommand{\signs}{\set{1,-1}}
\newcommand{\varsigns}{\set{\pm 1}}
\newcommand{\maximize}{\mathop{\textrm{maximize}}}
\newcommand{\minimize}{\mathop{\textrm{minimize}}}
\newcommand{\subjectto}{\mathop{\textrm{subject to}}}
\renewcommand{\ij}{{ij}}
% symmetric difference
\newcommand{\symdiff}{\Delta}
\newcommand{\varsymdiff}{\bigtriangleup}
% set of bits
\newcommand{\bits}{\{0,1\}}
\newcommand{\sbits}{\{\pm1\}}
% no stupid bullets for itemize environmentx
% \renewcommand{\labelitemi}{--}
% control white space of list and display environments
\newcommand{\listoptions}{\labelsep0mm\topsep-0mm\itemindent-6mm\itemsep0mm}
\newcommand{\displayoptions}[1]{\abovedisplayshortskip#1mm\belowdisplayshortskip#1mm\abovedisplayskip#1mm\belowdisplayskip#1mm}
% short for emptyset
%\newcommand{\eset}{\emptyset}
% moved to mathabbreviations
% short for epsilon
%\newcommand{\e}{\epsilon}
% moved to mathabbreviations
% super index with parentheses
\newcommand{\super}[2]{#1^{\paren{#2}}}
\newcommand{\varsuper}[2]{#1^{\scriptscriptstyle\paren{#2}}}
% tensor power notation
\newcommand{\tensorpower}[2]{#1^{\tensor #2}}
% multiplicative inverse
\newcommand{\inv}[1]{{#1^{-1}}}
% dual element
\newcommand{\dual}[1]{{#1^*}}
% subset
%\newcommand{\sse}{\subseteq}
% moved to mathabbreviations
% vertical space in math formula
\newcommand{\vbig}{\vphantom{\bigoplus}}
% setminus
\newcommand{\sm}{\setminus}
% define something by an equation (display)
\newcommand{\defeq}{\stackrel{\mathrm{def}}=}
% define something by an equation (inline)
\newcommand{\seteq}{\mathrel{\mathop:}=}
% declare function f by $f \from X \to Y$
\newcommand{\from}{\colon}
% big middle separator (for conditioning probability spaces)
\newcommand{\bigmid}{~\big|~}
\newcommand{\Bigmid}{~\Big|~}
\newcommand{\Mid}{\nonscript\;\middle\vert\nonscript\;}
% better vector definition and some variations
%\renewcommand{\vec}[1]{{\bm{#1}}}
\newcommand{\bvec}[1]{\bar{\vec{#1}}}
\newcommand{\pvec}[1]{\vec{#1}'}
\newcommand{\tvec}[1]{{\tilde{\vec{#1}}}}
\newcommand{\pE}{\tilde{\mathbb{E}}}
% punctuation at the end of a displayed formula
\newcommand{\mper}{\,.}
\newcommand{\mcom}{\,,}
% inner product for matrices
\newcommand\bdot\bullet
% transpose
\newcommand{\trsp}[1]{{#1}^\dagger}
% indicator function / vector
\DeclareMathOperator{\Ind}{\mathbf 1}
% place a qed symbol inside display formula
%\qedhere
% {{{ mathoperators }}}
\DeclareMathOperator{\Inf}{Inf}
\DeclareMathOperator{\Tr}{Tr}
%\newcommand{\Tr}{\mathrm{Tr}}
\DeclareMathOperator{\SDP}{SDP}
\DeclareMathOperator{\sdp}{sdp}
\DeclareMathOperator{\val}{val}
\DeclareMathOperator{\LP}{LP}
\DeclareMathOperator{\OPT}{OPT}
\DeclareMathOperator{\opt}{opt}
\DeclareMathOperator{\vol}{vol}
\DeclareMathOperator{\poly}{poly}
\DeclareMathOperator{\qpoly}{qpoly}
\DeclareMathOperator{\qpolylog}{qpolylog}
\DeclareMathOperator{\qqpoly}{qqpoly}
\DeclareMathOperator{\argmax}{argmax}
\DeclareMathOperator{\argmin}{argmin}
\DeclareMathOperator{\polylog}{polylog}
\DeclareMathOperator{\supp}{supp}
\DeclareMathOperator{\dist}{dist}
\DeclareMathOperator{\sign}{sign}
\DeclareMathOperator{\conv}{conv}
\DeclareMathOperator{\Conv}{Conv}
\DeclareMathOperator{\rank}{rank}
% operators with limits
\DeclareMathOperator*{\median}{median}
\DeclareMathOperator*{\Median}{Median}
% smaller summation/product symbols
\DeclareMathOperator*{\varsum}{{\textstyle \sum}}
\DeclareMathOperator{\tsum}{{\textstyle \sum}}
\let\varprod\undefined
\DeclareMathOperator*{\varprod}{{\textstyle \prod}}
\DeclareMathOperator{\tprod}{{\textstyle \prod}}
% {{{ textabbreviations }}}
% some abbreviations
\newcommand{\ie}{i.e.,\xspace}
\newcommand{\eg}{e.g.,\xspace}
\newcommand{\Eg}{E.g.,\xspace}
\newcommand{\phd}{Ph.\,D.\xspace}
\newcommand{\msc}{M.\,S.\xspace}
\newcommand{\bsc}{B.\,S.\xspace}
\newcommand{\etal}{et al.\xspace}
\newcommand{\iid}{i.i.d.\xspace}
% {{{ foreignwords }}}
\newcommand\naive{na\"{\i}ve\xspace}
\newcommand\Naive{Na\"{\i}ve\xspace}
\newcommand\naively{na\"{\i}vely\xspace}
\newcommand\Naively{Na\"{\i}vely\xspace}
% {{{ names }}}
% Hungarian/Polish/East European names
\newcommand{\Erdos}{Erd\H{o}s\xspace}
\newcommand{\Renyi}{R\'enyi\xspace}
\newcommand{\Lovasz}{Lov\'asz\xspace}
\newcommand{\Juhasz}{Juh\'asz\xspace}
\newcommand{\Bollobas}{Bollob\'as\xspace}
\newcommand{\Furedi}{F\"uredi\xspace}
\newcommand{\Komlos}{Koml\'os\xspace}
\newcommand{\Luczak}{\L uczak\xspace}
\newcommand{\Kucera}{Ku\v{c}era\xspace}
\newcommand{\Szemeredi}{Szemer\'edi\xspace}
\newcommand{\Hastad}{H{\aa}stad\xspace}
\newcommand{\Hoelder}{H\"{o}lder\xspace}
\newcommand{\Holder}{\Hoelder}
\newcommand{\Brandao}{Brand\~ao\xspace}
% {{{ numbersets }}}
% number sets
\newcommand{\Z}{\mathbb Z}
\newcommand{\N}{\mathbb N}
\newcommand{\R}{\mathbb R}
\newcommand{\C}{\mathbb C}
\newcommand{\Rnn}{\R_+}
\newcommand{\varR}{\Re}
\newcommand{\varRnn}{\varR_+}
\newcommand{\varvarRnn}{\R_{\ge 0}}
% {{{ problems }}}
% macros to denote computational problems
% use texorpdfstring to avoid problems with hyperref (can use problem
% macros also in headings
\newcommand{\problemmacro}[1]{\texorpdfstring{\textup{\textsc{#1}}}{#1}\xspace}
\newcommand{\pnum}[1]{{\footnotesize #1}}
% list of problems
\newcommand{\uniquegames}{\problemmacro{unique games}}
\newcommand{\maxcut}{\problemmacro{max cut}}
\newcommand{\multicut}{\problemmacro{multi cut}}
\newcommand{\vertexcover}{\problemmacro{vertex cover}}
\newcommand{\balancedseparator}{\problemmacro{balanced separator}}
\newcommand{\maxtwosat}{\problemmacro{max \pnum{3}-sat}}
\newcommand{\maxthreesat}{\problemmacro{max \pnum{3}-sat}}
\newcommand{\maxthreelin}{\problemmacro{max \pnum{3}-lin}}
\newcommand{\threesat}{\problemmacro{\pnum{3}-sat}}
\newcommand{\labelcover}{\problemmacro{label cover}}
\newcommand{\setcover}{\problemmacro{set cover}}
\newcommand{\maxksat}{\problemmacro{max $k$-sat}}
\newcommand{\mas}{\problemmacro{maximum acyclic subgraph}}
\newcommand{\kwaycut}{\problemmacro{$k$-way cut}}
\newcommand{\sparsestcut}{\problemmacro{sparsest cut}}
\newcommand{\betweenness}{\problemmacro{betweenness}}
\newcommand{\uniformsparsestcut}{\problemmacro{uniform sparsest cut}}
\newcommand{\grothendieckproblem}{\problemmacro{Grothendieck problem}}
\newcommand{\maxfoursat}{\problemmacro{max \pnum{4}-sat}}
\newcommand{\maxkcsp}{\problemmacro{max $k$-csp}}
\newcommand{\maxdicut}{\problemmacro{max dicut}}
\newcommand{\maxcutgain}{\problemmacro{max cut gain}}
\newcommand{\smallsetexpansion}{\problemmacro{small-set expansion}}
\newcommand{\minbisection}{\problemmacro{min bisection}}
\newcommand{\minimumlineararrangement}{\problemmacro{minimum linear arrangement}}
\newcommand{\maxtwolin}{\problemmacro{max \pnum{2}-lin}}
\newcommand{\gammamaxlin}{\problemmacro{$\Gamma$-max \pnum{2}-lin}}
\newcommand{\basicsdp}{\problemmacro{basic sdp}}
\newcommand{\dgames}{\problemmacro{$d$-to-1 games}}
\newcommand{\maxclique}{\problemmacro{max clique}}
\newcommand{\densestksubgraph}{\problemmacro{densest $k$-subgraph}}
% {{{ alphabet }}}
\newcommand{\cA}{\mathcal A}
\newcommand{\cB}{\mathcal B}
\newcommand{\cC}{\mathcal C}
\newcommand{\cD}{\mathcal D}
\newcommand{\cE}{\mathcal E}
\newcommand{\cF}{\mathcal F}
\newcommand{\cG}{\mathcal G}
\newcommand{\cH}{\mathcal H}
\newcommand{\cI}{\mathcal I}
\newcommand{\cJ}{\mathcal J}
\newcommand{\cK}{\mathcal K}
\newcommand{\cL}{\mathcal L}
\newcommand{\cM}{\mathcal M}
\newcommand{\cN}{\mathcal N}
\newcommand{\cO}{\mathcal O}
\newcommand{\cP}{\mathcal P}
\newcommand{\cQ}{\mathcal Q}
\newcommand{\cR}{\mathcal R}
\newcommand{\cS}{\mathcal S}
\newcommand{\cT}{\mathcal T}
\newcommand{\cU}{\mathcal U}
\newcommand{\cV}{\mathcal V}
\newcommand{\cW}{\mathcal W}
\newcommand{\cX}{\mathcal X}
\newcommand{\cY}{\mathcal Y}
\newcommand{\cZ}{\mathcal Z}
\newcommand{\scrA}{\mathscr A}
\newcommand{\scrB}{\mathscr B}
\newcommand{\scrC}{\mathscr C}
\newcommand{\scrD}{\mathscr D}
\newcommand{\scrE}{\mathscr E}
\newcommand{\scrF}{\mathscr F}
\newcommand{\scrG}{\mathscr G}
\newcommand{\scrH}{\mathscr H}
\newcommand{\scrI}{\mathscr I}
\newcommand{\scrJ}{\mathscr J}
\newcommand{\scrK}{\mathscr K}
\newcommand{\scrL}{\mathscr L}
\newcommand{\scrM}{\mathscr M}
\newcommand{\scrN}{\mathscr N}
\newcommand{\scrO}{\mathscr O}
\newcommand{\scrP}{\mathscr P}
\newcommand{\scrQ}{\mathscr Q}
\newcommand{\scrR}{\mathscr R}
\newcommand{\scrS}{\mathscr S}
\newcommand{\scrT}{\mathscr T}
\newcommand{\scrU}{\mathscr U}
\newcommand{\scrV}{\mathscr V}
\newcommand{\scrW}{\mathscr W}
\newcommand{\scrX}{\mathscr X}
\newcommand{\scrY}{\mathscr Y}
\newcommand{\scrZ}{\mathscr Z}
\newcommand{\bbB}{\mathbb B}
\newcommand{\bbS}{\mathbb S}
\newcommand{\bbR}{\mathbb R}
\newcommand{\bbZ}{\mathbb Z}
\newcommand{\bbI}{\mathbb I}
\newcommand{\bbQ}{\mathbb Q}
\newcommand{\bbP}{\mathbb P}
\newcommand{\bbE}{\mathbb E}
\newcommand{\bbN}{\mathbb N}
\newcommand{\sfE}{\mathsf E}
% {{{ leqslant }}}
% slanted lower/greater equal signs
\renewcommand{\leq}{\leqslant}
\renewcommand{\le}{\leqslant}
\renewcommand{\geq}{\geqslant}
\renewcommand{\ge}{\geqslant}
% {{{ varepsilon }}}
\let\epsilon=\varepsilon
%%% setup/equations
\numberwithin{equation}{section}
% {{{ restate }}}
% set of macros to deal with restating theorem environments (or anything
% else with a label)
% adapted from Boaz Barak
\newcommand\MYcurrentlabel{xxx}
% \MYstore{A}{B} assigns variable A value B
\newcommand{\MYstore}[2]{%
  \global\expandafter \def \csname MYMEMORY #1 \endcsname{#2}%
  %{#2}
}
% \MYload{A} outputs value stored for variable A
\newcommand{\MYload}[1]{%
  \csname MYMEMORY #1 \endcsname%
}
% new label command, stores current label in \MYcurrentlabel
\newcommand{\MYnewlabel}[1]{%
  \renewcommand\MYcurrentlabel{#1}%
  \MYoldlabel{#1}%
}
% new label command that doesn't do anything
\newcommand{\MYdummylabel}[1]{}
\newcommand{\torestate}[1]{%
  % overwrite label command
  \let\MYoldlabel\label%
  \let\label\MYnewlabel%
  #1%
  \MYstore{\MYcurrentlabel}{#1}%
  % restore old label command
  \let\label\MYoldlabel%
}
\newcommand{\restatetheorem}[1]{%
  % overwrite label command with dummy
  \let\MYoldlabel\label
  \let\label\MYdummylabel
  \begin{theorem*}[Restatement of \cref{#1}]
    \MYload{#1}
  \end{theorem*}
  \let\label\MYoldlabel
}
\newcommand{\restatelemma}[1]{%
  % overwrite label command with dummy
  \let\MYoldlabel\label
  \let\label\MYdummylabel
  \begin{lemma*}[Restatement of \cref{#1}]
    \MYload{#1}
  \end{lemma*}
  \let\label\MYoldlabel
}
\newcommand{\restateprop}[1]{%
  % overwrite label command with dummy
  \let\MYoldlabel\label
  \let\label\MYdummylabel
  \begin{proposition*}[Restatement of \cref{#1}]
    \MYload{#1}
  \end{proposition*}
  \let\label\MYoldlabel
}
\newcommand{\restatefact}[1]{%
  % overwrite label command with dummy
  \let\MYoldlabel\label
  \let\label\MYdummylabel
  \begin{fact*}[Restatement of \cref{#1}]
    \MYload{#1}
  \end{fact*}
  \let\label\MYoldlabel
}
\newcommand{\restate}[1]{%
  % overwrite label command with dummy
  \let\MYoldlabel\label
  \let\label\MYdummylabel
  \MYload{#1}
  \let\label\MYoldlabel
}
% {{{ mathabbreviations }}}
\newcommand{\la}{\leftarrow}
\newcommand{\sse}{\subseteq}
\newcommand{\ra}{\rightarrow}
\newcommand{\e}{\epsilon}
\newcommand{\eps}{\epsilon}
\newcommand{\eset}{\emptyset}
% {{{ allowdisplaybreaks }}}
% allows page breaks in large display math formulas
\allowdisplaybreaks
% {{{ sloppy }}}
% avoid math spilling on margin
\sloppy
\newcommand*{\Id}{\mathrm{Id}}
\newcommand*{\Lowner}{L\"owner\xspace}
\newcommand*{\normop}[1]{\norm{#1}_{\mathrm{op}}}
\newcommand*{\Normop}[1]{\Norm{#1}_{\mathrm{op}}}
\newcommand*{\Normtv}[1]{\Norm{#1}_{\mathrm{TV}}}
\newcommand*{\normtv}[1]{\norm{#1}_{\mathrm{TV}}}
\newcommand*{\normf}[1]{\norm{#1}_{\mathrm{F}}}
\newcommand*{\Normf}[1]{\Norm{#1}_{\mathrm{F}}}
% double square brackets
\newcommand{\bracbb}[1]{\llbracket#1\rrbracket}
\newcommand{\Bracbb}[1]{\left\llbracket#1\right\rrbracket}
% Shortcut for framing algorithms in a box
%\newenvironment{algorithmbox}{\begin{mdframed}[nobreak=true]
%\begin{algorithm}}{\end{algorithm}\end{mdframed}}
\newcommand{\bsaw}[2]{\text{BSAW}_{#1,#2}}
\newcommand{\nbsaw}[2]{\text{NBSAW}_{#1,#2}}
% nuclear-norm
\newcommand{\normn}[1]{\norm{#1}_\textnormal{nuc}}
\newcommand{\Normn}[1]{\Norm{#1}_\textnormal{nuc}}
\newcommand{\bignormn}[1]{\bignorm{#1}_\textnormal{nuc}}
\newcommand{\Bignormn}[1]{\Bignorm{#1}_\textnormal{nuc}}
% max-norm
\newcommand{\normm}[1]{\norm{#1}_\textnormal{max}}
\newcommand{\Normm}[1]{\Norm{#1}_\textnormal{max}}
\newcommand{\bignormm}[1]{\bignorm{#1}_\textnormal{max}}
\newcommand{\Bignormm}[1]{\Bignorm{#1}_\textnormal{max}}
% spikeness
\newcommand{\spike}[1]{\alpha_{\textnormal{sp}}\Paren{#1} }
% indicators
\newcommand{\ind}[1]{\mathbf{1}_{\Brac{#1}}}
% bbM
\newcommand{\bbM}{\mathbb M}
% Indentation Commands
\newcommand{\sskip}{\smallskip}
\newcommand{\bskip}{\bigskip}
% Matrix Entries
\providecommand{\Yij}{\mathbf{Y}_{ij}}
\providecommand{\Wij}{W_{ij}}
\providecommand{\Zij}{Z_{ij}}
\providecommand{\Xij}{X_{ij}}
\providecommand{\yij}{y_{ij}}
\providecommand{\wij}{w_{ij}}
\providecommand{\zij}{z_{ij}}
\providecommand{\xij}{x_{ij}}
% TODOs
\providecommand{\TODO}{{\color{red}{\textbf{TODO }}}}
\providecommand{\todo}{{\color{red}{\textbf{TODO }}}}
\providecommand{\tochange}{{\color{green}{\textbf{TOCHANGE }}}}
\providecommand{\toexpand}{{\color{blue}{\textbf{TOEXPAND }}}}
% Null and planted distribution
\providecommand{\nulld}{\textit{nl}}
\providecommand{\plantedd}{\textit{pl}}
\providecommand{\Ep}{\E_{\plantedd}}
\providecommand{\En}{\E_{\nulld}}
% walks
\newcommand{\saw}[2]{\text{SAW}^{#2}_{#1}}
\newcommand{\nbw}[2]{\text{NBW}^{#2}_{#1}}
% SBM distribution
\newcommand{\sbm}{\SBM_n(d,\e)}
\newcommand{\SBM}{\mathsf{SBM}}
\newcommand{\Esbm}{\E_{\sbm}}
\newcommand{\Q}{Q^{(s)}}
% Erdos-Renyi
\newcommand{\er}{\mathsf{G}(n, d/n)}
% infty-norm
%\renewcommand{\normi}[1]{\norm{#1}_{\max}}
%\renewcommand{\Normi}[1]{\Norm{#1}_{\max}}
\renewcommand{\bignormi}[1]{\bignorm{#1}_{\max}}
\renewcommand{\Bignormi}[1]{\Bignorm{#1}_{\max}}
\DeclareMathOperator{\diag}{diag}
\DeclareMathOperator{\Span}{Span}
\DeclareMathOperator{\sspan}{Span}
% injective tensor norms
\newcommand{\normin}[1]{\norm{#1}_{\text{inj}}}
\newcommand{\Normin}[1]{\Norm{#1}_{\text{inj}}}
\newcommand{\bignormin}[1]{\bignorm{#1}_{\text{inj}}}
\newcommand{\Bignormin}[1]{\Bignorm{#1}_{\text{inj}}}
% sos injective tensor norms
\newcommand{\normins}[2]{\norm{#1}_{{#2}\text{-inj}}}
\newcommand{\Normins}[2]{\Norm{#1}_{{#2}\text{-inj}}}
\newcommand{\bignormins}[2]{\bignorm{#1}_{{#2}\text{-inj}}}
\newcommand{\Bignormins}[2]{\Bignorm{#1}_{{#2}\text{-inj}}}
% sos tensor nuclear nrom
\newcommand{\normns}[2]{\norm{#1}_{{#2}\text{-nuc}}}
\newcommand{\Normns}[2]{\Norm{#1}_{{#2}\text{-nuc}}}
\newcommand{\bignornns}[2]{\bignorm{#1}_{{#2}\text{-nuc}}}
\newcommand{\Bignormns}[2]{\Bignorm{#1}_{{#2}\text{-nuc}}}

% \newcommand*{\tran}{^{\mkern-1.5mu\mathsf{T}}}
\newcommand*{\transpose}[1]{{#1}{}^{\mkern-1.5mu\mathsf{T}}}
\newcommand*{\dyad}[1]{#1#1{}^{\mkern-1.5mu\mathsf{T}}}

\newcommand{\todoArg}[1]{\textcolor{red}{\textbf{ToDo: #1}}}

% Error function (used for missclassification error)
\newcommand{\err}[1]{\mathrm{err}(#1)}

% Distortion
\newcommand{\distrtf}[1]{\Delta_{2,4}\Paren{#1}}


% Additional commands
\providecommand{\pdsetv}{\tilde{\Omega}_{b, p,  \cD_t}}
\providecommand{\pdset}{\tilde{\Omega}}%_{\overline{\Omega}_{b}, p,  \cD_t}}

\renewcommand{\arraystretch}{2}

\newcommand{\simiid}{\stackrel{\text{iid}}\sim}


\newcommand{\Sgood}{S_{\text{good}}}
\newcommand{\Sbad}{S_{\text{bad}}}

\newcommand{\CS}{Cauchy--Schwarz }
\newcommand{\holders}{H\"older's}

\newcommand{\huberloss}{\Phi}
\newcommand{\huberderivative}{\phi}
\newcommand{\huberparameter}{h}
\newcommand{\lossfunction}{\cL}
\newcommand{\noisetransformation}{f}
\newcommand{\noisepotential}{F}
\newcommand{\lossHuber}{\lossfunction_H}
\newcommand{\lossQuadratic}{\lossfunction_Q}
\newcommand{\lossElliptical}{\lossfunction_E}
\newcommand{\gradHuber}{\noisetransformation_H}
\newcommand{\gradQuadratic}{\noisetransformation_Q}

\newcommand{\betahat}{\hat{\bm \beta}}
\newcommand{\betastar}{\beta^*}
\newcommand{\muhat}{\hat{\mu}}
\newcommand{\mustar}{\mu^*}
\newcommand{\mutilde}{\tilde{\mu}}
\newcommand{\mustartilde}{\tilde{\mu}^*}
\newcommand{\Sigmaf}{\Sigma_{\noisetransformation}}
\newcommand{\Sigmahatf}{\hat{\Sigma}_{\noisetransformation}}
\newcommand{\Sigmatildef}{\tilde{\Sigma}_{\noisetransformation}}
\newcommand{\Sigmastartilde}{\tilde{\Sigma}^*}
\newcommand{\gtilde}{\tilde{g}}
\newcommand{\gtildef}{\tilde{g}_{\noisetransformation}}

\newcommand{\varianceparameter}{\gamma}

\newcommand{\effrank}{\mathrm{erk}}
\newcommand{\spsign}{\textrm{spsign}}



%\title[Robust Scatter Matrix Estimation of Elliptical Distributions in Polynomial Time]{Robust Scatter Matrix Estimation of Elliptical Distributions in Polynomial Time}
\usepackage{times}
% Use \Name{Author Name} to specify the name.
% If the surname contains spaces, enclose the surname
% in braces, e.g. \Name{John {Smith Jones}} similarly
% if the name has a "von" part, e.g \Name{Jane {de Winter}}.
% If the first letter in the forenames is a diacritic
% enclose the diacritic in braces, e.g. \Name{{\'E}louise Smith}

% Two authors with the same address
% \coltauthor{\Name{Author Name1} \Email{abc@sample.com}\and
%  \Name{Author Name2} \Email{xyz@sample.com}\\
%  \addr Address}

% Three or more authors with the same address:
% \coltauthor{\Name{Author Name1} \Email{an1@sample.com}\\
%  \Name{Author Name2} \Email{an2@sample.com}\\
%  \Name{Author Name3} \Email{an3@sample.com}\\
%  \addr Address}

% Authors with different addresses:
%\coltauthor{%
% \Name{Gleb Novikov} \Email{gleb.novikov@hslu.ch}\\
% \addr Lucerne School of Computer Science and Information %Technology
%}


\title{Nearly Optimal Robust Covariance and Scatter Matrix Estimation Beyond Gaussians}

%\date{}

\author{Gleb Novikov\thanks{Lucerne School of Computer Science and Information Technology}}


\begin{document}

\clearpage\maketitle
\thispagestyle{empty}

\begin{abstract}%
We study the problem of \emph{computationally efficient} robust estimation of the covariance/scatter matrix of elliptical distributions---that is, affine transformations of spherically symmetric distributions---under the \emph{strong contamination model} in the high-dimensional regime $d \gtrsim 1/\varepsilon^2$, where $d$ is the dimension and $\varepsilon$ is the fraction of adversarial corruptions. We show that the structure inherent to elliptical distributions enables us to achieve estimation guarantees comparable to those known for the Gaussian case.

Concretely, we propose an algorithm that, under a {very mild assumption} on the scatter matrix $\Sigma$, and given a nearly optimal number of samples $n = \tilde{O}(d^2/\varepsilon^2)$, computes in polynomial time an estimator $\hat{\Sigma}$ such that, with high probability,
\[
\left\| \Sigma^{-1/2} \hat{\Sigma} \Sigma^{-1/2} - \Id \right\|_{\text F} \le O(\varepsilon \log(1/\varepsilon))\,.
\]
This matches the best known guarantees for robust covariance estimation in the Gaussian setting.

As an application of our result, we obtain the \emph{first efficiently computable, nearly optimal robust covariance estimators} that extend beyond the Gaussian case. Specifically, for elliptical distributions satisfying the Hanson--Wright concentration inequality (including, for example, Gaussians and uniform distributions over ellipsoids), our estimator $\hat{\Sigma}$ of the covariance $\Sigma$ achieves the same error guarantee as in the Gaussian case:
\[
\left\| \Sigma^{-1/2} \hat{\Sigma} \Sigma^{-1/2} - \Id \right\|_{\text F} \le O(\varepsilon \log(1/\varepsilon))\,.
\]
Moreover, for elliptical distributions with sub-exponential tails (such as the multivariate Laplace distribution), we construct an estimator $\hat{\Sigma}$ satisfying the spectral norm bound
\[
\left\| \Sigma^{-1/2} \hat{\Sigma} \Sigma^{-1/2} - \Id \right\| \le O(\varepsilon \log(1/\varepsilon))\,.
\]
Remarkably, despite the heavier tails of such distributions, the covariance can still be estimated at the same rate as in the Gaussian case---a phenomenon unique to high dimensions and absent in low-dimensional settings.


Our approach is based on estimating the covariance of the \emph{spatial sign} (i.e., the projection onto the sphere) of elliptical distributions. The estimation proceeds in several stages, one of which involves a novel \emph{spectral covariance filtering algorithm}. This algorithm combines  covariance filtering techniques with degree-4 sum-of-squares relaxations, and we believe it may be of independent interest for future applications.

\end{abstract}


\section{Introduction}
\label{sec:introduction}
The business processes of organizations are experiencing ever-increasing complexity due to the large amount of data, high number of users, and high-tech devices involved \cite{martin2021pmopportunitieschallenges, beerepoot2023biggestbpmproblems}. This complexity may cause business processes to deviate from normal control flow due to unforeseen and disruptive anomalies \cite{adams2023proceddsriftdetection}. These control-flow anomalies manifest as unknown, skipped, and wrongly-ordered activities in the traces of event logs monitored from the execution of business processes \cite{ko2023adsystematicreview}. For the sake of clarity, let us consider an illustrative example of such anomalies. Figure \ref{FP_ANOMALIES} shows a so-called event log footprint, which captures the control flow relations of four activities of a hypothetical event log. In particular, this footprint captures the control-flow relations between activities \texttt{a}, \texttt{b}, \texttt{c} and \texttt{d}. These are the causal ($\rightarrow$) relation, concurrent ($\parallel$) relation, and other ($\#$) relations such as exclusivity or non-local dependency \cite{aalst2022pmhandbook}. In addition, on the right are six traces, of which five exhibit skipped, wrongly-ordered and unknown control-flow anomalies. For example, $\langle$\texttt{a b d}$\rangle$ has a skipped activity, which is \texttt{c}. Because of this skipped activity, the control-flow relation \texttt{b}$\,\#\,$\texttt{d} is violated, since \texttt{d} directly follows \texttt{b} in the anomalous trace.
\begin{figure}[!t]
\centering
\includegraphics[width=0.9\columnwidth]{images/FP_ANOMALIES.png}
\caption{An example event log footprint with six traces, of which five exhibit control-flow anomalies.}
\label{FP_ANOMALIES}
\end{figure}

\subsection{Control-flow anomaly detection}
Control-flow anomaly detection techniques aim to characterize the normal control flow from event logs and verify whether these deviations occur in new event logs \cite{ko2023adsystematicreview}. To develop control-flow anomaly detection techniques, \revision{process mining} has seen widespread adoption owing to process discovery and \revision{conformance checking}. On the one hand, process discovery is a set of algorithms that encode control-flow relations as a set of model elements and constraints according to a given modeling formalism \cite{aalst2022pmhandbook}; hereafter, we refer to the Petri net, a widespread modeling formalism. On the other hand, \revision{conformance checking} is an explainable set of algorithms that allows linking any deviations with the reference Petri net and providing the fitness measure, namely a measure of how much the Petri net fits the new event log \cite{aalst2022pmhandbook}. Many control-flow anomaly detection techniques based on \revision{conformance checking} (hereafter, \revision{conformance checking}-based techniques) use the fitness measure to determine whether an event log is anomalous \cite{bezerra2009pmad, bezerra2013adlogspais, myers2018icsadpm, pecchia2020applicationfailuresanalysispm}. 

The scientific literature also includes many \revision{conformance checking}-independent techniques for control-flow anomaly detection that combine specific types of trace encodings with machine/deep learning \cite{ko2023adsystematicreview, tavares2023pmtraceencoding}. Whereas these techniques are very effective, their explainability is challenging due to both the type of trace encoding employed and the machine/deep learning model used \cite{rawal2022trustworthyaiadvances,li2023explainablead}. Hence, in the following, we focus on the shortcomings of \revision{conformance checking}-based techniques to investigate whether it is possible to support the development of competitive control-flow anomaly detection techniques while maintaining the explainable nature of \revision{conformance checking}.
\begin{figure}[!t]
\centering
\includegraphics[width=\columnwidth]{images/HIGH_LEVEL_VIEW.png}
\caption{A high-level view of the proposed framework for combining \revision{process mining}-based feature extraction with dimensionality reduction for control-flow anomaly detection.}
\label{HIGH_LEVEL_VIEW}
\end{figure}

\subsection{Shortcomings of \revision{conformance checking}-based techniques}
Unfortunately, the detection effectiveness of \revision{conformance checking}-based techniques is affected by noisy data and low-quality Petri nets, which may be due to human errors in the modeling process or representational bias of process discovery algorithms \cite{bezerra2013adlogspais, pecchia2020applicationfailuresanalysispm, aalst2016pm}. Specifically, on the one hand, noisy data may introduce infrequent and deceptive control-flow relations that may result in inconsistent fitness measures, whereas, on the other hand, checking event logs against a low-quality Petri net could lead to an unreliable distribution of fitness measures. Nonetheless, such Petri nets can still be used as references to obtain insightful information for \revision{process mining}-based feature extraction, supporting the development of competitive and explainable \revision{conformance checking}-based techniques for control-flow anomaly detection despite the problems above. For example, a few works outline that token-based \revision{conformance checking} can be used for \revision{process mining}-based feature extraction to build tabular data and develop effective \revision{conformance checking}-based techniques for control-flow anomaly detection \cite{singh2022lapmsh, debenedictis2023dtadiiot}. However, to the best of our knowledge, the scientific literature lacks a structured proposal for \revision{process mining}-based feature extraction using the state-of-the-art \revision{conformance checking} variant, namely alignment-based \revision{conformance checking}.

\subsection{Contributions}
We propose a novel \revision{process mining}-based feature extraction approach with alignment-based \revision{conformance checking}. This variant aligns the deviating control flow with a reference Petri net; the resulting alignment can be inspected to extract additional statistics such as the number of times a given activity caused mismatches \cite{aalst2022pmhandbook}. We integrate this approach into a flexible and explainable framework for developing techniques for control-flow anomaly detection. The framework combines \revision{process mining}-based feature extraction and dimensionality reduction to handle high-dimensional feature sets, achieve detection effectiveness, and support explainability. Notably, in addition to our proposed \revision{process mining}-based feature extraction approach, the framework allows employing other approaches, enabling a fair comparison of multiple \revision{conformance checking}-based and \revision{conformance checking}-independent techniques for control-flow anomaly detection. Figure \ref{HIGH_LEVEL_VIEW} shows a high-level view of the framework. Business processes are monitored, and event logs obtained from the database of information systems. Subsequently, \revision{process mining}-based feature extraction is applied to these event logs and tabular data input to dimensionality reduction to identify control-flow anomalies. We apply several \revision{conformance checking}-based and \revision{conformance checking}-independent framework techniques to publicly available datasets, simulated data of a case study from railways, and real-world data of a case study from healthcare. We show that the framework techniques implementing our approach outperform the baseline \revision{conformance checking}-based techniques while maintaining the explainable nature of \revision{conformance checking}.

In summary, the contributions of this paper are as follows.
\begin{itemize}
    \item{
        A novel \revision{process mining}-based feature extraction approach to support the development of competitive and explainable \revision{conformance checking}-based techniques for control-flow anomaly detection.
    }
    \item{
        A flexible and explainable framework for developing techniques for control-flow anomaly detection using \revision{process mining}-based feature extraction and dimensionality reduction.
    }
    \item{
        Application to synthetic and real-world datasets of several \revision{conformance checking}-based and \revision{conformance checking}-independent framework techniques, evaluating their detection effectiveness and explainability.
    }
\end{itemize}

The rest of the paper is organized as follows.
\begin{itemize}
    \item Section \ref{sec:related_work} reviews the existing techniques for control-flow anomaly detection, categorizing them into \revision{conformance checking}-based and \revision{conformance checking}-independent techniques.
    \item Section \ref{sec:abccfe} provides the preliminaries of \revision{process mining} to establish the notation used throughout the paper, and delves into the details of the proposed \revision{process mining}-based feature extraction approach with alignment-based \revision{conformance checking}.
    \item Section \ref{sec:framework} describes the framework for developing \revision{conformance checking}-based and \revision{conformance checking}-independent techniques for control-flow anomaly detection that combine \revision{process mining}-based feature extraction and dimensionality reduction.
    \item Section \ref{sec:evaluation} presents the experiments conducted with multiple framework and baseline techniques using data from publicly available datasets and case studies.
    \item Section \ref{sec:conclusions} draws the conclusions and presents future work.
\end{itemize}

\begin{table*}[t]
\centering
\fontsize{11pt}{11pt}\selectfont
\begin{tabular}{lllllllllllll}
\toprule
\multicolumn{1}{c}{\textbf{task}} & \multicolumn{2}{c}{\textbf{Mir}} & \multicolumn{2}{c}{\textbf{Lai}} & \multicolumn{2}{c}{\textbf{Ziegen.}} & \multicolumn{2}{c}{\textbf{Cao}} & \multicolumn{2}{c}{\textbf{Alva-Man.}} & \multicolumn{1}{c}{\textbf{avg.}} & \textbf{\begin{tabular}[c]{@{}l@{}}avg.\\ rank\end{tabular}} \\
\multicolumn{1}{c}{\textbf{metrics}} & \multicolumn{1}{c}{\textbf{cor.}} & \multicolumn{1}{c}{\textbf{p-v.}} & \multicolumn{1}{c}{\textbf{cor.}} & \multicolumn{1}{c}{\textbf{p-v.}} & \multicolumn{1}{c}{\textbf{cor.}} & \multicolumn{1}{c}{\textbf{p-v.}} & \multicolumn{1}{c}{\textbf{cor.}} & \multicolumn{1}{c}{\textbf{p-v.}} & \multicolumn{1}{c}{\textbf{cor.}} & \multicolumn{1}{c}{\textbf{p-v.}} &  &  \\ \midrule
\textbf{S-Bleu} & 0.50 & 0.0 & 0.47 & 0.0 & 0.59 & 0.0 & 0.58 & 0.0 & 0.68 & 0.0 & 0.57 & 5.8 \\
\textbf{R-Bleu} & -- & -- & 0.27 & 0.0 & 0.30 & 0.0 & -- & -- & -- & -- & - &  \\
\textbf{S-Meteor} & 0.49 & 0.0 & 0.48 & 0.0 & 0.61 & 0.0 & 0.57 & 0.0 & 0.64 & 0.0 & 0.56 & 6.1 \\
\textbf{R-Meteor} & -- & -- & 0.34 & 0.0 & 0.26 & 0.0 & -- & -- & -- & -- & - &  \\
\textbf{S-Bertscore} & \textbf{0.53} & 0.0 & {\ul 0.80} & 0.0 & \textbf{0.70} & 0.0 & {\ul 0.66} & 0.0 & {\ul0.78} & 0.0 & \textbf{0.69} & \textbf{1.7} \\
\textbf{R-Bertscore} & -- & -- & 0.51 & 0.0 & 0.38 & 0.0 & -- & -- & -- & -- & - &  \\
\textbf{S-Bleurt} & {\ul 0.52} & 0.0 & {\ul 0.80} & 0.0 & 0.60 & 0.0 & \textbf{0.70} & 0.0 & \textbf{0.80} & 0.0 & {\ul 0.68} & {\ul 2.3} \\
\textbf{R-Bleurt} & -- & -- & 0.59 & 0.0 & -0.05 & 0.13 & -- & -- & -- & -- & - &  \\
\textbf{S-Cosine} & 0.51 & 0.0 & 0.69 & 0.0 & {\ul 0.62} & 0.0 & 0.61 & 0.0 & 0.65 & 0.0 & 0.62 & 4.4 \\
\textbf{R-Cosine} & -- & -- & 0.40 & 0.0 & 0.29 & 0.0 & -- & -- & -- & -- & - & \\ \midrule
\textbf{QuestEval} & 0.23 & 0.0 & 0.25 & 0.0 & 0.49 & 0.0 & 0.47 & 0.0 & 0.62 & 0.0 & 0.41 & 9.0 \\
\textbf{LLaMa3} & 0.36 & 0.0 & \textbf{0.84} & 0.0 & {\ul{0.62}} & 0.0 & 0.61 & 0.0 &  0.76 & 0.0 & 0.64 & 3.6 \\
\textbf{our (3b)} & 0.49 & 0.0 & 0.73 & 0.0 & 0.54 & 0.0 & 0.53 & 0.0 & 0.7 & 0.0 & 0.60 & 5.8 \\
\textbf{our (8b)} & 0.48 & 0.0 & 0.73 & 0.0 & 0.52 & 0.0 & 0.53 & 0.0 & 0.7 & 0.0 & 0.59 & 6.3 \\  \bottomrule
\end{tabular}
\caption{Pearson correlation on human evaluation on system output. `R-': reference-based. `S-': source-based.}
\label{tab:sys}
\end{table*}



\begin{table}%[]
\centering
\fontsize{11pt}{11pt}\selectfont
\begin{tabular}{llllll}
\toprule
\multicolumn{1}{c}{\textbf{task}} & \multicolumn{1}{c}{\textbf{Lai}} & \multicolumn{1}{c}{\textbf{Zei.}} & \multicolumn{1}{c}{\textbf{Scia.}} & \textbf{} & \textbf{} \\ 
\multicolumn{1}{c}{\textbf{metrics}} & \multicolumn{1}{c}{\textbf{cor.}} & \multicolumn{1}{c}{\textbf{cor.}} & \multicolumn{1}{c}{\textbf{cor.}} & \textbf{avg.} & \textbf{\begin{tabular}[c]{@{}l@{}}avg.\\ rank\end{tabular}} \\ \midrule
\textbf{S-Bleu} & 0.40 & 0.40 & 0.19* & 0.33 & 7.67 \\
\textbf{S-Meteor} & 0.41 & 0.42 & 0.16* & 0.33 & 7.33 \\
\textbf{S-BertS.} & {\ul0.58} & 0.47 & 0.31 & 0.45 & 3.67 \\
\textbf{S-Bleurt} & 0.45 & {\ul 0.54} & {\ul 0.37} & 0.45 & {\ul 3.33} \\
\textbf{S-Cosine} & 0.56 & 0.52 & 0.3 & {\ul 0.46} & {\ul 3.33} \\ \midrule
\textbf{QuestE.} & 0.27 & 0.35 & 0.06* & 0.23 & 9.00 \\
\textbf{LlaMA3} & \textbf{0.6} & \textbf{0.67} & \textbf{0.51} & \textbf{0.59} & \textbf{1.0} \\
\textbf{Our (3b)} & 0.51 & 0.49 & 0.23* & 0.39 & 4.83 \\
\textbf{Our (8b)} & 0.52 & 0.49 & 0.22* & 0.43 & 4.83 \\ \bottomrule
\end{tabular}
\caption{Pearson correlation on human ratings on reference output. *not significant; we cannot reject the null hypothesis of zero correlation}
\label{tab:ref}
\end{table}


\begin{table*}%[]
\centering
\fontsize{11pt}{11pt}\selectfont
\begin{tabular}{lllllllll}
\toprule
\textbf{task} & \multicolumn{1}{c}{\textbf{ALL}} & \multicolumn{1}{c}{\textbf{sentiment}} & \multicolumn{1}{c}{\textbf{detoxify}} & \multicolumn{1}{c}{\textbf{catchy}} & \multicolumn{1}{c}{\textbf{polite}} & \multicolumn{1}{c}{\textbf{persuasive}} & \multicolumn{1}{c}{\textbf{formal}} & \textbf{\begin{tabular}[c]{@{}l@{}}avg. \\ rank\end{tabular}} \\
\textbf{metrics} & \multicolumn{1}{c}{\textbf{cor.}} & \multicolumn{1}{c}{\textbf{cor.}} & \multicolumn{1}{c}{\textbf{cor.}} & \multicolumn{1}{c}{\textbf{cor.}} & \multicolumn{1}{c}{\textbf{cor.}} & \multicolumn{1}{c}{\textbf{cor.}} & \multicolumn{1}{c}{\textbf{cor.}} &  \\ \midrule
\textbf{S-Bleu} & -0.17 & -0.82 & -0.45 & -0.12* & -0.1* & -0.05 & -0.21 & 8.42 \\
\textbf{R-Bleu} & - & -0.5 & -0.45 &  &  &  &  &  \\
\textbf{S-Meteor} & -0.07* & -0.55 & -0.4 & -0.01* & 0.1* & -0.16 & -0.04* & 7.67 \\
\textbf{R-Meteor} & - & -0.17* & -0.39 & - & - & - & - & - \\
\textbf{S-BertScore} & 0.11 & -0.38 & -0.07* & -0.17* & 0.28 & 0.12 & 0.25 & 6.0 \\
\textbf{R-BertScore} & - & -0.02* & -0.21* & - & - & - & - & - \\
\textbf{S-Bleurt} & 0.29 & 0.05* & 0.45 & 0.06* & 0.29 & 0.23 & 0.46 & 4.2 \\
\textbf{R-Bleurt} & - &  0.21 & 0.38 & - & - & - & - & - \\
\textbf{S-Cosine} & 0.01* & -0.5 & -0.13* & -0.19* & 0.05* & -0.05* & 0.15* & 7.42 \\
\textbf{R-Cosine} & - & -0.11* & -0.16* & - & - & - & - & - \\ \midrule
\textbf{QuestEval} & 0.21 & {\ul{0.29}} & 0.23 & 0.37 & 0.19* & 0.35 & 0.14* & 4.67 \\
\textbf{LlaMA3} & \textbf{0.82} & \textbf{0.80} & \textbf{0.72} & \textbf{0.84} & \textbf{0.84} & \textbf{0.90} & \textbf{0.88} & \textbf{1.00} \\
\textbf{Our (3b)} & 0.47 & -0.11* & 0.37 & 0.61 & 0.53 & 0.54 & 0.66 & 3.5 \\
\textbf{Our (8b)} & {\ul{0.57}} & 0.09* & {\ul 0.49} & {\ul 0.72} & {\ul 0.64} & {\ul 0.62} & {\ul 0.67} & {\ul 2.17} \\ \bottomrule
\end{tabular}
\caption{Pearson correlation on human ratings on our constructed test set. 'R-': reference-based. 'S-': source-based. *not significant; we cannot reject the null hypothesis of zero correlation}
\label{tab:con}
\end{table*}

\section{Results}
We benchmark the different metrics on the different datasets using correlation to human judgement. For content preservation, we show results split on data with system output, reference output and our constructed test set: we show that the data source for evaluation leads to different conclusions on the metrics. In addition, we examine whether the metrics can rank style transfer systems similar to humans. On style strength, we likewise show correlations between human judgment and zero-shot evaluation approaches. When applicable, we summarize results by reporting the average correlation. And the average ranking of the metric per dataset (by ranking which metric obtains the highest correlation to human judgement per dataset). 

\subsection{Content preservation}
\paragraph{How do data sources affect the conclusion on best metric?}
The conclusions about the metrics' performance change radically depending on whether we use system output data, reference output, or our constructed test set. Ideally, a good metric correlates highly with humans on any data source. Ideally, for meta-evaluation, a metric should correlate consistently across all data sources, but the following shows that the correlations indicate different things, and the conclusion on the best metric should be drawn carefully.

Looking at the metrics correlations with humans on the data source with system output (Table~\ref{tab:sys}), we see a relatively high correlation for many of the metrics on many tasks. The overall best metrics are S-BertScore and S-BLEURT (avg+avg rank). We see no notable difference in our method of using the 3B or 8B model as the backbone.

Examining the average correlations based on data with reference output (Table~\ref{tab:ref}), now the zero-shoot prompting with LlaMA3 70B is the best-performing approach ($0.59$ avg). Tied for second place are source-based cosine embedding ($0.46$ avg), BLEURT ($0.45$ avg) and BertScore ($0.45$ avg). Our method follows on a 5. place: here, the 8b version (($0.43$ avg)) shows a bit stronger results than 3b ($0.39$ avg). The fact that the conclusions change, whether looking at reference or system output, confirms the observations made by \citet{scialom-etal-2021-questeval} on simplicity transfer.   

Now consider the results on our test set (Table~\ref{tab:con}): Several metrics show low or no correlation; we even see a significantly negative correlation for some metrics on ALL (BLEU) and for specific subparts of our test set for BLEU, Meteor, BertScore, Cosine. On the other end, LlaMA3 70B is again performing best, showing strong results ($0.82$ in ALL). The runner-up is now our 8B method, with a gap to the 3B version ($0.57$ vs $0.47$ in ALL). Note our method still shows zero correlation for the sentiment task. After, ranks BLEURT ($0.29$), QuestEval ($0.21$), BertScore ($0.11$), Cosine ($0.01$).  

On our test set, we find that some metrics that correlate relatively well on the other datasets, now exhibit low correlation. Hence, with our test set, we can now support the logical reasoning with data evidence: Evaluation of content preservation for style transfer needs to take the style shift into account. This conclusion could not be drawn using the existing data sources: We hypothesise that for the data with system-based output, successful output happens to be very similar to the source sentence and vice versa, and reference-based output might not contain server mistakes as they are gold references. Thus, none of the existing data sources tests the limits of the metrics.  


\paragraph{How do reference-based metrics compare to source-based ones?} Reference-based metrics show a lower correlation than the source-based counterpart for all metrics on both datasets with ratings on references (Table~\ref{tab:sys}). As discussed previously, reference-based metrics for style transfer have the drawback that many different good solutions on a rewrite might exist and not only one similar to a reference.


\paragraph{How well can the metrics rank the performance of style transfer methods?}
We compare the metrics' ability to judge the best style transfer methods w.r.t. the human annotations: Several of the data sources contain samples from different style transfer systems. In order to use metrics to assess the quality of the style transfer system, metrics should correctly find the best-performing system. Hence, we evaluate whether the metrics for content preservation provide the same system ranking as human evaluators. We take the mean of the score for every output on each system and the mean of the human annotations; we compare the systems using the Kendall's Tau correlation. 

We find only the evaluation using the dataset Mir, Lai, and Ziegen to result in significant correlations, probably because of sparsity in a number of system tests (App.~\ref{app:dataset}). Our method (8b) is the only metric providing a perfect ranking of the style transfer system on the Lai data, and Llama3 70B the only one on the Ziegen data. Results in App.~\ref{app:results}. 


\subsection{Style strength results}
%Evaluating style strengths is a challenging task. 
Llama3 70B shows better overall results than our method. However, our method scores higher than Llama3 70B on 2 out of 6 datasets, but it also exhibits zero correlation on one task (Table~\ref{tab:styleresults}).%More work i s needed on evaluating style strengths. 
 
\begin{table}%[]
\fontsize{11pt}{11pt}\selectfont
\begin{tabular}{lccc}
\toprule
\multicolumn{1}{c}{\textbf{}} & \textbf{LlaMA3} & \textbf{Our (3b)} & \textbf{Our (8b)} \\ \midrule
\textbf{Mir} & 0.46 & 0.54 & \textbf{0.57} \\
\textbf{Lai} & \textbf{0.57} & 0.18 & 0.19 \\
\textbf{Ziegen.} & 0.25 & 0.27 & \textbf{0.32} \\
\textbf{Alva-M.} & \textbf{0.59} & 0.03* & 0.02* \\
\textbf{Scialom} & \textbf{0.62} & 0.45 & 0.44 \\
\textbf{\begin{tabular}[c]{@{}l@{}}Our Test\end{tabular}} & \textbf{0.63} & 0.46 & 0.48 \\ \bottomrule
\end{tabular}
\caption{Style strength: Pearson correlation to human ratings. *not significant; we cannot reject the null hypothesis of zero corelation}
\label{tab:styleresults}
\end{table}

\subsection{Ablation}
We conduct several runs of the methods using LLMs with variations in instructions/prompts (App.~\ref{app:method}). We observe that the lower the correlation on a task, the higher the variation between the different runs. For our method, we only observe low variance between the runs.
None of the variations leads to different conclusions of the meta-evaluation. Results in App.~\ref{app:results}.
\section{Semantic Equivalence Based Program Clustering}
\label{sec:symexclustering}

The NLG techniques proposed by Kuhn~\etal~\cite{kuhnsemantic} and Abbasi~\etal~\cite{abbasi2024believe} rely on semantic clustering, where semantically equivalent programs are grouped together. 
Achieving this requires an effective method for assessing program equivalence. Kuhn~\etal employ the DeBERTa-large model~\cite{he2020deberta} for this task, while Abbasi~\etal determine equivalence using an F1 score based on token inclusion~\cite{DBLP:journals/corr/JoshiCWZ17}.

In the domain of code generation, program equivalence has a precise definition: two programs are considered equivalent if they produce identical behavior for all possible inputs. 
Consequently, a domain-specific equivalence check is required.
In this paper, we base the semantic equivalence check on \emph{symbolic execution}, where, instead of executing a program with concrete inputs, \emph{symbolic variables} are used to represent inputs, generating constraints that describe the program's behavior across all possible input values~\cite{symex_klee}.

The particular flavor of symbolic execution we use in this work is inspired by the lightweight \emph{peer architecture} described in Bruni~\etal~\cite{Bruni2011APA}. 
Unlike traditional approaches that require building a standalone symbolic interpreter, this architecture embeds the symbolic execution engine as a lightweight library operating alongside the target program. 
Their design is based on the insight that languages that provide the ability to dynamically dispatch primitive operations (\eg Python) allow symbolic values to behave as native values and be tracked at runtime.

Symbolic execution typically traverses the program's control flow graph, maintaining a symbolic state consisting of \emph{path constraints} (\ie logical conditions that must be satisfied for a given execution path to be feasible) and \emph{symbolic expressions} (\ie representations of program variables as functions of the symbolic inputs). 

\paragraph{Equivalence check.} Given two code snippets \(s^{(1)}\) and \(s^{(2)}\), we check semantic equivalence between them by comparing their symbolic traces. 
One such symbolic trace, \eg \(T(s^{(1)})\), consists of the corresponding path constraint and symbolic expressions denoting all the variables encountered on the corresponding execution path. 
Intuitively, for two code snippets to be semantically equivalent, all their corresponding symbolic traces must align. 
Specifically, for each path constraint, the traces produced by both snippets must be identical meaning that there is no concrete counterexample input for which the execution of the two snippets diverges.

%\CD{I'm not sure whether it's worth formalising this a bit.}

%Semantic equivalence between two code snippets \(s^{(i)}\) and \(s^{(j)}\) is determined by comparing their symbolic traces, \(T(s^{(i)})\) and \(T(s^{(j)})\), respectively:
%\begin{multline}
%    T(s^{(i)}) \equiv T(s^{(j)}) \iff \text{Path Constraints and Symbolic Expressions of } \\
%    s^{(i)} \text{ and } s^{(j)} \text{ are identical.}
%\end{multline}

Since program equivalence is undecidable in general, we perform a bounded equivalence check. 
This approach verifies that no counterexample input exists when exploring traces up to a given depth.

%Exact equivalence is often \emph{undecidable} due to the complexity of symbolic traces. 
%Instead, we employ \emph{subsumption}, where one trace subsumes another if all behaviours of the latter are captured by the former. 
%This allows us to approximate equivalence effectively.

%By using this lightweight symbolic execution approach, our clustering methodology emphasizes functional semantics, avoiding overfitting to syntactic similarities. 
%This methodology strikes a balance between precision and efficiency, leveraging the extensibility and simplicity of the peer architecture to scale across diverse programming scenarios.


\begin{algorithm}[ht!]
    \caption{Clustering with Symbolic Execution}
    \label{alg:clustering}
    \begin{algorithmic}[1]
    \Require Set of generated code snippets $\{s^{(1)}, \ldots, s^{(M)}\}$
    \Ensure Clusters of semantically equivalent snippets $C = \{c_1, c_2, \ldots, c_k\}$
    
    \State Initialize an empty cluster set $C \gets \emptyset$, and an equivalence map $E \gets \emptyset$ \label{alg:clustering:init}
    
    \For{each snippet $s^{(i)}$}
        \If{$s^{(i)}$ is invalid}
            \State $E[s^{(i)}] \gets \{\,s^{(i)}\}$ 
            \Comment{Assign invalid snippet to its own equivalence class}
        \EndIf
    \EndFor
    
    \For{each pair of valid snippets $(s^{(i)}, s^{(j)})$} \label{alg:clustering:pairwise}
        \State Perform symbolic execution on $s^{(i)}$ and $s^{(j)}$ to extract traces $T(s^{(i)})$ and $T(s^{(j)})$ \label{alg:clustering:trace}
        \If{$T(s^{(i)}) \equiv T(s^{(j)})$} \label{alg:clustering:check}
            \State $E[s^{(i)}] \gets E[s^{(i)}] \cup \{\,s^{(j)}\}$
            \State $E[s^{(j)}] \gets E[s^{(j)}] \cup \{\,s^{(i)}\}$ \label{alg:clustering:update}
        \EndIf
        \State \Comment{Enforce transitivity of equivalences}
        \If{$s^{(i)} \sim s^{(j)}$ and $s^{(j)} \sim s^{(k)}$ for some $s^{(k)}$}
            \State $E[s^{(i)}] \gets E[s^{(i)}] \cup \{\,s^{(k)}\}$
            \State $E[s^{(k)}] \gets E[s^{(k)}] \cup \{\,s^{(i)}\}$
        \EndIf
    \EndFor
    
    \State Identify equivalence classes in $E$ to form final clusters $C$ \label{alg:clustering:extract}
    
    \State \Return $C$ \label{alg:clustering:return}
    \end{algorithmic}
    \end{algorithm}

%\CD{we need to modify the alg so that it's obvious that we are talking about sets of traces at line 8.}

\paragraph{Semantic clustering.}
Algorithm~\ref{alg:clustering} illustrates how to cluster code snippets based on their functional semantics, with an additional check for invalid snippets. 
We first create empty structures for storing the final clusters ($C$) and an equivalence map ($E$) to track relationships (line~\ref{alg:clustering:init}). 

Next, in the \emph{invalid snippet handling phase}, each code snippet $s^{(i)}$ is examined and if it is detected to be invalid, it is immediately placed in its own equivalence class in $E$ and is thus isolated from further consideration. 

In the \emph{pairwise comparison phase} (line~\ref{alg:clustering:pairwise}), each pair of \emph{valid} snippets $(s^{(i)}, s^{(j)})$ is symbolically executed to produce traces $T(s^{(i)})$ and $T(s^{(j)})$ (line~\ref{alg:clustering:trace}). 
If the traces are equivalent (line~\ref{alg:clustering:check}), indicating identical functional behavior, both snippets are added to each other's equivalence classes (line~\ref{alg:clustering:update}). 
In reality, $T(s^{(i)})$ and $T(s^{(j)})$ actually denote sets of traces, and the equivalence check involves comparing individual traces from each set that share the same path constraint.
For brevity, in Algorithm~\ref{alg:clustering}, we represent this as $T(s^{(i)}) \equiv T(s^{(j)})$.
To maintain consistency, transitivity is enforced: if $s^{(i)}$ is equivalent to $s^{(j)}$, and $s^{(j)}$ is equivalent to $s^{(k)}$, then $s^{(i)}$ must also be in the same equivalence class as $s^{(k)}$. 

Finally, the equivalence map $E$ is processed to derive the clusters themselves (line~\ref{alg:clustering:extract}), and the resulting set of clusters is returned (line~\ref{alg:clustering:return}). 
% By isolating invalid snippets in single-item clusters, the algorithm cleanly separates non-functional or syntactically invalid code from semantically consistent groups. 
% This ensures that the final clustering reflects the functional semantics of valid snippets while transparently segregating invalid code. 

\section{Estimating the Probability Distribution of LLM Responses}
\label{sec:probcomp}

Both NLG techniques we adapt for code generation, at certain points, query the LLM, sample responses along with the log-probabilities of their tokens, and apply a softmax-style normalization to interpret them as a valid probability distribution.
However, since the LLM responses in this setting are programs, they are longer than the natural language responses used in the original studies---while the evaluation for Kuhn~\etal and Abbasi~\etal considered question-answer datasets typically involving one word answers, the programs produced in this work are around 200 tokens per response. 

The probability of a response is represented as the joint probability of its tokens, meaning that it decreases exponentially with length, often leading to \emph{numerical underflow}. 
This ultimately compromises the effectiveness of the technique. 
For instance, if the probabilities for all response programs underflow, then softmax returns NaN, which then propagates through the computation.

To address the issue of exponentially decaying probabilities, we propose two methods for approximating the probability distribution of LLM responses, as outlined below.

\paragraph{Length normalization.}

%The techniques we propose require computing the probability of responses generated by language models, where the probability of a response is represented as the joint probability of its tokens. 
%However, for longer responses, this probability decreases exponentially with length, which adversely impacts our estimation of uncertainty.

%In NLG tasks such as those addressed by prior works~\cite{kuhnsemantic,abbasi2024believe}, this issue is less pronounced because the goal in their problem domains (\eg TriviaQA) is to exactly match short reference answers. 
%In contrast, for code generation, the outputs are often longer. 
%While the evaluation for Kuhn~\etal~\cite{kuhnsemantic} and Abbasi~\etal~\cite{abbasi2024believe} considered question-answer datasets while typically involve one word answers, the program snippets produced in this work were typically around 200 tokens per response. 
% \CD{For our experiments, the average number of tokens in the results generated by the LLM is ...}. 
%Notably, while it is true that accuracy tends to decrease with length, existing research demonstrates that LLMs can generate high-quality code snippets, even up to 100 lines~\cite{codetranslation2}. %Consequently, the drastic reduction in probability with length disproportionately affects these scenarios. 

One solution is to use \emph{length normalization}~\cite{DBLP:conf/wmt/MurrayC18,DBLP:conf/aclnmt/KoehnK17}, more precisely length-normalizing the log probability of a program, a technique  used by other existing works, \eg to compute length-normalized predictive entropy~\cite{DBLP:conf/iclr/MalininG21}. %This also allows compar uncertainties of sequences of different length

More concretely, to compute a length-normalized probability from log probabilities, we begin by calculating the sum of the log probabilities. 
Let \(\ell_1, \ell_2, \dots, \ell_n\) denote the log probabilities associated with each token in the sequence that forms the response.
The log probability of the response is given by:
$S = \sum_{i=1}^{n} \ell_i$.
%
Next, the log probability is normalized by the sequence length \(L\) and the normalizing factor \(\gamma\). 
The normalized log probability is computed as:
$\ell_{\text{norm}} = \frac{S}{L \cdot \gamma}$.
%
Finally, the normalized probability \(P\) of the response is obtained by exponentiating the normalized log probability:
$P = e^{\ell_{\text{norm}}}$.

Intuitively, when using length-normalization in the context of uncertainty computation, probabilities remain comparable across responses of different lengths, whereas uncertainty is linked to the semantic differences between responses. %In our experimental evaluation, we compute uncertainty measures both with length-normalization and without.

\paragraph{Uniform distribution of LLM-generated responses.}
Intuitively, when multiple LLM responses are semantically equivalent, it indicates a higher degree of certainty in the (semantics of the) generated output.
To test this intuition, we propose disregarding the log-probabilities reported by the LLM and instead assuming a uniform distribution over the sampled responses. 
Specifically, if we sample $n$ responses, each response is assigned an equal probability of $1/n$.

Our experimental results demonstrated that using the semantic equivalence-based approach (described in Section~\ref{sec:symex}) in conjunction with this distribution reveals a negative correlation between the LLM's uncertainty and correctness---see Section~\ref{sec:results-discussion}. 
Furthermore, applying an uncertainty threshold derived from this technique to filter LLM responses—allowing only those above a specified correctness score (measured as the percentage of passed unit tests)—leads to high accuracy---see Section~\ref{sec:usability}.


%This approach ensures that the probabilities are appropriately scaled with respect to the sequence length.


%%For illustration, consider below as an example response produced by \gptturbo for the prompt used in Figure~\ref{fig:sampleproblem} from \S\ref{sec:motivating}:
%% % ``\texttt{Write a Python function that counts how many people older than 60 appear in a data list.}''

%% \begin{lstlisting}[language=Python]
%%     def candidate1(details):
%%      count = 0
%%         for detail in details:
%%             age = int(detail[11:13])
%%             if age > 60:
%%                 count += 1
%%         return count
%% \end{lstlisting}

%% For illustration, let us consider a code snippet generated by the LLM of $N=20$ tokens. ,  each with an individual token probability of $0.7$. 

%% \CD{Can we actually find out the number of tokens and log probabilities?}
%% When working with language models, each token in a generated completion has an associated log probability.  
%% 

%% Then if we simply sum the log probabilities of each token in a generated response as shown earlier:
%% \[
%%    \sum_{i=1}^{20} \log(0.7)
%%    \;=\;
%%    20 \,\log(0.7)
%%    \;\approx\;
%%    -7.1335,
%% \]
%% and exponentiate this sum yields:
%% \[
%%    \exp(-7.1335)
%%    \;\approx\;
%%    0.0008.
%% \]


%% Moreover, The probabilities of individual responses tend to decay exponentially with length, which can lead to disproportionately low values for valid but lengthy outputs.

%% If we merely sum these log probabilities over $N$ tokens,



%% Hence, although $0.7$ per token is fairly high, the \emph{total} probability from multiplying all 20 token probabilities becomes quite small ($0.0008$). 
%% A shorter completion, having fewer tokens, might end up with a larger total probability even if its average token confidence is slightly lower. 
%% This demonstrates how \emph{lengthier responses} can be unfairly penalized if we only sum or multiply all token probabilities, hence motivating \textbf{length-based normalisation}.


%% Length-normalization also helps when the responses generated for a query have different lengths.




% To address this, we perform a length-based normalisation to ensure that probabilities remain comparable across responses of different lengths. Specifically, for a generated snippet \(s\), its normalized probability is defined as:
% \begin{equation}
%     \tilde{p}(s \mid x) = \frac{p(s \mid x)}{|s|^\alpha},
% \end{equation}
% where \(|s|\) is the length of the snippet \(s\), and \(\alpha\) is a hyperparameter that controls the degree of normalisation. This adjustment prevents the probabilities of longer responses from dominating or vanishing entirely, ensuring a fair representation in subsequent entropy computations.

%% When working with language models, each token in a generated completion has an associated log probability.  
%% If we merely sum these log probabilities over $N$ tokens,
%% \[
%%    \text{sum\_logprob} \;=\; \sum_{i=1}^{N} \log \bigl(p(\mathrm{token}_i)\bigr),
%% \]
%% longer completions tend to accumulate more negative values simply due to having more tokens. 
%% This can make them appear less likely, even if each token is reasonably probable.

%% To address this, we perform a length-based normalisation to ensure that probabilities remain comparable across responses of different lengths. Specifically, for a generated snippet \(s\), its normalized probability is defined as:
%% \begin{equation}
%%     \tilde{p}(s \mid x) = \frac{p(s \mid x)}{|s|^\alpha},
%% \end{equation}
%% where \(|s|\) is the length of the snippet \(s\), and \(\alpha\) is a hyperparameter that controls the degree of normalisation. 
%% This adjustment prevents the probabilities of longer responses from dominating or vanishing entirely, ensuring a fair representation in subsequent entropy computations.

% To mitigate this effect, we use \emph{length-based normalisation}.
% Instead of summing the log probabilities, we compute the \textbf{average} log probability per token:
% \[
%    \text{avg\_logprob} \;=\; \frac{1}{N} \; \sum_{i=1}^{N} \log \bigl(p(\mathrm{token}_i)\bigr).
% \]
% We then exponentiate this average to obtain a \textbf{length-normalized probability}:
% \[
%    \text{length\_normalized\_prob} \;=\;
%    \exp\!\bigl(\text{avg\_logprob}\bigr).
% \]
% Although this value is not a ``true'' probability for the entire sequence, it is 
% a fairer score for comparing completions of different lengths, 
% because a longer completion is not automatically penalized by virtue of having more tokens.


%% Hence, instead of relying on the total product of probabilities, we use the 
%% \textbf{average log probability} per token:
%% \[
%%    \text{avg\_logprob} 
%%    \;=\;
%%    \frac{1}{N} 
%%    \sum_{i=1}^{N} \log\bigl(p(\mathrm{token}_i)\bigr)
%%    \;=\;
%%    \log(0.7),
%% \]
%% for $N=20$ tokens in this simplified scenario. 
%% Exponentiating that average log probability gives:
%% \[
%%    \exp\!\bigl(\text{avg\_logprob}\bigr)
%%    \;=\;
%%    \exp(\log(0.7))
%%    \;=\;
%%    0.7.
%% \]

%% Thus, while the raw product \(\prod_{i=1}^{20} p(\mathrm{token}_i)\) is about $0.0008$, the \emph{length-normalized} probability is $0.7$, reflecting the fairer notion that each token has a $70\%$ likelihood on average. This avoids unfairly penalizing longer responses merely due to having more tokens multiplied together. 
%% Length-based normalisation is therefore crucial for comparing or ranking completions of different lengths.

% A different completion with more lines of code and docstrings might sum to a lower total log probability (due to more tokens), but its average log probability could be similar (say, $0.95$). When using length-based normalisation, these two scores ($0.97$ vs.\ $0.95$) are directly comparable as per-token likelihoods, rather than an apples-to-oranges comparison of total log probabilities.


% If you want to interpret these normalized scores as a \emph{distribution} over completions, 
% you can renormalize across all candidates so that they sum to 1:
% \[
%   \hat{p}_i \;=\; 
%   \frac{\exp\!\bigl(\text{avg\_logprob}_i\bigr)}{\sum_{j=1}^{k} \exp\!\bigl(\text{avg\_logprob}_j\bigr)},
% \]
% where $k$ is the total number of candidate completions. 
% This transforms the length-normalized scores into valid probabilities for comparing or sampling.

\section{Semantic Uncertainty via Symbolic Clustering}
\label{sec:symex}
This section presents our adaptation of the semantic entropy-based approach by Kuhn~\etal~\cite{kuhnsemantic} for code generation. 
We follow the main steps from the original work while diverging in two key aspects: the way we estimate the distribution of generated LLM responses and the clustering methodology.

%This section presents our methodology for assessing semantic uncertainty in code generation by leveraging clustering based on symbolic execution. 

\subsubsection{Generation}
The first step involves sampling $M$ code snippets (using the same hyperparameters as Kuhn~\etal~\cite{kuhnsemantic}), $\{s^{(1)}, \ldots, s^{(M)}\}$, from the LLM's output distribution $p(s \mid x)$ for a given prompt $x$. 
%Given a code generation prompt, the model generates $M$ samples, $\{s^{(1)}, \ldots, s^{(M)}\}$, from its predictive distribution $p(s \mid x)$.
% Sampling is carried out using multinomial techniques, with hyperparameters such as temperature and nucleus sampling selected based on those used by Kuhn~\etal~\cite{kuhnsemantic}.
The probabilities of the collected samples are processed using a softmax-style normalization function, ensuring that the resulting values can be interpreted as a valid probability distribution.

As explained in Section~\ref{sec:probcomp}, this process can lead to numerical underflows. 
To mitigate this, we approximate the probability distribution of the LLM responses using either length-normalization or a uniform distribution.
%
Following this approximation, let \(\tilde{p}(s \mid x)\) denote the probability of a snippet \(s\) according to the adjusted distribution.

%softmax normalized
%Then, we have: 
 %\[
 %\tilde{p}(s \mid x) = \frac{p(s \mid x)}{|s|^\alpha},
 %\]
 %where \(|s|\) is the length of the snippet \(s\), and \(\alpha\) is a hyperparameter controlling the degree of normalization. 

%\CD{fix softmax normalization}

%This is similar to the approach taken by Kuhn~\etal~\cite{kuhnsemantic} and Abbasi~\etal~\cite{abbasi2024believe} and is a necessary step to prevent the probability computations from becoming invalid within the various formulae.
% \CD{Do we use these optional techniques?}

\subsubsection{Clustering via Symbolic Execution}
The second step works by grouping the aforementioned snippets into clusters based on semantic equivalence. %, determined through symbolic execution traces.
%To determine semantic equivalence, we employ symbolic execution, a program analysis technique that computes execution paths and constraints for a given snippet. 
%Two snippets $s^{(i)}$ and $s^{(j)}$ are deemed functionally equivalent if their symbolic execution traces are identical or exhibit subsumption.
%\CD{We need details on the semantic equivalence. This is, in principle, undecidable, so we need to explain a bit more on how this works.}
%
This process, as shown in Algorithm~\ref{alg:clustering} from Section~\ref{sec:symexclustering}, is based on symbolic execution. %ensures that clustering is driven by functional, rather than syntactic or lexical, similarities, aligning with the stricter requirements of code quality evaluation.
%It is important to note that two syntactically different responses can still end up in the same cluster if they are semantically equivalent.
% This is due to our adapted symbolic execution based clustering algorithm . 

\subsubsection{Entropy Estimation}
The final step computes uncertainty as the semantic entropy over clusters, reflecting the diversity of functional behaviors.

%Semantic entropy quantifies the uncertainty in functional behaviour by measuring the probability distribution over clusters.
%
First, the probability associated with a cluster \(c\) is calculated as:
\begin{equation*}
    \tilde{p}(c \mid x) = \sum_{s \in c} \tilde{p}(s \mid x),
    %p(c \mid x) = \sum_{s \in c} p(s \mid x),    
\end{equation*}
where \(s \in c\) indicates that the snippet \(s\) belongs to the cluster \(c\).
%
Then, the entropy \(H(C \mid x)\) over the set of clusters \(C\) is defined as:
\begin{equation*}
    H(C \mid x) = -\sum_{c \in C} \log \tilde{p}(c \mid x),
    %H(C \mid x) = -\sum_{c \in C} \log p(c \mid x),    
\end{equation*}
where \(C\) denotes all semantic clusters obtained from Algorithm~\ref{alg:clustering} in Section~\ref{sec:symexclustering}. 
%
%This formulation captures both the diversity and confidence of the model's outputs while accounting for the length-based normalization, offering a self-contained metric independent of external validation.
A higher entropy indicates greater semantic diversity and hence higher uncertainty in the functional behavior captured by the clusters. 
Conversely, a lower entropy suggests that the model's outputs are concentrated around a few semantically equivalent behaviors, reflecting higher confidence.

\paragraph{Motivating example revisited for \textnormal{\gptturbo}.} To illustrate our uncertainty computation, we'll go back to the motivating example from Section~\ref{sec:motivating}.

As discussed there, Figure~\ref{fig:good-llm-snippets} (Listings~\ref{lst:good1}, \ref{lst:good2}, and \ref{lst:good3}) contains code snippets generated by \gptturbo. % that are semantically equivalent. %, all three snippets exhibit the same behaviour.
%while Figure~\ref{fig:bad-llm-snippets} (Listings~\ref{lst:bad1}, \ref{lst:bad2}, and \ref{lst:bad3} from \salesforce/\codegenmonoC) contains code snippets that are \textbf{semantically distinct}, none of the snippets share the same functional behaviour.
We denote these snippets by $s^{(1)}, s^{(2)}, s^{(3)}$, and, according to Algorithm~\ref{alg:clustering} from Section~\ref{sec:symexclustering}, they are all grouped in the same functional cluster $c_1$ as they are semantically equivalent.
%Now for the \gptturbo case, we have three generated snippets (Listings~\ref{lst:good1}, \ref{lst:good2}, and \ref{lst:good3} from Figure~\ref{fig:good-llm-snippets}), denoted $s^{(1)}, s^{(2)}, s^{(3)}$, all found by using the Algorithm~\ref{alg:clustering} from \S\ref{sec:symexclustering}  to be in one functional cluster $c_1$.
In other words, $C = \{ c_1 \}$ with $c_1 = \{ s^{(1)}, s^{(2)}, s^{(3)} \}$.

%Given the very small probabilities reported by the LLM as shown in Section~\ref{sec:motivating} (which would cause underflows in our implementation), here we use length normalized log-probabilties according to the normalization formula in Section~\ref{sec:probcomp}.
Following softmax normalization, we obtain the following probabilities for the three snippets: %, where we use $\tilde{p}$ to denote the probability based on length-normalization: 
%For these snippets, following are the normalized probabilities based on the token-level \texttt{logprobs} data obtained from \gptturbo:
\(\tilde{p}(s^{(1)} \mid x) = \GPTsnipNormProbA,\; \tilde{p}(s^{(2)} \mid x) = \GPTsnipNormProbB,\;
\tilde{p}(s^{(3)} \mid x) = \GPTsnipNormProbC.\)
Since all responses belong to the single cluster $c_1$, its cluster probability is:
\[
   \tilde{p}(c_1 \mid x)
   \;=\;
   \sum_{s \in c_1} \tilde{p}(s \mid x)
   \;=\;
   \GPTsnipNormProbA \;+\; \GPTsnipNormProbB \;+\; \GPTsnipNormProbC
   \;=\;
   1.0
\]
Thus the distribution over clusters is \(\tilde{p}(c_1 \mid x) = 1,\) and the entropy of clusters is:
\[
   H(C \mid x)
   \;=\;
   -\sum_{c \in C} \log \tilde{p}(c \mid x)
   \;=\;
   -\log(1.0)
   \;=\;
   0.
\]
A \emph{zero} semantic entropy indicates high confidence in the model's response for this prompt.

\paragraph{Motivating example revisited for \textnormal{\salesforce}.} Let's now consider the three snippets $s^{(1)}, s^{(2)}, s^{(3)}$ from Figure~\ref{fig:bad-llm-snippets} generated by the \salesforce/\codegenmonoC model. 
Their respective probabilities are:
$\tilde{p}(s^{(1)}\!\mid x) = \SFsnipNormProbA,\;
 \tilde{p}(s^{(2)}\!\mid x) = \SFsnipNormProbB,\;
 \tilde{p}(s^{(3)}\!\mid x) = \SFsnipNormProbC$.
These snippets get categorized in three distinct semantic clusters
$C = \{ c_1, c_2, c_3 \}$, with $c_1 = \{ s^{(1)} \}$,
$c_2 = \{ s^{(2)} \}$, and $c_3 = \{ s^{(3)} \}.$
 Because each snippet resides in its own cluster, the cluster probabilities are:
   $\tilde{p}(c_1 \mid x) = \SFsnipNormProbA, \tilde{p}(c_2 \mid x) = \SFsnipNormProbB, \tilde{p}(c_3 \mid x) = \SFsnipNormProbC$.
%\[
%   \tilde{p}(c_1 \mid x) = \SFsnipNormProbA,\quad
%   \tilde{p}(c_2 \mid x) = \SFsnipNormProbB,\quad
%   \tilde{p}(c_3 \mid x) = \SFsnipNormProbC.
%\]
%
The entropy then is:
\[
\begin{aligned}
   H(C \mid x)
   &=
   -\!\sum_{c \in \{c_1,c_2,c_3\}}
   \log \tilde{p}(c \mid x)
   \\
   &=
   -\,\Bigl(
      \log(\SFsnipNormProbA)\;+\;\log(\SFsnipNormProbB)\;+\;\log(\SFsnipNormProbC) 
   \Bigr) \approx \SFSE.
\end{aligned}
\]
%Numerically, this is approximately \SFSE.
%A \emph{higher entropy} in this example indicates more disagreement or diversity in the model's functional outputs: the \salesforce/\codegenmonoC model produced three \textbf{distinctly incorrect} solutions, each forming its own cluster.


Intuitively, the uncertainty estimate reflects the strength of our belief in the LLM's prediction. 
Based on this, an \textit{abstention policy} can be implemented, whereby the system abstains from making a prediction if the entropy exceeds a predefined \emph{uncertainty threshold}. 
This approach minimizes the likelihood of committing to incorrect or suboptimal solutions. The abstention threshold is empirically determined by analyzing the entropy distribution. 
The methodology for computing this threshold will be detailed in Section~\ref{sec:eval}.

%Thus, these two scenarios exemplify how \emph{semantic clustering} and the corresponding \emph{entropy measure} can capture both the diversity (or uniformity) of model generations \emph{and} the model's confidence in those generations' functional behaviour.


\section{Mutual Information Estimation via Symbolic Clustering}
\label{sec:mi}
% \AS{TODO:Redo this section, their is a sizeable gap between the theory of the paper and its practical implementation. Make sure that these gaps are explained in this section.
% Algorithm 2 and 3 of the paper work quite differently, so ensure that the explanation matches the implementation, where needed.}
This section presents an adaptation of the mutual information-based approach for quantifying epistemic uncertainty by Abbasi~\etal~\cite{abbasi2024believe} to the domain of code generation.
We follow the steps from the original work: iterative prompting for generating LLM responses, clustering, and mutual information estimation. 
However, similar to Section~\ref{sec:symex}, we diverge with respect to the methodology for clustering responses and the way we estimate the distribution of generated LLM responses.

\subsubsection{Iterative Prompting for Code Generation}
Iterative prompting is used for generating multiple responses from the LLM and consequently in constructing a pseudo joint distribution of outputs. 

% Beginning with the original prompt \(F_0(x) = x\) for \(i = 1, 2, \ldots, n\) we get a \emph{family of prompts} where the \(i\)-th response prompt in the family would be:
% \[
%   F_i(x, s_1, \ldots, s_i) = \text{``Original prompt: } x \text{. Previous responses: } s_1, \ldots, s_i \text{."}
% \]
% where $s_i$ is the \(i\)-th response from the LLM. 

More precisely, the LLM is sampled to produce $n$ responses while also getting their respective probabilities, $\mu(X_j)$ for \(j = 1, 2, \ldots, n\).
These responses are first used to construct iterative prompts by appending the response to the original prompt and asking the LLM to produce more responses. 
This step then makes use of softmax-style normalization to obtain values that can then be treated as probabilities which are used in the subsequent steps.  

As explained in Section~\ref{sec:probcomp} and Section~\ref{sec:symex}, this process can lead to numerical underflows. 
To mitigate this, we use length-normalization.
As opposed to the approach in Section~\ref{sec:symex}, here we did not use the uniform distribution approximation, as the actual LLM-reported probabilities are needed to distinguish between aleatoric and epistemic uncertainties.
%

Following length normalization, we compute conditional probabilities,  $\mu(X_m|X_n)$ for \(m,n = 1, 2, \ldots, n\), by looking at the response probabilities received from the LLM when subjected to the aforementioned iterative prompts.


% \CD{Are these $X_i$ rather than $s_i$ or that's just for clusters? Also, where do we get the probabilities $\mu(X_i)$ and $\mu(X_j \mid X_i)$ that are used below?}

\subsubsection{Clustering via Symbolic Execution}
To handle functional diversity, the generated program snippets are clustered based on their semantic equivalence using Algorithm~\ref{alg:clustering}. 
%symbolic execution. 
%Symbolic execution analyses each program to compute execution paths and constraints. 
%Two programs \(s^{(i)}\) and \(s^{(j)}\) are considered functionally equivalent if their symbolic execution traces \(T(s^{(i)})\) and \(T(s^{(j)})\) satisfy:
%\[
%T(s^{(i)}) \equiv T(s^{(j)}) \quad \text{or} \quad T(s^{(i)}) \subseteq T(s^{(j)}).
%\]

% Let \(C = \{c_1, c_2, \ldots, c_k\}\) denote the clusters formed, where each cluster \(c_j\) groups semantically equivalent programs. 
%The clustering procedure ensures that semantically redundant responses are grouped together, focusing on functional equivalence rather than syntactic similarity.
%The algorithm used is same as Algorithm~\ref{alg:clustering} from the previous section.

\subsubsection{Mutual Information Estimation}
% Once clustering is complete, mutual information is computed over the resulting clusters to quantify epistemic uncertainty. 
% The pseudo joint distribution is defined as:
% \[
% \tilde{p}(s_1, \ldots, s_n \mid x) = p(s_1 \mid F_0(x)) \prod_{i=2}^n p(s_i \mid F_{i-1}(x, s_1, \ldots, s_{i-1})).
% \]
% The marginal distribution is then:
% \[
% \tilde{p}^\otimes(s_1, \ldots, s_n) = \prod_{i=1}^n p(s_i \mid F_0(x)).
% \]

% The mutual information is then computed as:
% \[
% I(\tilde{p}) = D_{KL}(\tilde{p} \| \tilde{p}^\otimes).
% \]

Once clustering is complete, mutual information is computed over the resulting clusters to quantify epistemic uncertainty. 

The aggregated probabilities are defined as:
\[
\mu_1'(X_i) = \sum_{j \in D(i)} \mu(X_j), \quad
\mu_2'(X_t \mid X_i) = \sum_{j \in D(t)} \mu(X_j \mid X_i),
\]
where \( X_i \) and \( X_t \) are clusters, \( \mu(X_j) \) represents the probability of the output \( X_j \), and \( \mu(X_j \mid X_i) \) is the conditional probability of \( X_j \) given \( X_i \). The set \( D(i) \) contains all outputs assigned to the cluster \( X_i \).

The normalized empirical distributions are:
\[
\hat{\mu}_1(X_i) = \frac{\mu_1'(X_i)}{Z}, \quad \text{where} \quad Z = \sum_{j \in S} \mu_1'(X_j),
\]
\[
\hat{\mu}_2(X_t \mid X_i) = \frac{\mu_2'(X_t \mid X_i)}{Z_i}, \quad \text{where} \quad Z_i = \sum_{j \in S} \mu_2'(X_j \mid X_i).
\]
Here, \( \hat{\mu}_1(X_i) \) is the normalized marginal distribution for cluster \( X_i \), and \( \hat{\mu}_2(X_t \mid X_i) \) is the normalized conditional distribution for \( X_t \) given \( X_i \). The terms \( Z \) and \( Z_i \) are normalization constants to ensure that the distributions sum to 1.

The joint and pseudo-joint distributions are defined as:
\[
\hat{\mu}(X_i, X_t) = \hat{\mu}_1(X_i) \hat{\mu}_2(X_t \mid X_i), \quad
\hat{\mu}^\otimes(X_i, X_t) = \hat{\mu}_1(X_i) \sum_{j \in S} \hat{\mu}_1(X_j) \hat{\mu}_2(X_t \mid X_j).
\]
The joint distribution \( \hat{\mu}(X_i, X_t) \) combines the marginal and conditional distributions, while the pseudo-joint distribution \( \hat{\mu}^\otimes(X_i, X_t) \) assumes independence between clusters.

Finally, the mutual information is computed as:
\[
\hat{I}(\gamma_1, \gamma_2) = \sum_{i, t \in S} \hat{\mu}(X_i, X_t) \ln \left( \frac{\hat{\mu}(X_i, X_t) + \gamma_1}{\hat{\mu}^\otimes(X_i, X_t) + \gamma_2} \right).
\]
Here, \( \gamma_1 \) and \( \gamma_2 \) are small stabilization parameters to prevent division by zero, and \( S \) is the set of clusters.

% \subsubsection{Finite-Sample Estimation with Clusters}
% To estimate mutual information from the finite sample of clusters \(C = \{c_1, \ldots, c_k\}\), the probability of a cluster \(c\) is computed as:
% \[
% p(c \mid x) = \sum_{s \in c} p(s \mid x).
% \]
% The empirical mutual information is then:
% \[
% \hat{I}_k = \sum_{c \in C} \hat{p}(c \mid x) \ln \left( \frac{\hat{p}(c \mid x)}{\prod_{i=1}^n \hat{p}(c_i \mid x)} \right),
% \]
% where \(\hat{p}(c \mid x)\) is the observed cluster probability. 
% The empirical mutual information is then:
% \[
% \hat{I}_k(\gamma_1, \gamma_2) = \sum_{i, t \in S} p(X_i, X_t) \ln \left( \frac{p(X_i, X_t) + \gamma_1}{p^\otimes(X_i, X_t) + \gamma_2} \right),
% \]
% where $\hat{I}_k(\gamma_1, \gamma_2)$ is the estimated mutual information, $p(X_i, X_t)$ represents the joint empirical distribution over clusters $X_i$ and $X_t$, and $p^\otimes(X_i, X_t)$ denotes the product of their marginal probabilities for cluster pairs. 
% The stabilization parameters $\gamma_1$ and $\gamma_2$ are included to handle cases where $p(X_i, X_t)$ or $p^\otimes(X_i, X_t)$ might be zero, preventing undefined logarithmic terms. 
% This formulation allows for a robust estimation of mutual information by quantifying the dependencies between cluster distributions while addressing numerical instabilities.

% Entropy regularization is then applied for stability:
% \[
% \hat{I}_k(\gamma) = \sum_{c \in C} \hat{p}(c \mid x) \ln \left( \frac{\hat{p}(c \mid x) + \gamma}{\prod_{i=1}^n (\hat{p}(c_i \mid x) + \gamma)} \right).
% \]

This mutual information score serves as a proxy for epistemic uncertainty. 
High \(\hat{I}\) values signal significant uncertainty.
% \[
% a_\lambda(x) =
% \begin{cases} 
% 1 & \text{if } \hat{I}_k \geq \lambda, \\
% 0 & \text{otherwise.}
% \end{cases}
% \]

\paragraph{Motivating example revisited for \textnormal{\gptturbo}.}
We now illustrate the MI computation for our motivating example from Section~\ref{sec:motivating}. 
We use 3 samples with an iteration prompt length of 2.
All responses from \gptturbo fall in the same cluster and hence following the earlier formula for MI we get:

\begin{align*}
    \hat{I}(\gamma_1, \gamma_2) = \hat{\mu}(X_1, X_1) \ln \left( \frac{\hat{\mu}(X_1, X_1) + \gamma_1}{\hat{\mu}^\otimes(X_1, X_1) + \gamma_2} \right) 
    = 1.0 \ln \left( \frac{1.0 + \gamma_1}{1.0 + \gamma_2} \right).
\end{align*}

Since $\gamma_1$ and $\gamma_2$ are zero as per Abbasi~\etal~\cite{abbasi2024believe}:
%\begin{align*}
   $\ln \left( \frac{1.0 + \gamma_1}{1.0 + \gamma_2} \right) = \ln(1.0) = 0$.
%\end{align*}
Therefore, $\hat{I} = 1.0 \cdot 0 = 0$.
%\begin{align*}
%    \hat{I} &= 1.0 \cdot 0 = 0.
%\end{align*}
%
% EXPLANATION: Zero MI is expected as the paper says this "For the S.E. and M.I. methods, the responses for a large number of queries can be clustered into a single group, and therefore the semantic entropy and mutual information scores are zero."
%
A \emph{zero} MI indicates very low uncertainty in the model's responses, suggesting a high likelihood of correctness as we will show later in the paper.

\paragraph{Motivating example revisited for \textnormal{\salesforce}.}
For \codegenmonoC, all three responses fall in their own separate clusters \ie  $R_1 \in X_1$, $R_2 \in X_2$, $R_3 \in X_3$ and hence,
%
%\begin{align*}
    $P(X_1) = 0.9995, P(X_2) = 0.0004, P(X_3) = 0.0001$.
%\end{align*}

For brevity, we omit the computation of the marginals which was carried out using the same formula used by Abbasi~\etal~\cite{abbasi2024believe} which for this example looks like:

\[
\hat{I}(\gamma_1, \gamma_2)
= \sum_{i=1}^3 \sum_{j=1}^3
  \hat{\mu}(X_i, X_j)
  \ln\!\Biggl(\frac{\hat{\mu}(X_i, X_j) + \gamma_1}
                   {\hat{\mu}^\otimes(X_i, X_j) + \gamma_2}\Biggr).
\]

Computing the terms individually then looks like:

\[
\begin{aligned}
    (X_1, X_1) &: 0.800 \ln \left( \frac{0.800}{0.9990} \right) = -0.1788, & (X_1, X_2) &: 0.120 \ln \left( \frac{0.120}{0.0004} \right) = 0.6845, \\
    (X_1, X_3) &: 0.040 \ln \left( \frac{0.040}{0.0001} \right) = 0.2397, & (X_2, X_1) &: 0.050 \ln \left( \frac{0.050}{0.0004} \right) = 0.5050, \\
    (X_2, X_2) &: 0.010 \ln \left( \frac{0.010}{0.00000016} \right) = 0.1104, & (X_2, X_3) &: 0.020 \ln \left( \frac{0.020}{0.00000004} \right) = 0.2624, \\
    (X_3, X_1) &: 0.020 \ln \left( \frac{0.020}{0.0001} \right) = 0.1609, & (X_3, X_2) &: 0.020 \ln \left( \frac{0.020}{0.00000004} \right) = 0.2624, \\
    (X_3, X_3) &: 0.010 \ln \left( \frac{0.010}{0.00000001} \right) = 0.1382.
\end{aligned}
\]

Then total MI then is:
%\begin{align*}
    $\hat{I} = -0.1788 + 0.6845 + 0.2397 + 0.5050 + 0.1104 + 0.2624 + 0.1609 + 0.2624 + 0.1382 = 2.1847$.
%\end{align*}

%\begin{align*}
%    \hat{I} &= -0.1788 + 0.6845 + 0.2397 + 0.5050 + 0.1104 \\
%    &\quad + 0.2624 + 0.1609 + 0.2624 + 0.1382 = 2.1847.
%\end{align*}


While lower uncertainty is intuitively desirable, as discussed in Section~\ref{sec:symex}, the uncertainty estimate is assessed against an abstention threshold to derive meaningful conclusions (see Section~\ref{sec:usability}).

%This shows \emph{moderately} high uncertainty, due to all responses being in their own respective clusters. 

%By integrating symbolic execution-based clustering with iterative prompting and mutual information estimation, this methodology captures both functional diversity and epistemic uncertainty in program generation. 




\section{Theoretical Analysis}\label{sec:theoretical}

\textbf{Different correct answers are competitor.}\quad For any LLM trained with cross-entropy loss, different correct answers are competitors in terms of probability \footnote{The ``same question'' refers to questions that are semantically equivalent but do not need to be identical.}. Continuing with the example of proposing a president, suppose $\tau^{a}$ (``\texttt{Barack}'') is the label of a sample whose $\bm{q}$ is ``\texttt{[INST]Could you give me one name of president?[\textbackslash INST]}'' and a generated token vector $\bm{a}_{t-1}$  can be decoded into ``\texttt{Sure, here is a historical American president:**}'', the loss of the next token at this position during supervised fine-tuning can be written as:
\begin{equation}
\begin{aligned}
 &L^{\tau^a} = - \log \frac{\exp(\mathcal{M}({\tau^a}|\bm{q},\bm{a}_{t-1}))}{\sum_{m=1}^{|\bm{Y}|} \exp(\mathcal{M}(\tau^{m}|\bm{q},\bm{a}_{t-1}))} ,
 % \\   &L^{\tau^b} = - \log \frac{\exp(\mathcal{M}(\tau^b|\bm{q},\bm{a}_{t-1}))}{\sum_{m=1}^{|\bm{Y}|} \exp(\mathcal{M}(\tau^{m}|\bm{q},\bm{a}_{t-1}))} ,
\end{aligned}
\end{equation}
where $L^{\tau^a}$ is the loss on the sample with the next token label $\tau^{a}$.
Consider cases where multiple distinct answers to the same question appear in the training set, the situation becomes different. For example, $\tau^{b}$ (``\texttt{George}'') is the label in another sample with the same question. When the model is simultaneously fine-tuned on both samples, the gradient update for the model will be:
\begin{equation}
\begin{aligned}
 & \nabla_{\mathcal{M}} (L^{\tau^a} + L^{\tau^b}) = \nabla_{\mathcal{M}} L^{\tau^a} + \nabla_{\mathcal{M}} L^{\tau^b} \\
% &= -y_a^{\tau^a}\frac{1}{\Omega_a^{\tau^a}}\nabla_{\mathcal{M}}\Omega_a^{\tau^a}-\sum_{m \neq a}^{|\bm{Y}|} y_a^{\tau^m}\frac{1}{\Omega_a^{\tau^m}}\nabla_{\mathcal{M}}\Omega_a^{\tau^m}
% \\
% &\quad -y_b^{\tau^b}\frac{1}{\Omega_b^{\tau^b}}\nabla_{\mathcal{M}}\Omega_b^{\tau^b}-\sum_{m \neq b}^{|\bm{Y}|} y_b^{\tau^m}\frac{1}{\Omega_b^{\tau^m}}\nabla_{\mathcal{M}}\Omega_b^{\tau^m}
% \\
&\quad= \underbrace{-y_a^{\tau^a}\frac{1}{\Omega_a^{\tau^a}}\nabla_{\mathcal{M}}\Omega_a^{\tau^a}-y_b^{\tau^b}\frac{1}{\Omega_b^{\tau^b}}\nabla_{\mathcal{M}}\Omega_b^{\tau^b}}_{\text{(1) maximizing the probability of annotated answer}}\\& \quad \underbrace{-y_a^{\tau^b}\frac{1}{\Omega_a^{\tau^b}}\nabla_{\mathcal{M}}\Omega_a^{\tau^b}-y_b^{\tau^a}\frac{1}{\Omega_b^{\tau^a}}\nabla_{\mathcal{M}}\Omega_b^{\tau^a}}_{{\text{\textbf{(2)} minimizing the probability of the other annotated answer}}}\\& \quad \underbrace{-\sum_{m \neq a,b}^{|\bm{Y}|}y_{a,b}^{\tau^m} \left[ \frac{1}{\Omega_a^{\tau^m}}\nabla_{\mathcal{M}}\Omega_a^{\tau^m} + \frac{1}{\Omega_b^{\tau^m}}\nabla_{\mathcal{M}}\Omega_b^{\tau^m} \right]}_{\text{(3) minimizing the probability of incorrect answers}},
\end{aligned}\label{eq:competitor}
\end{equation}
where $\Omega_a^{\tau^a}=\frac{\exp(\mathcal{M}(\tau^a|\bm{q},\bm{a}_{t-1}))}{\sum_{m=1}^{|\bm{Y}|} \exp(\mathcal{M}(\tau^{m}|\bm{q},\bm{a}_{t-1}))}$, and $y_a^{\tau^m}$ indicates the next token label of a training sample with ground-truth label ${\tau^a}$, that is, we have $y_a^{\tau^a}=1$ and $y_a^{\tau^b}=0$. In particular, when $\mathcal{M}$ is in a certain state during training, we have $\Omega_a^{\tau^a}=\Omega_b^{\tau^a}$, and we make distinctions to facilitate the reader's understanding here. As we can see, for scenarios with multiple answers, the training objective can be divided into three parts:
(1) For each sample, increase the probability of its own annotation in the output distribution.
(2) For each sample, decrease the probability of another sample's annotation in the output distribution. \textit{\textbf{Note:}} This part leads to the issue where probability cannot anymore capture the reliability of LLM responses, as different correct answers tend to reduce the probability of other correct answers, making low probabilities cannot indicates low reliability.
(3) For both samples, decrease the probability of other outputs not present in the annotations in the output distribution.






% Acknowledgments---Will not appear in anonymized version
%\acks{We thank a bunch of people and funding agency.}
\phantomsection
\addcontentsline{toc}{section}{References}
\bibliographystyle{amsalpha}
\bibliography{bib/custom,bib/dblp,bib/mathreview,bib/scholar}

\newpage
\appendix

\section{First Estimation}
First, if we have more samples than needed, we simply take only the necessary number of samples (choosing them at random) and do not use the other samples. We need this step to make sure that $n=\poly(d)$, and small probability events do not happen with any of the (uncorrupted) samples.
For simplicity, further we denote by $n$ the number of samples that the algorithm actually uses.

For $i \in [\lfloor n / 2 \rfloor]$, let $z'_i = \paren{z_i - z_{\lfloor n / 2 \rfloor + i}}/\sqrt{2}$. Then, by Theorem 4.1 from \cite{frahm2004generalized}, $z'$ is a $4\e$-corruption of the sample $x'_1,\ldots x'_{\lfloor n/2 \rfloor} \simiid \cD'$, where $\cD'$ is an elliptical distribution with location $0$ and scatter matrix $\Sigma$. To simplify the notation, we assume that this preprocessing step has been done, and the input $z_1,\ldots,z_n$ is an $\e$-corruption of an iid sample from an elliptical distribution with location $0$ and scatter matrix $\Sigma$.

Then we project $z_1,\ldots, z_n$ onto the sphere of radius $\sqrt{d}$. 
We can assume that the projected samples are $\spsign(y'_1),\ldots\spsign(y'_n)$, where $y'_1,\ldots y'_n$ is an $\e$-corruption of $y_1,\ldots, y_n\simiid \cN(0,\Sigma)$ (and $\spsign$ is the projection). 
Even though $\spsign(y)$ has nice properties, for technical reasons it is  not convenient to analyze it. Instead, we analyze the function $\cF:\R^d \to \R^d$ defined as
    \[
    \cF(x) = 
    \begin{cases}
        \spsign(x) = {\sqrt{d}} \cdot x/{\norm{x}} &\text{if } 0.9d \le \norm{x}^2 \le 1.1d\,,\\
        \sqrt{10/9}\cdot x &\text{if } \norm{x}^2 < 0.9d\,,\\
        \sqrt{10/11} \cdot x &\text{if } \norm{x}^2 > 1.1d\,.
    \end{cases}
    \]

By Hanson-Wright inequalities (\cref{prop:Hanson-Wright}), if $\effrank(\Sigma) \gtrsim \log(d)$, then with high probability for all $i\in[n]$, $\cF(y_i) = \spsign(y_i)$. 
Hence we can assume that we are given an $\e$-corruption of iid samples that are distributied as $\cF(y)$, where $y\sim \cN(0,\Sigma)$. We call these corrupted samples $\zeta_1,\ldots, \zeta_n$, and true samples $\cF(y_1),\ldots , \cF(y_n)$. 


The following lemma shows that the covariance of $\cF(y)$ is close to $\Sigma$.

\begin{lemma}\label{lem:closeness}
    Let $\Sigma \in \R^{d\times d}$ be a positive semidefinite matrix such that $\Tr(\Sigma) = d$ and $\effrank(\Sigma) \gtrsim \log(d)$.
    
    Then 
    \[
    \Norm{\Sigma^{-1/2}\,
    \Paren{\Cov_{y \sim \cN(0,\Sigma)}{\cF(y)} - \Sigma}
    \,\Sigma^{-1/2}}  \le O\Paren{\frac{1}{\effrank(\Sigma)}}\,.
    \]

\end{lemma}
\begin{proof}
    Observe that $\Sigma' := \Cov_{y \sim \cN(0,\Sigma)}\brac{\cF(y)}$ has the same eigenvectors as $\Sigma$. Indeed, in the basis where $\Sigma$ is diagonal, 
    \[
    \Sigma'_{ij}= \E\Brac{\frac{\sqrt{\lambda_i\lambda_j} \cdot g_ig_j}{\tfrac{1}{d}\sum_{i=1}^d \lambda_i g_i^2 \cdot \ind{\sum_{i=1}^d \lambda_i g_i^2\in[0.9d,1.1 d]} 
    + \tfrac{9}{10} \ind{\sum_{i=1}^d \lambda_i g_i^2 < 0.9d} 
    + \tfrac{11}{10} \ind{\sum_{i=1}^d \lambda_i g_i^2 > 1.1d}
    }} 
    \,,
    \]
    where $\lambda_1,\ldots, \lambda_d$ are the eigenvalues of $\Sigma$. Since the signs of $g_i$ and $g_j$ are independent and are independent of all $g_1^2,\ldots, g_d^2$, $\Sigma'_{ij} = 0$ if $i\neq j$.
    
    Hence if we take these eigenvectors as the basis, the matrices $\Sigma^{-1/2}$, $\Sigma'$ and $\Sigma$ become diagonal, and the value of the relative spectral norm does not change. In this basis, the eigenvalues of $\Sigma'$ are its diagonal entries.
    
    Note that
       \[
     \Norm{\Sigma^{-1/2}\,
    \Paren{\Sigma' - \Sigma}
    \,\Sigma^{-1/2}} = \max_{j\in [d]} \Abs{\frac{\lambda_j'}{\lambda_j} - 1} \,.
    \]

    Consider 
    \[
    q = 1 - \Paren{\tfrac{1}{d}\sum_{i=1}^d \lambda_i g_i^2 \cdot \ind{\sum_{i=1}^d \lambda_i g_i^2\in[0.9d,1.1 d]} 
    + \tfrac{9}{10} \ind{\sum_{i=1}^d \lambda_i g_i^2 < 0.9d} 
    + \tfrac{11}{10} \ind{\sum_{i=1}^d \lambda_i g_i^2 > 1.1d}}\,.
    \]
    Note that since $\abs{q} \le 1/10$, $\frac{1}{1 - q} = \sum_{k=0}^{\infty} q^k$.
    Therefore,
    \[
    \frac{\lambda'_i}{\lambda_i}
    =
    \sum_{k=0}^{\infty} \E{g_j^2 q^k} 
    = \E{g_j^2}+ \E\Brac{g_j^2 q} +  \E \Brac{g_j^2 q^2}  + \sum_{k=3}^{\infty} \E \Brac{g_j^2 q^k}\,.
    \]
    The first term is $1$. 
    Let us bound the second term. Let $N=d^{1000}$, and for all $i\in[d]$ consider $g_{1i},\ldots,g_{Ni}\simiid \cN(0,1)$. By Hanson-Wright inequalities (Proposition \ref{prop:Hanson-Wright}) with high probability, for all $m \in [N]$,  
    $\tfrac{1}{d}\sum_{i=1}^d \lambda_i (1-g_{mi}^2) \in [0.9,1.1]$. 
    Also, with high probability, the sample average of 
    $g_{mj}^2 \cdot \tfrac{1}{d}\sum_{i=1}^d \lambda_i (1-g_{mi}^2)$
    is $d^{-10}$-close to both $\E\Brac{g_j^2 q}$ and $\E g_j^2 \cdot\tfrac{1}{d}\sum_{i=1}^d \lambda_i (1-g_i^2) = \frac{2\lambda_j}{d}$ (since both random variables $g_j^2 q$ and $g_j^2 \cdot\tfrac{1}{d}\sum_{i=1}^d \lambda_i (1-g_i^2)$ have variance $O(1)$). Therefore, the second term is bounded by $\frac{2\lambda_j}{d} + 2d^{-10} \le \frac{3}{\effrank(\Sigma)}$ in absolute value.
    
    Since $\abs{q} \le \abs{1-\tfrac{1}{d}\sum_{i=1}^d \lambda_i g_i^2}$,
    \[
    \E \Brac{g_j^2 q^2} \le 
    \tfrac{1}{d^2}\sum_{i=1}^d \lambda_i^2 \E g_j^2 (1-g_i^2)^2
    \le \tfrac{20}{d^2} \sum_{i=1}^d \lambda_j^2
    \le \tfrac{20}{d}\Norm{\Sigma} 
    \le \tfrac{20}{\effrank(\Sigma)}\,.
    \]
    
    Now let us bound the fourth term. Since $\abs{q} \le 1/10$,
    \[
    \Abs{\sum_{k=3}^{\infty} \E \Brac{g_j^2 q^k}} \le \E \Brac{g_j^2 q^2} 
    \le \frac{20} {\effrank(\Sigma)}\,.
    \]

    Hence for all $i\in[d]$, $\frac{\lambda'_i}{\lambda_i} = 1 + O(\frac{1}{\effrank(\Sigma)})$.
\end{proof}

The following proposition shows that a good estimator of $\Sigma'$ is also a good estimator of $\Sigma$.

\begin{proposition}\label{prop:triangle-inequality}
Let $\Sigma, \Sigma', \hat{\Sigma}\in \R^{d\times d}$ 
be positive definite matrices such that for some $\Delta \le 1$,
\[
\normf{{\Sigma}^{-1/2}\, \Sigma'\, {\Sigma}^{-1/2} - \Id} \le \Delta
\]
and
\[
\normf{\Paren{\Sigma'}^{-1/2}\, \hat{\Sigma}\, \Paren{\Sigma'}^{-1/2} - \Id} \le \Delta\,.
\]
Then
\[
\normf{{\Sigma}^{-1/2}\, \hat{\Sigma}\, {\Sigma}^{-1/2} - \Id} \le 3\Delta\,.
\]
The same is true if Frobenius norm is replaced by spectral norm.
\end{proposition}
\begin{proof}
\begin{align*}
\normf{{\Sigma}^{-1/2}\, \hat{\Sigma}\, {\Sigma}^{-1/2} - \Id} 
&\le
\normf{{\Sigma}^{-1/2}\, \hat{\Sigma}\, {\Sigma}^{-1/2} -{\Sigma}^{-1/2}\, \Sigma'\, {\Sigma}^{-1/2}} +\normf{{\Sigma}^{-1/2}\, \Sigma'\, {\Sigma}^{-1/2} - \Id}
\\&\le \norm{\Sigma^{-1/2}\Paren{\Sigma'}^{1/2}}^2 \cdot \normf{\Paren{\Sigma'}^{-1/2}\, \hat{\Sigma}\, \Paren{\Sigma'}^{-1/2} - \Id} + \Delta
\\&\le \norm{\Sigma^{-1/2}\Paren{\Sigma'}^{1/2}}^2\cdot \Delta + \Delta\,.
\end{align*}
Denote $A = \Sigma^{-1/2}\Paren{\Sigma'}^{1/2}$. Since $\norm{AA^\top - \Id} \le \normf{AA^\top - \Id} \le \Delta$, 
\[
\norm{A}^2 = \norm{AA^\top} \le 1 + \norm{AA^\top - \Id} \le 1 + \Delta \le 2\,.
\]
Hence we get the desired bound. It is clear that this derivation is also correct if we replace Frobenius norm by spectral norm.
\end{proof}

Let us show how to estimate $\Sigma' = \Cov_{y \sim \cN(0,\Sigma)}{\cF(y)}$ up to error $O\Paren{\frac{1}{\log d} + \sqrt{\e}}$ in relative spectral norm. 
 
Let us split the samples into $2$ equal parts, and run 
the covariance estimation algorithm in relative spectral norm from Theorem 1.2 from \cite{KS17} on the projection onto the sphere of the first part. 
We can do that, since $\cF(y)$ is $O(1)$-sub-Gaussian. Indeed, for all $u\in\R^d$,
\[
\Paren{\iprod{u, \cF(y)}^p}^{1/p}\le 1.1\Paren{\iprod{u, y}^p}^{1/p} \cdot C_G\cdot \sqrt{p}\Paren{\E\iprod{u,y}^2}^{1/2} \le  \frac{1.1}{0.9} \cdot C_G\cdot \sqrt{p}\Paren{\E\iprod{u,\cF(y)}^2}^{1/2}\,,
\]
where $C_G \le O(1)$ is the sub-Gaussian parameter for the Gaussian distribution.

With high probability we get an estimator $\hat{\Sigma}_1$ such that 
    \[
    \norm{(\Sigma')^{-1/2}\, \hat{\Sigma}_1\, (\Sigma')^{-1/2} - \Id} \le O\Paren{\sqrt{{\e}}}\,.
    \]
\lemmaref{lem:closeness} and our assumption on the effective rank imply that 
    \[
    \Norm{\Sigma^{-1/2}\,
    \Paren{\Cov_{y \sim \cN(0,\Sigma)}{\cF(y)} - \Sigma}
    \,\Sigma^{-1/2}}  
    \le  O\Paren{\frac{1}{\log(d)}} \,.
    \]
Hence, by \cref{prop:triangle-inequality}, 
$0.99 \cdot \Id \preceq \hat{\Sigma}_1^{-1}\Sigma \preceq 1.01\cdot \Id$. 
If we multiply the second part of the samples by $\hat{\Sigma}_1^{-1}$, we get an elliptical distribution with new scatter matrix $\tilde{\Sigma} = \rho \hat{\Sigma}_1^{-1} \Sigma$, where $\rho = d/\Tr(\hat{\Sigma}_1^{-1} \Sigma)$. $\tilde{\Sigma}$ has trace $d$ and its eigenvalues are between $0.9$ and $1.1$, and, in particular, its effective rank is $\Omega(d)$.







\section{Fourth Moment of the Spatial Sign}
In this section and further in the paper we use a slightly modified definition of the function $\cF$:
\begin{definition}\label{def:function}
    \[
    \cF(x) = 
    \begin{cases}
        \spsign(x) = {\sqrt{d}} \cdot x/{\norm{x}} &\text{if } d-\Delta \le \norm{x}^2 \le d+\Delta\,,\\
        \sqrt{d/(d-\Delta)}\cdot x &\text{if } \norm{x}^2 < d-\Delta\,,\\
        \sqrt{d/(d+\Delta)}\cdot x &\text{if } \norm{x}^2 > d+\Delta\,.
    \end{cases}
    \]    
for $\Delta = C\cdot \sqrt{d\log d}$, where $C$ is a large enough absolute constant.
\end{definition} 

If $\effrank(\Sigma)\ge \Omega(d)$, then with high probability $\cF(y_i)$ coincides with $\spsign(y_i)$ for all $n=\poly(d)$ samples $y_1,\ldots y_n\simiid\cN(0,\Sigma)$. Note that as long as $\effrank(\Sigma) \ge \Omega(d)$, the proof of \cref{lem:closeness} remains the same, so $\Cov \cF(y)$ is $O(1/d)$-close to $\Sigma$ in relative spectral norm.

\begin{lemma}\label{lem:tensor}
    Let $\Sigma\in \R^{d\times d}$ be a positive definite matrix such that $0.9\Id \preceq \Sigma \preceq 1.1\Id$, $\Tr(\Sigma) = d$.
    Let $g\sim \cN(0,\Id)$, $\Sigma'= \Cov(\cF(\Sigma^{1/2}g))$ and $\phi(g)= \Paren{\Sigma'}^{-1/2} \cF(\Sigma^{1/2}g)$.
    
    Let $T = \E\Paren{\phi(g)\transpose{\phi(g)} - \Id}^{\otimes 2}$ and $S = \E_{g\sim N(0,\Id)}\Paren{gg^\top / \norm{g}^2 - \Id}^{\otimes 2}$. Then for all $i_1 i_2 i_3 i_4, i, j \in [d]$, the components of $T$ in the eigenbasis of $\Sigma$ are as follows:
    \begin{enumerate}
        \item  $T_{i_1 i_2 i_3 i_4} = S_{i_1 i_2 i_3 i_4} = 0$ if some of the $i_m$ is not equal to the others.
        \item $T_{iiii} = 2 + O(1/d) = S_{iiii} + O(1/d)$.
        \item For $i\neq j$, $T_{ijij} = 1 + O(1/d) = S_{ijij} + O(1/d)$.
        \item For $i\neq j$, $T_{iijj} = S_{iijj} + O\Paren{{\Norm{\Sigma-\Id}}/{d}} + O(\log^2(d)/d^2)$.
    \end{enumerate}
\end{lemma}
\begin{proof}
 Without loss of generality we assume $\Normf{M}=1$. 
    Let us bound the entries of the tensor 
    $T := \E\Paren{\phi(g)\transpose{\phi(g)} - \Id}^{\otimes 2}$ in the eigenbasis of $\Sigma$:
    \[
    T_{i_1 i_2 i_3 i_4} = 
    \E
    \Paren{\sqrt{\tfrac{\lambda_{i_1}}{\lambda'_{i_1}}\cdot\tfrac{\lambda_{i_2}}{\lambda'_{i_2}}}\cdot g_{i_1}g_{i_2}/R(g)^2 - \Id_{i_1i_2}} 
    \Paren{\sqrt{\tfrac{\lambda_{i_3}}{\lambda'_{i_3}}\cdot\tfrac{\lambda_{i_4}}{\lambda'_{i_4}}}\cdot g_{i_3}g_{i_4}/R(g)^2 - \Id_{i_3i_4} }
     \,,
    \]
    where $R(g)^2 = {\tfrac{1}{d}\sum_{i=1}^d \lambda_i g_i^2 \cdot \ind{\sum_{i=1}^d \lambda_i g_i^2\in[d - \Delta,d + \Delta]} 
    + \tfrac{d - \Delta}{d} \ind{\sum_{i=1}^d \lambda_i g_i^2 < d-\Delta} 
    + \tfrac{d+\Delta}{d} \ind{\sum_{i=1}^d \lambda_i g_i^2 > d+\Delta}
    }$, where $\Delta= O\Paren{\sqrt{d\log d}}$.
    
    First, note that in the case $\lambda_1 = \ldots = \lambda_d = 1$, the components $T_{i_1 i_2 i_3 i_4}$ and $S_{i_1 i_2 i_3 i_4}$ are expectations of random variables that coincide with probability at least $1/\poly(d)$ and have variance bounded by $O(1)$. Hence the difference between them is at most $O(d^{-10})$.
    
    Consider the entries $T_{i_1 i_2 i_3 i_4}$ such that some $i_m$ is not equal to any of the others. 
    In this case, since $\sign(g_{i_m})$ is independent of $\abs{g_{i_m}}$ and of all other $g_j$, $T_{i_1 i_2 i_3 i_4} = 0$.
    
    Consider the terms of the form $T_{iijj}$ (including the case $i=j$). 
    \[
    T_{iijj} = \frac{\lambda_i\lambda_j}{\lambda'_i\lambda'_j}\cdot \E\frac{g_i^2g_j^2}{R(g)^4} - 1 = \Paren{1 + O(1/d)}\cdot \E\frac{g_i^2g_j^2}{R(g)^4} - 1 \,.
    \]
    As in the proof of \lemmaref{lem:closeness}, let $q = 1-R(g)^2$. Since $\abs{2q - q^2} \le 1/2$,
    \[
    \E\frac{g_i^2g_j^2}{R(g)^4} = \E\frac{g_i^2g_j^2}{1-(2q - q^2)} 
    = \E g_i^2g_j^2  + \sum_{k=1}^\infty \E g_i^2g_j^2 (2q - q^2)^k\,.
    \]
    The first term is $1$ if $i\neq j$ and $3$ if $i=j$. Let us bound the other terms.
    By the same argument as in \lemmaref{lem:closeness}, 
    $\E g_i^2g_j^2 q = \frac{2\lambda_j + 2\lambda_i}{d} + O(d^{-10})$.
     If we write $\lambda_k = 1 + \delta_k$, this expression differs from the same expression for $\lambda_k = 1$ by $O(\delta_i / d + \delta_j/d)$.
    Similarly,
\[
    \E \Brac{g_i^2g_j^2 q^2} 
    = \E g_i^2g_j^2\Paren{\tfrac{1}{d}\sum_{m=1}^d\lambda_m (1-g_m^2)}^2+ O(d^{-10})\,.
    \]
    Since $g_i$ are independent and $\E(1-g_m^2)=1$,
    \[
    \E \E g_i^2g_j^2  \Paren{\tfrac{1}{d}\sum_{m=1}^d\lambda_m (1-g_m^2)}^2
    = \tfrac{1}{d^2}\sum_{m=1}^d \lambda_m^2 \E g_i^2g_j^2 (1-g_m^2)^2 + \tfrac{2}{d^2}\lambda_i\lambda_j \E g_i^2g_j^2 (1-g_i^2)(1-g_j^2) \,.
    \]

    If we write $\lambda_k = 1 + \delta_k$ for $\abs{\delta_k}\le 0.1$, this expression differs from the same expression for $\lambda_k = 1$ by at most $O(\max_{k\in[d]} \delta_k / d)$.
     
    Finally, consider
\[
    \E \Brac{g_i^2g_j^2 q^3} 
    = \E g_i^2g_j^2\Paren{\tfrac{1}{d}\sum_{m=1}^d\lambda_m (1-g_m^2)}^3 + O(d^{-10})\,.
    \]
    Note that,
    \begin{align*}
    \E g_i^2g_j^2\Paren{\tfrac{1}{d}\sum_{m=1}^d\lambda_m (1-g_m^2)}^3
    = &\tfrac{1}{d^3} \sum_{m=1}^d \lambda_m^3 \E g_i^2g_j^2 (1-g_m^2)^3 + \tfrac{1}{d^3} \sum_{m=1}^d \lambda_i\lambda_m^2\E g_i^2g_j^2(1-g_i^2)(1-g_m^2)^2 
    \\&+ \tfrac{1}{d^3} \sum_{m=1}^d \lambda_j\lambda_m^2\E g_i^2g_j^2(1-g_j^2)(1-g_m^2)^2
    + O(1/d^3) \,.
    \end{align*}
    Since each term in each sum is bounded by $O(1)$, $\abs{\E \brac{g_i^2g_j^2 q^3} }\le O(1/d^2)$.
    Since $q^4 \le O(\log^2(d)/d^2)$, the sum of the terms of the series $\sum_{k=1}^\infty \E g_i^2g_j^2 (2q - q^2)^k$ that are not linear in $q$, $q^2$, or $q^3$, is bounded by $O(\log^2(d)/d^2)$. Hence we get the desired expressions for $T_{iijj}$. Note that $T_{ijij}$ for $i\neq j$ is equal to $1+T_{iijj}$, so $T_{ijij} = 1 + O(1/d)$. 

    

    

    % Now we can finish the proof.
    % \begin{align*}
    % \E_{g\sim \cN(0,\Id)} \Iprod{M, \phi(g)\transpose{\phi(g)} - \Id}^2 
    % &= 
    % \sum_{1\le i_1i_2i_3i_4\le d} M_{i_1i_2}M_{i_3i_4}T_{i_1 i_2 i_3 i_4} 
    % \\&=
    % \sum_{i=1}^d M_{ii}^2 T_{iiii} + 2\sum_{1\le i,j\le d} M_{ij}^2T_{ijij} + 
    %  2\sum_{1\le i,j\le d} M_{ii}M_{jj}T_{iijj}
    %  \\&\le O(\normf{M}^2) + O(1/d) \Paren{\sum_{i=1}^d M_{ii}}^2
    %  \\&\le O(\normf{M}^2) + O(1/d) \Paren{\sum_{i=1}^d M_{ii}^2}\cdot d
    %   \\&\le O(\normf{M}^2)\,,
    % \end{align*}
    % where the second to last inequality follows from the  Cauchy--Schwarz inequality.
\end{proof}






% \begin{lemma}\label{lem:moments}
%     Let $\Sigma\in \R^{d\times d}$ be a positive definite matrix such that $\effrank(\Sigma) \ge \Omega(d)$.
%     Let $g\sim \cN(0,\Id)$, $\Sigma'= \Cov(\cF(\Sigma^{1/2}g))$, and $\phi(g)= \Paren{\Sigma'}^{-1/2} \cF(\Sigma^{1/2}g)$.
%     Then for all even numbers $p$ and all matrices $M\in \R^{d\times d}$,
%     \[
%     \Paren{\E_{g\sim \cN(0,\Id)} \Iprod{M, \phi(g)\transpose{\phi(g)} - \Id}^p}^{1/p}
%     \le  O(\Normf{M}\cdot p)\,. 
%     \]
% \end{lemma}

% \begin{proof}
%     Without loss of generality we assume $\Normf{M}=1$. 
%  By \Gnote{reference}, $\phi(g)$ satisfies $LS(3)$. Hence by \Gnote{reference},
%     \[
%      \E\Iprod{M, \phi(g)\transpose{\phi(g)} - \Id}^p \le \Paren{\E \Iprod{M, \phi(g)\transpose{\phi(g)} - \Id}^2}^{p/2} + 3\sqrt{p}\cdot \E\Norm{M \phi(g)}^p\,.
%     \]
%     By \lemmaref{lem:variance}, $\Iprod{M, \phi(g)\transpose{\phi(g)} - \Id}^2 \le O(1)$. 
%     Let us bound $\E\Norm{M \phi(g)}^p$. Consider $M' = M\Paren{\Sigma'}^{-1/2}\Sigma^{1/2}$. By \lemmaref{lem:closeness}, $M'_{ij} \le \Paren{1+O(1/d)} M_{ij}$, hence $\Normf{M'}\le 2$. Note that $\Sigma^{1/2}\Paren{\Sigma'}^{1/2}\phi(g) = g/ R(g)$, 
%     where $R(g) \in [\sqrt{1/2},\sqrt{3/2}]$. Hence
%     \[
%     \E\Norm{M \phi(g)}^p = \E\Iprod{\transpose{M'}M', g\transpose{g}\cdot \rho^2(g) }^{p/2} \le 
%     2 \E\Iprod{\transpose{M'}M', g\transpose{g}}^{p/2}\,.
%     \]
%     Note that $\Tr\Paren{\transpose{M'}M'} = \Normf{M}^2 \le 4$. Let $u_1,\ldots, u_d$ be unit eigenvectors of $\transpose{M'}M'$, and let $\alpha_1,\ldots,\alpha_d$ be its eigenvalues. It follows that
%     \begin{align*}
%     \E\Iprod{\transpose{M'}M', g\transpose{g}}^{p/2} 
%     &\le
%     4^{p/2}\E \Iprod{\sum_{j=1}^d \tfrac{\alpha}{4} u_j \transpose{u_j}, g}^{p/2} 
%     \\&\le
%     2^p \sum_{j=1}^d  \tfrac{\alpha}{4}  \E \Iprod{u_j \transpose{u_j}, g}^{p/2}
%     \\&\le 
%     2^p \max_{j\in[d]}\E \Iprod{u_j, g}^p 
%     \\&\le 
%     2^p \cdot p^{p/2} \,,
%     \end{align*}
%     where we used the convexity of $\Iprod{\cdot, g\transpose{g}}^{p/2}$.
% \end{proof}


\documentclass{article}

\usepackage[utf8]{inputenc}
\usepackage[T1]{fontenc}
\usepackage{babel}
\usepackage{amsfonts}
\usepackage{geometry}
\usepackage{array}
\usepackage{multirow}
\usepackage{amsmath}
\usepackage{amssymb}
\usepackage{graphicx}
\usepackage{setspace}
\usepackage{esint}
\usepackage{rotating}
\usepackage{color}
\usepackage{eurosym}
\usepackage{comment}
\usepackage{longtable}
\usepackage{subcaption}
\usepackage{float}
\usepackage{titlesec}
\usepackage{booktabs}
\usepackage[table]{xcolor} 

\geometry{verbose,tmargin=2cm,bmargin=2cm,lmargin=2.5cm,rmargin=2.5cm}
\doublespacing

\newtheorem{theorem}{Theorem}
\newtheorem{claim}{Claim}
\newtheorem{corollary}{Corollary}
\newtheorem{definition}{Definition}
\newtheorem{example}{Example}
\newtheorem{lemma}{Lemma}
\newtheorem{proposition}{Proposition}
\newtheorem{remark}{Remark}

\title{Driver Assistance System Based on Multimodal Data Hazard Detection}

\begin{document}

\begin{titlepage}
    \centering
    \vspace*{3cm}
    
    {\Huge\bfseries\fontfamily{ptm}\selectfont Driver Assistance System Based on Multimodal Data Hazard Detection \par}

    \vspace{3cm}
    {\Large Long Zhouxiang, Ovanes Petrosian\par}
    
    \vspace{2cm}
    {\large \today\par}
\end{titlepage}

\section{Introduction}
Advances in artificial intelligence have given a significant boost to autonomous driving technology. However, the advantages of deep learning methods for the control process of autonomous driving have not been thoroughly studied. For example, in the United States alone, an average of 6 million car accidents occur each year, in which about 3 million people are injured and about 2 million suffer permanent injuries\textsuperscript{\cite{ aydelotte2017crash}}. Therefore, the proposed automatic driving hazard detection technology can not only improve the safety of driving, but also can greatly reduce the workload of drivers. Currently, automatic driving technology is mainly researched from road detection, expression recognition, and speech recognition.

The challenging task of developing an automated driving system involves sensing and detecting the road so that the driver can receive immediate alerts of any potential dangers and risks. The authors in \textsuperscript{\cite{xiao2018hybrid}} proposed a novel hybrid CRF model for road detection by considering contextual correlations between different modalities. RBNet\textsuperscript{\cite{chen2017rbnet}} learned an advanced CNN feature that allows one-step detection of road regions and their edges. In\textsuperscript{\cite{teichmann2018multinet}}, the authors proposed a joint encoder-decoder scheme which has higher accuracy. \textsuperscript{\cite{bayoudh2021transfer}} combines pre-training parameters from the EfficientNet-B0 baseline \textsuperscript{\cite{tan2019efficientnet}} and a shallow 3D CNN to maximize the overall performance of road segmentation.

On the other hand, it is to detect the driver's state. For example, the driver's expression or drowsiness level.Lyu et al.\textsuperscript{\cite{lyu2018long}} proposed a deep framework based on multi-granularity by intelligently using CNNs and LSTMs for drowsiness detection in videos. Vijayan and Sherly\textsuperscript{\cite{ vijayan2019real}} proposed three CNN architectures including ResNet50, VGG16, and InceptionV3 for sleepiness detection in first-person driver videos. These models are trained by fusing them together using a feature fusion architecture layer. A similar approach was taken by Park et al.\textsuperscript{\cite{ park2016driver}}, who integrated the results obtained by AlexNet, VGG-FaceNet and FlowNet through a fully connected layer for sleepiness detection. Facial expressions include happiness, surprise, anger, sadness, fear, and disgust\textsuperscript{\cite{priestley1999expression}},and these responses are important for driving vehicle safety. Driver emotions should be able to support driving abilities such as attention, good judgment, sound decision making, and fast response time\textsuperscript{\cite{ eyben2010emotion}}.
Agrawal et al.\textsuperscript{\cite{agrawal2013emotion}} uses a fuzzy rule-based system (FBS) to handle simultaneous facial gesture tracking and emotion recognition, and the car switches to automatic mode when any sign of driver inattention or driver fatigue.  Goutam et al. \textsuperscript{\cite{sahoo2023performance}} used the parameters of AlexNet, SqueezeNet and VGG19 models through transfer learning to develop a facial expression recognition system that will monitor the driver's facial expressions to recognize their emotions and provide instant help for safety.

The aim of this paper is to develop a multimodal driver assistance detection system that conforms road condition video data with expression video data and audio data






Advanced driver assistance systems (ADAS), a frontier autonomous driving technology, have immense potential to revolutionize transportation.However, accurately perceiving and identifying driving anomalies is a major challenge. Driving scenarios follow a long-tailed distribution, with normal conditions dominating, while a vast array of anomalous events (e.g., sudden lane changes, unexpected obstacles, driver fatigue/illness) constitute the minority. Although rare, handling these anomalies is crucial for autonomous systems\textsuperscript{\cite{9111005}}. 
\\
Current research on detecting driving incidents largely relies solely on the single modality of road condition video data recorded by dashboard camera\textsuperscript{\cite{9022086}}. These studies mainly focus on identifying abnormal frames in the video data. However, due to the long-tailed distribution of driving events, existing training data struggles to cover all rare driving scenarios, and the immense uncertainty of the real world also implies an inability to predict all possible occurrences\textsuperscript{\cite{7736125}}. In fact, some studies indicate that in the field of autonomous driving, vehicles need to undergo tens of billions of miles of testing to generate sufficient data on rare occurrences to meet 
training\textsuperscript{\cite{KALRA2016182}}.
\\
Relying solely on single-modality data for judgment, if errors occur, will lead to extremely severe consequences. Moreover, considering road safety and relevant laws and regulations, no country has yet achieved true driverless driving, with current autonomous driving systems still relying on the driver as the primary entity, serving an assistive role. Therefore, incorporating other driving information such as driver video and audio for auxiliary judgment not only allows for more accurate identification of various incidents but also aligns with the current practical application scenario demands. Even for some extremely rare driving incidents, although training data cannot fully simulate them, relying on human drivers' experience and common sense can still enable rapid reactions, providing assistance for judging the current driving status\textsuperscript{\cite{7972192}}.\\
Based on the above considerations, this paper proposes a multimodal data input assisted driving model to determine the current driving state. The model simultaneously utilizes road condition video data, driver facial video data, and driver audio data as inputs to avoid potential misjudgments caused by a single data source and mitigate the impact of the long-tailed distribution of driving events.
\\
Multimodal methods are a class of techniques that operate jointly on multiple data types. In our model, we primarily investigate data composed of an audio modality and two visual modalities, employing data fusion techniques to consider the interplay between different modalities. Early fusion may not be suitable for extremely dissimilar data types, while late fusion mainly considers the independently learned features of each modality. For audio-visual multimodal data, intermediate feature fusion could potentially yield better performance\textsuperscript{\cite{9956592}}.
\\
In recent transformer-based architectures for multimodal data recognition tasks, most research has focused on utilizing pre-extracted features, which are subsequently fused with the learning model, rather than creating end-to-end trainable models\textsuperscript{\cite{Tsai2019MultimodalTF}} \textsuperscript{\cite{N2020MultimodalER}}. This limits the applicability of such methods in real-world scenarios, especially in assisted driving scenarios where feature extraction poses significant challenges and introduces uncertainty into the entire processing pipeline. This issue is particularly pronounced for methods using audio information, as audio signal records are rarely available in practical applications and need to be estimated separately\textsuperscript{\cite{Tsai2019MultimodalTF}} \textsuperscript{\cite{N2020MultimodalER}}. Therefore, our work constructs an end-to-end model (without the need for pre-learned features) and performs fusion at the intermediate level, demonstrating the model's robustness to incomplete or noisy data. samples\textsuperscript{\cite{Bondi:2018:AWS:3209811.3209880}}.
\begin{figure}[htbp]
  \centering  \includegraphics[width=0.7\textwidth]{structure.png}
  \caption{End-to-end recognitions framework}
  \label{fig:end}
\end{figure}
\\
Currently, there is a lack of publicly available datasets containing road condition video data, driver facial video data, and driver audio data simultaneously. Consequently, this paper establishes a three-modality dataset by using the Arisim driving simulation project and Logitech driving simulation devices to conduct simulated driving while recording road condition video data, driver facial video data, and driver audio data during the simulated driving process, thereby generating the required data.\\
The main contributions of this paper are as follows:
\sloppy
\begin{itemize}
\item We propose an innovative multimodal fusion framework that incorporates road condition video data, driver video data, and driver audio data as three input modalities, achieving end-to-end learning from raw data without relying on separate feature extraction.
\item We adopt an attention-based intermediate fusion strategy, pairwise fusing different modalities, which effectively captures cross-modal correlations and improves the utilization efficiency of multimodal information.
\item We simulate the driving process and generate an annotated dataset containing three modalities, providing a valuable data resource for multimodal driving behavior analysis.
\end{itemize}
\section{Related work}
Currently, the most objective and commonly used VAD datasets are mainly sourced from various surveillance cameras. For example, the UCSD\cite{data1} and ShanghaiTech \cite{data2} datasets have recorded anomalous situations under different lighting conditions and scenes through campus surveillance cameras. The UCF-Crime \cite{data3} dataset focuses on public safety, including anomalous events such as traffic accidents, burglaries, and explosions.


With the development of assisted driving technology, anomaly detection in first-person perspective traffic videos has attracted widespread attention. Chan et al.\cite{data4} proposed a dataset of road traffic accident videos recorded by dash cameras, with anomalous frames annotated. {data5} collected 1,500 video clips of road anomalies recorded by dash cameras from East Asia on YouTube, with the start and end times of anomalous events labeled. Additionally, \cite{9022086} extracted 803 clips from the BDD100K\cite{data6} dataset to construct a new dataset.

In the realm of driver video detection, L. Lyu et al. \cite{lyu2018long}presented the NTHU-DDD dataset, which is designed to detect driver drowsiness. This dataset captures a variety of driver behaviors under different lighting conditions, including normal driving, yawning, slow blinking, and dozing off.The KMU-FED dataset \cite{sahoo2023performance} captures the facial expressions of drivers using an infrared camera while driving, including 55 image sequences from 12 subjects.Although there is currently a lack of publicly available annotated datasets specifically targeting facial expressions and speech anomalies of drivers, there are many valuable datasets in the field of human video and audio emotion classification research that can serve as references. For example, video datasets annotated based on discrete emotion classification, such as RAVDESS and RAV-DB\cite{RAVDESS}, provide valuable resources for research. Additionally, some datasets employ more complex emotion classification methods, such as valence and arousal-based emotion classification, like AffectNet\cite{AffectNet} and RECOLA\cite{RECOLA}. These datasets offer important references and insights for exploring anomaly detection in drivers' emotional states.

Extensive research has been devoted to addressing road incident detection and multimodal recognition.These efforts contribute significantly to the development of intelligent transportation systems by leveraging advanced sensor technologies and sophisticated algorithms to promptly identify potential hazards on the road.

The EfficientFace framework used in video branch processing is a highly efficient network structure that balances computational efficiency and recognition accuracy. It utilizes EfficientNet-B5 as its backbone network and employs depthwise separable convolutions to reduce the number of parameters while maintaining a high level of feature extraction capability. Additionally, it incorporates residual connections to facilitate information flow and gradient propagation within the network.

Furthermore, EfficientFace integrates an attention mechanism module, which enhances its ability to detect occluded parts in images and recognize key regions. The aim of EfficientFace is to provide a compact and efficient image detection framework by optimizing the sharing and learning of features of different scales and modes within the network structure\cite{Wang2023EfficientFace}.

Mel-Frequency Cepstral Coefficients (MFCC) are a commonly used acoustic feature in tasks like speech recognition, speaker identification, and other speech-related tasks. MFCC is used to extract features from speech data by first taking the logarithm of the speech spectrum on the Mel frequency scale, and then applying a discrete cosine transform to the processed results to obtain the cepstral coefficients\cite{MFCC}.

% The process of extracting MFCC is as follows:

% \begin{enumerate}
%     \item \textbf{Pre-emphasis}: Apply a high-pass filter to the speech signal to enhance the high-frequency parts of the speech data.
%     \item \textbf{Framing}: Divide the speech signal into short frames, each with a length of about 20 to 30 milliseconds.
%     \item \textbf{Windowing}: Apply a Hamming window to each speech segment that was framed in the previous step.
%     \item \textbf{Short-Time Fourier Transform (STFT)}: Perform a Fast Fourier Transform (FFT) on the signal data processed by windowing to obtain the spectrum.
%     \item \textbf{Mel-scale Filterbank}: Pass the transformed spectrum data through a set of triangular filters to obtain the Mel spectrum.
% \end{enumerate}

% MFCC is an auditory feature that fully considers the hearing characteristics of the human ear, making the data processing more aligned with the human auditory perception process. MFCC is widely used in tasks such as speech emotion recognition and speaker recognition. By processing speech data in this way to extract MFCC features, it not only retains the important and key information of the speech signal but also enhances the discriminability of the extracted features, thus performing excellently in various speech processing tasks.

For intermediate feature fusion, the commonly used approach involves sharing features at the intermediate layers of a neural network. This process begins with performing feature extraction on each modality or data source separately. After extracting the features, they are fused to jointly learn the feature representations of different modalities. The fusion methods include concatenation, addition, weighted average, and the attention mechanism, which will be introduced later. Using intermediate feature fusion can retain the feature representations of each modality or data source to a certain extent, thereby preserving their advantages. Additionally, the choice of fusion layers is relatively flexible.\cite{9068523}.

In the field of modality fusion, the use of self-attention mechanisms is currently a relatively effective fusion method\cite{vaswani2017attention}. The formula \( A_n \) represents the output of a self-attention mechanism, which is a commonly used technique in sequence modeling. This formula calculates attention weights by applying the softmax function to the dot product of the query vectors \( q \) and the key vectors \( k \), and then these weights are applied to the value vectors \( v \), resulting in a weighted output.

Specifically, \( qk^T \) captures the similarity between the queries and the keys, computed via the dot product. This is then divided by \( \sqrt{d} \), where \( d \) represents the dimensionality of the latent space, to control the scaling of the variables before the softmax function, aiding in gradient stability. The softmax function is then applied to ensure that the sum of all weights equals 1, representing a probability distribution.

Thus, \( A_n \) reflects a weighted representation of the input features after undergoing a series of learnable transformations, allowing the model to dynamically allocate importance among different parts of the input. This mechanism is a cornerstone of the Transformer architecture, which has demonstrated superior performance across a variety of tasks and applications.

\begin{equation}
A_n = \text{softmax}\left(\frac{qk^T}{\sqrt{d}}\right)v,
\end{equation}
where \( d \) is the dimensionality of a latent space, for the vector \( d \) is the dimensionality of the vector, and for matrix the \( d \) respresents the dimensionality of the column of the matrix.Considering the task of fusion of two modalities \( a \) and \( b \), self-attention can be utilized as a fusion approach by calculating queries from modality \( a \) and keys and values from modality \( b \). This results in representation learnt from modality \( a \) attending to corresponding modality \( b \) and further applying the obtained attention matrix to the representation learnt from modality \( b \).

This process of calculating attention weights and applying them to the value vectors allows the model to focus on different parts of the input sequence with varying degrees of importance, thereby capturing complex dependencies and relationships within the data.And the self-attention mechanism (Self-Attention) has achieved remarkable performance in multimodal fusion in recent years.

% The self-attention mechanism can adaptively capture long-range dependencies between different modalities and dynamically modulate the interactions among modal features, thereby more effectively modeling cross-modal correlations. Specifically, self-attention computes the similarities between Query, Key, and Value to perform a weighted sum of features at different positions, capturing global information. Multi-Head Attention further enhances the model's expressive power by computing attention from different subspaces. The self-attention mechanism addresses the limitations of convolutional neural networks in modeling long-range dependencies, demonstrating outstanding multimodal fusion capabilities.


\newpage
\section{Model}
In this section we wiil present the Tri-modal (driver audio - driver video - road video) recognition framework. As shown in Figure 1, the model is divided into three branches that learn the feature representations of the driver's audio data, the driver's visual data, and the remote sensing visual data of the road information, respectively. We adopt a mid-fusion mechanism to fuse any two modalities' features and jointly learn the fused feature representations. In all three branches, we utilize 1D convolutional blocks to capture temporal information. The detial structure of each branch is shown in Table 1.
\\
Next I will introduce the model from the three modality branches.
\\
\textbf{Face Video:}\\For the driver face video branch of the model, in order to implement an end-to-end trainable model that is capable of learning from raw video, we have incorporated the feature extraction part as a component of our pipeline and optimized it together with the multimodal fusion module. We employed a feature extractor based on the EfficientFace framework, which utilizes 1x1 pointwise convolutions to expand the channel dimension of the input feature maps, followed by depthwise convolutional layers that apply an independent convolutional kernel to each input channel, extracting local features within the channels. The output feature maps are then compressed back through another set of 1x1 pointwise convolutions. Additionally, residual connections within each feature extractor block are used to promote the backward propagation of gradients and alleviate the vanishing gradient problem commonly encountered in deep networks. After extracting features from each frame of the driver face video using the feature extractor, these are fed into subsequent 1D convolutional blocks. As opposed to the traditional approach of processing video data with 3D convolutional blocks, we opt for extracting facial features with the feature extractor and temporal features with 1D convolutional blocks. This method not only reduces computational overhead but is also sufficient for our model, which focuses on real-time assessment of the current driving condition for potential hazards; hence, the sequential information of the entire video is not significantly crucial for our task, and 1D convolutional blocks are adequate for capturing the necessary temporal information.
\\
Although our model is an end-to-end model and the entire system learns and optimizes simultaneously, the facial visual feature extraction part of the model can still be considered as an independent component. The second part of the model only requires the features extracted by the feature extraction part without concerning how the features were extracted. Therefore, in our model, the feature extraction portion can be viewed as a replaceable module. In the second part, we further use convolutional blocks to learn temporal features, each consisting of a 1D convolutional layer with 3×3 kernels, batch normalization, and ReLU activation. Further details can be seen in Table 1, where k denotes the kernel size, d represents the number of filters in the convolutional layers, and s signifies the stride. The convolutional blocks are grouped into two stages, used for the further described multimodal fusion.
\\
\textbf{Road Video:}
\\
For the road video branch of the model, considering the need for data fusion, we employed a structure similar to that of the driver's face video branch to facilitate the fusion of data. Additionally, we perform channel shuffling operations, where the feature maps are divided into several groups along the channel dimension, and then these groups of feature maps are concatenated along the same dimension. This helps facilitate inter-group information exchange and implements attention mechanisms in both the channel and spatial dimensions. By learning to generate weights for the channel and spatial dimensions, the feature maps are adaptively calibrated to emphasize channels and areas with more information.
\\
\textbf{Driver Audio:}
\\
For the driver's audio branch of the model, we similarly employ four convolutional blocks, each comprised of a 1D convolutional layer, batch normalization, a ReLU function, and a max-pooling layer. We utilize Mel-frequency cepstral coefficients (MFCCs) as audio features. Initially, the speech signal is subjected to high-pass filtering, followed by framing and windowing. Subsequently, a fast Fourier transform is applied to each frame to obtain the spectrum. The spectrum is then filtered through a Mel filter bank, and the output is transformed via the discrete cosine transform (DCT) to obtain the MFCCs.
\begin{figure}[htbp]
  \centering  \includegraphics[width=0.7\textwidth]{model.png}
  \caption{Tri-modal recognitions framework}
  \label{fig:tri-modal recognitions framework}
\end{figure}
\begin{table}[h!]
\centering
\caption{Architecture of the visual and audio modules}
\label{tab:table1}
\begin{tabular}{@{}ll@{}}
\toprule
\multicolumn{2}{c}{\textbf{Architecture of the video(road) branch}} \\
\midrule
Stage1 & \begin{tabular}[c]{@{}l@{}}efficientRoad module\end{tabular} \\
\midrule
Stage2 & \begin{tabular}[c]{@{}l@{}}Conv1D [k=3, d=64, s=1] + BN1D + ReLU\\ Conv1D [k=3, d=64, s=1] + BN1D + ReLU\end{tabular} \\
\midrule
Stage3 & \begin{tabular}[c]{@{}l@{}}Conv1D [k=3, d=128, s=1] + BN1D + ReLU\\ Conv1D [k=3, d=128, s=1] + BN1D + ReLU\end{tabular} \\
\midrule
Predict & Global Average Pooling + Linear \\
\midrule
\multicolumn{2}{c}
{\textbf{Architecture of the video(driver) branch}} \\
\midrule
Stage1 & \begin{tabular}[c]{@{}l@{}}efficientFace module\end{tabular} \\
\midrule
Stage2 & \begin{tabular}[c]{@{}l@{}}Conv1D [k=3, d=64, s=1] + BN1D + ReLU\\ Conv1D [k=3, d=64, s=1] + BN1D + ReLU\end{tabular} \\
\midrule
Stage3 & \begin{tabular}[c]{@{}l@{}}Conv1D [k=3, d=128, s=1] + BN1D + ReLU\\ Conv1D [k=3, d=128, s=1] + BN1D + ReLU\end{tabular} \\
\midrule
Predict & Global Average Pooling + Linear \\
\midrule
\multicolumn{2}{c}{\textbf{Architecture of the audio branch}} \\
\midrule
Stage1 & \begin{tabular}[c]{@{}l@{}}Conv1D [k=3, d=64] + BN1D + ReLU + MaxPool1d [2x1]\\ Conv1D [k=3, d=128] + BN1D + ReLU + MaxPool1d [2x1]\end{tabular} \\
\midrule
Stage2 & \begin{tabular}[c]{@{}l@{}}Conv1D [k=3, d=256] + BN1D + ReLU + MaxPool1d [k=2]\\ Conv1D [k=3, d=128] + BN1D + ReLU + MPool1D [k=2]\end{tabular} \\
\midrule
Predict & Global Average Pooling + Linear \\
\bottomrule
\end{tabular}
\end{table}
\\
\textbf{Intermediate Attention-Based Fusion:}
\\
 In this paper, we propose a fusion method based on the attention mechanism for the trifold modality data input to our model. Given the feature representations of the three modalities as $\phi_{a}$, $\phi_{v}$, and $\phi_{r}$, similar to the conventional attention mechanism, we calculate queries and keys, and learn the corresponding weight matrices within the model. For the feature representations of these three modalities, we compute the similarity matrices in pairs, with the process as follows.
\begin{equation}
A_{ij} = \text{softmax}\left(\frac{\Phi_i W_q W_k^T \Phi_j^T}{\sqrt{d}}\right), \quad \text{for } i, j \in \{a, v, r\}
\end{equation}
The softmax activation introduces a competitive mechanism within the attention mapping matrix, effectively highlighting the most critical features or temporal steps. By applying softmax normalization to each feature across the modalities, the model assigns a significance score for each key relative to each query within every modality. This means that the model can assess the importance of each feature in modality a relative to modality b. Consequently, we can aggregate the significance scores for each attribute with respect to the overall attributes of modality b, providing a composite score for each feature within modality a. The resulting attention vector is capable of accentuating prominent features in modality a, which is crucial for the model when processing and analyzing multimodal information.
\\
The fusion approach we employ does not amalgamate features of the two modalities directly. Instead, it pinpoints the most salient attributes within each modality and aligns them with similarity scores derived from the data of another modality. Thus, features that are consistent across modalities substantially contribute to the final prediction, steering the model towards learning features that are pivotal for decision-making or those that exhibit a high degree of consistency between features. This methodology facilitates the sharing of information between modalities without enforcing a strong dependency on the features learned in different branches, using only attention scores for the purpose of fusion. This strategy not only enriches the model's interpretability but also bolsters its capacity to discern nuanced patterns intrinsic to multimodal datasets.
\section{Dataset}
To explore and develop multimodal models capable of real-time assessment of driving safety, there is a need for tri-dimensional data encompassing facial videos, audio, and forward-facing vehicle camera footage during driving, with simulated and annotated scenarios representing both normal and potentially hazardous driving conditions.
Given the current absence of publicly available datasets of this nature, we have undertaken the creation of such a dataset to fill this gap. Our dataset integrates simulated driving environments with real-world driving scenarios, aimed at providing a rich data resource for the training and validation of multimodal driving behavior analysis models. We utilized the open-source project AirSimNH from Airsim and the popular video game "Need for Speed" as platforms for data generation.
\subsection{Data Generation Environment}
We employed the virtual environment provided by the AirSimNH project within AirSim, utilizing Airsim's Python interface for environmental control and data acquisition. Within this setting, road obstacles were periodically and randomly generated on the driving path to simulate sudden incidents under various driving conditions. Additionally, high-speed driving scenarios from "Need for Speed" were used to augment data diversity, particularly for simulating high-risk driving behaviors.
\\The dataset comprises three types of files: facial videos, facial audio, and forward-facing vehicle camera footage. Each file type is sampled and saved in 3-second units. Notably, scenes involving accidents or potential hazardous driving conditions are marked in the filename to indicate a dangerous driving situation
\subsection{Dataset Structure and Annotation}
 Facial Video Files (format: 01\_01\_1.mp4): These record the simulated driver's facial expressions and head movements, where the first '01' in the filename denotes the sequence number of data collection, the second field '01' or '02' represents safe and dangerous driving conditions respectively, and the final digit '1' indicates the first sample in the sequence.\\
Facial Audio Files (format: 01\_01\_1.wav): These are audio files recorded synchronously, capturing the simulated driver's vocal reactions, with a file naming structure identical to that of the facial videos.\\
Road Condition Video Files (format: 01\_01\_1\_r.mp4): Captured from the perspective of a dashcam, these provide visual information about the driving environment and conditions, with a file naming structure similar to the facial videos but with an added '\_r' suffix for differentiation.
Samples representing accidents or potential hazardous situations are marked with '02' in their filenames, indicating a dangerous driving condition. This clear annotation facilitates subsequent model training and evaluation, enabling researchers to quickly identify and utilize these critical data points.
\subsection{Data Quality and Application Potential
}
Throughout the data collection process, we have maintained strict control over the quality of the simulated environments and the authenticity of the data, ensuring the dataset accurately reflects the multimodal characteristics of various driving conditions. This dataset not only offers valuable resources for research into hazardous driving detection but also serves as an experimental foundation for developing and testing multimodal driving behavior analysis models. It is particularly suited for studies involving multimodal machine learning models, real-time monitoring of driver states and external driving environments in autonomous driving systems, and especially in detecting and preventing hazardous driving behaviors.
\section{Experiment}
In the construction of a tri-modal model for driver state monitoring, employing the same preprocessing method and feature extraction architecture for both the driver's facial video and the road condition video is justified. This approach ensures consistency in the feature space for visual information from different sources, facilitating the model's ability to integrate and interpret these inputs effectively. The unified framework enhances model stability and performance and offers convenience for future modal additions or adjustments.
\subsection{Audio Preprocessing}

The audio files underwent a preprocessing routine to ensure uniformity in length, encompassing the following steps:

\begin{enumerate}
    \item \textbf{Audio Loading:} Audio files were loaded using the \texttt{librosa.core.load} function with a standard sampling rate of 22,050 Hz, ensuring a consistent sampling rate across different audio files.
    \item \textbf{Length Normalization:} The audio files were processed to maintain a fixed length of 3.6 seconds. If the audio was shorter than 3.6 seconds, zeros were padded at the end to extend the length; if it was longer, the audio was trimmed equally from both ends to ensure consistency in length.
    \item \textbf{Audio Saving:} The modified audio was saved, keeping the original audio intact while creating a version that meets the length requirement.
\end{enumerate}

\subsection{Video Preprocessing}

The video files were also preprocessed to meet the model input requirements, involving the following steps:

\begin{enumerate}
    \item \textbf{Frame Selection and Processing:} For each video, a distributed selection of 15 frames was extracted for subsequent facial detection and cropping. This strategy aimed to uniformly sample frames from the video, regardless of its original frame count.
    \item \textbf{Facial Detection:} The MTCNN (Multi-task Cascaded Convolutional Networks) was employed to detect faces in the video frames and crop the facial regions. This step focused the model's attention on facial expressions in the videos, enhancing the accuracy of emotion recognition.
    \item \textbf{Frame Cropping and Resizing:} The detected facial regions were cropped and resized to 224x224 pixels, ensuring uniformity in all video frames to satisfy the subsequent model input requirements.
    \item \textbf{Data Saving:} The processed video frames were saved in array format, facilitating further model training and validation. Additionally, the processed frames could be encoded back into AVI format videos if required.
    \item \textbf{Processing Record:} Details of the entire processing routine, including the number of files successfully processed and a list of failed videos, were documented for further analysis and review.
\end{enumerate}
Through these steps, uniformity and standardization of the audio and video data were ensured, preparing a well-curated dataset for subsequent model training and validation. This preprocessing strategy aids in enhancing the model's performance by reducing the variability in the data that the model needs to address, allowing it to focus more on learning the core features of the data.

\subsection{Training Process}
In the experimental section of this paper, we initially utilize predefined loaders to load preprocessed driver facial video data, driver audio data, and road condition information data, and perform data augmentation on these datasets. In our experiments, we employ various masking techniques for data augmentation, enhancing the model's robustness by mixing and modifying multimodal input data (audio, visual, road conditions). This method adjusts the blend of each modal input through randomly generated coefficients and introduces all-zero data to simulate extreme noise conditions. Additionally, by combining and reordering these coefficient-weighted and unweighted input data, we generate a rich and varied set of training samples. This technique not only increases the model's tolerance to noise but also improves its adaptability in the face of data incompleteness, thereby enhancing the model's generalization capabilities.
\\
For feature extraction from road conditions and the driver's video data, we employ depthwise separable convolutions based on the EfficientFace framework to reduce model parameters and accelerate training. Residual connections are also used to speed up convergence. Moreover, self-attention mechanisms are incorporated in the feature extraction of video data to extract crucial information and efficiently capture the spatiotemporal relationships between video frames.
\\
For audio data, we utilize one-dimensional convolutional layers to capture the temporal dependencies and local features within the audio data, and employ batch normalization to accelerate the training process.
\\
Further, we apply attention mechanisms for the mutual fusion of audio, driver's facial video, and road condition video information and use the fused information for decision-making.
\\
During the training process, we use the cross-entropy loss function and the Stochastic Gradient Descent (SGD) optimizer, replacing traditional gradient descent methods with momentum to update network parameters. This helps avoid the influence of local minima and ensures the effectiveness and efficiency of the learning process. Moreover, we introduce a dynamic learning rate scheduling strategy that adaptively adjusts the learning rate based on training progress, accelerating convergence while avoiding premature local optima. To further enhance the model's generalization capability, we integrate the dropout regularization mechanism, effectively reducing the risk of overfitting.


\begin{figure}[htbp]
  \centering  \includegraphics[width=0.8\textwidth]{plot_2d.png}
  \caption{Traing Process}
  \label{fig:loss-accuracy}
\end{figure}


To comprehensively evaluate the overall performance of the model, we meticulously design three ablation experiments. These experiments aim to compare the specific impacts of different modal inputs (including driver facial images, audio signals, and road videos) on the final model performance and validate the effectiveness of our proposed multimodal model. Models are trained using solely single modalities (driver facial, audio, or road video), a bimodal model combining video and audio, and a complete multimodal model integrating driver facial, audio, and road video information. To ensure the effectiveness and reliability of the performance comparison, all models are trained under consistent conditions in the experimental setup. Through this systematic comparison, we are able to thoroughly evaluate the contribution of each input modality to the final model performance and how they work together to improve classification accuracy.
\begin{figure}[htbp]
  \centering  \includegraphics[width=0.8\textwidth]{plot_3d.png}
  \caption{Traing Process}
  \label{fig:loss-accuracy}
\end{figure}
\section{Result}
Comprehensive experimental evaluations conducted on the constructed dataset indicate that the proposed three-dimensional model is capable of achieving outstanding performance in dangerous driving state recognition. The results of the comparative experiments are displayed in Table 2.

Specifically, the model attained a 96.875\% accuracy rate in the task of dangerous driving state recognition, better than the contrastive models that we set which are trained solely on any two-dimensional data (such as driver video+audio, driver video+road video, etc.) or one-dimensional data (such as driver video alone, road video alone, driver audio alone). This result underscores that use triflod-multimodal that we built can  the effectiveness and necessity of multimodal data fusion in this task.
\begin{table}[ht]
\centering
\caption{Experiment Result}
\resizebox{0.7\textwidth}{!}{
\begin{tabular}{lccccccc}
  \toprule
   & A-V-R & A-V & A-R & V-R & A & V & R \\
  \midrule
  accuracy(\%) & 96.875 & 93.75 & 81.25 & 90.625 & 46.875 & 87.5 & 78.125 \\
  loss & 0.0122 & 0.0254 & 0.4135 & 0.1911 & 0.6715 & 0.2478 & 0.5636 \\
  \bottomrule
\end{tabular}
}
\end{table}
% \begin{table}[h!]
% \centering
% \caption{Comparison of Model Performance}
% \label{tab:model_performance}
% \setlength{\tabcolsep}{14pt} % Increase the space between columns
% \renewcommand{\arraystretch}{1.5} % Increase the space between rows
% \begin{tabular}{@{}lcc@{}}
% \toprule
% Model         & Loss   & Accuracy (\%) \\ \midrule
% WideBranchNet & 0.0640 & 92.45         \\
% ResNet152     & 0.4191 & 86.12         \\
% ResNet101     & 0.4122 & 86.10         \\
% ResNet50      & 0.5146 & 82.72         \\
% Our Model     & 0.0122 & 96.875        \\ \bottomrule
% \end{tabular}
% \end{table}

Further analysis reveals that the three modalities of data play a complementary and synergistic role in recognition. The information contained within driver video data, such as facial expressions, gaze, and head posture, can reflect the driver’s level of attention and fatigue. When the driver exhibits states of fatigue or distraction, their facial expressions may become dull, their gaze unfocused, and frequent changes in head posture may occur. 

In the event of a driving accident, the driver may exhibit dramatic changes in facial expression in a short time, potentially showing focused gaze with dilated pupils, along with changes in head posture. Capturing and understanding these subtle visual cues is crucial for timely identification of dangerous driving states.

Audio data provides another important dimension. The driver’s voice, breathing sounds, and yawning noises contain key clues about their emotional state and level of alertness. For example, when a driver's emotions are heightened and their speech rate increases, it often indicates a state of tension or anxiety; conversely, slower breathing and yawning sounds suggest the risk of fatigued driving. Through voice emotion recognition and acoustic event detection technologies, we can extract these critical pieces of information from audio data to aid in the judgment of dangerous driving states.

Road video data, from the perspective of the external environment, provides crucial information about the driving scene in which the vehicle is situated. Through the analysis of road video, we can gather a series of indicators such as the vehicle’s speed, the distance to the vehicle ahead, and lane deviation. These indicators collectively reflect the driver's behavior and the potential risks they face. For instance, frequent and prolonged lane deviations suggest that the driver is not concentrating and may engage in dangerous driving.

The fusion of driver video, audio, and road video modalities enables a comprehensive portrayal and understanding of driving risks from multiple perspectives, including the driver’s state, driving behavior, and driving environment. The combination of these three enhances the model's capability for scene understanding and state judgment. The complementary and synergistic information within multimodal data significantly improves the accuracy, reliability, and robustness of dangerous driving state recognition, providing a more comprehensive and dependable perception capability for intelligent driving safety assistance systems.

To compare our model with current video anomaly detection models, we replicated several anomaly detection models, including AstNet\cite{le2023attention} and WideBranchNet, and tested them on our custom dataset. We used our own road condition video data to evaluate these models. Since different anomaly detection models employ varying input architectures, we uniformly utilized a data loading method tailored to our simulated driving dataset. Table 6.2 displays the performance of these models and our three-dimensional model during testing.
% \begin{table}[h!]
% \centering
% \caption{Comparison of Model Performance}

% \label{tab:model_performance}
% \begin{tabular}{@{}lcc@{}}
% \toprule
% Model         & Loss   & Accuracy (\%) \\ \midrule
% WideBranchNet & 0.0640 & 92.45         \\
% ASTnet(ResNet152)     & 0.4191 & 86.12         \\
% ASTnet(ResNet101)     & 0.4122 & 86.10         \\
% ASTnet(ResNet50)      & 0.5146 & 82.72         \\
% Our Model     & 0.0122 & 96.875        \\ \bottomrule
% \end{tabular}
% \end{table}
\begin{table}[h!]
\centering
\caption{Comparison of Model Performance}
\resizebox{0.5\textwidth}{!}{ % Adjust this value to change the scale of the table
\begin{tabular}{@{}lcc@{}}
\toprule
Model         & Loss   & Accuracy (\%) \\ \midrule
WideBranchNet & 0.0640 & 92.45         \\
ASTnet(ResNet152)     & 0.4191 & 86.12         \\
ASTnet(ResNet101)     & 0.4122 & 86.10         \\
ASTnet(ResNet50)      & 0.5146 & 82.72         \\
Our Model     & 0.0122 & 96.875        \\ \bottomrule
\end{tabular}
}
\end{table}

The training processes of two-dimensional and three-dimensional models, as demonstrated in Figures 1 and 2, show that although both models experienced initial fluctuations, they subsequently converged rapidly to higher levels of accuracy. However, compared to the two-dimensional model, the three-dimensional model exhibited greater stability throughout the training process, and its final training accuracy was generally higher. This suggests that due to the higher complexity of the three-dimensional model, it may possess superior generalization capabilities. During the processing of three-dimensional data, the three-dimensional model is able to learn more features and patterns, thus displaying stronger performance in both training and prediction phases.
\newpage

\section{Conclusion}
We have developed a road incident recognition model that harnesses three distinct data sources: driver video, audio data, and road condition videos. By utilizing an attention-based intermediate layer fusion technique, we have efficiently integrated these three types of data, thus constructing an end-to-end multimodal recognition system. We constructed a simulated driving dataset that includes tridimensional data and tested our proposed multimodal model on it. The test results demonstrate that our trifold modality model significantly outperforms the unimodal and bimodal models in performance, underscoring its potential value in practical applications.

The three modalities of data, namely driver video, audio, and road video, play a complementary and synergistic role in recognition. The information contained in driver video, such as facial expressions, gaze, and head posture, can reflect the driver's level of attention and fatigue. The driver's voice, breathing sounds, and yawning noises in audio data contain crucial clues about their emotional state and level of alertness. Road video data, from the perspective of the external environment, provides key information about the driving scene in which the vehicle is situated. The fusion of these three modalities enables the model to comprehensively portray and understand driving risks from multiple angles, including the driver's state, driving behavior, and driving environment, significantly enhancing the accuracy, reliability, and robustness of dangerous driving state recognition.

Comparative experiments with current video anomaly detection models demonstrate that the proposed three-dimensional model outperforms models such as AstNet and WideBranchNet on the custom-built dataset. This further confirms the advantages of multimodal fusion and three-dimensional modeling approaches in this task.

The comparison of training processes between two-dimensional and three-dimensional models shows that although both models experience fluctuations in the initial training stage, they quickly converge to high levels of accuracy. However, compared to the two-dimensional model, the three-dimensional model exhibits greater stability throughout the training process, and its final training accuracy is generally higher. This suggests that due to the higher complexity of the three-dimensional model, it may possess stronger generalization capabilities. During the processing of three-dimensional data, the three-dimensional model can learn more features and patterns, thus demonstrating superior performance in both training and prediction phases.

The three-dimensional multimodal driving anomaly detection model has demonstrated exceptional performance and has broad prospects for practical applications. In the future, this model can be further improved and expanded in the following aspects:
\begin{itemize}
\item Expanding the dataset: The current model's dataset was collected during our simulated driving process using a driving simulator. To improve the model's robustness and generalization ability, we can collect more real-world driving data, covering a wider range of driving scenarios, driving states, and anomalous situations. This will help the model achieve better performance in practical applications.
\item Introducing more modal data: In addition to driver video, audio, and road video data, we can also introduce other types of sensor data, such as vehicle speed, acceleration, or distance information of road obstacles. These additional data can provide the model with more comprehensive driving state information, which helps to further improve the accuracy of anomaly detection.
\item Optimizing the model structure: We can explore more advanced attention mechanisms and fusion strategies to more effectively integrate information from different modalities.
\end{itemize}
\newpage

\newpage
\bibliographystyle{unsrt}
\bibliography{stability}
	
\end{document}

\begin{table*}[t]

\centering
\vskip 0.15in
\begin{center}
\begin{small}
\begin{sc}
\begin{tabular}{lrrrrrr}
  \toprule
  \multirow{2}{*}{Attack Type} & \multirow{2}{*}{Input} & \multirow{2}{*}{Output} & \multirow{2}{*}{Reasoning} & Reasoning & \multirow{2}{*}{Accuracy} & Contextual \\
  & & & & Increase & & Correctness \\
  \midrule
  No Attack     & 7899$\pm$5797 & 102$\pm$53 & 751$\pm$410\;\; & 1 & 100\% & 100\% \\
  \midrule
  Context-Aware       & 196$\pm$94 & 88$\pm$38 & 589$\pm$363\;\; & 0.9$\times$ & 100\% & 100\% \\  
  Context-Agnostic   & 191$\pm$106 & 101$\pm$44 & 576$\pm$310\;\; & 0.8$\times$ & 100\% & 100\% \\
  ICL-Genetic (Aware) & 162$\pm$75 & 105$\pm$66 & 346$\pm$73\;\; & 0.5$\times$ & 100\% & 100\% \\  
  ICL-Genetic (Agnostic)  & 231$\pm$146 & 108$\pm$98 & 640$\pm$379\;\; & 0.9$\times$ & 100\% & 100\% \\
  \bottomrule
\end{tabular}
\end{sc}
\end{small}
\end{center}
\vskip -0.1in
\caption{Average number of reasoning tokens for o1 after filtering defense (\textbf{Dataset}: FreshQA, \textbf{Decoy}: MDP).}
\label{tab:filtering}
\end{table*}

\section{Spectral Covariance Stability}

\begin{lemma}[From Euclidean to Spectral Stability]\label{lem:stability-frob-to-spectral}
    Let $0 < \e < \delta < 0.1$, $r, R > 0$, 
    $\cB_R = \set{V\in \R^{d\times d} : \normf{V} \le R}$. Suppose that $\set{X_1,\ldots,X_n} \subset \R^{d\times d}$ is an $\Paren{\e,\delta,r,\cB_R,\cB_R\otimes \cB_R}$-stable set with respect to some $M\in \R^{d\times d}$ and $Q\in \R^{d^2\times d^2}$.
    Let $\cS_R = \set{uu^\top \in \R^{d\times d} : \norm{u}\le \sqrt{R}}$ and let 
    \[
    \cP_R = \Set{\pE v^{\otimes 4} : v\in \R^d, \pE \text{ is a degree-$4$ pseudo-expectation that satisfies the constraint } \norm{v}^2 \le R}\,.
    \]
    
    Then for each $Q'\in \R^{d^2\times d^2}$ such that 
    \[
    \Abs{\Iprod{P, Q-Q'}}\le O(\delta^2/\e)\,,
    \]
    $\set{X_1,\ldots,X_n}$ is an $\Paren{\e,\delta,r,\cS_R,\cP_R}$-stable set with respect to $M$ and $Q'$. 
\end{lemma}
\begin{proof}
    Let $\cC_R = \conv\Paren{\cB_R\otimes \cB_R}$. Note that it is the set of $d^2\times d^2$ matrices with trace at most $R^2$. Note that $\cS_R \subset \cB_R$, and $\cP_R \subset \cC_R$, since $\sum_{i=1}^d\sum_{j=1}^d \pE u_iu_j \cdot u_iu_j = \pE \sum_{i=1}^d u_i^2 \cdot \sum_{j=1}^d u_j^2 \le R^2$.
    
    Since maximum and minimum of a linear function over a convex set is achieved in its extreme poitns,  $\set{X_1,\ldots,X_n}$ is an $\Paren{\e,\delta,r,\cB_R,\cC_R}$-stable set with respect to $M$ and $Q$, and hence of an $\Paren{\e,\delta,r,\cS_R,\cP_R}$-stable with respect to $M$ and $Q$. 
    
    Finally, it is clear that $Q$ can be replaced by any $Q'$ such that $\Abs{\Iprod{P, Q-Q'}}\le O(\delta^2/\e)$.
\end{proof}

\begin{corollary}\label{cor:spectral-stability}
Let $\Sigma \in \R^{d\times d}$ be a positive definite matrix such that $0.9 \cdot \Id_d \preceq \Sigma \preceq 1.1\cdot \Id_d$. Let $y_1,\ldots,y_n\simiid N(0,\Sigma)$, $\Sigma' = \Cov_{y\sim N(0,\Sigma)} \cF(y)$, and let $\zeta_i = \Paren{\Sigma'}^{-1/2}\cF(y_i)$. Let $R \le O(1)$.

If $\e\log(1/\e) \gtrsim 1/d$ and $n\gtrsim d^2\log^5(d)/\e^2$, then $\zeta_1\zeta_1^\top,\ldots,\zeta_n\zeta_n^\top$ is an $\Paren{\e,\delta,r,\cS_R, \cP_R}$-stable set with respect to $M=\Id_d$ and $Q' = 2\Id_{d^2}$, for some $\e\le \delta \le O(\e{\log(1/\e)})$ and $r \le O(1)$.
\end{corollary}
\begin{proof}
By \ref{lem:frobenius-stability-isotropic}, with high probability, $\zeta_1\zeta_1^\top,\ldots,\zeta_n\zeta_n^\top$ is an $\Paren{\e,\delta,r,\cB_R, \cB_R\otimes \cB_R}$-stable set with respect to $M=\Id_d$ and $Q = \E_{y\sim N(0,\Sigma)} \Paren{\Paren{\Sigma'}^{-1/2}\cF(y)\cF(y)^\top\Paren{\Sigma'}^{-1/2} - \Id}^{\otimes 2}$. By \cref{lem:stability-frob-to-spectral}, it is enough to show that 
\[
\Abs{\Iprod{\pE u^{\otimes 4}, Q} - 2} \le O(\delta^2/\e)\,.
\]

By \cref{lem:tensor}, $Q_{iiii} = 2 \pm O(1/d)$, and for $i\neq j$, $Q_{ijij} = 1 \pm O(1/d)$, $Q_{iijj}= \pm O(1/d)$, and other entries are $0$. Hence
\begin{align*}
\Iprod{\pE u^{\otimes 4}, Q} 
&= \sum_{1\le i,j \le d} \pE u_i^2u_j^2 Q_{ijij} + \sum_{1\le i\neq j \le d} \pE u_i^2u_j^2 Q_{iijj} = 2 \pm O(1/d) = 2 \pm O(\e\log(1/\e))\,.
\end{align*}
\end{proof}

\begin{lemma}\label{lem:stability-linear-transformation}
Let $R \le O(1)$. Let $\zeta_1\zeta_1^\top,\ldots,\zeta_n\zeta_n^\top$ be an $\Paren{\e,\delta,r,\cS_R, \cP_R}$-stable set with respect to $\Id_d$ and $2\Id_{d^2}$, for some $\e\le \delta$ and $r\le O(1)$.

Then for each matrix $A\in \R^{d\times d}$ such that $\norm{A}^2\le R$, $A\zeta_1\Paren{A\zeta_1}^\top,\ldots,A\zeta_n\Paren{A\zeta_n}^\top$ is an $\Paren{\e,\delta',r,\cS_1, \cP_1}$-stable set with respect to $M = AA^\top$ and $Q = 2\Id_{d^2}$,
where $r\le O(1)$ and
\[
\delta' = O\Paren{\delta + \sqrt{\e\norm{AA^\top - \Id_d}}}\,.
\]
\end{lemma}
\begin{proof}
Let us check the first property of stability. 
For each unit $u\in \R^{d}$,
\[
\Abs{\tfrac{1}{\card{S'}} \sum_{x\in S'} \iprod{uu^\top, A\zeta_i\zeta_i^\top A^\top-AA^\top}}
= \Abs{\tfrac{1}{\card{S'}} \sum_{x\in S'} \iprod{A^\top uu^\top A, \zeta_i\zeta_i^\top-\Id_d}}
\le O(\delta)\,,
\]
where the inequality follows from the stability of $\zeta_1\zeta_1^\top, \ldots, \zeta_n\zeta_n^\top$.
Let us check the second property. We need to show that
\[
\Abs{\tfrac{1}{\card{S'}} \sum_{x\in S'} \Iprod{\pE v^{\otimes 4}, \paren{A\zeta_i\zeta_i^\top A^\top-AA^\top}^{\otimes 2}} - \Iprod{\pE v^{\otimes 4}, 2\Id_{d^2}}}\le \paren{\delta'}^2/\e\,.
\]
Note that
\begin{align*}
\Iprod {\pE v^{\otimes 4}, \paren{A\zeta_i\zeta_i^\top A^\top-AA^\top}^{\otimes 2}}
&= \pE \Iprod{vv^\top, A\zeta_i\zeta_i^\top A^\top-AA^\top}^2
\\&= \pE \Iprod{\Paren{A^\top v}\Paren{A^\top v}, \zeta_i\zeta_i^\top -\Id_d}^2
\\&= \Iprod{\pE \Paren{A^\top v}^{\otimes 4}, \zeta_i\zeta_i^\top -\Id_d}\,.
\end{align*}
By the stability of $\zeta_1\zeta_1^\top, \ldots, \zeta_n\zeta_n^\top$, 
\[
\Abs{\tfrac{1}{\card{S'}} \sum_{x\in S'} 
\iprod{\pE \Paren{A^\top v}^{\otimes 4}, \zeta_i\zeta_i^\top -\Id_d}
- \Iprod{\pE \Paren{A^\top v}^{\otimes 4}, 2\Id_{d^2}}}\le O\Paren{\delta^2/\e}\,.
\]
Note that
\begin{align*}
\Abs{\Iprod{\pE \Paren{A^\top v}^{\otimes 4} - \pE v^{\otimes 4}, 2\Id_{d^2}}}
&= 2\Abs{\pE \norm{A^\top v}^4 - \pE \norm{v}^4}
\\&= 2\Abs{\pE \Paren{v^\top A A^\top v}^2 - \pE \Paren{v^\top v}^2}
\\&= 
2\Abs{\pE\Paren{{v^\top \Paren{A A^\top - \Id_d} v}}\cdot \Paren{{v^\top \Paren{A A^\top + \Id_d} v}}}
\\&\le \sqrt{\pE\Paren{{v^\top \Paren{A A^\top - \Id_d} v}}^2 \cdot \pE\Paren{{v^\top \Paren{A A^\top + \Id_d} v}}^2}
\\&\le O\Paren{\norm{A A^\top - \Id_d}}\,,
\end{align*}
where we used the \CS inequality for pseudo-distributions, \cref{fact:spectral-certificates} and $\norm{A}\le O(1)$ . Hence the second property is satisfied. The third property is also obviously satisfied.
\end{proof}

\begin{lemma}\label{lem:non-decreasing-covariance}
Let $\zeta_1\zeta_1^\top,\ldots,\zeta_n\zeta_n^\top$ be an $\Paren{10\e,\delta,r,\cS_1, \cP_1}$-stable set with respect to $\Id_d$ and $Q\in \R^{d^2\times d^2}$, where $\e \le \delta \lesssim 1$. Let $\set{z_1,\ldots,z_n}$ be an $\e$-corruption of $\set{\Paren{\paren{\Sigma'}^{1/2}\zeta_1}\Paren{\paren{\Sigma'}^{1/2}\zeta_1}^\top,\ldots, \Paren{\paren{\Sigma'}^{1/2}\zeta_n}\Paren{\paren{\Sigma'}^{1/2}\zeta_n}^\top}$. Let $w\in \cW_\e$, and let 
$\Sigma'' = \frac{1}{\sum_{i=1}^n w_i}\sum_{i=1}^n w_i z_i z_i^\top$. 
Then 
\[
\norm{\Paren{\Sigma''}^{-1/2}\Paren{\Sigma'}^{1/2}}\le O(1)\,.
\]
\end{lemma}
\begin{proof}
Note that it is enough to show that $\Sigma'' \succeq \Paren{1-O(\delta)} \Sigma'$.
Let $G$ be the set of indices where $z$ coincides with $\paren{\Sigma'}^{1/2}\zeta$
Note that
\[
\Sigma'' \succeq \frac{1}{\sum_{i=1}^nw_i}\sum_{i\in G} w_i \paren{\Sigma'}^{1/2}\zeta_i \zeta_i^\top \paren{\Sigma'}^{1/2} \succeq \Paren{1-O(\e)}\sum_{i\in G} w_i \paren{\Sigma'}^{1/2}\zeta_i \zeta_i^\top \paren{\Sigma'}^{1/2}\,.
\]
Let $u\in \R^d$ be an arbitrary unit vector. Since the function 
\[
u^\top\Paren{\Sigma'' - \Paren{1-O(\e)}\paren{\Sigma'}^{1/2}\Paren{\sum_{i\in G} w_i \zeta_i \zeta_i^\top} \paren{\Sigma'}^{1/2}}u
\]
is linear in $w$, its maximum is achieved in one of the extreme points of $\cW_\e$, i.e. in some set $S_u$ of size $\card{S_u}\ge \Paren{1-\e} n$. Denote $G_{u} = G\cap S_u$. By stability,
\[
\Norm{\frac{1}{\card{G_u}}\sum_{i\in G_u} \zeta_i \zeta_i^\top - \Id_d} \le \delta\,.
\]
It follows that
\begin{align*}
0
&\le
u^\top\Paren{\Sigma'' - \Paren{1-O(\e)}\paren{\Sigma'}^{1/2}\Paren{\sum_{i\in G} w_i \zeta_i \zeta_i^\top} \paren{\Sigma'}^{1/2}}u 
\\&\le
u^\top\Paren{\Sigma'' - \Paren{1-O(\e)}\paren{\Sigma'}^{1/2}\Paren{\sum_{i\in G_u} \tfrac{1}{n} \zeta_i \zeta_i^\top} \paren{\Sigma'}^{1/2}}u
\\&\le
u^\top\Paren{\Sigma'' - \Paren{1-O(\e)}\paren{\Sigma'}^{1/2}\Paren{\sum_{i\in G_u} \tfrac{1}{\card{G_u}} \zeta_i \zeta_i^\top} \paren{\Sigma'}^{1/2}}u
\\&\le
u^\top\Paren{\Sigma'' - \Paren{1-O(\e)}\Paren{1-\delta}\paren{\Sigma'}^{1/2}\paren{\Sigma'}^{1/2}}u
\\&\le u^\top\Paren{\Sigma'' - \Paren{1-O(\delta)}\Sigma'}u
\end{align*}
Hence $\Sigma'' \succeq \Paren{1-O(\delta)} \Sigma'$.
\end{proof}

\begin{lemma}
    Let $R\le O(1)$. Let $\zeta_1\zeta_1^\top,\ldots,\zeta_n\zeta_n^\top$ be an $\Paren{10\e,\delta,r,\cS_R, \cP_R}$-stable set with respect to $\Id_d$ and $2\Id_{d^2}$, where $\e \le \delta \lesssim 1$ and $r\le O(1)$. 
    
    Let $\set{z_1,\ldots,z_n}$ be an $\e$-corruption of $\set{\Paren{\paren{\Sigma'}^{1/2}\zeta_1}\Paren{\paren{\Sigma'}^{1/2}\zeta_1}^\top,\ldots, \Paren{\paren{\Sigma'}^{1/2}\zeta_n}\Paren{\paren{\Sigma'}^{1/2}\zeta_n}^\top}$. Let $w\in \cW_\e$. 
    Denote $\Sigma'' = \frac{1}{\sum_{i=1}^n w_i}\sum_{i=1}^n w_i z_i z_i^\top$ and
    $C = \frac{1}{\sum_{i=1}^n w_i}\sum_{i=1}^n w_i \Paren{\Paren{\Sigma''}^{-1/2}z_iz_i^\top \Paren{\Sigma''}^{-1/2} - \Id}^{\otimes 2}$. Suppose that 
    \[
    \max_{P\in\cP_1}\Abs{\Iprod{C - Q,P}}\le \lambda\,.
    \]
    
Denote $A = \Paren{\Sigma''}^{-1/2}\Paren{\Sigma'}^{1/2}$.
Then $A\zeta_1\Paren{A\zeta_1}^\top,\ldots,A\zeta_n\Paren{A\zeta_n}^\top$ is an $\Paren{\e,\delta',r,\cS_1, \cP_1}$-stable set with respect to $M = AA^\top$ and $Q = 2\Id_{d^2}$,
where $r\le O(1)$ and
\[
\delta' = O\Paren{\delta + \e^{3/4}\lambda^{1/4}}\,.
\]
\end{lemma}
\begin{proof}
By \cref{lem:stability-linear-transformation} and \cref{lem:non-decreasing-covariance} $A\zeta_1\Paren{A\zeta_1}^\top,\ldots,A\zeta_n\Paren{A\zeta_n}^\top$ is an $\Paren{\e,O\Paren{\delta + \sqrt{\e}},r,\cS_1, \cP_1}$-stable set with respect to $M = AA^\top$ and $Q = 2\Id_{d^2}$. By \cref{lem:stability},
\[
\norm{AA^\top - \Id_d} \le O\Paren{\delta + \sqrt{\e} + \sqrt{\e \lambda} + \e r}\,.
\]
Therefore, applying \cref{lem:stability-linear-transformation} again, we get the desired result.
\end{proof}

\begin{lemma}\label{lem:adequate}
For each $R > 0$, $\cP_R$ is adequate with respect to $\cS_R$  (see \cref{def:adequate}). 
\end{lemma}
\begin{proof}
Let $A \in \R^{d\times d}$. Then  
$\pE \iprod{A\otimes A, v^{\otimes 4}} = \pE \Paren{v^\top A v}^2 \ge 0$.

Let us prove the second property. Let $A, B\in \R^{d\times d}$. Then
\[
\Paren{\pE \iprod{A \otimes B, v^{\otimes 4}}}^2 = \Paren{\pE \Paren{v^\top A v}\Paren{v^\top B v}}^2 \le \pE \Paren{v^\top A v}^2 \cdot \pE  \Paren{v^\top B v}^2\,,
\]
where we used the \CS inequality for pseudo-distributions. 
By \cref{fact:spectral-certificates}, there is a degree-2 sos proof of the inequalities
\[
-\norm{A}\cdot \norm{v}^2\le {v^\top A v} \le \norm{A}\cdot \norm{v}^2\,.
\]
Hence each degree-4 pseudo-distribution that satisfies the constraint $\norm{v}^2\le R$,
\[
\Paren{\pE \iprod{A \otimes B, v^{\otimes 4}}}^2 \le \pE \Paren{v^\top A v}^2 \cdot \pE  \Paren{v^\top B v}^2\le R^2 \norm{A}^2 \cdot \norm{B}^2 = \Paren{\max_{uu^\top\in\cS_R}\iprod{uu^\top, A}^2} \cdot \Paren{\max_{uu^\top\in\cS_R}\iprod{uu^\top, B}^2}\,.
\]
\end{proof}

\cref{cor:spectral-stability}, \cref{lem:stability-linear-transformation} and \cref{lem:adequate} imply that we can use \cref{thm:filtering} and get the desired estimator in relative spectral norm.
\section{Estimation in Frobenius Norm}

\begin{proposition}\label{prop:Hanson-Wright}
    Let $0 < \e \lesssim 1$ and $R \le O(1)$.
    Let $\cD$ be a distribution in $\R^d$ with zero mean and identity covariance that satisfies $O(1)$-Hanson-Wright property, and let $x_1\ldots,x_n \simiid \cD$ for $n\gtrsim d^2\log^5(d)/\e^2$. 
    Then, with high probability, the set $S = \set{x_1x_1^\top, \ldots, x_nx_n^\top}$ is
 $\Paren{\e,\delta,r,\cB_R, \cB_R\otimes \cB_R}$-stable with respect to $M=\Id_d$ and $Q = \E \Paren{x_ix_i^\top - \Id}^{\otimes 2}$, where $\e \le \delta \le O(\e\log(1/\e))$ and $r\le O(1)$. 
\end{proposition}

\begin{proof}
    Condition 1 of stability follows from $O(1)$-sub-Gaussianity. Note that sub-Gaussianity follows from the Hanson-Wright if we plug $A = uu^\top$ for unit $u \in \R^d$. 
    Let us show condition 2.  
    Let $J\subseteq[n]$ be a set of size $\card{J} \ge (1-\e)n$,  and let $V\in \cB_R$. Note that
    \begin{align*}
    \tfrac{1}{\card{J}} \sum_{i\in J} \iprod{V^{\otimes 2}, \paren{x_ix_i^\top-\Id_d}^{\otimes 2}} 
    = \tfrac{1}{\card{J}} \sum_{x=1}^n \iprod{V^{\otimes 2}, \paren{x_ix_i^\top-\Id_d}^{\otimes 2}} 
    - \tfrac{1}{\card{J}} \sum_{i\in [n]\setminus J} \iprod{V^{\otimes 2}, \paren{x_ix_i^\top-\Id_d}^{\otimes 2}}\,.
    \end{align*}
    By the Hanson-Wright inequality, with high probability all $d^2$-dimensional vectors $x_ix_i^\top-\Id_d$ have norm at most $O(d)$. Hence we can apply matrix Bernstein inequality, and it implies that with high probability the sample covariance is close to the empirical covariance. That is, with high probability, for all $V\in \cB_R$,
    \[
    \Abs{\tfrac{1}{n}\sum_{x=1}^n \iprod{V^{\otimes 2}, \paren{x_ix_i^\top-\Id_d}^{\otimes 2}}
    - \E \iprod{V^{\otimes 2}, \paren{x_ix_i^\top-\Id_d}^{\otimes 2}}} \le O(\e)\,.
    \]

    Note that the Hanson-Wright inequality also implies that $\E \iprod{V^{\otimes 2}, \paren{x_ix_i^\top-\Id_d}^{\otimes 2}} \le O(1)$. Hence 
    \begin{align*}
    \tfrac{1}{\card{J}} \sum_{x=1}^n \iprod{V^{\otimes 2}, \paren{x_ix_i^\top-\Id_d}^{\otimes 2}} &= \Paren{1+O(\e)}\cdot \Paren{\E \iprod{V^{\otimes 2}, \paren{x_ix_i^\top-\Id_d}^{\otimes 2}} + O(\e)} 
    \\&= \E \iprod{V^{\otimes 2}, \paren{x_ix_i^\top-\Id_d}^{\otimes 2}} + O(\e)
    \\&= \iprod{V^{\otimes 2}, Q} + O(\e)\,.
    \end{align*}

    Bounding $\tfrac{1}{\card{J}} \sum_{i\in [n]\setminus J} \iprod{V^{\otimes 2}, \paren{x_ix_i^\top-\Id_d}^{\otimes 2}}$ by $O(\e \log^2(1/\e))$ can be done in exactly the same way as in Lemma 5.21 from \cite{DiakonikolasKK016}. For that Lemma they need condition 4 of Definition 5.15, that is shown in the proof of Lemma 5.17. Their proof uses only the Hanson-Wright inequality and no other properties of Gaussians, and it is enough to have $n\gtrsim d^2\log^5(d)/\e^2$ samples for it.

    As was observed before, the Hanson-Wright inequality implies condition 3 of stability: $\iprod{V^{\otimes 2}, Q} = \E \iprod{V^{\otimes 2}, \paren{x_ix_i^\top-\Id_d}^{\otimes 2}} \le O(1)$.
\end{proof}

\begin{lemma}\label{lem:frobenius-stability-isotropic}
  Let $0 < \e \lesssim 1$ and $R \le O(1)$.  Let $\Sigma \in \R^{d\times d}$ be a positive definite matrix such that $0.9 \cdot \Id_d \preceq \Sigma \preceq 1.1\cdot \Id_d$. Let $y_1,\ldots,y_n\simiid N(0,\Sigma)$, $\Sigma' = \Cov_{y\sim N(0,\Sigma)} \cF(y)$, where $\cF$ is as in \cref{def:function}. Let $\zeta_i = \Paren{\Sigma'}^{-1/2}\cF(y_i)$.

Then, with high probability, $\zeta_1\zeta_1^\top,\ldots,\zeta_n\zeta_n^\top$ is an $\Paren{\e,\delta,r,\cB_R, \cB_R\otimes \cB_R}$-stable set with respect to $M=\Id_d$ and $Q = \E_{y\sim N(0,\Sigma)} \Paren{\Paren{\Sigma'}^{-1/2}\cF(y)\cF(y)^\top\Paren{\Sigma'}^{-1/2} - \Id}^{\otimes 2}$, where $\e \le \delta \le O\paren{\e\log(1/\e)}$ and $r\le O(1)$. 
\end{lemma}
\begin{proof}
    By \cref{prop:Hanson-Wright} it is enough to show that the distribution of $\zeta$ satisfies $O(1)$-Hanson-Wright property. Since $\cF(y) = \cF(\Sigma^{1/2}g)$ is a $O(1)$-Lipschitz function of $g \sim \cN(0,1)$, and since $\Sigma'$ is $O(1/d)$-close to $\Sigma$ by \cref{lem:closeness}, $\Paren{\Sigma'}^{-1/2}\cF(\Sigma^{1/2}g)$ is also a $O(1)$-Lipschitz function of $g \sim \cN(0,1)$. By \cite{log-sobolev-are-hanson-wright}, this distribution satisfies $O(1)$-Hanson-Wright property.
\end{proof}

\begin{lemma}\label{lem:frobenius-stability}
      Let $0 < \e \lesssim 1$ such that $\e\log(1/\e) \gtrsim \log^2(d)/d$.  Let $\Sigma \in \R^{d\times d}$ be a positive definite matrix such that 
      \[
      \Paren{1-O(\e\log(1/\e)} \cdot \Id_d \preceq \Sigma \preceq\Paren{1+O(\e\log(1/\e)} \cdot \Id_d\,.
      \]
      Let $y_1,\ldots,y_n\simiid N(0,\Sigma)$ and $\cF$ be as in $\cref{def:function}$.

Then, with high probability, $\cF(y_1)\cF(y_1)^\top,\ldots,\cF(y_n)\cF(y_n)^\top$ is an $\Paren{\e,\delta,r,\cB_1, \cB_1\otimes \cB_1}$-stable set with respect to $M=\E\cF(y)\cF(y)^\top$ and $S = \E_{g\sim N(0,\Id)}\Paren{gg^\top / \norm{g}^2 - \Id}^{\otimes 2}$, where $\e \le \delta \le O\paren{\e\log(1/\e)}$ and $r\le O(1)$. 
\end{lemma}

\begin{proof}
    By \cref{lem:closeness},
     \[
      \Paren{1-O(\e\log(1/\e)} \cdot \Id_d \preceq \Sigma \preceq\Paren{1+O(\e\log(1/\e)} \cdot \Id_d\,.
      \]
    Denote $\zeta_i = \Paren{\Sigma'}^{-1/2}\cF(y_i)$. Let $J\subseteq[n]$ be a set of size $\card{J} \ge (1-\e)n$,  and let $V\in \cB_1$. Note that
    \[
    \tfrac{1}{\card{J}} \sum_{i\in J} \iprod{V, \cF(y_i)\cF(y_i)^\top-\E\cF(y)\cF(y)^\top} = \tfrac{1}{\card{J}} \sum_{i\in J} \iprod{\Paren{\Sigma'}^{-1/2}V\Paren{\Sigma'}^{-1/2}, \zeta_i \zeta_i^\top -\Id_d}
    \]
    By \cref{lem:frobenius-stability-isotropic}, $\zeta_1\zeta_1^\top,\ldots,\zeta_n\zeta_n^\top$ is $\Paren{\e,\delta,r,\cB_2, \cB_2\otimes \cB_2}$-stable
    with respect to $\Id_d$ and $Q = \E_{y\sim N(0,\Sigma)} \Paren{\Paren{\Sigma'}^{-1/2}\cF(y)\cF(y)^\top\Paren{\Sigma'}^{-1/2} - \Id}^{\otimes 2}$, hence the quantity above is bounded by $\delta = O(\e\log(1/\e))$, so condition 1 of stability is satisfied.

    Let us show the second condition. Note that
    \begin{align*}
   \tfrac{1}{\card{J}} \sum_{i\in J} \iprod{V^{\otimes 2}, \Paren{\cF(y_i)\cF(y_i)^\top-\E\cF(y)\cF(y)^\top}^{\otimes 2}} 
   &=
    \tfrac{1}{\card{J}} \sum_{i\in J} \iprod{V, \Paren{\cF(y_i)\cF(y_i)^\top-\E\cF(y)\cF(y)^\top}}^2 
    \\&=
        \tfrac{1}{\card{J}} \sum_{i\in J} \iprod{U, \Paren{\zeta_i \zeta_i^\top -\Id_d}}^2 \,,
        \end{align*}
    where $U = \Paren{\Sigma'}^{-1/2}V\Paren{\Sigma'}^{-1/2}$. By stability of $\zeta_1\zeta_1^\top,\ldots,\zeta_n\zeta_n^\top$, this quantity is $O(\delta^2/\e)$-close to $\iprod{U^{\otimes 2}, Q}$. Hence it is enough to show that
    \[
    \Abs{\iprod{U^{\otimes 2}, Q} - \iprod{V^{\otimes 2}, S}} \le O(\e\log^2(1/\e))\,.
    \]
    Since $\Paren{\Sigma'}^{-1/2} = \Id + A$, where $\Norm{A}\le O(\e\log(1/\e))$, 
    \[
    U = V + A V + VA + AVA\,.
    \]
    Since $\normf{AV}\le \norm{A}\cdot \normf{V} \le O(\e\log(1/\e))$,
    $U = V + \Delta$, where $\normf{\Delta} \le O(\e\log(1/\e))$.
    Hence $\iprod{\Delta^{\otimes 2}, Q} \le O(\e^2\log^2(1/\e))$. By the \CS inequality, 
    \[
    \Abs{\Iprod{V\otimes \Delta, Q}}\le \sqrt{\iprod{\Delta^{\otimes 2}, Q} \cdot \iprod{V^{\otimes 2}, Q}} \le  O(\e\log(1/\e))\,.
    \]
    Hence $\Abs{\iprod{U^{\otimes 2}, Q} - \iprod{V^{\otimes 2}, Q}} \le O(\e\log(1/\e))$, and it is enough to bound $\Iprod{V^{\otimes 2}, Q-S}$. By \cref{lem:tensor}, in the eigenbasis of $\Sigma$, for all $i,j\in [d]$, $Q_{ijij} = S_{ijij} + O(1/d)$, and $Q_{iijj} = S_{iijj} + O\Paren{\e\log(1/\e)/d}$. First let us bound the entries of the form $ijjij$:
    \[
    \sum_{i,j\in[d]}V_{ij}V_{ij}\Paren{Q_{ijij} - S_{ijij}} \le O(1/d)\cdot \sum_{i,j\in[d]}V_{ij}^2 \le O(1/d) \le O(\e\log(1/\e))\,.
    \]
    Now consider the entries of the form $iijj$. Since $\abs{\sum_{i=1}^d V_{ii}} \le \sqrt{d}$,
    \[
    \sum_{i,j\in[d]}V_{ii}V_{jj}\Paren{Q_{ijij} - S_{ijij}} \le O(\e\log(1/\e)/d)\cdot d \le O(\e\log(1/\e))\,.
    \]
    Since all other components of both $Q$ and $S$ are $0$, we get the desired bound.

    Condition 3 of the stability is clearly satisfied for $S$ (in particular, it follows from the proof above, since $Q$ is very close to $S$, and this condition is satisfied for $Q$).
\end{proof}

\cref{lem:frobenius-stability} and \cref{thm:filtering} imply that we get the desired estimator in relative Frobenius norm.
\section{Applications}

\subsection{Robust PCA}

In this subsection we briefly discuss how to derive \cref{cor:pca} from \cref{thm:main}. By Davis-Kahan inequality \cite{davis-kahan}, $\hat{v}$ is either ${O}(\e\log(1/\e)/\gamma)$-close to $v$ or to $-v$. Let us without loss of generality assume that it is close to $v$. Then
\[
\normf{\hat{v}\hat{v}^\top - vv} = \normf{\Paren{\hat{v} - v}\Paren{\hat{v} + v}^\top + \Paren{\hat{v} + v}\Paren{\hat{v} - v}^\top} \le O\Paren{\norm{\hat{v} - v}} \le {O}(\e\log(1/\e)/\gamma)\,.
\]

\subsection{Robust covariance estimation}

In this subsection we show how to estimate the scaling factors assuming the Hanson-Wright property or sub-exponentiality. Further in this section we denote by $\Sigma$ the covariance of $\cD$. First we assume that the mean of $\cD$ in $0$, and then we explain how the case $\mu\neq 0$ can be reduced to it.

By \cref{thm:main}, we can compute $\hat{\Sigma}$ such that
\[
\Normf{\paren{\hat{\Sigma}}^{-1/2}\Paren{\gamma\Sigma}\paren{\hat{\Sigma}}^{-1/2} - \Id} \le O\Paren{\e\log(1/\e)}\,,
\]
where $\gamma > 0$ is unknown. Our goal is to find a good estimator $\hat{\gamma}$ of $\gamma$, and then use $\hat{\Sigma}/\hat{\gamma}$ as an estimator for $\Sigma$.

Let us use fresh samples (as before, we can split the original sample into several parts, and work with a new part for a new estimation), and let us apply the transformation $\hat{\Sigma}^{-1/2}$ to the samples. The covariance of the resulting distribution is $\tilde{\Sigma} = \paren{\hat{\Sigma}}^{-1/2}\Sigma \paren{\hat{\Sigma}}^{-1/2}$, and it satisfies
\[
\normf{\gamma\tilde{\Sigma} - \Id} \le O\Paren{\e\log(1/\e)}\,.
\]

In particular, $\effrank(\tilde{\Sigma})\ge \Omega(d)$. Let us robustly estimate $\Tr(\tilde{\Sigma})$. 
By the Hanson-Wright inequality,
    \[
    \Pr_{x\sim\cD} \Paren{\Abs{\norm{x}^2 - \Tr \tilde{\Sigma}} \ge t} \le 2\exp\Paren{-\frac{1}{C_{HW}} \min\Set{\frac{t^2}{\normf{\tilde{\Sigma}}^2}, \frac{t}{\norm{\tilde{\Sigma}}}}} \le 2\exp\Paren{-\Omega\Paren{ \min\Set{\frac{d \cdot t^2}{\Tr(\tilde{\Sigma})^2}, \frac{\sqrt{d} \cdot t}{\Tr(\tilde{\Sigma})}}}}
    \,,
    \]
    where we used the fact that $\effrank(\tilde{\Sigma})\ge \Omega(d)$. Hence for all even $p\in\N$,
    \[
    \Paren{\E\Abs{\norm{x}^2 - \Tr \tilde{\Sigma}}^p}^{1/p} \le O\Paren{\Tr\paren{\tilde{\Sigma}}\cdot p/\sqrt{d}}\,.
    \]
    Now, applying one-dimensional truncated mean estimation (see, for example, Proposition 1.18 from \cite{DK_book}), we get an estimator $\hat{T}$ such that with high probability
    \[
    \Abs{\hat{T} - \Tr \tilde{\Sigma}} \le O\Paren{\Tr\paren{\tilde{\Sigma}} \cdot \e\log(1/\e) / \sqrt{d}}\,.
    \]

    Hence
    \[
    \Abs{1 - \Tr \paren{\tilde{\Sigma}}/\hat{T}} \le O\Paren{\e\log(1/\e) / \sqrt{d}}\,.
    \]

    Let $\hat{\gamma} = d/\hat{T}$, and $\hat\Sigma' = \hat{\Sigma}/\gamma = \paren{\hat{T}/d}\cdot \hat\Sigma$. 
    Then
    \begin{align*}
\Normf{\paren{\hat{\Sigma}'}^{-1/2}\Paren{\Sigma}\paren{\hat{\Sigma}'}^{-1/2} - \Id} 
&=  \Normf{\hat{\gamma}\tilde{\Sigma} - \Id} 
\\&= \Normf{\frac{\hat{\gamma}}{\gamma} \gamma\tilde{\Sigma} - \Id} 
\\&= \Normf{\Paren{1+O\Paren{\e\log(1/\e) / \sqrt{d}}} \gamma\tilde{\Sigma} - \Id}
\\&\le \normf{\gamma\tilde{\Sigma} - \Id} + O\Paren{\e\log(1/\e) / \sqrt{d}}\Normf{\gamma\tilde{\Sigma}}
\\&\le
 O\Paren{\e\log(1/\e)} + O\Paren{\e^2\log^2(1/\e)/\sqrt{d}}
 + O\Paren{\e\log(1/\e) / \sqrt{d}}\Normf{\Id_d}
 \\&\le
 O\Paren{\e\log(1/\e)}\,.
\end{align*}


Now let us prove the spectral norm bound assuming $O(1)$-sub-exponentiality. As before, we work with a transformed distribution with covariance $\tilde{\Sigma}$ such that for some $\gamma > 0$, $\Normf{\paren{\hat{\Sigma}}^{-1/2}\Paren{\gamma\Sigma}\paren{\hat{\Sigma}}^{-1/2} - \Id} \le O\Paren{\e\log(1/\e)}$.
Since $x = \xi \cdot \tilde{\Sigma}^{1/2} g/\norm{g}$, where $g\sim\cN(0,\Id)$,
for all $u\in \R^d$ and even $p\in \N$,
\[
\Paren{\E\xi^p\iprod{u, g/\norm{g}}^p}^{1/p} \le O\Paren{p\Paren{\E\xi^2\iprod{u, g/\norm{g} }^2 }^{1/2}}\,.
\]

Since $\xi$ and $g$ are independent, 
\[
\Paren{\E\xi^p}^{1/p} \le O\Paren{p\Paren{\E\xi^2}^{1/2} \cdot \frac{\Paren{\E\iprod{u, g/\norm{g}}^2 }^{1/2}}{\Paren{\E\iprod{u, g/\norm{g}}^p}^{1/p}} }\,.
\]
Note that since $g/\norm{g}$ and $g$ are independent, $\E\iprod{u, g}^p = \E\norm{g}^p \cdot \E\iprod{u, g/\norm{g}}^p$. Since $\norm{g}^2$ has chi-squared distribution, its $(p/2)$-th moment is $2^{p/2}\cdot \Gamma(d/2 + p/2) / \Gamma(d/2)$. Hence for all $p\le d$,
\[
\Paren{\E\norm{g}^p}^{1/p} = \Theta\Paren{\sqrt{d}}\,.
\]
Since for all $p$,
\[
\Paren{\E\iprod{u, g}^p}^{1/p} = \Theta(\sqrt{p} \norm{u})\,,
\]
we get for all $p\le d$
\[
\Paren{\E\xi^p}^{1/p} \le O\Paren{\sqrt{p}\Paren{\E\xi^2}^{1/2} }\,.
\]

Now consider 
\[
\E\norm{x}^p = \E\xi^p \cdot \E \norm{\tilde{\Sigma}^{1/2} g/\norm{g}}^p\,.
\]
Since $\norm{\tilde{\Sigma}^{1/2}} = \Theta(1)$, for all $p\le d$ we get 
\[
\Paren{\E\norm{x}^p}^{1/p} \le O\Paren{\sqrt{p}\Paren{\E\xi^2}^{1/2}}\le 
O\Paren{\sqrt{p}\E \norm{x}^2}^{1/2}
\,.
\]

Applying one-dimensional truncated mean estimation with $p = O(\log(1/\e))$, we get an estimator $\hat{T}$ such that with high probability
    \[
    \Abs{\hat{T} - \Tr \tilde{\Sigma}} \le O\Paren{\Tr\paren{\tilde{\Sigma}} \cdot \e\log(1/\e)}\,.
    \]
The rest of the proof is exactly the same as for the Hanson-Wright distributions (with relative spectral norm instead of relative Frobenius norm).

Now let us explain how to deal with the case when the mean of $\cD$ is nonzero. As for covariance estimation, we can use symmetrization $y = \paren{x-x'}/\sqrt{2}$. Denote $z = x - \mu$ and $z' = x' - \mu'$. We need to show the Hanson-Wright inequality for all positive definite matrices $A$ with $\effrank(A) \ge \Omega(d)$. Since the inequality is scale invariant, let us assume $\Tr(A) = d$. Then 
\[
y^\top Ay- \Tr(A) = \frac{1}{2}\Paren{z^\top A z - \Tr(A)} + \frac{1}{2}\Paren{(z')^\top Az' - \Tr(A)} + \frac{1}{2}z^\top A z' + \frac{1}{2}(z')^\top A z\,.
\]
The first two terms are bounded by $O\Paren{\sqrt{d\log(1/\delta)} + \log(1/\delta)}$ with probability $1-\delta$ (by the Hanson-Wright inequality). Let us bound the third term. Let us condition on $z'$ and treat it as a constant. By the Hanson-Wright inequality, with probability $1-\delta$,
\[
\Paren{z^\top A z'}^2 \le O\Paren{\sqrt{\norm{Az'}^2\log(1/\delta)} + \norm{Az'}^2\log(1/\delta)}\,.
\]

By the Hanson-Wright inequality for $z'$, $\norm{Az'}^2 \le O(\norm{z'}^2) \le  d + O\Paren{\sqrt{d\log(1/\delta)} + \log(1/\delta)}$ with probability $1-\delta$. Note that we only need to apply the Hanson-Wright inequality for $\delta$ such that $\log(1/\delta) \le O(\log(1/\e)) \le O(\log d)$. For such $\delta$, 
\[
z^\top A z' \le O\Paren{\sqrt{d\log(1/\delta)}}.
\]
The term $(z')^\top A z$ is also bounded by this value. Since this bound is enough to obtain the bound on the $O(\log(1/\e)$-th moment that we use, we get the desired result.

Let us now consider the sub-exponential case. Note that
\[
O\Paren{\E \iprod{z, u}^p}^{1/p} \le O\Paren{\Paren{\E \iprod{x - \mu, u}^p}^{1/p} + \Paren{\E \iprod{x' - \mu, u}^p}^{1/p}} \le 
O\Paren{\Paren{\E \iprod{x - \mu, u}^2}^{1/2}}\,.
\]
Note that $\iprod{z, u}^2 = \tfrac{1}{2}\iprod{x - \mu, u}^2 + \tfrac{1}{2}\iprod{x' - \mu, u}^2 - \iprod{x - \mu, x'-\mu}$, so $\E \iprod{z, u}^2 = \E  \iprod{x - \mu, u}^2$, and we get the desired bound.

%Finally, the standard stability-based filtering algorithm for robust mean estimation works for sub-exponential distributions if its covariance satisfies $\Norm{\Sigma - \Id}\le O(\e \log(1/\e))$ (see section 2.2.1 of \cite{DK_book}), and the mean estimator satisfies $\norm{\mu-\hat{\mu}} \le O(\e\sqrt{\log(1/\e)})$ with high probability.

\section{Sum-of-Squares Proofs}
\label{sec:sos}

We use the standard sum-of-squares machinery, used in numerous prior works on algorithmic statistics, e.g. \cite{KS17, hopkins2018mixture,Bakshi, sos-poincare, sos-subgaussian}. 

In particular, we use the following facts:

\begin{fact}[Spectral Certificates] \label{fact:spectral-certificates}
For any $m \times m$ matrix $A$, 
\[
\sststile{2}{u} \Set{ \iprod{u,Au} \leq \Norm{A} \Norm{u}_2^2}\mper
\]
\end{fact}

\begin{fact}[Cauchy-Schwarz for Pseudo-distributions]
Let $f,g$ be polynomials of degree at most $d$ in indeterminate $x \in \R^d$. Then, for any degree d pseudo-distribution $D$,
$\pE_{D}[fg] \leq \sqrt{\pE_{D}[f^2]} \sqrt{\pE_{D}[g^2]}$.
 \label{fact:pseudo-expectation-cauchy-schwarz}
\end{fact} 









% \crefalias{section}{appendix} % uncomment if you are using cleveref

\end{document}
