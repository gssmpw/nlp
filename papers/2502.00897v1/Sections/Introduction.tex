\section{\textbf{Introduction}}
Accurate and efficient wavefield modeling is essential in seismology, with applications ranging from exploration to earthquake hazard analysis \citep{carcione2002seismic, fichtner2010full}. Among various modeling techniques, frequency-domain wavefield simulation has gained prominence due to its ability to handle steady-state wave propagation across multiple frequencies \citep{marfurt1984accuracy}. By focusing on specific frequency components, frequency-domain methods not only reduce the dimensionality of the problem, but also facilitate multi-frequency analysis, which is critical for multi-scale inversion strategies \citep{pratt1999seismicpart1, pratt1999seismicpart2}. These capabilities have positioned frequency-domain modeling as a powerful approach to better understand subsurface properties \citep{pratt1990frequency}. 

Despite its advantages, frequency-domain wavefield modeling faces significant challenges in practical applications, particularly in heterogeneous media. Solving the frequency-domain wave equation requires handling large and sparse linear systems for each frequency \citep{marfurt1984accuracy, pratt1999seismicpart1}, where the impedance matrix relates wavefield solutions to seismic sources. Traditional approaches rely on numerical techniques such as finite difference and finite element methods to discretize the model and solve these systems using either direct solvers \citep{operto20073d} or iterative methods \citep{riyanti2006new, plessix2009three, petrov20123d, wu2018efficient}. Direct solvers, though precise, are computationally expensive for large-scale models, for example 3D models, due to their reliance on matrix factorization \citep{plessix2009three}. On the other hand, iterative solvers offer a more memory-efficient alternative, but often struggle with convergence, particularly in complex and high-frequency scenarios \citep{yang2004optimal, huang2021finite}. Recent advancements in preconditioning techniques have partially alleviated these computational challenges \citep{sheikh2016accelerating, belonosov2017iterative}, yet the overall cost and complexity of traditional methods remain a bottleneck for large-scale applications. 

% Frequency-domain wavefield modeling plays a critical role in seismic applications, including subsurface imaging, full-waveform inversion, and resource exploration. Compared to time-domain approaches, frequency-domain methods reduce the dimensionality of the problem by solving for steady-state solutions at discrete frequencies, which makes them particularly effective for tasks requiring multi-frequency analysis. By directly linking frequency components to physical phenomena, these methods provide insights into the subsurface structure and dynamics, enabling high-resolution imaging and inversion. \\

% Traditional approaches to frequency-domain wavefield modeling involve solving the Helmholtz equation, which governs wave propagation in the frequency domain. These methods typically rely on numerical techniques such as finite difference or finite element methods, where the model domain is discretized into grids, and the resulting linear system is solved for each frequency. However, solving the Helmholtz equation becomes computationally prohibitive for large-scale models, high frequencies, or 3D applications due to the cost and complexity of matrix inversion. Additionally, these approaches require fine grid discretization to accurately capture high-frequency wavefields, further increasing the computational burden. These challenges have driven the exploration of alternative methods to address the scalability and efficiency limitations of traditional techniques. \\

To address the limitations of traditional numerical methods, physics-informed neural networks (PINNs) \citep{raissi2019physics} have emerged as a promising alternative for wavefield modeling \citep{rasht2022physics, waheed2022kronecker, sandhu2023multi, schuster2024review, chai2024modeling}. By embedding the governing partial differential equations (PDEs) into the training process of neural networks (NNs), PINNs eliminate the need for explicit grid discretization or labeled data. Instead, they leverage automatic differentiation to enforce physical constraints, enabling grid-free modeling and flexibility in handling complex wavefield scenarios. PINNs have demonstrated their ability to solve PDEs in various domains, offering a versatile framework for wavefield simulation. For example, \cite{song2021solving} proposed using PINNs to represent frequency-domain scattering wavefields in transversely isotropic media with a vertical symmetry axis. To further improve the performance of PINNs in wavefield representation, they also introduced an adaptive sinusoidal activation function to optimize the training process of PINNs \citep{song2022versatile}. \cite{bin2021pinneik} applied PINNs to solve the Eikonal equation for traveltime computation, which can better support tomographic imaging. \cite{huang2021modified} proposed using positional encoding to transform spatial coordinates into a higher-dimensional space, thereby accelerating the training process of PINNs and enhancing the accuracy of wavefield representation. \cite{wu2023helmholtz} incorporated perfectly matched layer (PML) boundary conditions into the loss function of PINNs and replaced the affine functions with quadratic functions to overcome challenges in representing wavefields for non-smooth velocity models. \cite{alkhalifah2024physics} used adaptive Gabor layers to enhance the fully connected NN model of vanilla PINNs, thereby improving training efficiency and the ability to represent high-frequency wavefields.

Despite these advantages, PINN-based methods still face significant challenges when applied to multi-frequency wavefields and diverse velocity models. A fundamental limitation lies in the fact that PINN is essentially a function approximator \citep{huang2022meta}. The network is trained to represent the wavefield solution for a specific velocity model and frequency configuration based on the given training data. However, when the frequency or velocity changes, the underlying physical relationships governing the wavefield also change. As a result, the trained PINN is no longer valid, and the network should be retrained from scratch to adapt to the new conditions. This retraining process is computationally expensive and undermines the scalability of PINN-based methods for practical seismic applications involving a wide range of frequencies and velocity models. 

Some prior studies have attempted to address this issue. For instance, \cite{huang2022single} proposed a reference frequency loss, utilizing the linear relationship between frequency and wavenumber to train a network capable of representing scattering wavefields with multiple frequency components simultaneously. \cite{song2023simulating} introduced Fourier-featured PINNs, which enhance the representation of multi-frequency wavefields by incorporating frequency information into the input and using positional encoding to map the input data to a higher-dimensional space. However, both approaches face the same limitation: they are only effective for the specific training velocity models. For new velocity models, training still needs to start from scratch. To improve the adaptability of PINNs to different velocities, \cite{taufik2024multiple} proposed LatentPINN, which uses an autoencoder trained across various velocity models to extract a latent representation of the velocity. The extracted latent representation, combined with spatial coordinates, serves as the input to PINNs, allowing the network to represent wavefields corresponding to different velocities. However, this approach requires aligning the size of the latent representation with the input spatial coordinates, making it applicable only to specific-sized velocity models once the autoencoder is trained. Moreover, LatentPINN may face generalization issues for out-of-distribution velocity models. In our previous work, we proposed Meta-PINN \citep{cheng2025meta}, which uses a meta-learning algorithm \citep{finn2017model} to learn a robust initialization on a very limited set of velocity models, allowing PINNs to converge quickly to an ideal accuracy on new velocity models. However, both \cite{taufik2024multiple}'s work and our previous work focused only on single-frequency wavefields, and they still require training from scratch for out-of-distribution frequency wavefields. 

Here, we extend our previous Meta-PINN to provide a generalized framework capable of representing multi-frequency wavefields for varying velocity models. We call our more general framework as Meta-LRPINN, where we integrate low-rank decomposition and frequency embedding to enhance efficiency and adaptability. By applying singular value decomposition (SVD) to the weight matrices of the PINN, we can significantly reduce the number of parameters, enabling efficient representation of complex wavefields. Meanwhile, we develop a frequency embedding hypernetwork (FEH) to embed frequency information directly into the singular values, allowing the network to dynamically adapt to multi-frequency wavefields. To further improve initialization and scalability, we employ a meta-learning strategy that optimizes the initialization parameters for both the LRPINN and the FEH. During the meta-training stage, the framework learns generalizable initialization parameters across various velocity models and frequencies, enabling rapid convergence during meta-testing phase with only a very small number of gradient updates. Moreover, by pruning the FEH in the meta-testing stage, we reduce computational complexity while retaining the frequency-aware capabilities of the network. Furthermore, to address the representation requirements for different frequency components while considering computational resource limitations, we propose an adaptive rank reduction strategy to further enhance training efficiency and reduce memory consumption, while still achieving competitive wavefield representations.

The proposed Meta-LRPINN framework provides a scalable, efficient, and adaptive solution for frequency-domain wavefield modeling, addressing key limitations of existing PINN methods. Our contributions are summarized as follows:
\begin{itemize}
    \item We propose a Meta-LRPINN framework that leverages singular value decomposition and frequency embedding to efficiently model multi-frequency seismic wavefields.
    \item We develop a meta-learning-based initialization strategy to enhance the scalability and adaptability of Meta-LRPINN across different velocity models and frequencies.
    \item We propose a rank-adaptive reduction strategy to accommodate the representation of different frequency wavefields while addressing computational resource constraints.
    \item We validate the proposed framework through extensive numerical examples, demonstrating its effectiveness for various frequencies and complex velocity scenarios. 
\end{itemize}




