\section{\textbf{Conclusions}}\label{conclusions}
We proposed Meta-LRPINN, a meta-learning-enhanced low-rank physics-informed neural network (PINN) framework, to address the challenges of modeling multi-frequency wavefields in variable velocity models. We first introduced a singular value decomposition (SVD)-based approach to decompose the hidden layer weights of PINN, significantly reducing the number of parameters, by reducing the rank (the number of singular values). To enhance the model's adaptability to varying frequencies, we proposed an innovative frequency embedding hypernetwork (FEH) that linked the input frequency to the singular values in the SVD decomposition. To improve training efficiency and optimization stability, we used meta-learning to provide a robust initialization. Furthermore, we introduced adaptive rank reduction and FEH pruning during the meta-testing phase to further enhance computational efficiency. Numerical examples demonstrated that Meta-LRPINN achieved faster convergence and higher accuracy than Meta-PINN, which did not include the SVD decomposition or the FEH, as well as vanilla PINN, with superior adaptability to varying frequencies and velocity models. The results also highlighted that adaptive rank reduction maintained competitive accuracy while significantly reducing the computational cost, and pruning FEH accelerated convergence without compromising performance. These findings demonstrate Meta-LRPINN as a robust and scalable framework for seismic wavefield representation.
