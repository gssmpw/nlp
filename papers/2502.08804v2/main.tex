\documentclass[11pt, letterpaper, reqno]{amsart}


\addtolength{\hoffset}{-1.95cm} \addtolength{\textwidth}{3.9cm}
\addtolength{\voffset}{-1.75cm}
\addtolength{\textheight}{2.4cm}

\renewcommand{\baselinestretch}{1.2}

\usepackage[numbers]{natbib}
%\usepackage[natbibapa]{apacite}
\setcitestyle{square}
%for patching citet
\usepackage{xpatch}
\usepackage{amsfonts}
\usepackage{amsmath}
\usepackage{amsthm}
\usepackage{amssymb}
\usepackage{dsfont}
\usepackage[normalem]{ulem}
\usepackage{bbm}
\usepackage{bm}
\usepackage{url}
\usepackage{mathrsfs} 
\usepackage{setspace,lipsum}
\usepackage[shortlabels]{enumitem}
\usepackage{subcaption}
\usepackage{graphicx}
\usepackage{xcolor}
\usepackage{hyperref}

\hypersetup{
    colorlinks = true,
    linkcolor={red!50!black},
    citecolor={red!50!black},
    urlcolor={blue!80!black}
}

\usepackage[nameinlink,capitalise]{cleveref}

\usepackage{thmtools, thm-restate}

\newtheorem{theorem}{Theorem}[section]
\newtheorem*{theorem*}{Theorem}
\newtheorem{corollary}[theorem]{Corollary}
\newtheorem{lemma}[theorem]{Lemma}
\newtheorem{proposition}[theorem]{Proposition}
\theoremstyle{definition}
\newtheorem{definition}[theorem]{Definition}
\newtheorem{assumption}{Assumption}
\let\origtheassumption\theassumption
\theoremstyle{remark}
\newtheorem{remark}[theorem]{Remark}
\newtheorem{example}[theorem]{Example}
\newtheorem{conjecture}[theorem]{Conjecture}




\usepackage{tikz}
\usetikzlibrary{shapes.geometric, arrows}

\tikzstyle{block} = [rectangle, rounded corners, minimum width=3cm, minimum height=1cm, text centered, draw=black]
\tikzstyle{arrow} = [thick,->,>=stealth]

\newcommand{\izzy}[1]{\textbf{Izzy: #1}}

% Define a shorter indicator function command

\newcommand\numberthis{\addtocounter{equation}{1}\tag{\theequation}}

\newenvironment{bfproof}[1][\proofname]{%
  \proof[\bfseries \scshape \normalsize #1]%
}{\endproof}

%%%%%%%%%%%%%%%%%%%%%%%%%%%%%%%%%
%%%%%%%%%%%%%%%%%%%%%%%%%%%%%%%%%

%Assumption A instead of Assumption 1
\renewcommand*{\theassumption}{\Alph{assumption}}
\crefname{assumption}{assumption}{assumptions}

\newcommand{\overbar}[1]{\mkern 1.5mu\overline{\mkern-1.5mu#1\mkern-1.5mu}\mkern 1.5mu}

\renewcommand{\qedsymbol}{$\blacksquare$} %fill the hollow QED symbol
\numberwithin{equation}{section} %numbering 1.1, 1.2, 2.1 rather than 1,2,3,4
%\allowdisinybreaks %allow the display break of aligned equations

\colorlet{yyred}{red!50!black}
\renewcommand{\qedsymbol}{\textcolor{yyred}{\rule{1.5ex}{1.5ex}}}

%%%%%%%%%%%%%%%%%%%%%%%%%%%%%%%%%
%%%%%%%%%%%%%%%%%%%%%%%%%%%%%%%%%

% have only the year linked in citation apacite
\makeatletter
\xpatchcmd\NAT@citex
 {%
  \@citea\NAT@hyper@{%
    \NAT@nmfmt{\NAT@nm}%
    \hyper@natlinkbreak{\NAT@aysep\NAT@spacechar}{\@citeb\@extra@b@citeb}%
    \NAT@date
  }%
 }
 {%
  \@citea
  \NAT@nmfmt{\NAT@nm}%
  \NAT@aysep\NAT@spacechar
  \NAT@hyper@{\NAT@date}%
 }
 {}{}
\xpatchcmd\NAT@citex
 {%
  \@citea\NAT@hyper@{%
    \NAT@nmfmt{\NAT@nm}%
    \hyper@natlinkbreak{\NAT@spacechar\NAT@@open\if*#1*\else#1\NAT@spacechar\fi}%
    {\@citeb\@extra@b@citeb}%
    \NAT@date
  }%
 }
 {
  \@citea
    \NAT@nmfmt{\NAT@nm}%
    \NAT@spacechar\NAT@@open\if*#1*\else#1\NAT@spacechar\fi
    \NAT@hyper@{\NAT@date}%
 }
 {}{}
\makeatother

%%%%%%%%%%%%%%%%%%%%%%%%%%%%%%%%%
%%%%%%%%%%%%%%%%%%%%%%%%%%%%%%%%%

\newcommand{\E}{\mathbb{E}}

\title[]{Novel Lower Bounds on $M/G/k$ Scheduling}

\author[]{Ziyuan Wang} 
\address{Department of Industrial Engineering and Management Science, Northwestern University}
\email{ziyuanwang2027@u.northwestern.edu}

\author[]{Isaac Grosof} \thanks{} 
\address{Department of Industrial Engineering and Management Science, Northwestern University}
\email{izzy.grosof@northwestern.edu}

\date{\today}
\keywords{M/G/k, Drift Method, variable speed queue, scheduling, SRPT, response time, multiserver scheduling, lower bound, queueing, BAR approach}
\subjclass[2010]{}

\begin{document}

\begin{abstract}
In queueing systems, effective scheduling algorithms are essential for optimizing performance in a wide range of modern applications. While the theory of optimal $M/G/1$ scheduling for mean response time is well established, many modern queueing systems operate with multiple servers. Recently, optimal scheduling for the $M/G/k$ queue has been explored in the heavy traffic limit, but much remains unknown about optimal scheduling in the intermediate regime.

In this paper, we give the first framework for proving nontrivial lower bounds on the mean response time of the $M/G/k$ system under arbitrary scheduling policies. These bounds significantly improve upon previous naive lower bounds, particularly for moderate loads. Key to our approach is a new variable-speed queue, which we call the Increasing Speed Queue, which more accurately captures the work completion behavior of multiserver systems. To analyze the expected work of this Increasing Speed Queue, we develop the DiffeDrift method, a novel manner of employing the drift method/BAR approach, by developing test functions via the solutions to a differential equation.
\end{abstract}

\maketitle



\section{Introduction}\label{sec:intro}

In queueing systems, effective scheduling algorithms are essential for optimizing performance in a wide range of modern applications. In the analysis of queueing systems, one of the most important performance metrics is the mean response time, denoted as $\E[T]$. An object of particular interest is the optimal scheduling policy, which minimizes the mean response time among all policies. In the single-server setting, optimality of scheduling policy is well understood. However, many modern queueing systems are multiserver systems, such as server farms and cloud computing services. Thus, a deeper understanding of optimal scheduling in the multiserver setting is required.

In the seminal work of \citet{Schrage1968Proof}, it is proven that the Shortest Remaining Processing Time (SRPT) minimizes the mean response time of jobs in an $M/G/1$ queue. \citet{Schrage1966} also characterize the mean response time of SRPT exactly. The single-server optimality of SRPT has been used to prove optimality results in the $M/G/k$ under extreme load conditions. When the job sizes are known upon arrival, \cite{Grosof2018} shows that the multiserver SRPT  is optimal in the heavy-traffic regime, as the load $\rho$ approaches the capacity of the system. Likewise, when job sizes are unknown, single-server results have been used to prove multiserver optimality.
If the job sizes are unknown but drawn from a known distribution, the optimal scheduling policy in the $M/G/1$ is known to be the Gittins index policy \cite{Gittins1979}.
Correspondingly, \cite{Scully2020} shows that the multiserver Gittins policy is optimal in the heavy-traffic regime. 
At the opposite extreme, in the light traffic limit, it is rare for nontrivial scheduling options to be available, so the choice of scheduling policy is less important.


However, much less is known about optimal $M/G/k$ scheduling under moderate loads. This intermediate regime often reflects the conditions faced by many real-world systems. Recently, the SEK policy introduced in \cite{Grosof2024BoMS} is empirically observed to beat SRPT-$k$ in the intermediate regime. The fact that SRPT-$k$ is not optimal raises an important question, 
\begin{quote}
    \textit{How much improvement beyond SRPT-$k$ is possible under moderate load?}
\end{quote}

Specifically, we aim to lower bound the mean response time of $M/G/k$ under arbitrary scheduling policies.

\subsection{Challenges in analyzing $M/G/k$ scheduling}
\label{sec:challenges}

The analysis of arbitrary scheduling policies in $M/G/k$ systems presents substantial challenges compared to the single-server case, and there have been limited results regarding the mean response time of $M/G/k$ systems under arbitrary scheduling policies.

The \textit{tagged-job} method \citep{Harchol-Balter2013} is a classical tool for analyzing single-server queues under many different scheduling policies, including the broad SOAP class policies \citep{Scully2018}. However, the tagged-job method breaks down in the analysis of multiserver scheduling. Unlike the single-server setting, the rate at which a multiserver queue completes work varies, making it intractable to quantify the random amount of work encountered by the tagged job in the system.

Another more flexible tool, the drift method \citep{Eryilmaz2012}, likewise encounters additional challenges when applied to queues with variable work completion rate, and has not been employed in multiserver systems with arbitrary scheduling.

As such, many results focus on the heavy-traffic limit, where the work completion behavior of an $M/G/k$ system is nearly identical to that of a single-server queue, allowing the tagged-job method to be reintroduced. For example, \cite{Grosof2018} proves that multiserver SRPT is heavy-traffic optimal using a multiserver tagged-job method. Unfortunately, such results are not tight when applied scheduling in the intermediate load regime. 

In addition, there have been approximation results for $M/G/k$ systems under specific policies, such as the First-Come First-Serve (FCFS) policy (see, e.g., \cite{Gupta2010} and \cite{Gupta2011}). There have also been attempts at an exact solution for multiserver queues using matrix analytic methods, but the state of the art is limited to FCFS scheduling and scheduling policies with no more than two priority classes \citep{Sleptchenko2005}.

\subsection{Our contributions}

In this paper, we present the first nontrivial lower bounds on the mean response time of any $M/G/k$ system. Our bounds hold for any arbitrary scheduling policy and hold across all system loads. We introduce a novel single-server variable-speed queue, the Increasing Speed Queue (ISQ-$k$). ISQ-$k$ allows us to bridge the multiserver and single-server queues in this challenging intermediate-load environment. Moreover, we develop a novel DiffeDrift extension to the drift method to analyze variable speed queues, including ISQ-$k$. 

We develop our lower bounds by leveraging the existing Work-Integral Number Equality (WINE) formula in a novel way \citep{Scully2020}. This equality allows us to establish a lower bound on mean response time by bounding the mean relevant work in the system (see \Cref{sec:wine}). The WINE formula applies to all systems: arbitrary scheduling, any number of servers, and any load.

Our key tool for lower bounding mean relevant work is a novel variable-speed single-server queue called the Increasing Speed Queue (ISQ-$k$). 
We prove that the mean total work of the ISQ-$k$ lower bounds the mean total work of any $M/G/k$ system under arbitrary scheduling policies and at all loads (see \Cref{sec:isq}).
We build on this result to lower bound the mean \emph{relevant} work of $M/G/k$ systems under arbitrary scheduling policies.

Our analysis of the ISQ-$k$ system is based on the drift method, which relies on the careful selection of test functions \citep{Eryilmaz2012}. However, most prior applications of drift methods either use standard test functions that are not customized to the system or employ rescaling techniques to eliminate variable rates \citep{Hurtado-Lange2020,Braverman2017}. Neither of these methods can accommodate complex variable work completion rates and give a tight analysis of the ISQ-$k$ system.

To overcome this limitation, we introduce a novel method of deriving test functions, which we term the DiffeDrift method. This consist of formulating our test functions as solutions to differential equations.

We discuss the DiffeDrift method in \Cref{sec:derive}. The test functions derived in this manner are not covered by existing approaches to the drift method, specifically the existing Basic Adjoint Relationship (BAR) results (see \cref{sub_sec:prior_bar}). The standard approach would involve a sequence of truncations of these test functions, which would be highly complex given their nontrivial form. Instead, we prove a novel BAR result that accommodates our test functions in \cref{sec:drift}.

Our DiffeDrift method allows us to select the drift first, then come up with the test function. We use this flexibility to handle the variable work completion rate of the ISQ-$k$ system.
We believe the DiffeDrift method and our BAR extension, \cref{prop:drift}, have broad applicability to queueing systems.


\setlength{\belowcaptionskip}{0pt} 

\begin{figure}[htbp!]
    \centering
    \includegraphics[width=0.85\textwidth]{Figures/intro_wine.png}
    \caption{$M/G/2$ mean response time under SRPT-2, compared against naive lower bounds ($M/G/1$/SRPT and $M/G/\infty$, and against our lower bounds. See \Cref{sec:wine} for the definition of WINE 3 and WINE 4 bounds.}
    
    \label{fig:intro_wine}
\end{figure}


\subsection{Improvement upon prior results}
We now show the massive improvement of our lower bound over the prior naive lower bounds. In the prior literature, only two lower bounds on mean response time for the $M/G/k$ queue under arbitrary scheduling policies have appeared. These bounds are the mean service time and the \emph{resource-pooled} single-server SRPT response time \citep{Grosof2018}. The resource-pooled system is a system which combines all $k$ servers from the $M/G/k$ into one giant server running at speed 1. Because the resource-pooled $M/G/1$ can simulate any $M/G/k$ scheduling policies, resource-pooled SRPT lower bounds the mean response time of the $M/G/k$ under any scheduling policy. The resource-pooled SRPT lower bound has been used to prove prior heavy-traffic bounds \citep{Grosof2018,Grosof2023Optimal}.

As an illustrative example, consider a two-server setting with an exponential job size distribution, as shown in \Cref{fig:intro_wine}. The blue line represents the mean response time of the resource pooled SRPT system, and the orange line represents the mean service time bound. The empirical mean response time of $M/G/2$-SRPT is given by the dashed green line -- while SRPT-2 is known to not be optimal in this setting \citep{Grosof2024}, it gives a useful sense of where the optimal policy may lie. Both lower bounds are observed to be loose in the intermediate regime.

The red line, labeled WINE 2, represents the combination of the mean service time bound and the resource-pooled SRPT bound via the WINE method. This bound significantly improves upon the naive bounds, particularly in the intermediate regime.

To illustrate the additional benefits of ISQ-$k$, we plot our best lower bound, which is represented by the brown line, labeled WINE 4, in \Cref{fig:intro_wine}. With the help of ISQ-$k$, our lower bound offers a major improvement over WINE 2, on top of WINE 2's improvement over the prior bounds. Collectively, the region where the optimal response might lie has shrunk by more than half across a wide range of loads, from 0.63 to 0.87, and by substantial margins across a much wider range. 


Our results apply to any job size distribution, any number of servers, and any load in the $M/G/k$ system, as well as any scheduling policy. For ease of exposition, we start by proving lower bounds in the 2-server setting in \Cref{thm:main_2} before generalizing our results to the $k$-server setting in \Cref{thm:main_k}.


\subsection{Outline of paper}
This paper is organized as follows:
\begin{itemize}
    \item \Cref{sec:prior_work}: We discuss prior work on multiserver SRPT, multiserver FCFS and the drift method.
    \item \Cref{sec:model}: We introduce the Increasing Speed Queue (ISQ-$k$), related auxiliary systems and notation.
    \item \Cref{sec:main}: We present our main results and provide a proof sketch. 
    \item \Cref{sec:drift}: We introduce the drift method and prove a version of the Basic Adjoint Relationship (BAR) applicable to our setting.
    \item \Cref{sec:isq}: We prove that the ISQ-$k$ system lower bounds the total work in the $M/G/k$ system. 
    \item \Cref{sec:mg2}: We prove our main results in the $k=2$ server case, lower bounding the $M/G/2$ under arbitrary scheduling policies.
    \item \Cref{sec:derive}: We discuss the DiffeDrift method, which we use to derive the ISQ-2 test functions, and how to generalize them to the $k$ server system.
    \item \Cref{sec:mgk}: We prove our main results in the general $k$ case,  lower bounding $M/G/k$ mean response time under arbitrary scheduling policies.
    \item \Cref{sec:wine}: We present a framework for lower bounding the response time of $M/G/k$ based on WINE.
    \item \Cref{sec:empirical}: We empirically demonstrate the tightness of our bounds via numerical calculation and simulation. 
    %\item \Cref{sec:conclusion}: We conclude our results and point our future directions.
\end{itemize}



\section{Prior Work}
\label{sec:prior_work}

\subsection{Shortest-remaining processing time}
It is well established that the Shortest-Remaining Processing Time (SRPT-1) policy is optimal in minimizing the mean response time in $M/G/1$ queues  \citep{Schrage1968Proof}. Furthermore, \cite{Schrage1966} characterizes the mean response time under SRPT in the $M/G/1$ exactly. 

In a multiserver setting, the SRPT policy, denoted by SRPT-$k$, has been shown to be heavy-traffic optimal, a result proven in \cite{Grosof2018}. This work provides an upper bound on the mean response time under SRPT-$k$ that is tight at very high loads.

At the opposite extreme, when there are no arrivals, it is well known that the Shortest Job First (SJF) policy is the optimal nonpreemptive policy. Furthermore, \cite{McNaughton1959} proves that no preemptive policy can outperform the optimal nonpreemptive policy. Since SRPT reduces to SJF when there are no arrivals, SRPT is also optimal in that setting.

However, SRPT-$k$ is not always the optimal policy across all load conditions. Notably, the SEK policy, recently introduced in \cite{Grosof2024BoMS}, empirically outperforms the response time of SRPT-$k$ under certain conditions. This raises the question of how much further improvement beyond SRPT-$k$ is possible.

The focus of this paper is on establishing lower bounds for the mean response time of $M/G/k$ systems under arbitrary policies, which has been noted as an open problem \citep{Grosof2019Open}.

\subsection{Multiserver mean response time analysis}
The analysis of multiserver FCFS systems, particularly in the context of $M/G/k$-FCFS queues, is relatively well understood. Early work by \cite{Kingman1970} and \cite{Daley1998} provide bounds on the mean response time based on the first two moments of the job size distribution. \cite{Harchol-Balter2005} conducts an exact analysis of the mean response time in $M/Ph/k$-FCFS systems. Since any arbitrary distribution can be closely approximated by a phase-type distribution, their results offer a useful approximation for mean response time in $M/G/k$-FCFS systems. 

More recently, \cite{Gupta2011} demonstrates how utilizing the full sequence of moments of the job size distribution can yield 
bounds on mean response time. \cite{Li2024} extends Kingman’s bounds \citep{Kingman1962}, to the multiserver $GI/GI/k$ queue and provide the first simple and explicit bounds for mean waiting time that scale with $1/(1-\rho)$.  These results allow for bounds on the mean response time in $M/G/k$-FCFS queues across all load levels. In contrast, our work considers a more general setting by allowing for any arbitrary scheduling policies and focuses on establishing lower bounds.

\cite{Scully2020} proves that the Gittins-$k$ policy is nearly optimal in the $M/G/k$ under extremely general conditions. Notably, the Gittins-$k$ policy coincides with SRPT-$k$ when there is perfect information about job sizes, which is the setting we consider in this paper. A key result of their work is the introduction of a new formula for calculating the mean response time, known as the Work Integral Number Equality (WINE). WINE provides a method to translate the exact expected relevant work into the exact mean response time, and is applicable to any queueing system under any scheduling policy. Our approach builds on this formula, using WINE as the basis for developing a framework to establish lower bounds on the mean response time. 

\subsection{Basic adjoint relationship (BAR) and drift method}
\label{sub_sec:prior_bar}

The well known Basic Adjoint Relationship (BAR) equation states that the stationary distribution $\pi$ of a continuous-time Markov chain $(Z_t)_{t\geq 0}$ with instantaneous generator $G$ satisfies
\begin{align}
\label{eqn:steady_state}
    \E_{\pi}[G\circ g(Z)]=0,
\end{align}
under suitable conditions on the Markov chain $Z$ and the function $g$ \citep{Glynn2008}. By carefully designing the test function $g$, one can solve the BAR equation \eqref{eqn:steady_state} to obtain either exact expressions for the moments of stationary variables or (asymptotically tight) bounds on these moments, depending on the system. This method is commonly referred to as the drift method. See, e.g.,  \cite{Eryilmaz2012,Maguluri2016,Grosof2023TRaM,Hong2023}. We use the drift method to obtain the exact expected total work of the ISQ-$k$ system,
and exact results and bounds for further systems we introduce in \cref{sec:model}. We use these exact results and bounds to lower bound mean response time in the $M/G/k$ under an arbitrary scheduling policy.
\subsubsection{Choosing test functions}

Recent applications of the drift method use simple exponential test functions of the form $e^{tQ}$, where $t$ is a general constant and $Q$ is the queue length, or $e^{tW}$, where $W$ is the total work in the system. For example, \cite{Braverman2017} and \cite{Braverman2024} use these test functions to prove
% have shown that the BAR is a more natural framework than the limit interchange approach for justifying 
heavy-traffic and steady-state approximations in both single-class and multi-class queueing systems. The use of exponential test functions is also known as the transform method, see, e.g., \cite{Hurtado-Lange2020} and \cite{jhunjhunwala2023}, which likewise uses test functions of the form $e^{tQ}$ or $e^{tW}$. Other recent works have focused on test functions of the form $Q^2$ or $W^2$, \cite{Grosof2022MSJ}, which  likewise depend on a constant rate of work completion.

However, these methods cannot be directly applied to queues with a variable work completion rate, as the test functions $e^{tQ}$ and $e^{tW}$ only yield useful information when the work completion rate is constant. Some attempts have been made to circumvent this problem by focusing on heavy traffic settings. For instance, \cite{Hurtado-Lange2020} relies heavily on state-space collapse to ensure a constant work completion rate except when the system is near-empty. As the system is rarely empty in heavy traffic, this state-space collapse allows the same straightforward test functions to be used, and heavy traffic results to be obtained.

However, in our intermediate-load setting, the variable work completion rate is fundamental, and we cannot eliminate that complexity from the problem. Instead, we introduce the DiffeDrift approach (see \cref{sec:derive}), inventing a new class of test functions which can accommodate the ISQ-$k$ system's variable work completion rate.

%%%%%%

\subsubsection{Sufficient conditions for BAR} Proposition 3 of \cite{Glynn2008} offers a proof of a set of sufficient conditions for the BAR equation \eqref{eqn:steady_state} to hold. This proposition has been widely applied in many recent papers in queueing theory. Their result requires the state space to be discrete and certain regularity conditions to hold on the rate matrix of the Markov jump process and the test function $g$. For example, \cite{Grosof2023TRaM} and \cite{Hong2023} invoke Proposition 3 of \cite{Glynn2008} by showing that their respective Markov chains have uniformly bounded transition rates. However, this result does not directly apply to our system because our system has a continuous state space.

Theorem 2 of \cite{Glynn2008} provides another set of sufficient conditions for the BAR equation \eqref{eqn:steady_state} to hold. This result allows continuous state spaces but requires \emph{bounded-drift} test functions. However, it is not directly applicable to our unbounded quadratic test function, which has unbounded drift. 
For test functions with unbounded drift, a standard approach in the existing literature is to truncate these functions, thereby producing a sequence of bounded test functions that approximate the original. See, e.g., \cite{Braverman2017} and \cite{Guang2024}. 

We take an alternative approach, proving a new BAR result for time-homogeneous Markov processes with unbounded continuous state spaces and unbounded test functions that grow at most quadratically, see \Cref{sec:drift}. We find this approach much easier to apply than the standard truncation method given the complexity of our test functions arising from the DiffeDrift method.

\section{Model}

In this section we specify our queueing model and introduce notation in \Cref{sec:notation}. We introduce a novel variable-speed queue called the Increasing Speed Queue (ISQ-$k$) in \Cref{subsec:isq}. Finally, we introduce two auxiliary queueing systems based on the ISQ-$k$ that we will use to lower bound the mean relevant work in the $M/G/k$, in \Cref{sec:aux_isq}. 

\label{sec:model}
\subsection{Queueing model and notation}
\label{sec:notation}
We study lower bounds on the mean response time $\E[T^\pi]$ of the $M/G/k$ queue under an arbitrary Markovian scheduling policy $\pi$. Let $k$ denote the number of servers and $\lambda$ the arrival rate. A job's \emph{size} is the inherent amount of work in the job. Let jobs have i.i.d. sizes sampled from a job size distribution with probability density function (pdf) $f_S$ and cumulative distribution function (cdf) $F_S$. Let $S$ denote the corresponding random variable for job size and let $\widetilde{S}$ denote the Laplace–Stieltjes transform of $S$.
Note that we assume for simplicity that the job size distribution is continuous and has a pdf, but our results can be straightforwardly extended to a more general setting.
The service rate of each server is $1/k$, and the entire system has a maximum service rate of $1$, when all $k$ servers are occupied. In particular, a job of size $s$ will require a total service time of $ks$ to complete. We define the \textit{load} of the system to be the long-term fraction of servers which are in use. Load is given by $\rho=\lambda\E[S]$, and we assume $\rho < 1$ for stability.

We define a \emph{scheduling policy} $\pi$ to map the set of jobs currently in the $M/G/k$ system, as well as some auxiliary Markovian state,
to a choice of at most $k$ jobs to serve.
For example, a few common scheduling policies are First-Come First-Serve (FCFS-$k$), which serves the $k$ jobs which arrived longest ago, and Shortest-Remaining-Processing-Time (SRPT-$k$), which serves the $k$ jobs of least remaining size.

We say a job is \textit{relevant} at some threshold $x$
if its remaining size is below $x$.
This may occur if the job arrives into the system with an initial size $<x$, or if a job with an initial size $>x$ reaches a remaining size below $x$.
We refer to the process of a job receiving service and lowering its remaining size
as ``aging''.

We use $W$ to denote the total work in the system,
namely the sum of the remaining sizes of all jobs in the system.
We use $W_x$ to denote the \textit{relevant work}, the total remaining size of all relevant jobs in the system.


Let $\lambda_x:=\lambda F_S(x)$ denote the arrival rate of jobs which are relevant from the moment they enter the system.
Let $S_x:=[S\mid S\leq x]$ denote the conditional job size distribution,
truncated at a threshold $x$. The conditional load, $\rho_x$, at a threshold $x$ is the average rate at which works with initial size $<x$ arrives into the system is given by $\rho_x:=\lambda_x\E[S_x]=\lambda\E[S\mathbbm{1}_{\{S\leq x\}}]$.

We also define the capped job size distribution, $S_{\bar{x}} := [\min\{S, x\}]$, which represents the amount of relevant work at some threshold $x$ that a random job will contribute over its time in the system. The capped load, $\rho_{\bar{x}}$, is similarly given by $\rho_{\bar{x}}=\lambda\E[S_{\bar{x}}]=\lambda\E[\min\{S,x\}]$.



\subsection{Increasing speed queue}
\label{subsec:isq}
The increasing-speed queue (ISQ-$k$) is a single-server, variable speed queue. The speed of the ISQ-$k$ system is defined as follows: When the first job arrives to an empty queue, the server initially runs at speed $1/k$. If another job arrives before the system empties, the server now runs at speed $2/k$. With each arrival during a busy period, the server's speed increases by $1/k$ until it reaches the maximum speed of 1. The server maintains the maximum speed of 1 until the system empties and a busy period ends, resetting the speed to 0.
The state of the ISQ-$k$ system is given by a pair $(w,i)$, where $w$ denotes the work in the system and $i$ denotes the speed of the system. In particular, $i\in\{0,1/k,2/k,\cdots,1\}$. We write $W(t), I(t)$ to denote the state of the system at a particular time $t$, and we write $W, I$ to denote the corresponding (correlated) stationary random variables.

\subsection{Recycling: auxiliary ISQ systems}
\label{sec:aux_isq}
Recall that we call a job relevant if its remaining size is less than some threshold $x$. There are two ways for a job to be relevant: it either enters the system with size less than $x$, or it ages down from a size larger than $x$ to a size less than $x$. We refer to the latter as \textit{recycling}, and we call such jobs \textit{recycled}.

We define two auxiliary systems to help us lower bound the response time of the $M/G/k$ system: the \textit{separate-recycling ISQ-$k$} and the \textit{arbitrary-recycling ISQ-$k$}.

The separate-recycling ISQ-$k$ system (Sep-ISQ-$k$) consists of two subsystems,
the truncated-ISQ-$k$ subsystem and the $M/G/\infty$ subsystem. 
Jobs arrive to the overall Sep-ISQ-$k$ system with the same arrival process as the $M/G/k$, jobs arriving into the system according to a Poisson process with rate $\lambda$ and with job size distribution $S$. Jobs with initial sizes $\leq x$ are routed to an ISQ-$k$ subsystem, which we call the truncated-ISQ-$k$ subsystem.
Specifically, the truncated-ISQ-$k$ system is an ISQ-$k$ system characterized by an arrival rate $\lambda_x = \lambda F_S(x)$, and job size distribution $S_x \sim [S \mid S \leq x ]$.

On the other hand, jobs with initial sizes $> x$ are routed to the separate $M/G/\infty$ subsystem, each server operating at a speed of $1/k$. That is, jobs which are not relevant on arrival are served at a separate subsystem.

The arbitrary-recycling-ISQ-$k$ system (AR-ISQ-$k$) is an ISQ-$k$ system with two arrival streams, a Poisson arrival process with a rate of $\lambda F_S(x)$ and job size distribution $S_x \sim [S\mid S\leq x]$, and a general arrival process with rate $\lambda(1-F_S(x))$ and job sizes exactly $x$.
The general arrival process is governed by some general Markovian process for which we only require stability and some general hidden state. The idea behind this general arrival process is that it models the moments at which jobs recycle in the M/G/$k$ system under a general Markovian scheduling policy.
We refer to the Poisson arrivals as the \emph{truncated stream} and the general arrival process as the \emph{recycling stream}.

We will use the triplet $(w, i, a)$ to denote a state of the arbitrary recycling system where $w$ and $i$ are defined similarly to the states of the full ISQ-$k$ system and $a$ denotes a separate Markovian state from the recycling stream's state space, which incorporates additional information about the recycling stream. As before, $A$ denotes the corresponding stationary random variable.

In \Cref{thm:isq_sep}, we show that the mean relevant work in the Sep-ISQ-$k$ system lower bounds the mean relevant work in the $M/G/k$ system under arbitrary scheduling policy, for any arrival rate $\lambda$, job size distribution $S$, and relevancy cutoff $x$.
We also show that the minimum mean relevant work in the AR-ISQ-$k$ system over all recycling streams lower bounds the mean relevant work in the $M/G/k$ system under an arbitrary scheduling policy, see \Cref{thm:ar_isq}.

\section{Main Results}
\label{sec:main}
In this paper, we give the first nontrivial lower bounds on the expected relevant work of the $M/G/k$ system under an arbitrary scheduling policy. We derive these bounds by analyzing the Increasing Speed Queue (ISQ-$k$). By the WINE formula (\Cref{prop:wine}), this yields the first non-trivial lower bounds on mean response time of the $M/G/k$ system. We start by lower bounding the 2-server system before moving on to the $k$-server system. For the 2-server system, these lower bounds on relevant work are as follows:

\begin{theorem}
\label{thm:main_2}
We have the following lower bounds of $\E[W_x^{M/G/2/OPT}]$:
\begin{align}
\label{eqn:main_2_1}
E[W_x^{M/G/2/OPT}]&\geq\dfrac{\lambda_x\E[S_x^2]}{2(1-\rho_x)}+\dfrac{\E[S_x]-(1-\widetilde{S_x}(2\lambda_x))/2\lambda_x}{3-\widetilde{S_x}(2\lambda_x)}+(\lambda-\lambda_x)x^2 \\
\label{eqn:main_2_2}
\E[W_x^{M/G/2/OPT}]&\geq \dfrac{\lambda_x E[S_x^2]}{2 (1 - \rho_x)} + \dfrac{\E[S_x]-(1-\widetilde{S_x}(2\lambda_x))/2\lambda_x}{3-\widetilde{S_x}(2\lambda_x)}\cdot\dfrac{1-\rho_{\overbar{x}}}{1-\rho_x}+\dfrac{(\lambda - \lambda_x)x^2}{2(1-\rho_x)}.
\end{align}
\end{theorem}

\Cref{thm:main_2} is proved in \Cref{sec:mg2}. \Cref{eqn:main_2_1,eqn:main_2_2} are derived by applying the drift method to the Sep-ISQ-2 system and AR-ISQ-2 system, respectively. These ISQ-2 variants were introduced in \Cref{sec:aux_isq}.

The two lower bounds have different strengths at different relevancy-cutoff levels. For small values of $x$, \eqref{eqn:main_2_1} is stronger than \eqref{eqn:main_2_2}. Because the second term in \eqref{eqn:main_2_2} is discounted by $\frac{1-\rho_{\bar{x}}}{1-\rho_x}$, when $x$ is small $\rho_{\bar{x}}$ is significantly larger than $\rho_x$, lowering the numerator. When $x$ is large, this fraction $\frac{1-\rho_{\bar{x}}}{1-\rho_x}$ will converge to 1. On the other hand, when $x$ is large, the last term of \eqref{eqn:main_2_2} will scale by $\frac{1}{2(1-\rho_x)}$, whereas this scaling factor is absent in \eqref{eqn:main_2_1}.

We now state our analogous result for the more general $k$-server system, which we prove in \Cref{sec:mgk}.

\begin{theorem}
\label{thm:main_k}
We have the following lower bounds of $\E[W_x^{M/G/k/OPT}]$ with $k\geq 3$,
\begin{align}
\label{eqn:main_k_1}
\E[W_x^{M/G/k/OPT}]&\geq \dfrac{\lambda_x\E[S_x^2]}{2(1-\rho_x)}+\dfrac{\lambda_x\E[v_1(S_x)]}{2+2\lambda_x\E[u_1(S_x)]} +\dfrac{k(\lambda-\lambda_x)x^2}{2},\\
\label{eqn:main_k_2}
\E[W_x^{M/G/k/OPT}]&\geq \dfrac{\lambda_x\E[S_x^2]}{2(1-\rho_x)}+\dfrac{\lambda_x\E[v_1(S_x)]}{2+2\lambda_x\E[u_1(S_x)]}\cdot\dfrac{1-\rho_{\bar{x}}}{1-\rho_x}+\frac{(\lambda - \lambda_x)J_x}{2(1-\rho_x)},
\end{align}
where $J_x:=\min\left\{x^2, \min_{i < 1}\left\{
        \inf_{w\in[0,kix]}h_{k,x}(w+x,i+1/k)-h_{k,x}(w,i)
        \right\}\right\}$.
\end{theorem}

In the above theorem, $u_1$ and $v_1$ is defined by the following recursive formulas,
\begin{align*}
    u_{q}(w)&=e^{-\frac{kw\lambda_x}{q}}\int_0^w\dfrac{e^{\frac{k\lambda_x y}{q}}(k-q+k\lambda_x \E[u_{q+1}(S_x+y)])}{q}\,dy,\\
    v_q(w)&=e^{-\frac{kw\lambda_x}{q}}\int_0^w \dfrac{e^{\frac{k\lambda_x y}{q}}(2ky-2qy+k\lambda_x \E[v_{q+1}(S_x+y)])}{q}\,dy,
\end{align*} 
with the initial condition $u_k(w)=0$ and $v_k(w)=0$. The function $h_{k,x}$ is defined in \Cref{def:isqk_modified}. \Cref{eqn:main_k_1} is based on the Sep-ISQ-$k$ system and \Cref{eqn:main_k_2} is based on the AR-ISQ-$k$ system.

Note that $u_1, v_1,$ and $h_{k,x}$ can all be exactly symbolically derived for any number of servers, see \Cref{apdx:isq3} for $u_1$ and $v_1$ under the 3-server case. See \Cref{sec:wine} for how to convert lower bounds on mean relevant work into lower bounds on mean response time.

\subsection{Proof overview}

We derive our main results, the lower bounds given in \Cref{thm:main_2,thm:main_k}, via two steps: First, we prove that the expected relevant work of each ISQ-$k$ system lower bounds the optimal expected relevant work in the $M/G/k$ under any scheduling policy. Second, we characterize the work in the ISQ-$k$ system by applying our novel DiffeDrift method. We now overview the proof in more detail.
\subsubsection*{Total work lower bound}

Our first step is to prove that the \emph{total work} in the ISQ-$k$ system lower bounds $M/G/k$ \emph{total work} under an arbitrary scheduling policy:

\begin{restatable}{theorem}{mainfive}
\label{thm:main}
For any job size distribution $S$ and arrival rate $\lambda$, the expected total work in the $M/G/k$ system under any scheduling policy is lower bounded by the expected total work in the ISQ-$k$ system.
\end{restatable}

We prove \cref{thm:main} using a sample path coupling argument in \Cref{sec:isq}.
In fact, we prove a more general lower bound under an arbitrary arrival sequence.

\subsubsection*{Relevant work lower bound}


Using \Cref{thm:main}, we prove in \Cref{thm:isq_sep,thm:ar_isq} that the expected \textit{relevant work} of the Sep-ISQ-$k$ system and the AR-ISQ-$k$ system each lower bound the $M/G/k$ expected \textit{relevant work} under any scheduling policy for all arrival rates $\lambda$.

\subsubsection*{Characterizing ISQ-$k$ total work}

Now that we've proven that Sep-ISQ-$k$ and AR-ISQ-$k$ lower bound mean relevant work in the M/G/$k$, our remaining goal is to characterize the expected relevant work of the Sep-ISQ-$k$ and AR-ISQ-$k$ systems. As an intermediate result, a key step is to compute the expected total work of the ISQ-$k$ system. We analyze the expected total work of the ISQ-$k$ system using the DiffeDrift method, see \Cref{sec:derive_affine}.

The expected total work of the ISQ-2 and ISQ-$k$ systems are given by:
\begin{align}
\label{eqn:main_isq2}
        \E[W^{\text{ISQ-2}}]&=\dfrac{\lambda\E[S^2]}{2(1-\lambda\E[S])}+\dfrac{\E[S]-(1-\widetilde{S}(2\lambda))/2\lambda}{3-\widetilde{S}(2\lambda)},\\
        \label{eqn:main_isqk}
        \E[W^{\text{ISQ-$k$}}]&=\dfrac{\lambda\E[S^2]}{2(1-\lambda\E[S])}+\dfrac{\lambda\E[v_1(S)]}{2+2\lambda\E[u_1(S)]}.
    \end{align} 
Here $v_1$ and $u_1$ are functions defined  as the solutions of differential equations in \Cref{def:isqk_contant,def:isqk_affine}.
\Cref{eqn:main_isq2,eqn:main_isqk} correspond to \Cref{prop:isq2_work,prop:isqk_work}.
These results are proved in \Cref{sec:mg2,sec:mgk}.



\subsubsection*{Sep-ISQ-$k$ and AR-ISQ-$k$ relevant work}
Now, it remains to analyze the mean relevant work in Sep-ISQ-$k$ and AR-ISQ-$k$. We exactly characterize the relevant work of the Sep-ISQ-$k$ system and derive a lower bound on the relevant work of the AR-ISQ-$k$ system.


We characterizing the exact expected relevant work of the Sep-ISQ-$k$ system in \Cref{thm:sep_2,thm:sep_k}.
As a result, we lower bound the expected relevant work of any $M/G/k$ system under any scheduling policies. In our main results, \cref{thm:main_2,thm:main_k}, our \Cref{eqn:main_2_1,eqn:main_k_1} are based on the Sep-ISQ-$k$ results.

For the AR-ISQ-$k$ system, arbitrary recycling presents multiple difficulties. First, the ISQ-$k$ total work formulas \eqref{eqn:main_isq2} and \eqref{eqn:main_isqk} cannot be applied directly, so we require specialized test functions, see \Cref{sec:derive_modified}. 
Additionally, the changes in the drift caused by the arbitrary recycling events are difficult to characterize exactly. Therefore, we uniformly lower bound the jumps of the test function during these events, allowing us to lower bound the expected relevant work of the AR-ISQ-$k$ system
as proved in \Cref{thm:ar_2,thm:ar_k} respectively.
We therefore lower bound the expected relevant work of any $M/G/k$ system under arbitrary scheduling policies, resulting in \Cref{eqn:main_2_2,eqn:main_k_2}  in our main results \Cref{thm:main_2,thm:main_k}.
Finally, we can translate lower bounds on the mean relevant work of the $M/G/k$ system into lower bounds on its mean response time under arbitrary scheduling policies.

\section{Drift Method}
\label{sec:drift}

In this section, we discuss the drift method/BAR approach, which we will use to analyze the ISQ-$k$ system.
We provide background on the drift method in \Cref{sec:drift_background}.
We then prove a novel instance of the BAR,  which holds for a class of unbounded test functions and Markov processes with continuous state spaces in \Cref{sec:novel_bar}.
Finally, we apply the drift method and our BAR result in particular to the ISQ-$k$ system in \Cref{sec:drift_isq}.

\subsection{Background on drift method}
\label{sec:drift_background}
The drift method \citep{Eryilmaz2012}, also known as the BAR \citep{Braverman2017} or the rate conservation law \citep{Miyazawa1994},
states that the average rate of increase and decrease of a random variable must be equal, for any random variable with finite expectation and satisfying certain regularity conditions.


To formalize this concept, we make use of the \textit{drift} of a random variable. The drift is the random variable's instantaneous rate of change, taken in expectation over the system's randomness. Let $G$ denote the \textit{instantaneous generator}, which acts as the stochastic counterpart to the derivative operator. We can also apply $G$ to functions of the system state, which implicitly define random variables. We call such functions \textit{test functions}: a function $f$ that maps system states $(w,i)$ to real values. The instantaneous generator takes $f$ and outputs $G\circ f$, the drift of $f$.

Let $G$ denote the generator operator for the ISQ-$k$ system. Recall that $W(t)$ denotes the work in the system at time $t$ and $I(t)$ denotes the speed of the system at time $t$. For any test function $f(w,i)$,
\begin{align*}
    G \circ f(w,i):=\lim_{t\to 0}\dfrac{1}{t}\E[f(W(t),I(t))-f(w,i)\,|\,W(0)=w, I(0)=i].
\end{align*}

We usually do not directly use the above equation to compute the drift of a test function when we know all the rates at which the system states change. For the ISQ-$k$ system, $w$ increases according to stochastic jumps of size $S$, which arrive according to a Poisson Process with rate $\lambda$. These jumps also cause $i$ to increase by $1/k$ as long as $i < 1$. When $i>0$, $w$ decreases at rate $i$ due to work completion. The next lemma is a special case of Section 6.1 in \cite{Braverman2024}, which states that for a piecewise deterministic Markov process, the drift can be characterized by jumps caused by a Poisson process.

\begin{lemma}
\label{lem:isq_drift}
    For any real-valued differentiable function $g$ of the state of the ISQ-$k$ system,
    \begin{align}
        G\circ g(w,i)=\lambda\E[g(w+S,\min\{1,i+1/k\})-g(w,i)]-\frac{d}{dw}g(w,i)i.
    \end{align}
\end{lemma}

A key fact about drift, known as the Basic Adjoint Relationship (BAR) \eqref{eqn:steady_state}, is that in a stationary system, the expected drift of any random variable or test function is zero, as long as the random variable or test function has finite expectation in stationarity, and satisfies certainty regularity conditions. 

Existing BAR results unfortunately do not cover the ISQ-$k$ system and the specific test functions that we will use to characterize mean workload.
We therefore prove a novel basic adjoint relationship in that applies to our system and test functions. Specifically, our new results, \Cref{prop:drift} and \Cref{lem:uniform}, cover a class of unbounded test functions in our continuous-state setting which were not previously covered by existing results. An alternative approach would be to truncate the test functions and take a limit of a series of truncated test functions, but we expect this approach to be significantly more complicated in our setting. 

\subsection{BAR for unbounded test functions of continuous Markov processes}
\label{sec:novel_bar}


We begin by establishing the general framework that serves as the starting point for our BAR.

Let $(X_t)_{t\geq 0}$ be a continuous-time, time-homogeneous Markov process with transition probability kernel $P(t, x, U)$.  We denote the state space as $\mathcal{S}$, and we emphasize that it can be continuous and unbounded.

We denote the generator as $G$,
\begin{align*}
    G \circ g(x):=\lim_{t\to 0}\frac{1}{t} \int_{y\in\mathcal{S}}\left( g(y)P(t,x,dy)-g(x)\right).
\end{align*}
We say that $g$ belongs to the domain of the generator $G$ of the process $X$ and write $g \in D(G)$ if the above limit exists for all $x\in \mathcal{S}$. The first piece of our novel BAR result is as follows. We prove this result in \Cref{apdx:sec5},

\begin{restatable}{proposition}{propdrift}
\label{prop:drift}
Suppose that $g\in D(G)$ and $(X_t)_{t\geq 0}$ has a stationary distribution $\pi$, for which $|g|$ and $|G\circ g|$ are $\pi$-integrable. Moreover, suppose that the following holds for all $t$ and $x$:
\begin{align}
\label{eqn:uniform}
    \frac{1}{t} \int_\mathcal{S}\left( g(y)P(t,x,dy)-g(x)\right)=A(t,x)+B(t,x),
\end{align}
where $\lim_{t\to 0}A(t,\cdot)\to G\circ g(\cdot)$ uniformly and $\lim_{t\to 0}\int_S B(t,x)\,\pi(dx)=0$. Then the BAR holds:
\begin{align}
\int_\mathcal{S}G\circ g(x)\,\pi(dx)=0.
\end{align}
\end{restatable}

\subsection{Applying drift method to ISQ-$k$ for quadratic test functions}
\label{sec:drift_isq}
Next, we derive the necessary conditions to apply the drift method and specifically \Cref{prop:drift}, to the ISQ-$k$ system. 
Throughout this paper, we will always use test functions with a leading $w^2$ term. Therefore, we present our result for functions of the form $g(w,i)=w^2+c(w,i),$ where $c(w, i)$ is linear in $w$. It is important for our result that the function is at most quadratic in $w$, and that the largest term whose coefficient depends on $i$ is at most linear in $w$. This is the setting in which we prove our novel BAR results in \Cref{lem:uniform}.

The first lemma provides sufficient conditions for the finiteness of $\E[W^2]$, namely that the job size distribution has a finite third moment. This holds for all phase-type distributions, as well as for the truncated distributions that arise in our analysis of relevant work. We prove the following lemma in \Cref{apdx:sec5}.

\begin{restatable}{lemma}{lemwsquare}
\label{lem:w2}
If the job size distribution $S$ has a finite third moment, $E[S^3] < \infty$, then $\E[W^2]<\infty$.
\end{restatable}

Next, we provide sufficient conditions on the test function $g$ to ensure that $G\circ g$ satisfies \Cref{prop:drift}'s \eqref{eqn:uniform}. This result makes use of the fact that the arrival process is Poisson and we prove in \Cref{apdx:sec5}.

\begin{restatable}{lemma}{lemuniform}
\label{lem:uniform}
Suppose $\E[S^3]<\infty$ and $g(w,i)=w^2+c(w,i)$ is a real-valued function of the ISQ-$k$ system which is twice-differentiable with respect to $w$ for each fixed $i$.
Suppose that $|c(w,i)|\leq C_1w+C_2$ for some constants $C_1$ and $C_2$ and $\lim_{w\to0^{+}}g(w,i)=g(0,0)$ for each fixed $i$. Moreover, we assume $|c'(w,i)|\leq M_1$ and $|c''(w,i)|\leq M_2$ for some constants $M_1$ and $M_2$. Then \begin{align}
   \dfrac{1}{t}\E[g(W(t),I(t))-g(w,i)\,|\,W(0)=w, I(0)=i]=A(t,w,i)+B(t,w,i),
\end{align}
where $\lim_{t \to 0} A(t,\cdot,\cdot)\to G\circ g(\cdot,\cdot)$ uniformly and $\lim_{t\to 0}\E[B(t,W,I)]=0$.
\end{restatable}

Now, we are ready to prove our novel BAR result. Combining \Cref{lem:w2,lem:uniform} we have the following BAR \eqref{eqn:steady_state} result which shows that in the ISQ-$k$ system, for test functions $g$ of the structure described here, the expected value of the $G \circ g$ in steady state is zero. We prove the following lemma in \Cref{apdx:sec5}.

\begin{lemma}
\label{lem:drift}
Suppose $\E[S^3]<\infty$ and $g(w,i)=w^2+c(w,i)$ is a real-valued function of the ISQ-$k$ system which is twice-differentiable with respect to $w$ for each fixed $i$.
Suppose that $|c(w,i)|\leq C_1w+C_2$ for some constants $C_1$ and $C_2$ and $\lim_{w\to0^{+}}g(w,i)=g(0,0)$ for each fixed $i$. Moreover, we assume $|c'(w,i)|\leq M_1$ and $|c''(w,i)|\leq M_2$ for some constants $M_1$ and $M_2$. Then,
\begin{align}
    \E[G\circ g(W,I)]=0,
\end{align}
where the expectation is taken over the stationary random variables $W$ and $I$.
\end{lemma}

We will often apply \Cref{lem:drift} when the job size distribution follows a truncated distribution such as $S_x$ and $S_{\bar{x}}$, for which the assumption that $\E[S^3] <\infty$ is automatically satisfied.

It is straightforward to apply \Cref{lem:drift} to the Sep-ISQ-$k$ system because each subsystem is Markovian and independent. For the AR-ISQ-$k$ system, we only apply \Cref{lem:drift} to the Poisson arrival stream and deal with the arbitrary recycling stream through a different approach, see \cite{Braverman2017,Braverman2024}.

In particular, we specify a version of \cref{lem:isq_drift} for the AR-ISQ-$k$ system using the Palm expectation $\E_r$ over the moments when jobs recycle, following \cite{Braverman2017, Braverman2024}. Note that the state of the AR-ISQ-$k$ system is $(w, i, a)$, where $w$ is the work, $i$ is the speed, and $a$ is the hidden arrival state. In this paper, we only consider test functions which do not depend on $a$. When it is clear, we write these test functions as $g(w, i)$. For clarity, we write test functions with three inputs $(w,i,a)$ in the following lemma:

\begin{lemma}
\label{lem:ar_drift}
    For any real-valued differentiable function $g$ of the state of the arbitrary recycling ISQ-$k$ system which does not depend on the hidden arrival state $a$,
    \begin{align*}
        G\circ g(w,i,a)=\lambda_x\E[g(w+S,\min\{1,i+1/k\},\cdot)-g(w,i,\cdot)]-\frac{d}{dw}g(w,i,\cdot)i + (\lambda - \lambda_x)\E_{r \mid w, i, a}[J(W,I)],
    \end{align*}
    where $J(w,i) =g(w+x,\min\{i+1/k, 1\},\cdot) - g(w,i,\cdot)$
    denotes the increase in the test function due to the arrival of a size-$x$ job,
    and $E_{r \mid w, i, a}[\cdot]$ denotes the conditional expectation of recycling with respect to the Palm measure in the immediate future of the state $(w, i, a)$ of the AR-ISQ-$k$ system.
\end{lemma}



\section{Increasing Speed Queue}
\label{sec:isq}
In this section, we prove that the Sep-ISQ-$k$ system and the AR-ISQ-$k$ system lower bound the relevant work of an $M/G/k$ system under an arbitrary scheduling policy.

We first show that, for any sequence of arrival times and job sizes,
the total work in a $k$-server system under an arbitrary scheduling policy is lower bounded by the total work in the ISQ-$k$ system, under the same sequence of arrival times and job sizes.
\begin{proposition}
\label{prop:work}
    For an arbitrary sequence of arrival times and job sizes and at any given point in time $t$, the total work in a $k$-server system under an arbitrary scheduling policy is lower bounded by the total work in an ISQ-$k$ system with the same arrival sequence.
\end{proposition}

\begin{proof}
    We will use the index $j\in \mathcal{J}=\{1,2,...\}$ to denote the busy periods of the ISQ-$k$ system. Specifically, a busy period begins when a job arrives to an empty system, and ends when the system is next empty.
    Let $W_j^{ISQ{\text -}k}(t)$ denote the total work in the ISQ-$k$ system consisting of jobs which arrived during busy period $j$ of the ISQ-$k$ system.
    Note that at all points in time, $W_j^{ISQ{\text -}k}(t)$ is positive for at most one value of $j$, namely the current busy period.
    Similarly, let $W_j^k(t)$ denote the total work remaining in the $k$-server system consisting of jobs which arrived during busy period $i$ of the ISQ-$k$ system.

    Note that the index $j$ always refers to busy periods of the ISQ-$k$ system -- we ignore busy periods of the $k$-server system. We will show that $W_{j}^{ISQ{\text -}k}(t)\leq W_j^{k}(t)$ for all $j\in \mathcal{J}$ and for all $t\geq 0$, which suffices to bound overall work at time $t$. Let  $A_j(t)$ denote the number of jobs which have arrived during busy period $j$ by time $t$. Let $B_j^{k}(t)$ denote the fraction of servers in the $k$-server system which are allocated to jobs which arrived during busy period $j$.
    
    Because arrivals to both systems are identical,
    it suffices to show that the rate of completion of $W_{j}^{ISQ{\text -}k}(t)$ exceeds that of $W_j^{k}(t)$ at all times $t$ at which $W_j^{ISQ{\text -}k}(t) > 0$. Note that the rate of completion of $W_j^{k}(t)$ is simply $B_j^{k}(t)$,
    thanks to our definition that each server in the $M/G/k$ completes work at rate $1/k$. Define $B_j^{ISQ{\text -}k}$ similarly.

    Thus, we want to show that $B_j^{ISQ{\text -}k}(t)$ is higher than $B_j^{k}(t)$ whenever $W_j^{ISQ{\text -}k}>0$ , $j$ must be the busy period currently active.
    Note that $W_j^{ISQ{\text -}k}(t) = W^{ISQ{\text -}k}(t)$,
    as all of the work in the ISQ-$k$ must have arrived during its current busy period,
    as the ISQ-$k$ system ends each busy period by completing all work.
    Thus, $B_j^{ISQ{\text -}k}(t)$ is  $\min\{A_j(t)/k, 1\}$.
        
    $B_j^{k}(t)$ is bounded above by $\min\{A_j(t)/k,1\}$, based on the number of jobs which have arrived during this busy period, namely $A_j(t)$. This is at most the completion rate in the ISQ-k system, as desired. This completes the proof.

\end{proof}

An immediate consequence of \Cref{prop:work} is \Cref{thm:main}, which states that for a Poisson arrival process, the expected work of the ISQ-$k$ system must lower bound that of the $M/G/k$ system. In \Cref{prop:isqk_work} of \Cref{sec:mgk} we provide an exact formula for the mean relevant work of the ISQ-$k$ system.

\mainfive*


Using \Cref{thm:main}, we prove \Cref{thm:isq_sep}, which states that the mean work in the Sep-ISQ-$k$ system lower bounds the mean relevant work of the $M/G/k$ system under an arbitrary scheduling policy. In \cref{thm:sep_k} of \Cref{sec:mgk}, we exactly characterize the mean relevant work for the Sep-ISQ-$k$ system.


\begin{theorem}
\label{thm:isq_sep}
     For an arbitrary job size distribution $S$ and arrival rate $\lambda$, the expected relevant work in the $M/G/k$ system under an arbitrary scheduling policy is lower bounded by the expected relevant work in the separate-recycling-ISQ-$k$ system.
\end{theorem}
\begin{proof}
    We want to lower bound the relevant work in the $M/G/k$. We will divide that work into two categories: Relevant work from jobs with original size $\leq x$, and relevant work from jobs with original size $> x$. We will show that the truncated-ISQ-$k$ system lower bounds the former category, and the M/G/$\infty$ with server speed $1/k$ lower bounds the latter category.
    
    Consider a coupled pair of systems: A truncated $M/G/k$ and the full $M/G/k$. Whenever a job arrives to the full $M/G/k$, if that job has size $\le x$, a job with the same size arrives to the truncated $M/G/k$. By Poisson splitting, the arrival process to the truncated $M/G/k$ is the desired process.
    
    To complete the coupling, we need to specify how the scheduling policies in the two systems relate to each other.
    Let an arbitrary scheduling policy be used in the full $M/G/k$. At any given point in time, we specify that the truncated $M/G/k$ serves each job with original size $\leq x$ that is in service in the full $M/G/k$ at that point in time. Note that this policy may waste servers, but it is an admissible policy.
    
    With this coupling in place,
    the total work in the truncated-$M/G/k$ system with this scheduling policy, all of which is relevant, lower bounds the total work from jobs with initial size $\leq x$ in the full-$M/G/k$ system at any given time, all of which is similarly relevant.
    
    From \Cref{thm:main}, we know that the total work in the truncated $M/G/k$ is lower bounded by the total work in the truncated-ISQ-$k$ system. This completes the bound for jobs with original size $\le x$.
    
    
    For jobs with initial size $> x$, the expected relevant work in the separate $M/G/\infty$ system lower bounds the expected relevant work of jobs with initial size $>x$ in the full-$M/G/k$ system because the rate of completion in the $M/G/\infty$ system is always at $1/k$ for each jobs whereas the rate of completion of each job in the full $M/G/k$ is never more than $1/k$.
    
    Thus we see that the expected relevant work in the full-$M/G/k$ system is lower bounded by the expected relevant work in the separate-recycling-ISQ-$k$ system.

\end{proof}

We now switch our focus to the arbitrary-recycling ISQ-$k$ system.
Recall that we define a recycling to occur when a job of original size greater than $x$ ages down to remaining size of exactly $x$ in the $M/G/k$. From the perspective of relevant work, a recycling event looks like a remaining-size-$x$ job popping into existence, i.e. becoming relevant for the first time. Recyclings happen with rate $\lambda (1-F_S(x))$, because every job with original size $> x$ eventually recycles.

In particular, imagine a job of size $x$ arriving into the AR-ISQ-$k$ system whenever a recycling occurs in the $M/G/k$ system. This arrival sequence has a rate $\lambda (1-F_S(x))$ and is Markovian.



\begin{theorem}
\label{thm:ar_isq}
    For an arbitrary job size distribution $S$ and arrival rate $\lambda$,
    and an arbitrary $M/G/k$ scheduling policy,
    there exists a Markovian recycling stream such that
    the expected relevant work in the $M/G/k$ under the given scheduling policy is lower bounded by the expected relevant work in an arbitrary-recycling-ISQ-$k$ with the given recycling stream. 
\end{theorem}
\begin{proof}
Consider the full $M/G/k$, and specifically consider the relevant work in the $M/G/k$.

Consider a specific realization of the Poisson arrival sequence into the full $M/G/k$ under an arbitrary scheduling policy. For jobs with original size greater than $x$, there is some time  when the job becomes relevant for the first time. Therefore, from the perspective of the relevant work in the $M/G/k$ this is equivalent to a job with size exactly $x$ arriving into the system according to some arbitrary arrival process. 

In other words, the relevant work in the full $M/G/k$ matches the total work in a $k$-server system with two arrival streams: Poisson arrivals for jobs with size $\le x$, and arbitrary arrivals at rate $\lambda(1-F_S(x))$ of size exactly $x$, with equivalent scheduling policies.

For this specific arbitrary arrival stream, by  \Cref{prop:work}, we have that the relevant work in the AR-ISQ-$k$ system with the same recycling stream lower bounds the relevant work in the arbitrary $k$-server system and thus lower bounds the relevant work in the full $M/G/k$ system. 

\end{proof}

Therefore, by \Cref{thm:ar_isq}, the minimum possible mean relevant work in the $M/G/k$ system under an arbitrary scheduling policy is lower bounded by the minimum possible mean work in the AR-ISQ-$k$ system under an arbitrary recycling stream. We lower bound the mean work in the AR-ISQ-$k$ system in \Cref{thm:ar_2}.

\section{Bounding Mean Relevant Work in the $M/G/2$}
\label{sec:mg2}

In this section, we derive explicit lower bounds on mean relevant work in the $M/G/2$ using the Sep-ISQ-2 and AR-ISQ-2 systems. We start with the 2-server system because it demonstrates the core idea of our proof before we move on to the more general results concerning the $M/G/k$.
In \cref{sec:numerical-isq2}, we numerically compare our novel bounds on mean relevant work to the existing bounds in the literature.

We will use the following test functions to bound mean relevant work in the Sep-ISQ-2 and AR-ISQ-2 systems. We explain the intuition behind these test functions in \Cref{sec:derive}.
\begin{definition}
\label{def:isq2_constant}
We define the ISQ-2 constant-drift test function $g_2$ as follows,
\begin{align}
\label{eqn:isq2_constant}
g_2(w,1)=w,\quad g_2(0,0)=0\quad \text{and}\quad g_2(w,1/2)=w+\dfrac{1-e^{-2\lambda w}}{2\lambda}.
\end{align}
\end{definition}


\begin{definition}
\label{def:isq2_affine}
 We define the ISQ-2 affine-drift test function $h_2$ as follows,
\begin{align}
\label{eqn:isq2_affine}
    h_2(w,1)=w^2,\quad h_2(0,0)=0\quad\text{and}\quad h_2(w,1/2)=w^2+\dfrac{w}{\lambda}-\dfrac{1-e^{-2w\lambda}}{2\lambda^2}.
\end{align}
\end{definition}

It is easy to verify that $g_2$ and $h_2$ satisfy the assumptions of \Cref{lem:drift}, our BAR result. We now characterize the mean work of the ISQ-2 system by applying \Cref{lem:drift} to the test functions $g_2$ and $h_2$. 
\begin{proposition}
\label{prop:isq2_work}
    For any job size distribution $S$ such that $E[S^3]$ is finite, and any arrival rate $\lambda$, the expected total work in the ISQ-$2$ system is given by 
    \begin{align}
    \label{eqn:isq2_work}
        \E[W^{\text{ISQ-2}}]=\dfrac{\lambda\E[S^2]}{2(1-\lambda\E[S])}+\dfrac{\E[S]-(1-\widetilde{S}(2\lambda))/2\lambda}{3-\widetilde{S}(2\lambda)}.
    \end{align}
\end{proposition}
\begin{proof}
    In order to characterize the mean work of the ISQ-$k$ system, we first need to characterize the fraction of the time that the system is idle, $\mathbb{P}(I=0)$.
    
    To do so, we first consider the constant-drift test function $g_2$,
    defined in
    \Cref{def:isq2_constant}.
    We want to calculate the drift $G \circ g_2(w, i)$
    for all possible values of $w$ and $i$.
    Recall from \Cref{sec:isq,def:isq2_constant} that
    the possible values of the speed $i$ are speeds $0, 1/2,$ and $1$.
    When $i=0$, the work must be $0$ by definition,
    while if $i>0$, the work must be positive.
    
    When $w>0$ and $i=1$, $G\circ g_2(w,i)=\lambda \E[S]-1.$ Arrivals cause a drift of $\lambda E[S]$,
    while work completion causes a drift of $-1$. When $w>0$ and $i=1/2$, applying \Cref{lem:isq_drift}, we have $G\circ g_2(w,i)=\lambda \E[S]-1.$
    The choice of the function $g_2$ ensures this drift property (See \Cref{sec:derive}). When $w=0$ and $i=0$, 
    \begin{align*}
    G\circ g_2(0,0)=\dfrac{1}{2}\left(1+2\lambda\E[S]-\widetilde{S}(2\lambda)) \right)=\lambda \E[S]-1+\left(\frac{3}{2}-\dfrac{1}{2}\widetilde{S}(2\lambda))\right).
    \end{align*}
    Thus, we can summarize the drift over all states as
    \begin{align*}
    G\circ g_2(w,i)=\lambda \E[S]-1 + \left(\frac{3}{2}-\frac{1}{2}\widetilde{S}(2\lambda))\right)\mathbbm{1}_{\{i=0\}}.
    \end{align*}
    Setting the expectation to zero by \cref{lem:drift}
    we have,
    \begin{align}
    \label{eqn:isq2_prob0}
    \mathbb{P}(I=0)=\dfrac{2(1-\lambda \E[S])}{3-\widetilde{S}(2\lambda)}.
    \end{align}
    Now we switch our focus to characterizing the mean work of the ISQ-$k$ system. We consider the affine-drift test function, $h_2$, defined in \Cref{def:isq2_affine}. When $w>0$ and $i=1$ or $1/2$,$ G\circ h_2(w,i)=\lambda\E[S^2]+2w(-1+\lambda\E[S]).$
    When $w=0$ we have, $G\circ h_2(0,0)=\lambda\E[S^2]+\E[S]+\frac{\widetilde{S}(2\lambda))-1}{2\lambda}.$
    Together, we can summarize the drift over all states as
    \begin{align*}
    G\circ h_2(w,i)=\lambda\E[S^2]+2w(-1+\lambda\E[S])+\left(\E[S]+\dfrac{\widetilde{S}(2\lambda))-1}{2\lambda}\right)\mathbbm{1}_{\{i=0\}}.
    \end{align*}
    Therefore, taking expectation of $G \circ h_2$ 
    and equating to zero using \Cref{lem:drift}, we have
    \begin{align*}
    \E[W^{ISQ{\text -}2}]&=\dfrac{\lambda\E[S^2]}{2(1-\lambda\E[S])}+\dfrac{\E[S]-(1-\widetilde{S}(2\lambda))/2\lambda}{2(1-\lambda\E[S])}\cdot\mathbb{P}(I=0)=\dfrac{\lambda\E[S^2]}{2(1-\lambda\E[S])}+\dfrac{\E[S]-(1-\widetilde{S}(2\lambda))/2\lambda}{3-\widetilde{S}(2\lambda)},
    \end{align*}
    where $\mathbb{P}(I=0)$ is given by \Cref{eqn:isq2_prob0}.

\end{proof}

Using \Cref{prop:isq2_work}, we can characterize the mean work of the truncated-ISQ-2 system in the Sep-ISQ-2 system. This provides us an exact mean relevant work formula for the Sep-ISQ-2 system. We therefore lower bound the mean relevant work of an $M/G/k$ system under arbitrary scheduling policy by \Cref{thm:ar_2}. This constitutes the first lower bound of \Cref{thm:main_2}, namely \Cref{eqn:main_2_1}. Our characterization of mean relevant work in the Sep-ISQ-2 system is as follows:

\begin{theorem}
\label{thm:sep_2}
    For an arbitrary threshold $x$, arbitrary job size distribution $S$, and arbitrary arrival rate $\lambda$, the expected relevant work in the separate-recycling-ISQ-2 system is exactly given by
    \begin{align}
    \label{eqn:isq2_sep}
        \E[W_x^{\text{sep-ISQ}}]=\dfrac{\lambda_x\E[S_x^2]}{2(1-\lambda_x\E[S_x])}+\dfrac{\E[S_x]-(1-\widetilde{S_x}(2\lambda_x))/2\lambda_x}{3-\widetilde{S_x}(2\lambda_x)}+\lambda(1-F_S(x))x^2.
    \end{align}
\end{theorem}


\begin{proof}
    The relevant work in the separate-recycling-ISQ-2 system is the sum of the total work in the truncated-ISQ-2 system and the relevant work in the separate $M/G/\infty$ system. Note that all work in the truncated-ISQ-2 system is relevant.
    
    First, let us handle the separate $M/G/\infty$ system.
    The servers at the $M/G/\infty$ system will operate at a speed of $1/2$, and the arrival rate into this system is determined by $\lambda(1-F_S(x))$. Thus the expected relevant work at this separate server is $\lambda(1-F_S(x))x^2$. 

    Jobs arrive into the truncated-ISQ-2 with a conditional size $S_x$, having a density given by $f_S(x)/F_S(x)$ and bounded third moment $\E[S_x^3]\leq x^3$. The arrival rate is $\lambda_x := \lambda F_S(x)$. In particular, $\E[S_x]=\int_0^x sf(s)/F_S(x) ds$, $\E[S^2_x]=\int_0^x s^2f(s)/F_S(x) ds$, and $\widetilde{S}_x(2\lambda_x)=\int_0^x e^{-2\lambda_x s}f_S(s)/F_S(x) ds$. By \Cref{prop:isq2_work}, we have the following formula for expected relevant work formula for the truncated-ISQ-2 system:
    \begin{align*}
    \E[W_x^{truncated{\text -}ISQ}]=\dfrac{\lambda_x\E[S_x^2]}{2(1-\lambda_x\E[S_x])}+\dfrac{\E[S_x]-(1-\widetilde{S_x}(2\lambda_x))/2\lambda_x}{3-\widetilde{S_x}(2\lambda_x)}.
    \end{align*} 
    Adding the two expected relevant work formulas together we have \Cref{eqn:isq2_sep}.

\end{proof}


To derive the ISQ-2 arbitrary-recycling lower bound, we will use the following modified ISQ-2 affine-drift test function $h_{2,x}$. Note that $h_{2,x}$ is a different function for each value of $x$, we provide intuition on how we derive $h_{2,x}$ in \Cref{sec:derive}. Again, it is easy to verify that the modified ISQ-2 affine-drift test function satisfies the assumptions of \Cref{lem:drift}.

\begin{definition}
\label{def:isq2_modified}
The ISQ-2 modified affine-drift test function $h_{2,x}$ is $h_{2,x}(0,0)=0$,
\begin{align}
\label{eqn:isq2_modified_affine}
h_{2,x}(w,1)=w^2\quad\text{and}\quad h_{2,x}(w,1/2)=w^2+\frac{w}{\lambda_x}-\frac{1-e^{-2w\lambda_x}}{2\lambda_x^2}-\frac{C_2(x,\lambda_x)(1-e^{-2w\lambda_x})}{\lambda_x},
\end{align} 
where $C_2(x,\lambda_x):=\frac{\E[S_x]-(1-\widetilde{S_x}(2\lambda_x))/2\lambda_x}{3-\widetilde{S_x}(2\lambda_x)}.$
Here, the first argument of $C_2$ is the relevancy cutoff level $x$ of the arbitrary-recycling ISQ-2 system.

\end{definition}


We are ready to state the main result for our AR-ISQ-2 lower bound, which provides us with another lower bound in mean relevant work in the $M/G/k$ by \Cref{thm:ar_2}. This constitutes the second lower bound of \Cref{thm:main_2}, namely \Cref{eqn:main_2_2}.

\begin{restatable}{theorem}{thmarisq}
\label{thm:ar_2}
 For an arbitrary threshold $x$, job size distribution $S$, arrival rate $\lambda$, and arbitrary recycling stream, expected relevant work in the arbitrary-recycling-ISQ-2 system is lower bounded by
    \begin{align}
    \label{eqn:isq2_arb}
        \E[W_{x}^{\text{arb-ISQ}}]\geq \dfrac{\lambda_x \E[S_x^2]}{2 (1 - \rho_x)} + \dfrac{\E[S_x]-(1-\widetilde{S_x}(2\lambda_x))/2\lambda_x}{3-\widetilde{S_x}(2\lambda_x)}\cdot\dfrac{1-\rho_{\overbar{x}}}{1-\rho_x}+\dfrac{(\lambda - \lambda_x)x^2}{2(1-\rho_x)}.
    \end{align}
\end{restatable}

\begin{proof}
Let $\E_r[\cdot]$ denote the Palm expectation taken over the moments when the arbitrary arrival of jobs of size $x$ occur. By \cref{lem:isq_drift}, the expected drift $G \circ h_{2,x}$ has two kinds of terms, the stochastic drift due to Poisson arrivals of rate $\lambda_x = \lambda F_S(x)$ and the recycling jumps due to the arbitrary arrivals of jobs of size $x$ with rate $\lambda (1-F_S(x))$.

Let $J_x(w, i) := h_{2,x}(w+x,\min(i+1/2, 1)) - h_{2,x}(w,i)$
denote the increase in the test function due to the arrival of a size-$x$ job.
Then $\E_r[J(W_x,I)]$ denotes the mean size of the recycling jump in stationarity. The expectation $\E_r[\cdot]$ can be interpreted as the Palm expectation associated with the random measure that records the cumulative number of recyclings.
We define $\E_{r \mid w, i, a}[\cdot]$ to be the conditional Palm expectation in the immediate future of a specific state of the AR-ISQ-$k$ system.
See, for example, \cite{Miyazawa1994}, \cite{Scully2020} and \cite{Braverman2024}.
We start by applying \Cref{lem:ar_drift} to find the drift in a specific state:
\begin{align*}
    G \circ h_{2,x}(w, i) =\lambda_x\E[S_x^2]+2w(-1+\lambda\E[S_x])+2C_2(x,\lambda_x)\left(\mathbbm{1}_{\{i=0\}}+\frac{1}{2}\mathbbm{1}_{\{i=1/2\}} 
    \right) + (\lambda - \lambda_x) \E_{r \mid w, i, a}[J_x(w,i)]
\end{align*}

Applying our BAR result \cref{lem:isq_drift} to the drift $G \circ h_{2,x}$, we find that
\begin{align*}
    0 = \E[G \circ h_{2,x}(W_x, I)] &=  \lambda_x\E[S_x^2]+2\E[W_x](-1+\lambda_x\E[S_x])+2C_2(x,\lambda)\left(\mathbb{P}(I^r=0)+\frac{1}{2}\mathbb{P}(I^r=1/2)\right)\\
    &\quad\quad + (\lambda - \lambda_x) \E_r[J_x(W_x,I)].
\end{align*}
Solving for $\E[W_x]$, we get
\begin{align}
\label{eqn:isq2_wx}
    \E[W_x] = \frac{\lambda_x E[S_x^2]}{2 (1 - \rho_x)} + \dfrac{C_2(x,\lambda_x)}{1-\rho_x}\left(\mathbb{P}(I^r=0)+\frac{1}{2}\mathbb{P}(I^r=1/2)\right) + \frac{(\lambda - \lambda_x) \E_r[J_x(W_x,I)]}{2(1-\rho_x)}.
\end{align}
In the above equation, $\mathbb{P}(I^r=0)$ and $\mathbb{P}(I^r=1/2)$ are difficult to evaluate. However, we show that we can relate the two probabilities to the capped load $\rho_{\overbar{x}}$.
To do so, we apply \Cref{lem:ar_drift} to the test function $g(w,i)=w$ for all $(w,i)$, we have
\begin{align*}
    0=\lambda_x\E[S_x]-1+\mathbb{P}(I^r=0)+\frac{1}{2}\mathbb{P}(I^r=1/2)+(\lambda-\lambda_x)x
\end{align*}
Re-arranging, and using the definition that $\rho_{\overbar{x}}=\lambda_x\E[S_x]+(\lambda-\lambda_x)x$, we get
\begin{align}
\label{eqn:isq2_load}
    \mathbb{P}(I^r=0)+\frac{1}{2}\mathbb{P}(I^r=1/2)=1-\rho_{\overbar{x}}.
\end{align}
Plugging in \Cref{eqn:isq2_load} into \Cref{eqn:isq2_wx}  we get
\begin{align*}
    \E[W_x] = \frac{\lambda_x E[S_x^2]}{2 (1 - \rho_x)} + \dfrac{C_2(x,\lambda_x)(1-\rho_{\overbar{x}})}{1-\rho_x} + \frac{(\lambda - \lambda_x) \E_r[J_x(W_x,I)]}{2(1-\rho_x)}.
    \end{align*}
    
Next, we want to lower bound $J_x(w,i)$ over all possible states in which a recycling could occur, and specifically the three cases $i=1,1/2,0$, respectively. In particular, we show that $J_x(w,i)$  is uniformly lower bounded by $x^2$ for any arbitrary job size distribution $S$ and arrival rate $\lambda$. 
\begin{itemize}
    \item Jump size $J_x(w,i)$ at speed $i=1$ is given by $h_{2,x}(w+x,1)-h_{2,x}(w,1)=2wx+x^2\geq x^2$. 

    \item Jump size $J_x(w,i)$ at speed $i=1/2$ is given by
    \begin{align*}
        h_{2,x}(w+x, 1) - h_{2,x}(w, 1/2) &=2wx + x^2 - \frac{w}{\lambda_x} + \dfrac{1-e^{-2w\lambda_x}}{2\lambda_x^2}+\dfrac{C_2(x,\lambda_x)(1-e^{-2x\lambda_x})}{\lambda_x}\\
        &\geq 2wx + x^2 - \frac{w}{\lambda_x} + \dfrac{1-e^{-2w\lambda_x}}{2\lambda_x^2}.
    \end{align*}
    Let $J_{lb}(w,x) = 2wx + x^2 - \frac{w}{\lambda_x} + \frac{1-e^{-2w\lambda_x}}{2\lambda_x^2}$ denote this lower bound.
    
    We now minimize $J_{lb}$ over $w$.
    To do so, we split into two cases: $x > \frac{1}{2\lambda_x}$,
    and $x \le \frac{1}{2\lambda_x}$.
    If $x>\frac{1}{2\lambda_x}$, then $J_{lb}(w,1/2)$ is a concave increasing function in $w$. Thus, the function attains its minimum when $w=0$. Otherwise, $J_{lb}(w,1/2)$ is a concave function with a unique maximum. Because $J_{lb}$ is a concave function, its minimum must be either $w=0$ or $w=x$. The following calculation shows that the minimum is always at least $x^2$.
    \begin{align*}
        J_{lb}(x,x) &= 3x^2-\frac{x}{\lambda_x}-\frac{e^{-2x\lambda_x}-1}{2\lambda_x^2}\geq 3x^2-\frac{x}{\lambda_x}-\frac{1-2x\lambda_x+2x^2\lambda_x^2-1}{2\lambda_x^2}=2x^2\geq x^2 \\
        J_{lb}(0, x) &= x^2  \ge x^2.
    \end{align*}


    \item Jump $J_x(w,i)$ at speed $i=0$ is given by $x^2 + \frac{x}{\lambda_x} - \frac{1 - e^{-2x\lambda_x}}{2\lambda_x^2}-\frac{C_2(x,\lambda_x)(1-e^{-2x\lambda_x})}{\lambda_x}$.

    To lower bound $J_x(0, 0)$, we must upper bound $C_2(x, \lambda_x)$, which is defined in \Cref{def:isq2_modified}.
    Using the fact that $\E[S_x] \le x$ and that $\widetilde{S_x}(\cdot) \le 1$, we find that
    \begin{align*}
        C_2(x,\lambda_x)=\dfrac{\E[S_x]-(1-\widetilde{S_x}(2\lambda_x))/2\lambda_x}{3-\widetilde{S_x}(2\lambda_x)}\leq\dfrac{x-(1-1)/2\lambda_x}{3-1}=\dfrac{x}{2}.
    \end{align*}    
    
    Now, we want to lower bound $J_x(0, 0)$. Note that $C_2(x, \lambda_x)$ has a negative coefficient in the formula for $J_x(0, 0)$, allowing us to apply our upper bound on $C_2(x, \lambda_x)$ to derive a lower bound on $J_x(0, 0)$:
    \begin{align*}
        J_x(0, 0) = x^2 + \frac{x}{\lambda_x} - \frac{1 - e^{-2x\lambda_x}}{2\lambda_x^2}-\frac{C_2(x,\lambda_x)(1-e^{-2x\lambda_x})}{\lambda_x}
        \ge x^2 + \frac{x}{\lambda_x} - \frac{1 - e^{-2x\lambda_x}}{2\lambda_x^2}-\frac{x(1-e^{-2x\lambda_x})}{2\lambda_x}.
    \end{align*}
    Therefore, it remains to show that the last three terms on the RHS of the above equation, which we define as the function $r(x)$, are nonnegative:
    \begin{align*}
     r(x):=\frac{x}{\lambda_x} - \frac{1 - e^{-2x\lambda_x}}{2\lambda_x^2}-\frac{x(1-e^{-2x\lambda_x})}{2\lambda_x}=\frac{2x\lambda_x-(1-e^{-2x\lambda_x})(1+x\lambda_x)}{2\lambda_x^2}.
    \end{align*}
    If we show that $r(x) \ge 0$ for all $x \ge 0$, then we have shown that $J(0,0)$ is lower bounded by $x^2$. 
    
    This is equivalent to showing that the numerator of the last term above is nonnegative, i.e.,
    \begin{align*}
        2x\lambda_x-(1-e^{-2x\lambda_x})(1+x\lambda_x)=(x\lambda_x-1)+(1+x\lambda_x)e^{-2x\lambda_x}\geq 0.
    \end{align*}
    Let $y:=x\lambda_x$. Note that $y>0$. Then the inequality becomes
    $(y-1)+(1+y)e^{-2y}\geq 0 \iff (1+y)e^{-2y}\geq 1-y$ which is trivially true if $y\geq 1$. For $y\in(0,1)$, we show that the equivalent inequality $\frac{1+y}{1-y}\geq e^{2y}$ holds.
    Because $\ln(\cdot)$ is increasing and $\frac{1+y}{1-y}$ is positive for all $y \in (0, 1)$, we equivalently show that $\ln \frac{1+y}{1-y} \geq 2y$.
    
    We start with the Taylor expansions $\ln(1+y)=y-y^2/2+y^3/3-\cdots$ and $\ln(1-y)=-y-y^2/2-y^3/3-\cdots$. We therefore have, $\ln\left(\frac{1+y}{1-y}\right)=\ln(1+y)-\ln(1-y)=2y+\frac{2y^3}{3}+\frac{2y^5}{5}+\cdots\geq 2y.$
    Exponentiating both sides, we have $\frac{1+y}{1-y}\geq e^{2y}$ for $y\in(0,1)$, so $r(x) \ge 0$ for all $x \ge 0$. Thus, $J(0, 0) \ge x^2$.
    
    
\end{itemize}    

Therefore, because $J_x(w,i)$ is lower bounded by $x^2$ for all $w,i$,
we know that $\E_r[J_x(W_x, I)]$ is similarly lower bounded by $x^2$.
We therefore have our desired lower bound on the mean relevant work in the arbitrary recycling ISQ-2 system:
    \begin{align*}
        \E[W_{x}^{arb{\text -}ISQ}]\ge\frac{\lambda_x \E[S_x^2]}{2 (1 - \rho_x)} + \dfrac{C_2(x,\lambda)(1-\rho_{\overbar{x}})}{1-\rho_x}+\frac{(\lambda - \lambda_x)x^2}{2(1-\rho_x)}. 
    \end{align*}
\end{proof}





\section{Deriving the Test Functions -- DiffeDrift}
\label{sec:derive}
In this section, we present our DiffeDrift method, which builds upon the drift method/BAR approach from prior literature. In DiffeDrift, we first select the desired drift and then derive the corresponding test function using differential equations. We focus on the ISQ-2 test functions introduced in \Cref{sec:mg2}, as they illustrate the main concept of the method. We generalize these test functions to the general case in \Cref{sec:mgk}.

We start by deriving the affine-drift test function (\Cref{def:isq2_affine}) in \Cref{sec:derive_affine} as well as the constant-drift test function (\Cref{def:isq2_constant}), which we used to characterize the mean total work of an ISQ-2 system. We then derive the modified affine-drift test function (\Cref{def:isq2_modified}) in \Cref{sec:derive_modified}, which is a specialized test function for the AR-ISQ-2 system.

\subsection{Affine-drift and constant-drift test functions} 
\label{sec:derive_affine}
Recall that the instantaneous drift operator $G$ is a stochastic version of a derivative operator. To find information about the mean work in the system, we consider test functions with a leading quadratic term $w^2$. The drift of such a test function is a linear function of $w$ and by applying \Cref{lem:drift} we plan to solve for $\E[W]$.


The possible states of the ISQ-2 system are $(0,0),(w,1/2)$ and $(w,1)$. There are three events in the system that can affect the state: stochastic arrivals, deterministic decrease in $w$ and the completion of the final job in the system. 

Due to arrivals, $w$ increases as stochastic jumps of size $S$ arrive at rate $\lambda$. When $i>0$, due to work completion $w$ decreases at rate $i$. Therefore, using \Cref{lem:isq_drift} we can write down the drift for any arbitrary test function $h$,
\begin{align*}
    G\circ h(w,i)=\lambda(\E[h(w+S,\min\{i+1/k,1\})-h(w,i)])-h'(w,i)\cdot i.
\end{align*}
Note that for $h$ to satisfy the conditions of \Cref{lem:drift}, it must change continuously if no stochastic events occur. In particular, when a busy period ends it must be that $\lim_{w\to0^+}h(w,i)=h(0,0):=0$ for all $i$.

To derive the affine-drift test function $h_2$, we start with a simple expression for $h_2(0, 0)$ and $h_2(w, 1)$, and solve for the necessary form of $h_2(w, 1/2)$. We define $h_2(w,0)=0$ as above, and let $h_2(w,1)=w^2$.

Now, our goal is to define $h_2(w, 1/2)$ to ensure that the drift at speed $1/2$ matches the drift at speed $1$. This allows us to isolate the complexity of the drift function to the case $i=0$ when applying \Cref{lem:drift}.

We calculate the drift at speed $1$,
\begin{align*}
    G\circ h_2(w,1)=\lambda(\E[h_2(w+S,1)]-\E[h_2(w+S,1)])-h'_2(w,1)=\lambda\E[S^2]+2\lambda w\E[S]-2w.
\end{align*}

We can also write down the drift at speed $1/2$,
\begin{align*}
    G\circ h_2(w,1/2) &= \lambda (\E[h_2(w+S,1)-h_2(w,1/2)])-h'_2(w,1/2)\cdot \frac{1}{2}\\
    &=\lambda\E[S^2]+\lambda w^2+2\lambda w\E[S]-\lambda h_2(w,1/2) -\frac{h'_2(w,1/2)}{2}.
\end{align*}

By comparing $G\circ h_2(w,1)$ with $G\circ h_2(w,1/2)$, we see that the two drifts match if and only if $h_2(w,1/2)$ solves the following differential equation,
\begin{align*}
    \lambda w^2+2w-\lambda h_2(w,1/2)-\frac{h_2'(w,1/2)}{2}=0,\quad h_2(0,1/2)=0,
\end{align*}
which has a unique solution given by $h_2(w,1/2)=w^2-\frac{1-e^{-2w\lambda}}{2\lambda^2}+\frac{w}{\lambda}.$ Ssolving this differential equation is the essence of our Diffedrift method. More complex in general $k$ case. This defines the affine-drift test function $h_2(w,i)$ over the three possible states $(0,0), (w,1/2)$ and $(w,1)$, as given in \Cref{def:isq2_affine}. 

In the proof of \Cref{prop:isq2_work}, we see that the affine-drift test function $h_2$ alone is insufficient to derive  $\E[W]$, we also need to determine $\mathbb{P}(I=0)$. Since $w$ has a constant drift, we can characterize $\mathbb{P}(I=0)$ using a test function with a leading linear term in $w$. By following similar differential-equation-based steps, one can derive the constant-drift test function $g_2$ defined in \Cref{def:isq2_constant}. We generalize the affine-drift test function to the setting of a general number of servers $k$ in \Cref{lem:isqk_affine}.

\subsection{Modified affine-drift test function}
\label{sec:derive_modified}
The affine-drift and constant-drift test functions are sufficient to determine the mean work in the ISQ-2 system. They also suffice to determine the mean relevant work in the Sep-ISQ-2 system, in the AR-ISQ-2 system an recycling stream with jobs of size $x$ arrives into the same system as the truncated stream. Therefore, $\mathbb{P}(I
=0)$ is no longer given by the expression derived in the proof of \Cref{prop:isq2_work} and we cannot apply the result of \Cref{prop:isq2_work} in determining the changes in the drift of the AR-ISQ-2 system due to the truncated stream. Letting $I^r$ denote the speed distribution in $\mathbb{P}(I^r=0)$, which now depends on the specific recycling stream, which is difficult to characterize exactly.

However, note that the load of the system in equilibrium does not depend on the recycling stream. Using the test function $g(w,i)=w$ for all $(w,i)$, we find that
\begin{align}
    \label{eq:unused-load}
    1-\rho_{\bar{x}}=\mathbb{P}(I^r=0)+\frac{1}{2}\mathbb{P}(I^r=1/2).
\end{align}
This is a characterization of the unused capacity in the AR-ISQ-2 system, which is unaffected by the details of recycling stream. Therefore, our plan is to modify the affine-drift test function $h_2$ so that we get an unused-load term matching \eqref{eq:unused-load}. Additionally, we use \Cref{lem:ar_drift} instead of \Cref{lem:drift} because it supports recyclings. We start with the following test functions:
\begin{align*}
h_{2,x}(w,1)=w^2,\quad h_{2,x}(0,0)=0\quad\text{and}\quad h_{2,x}(w,1/2)=w^2+\ell_1(w),
\end{align*} 
for a function $\ell_1$ to be determined.

We do not want to match the drift at speed $1/2$ with the drift at speed $1$ which would results in a $\mathbb{P}(I^r=0)$ term. Instead, to obtain an unused-load term we want to choose $\ell_1(w)$ to ensure that
\begin{align}
    \label{eq:drift-derivation}
    G\circ h_{2,x}(w,1/2) - G\circ h_{2,x}(w,1) =\frac{1}{2}\left( G\circ h_{2,x}(0,0) - (\lim_{w\to 0^+} G\circ h_{2,x}(w,1))\right) =: C_2(x, \lambda_x).
\end{align}
In particular, we want to ensure that \eqref{eq:drift-derivation} is a constant not depending on $w$. By doing so, we will create a term matching the unused-load \eqref{eq:unused-load}, allowing us to effectively bound the AR-ISQ-2 system. We call this constant $C_2(x, \lambda_x)$, though we do not yet know its exact value. The drift at speed $1/2$ is given by,
\begin{align*}
    G\circ h_{2,x}(w,1/2)&=\lambda_x(E[h_{2,x}(w+S_x,1)-h_{2,x}(w,1/2)])-\frac{1}{2}(h_{2,x}(w,1/2))'\\
    &=\lambda_x E[S_x^2]+2w(\lambda_x E[S_x]-1)+w-\lambda_x \ell_1(w)-\frac{\ell_1'(w)}{2}.
\end{align*}

Note that $G\circ h_{2,x}(w,1/2) - G\circ h_{2,x}(w,1) = w-\lambda_x \ell_1(w)-\frac{\ell_1'(w)}{2}$.
Our goal is to choose $\ell_1(w)$ to ensure that this quantity is a constant, which moreover matches $\frac{1}{2}\left( G\circ h_{2,x}(0,0) - G\circ h_{2,x}(0,1)\right)$. This is the heart of the DiffeDrift method.  

Solving the  differential equation $w-\lambda_x \ell_1(w)-\frac{\ell_1'(w)}{2}=C_2(x,\lambda_x)$ with $\ell_1(0)=0$, we find that,
\begin{align*}
    \ell_1(w)=\frac{w}{\lambda_x}-\frac{1-e^{-2w\lambda_x}}{2\lambda_x^2}-\frac{C_2(x,\lambda_x)(1-e^{-2w\lambda_x})}{\lambda_x}.
\end{align*}

Now, we can solve for $C_2(x, \lambda)$ as follows: By definition, we have $G\circ h_{2,x}(0,0)=\lambda_x \E[S_x^2]+\lambda_x \E[\ell_1(S_x)]$ and $\lim_{w \to 0^+} G\circ h_{2,x}(w,1) = \lambda_x E[S_x^2]$. Therefore,
\begin{align*}
    2C_2(x, \lambda_x) = \lambda_x \E[\ell_1(S_x)] \implies
    C_2(x, \lambda_x) = \dfrac{E[S_x]-(1-\widetilde{S_x}(2\lambda_x))/2\lambda_x}{3-\widetilde{S_x}(2\lambda_x)}.
\end{align*}

Now, we can exactly derive the expected value of the drift due to the truncated stream:
\begin{align*}
    \E[G\circ h_{2,x}(W_x,I)]&=\lambda_x\E[S_x^2]+2\E[W_x](-1+\lambda_x\E[S_x])+2C_2(x,\lambda_x)\left(\mathbb{P}(I^r=0)+\frac{1}{2}\mathbb{P}(I^r=1/2)\right)\\
    &=\lambda_x\E[S_x^2]+2\E[W_x](-1+\lambda_x\E[S_x])+2C_2(x,\lambda_x)(1-\rho_{\bar{x}}).
\end{align*}
We were able to apply the unused capacity formula \eqref{eq:unused-load} because our test function satisfied \eqref{eq:drift-derivation}. This is key to the strength of our AR-ISQ-2 bound \cref{thm:ar_2}.


We generalize the affine-drift test function to the setting of a general number of servers $k$ in \Cref{lem:isqk_modified}, using the same differential-equation structure for the DiffeDrift method.

\section{Bounding Mean Relevant Work in the $M/G/k$}
\label{sec:mgk}
In this section, we extend the results in \Cref{sec:mg2} for the ISQ-2 system to the general ISQ-$k$ system. The derivation of the test functions follow the same ideas outlined in \Cref{sec:derive}. We start by deriving our test functions using the DiffeDrift method. We first define the ISQ-$k$ constant-drift test function.

\begin{definition}
\label{def:isqk_contant}
We define the ISQ-$k$ constant-drift test function $g_k$ as follows:
\begin{align*}
    g_k(w,0/k)=0,\,g_k(w,1/k)=w+u_1(w),\,\cdots,\, g_k(w,\frac{k-1}{k})=w+u_{k-1}(w),\, g_k(w,k/k)=w,
\end{align*}
where for each $q\in\{k-1,\cdots,1\}$, we define $u_q(w)$ by the following recursive formula,
\begin{align}
\label{eqn:isqk_constant}
    u_{q}(w)=e^{-\frac{kw\lambda}{q}}\int_0^w\dfrac{e^{\frac{k\lambda y}{q}}(k-q+k\lambda \E[u_{q+1}(S+y)])}{q}\,dy,
\end{align}
where $u_k(w)=0$.
\end{definition}

Note that for $k=2$ this simplifies to the $g_2$ expression defined in \cref{def:isq2_constant}. Next, we define the ISQ-$k$ affine-drift test function.

\begin{definition}
\label{def:isqk_affine}
We define the ISQ-$k$ affine-drift test function as follows,
\begin{align*}
h_k(w,0/k)=0,\,h_k(w,1/k)=w^2+v_1(w),\,\cdots,\, h_k(w,\frac{k-1}{k})=w^2+v_{k-1}(w),\, h_k(w,k/k)=w^2.
\end{align*}
For each $q\in\{k-1,\cdots,1\}$, we define $v_q(w)$ by the following recursive formula,
\begin{align}
\label{eqn:isqk_affine}
    v_q(w)=e^{-\frac{kw\lambda}{q}}\int_0^w \dfrac{e^{\frac{k\lambda y}{q}}(2ky-2qy+k\lambda \E[v_{q+1}(S+y)])}{q}\,dy,
\end{align}
where $v_k(w)=0$. 
\end{definition}

Note that for $k=2$ this simplifies to the $h_2$ expression defined in \cref{def:isq2_affine} and that the $v_q(w)$ formulas have a similar recursive structure as $u_q(w)$, but with an additional $y$ coefficient in the numerator of the integrand fraction. \Cref{eqn:isqk_constant,eqn:isqk_affine} can be explicitly solved for any $k$, see \Cref{apdx:isq3} for the three server case. In \Cref{lem:isqk_constant_assumptions,lem:isqk_affined_bound} of \Cref{apdx:sec9} we prove the validity of these test functions.


Proceeding similarly as in the proof of \Cref{prop:isq2_work}, we compute the mean work of the ISQ-$k$. 

\begin{proposition}
\label{prop:isqk_work}
For an arbitrary job size distribution $S$ such that $E[S^3]$ is finite and arrival rate $\lambda$. The expected total work in an ISQ-$k$ system is given by 
    \begin{align}
    \label{eqn:isqk_work}
        \E[W^{\text{ISQ-k}}]=\dfrac{\lambda\E[S^2]}{2(1-\lambda\E[S])}+\dfrac{\lambda\E[v_1(S)]}{2+2\lambda\E[u_1(S)]}.
    \end{align}    
\end{proposition}
\begin{proof}
By \Cref{lem:isqk_constant} and \Cref{lem:isqk_affine} in \Cref{apdx:sec9} the drifts of the constant and affine test functions at speed $0$ are given by $G\circ g_k(0,0)=\lambda\E[S+u_1(S)]$ and $G\circ h_k(0,0)=\lambda\E[S^2+v_1(S)]$ respectively. At all other speeds $i \geq 1/k$, the drift of the constant-drift test function is given by $G\circ g_k(w,i)=\lambda\E[S]-1$ and the drift of the affine-drift test function is given by $G\circ h_k(w,i)=\lambda\E[S^2]+2w(\lambda\E[S]-1)$.

The rest of the proof can be completed in a similar fashion as in the proof of \Cref{prop:isq2_work}. Summarizing the drift of $g_k$ over all states, we get, $G\circ g_k(w,i)=\lambda\E[S]-1+(1+\lambda\E[u_1(S)])\cdot\mathbbm{1}_{\{i=0\}}.$
By \Cref{lem:isqk_constant} the constant-drift test function $g_k$ satisfies the assumption of \Cref{lem:drift}. Thus, we have $\E[G\circ g_k(W,I)]=0$ and solving for $\E[\mathbbm{1}_{\{i=0\}}]=\mathbb{P}(I=0)$ we get the probability that the system is in speed 0,
\begin{align}
\label{eqn:isqk_prob0}
    \mathbb{P}(I=0)=\dfrac{1-\lambda\E[S]}{1+\lambda\E[u_1(S)]}.
\end{align}
Similarly, we can summarize the drift of $h_k$ as
\begin{align*}
    G\circ h_k(w,i)=\lambda\E[S^2]+2w(\lambda\E[S]-1)+\lambda\E[v_1(S)]\cdot\mathbbm{1}_{\{i=0\}}.
\end{align*}
By \Cref{lem:isqk_affine} in \Cref{apdx:sec9} the affine-drift test function $h_k$ satisfies the assumption of \Cref{lem:drift}. Thus, we have $\E[G\circ h_k(W,I)]=0$. Solving for $\E[W]$ we get,
\begin{align*}
    \E[W^{ISQ{\text -}k}]&=\dfrac{\lambda\E[S^2]}{2(1-\lambda\E[S])}+\dfrac{\lambda\E[v_1(S)]}{\lambda\E[S]}\cdot\mathbb{P}(I=0)=\dfrac{\lambda\E[S^2]}{2(1-\lambda\E[S])}+\dfrac{\lambda\E[v_1(S)]}{2(1+\lambda\E[u_1(S)])}
\end{align*}    
where $\mathbb{P}(I=0)$ is given by \Cref{eqn:isqk_prob0}.

\end{proof}

Next, using \Cref{prop:isqk_work} we give the exact mean relevant work in the Sep-ISQ-$k$ system. This constitutes the first lower bound of \Cref{thm:main_k}.

\begin{theorem}
\label{thm:sep_k}
For an arbitrary job size distribution $S$ and arrival rate $\lambda$. The expected relevant work in the separate-recycling-ISQ-k system is given by,    
\begin{align}
    \label{eqn:isqk_sep}
    \E[W_x^{\text{sep-ISQ}}]=\dfrac{\lambda_x\E[S_x^2]}{2(1-\lambda_x\E[S_x])}+\dfrac{\lambda_x\E[v_1(S_x)]}{2+2\lambda_x\E[u_1(S_x)]} +\dfrac{k\lambda(1-F_S(x))x^2}{2},
\end{align}
where $v_1$ and $u_1$ are defined in \Cref{eqn:isqk_constant,eqn:isqk_affine}.
\end{theorem}
\begin{proof}
The relevant work in the separate-recycling-ISQ-$k$ system is the sum of the total work in the truncated-ISQ-$k$ and the relevant work in the separate $M/G/\infty$ system.

The derivation for the relevant work in the truncated-ISQ-$k$ system is exactly the same as in the proof of \Cref{thm:sep_2} which makes up the first two terms on the RHS of \Cref{eqn:isqk_sep}.

For the separate $M/G/\infty$ system, the servers will now operate at a speed of $1/k$, and the arrival rate into this system is $\lambda(1-F_S(x))$. Thus the mean relevant work at this separate server is $\frac{k\lambda(1-F_S(x))x^2}{2}$. 
\end{proof}

For the last part of this section, we derive the AR-ISQ-$k$ lower bound. We start with the ISQ-$k$ modified affine-test function, denoted by $h_{k,x}$.

\begin{definition}
\label{def:isqk_modified}
    We define the ISQ-$k$ modified affine-drift test function $h_{k,x}$ as follows:
    \begin{align*}
    h_{k,x}(w,0/k)=0,\,h_{k,x}(w,1/k)=w^2+\ell_1(w),\,\cdots,\, h_{k,x}(w,\frac{k-1}{k})=w^2+\ell_{k-1}(w),\, h_{k,x}(w,k/k)=w^2.
    \end{align*}
    For each $q\in\{k-1,\cdots,1\}$, we define $\ell_q(w)$ in terms of $\ell_{q+1}(w)$ by the following recursive formula,
    \begin{align}
    \label{eqn:isqk_modified}
        \ell_q(w)=e^{-\frac{kw\lambda_x}{q}}\int_0^w\dfrac{e^{\frac{k\lambda_x y}{q}}\left(k(q-k)C_k(x,\lambda_x)+2(k-q)y+k\lambda_x\E[\ell_{q+1}(S_x+y)] \right)}{q}\,dy
    \end{align}
    where $\ell_k(w)=0$ and 
    \begin{align}
    \label{eqn:C_k}
         C_k(x,\lambda_x)=\frac{2}{k}\cdot\dfrac{\lambda_x\E[v_1(S_x)]}{2+2\lambda_x\E[u_1(S_x)]}\geq 0,
    \end{align}
    where $v_1$ and $u_1$ are defined by \Cref{eqn:isqk_constant,eqn:isqk_affine}.
\end{definition}


The next theorem constitutes the second lower bound of \Cref{thm:main_k} and is based on the modified affine-drift test function. We prove this theorem in \Cref{apdx:sec9}.

\begin{restatable}{theorem}{mainarisqk}
\label{thm:ar_k}
For an arbitrary job size distribution $S$ and arrival rate $\lambda$. A lower bound on expected relevant work in the arbitrary-recycling-ISQ-$k$ system is given by,
    \begin{align}
    \label{eqn:isqk_jump}
    \E[W_x^{\text{arb-ISQ}}]\geq \dfrac{\lambda_x\E[S_x^2]}{2(1-\lambda_x\E[S_x])}+\dfrac{\lambda_x\E[v_1(S_x)]}{2+2\lambda_x\E[u_1(S_x)]}\cdot\dfrac{1-\rho_{\bar{x}}}{1-\rho_x}+\frac{(\lambda - \lambda_x)J_x}{2(1-\rho_x)}.
\end{align}
Here $J_x$ is the smallest jump size incurred by the recycling stream, i.e., 
\begin{align*}
    J_x:=\min\left\{x^2, \min_{i < 1}\left\{
        \inf_{w\in[0,kix]}h_{k,x}(w+x,i+1/k)-h_{k,x}(w,i)
        \right\}\right\}.
\end{align*}
\end{restatable}




We believe that $J_x=x^2$ for all thresholds $x$, numbers of servers $k$, and job size distributions $S$.
We proved in \cref{thm:ar_2} that $J_x=x^2$ whenever $k=2$.
However, formal proof of this assertion for $k \ge 3$ remains elusive, so we leave this as an open problem. We have empirically verified that this is true for the 3-server setting with exponential job sizes.

For theoretical evidence, one can also observe that each term in $J_x$ is of the form
\begin{align*}
    x^2 + 2wx + \E[l_{k}(x+S)]-l_{k-1}(x),
\end{align*}
and the contribution of $x^2+2wx$ should dominate the residual $\E[l_{k}(x+S)]-l_{k-1}(x)$ across all $x\geq 0$, because the residual grows at most linearly, by \Cref{lem:isqk_modified_bound} in \Cref{apdx:sec9}. Therefore we propose the following conjecture:
\begin{conjecture}
\label{conj:x2}
    For any job size distribution $S$ and any number of servers $k \ge 2, J_x=x^2$.
\end{conjecture}

\section{WINE Bounds}
\label{sec:wine}
To translate lower bounds on the mean relevant work of the $M/G/k$ system into lower bounds on its mean response time under arbitrary scheduling policies, we use the Work Integral Number Equality (WINE) introduced in Theorem 6.3 of \cite{Scully2020}. WINE allows us to characterize the mean response time of a generic queueing system under a generic scheduling policy in terms of the mean relevant work for each relevancy-cutoff level $x$.


\begin{proposition}[WINE Identity]
\label{prop:wine}
    For an arbitrary scheduling policy $\pi$, in an arbitrary system,
    \begin{align}
    \label{eqn:wine}
        \E[T^{\pi}]=\dfrac{1}{\lambda}\E[N^{\pi}]=\dfrac{1}{\lambda}\int_0^{\infty}\dfrac{\E[W_x^{\pi}]}{x^2}\,dx.
    \end{align}
\end{proposition}

WINE has been used to upper bound mean response time under complex scheduling policies in the $M/G/1$ \citep{Scully2022Uniform} as well as in multiserver systems (\cite{Scully2020} and \cite{Grosof2023Optimal}).

We use \Cref{prop:wine} in a novel fashion: rather than upper bounding relevant work for a specific policy to upper bound response time for that policy, we instead lower bound mean relevant work over all policies to lower bound mean response time over all policies.
In particular, if we can prove relevant work lower bounds at different relevancy cutoffs $x$ using different proof methods,
we can combine them by taking the strongest bound for every $x$ and integrating with \cref{prop:wine}. This will provide a stronger bound than any prior individual response time lower bound.

%because, for a family of positive functions $f_1,...,f_n$, we have $\int \max\{f_1(x), \dots, f_n(x)\} \, dx \geq \max\left\{\int f_1(x) \, dx, \dots, \int f_n(x) \, dx\right\}.$


Previously, only two naive lower bounds on the mean response time and mean relevant work of the $M/G/k$ under arbitrary scheduling policies have appeared in the literature. 

The first lower bound is the resource-pooled $M/G/1$ queue under the single-server-optimal SRPT policy. The optimal policy in the resource-pooled $M/G/1$ must lower bound the optimal policy in the $M/G/k$, as the resource-pooled $M/G/1$ is strictly more flexible, and could emulate the $M/G/k$ policy if desired. \cite{Schrage1968Proof} shows that SRPT is optimal for $M/G/1$, so it must lower bound the mean response time of the $M/G/k$ system. \cite{Grosof2018} proves that this bound is tight at high load, but it is empirically loose at medium and low load (see \Cref{fig:intro_wine}).

\cite{Schrage1966} characterizes the mean relevant work in the SRPT-1 system:
\begin{align}
    \E[W_x^{M/G/k}]\geq \E[W_x^{M/G/1 \text{-SRPT}}] = \dfrac{\lambda}{2}\dfrac{\E[S_{\bar{x}}^2]}{1-\rho_x}.
\end{align}

The second bound is the mean service time bound. To compute the mean relevant work, we imagine jobs arriving into an $M/G/\infty$ queue, whose servers run at the same speed as the $M/G/k$ servers. Each job of initial size $s$ will spend $k \min\{s, x\}$ amount of time as a relevant job, as its size shrinks linearly down to $0$. The job thus contribute $\frac{k}{2}\min\{s, x\}^2$ amount of relevant work. Therefore,
\begin{align}
\label{eqn:lower_2}
    \E[W_x^{M/G/k}]\geq \E[W_x^{M/G/\infty}]=\dfrac{k\lambda}{2}\E[S_{\bar{x}}^2].
\end{align}
The above bound is tight at low load but loose at medium and high load, see \Cref{fig:intro_wine}.

Taking the maximum of these two bounds for each $x$ and integrating using WINE (\Cref{prop:wine}) gives us our first improved lower bound on the mean response time of the $M/G/k$ under an arbitrary scheduling policy, which we call the WINE-2 bound. 

We further improve upon WINE-2 by incorporating the exact mean relevant work of the Sep-ISQ-$k$ system, giving an improved lower bound on mean relevant work in the $M/G/k$, namely \Cref{eqn:main_k_1}. Combining these bounds with WINE results in a new lower bound which we call WINE-3. Finally, WINE-3 is augmented with a lower bound on the mean relevant work of the AR-ISQ-$k$ system, giving a final lower bound on the $M/G/k$, namely \Cref{eqn:main_k_2}, resulting in the final WINE-4 bound. 

\subsection{Numerical Comparison of Bounds}
\label{sec:numerical-isq2}
We illustrate the impact of each of these new lower bounds (\Cref{eqn:main_2_1,eqn:main_2_2} of \Cref{thm:main_2}) in our WINE lower bounding framework at different thresholds $x$ in \Cref{fig:relevant_work}. Specifically, we show the weighted expected relevant work $\E[W_x]/x^2$ for each lower bound in an $M/G/2$ with exponential job size distribution setting. We plot the weighted expected relevant work lower bounds given by \eqref{eqn:main_2_1} and \eqref{eqn:main_2_2} against the $M/G/\infty$ and resource-pooled $M/G/1/SRPT$ queues. The job size distribution is $S \sim Exp(1)$ and the arrival rate is $\lambda = 0.9$.

We observe that $\E[W_x^{M/G/\infty}]$ and $\E[W_x^{sep-ISQ}]$ are almost identical for small values of $x$, but as $x$ increases, $\E[W_x^{sep-ISQ}]$ eventually overtakes $\E[W_x^{M/G/\infty}]$. Then, for large values of $x$, $\E[W_x^{arb-ISQ}]$ surpasses all of the other bounds. This occurs because, as $x$ increases, the factor $\frac{1-\rho_{\overbar{x}}}{1-\rho_x}$ in the second term of \eqref{eqn:main_2_2} approaches $1$, while the last term scales with the inverse of $1-\rho_x$. This contrasts with \eqref{eqn:main_2_1}, where such scaling behavior is missing.

\begin{figure}[htbp!]
    \centering
    \includegraphics[width=0.85\textwidth]{Figures/relevant_work.png}
    \caption{$\E[W_x]/x^2$ for four lower bounds on mean relevant work in the $M/G/2$ with $S \sim Exp(1)$ job size distribution and arrival rate $\lambda=0.9.$ The blue and orange curves are our novel lover bounds while the dashed red and green curves are naive lower bounds.
    }
    \label{fig:relevant_work}
    
\end{figure}


\section{Empirical and Numerical Results}
\label{sec:empirical}

In this section, we illustrate the effectiveness of our method by presenting empirical and numerical results on our novel lower bounds on the mean response time in the $M/G/k$ system under an arbitrary scheduling policy, based on the ISQ-$k$ system.


\subsection{Total work}
Recall that we exactly characterize total mean work in the ISQ-2 system in \cref{prop:isq2_work}.
We prove in \Cref{thm:main} that mean work in the ISQ-2 lower bounds mean work in the $M/G/2$ system under an arbitrary scheduling policy.
Here, we numerically and empirically verify this result for several scheduling policies.

\Cref{fig:total_work} plots the ratios between the empirical mean total work of an $M/G/2$ system under several scheduling policies,
as compared to the exact mean work in the matching ISQ-2 system.
We numerically and empirically evaluate these policies under an exponential job size distribution.
For FCFS-2 scheduling, the exact mean total work of an $M/G/2$ with exponential job size distribution is known. For SRPT-2 and Longest Remaining Processing Time (LRPT-2) scheduling, we evaluate the policies empirically.

\Cref{fig:total_work} shows that the total mean work of the ISQ-2 system serves as a lower bound for the total work of the $M/G/2$ system under the FCFS, SRPT, and LRPT policies, confirming our result.
In our exploration, it appears that the LRPT policy minimizes the total work of an $M/G/2$ system, across all scheduling policies, making this lower bound especially important.

We also compare the ISQ-2 lower bound to the prior resource-pooled $M/G/1$ lower bound, note that in an $M/G/1$, scheduling policy does not affect mean work. We see that the total work of ISQ-2 is significantly larger than the total work of the resource-pooled $M/G/1$ system, particularly for low to medium load, indicating that it is a much stronger bound.

\begin{figure}[htbp!]
    \centering
    \includegraphics[width=0.85\textwidth]{Figures/total_work_ratios.png}
    \caption{Ratios of the expected total work of $M/G/2$ under different scheduling policies compared to ISQ-2 and resource-pooled $M/G/1$.}
    \label{fig:total_work}
\end{figure}

\subsection{Response time bounds}
\label{sec:empirical response time}

We now illustrate the quality of the lower bounds on mean response time achievable by adding our lower bounds on mean response time under arbitrary scheduling policies, namely \Cref{thm:sep_k} and \ref{thm:ar_k} into our WINE framework (see \Cref{sec:wine}). In \Cref{fig:ratios_wine}, we plot the ratios of the WINE-2, WINE-3, and WINE-4 bounds defined in \Cref{sec:wine}   compared to the previous naive lower bounds, namely the resource-pooled SRPT-1 and the mean service time bounds. We consider two servers and an exponential job size distribution.

We observe that the WINE-2 bound, which combines naive lower bounds with our WINE framework significantly improves upon the existing naive response time lower bounds, achieving over a 20\% improvement for certain loads. By incorporating the ISQ-$k$ lower bounds, the WINE-3 and WINE-4 bounds offer an even larger improvement. They begin to outperform the baseline bound at much smaller loads, around $\rho = 0.3$, and reach improvements of over 25\%.


\begin{figure}[htbp!]
    \centering
    \includegraphics[width=0.85\textwidth]{Figures/ratios_wine.png}
    \caption{Ratios of WINE bounds over naive lower bounds for $M/G/2$ under arbitrary scheduling. ``Baseline'' is the maximum of $E[T^{SRPT\text{-}1}]$ and $2E[S]$.}
    \label{fig:ratios_wine}
\end{figure}

To more precisely measure the degree of improvement of our ISQ-based lower bounds beyond what we already achieved with WINE-2, in \Cref{fig:ratios} we plot the improvements of WINE-3 and WINE-4 over WINE-2 for two different job size distributions with differing variability. In \Cref{fig:erlang3}, we have a low-variability distribution, an Erlang-3 job size distribution with a squared coefficient of variation $C^2=\mathrm{Var}(S)/\E[S]^2=1/3$. In \Cref{fig:hyper5}, we have a high-variability distribution, a 2-branch hyperexponential job size distribution, with $C^2 = 5$.

\begin{figure}[h]
    \centering
    \begin{subfigure}[b]{0.49\textwidth}
        \centering
        \includegraphics[width=\textwidth]{Figures/ratios_erlang3.png}
        \caption{Erlang-3, $C^2=1/3$.}
        \label{fig:erlang3}
    \end{subfigure}
    \hfill
    \begin{subfigure}[b]{0.49\textwidth}
        \centering
        \includegraphics[width=\textwidth]{Figures/ratios_hyper5.png}
        \caption{Hyperexponential, $C^2=5$.}
        \label{fig:hyper5}
    \end{subfigure}
    \caption{Ratios of improvement of the WINE 3 and WINE 4 lower bounds over the baseline WINE-2 lower bound for various job size distributions and $k=2$ servers.}
    \label{fig:ratios}
\end{figure}
Two important observations become apparent: First, for the lower variability distribution in \Cref{fig:erlang3}, the greater the improvement of the ISQ-$k$ based lower bounds over WINE-2, reaching over $8\%$ for the Erlang-3 distribution. One intuitive explanation for this phenomenon is that the ISQ-$k$ system takes an optimistic view of work completion, focusing on the case where all jobs finish at the same time, so no servers are wasted prematurely. This optimistic view is most accurate when jobs have similar sizes, as in a low variability setting. Secondly, for the high variability job size distributions \Cref{fig:hyper5}, the AR-ISQ-2 based WINE-4 bound which incorporates recycling information offers a much greater improvement over WINE-2 than WINE-3 alone, justifying the need for \Cref{thm:ar_2}. This makes sense, as the larger, recycled jobs make up more of the load at intermediate thresholds under job size distributions with high variability.

\section{Conclusion}
\label{sec:conclusion}
We present the first nontrivial lower bounds on the mean response time of the $M/G/k$ system under arbitrary scheduling policies. We introduce the novel ISQ-$k$ queue which bridges the multiserver and single-server queues and helps us lower bound the mean relevant work of an $M/G/k$ system under arbitrary scheduling policies.

We introduce the novel DiffeDrift extension to the drift method to characterize the work of the increasing speed queue, introducing a new technique where we derive test functions via solutions of differential equations. Empirically, our lower bounds improve upon previous naive lower bounds for a wide range of loads, with the most significant improvement observed under moderate load conditions.

One direction for future work lies in proving \Cref{conj:x2} on jumps due to recycling for $k > 2$ servers. More ambitiously, one could try to derive similar lower bounds on the mean response time for the $G/G/k$ queue. While this paper relies on the Poisson arrival process, the underlying ISQ-$k$ system and the DiffeDrift method have the potential to be more broadly applicable. A final potential direction is to prove even tighter bounds by first proving that the LRPT scheduling policy minimizes remaining total work in the $M/G/k$.


\bibliographystyle{plainnat}

\bibliography{reference.bib}

\appendix

\section{Proofs for Section 5}
\label{apdx:sec5}
\propdrift*

\begin{proof}
We have  $\int_\mathcal{S}g(x)\pi(dx)=\int_\mathcal{S}\int_\mathcal{S}g(y)P(t,x,dy)\,\pi(dx).$ This equality holds because $\pi$ is a stationary distribution, so $\int_\mathcal{S} P(t,x,U)\, \pi(dx) = \pi(U)$ for any measurable set $U$. Subtracting the LHS, dividing both sides by $t$, and introducing a limit,
\begin{align*}
    \nonumber
    0&=\dfrac{1}{t}\int_\mathcal{S}\left( \int_\mathcal{S} g(y)P(t,x,dy)-g(x)\right)\pi(dx)=\lim_{t\to0}\int_\mathcal{S} \frac{1}{t} \left(\int_\mathcal{S} g(y)P(t,x,dy)-g(x)\right)\pi(dx).
\end{align*}
Therefore,
\begin{align}
    \label{eq:general-bar}
    0&=\lim_{t\to0}\int_\mathcal{S}  A(t,x)\,\pi(dx)+\lim_{t\to0}\int_\mathcal{S}B(t,x)\,\pi(dx)
\end{align}
For the first term on the RHS, we invoke the dominated convergence theorem (DCT) in order to take the limit under the integral. The assumption of the DCT is satisfied by the assumption that $\lim_{t \to 0} A(t,\cdot)\to G\circ g(\cdot)$ uniformly, so there exists some $t$ small enough such that the integrand can be bounded by, e.g., $2|G\circ g(x)|$ for all $x\in \mathcal{S}$, which is $\pi$-integrable by assumption.

The second term also vanishes by assumption. Therefore, \eqref{eq:general-bar} is equivalent to the following:
\begin{align*}
    0=\int_\mathcal{S}\lim_{t\to0} A(t,x)\pi(dx)+0=\int_\mathcal{S} G\circ g(x)\pi(dx),
\end{align*}
giving us the desired result. 
\end{proof}

\lemwsquare*
\begin{proof}
    Imagine an $M/G/1$ queue such that after each busy period, the servers stops working until the system accumulates $k$ jobs, then restarts and starts processing jobs at a rate of 1. We call this system an $M/G/1$ queue with server activation threshold.
    Then the expected work of an ISQ-$k$ queue will be bounded above by this $M/G/1$ queue with a server activation threshold. The expected work in this $M/G/1$ queue with a server activation threshold is equivalent to the expected waiting time of an $M/G/1$-FCFS queue with rest periods drawn from a distribution $T_0$, where $T_0$ is the sum of $k$ exponential random variables, and $T_0$ is coupled to the arrival process.

    $M/G/1$ queues with rest periods are studied in \cite{Scholl1983O}. Equation 15 of \cite{Scholl1983O} characterizes the expected second moment of waiting time of an $M/G/1$ queue with rest periods. In particular, this expectation is finite if $E[S^3]<\infty$ and $\E[T_0^3]<\infty$. Note that the sum of $k$ exponential distributions has finite moments of all orders, so the lemma holds by our assumption that $\E[S^3]<\infty$. Note that \cite{Scholl1983O} assumes that $T_0$ is independent of the arrival process, which is not the case in our setting. However the proof of Equation 15 proceeds by first characterizing the number of arrivals during the rest period, before proceeding with the rest of the proof. In our system, that number of arrivals always $k$, simplifying Equation 7 of \cite{Scholl1983O}, and remainder of the proof holds unchanged.
 \end{proof}

\lemuniform*
\begin{proof}
To illustrate the high-level idea, we first consider the case where the speed $i$ is 1. By the definition of the Poisson arrival process, for $t < w$,
\begin{align}
    \nonumber
    &\dfrac{1}{t}\E[g(W(t),1)-g(w,1)\,|\,W(0)=w, I(0)=1]\\
    \nonumber
    &=\mathbb{P}(N_t=0)g(w-t,1)+\mathbb{P}(N_t=1)\E[g(w-t+S,1)]+\mathbb{P}(N_t>1)\E[g(w-t+N_tS,1)\mid N_t>1]\\
    \nonumber
    &=\dfrac{1}{t}(1-\lambda t)g(w-t,1)+\frac{o(t)}{t}g(w-t,1) +\dfrac{1}{t}\lambda t \E[g(w-t+S,1)]+\frac{o(t)}{t}\E[g(w-t+S,1)]\\
    \nonumber
    &\quad + \frac{o(t)}{t}\E[g(w-t+N_t\cdot S,1)\mid N_t>1]\\
    \label{eq:uniform-poisson}
    &=\underbrace{-\frac{g(w,1)}{t}+\frac{g(w-t,1)}{t}+\lambda\E[g(S+w-t,1)]-\lambda g(w-t,1)}_{A(t,w,1)}+B(t,w,1)
\end{align}
Before we continue, note that to ensure uniform convergence, we must also consider the case that $t > w$, for $w$ close to 0. In this case, the only significant change is the at the $N_t = 0$ term results in a state after time $t$ of $(0, 0)$, rather than $(w-t, 1)$. By assumption, $\lim_{w \to 0^+} g(w, 1) = g(0, 0)$, so the proof proceeds unchanged.


Returning to \eqref{eq:uniform-poisson}, note that the $A(t,w,1)$ term contains the terms corresponding to the scenario where there is only one arrival during the period of length $t$ and the scenario where there is no arrival during the same period. $B(t,w,1)$ corresponds to the remaining negligible terms:
\begin{align*}
    B(t,w,1)=\frac{o(t)}{t}g(w-t,1)+\frac{o(t)}{t}\E[g(w-t+S,1)]+\frac{o(t)}{t}\E[g(w-t+N_t\cdot S,1)\mid N_t>1].
\end{align*}
By the independence of $S$ and $N_t$, it is easy to check that $\E[g(w-t+N_t\cdot S,1)\,|\, N_t>1]<\infty$ given the assumption that $g(w,i)=w^2+c(w,i)$ where $|c(w,i)|\leq C_1w+C_2$. Therefore, we have, $\lim_{t\to0} \E[B(t,W,1)]=0.$ We now want to show that $A(t,\cdot,1)$ converges uniformly to some limit as $t \to 0$, namely, $G\circ g$. We first consider the term $-g(w,1)/t+g(w-t,1)/t$. Since $g$ is twice-differentiable for all $w\in\mathbb{R}_{+}$, by Taylor expansion,
\begin{align}
\label{eqn:uniform_1}
    g(w,1)-g(w-t,1)=g'(w,1)t-R_1(w,t)=g'(w,1)t-g''(c,1)\frac{t^2}{2}
\end{align}
Recall that by our assumption, $c(w,i)$ has a bounded second derivative and $g(w,i)$ also has a bounded second derivative. Therefore, $\dfrac{1}{t}(-g(w,1)+g(w-t,1))=-g'(w,1)+\frac{g''(c,1)}{2}t\to -g'(w,1)$ uniformly for all $w$. Next, we want to show that $\lambda \E[g(S+w-t)]-\lambda g(w-t)\to \lambda\E[g(S+w)]-\lambda g(w)$ uniformly. Let $\hat{g}(w-t,1)$ be defined as this difference, namely $\E[g(S+w-t,1)]-g(w-t,1)$. Then again since $\hat{g}$ is smooth, we have by Taylor's theorem
\begin{align*}
\label{eqn:uniform_2}
    |\hat{g}(w-t,1)-\hat{g}(w,1)|=|-\hat{g}'(w,1)t+R_1(w,t)|\leq|\hat{g}'(w,1)|t+M_2\frac{t^2}{2},
\end{align*}
where $R_1(w, t)$ is the remainder term of the Taylor expansion, for the second-order and higher terms. Therefore the convergence is uniform if $|\hat{g}'(w,1)|$ is bounded. We have
\begin{align}
    \E[g(S+w-t,1)-g(w-t,1)]&=\E[S^2]-2\E[S]t+2\E[S]w+\hat{c}(w,t),
\end{align}
where $\hat{c}(w,t)$ is defined analogously to $\hat{g}$.
By our assumption that $|c'(w, i)| \le M_1$, the RHS has a bounded derivative, thus establishing the uniform convergence.

When the speed of the system is less than 1, the only necessary changes lie in \Cref{eqn:uniform_1,eqn:uniform_2}. \cref{eqn:uniform_1} becomes $g(w,i/k)-g(w-it/k,i/k)=g'(w,i/k)it/k-\bar{R}_1(w,t),$ for some remainder term $\bar{R}_1$. We now invoke our bounded second derivative assumption to bound $\bar{R}_1(w,t)$ in the same manner as $R_1(w, t)$ above.
Dividing by $t$ and taking the limit as $t \to 0$ yields the desired limit $g'(w,i/k)\frac{i}{k}$. Similarly, \Cref{eqn:uniform_2} can now be written as, 
\begin{align*}
    \E[g(S+w-t,(i+1/k))-g(w-t,i/k)]&=\E[S^2]-2\E[S]\bar{i}t/k+2\E[S]w+\bar{c}(w,t),
\end{align*}
using Taylor's theorem and the intermediate value theorem,
for some value $\bar{i}\in[i,i+1]$ and some function $\bar{c}(w,t)$ defined analogously to $\hat{c}$ with a bounded first derivative by our assumption. Therefore as $t\to 0$, $\E[g(S+w-t,(i+1/k))-g(w-t,i/k)]$ converges uniformly to some limit, establishing uniform convergence. 

\end{proof}

\section{Proofs for Section 9}
\label{apdx:sec9}
In this appendix, we provide proofs of lemmas needed in \Cref{sec:mgk} and \Cref{thm:ar_k}.
\begin{lemma}
\label{lem:isqk_constant_assumptions}
    $g_k(\cdot, \cdot)$ defined by \Cref{eqn:isqk_constant} satisfies the assumptions of \Cref{prop:drift} and \Cref{lem:uniform}. In particular, $\lim_{w^+\to0}g_k(w,i)=0$. Note also that $g_k(\cdot,\cdot) \geq 0$.
\end{lemma}
\begin{proof}
By \Cref{eqn:isqk_constant} we see that $u_q(0)=0$. Moreover, as we are integrating a non-negative function recursively starting from $u_k(\cdot) = 0$, each of $u_q(\cdot)$ must be non-negative, i.e., $u_q(\cdot)\geq 0$ for all $q\in\{1,...,k\}$.

Moreover, one can see that from the form of \Cref{eqn:isqk_constant} that each $u_q$ is a sum of linear and negative exponential terms in $w$. In particular, in the integrand, $\E[u_{q+1}(S+y)]$ preserves this structure during each recursive step, see, e.g., the ISQ-3 test functions \eqref{eq:isq3-constant} and \eqref{eq:isq3-affine} given in \Cref{apdx:isq3}. Therefore, $u_q$ must satisfy the assumptions of \Cref{prop:drift} and \Cref{lem:uniform}.

\end{proof}


Applying \Cref{lem:drift} to \Cref{eqn:isqk_constant} we have,

\begin{lemma}
\label{lem:isqk_constant}
     For any arbitrary job size distribution $S$ such that $E[S^3]$ is finite, the drift at speed 0 is $G\circ g_k(0,0)=\lambda \E[S+u_1(S)]$ and the drift at all other speeds $i \geq 1/k$ is given by $G\circ g_k(w,i)=\lambda\E[S]-1$ for all $i\geq 1/k$.
\end{lemma}
\begin{proof}
    Clearly, the drift at speed $1$ is given by $G\circ g_k(w,1)=\lambda\E[S]-1$ and the drift at speed $0$ is given by $G\circ g_k(0,0)=\lambda\E[S+u_1(S)]$. For all other speed $q/k$, the drift is given by,
    \begin{align*}
        G\circ g_k(w,q/k)&=\lambda\left((w+\E[S]+\E[u_{q+1}(w+S)])-(w+u_q(w)) \right)-(w+u_q(w))'\cdot \dfrac{q}{k}\\
        &=\lambda\E[S]-1+1-\frac{q}{k}-\lambda u_q(w)+\lambda\E[u_{q+1}(w+S)]-\dfrac{qu_q'(w)}{k}.
    \end{align*}
    We want to show that the above is equal to $\lambda\E[S]-1$. Therefore, we want prove that the set of $u_q(w)$ functions collectively solve the following first order linear ordinary differential equations:
    \begin{align*}
        1-\frac{q}{k}-\lambda u_q(w)+\lambda\E[u_{q+1}(w+S)]-\dfrac{q\cdot u_q'(w)}{k}=0,
    \end{align*}
    with initial condition $u_q(0)=0$.
    Solving this differential equation with $u_k(w)=0$ yields the formula for $u_{k-1}(w)$ in \cref{def:isqk_contant}, and solving for decreasing $q$ yields all of the functions $u_q(w)$.

\end{proof}

Note that the differential equations used to define the constant-drift and affine-drift test functions $g_k$ and $h_k$ are generalizations of the differential equations used to define the constant and affine-drift test functions $g_2$ and $h_2$ for the 2-server case in \Cref{sec:derive_affine}.

Via the same argument \Cref{lem:isqk_constant_assumptions}, we prove the following lemma, which allows us to apply \Cref{lem:drift}.

\begin{lemma} 
\label{lem:isqk_affined_bound}
    $h_k(\cdot,\cdot)$ defined by \Cref{eqn:isqk_affine} satisfies the assumptions of \Cref{prop:drift} and \Cref{lem:uniform}. In particular, $\lim_{w \to 0^+} h_k(w,i)=0$. Note also that $h_k(\cdot,\cdot)\geq0$.
\end{lemma}

Now, we examine the drift of the test function $h_k$.

\begin{lemma}
\label{lem:isqk_affine}
    For any arbitrary job size distribution $S$ such that $E[S^3]$ is finite, the drift at speed $0$ is given by $G \circ h_k(0, 0) = \lambda\E[S^2+v_1(S)]$ and the drift at all other speed $\geq 1/k$ is given by $G\circ h_k(w,i)=\lambda\E[S^2]+2w(\lambda\E[S]-1)$.
\end{lemma}
\begin{proof}
     Clearly, the drift at speed $1$ is given by $G\circ h_k(w,1)=\lambda\E[S^2]+2w(\lambda\E[S]-1)$ and the drift at speed $0$ is given by $G\circ h_k(0,0)=\lambda\E[S^2+v_1(S)]$. For all other speed $q/k$, the drift is given by,
     \begin{align*}
         G\circ h_k(w,q/k)&=\lambda\left(((w+\E[S])^2+\E[v_{q+1}(w+S)] )-(w^2+v_{q}(w))\right)-\left(w^2+v_q(w) \right)'\cdot\frac{q}{k}\\
         &=\lambda\E[S^2]+2w(\lambda\E[S]-1)+2w-\frac{2qw}{k}-\lambda v_q(w)+\lambda \E[v_{q+1}(S+w)]-\frac{qv'_q(w)}{k}.
     \end{align*}
     We want to show that the above is equal to $\lambda\E[S^2]+2w(\lambda\E[S]-1)$. Therefore, we want prove that the set of $v_q(w)$ functions collectively solve the first order linear ordinary differential equations $2w-\frac{2qw}{k}-\lambda v_q(w)+\lambda \E[v_{q+1}(S+w)]-\frac{qv'_q(w)}{k}=0,$ with initial condition $v_q(0)=0$.
    Solving this differential equation with $v_k(w)=0$ yields the formula for $v_{k-1}(w)$ in \cref{def:isqk_affine}, and solving for decreasing $q$ yields all of the functions $v_q(w)$. 

\end{proof}

The next set of lemmas concerns the modified affine-drift test function defined in \Cref{def:isqk_modified}. By the same reasoning as \Cref{lem:isqk_constant_assumptions} we have the following lemma. 

\begin{lemma}
\label{lem:isqk_modified_bound}
    $h_{k,x}(\cdot, \cdot)$ defined by \Cref{eqn:isqk_modified} satisfies the assumptions of \Cref{prop:drift} and \Cref{lem:uniform}.
\end{lemma}

The above lemma allows us to apply \Cref{lem:drift} to \Cref{eqn:isqk_modified}.
\begin{lemma}
\label{lem:isqk_modified}
    For any arbitrary truncated job size distribution $S_x$ and arrival rate $\lambda_x$, the drift is given by
    \begin{align*}
        G \circ h_{k,x}(w,1) &=\lambda_x\E[S_x^2]+2w(\lambda_x\E[S_x]-1) \\    
        \forall 0 < i < 1,
        G\circ h_{k,x}(w,i) &=\lambda_x\E[S_x^2]+2w(\lambda_x\E[S_x]-1)+\frac{k-i}{k}C_k(x,\lambda_x) \\
        G\circ h_{k,x}(0,0)&=\lambda_x(\E[S_x^2]+\E[\ell_1(S_x)]).
    \end{align*}
\end{lemma}

\begin{proof}
    Clearly, the drift at speed 1 is given by $G\circ h_{k,x}(w,1)=\lambda_x\E[S_x^2]+2w(\lambda_x\E[S_x]-1)$ and the drift at speed 0 is given by $G\circ h_{k,x}(0,0)=\lambda_x\E[S_x^2+\ell_1(S_x)]$. For all other speeds $i=q/k$, the drift is,
    \begin{align*}
        G\circ h_{k,x}(w,q/k)&=\lambda_x\left(((w+\E[S_x])^2+\E[\ell_{q+1}(w+S_x)] )-(w^2+\ell_{q}(w))\right)-\left(w^2+\ell_q(w) \right)'\cdot\frac{q}{k}\\
         &=\lambda_x\E[S_x^2]+2w(\lambda_x\E[S_x]-1)+2w-\frac{2qw}{k}-\lambda_x \ell_q(w)+\lambda_x \E[\ell_{q+1}(S_x+w)]-\frac{q\ell'_q(w)}{k}.
    \end{align*}
    We want to show that the above is equal to $\lambda_x\E[S_x^2]+2w(\lambda_x\E[S_x]-1)+(k-q)C_k(x,\lambda_x)$.  Therefore, we want to prove that the set of $\ell_q(w)$ functions collectively solve the following differential equations:
    \begin{align*}
        2w-\frac{2qw}{k}-\lambda_x \ell_q(w)+\lambda_x \E[\ell_{q+1}(S_x+w)]-\frac{q\ell'_q(w)}{k}=(k-q)C_k(x,\lambda_x),
    \end{align*}
    with initial condition $\ell_q(0)=0$. Solving this differential equation with $\ell_k(w)=0$ yields the formula for $\ell_{k-1}(w)$ in \cref{def:isqk_modified}, and solving for decreasing $q$ yields all of the functions $l_q(w)$.

\end{proof}

In the above lemma, the drift at speed 0 is given by $\lambda_x(\E[S_x^2]+\E[\ell_1(S_x)])$. However, because the recursive nature of $\ell_q$ makes $\ell_1$ very hard to work with, we now prove a lemma which gives us an alternative expression for $\E[\ell_1(S_x)]$.
\begin{lemma}
\label{lem:isqk_C}
   $C_k(x,\lambda_x)$ defined by \Cref{eqn:C_k} solves the following equation $\lambda_x\E[\ell_1(S_x)]=kC_k(x,\lambda_x)$.
\end{lemma}

\begin{proof}
    By \Cref{lem:isqk_modified}, the drift of $h_{k,x}$ at speed $i>0$ is given by $G\circ h_{k,x}(w,i)=\lambda_x\E[S_x^2]+2w\left(\rho_x-1 \right)+C_k(x,\lambda_x)\frac{k-i}{k}$ and the drift at speed 0 is given by $G\circ h_{k,x}(0,0)=\lambda_x\E[S_x^2]+\lambda_x\E[\ell_1(S_x)]$. Therefore, taking expectation over $W_x$ and $I$, we have,
    \begin{align*}
        &\E[G\circ h_{k,x}(W_x,I)] =\lambda\E[S_x^2]+2\E[W_x]\left(\rho_x-1 \right)+k C_k(x)\sum_{i=1}^k\dfrac{k-i}{k}\mathbb{P}(I=i/k)+\lambda\E[\ell_1(S_x)]\mathbb{P}(I=0/k)\\
        &= \lambda\E[S_x^2]+2\E[W_x]\left(\rho_x-1 \right)+k C_k(x)\sum_{i=0}^k\dfrac{k-i}{k}\mathbb{P}(I=i/k)+(\lambda\E[\ell_1(S_x)]-kC_k(x,\lambda))\mathbb{P}(I=0/k).
    \end{align*}
    %#where we used \Cref{lem:isqk_load} in the third equality for simplifying the sum as $1-\rho_x$.
    To simplify the above above equation, we apply \Cref{lem:drift} to the test function $g(w,i)=w$ for all $(w,i)$. We have, $1-\rho_{x}=\sum_{i=0}^k\frac{k-i}{k}\mathbb{P}(I=i/k)$. Therefore, 
    \begin{align*}
        \E[G\circ h_{k,x}(W,I)]=\lambda\E[S_x^2]+2\E[W_x]\left(\rho_x-1 \right)+k C_k(x)(1-\rho_{x})+ (\lambda\E[\ell_1(S_x)]-kC_k(x,\lambda))\mathbb{P}(I=0/k)
    \end{align*}
    
    By \Cref{lem:isqk_modified_bound}, the ISQ-$k$ modified affine-drift test function defined in \Cref{def:isqk_modified} satisfies the assumption of \Cref{lem:drift}. We have $\E[G\circ h_{k}(W_x,I)]=0$. Solving for $\E[W_x]$ we get,
    \begin{align}
    \label{eqn:isqk_work_modified}
        \E[W_x]=\dfrac{\lambda_x\E[S_x^2]}{2(1-\lambda_x\E[S_x])}+\dfrac{k}{2}\cdot C_k(x,\lambda_x)+\dfrac{(\lambda_x\E[\ell_1(S_x)]-kC_k(x,\lambda_x))\mathbb{P}(I=0/k)}{2(1-\rho_x)}.
    \end{align}
    Recall that $C_k(x,\lambda_x)$ is defined to be $C_k(x,\lambda_x)=\frac{2}{k}\cdot\frac{\lambda_x\E[v_1(S_x)]}{2+2\lambda_x\E[u_1(S_x)]}$.
    Recall also that by \Cref{prop:isqk_work}, we have the following alternative expression for mean relevant work for the same arrival rate $\lambda_x$ and truncated job size distribution $S_x$:
    \begin{align}
         \E[W_x]=\dfrac{\lambda_x\E[S_x^2]}{2(1-\lambda_x\E[S_x])}+\dfrac{\lambda_x\E[v_1(S_x)]}{2+2\lambda_x\E[u_1(S_x)]}
         = \dfrac{\lambda_x\E[S_x^2]}{2(1-\lambda_x\E[S_x])}+\dfrac{k}{2}\cdot C_k(x,\lambda_x)
    \end{align}
    Thus, taking the difference of these two expressions for $\E[W_x]$, we find that
    the third term of \eqref{eqn:isqk_work_modified} is zero:
    \begin{align*}
        0 = \dfrac{(\lambda_x\E[\ell_1(S_x)]-kC_k(x,\lambda_x))\mathbb{P}(I=0/k)}{2(1-\rho_x)}
    \end{align*}
    
    Note that $\mathbb{P}(I=0/k)$ is positive, by \Cref{eqn:isqk_prob0}. Thus solving $0 = \lambda\E[\ell_1(S_x)]-kC_k(x,\lambda_x)$ we get
    $\lambda_x\E[\ell_1(S_x)] =kC_k(x,\lambda_x)$
    as desired.

\end{proof}

Therefore, we can write the drift at speed 0 as $G\circ h_{k,x}(0,0)=\lambda_x\E[S_x^2]+kC_k(x,\lambda_x)$. We are now ready to state the general version of the AR-ISQ-$k$ lower bound. This is the second lower bound of \Cref{thm:main_k}.


\mainarisqk*

\begin{proof}
    The proof of this theorem follows the same argument as the proof of \Cref{thm:ar_2}. Let $\E_r[\cdot]$ denote the Palm expectation taken over the moments when the arbitrary arrival of jobs of size $x$ occur.

    Let $J_x(w, i) := h_{k,x}(w+x,\min(i+1/k, 1)) - h_{k,x}(w,i)$
    denote the increase in the test function due to the arrival of a size-$x$ job.
    Then $\E_r[J_x(W_x,I)]$ denotes the mean size of the recycling jump in stationarity. Thus, by \Cref{lem:ar_drift},$\E[G \circ h_{k,x}(W_x, I)] = \E[\text{stochastic drift}] + (\lambda - \lambda_x) \E_r[J_x(W_x,I)]$.
    By \Cref{lem:isqk_C} we can write the expected stochastic drift as,
    \begin{align*}
    \E[\text{stochastic drift}]&=\lambda_x\E[S_x^2]+2\E[W_x](-1+\rho_x)+kC_k(x,\lambda)\sum_{i=0}^k\frac{k-i}{k}\mathbb{P}(I^r=i/k)\\
    &=\lambda_x\E[S_x^2]+2\E[W_x](-1+\rho_x)+kC_k(x,\lambda)(1-\rho_{\bar{x}}).
    \end{align*}
    In the second equality, we used the fact that $1-\rho_{\overbar{x}}=\sum_{i=0}^k\frac{k-i}{k}\mathbb{P}(I^r=i/k)$, a result that can be derived similarly to \Cref{eqn:isq2_load} by applying \Cref{lem:ar_drift} to the test function $g(w,i)=w$ for all $(w,i)$.
    
    Therefore, substituting the above into $\E[G\circ h_{k,x}(W_x,I)]=0$ and solving for $\E[W_x]$, we have
    \begin{align*}
        \E[W_{x}^{arb{\text -}ISQ}]&=\frac{\lambda_x E[S_x^2]}{2 (1 - \rho_x)} + \dfrac{kC_k(x,\lambda_x)(1-\rho_{\overbar{x}})}{2(1-\rho_x)}+\frac{(\lambda_x - \lambda_x)\E_r[J_x(W_x,I)]}{2(1-\rho_x)}\\
        &=\dfrac{\lambda_x\E[S_x^2]}{2(1-\lambda_x\E[S_x])}+\dfrac{\lambda_x\E[v_1(S_x)]}{2+2\lambda_x\E[u_1(S_x)]}\cdot\dfrac{1-\rho_{\bar{x}}}{1-\rho_x}+\frac{(\lambda - \lambda_x)\E_r[J_x(W_x,I)]}{2(1-\rho_x)},
    \end{align*}
    where we use the definition of $C_k(x,\lambda)$ in the second equality. Recall the definition of an increasing speed queue. The current speed $i$ bounds the number of jobs in the system, which in turn bounds the total work, which consists of at most $ki$ jobs, each with size at most $x$. Therefore, for $0<i<1$, the system can only visit states $(w, i)$ where $w \le kix$.
    \begin{align*}
        J_x(w,i)=h_{k,x}(w+x,i+1/k)-h_{k,x}(w,i/k)\geq \inf_{w\in[0,k i x]}h_{k,x}(w+x,i+1/k)-h_{k,x}(w,i/k),
    \end{align*}
    along with $J_x(0,0)=h_{k,x}(x,1/k)$ and $J_x(w+x,1)\geq x^2$, we have $\E_r[J_x(W_x,I)]\geq J_x$.

\end{proof}

\section{Specific Results for the ISQ-3 System}
\label{apdx:isq3}
For the ISQ-3 system, the constant-drift test function $g_3$ defined in \cref{def:isqk_contant} is
\begin{align}
    \label{eq:isq3-constant}
&g_3(0, 0)=0,\quad g_3(w,1)=w,\quad g_3(w,2/3)=w+\frac{1}{3\lambda}(1-e^{-\frac{3}{2}w\lambda}), \\ &g_3(w,1/3)=w+\frac{1}{\lambda}+\frac{(2\widetilde{S}(\frac{3}{2}\lambda)-3)e^{-3w\lambda}}{3\lambda}-\frac{2\widetilde{S}(\frac{3}{2}\lambda)e^{-\frac{3}{2}w\lambda}}{3\lambda}.
\nonumber
\end{align} 
The affine-drift test function $h_3$ defined in \cref{def:isqk_affine} is
\begin{align}
    \label{eq:isq3-affine}&
h_3(0,0)=0,\quad h_3(w,1)=w^2,\quad h_3(w,2/3)=w^2+\frac{1}{9\lambda^2}(6w\lambda+4e^{-\frac{3}{2}w\lambda}-4),\\
\nonumber
&h_3(w,1/3)=w^2+\dfrac{1}{9\lambda^2}\left(-10+10e^{-3w\lambda}-8\widetilde{S}(\tfrac{3}{2}\lambda)e^{-3w\lambda}+8\widetilde{S}(\tfrac{3}{2}\lambda)e^{-\frac{3}{2}w\lambda} \right)+\dfrac{1}{3\lambda}\left(2\E[S]-2e^{-3w\lambda}\E[S]+6w \right). \\
\nonumber
\end{align}






\end{document}







