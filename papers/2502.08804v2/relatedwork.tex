\section{Prior Work}
\label{sec:prior_work}

\subsection{Shortest-remaining processing time}
It is well established that the Shortest-Remaining Processing Time (SRPT-1) policy is optimal in minimizing the mean response time in $M/G/1$ queues  \citep{Schrage1968Proof}. Furthermore, \cite{Schrage1966} characterizes the mean response time under SRPT in the $M/G/1$ exactly. 

In a multiserver setting, the SRPT policy, denoted by SRPT-$k$, has been shown to be heavy-traffic optimal, a result proven in \cite{Grosof2018}. This work provides an upper bound on the mean response time under SRPT-$k$ that is tight at very high loads.

At the opposite extreme, when there are no arrivals, it is well known that the Shortest Job First (SJF) policy is the optimal nonpreemptive policy. Furthermore, \cite{McNaughton1959} proves that no preemptive policy can outperform the optimal nonpreemptive policy. Since SRPT reduces to SJF when there are no arrivals, SRPT is also optimal in that setting.

However, SRPT-$k$ is not always the optimal policy across all load conditions. Notably, the SEK policy, recently introduced in \cite{Grosof2024BoMS}, empirically outperforms the response time of SRPT-$k$ under certain conditions. This raises the question of how much further improvement beyond SRPT-$k$ is possible.

The focus of this paper is on establishing lower bounds for the mean response time of $M/G/k$ systems under arbitrary policies, which has been noted as an open problem \citep{Grosof2019Open}.

\subsection{Multiserver mean response time analysis}
The analysis of multiserver FCFS systems, particularly in the context of $M/G/k$-FCFS queues, is relatively well understood. Early work by \cite{Kingman1970} and \cite{Daley1998} provide bounds on the mean response time based on the first two moments of the job size distribution. \cite{Harchol-Balter2005} conducts an exact analysis of the mean response time in $M/Ph/k$-FCFS systems. Since any arbitrary distribution can be closely approximated by a phase-type distribution, their results offer a useful approximation for mean response time in $M/G/k$-FCFS systems. 

More recently, \cite{Gupta2011} demonstrates how utilizing the full sequence of moments of the job size distribution can yield 
bounds on mean response time. \cite{Li2024} extends Kingman’s bounds \citep{Kingman1962}, to the multiserver $GI/GI/k$ queue and provide the first simple and explicit bounds for mean waiting time that scale with $1/(1-\rho)$.  These results allow for bounds on the mean response time in $M/G/k$-FCFS queues across all load levels. In contrast, our work considers a more general setting by allowing for any arbitrary scheduling policies and focuses on establishing lower bounds.

\cite{Scully2020} proves that the Gittins-$k$ policy is nearly optimal in the $M/G/k$ under extremely general conditions. Notably, the Gittins-$k$ policy coincides with SRPT-$k$ when there is perfect information about job sizes, which is the setting we consider in this paper. A key result of their work is the introduction of a new formula for calculating the mean response time, known as the Work Integral Number Equality (WINE). WINE provides a method to translate the exact expected relevant work into the exact mean response time, and is applicable to any queueing system under any scheduling policy. Our approach builds on this formula, using WINE as the basis for developing a framework to establish lower bounds on the mean response time. 

\subsection{Basic adjoint relationship (BAR) and drift method}
\label{sub_sec:prior_bar}

The well known Basic Adjoint Relationship (BAR) equation states that the stationary distribution $\pi$ of a continuous-time Markov chain $(Z_t)_{t\geq 0}$ with instantaneous generator $G$ satisfies
\begin{align}
\label{eqn:steady_state}
    \E_{\pi}[G\circ g(Z)]=0,
\end{align}
under suitable conditions on the Markov chain $Z$ and the function $g$ \citep{Glynn2008}. By carefully designing the test function $g$, one can solve the BAR equation \eqref{eqn:steady_state} to obtain either exact expressions for the moments of stationary variables or (asymptotically tight) bounds on these moments, depending on the system. This method is commonly referred to as the drift method. See, e.g.,  \cite{Eryilmaz2012,Maguluri2016,Grosof2023TRaM,Hong2023}. We use the drift method to obtain the exact expected total work of the ISQ-$k$ system,
and exact results and bounds for further systems we introduce in \cref{sec:model}. We use these exact results and bounds to lower bound mean response time in the $M/G/k$ under an arbitrary scheduling policy.
\subsubsection{Choosing test functions}

Recent applications of the drift method use simple exponential test functions of the form $e^{tQ}$, where $t$ is a general constant and $Q$ is the queue length, or $e^{tW}$, where $W$ is the total work in the system. For example, \cite{Braverman2017} and \cite{Braverman2024} use these test functions to prove
% have shown that the BAR is a more natural framework than the limit interchange approach for justifying 
heavy-traffic and steady-state approximations in both single-class and multi-class queueing systems. The use of exponential test functions is also known as the transform method, see, e.g., \cite{Hurtado-Lange2020} and \cite{jhunjhunwala2023}, which likewise uses test functions of the form $e^{tQ}$ or $e^{tW}$. Other recent works have focused on test functions of the form $Q^2$ or $W^2$, \cite{Grosof2022MSJ}, which  likewise depend on a constant rate of work completion.

However, these methods cannot be directly applied to queues with a variable work completion rate, as the test functions $e^{tQ}$ and $e^{tW}$ only yield useful information when the work completion rate is constant. Some attempts have been made to circumvent this problem by focusing on heavy traffic settings. For instance, \cite{Hurtado-Lange2020} relies heavily on state-space collapse to ensure a constant work completion rate except when the system is near-empty. As the system is rarely empty in heavy traffic, this state-space collapse allows the same straightforward test functions to be used, and heavy traffic results to be obtained.

However, in our intermediate-load setting, the variable work completion rate is fundamental, and we cannot eliminate that complexity from the problem. Instead, we introduce the DiffeDrift approach (see \cref{sec:derive}), inventing a new class of test functions which can accommodate the ISQ-$k$ system's variable work completion rate.

%%%%%%

\subsubsection{Sufficient conditions for BAR} Proposition 3 of \cite{Glynn2008} offers a proof of a set of sufficient conditions for the BAR equation \eqref{eqn:steady_state} to hold. This proposition has been widely applied in many recent papers in queueing theory. Their result requires the state space to be discrete and certain regularity conditions to hold on the rate matrix of the Markov jump process and the test function $g$. For example, \cite{Grosof2023TRaM} and \cite{Hong2023} invoke Proposition 3 of \cite{Glynn2008} by showing that their respective Markov chains have uniformly bounded transition rates. However, this result does not directly apply to our system because our system has a continuous state space.

Theorem 2 of \cite{Glynn2008} provides another set of sufficient conditions for the BAR equation \eqref{eqn:steady_state} to hold. This result allows continuous state spaces but requires \emph{bounded-drift} test functions. However, it is not directly applicable to our unbounded quadratic test function, which has unbounded drift. 
For test functions with unbounded drift, a standard approach in the existing literature is to truncate these functions, thereby producing a sequence of bounded test functions that approximate the original. See, e.g., \cite{Braverman2017} and \cite{Guang2024}. 

We take an alternative approach, proving a new BAR result for time-homogeneous Markov processes with unbounded continuous state spaces and unbounded test functions that grow at most quadratically, see \Cref{sec:drift}. We find this approach much easier to apply than the standard truncation method given the complexity of our test functions arising from the DiffeDrift method.