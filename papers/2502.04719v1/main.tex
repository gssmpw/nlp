% CVPR 2024 Paper Template; see https://github.com/cvpr-org/author-kit

\documentclass[10pt,twocolumn,letterpaper,pagebackref,breaklinks,colorlinks,allcolors=cvprblue]{article}

%%%%%%%%% PAPER TYPE  - PLEASE UPDATE FOR FINAL VERSION
% \usepackage{cvpr}              % To produce the CAMERA-READY version
% \usepackage[review]{cvpr}      % To produce the REVIEW version
\usepackage[pagenumbers]{cvpr} % To force page numbers, e.g. for an arXiv version

% Import additional packages in the preamble file, before hyperref
%
% --- inline annotations
%
\newcommand{\red}[1]{{\color{red}#1}}
\newcommand{\todo}[1]{{\color{red}#1}}
\newcommand{\TODO}[1]{\textbf{\color{red}[TODO: #1]}}
% --- disable by uncommenting  
% \renewcommand{\TODO}[1]{}
% \renewcommand{\todo}[1]{#1}



\newcommand{\VLM}{LVLM\xspace} 
\newcommand{\ours}{PeKit\xspace}
\newcommand{\yollava}{Yo’LLaVA\xspace}

\newcommand{\thisismy}{This-Is-My-Img\xspace}
\newcommand{\myparagraph}[1]{\noindent\textbf{#1}}
\newcommand{\vdoro}[1]{{\color[rgb]{0.4, 0.18, 0.78} {[V] #1}}}
% --- disable by uncommenting  
% \renewcommand{\TODO}[1]{}
% \renewcommand{\todo}[1]{#1}
\usepackage{slashbox}
% Vectors
\newcommand{\bB}{\mathcal{B}}
\newcommand{\bw}{\mathbf{w}}
\newcommand{\bs}{\mathbf{s}}
\newcommand{\bo}{\mathbf{o}}
\newcommand{\bn}{\mathbf{n}}
\newcommand{\bc}{\mathbf{c}}
\newcommand{\bp}{\mathbf{p}}
\newcommand{\bS}{\mathbf{S}}
\newcommand{\bk}{\mathbf{k}}
\newcommand{\bmu}{\boldsymbol{\mu}}
\newcommand{\bx}{\mathbf{x}}
\newcommand{\bg}{\mathbf{g}}
\newcommand{\be}{\mathbf{e}}
\newcommand{\bX}{\mathbf{X}}
\newcommand{\by}{\mathbf{y}}
\newcommand{\bv}{\mathbf{v}}
\newcommand{\bz}{\mathbf{z}}
\newcommand{\bq}{\mathbf{q}}
\newcommand{\bff}{\mathbf{f}}
\newcommand{\bu}{\mathbf{u}}
\newcommand{\bh}{\mathbf{h}}
\newcommand{\bb}{\mathbf{b}}

\newcommand{\rone}{\textcolor{green}{R1}}
\newcommand{\rtwo}{\textcolor{orange}{R2}}
\newcommand{\rthree}{\textcolor{red}{R3}}
\usepackage{amsmath}
%\usepackage{arydshln}
\DeclareMathOperator{\similarity}{sim}
\DeclareMathOperator{\AvgPool}{AvgPool}

\newcommand{\argmax}{\mathop{\mathrm{argmax}}}     



% It is strongly recommended to use hyperref, especially for the review version.
% hyperref with option pagebackref eases the reviewers' job.
% Please disable hyperref *only* if you encounter grave issues, 
% e.g. with the file validation for the camera-ready version.
%
% If you comment hyperref and then uncomment it, you should delete *.aux before re-running LaTeX.
% (Or just hit 'q' on the first LaTeX run, let it finish, and you should be clear).
\definecolor{cvprblue}{rgb}{0.21,0.49,0.74}
%\usepackage[pagebackref,breaklinks,colorlinks,citecolor=cvprblue]{hyperref}
% \usepackage[pagebackref,breaklinks,colorlinks,allcolors=cvprblue]{hyperref}
\usepackage{hyperref}
\hypersetup{colorlinks,allcolors=cvprblue}

\usepackage{multicol}

% Symbols
\usepackage{amsmath}
\usepackage{amssymb}
\usepackage{gensymb}
\usepackage{nicefrac}
\usepackage{fontawesome}
\usepackage{pifont}
\usepackage{textcomp}
\usepackage{verbatim}
\newcommand{\cmark}{\ding{51}}%
\newcommand{\xmark}{\ding{55}}%

% Tables
\usepackage{booktabs}
\usepackage{colortbl}
\usepackage{array}
\usepackage{ragged2e}
\usepackage{tabularx}
\usepackage{makecell}
\usepackage{multirow}
\usepackage{xspace}
\renewcommand\tabularxcolumn[1]{>{\Centering}m{#1}}
\renewcommand{\thefootnote}{\textasteriskcentered}

\newcommand{\tianfan}[1]{{\color{red}Tianfan: #1}}
\newcommand{\XINGE}[1]{{\color{orange}XINGE: #1}}
\newcommand{\Liqun}[1]{{\color{blue}Liqun: #1}}
\newcommand{\daijun}[1]{{\color{Green}Daijun: #1}}
\newcommand{\jinwei}[1]{{\color{Purple}Jinwei: #1}}

\newcommand{\greencheck}{{\color{Green}\cmark}\xspace}
\newcommand{\green}{\cellcolor{Green!12.5}\greencheck}
\newcommand{\yellowcheck}{{\color{YellowOrange}(\cmark)}\xspace}
\newcommand{\yellow}{\cellcolor{YellowOrange!12.5}\yellowcheck}
\newcommand{\redcheck}{{\color{red}\xmark}\xspace}
\newcommand{\red}{\cellcolor{red!12.5}\redcheck}

% Spacings
\usepackage{enumitem}
\setlist{nosep}
\renewcommand{\paragraph}[1]{\par\vspace{2pt plus 1pt minus 1pt}\noindent{\bfseries #1\enspace}}
\setlength{\textfloatsep}{7pt plus 2pt minus 4pt}
\setlength{\dbltextfloatsep}{7pt plus 2pt minus 4pt}
\setlength{\abovecaptionskip}{6pt}

%%%%%%%%% PAPER ID  - PLEASE UPDATE
\def\paperID{4777} % *** Enter the Paper ID here
\def\confName{CVPR}
\def\confYear{2025}

%%%%%%%%% TITLE - PLEASE UPDATE
\title{Tolerance-Aware Deep Optics}

%%%%%%%%% AUTHORS - PLEASE UPDATE
% CVPR template comments
% For a paper whose authors are all at the same institution,
% omit the following lines up until the closing ``}''.
% Additional authors and addresses can be added with ``\and'',
% just like the second author.
% To save space, use either the email address or home page, not both
\begin{comment}
\author{Jun Dai\\
Shanghai AI Laboratory\\
{\tt\small daijun@pjlab.org.cn}
\and
Liqun Chen\\
Shanghai AI Laboratory\\
{\tt\small chenliqun@pjlab.org.cn}
\and
Xinge Yang\\
King Abdullah University of Science and Technology\\
{\tt\small xinge.yang@kaust.edu.sa}
\and
Yuyao Hu\\
Shanghai Institute of Optics and Fine Mechanics\\
{\tt\small sthuyy@163.com}
\and
Jinwei Gu\\
NVIDIA\\
{\tt\small jinweig@nvidia.com}
\and
Tianfan Xue\\
CUHK\\
{\tt\small tfxue@ie.cuhk.edu.hk}
}
\end{comment}
% Below, arXiv version
\author{
    Jun Dai\textsuperscript{1} \hfill \quad
    Liqun Chen\textsuperscript{1}\footnote{test} \hfill \quad
    Xinge Yang\textsuperscript{2} \hfill \quad
    Yuyao Hu\textsuperscript{1} \hfill \quad
    Jinwei Gu\textsuperscript{3} \hfill \quad
    Tianfan Xue\textsuperscript{4,1}
    \\[0.4em]
    \textsuperscript{1}Shanghai AI Laboratory \quad
    \textsuperscript{2}KAUST \\
    \textsuperscript{3}NVIDIA \quad
    \textsuperscript{4}The Chinese University of Hong Kong
}
\begin{document}
\twocolumn[{%
\renewcommand\twocolumn[1][]{#1}%
\maketitle
\begin{center}
    \centering
    \captionsetup{type=figure}
    \includegraphics[width=1.0\textwidth,height=7.8cm]{figures/Teaser.pdf}
    \captionof{figure}{When subjected to random tolerances perturbations, the performance of conventional deep optics degraded severely, whereas deep optics with our tolerance-aware optimization maintained excellent computational imaging performance. \emph{Camera Captured} and \emph{Reconstructed}, representing the simulated camera imaging result and the reconstructed result after decoder, respectively.}
\end{center}%
}]
\begin{abstract}


The choice of representation for geographic location significantly impacts the accuracy of models for a broad range of geospatial tasks, including fine-grained species classification, population density estimation, and biome classification. Recent works like SatCLIP and GeoCLIP learn such representations by contrastively aligning geolocation with co-located images. While these methods work exceptionally well, in this paper, we posit that the current training strategies fail to fully capture the important visual features. We provide an information theoretic perspective on why the resulting embeddings from these methods discard crucial visual information that is important for many downstream tasks. To solve this problem, we propose a novel retrieval-augmented strategy called RANGE. We build our method on the intuition that the visual features of a location can be estimated by combining the visual features from multiple similar-looking locations. We evaluate our method across a wide variety of tasks. Our results show that RANGE outperforms the existing state-of-the-art models with significant margins in most tasks. We show gains of up to 13.1\% on classification tasks and 0.145 $R^2$ on regression tasks. All our code and models will be made available at: \href{https://github.com/mvrl/RANGE}{https://github.com/mvrl/RANGE}.

\end{abstract}

    
\section{Introduction}

Video generation has garnered significant attention owing to its transformative potential across a wide range of applications, such media content creation~\citep{polyak2024movie}, advertising~\citep{zhang2024virbo,bacher2021advert}, video games~\citep{yang2024playable,valevski2024diffusion, oasis2024}, and world model simulators~\citep{ha2018world, videoworldsimulators2024, agarwal2025cosmos}. Benefiting from advanced generative algorithms~\citep{goodfellow2014generative, ho2020denoising, liu2023flow, lipman2023flow}, scalable model architectures~\citep{vaswani2017attention, peebles2023scalable}, vast amounts of internet-sourced data~\citep{chen2024panda, nan2024openvid, ju2024miradata}, and ongoing expansion of computing capabilities~\citep{nvidia2022h100, nvidia2023dgxgh200, nvidia2024h200nvl}, remarkable advancements have been achieved in the field of video generation~\citep{ho2022video, ho2022imagen, singer2023makeavideo, blattmann2023align, videoworldsimulators2024, kuaishou2024klingai, yang2024cogvideox, jin2024pyramidal, polyak2024movie, kong2024hunyuanvideo, ji2024prompt}.


In this work, we present \textbf{\ours}, a family of rectified flow~\citep{lipman2023flow, liu2023flow} transformer models designed for joint image and video generation, establishing a pathway toward industry-grade performance. This report centers on four key components: data curation, model architecture design, flow formulation, and training infrastructure optimization—each rigorously refined to meet the demands of high-quality, large-scale video generation.


\begin{figure}[ht]
    \centering
    \begin{subfigure}[b]{0.82\linewidth}
        \centering
        \includegraphics[width=\linewidth]{figures/t2i_1024.pdf}
        \caption{Text-to-Image Samples}\label{fig:main-demo-t2i}
    \end{subfigure}
    \vfill
    \begin{subfigure}[b]{0.82\linewidth}
        \centering
        \includegraphics[width=\linewidth]{figures/t2v_samples.pdf}
        \caption{Text-to-Video Samples}\label{fig:main-demo-t2v}
    \end{subfigure}
\caption{\textbf{Generated samples from \ours.} Key components are highlighted in \textcolor{red}{\textbf{RED}}.}\label{fig:main-demo}
\end{figure}


First, we present a comprehensive data processing pipeline designed to construct large-scale, high-quality image and video-text datasets. The pipeline integrates multiple advanced techniques, including video and image filtering based on aesthetic scores, OCR-driven content analysis, and subjective evaluations, to ensure exceptional visual and contextual quality. Furthermore, we employ multimodal large language models~(MLLMs)~\citep{yuan2025tarsier2} to generate dense and contextually aligned captions, which are subsequently refined using an additional large language model~(LLM)~\citep{yang2024qwen2} to enhance their accuracy, fluency, and descriptive richness. As a result, we have curated a robust training dataset comprising approximately 36M video-text pairs and 160M image-text pairs, which are proven sufficient for training industry-level generative models.

Secondly, we take a pioneering step by applying rectified flow formulation~\citep{lipman2023flow} for joint image and video generation, implemented through the \ours model family, which comprises Transformer architectures with 2B and 8B parameters. At its core, the \ours framework employs a 3D joint image-video variational autoencoder (VAE) to compress image and video inputs into a shared latent space, facilitating unified representation. This shared latent space is coupled with a full-attention~\citep{vaswani2017attention} mechanism, enabling seamless joint training of image and video. This architecture delivers high-quality, coherent outputs across both images and videos, establishing a unified framework for visual generation tasks.


Furthermore, to support the training of \ours at scale, we have developed a robust infrastructure tailored for large-scale model training. Our approach incorporates advanced parallelism strategies~\citep{jacobs2023deepspeed, pytorch_fsdp} to manage memory efficiently during long-context training. Additionally, we employ ByteCheckpoint~\citep{wan2024bytecheckpoint} for high-performance checkpointing and integrate fault-tolerant mechanisms from MegaScale~\citep{jiang2024megascale} to ensure stability and scalability across large GPU clusters. These optimizations enable \ours to handle the computational and data challenges of generative modeling with exceptional efficiency and reliability.


We evaluate \ours on both text-to-image and text-to-video benchmarks to highlight its competitive advantages. For text-to-image generation, \ours-T2I demonstrates strong performance across multiple benchmarks, including T2I-CompBench~\citep{huang2023t2i-compbench}, GenEval~\citep{ghosh2024geneval}, and DPG-Bench~\citep{hu2024ella_dbgbench}, excelling in both visual quality and text-image alignment. In text-to-video benchmarks, \ours-T2V achieves state-of-the-art performance on the UCF-101~\citep{ucf101} zero-shot generation task. Additionally, \ours-T2V attains an impressive score of \textbf{84.85} on VBench~\citep{huang2024vbench}, securing the top position on the leaderboard (as of 2025-01-25) and surpassing several leading commercial text-to-video models. Qualitative results, illustrated in \Cref{fig:main-demo}, further demonstrate the superior quality of the generated media samples. These findings underscore \ours's effectiveness in multi-modal generation and its potential as a high-performing solution for both research and commercial applications.
\section{Related Work}

\subsection{Large 3D Reconstruction Models}
Recently, generalized feed-forward models for 3D reconstruction from sparse input views have garnered considerable attention due to their applicability in heavily under-constrained scenarios. The Large Reconstruction Model (LRM)~\cite{hong2023lrm} uses a transformer-based encoder-decoder pipeline to infer a NeRF reconstruction from just a single image. Newer iterations have shifted the focus towards generating 3D Gaussian representations from four input images~\cite{tang2025lgm, xu2024grm, zhang2025gslrm, charatan2024pixelsplat, chen2025mvsplat, liu2025mvsgaussian}, showing remarkable novel view synthesis results. The paradigm of transformer-based sparse 3D reconstruction has also successfully been applied to lifting monocular videos to 4D~\cite{ren2024l4gm}. \\
Yet, none of the existing works in the domain have studied the use-case of inferring \textit{animatable} 3D representations from sparse input images, which is the focus of our work. To this end, we build on top of the Large Gaussian Reconstruction Model (GRM)~\cite{xu2024grm}.

\subsection{3D-aware Portrait Animation}
A different line of work focuses on animating portraits in a 3D-aware manner.
MegaPortraits~\cite{drobyshev2022megaportraits} builds a 3D Volume given a source and driving image, and renders the animated source actor via orthographic projection with subsequent 2D neural rendering.
3D morphable models (3DMMs)~\cite{blanz19993dmm} are extensively used to obtain more interpretable control over the portrait animation. For example, StyleRig~\cite{tewari2020stylerig} demonstrates how a 3DMM can be used to control the data generated from a pre-trained StyleGAN~\cite{karras2019stylegan} network. ROME~\cite{khakhulin2022rome} predicts vertex offsets and texture of a FLAME~\cite{li2017flame} mesh from the input image.
A TriPlane representation is inferred and animated via FLAME~\cite{li2017flame} in multiple methods like Portrait4D~\cite{deng2024portrait4d}, Portrait4D-v2~\cite{deng2024portrait4dv2}, and GPAvatar~\cite{chu2024gpavatar}.
Others, such as VOODOO 3D~\cite{tran2024voodoo3d} and VOODOO XP~\cite{tran2024voodooxp}, learn their own expression encoder to drive the source person in a more detailed manner. \\
All of the aforementioned methods require nothing more than a single image of a person to animate it. This allows them to train on large monocular video datasets to infer a very generic motion prior that even translates to paintings or cartoon characters. However, due to their task formulation, these methods mostly focus on image synthesis from a frontal camera, often trading 3D consistency for better image quality by using 2D screen-space neural renderers. In contrast, our work aims to produce a truthful and complete 3D avatar representation from the input images that can be viewed from any angle.  

\subsection{Photo-realistic 3D Face Models}
The increasing availability of large-scale multi-view face datasets~\cite{kirschstein2023nersemble, ava256, pan2024renderme360, yang2020facescape} has enabled building photo-realistic 3D face models that learn a detailed prior over both geometry and appearance of human faces. HeadNeRF~\cite{hong2022headnerf} conditions a Neural Radiance Field (NeRF)~\cite{mildenhall2021nerf} on identity, expression, albedo, and illumination codes. VRMM~\cite{yang2024vrmm} builds a high-quality and relightable 3D face model using volumetric primitives~\cite{lombardi2021mvp}. One2Avatar~\cite{yu2024one2avatar} extends a 3DMM by anchoring a radiance field to its surface. More recently, GPHM~\cite{xu2025gphm} and HeadGAP~\cite{zheng2024headgap} have adopted 3D Gaussians to build a photo-realistic 3D face model. \\
Photo-realistic 3D face models learn a powerful prior over human facial appearance and geometry, which can be fitted to a single or multiple images of a person, effectively inferring a 3D head avatar. However, the fitting procedure itself is non-trivial and often requires expensive test-time optimization, impeding casual use-cases on consumer-grade devices. While this limitation may be circumvented by learning a generalized encoder that maps images into the 3D face model's latent space, another fundamental limitation remains. Even with more multi-view face datasets being published, the number of available training subjects rarely exceeds the thousands, making it hard to truly learn the full distibution of human facial appearance. Instead, our approach avoids generalizing over the identity axis by conditioning on some images of a person, and only generalizes over the expression axis for which plenty of data is available. 

A similar motivation has inspired recent work on codec avatars where a generalized network infers an animatable 3D representation given a registered mesh of a person~\cite{cao2022authentic, li2024uravatar}.
The resulting avatars exhibit excellent quality at the cost of several minutes of video capture per subject and expensive test-time optimization.
For example, URAvatar~\cite{li2024uravatar} finetunes their network on the given video recording for 3 hours on 8 A100 GPUs, making inference on consumer-grade devices impossible. In contrast, our approach directly regresses the final 3D head avatar from just four input images without the need for expensive test-time fine-tuning.


\section{Methodology}

\subsection{Problem Definition}

Given a multivariate time series input $X \in \mathbb{R}^{C  \times T}$, multivariate time series forecasting tasks are designed to predict its future $F$ time steps $\hat{Y}\in \mathbb{R}^{C \times F}$ using past $T$ steps. $C $ is the number of variates or channels.

\subsection{Preliminary Analysis}

This section presents why RevIN~\citep{Kim_revin,liu2022non}, High-pass, and Low-pass filters fail to address the Mid-Frequency Spectrum Gap. Let the input univariate time series be $ x(t) $ with length $ T $ and target $ y(t) $ with length $ F $. 

\begin{definition}[Frequency Spectral Energy]\label{def:energy}
The Fourier transform of $x(t)$, $X(f)$, and its spectral energy $E_X(f)$ is given by:
\vspace{-0.2cm}
\begin{align}
X(f) = \sum_{t=0}^{T-1} x(t) e^{-i 2 \pi f t / {T-1}}, \quad &f = 0, 1, \dots, T-1\notag\\
E_X(f) = |X(f)|^2.
\end{align}
\vspace{-0.2cm}
\end{definition}

\textbf{Impact of RevIN on Frequency Spectrum \quad}
\begin{definition}[Reversible Instance Normalization]\label{def:RevIN}
Given a \textbf{forecast model} $ f: \mathbb{R}^T \rightarrow \mathbb{R}^F $ that generates a forecast $ \hat{y}(t) $ from a given input $x(t)$, RevIN is defined as:
\vspace{-0.2cm}
\begin{align}
&\hat{x}(t) = \frac{x(t) - \mu}{\sigma},\quad t = 0, 1, \dots, T-1\notag\\
&\hat{y}(t) = f(\hat{x}(t)), \quad \hat{y}(t)_{rev}= \hat{y}(t) \cdot \sigma + \mu,\notag\\
&\mu = \frac{1}{T} \sum_{t=0}^{T-1} x(t), \quad \sigma = \sqrt{\frac{1}{T} \sum_{t=0}^{T-1} (x(t) - \mu)^2}.
\end{align}
\vspace{-0.2cm}
\end{definition}

\begin{theorem} [Frequency Spectrum after RevIN] \label{theorem:RevIN}
\vspace{-0.2cm}
The spectral energy of $\hat{x}(t)$ (transformed using RevIN):
\begin{align}
E_{\hat{X}}(0)=0,& \quad f=0, \notag\\
E_{\hat{X}}(f) = \left( \frac{1}{\sigma} \right)^2 |X(f)|^2,&\quad f = 1,2,\dots, T-1 . 
\end{align}
\vspace{-0.2cm}
\end{theorem}
The proof is in Appendix~\ref{app:RevIN}. Theorem~\ref{theorem:RevIN} suggests that RevIN scales the absolute spectral energy by $ \sigma^2 $ but does not affect its relative distribution except $E_{\hat{X}}(0)=0$. Thus, RevIN preserves the relative spectral energy distribution and leaves the Mid-Frequency Spectrum Gap unresolved. \textit{However, our experiments still employ RevIN to ensure a fair comparison with other baselines.}
\begin{figure*}[h]
  \centering
  \includegraphics[width=1.\linewidth]{Faker/source/assets/jpg/ReFocus.jpg}
  \caption{General structure of \textbf{ReFocus}. `Adaptive Mid-Frequency Energy Optimizer (AMEO)' enhances mid-frequency components modeling, and `Energy-based Key-Frequency Picking Block' (EKPB) effectively captures shared Key-Frequency across channels}
  \label{fig:refocus}
\end{figure*}

\begin{figure*}[h]
  \centering
  \includegraphics[width=0.7\linewidth]{Faker/source/assets/jpg/ket.jpg}
  \caption{General process of the \textbf{Key-Frequency Enhanced Training strategy (KET)}, where spectral information from other channels is randomly introduced into each channel, to enhance the extraction of the shared Key-Frequency.}
  \label{fig:reshuffle}
\end{figure*}
\textbf{Impact of High- and Low-pass filter \quad}
We still define $\hat{x}(t)$ to be the filtered (processed) signal, obtained by applying a filter $H(f)$ (High/Low-pass filter). The filter $ H(f) $ is 1 in the passband (High/Low frequency) and 0 in the stopband (Middle frequency). So $E_{\hat{X}}(f)=0,\quad E_{\hat{X}}\leq E_X(f)$ for middle frequencies, which creates even larger gap.

\subsection{Overall Structure of The Proposed ReFocus}

In this section, we elucidate the overall architecture of \textbf{ReFocus}, depicted in Figure \ref{fig:refocus}. We define frequency domain projection as $D1\rightarrow D2$ representing a projection from dimension $D1$ to $D2$ in the frequency domain~\citep{xu2024fits}. Initially, we apply \textbf{AMEO} to the input $X \in \mathbb{R}^{C \times T}$, yielding the processed spectrum $ X_{am} \in \mathbb{R}^{C  \times T} $. Next, we use a projection $T\rightarrow D$ to transform $ X_{am}$ into the Variate Embedding $ X_{em} \in \mathbb{R}^{C  \times D}$~\citep{LiuiTransformer}. Then, $X_{em}$ go through $N$ \textbf{EKPB} to generate representation $H_{N+1}$, which is projected to obtain final prediction $\hat{Y}$. 

\textbf{Adaptive Mid-Frequency Energy Optimizer \quad}
Building upon the \textbf{Preliminary Analysis}, we propose a convolution- and residual learning-based solution to address the Mid-Frequency Spectrum Gap, which we denoted as AMEO. 
\begin{definition}[Adaptive Mid-Frequency Energy Optimizer]\label{def:AMEO}
AMEO is defined as:
\begin{align}
&\hat{x}(t) = x(t)-\frac{\beta}{K}\sum_{k=0}^{K-1} \tilde{x}(t+K-1-k),\notag\\
&\tilde{x}(t) =\notag\\
&\begin{cases}
x(t-(\frac{K}{2}+1)), \quad \text{if } \frac{K}{2}+1 \leq t < T+\frac{K}{2}+1, \\
0,  \quad\text{if } 0 \leq t < \frac{K}{2}+1 \text{ or } T+\frac{K}{2}+1 \leq t < T+K.
\end{cases}
\end{align}
\vspace{-0.2cm}
\end{definition}

It is equivalent to $x=x-\beta \cdot Conv(x)$. $Conv$ is a 1D convolution (Zero-padding at both ends, stride $s=1$, kernel size $K$, with values initialized as $ \frac{1}{K} $). $\beta \in \mathbb{R}^{1}$ is a hyperparameter.

\begin{theorem} [Frequency Spectrum after AMEO] \label{theorem:AMEO}
The spectral energy of $\hat{x}(t)$ obtained using AMEO:
\begin{align}
E_{\hat{X}}(f) =|X(f)|^2 \left\{1 - \beta \cdot \underbrace{\frac{1}{K} \sum_{k=0}^{K-1} e^{i 2 \pi f (\frac{3K}{2}-k -2) / {T-1}}}_{G(f)}\right\}^2
\end{align}
\vspace{-0.2cm}
\end{theorem}

The proof is in Appendix~\ref{app:AMEO}. We have $E_{\hat{X}}(f) =|X(f)|^2(1-\beta  \cdot G(f))^2$. Generally, $ G(f) $ behaves as a decay function, gradually reducing its value from \textbf{One} to \textbf{Zero}. Such \textbf{decay behavior} makes AMEO relatively enhances mid-frequency components, thus addressing the Mid-Frequency Spectrum Gap.

\textbf{Energy-based Key-Frequency Picking Block \quad} In each \textbf{EKPB}, the input $ H_i \in \mathbb{R}^{C  \times D} (H_1=X_{em}) $ is first processed through an MLP to generate $ H_i^k \in \mathbb{R}^{C  \times Q}$. Then, FFT is applied to get $ H_i^f \in \mathbb{R}^{C  \times (Q/2+1)}$. For $ H_i^f$, we calculate its energy, denoted as $ H_i^e \in \mathbb{R}^{C  \times (Q/2+1)}$. A cross-channel softmax is then applied to $ H_i^e$ per frequency to obtain a probability distribution $ H_i^{soft} \in \mathbb{R}^{C  \times (Q/2+1)}$. Using $H_i^{soft}$, we select values from $ H_i^f$ across channels for each frequency, resulting in $K^f_i \in \mathbb{R}^{1  \times (Q/2+1)}$, which represents the Shared Key-Frequency across all channels. Then iFFT is performed on $K^f_i$ to get $K_i\in \mathbb{R}^{1  \times Q}$, followed by projection $Q\rightarrow D$ and repeating (C times) to get $\hat{K}_i \in \mathbb{R}^{C  \times D}$. This $\hat{K}_i$ is point-wisely added to $\hat{H_i}\in \mathbb{R}^{C  \times D}$ , which is the projection of $ H_i$ using projection $D\rightarrow D$. Then, an MLP and $Add\&Norm$ is applied to the result $HK\in \mathbb{R}^{C  \times D}$ to fuse inter-series dependencies information, and another MLP and $Add\&Norm$ is used to capture intra-series variations~\citep{LiuiTransformer}. The output of each \textbf{EKPB} is $\hat{O_i} \in \mathbb{R}^{C  \times D}$, where $H_{i+1}=\hat{O_i}$.

\subsection{Key-Frequency Enhanced Training strategy}

In real-world time series, certain channels often exhibit spectral dependencies, which may not be fully captured in the training set, and the specific channels with such dependencies are also unknown~\citep{geweke1984freqchannel,Zhao2024freqchannel}. So this work borrows insight from recent advancement of mix-up in time series~\citep{zhou2023mixup,ansari2024mixup}, randomly introducing spectral information from other channels into each channel, to enhance the extraction of the shared Key-Frequency, as in Figure~\ref{fig:reshuffle}. Given a multivariate time series input $X \in \mathbb{R}^{C \times T}$ and its ground-truth $Y \in \mathbb{R}^{C \times F}$, we generate a pseudo sample pair: 

\begin{align}
X' = iFFT(FFT(X) +\alpha \cdot FFT(X[\text{perm},:]))&,  \notag\\ 
Y' = iFFT(FFT(Y) +\alpha \cdot FFT(Y[\text{perm},:]))&.
\end{align}

$\alpha \in \mathbb{R}^{C \times 1}$ is a weight vector sampled from a normal distribution, $\text{perm}$ is a reshuffled channel index. Since $FFT$ and $iFFT$ are linear operations, this mix-up process can be equivalently simplified in the \textbf{Time Domain}:
\begin{align}
X' = X +\alpha \cdot X[\text{perm},:]&,  \notag\\
Y' = Y +\alpha \cdot Y[\text{perm},:]&
 \end{align}
We alternate training between real and synthetic data to preserve the spectral dependencies in real samples. This combines the advantages of data augmentation, such as improved generalization, while mitigating potential drawbacks like over-smoothing and training instability~\citep{ryu2024tf,alkhalifah2022tf}.












\section{Experimental Results}
In this section, we present the main results in~\secref{sec:main}, followed by ablation studies on key design choices in~\secref{sec:ablation}.

\begin{table*}[t]
\renewcommand\arraystretch{1.05}
\centering
\setlength{\tabcolsep}{2.5mm}{}
\begin{tabular}{l|l|c|cc|cc}
type & model     & \#params      & FID$\downarrow$ & IS$\uparrow$ & Precision$\uparrow$ & Recall$\uparrow$ \\
\shline
GAN& BigGAN~\cite{biggan} & 112M & 6.95  & 224.5       & 0.89 & 0.38     \\
GAN& GigaGAN~\cite{gigagan}  & 569M      & 3.45  & 225.5       & 0.84 & 0.61\\  
GAN& StyleGan-XL~\cite{stylegan-xl} & 166M & 2.30  & 265.1       & 0.78 & 0.53  \\
\hline
Diffusion& ADM~\cite{adm}    & 554M      & 10.94 & 101.0        & 0.69 & 0.63\\
Diffusion& LDM-4-G~\cite{ldm}   & 400M  & 3.60  & 247.7       & -  & -     \\
Diffusion & Simple-Diffusion~\cite{diff1} & 2B & 2.44 & 256.3 & - & - \\
Diffusion& DiT-XL/2~\cite{dit} & 675M     & 2.27  & 278.2       & 0.83 & 0.57     \\
Diffusion&L-DiT-3B~\cite{dit-github}  & 3.0B    & 2.10  & 304.4       & 0.82 & 0.60    \\
Diffusion&DiMR-G/2R~\cite{liu2024alleviating} &1.1B& 1.63& 292.5& 0.79 &0.63 \\
Diffusion & MDTv2-XL/2~\cite{gao2023mdtv2} & 676M & 1.58 & 314.7 & 0.79 & 0.65\\
Diffusion & CausalFusion-H$^\dag$~\cite{deng2024causal} & 1B & 1.57 & - & - & - \\
\hline
Flow-Matching & SiT-XL/2~\cite{sit} & 675M & 2.06 & 277.5 & 0.83 & 0.59 \\
Flow-Matching&REPA~\cite{yu2024representation} &675M& 1.80 & 284.0 &0.81 &0.61\\    
Flow-Matching&REPA$^\dag$~\cite{yu2024representation}& 675M& 1.42&  305.7& 0.80& 0.65 \\
\hline
Mask.& MaskGIT~\cite{maskgit}  & 227M   & 6.18  & 182.1        & 0.80 & 0.51 \\
Mask. & TiTok-S-128~\cite{yu2024image} & 287M & 1.97 & 281.8 & - & - \\
Mask. & MAGVIT-v2~\cite{yu2024language} & 307M & 1.78 & 319.4 & - & - \\ 
Mask. & MaskBit~\cite{weber2024maskbit} & 305M & 1.52 & 328.6 & - & - \\
\hline
AR& VQVAE-2~\cite{vqvae2} & 13.5B    & 31.11           & $\sim$45     & 0.36           & 0.57          \\
AR& VQGAN~\cite{vqgan}& 227M  & 18.65 & 80.4         & 0.78 & 0.26   \\
AR& VQGAN~\cite{vqgan}   & 1.4B     & 15.78 & 74.3   & -  & -     \\
AR&RQTran.~\cite{rq}     & 3.8B    & 7.55  & 134.0  & -  & -    \\
AR& ViTVQ~\cite{vit-vqgan} & 1.7B  & 4.17  & 175.1  & -  & -    \\
AR & DART-AR~\cite{gu2025dart} & 812M & 3.98 & 256.8 & - & - \\
AR & MonoFormer~\cite{zhao2024monoformer} & 1.1B & 2.57 & 272.6 & 0.84 & 0.56\\
AR & Open-MAGVIT2-XL~\cite{luo2024open} & 1.5B & 2.33 & 271.8 & 0.84 & 0.54\\
AR&LlamaGen-3B~\cite{llamagen}  &3.1B& 2.18& 263.3 &0.81& 0.58\\
AR & FlowAR-H~\cite{flowar} & 1.9B & 1.65 & 296.5 & 0.83 & 0.60\\
AR & RAR-XXL~\cite{yu2024randomized} & 1.5B & 1.48 & 326.0 & 0.80 & 0.63 \\
\hline
MAR & MAR-B~\cite{mar} & 208M & 2.31 &281.7 &0.82 &0.57 \\
MAR & MAR-L~\cite{mar} &479M& 1.78 &296.0& 0.81& 0.60 \\
MAR & MAR-H~\cite{mar} & 943M&1.55& 303.7& 0.81 &0.62 \\
\hline
VAR&VAR-$d16$~\cite{var}   & 310M  & 3.30& 274.4& 0.84& 0.51    \\
VAR&VAR-$d20$~\cite{var}   &600M & 2.57& 302.6& 0.83& 0.56     \\
VAR&VAR-$d30$~\cite{var}   & 2.0B      & 1.97  & 323.1 & 0.82 & 0.59      \\
\hline
\modelname& \modelname-B    &172M   &1.72&280.4&0.82&0.59 \\
\modelname& \modelname-L   & 608M   & 1.28& 292.5&0.82&0.62\\
\modelname& \modelname-H    & 1.1B    & 1.24 &301.6&0.83&0.64\\
\end{tabular}
\caption{
\textbf{Generation Results on ImageNet-256.}
Metrics include Fréchet Inception Distance (FID), Inception Score (IS), Precision, and Recall. $^\dag$ denotes the use of guidance interval sampling~\cite{guidance}. The proposed \modelname-H achieves a state-of-the-art 1.24 FID on the ImageNet-256 benchmark without relying on vision foundation models (\eg, DINOv2~\cite{dinov2}) or guidance interval sampling~\cite{guidance}, as used in REPA~\cite{yu2024representation}.
}\label{tab:256}
\end{table*}

\subsection{Main Results}
\label{sec:main}
We conduct experiments on ImageNet~\cite{deng2009imagenet} at 256$\times$256 and 512$\times$512 resolutions. Following prior works~\cite{dit,mar}, we evaluate model performance using FID~\cite{fid}, Inception Score (IS)~\cite{is}, Precision, and Recall. \modelname is trained with the same hyper-parameters as~\cite{mar,dit} (\eg, 800 training epochs), with model sizes ranging from 172M to 1.1B parameters. See Appendix~\secref{sec:sup_hyper} for hyper-parameter details.





\begin{table}[t]
    \centering
    \begin{tabular}{c|c|c|c}
      model    &  \#params & FID$\downarrow$ & IS$\uparrow$ \\
      \shline
      VQGAN~\cite{vqgan}&227M &26.52& 66.8\\
      BigGAN~\cite{biggan}& 158M&8.43 &177.9\\
      MaskGiT~\cite{maskgit}& 227M&7.32& 156.0\\
      DiT-XL/2~\cite{dit} &675M &3.04& 240.8 \\
     DiMR-XL/3R~\cite{liu2024alleviating}& 525M&2.89 &289.8 \\
     VAR-d36~\cite{var}  & 2.3B& 2.63 & 303.2\\
     REPA$^\ddagger$~\cite{yu2024representation}&675M &2.08& 274.6 \\
     \hline
     \modelname-L & 608M&1.70& 281.5 \\
    \end{tabular}
    \caption{
    \textbf{Generation Results on ImageNet-512.} $^\ddagger$ denotes the use of DINOv2~\cite{dinov2}.
    }
    \label{tab:512}
\end{table}

\noindent\textbf{ImageNet-256.}
In~\tabref{tab:256}, we compare \modelname with previous state-of-the-art generative models.
Out best variant, \modelname-H, achieves a new state-of-the-art-performance of 1.24 FID, outperforming the GAN-based StyleGAN-XL~\cite{stylegan-xl} by 1.06 FID, masked-prediction-based MaskBit~\cite{maskgit} by 0.28 FID, AR-based RAR~\cite{yu2024randomized} by 0.24 FID, VAR~\cite{var} by 0.73 FID, MAR~\cite{mar} by 0.31 FID, and flow-matching-based REPA~\cite{yu2024representation} by 0.18 FID.
Notably, \modelname does not rely on vision foundation models~\cite{dinov2} or guidance interval sampling~\cite{guidance}, both of which were used in REPA~\cite{yu2024representation}, the previous best-performing model.
Additionally, our lightweight \modelname-B (172M), surpasses DiT-XL (675M)~\cite{dit} by 0.55 FID while achieving an inference speed of 9.8 images per second—20$\times$ faster than DiT-XL (0.5 images per second). Detailed speed comparison can be found in Appendix \ref{sec:speed}.



\noindent\textbf{ImageNet-512.}
In~\tabref{tab:512}, we report the performance of \modelname on ImageNet-512.
Similarly, \modelname-L sets a new state-of-the-art FID of 1.70, outperforming the diffusion based DiT-XL/2~\cite{dit} and DiMR-XL/3R~\cite{liu2024alleviating} by a large margin of 1.34 and 1.19 FID, respectively.
Additionally, \modelname-L also surpasses the previous best autoregressive model VAR-d36~\cite{var} and flow-matching-based REPA~\cite{yu2024representation} by 0.93 and 0.38 FID, respectively.




\noindent\textbf{Qualitative Results.}
\figref{fig:qualitative} presents samples generated by \modelname (trained on ImageNet) at 512$\times$512 and 256$\times$256 resolutions. These results highlight \modelname's ability to produce high-fidelity images with exceptional visual quality.

\begin{figure*}
    \centering
    \vspace{-6pt}
    \includegraphics[width=1\linewidth]{figures/qualitative.pdf}
    \caption{\textbf{Generated Samples.} \modelname generates high-quality images at resolutions of 512$\times$512 (1st row) and 256$\times$256 (2nd and 3rd row).
    }
    \label{fig:qualitative}
\end{figure*}

\subsection{Ablation Studies}
\label{sec:ablation}
In this section, we conduct ablation studies using \modelname-B, trained for 400 epochs to efficiently iterate on model design.

\noindent\textbf{Prediction Entity X.}
The proposed \modelname extends next-token prediction to next-X prediction. In~\tabref{tab:X}, we evaluate different designs for the prediction entity X, including an individual patch token, a cell (a group of surrounding tokens), a subsample (a non-local grouping), a scale (coarse-to-fine resolution), and an entire image.

Among these variants, cell-based \modelname achieves the best performance, with an FID of 2.48, outperforming the token-based \modelname by 1.03 FID and surpassing the second best design (scale-based \modelname) by 0.42 FID. Furthermore, even when using standard prediction entities such as tokens, subsamples, images, or scales, \modelname consistently outperforms existing methods while requiring significantly fewer parameters. These results highlight the efficiency and effectiveness of \modelname across diverse prediction entities.






\begin{table}[]
    \centering
    \scalebox{0.92}{
    \begin{tabular}{c|c|c|c|c}
        model & \makecell[c]{prediction\\entity} & \#params & FID$\downarrow$ & IS$\uparrow$\\
        \shline
        LlamaGen-L~\cite{llamagen} & \multirow{2}{*}{token} & 343M & 3.80 &248.3\\
        \modelname-B& & 172M&3.51&251.4\\
        \hline
        PAR-L~\cite{par} & \multirow{2}{*}{subsample}& 343M & 3.76 & 218.9\\
        \modelname-B&  &172M& 3.58&231.5\\
        \hline
        DiT-L/2~\cite{dit}& \multirow{2}{*}{image}& 458M&5.02&167.2 \\
         \modelname-B& & 172M&3.13&253.4 \\
        \hline
        VAR-$d16$~\cite{var} & \multirow{2}{*}{scale} & 310M&3.30 &274.4\\
        \modelname-B& &172M&2.90&262.8\\
        \hline
        \baseline{\modelname-B}& \baseline{cell} & \baseline{172M}&\baseline{2.48}&\baseline{269.2} \\
    \end{tabular}
    }
    \caption{\textbf{Ablation on Prediction Entity X.} Using cells as the prediction entity outperforms alternatives such as tokens or entire images. Additionally, under the same prediction entity, \modelname surpasses previous methods, demonstrating its effectiveness across different prediction granularities. }%
    \label{tab:X}
\end{table}

\noindent\textbf{Cell Size.}
A prediction entity cell is formed by grouping spatially adjacent $k\times k$ tokens, where a larger cell size incorporates more tokens and thus captures a broader context within a single prediction step.
For a $256\times256$ input image, the encoded continuous latent representation has a spatial resolution of $16\times16$. Given this, the image can be partitioned into an $m\times m$ grid, where each cell consists of $k\times k$ neighboring tokens. As shown in~\tabref{tab:cell}, we evaluate different cell sizes with $k \in \{1,2,4,8,16\}$, where $k=1$ represents a single token and $k=16$ corresponds to the entire image as a single entity. We observe that performance improves as $k$ increases, peaking at an FID of 2.48 when using cell size $8\times8$ (\ie, $k=8$). Beyond this, performance declines, reaching an FID of 3.13 when the entire image is treated as a single entity.
These results suggest that using cells rather than the entire image as the prediction unit allows the model to condition on previously generated context, improving confidence in predictions while maintaining both rich semantics and local details.





\begin{table}[t]
    \centering
    \scalebox{0.98}{
    \begin{tabular}{c|c|c|c}
    cell size ($k\times k$ tokens) & $m\times m$ grid & FID$\downarrow$ & IS$\uparrow$ \\
       \shline
       $1\times1$ & $16\times16$ &3.51&251.4 \\
       $2\times2$ & $8\times8$ & 3.04& 253.5\\
       $4\times4$ & $4\times4$ & 2.61&258.2 \\
       \baseline{$8\times8$} & \baseline{$2\times2$} & \baseline{2.48} & \baseline{269.2}\\
       $16\times16$ & $1\times1$ & 3.13&253.4  \\
    \end{tabular}
    }
    \caption{\textbf{Ablation on the cell size.}
    In this study, a $16\times16$ continuous latent representation is partitioned into an $m\times m$ grid, where each cell consits of $k\times k$ neighboring tokens.
    A cell size of $8\times8$ achieves the best performance, striking an optimal balance between local structure and global context.
    }
    \label{tab:cell}
\end{table}



\begin{table}[t]
    \centering
    \scalebox{0.95}{
    \begin{tabular}{c|c|c|c}
      previous cell & noise time step &  FID$\downarrow$ & IS$\uparrow$ \\
       \shline
       clean & $t_i=0, \forall i<n$& 3.45& 243.5\\
       increasing noise & $t_1<t_2<\cdots<t_{n-1}$& 2.95&258.8 \\
       decreasing noise & $t_1>t_2>\cdots>t_{n-1}$&2.78 &262.1 \\
      \baseline{random noise}  & \baseline{no constraint} &\baseline{2.48} & \baseline{269.2}\\
    \end{tabular}
    }
    \caption{
    \textbf{Ablation on Noisy Context Learning.}
    This study examines the impact of noise time steps ($t_1, \cdots, t_{n-1} \subset [0, 1]$) in previous entities ($t=0$ represents pure Gaussian noise).
    Conditioning on all clean entities (the ``clean'' variant) results in suboptimal performance.
    Imposing an order on noise time steps, either ``increasing noise'' or ``decreasing noise'', also leads to inferior results. The best performance is achieved with the "random noise" setting, where no constraints are imposed on noise time steps.
    }
    \label{tab:ncl}
\end{table}


\noindent\textbf{Noisy Context Learning.}
During training, \modelname employs Noisy Context Learning (NCL), predicting $X_n$ by conditioning on all previous noisy entities, unlike Teacher Forcing.
The noise intensity of previous entities is contorlled by noise time steps $\{t_1, \dots, t_{n-1}\} \subset [0, 1]$, where $t=0$ corresponds to pure Gaussian noise.
We analyze the impact of NCL in~\tabref{tab:ncl}.
When conditioning on all clean entities (\ie, the ``clean'' variant, where $t_i=0, \forall i<n$), which is equivalent to vanilla AR (\ie, Teacher Forcing), the suboptimal performance is obtained.
We also evaluate two constrained noise schedules: the ``increasing noise'' variant, where noise time steps increase over AR steps ($t_1<t_2< \cdots < t_{n-1}$), and the `` decreasing noise'' variant, where noise time steps decrease ($t_1>t_2> \cdots > t_{n-1}$).
While both settings improve over the ``clean'' variant, they remain inferior to our final ``random noise'' setting, where no constraints are imposed on noise time steps, leading to the best performance.




        

\section{Discussion and Ablation Study}
\label{sec:ablations}
We provide an in-depth analysis of our optimization setting, the impact of different loss functions, and the number of tolerances sampled during optimization. This analysis aims to offer more guidance for future work.

%--------------------------------------------------------
\subsection{Partially Tolerance Optimization}
In deep optics, the optimization of optical and decoder parameters may vary~\cite{tseng2021differentiable}. Tolerance-aware optimization adds further complexity by introducing random perturbations in each forward pass. To delve into the distinctions between these components, we selectively control the optimization: (a) optimizing the optics, and (b) only the computational decoder. Through experiments, we find that the deep optics achieves improvement of robustness only when the optics and decoder are jointly optimized simultaneously, shown in \cref{tab:partly-tolerance}. 
Optimizing only the decoder can enhance the decoder's tolerance-aware ability, but still fails to fully address the impact of random tolerances. Additionally, when optimizing the optics alone, the global influence of optical parameters on the subsequent decoder, along with the limited number of optical parameters, makes the optimization challenging and risks compromising the results of the initial pre-training stage. Only through the joint optimization of the optical system and decoder can we achieve a more stable optical system with enhanced pairing capability from the decoder, significantly improving the robustness of deep optics against potential tolerances.

%--------------------------------------------------------

\begin{table}
\centering
\caption{\emph{Optics/Decoder-only} means that only optimize the optics/decoder part of parameters during tolerance optimization and \emph{Both} means optimized both parts. The results are average of 100 times of random sampled tolerances experiments (excluding Spot Size).}
\begin{tabular}{c|c c c c}
\hline
 & Optics-only & Decoder-only & Both \\
\hline
PSNR$\uparrow$ & 22.06 & 25.94 & \bf{28.08} \\
SSIM$\uparrow$ & 0.376 & 0.734 & \bf{0.845} \\
LPIPS$\downarrow$ & 0.664 & 0.379 & \bf{0.225} \\
Spot Size ($\mu m$)$\downarrow$ & 34.2 & \textbf{14.2} & 38.7 \\
\hline
\end{tabular}
\label{tab:partly-tolerance}
\end{table}

%--------------------------------------------------------
\subsection{Analysis of Loss Function Impact}
To analyze the impact of the two loss functions, we conduct ablation studies on Spot loss and PSF loss. The experimental results are shown in \cref{tab:loss_comparison}. The results demonstrate that the proposed Spot loss and PSF loss significantly improve the performance. It is worth noting that Spot loss and PSF similarity loss can only be used together to maximize the performance of tolerance-aware optimization. The two losses play different roles respectively, the Spot loss is helpful to ensure that the overall imaging quality of the optics is not seriously degraded during the tolerance optimization, while the PSF loss is able to significantly improve the robustness of the optics to random tolerances on this basis. Without the constraint of Spot loss, PSF loss can lead to significant degradation in the imaging performance of the optical system, thereby impacting overall tolerance optimization. As shown in \cref{tab:loss_comparison}, the spot size is extremely large, indicating very low imaging quality.
\begin{figure}[t]
  \centering
   \includegraphics[width=1.0\linewidth]{figures/perb_num_ablation.pdf}

   \caption{Ablation study on the number of sampled tolerances pattern in every iteration. Average PSNR, SSIM, and LPIPS of 100 times random tolerances test for different sampling numbers are shown, along with the PSFs comparison. Scale bar: $15\mu m$ .}
   \label{fig:perb_num}
\end{figure}

\begin{table}
\centering
\caption{Ablation study on Spot loss and PSF loss, each incorporating basic image quality loss. The results represent the averages from 100 random sampling tolerance experiments (excluding Spot Size).}
\begin{tabular}{c|c c c c}
\hline
 & - & $\mathcal{L}_{\text{Spot}}$ & $\mathcal{L}_{\text{PSF}}$ & $\mathcal{L}_{\text{Spot}}\&\mathcal{L}_{\text{PSF}}$ \\
\hline
PSNR$\uparrow$ & 24.83 & 26.63 & 20.49 & \bf{28.08} \\
SSIM$\uparrow$ & 0.738 & 0.782 & 0.601 & \bf{0.850} \\
LPIPS$\downarrow$ & 0.378 & 0.248 & 0.529 & \bf{0.225} \\
Spot Size ($\mu m$)$\downarrow$ & 52.6 & \bf{12.9} & 637.9 & 38.7 \\
\hline
\end{tabular}
\label{tab:loss_comparison}
\end{table}

%--------------------------------------------------------
\subsection{Number of Sampled Tolerance Patterns}
In the tolerance-aware optimization process, the number of sampled tolerance patterns is a critical hyperparameter that significantly impacts the performance. If the number is too low, the optimization may stuck in local optima. However, increasing the sampled number leads to a higher consumption of GPU memory. Therefore, selecting an appropriate sample number is of great importance to balance the trade-off between optimization effectiveness and memory usage. Therefore, we employ different number of sampled tolerance pattern and conduct quantitative analysis to determine a reasonable lower limit, see in \cref{fig:perb_num}. This experiment significantly streamlines subsequent research efforts.

\paragraph{Summary}
Our findings provide significant insights into the influence of correctness, explanations, and refinement on evaluation accuracy and user trust in AI-based planners. 
In particular, the findings are three-fold: 
(1) The \textbf{correctness} of the generated plans is the most significant factor that impacts the evaluation accuracy and user trust in the planners. As the PDDL solver is more capable of generating correct plans, it achieves the highest evaluation accuracy and trust. 
(2) The \textbf{explanation} component of the LLM planner improves evaluation accuracy, as LLM+Expl achieves higher accuracy than LLM alone. Despite this improvement, LLM+Expl minimally impacts user trust. However, alternative explanation methods may influence user trust differently from the manually generated explanations used in our approach.
% On the other hand, explanations may help refine the trust of the planner to a more appropriate level by indicating planner shortcomings.
(3) The \textbf{refinement} procedure in the LLM planner does not lead to a significant improvement in evaluation accuracy; however, it exhibits a positive influence on user trust that may indicate an overtrust in some situations.
% This finding is aligned with prior works showing that iterative refinements based on user feedback would increase user trust~\cite{kunkel2019let, sebo2019don}.
Finally, the propensity-to-trust analysis identifies correctness as the primary determinant of user trust, whereas explanations provided limited improvement in scenarios where the planner's accuracy is diminished.

% In conclusion, our results indicate that the planner's correctness is the dominant factor for both evaluation accuracy and user trust. Therefore, selecting high-quality training data and optimizing the training procedure of AI-based planners to improve planning correctness is the top priority. Once the AI planner achieves a similar correctness level to traditional graph-search planners, strengthening its capability to explain and refine plans will further improve user trust compared to traditional planners.

\paragraph{Future Research} Future steps in this research include expanding user studies with larger sample sizes to improve generalizability and including additional planning problems per session for a more comprehensive evaluation. Next, we will explore alternative methods for generating plan explanations beyond manual creation to identify approaches that more effectively enhance user trust. 
Additionally, we will examine user trust by employing multiple LLM-based planners with varying levels of planning accuracy to better understand the interplay between planning correctness and user trust. 
Furthermore, we aim to enable real-time user-planner interaction, allowing users to provide feedback and refine plans collaboratively, thereby fostering a more dynamic and user-centric planning process.


{
    \small
    \bibliographystyle{ieeenat_fullname}
    \bibliography{main}
}

% WARNING: do not forget to delete the supplementary pages from your submission 
\clearpage
\pagenumbering{gobble}
\maketitlesupplementary

\section{Additional Results on Embodied Tasks}

To evaluate the broader applicability of our EgoAgent's learned representation beyond video-conditioned 3D human motion prediction, we test its ability to improve visual policy learning for embodiments other than the human skeleton.
Following the methodology in~\cite{majumdar2023we}, we conduct experiments on the TriFinger benchmark~\cite{wuthrich2020trifinger}, which involves a three-finger robot performing two tasks: reach cube and move cube. 
We freeze the pretrained representations and use a 3-layer MLP as the policy network, training each task with 100 demonstrations.

\begin{table}[h]
\centering
\caption{Success rate (\%) on the TriFinger benchmark, where each model's pretrained representation is fixed, and additional linear layers are trained as the policy network.}
\label{tab:trifinger}
\resizebox{\linewidth}{!}{%
\begin{tabular}{llcc}
\toprule
Methods       & Training Dataset & Reach Cube & Move Cube \\
\midrule
DINO~\cite{caron2021emerging}         & WT Venice        & 78.03     & 47.42     \\
DoRA~\cite{venkataramanan2023imagenet}          & WT Venice        & 81.62     & 53.76     \\
DoRA~\cite{venkataramanan2023imagenet}          & WT All           & 82.40     & 48.13     \\
\midrule
EgoAgent-300M & WT+Ego-Exo4D      & 82.61    & 54.21      \\
EgoAgent-1B   & WT+Ego-Exo4D      & \textbf{85.72}      & \textbf{57.66}   \\
\bottomrule
\end{tabular}%
}
\end{table}

As shown in Table~\ref{tab:trifinger}, EgoAgent achieves the highest success rates on both tasks, outperforming the best models from DoRA~\cite{venkataramanan2023imagenet} with increases of +3.32\% and +3.9\% respectively.
This result shows that by incorporating human action prediction into the learning process, EgoAgent demonstrates the ability to learn more effective representations that benefit both image classification and embodied manipulation tasks.
This highlights the potential of leveraging human-centric motion data to bridge the gap between visual understanding and actionable policy learning.



\section{Additional Results on Egocentric Future State Prediction}

In this section, we provide additional qualitative results on the egocentric future state prediction task. Additionally, we describe our approach to finetune video diffusion model on the Ego-Exo4D dataset~\cite{grauman2024ego} and generate future video frames conditioned on initial frames as shown in Figure~\ref{fig:opensora_finetune}.

\begin{figure}[b]
    \centering
    \includegraphics[width=\linewidth]{figures/opensora_finetune.pdf}
    \caption{Comparison of OpenSora V1.1 first-frame-conditioned video generation results before and after finetuning on Ego-Exo4D. Fine-tuning enhances temporal consistency, but the predicted pixel-space future states still exhibit errors, such as inaccuracies in the basketball's trajectory.}
    \label{fig:opensora_finetune}
\end{figure}

\subsection{Visualizations and Comparisons}

More visualizations of our method, DoRA, and OpenSora in different scenes (as shown in Figure~\ref{fig:supp pred}). For OpenSora, when predicting the states of $t_k$, we use all the ground truth frames from $t_{0}$ to $t_{k-1}$ as conditions. As OpenSora takes only past observations as input and neglects human motion, it performs well only when the human has relatively small motions (see top cases in Figure~\ref{fig:supp pred}), but can not adjust to large movements of the human body or quick viewpoint changes (see bottom cases in Figure~\ref{fig:supp pred}).

\begin{figure*}
    \centering
    \includegraphics[width=\linewidth]{figures/supp_pred.pdf}
    \caption{Retrieval and generation results for egocentric future state prediction. Correct and wrong retrieval images are marked with green and red boundaries, respectively.}
    \label{fig:supp pred}
\end{figure*}

\begin{figure*}[t]
    \centering
    \includegraphics[width=0.9\linewidth]{figures/motion_prediction.pdf}
    \vspace{-0.5mm}
    \caption{Motion prediction results in scenes with minor changes in observation.}
    \vspace{-1.5mm}
    \label{fig:motion_prediction}
\end{figure*}

\subsection{Finetuning OpenSora on Ego-Exo4D}

OpenSora V1.1~\cite{opensora}, initially trained on internet videos and images, produces severely inconsistent results when directly applied to infer future videos on the Ego-Exo4D dataset, as illustrated in Figure~\ref{fig:opensora_finetune}.
To address the gap between general internet content and egocentric video data, we fine-tune the official checkpoint on the Ego-Exo4D training set for 50 epochs.
OpenSora V1.1 proposed a random mask strategy during training to enable video generation by image and video conditioning. We adopted the default masking rate, which applies: 75\% with no masking, 2.5\% with random masking of 1 frame to 1/4 of the total frames, 2.5\% with masking at either the beginning or the end for 1 frame to 1/4 of the total frames, and 5\% with random masking spanning 1 frame to 1/4 of the total frames at both the beginning and the end.

As shown in Fig.~\ref{fig:opensora_finetune}, despite being trained on a large dataset, OpenSora struggles to generalize to the Ego-Exo4D dataset, producing future video frames with minimal consistency relative to the conditioning frame. While fine-tuning improves temporal consistency, the moving trajectories of objects like the basketball and soccer ball still deviate from realistic physical laws. Compared with our feature space prediction results, this suggests that training world models in a reconstructive latent space is more challenging than training them in a feature space.


\section{Additional Results on 3D Human Motion Prediction}

We present additional qualitative results for the 3D human motion prediction task, highlighting a particularly challenging scenario where egocentric observations exhibit minimal variation. This scenario poses significant difficulties for video-conditioned motion prediction, as the model must effectively capture and interpret subtle changes. As demonstrated in Fig.~\ref{fig:motion_prediction}, EgoAgent successfully generates accurate predictions that closely align with the ground truth motion, showcasing its ability to handle fine-grained temporal dynamics and nuanced contextual cues.

\section{OpenSora for Image Classification}

In this section, we detail the process of extracting features from OpenSora V1.1~\cite{opensora} (without fine-tuning) for an image classification task. Following the approach of~\cite{xiang2023denoising}, we leverage the insight that diffusion models can be interpreted as multi-level denoising autoencoders. These models inherently learn linearly separable representations within their intermediate layers, without relying on auxiliary encoders. The quality of the extracted features depends on both the layer depth and the noise level applied during extraction.


\begin{table}[h]
\centering
\caption{$k$-NN evaluation results of OpenSora V1.1 features from different layer depths and noising scales on ImageNet-100. Top1 and Top5 accuracy (\%) are reported.}
\label{tab:opensora-knn}
\resizebox{0.95\linewidth}{!}{%
\begin{tabular}{lcccccc}
\toprule
\multirow{2}{*}{Timesteps} & \multicolumn{2}{c}{First Layer} & \multicolumn{2}{c}{Middle Layer} & \multicolumn{2}{c}{Last Layer} \\
\cmidrule(r){2-3}   \cmidrule(r){4-5}  \cmidrule(r){6-7}  & Top1           & Top5           & Top1            & Top5           & Top1           & Top5          \\
\midrule
32        &  6.10           & 18.20             & 34.04               & 59.50             & 30.40             & 55.74             \\
64        & 6.12              & 18.48              & 36.04               & 61.84              & 31.80         & 57.06         \\
128       & 5.84             & 18.14             & 38.08               & 64.16              & 33.44       & 58.42 \\
256       & 5.60             & 16.58              & 30.34               & 56.38              &28.14          & 52.32        \\
512       & 3.66              & 11.70            & 6.24              & 17.62              & 7.24              & 19.44  \\ 
\bottomrule
\end{tabular}%
}
\end{table}

As shown in Table~\ref{tab:opensora-knn}, we first evaluate $k$-NN classification performance on the ImageNet-100 dataset using three intermediate layers and five different noise scales. We find that a noise timestep of 128 yields the best results, with the middle and last layers performing significantly better than the first layer.
We then test this optimal configuration on ImageNet-1K and find that the last layer with 128 noising timesteps achieves the best classification accuracy.

\section{Data Preprocess}
For egocentric video sequences, we utilize videos from the Ego-Exo4D~\cite{grauman2024ego} and WT~\cite{venkataramanan2023imagenet} datasets.
The original resolution of Ego-Exo4D videos is 1408×1408, captured at 30 fps. We sample one frame every five frames and use the original resolution to crop local views (224×224) for computing the self-supervised representation loss. For computing the prediction and action loss, the videos are downsampled to 224×224 resolution.
WT primarily consists of 4K videos (3840×2160) recorded at 60 or 30 fps. Similar to Ego-Exo4D, we use the original resolution and downsample the frame rate to 6 fps for representation loss computation.
As Ego-Exo4D employs fisheye cameras, we undistort the images to a pinhole camera model using the official Project Aria Tools to align them with the WT videos.

For motion sequences, the Ego-Exo4D dataset provides synchronized 3D motion annotations and camera extrinsic parameters for various tasks and scenes. While some annotations are manually labeled, others are automatically generated using 3D motion estimation algorithms from multiple exocentric views. To maximize data utility and maintain high-quality annotations, manual labels are prioritized wherever available, and automated annotations are used only when manual labels are absent.
Each pose is converted into the egocentric camera's coordinate system using transformation matrices derived from the camera extrinsics. These matrices also enable the computation of trajectory vectors for each frame in a sequence. Beyond the x, y, z coordinates, a visibility dimension is appended to account for keypoints invisible to all exocentric views. Finally, a sliding window approach segments sequences into fixed-size windows to serve as input for the model. Note that we do not downsample the frame rate of 3D motions.

\section{Training Details}
\subsection{Architecture Configurations}
In Table~\ref{tab:arch}, we provide detailed architecture configurations for EgoAgent following the scaling-up strategy of InternLM~\cite{team2023internlm}. To ensure the generalization, we do not modify the internal modules in InternML, \emph{i.e.}, we adopt the RMSNorm and 1D RoPE. We show that, without specific modules designed for vision tasks, EgoAgent can perform well on vision and action tasks.

\begin{table}[ht]
  \centering
  \caption{Architecture configurations of EgoAgent.}
  \resizebox{0.8\linewidth}{!}{%
    \begin{tabular}{lcc}
    \toprule
          & EgoAgent-300M & EgoAgent-1B \\
          \midrule
    Depth & 22    & 22 \\
    Embedding dim & 1024  & 2048 \\
    Number of heads & 8     & 16 \\
    MLP ratio &    8/3   & 8/3 \\
    $\#$param.  & 284M & 1.13B \\
    \bottomrule
    \end{tabular}%
    }
  \label{tab:arch}%
\end{table}%

Table~\ref{tab:io_structure} presents the detailed configuration of the embedding and prediction modules in EgoAgent, including the image projector ($\text{Proj}_i$), representation head/state prediction head ($\text{MLP}_i$), action projector ($\text{Proj}_a$) and action prediction head ($\text{MLP}_a$).
Note that the representation head and the state prediction head share the same architecture but have distinct weights.

\begin{table}[t]
\centering
\caption{Architecture of the embedding ($\text{Proj}_i$, $\text{Proj}_a$) and prediction ($\text{MLP}_i$, $\text{MLP}_a$) modules in EgoAgent. For details on module connections and functions, please refer to Fig.~2 in the main paper.}
\label{tab:io_structure}
\resizebox{\linewidth}{!}{%
\begin{tabular}{lcl}
\toprule
       & \multicolumn{1}{c}{Norm \& Activation} & \multicolumn{1}{c}{Output Shape}  \\
\midrule
\multicolumn{3}{l}{$\text{Proj}_i$ (\textit{Image projector})} \\
\midrule
Input image  & -          & 3$\times$224$\times$224 \\
Conv 2D (16$\times$16) & -       & Embedding dim$\times$14$\times$14    \\
\midrule
\multicolumn{3}{l}{$\text{MLP}_i$ (\textit{State prediction head} \& \textit{Representation head)}} \\
\midrule
Input embedding  & -          & Embedding dim \\
Linear & GELU       & 2048          \\
Linear & GELU       & 2048          \\
Linear & -          & 256           \\
Linear & -          & 65536     \\
\midrule
\multicolumn{3}{l}{$\text{Proj}_a$ (\textit{Action projector})} \\
\midrule
Input pose sequence  & -          & 4$\times$5$\times$17 \\
Conv 2D (5$\times$17) & LN, GELU   & Embedding dim$\times$1$\times$1    \\
\midrule
\multicolumn{3}{l}{$\text{MLP}_a$ (\textit{Action prediction head})} \\
\midrule
Input embedding  & -          & Embedding dim$\times$1$\times$1 \\
Linear & -          & 4$\times$5$\times$17     \\
\bottomrule
\end{tabular}%
}
\end{table}


\subsection{Training Configurations}
In Table~\ref{tab:training hyper}, we provide the detailed training hyper-parameters for experiments in the main manuscripts.

\begin{table}[ht]
  \centering
  \caption{Hyper-parameters for training EgoAgent.}
  \resizebox{0.86\linewidth}{!}{%
    \begin{tabular}{lc}
    \toprule
    Training Configuration & EgoAgent-300M/1B \\
    \midrule
    Training recipe: &  \\
    optimizer & AdamW~\cite{loshchilov2017decoupled} \\
    optimizer momentum & $\beta_1=0.9, \beta_2=0.999$ \\
    \midrule
    Learning hyper-parameters: &  \\
    base learning rate & 6.0E-04 \\
    learning rate schedule & cosine \\
    base weight decay & 0.04 \\
    end weight decay & 0.4 \\
    batch size & 1920 \\
    training iters & 72,000 \\
    lr warmup iters & 1,800 \\
    warmup schedule & linear \\
    gradient clip & 1.0 \\
    data type & float16 \\
    norm epsilon & 1.0E-06 \\
    \midrule
    EMA hyper-parameters: &  \\
    momentum & 0.996 \\
    \bottomrule
    \end{tabular}%
    }
  \label{tab:training hyper}%
\end{table}%

\clearpage


\end{document}
