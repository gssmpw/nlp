\clearpage

\appendix
\renewcommand\thefigure{A\arabic{figure}}
\renewcommand\thetable{A\arabic{table}}  
\renewcommand\theequation{A\arabic{equation}}
\setcounter{section}{0}
\setcounter{equation}{0}
\setcounter{table}{0}
\setcounter{figure}{0}

\setcounter{page}{1}
\maketitlesupplementary


\section{Tolerances sampler}
\label{sec:supp}
In this section, we present more implementation details of our tolerance sampler and its integration into the ray tracing. We model four common kinds of tolerances: decentration, tilt, central thickness, and curvature errors, and the specific tolerance ranges we used see \cref{tab:tolr_range}. Considering the perturbations arising from tolerances in the design parameters, we can express the parameters as follows:
\begin{equation}
\theta_{real} = \theta_{ideal} + \theta_{\Delta},
\end{equation}
where $\theta_{ideal}$ is the design parameters from pretrained without considering tolerances, $\theta_{\Delta}$ is the perturbations come from random tolerances, $\theta_{real}$ is the optics parameters after fabrication. 

We firstly random sample tolerances from given ranges, thus, $\theta_{\Delta}$ obeys the normal distribution, $\theta_{\Delta} \sim \mathcal{N}(0, \frac{max^{2}}{9})$, so that the range of tolerances is approximately 3 times the standard deviation of the normal distribution, \eg, for decentration, $max=0.04mm$, and we clamp the tolerances range if sampled tolerance values exceed the $max$. Secondly, for each individual lens (or double glued structure) in the lens system, we independently sample four tolerances.

Given that we are conducting ray tracing on a surface-by-surface basis, we convert the sampled per-lens tolerances into the necessary spatial transformations and curvature offsets for each surface. First, we categorize the tolerances into two types: spatial transformations, which include decentration, tilt, and central thickness error, and curvature errors, which can be implemented as offsets in the ray tracing process. Notably, both decentration and central thickness errors are consolidated into a single translation of the lens surfaces, expressed as translation vector $\boldsymbol{T} = [\Delta X, \Delta Y, \Delta Z]^T$.  While tilt is represented as a rotation of the lens surfaces, expressed as Rotation matrix, $\boldsymbol{R} = \boldsymbol{R}_z(\gamma)\cdot \boldsymbol{R}_y(\beta)\cdot \boldsymbol{R}_x(\alpha)$, more specific:
\begin{align}
\boldsymbol{R}_x(\alpha) &= \begin{bmatrix}
1 & 0 & 0 \\
0 & \cos{\alpha} & \sin{\alpha} \\
0 & \sin{\alpha} & \cos{\alpha} \\
\end{bmatrix}, \\
\boldsymbol{R}_y(\beta) &= \begin{bmatrix}
\cos{\beta} & 0 & \sin{\beta} \\
0 & 1 & 0 \\
-\sin{\beta} & 0 & \cos{\beta}
\end{bmatrix}, \\
\boldsymbol{R}_z(\gamma) &= \begin{bmatrix}
\cos{\gamma} & -\sin{\gamma} & 0 \\
\sin{\gamma} & \cos{\gamma} & 0 \\
0 & 0 & 1
\end{bmatrix}.
\end{align}
where $\alpha, \beta, \gamma$ are sampled tilt degrees from normal distribution, around three axes.

\begin{table}
\centering
\caption{Specific tolerance ranges use in our paper. Decentration and central thickness error tolerances are in \emph{millimeters}, curvature error is in \emph{percentrage} and tilt is in \emph{degree}.}
\begin{tabular}{c|c}
\hline
& TOLR Range \\
\hline
Decenteration & $(-0.04, 0.04)$ \\
Tilt & $(-0.05, 0.05)$ \\
Central thickness error & $(-0.04, 0.04)$ \\
Curvature error & $(-0.3\%, +0.3\%)$ \\
\hline
\end{tabular}
\label{tab:tolr_range}
\end{table}

As we convert the surface transformations into equivalent coordinate system transformations, we need to invert the signs of the formulas above. In the ray tracing process, we first apply the coordinate system transformations in the order of translation followed by rotation. We then calculate the intersection point between the ray and the surface, incorporating the curvature offset to finalize the application of Snell's law. Finally, we revert the coordinate system back to its original configuration.

\section{Deep optics pipeline}
In this section, we present more implementation details for deep optics pipeline in both training and testing.

\subsection{Rendering by PSF map}
Rather than rendering the imaging result pixel by pixel, we first generate the Point Spread Function (PSF) Map through ray tracing. Once the PSF Map is obtained, we compute the camera imaging result using spatially-variant convolution between the PSF Map and the sharp image, illustrated by \cref{fig:spatially-conv}.

\begin{figure*}[tb] \centering
  \includegraphics[width=0.9\textwidth,height=0.32\textwidth]{figures/spatially-variant-convolution.pdf}
  \caption{Initially, the original image is populated with reflections. Subsequently, images at various field-of-view positions within the complete large image are convolved using the point spread function (PSF) corresponding to each field-of-view as the convolution kernel. Finally, the convolved images are integrated to produce a cohesive composite.} 
  \label{fig:spatially-conv}
\end{figure*}

\begin{figure*}[tb] \centering
  \includegraphics[width=0.8\textwidth,height=0.32\textwidth]{figures/combined_PSF.pdf}
  \caption{For each local field of view (FoV), a random tolerance pattern is sampled, and the local Point Spread Function (PSF) is obtained through differentiable ray tracing. These local PSFs are then stitched together to create a PSF map that incorporates multiple tolerance patterns.} 
  \label{fig:combine_PSF}
\end{figure*}

In this paper, we adopt the differentiable PSF map method from \cite{yang2024curriculum}. In our experiments, we utilize $51 \times 51$ PSF kernel size and $8\times 8$ PSF grid map.

\subsection{Tolerance optimization and evaluation}
\textbf{Tolerance optimization.} During the tolerance optimization process, we need to simultaneously sample various tolerance modes and perform ray tracing. To mitigate computational overhead, we conduct local ray tracing within a distinct field of view for each sampled tolerance mode, allowing us to obtain the PSF at specific local field of view angles. In our experiments, we sample 64 tolerance patterns, rendering a local PSF for each pattern. Finally, we combine all 64 local PSFs into an integrated PSF map that encapsulates the effects of all tolerance patterns, demonstrated in \cref{fig:combine_PSF}. It is important to note that we utilize resized images ($256 \times 256$) from the DIV2K dataset \cite{agustsson2017ntire} during the training process to mitigate memory overhead, batch size is set in 64. During tolerance optimization, we employ each localized PSF from the PSF map alongside a $256 \times 256$ image for convolution to obtain simulated imaging results. Finally, we add $1\%$ Gaussian noise to simulate sensor noise, as shown in \cref{fig:pipeline}.

\vspace{0.8em}
\noindent\textbf{Evaluation.} During evaluation, in contrast to the training phase, we directly utilize $2048 \times 2048$ images along with a PSF map obtained from ray tracing, applying spatially-variant convolution to generate simulated imaging results, batch size is set in 2. We sampled randomized tolerances multiple times, and for each sampled tolerance pattern, we computed the PSNR, SSIM and LPIPS, across the entire test dataset. 

We visualize the frequency histograms of the complete test results for Lens 1 and Lens 2, see \cref{fig:distribution}.
We provide a visual flipping comparison in the local web page, please refer to the file \textit{visualization.html} in the supplement.

\begin{figure*}[tb] \centering
  \includegraphics[width=\textwidth,height=0.64\textwidth]{figures/supp_distribution.pdf}
  \caption{Top: 100 times of random tolerances test for Lens1. Bottom: 200 times of random tolerances test for Lens2. The frequency distribution of the overall test indicates that the deep optics design, following tolerance-aware optimization, is significantly less impacted by tolerances.} 
  \label{fig:distribution}
\end{figure*}
% --------------------------------------------------------------------
\section{Lens1 and Lens2 parameters}
In this section, we list all parameters of lens1 and lens2 in \cref{tab:lens1_param} and \cref{tab:lens2_param}, and the layouts for the two lenses are as shown in \cref{fig:lens1_layout} and \cref{fig:lens2_layout}.

\begin{figure*}[tb]
\centering
  \includegraphics[width=0.7\textwidth,height=0.25\textwidth]{figures/Lens1_v1.pdf}
  \caption{Lens1 layout.} 
  \label{fig:lens1_layout}
\end{figure*}

\begin{table*}
\centering
\caption{The lens1 parameters used in our paper.}
\label{tab:lens1_param}
\begin{tabular}{cccccc}
\toprule
Surface No. & Radius (mm) & Distance (mm) & Diameter (mm) & $n_d$ & $\nu_d$ \\
\toprule
OBJ & INFINITY & 1000.0 & INFINITY & AIR &  \\
1 & 22.01 & 3.26 & 19.0 & 1.620410 & 60.323649 \\
2 & -435.76 & 6.01 & 19.0 & AIR & \\
3 & -22.21 & 1.0 & 10.0 & 1.620040 & 36.376491 \\
4 & 20.29 & 1.0 & 10.0 & AIR & \\
STO & INFINITY & 4.75 & 10.0 & AIR & \\
5 & 79.68 & 2.95 & 15.0 & 1.620410 & 60.323649 \\
6 & -18.40 & 41.55 & 15.0 & AIR & \\
IMA & INFINITY & - & 10.14 & AIR & \\
\bottomrule
\end{tabular}
\end{table*}

\begin{figure*}[tb]
\centering
  \includegraphics[width=0.7\textwidth,height=0.25\textwidth]{figures/Lens2_v1.pdf}
  \caption{Lens2 layout.} 
  \label{fig:lens2_layout}
\end{figure*}

\begin{table*}[htp]
\centering
\caption{The lens2 parameters used in our paper.}
\label{tab:lens2_param}
\begin{tabular}{cccccc}
\toprule
Surface No. & Radius (mm) & Distance (mm) & Diameter (mm) & $n_d$ & $\nu_d$ \\
\toprule
OBJ & INFINITY & 2000.0 & INFINITY & AIR &  \\
1 & 37.789 & 6.544 & 42.0 & 1.6400 & 60.20 \\
2 & 23.941 & 18.170 & 35.984 & AIR & \\
3 & 27.340 & 8.800 & 34.400 & 1.7880 & 47.49 \\
4 & 2541.820 & 2.432 & 34.400 & AIR & \\
STO & INFINITY & 3.800 & 23.200 & AIR & \\
5 & -50.594 & 4.000 & 24.640 & 1.7552 & 27.53 \\
6 & 28.004 & 3.994 & 30.862 & AIR & \\
7 & 149.795 & 8.285 & 34.400 &  1.6400 & 60.20 \\
8 & -28.474 & 0.832 & 34.400 & AIR & \\
9 & 33.206 & 9.275 & 34.400 &  1.6584 & 50.85 \\
10 & 47.804 & 35.074 & 29.560 & AIR & \\
IMA & INFINITY & - & 12.454 & AIR & \\
\bottomrule
\end{tabular}
\end{table*}