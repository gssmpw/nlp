\section{Related work}
There are several other works that deal not with the diameter, but rather with the problem of computing the \emph{distance} of two given vertices of some polytope $P$.
In the case of the distance this problem was shown to be NP-hard for the bipartite perfect matching polytope $P_G$ independently by Aichholzer et al.\ \cite{DBLP:journals/algorithmica/AichholzerCHKMS21} and Ito et al.\ \cite{DBLP:journals/siamdm/ItoKKKO22}. Additionally, Borgward et al.\ \cite{borgwardt2025hardness} show NP-hardness of computing circuit distances in 0/1-network flow polytopes, and Cardinal and Steiner show the same for polymatroids \cite{cardinal2023shortest}.
These results were strengthened by Cardinal and Steiner \cite{DBLP:conf/ipco/CardinalS23} to a $O(\log n / \log \log n)$-hardness of approximation result for computing a short path between two 
vertices of distance two on the bipartite perfect matching polytope, assuming ETH.
As explained in \cite{nobel2025complexity}, the techniques for handling polytope diameters and distances are very different, since distances can be very short (which is often helpful for proofs), 
but diameters are usually large.

Finally, a related series of work deals with \emph{combinatorial reconfiguration} of perfect matchings \cite{DBLP:journals/tcs/ItoDHPSUU11,kaminski2012complexity,DBLP:conf/sofsem/GuptaKM19,DBLP:conf/wg/BousquetHIM19,bonamy2019perfect, binucci2025flipping}. 
In these papers, the notion of adjacency between two perfect matchings differs from the adjacency on the bipartite perfect matching polytope considered in the present paper.