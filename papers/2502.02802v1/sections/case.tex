\section{Case Study}
\label{app:case}

In this section, we present the generated counseling sessions using simulated clients of different methods. First, Table~\ref{tab:real conv} and Table~\ref{tab:real profile} show the real conversation and the extracted profile, respectively.

Table~\ref{tab:example-based conv} shows the session generated by the example-based method. In this case, the simulated client is motivated by avoiding trouble situations and plans to distance from drinking environments. This is inconsistent with the real client's motivation related to sports and non-existent preferred plan to change. 

Table~\ref{tab:profile-based conv} shows the session generated by the profile-based method. In this case, the client proactively mentions motivation and knowledge about the risks of drinking, which is inconsistent with the real client's profile. Moreover, the simulated client plans to refrain from alcohol, which is not mentioned in the given profile. 

Table~\ref{tab:action-based conv} shows the session generated by the action-based method. The simulated client in this session is also motivated by an inconsistent reason and adopt a non-existent preferred plan. Although the action-based prompt provides action options, it still results in a longer counseling session. The topic about friends is also inconsistent with the given profile. 

Finally, Table~\ref{tab:our conv} shows the session generated by our framework-based method. The motivation of the simulated client is consistent with the given profile, and the client does not express the motivation on its own. Moreover, the simulated client ends the conversation proactively instead of proposing a plan, which is aligned with the given profile.

Incidentally, these session examples also reveal the common shortcomings of LLM-based counselors which almost always employ a reflection followed by a question, which may not be effective in counseling. To make progress, the simulated counselor may incorrectly determine the client's state and ask the client direct questions about motivation or plans.

\begin{table*}[tb]
\begin{tabularx}{\textwidth}{X}
\toprule
Counselor: Hi, Nick. Thank you for completing the CRAFFT questionnaire. I also appreciate you sharing some information about yourself. \newline Client: You're welcome. \newline Counselor: I see from the questionnaire, did you have got in trouble while using alcohol? Can you tell me more about that? \newline Client: Well, a friend and I would snuck out one night to go to the park, and we got a couple of beers from his house, but the cop came by, and so we threw them in the bushes. He was originally gonna give us a curfew ticket but ended up not to. Luckily, he didn't see the beer. \newline Counselor: That must be really scary. \newline Client: Yeah, it was. If my mom would have found out, she would've killed me. \newline Counselor: I understand. So tell me, how often do you drink? \newline Client: Well, I've only tried drinking once or twice. \newline Counselor: I'm gonna make sure I understand. You have tried drinking a couple of times, and you almost got in trouble once. Did I get that right? \newline Client: Yeah, that's right. \newline Counselor: As your provider, I'm a little concerned about your health. Drinking alcohol could harm you, and I would like to share some information with you. Is that okay? \newline Client: Well, I'm not sure if this information is for me, since I've only been drinking once or twice. \newline Counselor: I understand. You're just experimenting with alcohol. Then, how about if you share with me what the effects of alcohol are? \newline Client: Well, all I really know is that it's not good for me. \newline Counselor: Are you surprised what that might be true? \newline Client: Yeah, and a couple of my friends drink too. \newline Counselor: Well, you might not be drinking that much, and other kids are also trying alcohol. I'd like to share with you the risk of using. Alcohol and drugs could really harm you because your brain is still changing. It also-- you're very high risk for becoming addicted. Alcohol and drugs could also interfere with your role in life and your goals, especially in sports, and it could cause unintended sex. How do you feel about this information? \newline Client: I really didn't think about how this would affect me playing sports. \newline Counselor: Based on your reason for your visit today, I feel like you really wanted to play soccer, and you have great goals about that. \newline Client: Yeah, I would like to play soccer in college. \newline Counselor: That's a great goal to have for yourself. \newline Client: Yeah, I would also like to get scholarships. \newline Counselor: Scholarship will really help you achieving your goals. \newline Client: Yeah. \newline Counselor: I would like to share with you these materials to show the risk of drinking at younger age. I also wanted to encourage you to think about these risks. What do you think about that? \newline Client: Well, I will take a look at it, and I'll definitely read it, and I don't wanna become addicted, and I would like to continue to play sports as long as possible. \newline Counselor: It sounds like you're in the right track, looks like you care about your health, and you care about your goals in life. I'm here to answer any questions. \newline Client: Thanks for all the information. \newline Counselor: You're welcome.  \\ \bottomrule
\end{tabularx}
\caption{A real conversation between Counselor and Client in AnnoMI.}
\label{tab:real conv}
\end{table*}

\begin{table*}[tb]
\begin{tabularx}{\textwidth}{X}
\toprule
Behavioral Problem: Drinking \newline Receptivity: 3 \newline Personas: \newline - The client sneaked out with a friend to the park to has a couple of beers. During this incident, a police officer came by, but he managed to avoid getting into trouble as he threw the beers away before the policy saw them. \newline - If the client's mother found out he was in trouble because of his drinking, she would kill the client. \newline - The client has an interest in soccer and have set a goal to play in college. \newline - The client has aspirations of receiving scholarships to support his soccer career. \newline Motivation: \newline - The client is motivated because alcohol would affect the client playing sports. \newline Beliefs: \newline - The client can handle the dangerous situation to avoid getting into trouble. \newline - The client believes that he doesn't have a problem with alcohol because he has only drunk once or twice. \newline - The client believes that drinking is normal because a couple of the client’s friends drink too. \newline Acceptable Plans: None  \\ \bottomrule
\end{tabularx}
\caption{The extracted profile of the real conversation.}
\label{tab:real profile}
\end{table*}


\begin{table*}[tb]
\begin{tabularx}{\textwidth}{X}
\toprule
Counselor: Hello. How are you? \newline 
Client: I am good. What about you? \newline 
Counselor: I'm here to support you. Can you tell me more about what brings you in today? \newline 
Client: I recently had a close call with the police while drinking at the park with a friend. \newline 
Counselor: It sounds like that was a scary experience for you. Can you share more about how you're feeling about what happened? \newline 
Client: I was really scared when the police officer came by, but I managed to avoid getting into trouble. \newline 
Counselor: It's understandable to feel scared in that situation. It's great that you were able to avoid any serious consequences. Have you thought about what might have happened if things had gone differently? \newline 
Client: If the police had caught us, it could have been a lot worse. My mom would have been furious if she found out. \newline 
Counselor: It sounds like you're considering the potential consequences of your actions. How do you feel about your mom's reaction if she had found out? \newline 
Client: My mom would definitely not be happy if she found out about me drinking. She would be really upset and disappointed in me. \newline 
Counselor: It seems like you're concerned about how your mom would react. What are your thoughts on this situation and how it's affecting your relationship with her? \newline 
Client: I know my mom would be really disappointed if she found out. It would definitely strain our relationship, and I don't want that to happen. \newline 
Counselor: It sounds like you care a lot about your relationship with your mom and want to avoid any strain on it. What are some ways you think you could prevent this from happening in the future? \newline 
Client: {\color{red} I think one way to prevent this from happening again is to avoid situations where drinking might be involved (Inconsistent with Motivation)} . \newline  
Counselor: That's a proactive step you're considering. What specific steps do you think you can take to avoid these situations? \newline  
Client: {\color{red} I could make plans with friends that don't involve drinking, and if I find myself in a situation where drinking is present, I can choose to remove myself from that environment. (Inconsistent with Plan)} 
\\ \bottomrule
\end{tabularx}
\caption{A generated conversation from simulated example-based client.}
\label{tab:example-based conv}
\end{table*}

\begin{table*}[tb]
\begin{tabularx}{\textwidth}{X}
\toprule
Counselor: Hello. How are you? \newline Client: I am good. What about you? \newline
Counselor: I'm well, thank you for asking. What brings you in today? \newline Client: I wanted to talk about some concerns I have regarding alcohol and its effects. \newline
Counselor: It sounds like you've been reflecting on your relationship with alcohol. Can you share a bit more about what specific concerns you have? \newline Client: Well, I've only tried drinking once or twice, but after some recent events, I've been thinking about the potential risks and how it could impact my goals in life. \newline
Counselor: It's great to hear you're considering how drinking might align with your life goals. What are some of those goals you feel could be impacted? \newline
Client: I'm really passionate about playing soccer and aiming to secure scholarships to play in college. {\color{red} I've realized that drinking could potentially harm my athletic performance and jeopardize my chances of achieving my goals.} {\color{red} (Self-Motivated)} \newline
Counselor: It’s commendable that you're looking ahead and evaluating how your choices affect your dreams. How do you feel about these realizations impacting your decisions about drinking in the future? \newline Client: I feel determined to prioritize my goals and make responsible choices that align with my aspirations. It means {\color{red} refraining from alcohol} to safeguard my future in soccer. {\color{red} (Inconsistent with Plan)} \\ \bottomrule
\end{tabularx}
\caption{A generated conversation from simulated profile-based client.}
\label{tab:profile-based conv}
\end{table*}


\begin{table*}[tb]
\begin{tabularx}{\textwidth}{X}
\toprule
Counselor: Hello. How are you? \newline
Client: I am good. What about you? \newline
Counselor: I'm here to support you. Can you tell me more about what brings you in today? \newline
Client: I got into a situation with some beers at the park with a friend, but we managed to avoid trouble. \newline
Counselor: It sounds like you had a challenging experience. How do you feel about what happened? \newline
Client: I handled the situation well by throwing the beers away before the police saw us. \newline
Counselor: It seems like you were able to make a quick decision in the moment. What are your thoughts on how you handled the situation? \newline
Client: I think I made the right choice in the moment to avoid getting into trouble. \newline
Counselor: It sounds like you value staying out of trouble. Can you think of any other ways to handle similar situations in the future? \newline
Client: I guess I could be more careful about where and when I decide to have a drink with friends. \newline
Counselor: It sounds like you're considering making some changes in your socializing habits. What are your reasons for wanting to be more careful about when and where you drink with friends? \newline
Client: {\color{red} I understand that being more cautious about my drinking habits can help me avoid risky situations in the future. (Inconsistent with Motivation \& Plan)} \newline
Counselor: It's great that you're recognizing the benefits of being more cautious. How do you think this change could positively impact your life? \newline
Client: I believe being more cautious about drinking can help me to avoid such situation. \newline
Counselor: It's great that you're thinking about how your choices can help you to avoid such situation. Are there any concerns or challenges you anticipate in being more cautious about drinking? \newline
Client: I'm worried that some of my friends who drink might not understand why I'm being more cautious and could {\color{red} pressure me to join them}. {\color{red} (Inconsistent with Personas)} \newline
Counselor: It's understandable to be concerned about how your friends might react to your decision. How do you think you could handle situations where your friends pressure you to drink? \newline
Client: I guess I could try to explain to them why I'm being more cautious and ask for their support in my decision. \newline
Counselor: It sounds like communication and seeking support from your friends could be helpful. How do you feel about having that conversation with them? \newline
Client: I hesitate because I'm not sure how they will react to my decision to be more cautious about drinking. \newline
Counselor: It's normal to feel hesitant about how your friends might respond. What are some ways you could address your concerns about their reactions before having that conversation? \newline
Client: I could try to choose the right time and place to have that conversation with my friends, maybe when we're all relaxed and not in a drinking environment. \newline
Counselor: That sounds like a thoughtful approach. Planning the conversation carefully could help make it more productive. How do you feel about taking this step towards being more cautious about drinking with your friends? \newline
Client: {\color{red} I plan to talk to my friends about being more cautious about drinking and hope they will understand and support my decision. (Inconsistent with Plan)} \\ \bottomrule
\end{tabularx}
\caption{A generated conversation from simulated action-based client.}
\label{tab:action-based conv}
\end{table*}


\begin{table*}[tb]
\begin{tabularx}{\textwidth}{X}
\toprule
Counselor: Hello. How are you?  \newline
Client: {\color{blue} [State: The client is unaware of or underestimates the need for change. Action: The client interacts politely with the counselor, through greetings, expressions of gratitude, or comments on the counselor's views.] }I am good. What about you? \newline
Counselor: I'm well, thank you for asking. What brings you in today? \newline
Client: {\color{blue} [State: The client is unaware of or underestimates the need for change. Action: The client shares details about their background, experiences, or emotions. Information: The client sneaked out with a friend to the park to has a couple of beers. During this incident, a policeofficer came by, but he managed to avoid getting into trouble as he threw the beers away before the policy saw them.] }Well, I had an incident with a friend a few days ago. We sneaked out to the park for a couple of beers. Suddenly, a police officer was nearby, but luckily we managed to throw the beers away before he saw them. We absolutely got away with it! \newline
Counselor: It sounds like that was quite an experience for you. How did you feel about the situation afterwards?  \newline
Client: {\color{blue} [State: The client is unaware of or underestimates the need for change. Action: The client downplays the importance or impact of their behavior or situation. Information: The client can handle the dangerous situation to avoid getting into trouble.]} Honestly, I didn't think much of it. I'm pretty good at handling such situations and avoiding trouble. I think it was nothing too serious. \newline
Counselor: It's good to hear that you felt in control of the situation. Do you find yourself often in similar situations where you need to act quickly to avoid trouble?  \newline
Client: {\color{blue} [State: The client is unaware of or underestimates the need for change. Action: The client directly refuses to admit their behavior is problematic or needs change.]} This situation was more of an exception. I do not often get in such sorts of troubles. And I don't really think there's any problem with my behavior. Occasional drinking doesn't harm anyone, does it? \newline
Counselor: It sounds like you see these occasions as just a part of life without much risk. When you do choose to have a drink, what are some of the reasons you find it enjoyable or beneficial?  \newline
Client: {\color{blue} [State: The client is unaware of or underestimates the need for change. Action: The client attributes their issues to external factors, such as stressful life or other people. Information: The client believes that drinking is normal because a couple of the client’s friends drink too.]} A couple of my friends tend to drink, and I find it socially easier to fit in when I do the same. I don't think it's a big deal as I'm not hurting anyone. \newline
Counselor: It's understandable to seek relaxation and social comfort. How do you feel about your ways of unwinding and socializing? Do they generally give you a sense of satisfaction? 
\\ \bottomrule
\end{tabularx}
\caption{A generated conversation from our simulated client, Part 1. For better presentation, the instruction (in {\color{blue} blue}) is placed before the client's utterance, which should follow the counselor's utterance.}
\label{tab:our conv}
\end{table*}


\begin{table*}[tb]
\begin{tabularx}{\textwidth}{X}
\toprule
Client: {\color{blue} [State: The client is unaware of or underestimates the need for change. Action: The client interacts politely with the counselor, through greetings, expressions of gratitude, or comments on the counselor's views.]} Yes, they do. I feel fine about it. It's just a way of enhancing social interactions. And I must say, I appreciate your understanding. \newline
Counselor: I'm glad to hear that it works for you and that you feel understood. When thinking about the future, do you see yourself continuing with these same strategies, or are there new ways you might be curious to explore for dealing with stress and socializing?  \newline
Client: {\color{blue}[State: The client is unaware of or underestimates the need for change. Action: The client directly refuses to admit their behavior is problematic or needs change.].} Honestly, I don't see why I would need to change anything. My methods for socializing seem perfectly fine to me. I don't think there's any problem that needs addressing really. \newline
Counselor: It's great that you feel confident about your current methods. Reflecting on our overall well-being can sometimes open up new perspectives, even when things are going well. Are there any areas in your life where you’re looking to enhance your experiences or make any improvements?  \newline
Client: {\color{blue} [State: The client is unaware of or underestimates the need for change. Action: The client shares details about their background, experiences, or emotions. Information: The client has an interest in soccer and have set a goal to play in college.]} Well, I've been playing soccer for some time now and I have this goal of playing in college. I feel like it's a big part of what shapes me as a person, besides the friendships and the bonding, of course. \newline
Counselor: That sounds like a wonderful goal and a meaningful part of your life. How do you think your current habits support your aspirations of playing soccer in college?  \newline
Client: {\color{blue} [State: The client is unaware of or underestimates the need for change. Action: The client interacts politely with the counselor, through greetings, expressions of gratitude, or comments on the counselor's views.]} Well, I guess my social habits help me bond better with my teammates. And honestly, I can't see how they could be detrimental to my goal of playing college soccer. Is there something you think I should be concerned about? \newline
Counselor: It's great to hear how your social life positively impacts your team relationships. It's all about finding the right balance that supports your goals.  Have you noticed any impact on your training or energy levels after nights involving drinking? \newline
\\ \bottomrule
\end{tabularx}
\caption{A generated conversation from our simulated client, Part 2. For better presentation, the instruction (in {\color{blue} blue}) is placed before the client's utterance, which should follow the counselor's utterance.}
\end{table*}


\begin{table*}[tb]
\begin{tabularx}{\textwidth}{X}
\toprule
Client: {\color{blue} [State: The client acknowledges the need for change but remains ambivalent.  Action: The client highlight the importance, benefit or confidence to change. Information: The client is motivated because alcohol would affect the client playing sports.]} Now that you've mentioned it, I have noticed feeling less energetic during my training sessions after nights that involved drinking. I guess I never really thought about the connection there. It's definitely making me rethink my habits and how they can affect my sports performance. That really is important to me. \newline
Counselor: It's great that you're making these connections and thinking about how your choices affect your goals. It takes a lot of self-awareness to recognize and consider adjusting your habits. What are some steps you might consider to ensure your habits support your soccer ambitions more effectively? \newline
Client: {\color{blue}  [State: The client acknowledges the need for change but remains ambivalent.  Action: The client shows uncertainty, indicating ambivalence about change. Information: The client believes that he doesn’t have a problem with alcohol because he has only drunk once or twice.]} Well, I suppose there could be some merit to cutting back on the amount I drink. But since I've only been drinking on occasion, I'm not entirely sure if it's absolutely necessary for me to change. After all, it's not like I have a problem with alcohol or anything. \newline
Counselor: It's good to evaluate the scale and impact of your habits carefully. Moderation is key, and it sounds like you're considering how even occasional drinking might play a role in your overall performance and health. As you think about these adjustments, what kind of support do you think would be helpful for you in making such changes, if you decide to pursue them? \newline
Client: {\color{blue} [State: The client acknowledges the need for change but remains ambivalent.  Action: The client interacts politely with the counselor, such as greeting, thanking or ask questions.]} Well, I suppose there could be some merit to cutting back on the amount I drink. But since I've only been drinking on occasion, I'm not entirely sure if it's absolutely necessary for me to change. After all, it's not like I have a problem with alcohol or anything. \newline
Counselor: It's good to evaluate the scale and impact of your habits carefully. Moderation is key, and it sounds like you're considering how even occasional drinking might play a role in your overall performance and health. As you think about these adjustments, what kind of support do you think would be helpful for you in making such changes, if you decide to pursue them? \newline
Client: {\color{blue} [State: The client acknowledges the need for change but remains ambivalent. Action: The client highlights current state, expresses a desire to end the current session, and suggests further discussion be deferred to a later time.]} I'm not completely sure yet, I guess I'll need some time to think about what kind of support would be helpful. Perhaps we can revisit this in our next session? I think for now, I need some time to digest everything we've talked about. \newline
Counselor: Absolutely, taking some time to reflect on this is important. I'm here to support you through this process, and we can definitely explore this further in our next session. If any thoughts or questions come up in the meantime, feel free to bring them up when we meet again.
\\ \bottomrule
\end{tabularx}
\caption{A generated conversation from our simulated client, Part 3. For better presentation, the instruction (in {\color{blue} blue}) is placed before the client's utterance, which should follow the counselor's utterance.}
\end{table*}


\begin{table*}[tb]
\begin{tabularx}{\textwidth}{X}
\toprule
Counselor: Hi, How are you? \newline
Client: {\color{blue} [State: The client is unaware of or underestimates the need for change. Action: The client interacts politely with the counselor, through greetings, expressions of gratitude, or comments on the counselor’s views.]} I am good. What about you? \newline
Counselor: I'm here to support you today. What brings you here?  \newline
Client: {\color{blue} [State: The client is unaware of or underestimates the need for change. Action: The client shares details about their background, experiences, or emotions. Information: The client sneaked out with a friend to the park to has a couple of beers. During this incident, a police officer came by, but he managed to avoid getting into trouble as he threw the beers away before the policy saw them.]} Just hanging out, you know. The other day, I was at the park with a buddy, and we were just having a couple of beers. Nothing crazy. But then a cop showed up, and I had to ditch the beers quick. Luckily, he didn’t see anything so no trouble there. \newline
Counselor: It sounds like you managed the situation calmly. Can you share any situations where your drinking might have had an impact on your life, positive or negative? \newline 
Client: {\color{blue} [State: The client is unaware of or underestimates the need for change. Action: The client directly refuses to admit their behavior is problematic or needs change.]} Honestly, I really don’t think it’s had any major impact. Like I said, it’s very occasional, and I haven’t seen any real negatives from it. \newline
Counselor: It sounds like you haven't noticed any significant negative effects from your drinking. Have you ever thought about what benefits you might see from cutting back or stopping altogether?  \newline
Client: {\color{blue} [State: The client is unaware of or underestimates the need for change. Action: The client directly refuses to admit their behavior is problematic or needs change.]} I haven’t really thought about stopping since it’s not a frequent thing. I don't see how it would make a big difference for me to cut back or stop. \newline
Counselor: It's understandable that you might not see a need for change if you don't perceive any negative impacts. Can you imagine any potential benefits that might come from reducing your drinking, even if it's occasional?  \newline
Client: {\color{blue} [State: The client is unaware of or underestimates the need for change. Action: The client directly refuses to admit their behavior is problematic or needs change.]} Not really, I mean, I play soccer and I'm pretty good at it. I don’t think having a few beers now and then affects that, so I don’t see why I’d need to change anything. \newline
Counselor: It seems like you're feeling pretty confident in your current approach to drinking. Remember, I'm here whenever you're ready to talk more or if you have any other concerns. \newline
\\ \bottomrule
\end{tabularx}
\caption{A failure example from our simulated client. For better presentation, the instruction (in {\color{blue} blue}) is placed before the client's utterance, which should follow the counselor's utterance. The counselor fails to find out the motivation by asking self-exploration and ends the counseling.}
\label{tab:failure example}
\end{table*}



