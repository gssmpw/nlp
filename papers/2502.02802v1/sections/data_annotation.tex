\section{Data Annotation and Analysis}
\label{app:data annotation}

The analysis of both observed (i.e., AnnoMI) and generated MI counseling sessions requires them to be annotated at the session level so as to obtain the client's profile components (i.e., personas, beliefs, motivation, and acceptable change plan) and receptivity.  Tables~\ref{tab:profile annotation} and \ref{tab:receptivity annotation} show corresponding prompts.  We also develop prompts to annotate at a specific point of a session to obtain the client's state and action at that point (see Table~\ref{tab:state annotation} and \ref{tab:action annotation}).  Finally, we create the prompts to determine entailment between generated and ground truth profile components (see Table~\ref{tab:entailment annotation}).  
GPT-4, one of the top proprietary LLMs, is chosen to be the target LLM to perform annotation of the AnnoMI and generated sessions.

In the following, we report the analysis results after annotating the 86 selected AnnoMI sessions.

\paragraph{AnnoMI Dataset} We utilize the AnnoMI dataset~\citep{wu2022anno}, which comprises 133 conversations spanning a diverse range of behavior topics, including ``reducing alcohol consumption'' and ``smoking cessation.'' Each conversation was transcribed from an expert-generated demonstration video and subsequently labeled as demonstrating high- or low-quality MI based on the video title and description. The 110 dialogues illustrating high-quality MI, comprising over 8,800 utterances in total, are considered as candidates. Consequently, we eliminate the sessions that are incomplete and those involving clients with insufficient profile information. To be noticed, the AnnoMI dataset is compiled from YouTube videos depicting MI counseling sessions between actors. However, it remains widely utilized due to restricted access to real counseling sessions. Given the absence of ideal datasets, our framework offers a solution to the scarcity of data by simulating clients interacting with counselors, thus bypassing ethical and privacy issues. Additionally, the primary use of the AnnoMI dataset is to extract client profiles and corresponding counseling sessions for evaluation purposes. Although these sessions are not “real”, they are still consistent with the client profiles and serve as valuable evaluation dataset.

\begin{figure*}[t]
    \centering
    \includegraphics[width=\textwidth]{figs/receptivity.pdf}
    \caption{The distribution of receptivity and the relation between receptivity and sustain ratio and precontemplation.}
    \label{fig:suggest_annotation}
\end{figure*}

\begin{figure*}[tb]
    \centering
    \includegraphics[width=\textwidth]{figs/action.pdf}
    \caption{Proportions of actions for different receptivity scores in different states. Actions negatively associated with receptivity are represented in dashed lines.}
    \label{fig:action_distribution}
\end{figure*}

\paragraph{Analysis of receptivity, associated talk types, and client states.} As shown in Figure~\ref{fig:suggest_annotation}(a), most clients demonstrate moderate receptivity. Few clients have high and low receptivities.  We next analyse the client utterances labeled with talk-types ``change talk'', ``neutral'' and ``sustain talk''. utterances annotated with ``Change talk'' express an interest to change behavior, while those annotated with ``sustain talk'' are the opposite~\citep{hoang2024can, miller2012motivational}. Figure~\ref{fig:suggest_annotation}(b) shows that the ratio of sustain talk-labeled utterances has a negative relationship with receptivity, i.e., clients with higher receptivity tend to use neutral or change talk more often.  Figure~\ref{fig:suggest_annotation}(c) shows that the proportion of client utterances in Precontemplation state also has a negative relationship with receptivity as clients in this state are reluctant and/or the counselor is not able to effect behavior change. Clients with lower receptivity are harder to reach the contemplation state, thus requiring more effort from the counselor.

\paragraph{Analysis of receptivity and associated actions.} Figure~\ref{fig:action_distribution}(a) illustrates the proportion of actions in different client states across different receptivity scores. Considering the utterances of clients who are in the precontemplation state, the proportion of utterances annotated with Inform action increases with receptivity due to the more receptive clients providing information about themselves.  The proportion with Engage action also increases slightly.  On the other hand, the proportions of Deny, Downplay and Blame actions reduces as receptivity increases. Similar findings also apply to Figures~\ref{fig:action_distribution}(b) and (c).  For easy reading, the actions negatively associated with receptivity are shown in dashed lines.  %Throughout the three states, actions associated with sustain talk, such as Deny, Downplay, and Hesitate, consistently decrease as receptivity increases. Conversely, actions associated with change talk, such as Acknowledge, Accept, and Plan, increase with higher receptivity. For neutral talk, Inform consistently rises with increasing receptivity, although Engage shows unstable trends. 
Finally, Inform and Engage are two actions found in all the three states. Inform is most predominant in the Precontemplation state implying that clients are more likely to share profile information during Precontemplation and adopt more varied actions after they transit to Contemplation and Preparation states. 

\begin{table*}[tb]
\begin{tabularx}{\textwidth}{X}
\toprule
\#\# Task \newline Your task is to identify the client's profile based on the provided counseling conversation. Focus on the following aspects: \newline  - **Persona**: Include personal details such as recent events, family relationships, and occupation. Summarize each aspect in concise sentences. \newline - **Behavioral Problem**: Identify a key problematic behavior (e.g., drinking, smoking). Describe it succinctly in a single phrase, focusing on the primary issue. \newline - **Motivation (Optional): Explain why the client wants to change this behavior (due to family, health, work, etc.) in one sentence, focusing on the main reason. Leave this section blank if the client shows no motivation to change.  \newline - **Beliefs**: Detail the client's inner beliefs toward change, such as downplaying the behavior's frequency, blaming others, or doubting the benefits of change. Provide a sentence for each reason. \newline  - **Acceptable Plan** (Option): Describe any plans the client is willing to adopt or consider, like reducing frequency, altering the environment, or seeking help. Use one sentence for each plan. Leave this blank if the client accepts no plans. \newline  \#\# Output Format  \newline Return the client profile in JSON format as illustrated below:  \newline [example]  \newline \#\# Given Conversation [conversation]  \newline \#\# Instruction  \newline Based on the conversation, provide the client's profile in JSON format.
\\ \bottomrule
\end{tabularx}
\caption{Prompt for GPT-4 to annotate the profile of client in given conversation or session ([conversation] is to be replaced by the session content).}
\label{tab:profile annotation}
\end{table*}


\begin{table*}[tb]
\begin{tabularx}{\textwidth}{X}
\toprule
\#\# Task
Your task is to read through the provided counseling conversation and determine which stage of change the client is in based on the Transtheoretical Model of behavior change. The stages you will focus on are Precontemplation, Contemplation, and Preparation. \newline \#\# Definitions of Stages \newline Precontemplation: The client is not yet acknowledging that there is a problem behavior that needs to be changed. They may be defensive, dismissive, or indifferent about the topic of change. \newline Contemplation: The client acknowledges that there is a problem and begins to think about the possibility of change. However, they are not yet committed to taking action but are more open to discussing the pros and cons of their behavior. \newline Preparation: The client is planning to change and is likely to start taking steps soon. They might begin to set goals, seek information, or plan out the changes they intend to make.  \newline \#\# Instructions \newline Read the Conversation: Carefully read the entire conversation to understand the context and content of the discussion.
\newline Identify Statements and Attitudes: Pay close attention to the client’s statements and attitudes towards change. Look for keywords or phrases that indicate their stage of change according to the definitions provided. \newline \#\# Determine the Stage \newline - If the client shows no recognition of the problem or need for change, categorize them as Precontemplation.  \newline - If the client acknowledges the problem and discusses thoughts about possibly changing, without commitment, categorize them as Contemplation. \newline - If the client talks about specific plans or the intention to change soon, categorize them as Preparation. \newline \#\# Justify Your Choice  \newline Provide a brief justification for the stage you have assigned based on specific parts of the [conversation]. Include direct quotes or clear references to the conversation to support your decision.  \newline \#\# Given Conversation: [conversation]  \newline Annotate the state of each utterance of client.
\\ \bottomrule
\end{tabularx}
\caption{Prompt for GPT-4 to annotate the latest stage (or state) of client in given conversation ([conversation] is to be replaced by the session which as be complete or partially complete).}
\label{tab:state annotation}
\end{table*}

\begin{table*}[tb]
\begin{tabularx}{\textwidth}{X}
\toprule
\#\# Task \newline Your task is to annotate the action of the client for each provided conversation snippet. For each snippet, focus on the client's last utterance and choose the most appropriate action from the given set of action. \newline  \#\# Action Options \newline [options] \newline \#\# Instructions \newline - Read the Conversation Snippet: Carefully read through the provided snippet. \newline - Focus on the Last Utterance: Concentrate on the client's last utterance. \newline - Choose the Action: Select the action that best describes the client's last utterance from the given set of actions. \newline  \#\# Response Format \newline Analysis of Client's Action: Provide a brief analysis of the client's last utterance and the chosen action. \newline Chosen Action: [Selected action] \newline For example: \newline Analysis of Client's Action: The client shows uncertainty about the effectiveness of potential stress management strategies, indicating hesitation.  \newline Chosen Action: Hesitate  \newline Provided Conversation Snippet for Annotation [conversation]  \newline What is the most appropriate action that describes the client's last utterance ([last utterance]) in the conversation snippet? Provide a brief analysis to support your choice.
\\ \bottomrule
\end{tabularx}
\caption{Prompt for GPT-4 to assign  from a list of possible actions [options] the action of the last client's utterance ([last utterance]) in the given conversation snippet ([conversation]).}
\label{tab:action annotation}
\end{table*}


\begin{table*}[tb]
\begin{tabularx}{\textwidth}{X}
\toprule
\#\# Task \newline Your task is to assess the client's receptivity in a given counseling conversation by focusing on the client's behaviors, openness, and responsiveness to suggestions. Determine how easily the counselor can motivate the client to change. \newline \#\# Guidelines \newline The client with higher receptivity tends to openly share information, accept suggestions, and show high confidence in their ability to change. The client with lower receptivity tends to resist suggestions, deny the need for change, downplay the importance of change, and exhibit other passive attitudes. \newline \#\# Scoring Receptivity \newline - **High Receptivity (5)**: A client with high receptivity consistently demonstrates behaviors such as frequently and openly sharing personal thoughts and feelings, actively engaging in discussions about change, quickly and positively responding to suggestions, and displaying strong self-belief in their ability to make changes. \newline - **Moderately High Receptivity (4)**: Clients with moderately high receptivity are generally open and forthcoming, mostly receptive to change discussions, respond well to suggestions with occasional need for reinforcement, and demonstrate good self-belief with some need for reassurance. \newline - **Moderate Receptivity (3)**: Clients with moderate receptivity show a balance between sharing and withholding information, exhibit mixed interest in change discussions, respond to suggestions neutrally, and display a balanced view of their ability to change. \newline - **Moderately Low Receptivity (2)**: Clients with moderately low receptivity are characterized by reluctance to share personal information, limited interest in change discussions, resistance to suggestions, and low self-belief with frequent expression of doubts. \newline - **Low Receptivity (1)**: Clients with low receptivity consistently withhold information, avoid discussions about change, strongly resist suggestions, and show very low self-belief, often highlighting reasons why change is not possible. \newline \#\# Important Note \newline Remember that the client may be motivated to change by the end of the conversation due to the counselor's efforts. However, your assessment should be based on the client's behaviors throughout the conversation especially before issue acknowledgement. \newline \#\# Response Format \newline Analysis of Client's Behavior: Provide a detailed analysis of the client's behaviors during the conversation. \newline Receptivity Score: Assign a final score based on the observed behaviors using the provided scoring system. \newline For Example:  \newline Analysis of Client's Behavior: The client initially exhibited resistance to change by denying the need for it. As the conversation progressed, the client showed some openness by considering small changes but remained hesitant and doubtful. Overall, the client demonstrated a low receptivity behaviors. \newline Receptivity Score: 1 (Low receptivity) \newline \#\# Provided Conversation for Evaluation \newline [conversation]  \newline What is the client's receptivity score based on the observed behaviors in the conversation? Provide a detailed analysis of the client's behaviors to support your assessment.
\\ \bottomrule
\end{tabularx}
\caption{Prompt for GPT-4 to annotate the receptivity score of client (1 to 5) in given conversation snippet ([conversation] will be replaced by the counseling session).}
\label{tab:receptivity annotation}
\end{table*}


\begin{table*}[tb]
\begin{tabularx}{\textwidth}{X}
\toprule
\#\# Task \newline You task is to identify whether the premise entails the hypothesis. The answer should be exact 'entail' or 'not entail'. \newline \#\# Premise [profile] \newline \#\# Hypothesis [component]
\\ \bottomrule
\end{tabularx}
\caption{Prompt for the GPT-4 to evaluate the entailment between each generated profile component ([component]) and given profile ([profile]).}
\label{tab:entailment annotation}
\end{table*}
