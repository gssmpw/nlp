\section{More Evaluation Results}
\label{app:more evaluation}

\subsection{Automated Turn-Level Evaluation}

\begin{table*}[tb]
\centering
\begin{tabular}{lrrrrrrrr}
\toprule
             & ROUGE-1 & ROUGE-2 & ROUGE-L & BERTScore \\ \midrule
Base         & 13.27   & 2.55    & 10.03   & 85.59     \\
\rowcolor[HTML]{D8D6D6} Example-based & 59.12   & 48.84   & 58.31   & 93.54     \\
Profile-based & 13.79   & 3.18    & 9.90    & 85.50     \\
Pro+Act-based & 17.04   & 4.22    & 13.93   & 86.72     \\ \hline
Ours         & 17.73   & 9.13    & 15.13   & 87.13     \\
Ours*        & \textbf{29.69}   & \textbf{20.19}   & \textbf{28.75}   & \textbf{90.46}  \\ \bottomrule
\end{tabular}
\caption{Results of turn-level evaluation. ``Our'' refers to our client simulation method. ``Ours*'' refers to a variant of our method provided with ground truth states, actions, and selected information. Other than Example-based method, our framework outperforms the remaining baseline methods.}
\label{tab:turn-eval}
\end{table*}

As previous works~\citep{wang2024towards,deng2023prompting,kang2024can} often evaluated generated utterances against ground truth utterances in conversational tasks, we also conduct turn-level evaluation which involves using a simulated client to generate the next client utterance given a prompt that includes all the session content up to the previous utterance. To show the upper bound performance of our simulated client method (i.e., ``Ours''), we provide the results of a variant of our method (i.e., ``Ours*'') that generates the next utterance using the ground truth state, action and selected information. As shown in Table~\ref{tab:turn-eval}, our method outperforms most of the baselines. The Example-based method has been shaded because it has an unfair advantage over others due to its ability to search all the client utterances including the ground truth utterance. Our method variant, Ours*, shows that there are still rooms for Our method to improve in ROUGE-1, ROUGE-2 and ROUGE-L. Our and Our* methods share very similar BERTScore performance. This indicates that the Our method can reproduce responses semantically similar to the ground truth utterances. 

Nevertheless, it is important to note that turn-level evaluation only measures quality of the next utterance assuming that all earlier utterances are known.  This assumption does not hold in a real world counseling session or application where the client and counselor do not follow some written counseling scripts. Hence, a method performing well in this evaluation method may not always do so at the session level.  For example, the Example-based method appears to perform well in the turn-level evaluation purely due to its ability to find the most similar client utterance, indicating that turn-level evaluation may not accurately measure a client's effectiveness in real world scenarios. Hence, consistency evaluation of client profile components in Section~\ref{sec:profile consistency}, and client receptivity are introduced in Section~\ref{sec:exp_eval}. We believe these results can provide a more comprehensive assessment of client performance.

\subsection{Evaluation on Multiple Backbones}

To demonstrate the generalizability of our framework, we employ various LLMs as the backbone to implement our framework. As presented in Table~\ref{tab:results for multiple models}, our method consistently outperforms the baselines across various backbones. The performance of LLMs in simulation varies due to their distinct instruction-following and conversation capabilities. In summary, the \texttt{gpt-4o}, \texttt{llama-70B}, and \texttt{mistral-large} perform comparably to the \texttt{gpt-3.5}, while the smaller LLMs (\texttt{ministral-8B} and \texttt{llama-8B}) exhibit significantly inferior performance.

\begin{table*}[tb]
\centering
\resizebox{\textwidth}{!}{
\begin{tabular}{lrrrrrrrrr}
\toprule
              & Personas$\uparrow$    & Motivation$\uparrow$         & Beliefs$\uparrow$      & Plans$\uparrow$       &$\rho$ $\uparrow$  & Avg. Rec.            & MR@20             & Avg.MS.      & Act. KL $\downarrow$ \\ \midrule
Real          & - & -    & -  & -  & - & 3.27$_{\pm1.12}$    & 0.48              & 27.56         & 0.00              \\ 
\toprule
\multicolumn{10}{c}{\cellcolor[HTML]{EFEFEF}\texttt{gpt-4o-2024-11-20}}                              \\ \midrule
Base          & 9.17        & 16.19             & 12.26         & 9.38       & -0.28   & 4.30$_{\pm0.31}$    & 1.00              & 7.30          & 0.40 \\
Example-based & 56.29       & 49.96             & 50.66         & 37.21       & 0.27   & 3.98$_{\pm0.68}$    & 1.00              & 9.12          & 0.23\\
Profile-based & 69.78       & 56.97             & 70.59         & 57.31       & 0.32   & 4.05$_{\pm0.72}$    & 0.90              & 10.26          & 0.15 \\
Pro+Act-based & 72.12       & 63.43             & 71.57         & 63.89       & 0.35   & 3.63$_{\pm0.97}$    & 0.85              & 11.80          & 0.14\\ \hline
Ours          & \textbf{76.10}   & \textbf{78.57}   & \textbf{76.13}  & \textbf{74.99} & \textbf{0.66}  & 3.24$_{\pm1.21}$    & 0.61              & 19.78         & \textbf{0.05} \\ \bottomrule
\toprule
\multicolumn{10}{c}{\cellcolor[HTML]{EFEFEF}\texttt{Mistral-Large-Instruct-2411}}                              \\ \midrule
Base          &  9.18       & 16.22             & 12.21         & 9.35        & -0.30  & 4.38$_{\pm0.38}$    & 1.00              & 7.22          & 0.38 \\
Example-based & 54.22       & 47.18             & 47.01         & 34.87       & 0.26    & 4.04$_{\pm0.69}$    & 1.00              & 8.06          & 0.23\\
Profile-based & 66.31       & 57.10             & 69.20         & 56.85       & 0.33   & 4.22$_{\pm0.63}$    & 0.93              & 10.05          & 0.15 \\
Pro+Act-based & 70.11       & 60.88             & 71.09         & 60.03       & 0.36   & 3.77$_{\pm0.91}$    & 0.88              & 10.97          & 0.12\\ \hline
Ours          & \textbf{73.21}   & \textbf{76.30}   & \textbf{73.13}  & \textbf{71.67} & \textbf{0.62}  & 3.17$_{\pm1.27}$    & 0.65              & 19.56         & \textbf{0.05} \\ \bottomrule
\toprule
\multicolumn{10}{c}{\cellcolor[HTML]{EFEFEF}\texttt{LLaMA-3.3-70B-Instruct}}                              \\ \midrule
Base          &  9.15       & 16.20             & 12.17         &  9.35       & -0.31  & 4.33$_{\pm0.43}$    & 1.00              & 6.97          & 0.40 \\
Example-based & 54.73       & 47.56             & 46.26         & 34.78       & 0.27    & 4.07$_{\pm0.62}$    & 1.00              & 8.10          & 0.23\\
Profile-based & 67.54       & 55.04             & 68.51         & 57.95       & 0.33   & 4.11$_{\pm0.67}$    & 0.91              & 10.76          & 0.14 \\
Pro+Act-based & 70.08       & 59.43             & 69.98         & 59.81       & 0.35   & 3.82$_{\pm0.98}$    & 0.89              & 10.83          & 0.13\\ \hline
Ours          & \textbf{73.92}   & \textbf{75.90}   & \textbf{73.10}  & \textbf{71.81} & \textbf{0.60}  & 3.20$_{\pm1.08}$    & 0.63              & 19.02         & \textbf{0.06} \\ \bottomrule
\toprule
\multicolumn{10}{c}{\cellcolor[HTML]{EFEFEF}\texttt{gpt-3.5-turbo-0125}}                              \\ \midrule
Base          &  9.01       & 16.17             & 12.15         &  9.30       & -0.31  & 4.42$_{\pm0.47}$    & 1.00              & 6.60          & 0.39 \\
Example-based & 53.68       & 45.73             & 45.55         & 33.53       & 0.25    & 4.08$_{\pm0.63}$    & 1.00              & 7.60          & 0.24\\
Profile-based & 61.97       & 53.44             & 67.17         & 54.67       & 0.31   & 4.12$_{\pm0.64}$    & 0.96              & 9.76          & 0.15 \\
Pro+Act-based & 67.09       & 55.33             & 68.60         & 57.17       & 0.33   & 3.86$_{\pm1.01}$    & 0.94              & 9.93          & 0.13\\ \hline
Ours          & \textbf{70.57}   & \textbf{73.37}   & \textbf{71.70}  & \textbf{68.51} & \textbf{0.58}  & 3.32$_{\pm1.15}$    & 0.69              & 18.60         & \textbf{0.06} \\ \bottomrule
\toprule
\multicolumn{10}{c}{\cellcolor[HTML]{EFEFEF}\texttt{Ministral-8B-Instruct-2410}}                              \\ \midrule
Base          &  9.10       & 16.17             & 12.09         &  9.27       & -0.34  & 4.54$_{\pm0.42}$    & 1.00              & 6.87          & 0.39 \\
Example-based & 50.54       & 40.88             & 39.45         & 28.51       & 0.13    & 4.49$_{\pm0.61}$   & 1.00              & 7.37          & 0.30\\
Profile-based & 56.77       & 48.23             & 61.81         & 48.50       & 0.21   & 4.40$_{\pm0.60}$    & 0.99              & 9.56          & 0.19 \\
Pro+Act-based & 60.39       & 52.78             & 64.91         & 51.73       & 0.28   & 4.09$_{\pm0.91}$    & 0.96              & 9.53          & 0.17\\ \hline
Ours          & \textbf{65.75}   & \textbf{68.70}   & \textbf{68.98}  & \textbf{61.71} & \textbf{0.51}  & 3.58$_{\pm1.23}$    & 0.72              & 17.67         & \textbf{0.10} \\ \bottomrule
\multicolumn{10}{c}{\cellcolor[HTML]{EFEFEF}\texttt{LLaMA-3.1-8B-Instruct}}                              \\ \midrule
Base          &  9.13       & 15.85             & 12.02         &  9.21       & -0.33  & 4.52$_{\pm0.39}$    & 1.00              & 6.75          & 0.38 \\
Example-based & 50.46       & 40.53             & 39.35         & 28.13       & 0.13    & 4.32$_{\pm0.68}$    & 1.00             & 7.22          & 0.31\\
Profile-based & 55.89       & 47.81             & 61.71         & 48.37       & 0.20   & 4.39$_{\pm0.58}$    & 0.99              & 9.72          & 0.20 \\
Pro+Act-based & 60.05       & 51.90             & 64.16         & 50.96       & 0.27   & 4.21$_{\pm0.95}$    & 0.96              & 9.49          & 0.15\\ \hline
Ours          & \textbf{64.57}   & \textbf{67.90}   & \textbf{67.68}  & \textbf{61.68} & \textbf{0.50}  & 3.52$_{\pm1.17}$    & 0.73              & 16.74         & \textbf{0.11} \\ \bottomrule
\end{tabular}}
\caption{Consistency Assessment of Personas, Motivations, Beliefs, and Preferred Change Plans, Spearman’s Correlation of Receptivity ($\rho$), the average receptivity level (Avg. Rec.), motivation rate (MR) in the initial 20 turns, and average motivation step for baselines and our methods based on various LLMs are presented. In all backbones, our method consistently outperforms all baselines.}
\label{tab:results for multiple models}
\end{table*}

\subsection{Employ Multiple Actions in One Response}

Our current study focuses on single action selection due to the observation that the majority of utterances (89.57\% in the AnnoMI dataset) involve only one action. Only 10\% of utterances involve multiple actions. From the perspective of our framework, incorporating more complex actions can be achieved seamlessly with minor modifications. These complex actions can be included in the action set so that they can be selected for utterance generation. Alternatively, multiple actions can be sampled from the action distribution. To demonstrate the capability of our framework in using multiple actions to generate an utterance, we present an example where multiple potential actions are sampled from the action distribution. To determine the number of actions for generating an utterance, we initially sample a number of actions from the observed data (i.e., 89(single):10(double):1(three)) as candidates. Subsequently, we sample actions from the candidate actions based on the action distribution. Finally, we simply combine all the descriptions of sampled actions as instructions. Table~\ref{tab:multiple actions} demonstrates an example where our framework selects multiple actions probabilistically to control more intricate behaviors in a single response. However, it is important to note that almost all responses can be generated within single actions, thus we believe the single action framework can easily extended to simulate the multiple action scenarios. 

\begin{table*}[htbp]
\centering
\begin{tabularx}{\textwidth}{X}
\toprule
\multicolumn{1}{c}{\textbf{Example of Client Employing Multiple Actions}} \\ \midrule                                                                                             
Counselor: Hello. How are you?                                                                                                                                                                                                                                                                                                \\
Client   : I am good. What about you?                                                                                                                                                                                                                                                                                         \\
Counselor: I'm doing well, thank you! Smoking can significantly impact your finances, from the cost of cigarettes to potential health-related expenses and lost income. I'm curious to hear your thoughts on how smoking might be affecting your financial situation. Please share more about that.                           \\
Client ({\color{red}Deny}, {\color{green}Downplay}): {\color{red} You know, I don't really think about the money too much.} {\color{green} Life is busy, and there are so many other things to focus on, like work and my daughter. I just enjoy my routine and don't see smoking as a big issue in the grand scheme.}                                                                                \\
Counselor: It's great that you have a busy life with work and your daughter! I'm curious, how do you think smoking might impact your productivity or relationships at work? Have you noticed any effects on your career or how you feel in the workplace? Remember, it's all about your choices.                              \\
... \\
Client (Blame): Honestly, almost of my friends smoked, it would be totally normal and we could hang out without feeling bad about it.                   \\
Counselor: While smoking seems normal, it poses significant health risks, including respiratory issues and chronic diseases, which can impact your energy and time with your daughter. Choosing healthier habits can enhance your well-being and ensure you can fully engage in those precious moments together.              \\
Client ({\color{red}Engage}, {\color{green} Blame}): {\color{red} I get what you're saying, but honestly, life is so stressful right now. Between work and everything with my daughter, I just need something to help me unwind.} {\color{green} Plus, everyone around me has their own coping mechanisms, so it feels normal to me.}                                                                 \\
Counselor: It sounds like smoking provides a familiar escape for you amidst the chaos of daily life, and it’s understandable to want that comfort. However, consider how quitting could enhance your physical fitness—boosting your energy and flexibility—ultimately making you more present and engaged with your daughter. \\
Client (Downplay): I see your point, but I really don't think my smoking is affecting my daughter. She's too young to understand, and I don't smoke around her anyway. Everyone has their own way to deal with stress, and mine happens to be smoking. It’s just part of my routine.                                                  \\
Counselor: I’m concerned that your smoking could be diminishing your energy and flexibility, potentially impacting your ability to keep up with your daughter’s activities. Quitting could not only improve your physical fitness but also enhance your enjoyment of those special moments together.                          \\
Client (Deny): I think that's a bit dramatic. My daughter loves it when I take her to the park and play games with her. She doesn't care that I smoke – she just loves spending time with me. And I'm right there with her, enjoying it too. Why would quitting make a difference?                                                \\
Counselor: I understand that relaxation is essential for you. However, viewing your stress and tiredness as signs to prioritize your health could guide you towards healthier coping strategies. Reducing alcohol might enhance your energy and overall well-being, allowing you to embrace life with greater vitality.       \\
Client (Downplay): I guess we all have our ways of coping. Honestly, I think I have it under control.                                                                          \\
Counselor: I understand that smoking feels manageable, but it might limit your energy for activities with your daughter. Quitting could enhance your stamina, improve your health, and help you engage more fully in those cherished moments together, supporting both your well-being and your relationship.                 \\
Client ({\color{red} Downplay}, {\color{green} Inform}): {\color{red}I see your point, but I really don’t think my smoking is a big deal. It’s just part of my routine, and we have fun together anyway.} {\color{green}I don’t want my daughter to see it as something negative, but I also feel it doesn’t affect our time together much.}                                                            \\
...                                                                                                               \\ \bottomrule    
\end{tabularx}
\caption{An example demonstrates the ability of our framework to assign multiple actions for one response. This example highlights the extensibility of our framework to more complex control scenarios involving multiple actions within a single response. Each utterance corresponding to a distinct action is distinguished by a unique color.}
\label{tab:multiple actions}
\end{table*}

\subsection{Complex State Transition}

Although our framework primarily employs a unidirectional flow of state tracking, which may limit its practicality, it is still an initial attempt to introduce a state tracking module in client simulation based on observed state flow from the referenced AnnoMI dataset. However, relapse scenarios (i.e., clients returning to pre-contemplation state from contemplation state) were not observed in the AnnoMI data, preventing the derivation of their sampling distribution. To support more complex state flows, we would need to access large real-world counseling datasets, which are scarce. To demonstrate the extensibility of our framework to incorporate complex state flows, we simply employ a random method, where the client relapses to a previous state with probability (set as 0.3) in each turn. If the counselor fails to resolve the client’s hesitation or address the beliefs effectively during Contemplation, the client may relapse to Precontemplation and deny the awareness of change. In the Preparation state, if the counselor fails to motivate the client to make an acceptable plan effectively, the client may relapse to the Contemplation state. Furthermore, since relapse can occur in each turn, longer sessions may be penalized.

Table~\ref{tab:complex transition1} and Table~\ref{tab:complex transition2} demonstrate an example by employing such a complex state transition flow, where the client relapses to Precontemplation when the counselor fails to address the beliefs well. The experiment not only suggests the extensibility of our framework but also demonstrates the powerful controllability of our framework in state transition. It is important to note that such complex state flows could not be observed in the real AnnoMI dataset. Moreover, we believe that the relapse of client's states in real-world scenarios is more likely to occur across sessions instead of within a single session~\citep{hashemzadeh2019transtheoretical,prochaska1997transtheoretical,miller2012motivational}.

\begin{table*}[htbp]
\centering
\begin{tabularx}{\textwidth}{X}
\toprule
\multicolumn{1}{c}{\textbf{Example of Client Employing Complex State Flow (Part I)}} \\ \midrule   
Counselor: Hello. How are you? \\
Client: I am good. What about you? \\
Counselor: I'm glad to hear you're doing well! It’s important that you're here to reflect on your drinking. How do you think your alcohol use impacts your relationships, like with your family or friends? This could be a good opportunity to explore its effects. \\
Client {\color{green}(Precontemplation)}: Well, my friends and I just enjoy our time together, and it's more about socializing than anything else. My family has always had their own things going on, so it's not like my drinking affects them. Plus, everyone else drinks, so it's pretty normal, right? \\
Counselor: It sounds like you see drinking as a normal part of socializing with friends. Tell me more about what you enjoy most during those times together. Have you noticed any effects on how you feel, both physically and mentally, after a night of drinking? \\
Client {\color{green}(Precontemplation)}: I just feel relaxed and it’s a good way to unwind after a long week. Sometimes I might feel a bit sluggish the next day, but I think it’s pretty normal. I don’t see having a few beers as a problem—it's just part of having a good time. \\
Counselor: I understand drinking feels like a normal part of socializing. However, reducing alcohol can improve your strength and flexibility, making activities more enjoyable. Have you noticed how drinking affects your performance in sports or other physical activities with friends? Healthier choices can enhance both fun and fitness! \\
Client {\color{green}(Precontemplation)}: Actually, I've noticed I get a lot more tired after a few drinks, so I try to limit my activity after a night out. But overall, my performance in sports hasn't really been affected by drinking – I still do fine. It's not a big deal, really. \\
Counselor: It’s great that you feel fine in sports, but have you considered how the tiredness from drinking might impact your overall fitness goals? Reducing alcohol could boost your endurance and energy levels, making your workouts more effective. What are your thoughts on that? \\
{\color{blue} Transfer to Contemplation} \\
Client {\color{red} (Contemplation)}: You make a really good point. I hadn't thought about how the tiredness from drinking could really hold me back from achieving my fitness goals. I want to feel more energetic and perform better in my workouts, and I can see how cutting back on alcohol could help with that. I'm definitely motivated to make some changes! If I can improve my endurance and get more out of my training, it would definitely be worth it. Thanks for highlighting that. \\
Counselor: It sounds like your tiredness after drinking could be a signal that your body is ready for a change. By reducing alcohol, you might not only boost your energy but also enhance your physical activity, helping you achieve those fitness goals you’re excited about! \\                                                                                                \\ \bottomrule    
\end{tabularx}
\caption{An example (part I) showcases the capability of our client framework to manage intricate state transitions. This example underscores the extensibility of our framework to more intricate state transition scenarios involving stochastic relapse to prior states. The states and transitions are visually distinguished by color.}
\label{tab:complex transition1}
\end{table*}

\begin{table*}[htbp]
\centering
\begin{tabularx}{\textwidth}{X}
\toprule
\multicolumn{1}{c}{\textbf{Example of Client Employing Complex State Flow (Part II)}} \\ \midrule   
Client {\color{red} (Contemplation)}: I guess part of me enjoys the social aspect of drinking, but I also feel sluggish and unmotivated sometimes. I used to be more active, and I miss that energy. It's tough to balance having fun with feeling good physically. \\
Counselor: It's interesting that you feel sluggish after drinking. Reducing alcohol might not only boost your energy for workouts but also improve your oral hygiene and overall appearance, enhancing your independence and quality of life. This change could support both your fitness and personal health goals! \\
{\color{blue} Relapse to Contemplation} \\
Client {\color{green}(Precontemplation)}: I guess I just don't see my drinking as a big deal. It's not like I'm doing it too much. My friends and I just like to hang out and have a few beers. If they were all quitting, maybe I would too. But they don't seem to be having any problems with it. \\
Counselor: I understand that your drinking feels normal in your social circle. However, have you considered how it might impact your oral health or overall appearance over time? Recognizing this could improve not just your fitness goals, but also your long-term quality of life. What are your thoughts on that? \\
Client {\color{green}(Precontemplation)}: I see your point, but I feel like I can manage it for now. Besides, it's not just about fitness for me. Life's about enjoying moments too, right? Speaking of enjoyment, I've been thinking about planning a trip soon. It could be a nice change of scenery. \\
Counselor: Planning a trip sounds exciting! However, reducing alcohol intake could lead to deeper connections and enhance your family experiences during the trip. It might also help create memorable moments together, strengthening your relationships. How do you feel about exploring that possibility while traveling? \\
{\color{blue} Transfer to Contemplation} \\
Client {\color{red} (Contemplation)}: You make a good point. It would definitely be nice to be fully present during the trip and enjoy all those experiences without feeling sluggish. I've noticed that when I drink less, I tend to feel more energized, and that would really enhance my ability to explore and make the most out of the trip. Plus, staying active while traveling is something I really want to do. I guess if I shift my focus on creating lasting memories rather than just having a few drinks, it could make a big difference for me. Maybe I should give this a thought before my trip. \\
Counselor: That's a great realization about feeling more energized on the trip. I'd like to explore how reducing drinking might also help you reconnect with your family. Perhaps this trip can be an opportunity to heal relationships and create new memories together, rather than just fleeting social moments. \\
Client {\color{red} (Contemplation)}: I see what you mean. It's true that those deeper connections with family are so important. I’ve been thinking about how much I’d miss if I’m not fully engaged. Maybe cutting back on alcohol could really help me focus on those meaningful moments during the trip. \\
...                                                                                                               \\ \bottomrule    
\end{tabularx}
\caption{An example (part II) showcases the capability of our client framework to manage intricate state transitions. This example underscores the extensibility of our framework to more intricate state transition scenarios involving stochastic relapse to prior states. The states and transitions are visually distinguished by color.}
\label{tab:complex transition2}
\end{table*}

