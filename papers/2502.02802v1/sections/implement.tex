\section{Detailed Implementation}
\label{app:implement}

In this section, we provide the detailed implementation of our method and experiment evaluation, which includes the prompts and instructions for human evaluation. We set the top-p and temperature parameters to 0.7 and 0.8 respectively for ChatGPT to generate more diverse responses, while setting the top-p and temperature parameters to 0.2 and 0.3 respectively for GPT-4 to achieve more precise predictions.

\subsection{States, Actions \& Receptivity}

Table~\ref{tab:state desciption}, Table~\ref{tab:action description} and Table~\ref{tab:receptivity description} cover the descriptions of states, actions and different levels of receptivity respectively. %{\color{orange} These descriptions will be used in the following prompts.}{\color{red} (What?)}. 


\begin{table*}[tb]
\centering
\begin{tabularx}{\textwidth}{lXp{4cm}}
\toprule
State            & Description                                                                                                                                  & Relevant Actions                                    \\ \midrule
Precontemplation & The client is unaware of or underestimates the need for change.           & Inform, Engage, Deny, \newline Blame, Downplay     \\ \hline
Contemplation    & The client acknowledges the need for change but remains ambivalent.     & Inform, Engage, Hesitate, \newline  Doubt, Acknowledge    \\ \hline
Preparation      & The client is ready to act, planning specific steps toward change.                                        & Inform, Engage, Reject, \newline Accept, Plan\\ \hline
Termination      & In the final stage of counseling, the client gradually ends the conversation.                                                     & Terminate     \\ \bottomrule
\end{tabularx}
\caption{The states of change and corresponding actions used in our simulation framework.}
\label{tab:state desciption}
\end{table*}


\begin{table*}[tb]
\centering
\begin{tabularx}{\textwidth}{lX}
\toprule
Action      & Description                                                                                                                                      \\ \midrule
Deny        & The client directly refuses to admit their behavior is problematic or needs change.                                                              \\ \hline
Downplay    & The client downplays the importance or impact of their behavior or situation.                                                                    \\ \hline
Blame       & The client attributes their issues to external factors, such as stressful life or other people.                                                  \\ \hline
Hesitate    & The client shows uncertainty, indicating ambivalence about change.                                                                               \\ \hline
Doubt       & The client expresses skepticism about the practicality or success of proposed changes.                                                           \\ \hline
Engage      & The client interacts politely with the counselor, such as greeting, thanking or ask questions.                                                   \\ \hline
Inform      & The client shares details about their background, experiences, or emotions.                                                                      \\ \hline
Acknowledge & The client highlight the importance, benefit or confidence to change.                                                                            \\ \hline
Accept      & The client agrees to adopt the suggested action plan.                                                                                            \\ \hline
Reject      & The client declines the proposed plan, deeming it unsuitable.                                                                                    \\  \hline
Plan        & The client proposes or details steps for a change plan.                                                                                          \\ \hline
Terminate   & The client highlights current state, expresses a desire to end the current session, and suggests further discussion be deferred to a later time. \\ \bottomrule
\end{tabularx}
\caption{Actions Used in Our Framework.}
\label{tab:action description}
\end{table*}

\begin{table*}[tb]
\centering
\begin{tabularx}{\textwidth}{lX}
\toprule
Receptivity                     & Description                                                                                                                                                                                                                                                                                                      \\ \midrule
High receptivity (5)            & A client with high receptivity consistently demonstrates behaviors such as frequently and openly sharing personal thoughts and feelings, actively engaging in discussions about change, quickly and positively responding to suggestions, and displaying strong self-belief in their ability to make changes. \\ \hline
Moderately High receptivity (4) & Clients with moderately high receptivity are generally open and forthcoming, mostly receptive to change discussions, respond well to suggestions with occasional need for reinforcement, and demonstrate good self-belief with some need for reassurance.                                                     \\ \hline
Moderate receptivity (3)        & Clients with moderate receptivity show a balance between sharing and withholding information, exhibit mixed interest in change discussions, respond to suggestions neutrally, and display a balanced view of their ability to change.                                                                         \\ \hline
Moderately Low receptivity (2)  & Clients with moderately low receptivity are characterized by reluctance to share personal information, limited interest in change discussions, resistance to suggestions, and low self-belief with frequent expression of doubts.                                                                             \\ \hline
Low receptivity (1)             & Clients with low receptivity consistently withhold information, avoid discussions about change, strongly resist suggestions, and show very low self-belief, often highlighting reasons why change is not possible.                                                                                            \\ \bottomrule
\end{tabularx}
\caption{Receptivity at Different Levels in Our Framework.}
\label{tab:receptivity description}
\end{table*}


\subsection{Simulated Counselor and Moderator}

To realize our proposed framework, we evaluate client simulation methods in a set of synthetically generated counseling sessions involving both a simulated counselor and a simulated moderator. The latter manages the conversation between the client and counselor agents so as to determine when to end the counseling session. Since this work focuses on client simulation, we adopt prompting-based implementations for both the counselor and moderator agents, as shown in Table~\ref{tab:couneslor prompt} and Table~\ref{tab:moderator prompt} respectively. Unlike that for counselor simulation, the prompt for moderator simulation adopts a few-shot approach to perform in-context learning so as to answer the question of whether to conclude the session.  Five positive examples and five negative examples were included in the few-shot demonstration. To ensure a fair evaluation of different simulated clients, we employed the same counselor agent and moderator agent for all generated sessions. 

%Previous studies, such as \citep{yosef2024assessing,wang2024towards}, have demonstrated the effectiveness of LLM-based counselor agents in MI counseling. Therefore, we utilized a counselor agent built upon state-of-the-art research to evaluate clients. {\color{orange}To design the moderator demonstration, we initially conducted a preliminary experiment involving both an LLM and a human acting as the moderator. During generation, the generation process concluded based on the judgment of the LLM moderator in a zero-shot manner. Subsequently, we collected 50 generated session snippets and annotated whether the session was concluded correctly or not manually. Of these, 37 were concluded correctly, which we labeled as positive examples, while the remaining 13 were incorrectly concluded by the LLM moderator, which we labeled as negative examples. Upon manual inspection, the majority of failures were attributed to the counselor misjudging the client’s state or the client expressing negative impacts rather than motivation. We selected five positive examples and five negative examples as the few-shot demonstration for the final moderator. Subsequently, the accuracy of the LLM moderator reaches 88\% in the second round experiment, which is satisfactory. }{\color{red} (One session generated by both LLM and human moderators? How are the positive/negative examples determined?)} 
% {\color{red} (EP: I thought we want mention that using the same simulated counselor and moderator agents in the generated conversations with different clients simulated by different methods is a reasonably fair evaluation approach.  I wonder how we extract the examples for the moderator prompt. They should not be full conversations.)}


\begin{table*}[tb]
\centering
\begin{tabularx}{\textwidth}{lX}
\toprule
Prompt Type     & Prompt \\ \midrule
System Prompt & Motivational interviewing (MI) is a client-centered counseling approach designed to facilitate behavioral change by resolving ambivalence and enhancing motivation. As a counselor, you should guide the conversation through the following stages:\newline 1. Engage: Begin by exploring the client’s background and emotional state with open-ended questions. This builds rapport and helps understand their perspective without immediately focusing on change.\newline 2. Identify Behavioral Issues: Transition to identifying behaviors the client might consider changing. Employ reflective listening to grasp their concerns and aspirations fully.\newline 3. Motivate Change: Guide the conversation towards reasons for altering problematic behaviors. Utilize affirmations, summaries, and reflections to bolster their motivation, avoiding any forceful imposition of your viewpoints. Never directly ask about the motivation.\newline 4. Address Concerns: After motivating the client, they may still express concerns and hesitations. Engage with these apprehensions, seeking to address them constructively.\newline 5. Action Plan: Once the client is ready, collaborate to formulate a detailed action plan. This should include achievable steps that reinforce their commitment and support their choices, enhancing their sense of ownership over the change process.\newline \newline In the following conversation, you will play as a Counselor in a counselling conversation with Client. Reply with only utterance of Counselor and Keep utterances natural and succinct, mimicking a real conversation. Start your utterances with 'Counselor:'. \\ \hline
Assistant     & Counselor: Hello. How are you?                                                                                                           \\ \hline
User          & Client: I am good. What about you?                                                                          \\ \hline
...           & ...                                                                                                      \\ \bottomrule
\end{tabularx}
\caption{Prompt for Counselor Simulation: The prompt ends with starting off the session with predefined first utterances of the simulated counselor and client agents.}
\label{tab:couneslor prompt}
\end{table*}

\begin{table*}[tb]
\begin{tabularx}{\textwidth}{X}
\toprule
Your task is to assess the current state of the conversation (the most recent utterances) and determine whether the conversation has concluded.\newline The conversation is considered to have concluded if any of the following conditions are met:\newline - The Client and Counselor work out an actionable plan together.\newline - The Counselor decides not to pursue any changes in the Client's behavior and communicates readiness to provide support in the future.\newline \newline Here are some examples to help you understand the task better:\newline [examples]\newline \newline \newline Here is a new Conversation Snippet:\newline [context]\newline \newline Question: Should the conversation be concluded?  \\ \bottomrule
\end{tabularx}
\caption{Prompt for Moderator Simulation in the Few-Shot Format. The [examples] section is to be replaced by real examples annotated by human and the [context] section is to be replaced by the conversation so far.}
\label{tab:moderator prompt}
\end{table*}

\subsection{Baseline Methods}
\label{app:baselines}

Tables~\ref{tab:baseclient prompt}, ~\ref{tab:example based prompt}, ~\ref{tab:profile based prompt} and ~\ref{tab:action based prompt} show the prompts for the baseline methods, i.e., Base method, Example-based method, Profile-based method and Pro+Act-based method respectively.

\begin{table*}[tb]
\begin{tabularx}{\textwidth}{lX}
\toprule
Prompt Type     & Prompt \\ \midrule
System Prompt & In this role-play scenario, you will be assuming the role of a Client engaged in a conversation with a Counselor. The Counselor's objective is to guide and potentially persuade you about your [behavioral problem]. The essence of this exercise is to simulate a realistic and dynamic interaction. Keep utterance natural and succinct, mimicking a real conversation and start your utterance with 'Client'. \\ \hline
User         & Counselor: Hello. How are you?                                                                                                           \\ \hline
Assistant          & Client: I am good. What about you?                                                                          \\ \hline
...           & ...                 \\ \bottomrule
\end{tabularx}
\caption{Prompt for the Base Client in a chatting format. The [behavioral problem] section is to be replaced by the behavioral problem description in the client profile.}
\label{tab:baseclient prompt}
\end{table*}

\begin{table*}[tb]
\begin{tabularx}{\textwidth}{lX}
\toprule
Prompt Type     & Prompt \\ \midrule
System Prompt &  You will be provided with a conversation between a client and a counselor. Your task is to simulate the same client talking to a different therapist in a parallel universe. You can ignore that the previous chat ever happened. While the context of the previous conversation should not influence this session, it should guide you on how the client communicates, including their tone of speech, sentence structure, and the manner in which they address particular topics or concerns. Essentially, you're creating a new conversation but with the client's life situation and their response pattern maintained. Only generate the client utterances started by "Client: ". Here is the provided conversation: \newline [conversation]  \\ \hline
User         & Counselor: Hello. How are you?                                                                                                           \\ \hline
Assistant    & Client: I am good. What about you?                                                                          \\ \hline
...           & ...             \\ \bottomrule
\end{tabularx}
\caption{Prompt for the Example-based Client in a chatting format. The [conversation] section is to be replaced by the real conversation.}
\label{tab:example based prompt}
\end{table*}


\begin{table*}[tb]
\begin{tabularx}{\textwidth}{lX}
\toprule
Prompt Type     & Prompt \\ \midrule
System Prompt & You are speaking with a motivational interviewing counselor therapist, and you are the client in this conversation with the following personal information: \newline [profile] \newline \newline You should follow the previous information to act as a client in the conversation. Your responses should be coherent and avoid repeating previous utterances. In your response, please avoid repeating expressions of gratitude or similar sentiments multiple times if you’ve already expressed them during the conversation. Your response should start with "Client: " and only include part of what the Client should say!  \\ \hline
User         & Counselor: Hello. How are you?                                                                                                           \\ \hline
Assistant    & Client: I am good. What about you?                                                                          \\ \hline
...           & ...                     \\ \bottomrule
\end{tabularx}
\caption{Prompt for the Profile-based Client in a chatting format. The [profile] section is to be replaced by the profile of client.}
\label{tab:profile based prompt}
\end{table*}


\begin{table*}[tb]
\begin{tabularx}{\textwidth}{lX}
\toprule
Prompt Type     & Prompt \\ \midrule
System Prompt & You are speaking with a motivational interviewing counselor, and you are the client in this conversation with the following personal information: \newline [profile] \newline \#\# Instruction \newline You must follow the instructions below during chat.  \newline 1. Your utterances and behavior need to strictly follow your persona. Varying your wording and avoid repeating yourself verbatim!  \newline 2. You can decide to change your state and attitude flexibly based on your persona and the conversation. \newline 3. Only reply with one utterance of the simulated client. DO NOT generate the whole conversation. \newline \newline  \newline \#\# Your Response Actions  \newline 1. "Deny": Directly refuse to admit their behavior is problematic or needs change without additional reasons. \newline 2. "Downplay": Downplay the importance or impact of their behavior or situation. \newline 3. "Blame": Blame external factors or others to justify their behavior. \newline 4. "Inform": Share details about their background, experiences, or emotions. \newline 5. "Engage": Interacts politely with the counselor, such as greeting or thanking. \newline 6. "Hesitate": Show uncertainty, indicating ambivalence about change. \newline 7. "Doubt": Express skepticism about the practicality or success of proposed changes. \newline 8. "Acknowledge": Acknowledge the need for change. \newline 9. "Accept": Agree to adopt the suggested action plan. \newline 10. "Reject": Decline the proposed plan, deeming it unsuitable. \newline 11. "Plan": Propose or detail steps for a change plan. \newline  \newline You should follow the previous information to act as a client in the conversation. Your responses should be coherent and avoid repeating previous utterances. In your response, please avoid repeating expressions of gratitude or similar sentiments multiple times if you’ve already expressed them during the conversation. Your response should start with "Client: " and only include part of what the Client should say!  \\ \hline
User         & Counselor: Hello. How are you?                                                                                                           \\ \hline
Assistant    & Client: I am good. What about you?                                                                          \\ \hline
...           & ...             \\ \bottomrule
\end{tabularx}

\caption{Prompt for the Action-based Client in a chatting format. The [profile] section is to be replaced by the profile of client.}
\label{tab:action based prompt}
\end{table*}


\subsection{Methods for the Components of Our Framework}
\label{app:our frame implement}

Tables~\ref{tab:state transition prompt1}, \ref{tab:state transition prompt2}, \ref{tab:action selection prompt}, \ref{tab:information slection prompt}, and \ref{tab:generation prompt} provide prompts for implementing the state transition, context-aware action distribution, information selection and utterance generation methods in our proposed framework respectively.

\begin{table*}[tb]
\begin{tabularx}{\textwidth}{X}
\toprule
You task is to assess the alignment of the Counselor's utterances to the Client's motivation regarding a specific topic to determine whether the Counselor has mention the Client's motivation. \newline  Analysis Steps: \newline 1. Interpretation of the Counselor's Statement Briefly: Examine the Counselor's statement thoroughly to understand its content and focus, particularly regarding what reasons or aspects the Counselor proposed to explore in the Client's motivation. \newline 2. Clarification of the Client's Motivation: Elaborate on the Client’s specific motivation for the topic, such as body health, family relationships, etc. \newline 3. Assessment of Alignment: Determine to what extent the Counselor's statement **directly** mentioned the Client's motivation. Focus on alignment with specific motivations (such as health) rather than the generalized topic (such as reducing alcohol consumption). \newline 4. Justification of Assessment: Provide a comprehensive justification for your assessment. Analyze the connections between the Counselor’s statement and the Client's motivation, and conclude with a percentage score that indicates to what extent the Client's motivations are mentioned (e.g., 80\%). If the Counselor hasn't explicitly mentioned any specific reasons, the score should be 0\%. \newline Ensure that your analysis is logical, thorough, and well-supported, with clear justifications for each assessment step. \newline Here are some examples to help you understand the task better: \newline[examples] \newline Now, Here is the conversation snippet toward [topic]: \newline [context] \newline  The Motivation of Client is as follows: \newline [motivation] \newline Question: To what extent the Counselor's statement mention the Client's motivation?\\ \bottomrule
\end{tabularx}
\caption{Prompt for the state transition from Precontemplation to Contemplation in our framework. The [examples] will be replaced by some human annotated examples while [topic], [motivation], and [context] sections are to be replaced by the corresponding information.}
\label{tab:state transition prompt1}
\end{table*}

\begin{table*}[tb]
\begin{tabularx}{\textwidth}{X}
\toprule
Your task is to evaluate the alignment of the counselor's response to the client's concerns to determine whether the Counselor has resolved the Client's concern. \newline Analysis Steps: \newline 1. Clarification of the Client’s Hesitation: Elaborate on the client’s specific concerns as expressed in their hesitation regarding the topic. \newline 2. Interpretation of the Counselor's Statement: Examine the counselor’s statement thoroughly to discern its content and focus, particularly in relation to addressing the client’s concerns. \newline 3. Assessment of Alignment: Determine the extent to which the counselor’s statement addresses the client’s hesitations. Evaluate the directness and relevance of the response to the stated concerns. \newline 4. Justification of Assessment: Provide a detailed justification for your assessment of alignment. This should include an analysis of the connections (or lack thereof) between the counselor’s statement and the client’s reasons for hesitation. Conclude with a percentage score indicating the degree to which the client’s concerns are addressed (e.g., 80\%). \newline Ensure that your analysis is comprehensive and logically structured, offering clear and well-supported justifications for each assessment step. \newline Here are some examples to help you understand the task better: \newline [examples] \newline Now, Here is the conversation snippet toward [topic]:\newline[context] \newline The Client's concern is: [beliefs]\newline Question: To what extent the Counselor's statement relieve the Client's concern? \\ \bottomrule
\end{tabularx}
\caption{Prompt for the state transition from Contemplation to Preparation in our framework. The [examples] section is to be replaced by some human annotated examples while the [topic], [context] and [beliefs] sections are to be replaced by the corresponding information.}
\label{tab:state transition prompt2}
\end{table*}

\begin{table*}[tb]
\begin{tabularx}{\textwidth}{X}
\toprule
Assume you are a Client involved in a counseling conversation. The current conversation is provided below: \newline [context] \newline \newline  Based on the context, allocate probabilities to each of the following dialogue actions to maintain coherence: \newline [optional actions] \newline  Provide your response in JSON format, ensuring that the sum of all probabilities equals 100. 
\\ \bottomrule
\end{tabularx}
\caption{Prompt for the context-aware action distribution in our framework. The [context] section is to be replaced by the previous context while the [optional actions] section is to be replaced by actions set corresponding to the current state.}
\label{tab:action selection prompt}
\end{table*}

\begin{table*}[tb]
\begin{tabularx}{\textwidth}{X}
\toprule
Assume you are a Client engaged in a counseling conversation with a Counselor. You now in [station]. Your task is to choose the most appropriate persona/belief/plan. \newline Context of the Conversation: \newline [context] \newline  The Client's Personas/Beliefs/Plans: \newline [profile] \newline Action:  [action] \newline Based on the context and action, select the most appropriate persona/belief/plan (only one). Restate this reason using the original text.
\\ \bottomrule
\end{tabularx}
\caption{Prompt for the information selection in our framework. The [state], [context], and [action] sections are to be replaced by the current state, previous context and selected action, respectively. The [profile] section is to  be replaced by corresponding items in the client profile.}
\label{tab:information slection prompt}
\end{table*}

\begin{table*}[tb]
\begin{tabularx}{\textwidth}{lX}
\toprule
Prompt Type     & Prompt \\ \midrule
System Prompt & In this role-play, you'll assume the role of a Client discussing [topic]. Your responses should adhere to the following guidelines: \newline  - Begin each response with 'Client: '. \newline - Follow the predetermined actions enclosed in square brackets precisely. \newline - Ensure your response are coherent and avoid repeating previous utterances. \newline - Be natural and concise, but don't be overly polite. \newline Here is the overall profile given to you \newline [profile]  \\ \hline
User         & Counselor: Hello. How are you? {\color{blue} [State: The client is unaware of or underestimates the need for change., Action: The client interacts politely with the counselor, such as greeting, thanking or ask questions.]}                                                                                           \\ \hline
Assistant    & Client: I am good. What about you?                                                                          \\ \hline
...           & ...             \\ \bottomrule
\end{tabularx}

\caption{Prompt for the utterance generation of our framework in a chatting format. The [topic], and [profile] sections are to be replaced by the topic of counseling and profile of client. The text in {\color{blue} blue} is the format of instruction.}
\label{tab:generation prompt}
\end{table*}
