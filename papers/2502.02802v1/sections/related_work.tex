\section{Related Work}

\paragraph{Motivational Interviewing}
Motivational Interviewing (MI) is a client-centered counseling approach for eliciting behavior change by helping clients to explore and resolve ambivalence~\citep{miller2012motivational}. Recent works on automated MI-counselling ~\citep{chiu2024computational,yosef2024assessing,wang2024towards} predominantly focus on the evaluation of counselor agents, but ignoring the simulation of clients with strong consistency. As these counselor agents interact with hypothetical clients, their effectiveness in the simulated MI-counselling sessions would not accurately reflect that involving real human clients. To the best of our knowledge, our work is the first to address the consistency of client simulations in MI counseling. MI outlines a progression through five stages of change: precontemplation, contemplation, preparation, action, and maintenance~\citep{prochaska1997transtheoretical,hashemzadeh2019transtheoretical}. In our work, we focus on the states from Precontemplation to Preparation which are most challenging to counselors.

\paragraph{Client Agent Simulation}
Client agent simulation provides a cost-effective way to mimic the behaviors of real users and is extensively employed across different patient simulation applications~\citep{wang2024patient}. \citet{wang2024patient} introduced PATIENT-$\Psi$, integrating cognitive models with LLMs to accurately mimic the communication behaviors of real patients. However, they rely on the GPT-4 to infer patients' behaviors from their personas instead of staying consistent with real human client behaviors in counseling sessions. In our method, we integrate state tracking and action selection to enhance the consistency of the client simulations. \citet{liao2024automatic} introduced a State-Aware Patient Simulator that tracks the action of a doctor before selecting specific actions and relevant information to disclose. While their state tracking and action selection approach is similar to ours, we differ by learning the client's state transitions and actions from real datasets and adopting turn-by-turn response generation. Furthermore, patient simulators are designed to accurately disclose information, whereas counseling clients may be reticent or exhibit non-collaborative behaviors. Thus, we incorporate receptivity controls to simulate such realistic interaction. 


