\section{Client Simulation Framework}
\label{sec:framework}

\begin{figure*}[tb]
    \centering
    \includegraphics[width=\textwidth]{figs/framework.pdf}
    \caption{Proposed Client Simulation Framework.}
    \label{fig:framework}
\end{figure*}


We first give an overview of our proposed client simulation framework to generate client utterances consistent to client's profile and behavior in a MI-counseling session.  The framework consists of modules for: (a) {\em state transition}, (b) {\em action selection} (c) {\em information selection} , and (d) {\em response generation} as shown in Figure~\ref{fig:framework}.\footnote{The detailed prompt design of our framework can be found in Appendix~\ref{app:implement}.} Finally, we annotate and analyze the real world MI-based counseling data for deriving the knowledge required for developing modules (a) and (b).

\subsection{Overview}

Our framework takes a {\em client's profile} as input. It consists of the client's behavioral problem, initial state, final state, persona, motivation, beliefs, preferred change plans, and receptivity.  The client's behavioral problem, persona, motivation, beliefs, and preferred change plans are expressed in text. The initial and final states refer to the client's initial and final state-of-mind towards the behavioral problem before and after the counseling session, respectively. We utilize the transtheoretical model of health behavior change ~\citep{abuse2019enhancing} to define three possible states: {\em Precontemplation}, {\em Contemplation}, and {\em Preparation}~\citep{prochaska1997transtheoretical, hashemzadeh2019transtheoretical}. Note that there also exists Action and Maintenance states in the transtheoretical model. As MI-counseling is particularly useful when clients are in the Precontemplation and Contemplation states for them to reach the Preparation state, we consider only the three primary states henceforth, which also aligns well with the real dataset. To end the counseling session, we also include the {\em Termination} state.  

Similar to the earlier work, the persona covers background information about the client. These are useful information the counselor may need to uncover during the session. Motivation refers to a specific reason that can lead the client to consider making changes. In contrast, the client's beliefs are the ones which may obstruct behavioral changes. Preferred change plans (or plan) refer to the set of specific behavior changes the client may agree to. Receptivity captures how receptive the client is towards the counselor and is categorized into five levels from lowest (1) to highest (5). 

The client simulator generates one utterance at a time in the counseling session.  Right after each counselor's utterance, the simulator first determines the client's next state using the state transition module which has the current counseling session text (also known as context) and current state as input.  Conditioned on the next state, the action selection module merges the context-aware and (state, receptivity)-aware action distributions so that an action representing a type of utterance can be sampled.  The sampled (or selected) action, together with the new state and current context, will then be used by the response generation module to generate the next utterance.  If the selected action requires some client's profile knowledge (e.g., Inform, Hesitate, and Blame actions), the simulator will perform an additional information selection step to search for relevant reference information from the profile for response generation.

\subsection{State Transition}
\label{sec:state transition}

% \begin{table}[tb]
% \resizebox{0.47\textwidth}{!}{
% \begin{tabular}{lp{0.6\columnwidth}}
% \toprule
% State            & Corresponding Actions                                    \\ \midrule
% Precontemplation & Inform, Engage, Deny, Blame, Downplay     \\ \hline
% Contemplation    & Inform, Engage, Hesitate, Doubt, Acknowledge    \\ \hline
% Preparation      & Inform, Engage, Reject, Accept, Plan\\ \hline
% Termination      & Terminate     \\ \bottomrule
% \end{tabular}}
% \caption{The states of change and corresponding actions used in our simulation framework.}
% \label{tab:state desc}
% \end{table}


During MI counseling, a client is guided through different states to identify changes that can address his/her behavioral problem. The goal of the state transition module is thus to maintain consistency with the way the client may change states during MI counseling as well as the client's profiles~\citep{abuse2019enhancing}.

In the Precontemplation state, the counselor focuses on building trust, establishing change goals, understanding the client's motivations, and address some wrong beliefs, before evoking the client to change. The client is expected to enter the Contemplation state only when motivated by a specific reason which has also been brought up by the counselor. The state transition module thus seeks to be consistent by analyzing the counselor's utterances to search for mention(s) of client's motivation. If such a mention is found, the next client's state is Contemplation; otherwise, the state remains unchanged. In the Contemplation state, the client recognizes his/her behavioral problem but is hesitate to change, often due to some beliefs which are part of client's profile. The state transition module tracks these reasons from the client's profile and assesses whether they have been adequately addressed. If so, the client will transit to the Preparation state. In the Preparation phase, the client discusses specific plans for change while counselor provides the necessary information and tools to guide this discussion. When the client's preferred change plan has been discussed, the state transition module should output Termination as the next state. The state transition in our framework align with the observations in a real dataset. Nevertheless, our proposed framework can accommodate more complex flows and additional states, provided that further comprehensive datasets are available.
% {\color{blue} The state transition in our framework align with the observation in a real dataset. Nevertheless, our proposed framework can accommodate more complex flows and additional states, provided that further comprehensive datasets are available.}
% {\color{red} (EP: Need to explain that the current state transitions are based on transitions known in AnnoMI dataset, and that other new state transitions can be accommodated.)}

\subsection{Action Selection}
\label{sec:action selection}

Inspired by \citet{dutt2021resper}, we tailor client actions to MI counseling and integrate a simple yet effective action sampling method into the action selection module. We meticulously design candidate actions for each state based on MI counseling theory and analysis of a real dataset.\footnote{The detailed candidate actions are outlined in the Appendix~\ref{app:implement}.} Similarly, our proposed framework can accommodate additional actions, provided further datasets are available. We recognize the need to consider both the latest counselling session context, client's state and receptivity in action selection. We first infer the context-aware action distribution based on the latest session context using LLM. We next derive from real world MI-counseling data the action distribution for each (state,receptivity) combination. To ensure both context coherence and profile consistency, we finally merge the context-aware action distribution and (state,receptivity)-aware action distribution by averaging the two.  When sampling from the merged action distribution, we only select among among actions relevant to the next state.

\subsection{Information Selection}
\label{sec:information selection}

Our framework divides actions into two types. The type-1 actions, e.g., Deny, Engage, and Accept, do not require profile information to generate the utterances. The type-2 actions (e.g., Inform, Blame, Hesitate, and Plan) utilize additional information from the client's profile for utterance generation. The information selection module thus selects appropriate information from client profile for type-2 actions using LLM prompting similar to that in \citet{tu2023characterchat}, thereby enhancing the consistency with the client profile. This selection mechanism also prevents the simulated clients from sharing too much profile information unnecessarily shortening the counseling sessions.

\subsection{Response Generation}
\label{sec:response generation}

Finally, the response generation module produces a client's utterance using a {\em turn-by-turn} generation approach. Specifically, using the same application session\footnote{This covers the entire history of prompt instructions, generated counselor and client utterances including the initial role-playing instruction and client profile.} with the LLM, the module submits the prompt instruction to generate the client utterance of the next turn.  The prompt instruction includes a description of the next state and selected action, along with selected information if the action is type-2. 


\subsection{Data Annotation}
\label{sec:data annotation}

\paragraph{Annotation of AnnoMI Dataset.} Our framework assumes that state transition diagram and client profile knowledge exist to develop the state transition, action selection, information selection, and response generation modules. In the absence of available expert-curated knowledge and to allow the framework to adapt to different counseling approaches, we choose to derive these knowledge from AnnoMI~\citep{wu2022anno} for the purpose of implementing and evaluating the framework. We finally select 86 clients and their sessions from the dataset\footnote{The detailed description is provided in Appendix~\ref{app:data annotation}.} then utilized GPT-4\footnote{GPT-4: \texttt{gpt-4-0125}.} to annotate utterances of these selected sessions.
Other than the purpose of framework implementation, we also aim to perform the annotation as accurate as possible so that one can automatically evaluate the consistency of profiles and receptivity of clients in generated sessions against their ground truth ones, and analyze the distribution of client states and actions.

\paragraph{Annotation of Client Profile, States, Actions and Receptivity.} We prompt GPT-4 to summarize the four profile components: persona, motivation, beliefs, and preferred change plan, for the client given a counseling session. These prompts can be found in Table~\ref{tab:profile annotation}.  We also design prompts to annotate the client's state and action (conditioned on the state) at the utterance level (see Tables~\ref{tab:state annotation} and \ref{tab:action annotation} respectively). To determine a client's receptivity in a counseling session, we provide GPT-4 the session labeled with client states and actions followed by prompting GPT-4 to assign a receptivity score (between 1 and 5 as shown in Table~\ref{tab:receptivity annotation}). We repeat five rounds of such assignments and obtain the average receptivity score. We take the floor of the average score to obtain the final receptivity integer score. 

\paragraph{Annotation Validity} To determine the validity of above GPT-4 annotations, two annotators manually reviewed the annotations together and derived the commonly agreed annotation labels.  Against these manually derived ground truth annotations, the GPT-4 annotations achieve high accuracy rates: 87.31\% for states, 85.20\% for actions, and 80.32\% for receptivity scores. The precision and recall of persona annotation are 81.93\% and 80.07\% respectively, that of belief annotation are 77.48\% and 77.38\% respectively. The precision and recall of motivation annotation are 82.72\% and 79.98\% respectively, while that of plan annotation are 78.38\% and 76.24\% respectively. Furthermore, all profile items are factually accurate. Overall, these results confirm the high reliability of GPT-4 annotation method which makes it feasible to automate the evaluation of generated sessions. 

\paragraph{Client Behavior Analysis in AnnoMI sessions.} We analyze the our annotations to reveal client behaviors in AnnoMI sessions. Most clients, ~70\% of them, demonstrate moderate receptivity (score=2 to 4), while a few exhibit high (score=5) or low receptivity (score=1). Clients with higher receptivity tend to use neutral or change talk more often. Conversely, clients with lower receptivity are harder to move to the Contemplation state, requiring more effort from the counselor. Some actions, such as Deny, Downplay, and Reject, are negatively associated with receptivity. The details are provided in Figures~\ref{fig:suggest_annotation} and \ref{fig:action_distribution}.

