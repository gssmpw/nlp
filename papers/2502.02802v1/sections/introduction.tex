\section{Introduction}

Traditionally, mental health counselors are trained with human clients. To address the cost and limited access to human clients, researchers begin to explore the use of large language models (LLMs) to create agent-simulated clients~\citep{yosef2024assessing, chiu2024computational, wang2024towards}. Effective mental health counseling requires counselors to apply some principled approach tailored to the given client's problem. Motivational Interviewing (MI) is one such approach that aims to elicit client-driven motivation towards behavioral change~\citep{miller2012motivational,prochaska1997transtheoretical, hashemzadeh2019transtheoretical}. In MI, a counselor aims to guide a client in the Precontemplation state denying, and downplaying his/her behavioral problem towards the Preparation state where the client starts to plan for behavioral change~\citep{abuse2019enhancing}. To train a counselor who can adapt to the unique nuances of many different types of clients\citep{hoang2024can}, we require agent-simulated clients to closely mimic human clients of different profiles and behaviors in highly interactive counseling sessions.

\begin{figure}[tb]
    \centering
    \includegraphics[width=0.49\textwidth]{figs/inconsistency.pdf}
    \caption{Types of inconsistency in existing client simulation approaches (Left: simulated client, Right: counselor):  client response inconsistent with the (a) client's motivation to change,  (b) beliefs, (c) preferred plans to change, and (d) receptivity.}
    \label{fig:inconsistency}
\end{figure}

Nevertheless, the current client simulation methods predominantly utilize simple personas~\citep{yosef2024assessing,wang2024towards} or conversation examples of the clients~\citep{chiu2024computational} as LLM prompts. These prompts focus on personas but ignore the diverse dialogue actions and nuanced state transitions during counseling, which in turn leads to generated utterances inconsistent with client profiles and behaviors. Moreover, LLMs like ChatGPT are aligned to generate compliant responses~\citep{shen2023large, kopf2024openassistant}. Simulated clients using simple LLM prompting are found to demonstrate overly-compliant behavior or select overly-narrow set of actions compared to human clients~\citep{kang2024can}. 

Based on our empirical evaluation of previous client simulation approaches~\citep{yosef2024assessing,chiu2024computational,wang2024towards}, we determine four types of inconsistencies as depicted in Figure~\ref{fig:inconsistency}. The simulated clients are likely agree to behavior change for reasons not aligned with their input motivation or accept plans inconsistent with the preferred plans as shown in Figures~\ref{fig:inconsistency}(a) and  (c) respectively. Figure~\ref{fig:inconsistency}(b) also shows a simulated client not adhering to the given beliefs. Moreover, the simulated client may fail to demonstrate the stipulated receptivity to others' opinions as shown in Figure~\ref{fig:inconsistency}(d).

To tackle these challenges, we develop a comprehensive framework for consistent client simulation, as illustrated in Figure~\ref{fig:framework}. Our framework comprises four key modules: state transition, action selection, information selection, and response generation modules. Unlike the previous approaches, our framework explicitly models four aspects of client profiles, i.e., motivation, beliefs, preferred change plans, and receptivity. The framework's modules jointly control the simulated client's state and behavior at a fine-grained level thereby achieving a high degree of consistency with respect to input client profile and MI-counseling process. To achieve high degree of realism, the modules utilize knowledge extracted from a real world MI-counseling dataset.

We summarize our contributions as follows:

\begin{itemize}
    \item Our work is the first to define different aspects of consistency. We incorporate state tracking, action selection, and information selection in our proposed client simulation framework to  jointly support the overall consistency of a simulated client's utterances with respect to his/her profile and expected behavior in MI-counseling.
    \item By utilizing action distribution and state transition knowledge of clients derived from a real world counseling dataset, our client simulation framework can be easily adapted to different counseling use cases where relevant real world datasets are available. %demonstrates higher consistency with real client behaviors.
    \item An extensive set of experiments involving both automatic and expert evaluation shows that our client simulation method achieves higher consistency and creates more human-like simulated clients than other baseline methods.
\end{itemize}
