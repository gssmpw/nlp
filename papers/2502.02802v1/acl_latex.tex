% This must be in the first 5 lines to tell arXiv to use pdfLaTeX, which is strongly recommended.
\pdfoutput=1
% In particular, the hyperref package requires pdfLaTeX in order to break URLs across lines.

\documentclass[11pt]{article}

% Change "review" to "final" to generate the final (sometimes called camera-ready) version.
% Change to "preprint" to generate a non-anonymous version with page numbers.
% \usepackage[review]{acl}
\usepackage[final]{acl}

% Standard package includes
\usepackage{times}
\usepackage{latexsym}

% For proper rendering and hyphenation of words containing Latin characters (including in bib files)
\usepackage[T1]{fontenc}
% For Vietnamese characters
% \usepackage[T5]{fontenc}
% See https://www.latex-project.org/help/documentation/encguide.pdf for other character sets

% This assumes your files are encoded as UTF8
\usepackage[utf8]{inputenc}

% This is not strictly necessary, and may be commented out,
% but it will improve the layout of the manuscript,
% and will typically save some space.
\usepackage{microtype}

% This is also not strictly necessary, and may be commented out.
% However, it will improve the aesthetics of text in
% the typewriter font.
\usepackage{inconsolata}

%Including images in your LaTeX document requires adding
%additional package(s)
\usepackage{graphicx}

%custom packages
\usepackage{algorithm}
\usepackage{algorithmic}
\usepackage{tabularx}
\usepackage{booktabs}
\usepackage{colortbl}
\usepackage{multirow}
\usepackage{amsmath}

% If the title and author information does not fit in the area allocated, uncomment the following
%
%\setlength\titlebox{<dim>}
%
% and set <dim> to something 5cm or larger.

\title{Consistent Client Simulation for Motivational Interviewing-based Counseling}

% Author information can be set in various styles:
% For several authors from the same institution:
% \author{Author 1 \and ... \and Author n \\
%         Address line \\ ... \\ Address line}
% if the names do not fit well on one line use
%         Author 1 \\ {\bf Author 2} \\ ... \\ {\bf Author n} \\
% For authors from different institutions:
% \author{Author 1 \\ Address line \\  ... \\ Address line
%         \And  ... \And
%         Author n \\ Address line \\ ... \\ Address line}
% To start a separate ``row'' of authors use \AND, as in
% \author{Author 1 \\ Address line \\  ... \\ Address line
%         \AND
%         Author 2 \\ Address line \\ ... \\ Address line \And
%         Author 3 \\ Address line \\ ... \\ Address line}

% \author{First Author \\
%   Affiliation / Address line 1 \\
%   Affiliation / Address line 2 \\
%   Affiliation / Address line 3 \\
%   \texttt{email@domain} \\\And
%   Second Author \\
%   Affiliation / Address line 1 \\
%   Affiliation / Address line 2 \\
%   Affiliation / Address line 3 \\
%   \texttt{email@domain} \\}

\author{
 \textbf{Yizhe Yang\textsuperscript{1}~\thanks{Work was done during a visit at SMU.}},
 \textbf{Palakorn Achananuparp\textsuperscript{2}},
 \textbf{Heyan Huang\textsuperscript{1}~\thanks{Corresponding Author}},
 \textbf{Jing Jiang\textsuperscript{3}},
\\
 \textbf{John Pinto\textsuperscript{4}},
 \textbf{Jenny Giam\textsuperscript{5}},
 \textbf{Kit Phey Leng \textsuperscript{6}},
 \textbf{Nicholas Gabriel Lim \textsuperscript{7}},
 \\
 \textbf{Cameron Tan Shi Ern \textsuperscript{8}},
 \textbf{Ee-peng Lim\textsuperscript{2}}
% \\
%  \textbf{Ninth Author\textsuperscript{1}},
%  \textbf{Tenth Author\textsuperscript{1}},
%  \textbf{Eleventh E. Author\textsuperscript{1,2,3,4,5}},
%  \textbf{Twelfth Author\textsuperscript{1}},
% \\
%  \textbf{Thirteenth Author\textsuperscript{3}},
%  \textbf{Fourteenth F. Author\textsuperscript{2,4}},
%  \textbf{Fifteenth Author\textsuperscript{1}},
%  \textbf{Sixteenth Author\textsuperscript{1}},
% \\
%  \textbf{Seventeenth S. Author\textsuperscript{4,5}},
%  \textbf{Eighteenth Author\textsuperscript{3,4}},
%  \textbf{Nineteenth N. Author\textsuperscript{2,5}},
%  \textbf{Twentieth Author\textsuperscript{1}}
\\
\\
 \textsuperscript{1}Beijing Institute of Technology,
 \textsuperscript{2}Singapore Management University,
 \textsuperscript{3}Australian National University, 
 \\
 \textsuperscript{4}ThoughtFull,
 \textsuperscript{5}Singapore Institute of Technology,
 \textsuperscript{6}National Institute of Education,
 \\
 \textsuperscript{7}Singapore University of Social Sciences,
 \textsuperscript{8}National University of Singapore
\\
 \small{
   % \textbf{Correspondence:} \href{mailto:email@domain}{email@domain}
   \{yizheyang,hhy63\}@bit.edu.cn, \{palakorna,eplim\}@smu.edu.sg, jing.jiang@anu.edu.au
 }
}

\begin{document}
\maketitle
\begin{abstract}
Simulating human clients in mental health counseling is crucial for training and evaluating counselors (both human or simulated) in a scalable manner. Nevertheless, past research on client simulation did not focus on complex conversation tasks such as mental health counseling. In these tasks, the challenge is to ensure that the client's actions (i.e., interactions with the counselor) are consistent with with its stipulated profiles and negative behavior settings. In this paper, we propose a novel framework that supports consistent client simulation for mental health counseling. Our framework tracks the mental state of a simulated client, controls its state transitions, and generates for each state behaviors consistent with the client's motivation, beliefs, preferred plan to change, and receptivity.  By varying the client profile and receptivity, we demonstrate that consistent simulated clients for different counseling scenarios can be effectively created.  Both our automatic and expert evaluations on the generated counseling sessions also show that our client simulation method achieves higher consistency than previous methods.
\end{abstract}

\documentclass[../main.tex]{subfiles}
\graphicspath{{../images/}}
\makeatletter
\def\input@path{{../images/}}
\makeatother
\begin{document}
\section{Introduction}
\begin{figure}
\centering
\begin{tikzpicture}
\node[inner sep=0pt] (ws) at (0, 0) {
\includegraphics[height=.4\textwidth, trim={10cm 0 10cm 0},clip]{world_space.png}};
\node[inner sep=0pt] (cs) at (6,0) {\includegraphics[height=.4\textwidth, trim={10cm 1cm 10cm 4cm},clip]{conf_space.png}};
\end{tikzpicture}
\vspace{-5pt}
\label{fig:pbrm_intro}
\caption{\textbf{Left}: Shows world space obstacles as grey spheres. Robots start and goal configuration is colored red and green, respectively. Configurations along the computed path are colored transparent blue. \textbf{Right:} Mapped world space scenario to configuration space. Obstacle region is the grey mesh. Red spheres are collision-free regions computed by the neural SCDF. The optimized shortest path in the convex corridor is the blue curve.}
\vspace{-25pt}
\end{figure}
Motion planning is the problem of finding a collision-free trajectory that connects a given start and goal configuration. The planning takes place in the configuration space of the robot. For single body robots, like mobile robots or drones, the configuration space and the world space are usually the same. This simplifies the planning, since explicit obstacle representations are available which enables geometrical tools like separating hyperplanes, smallest distance to obstacles etc., to be used when designing motion planning algorithms. For multi-body robots like manipulators, the situation is completely different. The world space obstacles are usually mapped to non-convex regions, and to make the problem even harder, the mapping is usually not known. Forming explicit representations of the obstacle region in the configuration space is usually too expensive or intractable. Despite all of this, sampling based planners are used with great success, which mainly is due to their use of implicit representations of the obstacle region. The basic idea is to construct a graph in the configuration space that covers and connects the collision-free region. From this graph, a path can be extracted that connects a given start and goal configuration. The approach is computationally expensive, since the graph is constructed with the smallest geometrical building block available, points, which represents a collision-check. Furthermore, the extracted paths from the graph are non-smooth and jagged due to the stochastic nature of the approach. This adds an additional post-processing step to the process, where the paths are shortcutted and smoothened, before the path can be used for tracking. Clearly a lot of time is invested to form this graph and produce smooth paths. Thus, if the obstacles start to move, then all of this work is done in no use, since all points that make up this graph need to be re-verified, which is simply too time consuming to be done in real time.
\\\\
In this work, we want to address the existing drawbacks of the sampling based planners. Our main contribution is an improved motion planner where each vertex in the graph covers a collision-free region in the form of a sphere instead of a point and where the edges are formed with neighboring intersecting spheres. This representation has the advantage of instead of returning piecewise linear paths, returning a sequence of overlapping spheres, i.e. a convex corridor, that connects a given start and goal configuration, illustrated in Figure \ref{fig:pbrm_intro}. This convex corridor allows us to use convex optimization to produce smooth trajectories, instead of computationally expensive post-processing methods. The representation further allows us to estimate the coverage of the collision-free space, which gives us awareness and feedback in the offline roadmap construction phase. Finally, our representation is simple to adapt to moving obstacles, simply requery for the new radii and recheck for intersections. 
\\\\
The spherical collision-free regions are formed using a signed distance function (SDF), which is a function that returns the smallest distance from an arbitrary point to the boundary of an obstacle. As the name implies, the distance is signed, thus if the point is inside the obstacle it is negative otherwise positive. If the distance is positive, a sphere with radius equal to the distance is guaranteed to cover a collision-free region. Using an SDF in motion planning is not new, but what is novel about our approach is that we express the distance in the configuration space instead of the world space and by doing so allows us to form these convex collision-free regions. We refer to the resulting SDF as a signed configuration distance function (SCDF). Computing an SCDF analytically is non-trivial, our approach is therefore to parameterize the SCDF with a deep neural network and learn the mapping by supervised learning. Our resulting neural SCDF can compute distances for different parameter values of obstacle shapes and we also show how multiple distances can be combined, thus making our approach flexible.
\section{Related work}
Motion planning algorithms can roughly be divided into three families, grid-based, sampling based and optimization based methods. Grid-based methods (GBM) discretize the planning space from which a graph is then compiled. A standard search method is A$^\star$ \citep{a_star}, which is classified as an \textit{informed} search method, since it employs a heuristic function to speed up the search. A$^\star$ guarantees to return an optimal path at the level of discretization used. GBMs usually discretize the planning space by a regular lattice and this limits the GBMs to problems with low dimensionality due to the curse of dimensionality. Thus, GBMs are usually limited to single-body robots where the degrees of freedom (DOF) are low. To overcome the inherent scaling problem with the GBMs, stochastic methods are usually used for multi-body robots. These methods are termed as sampling-based methods (SBM) and core members within this family are the rapidly-exploring random trees (RRT) \citep{rrt} and the probabilistic roadmap (PRM) \citep{prm}. RRT grows a tree from the start configuration and explores the collision-free region in a rapid way until it is able to connect to the goal region. RRT is usually improved by bi-directional planning \citep{rrt_connect}, i.e. an additional tree is grown from the goal configuration and the trees are tested for connection after any tree has been expanded. RRT is a single-query method, thus it searches for a path from scratch each time it is queried. Contrary to this, PRM is a multi-query method, which solves for multiple queries without starting from scratch. PRM does this by creating a roadmap (graph) that covers the collision-free space as an offline step. The graph is then used to solve for multiple queries. PRMs are used in cases where the environment does not change since the extra offline step is too computationally costly and needs to be re-done if the environment is changed. In our work, we address this inherent issue by using a different roadmap representation. Our vertices in the graph cover a collision-free region in the form of spheres and we form the edges by checking for intersecting spheres. If something in the environment changes, we recompute the spheres radii and recheck the intersections, without relying on collision detection. We use a trained neural network to compute the sphere radius, therefore querying for the radius can be done fast, hence our representation enables the PRM for dynamic environments.
\\\\
In the recent decades, optimization based methods (OBM) \citep{chomp, schulman, itomp, stomp} have been introduced as an alternative to SBM for multi-body robots. Like the SBM, the OBMs scale well to higher dimensional problems and produce smoother motion. It is common to use a SDF in the optimization since it is a smooth function, thus enabling gradient-based methods. However, the standard way of expressing the SDF is in world space. The distance therefore needs to be mapped to the configuration space by the forward kinematics. This mapping makes the optimization problem a non-linear program (NLP), which is computationally expensive to solve. Recently, a different approach has been proposed. In \cite{mp_gcs} motion planning is formulated as a convex optimization problem by using the graph of convex sets framework \citep{gcs}. The underlying idea is to decompose the collision-free space into intersecting convex sets from which a convex optimization problem is formulated. In cases where an explicit representation of the obstacles in the configuration space exists, like for single-body robots, creating collision-free convex regions can be done fast \citep{iris}. For multi-body robots, this is non-trivial. Existing work does this successfully \citep{iris_nlp, iris_c} by an optimization based approach, but the methods are still too time consuming to be used in the presence of moving obstacles. Our approach is instead to use deep learning to learn an SDF expressed in the configuration space. With this, we can query for shortest distances to the collision boundary, which allows us to expand spherical regions which are collision-free. Our approach is fast and therefore enables our suggested roadmap planner to be used in dynamic environments.
\\\\
Recent research has focused on learning collision detection \citep{fk_kernel_distance, diffco, graphdistnet} by predicting the signed distance between the robot links and the surrounding obstacles in the world space. The learned SDF is used in trajectory optimization but since the distance is expressed in the world space, the problem becomes an NLP and therefore takes a long time to solve. We take a novel approach and suggest to instead express the signed distance in the configuration space. This allows us to improve the PRM at the same time as it enables convex optimization for trajectory optimization, which runs faster and is more reliable than NLP solvers. In \cite{cspf} a learned signed distance function in the configuration space is proposed similar to our approach. However, their approach is restricted to point cloud representations, while we propose to represent the obstacles as parameterized geometric shapes, e.g. spheres. Furthermore, we also show how to use our learned SCDF to improve an existing roadmap planner.
\section{Problem formulation}
A robot is located in the world space, $\W \subset \R^3 $. The unique location of the robot is given by its configuration $\q \in \C$, where $\C$ is the configuration space. The set of points covered by the robots bodies at a certain configuration is expressed as $\B(\q) \subset \W$. The robot is surrounded by $\NrObst$ obstacles $\O = \bigcup_{i=1}^{\NrObst} \O_i$, where  $\O_i \subset \W$. The representation of the obstacle in the configuration space is the set $\C\O_i = \{\q \in \C \: |\: \B(\q) \cap \O_i \neq \emptyset \}$. The obstacle space is formed as $\Co = \bigcup_{i=1}^{\NrObst} \C \O_i$. The complement is referred to as the free space, $\Cf = \C \setminus \Co$. The path planning problem is a tuple, ($\Cf$, $\qStart$, $\qGoal$), where we want to connect a query pair, consisting of a start, $\qStart$, and goal configuration, $\qGoal$, with a geometric path, $\q(s): [0, 1] \mapsto \Cf$, such that $\q(0)=\qStart$ and $\q(1)=\qGoal$, or report correctly when such a path does not exist.
\end{document}


\section{Related Work}
% \subsection{Vision Language Model}
% 시각장애인에서 상황을 설명할 DB가 없으니 만들었다. 그리고 이를 VLM에 튜닝했다.
\subsection{Technical approaches for assisting the visually-impaired}


\subsection{Datasets for visual instruction tuning}


\section{Client Simulation Framework}
\label{sec:framework}

\begin{figure*}[tb]
    \centering
    \includegraphics[width=\textwidth]{figs/framework.pdf}
    \caption{Proposed Client Simulation Framework.}
    \label{fig:framework}
\end{figure*}


We first give an overview of our proposed client simulation framework to generate client utterances consistent to client's profile and behavior in a MI-counseling session.  The framework consists of modules for: (a) {\em state transition}, (b) {\em action selection} (c) {\em information selection} , and (d) {\em response generation} as shown in Figure~\ref{fig:framework}.\footnote{The detailed prompt design of our framework can be found in Appendix~\ref{app:implement}.} Finally, we annotate and analyze the real world MI-based counseling data for deriving the knowledge required for developing modules (a) and (b).

\subsection{Overview}

Our framework takes a {\em client's profile} as input. It consists of the client's behavioral problem, initial state, final state, persona, motivation, beliefs, preferred change plans, and receptivity.  The client's behavioral problem, persona, motivation, beliefs, and preferred change plans are expressed in text. The initial and final states refer to the client's initial and final state-of-mind towards the behavioral problem before and after the counseling session, respectively. We utilize the transtheoretical model of health behavior change ~\citep{abuse2019enhancing} to define three possible states: {\em Precontemplation}, {\em Contemplation}, and {\em Preparation}~\citep{prochaska1997transtheoretical, hashemzadeh2019transtheoretical}. Note that there also exists Action and Maintenance states in the transtheoretical model. As MI-counseling is particularly useful when clients are in the Precontemplation and Contemplation states for them to reach the Preparation state, we consider only the three primary states henceforth, which also aligns well with the real dataset. To end the counseling session, we also include the {\em Termination} state.  

Similar to the earlier work, the persona covers background information about the client. These are useful information the counselor may need to uncover during the session. Motivation refers to a specific reason that can lead the client to consider making changes. In contrast, the client's beliefs are the ones which may obstruct behavioral changes. Preferred change plans (or plan) refer to the set of specific behavior changes the client may agree to. Receptivity captures how receptive the client is towards the counselor and is categorized into five levels from lowest (1) to highest (5). 

The client simulator generates one utterance at a time in the counseling session.  Right after each counselor's utterance, the simulator first determines the client's next state using the state transition module which has the current counseling session text (also known as context) and current state as input.  Conditioned on the next state, the action selection module merges the context-aware and (state, receptivity)-aware action distributions so that an action representing a type of utterance can be sampled.  The sampled (or selected) action, together with the new state and current context, will then be used by the response generation module to generate the next utterance.  If the selected action requires some client's profile knowledge (e.g., Inform, Hesitate, and Blame actions), the simulator will perform an additional information selection step to search for relevant reference information from the profile for response generation.

\subsection{State Transition}
\label{sec:state transition}

% \begin{table}[tb]
% \resizebox{0.47\textwidth}{!}{
% \begin{tabular}{lp{0.6\columnwidth}}
% \toprule
% State            & Corresponding Actions                                    \\ \midrule
% Precontemplation & Inform, Engage, Deny, Blame, Downplay     \\ \hline
% Contemplation    & Inform, Engage, Hesitate, Doubt, Acknowledge    \\ \hline
% Preparation      & Inform, Engage, Reject, Accept, Plan\\ \hline
% Termination      & Terminate     \\ \bottomrule
% \end{tabular}}
% \caption{The states of change and corresponding actions used in our simulation framework.}
% \label{tab:state desc}
% \end{table}


During MI counseling, a client is guided through different states to identify changes that can address his/her behavioral problem. The goal of the state transition module is thus to maintain consistency with the way the client may change states during MI counseling as well as the client's profiles~\citep{abuse2019enhancing}.

In the Precontemplation state, the counselor focuses on building trust, establishing change goals, understanding the client's motivations, and address some wrong beliefs, before evoking the client to change. The client is expected to enter the Contemplation state only when motivated by a specific reason which has also been brought up by the counselor. The state transition module thus seeks to be consistent by analyzing the counselor's utterances to search for mention(s) of client's motivation. If such a mention is found, the next client's state is Contemplation; otherwise, the state remains unchanged. In the Contemplation state, the client recognizes his/her behavioral problem but is hesitate to change, often due to some beliefs which are part of client's profile. The state transition module tracks these reasons from the client's profile and assesses whether they have been adequately addressed. If so, the client will transit to the Preparation state. In the Preparation phase, the client discusses specific plans for change while counselor provides the necessary information and tools to guide this discussion. When the client's preferred change plan has been discussed, the state transition module should output Termination as the next state. The state transition in our framework align with the observations in a real dataset. Nevertheless, our proposed framework can accommodate more complex flows and additional states, provided that further comprehensive datasets are available.
% {\color{blue} The state transition in our framework align with the observation in a real dataset. Nevertheless, our proposed framework can accommodate more complex flows and additional states, provided that further comprehensive datasets are available.}
% {\color{red} (EP: Need to explain that the current state transitions are based on transitions known in AnnoMI dataset, and that other new state transitions can be accommodated.)}

\subsection{Action Selection}
\label{sec:action selection}

Inspired by \citet{dutt2021resper}, we tailor client actions to MI counseling and integrate a simple yet effective action sampling method into the action selection module. We meticulously design candidate actions for each state based on MI counseling theory and analysis of a real dataset.\footnote{The detailed candidate actions are outlined in the Appendix~\ref{app:implement}.} Similarly, our proposed framework can accommodate additional actions, provided further datasets are available. We recognize the need to consider both the latest counselling session context, client's state and receptivity in action selection. We first infer the context-aware action distribution based on the latest session context using LLM. We next derive from real world MI-counseling data the action distribution for each (state,receptivity) combination. To ensure both context coherence and profile consistency, we finally merge the context-aware action distribution and (state,receptivity)-aware action distribution by averaging the two.  When sampling from the merged action distribution, we only select among among actions relevant to the next state.

\subsection{Information Selection}
\label{sec:information selection}

Our framework divides actions into two types. The type-1 actions, e.g., Deny, Engage, and Accept, do not require profile information to generate the utterances. The type-2 actions (e.g., Inform, Blame, Hesitate, and Plan) utilize additional information from the client's profile for utterance generation. The information selection module thus selects appropriate information from client profile for type-2 actions using LLM prompting similar to that in \citet{tu2023characterchat}, thereby enhancing the consistency with the client profile. This selection mechanism also prevents the simulated clients from sharing too much profile information unnecessarily shortening the counseling sessions.

\subsection{Response Generation}
\label{sec:response generation}

Finally, the response generation module produces a client's utterance using a {\em turn-by-turn} generation approach. Specifically, using the same application session\footnote{This covers the entire history of prompt instructions, generated counselor and client utterances including the initial role-playing instruction and client profile.} with the LLM, the module submits the prompt instruction to generate the client utterance of the next turn.  The prompt instruction includes a description of the next state and selected action, along with selected information if the action is type-2. 


\subsection{Data Annotation}
\label{sec:data annotation}

\paragraph{Annotation of AnnoMI Dataset.} Our framework assumes that state transition diagram and client profile knowledge exist to develop the state transition, action selection, information selection, and response generation modules. In the absence of available expert-curated knowledge and to allow the framework to adapt to different counseling approaches, we choose to derive these knowledge from AnnoMI~\citep{wu2022anno} for the purpose of implementing and evaluating the framework. We finally select 86 clients and their sessions from the dataset\footnote{The detailed description is provided in Appendix~\ref{app:data annotation}.} then utilized GPT-4\footnote{GPT-4: \texttt{gpt-4-0125}.} to annotate utterances of these selected sessions.
Other than the purpose of framework implementation, we also aim to perform the annotation as accurate as possible so that one can automatically evaluate the consistency of profiles and receptivity of clients in generated sessions against their ground truth ones, and analyze the distribution of client states and actions.

\paragraph{Annotation of Client Profile, States, Actions and Receptivity.} We prompt GPT-4 to summarize the four profile components: persona, motivation, beliefs, and preferred change plan, for the client given a counseling session. These prompts can be found in Table~\ref{tab:profile annotation}.  We also design prompts to annotate the client's state and action (conditioned on the state) at the utterance level (see Tables~\ref{tab:state annotation} and \ref{tab:action annotation} respectively). To determine a client's receptivity in a counseling session, we provide GPT-4 the session labeled with client states and actions followed by prompting GPT-4 to assign a receptivity score (between 1 and 5 as shown in Table~\ref{tab:receptivity annotation}). We repeat five rounds of such assignments and obtain the average receptivity score. We take the floor of the average score to obtain the final receptivity integer score. 

\paragraph{Annotation Validity} To determine the validity of above GPT-4 annotations, two annotators manually reviewed the annotations together and derived the commonly agreed annotation labels.  Against these manually derived ground truth annotations, the GPT-4 annotations achieve high accuracy rates: 87.31\% for states, 85.20\% for actions, and 80.32\% for receptivity scores. The precision and recall of persona annotation are 81.93\% and 80.07\% respectively, that of belief annotation are 77.48\% and 77.38\% respectively. The precision and recall of motivation annotation are 82.72\% and 79.98\% respectively, while that of plan annotation are 78.38\% and 76.24\% respectively. Furthermore, all profile items are factually accurate. Overall, these results confirm the high reliability of GPT-4 annotation method which makes it feasible to automate the evaluation of generated sessions. 

\paragraph{Client Behavior Analysis in AnnoMI sessions.} We analyze the our annotations to reveal client behaviors in AnnoMI sessions. Most clients, ~70\% of them, demonstrate moderate receptivity (score=2 to 4), while a few exhibit high (score=5) or low receptivity (score=1). Clients with higher receptivity tend to use neutral or change talk more often. Conversely, clients with lower receptivity are harder to move to the Contemplation state, requiring more effort from the counselor. Some actions, such as Deny, Downplay, and Reject, are negatively associated with receptivity. The details are provided in Figures~\ref{fig:suggest_annotation} and \ref{fig:action_distribution}.



\section{Experiments}\label{sec_exp}
%\hp{Accelerating IM simulation~\cite{tang2015influence}}

% \begin{itemize}
%     \item 6.1. Problem setting of three COPs, including the general model and three specific CO problems 
%     \item 6.2. Experiment Setting (hyperparameters, details of training, evaluation, and test) 写在appendix里吧
%     \item 6.3. Performance analysis 这个要占半页
% \end{itemize}

%\hp{need to think of a way to compress these tables / visuals.} 

%\hp{\cancel{Baselines}; hyperparamters; \cancel{metrics}; etc.}

With theoretical guarantees on the existence and convergence of NE for ACCES games, we are also interested in how our proposed algorithm CCDO-RL works empirically. To evaluate this, we conduct experiments of CCDO-RL on three distinct ACCES game instances introduced in Section \ref{sub_exp_ins} and analyze the performance of CCDO-RL in Section \ref{sub_train_eval}. Section 6.2.1 aims to empirically demonstrate the convergence (Figures \ref{fig_exploit_20} and \ref{fig_exploit_50}) of the algorithm CCDO-RL over realistic CO problems, and show its consistency with Theorem \ref{CCDOA}. Section 6.2.2 intends to show the average reward (to seen training graphs) as well as the generalizability (to unseen test graphs) of the combinatorial player in real-world ACCES games (shown in Tables \ref{tab_aver}, and \ref{tab_gene}).

\subsection{Three Instances of ACCES Games} \label{sub_exp_ins}
% \hp{This para does not make much sense. Need to follow the framework in the Preliminaries section.}
% For combinatorial optimization problems in real-world applications, situations are more complicated and intractable due to changeable environmental or physical parameters. The form of parameter sets is very crucial because different types have different solvability and computation complexity. Forms of parameter sets mainly contain discrete sets, interval sets \cite{buchheim2018robust} like polyhedral and ellipsoid, probability distributions \cite{carlsson2018wasserstein}, and variable functions \cite{krause2008robust}.

% In reality, these parameters are often impacted by some common factors, such as conditions of weather, transportation, and individual personalities. \cite{kalimeris2019robust} proposed an assumption that real instances (e.g. demands in CVRP, coverages in CSP) 
%Considering affected or attacked COPs, the real instance $\{\theta_{i}\}$ always relied on the estimated value $\{\hat{\theta}_{i}$\} and the variation determined by independent factors $\{g_{i}\}$ and environment/physical parameters/attacker actions $\{\eta\}$. The concrete parameter influence model is stated as follows:

We consider a certain COP which is parameterized with $\{\theta_{i}\}$, where $i$ is the index of nodes (such as a target in security games) -- e.g., such parameters can be interpreted as attack probability of targets.
%coverage radius, customer's demands, or attack probability of targets. 
In real-world applications, we often need to estimate such parameters before solving the COPs. Unfortunately, the estimation $\{\hat{\theta}_{i}\}$ often bears a gap to the true value $\{\theta_{i}\}$, which derives from e.g. environment (aleatoric) uncertainty, model (epistemic) uncertainty, or an attacker trying to manipulate the defender's utility. We use a generic model to formulate this gap:
\begin{equation}\label{linrob}
    \theta_{i} = \hat{\theta}_{i} + y \cdot \tau_{i},
\end{equation}
where $y$ represents the strategy of the nature/attacker, $\tau_{i}$ is the environment factors like weather and transportation conditions, or human subjective factors like the preference of the attacker. 
Such abstraction can represent a wide range of ACCES games, such as facility location covering problems \cite{an2020battery, TIRKOLAEE2020340}, CVRP \cite{vehiclerouting.ch8,dinh2018exact, FLORIO20231081}, security patrolling (OP) \citep{xu2021robust}, and influence maximization problem \cite{kalimeris2019robust}. We describe three instances of ACCES games based on the model (\ref{linrob}).%Based on this model (\ref{linrob}), we focus on three combinatorial optimization problems with attacks or environmental/physical influence.

% \hp{Hard to follow. We should point out what are the two players, what are X, Y, u etc}

\textbf{Adversarial Covering Salesman Problem (ACSP):} In a map of cities, every city $i$ has a coverage $\theta_{i}$. A salesman finds the shortest path such that all cities are visited or covered, with $\theta_{i}$ influenced by physical factors $\tau_i$ and transportation parameters $y$ based on Eq.(\ref{linrob}). The salesman is Player 1 where $X$ consists of the feasible paths of the salesman. Nature is Player 2 with $Y$ = $[0, 1]^K \ni y, K \in \mathbb{N}$. The utility function of Player 1 $u$ is the opposite of the total traveling distance.

\textbf{Adversarial Capacitated Vehicle Routing Problem (ACVRP):} A vehicle with a constrained capacity of goods finds the shortest path under the worst case with the $i_{th}$ customer's demand $\theta_i$ changed by environmental factors $\tau_i$ and weather parameter $y$ on Eq.(\ref{linrob}). The vehicle is Player 1 where $X$ is the set of the feasible path $x$. Nature is Player 2 where $Y$ is $[0, 1]^K \ni y, K \in \mathbb{N}$. The utility function of Player 1  $u$ is the opposite of total delivery distance satisfying all the demands of customers.


\textbf{Patrolling Game (PG):} The patrolling game is described in the introduction.

For all the problem instances, we run our algorithm on two problem sizes: 20 nodes and 50 nodes. The detailed description and problem parameters of the three game instances are in Appendix \ref{app_ex_para_set}.

% Similarly, in the vehicle route problem (VRP), conditions with correlated parameters arouse broad attention from scholars \cite{vehiclerouting.ch8,dinh2018exact,FLORIO20231081}. \cite{dinh2018exact} considered the demand correlation by geographical proximity of nodes, described by some independent random variables in the fractional form. \cite{FLORIO20231081} utilized 'external factors' to stand for unknown covariates affecting all demands and presented a Bayesian model to learn correlations. Further more, about IM problems, \cite{kalimeris2019robust} combined node features and uncertain hyperparameters to fit the influence probability on each edge.

% \subsection{Training CCDO-RL}

% For all the problems, CCDO-RL adopts the REINFORCE algorithm with an attention-based encoder-decoder framework \cite{kool2018attention} (used as an inductive graph representation component) to learn a (generalizable) COP solver for one player (protagonist), and PPO \cite{schulman2017proximal} to train a policy for the other player (adversary) whose strategy space is continuous. CCDO-RL is trained with 50 epochs on a set of 10,000 graphs (with 20 or 50 nodes). The hyperparameters of CCDO-RL are specified in Appendix \ref{app_ex_para_set} (Table \ref{tab_hyper_ccdorl}). Our code is included as supplementary material for ease of reproduction. 
% % \hp{need to specify hyperparas}

\subsection{Performance of CCDO-RL}\label{sub_train_eval}

Two aspects are evaluated for the performance of CCDO-RL, i.e., i) Convergence to NE (Section \ref{sub_per_conver}) exploring whether CCDO-RL can compute the NE, and ii) Protagonist policy's average reward and generalizability (Section \ref{sub_per_rob}). Generalizability refers to the ability of RL models trained on previously seen graphs (problem instances), to perform well on a new set of unseen test graphs. The model’s usability is enhanced by generalizability, rather than focusing solely on the average reward, which is a critical motivation in the literature on RL for COPs \citep{khalil2017learning, kool2018attention}.

For all the problems, CCDO-RL adopts the REINFORCE algorithm with an attention-based encoder-decoder framework \citep{kool2018attention} (used as an inductive graph representation component) to learn a generalizable COP solver for Player 1 (protagonist), and PPO to train a policy for Player 2 (adversary) whose strategy space is continuous. CCDO-RL is trained on a set of 10,000 graphs (with 20 or 50 nodes). The hyperparameters of CCDO-RL are specified in Appendix \ref{app_ex_para_set} (Table \ref{tab_hyper_ccdorl}). Our code is included as supplementary material and will be open-sourced for ease of reproduction. 

% \textbf{Training.} For all the problems, CCDO-RL adopts the REINFORCE algorithm with attention-based encoder-decoder framework \cite{kool2018attention} (used as an inductive graph representation component) to learn a (generalizable) COP solver for one player (protagonist), and PPO \cite{schulman2017proximal} to train a policy for the other player (adversary) whose strategy space is continuous. CCDO-RL is trained with 50 epochs on a set of 10,000 graphs (with 20 or 50 nodes). 

% \hp{We should first present results about convergence as it is mostly aligned with the theory.}

\subsubsection{Convergence to NE} \label{sub_per_conver}

Exploitability is a common metric to describe the closeness to true NE by calculating the sum of performance distances between each new best response and subgame NE, i.e. $\sum_{i=1,2} U(\pi_{i,k}^{br}, \sigma_{-i,k}) - U(\sigma)$ in the general two-player game. Since our game is zero-sum, the calculation is as follows:
\begin{equation*}
   \text{Exploitability}(\sigma) = \max_{\pi_1 \in \Sigma_1} U(\pi_1, \sigma_{2}) - \min_{\pi_2 \in \Sigma_2} U(\sigma_1, \pi_2).
\end{equation*}
From Figure \ref{fig_exploit_20}, we can see that CCDO-RL can converge to approximate NE in 25 iterations or less (in the PG setting), reaching 0.05 in ACSP, 0.10 in ACVRP, and 0.03 in PG with 20 nodes. Similar results are observed in problems with 50 nodes (see Figure \ref{fig_exploit_50} in Appendix \ref{app_exp}). These results validate the effectiveness of CCDO-RL in finding the NE for various types of games.

%Similarly, the exploitability of three COPs in 50 nodes is provided in the appendix \ref{app_exp}.
\vspace{-\baselineskip}
\begin{figure}[htbp]
	\centering
    \subfigure[ACSP20]{
    \label{csp20_nashconv}
    \includegraphics[scale=0.20]{Figures/nashconv_log_csp20_sm_7.eps}
    }
    \subfigure[ACVRP20]{
    \label{cvrp20_nashconv}%文中引用该图片代号
    \includegraphics[scale=0.20]{Figures/nashconv_log_svrp20_sm_7.eps}
    }
    \subfigure[PG20]{
    \label{opsa20_nashconv}
    \includegraphics[scale=0.20]{Figures/nashconv_log_pg20_sm_7.eps}
    }
    \caption{Exploitability curve of CCDO-RL on three games of 20 nodes}
    \label{fig_exploit_20}
\end{figure}
\vspace{-\baselineskip}
\subsubsection{Average reward and Generalizability of Combinatorial player} \label{sub_per_rob}
% \subsubsection{Robustness and Generalizability of Protagonist Policy} \label{sub_per_rob}
%\hp{CCDO-RL being better in these following metrics is only kind of a by-product.}

% \textbf{Evaluation.} The learned policies are then tested on 200 graphs, where 100 of them are randomly selected from the 10,000 training graphs, and the other 100 are unseen graphs. 
% We use two metrics to evaluate the performance of different policies for the protagonist player: \textbf{Average proportional loss} $R-$ describes the policy overfitting degree \citep{lanctot2017unified}; \textbf{Reward} evaluates the performance of the protagonist with the adversary under three COPs.  
% \begin{eqnarray}
%         &R- = (\hat{D} - \hat{O}) / \hat{D}.
% \end{eqnarray}
% in which $\hat{D}$ is the mean value of the diagonals and $\hat{O}$ is the mean value of the off-diagonals in the payoff matrix provided in the Appendix \ref{app_exp}.

% Because the protagonist policy is trained against a powerful adversary under our ACCES game setting, the obtained policy is naturally robust against adversarial perturbations. This subsection sheds a bit of light on this perspective and quantifies the extent of robustness of CCDO-RL as well as the ability of RL to generalize to unseen test graphs.

\textbf{Evaluation.} The learned policies are tested on 200 graphs, with 100 being randomly selected from the 10,000 training graphs (to show the average reward), and the other 100 being unseen graphs (to test policy generalization). We evaluate the performance of the protagonist with the adversary under three COPs. For each COP, the performance is considered both on the 20-node and 50-node map.
% We use two metrics to evaluate the performance of different policies for the protagonist player: \textbf{Average proportional loss} $R-$ describes the policy overfitting degree \citep{lanctot2017unified}; \textbf{Reward} evaluates the performance of the protagonist with the adversary under three COPs.

\textbf{Baselines.} There are heuristic algorithms for each game instance (Heuristic in Table \ref{tab_aver} and \ref{tab_gene}) and a single-player RL algorithm. For ACVRP, we adopt the Tabu Search algorithm (Tabu) \citep{li2020improved} as the heuristic algorithm, which is widely applied in the routing problem. For ACSP, the common benchmark local search algorithm, LS2 \citep{golden2012generalized}, is used. For PG, we choose the greedy algorithm as the baseline. The "RL against Stoc" algorithm in Tables \ref{tab_aver} and \ref{tab_gene} is identical to the protagonist model in CCDO-RL but trained in environments with stochastic adversarial perturbations.

% \textbf{Baselines.} There are a heuristic algorithms for each game instance {\color{red} (Heuristic mentioned in the Table \ref{tab_aver} and \ref{tab_gene})} and a single-player RL algorithm. For ACVRP, we adopt the Clarke-Wright (CW) algorithm \citep{pichpibul2013heuristic} and the Tabu Search algorithm (Tabu) \citep{li2020improved} as heuristics, which are applied widely in the routing problem. For ACSP, two common benchmark local search algorithms, LS1 and LS2 \citep{golden2012generalized}, are used. For PG, we choose a local search algorithm \citep{vansteenwegen2009iterated} and the greedy algorithm as the heuristic baselines. {\color{red} The "RL  against Stoc" algorithm referred to Tables \ref{tab_aver} and \ref{tab_gene}} is identical to the protagonist model in CCDO-RL {\color{red} but trained on environments with stochastic adversarial perturbations.} 

\textbf{Average Reward.}  As illustrated in Table \ref{tab_aver}, our algorithm achieves a better average reward than baselines (10.08\% improvement on average of all settings against two baselines), regardless of CO instance or problem size, when confronting the adversary trained by CCDO-RL. In the setting of CSP-20 nodes, the average reward is improved by 46.98\% compared to the heuristic and by 7.14\% compared with the RL against Stoc. For the 50-node setting, the improvements are 45.91\% and 5.28\% respectively. Similarly, the improvements in contrast to Heuristic and RL against Stoc are as follows: 1.72\% and 3.01\%  for CVRP-20 nodes, 0.75\% and 4.46\% for CVRP-50 nodes, 4.17\% and 1.48\% for PG-20 nodes, and 10.60\% and 4.38\% for PG-50 nodes.

\textbf{Generalizability.} From Table \ref{tab_gene}, CCDO-RL continues to achieve a better average reward when facing the adversary, demonstrating that the learned RL policies generalize well to unseen graphs. Even though the non-RL baselines do have access to the graph structures and other problem information of the unseen problem instances, CCDO-RL can obtain comparable performances without re-training on the new problem instances. The improvements versus Heuristic and RL against Stoc are 46.61\% and 7.02\% for CSP-20 nodes, 42.24\% and 3.94\% for CSP-50 nodes, 1.12\% and 1.56\% for CVRP-20 nodes, 0.90\% and 5.05\% for CVRP-50 nodes, 5.35\% and 2.40\% for PG-20 nodes, and 12.17\% and 10.33\% for PG-50 nodes. Even when confronting the stochastic adversary, CCDO shows superior generalizability compared to two baselines across three COPs, with average improvements of 6.31\%, 3.42\%, and 3.95\% respectively. Detailed results are provided in Appendix \ref{app_exp} (Tables \ref{tab_csp_full_20} - \ref{tab_op_full_50}). 
% The model’s usability is enhanced by the ability to generalize rather than focusing solely on the average reward, which is a critical motivation of the RL for combinatorial optimization literature \citep{khalil2017learning, kool2018attention}.  

\begin{remark}
    In CO problems (or more broadly, operations research and economics), it is known that achieving solution quality improvements against strong baselines (e.g., the RL methods trained with a stochastic adversary) is very challenging, and the margins are usually small \citep{kool2018attention}, sometimes even less than 1\%. However, these “tiny” marginal improvements in profits keep small business owners in the real world alive. Last, the improvement depends a lot on the problem settings, and we show that sometimes the improvement can be much more significant.
\end{remark}
\vspace{-\baselineskip}
% \textbf{Performance analysis.} The robustness results of CCDO-RL for ACSP are shown in Table \ref{tab_csp}. We have the following observations: 1) On both of the 100 seen/unseen graphs, single-player RL performs better than heuristic algorithms no matter whether attacked or not. (2) When confronting the adversary trained by CCDO-RL, CCDO-RL exceeds RL by 0.25 and 0.24 on the training set, and by 0.25 and 0.18 on the test set, respectively under the 20-node and 50-node graphs. This demonstrates the robustness of CCDO-RL. 3) Compared to the performance of the training set with that of the test set, we can see that RL and CCDO-RL both maintain a certain degree of generalization. Similar results for ACVRP (Table \ref{tab_cvrp}) and SPG (Table \ref{tab_op}) are provided in Appendix \ref{app_exp}. 

\begin{table}[ht]
  \caption{Average reward against CCDO-RL's adversary (on seen graphs)}
  \vspace{\baselineskip}
  \label{tab_aver}
  \centering
  \small
  \begin{tabular}{lllllll}
    \toprule
    \multirow{2}{*}{method} & \multicolumn{2}{c}{ACSP (Mean$\pm$Std)} & \multicolumn{2}{c}{ACVRP (Mean$\pm$Std)} & \multicolumn{2}{c}{PG (Mean$\pm$Std)} \\
    \cmidrule(r){2-3} \cmidrule{4-5} \cmidrule(r){6-7}
                            & 20 nodes & 50 nodes & 20 nodes & 50 nodes & 20 nodes & 50 nodes\\
    \midrule
    Heuristic & 6.13$\pm$1.20 & 7.55$\pm$1.42 & 7.65$\pm$1.23  & 13.38$\pm$1.70 & 2.64$\pm$1.03 & 4.53$\pm$1.84   \\
    RL against Stoc    & 3.50$\pm$0.47  & 4.55$\pm$0.62  & 7.55$\pm$1.16  & 13.90$\pm$1.63 & 2.71$\pm$0.90 & 4.80$\pm$2.18   \\
    CCDO-RL   & $\pmb{3.25}$$\pm$0.42 & $\pmb{4.31}$$\pm$0.51  & $\pmb{7.42}$$\pm$1.21  & $\pmb{13.28}$$\pm$1.52 &  $\pmb{2.75}$$\pm$0.87 & $\pmb{5.01}$$\pm$1.91  \\
    \bottomrule
  \end{tabular}
\end{table}
\vspace{-\baselineskip}

\begin{table}[htp]
  \caption{Generalizability against CCDO-RL's adversary (on unseen graphs)}
  \vspace{\baselineskip}
  \label{tab_gene}
  \centering
  \small
  \begin{threeparttable}
  \begin{tabular}{lllllll}
    \toprule
    \multirow{2}{*}{method} & \multicolumn{2}{c}{ACSP (Mean$\pm$Std)} & \multicolumn{2}{c}{ACVRP (Mean$\pm$Std)} & \multicolumn{2}{c}{PG (Mean$\pm$Std)} \\
    \cmidrule(r){2-3} \cmidrule{4-5} \cmidrule(r){6-7}
                            & 20 nodes & 50 nodes & 20 nodes & 50 nodes & 20 nodes & 50 nodes\\
    \midrule
    Heuristic & 6.20$\pm$1.33 & 7.60$\pm$1.37   & 7.64$\pm$1.30  & 13.27$\pm$1.87 & 2.43$\pm$0.98 & 4.19$\pm$1.69    \\
    RL against Stoc  & 3.56$\pm$0.37  & 4.57$\pm$0.58  & 7.67$\pm$1.30  & 13.85$\pm$1.53 &  2.50$\pm$0.95 & 4.26$\pm$2.17 \\
    CCDO-RL   & $\pmb{3.31}$$\pm$0.35 & $\pmb{4.39}$$\pm$0.52  & $\pmb{7.55}$$\pm$1.28  & $\pmb{13.15}$$\pm$1.59 & $\pmb{2.56}$$\pm$0.92 & $\pmb{4.70}$$\pm$1.94\\

    \bottomrule
  \end{tabular}
  \begin{tablenotes}
      \footnotesize
      \item[1] For the average reward of ACSP and ACVRP, smaller is better while for that of PG larger is better.
  \end{tablenotes}
  \end{threeparttable}
\end{table}
\vspace{-\baselineskip}
% two heuristics and one RL
% \begin{table}[ht]
%   \caption{{\color{red} Average reward of CCDO-RL (on seen graphs). For the value of CSP and CVRP, larger is better while for that of PG smaller is better.}}
%   \label{tab_aver}
%   \centering
%   \small
%   \begin{tabular}{lllllll}
%     \toprule
%     \multirow{2}{*}{method} & \multicolumn{2}{c}{CSP (Mean$\pm$Std)} & \multicolumn{2}{c}{CVRP (Mean$\pm$Std)} & \multicolumn{2}{c}{PG (Mean$\pm$Std)} \\
%     \cmidrule(r){2-3} \cmidrule{4-5} \cmidrule(r){6-7}
%                             & 20 nodes & 50 nodes & 20 nodes & 50 nodes & 20 nodes & 50 nodes\\
%     \midrule
%     Baseline 1 & 4.52$\pm$0.71  & 5.98$\pm$0.94 & 7.64$\pm$1.56  & 13.49$\pm$2.10 & 2.71$\pm$1.10 & 1.82$\pm$1.40   \\
%     Baseline 2 & 6.13$\pm$1.20 & 7.55$\pm$1.42   & 7.65$\pm$1.23  & 13.38$\pm$1.70 & 2.64$\pm$1.03 & 1.47$\pm$0.99  \\
%     RL {\color{red}against Stoc}    & 3.50$\pm$0.47  & 4.55$\pm$0.62  & 7.55$\pm$1.16  & 13.90$\pm$1.63 & 2.71$\pm$0.90 & 1.54$\pm$1.03   \\
%     CCDO-RL   & $\pmb{3.25}$$\pm$0.42 & $\pmb{4.31}$$\pm$0.51  & $\pmb{7.42}$$\pm$1.21  & $\pmb{13.28}$$\pm$1.52 &  $\pmb{2.75}$$\pm$0.87 & $\pmb{1.87}$$\pm$1.22  \\
%     \bottomrule
%   \end{tabular}
% \end{table}


% \begin{table}[htp]
%   \caption{{\color{red}Generalizability of CCDO-RL (on unseen graphs)}}
%   \label{tab_gene}
%   \centering
%   \small
%   \begin{threeparttable}
%   \begin{tabular}{lllllll}
%     \toprule
%     \multirow{2}{*}{method} & \multicolumn{2}{c}{CSP (Mean$\pm$Std)} & \multicolumn{2}{c}{CVRP (Mean$\pm$Std)} & \multicolumn{2}{c}{PG (Mean$\pm$Std)} \\
%     \cmidrule(r){2-3} \cmidrule{4-5} \cmidrule(r){6-7}
%                             & 20 nodes & 50 nodes & 20 nodes & 50 nodes & 20 nodes & 50 nodes\\
%     \midrule
%     Baseline 1 & 4.53$\pm$0.79  & 5.95$\pm$0.96 & 7.55$\pm$1.39  & 13.35$\pm$2.04 & 2.52$\pm$1.08 & $\pmb{1.86}$$\pm$1.44  \\
%     Baseline 2 & 6.20$\pm$1.33 & 7.60$\pm$1.37   & 7.64$\pm$1.3  & 13.27$\pm$1.87 & 2.43$\pm$0.98 & 1.52$\pm$1.20    \\
%     RL {\color{red}against Stoc}  & 3.56$\pm$0.37  & 4.57$\pm$0.58  & 7.67$\pm$1.30  & 13.85$\pm$1.53 &  2.50$\pm$0.95 & 1.03$\pm$5.05 \\
%     CCDO-RL   & $\pmb{3.31}$$\pm$0.35 & $\pmb{4.39}$$\pm$0.52  & $\pmb{7.55}$$\pm$1.28  & $\pmb{13.15}$$\pm$1.59 & $\pmb{2.56}$$\pm$0.92 & 1.35$\pm$5.09\\

%     \bottomrule
%   \end{tabular}
%   \begin{tablenotes}
%       \footnotesize
%       \item[1] For the value of CSP and CVRP, larger is better while for that of PG smaller is better.
%   \end{tablenotes}
%   \end{threeparttable}
% \end{table}


\section{Conclusion}
In this paper, we focus on client simulation for Motivational Interviewing (MI) counseling, a complex scenario where the client should exhibit varied behaviors in different counseling stages. We introduce a consistent client simulation framework that integrates state tracking and action selection with an LLM to accurately mimic the behaviors of real clients according to given profiles. Based on our framework, we develop a client simulation method that supports changes of client states consistent with real world MI-counseling using state transition control, and utterances consistent with real world client's profile using action and information selection modules. These modules utilizes action distribution from real datasets for different levels of receptivity. The action selection mechanism incorporates both context-aware and (state,receptivity)-aware action distributions, to align with the expected client behavior. Extensive evaluations using GPT-4 and human assessments demonstrate the high consistency of our method with real clients, outperforming the existing client simulation methods. With our client simulation framework, we believe that more realistic simulated clients can be implemented for training and evaluating counselors.

\section*{Limitations}
Although our simulation framework shows potential for future use in counseling, there are three main limitations to this work. 

Similar to other LLM prompting approaches, our framework also suffers from prompt sensitivity, as it incorporates multiple steps of prompting, which may lead to error accumulation. To mitigate such sensitivity, we employ a few-shot method to formulate our prompts. The design of effective prompts and their evaluation are therefore important areas for further research.

This paper focuses on client simulation only.  As a good counseling session also depends on the performance of counselor, we also need to develop highly competent counselor agents and evaluate them. Nevertheless, without a diverse set of clients, it is very difficult to meet this objective. In our future work, we will therefore leverage our client simulation framework to design better counselor agents and to develop a robust approach to evaluate their counseling skills.

Our current study focuses on real client simulation and conduct experiments on a single dataset (i.e. AnnoMI). We also restrict our study to MI approach to counseling.  Our work however can be easily extended to involve more real world counseling datasets as long as privacy and ethical concerns are addressed. We can also extend our framework to accommodate other counseling approaches (e.g., Cognitive behavioral therapy) as it is quite common for experienced counselors to apply a mixture of different approaches in a single counseling session so that to both personalize to the real clients and achieve the best outcome. 

\section*{Ethics Statement}
For the revision of the framework and any related annotation work, we conducted a thorough survey of psychological knowledge and datasets pertaining to motivational interviewing counseling. All annotations were performed by our co-authors.

%\section{Acknowledgment}
%This research / project is supported by the Singapore’s Ministry of Health’s National Medical Research Council under its Population Health Research Grant Thematic Category (PHRGTC-5-005). Any opinions, findings and conclusions or recommendations expressed in this material are those of the author(s) and do not reflect the views of MOH/NMRC

% Bibliography entries for the entire Anthology, followed by custom entries
%\bibliography{anthology,custom}
% Custom bibliography entries only
\bibliography{custom}

\appendix

\section{Data Annotation and Analysis}
\label{app:data annotation}

The analysis of both observed (i.e., AnnoMI) and generated MI counseling sessions requires them to be annotated at the session level so as to obtain the client's profile components (i.e., personas, beliefs, motivation, and acceptable change plan) and receptivity.  Tables~\ref{tab:profile annotation} and \ref{tab:receptivity annotation} show corresponding prompts.  We also develop prompts to annotate at a specific point of a session to obtain the client's state and action at that point (see Table~\ref{tab:state annotation} and \ref{tab:action annotation}).  Finally, we create the prompts to determine entailment between generated and ground truth profile components (see Table~\ref{tab:entailment annotation}).  
GPT-4, one of the top proprietary LLMs, is chosen to be the target LLM to perform annotation of the AnnoMI and generated sessions.

In the following, we report the analysis results after annotating the 86 selected AnnoMI sessions.

\paragraph{AnnoMI Dataset} We utilize the AnnoMI dataset~\citep{wu2022anno}, which comprises 133 conversations spanning a diverse range of behavior topics, including ``reducing alcohol consumption'' and ``smoking cessation.'' Each conversation was transcribed from an expert-generated demonstration video and subsequently labeled as demonstrating high- or low-quality MI based on the video title and description. The 110 dialogues illustrating high-quality MI, comprising over 8,800 utterances in total, are considered as candidates. Consequently, we eliminate the sessions that are incomplete and those involving clients with insufficient profile information. To be noticed, the AnnoMI dataset is compiled from YouTube videos depicting MI counseling sessions between actors. However, it remains widely utilized due to restricted access to real counseling sessions. Given the absence of ideal datasets, our framework offers a solution to the scarcity of data by simulating clients interacting with counselors, thus bypassing ethical and privacy issues. Additionally, the primary use of the AnnoMI dataset is to extract client profiles and corresponding counseling sessions for evaluation purposes. Although these sessions are not “real”, they are still consistent with the client profiles and serve as valuable evaluation dataset.

\begin{figure*}[t]
    \centering
    \includegraphics[width=\textwidth]{figs/receptivity.pdf}
    \caption{The distribution of receptivity and the relation between receptivity and sustain ratio and precontemplation.}
    \label{fig:suggest_annotation}
\end{figure*}

\begin{figure*}[tb]
    \centering
    \includegraphics[width=\textwidth]{figs/action.pdf}
    \caption{Proportions of actions for different receptivity scores in different states. Actions negatively associated with receptivity are represented in dashed lines.}
    \label{fig:action_distribution}
\end{figure*}

\paragraph{Analysis of receptivity, associated talk types, and client states.} As shown in Figure~\ref{fig:suggest_annotation}(a), most clients demonstrate moderate receptivity. Few clients have high and low receptivities.  We next analyse the client utterances labeled with talk-types ``change talk'', ``neutral'' and ``sustain talk''. utterances annotated with ``Change talk'' express an interest to change behavior, while those annotated with ``sustain talk'' are the opposite~\citep{hoang2024can, miller2012motivational}. Figure~\ref{fig:suggest_annotation}(b) shows that the ratio of sustain talk-labeled utterances has a negative relationship with receptivity, i.e., clients with higher receptivity tend to use neutral or change talk more often.  Figure~\ref{fig:suggest_annotation}(c) shows that the proportion of client utterances in Precontemplation state also has a negative relationship with receptivity as clients in this state are reluctant and/or the counselor is not able to effect behavior change. Clients with lower receptivity are harder to reach the contemplation state, thus requiring more effort from the counselor.

\paragraph{Analysis of receptivity and associated actions.} Figure~\ref{fig:action_distribution}(a) illustrates the proportion of actions in different client states across different receptivity scores. Considering the utterances of clients who are in the precontemplation state, the proportion of utterances annotated with Inform action increases with receptivity due to the more receptive clients providing information about themselves.  The proportion with Engage action also increases slightly.  On the other hand, the proportions of Deny, Downplay and Blame actions reduces as receptivity increases. Similar findings also apply to Figures~\ref{fig:action_distribution}(b) and (c).  For easy reading, the actions negatively associated with receptivity are shown in dashed lines.  %Throughout the three states, actions associated with sustain talk, such as Deny, Downplay, and Hesitate, consistently decrease as receptivity increases. Conversely, actions associated with change talk, such as Acknowledge, Accept, and Plan, increase with higher receptivity. For neutral talk, Inform consistently rises with increasing receptivity, although Engage shows unstable trends. 
Finally, Inform and Engage are two actions found in all the three states. Inform is most predominant in the Precontemplation state implying that clients are more likely to share profile information during Precontemplation and adopt more varied actions after they transit to Contemplation and Preparation states. 

\begin{table*}[tb]
\begin{tabularx}{\textwidth}{X}
\toprule
\#\# Task \newline Your task is to identify the client's profile based on the provided counseling conversation. Focus on the following aspects: \newline  - **Persona**: Include personal details such as recent events, family relationships, and occupation. Summarize each aspect in concise sentences. \newline - **Behavioral Problem**: Identify a key problematic behavior (e.g., drinking, smoking). Describe it succinctly in a single phrase, focusing on the primary issue. \newline - **Motivation (Optional): Explain why the client wants to change this behavior (due to family, health, work, etc.) in one sentence, focusing on the main reason. Leave this section blank if the client shows no motivation to change.  \newline - **Beliefs**: Detail the client's inner beliefs toward change, such as downplaying the behavior's frequency, blaming others, or doubting the benefits of change. Provide a sentence for each reason. \newline  - **Acceptable Plan** (Option): Describe any plans the client is willing to adopt or consider, like reducing frequency, altering the environment, or seeking help. Use one sentence for each plan. Leave this blank if the client accepts no plans. \newline  \#\# Output Format  \newline Return the client profile in JSON format as illustrated below:  \newline [example]  \newline \#\# Given Conversation [conversation]  \newline \#\# Instruction  \newline Based on the conversation, provide the client's profile in JSON format.
\\ \bottomrule
\end{tabularx}
\caption{Prompt for GPT-4 to annotate the profile of client in given conversation or session ([conversation] is to be replaced by the session content).}
\label{tab:profile annotation}
\end{table*}


\begin{table*}[tb]
\begin{tabularx}{\textwidth}{X}
\toprule
\#\# Task
Your task is to read through the provided counseling conversation and determine which stage of change the client is in based on the Transtheoretical Model of behavior change. The stages you will focus on are Precontemplation, Contemplation, and Preparation. \newline \#\# Definitions of Stages \newline Precontemplation: The client is not yet acknowledging that there is a problem behavior that needs to be changed. They may be defensive, dismissive, or indifferent about the topic of change. \newline Contemplation: The client acknowledges that there is a problem and begins to think about the possibility of change. However, they are not yet committed to taking action but are more open to discussing the pros and cons of their behavior. \newline Preparation: The client is planning to change and is likely to start taking steps soon. They might begin to set goals, seek information, or plan out the changes they intend to make.  \newline \#\# Instructions \newline Read the Conversation: Carefully read the entire conversation to understand the context and content of the discussion.
\newline Identify Statements and Attitudes: Pay close attention to the client’s statements and attitudes towards change. Look for keywords or phrases that indicate their stage of change according to the definitions provided. \newline \#\# Determine the Stage \newline - If the client shows no recognition of the problem or need for change, categorize them as Precontemplation.  \newline - If the client acknowledges the problem and discusses thoughts about possibly changing, without commitment, categorize them as Contemplation. \newline - If the client talks about specific plans or the intention to change soon, categorize them as Preparation. \newline \#\# Justify Your Choice  \newline Provide a brief justification for the stage you have assigned based on specific parts of the [conversation]. Include direct quotes or clear references to the conversation to support your decision.  \newline \#\# Given Conversation: [conversation]  \newline Annotate the state of each utterance of client.
\\ \bottomrule
\end{tabularx}
\caption{Prompt for GPT-4 to annotate the latest stage (or state) of client in given conversation ([conversation] is to be replaced by the session which as be complete or partially complete).}
\label{tab:state annotation}
\end{table*}

\begin{table*}[tb]
\begin{tabularx}{\textwidth}{X}
\toprule
\#\# Task \newline Your task is to annotate the action of the client for each provided conversation snippet. For each snippet, focus on the client's last utterance and choose the most appropriate action from the given set of action. \newline  \#\# Action Options \newline [options] \newline \#\# Instructions \newline - Read the Conversation Snippet: Carefully read through the provided snippet. \newline - Focus on the Last Utterance: Concentrate on the client's last utterance. \newline - Choose the Action: Select the action that best describes the client's last utterance from the given set of actions. \newline  \#\# Response Format \newline Analysis of Client's Action: Provide a brief analysis of the client's last utterance and the chosen action. \newline Chosen Action: [Selected action] \newline For example: \newline Analysis of Client's Action: The client shows uncertainty about the effectiveness of potential stress management strategies, indicating hesitation.  \newline Chosen Action: Hesitate  \newline Provided Conversation Snippet for Annotation [conversation]  \newline What is the most appropriate action that describes the client's last utterance ([last utterance]) in the conversation snippet? Provide a brief analysis to support your choice.
\\ \bottomrule
\end{tabularx}
\caption{Prompt for GPT-4 to assign  from a list of possible actions [options] the action of the last client's utterance ([last utterance]) in the given conversation snippet ([conversation]).}
\label{tab:action annotation}
\end{table*}


\begin{table*}[tb]
\begin{tabularx}{\textwidth}{X}
\toprule
\#\# Task \newline Your task is to assess the client's receptivity in a given counseling conversation by focusing on the client's behaviors, openness, and responsiveness to suggestions. Determine how easily the counselor can motivate the client to change. \newline \#\# Guidelines \newline The client with higher receptivity tends to openly share information, accept suggestions, and show high confidence in their ability to change. The client with lower receptivity tends to resist suggestions, deny the need for change, downplay the importance of change, and exhibit other passive attitudes. \newline \#\# Scoring Receptivity \newline - **High Receptivity (5)**: A client with high receptivity consistently demonstrates behaviors such as frequently and openly sharing personal thoughts and feelings, actively engaging in discussions about change, quickly and positively responding to suggestions, and displaying strong self-belief in their ability to make changes. \newline - **Moderately High Receptivity (4)**: Clients with moderately high receptivity are generally open and forthcoming, mostly receptive to change discussions, respond well to suggestions with occasional need for reinforcement, and demonstrate good self-belief with some need for reassurance. \newline - **Moderate Receptivity (3)**: Clients with moderate receptivity show a balance between sharing and withholding information, exhibit mixed interest in change discussions, respond to suggestions neutrally, and display a balanced view of their ability to change. \newline - **Moderately Low Receptivity (2)**: Clients with moderately low receptivity are characterized by reluctance to share personal information, limited interest in change discussions, resistance to suggestions, and low self-belief with frequent expression of doubts. \newline - **Low Receptivity (1)**: Clients with low receptivity consistently withhold information, avoid discussions about change, strongly resist suggestions, and show very low self-belief, often highlighting reasons why change is not possible. \newline \#\# Important Note \newline Remember that the client may be motivated to change by the end of the conversation due to the counselor's efforts. However, your assessment should be based on the client's behaviors throughout the conversation especially before issue acknowledgement. \newline \#\# Response Format \newline Analysis of Client's Behavior: Provide a detailed analysis of the client's behaviors during the conversation. \newline Receptivity Score: Assign a final score based on the observed behaviors using the provided scoring system. \newline For Example:  \newline Analysis of Client's Behavior: The client initially exhibited resistance to change by denying the need for it. As the conversation progressed, the client showed some openness by considering small changes but remained hesitant and doubtful. Overall, the client demonstrated a low receptivity behaviors. \newline Receptivity Score: 1 (Low receptivity) \newline \#\# Provided Conversation for Evaluation \newline [conversation]  \newline What is the client's receptivity score based on the observed behaviors in the conversation? Provide a detailed analysis of the client's behaviors to support your assessment.
\\ \bottomrule
\end{tabularx}
\caption{Prompt for GPT-4 to annotate the receptivity score of client (1 to 5) in given conversation snippet ([conversation] will be replaced by the counseling session).}
\label{tab:receptivity annotation}
\end{table*}


\begin{table*}[tb]
\begin{tabularx}{\textwidth}{X}
\toprule
\#\# Task \newline You task is to identify whether the premise entails the hypothesis. The answer should be exact 'entail' or 'not entail'. \newline \#\# Premise [profile] \newline \#\# Hypothesis [component]
\\ \bottomrule
\end{tabularx}
\caption{Prompt for the GPT-4 to evaluate the entailment between each generated profile component ([component]) and given profile ([profile]).}
\label{tab:entailment annotation}
\end{table*}


\section{Expert Annotation}
\label{app:human evaluation}


\subsection{Demographics for Annotators}

Our annotation team comprises four experts who are co-authors of this paper. Three of them hold postgraduate degrees in counseling and one is a senior-year university student pursuing a psychology major. They all have relevant experience in mental health counseling.

\subsection{Annotation Instruction}

Table~\ref{tab:human_instruction1} presents the profile consistency rating score descriptions and client profile components for an expert to evaluate the profile consistency scores of a generated counseling session in all the four profile aspects (i.e., persona, beliefs, motivation, and acceptable plan). Table~\ref{tab:human_instruction2} shows the realism rating score descriptions for an expert to assess the realism of client in a generated session. Finally, Table~\ref{tab:human_instruction3} shows the score descriptions for an expert to determine the receptivity level of client in a generated counseling session.

\begin{table*}[tb]
\begin{tabularx}{\textwidth}{X}
\toprule
\textbf{Consistency Rating (To rate how consistent is the client with the four aspects of profile)}\newline
5 (High): Perfectly aligns with profile across all aspects \newline
4 (Moderately High): Mostly aligns with profile with minor variations \newline
3 (Moderate): Partially aligns with profile with some discrepancies \newline
2 (Moderately Low): Limited alignment with profile with multiple discrepancies \newline
1 (Low): Consistently contradicts profile information \newline
\textbf{CLIENT PROFILE} \newline
\textbf{Behavior:} drinking alcohol \newline
\textbf{Personas:}\newline
- The client sneaked out with a friend to the park to has a couple of beers.\newline
- The client managed to avoid getting into trouble as he threw the beers away before the policy saw them.\newline
- If the the client's mother found out he was in trouble because of his drinking, she would kill him.\newline
- The client has an interest in soccer and has set a goal to play in college.\newline
- The client has aspirations of receiving scholarships to support his soccer career.\newline
\textbf{Beliefs:} \newline
- The client believes he can handle the dangerous situation to avoid getting into trouble. \newline
- The client has only drunk once or twice. \newline
- A couple of the client's friends drink too. \newline
\textbf{Motivation:} The client is motivated because of the risk of drinking alcohol in sports, as alcohol would affect the ability to play soccer. \newline
\textbf{Acceptable Plans:} None \newline
\textbf{SESSION} \newline
[conversation]
\\ \bottomrule
\end{tabularx}
\caption{Description of profile consistency rating scores, ground truth client profile, and counseling session for expert annotation.  The expert is required to give a rating for each profile aspect. ([conversation] will be replaced by the real or generated session.)}
\label{tab:human_instruction1}
\end{table*}


\begin{table*}[tb]
\begin{tabularx}{\textwidth}{X}
\toprule
\textbf{Realism Rating (To rate how realistic is the client compared to a human client)} \newline
5 (High): Shows authentic responses with natural flow and rich personal context\newline
4 (Moderately High): Shows mostly genuine responses with occasional scripted moments\newline
3 (Moderate): Shows mixed authentic and artificial responses\newline
2 (Moderately Low): Shows mostly artificial responses with few authentic moments\newline
1 (Low): Shows completely artificial responses with no authentic elements\newline
\textbf{SESSION}\newline
[conversation]
\\ \bottomrule
\end{tabularx}
\caption{Description of realism rating scores, and counseling session for expert annotation. ([conversation] will be replaced by the real or generated session.)}
\label{tab:human_instruction2}
\end{table*}

\begin{table*}[tb]
\begin{tabularx}{\textwidth}{X}
\toprule
\textbf{Receptivity Rating (To rate how receptive is the client in the session)} \newline
5 (High): Actively engages, shares openly, and responds positively \newline
3 (Moderate): Shows balanced sharing and mixed response patterns \newline
1 (Low): Consistently withholds and strongly resists engagement \newline
\textbf{SESSION} \newline
[conversation]
\\ \bottomrule
\end{tabularx}
\caption{Description of receptivity ratings, and counseling session for expert annotation. ([conversation] will be replaced by the real or generated session.)}
\label{tab:human_instruction3}
\end{table*}



\section{More Evaluation Results}
\label{app:more evaluation}

\subsection{Automated Turn-Level Evaluation}

\begin{table*}[tb]
\centering
\begin{tabular}{lrrrrrrrr}
\toprule
             & ROUGE-1 & ROUGE-2 & ROUGE-L & BERTScore \\ \midrule
Base         & 13.27   & 2.55    & 10.03   & 85.59     \\
\rowcolor[HTML]{D8D6D6} Example-based & 59.12   & 48.84   & 58.31   & 93.54     \\
Profile-based & 13.79   & 3.18    & 9.90    & 85.50     \\
Pro+Act-based & 17.04   & 4.22    & 13.93   & 86.72     \\ \hline
Ours         & 17.73   & 9.13    & 15.13   & 87.13     \\
Ours*        & \textbf{29.69}   & \textbf{20.19}   & \textbf{28.75}   & \textbf{90.46}  \\ \bottomrule
\end{tabular}
\caption{Results of turn-level evaluation. ``Our'' refers to our client simulation method. ``Ours*'' refers to a variant of our method provided with ground truth states, actions, and selected information. Other than Example-based method, our framework outperforms the remaining baseline methods.}
\label{tab:turn-eval}
\end{table*}

As previous works~\citep{wang2024towards,deng2023prompting,kang2024can} often evaluated generated utterances against ground truth utterances in conversational tasks, we also conduct turn-level evaluation which involves using a simulated client to generate the next client utterance given a prompt that includes all the session content up to the previous utterance. To show the upper bound performance of our simulated client method (i.e., ``Ours''), we provide the results of a variant of our method (i.e., ``Ours*'') that generates the next utterance using the ground truth state, action and selected information. As shown in Table~\ref{tab:turn-eval}, our method outperforms most of the baselines. The Example-based method has been shaded because it has an unfair advantage over others due to its ability to search all the client utterances including the ground truth utterance. Our method variant, Ours*, shows that there are still rooms for Our method to improve in ROUGE-1, ROUGE-2 and ROUGE-L. Our and Our* methods share very similar BERTScore performance. This indicates that the Our method can reproduce responses semantically similar to the ground truth utterances. 

Nevertheless, it is important to note that turn-level evaluation only measures quality of the next utterance assuming that all earlier utterances are known.  This assumption does not hold in a real world counseling session or application where the client and counselor do not follow some written counseling scripts. Hence, a method performing well in this evaluation method may not always do so at the session level.  For example, the Example-based method appears to perform well in the turn-level evaluation purely due to its ability to find the most similar client utterance, indicating that turn-level evaluation may not accurately measure a client's effectiveness in real world scenarios. Hence, consistency evaluation of client profile components in Section~\ref{sec:profile consistency}, and client receptivity are introduced in Section~\ref{sec:exp_eval}. We believe these results can provide a more comprehensive assessment of client performance.

\subsection{Evaluation on Multiple Backbones}

To demonstrate the generalizability of our framework, we employ various LLMs as the backbone to implement our framework. As presented in Table~\ref{tab:results for multiple models}, our method consistently outperforms the baselines across various backbones. The performance of LLMs in simulation varies due to their distinct instruction-following and conversation capabilities. In summary, the \texttt{gpt-4o}, \texttt{llama-70B}, and \texttt{mistral-large} perform comparably to the \texttt{gpt-3.5}, while the smaller LLMs (\texttt{ministral-8B} and \texttt{llama-8B}) exhibit significantly inferior performance.

\begin{table*}[tb]
\centering
\resizebox{\textwidth}{!}{
\begin{tabular}{lrrrrrrrrr}
\toprule
              & Personas$\uparrow$    & Motivation$\uparrow$         & Beliefs$\uparrow$      & Plans$\uparrow$       &$\rho$ $\uparrow$  & Avg. Rec.            & MR@20             & Avg.MS.      & Act. KL $\downarrow$ \\ \midrule
Real          & - & -    & -  & -  & - & 3.27$_{\pm1.12}$    & 0.48              & 27.56         & 0.00              \\ 
\toprule
\multicolumn{10}{c}{\cellcolor[HTML]{EFEFEF}\texttt{gpt-4o-2024-11-20}}                              \\ \midrule
Base          & 9.17        & 16.19             & 12.26         & 9.38       & -0.28   & 4.30$_{\pm0.31}$    & 1.00              & 7.30          & 0.40 \\
Example-based & 56.29       & 49.96             & 50.66         & 37.21       & 0.27   & 3.98$_{\pm0.68}$    & 1.00              & 9.12          & 0.23\\
Profile-based & 69.78       & 56.97             & 70.59         & 57.31       & 0.32   & 4.05$_{\pm0.72}$    & 0.90              & 10.26          & 0.15 \\
Pro+Act-based & 72.12       & 63.43             & 71.57         & 63.89       & 0.35   & 3.63$_{\pm0.97}$    & 0.85              & 11.80          & 0.14\\ \hline
Ours          & \textbf{76.10}   & \textbf{78.57}   & \textbf{76.13}  & \textbf{74.99} & \textbf{0.66}  & 3.24$_{\pm1.21}$    & 0.61              & 19.78         & \textbf{0.05} \\ \bottomrule
\toprule
\multicolumn{10}{c}{\cellcolor[HTML]{EFEFEF}\texttt{Mistral-Large-Instruct-2411}}                              \\ \midrule
Base          &  9.18       & 16.22             & 12.21         & 9.35        & -0.30  & 4.38$_{\pm0.38}$    & 1.00              & 7.22          & 0.38 \\
Example-based & 54.22       & 47.18             & 47.01         & 34.87       & 0.26    & 4.04$_{\pm0.69}$    & 1.00              & 8.06          & 0.23\\
Profile-based & 66.31       & 57.10             & 69.20         & 56.85       & 0.33   & 4.22$_{\pm0.63}$    & 0.93              & 10.05          & 0.15 \\
Pro+Act-based & 70.11       & 60.88             & 71.09         & 60.03       & 0.36   & 3.77$_{\pm0.91}$    & 0.88              & 10.97          & 0.12\\ \hline
Ours          & \textbf{73.21}   & \textbf{76.30}   & \textbf{73.13}  & \textbf{71.67} & \textbf{0.62}  & 3.17$_{\pm1.27}$    & 0.65              & 19.56         & \textbf{0.05} \\ \bottomrule
\toprule
\multicolumn{10}{c}{\cellcolor[HTML]{EFEFEF}\texttt{LLaMA-3.3-70B-Instruct}}                              \\ \midrule
Base          &  9.15       & 16.20             & 12.17         &  9.35       & -0.31  & 4.33$_{\pm0.43}$    & 1.00              & 6.97          & 0.40 \\
Example-based & 54.73       & 47.56             & 46.26         & 34.78       & 0.27    & 4.07$_{\pm0.62}$    & 1.00              & 8.10          & 0.23\\
Profile-based & 67.54       & 55.04             & 68.51         & 57.95       & 0.33   & 4.11$_{\pm0.67}$    & 0.91              & 10.76          & 0.14 \\
Pro+Act-based & 70.08       & 59.43             & 69.98         & 59.81       & 0.35   & 3.82$_{\pm0.98}$    & 0.89              & 10.83          & 0.13\\ \hline
Ours          & \textbf{73.92}   & \textbf{75.90}   & \textbf{73.10}  & \textbf{71.81} & \textbf{0.60}  & 3.20$_{\pm1.08}$    & 0.63              & 19.02         & \textbf{0.06} \\ \bottomrule
\toprule
\multicolumn{10}{c}{\cellcolor[HTML]{EFEFEF}\texttt{gpt-3.5-turbo-0125}}                              \\ \midrule
Base          &  9.01       & 16.17             & 12.15         &  9.30       & -0.31  & 4.42$_{\pm0.47}$    & 1.00              & 6.60          & 0.39 \\
Example-based & 53.68       & 45.73             & 45.55         & 33.53       & 0.25    & 4.08$_{\pm0.63}$    & 1.00              & 7.60          & 0.24\\
Profile-based & 61.97       & 53.44             & 67.17         & 54.67       & 0.31   & 4.12$_{\pm0.64}$    & 0.96              & 9.76          & 0.15 \\
Pro+Act-based & 67.09       & 55.33             & 68.60         & 57.17       & 0.33   & 3.86$_{\pm1.01}$    & 0.94              & 9.93          & 0.13\\ \hline
Ours          & \textbf{70.57}   & \textbf{73.37}   & \textbf{71.70}  & \textbf{68.51} & \textbf{0.58}  & 3.32$_{\pm1.15}$    & 0.69              & 18.60         & \textbf{0.06} \\ \bottomrule
\toprule
\multicolumn{10}{c}{\cellcolor[HTML]{EFEFEF}\texttt{Ministral-8B-Instruct-2410}}                              \\ \midrule
Base          &  9.10       & 16.17             & 12.09         &  9.27       & -0.34  & 4.54$_{\pm0.42}$    & 1.00              & 6.87          & 0.39 \\
Example-based & 50.54       & 40.88             & 39.45         & 28.51       & 0.13    & 4.49$_{\pm0.61}$   & 1.00              & 7.37          & 0.30\\
Profile-based & 56.77       & 48.23             & 61.81         & 48.50       & 0.21   & 4.40$_{\pm0.60}$    & 0.99              & 9.56          & 0.19 \\
Pro+Act-based & 60.39       & 52.78             & 64.91         & 51.73       & 0.28   & 4.09$_{\pm0.91}$    & 0.96              & 9.53          & 0.17\\ \hline
Ours          & \textbf{65.75}   & \textbf{68.70}   & \textbf{68.98}  & \textbf{61.71} & \textbf{0.51}  & 3.58$_{\pm1.23}$    & 0.72              & 17.67         & \textbf{0.10} \\ \bottomrule
\multicolumn{10}{c}{\cellcolor[HTML]{EFEFEF}\texttt{LLaMA-3.1-8B-Instruct}}                              \\ \midrule
Base          &  9.13       & 15.85             & 12.02         &  9.21       & -0.33  & 4.52$_{\pm0.39}$    & 1.00              & 6.75          & 0.38 \\
Example-based & 50.46       & 40.53             & 39.35         & 28.13       & 0.13    & 4.32$_{\pm0.68}$    & 1.00             & 7.22          & 0.31\\
Profile-based & 55.89       & 47.81             & 61.71         & 48.37       & 0.20   & 4.39$_{\pm0.58}$    & 0.99              & 9.72          & 0.20 \\
Pro+Act-based & 60.05       & 51.90             & 64.16         & 50.96       & 0.27   & 4.21$_{\pm0.95}$    & 0.96              & 9.49          & 0.15\\ \hline
Ours          & \textbf{64.57}   & \textbf{67.90}   & \textbf{67.68}  & \textbf{61.68} & \textbf{0.50}  & 3.52$_{\pm1.17}$    & 0.73              & 16.74         & \textbf{0.11} \\ \bottomrule
\end{tabular}}
\caption{Consistency Assessment of Personas, Motivations, Beliefs, and Preferred Change Plans, Spearman’s Correlation of Receptivity ($\rho$), the average receptivity level (Avg. Rec.), motivation rate (MR) in the initial 20 turns, and average motivation step for baselines and our methods based on various LLMs are presented. In all backbones, our method consistently outperforms all baselines.}
\label{tab:results for multiple models}
\end{table*}

\subsection{Employ Multiple Actions in One Response}

Our current study focuses on single action selection due to the observation that the majority of utterances (89.57\% in the AnnoMI dataset) involve only one action. Only 10\% of utterances involve multiple actions. From the perspective of our framework, incorporating more complex actions can be achieved seamlessly with minor modifications. These complex actions can be included in the action set so that they can be selected for utterance generation. Alternatively, multiple actions can be sampled from the action distribution. To demonstrate the capability of our framework in using multiple actions to generate an utterance, we present an example where multiple potential actions are sampled from the action distribution. To determine the number of actions for generating an utterance, we initially sample a number of actions from the observed data (i.e., 89(single):10(double):1(three)) as candidates. Subsequently, we sample actions from the candidate actions based on the action distribution. Finally, we simply combine all the descriptions of sampled actions as instructions. Table~\ref{tab:multiple actions} demonstrates an example where our framework selects multiple actions probabilistically to control more intricate behaviors in a single response. However, it is important to note that almost all responses can be generated within single actions, thus we believe the single action framework can easily extended to simulate the multiple action scenarios. 

\begin{table*}[htbp]
\centering
\begin{tabularx}{\textwidth}{X}
\toprule
\multicolumn{1}{c}{\textbf{Example of Client Employing Multiple Actions}} \\ \midrule                                                                                             
Counselor: Hello. How are you?                                                                                                                                                                                                                                                                                                \\
Client   : I am good. What about you?                                                                                                                                                                                                                                                                                         \\
Counselor: I'm doing well, thank you! Smoking can significantly impact your finances, from the cost of cigarettes to potential health-related expenses and lost income. I'm curious to hear your thoughts on how smoking might be affecting your financial situation. Please share more about that.                           \\
Client ({\color{red}Deny}, {\color{green}Downplay}): {\color{red} You know, I don't really think about the money too much.} {\color{green} Life is busy, and there are so many other things to focus on, like work and my daughter. I just enjoy my routine and don't see smoking as a big issue in the grand scheme.}                                                                                \\
Counselor: It's great that you have a busy life with work and your daughter! I'm curious, how do you think smoking might impact your productivity or relationships at work? Have you noticed any effects on your career or how you feel in the workplace? Remember, it's all about your choices.                              \\
... \\
Client (Blame): Honestly, almost of my friends smoked, it would be totally normal and we could hang out without feeling bad about it.                   \\
Counselor: While smoking seems normal, it poses significant health risks, including respiratory issues and chronic diseases, which can impact your energy and time with your daughter. Choosing healthier habits can enhance your well-being and ensure you can fully engage in those precious moments together.              \\
Client ({\color{red}Engage}, {\color{green} Blame}): {\color{red} I get what you're saying, but honestly, life is so stressful right now. Between work and everything with my daughter, I just need something to help me unwind.} {\color{green} Plus, everyone around me has their own coping mechanisms, so it feels normal to me.}                                                                 \\
Counselor: It sounds like smoking provides a familiar escape for you amidst the chaos of daily life, and it’s understandable to want that comfort. However, consider how quitting could enhance your physical fitness—boosting your energy and flexibility—ultimately making you more present and engaged with your daughter. \\
Client (Downplay): I see your point, but I really don't think my smoking is affecting my daughter. She's too young to understand, and I don't smoke around her anyway. Everyone has their own way to deal with stress, and mine happens to be smoking. It’s just part of my routine.                                                  \\
Counselor: I’m concerned that your smoking could be diminishing your energy and flexibility, potentially impacting your ability to keep up with your daughter’s activities. Quitting could not only improve your physical fitness but also enhance your enjoyment of those special moments together.                          \\
Client (Deny): I think that's a bit dramatic. My daughter loves it when I take her to the park and play games with her. She doesn't care that I smoke – she just loves spending time with me. And I'm right there with her, enjoying it too. Why would quitting make a difference?                                                \\
Counselor: I understand that relaxation is essential for you. However, viewing your stress and tiredness as signs to prioritize your health could guide you towards healthier coping strategies. Reducing alcohol might enhance your energy and overall well-being, allowing you to embrace life with greater vitality.       \\
Client (Downplay): I guess we all have our ways of coping. Honestly, I think I have it under control.                                                                          \\
Counselor: I understand that smoking feels manageable, but it might limit your energy for activities with your daughter. Quitting could enhance your stamina, improve your health, and help you engage more fully in those cherished moments together, supporting both your well-being and your relationship.                 \\
Client ({\color{red} Downplay}, {\color{green} Inform}): {\color{red}I see your point, but I really don’t think my smoking is a big deal. It’s just part of my routine, and we have fun together anyway.} {\color{green}I don’t want my daughter to see it as something negative, but I also feel it doesn’t affect our time together much.}                                                            \\
...                                                                                                               \\ \bottomrule    
\end{tabularx}
\caption{An example demonstrates the ability of our framework to assign multiple actions for one response. This example highlights the extensibility of our framework to more complex control scenarios involving multiple actions within a single response. Each utterance corresponding to a distinct action is distinguished by a unique color.}
\label{tab:multiple actions}
\end{table*}

\subsection{Complex State Transition}

Although our framework primarily employs a unidirectional flow of state tracking, which may limit its practicality, it is still an initial attempt to introduce a state tracking module in client simulation based on observed state flow from the referenced AnnoMI dataset. However, relapse scenarios (i.e., clients returning to pre-contemplation state from contemplation state) were not observed in the AnnoMI data, preventing the derivation of their sampling distribution. To support more complex state flows, we would need to access large real-world counseling datasets, which are scarce. To demonstrate the extensibility of our framework to incorporate complex state flows, we simply employ a random method, where the client relapses to a previous state with probability (set as 0.3) in each turn. If the counselor fails to resolve the client’s hesitation or address the beliefs effectively during Contemplation, the client may relapse to Precontemplation and deny the awareness of change. In the Preparation state, if the counselor fails to motivate the client to make an acceptable plan effectively, the client may relapse to the Contemplation state. Furthermore, since relapse can occur in each turn, longer sessions may be penalized.

Table~\ref{tab:complex transition1} and Table~\ref{tab:complex transition2} demonstrate an example by employing such a complex state transition flow, where the client relapses to Precontemplation when the counselor fails to address the beliefs well. The experiment not only suggests the extensibility of our framework but also demonstrates the powerful controllability of our framework in state transition. It is important to note that such complex state flows could not be observed in the real AnnoMI dataset. Moreover, we believe that the relapse of client's states in real-world scenarios is more likely to occur across sessions instead of within a single session~\citep{hashemzadeh2019transtheoretical,prochaska1997transtheoretical,miller2012motivational}.

\begin{table*}[htbp]
\centering
\begin{tabularx}{\textwidth}{X}
\toprule
\multicolumn{1}{c}{\textbf{Example of Client Employing Complex State Flow (Part I)}} \\ \midrule   
Counselor: Hello. How are you? \\
Client: I am good. What about you? \\
Counselor: I'm glad to hear you're doing well! It’s important that you're here to reflect on your drinking. How do you think your alcohol use impacts your relationships, like with your family or friends? This could be a good opportunity to explore its effects. \\
Client {\color{green}(Precontemplation)}: Well, my friends and I just enjoy our time together, and it's more about socializing than anything else. My family has always had their own things going on, so it's not like my drinking affects them. Plus, everyone else drinks, so it's pretty normal, right? \\
Counselor: It sounds like you see drinking as a normal part of socializing with friends. Tell me more about what you enjoy most during those times together. Have you noticed any effects on how you feel, both physically and mentally, after a night of drinking? \\
Client {\color{green}(Precontemplation)}: I just feel relaxed and it’s a good way to unwind after a long week. Sometimes I might feel a bit sluggish the next day, but I think it’s pretty normal. I don’t see having a few beers as a problem—it's just part of having a good time. \\
Counselor: I understand drinking feels like a normal part of socializing. However, reducing alcohol can improve your strength and flexibility, making activities more enjoyable. Have you noticed how drinking affects your performance in sports or other physical activities with friends? Healthier choices can enhance both fun and fitness! \\
Client {\color{green}(Precontemplation)}: Actually, I've noticed I get a lot more tired after a few drinks, so I try to limit my activity after a night out. But overall, my performance in sports hasn't really been affected by drinking – I still do fine. It's not a big deal, really. \\
Counselor: It’s great that you feel fine in sports, but have you considered how the tiredness from drinking might impact your overall fitness goals? Reducing alcohol could boost your endurance and energy levels, making your workouts more effective. What are your thoughts on that? \\
{\color{blue} Transfer to Contemplation} \\
Client {\color{red} (Contemplation)}: You make a really good point. I hadn't thought about how the tiredness from drinking could really hold me back from achieving my fitness goals. I want to feel more energetic and perform better in my workouts, and I can see how cutting back on alcohol could help with that. I'm definitely motivated to make some changes! If I can improve my endurance and get more out of my training, it would definitely be worth it. Thanks for highlighting that. \\
Counselor: It sounds like your tiredness after drinking could be a signal that your body is ready for a change. By reducing alcohol, you might not only boost your energy but also enhance your physical activity, helping you achieve those fitness goals you’re excited about! \\                                                                                                \\ \bottomrule    
\end{tabularx}
\caption{An example (part I) showcases the capability of our client framework to manage intricate state transitions. This example underscores the extensibility of our framework to more intricate state transition scenarios involving stochastic relapse to prior states. The states and transitions are visually distinguished by color.}
\label{tab:complex transition1}
\end{table*}

\begin{table*}[htbp]
\centering
\begin{tabularx}{\textwidth}{X}
\toprule
\multicolumn{1}{c}{\textbf{Example of Client Employing Complex State Flow (Part II)}} \\ \midrule   
Client {\color{red} (Contemplation)}: I guess part of me enjoys the social aspect of drinking, but I also feel sluggish and unmotivated sometimes. I used to be more active, and I miss that energy. It's tough to balance having fun with feeling good physically. \\
Counselor: It's interesting that you feel sluggish after drinking. Reducing alcohol might not only boost your energy for workouts but also improve your oral hygiene and overall appearance, enhancing your independence and quality of life. This change could support both your fitness and personal health goals! \\
{\color{blue} Relapse to Contemplation} \\
Client {\color{green}(Precontemplation)}: I guess I just don't see my drinking as a big deal. It's not like I'm doing it too much. My friends and I just like to hang out and have a few beers. If they were all quitting, maybe I would too. But they don't seem to be having any problems with it. \\
Counselor: I understand that your drinking feels normal in your social circle. However, have you considered how it might impact your oral health or overall appearance over time? Recognizing this could improve not just your fitness goals, but also your long-term quality of life. What are your thoughts on that? \\
Client {\color{green}(Precontemplation)}: I see your point, but I feel like I can manage it for now. Besides, it's not just about fitness for me. Life's about enjoying moments too, right? Speaking of enjoyment, I've been thinking about planning a trip soon. It could be a nice change of scenery. \\
Counselor: Planning a trip sounds exciting! However, reducing alcohol intake could lead to deeper connections and enhance your family experiences during the trip. It might also help create memorable moments together, strengthening your relationships. How do you feel about exploring that possibility while traveling? \\
{\color{blue} Transfer to Contemplation} \\
Client {\color{red} (Contemplation)}: You make a good point. It would definitely be nice to be fully present during the trip and enjoy all those experiences without feeling sluggish. I've noticed that when I drink less, I tend to feel more energized, and that would really enhance my ability to explore and make the most out of the trip. Plus, staying active while traveling is something I really want to do. I guess if I shift my focus on creating lasting memories rather than just having a few drinks, it could make a big difference for me. Maybe I should give this a thought before my trip. \\
Counselor: That's a great realization about feeling more energized on the trip. I'd like to explore how reducing drinking might also help you reconnect with your family. Perhaps this trip can be an opportunity to heal relationships and create new memories together, rather than just fleeting social moments. \\
Client {\color{red} (Contemplation)}: I see what you mean. It's true that those deeper connections with family are so important. I’ve been thinking about how much I’d miss if I’m not fully engaged. Maybe cutting back on alcohol could really help me focus on those meaningful moments during the trip. \\
...                                                                                                               \\ \bottomrule    
\end{tabularx}
\caption{An example (part II) showcases the capability of our client framework to manage intricate state transitions. This example underscores the extensibility of our framework to more intricate state transition scenarios involving stochastic relapse to prior states. The states and transitions are visually distinguished by color.}
\label{tab:complex transition2}
\end{table*}



\section{\sys's Implementation}
\label{sec:implement}


We prototype \sys on top of Huggingface Transformers framework~\cite{wolf2020huggingface} using MoE-Infinity codebase~\cite{moe-infinity-code}.
% Since the open-sourced MoE-Infinity codebase~\cite{moe-infinity-code} lacks many features described in its original paper, such as we 
The implementation of \sys is described as follows.


\textbf{Expert Map Store} is implemented in Python using PyTorch~\cite{paszke2019pytorch} and NumPy~\cite{harris2020array} libraries.
We store all semantic embeddings and expert maps using \texttt{ndarrays} data structure for efficient array operations. The arrays are converted to tensors to compute similarity for expert map matching.


\textbf{Expert Map Matcher} is implemented in Python using PyTorch~\cite{paszke2019pytorch} and TorchMetrics~\cite{detlefsen2022torchmetrics} libraries.
We implement the pairwise computations, including similarity (\S\ref{subsec:design-similarity-match}) and redundancy (\S~\ref{subsec:design-expert-map-store}) scores, using the Cosine Similarity interfaces in TorchMetrics. 
We use the Python multithreading library to implement the asynchronous expert map matching and expert prefetching, where the threads share the same memory space with the Expert Map Store for efficient reading and writing.


\textbf{Expert Cache} is implemented in C++ based on MoE-Infinity codebase~\cite{moe-infinity-code}. 
% However, as the codebase lacks many features described in the original paper and contains some bugs, such as expert on-demand loading, we had to refactor the code.
The expert management in GPUs is implemented with the CUDA Runtime APIs~\cite{cuda-runtime-api}.
We implement the caching logic of \sys and fix critical bugs in the MoE-Infinity codebase to enable expert offloading.
Same with MoE-Infinity, \sys supports multi-GPU inference with expert parallelism, where the experts are mapped to different GPU devices for loading and offloading. We use a hash map to assign expert IDs to different GPUs and retrieve them during inference.
The expert assignment follows a round-robin manner to balance the overall GPU load.
Additionally, we use a multi-thread task pool in the GPU space to schedule and execute expert prefetching and on-demand loading tasks.

\begin{table*}[t]
\centering
\caption{A case study examining \model's response to a query about ``\emph{graders}'' in OpenAI's 12-day edu    cational series demonstrates its video retrieval capabilities and knowledge synthesis.}
\label{tab:case study}
\small
\begin{tabularx}{\textwidth}{X X}

\toprule
\textbf{Query:} Explain the purpose and functionality of 'graders' in the context of reinforcement fine-tuning.\\

\midrule
\textbf{Input Videos:} \href{https://www.youtube.com/playlist?list=PLOXw6I10VTv9lin5AzsHAHCTrC7BdVdEM}{12 Days of OpenAI} (A total of 12 videos) \\

\midrule

\textbf{\model\ Response:} 

\texttt{\#\#\# Understanding Gradation in Reinforcement Fine-Tuning}

\texttt{**Purpose of Graders**}

\texttt{Graders are integral components in the process of reinforcement fine-tuning, mainly utilized to evaluate and improve the performance of machine learning models...} (Omitted for brevity)

\texttt{**Functionality of Graders**}

\texttt{\colorbox{lightgreen}{The grading system operates by comparing a model's outputs against those of}}

\texttt{\colorbox{lightgreen}{verified correct responses. The mechanics of this assessment involve assigning}}

\texttt{\colorbox{lightgreen}{scores to the model's outputs based on their correctness. These scores range}}

\texttt{\colorbox{lightgreen}{from zero to one, where:}}


\texttt{\colorbox{lightblue}{- A score of **0** indicates that the model's answer is completely incorrect.}}

\texttt{\colorbox{lightblue}{- A score of **1** signals that the answer is fully correct.}}

\texttt{- Scores can also be fractional, allowing for partial credit. For instance, \colorbox{lightpink}{a score of **0.7** may denote that the model's suggestion was }}
\texttt{\colorbox{lightpink}{correct but not the highest in a ranked list of possible answers.}} 

\\

\texttt{The process includes:}

\texttt{1. **Input Comparison**: The grader takes both the model’s output and the correct answer, analyzing them to derive a score.}

\texttt{2. **Graded Outputs**: This interaction allows for a dynamic reflection of the model's performance, where the scores directly influence how the model is fine-tuned in subsequent training iterations.}

\texttt{3. **Feedback Loop**: The scores obtained from graders provide feedback that can be used to adjust learning parameters and improve the model’s future responses. This highlights the feedback mechanism that is crucial for reinforcement learning.}

\\

(The remaining content are omitted for brevity.)

\\

\midrule
\textbf{Retrieved Sources by \model:}

\model\ accurately retrieved four continuous video segments from the OpenAI's 12-day show, ranging from 10:00 to 12:00 in Day 2 of the video "\href{https://www.youtube.com/watch?v=yCIYS9fx56U}{Reinforcement Fine-Tuning}." Here, we highlight key moments relevant to the detailed content in the answer. From left to right, these are retrieved moments at timestamps \colorbox{lightgreen}{10:35}, \colorbox{lightblue}{10:39}, and \colorbox{lightpink}{11:10}, which provide informative insights that help \model\ give a comprehensive answer to the query.

\\

\begin{tabular}{ccc}
    {\includegraphics[width=0.3\textwidth]{figs/openai-1.png}} &
    {\includegraphics[width=0.3\textwidth]{figs/openai-2.png}} &
    {\includegraphics[width=0.3\textwidth]{figs/openai-3.png}} \\
\end{tabular}

\\

\bottomrule

\end{tabularx}
\vspace{-0.2in}
\end{table*}

\end{document}
