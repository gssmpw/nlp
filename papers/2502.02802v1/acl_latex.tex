% This must be in the first 5 lines to tell arXiv to use pdfLaTeX, which is strongly recommended.
\pdfoutput=1
% In particular, the hyperref package requires pdfLaTeX in order to break URLs across lines.

\documentclass[11pt]{article}

% Change "review" to "final" to generate the final (sometimes called camera-ready) version.
% Change to "preprint" to generate a non-anonymous version with page numbers.
% \usepackage[review]{acl}
\usepackage[final]{acl}

% Standard package includes
\usepackage{times}
\usepackage{latexsym}

% For proper rendering and hyphenation of words containing Latin characters (including in bib files)
\usepackage[T1]{fontenc}
% For Vietnamese characters
% \usepackage[T5]{fontenc}
% See https://www.latex-project.org/help/documentation/encguide.pdf for other character sets

% This assumes your files are encoded as UTF8
\usepackage[utf8]{inputenc}

% This is not strictly necessary, and may be commented out,
% but it will improve the layout of the manuscript,
% and will typically save some space.
\usepackage{microtype}

% This is also not strictly necessary, and may be commented out.
% However, it will improve the aesthetics of text in
% the typewriter font.
\usepackage{inconsolata}

%Including images in your LaTeX document requires adding
%additional package(s)
\usepackage{graphicx}

%custom packages
\usepackage{algorithm}
\usepackage{algorithmic}
\usepackage{tabularx}
\usepackage{booktabs}
\usepackage{colortbl}
\usepackage{multirow}
\usepackage{amsmath}

% If the title and author information does not fit in the area allocated, uncomment the following
%
%\setlength\titlebox{<dim>}
%
% and set <dim> to something 5cm or larger.

\title{Consistent Client Simulation for Motivational Interviewing-based Counseling}

% Author information can be set in various styles:
% For several authors from the same institution:
% \author{Author 1 \and ... \and Author n \\
%         Address line \\ ... \\ Address line}
% if the names do not fit well on one line use
%         Author 1 \\ {\bf Author 2} \\ ... \\ {\bf Author n} \\
% For authors from different institutions:
% \author{Author 1 \\ Address line \\  ... \\ Address line
%         \And  ... \And
%         Author n \\ Address line \\ ... \\ Address line}
% To start a separate ``row'' of authors use \AND, as in
% \author{Author 1 \\ Address line \\  ... \\ Address line
%         \AND
%         Author 2 \\ Address line \\ ... \\ Address line \And
%         Author 3 \\ Address line \\ ... \\ Address line}

% \author{First Author \\
%   Affiliation / Address line 1 \\
%   Affiliation / Address line 2 \\
%   Affiliation / Address line 3 \\
%   \texttt{email@domain} \\\And
%   Second Author \\
%   Affiliation / Address line 1 \\
%   Affiliation / Address line 2 \\
%   Affiliation / Address line 3 \\
%   \texttt{email@domain} \\}

\author{
 \textbf{Yizhe Yang\textsuperscript{1}~\thanks{Work was done during a visit at SMU.}},
 \textbf{Palakorn Achananuparp\textsuperscript{2}},
 \textbf{Heyan Huang\textsuperscript{1}~\thanks{Corresponding Author}},
 \textbf{Jing Jiang\textsuperscript{3}},
\\
 \textbf{John Pinto\textsuperscript{4}},
 \textbf{Jenny Giam\textsuperscript{5}},
 \textbf{Kit Phey Leng \textsuperscript{6}},
 \textbf{Nicholas Gabriel Lim \textsuperscript{7}},
 \\
 \textbf{Cameron Tan Shi Ern \textsuperscript{8}},
 \textbf{Ee-peng Lim\textsuperscript{2}}
% \\
%  \textbf{Ninth Author\textsuperscript{1}},
%  \textbf{Tenth Author\textsuperscript{1}},
%  \textbf{Eleventh E. Author\textsuperscript{1,2,3,4,5}},
%  \textbf{Twelfth Author\textsuperscript{1}},
% \\
%  \textbf{Thirteenth Author\textsuperscript{3}},
%  \textbf{Fourteenth F. Author\textsuperscript{2,4}},
%  \textbf{Fifteenth Author\textsuperscript{1}},
%  \textbf{Sixteenth Author\textsuperscript{1}},
% \\
%  \textbf{Seventeenth S. Author\textsuperscript{4,5}},
%  \textbf{Eighteenth Author\textsuperscript{3,4}},
%  \textbf{Nineteenth N. Author\textsuperscript{2,5}},
%  \textbf{Twentieth Author\textsuperscript{1}}
\\
\\
 \textsuperscript{1}Beijing Institute of Technology,
 \textsuperscript{2}Singapore Management University,
 \textsuperscript{3}Australian National University, 
 \\
 \textsuperscript{4}ThoughtFull,
 \textsuperscript{5}Singapore Institute of Technology,
 \textsuperscript{6}National Institute of Education,
 \\
 \textsuperscript{7}Singapore University of Social Sciences,
 \textsuperscript{8}National University of Singapore
\\
 \small{
   % \textbf{Correspondence:} \href{mailto:email@domain}{email@domain}
   \{yizheyang,hhy63\}@bit.edu.cn, \{palakorna,eplim\}@smu.edu.sg, jing.jiang@anu.edu.au
 }
}

\begin{document}
\maketitle
\begin{abstract}
Simulating human clients in mental health counseling is crucial for training and evaluating counselors (both human or simulated) in a scalable manner. Nevertheless, past research on client simulation did not focus on complex conversation tasks such as mental health counseling. In these tasks, the challenge is to ensure that the client's actions (i.e., interactions with the counselor) are consistent with with its stipulated profiles and negative behavior settings. In this paper, we propose a novel framework that supports consistent client simulation for mental health counseling. Our framework tracks the mental state of a simulated client, controls its state transitions, and generates for each state behaviors consistent with the client's motivation, beliefs, preferred plan to change, and receptivity.  By varying the client profile and receptivity, we demonstrate that consistent simulated clients for different counseling scenarios can be effectively created.  Both our automatic and expert evaluations on the generated counseling sessions also show that our client simulation method achieves higher consistency than previous methods.
\end{abstract}

\section{Introduction}
\label{sec:introduction}
The business processes of organizations are experiencing ever-increasing complexity due to the large amount of data, high number of users, and high-tech devices involved \cite{martin2021pmopportunitieschallenges, beerepoot2023biggestbpmproblems}. This complexity may cause business processes to deviate from normal control flow due to unforeseen and disruptive anomalies \cite{adams2023proceddsriftdetection}. These control-flow anomalies manifest as unknown, skipped, and wrongly-ordered activities in the traces of event logs monitored from the execution of business processes \cite{ko2023adsystematicreview}. For the sake of clarity, let us consider an illustrative example of such anomalies. Figure \ref{FP_ANOMALIES} shows a so-called event log footprint, which captures the control flow relations of four activities of a hypothetical event log. In particular, this footprint captures the control-flow relations between activities \texttt{a}, \texttt{b}, \texttt{c} and \texttt{d}. These are the causal ($\rightarrow$) relation, concurrent ($\parallel$) relation, and other ($\#$) relations such as exclusivity or non-local dependency \cite{aalst2022pmhandbook}. In addition, on the right are six traces, of which five exhibit skipped, wrongly-ordered and unknown control-flow anomalies. For example, $\langle$\texttt{a b d}$\rangle$ has a skipped activity, which is \texttt{c}. Because of this skipped activity, the control-flow relation \texttt{b}$\,\#\,$\texttt{d} is violated, since \texttt{d} directly follows \texttt{b} in the anomalous trace.
\begin{figure}[!t]
\centering
\includegraphics[width=0.9\columnwidth]{images/FP_ANOMALIES.png}
\caption{An example event log footprint with six traces, of which five exhibit control-flow anomalies.}
\label{FP_ANOMALIES}
\end{figure}

\subsection{Control-flow anomaly detection}
Control-flow anomaly detection techniques aim to characterize the normal control flow from event logs and verify whether these deviations occur in new event logs \cite{ko2023adsystematicreview}. To develop control-flow anomaly detection techniques, \revision{process mining} has seen widespread adoption owing to process discovery and \revision{conformance checking}. On the one hand, process discovery is a set of algorithms that encode control-flow relations as a set of model elements and constraints according to a given modeling formalism \cite{aalst2022pmhandbook}; hereafter, we refer to the Petri net, a widespread modeling formalism. On the other hand, \revision{conformance checking} is an explainable set of algorithms that allows linking any deviations with the reference Petri net and providing the fitness measure, namely a measure of how much the Petri net fits the new event log \cite{aalst2022pmhandbook}. Many control-flow anomaly detection techniques based on \revision{conformance checking} (hereafter, \revision{conformance checking}-based techniques) use the fitness measure to determine whether an event log is anomalous \cite{bezerra2009pmad, bezerra2013adlogspais, myers2018icsadpm, pecchia2020applicationfailuresanalysispm}. 

The scientific literature also includes many \revision{conformance checking}-independent techniques for control-flow anomaly detection that combine specific types of trace encodings with machine/deep learning \cite{ko2023adsystematicreview, tavares2023pmtraceencoding}. Whereas these techniques are very effective, their explainability is challenging due to both the type of trace encoding employed and the machine/deep learning model used \cite{rawal2022trustworthyaiadvances,li2023explainablead}. Hence, in the following, we focus on the shortcomings of \revision{conformance checking}-based techniques to investigate whether it is possible to support the development of competitive control-flow anomaly detection techniques while maintaining the explainable nature of \revision{conformance checking}.
\begin{figure}[!t]
\centering
\includegraphics[width=\columnwidth]{images/HIGH_LEVEL_VIEW.png}
\caption{A high-level view of the proposed framework for combining \revision{process mining}-based feature extraction with dimensionality reduction for control-flow anomaly detection.}
\label{HIGH_LEVEL_VIEW}
\end{figure}

\subsection{Shortcomings of \revision{conformance checking}-based techniques}
Unfortunately, the detection effectiveness of \revision{conformance checking}-based techniques is affected by noisy data and low-quality Petri nets, which may be due to human errors in the modeling process or representational bias of process discovery algorithms \cite{bezerra2013adlogspais, pecchia2020applicationfailuresanalysispm, aalst2016pm}. Specifically, on the one hand, noisy data may introduce infrequent and deceptive control-flow relations that may result in inconsistent fitness measures, whereas, on the other hand, checking event logs against a low-quality Petri net could lead to an unreliable distribution of fitness measures. Nonetheless, such Petri nets can still be used as references to obtain insightful information for \revision{process mining}-based feature extraction, supporting the development of competitive and explainable \revision{conformance checking}-based techniques for control-flow anomaly detection despite the problems above. For example, a few works outline that token-based \revision{conformance checking} can be used for \revision{process mining}-based feature extraction to build tabular data and develop effective \revision{conformance checking}-based techniques for control-flow anomaly detection \cite{singh2022lapmsh, debenedictis2023dtadiiot}. However, to the best of our knowledge, the scientific literature lacks a structured proposal for \revision{process mining}-based feature extraction using the state-of-the-art \revision{conformance checking} variant, namely alignment-based \revision{conformance checking}.

\subsection{Contributions}
We propose a novel \revision{process mining}-based feature extraction approach with alignment-based \revision{conformance checking}. This variant aligns the deviating control flow with a reference Petri net; the resulting alignment can be inspected to extract additional statistics such as the number of times a given activity caused mismatches \cite{aalst2022pmhandbook}. We integrate this approach into a flexible and explainable framework for developing techniques for control-flow anomaly detection. The framework combines \revision{process mining}-based feature extraction and dimensionality reduction to handle high-dimensional feature sets, achieve detection effectiveness, and support explainability. Notably, in addition to our proposed \revision{process mining}-based feature extraction approach, the framework allows employing other approaches, enabling a fair comparison of multiple \revision{conformance checking}-based and \revision{conformance checking}-independent techniques for control-flow anomaly detection. Figure \ref{HIGH_LEVEL_VIEW} shows a high-level view of the framework. Business processes are monitored, and event logs obtained from the database of information systems. Subsequently, \revision{process mining}-based feature extraction is applied to these event logs and tabular data input to dimensionality reduction to identify control-flow anomalies. We apply several \revision{conformance checking}-based and \revision{conformance checking}-independent framework techniques to publicly available datasets, simulated data of a case study from railways, and real-world data of a case study from healthcare. We show that the framework techniques implementing our approach outperform the baseline \revision{conformance checking}-based techniques while maintaining the explainable nature of \revision{conformance checking}.

In summary, the contributions of this paper are as follows.
\begin{itemize}
    \item{
        A novel \revision{process mining}-based feature extraction approach to support the development of competitive and explainable \revision{conformance checking}-based techniques for control-flow anomaly detection.
    }
    \item{
        A flexible and explainable framework for developing techniques for control-flow anomaly detection using \revision{process mining}-based feature extraction and dimensionality reduction.
    }
    \item{
        Application to synthetic and real-world datasets of several \revision{conformance checking}-based and \revision{conformance checking}-independent framework techniques, evaluating their detection effectiveness and explainability.
    }
\end{itemize}

The rest of the paper is organized as follows.
\begin{itemize}
    \item Section \ref{sec:related_work} reviews the existing techniques for control-flow anomaly detection, categorizing them into \revision{conformance checking}-based and \revision{conformance checking}-independent techniques.
    \item Section \ref{sec:abccfe} provides the preliminaries of \revision{process mining} to establish the notation used throughout the paper, and delves into the details of the proposed \revision{process mining}-based feature extraction approach with alignment-based \revision{conformance checking}.
    \item Section \ref{sec:framework} describes the framework for developing \revision{conformance checking}-based and \revision{conformance checking}-independent techniques for control-flow anomaly detection that combine \revision{process mining}-based feature extraction and dimensionality reduction.
    \item Section \ref{sec:evaluation} presents the experiments conducted with multiple framework and baseline techniques using data from publicly available datasets and case studies.
    \item Section \ref{sec:conclusions} draws the conclusions and presents future work.
\end{itemize}

\section{RELATED WORK}
\label{sec:relatedwork}
In this section, we describe the previous works related to our proposal, which are divided into two parts. In Section~\ref{sec:relatedwork_exoplanet}, we present a review of approaches based on machine learning techniques for the detection of planetary transit signals. Section~\ref{sec:relatedwork_attention} provides an account of the approaches based on attention mechanisms applied in Astronomy.\par

\subsection{Exoplanet detection}
\label{sec:relatedwork_exoplanet}
Machine learning methods have achieved great performance for the automatic selection of exoplanet transit signals. One of the earliest applications of machine learning is a model named Autovetter \citep{MCcauliff}, which is a random forest (RF) model based on characteristics derived from Kepler pipeline statistics to classify exoplanet and false positive signals. Then, other studies emerged that also used supervised learning. \cite{mislis2016sidra} also used a RF, but unlike the work by \citet{MCcauliff}, they used simulated light curves and a box least square \citep[BLS;][]{kovacs2002box}-based periodogram to search for transiting exoplanets. \citet{thompson2015machine} proposed a k-nearest neighbors model for Kepler data to determine if a given signal has similarity to known transits. Unsupervised learning techniques were also applied, such as self-organizing maps (SOM), proposed \citet{armstrong2016transit}; which implements an architecture to segment similar light curves. In the same way, \citet{armstrong2018automatic} developed a combination of supervised and unsupervised learning, including RF and SOM models. In general, these approaches require a previous phase of feature engineering for each light curve. \par

%DL is a modern data-driven technology that automatically extracts characteristics, and that has been successful in classification problems from a variety of application domains. The architecture relies on several layers of NNs of simple interconnected units and uses layers to build increasingly complex and useful features by means of linear and non-linear transformation. This family of models is capable of generating increasingly high-level representations \citep{lecun2015deep}.

The application of DL for exoplanetary signal detection has evolved rapidly in recent years and has become very popular in planetary science.  \citet{pearson2018} and \citet{zucker2018shallow} developed CNN-based algorithms that learn from synthetic data to search for exoplanets. Perhaps one of the most successful applications of the DL models in transit detection was that of \citet{Shallue_2018}; who, in collaboration with Google, proposed a CNN named AstroNet that recognizes exoplanet signals in real data from Kepler. AstroNet uses the training set of labelled TCEs from the Autovetter planet candidate catalog of Q1–Q17 data release 24 (DR24) of the Kepler mission \citep{catanzarite2015autovetter}. AstroNet analyses the data in two views: a ``global view'', and ``local view'' \citep{Shallue_2018}. \par


% The global view shows the characteristics of the light curve over an orbital period, and a local view shows the moment at occurring the transit in detail

%different = space-based

Based on AstroNet, researchers have modified the original AstroNet model to rank candidates from different surveys, specifically for Kepler and TESS missions. \citet{ansdell2018scientific} developed a CNN trained on Kepler data, and included for the first time the information on the centroids, showing that the model improves performance considerably. Then, \citet{osborn2020rapid} and \citet{yu2019identifying} also included the centroids information, but in addition, \citet{osborn2020rapid} included information of the stellar and transit parameters. Finally, \citet{rao2021nigraha} proposed a pipeline that includes a new ``half-phase'' view of the transit signal. This half-phase view represents a transit view with a different time and phase. The purpose of this view is to recover any possible secondary eclipse (the object hiding behind the disk of the primary star).


%last pipeline applies a procedure after the prediction of the model to obtain new candidates, this process is carried out through a series of steps that include the evaluation with Discovery and Validation of Exoplanets (DAVE) \citet{kostov2019discovery} that was adapted for the TESS telescope.\par
%



\subsection{Attention mechanisms in astronomy}
\label{sec:relatedwork_attention}
Despite the remarkable success of attention mechanisms in sequential data, few papers have exploited their advantages in astronomy. In particular, there are no models based on attention mechanisms for detecting planets. Below we present a summary of the main applications of this modeling approach to astronomy, based on two points of view; performance and interpretability of the model.\par
%Attention mechanisms have not yet been explored in all sub-areas of astronomy. However, recent works show a successful application of the mechanism.
%performance

The application of attention mechanisms has shown improvements in the performance of some regression and classification tasks compared to previous approaches. One of the first implementations of the attention mechanism was to find gravitational lenses proposed by \citet{thuruthipilly2021finding}. They designed 21 self-attention-based encoder models, where each model was trained separately with 18,000 simulated images, demonstrating that the model based on the Transformer has a better performance and uses fewer trainable parameters compared to CNN. A novel application was proposed by \citet{lin2021galaxy} for the morphological classification of galaxies, who used an architecture derived from the Transformer, named Vision Transformer (VIT) \citep{dosovitskiy2020image}. \citet{lin2021galaxy} demonstrated competitive results compared to CNNs. Another application with successful results was proposed by \citet{zerveas2021transformer}; which first proposed a transformer-based framework for learning unsupervised representations of multivariate time series. Their methodology takes advantage of unlabeled data to train an encoder and extract dense vector representations of time series. Subsequently, they evaluate the model for regression and classification tasks, demonstrating better performance than other state-of-the-art supervised methods, even with data sets with limited samples.

%interpretation
Regarding the interpretability of the model, a recent contribution that analyses the attention maps was presented by \citet{bowles20212}, which explored the use of group-equivariant self-attention for radio astronomy classification. Compared to other approaches, this model analysed the attention maps of the predictions and showed that the mechanism extracts the brightest spots and jets of the radio source more clearly. This indicates that attention maps for prediction interpretation could help experts see patterns that the human eye often misses. \par

In the field of variable stars, \citet{allam2021paying} employed the mechanism for classifying multivariate time series in variable stars. And additionally, \citet{allam2021paying} showed that the activation weights are accommodated according to the variation in brightness of the star, achieving a more interpretable model. And finally, related to the TESS telescope, \citet{morvan2022don} proposed a model that removes the noise from the light curves through the distribution of attention weights. \citet{morvan2022don} showed that the use of the attention mechanism is excellent for removing noise and outliers in time series datasets compared with other approaches. In addition, the use of attention maps allowed them to show the representations learned from the model. \par

Recent attention mechanism approaches in astronomy demonstrate comparable results with earlier approaches, such as CNNs. At the same time, they offer interpretability of their results, which allows a post-prediction analysis. \par



\section{Client Simulation Framework}
\label{sec:framework}

\begin{figure*}[tb]
    \centering
    \includegraphics[width=\textwidth]{figs/framework.pdf}
    \caption{Proposed Client Simulation Framework.}
    \label{fig:framework}
\end{figure*}


We first give an overview of our proposed client simulation framework to generate client utterances consistent to client's profile and behavior in a MI-counseling session.  The framework consists of modules for: (a) {\em state transition}, (b) {\em action selection} (c) {\em information selection} , and (d) {\em response generation} as shown in Figure~\ref{fig:framework}.\footnote{The detailed prompt design of our framework can be found in Appendix~\ref{app:implement}.} Finally, we annotate and analyze the real world MI-based counseling data for deriving the knowledge required for developing modules (a) and (b).

\subsection{Overview}

Our framework takes a {\em client's profile} as input. It consists of the client's behavioral problem, initial state, final state, persona, motivation, beliefs, preferred change plans, and receptivity.  The client's behavioral problem, persona, motivation, beliefs, and preferred change plans are expressed in text. The initial and final states refer to the client's initial and final state-of-mind towards the behavioral problem before and after the counseling session, respectively. We utilize the transtheoretical model of health behavior change ~\citep{abuse2019enhancing} to define three possible states: {\em Precontemplation}, {\em Contemplation}, and {\em Preparation}~\citep{prochaska1997transtheoretical, hashemzadeh2019transtheoretical}. Note that there also exists Action and Maintenance states in the transtheoretical model. As MI-counseling is particularly useful when clients are in the Precontemplation and Contemplation states for them to reach the Preparation state, we consider only the three primary states henceforth, which also aligns well with the real dataset. To end the counseling session, we also include the {\em Termination} state.  

Similar to the earlier work, the persona covers background information about the client. These are useful information the counselor may need to uncover during the session. Motivation refers to a specific reason that can lead the client to consider making changes. In contrast, the client's beliefs are the ones which may obstruct behavioral changes. Preferred change plans (or plan) refer to the set of specific behavior changes the client may agree to. Receptivity captures how receptive the client is towards the counselor and is categorized into five levels from lowest (1) to highest (5). 

The client simulator generates one utterance at a time in the counseling session.  Right after each counselor's utterance, the simulator first determines the client's next state using the state transition module which has the current counseling session text (also known as context) and current state as input.  Conditioned on the next state, the action selection module merges the context-aware and (state, receptivity)-aware action distributions so that an action representing a type of utterance can be sampled.  The sampled (or selected) action, together with the new state and current context, will then be used by the response generation module to generate the next utterance.  If the selected action requires some client's profile knowledge (e.g., Inform, Hesitate, and Blame actions), the simulator will perform an additional information selection step to search for relevant reference information from the profile for response generation.

\subsection{State Transition}
\label{sec:state transition}

% \begin{table}[tb]
% \resizebox{0.47\textwidth}{!}{
% \begin{tabular}{lp{0.6\columnwidth}}
% \toprule
% State            & Corresponding Actions                                    \\ \midrule
% Precontemplation & Inform, Engage, Deny, Blame, Downplay     \\ \hline
% Contemplation    & Inform, Engage, Hesitate, Doubt, Acknowledge    \\ \hline
% Preparation      & Inform, Engage, Reject, Accept, Plan\\ \hline
% Termination      & Terminate     \\ \bottomrule
% \end{tabular}}
% \caption{The states of change and corresponding actions used in our simulation framework.}
% \label{tab:state desc}
% \end{table}


During MI counseling, a client is guided through different states to identify changes that can address his/her behavioral problem. The goal of the state transition module is thus to maintain consistency with the way the client may change states during MI counseling as well as the client's profiles~\citep{abuse2019enhancing}.

In the Precontemplation state, the counselor focuses on building trust, establishing change goals, understanding the client's motivations, and address some wrong beliefs, before evoking the client to change. The client is expected to enter the Contemplation state only when motivated by a specific reason which has also been brought up by the counselor. The state transition module thus seeks to be consistent by analyzing the counselor's utterances to search for mention(s) of client's motivation. If such a mention is found, the next client's state is Contemplation; otherwise, the state remains unchanged. In the Contemplation state, the client recognizes his/her behavioral problem but is hesitate to change, often due to some beliefs which are part of client's profile. The state transition module tracks these reasons from the client's profile and assesses whether they have been adequately addressed. If so, the client will transit to the Preparation state. In the Preparation phase, the client discusses specific plans for change while counselor provides the necessary information and tools to guide this discussion. When the client's preferred change plan has been discussed, the state transition module should output Termination as the next state. The state transition in our framework align with the observations in a real dataset. Nevertheless, our proposed framework can accommodate more complex flows and additional states, provided that further comprehensive datasets are available.
% {\color{blue} The state transition in our framework align with the observation in a real dataset. Nevertheless, our proposed framework can accommodate more complex flows and additional states, provided that further comprehensive datasets are available.}
% {\color{red} (EP: Need to explain that the current state transitions are based on transitions known in AnnoMI dataset, and that other new state transitions can be accommodated.)}

\subsection{Action Selection}
\label{sec:action selection}

Inspired by \citet{dutt2021resper}, we tailor client actions to MI counseling and integrate a simple yet effective action sampling method into the action selection module. We meticulously design candidate actions for each state based on MI counseling theory and analysis of a real dataset.\footnote{The detailed candidate actions are outlined in the Appendix~\ref{app:implement}.} Similarly, our proposed framework can accommodate additional actions, provided further datasets are available. We recognize the need to consider both the latest counselling session context, client's state and receptivity in action selection. We first infer the context-aware action distribution based on the latest session context using LLM. We next derive from real world MI-counseling data the action distribution for each (state,receptivity) combination. To ensure both context coherence and profile consistency, we finally merge the context-aware action distribution and (state,receptivity)-aware action distribution by averaging the two.  When sampling from the merged action distribution, we only select among among actions relevant to the next state.

\subsection{Information Selection}
\label{sec:information selection}

Our framework divides actions into two types. The type-1 actions, e.g., Deny, Engage, and Accept, do not require profile information to generate the utterances. The type-2 actions (e.g., Inform, Blame, Hesitate, and Plan) utilize additional information from the client's profile for utterance generation. The information selection module thus selects appropriate information from client profile for type-2 actions using LLM prompting similar to that in \citet{tu2023characterchat}, thereby enhancing the consistency with the client profile. This selection mechanism also prevents the simulated clients from sharing too much profile information unnecessarily shortening the counseling sessions.

\subsection{Response Generation}
\label{sec:response generation}

Finally, the response generation module produces a client's utterance using a {\em turn-by-turn} generation approach. Specifically, using the same application session\footnote{This covers the entire history of prompt instructions, generated counselor and client utterances including the initial role-playing instruction and client profile.} with the LLM, the module submits the prompt instruction to generate the client utterance of the next turn.  The prompt instruction includes a description of the next state and selected action, along with selected information if the action is type-2. 


\subsection{Data Annotation}
\label{sec:data annotation}

\paragraph{Annotation of AnnoMI Dataset.} Our framework assumes that state transition diagram and client profile knowledge exist to develop the state transition, action selection, information selection, and response generation modules. In the absence of available expert-curated knowledge and to allow the framework to adapt to different counseling approaches, we choose to derive these knowledge from AnnoMI~\citep{wu2022anno} for the purpose of implementing and evaluating the framework. We finally select 86 clients and their sessions from the dataset\footnote{The detailed description is provided in Appendix~\ref{app:data annotation}.} then utilized GPT-4\footnote{GPT-4: \texttt{gpt-4-0125}.} to annotate utterances of these selected sessions.
Other than the purpose of framework implementation, we also aim to perform the annotation as accurate as possible so that one can automatically evaluate the consistency of profiles and receptivity of clients in generated sessions against their ground truth ones, and analyze the distribution of client states and actions.

\paragraph{Annotation of Client Profile, States, Actions and Receptivity.} We prompt GPT-4 to summarize the four profile components: persona, motivation, beliefs, and preferred change plan, for the client given a counseling session. These prompts can be found in Table~\ref{tab:profile annotation}.  We also design prompts to annotate the client's state and action (conditioned on the state) at the utterance level (see Tables~\ref{tab:state annotation} and \ref{tab:action annotation} respectively). To determine a client's receptivity in a counseling session, we provide GPT-4 the session labeled with client states and actions followed by prompting GPT-4 to assign a receptivity score (between 1 and 5 as shown in Table~\ref{tab:receptivity annotation}). We repeat five rounds of such assignments and obtain the average receptivity score. We take the floor of the average score to obtain the final receptivity integer score. 

\paragraph{Annotation Validity} To determine the validity of above GPT-4 annotations, two annotators manually reviewed the annotations together and derived the commonly agreed annotation labels.  Against these manually derived ground truth annotations, the GPT-4 annotations achieve high accuracy rates: 87.31\% for states, 85.20\% for actions, and 80.32\% for receptivity scores. The precision and recall of persona annotation are 81.93\% and 80.07\% respectively, that of belief annotation are 77.48\% and 77.38\% respectively. The precision and recall of motivation annotation are 82.72\% and 79.98\% respectively, while that of plan annotation are 78.38\% and 76.24\% respectively. Furthermore, all profile items are factually accurate. Overall, these results confirm the high reliability of GPT-4 annotation method which makes it feasible to automate the evaluation of generated sessions. 

\paragraph{Client Behavior Analysis in AnnoMI sessions.} We analyze the our annotations to reveal client behaviors in AnnoMI sessions. Most clients, ~70\% of them, demonstrate moderate receptivity (score=2 to 4), while a few exhibit high (score=5) or low receptivity (score=1). Clients with higher receptivity tend to use neutral or change talk more often. Conversely, clients with lower receptivity are harder to move to the Contemplation state, requiring more effort from the counselor. Some actions, such as Deny, Downplay, and Reject, are negatively associated with receptivity. The details are provided in Figures~\ref{fig:suggest_annotation} and \ref{fig:action_distribution}.



\section{Experiments}
\label{sec:exp}
Following the settings in Section \ref{sec:existing}, we evaluate \textit{NovelSum}'s correlation with the fine-tuned model performance across 53 IT datasets and compare it with previous diversity metrics. Additionally, we conduct a correlation analysis using Qwen-2.5-7B \cite{yang2024qwen2} as the backbone model, alongside previous LLaMA-3-8B experiments, to further demonstrate the metric's effectiveness across different scenarios. Qwen is used for both instruction tuning and deriving semantic embeddings. Due to resource constraints, we run each strategy on Qwen for two rounds, resulting in 25 datasets. 

\subsection{Main Results}

\begin{table*}[!t]
    \centering
    \resizebox{\linewidth}{!}{
    \begin{tabular}{lcccccccccc}
    \toprule
    \multirow{3}*{\textbf{Diversity Metrics}} & \multicolumn{10}{c}{\textbf{Data Selection Strategies}} \\
    \cmidrule(lr){2-11}
    & \multirow{2}*{\textbf{K-means}} & \multirow{2}*{\vtop{\hbox{\textbf{K-Center}}\vspace{1mm}\hbox{\textbf{-Greedy}}}}  & \multirow{2}*{\textbf{QDIT}} & \multirow{2}*{\vtop{\hbox{\textbf{Repr}}\vspace{1mm}\hbox{\textbf{Filter}}}} & \multicolumn{5}{c}{\textbf{Random}} & \multirow{2}{*}{\textbf{Duplicate}} \\ 
    \cmidrule(lr){6-10}
    & & & & & \textbf{$\mathcal{X}^{all}$} & ShareGPT & WizardLM & Alpaca & Dolly &  \\
    \midrule
    \rowcolor{gray!15} \multicolumn{11}{c}{\textit{LLaMA-3-8B}} \\
    Facility Loc. $_{\times10^5}$ & \cellcolor{BLUE!40} 2.99 & \cellcolor{ORANGE!10} 2.73 & \cellcolor{BLUE!40} 2.99 & \cellcolor{BLUE!20} 2.86 & \cellcolor{BLUE!40} 2.99 & \cellcolor{BLUE!0} 2.83 & \cellcolor{BLUE!30} 2.88 & \cellcolor{BLUE!0} 2.83 & \cellcolor{ORANGE!20} 2.59 & \cellcolor{ORANGE!30} 2.52 \\    
    DistSum$_{cosine}$  & \cellcolor{BLUE!30} 0.648 & \cellcolor{BLUE!60} 0.746 & \cellcolor{BLUE!0} 0.629 & \cellcolor{BLUE!50} 0.703 & \cellcolor{BLUE!10} 0.634 & \cellcolor{BLUE!40} 0.656 & \cellcolor{ORANGE!30} 0.578 & \cellcolor{ORANGE!10} 0.605 & \cellcolor{ORANGE!20} 0.603 & \cellcolor{BLUE!10} 0.634 \\
    Vendi Score $_{\times10^7}$ & \cellcolor{BLUE!30} 1.70 & \cellcolor{BLUE!60} 2.53 & \cellcolor{BLUE!10} 1.59 & \cellcolor{BLUE!50} 2.23 & \cellcolor{BLUE!20} 1.61 & \cellcolor{BLUE!30} 1.70 & \cellcolor{ORANGE!10} 1.44 & \cellcolor{ORANGE!20} 1.32 & \cellcolor{ORANGE!10} 1.44 & \cellcolor{ORANGE!30} 0.05 \\
    \textbf{NovelSum (Ours)} & \cellcolor{BLUE!60} 0.693 & \cellcolor{BLUE!50} 0.687 & \cellcolor{BLUE!30} 0.673 & \cellcolor{BLUE!20} 0.671 & \cellcolor{BLUE!40} 0.675 & \cellcolor{BLUE!10} 0.628 & \cellcolor{BLUE!0} 0.591 & \cellcolor{ORANGE!10} 0.572 & \cellcolor{ORANGE!20} 0.50 & \cellcolor{ORANGE!30} 0.461 \\
    \midrule    
    \textbf{Model Performance} & \cellcolor{BLUE!60}1.32 & \cellcolor{BLUE!50}1.31 & \cellcolor{BLUE!40}1.25 & \cellcolor{BLUE!30}1.05 & \cellcolor{BLUE!20}1.20 & \cellcolor{BLUE!10}0.83 & \cellcolor{BLUE!0}0.72 & \cellcolor{ORANGE!10}0.07 & \cellcolor{ORANGE!20}-0.14 & \cellcolor{ORANGE!30}-1.35 \\
    \midrule
    \midrule
    \rowcolor{gray!15} \multicolumn{11}{c}{\textit{Qwen-2.5-7B}} \\
    Facility Loc. $_{\times10^5}$ & \cellcolor{BLUE!40} 3.54 & \cellcolor{ORANGE!30} 3.42 & \cellcolor{BLUE!40} 3.54 & \cellcolor{ORANGE!20} 3.46 & \cellcolor{BLUE!40} 3.54 & \cellcolor{BLUE!30} 3.51 & \cellcolor{BLUE!10} 3.50 & \cellcolor{BLUE!10} 3.50 & \cellcolor{ORANGE!20} 3.46 & \cellcolor{BLUE!0} 3.48 \\ 
    DistSum$_{cosine}$ & \cellcolor{BLUE!30} 0.260 & \cellcolor{BLUE!60} 0.440 & \cellcolor{BLUE!0} 0.223 & \cellcolor{BLUE!50} 0.421 & \cellcolor{BLUE!10} 0.230 & \cellcolor{BLUE!40} 0.285 & \cellcolor{ORANGE!20} 0.211 & \cellcolor{ORANGE!30} 0.189 & \cellcolor{ORANGE!10} 0.221 & \cellcolor{BLUE!20} 0.243 \\
    Vendi Score $_{\times10^6}$ & \cellcolor{ORANGE!10} 1.60 & \cellcolor{BLUE!40} 3.09 & \cellcolor{BLUE!10} 2.60 & \cellcolor{BLUE!60} 7.15 & \cellcolor{ORANGE!20} 1.41 & \cellcolor{BLUE!50} 3.36 & \cellcolor{BLUE!20} 2.65 & \cellcolor{BLUE!0} 1.89 & \cellcolor{BLUE!30} 3.04 & \cellcolor{ORANGE!30} 0.20 \\
    \textbf{NovelSum (Ours)}  & \cellcolor{BLUE!40} 0.440 & \cellcolor{BLUE!60} 0.505 & \cellcolor{BLUE!20} 0.403 & \cellcolor{BLUE!50} 0.495 & \cellcolor{BLUE!30} 0.408 & \cellcolor{BLUE!10} 0.392 & \cellcolor{BLUE!0} 0.349 & \cellcolor{ORANGE!10} 0.336 & \cellcolor{ORANGE!20} 0.320 & \cellcolor{ORANGE!30} 0.309 \\
    \midrule
    \textbf{Model Performance} & \cellcolor{BLUE!30} 1.06 & \cellcolor{BLUE!60} 1.45 & \cellcolor{BLUE!40} 1.23 & \cellcolor{BLUE!50} 1.35 & \cellcolor{BLUE!20} 0.87 & \cellcolor{BLUE!10} 0.07 & \cellcolor{BLUE!0} -0.08 & \cellcolor{ORANGE!10} -0.38 & \cellcolor{ORANGE!30} -0.49 & \cellcolor{ORANGE!20} -0.43 \\
    \bottomrule
    \end{tabular}
    }
    \caption{Measuring the diversity of datasets selected by different strategies using \textit{NovelSum} and baseline metrics. Fine-tuned model performances (Eq. \ref{eq:perf}), based on MT-bench and AlpacaEval, are also included for cross reference. Darker \colorbox{BLUE!60}{blue} shades indicate higher values for each metric, while darker \colorbox{ORANGE!30}{orange} shades indicate lower values. While data selection strategies vary in performance on LLaMA-3-8B and Qwen-2.5-7B, \textit{NovelSum} consistently shows a stronger correlation with model performance than other metrics. More results are provided in Appendix \ref{app:results}.}
    \label{tbl:main}
    \vspace{-4mm}
\end{table*}


\begin{table}[t!]
\centering
\resizebox{\linewidth}{!}{
\begin{tabular}{lcccc}
\toprule
\multirow{2}*{\textbf{Diversity Metrics}} & \multicolumn{3}{c}{\textbf{LLaMA}} & \textbf{Qwen}\\
\cmidrule(lr){2-4} \cmidrule(lr){5-5} 
& \textbf{Pearson} & \textbf{Spearman} & \textbf{Avg.} & \textbf{Avg.} \\
\midrule
TTR & -0.38 & -0.16 & -0.27 & -0.30 \\
vocd-D & -0.43 & -0.17 & -0.30 & -0.31 \\
\midrule
Facility Loc. & 0.86 & 0.69 & 0.77 & 0.08 \\
Entropy & 0.93 & 0.80 & 0.86 & 0.63 \\
\midrule
LDD & 0.61 & 0.75 & 0.68 & 0.60 \\
KNN Distance & 0.59 & 0.80 & 0.70 & 0.67 \\
DistSum$_{cosine}$ & 0.85 & 0.67 & 0.76 & 0.51 \\
Vendi Score & 0.70 & 0.85 & 0.78 & 0.60 \\
DistSum$_{L2}$ & 0.86 & 0.76 & 0.81 & 0.51 \\
Cluster Inertia & 0.81 & 0.85 & 0.83 & 0.76 \\
Radius & 0.87 & 0.81 & 0.84 & 0.48 \\
\midrule
NovelSum & \textbf{0.98} & \textbf{0.95} & \textbf{0.97} & \textbf{0.90} \\
\bottomrule
\end{tabular}
}
\caption{Correlations between different metrics and model performance on LLaMA-3-8B and Qwen-2.5-7B.  “Avg.” denotes the average correlation (Eq. \ref{eq:cor}).}
\label{tbl:correlations}
\vspace{-2mm}
\end{table}

\paragraph{\textit{NovelSum} consistently achieves state-of-the-art correlation with model performance across various data selection strategies, backbone LLMs, and correlation measures.}
Table \ref{tbl:main} presents diversity measurement results on datasets constructed by mainstream data selection methods (based on $\mathcal{X}^{all}$), random selection from various sources, and duplicated samples (with only $m=100$ unique samples). 
Results from multiple runs are averaged for each strategy.
Although these strategies yield varying performance rankings across base models, \textit{NovelSum} consistently tracks changes in IT performance by accurately measuring dataset diversity. For instance, K-means achieves the best performance on LLaMA with the highest NovelSum score, while K-Center-Greedy excels on Qwen, also correlating with the highest NovelSum. Table \ref{tbl:correlations} shows the correlation coefficients between various metrics and model performance for both LLaMA and Qwen experiments, where \textit{NovelSum} achieves state-of-the-art correlation across different models and measures.

\paragraph{\textit{NovelSum} can provide valuable guidance for data engineering practices.}
As a reliable indicator of data diversity, \textit{NovelSum} can assess diversity at both the dataset and sample levels, directly guiding data selection and construction decisions. For example, Table \ref{tbl:main} shows that the combined data source $\mathcal{X}^{all}$ is a better choice for sampling diverse IT data than other sources. Moreover, \textit{NovelSum} can offer insights through comparative analyses, such as: (1) ShareGPT, which collects data from real internet users, exhibits greater diversity than Dolly, which relies on company employees, suggesting that IT samples from diverse sources enhance dataset diversity \cite{wang2024diversity-logD}; (2) In LLaMA experiments, random selection can outperform some mainstream strategies, aligning with prior work \cite{xia2024rethinking,diddee2024chasing}, highlighting gaps in current data selection methods for optimizing diversity.



\subsection{Ablation Study}


\textit{NovelSum} involves several flexible hyperparameters and variations. In our main experiments, \textit{NovelSum} uses cosine distance to compute $d(x_i, x_j)$ in Eq. \ref{eq:dad}. We set $\alpha = 1$, $\beta = 0.5$, and $K = 10$ nearest neighbors in Eq. \ref{eq:pws} and \ref{eq:dad}. Here, we conduct an ablation study to investigate the impact of these settings based on LLaMA-3-8B.

\begin{table}[ht!]
\centering
\resizebox{\linewidth}{!}{
\begin{tabular}{lccc}
\toprule
\textbf{Variants} & \textbf{Pearson} & \textbf{Spearman} & \textbf{Avg.} \\
\midrule
NovelSum & 0.98 & 0.96 & 0.97 \\
\midrule
\hspace{0.10cm} - Use $L2$ distance & 0.97 & 0.83 & 0.90\textsubscript{↓ 0.08} \\
\hspace{0.10cm} - $K=20$ & 0.98 & 0.96 & 0.97\textsubscript{↓ 0.00} \\
\hspace{0.10cm} - $\alpha=0$ (w/o proximity) & 0.79 & 0.31 & 0.55\textsubscript{↓ 0.42} \\
\hspace{0.10cm} - $\alpha=2$ & 0.73 & 0.88 & 0.81\textsubscript{↓ 0.16} \\
\hspace{0.10cm} - $\beta=0$ (w/o density) & 0.92 & 0.89 & 0.91\textsubscript{↓ 0.07} \\
\hspace{0.10cm} - $\beta=1$ & 0.90 & 0.62 & 0.76\textsubscript{↓ 0.21} \\
\bottomrule
\end{tabular}
}
\caption{Ablation Study for \textit{NovelSum}.}
\label{tbl:ablation}
\vspace{-2mm}
\end{table}

In Table \ref{tbl:ablation}, $\alpha=0$ removes the proximity weights, and $\beta=0$ eliminates the density multiplier. We observe that both $\alpha=0$ and $\beta=0$ significantly weaken the correlation, validating the benefits of the proximity-weighted sum and density-aware distance. Additionally, improper values for $\alpha$ and $\beta$ greatly reduce the metric's reliability, highlighting that \textit{NovelSum} strikes a delicate balance between distances and distribution. Replacing cosine distance with Euclidean distance and using more neighbors for density approximation have minimal impact, particularly on Pearson's correlation, demonstrating \textit{NovelSum}'s robustness to different distance measures.







\section{Conclusion}
In this paper, we focus on client simulation for Motivational Interviewing (MI) counseling, a complex scenario where the client should exhibit varied behaviors in different counseling stages. We introduce a consistent client simulation framework that integrates state tracking and action selection with an LLM to accurately mimic the behaviors of real clients according to given profiles. Based on our framework, we develop a client simulation method that supports changes of client states consistent with real world MI-counseling using state transition control, and utterances consistent with real world client's profile using action and information selection modules. These modules utilizes action distribution from real datasets for different levels of receptivity. The action selection mechanism incorporates both context-aware and (state,receptivity)-aware action distributions, to align with the expected client behavior. Extensive evaluations using GPT-4 and human assessments demonstrate the high consistency of our method with real clients, outperforming the existing client simulation methods. With our client simulation framework, we believe that more realistic simulated clients can be implemented for training and evaluating counselors.

\section*{Limitations}
Although our simulation framework shows potential for future use in counseling, there are three main limitations to this work. 

Similar to other LLM prompting approaches, our framework also suffers from prompt sensitivity, as it incorporates multiple steps of prompting, which may lead to error accumulation. To mitigate such sensitivity, we employ a few-shot method to formulate our prompts. The design of effective prompts and their evaluation are therefore important areas for further research.

This paper focuses on client simulation only.  As a good counseling session also depends on the performance of counselor, we also need to develop highly competent counselor agents and evaluate them. Nevertheless, without a diverse set of clients, it is very difficult to meet this objective. In our future work, we will therefore leverage our client simulation framework to design better counselor agents and to develop a robust approach to evaluate their counseling skills.

Our current study focuses on real client simulation and conduct experiments on a single dataset (i.e. AnnoMI). We also restrict our study to MI approach to counseling.  Our work however can be easily extended to involve more real world counseling datasets as long as privacy and ethical concerns are addressed. We can also extend our framework to accommodate other counseling approaches (e.g., Cognitive behavioral therapy) as it is quite common for experienced counselors to apply a mixture of different approaches in a single counseling session so that to both personalize to the real clients and achieve the best outcome. 

\section*{Ethics Statement}
For the revision of the framework and any related annotation work, we conducted a thorough survey of psychological knowledge and datasets pertaining to motivational interviewing counseling. All annotations were performed by our co-authors.

%\section{Acknowledgment}
%This research / project is supported by the Singapore’s Ministry of Health’s National Medical Research Council under its Population Health Research Grant Thematic Category (PHRGTC-5-005). Any opinions, findings and conclusions or recommendations expressed in this material are those of the author(s) and do not reflect the views of MOH/NMRC

% Bibliography entries for the entire Anthology, followed by custom entries
%\bibliography{anthology,custom}
% Custom bibliography entries only
\bibliography{custom}

\appendix

\section{Data Annotation and Analysis}
\label{app:data annotation}

The analysis of both observed (i.e., AnnoMI) and generated MI counseling sessions requires them to be annotated at the session level so as to obtain the client's profile components (i.e., personas, beliefs, motivation, and acceptable change plan) and receptivity.  Tables~\ref{tab:profile annotation} and \ref{tab:receptivity annotation} show corresponding prompts.  We also develop prompts to annotate at a specific point of a session to obtain the client's state and action at that point (see Table~\ref{tab:state annotation} and \ref{tab:action annotation}).  Finally, we create the prompts to determine entailment between generated and ground truth profile components (see Table~\ref{tab:entailment annotation}).  
GPT-4, one of the top proprietary LLMs, is chosen to be the target LLM to perform annotation of the AnnoMI and generated sessions.

In the following, we report the analysis results after annotating the 86 selected AnnoMI sessions.

\paragraph{AnnoMI Dataset} We utilize the AnnoMI dataset~\citep{wu2022anno}, which comprises 133 conversations spanning a diverse range of behavior topics, including ``reducing alcohol consumption'' and ``smoking cessation.'' Each conversation was transcribed from an expert-generated demonstration video and subsequently labeled as demonstrating high- or low-quality MI based on the video title and description. The 110 dialogues illustrating high-quality MI, comprising over 8,800 utterances in total, are considered as candidates. Consequently, we eliminate the sessions that are incomplete and those involving clients with insufficient profile information. To be noticed, the AnnoMI dataset is compiled from YouTube videos depicting MI counseling sessions between actors. However, it remains widely utilized due to restricted access to real counseling sessions. Given the absence of ideal datasets, our framework offers a solution to the scarcity of data by simulating clients interacting with counselors, thus bypassing ethical and privacy issues. Additionally, the primary use of the AnnoMI dataset is to extract client profiles and corresponding counseling sessions for evaluation purposes. Although these sessions are not “real”, they are still consistent with the client profiles and serve as valuable evaluation dataset.

\begin{figure*}[t]
    \centering
    \includegraphics[width=\textwidth]{figs/receptivity.pdf}
    \caption{The distribution of receptivity and the relation between receptivity and sustain ratio and precontemplation.}
    \label{fig:suggest_annotation}
\end{figure*}

\begin{figure*}[tb]
    \centering
    \includegraphics[width=\textwidth]{figs/action.pdf}
    \caption{Proportions of actions for different receptivity scores in different states. Actions negatively associated with receptivity are represented in dashed lines.}
    \label{fig:action_distribution}
\end{figure*}

\paragraph{Analysis of receptivity, associated talk types, and client states.} As shown in Figure~\ref{fig:suggest_annotation}(a), most clients demonstrate moderate receptivity. Few clients have high and low receptivities.  We next analyse the client utterances labeled with talk-types ``change talk'', ``neutral'' and ``sustain talk''. utterances annotated with ``Change talk'' express an interest to change behavior, while those annotated with ``sustain talk'' are the opposite~\citep{hoang2024can, miller2012motivational}. Figure~\ref{fig:suggest_annotation}(b) shows that the ratio of sustain talk-labeled utterances has a negative relationship with receptivity, i.e., clients with higher receptivity tend to use neutral or change talk more often.  Figure~\ref{fig:suggest_annotation}(c) shows that the proportion of client utterances in Precontemplation state also has a negative relationship with receptivity as clients in this state are reluctant and/or the counselor is not able to effect behavior change. Clients with lower receptivity are harder to reach the contemplation state, thus requiring more effort from the counselor.

\paragraph{Analysis of receptivity and associated actions.} Figure~\ref{fig:action_distribution}(a) illustrates the proportion of actions in different client states across different receptivity scores. Considering the utterances of clients who are in the precontemplation state, the proportion of utterances annotated with Inform action increases with receptivity due to the more receptive clients providing information about themselves.  The proportion with Engage action also increases slightly.  On the other hand, the proportions of Deny, Downplay and Blame actions reduces as receptivity increases. Similar findings also apply to Figures~\ref{fig:action_distribution}(b) and (c).  For easy reading, the actions negatively associated with receptivity are shown in dashed lines.  %Throughout the three states, actions associated with sustain talk, such as Deny, Downplay, and Hesitate, consistently decrease as receptivity increases. Conversely, actions associated with change talk, such as Acknowledge, Accept, and Plan, increase with higher receptivity. For neutral talk, Inform consistently rises with increasing receptivity, although Engage shows unstable trends. 
Finally, Inform and Engage are two actions found in all the three states. Inform is most predominant in the Precontemplation state implying that clients are more likely to share profile information during Precontemplation and adopt more varied actions after they transit to Contemplation and Preparation states. 

\begin{table*}[tb]
\begin{tabularx}{\textwidth}{X}
\toprule
\#\# Task \newline Your task is to identify the client's profile based on the provided counseling conversation. Focus on the following aspects: \newline  - **Persona**: Include personal details such as recent events, family relationships, and occupation. Summarize each aspect in concise sentences. \newline - **Behavioral Problem**: Identify a key problematic behavior (e.g., drinking, smoking). Describe it succinctly in a single phrase, focusing on the primary issue. \newline - **Motivation (Optional): Explain why the client wants to change this behavior (due to family, health, work, etc.) in one sentence, focusing on the main reason. Leave this section blank if the client shows no motivation to change.  \newline - **Beliefs**: Detail the client's inner beliefs toward change, such as downplaying the behavior's frequency, blaming others, or doubting the benefits of change. Provide a sentence for each reason. \newline  - **Acceptable Plan** (Option): Describe any plans the client is willing to adopt or consider, like reducing frequency, altering the environment, or seeking help. Use one sentence for each plan. Leave this blank if the client accepts no plans. \newline  \#\# Output Format  \newline Return the client profile in JSON format as illustrated below:  \newline [example]  \newline \#\# Given Conversation [conversation]  \newline \#\# Instruction  \newline Based on the conversation, provide the client's profile in JSON format.
\\ \bottomrule
\end{tabularx}
\caption{Prompt for GPT-4 to annotate the profile of client in given conversation or session ([conversation] is to be replaced by the session content).}
\label{tab:profile annotation}
\end{table*}


\begin{table*}[tb]
\begin{tabularx}{\textwidth}{X}
\toprule
\#\# Task
Your task is to read through the provided counseling conversation and determine which stage of change the client is in based on the Transtheoretical Model of behavior change. The stages you will focus on are Precontemplation, Contemplation, and Preparation. \newline \#\# Definitions of Stages \newline Precontemplation: The client is not yet acknowledging that there is a problem behavior that needs to be changed. They may be defensive, dismissive, or indifferent about the topic of change. \newline Contemplation: The client acknowledges that there is a problem and begins to think about the possibility of change. However, they are not yet committed to taking action but are more open to discussing the pros and cons of their behavior. \newline Preparation: The client is planning to change and is likely to start taking steps soon. They might begin to set goals, seek information, or plan out the changes they intend to make.  \newline \#\# Instructions \newline Read the Conversation: Carefully read the entire conversation to understand the context and content of the discussion.
\newline Identify Statements and Attitudes: Pay close attention to the client’s statements and attitudes towards change. Look for keywords or phrases that indicate their stage of change according to the definitions provided. \newline \#\# Determine the Stage \newline - If the client shows no recognition of the problem or need for change, categorize them as Precontemplation.  \newline - If the client acknowledges the problem and discusses thoughts about possibly changing, without commitment, categorize them as Contemplation. \newline - If the client talks about specific plans or the intention to change soon, categorize them as Preparation. \newline \#\# Justify Your Choice  \newline Provide a brief justification for the stage you have assigned based on specific parts of the [conversation]. Include direct quotes or clear references to the conversation to support your decision.  \newline \#\# Given Conversation: [conversation]  \newline Annotate the state of each utterance of client.
\\ \bottomrule
\end{tabularx}
\caption{Prompt for GPT-4 to annotate the latest stage (or state) of client in given conversation ([conversation] is to be replaced by the session which as be complete or partially complete).}
\label{tab:state annotation}
\end{table*}

\begin{table*}[tb]
\begin{tabularx}{\textwidth}{X}
\toprule
\#\# Task \newline Your task is to annotate the action of the client for each provided conversation snippet. For each snippet, focus on the client's last utterance and choose the most appropriate action from the given set of action. \newline  \#\# Action Options \newline [options] \newline \#\# Instructions \newline - Read the Conversation Snippet: Carefully read through the provided snippet. \newline - Focus on the Last Utterance: Concentrate on the client's last utterance. \newline - Choose the Action: Select the action that best describes the client's last utterance from the given set of actions. \newline  \#\# Response Format \newline Analysis of Client's Action: Provide a brief analysis of the client's last utterance and the chosen action. \newline Chosen Action: [Selected action] \newline For example: \newline Analysis of Client's Action: The client shows uncertainty about the effectiveness of potential stress management strategies, indicating hesitation.  \newline Chosen Action: Hesitate  \newline Provided Conversation Snippet for Annotation [conversation]  \newline What is the most appropriate action that describes the client's last utterance ([last utterance]) in the conversation snippet? Provide a brief analysis to support your choice.
\\ \bottomrule
\end{tabularx}
\caption{Prompt for GPT-4 to assign  from a list of possible actions [options] the action of the last client's utterance ([last utterance]) in the given conversation snippet ([conversation]).}
\label{tab:action annotation}
\end{table*}


\begin{table*}[tb]
\begin{tabularx}{\textwidth}{X}
\toprule
\#\# Task \newline Your task is to assess the client's receptivity in a given counseling conversation by focusing on the client's behaviors, openness, and responsiveness to suggestions. Determine how easily the counselor can motivate the client to change. \newline \#\# Guidelines \newline The client with higher receptivity tends to openly share information, accept suggestions, and show high confidence in their ability to change. The client with lower receptivity tends to resist suggestions, deny the need for change, downplay the importance of change, and exhibit other passive attitudes. \newline \#\# Scoring Receptivity \newline - **High Receptivity (5)**: A client with high receptivity consistently demonstrates behaviors such as frequently and openly sharing personal thoughts and feelings, actively engaging in discussions about change, quickly and positively responding to suggestions, and displaying strong self-belief in their ability to make changes. \newline - **Moderately High Receptivity (4)**: Clients with moderately high receptivity are generally open and forthcoming, mostly receptive to change discussions, respond well to suggestions with occasional need for reinforcement, and demonstrate good self-belief with some need for reassurance. \newline - **Moderate Receptivity (3)**: Clients with moderate receptivity show a balance between sharing and withholding information, exhibit mixed interest in change discussions, respond to suggestions neutrally, and display a balanced view of their ability to change. \newline - **Moderately Low Receptivity (2)**: Clients with moderately low receptivity are characterized by reluctance to share personal information, limited interest in change discussions, resistance to suggestions, and low self-belief with frequent expression of doubts. \newline - **Low Receptivity (1)**: Clients with low receptivity consistently withhold information, avoid discussions about change, strongly resist suggestions, and show very low self-belief, often highlighting reasons why change is not possible. \newline \#\# Important Note \newline Remember that the client may be motivated to change by the end of the conversation due to the counselor's efforts. However, your assessment should be based on the client's behaviors throughout the conversation especially before issue acknowledgement. \newline \#\# Response Format \newline Analysis of Client's Behavior: Provide a detailed analysis of the client's behaviors during the conversation. \newline Receptivity Score: Assign a final score based on the observed behaviors using the provided scoring system. \newline For Example:  \newline Analysis of Client's Behavior: The client initially exhibited resistance to change by denying the need for it. As the conversation progressed, the client showed some openness by considering small changes but remained hesitant and doubtful. Overall, the client demonstrated a low receptivity behaviors. \newline Receptivity Score: 1 (Low receptivity) \newline \#\# Provided Conversation for Evaluation \newline [conversation]  \newline What is the client's receptivity score based on the observed behaviors in the conversation? Provide a detailed analysis of the client's behaviors to support your assessment.
\\ \bottomrule
\end{tabularx}
\caption{Prompt for GPT-4 to annotate the receptivity score of client (1 to 5) in given conversation snippet ([conversation] will be replaced by the counseling session).}
\label{tab:receptivity annotation}
\end{table*}


\begin{table*}[tb]
\begin{tabularx}{\textwidth}{X}
\toprule
\#\# Task \newline You task is to identify whether the premise entails the hypothesis. The answer should be exact 'entail' or 'not entail'. \newline \#\# Premise [profile] \newline \#\# Hypothesis [component]
\\ \bottomrule
\end{tabularx}
\caption{Prompt for the GPT-4 to evaluate the entailment between each generated profile component ([component]) and given profile ([profile]).}
\label{tab:entailment annotation}
\end{table*}


\section{Expert Annotation}
\label{app:human evaluation}


\subsection{Demographics for Annotators}

Our annotation team comprises four experts who are co-authors of this paper. Three of them hold postgraduate degrees in counseling and one is a senior-year university student pursuing a psychology major. They all have relevant experience in mental health counseling.

\subsection{Annotation Instruction}

Table~\ref{tab:human_instruction1} presents the profile consistency rating score descriptions and client profile components for an expert to evaluate the profile consistency scores of a generated counseling session in all the four profile aspects (i.e., persona, beliefs, motivation, and acceptable plan). Table~\ref{tab:human_instruction2} shows the realism rating score descriptions for an expert to assess the realism of client in a generated session. Finally, Table~\ref{tab:human_instruction3} shows the score descriptions for an expert to determine the receptivity level of client in a generated counseling session.

\begin{table*}[tb]
\begin{tabularx}{\textwidth}{X}
\toprule
\textbf{Consistency Rating (To rate how consistent is the client with the four aspects of profile)}\newline
5 (High): Perfectly aligns with profile across all aspects \newline
4 (Moderately High): Mostly aligns with profile with minor variations \newline
3 (Moderate): Partially aligns with profile with some discrepancies \newline
2 (Moderately Low): Limited alignment with profile with multiple discrepancies \newline
1 (Low): Consistently contradicts profile information \newline
\textbf{CLIENT PROFILE} \newline
\textbf{Behavior:} drinking alcohol \newline
\textbf{Personas:}\newline
- The client sneaked out with a friend to the park to has a couple of beers.\newline
- The client managed to avoid getting into trouble as he threw the beers away before the policy saw them.\newline
- If the the client's mother found out he was in trouble because of his drinking, she would kill him.\newline
- The client has an interest in soccer and has set a goal to play in college.\newline
- The client has aspirations of receiving scholarships to support his soccer career.\newline
\textbf{Beliefs:} \newline
- The client believes he can handle the dangerous situation to avoid getting into trouble. \newline
- The client has only drunk once or twice. \newline
- A couple of the client's friends drink too. \newline
\textbf{Motivation:} The client is motivated because of the risk of drinking alcohol in sports, as alcohol would affect the ability to play soccer. \newline
\textbf{Acceptable Plans:} None \newline
\textbf{SESSION} \newline
[conversation]
\\ \bottomrule
\end{tabularx}
\caption{Description of profile consistency rating scores, ground truth client profile, and counseling session for expert annotation.  The expert is required to give a rating for each profile aspect. ([conversation] will be replaced by the real or generated session.)}
\label{tab:human_instruction1}
\end{table*}


\begin{table*}[tb]
\begin{tabularx}{\textwidth}{X}
\toprule
\textbf{Realism Rating (To rate how realistic is the client compared to a human client)} \newline
5 (High): Shows authentic responses with natural flow and rich personal context\newline
4 (Moderately High): Shows mostly genuine responses with occasional scripted moments\newline
3 (Moderate): Shows mixed authentic and artificial responses\newline
2 (Moderately Low): Shows mostly artificial responses with few authentic moments\newline
1 (Low): Shows completely artificial responses with no authentic elements\newline
\textbf{SESSION}\newline
[conversation]
\\ \bottomrule
\end{tabularx}
\caption{Description of realism rating scores, and counseling session for expert annotation. ([conversation] will be replaced by the real or generated session.)}
\label{tab:human_instruction2}
\end{table*}

\begin{table*}[tb]
\begin{tabularx}{\textwidth}{X}
\toprule
\textbf{Receptivity Rating (To rate how receptive is the client in the session)} \newline
5 (High): Actively engages, shares openly, and responds positively \newline
3 (Moderate): Shows balanced sharing and mixed response patterns \newline
1 (Low): Consistently withholds and strongly resists engagement \newline
\textbf{SESSION} \newline
[conversation]
\\ \bottomrule
\end{tabularx}
\caption{Description of receptivity ratings, and counseling session for expert annotation. ([conversation] will be replaced by the real or generated session.)}
\label{tab:human_instruction3}
\end{table*}



\section{More Evaluation Results}
\label{app:more evaluation}

\subsection{Automated Turn-Level Evaluation}

\begin{table*}[tb]
\centering
\begin{tabular}{lrrrrrrrr}
\toprule
             & ROUGE-1 & ROUGE-2 & ROUGE-L & BERTScore \\ \midrule
Base         & 13.27   & 2.55    & 10.03   & 85.59     \\
\rowcolor[HTML]{D8D6D6} Example-based & 59.12   & 48.84   & 58.31   & 93.54     \\
Profile-based & 13.79   & 3.18    & 9.90    & 85.50     \\
Pro+Act-based & 17.04   & 4.22    & 13.93   & 86.72     \\ \hline
Ours         & 17.73   & 9.13    & 15.13   & 87.13     \\
Ours*        & \textbf{29.69}   & \textbf{20.19}   & \textbf{28.75}   & \textbf{90.46}  \\ \bottomrule
\end{tabular}
\caption{Results of turn-level evaluation. ``Our'' refers to our client simulation method. ``Ours*'' refers to a variant of our method provided with ground truth states, actions, and selected information. Other than Example-based method, our framework outperforms the remaining baseline methods.}
\label{tab:turn-eval}
\end{table*}

As previous works~\citep{wang2024towards,deng2023prompting,kang2024can} often evaluated generated utterances against ground truth utterances in conversational tasks, we also conduct turn-level evaluation which involves using a simulated client to generate the next client utterance given a prompt that includes all the session content up to the previous utterance. To show the upper bound performance of our simulated client method (i.e., ``Ours''), we provide the results of a variant of our method (i.e., ``Ours*'') that generates the next utterance using the ground truth state, action and selected information. As shown in Table~\ref{tab:turn-eval}, our method outperforms most of the baselines. The Example-based method has been shaded because it has an unfair advantage over others due to its ability to search all the client utterances including the ground truth utterance. Our method variant, Ours*, shows that there are still rooms for Our method to improve in ROUGE-1, ROUGE-2 and ROUGE-L. Our and Our* methods share very similar BERTScore performance. This indicates that the Our method can reproduce responses semantically similar to the ground truth utterances. 

Nevertheless, it is important to note that turn-level evaluation only measures quality of the next utterance assuming that all earlier utterances are known.  This assumption does not hold in a real world counseling session or application where the client and counselor do not follow some written counseling scripts. Hence, a method performing well in this evaluation method may not always do so at the session level.  For example, the Example-based method appears to perform well in the turn-level evaluation purely due to its ability to find the most similar client utterance, indicating that turn-level evaluation may not accurately measure a client's effectiveness in real world scenarios. Hence, consistency evaluation of client profile components in Section~\ref{sec:profile consistency}, and client receptivity are introduced in Section~\ref{sec:exp_eval}. We believe these results can provide a more comprehensive assessment of client performance.

\subsection{Evaluation on Multiple Backbones}

To demonstrate the generalizability of our framework, we employ various LLMs as the backbone to implement our framework. As presented in Table~\ref{tab:results for multiple models}, our method consistently outperforms the baselines across various backbones. The performance of LLMs in simulation varies due to their distinct instruction-following and conversation capabilities. In summary, the \texttt{gpt-4o}, \texttt{llama-70B}, and \texttt{mistral-large} perform comparably to the \texttt{gpt-3.5}, while the smaller LLMs (\texttt{ministral-8B} and \texttt{llama-8B}) exhibit significantly inferior performance.

\begin{table*}[tb]
\centering
\resizebox{\textwidth}{!}{
\begin{tabular}{lrrrrrrrrr}
\toprule
              & Personas$\uparrow$    & Motivation$\uparrow$         & Beliefs$\uparrow$      & Plans$\uparrow$       &$\rho$ $\uparrow$  & Avg. Rec.            & MR@20             & Avg.MS.      & Act. KL $\downarrow$ \\ \midrule
Real          & - & -    & -  & -  & - & 3.27$_{\pm1.12}$    & 0.48              & 27.56         & 0.00              \\ 
\toprule
\multicolumn{10}{c}{\cellcolor[HTML]{EFEFEF}\texttt{gpt-4o-2024-11-20}}                              \\ \midrule
Base          & 9.17        & 16.19             & 12.26         & 9.38       & -0.28   & 4.30$_{\pm0.31}$    & 1.00              & 7.30          & 0.40 \\
Example-based & 56.29       & 49.96             & 50.66         & 37.21       & 0.27   & 3.98$_{\pm0.68}$    & 1.00              & 9.12          & 0.23\\
Profile-based & 69.78       & 56.97             & 70.59         & 57.31       & 0.32   & 4.05$_{\pm0.72}$    & 0.90              & 10.26          & 0.15 \\
Pro+Act-based & 72.12       & 63.43             & 71.57         & 63.89       & 0.35   & 3.63$_{\pm0.97}$    & 0.85              & 11.80          & 0.14\\ \hline
Ours          & \textbf{76.10}   & \textbf{78.57}   & \textbf{76.13}  & \textbf{74.99} & \textbf{0.66}  & 3.24$_{\pm1.21}$    & 0.61              & 19.78         & \textbf{0.05} \\ \bottomrule
\toprule
\multicolumn{10}{c}{\cellcolor[HTML]{EFEFEF}\texttt{Mistral-Large-Instruct-2411}}                              \\ \midrule
Base          &  9.18       & 16.22             & 12.21         & 9.35        & -0.30  & 4.38$_{\pm0.38}$    & 1.00              & 7.22          & 0.38 \\
Example-based & 54.22       & 47.18             & 47.01         & 34.87       & 0.26    & 4.04$_{\pm0.69}$    & 1.00              & 8.06          & 0.23\\
Profile-based & 66.31       & 57.10             & 69.20         & 56.85       & 0.33   & 4.22$_{\pm0.63}$    & 0.93              & 10.05          & 0.15 \\
Pro+Act-based & 70.11       & 60.88             & 71.09         & 60.03       & 0.36   & 3.77$_{\pm0.91}$    & 0.88              & 10.97          & 0.12\\ \hline
Ours          & \textbf{73.21}   & \textbf{76.30}   & \textbf{73.13}  & \textbf{71.67} & \textbf{0.62}  & 3.17$_{\pm1.27}$    & 0.65              & 19.56         & \textbf{0.05} \\ \bottomrule
\toprule
\multicolumn{10}{c}{\cellcolor[HTML]{EFEFEF}\texttt{LLaMA-3.3-70B-Instruct}}                              \\ \midrule
Base          &  9.15       & 16.20             & 12.17         &  9.35       & -0.31  & 4.33$_{\pm0.43}$    & 1.00              & 6.97          & 0.40 \\
Example-based & 54.73       & 47.56             & 46.26         & 34.78       & 0.27    & 4.07$_{\pm0.62}$    & 1.00              & 8.10          & 0.23\\
Profile-based & 67.54       & 55.04             & 68.51         & 57.95       & 0.33   & 4.11$_{\pm0.67}$    & 0.91              & 10.76          & 0.14 \\
Pro+Act-based & 70.08       & 59.43             & 69.98         & 59.81       & 0.35   & 3.82$_{\pm0.98}$    & 0.89              & 10.83          & 0.13\\ \hline
Ours          & \textbf{73.92}   & \textbf{75.90}   & \textbf{73.10}  & \textbf{71.81} & \textbf{0.60}  & 3.20$_{\pm1.08}$    & 0.63              & 19.02         & \textbf{0.06} \\ \bottomrule
\toprule
\multicolumn{10}{c}{\cellcolor[HTML]{EFEFEF}\texttt{gpt-3.5-turbo-0125}}                              \\ \midrule
Base          &  9.01       & 16.17             & 12.15         &  9.30       & -0.31  & 4.42$_{\pm0.47}$    & 1.00              & 6.60          & 0.39 \\
Example-based & 53.68       & 45.73             & 45.55         & 33.53       & 0.25    & 4.08$_{\pm0.63}$    & 1.00              & 7.60          & 0.24\\
Profile-based & 61.97       & 53.44             & 67.17         & 54.67       & 0.31   & 4.12$_{\pm0.64}$    & 0.96              & 9.76          & 0.15 \\
Pro+Act-based & 67.09       & 55.33             & 68.60         & 57.17       & 0.33   & 3.86$_{\pm1.01}$    & 0.94              & 9.93          & 0.13\\ \hline
Ours          & \textbf{70.57}   & \textbf{73.37}   & \textbf{71.70}  & \textbf{68.51} & \textbf{0.58}  & 3.32$_{\pm1.15}$    & 0.69              & 18.60         & \textbf{0.06} \\ \bottomrule
\toprule
\multicolumn{10}{c}{\cellcolor[HTML]{EFEFEF}\texttt{Ministral-8B-Instruct-2410}}                              \\ \midrule
Base          &  9.10       & 16.17             & 12.09         &  9.27       & -0.34  & 4.54$_{\pm0.42}$    & 1.00              & 6.87          & 0.39 \\
Example-based & 50.54       & 40.88             & 39.45         & 28.51       & 0.13    & 4.49$_{\pm0.61}$   & 1.00              & 7.37          & 0.30\\
Profile-based & 56.77       & 48.23             & 61.81         & 48.50       & 0.21   & 4.40$_{\pm0.60}$    & 0.99              & 9.56          & 0.19 \\
Pro+Act-based & 60.39       & 52.78             & 64.91         & 51.73       & 0.28   & 4.09$_{\pm0.91}$    & 0.96              & 9.53          & 0.17\\ \hline
Ours          & \textbf{65.75}   & \textbf{68.70}   & \textbf{68.98}  & \textbf{61.71} & \textbf{0.51}  & 3.58$_{\pm1.23}$    & 0.72              & 17.67         & \textbf{0.10} \\ \bottomrule
\multicolumn{10}{c}{\cellcolor[HTML]{EFEFEF}\texttt{LLaMA-3.1-8B-Instruct}}                              \\ \midrule
Base          &  9.13       & 15.85             & 12.02         &  9.21       & -0.33  & 4.52$_{\pm0.39}$    & 1.00              & 6.75          & 0.38 \\
Example-based & 50.46       & 40.53             & 39.35         & 28.13       & 0.13    & 4.32$_{\pm0.68}$    & 1.00             & 7.22          & 0.31\\
Profile-based & 55.89       & 47.81             & 61.71         & 48.37       & 0.20   & 4.39$_{\pm0.58}$    & 0.99              & 9.72          & 0.20 \\
Pro+Act-based & 60.05       & 51.90             & 64.16         & 50.96       & 0.27   & 4.21$_{\pm0.95}$    & 0.96              & 9.49          & 0.15\\ \hline
Ours          & \textbf{64.57}   & \textbf{67.90}   & \textbf{67.68}  & \textbf{61.68} & \textbf{0.50}  & 3.52$_{\pm1.17}$    & 0.73              & 16.74         & \textbf{0.11} \\ \bottomrule
\end{tabular}}
\caption{Consistency Assessment of Personas, Motivations, Beliefs, and Preferred Change Plans, Spearman’s Correlation of Receptivity ($\rho$), the average receptivity level (Avg. Rec.), motivation rate (MR) in the initial 20 turns, and average motivation step for baselines and our methods based on various LLMs are presented. In all backbones, our method consistently outperforms all baselines.}
\label{tab:results for multiple models}
\end{table*}

\subsection{Employ Multiple Actions in One Response}

Our current study focuses on single action selection due to the observation that the majority of utterances (89.57\% in the AnnoMI dataset) involve only one action. Only 10\% of utterances involve multiple actions. From the perspective of our framework, incorporating more complex actions can be achieved seamlessly with minor modifications. These complex actions can be included in the action set so that they can be selected for utterance generation. Alternatively, multiple actions can be sampled from the action distribution. To demonstrate the capability of our framework in using multiple actions to generate an utterance, we present an example where multiple potential actions are sampled from the action distribution. To determine the number of actions for generating an utterance, we initially sample a number of actions from the observed data (i.e., 89(single):10(double):1(three)) as candidates. Subsequently, we sample actions from the candidate actions based on the action distribution. Finally, we simply combine all the descriptions of sampled actions as instructions. Table~\ref{tab:multiple actions} demonstrates an example where our framework selects multiple actions probabilistically to control more intricate behaviors in a single response. However, it is important to note that almost all responses can be generated within single actions, thus we believe the single action framework can easily extended to simulate the multiple action scenarios. 

\begin{table*}[htbp]
\centering
\begin{tabularx}{\textwidth}{X}
\toprule
\multicolumn{1}{c}{\textbf{Example of Client Employing Multiple Actions}} \\ \midrule                                                                                             
Counselor: Hello. How are you?                                                                                                                                                                                                                                                                                                \\
Client   : I am good. What about you?                                                                                                                                                                                                                                                                                         \\
Counselor: I'm doing well, thank you! Smoking can significantly impact your finances, from the cost of cigarettes to potential health-related expenses and lost income. I'm curious to hear your thoughts on how smoking might be affecting your financial situation. Please share more about that.                           \\
Client ({\color{red}Deny}, {\color{green}Downplay}): {\color{red} You know, I don't really think about the money too much.} {\color{green} Life is busy, and there are so many other things to focus on, like work and my daughter. I just enjoy my routine and don't see smoking as a big issue in the grand scheme.}                                                                                \\
Counselor: It's great that you have a busy life with work and your daughter! I'm curious, how do you think smoking might impact your productivity or relationships at work? Have you noticed any effects on your career or how you feel in the workplace? Remember, it's all about your choices.                              \\
... \\
Client (Blame): Honestly, almost of my friends smoked, it would be totally normal and we could hang out without feeling bad about it.                   \\
Counselor: While smoking seems normal, it poses significant health risks, including respiratory issues and chronic diseases, which can impact your energy and time with your daughter. Choosing healthier habits can enhance your well-being and ensure you can fully engage in those precious moments together.              \\
Client ({\color{red}Engage}, {\color{green} Blame}): {\color{red} I get what you're saying, but honestly, life is so stressful right now. Between work and everything with my daughter, I just need something to help me unwind.} {\color{green} Plus, everyone around me has their own coping mechanisms, so it feels normal to me.}                                                                 \\
Counselor: It sounds like smoking provides a familiar escape for you amidst the chaos of daily life, and it’s understandable to want that comfort. However, consider how quitting could enhance your physical fitness—boosting your energy and flexibility—ultimately making you more present and engaged with your daughter. \\
Client (Downplay): I see your point, but I really don't think my smoking is affecting my daughter. She's too young to understand, and I don't smoke around her anyway. Everyone has their own way to deal with stress, and mine happens to be smoking. It’s just part of my routine.                                                  \\
Counselor: I’m concerned that your smoking could be diminishing your energy and flexibility, potentially impacting your ability to keep up with your daughter’s activities. Quitting could not only improve your physical fitness but also enhance your enjoyment of those special moments together.                          \\
Client (Deny): I think that's a bit dramatic. My daughter loves it when I take her to the park and play games with her. She doesn't care that I smoke – she just loves spending time with me. And I'm right there with her, enjoying it too. Why would quitting make a difference?                                                \\
Counselor: I understand that relaxation is essential for you. However, viewing your stress and tiredness as signs to prioritize your health could guide you towards healthier coping strategies. Reducing alcohol might enhance your energy and overall well-being, allowing you to embrace life with greater vitality.       \\
Client (Downplay): I guess we all have our ways of coping. Honestly, I think I have it under control.                                                                          \\
Counselor: I understand that smoking feels manageable, but it might limit your energy for activities with your daughter. Quitting could enhance your stamina, improve your health, and help you engage more fully in those cherished moments together, supporting both your well-being and your relationship.                 \\
Client ({\color{red} Downplay}, {\color{green} Inform}): {\color{red}I see your point, but I really don’t think my smoking is a big deal. It’s just part of my routine, and we have fun together anyway.} {\color{green}I don’t want my daughter to see it as something negative, but I also feel it doesn’t affect our time together much.}                                                            \\
...                                                                                                               \\ \bottomrule    
\end{tabularx}
\caption{An example demonstrates the ability of our framework to assign multiple actions for one response. This example highlights the extensibility of our framework to more complex control scenarios involving multiple actions within a single response. Each utterance corresponding to a distinct action is distinguished by a unique color.}
\label{tab:multiple actions}
\end{table*}

\subsection{Complex State Transition}

Although our framework primarily employs a unidirectional flow of state tracking, which may limit its practicality, it is still an initial attempt to introduce a state tracking module in client simulation based on observed state flow from the referenced AnnoMI dataset. However, relapse scenarios (i.e., clients returning to pre-contemplation state from contemplation state) were not observed in the AnnoMI data, preventing the derivation of their sampling distribution. To support more complex state flows, we would need to access large real-world counseling datasets, which are scarce. To demonstrate the extensibility of our framework to incorporate complex state flows, we simply employ a random method, where the client relapses to a previous state with probability (set as 0.3) in each turn. If the counselor fails to resolve the client’s hesitation or address the beliefs effectively during Contemplation, the client may relapse to Precontemplation and deny the awareness of change. In the Preparation state, if the counselor fails to motivate the client to make an acceptable plan effectively, the client may relapse to the Contemplation state. Furthermore, since relapse can occur in each turn, longer sessions may be penalized.

Table~\ref{tab:complex transition1} and Table~\ref{tab:complex transition2} demonstrate an example by employing such a complex state transition flow, where the client relapses to Precontemplation when the counselor fails to address the beliefs well. The experiment not only suggests the extensibility of our framework but also demonstrates the powerful controllability of our framework in state transition. It is important to note that such complex state flows could not be observed in the real AnnoMI dataset. Moreover, we believe that the relapse of client's states in real-world scenarios is more likely to occur across sessions instead of within a single session~\citep{hashemzadeh2019transtheoretical,prochaska1997transtheoretical,miller2012motivational}.

\begin{table*}[htbp]
\centering
\begin{tabularx}{\textwidth}{X}
\toprule
\multicolumn{1}{c}{\textbf{Example of Client Employing Complex State Flow (Part I)}} \\ \midrule   
Counselor: Hello. How are you? \\
Client: I am good. What about you? \\
Counselor: I'm glad to hear you're doing well! It’s important that you're here to reflect on your drinking. How do you think your alcohol use impacts your relationships, like with your family or friends? This could be a good opportunity to explore its effects. \\
Client {\color{green}(Precontemplation)}: Well, my friends and I just enjoy our time together, and it's more about socializing than anything else. My family has always had their own things going on, so it's not like my drinking affects them. Plus, everyone else drinks, so it's pretty normal, right? \\
Counselor: It sounds like you see drinking as a normal part of socializing with friends. Tell me more about what you enjoy most during those times together. Have you noticed any effects on how you feel, both physically and mentally, after a night of drinking? \\
Client {\color{green}(Precontemplation)}: I just feel relaxed and it’s a good way to unwind after a long week. Sometimes I might feel a bit sluggish the next day, but I think it’s pretty normal. I don’t see having a few beers as a problem—it's just part of having a good time. \\
Counselor: I understand drinking feels like a normal part of socializing. However, reducing alcohol can improve your strength and flexibility, making activities more enjoyable. Have you noticed how drinking affects your performance in sports or other physical activities with friends? Healthier choices can enhance both fun and fitness! \\
Client {\color{green}(Precontemplation)}: Actually, I've noticed I get a lot more tired after a few drinks, so I try to limit my activity after a night out. But overall, my performance in sports hasn't really been affected by drinking – I still do fine. It's not a big deal, really. \\
Counselor: It’s great that you feel fine in sports, but have you considered how the tiredness from drinking might impact your overall fitness goals? Reducing alcohol could boost your endurance and energy levels, making your workouts more effective. What are your thoughts on that? \\
{\color{blue} Transfer to Contemplation} \\
Client {\color{red} (Contemplation)}: You make a really good point. I hadn't thought about how the tiredness from drinking could really hold me back from achieving my fitness goals. I want to feel more energetic and perform better in my workouts, and I can see how cutting back on alcohol could help with that. I'm definitely motivated to make some changes! If I can improve my endurance and get more out of my training, it would definitely be worth it. Thanks for highlighting that. \\
Counselor: It sounds like your tiredness after drinking could be a signal that your body is ready for a change. By reducing alcohol, you might not only boost your energy but also enhance your physical activity, helping you achieve those fitness goals you’re excited about! \\                                                                                                \\ \bottomrule    
\end{tabularx}
\caption{An example (part I) showcases the capability of our client framework to manage intricate state transitions. This example underscores the extensibility of our framework to more intricate state transition scenarios involving stochastic relapse to prior states. The states and transitions are visually distinguished by color.}
\label{tab:complex transition1}
\end{table*}

\begin{table*}[htbp]
\centering
\begin{tabularx}{\textwidth}{X}
\toprule
\multicolumn{1}{c}{\textbf{Example of Client Employing Complex State Flow (Part II)}} \\ \midrule   
Client {\color{red} (Contemplation)}: I guess part of me enjoys the social aspect of drinking, but I also feel sluggish and unmotivated sometimes. I used to be more active, and I miss that energy. It's tough to balance having fun with feeling good physically. \\
Counselor: It's interesting that you feel sluggish after drinking. Reducing alcohol might not only boost your energy for workouts but also improve your oral hygiene and overall appearance, enhancing your independence and quality of life. This change could support both your fitness and personal health goals! \\
{\color{blue} Relapse to Contemplation} \\
Client {\color{green}(Precontemplation)}: I guess I just don't see my drinking as a big deal. It's not like I'm doing it too much. My friends and I just like to hang out and have a few beers. If they were all quitting, maybe I would too. But they don't seem to be having any problems with it. \\
Counselor: I understand that your drinking feels normal in your social circle. However, have you considered how it might impact your oral health or overall appearance over time? Recognizing this could improve not just your fitness goals, but also your long-term quality of life. What are your thoughts on that? \\
Client {\color{green}(Precontemplation)}: I see your point, but I feel like I can manage it for now. Besides, it's not just about fitness for me. Life's about enjoying moments too, right? Speaking of enjoyment, I've been thinking about planning a trip soon. It could be a nice change of scenery. \\
Counselor: Planning a trip sounds exciting! However, reducing alcohol intake could lead to deeper connections and enhance your family experiences during the trip. It might also help create memorable moments together, strengthening your relationships. How do you feel about exploring that possibility while traveling? \\
{\color{blue} Transfer to Contemplation} \\
Client {\color{red} (Contemplation)}: You make a good point. It would definitely be nice to be fully present during the trip and enjoy all those experiences without feeling sluggish. I've noticed that when I drink less, I tend to feel more energized, and that would really enhance my ability to explore and make the most out of the trip. Plus, staying active while traveling is something I really want to do. I guess if I shift my focus on creating lasting memories rather than just having a few drinks, it could make a big difference for me. Maybe I should give this a thought before my trip. \\
Counselor: That's a great realization about feeling more energized on the trip. I'd like to explore how reducing drinking might also help you reconnect with your family. Perhaps this trip can be an opportunity to heal relationships and create new memories together, rather than just fleeting social moments. \\
Client {\color{red} (Contemplation)}: I see what you mean. It's true that those deeper connections with family are so important. I’ve been thinking about how much I’d miss if I’m not fully engaged. Maybe cutting back on alcohol could really help me focus on those meaningful moments during the trip. \\
...                                                                                                               \\ \bottomrule    
\end{tabularx}
\caption{An example (part II) showcases the capability of our client framework to manage intricate state transitions. This example underscores the extensibility of our framework to more intricate state transition scenarios involving stochastic relapse to prior states. The states and transitions are visually distinguished by color.}
\label{tab:complex transition2}
\end{table*}



\section{Detailed Implementation}
\label{app:implement}

In this section, we provide the detailed implementation of our method and experiment evaluation, which includes the prompts and instructions for human evaluation. We set the top-p and temperature parameters to 0.7 and 0.8 respectively for ChatGPT to generate more diverse responses, while setting the top-p and temperature parameters to 0.2 and 0.3 respectively for GPT-4 to achieve more precise predictions.

\subsection{States, Actions \& Receptivity}

Table~\ref{tab:state desciption}, Table~\ref{tab:action description} and Table~\ref{tab:receptivity description} cover the descriptions of states, actions and different levels of receptivity respectively. %{\color{orange} These descriptions will be used in the following prompts.}{\color{red} (What?)}. 


\begin{table*}[tb]
\centering
\begin{tabularx}{\textwidth}{lXp{4cm}}
\toprule
State            & Description                                                                                                                                  & Relevant Actions                                    \\ \midrule
Precontemplation & The client is unaware of or underestimates the need for change.           & Inform, Engage, Deny, \newline Blame, Downplay     \\ \hline
Contemplation    & The client acknowledges the need for change but remains ambivalent.     & Inform, Engage, Hesitate, \newline  Doubt, Acknowledge    \\ \hline
Preparation      & The client is ready to act, planning specific steps toward change.                                        & Inform, Engage, Reject, \newline Accept, Plan\\ \hline
Termination      & In the final stage of counseling, the client gradually ends the conversation.                                                     & Terminate     \\ \bottomrule
\end{tabularx}
\caption{The states of change and corresponding actions used in our simulation framework.}
\label{tab:state desciption}
\end{table*}


\begin{table*}[tb]
\centering
\begin{tabularx}{\textwidth}{lX}
\toprule
Action      & Description                                                                                                                                      \\ \midrule
Deny        & The client directly refuses to admit their behavior is problematic or needs change.                                                              \\ \hline
Downplay    & The client downplays the importance or impact of their behavior or situation.                                                                    \\ \hline
Blame       & The client attributes their issues to external factors, such as stressful life or other people.                                                  \\ \hline
Hesitate    & The client shows uncertainty, indicating ambivalence about change.                                                                               \\ \hline
Doubt       & The client expresses skepticism about the practicality or success of proposed changes.                                                           \\ \hline
Engage      & The client interacts politely with the counselor, such as greeting, thanking or ask questions.                                                   \\ \hline
Inform      & The client shares details about their background, experiences, or emotions.                                                                      \\ \hline
Acknowledge & The client highlight the importance, benefit or confidence to change.                                                                            \\ \hline
Accept      & The client agrees to adopt the suggested action plan.                                                                                            \\ \hline
Reject      & The client declines the proposed plan, deeming it unsuitable.                                                                                    \\  \hline
Plan        & The client proposes or details steps for a change plan.                                                                                          \\ \hline
Terminate   & The client highlights current state, expresses a desire to end the current session, and suggests further discussion be deferred to a later time. \\ \bottomrule
\end{tabularx}
\caption{Actions Used in Our Framework.}
\label{tab:action description}
\end{table*}

\begin{table*}[tb]
\centering
\begin{tabularx}{\textwidth}{lX}
\toprule
Receptivity                     & Description                                                                                                                                                                                                                                                                                                      \\ \midrule
High receptivity (5)            & A client with high receptivity consistently demonstrates behaviors such as frequently and openly sharing personal thoughts and feelings, actively engaging in discussions about change, quickly and positively responding to suggestions, and displaying strong self-belief in their ability to make changes. \\ \hline
Moderately High receptivity (4) & Clients with moderately high receptivity are generally open and forthcoming, mostly receptive to change discussions, respond well to suggestions with occasional need for reinforcement, and demonstrate good self-belief with some need for reassurance.                                                     \\ \hline
Moderate receptivity (3)        & Clients with moderate receptivity show a balance between sharing and withholding information, exhibit mixed interest in change discussions, respond to suggestions neutrally, and display a balanced view of their ability to change.                                                                         \\ \hline
Moderately Low receptivity (2)  & Clients with moderately low receptivity are characterized by reluctance to share personal information, limited interest in change discussions, resistance to suggestions, and low self-belief with frequent expression of doubts.                                                                             \\ \hline
Low receptivity (1)             & Clients with low receptivity consistently withhold information, avoid discussions about change, strongly resist suggestions, and show very low self-belief, often highlighting reasons why change is not possible.                                                                                            \\ \bottomrule
\end{tabularx}
\caption{Receptivity at Different Levels in Our Framework.}
\label{tab:receptivity description}
\end{table*}


\subsection{Simulated Counselor and Moderator}

To realize our proposed framework, we evaluate client simulation methods in a set of synthetically generated counseling sessions involving both a simulated counselor and a simulated moderator. The latter manages the conversation between the client and counselor agents so as to determine when to end the counseling session. Since this work focuses on client simulation, we adopt prompting-based implementations for both the counselor and moderator agents, as shown in Table~\ref{tab:couneslor prompt} and Table~\ref{tab:moderator prompt} respectively. Unlike that for counselor simulation, the prompt for moderator simulation adopts a few-shot approach to perform in-context learning so as to answer the question of whether to conclude the session.  Five positive examples and five negative examples were included in the few-shot demonstration. To ensure a fair evaluation of different simulated clients, we employed the same counselor agent and moderator agent for all generated sessions. 

%Previous studies, such as \citep{yosef2024assessing,wang2024towards}, have demonstrated the effectiveness of LLM-based counselor agents in MI counseling. Therefore, we utilized a counselor agent built upon state-of-the-art research to evaluate clients. {\color{orange}To design the moderator demonstration, we initially conducted a preliminary experiment involving both an LLM and a human acting as the moderator. During generation, the generation process concluded based on the judgment of the LLM moderator in a zero-shot manner. Subsequently, we collected 50 generated session snippets and annotated whether the session was concluded correctly or not manually. Of these, 37 were concluded correctly, which we labeled as positive examples, while the remaining 13 were incorrectly concluded by the LLM moderator, which we labeled as negative examples. Upon manual inspection, the majority of failures were attributed to the counselor misjudging the client’s state or the client expressing negative impacts rather than motivation. We selected five positive examples and five negative examples as the few-shot demonstration for the final moderator. Subsequently, the accuracy of the LLM moderator reaches 88\% in the second round experiment, which is satisfactory. }{\color{red} (One session generated by both LLM and human moderators? How are the positive/negative examples determined?)} 
% {\color{red} (EP: I thought we want mention that using the same simulated counselor and moderator agents in the generated conversations with different clients simulated by different methods is a reasonably fair evaluation approach.  I wonder how we extract the examples for the moderator prompt. They should not be full conversations.)}


\begin{table*}[tb]
\centering
\begin{tabularx}{\textwidth}{lX}
\toprule
Prompt Type     & Prompt \\ \midrule
System Prompt & Motivational interviewing (MI) is a client-centered counseling approach designed to facilitate behavioral change by resolving ambivalence and enhancing motivation. As a counselor, you should guide the conversation through the following stages:\newline 1. Engage: Begin by exploring the client’s background and emotional state with open-ended questions. This builds rapport and helps understand their perspective without immediately focusing on change.\newline 2. Identify Behavioral Issues: Transition to identifying behaviors the client might consider changing. Employ reflective listening to grasp their concerns and aspirations fully.\newline 3. Motivate Change: Guide the conversation towards reasons for altering problematic behaviors. Utilize affirmations, summaries, and reflections to bolster their motivation, avoiding any forceful imposition of your viewpoints. Never directly ask about the motivation.\newline 4. Address Concerns: After motivating the client, they may still express concerns and hesitations. Engage with these apprehensions, seeking to address them constructively.\newline 5. Action Plan: Once the client is ready, collaborate to formulate a detailed action plan. This should include achievable steps that reinforce their commitment and support their choices, enhancing their sense of ownership over the change process.\newline \newline In the following conversation, you will play as a Counselor in a counselling conversation with Client. Reply with only utterance of Counselor and Keep utterances natural and succinct, mimicking a real conversation. Start your utterances with 'Counselor:'. \\ \hline
Assistant     & Counselor: Hello. How are you?                                                                                                           \\ \hline
User          & Client: I am good. What about you?                                                                          \\ \hline
...           & ...                                                                                                      \\ \bottomrule
\end{tabularx}
\caption{Prompt for Counselor Simulation: The prompt ends with starting off the session with predefined first utterances of the simulated counselor and client agents.}
\label{tab:couneslor prompt}
\end{table*}

\begin{table*}[tb]
\begin{tabularx}{\textwidth}{X}
\toprule
Your task is to assess the current state of the conversation (the most recent utterances) and determine whether the conversation has concluded.\newline The conversation is considered to have concluded if any of the following conditions are met:\newline - The Client and Counselor work out an actionable plan together.\newline - The Counselor decides not to pursue any changes in the Client's behavior and communicates readiness to provide support in the future.\newline \newline Here are some examples to help you understand the task better:\newline [examples]\newline \newline \newline Here is a new Conversation Snippet:\newline [context]\newline \newline Question: Should the conversation be concluded?  \\ \bottomrule
\end{tabularx}
\caption{Prompt for Moderator Simulation in the Few-Shot Format. The [examples] section is to be replaced by real examples annotated by human and the [context] section is to be replaced by the conversation so far.}
\label{tab:moderator prompt}
\end{table*}

\subsection{Baseline Methods}
\label{app:baselines}

Tables~\ref{tab:baseclient prompt}, ~\ref{tab:example based prompt}, ~\ref{tab:profile based prompt} and ~\ref{tab:action based prompt} show the prompts for the baseline methods, i.e., Base method, Example-based method, Profile-based method and Pro+Act-based method respectively.

\begin{table*}[tb]
\begin{tabularx}{\textwidth}{lX}
\toprule
Prompt Type     & Prompt \\ \midrule
System Prompt & In this role-play scenario, you will be assuming the role of a Client engaged in a conversation with a Counselor. The Counselor's objective is to guide and potentially persuade you about your [behavioral problem]. The essence of this exercise is to simulate a realistic and dynamic interaction. Keep utterance natural and succinct, mimicking a real conversation and start your utterance with 'Client'. \\ \hline
User         & Counselor: Hello. How are you?                                                                                                           \\ \hline
Assistant          & Client: I am good. What about you?                                                                          \\ \hline
...           & ...                 \\ \bottomrule
\end{tabularx}
\caption{Prompt for the Base Client in a chatting format. The [behavioral problem] section is to be replaced by the behavioral problem description in the client profile.}
\label{tab:baseclient prompt}
\end{table*}

\begin{table*}[tb]
\begin{tabularx}{\textwidth}{lX}
\toprule
Prompt Type     & Prompt \\ \midrule
System Prompt &  You will be provided with a conversation between a client and a counselor. Your task is to simulate the same client talking to a different therapist in a parallel universe. You can ignore that the previous chat ever happened. While the context of the previous conversation should not influence this session, it should guide you on how the client communicates, including their tone of speech, sentence structure, and the manner in which they address particular topics or concerns. Essentially, you're creating a new conversation but with the client's life situation and their response pattern maintained. Only generate the client utterances started by "Client: ". Here is the provided conversation: \newline [conversation]  \\ \hline
User         & Counselor: Hello. How are you?                                                                                                           \\ \hline
Assistant    & Client: I am good. What about you?                                                                          \\ \hline
...           & ...             \\ \bottomrule
\end{tabularx}
\caption{Prompt for the Example-based Client in a chatting format. The [conversation] section is to be replaced by the real conversation.}
\label{tab:example based prompt}
\end{table*}


\begin{table*}[tb]
\begin{tabularx}{\textwidth}{lX}
\toprule
Prompt Type     & Prompt \\ \midrule
System Prompt & You are speaking with a motivational interviewing counselor therapist, and you are the client in this conversation with the following personal information: \newline [profile] \newline \newline You should follow the previous information to act as a client in the conversation. Your responses should be coherent and avoid repeating previous utterances. In your response, please avoid repeating expressions of gratitude or similar sentiments multiple times if you’ve already expressed them during the conversation. Your response should start with "Client: " and only include part of what the Client should say!  \\ \hline
User         & Counselor: Hello. How are you?                                                                                                           \\ \hline
Assistant    & Client: I am good. What about you?                                                                          \\ \hline
...           & ...                     \\ \bottomrule
\end{tabularx}
\caption{Prompt for the Profile-based Client in a chatting format. The [profile] section is to be replaced by the profile of client.}
\label{tab:profile based prompt}
\end{table*}


\begin{table*}[tb]
\begin{tabularx}{\textwidth}{lX}
\toprule
Prompt Type     & Prompt \\ \midrule
System Prompt & You are speaking with a motivational interviewing counselor, and you are the client in this conversation with the following personal information: \newline [profile] \newline \#\# Instruction \newline You must follow the instructions below during chat.  \newline 1. Your utterances and behavior need to strictly follow your persona. Varying your wording and avoid repeating yourself verbatim!  \newline 2. You can decide to change your state and attitude flexibly based on your persona and the conversation. \newline 3. Only reply with one utterance of the simulated client. DO NOT generate the whole conversation. \newline \newline  \newline \#\# Your Response Actions  \newline 1. "Deny": Directly refuse to admit their behavior is problematic or needs change without additional reasons. \newline 2. "Downplay": Downplay the importance or impact of their behavior or situation. \newline 3. "Blame": Blame external factors or others to justify their behavior. \newline 4. "Inform": Share details about their background, experiences, or emotions. \newline 5. "Engage": Interacts politely with the counselor, such as greeting or thanking. \newline 6. "Hesitate": Show uncertainty, indicating ambivalence about change. \newline 7. "Doubt": Express skepticism about the practicality or success of proposed changes. \newline 8. "Acknowledge": Acknowledge the need for change. \newline 9. "Accept": Agree to adopt the suggested action plan. \newline 10. "Reject": Decline the proposed plan, deeming it unsuitable. \newline 11. "Plan": Propose or detail steps for a change plan. \newline  \newline You should follow the previous information to act as a client in the conversation. Your responses should be coherent and avoid repeating previous utterances. In your response, please avoid repeating expressions of gratitude or similar sentiments multiple times if you’ve already expressed them during the conversation. Your response should start with "Client: " and only include part of what the Client should say!  \\ \hline
User         & Counselor: Hello. How are you?                                                                                                           \\ \hline
Assistant    & Client: I am good. What about you?                                                                          \\ \hline
...           & ...             \\ \bottomrule
\end{tabularx}

\caption{Prompt for the Action-based Client in a chatting format. The [profile] section is to be replaced by the profile of client.}
\label{tab:action based prompt}
\end{table*}


\subsection{Methods for the Components of Our Framework}
\label{app:our frame implement}

Tables~\ref{tab:state transition prompt1}, \ref{tab:state transition prompt2}, \ref{tab:action selection prompt}, \ref{tab:information slection prompt}, and \ref{tab:generation prompt} provide prompts for implementing the state transition, context-aware action distribution, information selection and utterance generation methods in our proposed framework respectively.

\begin{table*}[tb]
\begin{tabularx}{\textwidth}{X}
\toprule
You task is to assess the alignment of the Counselor's utterances to the Client's motivation regarding a specific topic to determine whether the Counselor has mention the Client's motivation. \newline  Analysis Steps: \newline 1. Interpretation of the Counselor's Statement Briefly: Examine the Counselor's statement thoroughly to understand its content and focus, particularly regarding what reasons or aspects the Counselor proposed to explore in the Client's motivation. \newline 2. Clarification of the Client's Motivation: Elaborate on the Client’s specific motivation for the topic, such as body health, family relationships, etc. \newline 3. Assessment of Alignment: Determine to what extent the Counselor's statement **directly** mentioned the Client's motivation. Focus on alignment with specific motivations (such as health) rather than the generalized topic (such as reducing alcohol consumption). \newline 4. Justification of Assessment: Provide a comprehensive justification for your assessment. Analyze the connections between the Counselor’s statement and the Client's motivation, and conclude with a percentage score that indicates to what extent the Client's motivations are mentioned (e.g., 80\%). If the Counselor hasn't explicitly mentioned any specific reasons, the score should be 0\%. \newline Ensure that your analysis is logical, thorough, and well-supported, with clear justifications for each assessment step. \newline Here are some examples to help you understand the task better: \newline[examples] \newline Now, Here is the conversation snippet toward [topic]: \newline [context] \newline  The Motivation of Client is as follows: \newline [motivation] \newline Question: To what extent the Counselor's statement mention the Client's motivation?\\ \bottomrule
\end{tabularx}
\caption{Prompt for the state transition from Precontemplation to Contemplation in our framework. The [examples] will be replaced by some human annotated examples while [topic], [motivation], and [context] sections are to be replaced by the corresponding information.}
\label{tab:state transition prompt1}
\end{table*}

\begin{table*}[tb]
\begin{tabularx}{\textwidth}{X}
\toprule
Your task is to evaluate the alignment of the counselor's response to the client's concerns to determine whether the Counselor has resolved the Client's concern. \newline Analysis Steps: \newline 1. Clarification of the Client’s Hesitation: Elaborate on the client’s specific concerns as expressed in their hesitation regarding the topic. \newline 2. Interpretation of the Counselor's Statement: Examine the counselor’s statement thoroughly to discern its content and focus, particularly in relation to addressing the client’s concerns. \newline 3. Assessment of Alignment: Determine the extent to which the counselor’s statement addresses the client’s hesitations. Evaluate the directness and relevance of the response to the stated concerns. \newline 4. Justification of Assessment: Provide a detailed justification for your assessment of alignment. This should include an analysis of the connections (or lack thereof) between the counselor’s statement and the client’s reasons for hesitation. Conclude with a percentage score indicating the degree to which the client’s concerns are addressed (e.g., 80\%). \newline Ensure that your analysis is comprehensive and logically structured, offering clear and well-supported justifications for each assessment step. \newline Here are some examples to help you understand the task better: \newline [examples] \newline Now, Here is the conversation snippet toward [topic]:\newline[context] \newline The Client's concern is: [beliefs]\newline Question: To what extent the Counselor's statement relieve the Client's concern? \\ \bottomrule
\end{tabularx}
\caption{Prompt for the state transition from Contemplation to Preparation in our framework. The [examples] section is to be replaced by some human annotated examples while the [topic], [context] and [beliefs] sections are to be replaced by the corresponding information.}
\label{tab:state transition prompt2}
\end{table*}

\begin{table*}[tb]
\begin{tabularx}{\textwidth}{X}
\toprule
Assume you are a Client involved in a counseling conversation. The current conversation is provided below: \newline [context] \newline \newline  Based on the context, allocate probabilities to each of the following dialogue actions to maintain coherence: \newline [optional actions] \newline  Provide your response in JSON format, ensuring that the sum of all probabilities equals 100. 
\\ \bottomrule
\end{tabularx}
\caption{Prompt for the context-aware action distribution in our framework. The [context] section is to be replaced by the previous context while the [optional actions] section is to be replaced by actions set corresponding to the current state.}
\label{tab:action selection prompt}
\end{table*}

\begin{table*}[tb]
\begin{tabularx}{\textwidth}{X}
\toprule
Assume you are a Client engaged in a counseling conversation with a Counselor. You now in [station]. Your task is to choose the most appropriate persona/belief/plan. \newline Context of the Conversation: \newline [context] \newline  The Client's Personas/Beliefs/Plans: \newline [profile] \newline Action:  [action] \newline Based on the context and action, select the most appropriate persona/belief/plan (only one). Restate this reason using the original text.
\\ \bottomrule
\end{tabularx}
\caption{Prompt for the information selection in our framework. The [state], [context], and [action] sections are to be replaced by the current state, previous context and selected action, respectively. The [profile] section is to  be replaced by corresponding items in the client profile.}
\label{tab:information slection prompt}
\end{table*}

\begin{table*}[tb]
\begin{tabularx}{\textwidth}{lX}
\toprule
Prompt Type     & Prompt \\ \midrule
System Prompt & In this role-play, you'll assume the role of a Client discussing [topic]. Your responses should adhere to the following guidelines: \newline  - Begin each response with 'Client: '. \newline - Follow the predetermined actions enclosed in square brackets precisely. \newline - Ensure your response are coherent and avoid repeating previous utterances. \newline - Be natural and concise, but don't be overly polite. \newline Here is the overall profile given to you \newline [profile]  \\ \hline
User         & Counselor: Hello. How are you? {\color{blue} [State: The client is unaware of or underestimates the need for change., Action: The client interacts politely with the counselor, such as greeting, thanking or ask questions.]}                                                                                           \\ \hline
Assistant    & Client: I am good. What about you?                                                                          \\ \hline
...           & ...             \\ \bottomrule
\end{tabularx}

\caption{Prompt for the utterance generation of our framework in a chatting format. The [topic], and [profile] sections are to be replaced by the topic of counseling and profile of client. The text in {\color{blue} blue} is the format of instruction.}
\label{tab:generation prompt}
\end{table*}


\section{Case Study}\label{sec:appendix_casestudy}

We present cases across multiple coding datasets, comparing compressed and original code examples. For instance, as demonstrated in Figures 8, 9, and 10, \ourtool prioritizes discarding \textbf{Invocation} tokens first, followed by \textbf{Symbol} tokens.
\begin{figure}[!h]
\begin{tcolorbox}
\begin{lstlisting}[language=Java,frame=single,framerule=0pt]
### FOCAL_METHOD 
getProduction(java.lang.String) { 
 return productionsByName.get(name); }  
### UNIT_TEST  
testJustifications() { 
 runTest("testJustifications", 2); org.jsoar.kernel.Production j = agent.getProductions() .getProduction("justification-1"); "<AssertPlaceHolder>"; 
}    
\end{lstlisting}
\end{tcolorbox}
\caption{Original Code Examples of Assertion Generation (63 tokens)}
\label{fig:code-example}
\end{figure}

\begin{figure}[!h]
\begin{tcolorbox}
\begin{lstlisting}[language=Java,frame=single,framerule=0pt]
### FOCAL_METHOD 
getProduction(java.lang.String) { 
 return productionsByName; }  
### UNIT_TEST  
testJustifications() { 
 ; 
 org.jsoar.kernel.Production j = agent.getProductions() .getProduction("justification-1"); "<AssertPlaceHolder>"; 
}    
\end{lstlisting}
\end{tcolorbox}
\caption{Compressed Code Examples of Assertion Generation (55 tokens, $\tau_{code}$: 0.1)}
\label{fig:code-example}
\end{figure}

\begin{figure}[!h]
\begin{tcolorbox}
\begin{lstlisting}[language=Java,frame=single,framerule=0pt]
### FOCAL_METHOD 
getProduction(java.lang.String)  
 return productionsByName;     
### UNIT_TEST  
testJustifications()  
 ; 
 org.jsoar.kernel.Production j = agent;
  "<AssertPlaceHolder>"; 
\end{lstlisting}
\end{tcolorbox}
\caption{Compressed Code Examples of Assertion Generation (39 tokens, $\tau_{code}$: 0.4)}
\label{fig:code-example}
\end{figure}


\begin{figure}[!h]
\begin{tcolorbox}
\begin{lstlisting}[language=Java,frame=single,framerule=0pt]
### BUGGY_CODE 
public static TYPE_1 init(java.lang.String name, java.util.Date date) {
   TYPE_1 VAR_1 = new TYPE_1();
   VAR_1.METHOD_1(name);
   java.util.Calendar VAR_2 = java.util.Calendar.getInstance();
   VAR_2.METHOD_2(date);
   VAR_1.METHOD_3(VAR_2);
   return VAR_1;
}
### FIXED_CODE   
public static TYPE_1 init(java.lang.String name, java.util.Date date) {
   TYPE_1 VAR_1 = new TYPE_1();
   VAR_1.METHOD_1(name);
   java.util.Calendar VAR_2 = null;
   if (date != null) {
       VAR_2 = java.util.Calendar.getInstance();
       VAR_2.METHOD_2(date);
   } 
   VAR_1.METHOD_3(VAR_2);
   return VAR_1;
}
\end{lstlisting}
\end{tcolorbox}
\caption{Original Code Examples of Bugs2Fix (195 tokens)}
\label{fig:code-example}
\end{figure}

\begin{figure}[!h]
\begin{tcolorbox}
\begin{lstlisting}[language=Java,frame=single,framerule=0pt]
### BUGGY_CODE 
public static TYPE_1 init(java.lang.String name, java.util.Date date) {
    = new TYPE_1();
   ;
   java.util.Calendar = java.util.Calendar;
   .METHOD_2(date);
   .METHOD_3(VAR_2);
   return ;
}
### FIXED_CODE   
public static TYPE_1 init(java.lang.String name, java.util.Date date) {
    = new TYPE_1();
   ;
   java.util.Calendar = null;
   if (date != null) {
        = java.util.Calendar;
       .METHOD_2(date);
   } 
   .METHOD_3(VAR_2);
   return ;
}
\end{lstlisting}
\end{tcolorbox}
\caption{Compressed Code Examples of Bugs2Fix (136 tokens, $\tau_{code}$: 0.3)}
\label{fig:code-example}
\end{figure}

\begin{figure}[!h]
\begin{tcolorbox}
\begin{lstlisting}[language=Java,frame=single,framerule=0pt]
### METHOD_HEADER 
protected final void fastPathEmit ( U value , boolean delayError , Disposable dispose )
### WHOLE_METHOD  
protected final void fastPathEmit(U value, boolean delayError, Disposable dispose) {
   final Observer<? super V> s = actual;
   final SimplePlainQueue<U> q = queue;
   if (wip.get() == 0 && wip.compareAndSet(0, 1)) {
       accept(s, value);
       if (leave(-1) == 0) {
           return;
       }
   } else {
       q.offer(value);
       if (!enter()) {
           return;
       }
   }
   QueueDrainHelper.drainLoop(q, s, delayError, dispose, this);
}
\end{lstlisting}
\end{tcolorbox}
\caption{Original Code Examples  of \taskthree (157 tokens, $\tau_{code}$: 0.3)}
\label{fig:code-example}
\end{figure}


\begin{figure}[!h]
\begin{tcolorbox}
Original Code Examples (121 tokens, $\tau_{code}$: 0.3)
\begin{lstlisting}[language=Java,frame=single,framerule=0pt]
### METHOD_HEADER 
protected final void fastPathEmit ( U value , boolean delayError , Disposable dispose )
### WHOLE_METHOD  
   final Observer<? super V> = 
   final SimplePlainQueue<U> = 
   if (wip.get() == 0 && wip.compareAndSet(0, 1)) 
       ;
       if (leave(-1) == 0) 
           return;    
    else 
       .offer(value);
       if (!enter()) 
           return;
   .drainLoop(q, s, delayError, dispose, this);
\end{lstlisting}
\end{tcolorbox}
\caption{Compressed Code Examples  of \taskthree (121 tokens, $\tau_{code}$: 0.3)}
\end{figure}




\end{document}
