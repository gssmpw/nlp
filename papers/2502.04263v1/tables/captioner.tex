\begin{table}
    \caption{Performance (mAP) comparison between the proposed \bluetext{OTI} technique and the captioning-based approach on the image-to-image retrieval task. \bluetext{Blue} rows indicate the usage of \bluetext{OTI}-inverted features. $\cmark$ and $\xmark$ denote inter-modal and intra-modal approaches, respectively.}
    \label{tab:captioners}
    \vspace{-5pt}
    \centering
    \resizebox{0.65\linewidth}{!}{
           \begin{tabular}{lccccccc}
                \toprule
                Method & \shortstack{Inter \\ modal} & \rotatebox{90}{\smash{CUB}} & \rotatebox{90}{\smash{SOP}} & \rotatebox{90}{\smash{$\mathcal{R}$Oxford}} & \rotatebox{90}{\smash{$\mathcal{R}$Paris}} & \rotatebox{90}{\smash{Cars}} & \rotatebox{90}{\smash{\textit{Average}}} \\
                \midrule
                
                Baseline & \xmark & 22.9 & 34.4 & 42.6 & 67.9 & 24.6 & 38.5\\
                
                DeCap & \cmark & 4.4 & 2.0 & 0.1 & 1.2 & 2.5 & 2.0\\

                CoCa (COCO) & \cmark & 3.5 & 0.8 & 0.0 & 0.7 & 1.8 & 1.4\\

                CoCa (LAION) & \cmark & 17.6 & 3.9 & 8.4 & 28.2 & 23.6 & 16.3\\


                \bluetext{OTI} & \cmark & \cellcolor{tabhighlight}\textbf{24.6} & \cellcolor{tabhighlight}\textbf{35.1} & \cellcolor{tabhighlight}\textbf{43.0} & \cellcolor{tabhighlight}\textbf{70.3} & \cellcolor{tabhighlight}\textbf{28.0} & \cellcolor{tabhighlight}\hgreen{40.2}\\

                \bottomrule
            \end{tabular}
    }
\end{table}
