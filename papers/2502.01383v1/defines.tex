% \def\defeq{\triangleq}
\def\defeq{\stackrel{\text{def}}{=}}

% Операторы
\DeclareMathOperator{\Id}{\mathrm{Id}}
\DeclareMathOperator{\proba}{\mathbb{P}}
\DeclareMathOperator{\qroba}{\mathbb{Q}}
\newcommand{\jproba}[2]{\proba_{#1,#2}}
\newcommand{\mproba}[2]{\proba_{#1} \! \otimes \proba_{#2}}

\DeclareMathOperator{\expect}{\mathbb{E}} % Математическое ожидание.
\DeclareMathOperator{\dispersion}{\mathbb{D}} % Дисперсия.
\DeclareMathOperator{\variance}{\mathrm{Var}} % Дисперсия.
\DeclareMathOperator{\cov}{\mathrm{cov}}   % Ковариация.
\DeclareMathOperator{\supp}{\mathrm{supp}} % Носитель меры.
\DeclareMathOperator{\range}{\mathrm{ran}} % Область значений.
\DeclareMathOperator{\rank}{\mathrm{rank}} % Ранг.
\DeclareMathOperator{\trace}{\mathrm{Tr}} % Ранг.
%\DeclareMathOperator*{\argmax}{\arg\max}
%\DeclareMathOperator*{\argmin}{\arg\min}
\DeclareMathOperator{\diag}{diag}
\def\identity{\mathrm{I}}

% Сходимости.
\def\limproba{\underset{N \to \infty}{\overset{\proba}{\longrightarrow}}}
\newcommand{\almost}[1]{\overset{\mathrm{a.s.}}{#1}}
\def\almosteq{\almost{=}}

% Поточечная взаимная информация.
\DeclareMathOperator{\PMI}{\mathrm{PMI}}

% Расстояние Кульбака-Лейблера.
\DeclareMathOperator*{\DKLoperator}{D_{\mathrm{KL}}}
\newcommand{\DKL}[2]{\DKLoperator \left( #1 \, \Vert \, #2 \right)}

% Вариационная оценка Донскера-Варадхана:
\DeclareMathOperator*{\DVoperator}{\mathnormal{\widehat{I}^{(DV)}}}
\newcommand{\DV}[2]{\DVoperator \left( #1 ; #2 \right)}

%\widehat{I}^{(DV)}(X; Y)

% Функция логарифма правдоподобия.
\def\loglikelihood{\mathcal{L}}

% Множества.
\def\naturals{\mathbb{N}}
\def\integers{\mathbb{Z}}
\def\reals{\mathbb{R}}

% Распределения.
\def\PDF{p}
\def\normal{\mathcal{N}}
\def\uniform{\mathrm{U}}

% Фигурная скобка под матрицей.
\newcommand\undermat[2]{%
    \makebox[0pt][l]{$\smash{\underbrace{\phantom{%
        \begin{matrix}#2\end{matrix}}}_{\text{$#1$}}}$
    }#2%
}

% Линейная алгебра.
\newcommand{\dotprod}[2]{\left\langle #1 , #2 \right\rangle}

% Гиперболические функции: дополнительно.
\DeclareMathOperator{\sech}{sech}
\DeclareMathOperator{\csch}{csch}
\DeclareMathOperator{\arcsec}{asec}
\DeclareMathOperator{\arccot}{acot}
\DeclareMathOperator{\arccsc}{acsc}
\DeclareMathOperator{\arccosh}{aCosh}
\DeclareMathOperator{\arcsinh}{asinh}
\DeclareMathOperator{\arctanh}{atanh}
\DeclareMathOperator{\arcsech}{asech}
\DeclareMathOperator{\arccsch}{aCsch}
\DeclareMathOperator{\arccoth}{aCoth}

% Другое
\def\const{\textnormal{const}}
\def\Mid{\,\middle\vert\,}
\def\blankarg{\,\cdot\,}