\usepackage[OT4]{fontenc}
\usepackage[margin=1.2in]{geometry}
\usepackage{amssymb,amsfonts,amsmath,amsthm,amscd,dsfont,mathrsfs,pifont}
\usepackage{blkarray}
\usepackage{graphicx,float,psfrag,epsfig,color}
\usepackage{qtree}
\usepackage{comment}
% \usepackage[dvipsnames]{xcolor}
\usepackage{authblk,textcomp}
\usepackage{caption} 
\usepackage{subcaption}



%%Packages
\usepackage{latexsym, amsthm, amscd, amsfonts, mathrsfs, amsmath, amssymb, stmaryrd, tikz-cd, mathrsfs, bbm, url, esint, mathtools, enumerate, caption, subcaption}
 \usepackage{dsfont}
\usepackage{blkarray}
\usepackage[utf8]{inputenc} % allow utf-8 input
\usepackage[T1]{fontenc}    % use 8-bit T1 fonts
\usepackage{hyperref}       % hyperlinks
\usepackage{url}            % simple URL typesetting
\usepackage{booktabs}       % professional-quality tables
\usepackage{nicefrac}       % compact symbols for 1/2, etc.
\usepackage{microtype}      % microtypography
% \usepackage{xcolor}  
%%algorithm

\usepackage{algpseudocode}
\usepackage{enumitem}
\usepackage{comment}
\usepackage{bbm}
\usepackage[ruled,vlined]{algorithm2e}
% \usepackage{algorithm}


\footnotesep 14pt
\floatsep 27pt plus 2pt minus 4pt      % Nominal is double what is in art12.sty
\textfloatsep 40pt plus 2pt minus 4pt
\intextsep 27pt plus 4pt minus 4pt

\newcommand{\bookhskip}{\hskip 1em plus 0.5em minus 0.4em\relax}

%%Commands and operators
\newcommand{\Corr}{\mathrm{corr}}
\newcommand{\Cov}{\mathrm{cov}}
\newcommand{\pr}{\mathbb{P}}
\newcommand{\E}{\mathbb{E}}
\DeclareMathOperator{\CP}{CP}
\DeclareMathOperator{\Var}{Var}
\newcommand{\Prr}{\mathop{\rm Pr}\nolimits}
\DeclareMathOperator{\argmin}{argmin}
\DeclareMathOperator{\argmax}{argmax}
\DeclareMathOperator{\Maj}{Maj}
\DeclareMathOperator{\Ramp}{Ramp}
\DeclareMathOperator{\ReLU}{ReLU}
\DeclareMathOperator{\Unif}{Unif}
\DeclareMathOperator{\Rad}{Rad}
\DeclareMathOperator{\INAL}{INAL}
\DeclareMathOperator{\sign}{sgn}
\DeclareMathOperator{\erf}{erf}
\DeclareMathOperator{\erfc}{erfc}
\DeclareMathOperator{\orb}{orb}
\DeclareMathOperator{\Span}{span}
\newcommand{\NN}{N}
\DeclareMathOperator{\poly}{poly}
\DeclareMathOperator{\TV}{TV}
\DeclareMathOperator{\Inf}{Inf}
\newcommand{\cF}{\mathcal{F}}
\newcommand{\cX}{\mathcal{X}}
\newcommand{\cU}{\mathcal{U}}
\newcommand{\cD}{\mathcal{D}}
\newcommand{\cN}{\mathcal{N}}
\newcommand{\cH}{\mathcal{H}}
\newcommand{\cP}{\mathcal{P}}
\newcommand{\bR}{\mathbb{R}}
\newcommand{\bI}{\mathbb{I}}
\newcommand{\bN}{\mathbb{N}}
\newcommand{\cK}{\mathcal{K}}
\newcommand{\cL}{\mathcal{L}}
\newcommand{\cB}{\mathcal{B}}
\newcommand{\bS}{\mathbb{S}}
\newcommand{\w}{\mathbf{w}}
\newcommand{\x}{\mathbf{x}}
\newcommand{\dm}[1]{{\color{red}{[[{\bf Dan:} #1]]}}}

\newcommand{\bmu}{\boldsymbol{\mu}}



%Colors
\definecolor{mydarkblue}{rgb}{0,0.08,0.45}
  \hypersetup{ %
    colorlinks=true,
    linkcolor=mydarkblue,
    citecolor=mydarkblue,
    filecolor=mydarkblue,
    urlcolor=mydarkblue,
    }
\definecolor{DSgray}{cmyk}{0,0,0,0.7}
\definecolor{DSred}{cmyk}{0,0.7,0,0.7}


%%Author notes
\newcommand{\Authornote}[2]{\noindent{\small\textcolor{DSgray}{\sf{
\textcolor{purple}{[#1: #2]}}}}}
\newcommand{\Authornoteo}[2]{\noindent{\small\textcolor{DSgray}{\sf{
\textcolor{orange}{[#1: #2]\marginpar{\textcolor{orange}{\fbox{\Large !}}}}}}}}
\newcommand{\Authornotec}[2]{\noindent{\small\textcolor{DSgray}{\sf{
\textcolor{blue}{[#1: #2]\marginpar{\textcolor{blue}{\fbox{\Large !}}}}}}}}
\newcommand{\enote}{\Authornote{Elisabetta}}
\newcommand{\Lnote}{\Authornote{Elchanan:}}
\newcommand{\dnote}{\Authornote{Dan:}}


%%Theorem styles
\theoremstyle{plain}
\newtheorem{definition}{Definition}%[section]
\theoremstyle{plain}
\newtheorem{theorem}{Theorem}%[section]
\theoremstyle{plain}
\newtheorem{proposition}[theorem]{Proposition}%[section]
\theoremstyle{plain}
\newtheorem{exe}{Exercise}%[section]
\theoremstyle{plain}
\newtheorem{exa}{Example}%[section]
\theoremstyle{plain}
\newtheorem{fact}{Fact}%[section]
\theoremstyle{plain}
\newtheorem{claim}{Claim}%[section]
\theoremstyle{plain}
\newtheorem{lemma}{Lemma}%[section]
\theoremstyle{plain}
\newtheorem{corollary}{Corollary}%[thm]
\theoremstyle{plain}
\newtheorem{conjecture}{Conjecture}%[section]
\theoremstyle{plain}
\newtheorem{remark}{Remark}%[section]
\theoremstyle{plain}
\newtheorem{assumption}{Assumption}%[section]
\theoremstyle{plain}
\newtheorem{note}{Note}%[section]
% \theoremstyle{discussion}
\newtheorem{discussion}{Discussion}%[section]
\theoremstyle{plain}
\newtheorem{conj}{Conjecture}%[section]