
%% bare_jrnl.tex
%% V1.4b
%% 2015/08/26
%% by Michael Shell
%% see http://www.michaelshell.org/
%% for current contact information.
%%
%% This is a skeleton file demonstrating the use of IEEEtran.cls
%% (requires IEEEtran.cls version 1.8b or later) with an IEEE
%% journal paper.
%%
%% Support sites:
%% http://www.michaelshell.org/tex/ieeetran/
%% http://www.ctan.org/pkg/ieeetran
%% and
%% http://www.ieee.org/

%%*************************************************************************
%% Legal Notice:
%% This code is offered as-is without any warranty either expressed or
%% implied; without even the implied warranty of MERCHANTABILITY or
%% FITNESS FOR A PARTICULAR PURPOSE! 
%% User assumes all risk.
%% In no event shall the IEEE or any contributor to this code be liable for
%% any damages or losses, including, but not limited to, incidental,
%% consequential, or any other damages, resulting from the use or misuse
%% of any information contained here.
%%
%% All comments are the opinions of their respective authors and are not
%% necessarily endorsed by the IEEE.
%%
%% This work is distributed under the LaTeX Project Public License (LPPL)
%% ( http://www.latex-project.org/ ) version 1.3, and may be freely used,
%% distributed and modified. A copy of the LPPL, version 1.3, is included
%% in the base LaTeX documentation of all distributions of LaTeX released
%% 2003/12/01 or later.
%% Retain all contribution notices and credits.
%% ** Modified files should be clearly indicated as such, including  **
%% ** renaming them and changing author support contact information. **
%%*************************************************************************


% *** Authors should verify (and, if needed, correct) their LaTeX system  ***
% *** with the testflow diagnostic prior to trusting their LaTeX platform ***
% *** with production work. The IEEE's font choices and paper sizes can   ***
% *** trigger bugs that do not appear when using other class files.       ***                          ***
% The testflow support page is at:
% http://www.michaelshell.org/tex/testflow/



\documentclass[journal]{IEEEtran}
%
% If IEEEtran.cls has not been installed into the LaTeX system files,
% manually specify the path to it like:
% \documentclass[journal]{../sty/IEEEtran}





% Some very useful LaTeX packages include:
% (uncomment the ones you want to load)


% *** MISC UTILITY PACKAGES ***
%
%\usepackage{ifpdf}
% Heiko Oberdiek's ifpdf.sty is very useful if you need conditional
% compilation based on whether the output is pdf or dvi.
% usage:
% \ifpdf
%   % pdf code
% \else
%   % dvi code
% \fi
% The latest version of ifpdf.sty can be obtained from:
% http://www.ctan.org/pkg/ifpdf
% Also, note that IEEEtran.cls V1.7 and later provides a builtin
% \ifCLASSINFOpdf conditional that works the same way.
% When switching from latex to pdflatex and vice-versa, the compiler may
% have to be run twice to clear warning/error messages.






% *** CITATION PACKAGES ***
%
%\usepackage{cite}
% cite.sty was written by Donald Arseneau
% V1.6 and later of IEEEtran pre-defines the format of the cite.sty package
% \cite{} output to follow that of the IEEE. Loading the cite package will
% result in citation numbers being automatically sorted and properly
% "compressed/ranged". e.g., [1], [9], [2], [7], [5], [6] without using
% cite.sty will become [1], [2], [5]--[7], [9] using cite.sty. cite.sty's
% \cite will automatically add leading space, if needed. Use cite.sty's
% noadjust option (cite.sty V3.8 and later) if you want to turn this off
% such as if a citation ever needs to be enclosed in parenthesis.
% cite.sty is already installed on most LaTeX systems. Be sure and use
% version 5.0 (2009-03-20) and later if using hyperref.sty.
% The latest version can be obtained at:
% http://www.ctan.org/pkg/cite
% The documentation is contained in the cite.sty file itself.
\usepackage{cite}





% *** GRAPHICS RELATED PACKAGES ***
%
\ifCLASSINFOpdf
  % \usepackage[pdftex]{graphicx}
  % declare the path(s) where your graphic files are
  % \graphicspath{{../pdf/}{../jpeg/}}
  % and their extensions so you won't have to specify these with
  % every instance of \includegraphics
  % \DeclareGraphicsExtensions{.pdf,.jpeg,.png}
\else
  % or other class option (dvipsone, dvipdf, if not using dvips). graphicx
  % will default to the driver specified in the system graphics.cfg if no
  % driver is specified.
  % \usepackage[dvips]{graphicx}
  % declare the path(s) where your graphic files are
  % \graphicspath{{../eps/}}
  % and their extensions so you won't have to specify these with
  % every instance of \includegraphics
  % \DeclareGraphicsExtensions{.eps}
\fi
% graphicx was written by David Carlisle and Sebastian Rahtz. It is
% required if you want graphics, photos, etc. graphicx.sty is already
% installed on most LaTeX systems. The latest version and documentation
% can be obtained at: 
% http://www.ctan.org/pkg/graphicx
% Another good source of documentation is "Using Imported Graphics in
% LaTeX2e" by Keith Reckdahl which can be found at:
% http://www.ctan.org/pkg/epslatex
%
% latex, and pdflatex in dvi mode, support graphics in encapsulated
% postscript (.eps) format. pdflatex in pdf mode supports graphics
% in .pdf, .jpeg, .png and .mps (metapost) formats. Users should ensure
% that all non-photo figures use a vector format (.eps, .pdf, .mps) and
% not a bitmapped formats (.jpeg, .png). The IEEE frowns on bitmapped formats
% which can result in "jaggedy"/blurry rendering of lines and letters as
% well as large increases in file sizes.
%
% You can find documentation about the pdfTeX application at:
% http://www.tug.org/applications/pdftex





% *** MATH PACKAGES ***
%
%\usepackage{amsmath}
% A popular package from the American Mathematical Society that provides
% many useful and powerful commands for dealing with mathematics.
%
% Note that the amsmath package sets \interdisplaylinepenalty to 10000
% thus preventing page breaks from occurring within multiline equations. Use:
%\interdisplaylinepenalty=2500
% after loading amsmath to restore such page breaks as IEEEtran.cls normally
% does. amsmath.sty is already installed on most LaTeX systems. The latest
% version and documentation can be obtained at:
% http://www.ctan.org/pkg/amsmath





% *** SPECIALIZED LIST PACKAGES ***
%
%\usepackage{algorithmic}
% algorithmic.sty was written by Peter Williams and Rogerio Brito.
% This package provides an algorithmic environment fo describing algorithms.
% You can use the algorithmic environment in-text or within a figure
% environment to provide for a floating algorithm. Do NOT use the algorithm
% floating environment provided by algorithm.sty (by the same authors) or
% algorithm2e.sty (by Christophe Fiorio) as the IEEE does not use dedicated
% algorithm float types and packages that provide these will not provide
% correct IEEE style captions. The latest version and documentation of
% algorithmic.sty can be obtained at:
% http://www.ctan.org/pkg/algorithms
% Also of interest may be the (relatively newer and more customizable)
% algorithmicx.sty package by Szasz Janos:
% http://www.ctan.org/pkg/algorithmicx




% *** ALIGNMENT PACKAGES ***
%
%\usepackage{array}
% Frank Mittelbach's and David Carlisle's array.sty patches and improves
% the standard LaTeX2e array and tabular environments to provide better
% appearance and additional user controls. As the default LaTeX2e table
% generation code is lacking to the point of almost being broken with
% respect to the quality of the end results, all users are strongly
% advised to use an enhanced (at the very least that provided by array.sty)
% set of table tools. array.sty is already installed on most systems. The
% latest version and documentation can be obtained at:
% http://www.ctan.org/pkg/array


% IEEEtran contains the IEEEeqnarray family of commands that can be used to
% generate multiline equations as well as matrices, tables, etc., of high
% quality.




% *** SUBFIGURE PACKAGES ***
%\ifCLASSOPTIONcompsoc
%  \usepackage[caption=false,font=normalsize,labelfont=sf,textfont=sf]{subfig}
%\else
%  \usepackage[caption=false,font=footnotesize]{subfig}
%\fi
% subfig.sty, written by Steven Douglas Cochran, is the modern replacement
% for subfigure.sty, the latter of which is no longer maintained and is
% incompatible with some LaTeX packages including fixltx2e. However,
% subfig.sty requires and automatically loads Axel Sommerfeldt's caption.sty
% which will override IEEEtran.cls' handling of captions and this will result
% in non-IEEE style figure/table captions. To prevent this problem, be sure
% and invoke subfig.sty's "caption=false" package option (available since
% subfig.sty version 1.3, 2005/06/28) as this is will preserve IEEEtran.cls
% handling of captions.
% Note that the Computer Society format requires a larger sans serif font
% than the serif footnote size font used in traditional IEEE formatting
% and thus the need to invoke different subfig.sty package options depending
% on whether compsoc mode has been enabled.
%
% The latest version and documentation of subfig.sty can be obtained at:
% http://www.ctan.org/pkg/subfig




% *** FLOAT PACKAGES ***
%
%\usepackage{fixltx2e}
% fixltx2e, the successor to the earlier fix2col.sty, was written by
% Frank Mittelbach and David Carlisle. This package corrects a few problems
% in the LaTeX2e kernel, the most notable of which is that in current
% LaTeX2e releases, the ordering of single and double column floats is not
% guaranteed to be preserved. Thus, an unpatched LaTeX2e can allow a
% single column figure to be placed prior to an earlier double column
% figure.
% Be aware that LaTeX2e kernels dated 2015 and later have fixltx2e.sty's
% corrections already built into the system in which case a warning will
% be issued if an attempt is made to load fixltx2e.sty as it is no longer
% needed.
% The latest version and documentation can be found at:
% http://www.ctan.org/pkg/fixltx2e


%\usepackage{stfloats}
% stfloats.sty was written by Sigitas Tolusis. This package gives LaTeX2e
% the ability to do double column floats at the bottom of the page as well
% as the top. (e.g., "\begin{figure*}[!b]" is not normally possible in
% LaTeX2e). It also provides a command:
%\fnbelowfloat
% to enable the placement of footnotes below bottom floats (the standard
% LaTeX2e kernel puts them above bottom floats). This is an invasive package
% which rewrites many portions of the LaTeX2e float routines. It may not work
% with other packages that modify the LaTeX2e float routines. The latest
% version and documentation can be obtained at:
% http://www.ctan.org/pkg/stfloats
% Do not use the stfloats baselinefloat ability as the IEEE does not allow
% \baselineskip to stretch. Authors submitting work to the IEEE should note
% that the IEEE rarely uses double column equations and that authors should try
% to avoid such use. Do not be tempted to use the cuted.sty or midfloat.sty
% packages (also by Sigitas Tolusis) as the IEEE does not format its papers in
% such ways.
% Do not attempt to use stfloats with fixltx2e as they are incompatible.
% Instead, use Morten Hogholm'a dblfloatfix which combines the features
% of both fixltx2e and stfloats:
%
% \usepackage{dblfloatfix}
% The latest version can be found at:
% http://www.ctan.org/pkg/dblfloatfix




%\ifCLASSOPTIONcaptionsoff
%  \usepackage[nomarkers]{endfloat}
% \let\MYoriglatexcaption\caption
% \renewcommand{\caption}[2][\relax]{\MYoriglatexcaption[#2]{#2}}
%\fi
% endfloat.sty was written by James Darrell McCauley, Jeff Goldberg and 
% Axel Sommerfeldt. This package may be useful when used in conjunction with 
% IEEEtran.cls'  captionsoff option. Some IEEE journals/societies require that
% submissions have lists of figures/tables at the end of the paper and that
% figures/tables without any captions are placed on a page by themselves at
% the end of the document. If needed, the draftcls IEEEtran class option or
% \CLASSINPUTbaselinestretch interface can be used to increase the line
% spacing as well. Be sure and use the nomarkers option of endfloat to
% prevent endfloat from "marking" where the figures would have been placed
% in the text. The two hack lines of code above are a slight modification of
% that suggested by in the endfloat docs (section 8.4.1) to ensure that
% the full captions always appear in the list of figures/tables - even if
% the user used the short optional argument of \caption[]{}.
% IEEE papers do not typically make use of \caption[]'s optional argument,
% so this should not be an issue. A similar trick can be used to disable
% captions of packages such as subfig.sty that lack options to turn off
% the subcaptions:
% For subfig.sty:
% \let\MYorigsubfloat\subfloat
% \renewcommand{\subfloat}[2][\relax]{\MYorigsubfloat[]{#2}}
% However, the above trick will not work if both optional arguments of
% the \subfloat command are used. Furthermore, there needs to be a
% description of each subfigure *somewhere* and endfloat does not add
% subfigure captions to its list of figures. Thus, the best approach is to
% avoid the use of subfigure captions (many IEEE journals avoid them anyway)
% and instead reference/explain all the subfigures within the main caption.
% The latest version of endfloat.sty and its documentation can obtained at:
% http://www.ctan.org/pkg/endfloat
%
% The IEEEtran \ifCLASSOPTIONcaptionsoff conditional can also be used
% later in the document, say, to conditionally put the References on a 
% page by themselves.




% *** PDF, URL AND HYPERLINK PACKAGES ***
%
%\usepackage{url}
% url.sty was written by Donald Arseneau. It provides better support for
% handling and breaking URLs. url.sty is already installed on most LaTeX
% systems. The latest version and documentation can be obtained at:
% http://www.ctan.org/pkg/url
% Basically, \url{my_url_here}.




% *** Do not adjust lengths that control margins, column widths, etc. ***
% *** Do not use packages that alter fonts (such as pslatex).         ***
% There should be no need to do such things with IEEEtran.cls V1.6 and later.
% (Unless specifically asked to do so by the journal or conference you plan
% to submit to, of course. )


% correct bad hyphenation here
\hyphenation{op-tical net-works semi-conduc-tor}
\usepackage{graphicx}
\usepackage{makecell}
\usepackage{xcolor}

\newcommand{\argmin}{\mathop{\mathrm{argmin}}}
\newcommand{\norm}[1]{\Vert#1\Vert}
\newcommand{\hl}[1]{\textcolor{red}{#1}}

\begin{document}
%
% paper title
% Titles are generally capitalized except for words such as a, an, and, as,
% at, but, by, for, in, nor, of, on, or, the, to and up, which are usually
% not capitalized unless they are the first or last word of the title.
% Linebreaks \\ can be used within to get better formatting as desired.
% Do not put math or special symbols in the title.
\title{Noise Controlled CT Super-Resolution with Conditional Diffusion Model}
%
%
% author names and IEEE memberships
% note positions of commas and nonbreaking spaces ( ~ ) LaTeX will not break
% a structure at a ~ so this keeps an author's name from being broken across
% two lines.
% use \thanks{} to gain access to the first footnote area
% a separate \thanks must be used for each paragraph as LaTeX2e's \thanks
% was not built to handle multiple paragraphs
%

\author{Yuang~Wang,
        Siyeop Yoon,
        Rui Hu,
        Baihui Yu,
        Duhgoon Lee,
        Rajiv Gupta,
        Li Zhang,
        Zhiqiang Chen,
        and Dufan~Wu
        % <-this % stops a space
\thanks{Y. Wang, S. Yoon, R. Hu, B. Yu, R. Gupta, and D. Wu are (were) with the Department of Radiology, Massachusetts General Hospital and Harvard Medical School, Boston MA 02114, USA. E-mail: (dwu6@mgh.harvard.edu).}% <-this % stops a space
\thanks{D. Lee is with Neurologica Corp., Danvers MA 01923, USA. }
\thanks{Y. Wang, L. Zhang, and Z. Chen are with the Department of Engineering Physics, Tsinghua University, Beijing 100084, China.}
}

% note the % following the last \IEEEmembership and also \thanks - 
% these prevent an unwanted space from occurring between the last author name
% and the end of the author line. i.e., if you had this:
% 
% \author{....lastname \thanks{...} \thanks{...} }
%                     ^------------^------------^----Do not want these spaces!
%
% a space would be appended to the last name and could cause every name on that
% line to be shifted left slightly. This is one of those "LaTeX things". For
% instance, "\textbf{A} \textbf{B}" will typeset as "A B" not "AB". To get
% "AB" then you have to do: "\textbf{A}\textbf{B}"
% \thanks is no different in this regard, so shield the last } of each \thanks
% that ends a line with a % and do not let a space in before the next \thanks.
% Spaces after \IEEEmembership other than the last one are OK (and needed) as
% you are supposed to have spaces between the names. For what it is worth,
% this is a minor point as most people would not even notice if the said evil
% space somehow managed to creep in.



% The paper headers
%\markboth{Journal of \LaTeX\ Class Files,~Vol.~14, No.~8, August~2015}%
%{Shell \MakeLowercase{\textit{et al.}}: Bare Demo of IEEEtran.cls for IEEE Journals}
% The only time the second header will appear is for the odd numbered pages
% after the title page when using the twoside option.
% 
% *** Note that you probably will NOT want to include the author's ***
% *** name in the headers of peer review papers.                   ***
% You can use \ifCLASSOPTIONpeerreview for conditional compilation here if
% you desire.




% If you want to put a publisher's ID mark on the page you can do it like
% this:
%\IEEEpubid{0000--0000/00\$00.00~\copyright~2015 IEEE}
% Remember, if you use this you must call \IEEEpubidadjcol in the second
% column for its text to clear the IEEEpubid mark.



% use for special paper notices
%\IEEEspecialpapernotice{(Invited Paper)}




% make the title area
\maketitle
\pagestyle{empty}  % no page number for the second and the later pages
\thispagestyle{empty} % no page number for the first page
% As a general rule, do not put math, special symbols or citations
% in the abstract or keywords.
\begin{abstract}
Improving the spatial resolution of CT images is a meaningful yet challenging task, often accompanied by the issue of noise amplification. This article introduces an innovative framework for noise-controlled CT super-resolution utilizing the conditional diffusion model. The model is trained on hybrid datasets, combining noise-matched simulation data with segmented details from real data. Experimental results with real CT images validate the effectiveness of our proposed framework, showing its potential for practical applications in CT imaging.
\end{abstract}

% Note that keywords are not normally used for peerreview papers.
\begin{IEEEkeywords}
Super-Resolution, Conditional Diffusion Model, Noise Controlling
\end{IEEEkeywords}






% For peer review papers, you can put extra information on the cover
% page as needed:
% \ifCLASSOPTIONpeerreview
% \begin{center} \bfseries EDICS Category: 3-BBND \end{center}
% \fi
%
% For peerreview papers, this IEEEtran command inserts a page break and
% creates the second title. It will be ignored for other modes.
\IEEEpeerreviewmaketitle


\begin{figure*}
\centering
   \includegraphics[width=18cm]{fig2.pdf}
   \caption
   {Framework of Noise Controlled CT Super-Resolution
   \label{fig:framework}
    }  %note label inside caption
\end{figure*}
\section{Introduction}
% The very first letter is a 2 line initial drop letter followed
% by the rest of the first word in caps.
% 
% form to use if the first word consists of a single letter:
% \IEEEPARstart{A}{demo} file is ....
% 
% form to use if you need the single drop letter followed by
% normal text (unknown if ever used by the IEEE):
% \IEEEPARstart{A}{}demo file is ....
% 
% Some journals put the first two words in caps:
% \IEEEPARstart{T}{his demo} file is ....
% 
% Here we have the typical use of a "T" for an initial drop letter
% and "HIS" in caps to complete the first word.
\IEEEPARstart{M}{edical} imaging, especially Computed Tomography (CT), is crucial for diagnosing and treating health conditions. Achieving higher resolution in CT scans is an ongoing challenge, and super-resolution techniques play a vital role in enhancing spatial resolution. This improvement promises more detailed information for clinicians, leading to better diagnostic accuracy and improved patient care. Diffusion models \cite{song2019generative, ho2020denoising}, rooted in probabilistic modeling and diffusion processes, offer significant advancements in natural imaging tasks, including super-resolution\cite{choi2021ilvr, saharia2022image, li2022srdiff}. Their stability, unlike Generative Adversarial Networks (GANs), ensures reliable image generation without the pitfalls of mode collapse or unrealistic artifacts. In the context of CT super-resolution, diffusion models stand as a source of inspiration, paving the way for novel developments in medical imaging.
%\hfill mds

Achieving super-resolution in CT images presents a complex challenge in noise control, particularly when compared to natural images. The training of conditional diffusion models for super-resolution necessitates paired sets of high resolution (HR) and low resolution (LR) CT images. 
Achieving super-resolution without noise amplification requires meticulous matching of noise levels in HR and LR CT images. The scarcity of noise-matched pairs, prompted by radiation exposure, has led certain methodologies to resort to downsampling HR CT images in either the image domain or projection domain to obtain corresponding LR CT images for model training \cite{zhang2021ct}. However, these approaches inadvertently exacerbate noise while improving spatial resolution. Some techniques introduce Gaussian noise to LR CT images\cite{8736838} before or after downsampling to harmonize with the noise level, yet their efficacy in super-resolution may waver when applied to real data due to the introduction of unrealistically distributed noise.

%\hfill August 26, 2015
In this article, we present an innovative framework for noise-controlled CT super-resolution, utilizing a conditional diffusion model trained on hybrid datasets. Numerical phantoms are employed to generate noise-matched simulation pairs of HR and LR CT images. Furthermore, details absent in numerical phantoms are segmented from noise-unmatched real pairs and integrated into the training process. Testing using real CT images validates the effectiveness of the proposed framework in real-world scenarios.
\section{Method}

\subsection{Conditional Diffusion Model}
In our article, we employ the Conditional Denoising Diffusion Probabilistic Model (DDPM)\cite{saharia2022image, li2022srdiff} for super-resolution in CT images. Here, the set of LR images serves as the condition $y$, and the set of HR images forms the generation target $x_0$. The Conditional DDPM contains forward and reverse processes. The forward process, a Markov chain, gradually introduces noise into the image until it becomes standard Gaussian noise, with the following process:

\begin{equation}
q\left(x_t|x_{t-1}\right)=N\left(x_t|\sqrt{1-\beta_t}x_{t-1},\beta_tI\right),
\label{eq:q(xt|xt-1)}
\end{equation}
where $t$ is the diffusion time step from 0 to T, and $\beta_t$ is a small positive hyperparameter determining the speed of the diffusion. $x_0$ is the original image and $x_T$ approximately follows the standard normal distribution.

From (\ref{eq:q(xt|xt-1)}), we can derive $q\left(x_t|x_0\right)$ and $q\left(x_{t-1}|x_0,x_t\right)$:

\begin{equation}
q\left(x_t|x_0\right)=N\left(x_t|\sqrt{\gamma_t}x_0,\left(1-\gamma_t\right)I\right),
\label{eq:q(xt|x0)}
\end{equation}
\begin{equation}
q\left(x_{t-1}|x_t,x_0\right)=N\left(x_{t-1}|\widetilde{\mu_t}\left(x_t,x_0\right),\widetilde{\beta_t}I\right),
\label{eq:q(xt-1|xt,x0))}
\end{equation}
where
\begin{equation}
\gamma_t=\prod_{s=1}^{t}\alpha_s,
\label{eq:gamma_t}
\end{equation}
\begin{equation}
\alpha_t = 1-\beta_t,
\label{eq:alpha_t}
\end{equation}
\begin{equation}
\widetilde{\mu_t}\left(x_t,x_0\right)=\frac{\sqrt{\gamma_{t-1}}\beta_t}{1-\gamma_t}x_0+\frac{\sqrt{\alpha_t}\left(1-\gamma_{t-1}\right)}{1-\gamma_t}x_t,
\label{eq:mu_t}
\end{equation}
and
\begin{equation}
\widetilde{\beta_t}=\frac{1-\gamma_{t-1}}{1-\gamma_t}\beta_t.
\label{eq:beta_t}
\end{equation}

The reverse process, a Markov chain aimed at generating $x_0$ from $x_T \sim N(0, I)$, is modeled as a Gaussian process $p_\theta$:
\begin{equation}
p_\theta\left(x_{t-1}|x_t,y\right)=N\left(x_{t-1}|\mu_\theta\left(x_t,y,t\right),\sigma_t^2\right),
\label{eq:p(xt-1|xt,y)}
\end{equation}
where $\sigma_t^2$ is a hyperparameter and is usually set as $\widetilde{\beta_t}$, and $\mu_\theta$ is trained to match $\widetilde{\mu_t}$ to minimize the Kullback–Leibler (K-L) divergence between $p_\theta\left(x_{t-1}|x_t,y\right)$ and $q\left(x_{t-1}|x_t,x_0\right)$:
\begin{equation}
\theta^*=\argmin_\theta E_q\left\{\frac{1}{2\sigma_t^2}\norm{\mu_\theta(x_t,y,t)-\widetilde{\mu_t}(x_t,x_0)}\right\}.
\label{eq:theta_train_by_mu}
\end{equation}

To simplify the training, we can parameterize $\mu_\theta(x_t,y,t)$ as:
\begin{equation}
\mu_\theta(x_t,y,t)=\widetilde{\mu_t}(x_t,\widehat{x}_{0,\theta}(x_t, y, t)),
\label{eq:mu_theta_reparam}
\end{equation}

where
\begin{equation}
\widehat{x}_{0,\theta}\left(x_t,y,t\right)=\frac{1}{\sqrt{\gamma_t}}\left(x_t-\sqrt{1-\gamma_t}\epsilon_\theta\left(x_t,y,t\right)\right).
\label{eq:x_0_hat}
\end{equation}

It leads to our training loss function
\begin{equation}
\theta^*=\argmin_\theta E_{x_0,y}E_{t,\epsilon}\norm{\epsilon_\theta\left(x_t,y,t\right)-\epsilon}_2^2,
\label{eq:training_loss}
\end{equation}
where
\begin{equation}
x_t=\sqrt{\gamma_t}x_0 + \sqrt{1 - \gamma_t}\epsilon, \epsilon \sim N(0, I).
\label{eq:x_t}
\end{equation}

After the network $\epsilon_\theta$ is trained, one can predict $x_0$ from $x_T \sim N(0,I)$ and the condition $y$ following (\ref{eq:p(xt-1|xt,y)}).

\begin{figure*}
\centering
   \includegraphics[width=15.5cm]{fig1.pdf}
   \caption
   {Architecture of the conditional noise predictor. The content in parentheses (c, 2c, 4c, 8c and 16c) after the block name indicates the number of output channels of each block, and in our implementation c is set to 32.
   \label{fig:network}
    }  %note label inside caption
\end{figure*}
\subsection{Framework of Noise-Controlled CT Super Resolution}
The framework for noise-controlled CT super-resolution is illustrated in Fig. \ref{fig:framework}. The HR images are the labels $x_0$, and the corresponding LR images are the conditions $y$. Instead of directly training the conditional diffusion model using noise-unmatched real data, which is prone to amplifying noise while enhancing spatial resolution, our approach integrates training with hybrid datasets. These datasets comprise both noise-matched simulation data and segmented bones from real data. 

Numerical phantoms are utilized to simulate CT scans with the same geometry parameters as real CT images, yielding noise-free projections. Assuming that the quantum noise is Gaussian, the following equations are used to inject correlated noises into the HR and LR images:
\begin{equation}
\widehat{p}_{HR}=p_{HR}+\frac{k_{HR}}{\sqrt{N_{HR}}}z
\label{eq:noise_hr}
\end{equation}
\begin{equation}
\widehat{p}_{LR}=p_{LR}+\frac{k_{LR}}{\sqrt{N_{LR}}}downsample\left(z\right),
\label{eq:noise_lr}
\end{equation}
where $z \sim N(0, I)$. $\widehat{p}$ and $p$ are the noisy and noiseless post-log projections, respectively. $N$ is the number of photons per ray that reached the detector, which is different for each detector. $k_{HR}$ and $k_{LR}$ are hyperparameters to match the noise levels of both simulated HR and LR images to that of the real LR image.

Because of the lack of details such as trabecular bone structures in the simulation phantoms, models trained with the simulation data alone would lead to significant oversmooth of the bony structures when applied to the real data. The bone regions from the real data are segmented using threshold followed by hole filling algorithms and opening and closing morphological operations, and serve as part of the training data, as demonstrated in Fig. \ref{fig:framework}. During testing, the trained model is used on both the original LR image and segmented bones. The pixels within the bone mask on the non-segmented super-resolved image are replaced by those in the bone-only super-resolution image. 

In summary, the noise-matched simulated images provide the model the capability of super-resolution without amplification of the noise. Meanwhile, the segmented HR and LR bone pairs encourage the preservation of the detailed bony structures in reality without promoting noise amplification too much. By adopting both training data, the model is trained to enhance spatial resolution without noise amplification and good preservation of the bony structures.


\subsection{Implementation Details}
The real training and testing data were acquired using the OmniTom PCD portable photon counting CT system (Neurologica, Danvers USA) at Massachusetts General Hospital. Acquisition parameters included 120 kVp and 40 mAs for 1$\times$1 binning data, with an effective detector size of 0.12 $\times$ 0.14 mm$^2$ at the isocenter. The projection data were rebinned to 3$\times$3 and 6$\times$5 to generate paired HR and LR real data. The real training dataset comprises 64 slices of a cadaver head, and the real testing dataset includes 32 slices of a separate temporal bone.

Simulation data were generated using the XCIST simulation tool\cite{wu2022xcist}, utilizing head parts from five whole-body NURBS phantoms it provides, resulting in a total of 1040 slices. The hyperparameters $k_{HR}$ and $k_{LR}$ in equations (\ref{eq:noise_hr}) and (\ref{eq:noise_lr}) are set to 0.12 and 0.60, respectively, to match the noise level of the simulation data to that of the real LR image.

Both real and simulation data were reconstructed into 0.3$\times$0.3 mm$^2$ HR images and 0.6$\times$0.6 mm$^2$ LR images using edge-enhancing bone filter. To align their sizes for the conditional diffusion model training, the LR images were upsampled by two times using sinc-interpolation before being fed into the network as condition $y$ \cite{wang2021real}.

The architecture of the conditional noise predictor we employed is illustrated in Fig. \ref{fig:network}. The  core structure of the conditional noise predictor is a U-Net, similar to the one utilized in the DDPM \cite{ho2020denoising}. $y$ and $x_t$ undergo a convolution layer separately and are then concatenated in the channel dimension. This structure is propagated through skip and residual connections across the entire U-Net, allowing the conditional noise predictor to extract information from $y$ and $x_t$ both separately and jointly.

We used $T=1,000$ as the total number of diffusion steps, and $\beta_t$ was selected according to the sigmoid schedule\cite{jabri2022scalable}. The model was trained on randomly cropped patches of 128 $\times$ 128, and tested on the full image. A batch size of 64 was used during the training, with 48 slices from the noise-matched simulation and 16 from the real segmented bone images. The network was trained by the Adam algorithm with a learning rate of $8\times10^{-5}$ for 100,000 iterations.


\section{Result}
\begin{figure*}
\centering
   \includegraphics[width=18cm]{fig3.pdf}
   \caption
   {Results on LR CT images of a temporal bone using our proposed framework and two comparison methods M1 and M2.
   \label{fig3}
    }  %note label inside caption
\end{figure*}

\begin{table}[htbp]
\begin{center}
\caption{Quantitative analysis for all tested methods on cases and ROIs in Fig. \ref{fig3}. 
\label{table 1}
\vspace*{0.1ex}
}
\begin{tabular} {|l|c|c|c|c|c|c|}
%\begin{edtable}{tabular} {|l|c|c|c|c|}
\hline
Method &\multicolumn{2}{c|}{\makecell[c]{STD (HU)}}&\multicolumn{2}{c|}{\makecell[c]{Haralick feature\\ distance}}&\multicolumn{2}{c|}{PSNR (dB)} \\
\cline{2-7}
        & ROI 1          & ROI 2 & ROI 3          & ROI 4  & Case 1          & Case 2   \\
\hline
        &               &               &               &       & & \\
HR      & 60.4           & 61.2          &  N/A         & N/A   & N/A & N/A        \\
&&&&&&  \vspace{-2mm}\\
LR      & 45.7           & 38.1          & 1367 & 1148    &  23.88         & 24.30     \\
&&&&&&  \vspace{-2mm}\\
Proposed      & 41.5           & 39.9         &325&535     &  23.79         & 24.33     \\
&&&&&&  \vspace{-2mm}\\
M1      & 43.3           & 40.3           &857&976  &  21.46         & 21.62      \\
&&&&&&  \vspace{-2mm}\\
M2      & 54.1          & 51.9          &324&550  &  22.97        & 24.26       \\
        &               &               &               &  &&      \\
\hline
\end{tabular}
%\end{edtable}
\end{center}
\end{table}

The representative results of noise-controlled super resolution from our proposed framework on LR CT images of a temporal bone are presented in Fig. \ref{fig3}, with quantitative analyses in Table \ref{table 1}. The proposed method was further compared to two other methods: M1, which was trained with simulation data only; and M2, which was similar to the proposed method but without matched noise levels in the simulation. 

Regions of interests (ROIs) are selected for quantitative analyses and marked in Fig. \ref{fig3} using yellow ellipses and rectangles. Uniform ROIs 1 and 2 are used to calculate the standard deviation for evaluating the noise level, and ROIs 3 and 4, which contain intricate details, are used to calculate Haralick feature \cite{haralick1973textural} distances to evaluate the similarity of the textures. Peak signal-to-noise ratio (PSNR) is also calculated to assess the consistency of the generated HR images. 

Notably, M1 oversmoothes the trabecular bone structures due to the lack of such structures in the training data. Due to the oversmoothing, it has the worst Haralick feature distances in the bone ROIs (3 and 4) and the worst PSNR compared to the proposed method and M2. M2 has the similar enhancement of the spatial resolution and bony details compared to the proposed method, but it amplifies the noise as shown by the increased standard deviations in ROIs 1 and 2, leading to slightly worse PSNR compared to the proposed method. The proposed method achieved spatial resolution improvement without amplification of the noise. Both M2 and the proposed method had similar PSNR with the LR image, demonstrating that they are not drastically changing the structures as in M1. No significant improvement of the PSNR was observed due to the presence of noise in the HR images.


% needed in second column of first page if using \IEEEpubid
%\IEEEpubidadjcol



% An example of a floating figure using the graphicx package.
% Note that \label must occur AFTER (or within) \caption.
% For figures, \caption should occur after the \includegraphics.
% Note that IEEEtran v1.7 and later has special internal code that
% is designed to preserve the operation of \label within \caption
% even when the captionsoff option is in effect. However, because
% of issues like this, it may be the safest practice to put all your
% \label just after \caption rather than within \caption{}.
%
% Reminder: the "draftcls" or "draftclsnofoot", not "draft", class
% option should be used if it is desired that the figures are to be
% displayed while in draft mode.
%
%\begin{figure}[!t]
%\centering
%\includegraphics[width=2.5in]{myfigure}
% where an .eps filename suffix will be assumed under latex, 
% and a .pdf suffix will be assumed for pdflatex; or what has been declared
% via \DeclareGraphicsExtensions.
%\caption{Simulation results for the network.}
%\label{fig_sim}
%\end{figure}

% Note that the IEEE typically puts floats only at the top, even when this
% results in a large percentage of a column being occupied by floats.


% An example of a double column floating figure using two subfigures.
% (The subfig.sty package must be loaded for this to work.)
% The subfigure \label commands are set within each subfloat command,
% and the \label for the overall figure must come after \caption.
% \hfil is used as a separator to get equal spacing.
% Watch out that the combined width of all the subfigures on a 
% line do not exceed the text width or a line break will occur.
%
%\begin{figure*}[!t]
%\centering
%\subfloat[Case I]{\includegraphics[width=2.5in]{box}%
%\label{fig_first_case}}
%\hfil
%\subfloat[Case II]{\includegraphics[width=2.5in]{box}%
%\label{fig_second_case}}
%\caption{Simulation results for the network.}
%\label{fig_sim}
%\end{figure*}
%
% Note that often IEEE papers with subfigures do not employ subfigure
% captions (using the optional argument to \subfloat[]), but instead will
% reference/describe all of them (a), (b), etc., within the main caption.
% Be aware that for subfig.sty to generate the (a), (b), etc., subfigure
% labels, the optional argument to \subfloat must be present. If a
% subcaption is not desired, just leave its contents blank,
% e.g., \subfloat[].


% An example of a floating table. Note that, for IEEE style tables, the
% \caption command should come BEFORE the table and, given that table
% captions serve much like titles, are usually capitalized except for words
% such as a, an, and, as, at, but, by, for, in, nor, of, on, or, the, to
% and up, which are usually not capitalized unless they are the first or
% last word of the caption. Table text will default to \footnotesize as
% the IEEE normally uses this smaller font for tables.
% The \label must come after \caption as always.
%
%\begin{table}[!t]
%% increase table row spacing, adjust to taste
%\renewcommand{\arraystretch}{1.3}
% if using array.sty, it might be a good idea to tweak the value of
% \extrarowheight as needed to properly center the text within the cells
%\caption{An Example of a Table}
%\label{table_example}
%\centering
%% Some packages, such as MDW tools, offer better commands for making tables
%% than the plain LaTeX2e tabular which is used here.
%\begin{tabular}{|c||c|}
%\hline
%One & Two\\
%\hline
%Three & Four\\
%\hline
%\end{tabular}
%\end{table}


% Note that the IEEE does not put floats in the very first column
% - or typically anywhere on the first page for that matter. Also,
% in-text middle ("here") positioning is typically not used, but it
% is allowed and encouraged for Computer Society conferences (but
% not Computer Society journals). Most IEEE journals/conferences use
% top floats exclusively. 
% Note that, LaTeX2e, unlike IEEE journals/conferences, places
% footnotes above bottom floats. This can be corrected via the
% \fnbelowfloat command of the stfloats package.




\section{Conclusion and Discussion}
In conclusion, our proposed framework for noise controlled CT super resolution, leveraging the Conditional DDPM, presents a promising approach to enhance the spatial resolution of CT images while effectively controlling noise. By incorporating segmented details from real data and ensuring a matched noise level in simulation data during training, our model achieved image super-resolution without noise amplification. These results show its potential for practical applications in CT imaging, contributing to advancements in medical imaging technology.

One major limitation of the current method is that the proposed model depends on the precision of details segmentation, and errors in this process may introduce artifacts or result in the loss of details in the final output. Additionally, the training using segmented details featuring a completely black background may pose challenges to the performance of the conditional diffusion model. In our future work, we aim to investigate more robust and effective approaches to seamlessly integrate details unique to real data with noise-matched simulation data, thereby further enhancing the efficacy of noise-controlled CT super-resolution.



% if have a single appendix:
%\appendix[Proof of the Zonklar Equations]
% or
%\appendix  % for no appendix heading
% do not use \section anymore after \appendix, only \section*
% is possibly needed

% use appendices with more than one appendix
% then use \section to start each appendix
% you must declare a \section before using any
% \subsection or using \label (\appendices by itself
% starts a section numbered zero.)
%


%\appendices
%\section{Proof of the First Zonklar Equation}
%Appendix one text goes here.

% you can choose not to have a title for an appendix
% if you want by leaving the argument blank
%\section{}
%Appendix two text goes here.


% use section* for acknowledgment
%\section*{Acknowledgment}


%The authors would like to thank...


% Can use something like this to put references on a page
% by themselves when using endfloat and the captionsoff option.
\ifCLASSOPTIONcaptionsoff
  \newpage
\fi



% trigger a \newpage just before the given reference
% number - used to balance the columns on the last page
% adjust value as needed - may need to be readjusted if
% the document is modified later
%\IEEEtriggeratref{8}
% The "triggered" command can be changed if desired:
%\IEEEtriggercmd{\enlargethispage{-5in}}

% references section

% can use a bibliography generated by BibTeX as a .bbl file
% BibTeX documentation can be easily obtained at:
% http://mirror.ctan.org/biblio/bibtex/contrib/doc/
% The IEEEtran BibTeX style support page is at:
% http://www.michaelshell.org/tex/ieeetran/bibtex/
\bibliographystyle{IEEEtran}
% argument is your BibTeX string definitions and bibliography database(s)
\bibliography{IEEEabrv}
%
% <OR> manually copy in the resultant .bbl file
% set second argument of \begin to the number of references
% (used to reserve space for the reference number labels box)
% \begin{thebibliography}{1}

% \bibitem{IEEEhowto:kopka}
%H.~Kopka and P.~W. Daly, \emph{A Guide to \LaTeX}, %3rd~ed.\hskip 1em plus
  %0.5em minus 0.4em\relax Harlow, England: Addison-Wesley, %1999.

%\end{thebibliography}

% biography section
% 
% If you have an EPS/PDF photo (graphicx package needed) extra braces are
% needed around the contents of the optional argument to biography to prevent
% the LaTeX parser from getting confused when it sees the complicated
% \includegraphics command within an optional argument. (You could create
% your own custom macro containing the \includegraphics command to make things
% simpler here.)
%\begin{IEEEbiography}[{\includegraphics[width=1in,height=1.25in,clip,keepaspectratio]{mshell}}]{Michael Shell}
% or if you just want to reserve a space for a photo:

%\begin{IEEEbiography}{Michael Shell}
%Biography text here.
%\end{IEEEbiography}

% if you will not have a photo at all:
%\begin{IEEEbiographynophoto}{John Doe}
%Biography text here.
%\end{IEEEbiographynophoto}

% insert where needed to balance the two columns on the last page with
% biographies
%\newpage

%\begin{IEEEbiographynophoto}{Jane Doe}
%Biography text here.
%\end{IEEEbiographynophoto}

% You can push biographies down or up by placing
% a \vfill before or after them. The appropriate
% use of \vfill depends on what kind of text is
% on the last page and whether or not the columns
% are being equalized.

%\vfill

% Can be used to pull up biographies so that the bottom of the last one
% is flush with the other column.
%\enlargethispage{-5in}



% that's all folks
\end{document}


