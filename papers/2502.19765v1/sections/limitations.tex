\section*{Limitations}
The proposed editing framework, EdiText, is built upon generative text models trained with attribute-specific text data. As such, it presents two key limitations. First, the performance of EdiText heavily depends on the generative capabilities of the backbone model. Since our approach assumes that the backbone model, LD4LG, is capable of label-conditional generation for text data with specific attributes, the quality of edited samples generated by EdiText is closely tied to the performance of LD4LG. Second, our methodology involves transferring the generative capability of a pre-trained model to editing tasks without introducing an explicit objective for editing during training. This could potentially result in lower performance compared to methods explicitly trained for editing objectives.

Nevertheless, EdiText offers notable advantages. It is versatile, as it can be applied to any backbone text model that utilizes (1) diffusion in continuous embedding spaces and (2) self-conditioning techniques. Additionally, it operates in a training-free manner for the editing task itself. We recognize this dual nature—as both a limitation of our current study and a promising direction for future research.

\section*{Ethics Statement}

We propose EdiText, an editing framework designed to modify text based on user-specified attributes when provided with a reference text. EdiText aims to offer a highly versatile approach to text editing, such as removing harmful content or eliminating unnecessary nuances in whole sentences to enhance overall clarity and appropriateness. 
However, there are concerns about its potential misuse, including generating hateful content, jailbreaking human-aligned text models, or producing toxic responses.
