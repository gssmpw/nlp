\section{Results}
\label{sec:results}
We present the effectiveness of our proposed editing methods—coarse and fine-level editing—using various metrics in Section \ref{results:respective}. The results for the detoxifying and toxifying tasks are presented in Tables \ref{tab:table1} and \ref{tab:table2}, and the results for the two sentiment control tasks are shown in Table \ref{tab:table3}. The results from Table \ref{tab:table1} and Table \ref{tab:table2} are also plotted in Figure \ref{fig:curve}, which shows curves for each model in terms of Hamming Distance and Moderation Toxicity. In the upper panel, representing the detoxifying task, a lower reflection rate (i.e., a lower curve) is preferred at the same retention rate. Conversely, in the lower panel for the toxifying task, higher reflection rates are preferred. Section \ref{results:fused} further explores the integration of the two proposed editing methods. Additional results—including human evaluation, qualitative samples, and results on additional data—are presented in Appendix \ref{app:additional_results}.


\begin{table*}[ht]
\fontsize{8}{11}\selectfont
\centering
\begin{tabular}{l|ccc|ccc}
\toprule
\multirow{3}{*}{\textbf{Method}} & \multicolumn{3}{c|}{\textbf{Negative $\rightarrow$ Positive}}                   & \multicolumn{3}{c}{\textbf{Positive $\rightarrow$ Negative}}                    \\ \cmidrule{2-7} 
                                 & \multicolumn{2}{c|}{\textbf{Retention Rate}}               & \textbf{Sentiment} & \multicolumn{2}{c|}{\textbf{Retention Rate}}               & \textbf{Sentiment} \\ \cmidrule{2-7} 
                                 & \textbf{Ham $(\downarrow)$} & \multicolumn{1}{c|}{\textbf{BERTScore $(\uparrow)$}} & \textbf{Accuracy $(\uparrow)$}  & \textbf{Ham $(\downarrow)$} & \multicolumn{1}{c|}{\textbf{BERTScore $(\uparrow)$}} & \textbf{Accuracy $(\uparrow)$}  \\ \midrule
ParaGuide ($\lambda=200$)        & 13.962           & \multicolumn{1}{c|}{0.9026}             & 0.64               & 13.728           & \multicolumn{1}{c|}{0.8872}             & 0.60               \\
ParaGuide ($\lambda=1,000$)      & 15.820           & \multicolumn{1}{c|}{0.8813}             & 0.80               & 17.114           & \multicolumn{1}{c|}{0.8489}             & 0.69               \\
ParaGuide ($\lambda=10,000$)     & 17.966           & \multicolumn{1}{c|}{0.8565}             & 0.89               & 20.440           & \multicolumn{1}{c|}{0.8200}             & 0.73               \\ \midrule
Qwen2.5-0.5B-Instruct            & 23.900           & \multicolumn{1}{c|}{0.8812}             & 0.60               & 19.464           & \multicolumn{1}{c|}{0.8785}             & 0.72              \\
Qwen2.5-0.5B-Instruct (Amp.)     & 22.306           & \multicolumn{1}{c|}{0.8759}             & 0.65               & 18.592           & \multicolumn{1}{c|}{0.8789}             & 0.76               \\ \midrule
EdiText-CE ($t_{CE}=175$)        & 9.870            & \multicolumn{1}{c|}{0.9257}             & 0.56               & 9.992            & \multicolumn{1}{c|}{0.9264}             & 0.61               \\
EdiText-CE ($t_{CE}=200$)        & 15.094           & \multicolumn{1}{c|}{0.8788}             & 0.77               & 14.784           & \multicolumn{1}{c|}{0.8810}             & 0.79               \\
EdiText-CE ($t_{CE}=225$)        & 19.460           & \multicolumn{1}{c|}{0.8464}             & 0.90               & 18.848           & \multicolumn{1}{c|}{0.8434}             & 0.86               \\ \midrule
EdiText-FE ($t_{FE}=25$)         & 10.744           & \multicolumn{1}{c|}{0.9159}             & 0.60               & 11.636           & \multicolumn{1}{c|}{0.9004}             & 0.73               \\
EdiText-FE ($t_{FE}=75$)         & 11.758           & \multicolumn{1}{c|}{0.9064}             & 0.62               & 12.678           & \multicolumn{1}{c|}{0.8935}             & 0.73               \\
\bottomrule
\end{tabular}
\vskip -0.1in
\caption{Quantitative results of coarse- and fine-level sentiment editing on Enron data. \textbf{Ham} indicates Hamming Distance, and \textbf{Sentiment} indicates sentiment classification accuracy. $\lambda$ refers to guidance strength for ParaGuide.}
\label{tab:table3}
\vskip -0.2in
\end{table*}

\begin{figure}[!ht]
    \centering
    \includegraphics[width=1.0\linewidth]{figures/Detoxify_fused_Extended_noMnM_EdiText.pdf}
    \includegraphics[width=1.0\linewidth]{figures/Toxify_fused_Extended_noMnM_EdiText.pdf}
    \vskip -0.1in
    \caption{Hamming Distance - Moderation Toxicity curve for toxicity control tasks. The upper maps to detoxifying task while the lower maps to toxifying task.}
    \label{fig:curve}
    \vskip -0.25in
\end{figure}



\subsection{Model Comparison}
\label{results:respective}


As shown by the retention rate metrics in Table \ref{tab:table1}, \ref{tab:table2}, and \ref{tab:table3}, EdiText-CE can adjust the degree of text editing over a much wider range by modifying $t_{CE}$. In contrast, the NAR baseline, ParaGuide, exhibits minimal variation in both the retention rate and its corresponding reflection rate, even when its control parameter $\lambda$ is adjusted from $200$ to $10$K. Similarly, our AR baseline, Qwen2.5-0.5B-Instruct, shows limited responsiveness to control adjustments. When we attempt to amplify the editing degree by modifying the instruction to "as much as possible," the reflection rate rather deteriorates in toxicity-related tasks and only marginally increases in sentiment experiments. These findings indicate that the control range of other baselines is narrow compared to that of EdiText-CE. Figure \ref{fig:curve} further supports this observation by revealing the confined curves and limited coverage of the baselines.

Furthermore, EdiText-CE performs on par with or even outperforms other baselines across all reflection rate metrics related to toxicity and sentiment, particularly within the range where the text is edited to a similar extent as baselines in terms of retention rate. These results confirm that our coarse-level editing method not only offers a broader spectrum of control but also achieves superior editing performance within each range.

Additionally, unlike the coarse-level method, the proposed fine-level editing method—although it exhibits a narrower variation in retention and reflection rates compared to its coarse-level counterpart—provides much more granular control over the degree of text editing based on $t_{FE}$. The results for EdiText-FE indicate that even when using the fine-level technique alone, the desired attribute is better reflected while maintaining a retention rate similar to that of the baselines. Overall, these findings show that our two proposed techniques offer both coarse and fine control options, allowing users to tailor the approach to their needs, with each method individually achieving superior performance compared to baselines across various tasks.

\subsection{Integrated Method with EdiText}
\label{results:fused}

In the previous section, we showed that both variants of EdiText perform comparably to—or even better than—their baseline when used individually, though each exhibits distinct strengths and weaknesses. For instance, EdiText-CE offers a broad editing range, but its wide intervals can make it challenging to select a precise editing degree. In contrast, EdiText-FE has a narrower scope but allows for granular control. Motivated by these complementary characteristics, we explore integrating coarse- and fine-level editing—combining a wide editing range with precise modifications.

We apply integrated editing method to EdiText by setting $t_{CE}$ to either $200$ or $225$, while varying $t_{FE}$ in 10 equally spaced timesteps from $t_{CE}$ down to 0. To clearly observe the trade-off curve, we also extend the editing results of ParaGuide by adjusting the control factor $\lambda$ to values of $50$, $100$, $200$, $1$K, $10$K, $100$K and $1$M. Even with these expanded settings, we observe that ParaGuide's editing scope remains more limited than that of EdiText. 


Our integrated editing method, as shown in Figure \ref{fig:curve}, reveals two key insights. First, regarding the trade-off, our integrated method outperforms other baselines. In the detoxifying task, the curve for integrated EdiText consistently lies below those of the other baselines—even below EdiText-CE—demonstrating a superior trade-off. A similar trend is observed in the toxifying task, where a higher Moderation Toxicity is preferrable for a given Hamming Distance, and once again, our integrated EdiText exhibits this favorable behavior.

Furthermore, our method enhances the granularity of editing. In EdiText-CE, achieving text editing with a Hamming distance between 10 and 25—desirable in some cases—is rarely possible due to its wide intervals, as seen in Figure \ref{fig:curve}. In contrast, the integrated EdiText benefits from fine-level editing within the same range, enabling much more precise text modifications and thereby improving the overall completeness of its editing scope.