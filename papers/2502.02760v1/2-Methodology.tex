\section{Methodology}
\label{sec:methodology}
We performed a two-step methodology—filtering and investigation—to construct the Rust flaky test dataset.

\subsection{Filtering}
To accurately investigate the root causes of flakiness, we focus on \textit{already fixed} flaky tests. Our approach involves leveraging the GitHub REST API to search for issues in repositories using Rust, specifically querying for the term \textit{'flaky'}. The search was narrowed to only closed issues that have linked pull requests. Finally, we create a dataset containing all issues captured on October 29, 2024. 
This process resulted in a dataset of 1,146 issues potentially related to Rust flaky tests. 
% \yang{from my understanding, \hl{1,146} is the number of issues, then what's the number of tests?}\tom{Unknown. Only by reviewing an issue can we know the actual number of flaky tests.}

% \tom{This might be a good spot to mention that the results of the pull from GH were shuffled before serializing to the initial CSV file}\yang{mentioned below}

\subsection{Investigation}
Manually investigating over 1,000 issues is time-consuming. To ensure the diversity of tests in our study within a limited time, we shuffled the dataset and performed our analysis from start to finish. At the time of submission, we have manually investigated a subset of 53 tests from 49 projects of their root causes and fix strategies to conduct a deep analysis through the following process: 
We first reviewed the description provided in each issue, which may include the test failure scenario, the assertion failure, and the hypothesized source of the issue. We then reviewed the linked pull request to verify it as the fix for flakiness. Finally, we categorized the cause of flakiness and its fixing strategy. Note that we had to exclude certain cases because (1) some issues identified in our GitHub search using 'flaky' or its variations were actually not relevant to flaky tests, and (2) the issues pertained to parts of the repository not utilizing Rust, such as Python bindings or CI infrastructure.

% \yang{@Tom, how do you filter those N/A cases? Please elaborate on details of filtering out those flaky tests} 
% \tom{Captured general process and filtering of N/A below}
% We first reviewed the description provided in each issue, which may include the test failure scenario, the assertion failure, and the hypothesized source of the issue. We then reviewed its linked pull request to mark it as fixing the issue. Finally, we determined the category of the cause of the flakiness and the strategy for fixing it.

% In some cases, issues identified in the GitHub search query that we analyzed used the term 'flaky' or some variation but were not flaky tests. In other cases, the issue related to a portion of the repository that was not using Rust and instead was, for example, Python bindings or CI infrastructure. These issues were noted as 'not applicable' and filtered out of the final results for these cases.