\documentclass[10pt,conference]{IEEEtran}
\IEEEoverridecommandlockouts

\usepackage{cite}
\usepackage{amsmath,amssymb,amsfonts}
\usepackage{graphicx}
\usepackage{textcomp}
% \usepackage{xcolor}

\def\BibTeX{{\rm B\kern-.05em{\sc i\kern-.025em b}\kern-.08em
    T\kern-.1667em\lower.7ex\hbox{E}\kern-.125emX}}

\newcommand{\thought}[1]{{\color[rgb]{0.2,0.39,0.66}(#1)}}
\newcommand{\todo}[1]{{\color[rgb]{1.0,0.0,0.0}(#1)}}
\newcommand{\hsh}[1]{{\color{green!50!black} Henrik: #1}}
\newcommand{\st}[1]{{\color{red!50!black} Sebastian: #1}}

\newcommand{\ulm}[1]{_{\scaleto{\mathrm{#1}}{3pt}}}
\newcommand\at[2]{\left.#1\right|_{#2}}











\newtheorem{assumption}{Assumption}

\DeclareMathOperator*{\argmax}{arg\,max}
\DeclareMathOperator*{\argmin}{arg\,min}

\newcommand{\swname}[1]{\texttt{#1}}
\newcommand{\ie}{i\/.\/e\/.,\/~}
\newcommand{\eg}{e\/.\/g\/.,\/~}
\newcommand{\cf}{cf\/.\/~}

\newcommand{\fig}{Fig\/.\/~}
\newcommand{\defn}{Def\/.\/~}
\newcommand{\sect}{Sec\/.\/~}
\newcommand{\tabl}{Tab\/.\/~}
\newcommand{\algo}{Algorithm~}
\newcommand{\theo}{Theorem~}

\newcommand{\bnnl}{3 hidden layers}
\newcommand{\bnnn}{50 neurons}
\newcommand{\bnna}{tanh activations}

\newcommand{\capt}[1]{\mdseries{\emph{#1}}}

\newcommand{\videolink}{at \url{https://youtu.be/_d7AqTRjz6g}}
\newcommand{\codelink}{\url{https://github.com/wheelbot/mini-wheelbot}}

\newcommand{\fakepar}[1]{\vspace{0mm}\noindent\textbf{#1.}}

\newcommand{\needref}{\textcolor{red}{[REF]}}

\newcommand{\plotfontsize}{9pt}
    
\begin{document}

\title{A Preliminary Study of Fixed Flaky Tests in Rust Projects on GitHub}


\author{
\IEEEauthorblockN{Tom Schroeder}
\IEEEauthorblockA{\small{\textit{University of Illinois Urbana-Champaign}} \\
Urbana, IL, USA \\
jts13@illinois.edu}
\and
\IEEEauthorblockN{Minh Phan}
\IEEEauthorblockA{\small{\textit{University of Illinois Urbana-Champaign}} \\
Urbana, IL, USA \\
minhnp2@illinois.edu}
\and
\IEEEauthorblockN{Yang Chen}
\IEEEauthorblockA{\small{\textit{University of Illinois Urbana-Champaign}} \\
Urbana, IL, USA \\
yangc9@illinois.edu}
}

% \vspace{-100pt}

\maketitle

\begin{abstract}
Prior research has extensively studied flaky tests in various domains, such as web applications, mobile applications, and other open-source projects in a range of multiple programming languages, including Java, JavaScript, Python, Ruby, and more.
However, little attention has been given to flaky tests in Rust—an emerging popular language known for its safety features relative to C/C++. Rust incorporates interesting features that make it easy to detect some flaky tests, e.g., the Rust standard library randomizes the order of elements in hash tables, effectively exposing implementation-dependent flakiness.
However, Rust still has several sources of nondeterminism that can lead to flaky tests.

We present our work-in-progress on studying flaky tests in Rust projects on GitHub. Searching through the closed GitHub issues and pull requests, we identified 1,146 issues potentially related to Rust flaky tests.
We focus on flaky tests that are fixed, not just reported, as the fixes can offer valuable information on root causes, manifestation characteristics, and strategies of fixes.
By far, we have inspected 53 tests. Our initial findings indicate that the predominant root causes include asynchronous wait (33.9\%), concurrency issues (24.5\%), logic errors (9.4\%), and network-related problems (9.4\%). Our artifact is publicly available at ~\cite{artifact}.

% \yang{@Tom\&Minh please confirm: "...we have inspected 87 tests...": is it correct that "we inspected 87 tests, and after filtering out NA cases, we studied 53 tests for detailed root cause analysis (excluded one with "hard to classify")? }
% \tom{Sorry, yes, you're right. I was including all issues in that number.} \yang{should I change the above \hl{87} to \hl{53}?}\tom{Yes, 53.}
\end{abstract}

\begin{IEEEkeywords}
Flaky Tests, Rust
\end{IEEEkeywords}

\section{Introduction}

Chain-of-Thought (CoT) prompting~\cite{Nye:2021, cot, Kojima:2022cotzero} has emerged as a cornerstone strategy for enhancing Large Language Models (LLMs) in complex reasoning tasks. By eliciting step-by-step inference, CoT enables LLMs to decompose intricate problems into manageable subtasks, thereby improving their problem-solving performance~\cite{Yao:2023tot, Wang:2023self-consistency, Zhou:2023least, Shinn:2023Reflexion}. Recent advancements, such as OpenAI's o1~\cite{o1} and DeepSeek-R1~\cite{deepseekr1}, further demonstrate that scaling up CoT lengths from hundreds to thousands of reasoning steps could continuously improve LLM reasoning. These breakthroughs have underscored CoT’s potential to advance LLM capabilities, expanding the boundaries of AI-driven problem-solving.

\begin{figure}[t]
\centering
    \includegraphics[width=0.95\columnwidth]{fig/intro.pdf}
    \caption{In contrast to vanilla CoT that generates all reasoning tokens sequentially, \method enables LLMs to \textit{skip} tokens with less semantic importance (\textit{e.g.,} \includegraphics[width=7pt]{fig/token.pdf}~) and learn shortcuts between critical reasoning tokens, facilitating controllable CoT compression.}
    \label{fig:intro}
\end{figure}

Despite its effectiveness, the increased length of CoT sequences introduces substantial computational overhead. Due to the autoregressive nature of LLM decoding, longer CoT outputs lead to proportional increases in both inference latency and memory footprints of key-value cache. Additionally, the quadratic computational cost of attention layers further exacerbates this burden. These issues become particularly pronounced when CoT sequences extend into thousands of reasoning steps, resulting in significant computational costs and prolonged response times. While prior research has explored methods for selectively skipping reasoning steps~\cite{Ding:2024cotshortcut, liu2024skipstep}, recent findings~\cite{jin:2024cotlength, Merrill:2024cotlength} suggest that such reductions may conflict with test-time scaling~\cite{o1-blog, snell2025scaling}, ultimately impairing LLM reasoning performance. Therefore, striking an optimal balance between CoT efficiency and reasoning accuracy remains a critical open challenge.

In this work, we delve into CoT efficiency and seek the answer to an important question: \textit{``Does every token in the CoT output contribute equally to deriving the answer?''} We empirically analyze the semantic importance of tokens within CoT outputs and reveal that their contributions to the reasoning performance vary, as depicted in Figure 2. Building on this insight, we introduce \method, a simple yet effective approach that enables LLMs to \textit{skip} less important tokens within CoT sequences and learn shortcuts between critical reasoning tokens, thereby allowing for controllable CoT compression with adjustable ratios. Specifically, as shown in Figure~\ref{fig:intro}, \method constructs compressed CoT training data with various compression ratios, by pruning unimportance tokens from original LLM CoT trajectories. Then, it conducts a general supervised fine-tuning process on target LLMs with this training data, facilitating LLMs to automatically trim redundant tokens during reasoning.

We conduct extensive experiments across various models, including LLaMA-3.1-8B-Instruct and the Qwen2.5-Instruct series, using two widely recognized math reasoning benchmarks: GSM8K and MATH-500. The results validate the effectiveness of \method in compressing CoT outputs while maintaining robust reasoning performance. Notably, Qwen2.5-14B-Instruct exhibits almost \textbf{NO} performance drop (less than $0.4\%$) with a $\bm{40\%}$ reduction in token usage on GSM8K. On the challenging MATH-500 dataset, LLaMA-3.1-8B-Instruct effectively reduces CoT token usage by $\bm{30}\%$ with a performance decline of less than $4\%$, resulting in a $\bm{1.4}\times$ inference speedup. Further analysis underscores the coherence of \method in specified compression ratios and its potential scalability with stronger compression techniques.

\method is distinguished by its low training cost. For Qwen2.5-14B-Instruct, \method fine-tunes only 0.2\% of the model's parameters using LoRA. The size of the compressed CoT training data is no larger than that of the original training set, with 7,473 examples in GSM8K and 7,500 in MATH. The training is completed in approximately 2 hours for the 7B model and 2.5 hours for the 14B model on two 3090 GPUs. These characteristics make \method an efficient and reproducible approach, suitable for use in efficient and cost-effective LLM deployment.

To sum up, our key contributions are:
\begin{enumerate}
    \item To the best of our knowledge, this work is the \textit{first} to investigate the potential of enhancing CoT efficiency through \textit{token skipping}, inspired by the varying semantic importance of tokens in CoT trajectories of LLMs.
    \item We introduce \method, a simple yet effective approach that enables LLMs to skip redundant tokens within CoTs and learn shortcuts between critical tokens, facilitating CoT compression with adjustable ratios.
    \item Our experiments validate the effectiveness of \method. When applied to Qwen2.5-14B-Instruct, \method reduces reasoning tokens by $40\%$ (from 313 to 181) on GSM8K, with less than a $0.4\%$ performance drop.
\end{enumerate}

\section{Methodology}
\label{sec:methodology}

In this section, we provide an overview of our methodology. Our goal is to answer the following research questions (RQs):
\begin{itemize}
    \item \textbf{RQ1:} \emph{What are the main characteristics and concerns in the code reviews dataset?}
    \item \textbf{RQ2:} \emph{How can our data curation pipeline improve the overall quality of the dataset, specifically in terms of reducing noise and irrelevant reviews?}
    \item \textbf{RQ3:} \emph{What is the impact of dataset curation on the performance of LLMs for automated comment generation?}
    \item \textbf{RQ4:} \emph{What is the impact of dataset curation on the comments' usefulness and LLMs performance for automated code refinement?}
\end{itemize}

\begin{figure*}[!h]
    \centering
    \includegraphics[width=0.9\linewidth]{figures/methodology.pdf}
    \caption{Overview of our methodology. We use a large code review dataset of samples comprising pre-commit and post-commit codes along with review comments. For each sample, we use LLM-as-a-Judge with Llama-3.1-70B to generate a reformulated review comment, a categorization of the review, and a score for the original review comment. Next, we use the reformulated review comments to create our curated dataset, while filtering out irrelevant samples. Finally, we compare the effectiveness of LLMs fine-tuned on the original and curated datasets on two downstream tasks: comment generation and code refinement.}
    \label{fig:methodology}
    \vspace{-.5em}
\end{figure*}

\begin{comment}
\begin{figure*}[!htbp]
    \centering
    \includegraphics[width=\textwidth]{figures/methodology.png}
    \caption{Overview of the proposed methodology.}
    \label{fig:methodology}
\end{figure*}
\end{comment}

\Fig{fig:methodology} illustrates the workflow of the proposed methodology. 
The process begins with selecting a large code review dataset \cite{li2022automating}, which forms the basis for subsequent analyses.
Then, we define an evaluation framework (see \Fig{fig:eval_framework}) to classify review comments according to multiple categories (\ie type, nature, civility) and to score them based on various criteria (\ie relevance, clarity, conciseness). We conduct a quality assessment and characterization of the dataset, utilizing an LLM, \emph{Llama-3.1-70B}, as an automated judge using the defined evaluation framework (\emph{RQ1}).

\begin{figure}[!t]
    \centering
    \includegraphics[width=\linewidth]{figures/eval_framework.pdf}
    \caption{Overview of our evaluation framework.}
    % \vspace{-2em}
    \label{fig:eval_framework}
\end{figure}

Subsequently, we propose a curation pipeline aimed at refining the review comments. This step involves employing \emph{Llama-3.1-70B} to reformulate the comments based on some criteria to enhance their quality. A comparative analysis of both the original and curated datasets is then performed using the defined evaluation framework, focusing on classification categories and scoring criteria (\emph{RQ2}).


The next phase consists of fine-tuning two separate language models, each trained on either the original or curated dataset, to automate the task of comment generation. The performance of these models is evaluated and compared to determine which dataset better facilitates the learning process for generating high-quality review comments (\emph{RQ3}).


Finally, we assess the usefulness of the review comments for the subsequent task (\ie code refinement). We use two versions of the same language model, each provided with one version of the review comments (\ie original or curated), and compare the accuracy of the generated code. This comparison allows us to evaluate the impact of the curated versus original review comments on the correctness and effectiveness of automated code refinement (\emph{RQ4}).





\section{Analysis and Future Work}
\label{sec:analysis}

\subsection{Categories of Flakiness Root Causes}

Table~\ref{root-cause} presents the categories identified for the 53 tests through manual analysis in our study. We found that 33.9\% of these tests are affected by asynchronous waits. Among these, 38\% specifically are related to issues of \textit{improper wait}, which means tests do not properly wait for asynchronous operations to complete before moving on to assertions.
% \yang{is this correct?}
% \tom{It's basically just shorthand for 'missing synchronization point' - the code/test was missing some condition for waiting on some control/data dependency. The term 'wait' doesn't hold any official significance within Rust.}. 
Flakiness due to concurrency, logic, and network issues also appear frequently among the top causes. Additional root causes include I/O operations, randomness, time issues, unordered data and environment.

\begin{table}[htbp]
\centering
\scriptsize
\caption{Root causes of flakiness }
%\yang{reformat naming: consistent capital...do we need to rename the categories for better understanding?}
\begin{tabular}{@{}llll@{}}
\toprule
\textbf{Root Cause (Percent)} & \textbf{Total} & \textbf{Root Cause (Percent)} & \textbf{Total} \\ \midrule
\textbf{Async Wait (33.9\%)} & 18 & \textbf{Concurrency (24.5\%)} & 13 \\
\quad Improper Wait & 8 & \quad Robustness & 4 \\
\quad Tweak Duration & 2 & \quad Retry & 2 \\
\quad Sleep & 2 & \quad Lock & 1 \\
\quad Increase Timeout & 1 & \quad Channel & 1 \\
\quad Retry & 1 & \quad Race Condition & 1 \\
\quad Configuration & 1 & \quad Non-deterministic Environment & 1 \\
\quad Deadlock & 1 & \quad Reorder Initialization & 1 \\
\quad Channel & 1 & \quad Shared State & 1 \\
\quad Unknown State & 1 & \quad Atomics & 1 \\
% \quad Wait/Lock & 1 & & \\
\textbf{Logic (9.4\%)} & 5 & \textbf{Network (9.4\%)} & 5 \\
\quad Off by One & 2 & \quad Certs & 1 \\
\quad Inverted Conditional & 1 & \quad Connection Loss & 1 \\
\quad Deserialization & 1 & \quad Empty Header & 1 \\
\quad Sorting & 1 & \quad Retry & 1 \\
 & & \quad Reused Resource (port) & 1 \\
\textbf{I/O (5.7\%)} & 3 & \textbf{Randomness (5.7\%)} & 3 \\
\quad Flush & 1 & \quad Ranges & 1 \\
\quad Temp Files & 1 & \quad RNG & 1 \\
\quad Resource Limit & 1 & \quad Weighting & 1 \\
\textbf{Time (3.8\%)} & 2 & \textbf{Unordered data (5.7\%)} & 3 \\
\quad Inconsistent Clock & 1 & \quad Sorting & 2 \\
\quad Off by One & 1 & \quad Non-Deterministic Environment & 1 \\
\textbf{Environment (1.9\%)} & 1 & & \\
\quad Shared libraries & 1 & & \\
\bottomrule
\end{tabular}
\label{root-cause}
\end{table}


\begin{figure}
    \centering
    \includegraphics[scale=0.65]
    {Figures/fixfig.pdf}
    \vspace{-10pt}
    \caption{\small Categories of fixes. (a) shows the effectiveness of fixes per root cause category; (b) outlines the scope of changes for fixes per root cause category.}
    % \vspace{-5mm}
    \label{fig:fix}
    \vspace{-4pt}
\end{figure}

\subsection{Categories of Fix Strategies}
To investigate the fixes of flaky tests, we categorize them based on the \textit{effectiveness of fixes} by assessing whether each fix completely removes flakiness or only reduces the likelihood of failure (Figure~\ref{fig:fix}(a)). Additionally, we evaluate the \textit{scope of changes}, determining whether the fix modifies test code or main code (Figure~\ref{fig:fix}(b)).
Four tests affected by async waits and one test caused by time issues have not been completely fixed but only have a reduced likelihood of test failures. The potential reason could be that the actual root cause was not fully understood or that completely removing flakiness through a patch is programmatically challenging.
% \yang{any reason here?}\tom{The most common reason is that the dependency between two actors is not fully understood or not available programmatically, so something like a timeout duration always has the possibility of failing}. 
Additionally, of the 53 tests, 27 were fixed by modifying the main code to address common root causes such as async waits, concurrency, and logic. The remaining tests were patched within the test code. 
% \yang{any observation of differences between fixes to main code and test code?}\tom{We're not explicitly tracking this currently. In general, the fixes are typically similar.}


% \subsection{Future Work}
\section{Future Work}
\label{sec:futurework}
We present the first study of flaky tests in Rust from GitHub projects. In future work, we aim to continue our study of the dataset with 1,146 issues related to potential Rust flaky tests and explore methods for detecting and addressing flakiness in Rust. Additionally, we will investigate how language features influence flaky tests across different programming languages.


\bibliographystyle{IEEEtran}
\bibliography{reference}

\end{document}
