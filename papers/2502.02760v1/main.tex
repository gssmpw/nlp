\documentclass[10pt,conference]{IEEEtran}
\IEEEoverridecommandlockouts

\usepackage{cite}
\usepackage{amsmath,amssymb,amsfonts}
\usepackage{graphicx}
\usepackage{textcomp}
% \usepackage{xcolor}

\def\BibTeX{{\rm B\kern-.05em{\sc i\kern-.025em b}\kern-.08em
    T\kern-.1667em\lower.7ex\hbox{E}\kern-.125emX}}


%
\setlength\unitlength{1mm}
\newcommand{\twodots}{\mathinner {\ldotp \ldotp}}
% bb font symbols
\newcommand{\Rho}{\mathrm{P}}
\newcommand{\Tau}{\mathrm{T}}

\newfont{\bbb}{msbm10 scaled 700}
\newcommand{\CCC}{\mbox{\bbb C}}

\newfont{\bb}{msbm10 scaled 1100}
\newcommand{\CC}{\mbox{\bb C}}
\newcommand{\PP}{\mbox{\bb P}}
\newcommand{\RR}{\mbox{\bb R}}
\newcommand{\QQ}{\mbox{\bb Q}}
\newcommand{\ZZ}{\mbox{\bb Z}}
\newcommand{\FF}{\mbox{\bb F}}
\newcommand{\GG}{\mbox{\bb G}}
\newcommand{\EE}{\mbox{\bb E}}
\newcommand{\NN}{\mbox{\bb N}}
\newcommand{\KK}{\mbox{\bb K}}
\newcommand{\HH}{\mbox{\bb H}}
\newcommand{\SSS}{\mbox{\bb S}}
\newcommand{\UU}{\mbox{\bb U}}
\newcommand{\VV}{\mbox{\bb V}}


\newcommand{\yy}{\mathbbm{y}}
\newcommand{\xx}{\mathbbm{x}}
\newcommand{\zz}{\mathbbm{z}}
\newcommand{\sss}{\mathbbm{s}}
\newcommand{\rr}{\mathbbm{r}}
\newcommand{\pp}{\mathbbm{p}}
\newcommand{\qq}{\mathbbm{q}}
\newcommand{\ww}{\mathbbm{w}}
\newcommand{\hh}{\mathbbm{h}}
\newcommand{\vvv}{\mathbbm{v}}

% Vectors

\newcommand{\av}{{\bf a}}
\newcommand{\bv}{{\bf b}}
\newcommand{\cv}{{\bf c}}
\newcommand{\dv}{{\bf d}}
\newcommand{\ev}{{\bf e}}
\newcommand{\fv}{{\bf f}}
\newcommand{\gv}{{\bf g}}
\newcommand{\hv}{{\bf h}}
\newcommand{\iv}{{\bf i}}
\newcommand{\jv}{{\bf j}}
\newcommand{\kv}{{\bf k}}
\newcommand{\lv}{{\bf l}}
\newcommand{\mv}{{\bf m}}
\newcommand{\nv}{{\bf n}}
\newcommand{\ov}{{\bf o}}
\newcommand{\pv}{{\bf p}}
\newcommand{\qv}{{\bf q}}
\newcommand{\rv}{{\bf r}}
\newcommand{\sv}{{\bf s}}
\newcommand{\tv}{{\bf t}}
\newcommand{\uv}{{\bf u}}
\newcommand{\wv}{{\bf w}}
\newcommand{\vv}{{\bf v}}
\newcommand{\xv}{{\bf x}}
\newcommand{\yv}{{\bf y}}
\newcommand{\zv}{{\bf z}}
\newcommand{\zerov}{{\bf 0}}
\newcommand{\onev}{{\bf 1}}

% Matrices

\newcommand{\Am}{{\bf A}}
\newcommand{\Bm}{{\bf B}}
\newcommand{\Cm}{{\bf C}}
\newcommand{\Dm}{{\bf D}}
\newcommand{\Em}{{\bf E}}
\newcommand{\Fm}{{\bf F}}
\newcommand{\Gm}{{\bf G}}
\newcommand{\Hm}{{\bf H}}
\newcommand{\Id}{{\bf I}}
\newcommand{\Jm}{{\bf J}}
\newcommand{\Km}{{\bf K}}
\newcommand{\Lm}{{\bf L}}
\newcommand{\Mm}{{\bf M}}
\newcommand{\Nm}{{\bf N}}
\newcommand{\Om}{{\bf O}}
\newcommand{\Pm}{{\bf P}}
\newcommand{\Qm}{{\bf Q}}
\newcommand{\Rm}{{\bf R}}
\newcommand{\Sm}{{\bf S}}
\newcommand{\Tm}{{\bf T}}
\newcommand{\Um}{{\bf U}}
\newcommand{\Wm}{{\bf W}}
\newcommand{\Vm}{{\bf V}}
\newcommand{\Xm}{{\bf X}}
\newcommand{\Ym}{{\bf Y}}
\newcommand{\Zm}{{\bf Z}}

% Calligraphic

\newcommand{\Ac}{{\cal A}}
\newcommand{\Bc}{{\cal B}}
\newcommand{\Cc}{{\cal C}}
\newcommand{\Dc}{{\cal D}}
\newcommand{\Ec}{{\cal E}}
\newcommand{\Fc}{{\cal F}}
\newcommand{\Gc}{{\cal G}}
\newcommand{\Hc}{{\cal H}}
\newcommand{\Ic}{{\cal I}}
\newcommand{\Jc}{{\cal J}}
\newcommand{\Kc}{{\cal K}}
\newcommand{\Lc}{{\cal L}}
\newcommand{\Mc}{{\cal M}}
\newcommand{\Nc}{{\cal N}}
\newcommand{\nc}{{\cal n}}
\newcommand{\Oc}{{\cal O}}
\newcommand{\Pc}{{\cal P}}
\newcommand{\Qc}{{\cal Q}}
\newcommand{\Rc}{{\cal R}}
\newcommand{\Sc}{{\cal S}}
\newcommand{\Tc}{{\cal T}}
\newcommand{\Uc}{{\cal U}}
\newcommand{\Wc}{{\cal W}}
\newcommand{\Vc}{{\cal V}}
\newcommand{\Xc}{{\cal X}}
\newcommand{\Yc}{{\cal Y}}
\newcommand{\Zc}{{\cal Z}}

% Bold greek letters

\newcommand{\alphav}{\hbox{\boldmath$\alpha$}}
\newcommand{\betav}{\hbox{\boldmath$\beta$}}
\newcommand{\gammav}{\hbox{\boldmath$\gamma$}}
\newcommand{\deltav}{\hbox{\boldmath$\delta$}}
\newcommand{\etav}{\hbox{\boldmath$\eta$}}
\newcommand{\lambdav}{\hbox{\boldmath$\lambda$}}
\newcommand{\epsilonv}{\hbox{\boldmath$\epsilon$}}
\newcommand{\nuv}{\hbox{\boldmath$\nu$}}
\newcommand{\muv}{\hbox{\boldmath$\mu$}}
\newcommand{\zetav}{\hbox{\boldmath$\zeta$}}
\newcommand{\phiv}{\hbox{\boldmath$\phi$}}
\newcommand{\psiv}{\hbox{\boldmath$\psi$}}
\newcommand{\thetav}{\hbox{\boldmath$\theta$}}
\newcommand{\tauv}{\hbox{\boldmath$\tau$}}
\newcommand{\omegav}{\hbox{\boldmath$\omega$}}
\newcommand{\xiv}{\hbox{\boldmath$\xi$}}
\newcommand{\sigmav}{\hbox{\boldmath$\sigma$}}
\newcommand{\piv}{\hbox{\boldmath$\pi$}}
\newcommand{\rhov}{\hbox{\boldmath$\rho$}}
\newcommand{\upsilonv}{\hbox{\boldmath$\upsilon$}}

\newcommand{\Gammam}{\hbox{\boldmath$\Gamma$}}
\newcommand{\Lambdam}{\hbox{\boldmath$\Lambda$}}
\newcommand{\Deltam}{\hbox{\boldmath$\Delta$}}
\newcommand{\Sigmam}{\hbox{\boldmath$\Sigma$}}
\newcommand{\Phim}{\hbox{\boldmath$\Phi$}}
\newcommand{\Pim}{\hbox{\boldmath$\Pi$}}
\newcommand{\Psim}{\hbox{\boldmath$\Psi$}}
\newcommand{\Thetam}{\hbox{\boldmath$\Theta$}}
\newcommand{\Omegam}{\hbox{\boldmath$\Omega$}}
\newcommand{\Xim}{\hbox{\boldmath$\Xi$}}


% Sans Serif small case

\newcommand{\Gsf}{{\sf G}}

\newcommand{\asf}{{\sf a}}
\newcommand{\bsf}{{\sf b}}
\newcommand{\csf}{{\sf c}}
\newcommand{\dsf}{{\sf d}}
\newcommand{\esf}{{\sf e}}
\newcommand{\fsf}{{\sf f}}
\newcommand{\gsf}{{\sf g}}
\newcommand{\hsf}{{\sf h}}
\newcommand{\isf}{{\sf i}}
\newcommand{\jsf}{{\sf j}}
\newcommand{\ksf}{{\sf k}}
\newcommand{\lsf}{{\sf l}}
\newcommand{\msf}{{\sf m}}
\newcommand{\nsf}{{\sf n}}
\newcommand{\osf}{{\sf o}}
\newcommand{\psf}{{\sf p}}
\newcommand{\qsf}{{\sf q}}
\newcommand{\rsf}{{\sf r}}
\newcommand{\ssf}{{\sf s}}
\newcommand{\tsf}{{\sf t}}
\newcommand{\usf}{{\sf u}}
\newcommand{\wsf}{{\sf w}}
\newcommand{\vsf}{{\sf v}}
\newcommand{\xsf}{{\sf x}}
\newcommand{\ysf}{{\sf y}}
\newcommand{\zsf}{{\sf z}}


% mixed symbols

\newcommand{\sinc}{{\hbox{sinc}}}
\newcommand{\diag}{{\hbox{diag}}}
\renewcommand{\det}{{\hbox{det}}}
\newcommand{\trace}{{\hbox{tr}}}
\newcommand{\sign}{{\hbox{sign}}}
\renewcommand{\arg}{{\hbox{arg}}}
\newcommand{\var}{{\hbox{var}}}
\newcommand{\cov}{{\hbox{cov}}}
\newcommand{\Ei}{{\rm E}_{\rm i}}
\renewcommand{\Re}{{\rm Re}}
\renewcommand{\Im}{{\rm Im}}
\newcommand{\eqdef}{\stackrel{\Delta}{=}}
\newcommand{\defines}{{\,\,\stackrel{\scriptscriptstyle \bigtriangleup}{=}\,\,}}
\newcommand{\<}{\left\langle}
\renewcommand{\>}{\right\rangle}
\newcommand{\herm}{{\sf H}}
\newcommand{\trasp}{{\sf T}}
\newcommand{\transp}{{\sf T}}
\renewcommand{\vec}{{\rm vec}}
\newcommand{\Psf}{{\sf P}}
\newcommand{\SINR}{{\sf SINR}}
\newcommand{\SNR}{{\sf SNR}}
\newcommand{\MMSE}{{\sf MMSE}}
\newcommand{\REF}{{\RED [REF]}}

% Markov chain
\usepackage{stmaryrd} % for \mkv 
\newcommand{\mkv}{-\!\!\!\!\minuso\!\!\!\!-}

% Colors

\newcommand{\RED}{\color[rgb]{1.00,0.10,0.10}}
\newcommand{\BLUE}{\color[rgb]{0,0,0.90}}
\newcommand{\GREEN}{\color[rgb]{0,0.80,0.20}}

%%%%%%%%%%%%%%%%%%%%%%%%%%%%%%%%%%%%%%%%%%
\usepackage{hyperref}
\hypersetup{
    bookmarks=true,         % show bookmarks bar?
    unicode=false,          % non-Latin characters in AcrobatÕs bookmarks
    pdftoolbar=true,        % show AcrobatÕs toolbar?
    pdfmenubar=true,        % show AcrobatÕs menu?
    pdffitwindow=false,     % window fit to page when opened
    pdfstartview={FitH},    % fits the width of the page to the window
%    pdftitle={My title},    % title
%    pdfauthor={Author},     % author
%    pdfsubject={Subject},   % subject of the document
%    pdfcreator={Creator},   % creator of the document
%    pdfproducer={Producer}, % producer of the document
%    pdfkeywords={keyword1} {key2} {key3}, % list of keywords
    pdfnewwindow=true,      % links in new window
    colorlinks=true,       % false: boxed links; true: colored links
    linkcolor=red,          % color of internal links (change box color with linkbordercolor)
    citecolor=green,        % color of links to bibliography
    filecolor=blue,      % color of file links
    urlcolor=blue           % color of external links
}
%%%%%%%%%%%%%%%%%%%%%%%%%%%%%%%%%%%%%%%%%%%

    
\begin{document}

\title{A Preliminary Study of Fixed Flaky Tests in Rust Projects on GitHub}


\author{
\IEEEauthorblockN{Tom Schroeder}
\IEEEauthorblockA{\small{\textit{University of Illinois Urbana-Champaign}} \\
Urbana, IL, USA \\
jts13@illinois.edu}
\and
\IEEEauthorblockN{Minh Phan}
\IEEEauthorblockA{\small{\textit{University of Illinois Urbana-Champaign}} \\
Urbana, IL, USA \\
minhnp2@illinois.edu}
\and
\IEEEauthorblockN{Yang Chen}
\IEEEauthorblockA{\small{\textit{University of Illinois Urbana-Champaign}} \\
Urbana, IL, USA \\
yangc9@illinois.edu}
}

% \vspace{-100pt}

\maketitle

\begin{abstract}
Prior research has extensively studied flaky tests in various domains, such as web applications, mobile applications, and other open-source projects in a range of multiple programming languages, including Java, JavaScript, Python, Ruby, and more.
However, little attention has been given to flaky tests in Rust—an emerging popular language known for its safety features relative to C/C++. Rust incorporates interesting features that make it easy to detect some flaky tests, e.g., the Rust standard library randomizes the order of elements in hash tables, effectively exposing implementation-dependent flakiness.
However, Rust still has several sources of nondeterminism that can lead to flaky tests.

We present our work-in-progress on studying flaky tests in Rust projects on GitHub. Searching through the closed GitHub issues and pull requests, we identified 1,146 issues potentially related to Rust flaky tests.
We focus on flaky tests that are fixed, not just reported, as the fixes can offer valuable information on root causes, manifestation characteristics, and strategies of fixes.
By far, we have inspected 53 tests. Our initial findings indicate that the predominant root causes include asynchronous wait (33.9\%), concurrency issues (24.5\%), logic errors (9.4\%), and network-related problems (9.4\%). Our artifact is publicly available at ~\cite{artifact}.

% \yang{@Tom\&Minh please confirm: "...we have inspected 87 tests...": is it correct that "we inspected 87 tests, and after filtering out NA cases, we studied 53 tests for detailed root cause analysis (excluded one with "hard to classify")? }
% \tom{Sorry, yes, you're right. I was including all issues in that number.} \yang{should I change the above \hl{87} to \hl{53}?}\tom{Yes, 53.}
\end{abstract}

\begin{IEEEkeywords}
Flaky Tests, Rust
\end{IEEEkeywords}

\section{Introduction}

Deep Reinforcement Learning (DRL) has emerged as a transformative paradigm for solving complex sequential decision-making problems. By enabling autonomous agents to interact with an environment, receive feedback in the form of rewards, and iteratively refine their policies, DRL has demonstrated remarkable success across a diverse range of domains including games (\eg Atari~\citep{mnih2013playing,kaiser2020model}, Go~\citep{silver2018general,silver2017mastering}, and StarCraft II~\citep{vinyals2019grandmaster,vinyals2017starcraft}), robotics~\citep{kalashnikov2018scalable}, communication networks~\citep{feriani2021single}, and finance~\citep{liu2024dynamic}. These successes underscore DRL's capability to surpass traditional rule-based systems, particularly in high-dimensional and dynamically evolving environments.

Despite these advances, a fundamental challenge remains: DRL agents typically rely on deep neural networks, which operate as black-box models, obscuring the rationale behind their decision-making processes. This opacity poses significant barriers to adoption in safety-critical and high-stakes applications, where interpretability is crucial for trust, compliance, and debugging. The lack of transparency in DRL can lead to unreliable decision-making, rendering it unsuitable for domains where explainability is a prerequisite, such as healthcare, autonomous driving, and financial risk assessment.

To address these concerns, the field of Explainable Deep Reinforcement Learning (XRL) has emerged, aiming to develop techniques that enhance the interpretability of DRL policies. XRL seeks to provide insights into an agent’s decision-making process, enabling researchers, practitioners, and end-users to understand, validate, and refine learned policies. By facilitating greater transparency, XRL contributes to the development of safer, more robust, and ethically aligned AI systems.

Furthermore, the increasing integration of Reinforcement Learning (RL) with Large Language Models (LLMs) has placed RL at the forefront of natural language processing (NLP) advancements. Methods such as Reinforcement Learning from Human Feedback (RLHF)~\citep{bai2022training,ouyang2022training} have become essential for aligning LLM outputs with human preferences and ethical guidelines. By treating language generation as a sequential decision-making process, RL-based fine-tuning enables LLMs to optimize for attributes such as factual accuracy, coherence, and user satisfaction, surpassing conventional supervised learning techniques. However, the application of RL in LLM alignment further amplifies the explainability challenge, as the complex interactions between RL updates and neural representations remain poorly understood.

This survey provides a systematic review of explainability methods in DRL, with a particular focus on their integration with LLMs and human-in-the-loop systems. We first introduce fundamental RL concepts and highlight key advances in DRL. We then categorize and analyze existing explanation techniques, encompassing feature-level, state-level, dataset-level, and model-level approaches. Additionally, we discuss methods for evaluating XRL techniques, considering both qualitative and quantitative assessment criteria. Finally, we explore real-world applications of XRL, including policy refinement, adversarial attack mitigation, and emerging challenges in ensuring interpretability in modern AI systems. Through this survey, we aim to provide a comprehensive perspective on the current state of XRL and outline future research directions to advance the development of interpretable and trustworthy DRL models.
\section{Methodology}
\label{sec:methodology}

In this section, we provide an overview of our methodology. Our goal is to answer the following research questions (RQs):
\begin{itemize}
    \item \textbf{RQ1:} \emph{What are the main characteristics and concerns in the code reviews dataset?}
    \item \textbf{RQ2:} \emph{How can our data curation pipeline improve the overall quality of the dataset, specifically in terms of reducing noise and irrelevant reviews?}
    \item \textbf{RQ3:} \emph{What is the impact of dataset curation on the performance of LLMs for automated comment generation?}
    \item \textbf{RQ4:} \emph{What is the impact of dataset curation on the comments' usefulness and LLMs performance for automated code refinement?}
\end{itemize}

\begin{figure*}[!h]
    \centering
    \includegraphics[width=0.9\linewidth]{figures/methodology.pdf}
    \caption{Overview of our methodology. We use a large code review dataset of samples comprising pre-commit and post-commit codes along with review comments. For each sample, we use LLM-as-a-Judge with Llama-3.1-70B to generate a reformulated review comment, a categorization of the review, and a score for the original review comment. Next, we use the reformulated review comments to create our curated dataset, while filtering out irrelevant samples. Finally, we compare the effectiveness of LLMs fine-tuned on the original and curated datasets on two downstream tasks: comment generation and code refinement.}
    \label{fig:methodology}
    \vspace{-.5em}
\end{figure*}

\begin{comment}
\begin{figure*}[!htbp]
    \centering
    \includegraphics[width=\textwidth]{figures/methodology.png}
    \caption{Overview of the proposed methodology.}
    \label{fig:methodology}
\end{figure*}
\end{comment}

\Fig{fig:methodology} illustrates the workflow of the proposed methodology. 
The process begins with selecting a large code review dataset \cite{li2022automating}, which forms the basis for subsequent analyses.
Then, we define an evaluation framework (see \Fig{fig:eval_framework}) to classify review comments according to multiple categories (\ie type, nature, civility) and to score them based on various criteria (\ie relevance, clarity, conciseness). We conduct a quality assessment and characterization of the dataset, utilizing an LLM, \emph{Llama-3.1-70B}, as an automated judge using the defined evaluation framework (\emph{RQ1}).

\begin{figure}[!t]
    \centering
    \includegraphics[width=\linewidth]{figures/eval_framework.pdf}
    \caption{Overview of our evaluation framework.}
    % \vspace{-2em}
    \label{fig:eval_framework}
\end{figure}

Subsequently, we propose a curation pipeline aimed at refining the review comments. This step involves employing \emph{Llama-3.1-70B} to reformulate the comments based on some criteria to enhance their quality. A comparative analysis of both the original and curated datasets is then performed using the defined evaluation framework, focusing on classification categories and scoring criteria (\emph{RQ2}).


The next phase consists of fine-tuning two separate language models, each trained on either the original or curated dataset, to automate the task of comment generation. The performance of these models is evaluated and compared to determine which dataset better facilitates the learning process for generating high-quality review comments (\emph{RQ3}).


Finally, we assess the usefulness of the review comments for the subsequent task (\ie code refinement). We use two versions of the same language model, each provided with one version of the review comments (\ie original or curated), and compare the accuracy of the generated code. This comparison allows us to evaluate the impact of the curated versus original review comments on the correctness and effectiveness of automated code refinement (\emph{RQ4}).





\section{Analysis and Future Work}
\label{sec:analysis}

\subsection{Categories of Flakiness Root Causes}

Table~\ref{root-cause} presents the categories identified for the 53 tests through manual analysis in our study. We found that 33.9\% of these tests are affected by asynchronous waits. Among these, 38\% specifically are related to issues of \textit{improper wait}, which means tests do not properly wait for asynchronous operations to complete before moving on to assertions.
% \yang{is this correct?}
% \tom{It's basically just shorthand for 'missing synchronization point' - the code/test was missing some condition for waiting on some control/data dependency. The term 'wait' doesn't hold any official significance within Rust.}. 
Flakiness due to concurrency, logic, and network issues also appear frequently among the top causes. Additional root causes include I/O operations, randomness, time issues, unordered data and environment.

\begin{table}[htbp]
\centering
\scriptsize
\caption{Root causes of flakiness }
%\yang{reformat naming: consistent capital...do we need to rename the categories for better understanding?}
\begin{tabular}{@{}llll@{}}
\toprule
\textbf{Root Cause (Percent)} & \textbf{Total} & \textbf{Root Cause (Percent)} & \textbf{Total} \\ \midrule
\textbf{Async Wait (33.9\%)} & 18 & \textbf{Concurrency (24.5\%)} & 13 \\
\quad Improper Wait & 8 & \quad Robustness & 4 \\
\quad Tweak Duration & 2 & \quad Retry & 2 \\
\quad Sleep & 2 & \quad Lock & 1 \\
\quad Increase Timeout & 1 & \quad Channel & 1 \\
\quad Retry & 1 & \quad Race Condition & 1 \\
\quad Configuration & 1 & \quad Non-deterministic Environment & 1 \\
\quad Deadlock & 1 & \quad Reorder Initialization & 1 \\
\quad Channel & 1 & \quad Shared State & 1 \\
\quad Unknown State & 1 & \quad Atomics & 1 \\
% \quad Wait/Lock & 1 & & \\
\textbf{Logic (9.4\%)} & 5 & \textbf{Network (9.4\%)} & 5 \\
\quad Off by One & 2 & \quad Certs & 1 \\
\quad Inverted Conditional & 1 & \quad Connection Loss & 1 \\
\quad Deserialization & 1 & \quad Empty Header & 1 \\
\quad Sorting & 1 & \quad Retry & 1 \\
 & & \quad Reused Resource (port) & 1 \\
\textbf{I/O (5.7\%)} & 3 & \textbf{Randomness (5.7\%)} & 3 \\
\quad Flush & 1 & \quad Ranges & 1 \\
\quad Temp Files & 1 & \quad RNG & 1 \\
\quad Resource Limit & 1 & \quad Weighting & 1 \\
\textbf{Time (3.8\%)} & 2 & \textbf{Unordered data (5.7\%)} & 3 \\
\quad Inconsistent Clock & 1 & \quad Sorting & 2 \\
\quad Off by One & 1 & \quad Non-Deterministic Environment & 1 \\
\textbf{Environment (1.9\%)} & 1 & & \\
\quad Shared libraries & 1 & & \\
\bottomrule
\end{tabular}
\label{root-cause}
\end{table}


\begin{figure}
    \centering
    \includegraphics[scale=0.65]
    {Figures/fixfig.pdf}
    \vspace{-10pt}
    \caption{\small Categories of fixes. (a) shows the effectiveness of fixes per root cause category; (b) outlines the scope of changes for fixes per root cause category.}
    % \vspace{-5mm}
    \label{fig:fix}
    \vspace{-4pt}
\end{figure}

\subsection{Categories of Fix Strategies}
To investigate the fixes of flaky tests, we categorize them based on the \textit{effectiveness of fixes} by assessing whether each fix completely removes flakiness or only reduces the likelihood of failure (Figure~\ref{fig:fix}(a)). Additionally, we evaluate the \textit{scope of changes}, determining whether the fix modifies test code or main code (Figure~\ref{fig:fix}(b)).
Four tests affected by async waits and one test caused by time issues have not been completely fixed but only have a reduced likelihood of test failures. The potential reason could be that the actual root cause was not fully understood or that completely removing flakiness through a patch is programmatically challenging.
% \yang{any reason here?}\tom{The most common reason is that the dependency between two actors is not fully understood or not available programmatically, so something like a timeout duration always has the possibility of failing}. 
Additionally, of the 53 tests, 27 were fixed by modifying the main code to address common root causes such as async waits, concurrency, and logic. The remaining tests were patched within the test code. 
% \yang{any observation of differences between fixes to main code and test code?}\tom{We're not explicitly tracking this currently. In general, the fixes are typically similar.}


% \subsection{Future Work}
\section{Future Work}
\label{sec:futurework}
We present the first study of flaky tests in Rust from GitHub projects. In future work, we aim to continue our study of the dataset with 1,146 issues related to potential Rust flaky tests and explore methods for detecting and addressing flakiness in Rust. Additionally, we will investigate how language features influence flaky tests across different programming languages.


\bibliographystyle{IEEEtran}
\bibliography{reference}

\end{document}
