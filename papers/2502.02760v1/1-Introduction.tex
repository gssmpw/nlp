\section{Introduction}
\label{sec:introduction}
Regression testing is crucial to ensure software reliability by detecting potential bugs after code changes.
However, flaky tests can non-deterministically pass or fail when run on the same code version, significantly impacting the quality of regression test suites~\cite{luo2014empirical}. 
Common flaky tests can be categorized into order-dependent (OD) and non-order-dependent (NOD) tests.
OD tests can be flaky due to dependencies within the test suite—passing if run before specific tests and failing if run after.
NOD tests, on the other hand, may become flaky for reasons other than order dependencies, such as concurrency, asynchronous waits, network, and I/O operations.  
Implementation-dependent (ID) tests are a subcategory of NOD caused by wrong assumptions of unordered collections in the implementation of tests.

Since the seminal study of flaky tests in Apache projects done by Luo et al. in 2014~\cite{luo2014empirical}, several studies have been proposed for characterizing~\cite{thorve2018empirical, gruber2021empirical, gruber2024do, owain2021survey,lam2019root,lam2020study,lam2020large,lam2020understanding,chen2023transforming, hashemi2022empirical, barbosa2022test}, detecting
%~\cite{ziftci2020flake,wang2022ipflakies,person2015test,mascheroni2018identifying,king2018towards,pinto2020vocabulary,bell2018deflaker,lam2019idflakies,wei2022preempting,verdecchia2021know,yi2021finding, dutta2020detecting}, 
and fixing
% ~\cite{shi2019ifixflakies, wang2022ipflakies,li2022repairing, zhang2021domain, chen2024neurosymbolic} 
flaky tests across various programming languages and platforms, including Android, Java, JavaScript, Python, and Ruby. However, no prior work has specifically focused on flaky tests in Rust projects. Given Rust's emerging popularity and its interesting features that could introduce non-determinism, investigating flakiness in Rust projects is a crucial area for future research.

In this study, we investigate test flakiness in Rust projects. Starting by constructing a dataset of flaky tests from Rust projects on GitHub, we identified 1,146 issues potentially related to Rust flaky tests.
% we identified \hl{X} fixed flaky tests from \hl{X} GitHub issues.
By further inspecting the characteristics of 53 tests, we identified nine common root causes of test flakiness. To our knowledge, we presented the first study to investigate flaky tests in Rust on Github projects. Our findings offer valuable insights that could guide future research.
