\usepackage{tabularray}
\usepackage[noend]{algpseudocode}
\usepackage{multirow}
\usepackage{xurl}
\usepackage{booktabs} % For formal tables
\usepackage{amsmath}
\usepackage{graphicx}
\usepackage[belowskip=-12pt]{caption}
\usepackage{subcaption}
\usepackage{rotating}
\usepackage{tablefootnote}
\usepackage{adjustbox}
\usepackage{array}
 \usepackage{float}
\usepackage{tabularx}
% \usepackage[dvipsnames]{xcolor}
\usepackage{listings}
\lstset{escapeinside={<@}{@>}}
\usepackage{color, colortbl}
\usepackage{balance} % to balance the bibliography
\usepackage{lscape}
\usepackage{xspace}
% \usepackage[table,x11names,dvipsnames]{xcolor}
% \usepackage[dvipsnames]{xcolor}
\usepackage{framed}
\usepackage{syntax}
\usepackage{parcolumns}
\usepackage{pifont}
\usepackage{soul}
\usepackage[tikz]{bclogo}
\usepackage{enumitem}
\usepackage{soul}
\usepackage{makecell}
\usepackage[many]{tcolorbox}
\usepackage{hyperref}
\usepackage{cleveref}
% \usepackage{emoji}
\usepackage{tikzsymbols}
\usepackage{textcomp}
% \usepackage{parskip}
\usepackage{wrapfig}
\usepackage{tikz}
% \newcommand{\circled}[1]{\tikz[baseline=(char.base)]{
%             \node[shape=circle,draw,inner sep=1pt] (char) {#1};}}
% \newcommand{\circled}[1]{\tikz[baseline=(char.base)]{
%             \node[shape=circle,draw,inner sep=1pt] (char) {#2};}}
% \newcommand{\circled}[1]{\tikz[baseline=(char.base)]{
%             \node[shape=circle,draw,inner sep=1pt] (char) {#3};}}
% \newcommand{\circled}[1]{\tikz[baseline=(char.base)]{
%             \node[shape=circle,draw,inner sep=1pt] (char) {#4};}}
% \newcommand{\circled}[1]{\tikz[baseline=(char.base)]{
%             \node[shape=circle,draw,inner sep=1pt] (char) {#5};}}
% \newcommand{\circled}[1]{\tikz[baseline=(char.base)]{
%             \node[shape=circle,draw,inner sep=1pt] (char) {#6};}}
% \newcommand{\circled}[1]{\tikz[baseline=(char.base)]{
%             \node[shape=circle,draw,inner sep=1pt] (char) {#7};}}            
% \usepackage[hidelinks]{hyperref}

\crefformat{section}{\S#2#1#3} % see manual of cleveref, section 8.2.1
\crefformat{subsection}{\S#2#1#3}
\crefformat{subsubsection}{\S#2#1#3}

\definecolor{maroon}{cmyk}{0,0.87,0.68,0.00}
\definecolor{mygreen}{cmyk}{0.41,0.1,0.5,0}

\newcommand*\rotninty{\rotatebox{90}}
\newcolumntype{R}[2]{%
    >{\adjustbox{angle=#1,lap=\width-(#2)}\bgroup}%
    l%
    <{\egroup}%
}
\newcommand*\rot{\multicolumn{1}{R{30}{1em}}}% no optional argument here, please!

%%% Packages for Algorithms %%%
\usepackage[linesnumbered,ruled,vlined]{algorithm2e}
\SetKwInput{KwInputs}{Inputs}                % Set the Input
\SetKwInput{KwOutputs}{Outputs}              % set the Output
\SetKwInput{KwOutput}{Output}              % set the Output


% \setcopyright{acmcopyright}
% \copyrightyear{2023}
% \acmYear{2023}
% \acmDOI{XXXXXXX.XXXXXXX}

%% These commands are for a PROCEEDINGS abstract or paper.
% \acmConference[FSE 2024]{International Conference on the Foundations of Software Engineering}{July 2024}{Porto de Galinhas, Brazil}

% \settopmatter{printacmref=false, printccs=false, printfolios=false}
% \setcopyright{none} 
% \renewcommand\footnotetextcopyrightpermission[1]{}

%
%  Uncomment \acmBooktitle if th title of the proceedings is different
%  from ``Proceedings of ...''!
%
%\acmBooktitle{Woodstock '18: ACM Symposium on Neural Gaze Detection,
%  June 03--05, 2018, Woodstock, NY} 
% \acmPrice{15.00}

\definecolor{patchgreen}{HTML}{ccffcc}
\definecolor{patchred}{HTML}{ffcccc}
\definecolor{myyellow}{HTML}{FFFFCC}
\definecolor{mygray}{HTML}{F0F0F0}

\newcommand{\addgreen}{\makebox[0pt][l]{\color{patchgreen}\rule[-1ex]{\linewidth}{3ex}}}
\newcommand{\addred}{\makebox[0pt][l]{\color{patchred}\rule[-1ex]{\linewidth}{3ex}}}

\newcommand{\yang}[1]{\textcolor{blue}{[Yang: #1]}}
\newcommand{\tom}[1]{\textcolor{purple}{[Tom: #1]}}
\newcommand{\darko}[1]{\textcolor{red}{[Darko: #1]}}
\newcommand\printpercent[2]{\the\numexpr#1*100/#2\%}
\newcommand{\mybox}[1]{\begin{tcolorbox}[enhanced, frame hidden, boxsep=0pt]\emph{#1}\end{tcolorbox}}
\preto\tabular{\setcounter{rownumbers}{0}}
\newcounter{rownumbers}
\newcommand\rownumber{\stepcounter{rownumbers}\arabic{rownumbers}}
\newcommand{\ifixflakies}{iFixFlakies\xspace}
\newcommand{\ipflakies}{iPFlakies\xspace}
\newcommand{\idflakies}{iDFlakies\xspace}
\newcommand{\odrepair}{ODRepair\xspace}
\newcommand{\idoft}{IDoFT\xspace}
\newcommand{\nondex}{NonDex\xspace}
\newcommand{\dexfix}{DexFix\xspace}
\newcommand{\inspector}{\textit{Inspector}\xspace}
\newcommand{\promptgeneration}{\textit{Prompt Generator}\xspace}
\newcommand{\repair}{\textit{Tailor}\xspace}
\newcommand{\validation}{\textit{Validator}\xspace}
\newcommand{\feedback}{\textit{Feedback Loop}\xspace}
\newcommand{\stitch}{\textit{Stitching}\xspace}
\newcommand{\gpt}{GPT-4\xspace}
\newcommand{\magic}{Magicoder\xspace}

\definecolor{problemblue}{RGB}{100,134,158}
\definecolor{idiomsgreen}{RGB}{0,162,0}
\definecolor{exercisebgblue}{rgb}{0,  .69,  .941}
\definecolor{deepgreen}{rgb}{0.0, 0.5, 0.0}
\definecolor{codegreen}{rgb}{0,0.6,0}
\definecolor{codegray}{rgb}{0.5,0.5,0.5}
\definecolor{codepurple}{rgb}{0.58,0,0.82}
\definecolor{backcolour}{rgb}{0.95,0.95,0.92}
\definecolor{redColor}{RGB}{255,0,0}
\definecolor{Gray}{gray}{0.1}
\definecolor{CadetBlue}{RGB}{243.0,42.1,134.0}


\newtcbtheorem[]{Theorem}{}{
        enhanced,
        sharp corners,
        colback=white,
        colframe=exercisebgblue
    }{thm}
    
\newenvironment{findingBox1}[2]{ \begin{Theorem}{}{}{#2}}{\end{Theorem}}

\lstdefinestyle{code}{
  backgroundcolor=\color{gray!4},
  commentstyle=\color{codegray},
  keywordstyle=\color{codepurple},
  numberstyle=\tiny\color{codegray},
  stringstyle=\color{codegray},
  basicstyle=\ttfamily\footnotesize,
  breakatwhitespace=false,         
  breaklines=true,                 
  captionpos=b,                    
  keepspaces=true,                 
  numbers=left,                    
  numbersep=5pt,                  
  showspaces=false,                
  showstringspaces=false,
  showtabs=false,                  
  tabsize=2,
  % escapeinside=||
}
\lstset{style=code}

\lstdefinelanguage{test}{%
	language     = python,
	breaklines = true,backgroundcolor=\color{white},escapechar=!,rulecolor=\color{black}, breaklines=true,sensitive=true,  numbersep=5pt, xleftmargin=.015\textwidth, frame=tb,label=test
}
\lstdefinelanguage{source}{%
	language     = python,
	breaklines = true,
firstnumber=0,numberfirstline=false,columns=fullflexible,numbers=left,backgroundcolor=\color{white},
    rulecolor=\color{black}, 
    breaklines=true,sensitive=true, numbersep=5pt, xleftmargin=.015\textwidth, label=test
}
\newcommand*\Suppressnumber{%
  \lst@AddToHook{OnNewLine}{%
    \let\thelstnumber\relax%
     \advance\c@lstnumber-\@ne\relax%
    }%
}

\newcommand*\Reactivatenumber{%
  \lst@AddToHook{OnNewLine}{%
   \let\thelstnumber\origthelstnumber%
   \advance\c@lstnumber\@ne\relax}%
}
\newcommand*{\Perm}[2]{{}^{#1}\!P_{#2}}%

\definecolor{diffstart}{named}{codegreen}
\definecolor{diffincl}{named}{redColor}
%\definecolor{diffrem}{named}{Orange}

\newcommand{\centered}[1]{\begin{tabular}{l} #1 \end{tabular}}

% \lstdefinestyle{customc}{
% 	belowcaptionskip=1\baselineskip,
% 	breaklines=false,
% 	frame= single,
% 	%numbers = left,
% 	breaklines = true,
% 	xleftmargin=\parindent,
% 	language= Python,
% 	showstringspaces=false,
% 	basicstyle=\footnotesize\ttfamily,
% 	keywordstyle=\bfseries\color{green!40!black},
% 	commentstyle=\itshape\color{purple!40!black},
% 	identifierstyle=\color{blue},
% 	stringstyle=\color{codegreen},
% 	backgroundcolor=\color{gray!4}
% }

\DeclareCaptionFormat{listing}{#3}
\captionsetup[lstlisting]{format=listing}
% \lstset{style=customc}
\newcommand{\change}[1]{{\color{blue!80!white}\textbf{}~#1}\xspace}
%\usepackage{pdflscape}
\newcommand{\centerhfill}[1][\quad]{\hspace{\stretch{0.5}}#1\hspace{\stretch{0.5}}}
\graphicspath{{./figures/}}

\newcounter{NumObservations}
\stepcounter{NumObservations}
\definecolor{shadecolor}{rgb}{.9,.9,.9}

\newcommand{\findingold}[1]{%
	\begin{mdframed}[
		backgroundcolor= gray!10,
		linewidth=.2pt,
		linecolor = gray!110,
		roundcorner=2pt,
		innertopmargin=2pt,
		innerbottommargin=0pt,
		innerrightmargin=4pt,
		innerleftmargin=2pt,
		leftmargin = 0pt,
		rightmargin = 0pt]%
		\noindent{\textbf{Finding \arabic{NumObservations}}: \em #1} \\
	\end{mdframed}%
	\stepcounter{NumObservations}
}

\newcommand{\findingsold}[1]{
	
	\begin{shaded}
		\vspace{-.85em}
		%\hspace{1.5em}
		\textit{{\bf{Finding \arabic{NumObservations}}}: #1.}\\
		\vspace{-1.75em}
	\end{shaded}
	\stepcounter{NumObservations}	
}

\definecolor{msftBlue}{RGB}{0,164,239}
\definecolor{msftGreen}{RGB}{127,186,0}
\definecolor{msftYello}{RGB}{255,185,0}
\usepackage{bigstrut}
\newenvironment{findingBox}[2]{%
	\begin{tcolorbox}[colframe=black!90,colback=exercisebgblue!10,boxrule=.75pt, left=4pt, right = 4pt, top=1pt, bottom=1pt]{\textbf{Finding #1:} #2} %~\itshape %
}{%
	\end{tcolorbox}
}


%%% Annotations
\newcommand{\nbc}[3]{
	{\colorbox{#3}{\bfseries\sffamily\scriptsize\textcolor{white}{#1}}}
	{\textcolor{#3}{\sf\small$\blacktriangleright$\textit{#2}$\blacktriangleleft$}}
}
\definecolor{vlcolor}{rgb}{0.9,0.1,0.1}
\newcommand\vule[1]{\nbc{VL}{#1}{vlcolor}}




