\section{Introduction}
\label{sec:intro}
 \textit{Pose Estimation Systems} has applications in medicine~\cite{LITJENS201760}, sign language recognition~\cite{holmes2023scarcity} and high stakes sports events~\cite{martin2021automated},  requiring them to be well-tested in order to provide a substantiated assessment of the correct behaviour when applied to sensitive domains. Such systems need to perform correctly and as expected under a variety of conditions. This work aims to provide practitioners with a means of assessing this.

In this work we propose \proposed, a metamorphic testing framework to test pose estimation systems without the significant cost of manual data labelling. Additionally, we propose a non exhaustive set of metamorphic rules for \proposed, including flexible metrics to assess violations. These rules allow practitioners to apply various commonly encountered image changes and can easily be extended by users, should they want to explore different aspects of the system.

We apply \proposed to Mediapipe Hollistic~\cite{researchMediaPipeHolistic}, a widely used state-of-the-art pose estimation system, on datasets from the literature used to train and assess human pose estimation systems in different domains. We show that our proposed framework can find numerous faulty outputs from the system. Results show that \proposed provides results on par with classic, human annotated ground truth-based testing on the \FLIC dataset. Furthermore, we illustrate how analysis of the results can highlight elements of the input that impact the system's performance. Such analysis can then help practitioners better understand the settings under which the system functions properly, and can thus be used with confidence.

The remainder of this paper is organised as follows. First, Section~\ref{sec:background} provides an overview of the context and concepts that underpin this work. Next, Section~\ref{sec:contribution} describes our proposed metamorphic testing framework for pose estimation systems, and Section~\ref{sec:metrules} describes a non-exhaustive set of possible metamorphic relations for this system. Section~\ref{sec:exps} describes the experiments performed to evaluate the framework, and Section~\ref{sec:results} describes and analyses the results of these experiments. Section~\ref{sec:rw} gives an overview of related work, while Section~\ref{sec:threats} describes threats to the validity of this work, and steps taken to address them. Finally, Section~\ref{sec:concAndFuture} concludes the paper and proposes avenues for future work.