\section{Background}
\label{sec:background}

This section provides an overview and definition of the concepts used in this work. First, Section~\ref{sec:bg:pose} defines human pose estimation systems, which the proposed framework tests, then Section~\ref{sec:bg:MLTesting} defines challenges specific to testing Machine Learning (ML) systems, and Section~\ref{sec:bg:metamorphic} explains the idea of metamorphic testing, the basis of the proposed framework. 

\subsection{Pose Estimation}
\label{sec:bg:pose}

Pose estimation is the task of estimating the locations of different landmarks on a subject from images or video frames, with the overall aim of providing an overview of their pose. This task is important to a wide variety of fields including human activity recognition~\cite{luvizon20182d}, sign language recognition~\cite{holmes2023scarcity}, and sports analytics~\cite{martin2021automated}. It is typically solved using DL-based regression algorithms which estimate the coordinates of each body part. One of the most common open-source pose estimation frameworks available is MediaPipe's Pose Landmark Detection system \cite{mediapipePose}, which is often incorporated directly into larger ML frameworks for a variety of tasks. This is typically done without fine-tuning, as this model is trained on a large and diverse set of people -- often a far larger number of individuals than would be available for the lower-resource tasks on which this model is typically applied. It is thus important that these pose estimation systems be tested under different conditions when integrated into different systems that target different use cases and thus will provide images of different natures (e.g., dynamic applications will feed the pose estimation system images that are more blurry).


\subsection{Challenges in Testing ML-based Systems}
\label{sec:bg:MLTesting}
Deep Learning (DL) research has gained enormous momentum over the last decade, with a majority of the development in this area being centred around computer vision. However, despite their success, these deep architectures have also demonstrated harmful decision-making~\cite{birhane2024dark, birhane2022auditing,weidinger2021ethical}, bias~\cite{raji2020saving} and a lack of ``understanding" of common-sense concepts such as basic spatial relations~\cite{hoehing2023s, liu2023visual}. These issues are made all the more challenging to identify given that DL methods lack interpretability, with the steps that lead to a particular decision often being unclear. This lack of transparency means that it is crucial that these systems are systematically tested and their results compared to their expected behaviour in order to understand their sensitivities, along with the conditions under which they can be expected to behave correctly. 

Another aspect that complicates the testing of complex systems is that they often present what is known as the \textit{Oracle Problem}~\cite{barr2014oracle}, which describes a scenario where the correct output of the system under test is not known. Sometimes there are no automated ways to compute this oracle, or the act of querying this oracle can be prohibitively expensive. In the case of pose estimation, labelling keypoints in videos is extremely time consuming, making it expensive to collect this form of ground truth data for many applications. This complicates the testing of these systems and requires techniques that can assess the correct functioning of the system without needing an oracle for the correctness of a single execution of the system. 


Though testing techniques for computer vision-based classification systems have been extensively explored~\cite{ma2018deepmutation, dwarakanath2018identifying, ma2018deepgauge}, less attention has been given to more complex computer vision tasks such as pose estimation. For example, while MediaPipe does evaluate its pose estimation framework for bias related to the demographics of individuals -- an admirable activity -- this analysis is expensive as it requires ground truth information. Additionally, though their evaluation states that degradation can be expected as video quality and lighting gets worse, a comprehensive testing procedure is not outlined. For many applications, it is crucial that we understand the \textit{specific} parameters within which a system can be expected to operate accurately, especially where mistakes are costly.  There has been some work in this area~\cite{hand-pose}


\subsection{Metamorphic testing}
\label{sec:bg:metamorphic}


\textit{Metamorphic Testing} (\MetTest) has been used to address the oracle problem for both classical programs~\cite{xie2011testing} and ML applications~\cite{dwarakanath2018identifying} and has demonstrated promising results at a relatively low cost. \MetTest relies on \emph{metamorphic rules} -- relations between certain changes to the input of the system (e.g., doubling an integer input) and their effect on its output (e.g., the output should be doubled too) -- to circumvent the oracle problem. If one of these relations is violated for a given input, then the system is not behaving correctly. Note that the violation of the metamorphic rule can be verified without knowing what the correct output for either of the inputs is, i.e., without an oracle for the correctness of each output.

This method is well suited for testing pose estimation systems as it does not require ground truth labels, which come from expensive manual annotation of data. Additionally, since the inputs can be modified in numerous ways, it can help understand the boundaries of the input space under which the system performs as expected. 

