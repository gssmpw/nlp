\begin{table*}[h!]
\centering
\caption{Summary of The European Union’s Artificial Intelligence Act.}
\label{tab:summary_ai_gpai}
\resizebox{\textwidth}{!}{%
\begin{tabular}{|l|l|p{10cm}|}
\hline
\textbf{Category} & \textbf{Key Areas} & \textbf{Details} \\ \hline
\multirow{3}{*}{AI Systems} 
    & Definition & Machine-based systems with autonomy and adaptiveness. Generates outputs like predictions, recommendations, or decisions that impact environments. \\ \cline{2-3}
    & Regulation & Risk-based approach, with stricter rules for high-risk systems. High-risk applications include critical infrastructure, education, and vocational training. \\ \cline{2-3}
    & Key Requirements & 
      - \textbf{Risk Management}: Identify and mitigate risks to health, safety, and fundamental rights. Mitigate risks through appropriate design and development measures. Balance risk minimization with other compliance requirements. \\
      & & - \textbf{Data Governance}: Datasets must be relevant, representative, error-free, and contextually appropriate. Prevent biases and address data gaps or shortcomings. \\
      & & - \textbf{Accuracy and Cybersecurity}: Ensure high accuracy and resilience to threats throughout the system's lifecycle. Protect against vulnerabilities, including data and model poisoning or adversarial attacks. \\ \hline
\multirow{3}{*}{GPAI Models} 
    & Definition & General-purpose AI models trained on large datasets, capable of diverse tasks. Example: GPT-3.5. \\ \cline{2-3}
    & Regulation & Risk-based approach, with stricter rules for models posing systemic risks. Systemic risks include societal impact, public health, and large-scale discrimination. \\ \cline{2-3}
    & Key Requirements & 
      - \textbf{Copyright Compliance}: Adhere to EU copyright laws and respect text and data mining restrictions. \\
      & & - \textbf{Systemic Risk Mitigation}: Address risks like manipulation, discrimination, or public harm. \\
      & & - \textbf{Cybersecurity}: Ensure adequate protection against systemic threats, such as adversarial attacks. \\ \hline
\end{tabular}%
}

\end{table*}