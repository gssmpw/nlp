% Machine unlearning (MU) refers to a set of methods for removing, from a machine learning (ML) model, the influence of particular examples in that model's training data \citep{CaoYang2015}. 
Since its inception, Machine Unlearning (MU) has been motivated by the so-called ``right to be forgotten'' (RTBF) \citep{CaoYang2015}, which is now encoded in data privacy laws like the European Union's General Data Protection Regulation (GDPR) \citep[Art. 17]{european_union_gdpr_2016}. 
% The logic behind this motivation typical holds that, because a data subject's right to request deletion of their data extends to ``downstream derivatives'' like models trained on that data \citep{JULIUSSEN2023105885}, using MU to remove the influence of that data from the model implements RTBF and, thus, aids compliance with these data privacy laws \citep{DBLP:conf/sp/BourtouleCCJTZL21, yang2024machinelearningmachineunlearning}. 
However, recent works call into question whether this motivation really holds water \citep{JULIUSSEN2023105885} or, if it does, whether MU can ever fully deliver for that use case \citep{cooper2024machineunlearningdoesntthink}.\footnote{We note that this conversation is not entirely irrelevant to the AIA, because AI developers who process personal data within the meaning of GDPR must abide by that law  
\citep{mazzini2023proposal}. However, because our goal here is to focus on the ``net new'' applications of MU invoked by AI regulation, we do not revisit the discussion about MU and RTBF or GDPR.}

In the meantime, a new --- and perhaps more compelling -- motivation for MU has emerged: artificial intelligence (AI) regulation. Worldwide, multiple AI regulation efforts are working their way through the legislative process \citep{BELLI2023105767, Beardwood+2024+129+137, Zhang+2024+162+165} or have graduated it and gone into effect \citep{european_union_ai_act_2024, colorado_ai_act_2024}. Perhaps presaging the multitude of new use cases for MU these laws invoke, scholars have begun scratching the surface of how MU interacts with AI regulation (or, at least, the trustworthy AI principles it often inscribes) \citep{hine_supporting_2024, DIAZRODRIGUEZ2023101896, li2024wmdpbenchmarkmeasuringreducing}.  

% Like the data privacy laws that precede them, these AI regulations sometimes regulate data \citep[Art. 10]{european_union_ai_act_2024}. Unlike data privacy laws, however, these AI regulations sometimes also regulate the models that are the direct target of MU \citep[Art. 15]{european_union_ai_act_2024}. In many cases, this arguably makes using of MU to aid compliance with AI regulation more straightforward than using it to aid compliance with data privacy laws. 

In this position paper, we argue that, while MU indeed has great promise as a tool for complying with AI regulation, this potential will only be realized if researchers and policymakers collaboratively close the gaps between MU's state of the art and these prospective new applications. We use the European Union's Artificial Intelligence Act (AIA) \citep{european_union_ai_act_2024} as an example to support our argument. 
% This paper aims first and foremost to make this link between MU and AI regulation more concrete, with the European Union's Artificial Intelligence Act (AIA) \citep{european_union_ai_act_2024} as a proof point. 
This starts with a thorough cataloging of the AIA requirements that MU can assist compliance with.
% As it will be shown, this catalog has two parts. The first focuses on the AIA's \textit{systems- and model-centered requirements}, which we argue make more straightforward targets for MU as a compliance tool. The second, meanwhile, covers the AIA's \textit{data-centered requirements}, where, we argue, the viability of MU as a compliance tool is less obvious.
% Because the AIA does not directly reference MU, establishing these connections sometimes requires legal interpretation of the text of the AIA's requirements. Though very different in substance, these legal analyses echo earlier ones connecting MU to data privacy laws \citep{JULIUSSEN2023105885} 
For each potential use case, we flag any legal ambiguities that lawmakers, in order to clarify MU's potential as an AIA compliance tool, should address when amending, updating, or translating the AIA into codes of practice or technical specifications. What is more, we scrutinize, from a technical perspective, whether the state of the art (SOTA) of MU can really support the hypothesized application. In many cases, we identify critical gaps between the two and call for the research community to help resolve them. Finally, we conclude with a pointed call for the AI research and legislative communities to act together to fill these gaps.

% Perhaps  surprisingly, given how quickly MU is given as as key tool in addressing AI regulation needs — we discover MU can not meet the needs of AIA requirements in the majority of cases given. This finding is supported by our first of its kind systematic analysis that shows this gap between today’s technical capabilities of MU  and informal expectations based on regulatory needs. 

% As we do all of this, we are mindful of key features of the contemporary AI landscape, such as the large language models (LLMs) and other foundation models that are the ``centerpiece[s] of the modern AI ecosystem'' \citep{Klyman2023How} and the rising importance of open source models and datasets \citep{DBLP:journals/corr/abs-2108-07258, DBLP:journals/corr/abs-2405-13058}. 



% In sum, our contributions are:
% \begin{itemize}
%     \item {A thorough mapping of AIA requirements onto potential applications of MU to aid compliance with those requirements;}
%     \item {A set of action items for the legislative community, to help clarify the viability of these potential applications;}
%     \item {A technical analysis of whether the SOTA of MU supports each potential application;}
%     % \item {technical specifications for how to measure and navigate trade-offs while implementing MU for AI regulation compliance;}
%     \item {A set of technical research priorities to help fill any gaps between the potential applications and MU's SOTA;}
% \end{itemize}