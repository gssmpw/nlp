The ``right to be forgotten'' and the data privacy laws that encode it have motivated machine unlearning since its earliest days. Now, an inbound wave of artificial intelligence regulations --- like the European Union's Artificial Intelligence Act (AIA) --- potentially offer important new use cases for machine unlearning.
%Now, as we enter the era of AI regulation, machine unlearning may find its second wind as a tool for complying with the European Union's Artificial Intelligence Act (AIA) and other laws like it. 
%  
However, this position paper argues, this opportunity will only be realized if researchers, aided by policymakers, proactively bridge the (sometimes sizable) gaps between machine unlearning's state of the art and its potential applications to AI regulation. To demonstrate this point, we use the AIA as an example. Specifically, we deliver a ``state of the union'' as regards machine unlearning's current potential for aiding compliance with the AIA. This starts with a precise cataloging of the potential applications of machine unlearning to AIA compliance. For each, we flag any legal ambiguities clouding the potential application and, moreover, flag the technical gaps that exist between the potential application and the state of the art of machine unlearning. Finally, we end with a call to action: for both machine learning researchers and policymakers, to, respectively, solve the open technical and legal questions that will unlock machine unlearning's potential to assist compliance with the AIA --- and other AI regulation like it.