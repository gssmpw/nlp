% \begin{figure}[h!]
%     \centering
%     \resizebox{\columnwidth}{!}{% Resize the entire TikZ picture to fit one column
%         \begin{tikzpicture}[
%             % Define styles for different elements
%             nodeStylePrimary/.style={
%                 draw, 
%                 ellipse, 
%                 text centered, 
%                 minimum height=1.5cm, 
%                 minimum width=3cm, 
%                 font=\bfseries
%             },
%             nodeStyleSecondary/.style={
%                 draw, 
%                 rectangle, 
%                 rounded corners, 
%                 text centered, 
%                 minimum height=1cm, 
%                 minimum width=2cm, 
%                 font=\bfseries
%             },
%             arrowStyleTradeoff/.style={
%                 <->, 
%                 thick, 
%                 solid, 
%                 -latex
%             },
%             arrowStylePrivacy/.style={
%                 <->, 
%                 thick, 
%                 dashed, 
%                 -latex
%             },
%             arrowStyleControl/.style={
%                 ->, 
%                 thick, 
%                 dotted, 
%                 -latex
%             },
%             legendBox/.style={
%                 draw, 
%                 rectangle, 
%                 rounded corners, 
%                 fill=white, 
%                 font=\small,
%                 inner sep=5pt
%             }
%         ]
        
%         % Primary Objectives
%         \node[nodeStylePrimary, fill=blue!30] (Utility) at (0, 0) {Model Utility};
%         \node[nodeStylePrimary, fill=green!30] (Forgetting) at (4, 0) {Forgetting Quality};
%         \node[nodeStylePrimary, fill=orange!30] (Efficiency) at (2, 4) {Efficiency};
        
%         % Privacy as external factor (shifted down slightly)
%         \node[nodeStylePrimary, fill=red!30] (Privacy) at (2, -4) {Privacy};
        
%         % Hyperparameters node (shifted slightly right)
%         \node[nodeStyleSecondary, fill=yellow!30] (Hyperparams) at (6.5, 1.75) {Hyperparameters};
        
%         % Trade-off arrows (solid) with adjusted label positions
%         \draw[arrowStyleTradeoff] (Utility) -- (Forgetting) node[pos=0.6, above] {Balance};
%         \draw[arrowStyleTradeoff] (Forgetting) -- (Efficiency) node[pos=0.6, right] {Trade-off};
%         \draw[arrowStyleTradeoff] (Efficiency) -- (Utility) node[pos=0.4, left] {Trade-off};
        
%         % Privacy relationships (dashed) with shifted labels
%         \draw[arrowStylePrivacy] (Privacy) -- (Forgetting) node[pos=0.4, left] {Conflict};
%         \draw[arrowStylePrivacy] (Privacy) -- (Utility) node[pos=0.6, right] {Impact};
        
%         % Hyperparameters relationships (dotted) with adjusted angles
%         \draw[arrowStyleControl] (Hyperparams) -- (Forgetting) node[pos=0.6, above, sloped] {Adjust};
%         \draw[arrowStyleControl] (Hyperparams) -- (Efficiency) node[pos=0.4, below, sloped] {Tune};
        
%         % Annotations for Privacy Conflicts (shifted slightly outward)
%         \node[draw, diamond, fill=gray!20, minimum size=1cm, above left=0.5cm and 0.5cm of Privacy] (Conflict1) {Info Leak};
%         \node[draw, diamond, fill=gray!20, minimum size=1cm, below left=0.5cm and 0.5cm of Privacy] (Conflict2) {Membership Reveal};
%         \draw[->, dashed] (Privacy) -- (Conflict1);
%         \draw[->, dashed] (Privacy) -- (Conflict2);
        
%         \end{tikzpicture}
%     }
%     \caption{Illustration of the trade-offs in unlearning algorithms among Model Utility, Forgetting Quality, Efficiency, and Privacy. Hyperparameters serve as control elements to navigate these trade-offs. Solid arrows represent direct trade-offs, dashed arrows indicate privacy-related relationships, and dotted arrows denote the influence of hyperparameters.}
%     \label{fig:enhanced-tradeoffs-diagram}
% \end{figure}

\begin{figure}[h!]
    \centering
    \resizebox{0.8\columnwidth}{!}{%
        \begin{tikzpicture}[
            % Define styles for different elements
            nodeStylePrimary/.style={
                draw, 
                ellipse, 
                text centered, 
                minimum height=1.5cm, 
                minimum width=3cm, 
                font=\bfseries
            },
            arrowStyleTradeoff/.style={
                <->, 
                thick, 
                solid
            },
            arrowStylePrivacy/.style={
                <->, 
                thick, 
                dashed
            }
        ]
        
        % Repositioned Primary Objectives
        \node[nodeStylePrimary, fill=orange!30] (Efficiency) at (-2, 2) {Efficiency};
        \node[nodeStylePrimary, fill=blue!30] (Utility)    at ( 2, 2) {Model Utility};
        \node[nodeStylePrimary, fill=green!30] (Forgetting) at ( 0,-1.5) {Forgetting Quality};
        
        % Privacy below Forgetting
        \node[nodeStylePrimary, fill=red!30] (Privacy) at (0, -4) {Privacy};
        
        % Trade-off arrows among Efficiency, Utility, and Forgetting (no labels)
        \draw[arrowStyleTradeoff] (Efficiency) -- (Utility);
        \draw[arrowStyleTradeoff] (Efficiency) -- (Forgetting);
        \draw[arrowStyleTradeoff] (Utility)    -- (Forgetting);
        
        % Privacy relationship (dashed) ONLY with Forgetting (no label)
        \draw[arrowStylePrivacy] (Privacy) -- (Forgetting);
        
        % Label inside the triangle
        \node[font=\bfseries] at (0, 0.85) {Hyperparameters};
        
        % (Optional) Privacy conflict annotations (if you still want them)
        \node[draw, diamond, fill=gray!20, minimum size=1cm, above left=-0.1   cm and 0.7cm of Privacy] (Conflict1) {Onion effect};
        \node[draw, diamond, fill=gray!20, minimum size=1cm, below left=0.5cm and 0.6cm of Privacy] (Conflict2) {Streisand Effect};
        \draw[->, dashed] (Privacy) -- (Conflict1);
        \draw[->, dashed] (Privacy) -- (Conflict2);
        
        \end{tikzpicture}
    }
    \caption{\textit{\textbf{Illustration of MU trade-offs.}}}
    \label{fig:updated-tradeoffs-diagram}
\end{figure}

