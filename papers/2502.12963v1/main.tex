%%%%%%%%%%%%%%%%%%%%%%%%%%%%%%%%%%%%%%%%%%%%%%%%%%%%%%%%%%%%%%%%%%%%%%%%%%%%%%%%
%2345678901234567890123456789012345678901234567890123456789012345678901234567890
%        1         2         3         4         5         6         7         8

\documentclass[letterpaper, 10 pt, conference]{ieeeconf}  % Comment this line out if you need a4paper

%\documentclass[a4paper, 10pt, conference]{ieeeconf}      % Use this line for a4 paper

\IEEEoverridecommandlockouts                              % This command is only needed if 
                                                          % you want to use the \thanks command

\overrideIEEEmargins                                      % Needed to meet printer requirements.

%In case you encounter the following error:
%Error 1010 The PDF file may be corrupt (unable to open PDF file) OR
%Error 1000 An error occurred while parsing a contents stream. Unable to analyze the PDF file.
%This is a known problem with pdfLaTeX conversion filter. The file cannot be opened with acrobat reader
%Please use one of the alternatives below to circumvent this error by uncommenting one or the other
%\pdfobjcompresslevel=0
%\pdfminorversion=4

% See the \addtolength command later in the file to balance the column lengths
% on the last page of the document

% The following packages can be found on http:\\www.ctan.org
%\usepackage{graphics} % for pdf, bitmapped graphics files
%\usepackage{epsfig} % for postscript graphics files
%\usepackage{mathptmx} % assumes new font selection scheme installed
%\usepackage{times} % assumes new font selection scheme installed
%\usepackage{amsmath} % assumes amsmath package installed
%\usepackage{amssymb}  % assumes amsmath package installed
\let\labelindent\relax
\usepackage{cite}
\usepackage{graphicx}
\usepackage{subfigure}
\usepackage{dblfloatfix}
\usepackage{amsmath}
\usepackage{booktabs}
\usepackage{multirow}
\usepackage{enumitem}
\usepackage{amsfonts}
\usepackage{CJK}
\usepackage{verbatim}
\usepackage{hyperref}


\title{\LARGE \bf
D3-ARM: High-Dynamic, Dexterous and Fully Decoupled 

Cable-driven  Robotic Arm
}


\author{Hong Luo, Jianle Xu, Shoujie Li, Huayue Liang, Yanbo Chen, Chongkun Xia and Xueqian Wang % <-this % stops a space
\thanks{This work was supported by the the National Key R\&D Program of China (2022YFB4701400/4701402), National Natural Science Foundation of China (No.62103225, U21B6002, 62203260, 92248304), Natural Science Foundation of Shenzhen (No. JCYJ20230807111604008), Natural Science Foundation of Guangdong Province (No.2024A1515010003) and Guangdong Basic and Applied Basic Research Foundation (2023A1515011773).}% <-this % stops a space
\thanks{Hong Luo, Jianle Xu, Shoujie Li, Huayue Liang, Yanbo Chen and Xueqian Wang are with the Center for Artificial Intelligence and Robotics, Shenzhen International Graduate School, Tsinghua University, Shenzhen 518055, China.}
\thanks{Chongkun Xia is with School of Advanced Manufacturing, Sun Yat-sen University, shenzhen 518107, China.}
\thanks{Corresponding author: Chongkun Xia (xiachk5@mail.sysu.edu.cn), Xueqian Wang (wang.xq@sz.tsinghua.edu.cn)}
}


\begin{document}

\maketitle
\thispagestyle{empty}
\pagestyle{empty}


%%%%%%%%%%%%%%%%%%%%%%%%%%%%%%%%%%%%%%%%%%%%%%%%%%%%%%%%%%%%%%%%%%%%%%%%%%%%%%%%
\begin{abstract}

% 通过绳索的远程驱动特性,绳驱臂不仅能有效降低运动部分的惯量提高动态性能,还能使电机远离操作环境从而提高交互安全性。
% 然而运动耦合和布线等问题限制了这两个优势的进一步结合,即将绳驱臂所有电机全部后置在基座并不影响机械臂的控制精度和性能是一个难题。

%Cable transmission enables motors to remotely drive lightweight joints for operational tasks, allowing cable-driven robotic arms to be widely applied in interactive and extreme environments. 
%However, this transmission method brings motion coupling and cable routing problems, thereby affect the control precision and motion performance of cable-driven arms in these environments.
Cable transmission enables motors of robotic arm to operate lightweight and low-inertia joints remotely in various environments, but it also creates issues with motion coupling and cable routing that can reduce arm's control precision and performance.
In this paper, we present a novel motion decoupling mechanism with low-friction to align the cables and efficiently transmit the motor's power. By arranging these mechanisms at the joints, we fabricate a fully decoupled and lightweight cable-driven robotic arm called D3-Arm with all the electrical components be placed at the base. 
Its 776 mm length moving part boasts six degrees of freedom (DOF) and only 1.6 kg weights. To address the issue of cable slack, a cable-pretension mechanism is integrated to enhance the stability of long-distance cable transmission. Through a series of comprehensive tests, D3-Arm demonstrated 1.29 mm average positioning error and 2.0 kg payload capacity, proving the practicality of the proposed decoupling mechanisms in cable-driven robotic arm. 

% 大致结构按这样写,结尾的performance再斟酌斟酌

\end{abstract}


%%%%%%%%%%%%%%%%%%%%%%%%%%%%%%%%%%%%%%%%%%%%%%%%%%%%%%%%%%%%%%%%%%%%%%%%%%%%%%%%
\section{INTRODUCTION}
% 先讲绳驱臂在轻量化方面(机电分离的其它好处)的研究,引到绳驱对提高机械臂动态性能的帮助,然后指出目前的耦合问题
% 然后是绳驱对灵活性的帮助
% 最后讲到绳驱虽然能提高这些性能,但也带来了coupled相关问题,这些问题会影响到控制性能和精度,然后讲解耦的研究


% 绳驱臂广泛应用于机械臂的轻量化设计中,但对某些如辐射环境或高低温环境中的机械臂,还需要对电机等进行集中保护,从而增加机械臂的耐用性。这就要求绳驱臂的电机能全部放置在基座中与操作环境隔离。

With the features of backlash-free actuation and remote transmission, cable-driven joints are widely used in the lightweight design of robotic arms by placing motors at the proximal part\cite{lightweight_cb_manipulator,ozawa2013analysis,6386236}. However, for certain environments such as high radiation and underwater tasks, centralized protection of components like motors is necessary to enhance the durability like radiation resistance of the robotic arms\cite{jiang2013mechanism}. This necessitates that all motors of the cable-driven arm be housed within the base to ensure isolation from the operating environment\cite{Dragon,6907730}. In this paper, we design a cable-driven robotic arm suitable for such environments, as illustrated in Fig. \ref{fig:enter-label}.


%In this configuration, the motors can remotely drive each joint by cables through pulleys\cite{low-cost}\cite{wam} or bowden cables\cite{bowden}, which can also significantly reduce the mass and inertia of robotic arm's moving parts and improve their acceleration performance and high-dynamic performance.

To concentrate all the motors of the cable-driven robotic arm at the base and effectively drive the joints, the primary challenge is to design an efficient cable transmission routing. One of the most straightforward methods is to pass the driving cables directly through holes in the joint components and arrange these modular joints in a series to form a cable-driven continuum arm\cite{continuum1,continuum2,continuum3,continuum4} or use tubes to constrain the transmission path of the cables\cite{Self-calibration,self-calibration_2,bowden}. Despite its simplicity, this transmission method can introduce significant friction influence, which in turn reduces the control precision and efficiency of the cable-driven arm\cite{Parallel_Continuum}. To avoid this friction issues, grooved pulleys are widely used along the cable transmission path\cite{wam,constant_tendon_length}. The Twist Snake \cite{snake} designed a joint composition where multiple pulleys shared one joint axis to send the cables from motors at the base to driven joints. SAQIEL\cite{SAQIEL} introduced a passive 3D wire alignment mechanism composed by grooved pulleys to achieve a supple drive from actuators onto the root link by minimal friction. This transmission method can effectively solve the energy loss issue along the long transmission path of cable-driven arms.
%Generally, these methods require a cable-pretension mechanism to load tension of the cable, thereby avoiding the cable from loosening\cite{8593679,9349131}.

%Mustafa \textit{et al.} designed a lightweight cable-driven arm by directly connecting the joints to the motors located at the base using cables and tubes\cite{Self-calibration}, where the friction between them bring significant challenges to transmission efficiency and modeling. To avoid this friction issues, grooved pulley is also widely used in cable transmission. Temma Suzuki \textit{et al.} introduced a low-friction cable-driven manipulator using a passive 3D wire aligner, which is composed by bearings and grooved pulleys\cite{SAQIEL}. Shunichi Sakurai \textit{et al.} designed a 4-DOF tendon-driven manipulator with constant tendon-length routing through pulleys at every joints\cite{constant_tendon_length}. Kazutoshi Tanaka and Masashi Hamaya designed a 7-DOF cable-driven robotic arm with multiple pulleys at joint axis\cite{snake}. These cable-driven arms not only achieve efficient remote transmission through cables and pulleys but also successfully centralize the motors at the base, resulting in a lightweight and safe design for the entire robotic arm.

%A key design focus of the cable-driven robotic arm is to fully leverage the flexibility and elasticity of the cables in the development of transmission method. One easy way to transmit power from motors at the base to joints is using wires and tubes\cite{Self-calibration}, \cite{self-calibration_2}. With this method, Mustafa \textit{et al.} developed a lightweight cable-driven robotic arm with a great modeling challenge from the friction between the cables and tubes\cite{Self-calibration}. To avoid the friction issues during the cable transmission, grooved pulley are widely used in cable transmission. Pang \textit{et al.} developed a cable-driven humanoid manipulator with a high-stiffness cable-driven joint composed by movable pulley groups, which can avoid the impact of friction on joint movement\cite{stiffness}.

\begin{figure}[t]
    \centering
    \includegraphics[width=1\linewidth]{picture/Overview_of_D3_arm_pro.pdf}
    \caption{D3-Arm: a fully decoupled cable-driven robotic arm with all its electrical components located at the base.}
    \label{fig:enter-label}
    \vspace{-0.5cm}
\end{figure}

\begin{figure*}[t] % 使用 figure* 环境
    \centering
    \includegraphics[width=1\linewidth]{picture/Proportion_and_Composition_of_D3-Arm_pro.pdf}
    \caption{Proportion and joint composition of D3-Arm. Pairs of lines in the same color indicate a group of cables that drive the same joint. The side view shows the cable routing of the decoupling cable aligner mechanism arranged in Joint1. The front view shows the cable routing of decoupling rolling pair mechanism arranged in Joint2 and Joint3.}
    \label{fig:Proportion}
    \vspace{-0.5cm}
\end{figure*}
However, the abovementioned cable transmission methods inevitably bring coupling problems where the movement of the certain link can affect the length and tension of the cables driving other links. This deficiency greatly limits the control precision and durability of the robotic arm\cite{motion_decoupled}. To solve this coupling problem, many research efforts have focused on compensating the length variations in cable coupling by implementing decoupling mechanisms\cite{jiang2018design,choi2024design}. The follower decoupling mechanism that synchronously drags the cables with joint movement ensures the length of the cable driving other joints remain constant\cite{constant_length}. The alignment decoupling mechanism, which keeps the cable consistently aligned with the joint axis, is also an effective method to prevent coupling effects on the cable\cite{RoboRay_hand}. Based on these strategies, the LIMS arm used a rolling joint as its elbow, successfully decoupling the elbow motion and the movement of the wrist tendon\cite{kim,kim2018development}. The CDSSR achieved full-joint decoupling through the cable length compensator in each articulation\cite{xu2024design}. However, this compensation-based decoupling method adversely affects cable tension stability, thus limiting joint dynamic performance. While the CDSSR demonstrates the feasibility of full-arm decoupling, achieving high-speed and fully decoupled motion still requires further improvements in transmission system design.

% Xu \textit{et al.} proposed a cable-driven manipulator equipped with a decoupling link placed at the elbow joint to minimize the motion interference to the cables\cite{xu2024design}. But these decoupling mechanisms can only address the coupling issue of one single joint in aforementioned cable-driven arms, with their motors still positioned in the moving parts. Achieving decoupling for all joints in cable-driven arms presents significant challenges.

%Kim \textit{et al.} designed an decoupled elbow rolling joint with four idling pulleys placed symmetrically and integrated it into a 7 DOF humanoid cable-driven manipulator named LIMS\cite{kim},\cite{kim2018development}. But this mechanism can only decouple the motion of single DOF and the actuators that drive the wrist and elbow are still mounted on the upper arm. Lee \textit{et al.} designed a tendon-driven servo-manipulator with a novel motion decoupling mechanism\cite{lee2008design}. Surong Jiang \textit{et al.} designed a novel motion-decoupling modular joint to effectively solve the motion-coupling, hysteresis and the dead zone problems in cable-driven motion\cite{jiang2018design}. Both of the decoupling mechanisms compensate the variations in the coupled length of the cables using following link or pulleys, however require a relatively large bulk, which affects the working space of the robotic arm.

%Unlike the decoupled strategies mentioned above which is focus on the joint motion relative to the radial direction of the cables, there are other strategies that achieve axial decoupled motions. RoboRay Hand used grooved pulleys fixed with the joint to align the cables on it be collinear with the axis of the joint at all angle\cite{RoboRay_hand}. Divya Shah presented a constant-length tendon routing mechanism for an axial joint that achieved decoupled motions\cite{constant_length}. But this decoupled method poses certain challenges when dealing with multiple cables.

In this paper, we aim to develop a fully decoupled cable-driven robotic arm with all its motors mounted at the base. To achieve this goal, we propose lightweight cable-driven joints equipped with decoupling mechanisms, utilizing a cable-pulley configuration to facilitate low-friction power transmission, and assemble these joints in series to form a 6 DOF cable-driven robotic arm. This cable-driven robotic arm can not only facilitate the isolation and protection of components such as motors to adapt to various environments, its fully decoupled system also enhance the control precision and sufficient dynamic performance. The main contributions of this work are as follows:

\begin{enumerate}
    \item A fully decoupled and low-friction transmission mechanism is proposed to enable high-efficiency and high-precision remote transmission via cables.
    \item A cable-driven arm is designed based on the decoupling mechanisms, with all the electrical components such as motors mounted at the base, ensuring the operational safety in interactive or extreme environments.
    \item The effectiveness of the decoupling mechanisms in enhancing the control precision of the cable-driven arm is validated through end-effector position repeatability experiment.
\end{enumerate}


% 列举创新点
% 提出了一种全解耦低摩擦的传动解耦装置,实现了高效率且高精度的绳索远程传动;
% 将这种装置应用到一款电机全部后置的机械臂中,实现了机电分离的绳驱臂
% 实现了全臂的轻量化设计,使机械臂具备高速运动的基础

%%%%%%%%%%%%%%%%%%%%%%%%%%%%%%%%%%%%%%%%%%%%%%%%%%%%%%%%%%%%%%%%%%%%%%%%%%%%%%%%
%\section{RELATED WORKS}
% 


%\subsection{Cable-driven Robotic Arm}
% wam lims等,体现出绳索后置对绳驱臂性能提升好的好处,引出解耦装置的必要性
%As one of the earliest commercially applied cable-driven robotic arm, WAM employed low ratio cable transmission and comprised cable differential transmission systems\cite{wam}. Kim developed an anthropomorphic and low-inertia cable-driven manipulator using tension-amplifying mechanism, which can decouple the motion between the elbow joint and wrist joint\cite{kim}. With that decouple mechanism, all the joints of this manipulator are remotely driven by the motors placed at the shoulder joint. Shunichi Sakurai \textit{et al.} designed a 4-DOF tendon-driven manipulator with constant tendon-length routing and mounted all its motors at the base. Temma Suzuki \textit{et al.} introduced a low-friction cable-driven manipulator using a passive 3D wire aligner, which is composed by bearings and grooved pulleys\cite{SAQIEL}. Both of the above manipulators have achieved extreme lightness of moving part by placing the motors and actuators at the base, but their motion accuracy and dynamic performance are still limited by coupled cable-driven system.

%\subsection{decoupling mechanism}
% RoboRay hand: A highly backdrivable robotic hand with sensorless contact force measurements
% 轴线解耦装置
% Towards simplicity: On the design of a 2-DOFs wrist mechanism for tendon-driven robotic hands
% Design and Evaluation of a Cable-Actuated Palletizing Robot With Geared Rolling Joints
%In order to solve the coupling problem of cable-driven joints and let motors to be separated from the joints without affecting performance, various decoupling mechanism have been developed. Wontae Choi \textit{et al.} design a 4-DOF cable-actuated robot with geared rolling joints arranged by series\cite{choi2024design}. However, this single decoupling routing mechanism limits the dexterity of the robot. Jong Kwang Lee \textit{et al.} designed a tendon-driven servo-manipulator with a novel motion decoupling mechanism\cite{lee2008design}. They used moving pulleys and a decoupling link to successfully compensate the length variation of the cables. Surong Jiang \textit{et al.} designed a novel motion-decoupling modular joint to effectively solve the motion-coupling, hysteresis and the dead zone problems in cable-driven motion\cite{jiang2018design}. They used the motion of following wheel and decoupling cable to compensate the coupling length of the cables. But those decoupling mechanisms require a relatively large bulk, which affects the working space of the robotic arm. 

%Unlike the decoupled strategies mentioned above which is focus on the joint motion relative to the radial direction of the cables, there are other strategies that achieve axial decoupled motions. RoboRay Hand used grooved pulleys fixed with the joint to align the cables on it be collinear with the axis of the joint at all angle\cite{RoboRay_hand}. By combing this routing mechanism and rolling motion, RoboRay Hand implemented a 2-DOF cable-driven decoupling mechanism. Divya Shah presented a constant-length tendon routing mechanism for an axial joint that achieved decoupled motions\cite{constant_length}. Based on the idea of moving pulleys, this mechanism can allow tendon routing of all four wrist actuation tendons. 



%%%%%%%%%%%%%%%%%%%%%%%%%%%%%%%%%%%%%%%%%%%%%%%%%%%%%%%%%%%%%%%%%%%%%%%%%%%%%%%%
\section{Mechanism Design of D3-Arm}


\subsection{Design Requirement} 
In this chapter, we provide a detailed description for the design of the 6-DOF cable-driven robotic arm named D3-Arm(High-\textbf{D}ynamic,\textbf{D}exterous and fully \textbf{D}ecoupled). 


From the aforementioned research, it is clear that to enhance the durability of the cable-driven arm in extreme environments, the isolation for the electrical components like motors from the operating environment is essential. 
To achieve this configuration, it is necessary to design a high-efficiency and fully decoupled cable transmission path to enhance remote control accuracy, which implies the integration of low-friction and lightweight decoupling mechanisms within the joints. In addition, the decoupling mechanism within each joint must be able to simultaneously decouple multiple driving cables to enhance the flexibility of the robotic arm. Finally, to fully leverage the advantage of the remote driving feature of cables, the mass and inertia of the robotic arm's moving parts should be sufficiently low while ensuring a certain load capacity to enhance its dynamic motion capabilities and collision safety. From this, the following design objectives for the D3-Arm can be summarized:
%From the above research status, it is evident that enhancing the dynamic performance of cable-driven robotic arms hinges on minimizing the mass and inertia of the moving parts while ensuring a certain level of stiffness and payload capacity. Based on this premise, it becomes essential to centralize all driving motors and reducers in the base of the robotic arm. In this configuration, the cables will pass through the remaining joints along their transmission path, leading to issues such as coupling and friction. Therefore, it is essential to design a decoupling mechanism to enhance the control precision of the cable-driven arm. Considering the driving resistance, it is preferable for the decoupling mechanism to be composed of grooved bearings or pulleys in order to minimize friction caused by cable displacement as much as possible. From this, the following design objectives for the D3-Arm can be summarized:

\begin{enumerate}
    \item \emph{Safety in multiple environments}: All motors and other electrical components are centrally arranged at the base, ensuring the durability and applicability in human-robot interactive or extreme environments.
    \item \emph{Accuracy in aforementioned configuration}: Implementing decoupling mechanisms in the cable routing to make all the joint motions be completely independent.
    \item \emph{Utilization of the features in cable-driven system}: Minimizing the friction in the cable transmission system and the inertia of moving parts to maximize the advantages of cable-driven systems.

\end{enumerate}



% 根据篇幅考虑一下要不要解释为什么一个关节需要两根绳索
\subsection{Joint Composition}
Fig.\ref{fig:Proportion} shows the proportion and the joint composition of D3-Arm. The six driving motors are positioned at the base and remotely actuate the corresponding six joints through six pairs of driving cables. Among these, Joint1, Joint2, and Joint3 are rotational or rolling joints. To ensure that the robotic arm has a sufficiently large working space, the motion axis of Joint1 is arranged to be perpendicular to that of Joint2. In contrast, the remaining terminal joints employs a 3-DOF quaternion joint due to considerations of size and integration\cite{quternion}. This parallel joint not only allows for independent motions but also provides ample movement space, making it highly suitable as a end-effector for the fully decoupled cable-driven system.

In the transmission path, only the driving cable of Joint1 is directly connected to the motor, while the remaining driving cables must continuously pass through at least one joint before they can connect to the motors. In this configuration, Joint1, Joint2, and Joint3 require the design of decoupling mechanism to prevent the influence of joint motions on the driving cables. Besides, the decoupling mechanisms of those three joints should allow cable routing of at least six actuation cables, making the modular design of the mechanism extremely important. To this end, we have designed and implemented two types of decoupling mechanisms, referred to 1-DOF decoupling cable aligner mechanism and 1-DOF decoupling rolling pair mechanism.

\begin{figure}[t]
    \centering
    \includegraphics[width=1\linewidth]{picture/Schematics_of_the_motion_coupling_problem_and_decoupled_way_during_a_pitching_motion.pdf}
    \caption{Schematics of the motion coupling problem and decoupling way during a pitching motion. (a) Normal routing of cables in two adjoining joints. (b) Motion coupling in normal routing. (c) Decoupled routing of cables in two adjoining joints. (d) Motion situation after decoupling.}
    \label{fig:Schematics}
    \vspace{-0.5cm}
\end{figure}

\begin{comment}
\begin{figure}[t]
    \centering
    \subfigure[]{
        \includegraphics[width=0.21\textwidth]{picture/a-Schematics of the motion coupling problem and decoupled way during a pitching motion.png}
        \label{fig:subfig1}
    }
    \subfigure[]{
        \includegraphics[width=0.21\textwidth]{picture/b-Schematics of the motion coupling problem and decoupled way during a pitching motion.png}
        \label{fig:subfig2}
    }
    \subfigure[]{
        \includegraphics[width=0.21\textwidth]{picture/c-Schematics of the motion coupling problem and decoupled way during a pitching motion.png}
        \label{fig:subfig3}
    }
    \subfigure[]{
        \includegraphics[width=0.21\textwidth]{picture/d-Schematics of the motion coupling problem and decoupled way during a pitching motion.png}
        \label{fig:subfig4}
    }
    \caption{Schematics of the motion coupling problem and decoupled way during a pitching motion. (a) Normal routing of cables in two adjoining joints. (b) Motion coupling in normal routing. (c) Decoupled routing of cables in two adjoining joints. (d) Motion situation after decoupling.}
    \label{fig:main}
\end{figure}    
\end{comment}


\subsection{Decoupling Cable Aligner Mechanism}
The coupling problems caused by remote driving with the cables during pitch motion are shown in the Fig. \ref{fig:Schematics}(a). The driving cables of Joint2 need to pass through Joint1 before connecting to Motor2, causing its winding section on Joint1 to become a coupled part (highlighted in red in Fig. \ref{fig:Schematics}(a)). When Motor1 drives Joint1 to move by an angle $\Delta \theta_1$, the length of the coupled portion of the driving cable for Joint2 change accordingly, causing Joint2 to also move by an angle $\Delta \theta_2$, even though Motor2 is not actively driving it. Finally, as shown in Fig. \ref{fig:Schematics}(b), this coupled angle causes the end effector to deviate from its originally intended position (the red position in Fig. \ref{fig:Schematics}(b)).

To solve this problem, an effective and easily extendable decoupling method is to align the driving cable with the motion axis of Joint1 using a pulley system (as shown in Fig. \ref{fig:Schematics}(c)). This system requires at least two grooved pulleys, one of which is rigidly connected to the base to tension and pull the cable to the motion axis of Joint1, while the other is rigidly connected to Link1, ensuring the cable remains aligned with the axis at any angle of movement as shown in Fig. \ref{fig:Schematics}(d). Thus, the motion of Joint1 does not affect the length of the driving cable for Joint2, achieving decoupled motion.

\begin{figure}[t]
    \centering
    \includegraphics[width=1\linewidth]{picture/Detailed_design_of_decoupled_cable_aligner_Mechanism.pdf}
    \caption{Detailed design of decoupling cable aligner Mechanism. The mechanism consists of fixed pulleys and movable pulleys, ensuring that the driving cables remain aligned with the motion axis of Joint1.}
    \label{fig:design of aligner}
    \vspace{-0.5cm}
\end{figure}

Based on the aforementioned principle, we design a decoupling cable aligner mechanism and applied it into Joint1 of the D3-Arm to decouple the 10 driving cables that pass through it, as shown in Fig. \ref{fig:design of aligner}. The mechanism is composed of fixed pulleys and movable pulleys, where the fixed pulleys are rigidly connected to the base, and movable pulleys are rigidly connected to the Joint1. They align the driving cables for five joints from the motors to the motion axis of Joint1. Considering the interference between this driving cables, it is necessary to arrange these pulleys sequentially and route the cables radially. The top view of the cable routing within the decoupled device is illustrated as shown in side view of Fig. \ref{fig:Proportion}, where the driving cables for the same joint are arranged symmetrically about the center to facilitate the subsequent configuration of the cables for other joints. With the correct cable routing and pulley configuration, the driving cables can always follow the motion axis of Joint1 during its rotational movement.


\subsection{Decoupling Rolling Pair Mechanism}
To address the issue of cable coupling in multi-joint systems, it is also essential to minimize the number of transmission components that the cables pass through, thereby reducing the risk of cable slackening and detachment\cite{snake}. Therefore, to accommodate the cable routings in Joint1, the decoupling mechanism in Joint2 and Joint3 need to decouple the driving cables distributed along their axial directions. 

In addition to the decoupling method that aligns the cables with the joint axes, another approach involves using rolling constraints to convert 1-DOF rotational joints into rolling joints. Under rolling contact, the motion of the joint does not affect the length of the cable passing through it. The cable routings under the rolling joint are shown in Fig. \ref{fig:rolling}(a) and Fig. \ref{fig:rolling}(b). Link0 and Link1 are articulated through an intermediate link and are subjected to rolling constraints by two linkage cables. The relationship between the cable motion and joint angle is shown in Fig. \ref{fig:rolling}(c), where the the solid and dashed yellow line represent the driving cables for the rolling joint. Based on the rolling constraint, the intermediate link and Link1 can move through the same angle $\theta_e/2$. Finally, the angle of Link1 relative to Link0 is related to the length of the driving cables $\Delta l_1$ by the following relationship:
\begin{equation}
\Delta l_1 = \Delta l_{1,1} + \Delta l_{1,2} = \frac{R\theta_e}{2}
\end{equation}
Similarly, the length change of another driving cable $l_2$ can also be determined as
\begin{equation}
\Delta l_2 = \Delta l_1 = \frac{R\theta_e}{2}
\end{equation}
When the rolling joint is in motion, the length changes of the driving cables for the remaining joints are illustrated in Fig. \ref{fig:rolling}(d). Since these cables can pass through the rolling contact point at any angle of joint movement, the length changes of the cable in the two pulleys always offset each other. Therefore, the length of the cables passing through this joint remains constant.

\begin{figure}[t]
    \centering
    \includegraphics[width=1\linewidth]{picture/Detailed_design_of_decoupled_rolling_pair_Mechanism.pdf}
    \caption{Detailed design of decoupled rolling pair Mechanism. (a) cable routings in joint motion. (b) cable routings in decoupling way. (c) Relationship between the cable motion and joint angle. (d) Cable displacement during joint movement.}
    \label{fig:rolling}
    \vspace{-0.5cm}
\end{figure}

\begin{comment}
\begin{figure}[t]
    \centering
    \subfigure[]{
        \includegraphics[width=0.21\textwidth]{picture/a-Detailed design of decoupled rolling pair Mechanism.png}
        \label{fig:a-rolling}
    }
    \subfigure[]{
        \includegraphics[width=0.21\textwidth]{picture/b-Detailed design of decoupled rolling pair Mechanism.png}
        \label{fig:b-rolling}
    }
    \subfigure[]{
        \includegraphics[width=0.21\textwidth]{picture/c-Detailed design of decoupled rolling pair Mechanism.png}
        \label{fig:c-rolling}
    }
    \subfigure[]{
        \includegraphics[width=0.21\textwidth]{picture/d-Detailed design of decoupled rolling pair Mechanism.png}
        \label{fig:d-rolling}
    }
    \caption{Detailed design of decoupled rolling pair Mechanism. (a) cable routings in joint motion. (b) cable routings in decoupling. (c) Relationship between the cable motion and joint angle. (d) Cable displacement during joint movement.}
    \label{fig:main}
    \vspace{-0.4cm}
\end{figure}    
\end{comment}


Similar to the advantages of the decoupling cable aligner mechanism, this decoupling mechanism enables multi-cable decoupling through axial expansion. We arrange this decoupling mechanism within Joint2 and Joint3 to decouple the 6 driving cables of the 3-DOF quaternion joint. The routing of the cables for each joint on the cable-driven arm is illustrated in the Front view of Fig. \ref{fig:Proportion}.

\subsection{Cable-pretension Module Mechanism}

The proposed decoupling mechanisms are all composed of pulleys to reduce frictional resistance during cable movement; however, this also increases the risk of cable being slack and coming off the pulleys\cite{8593679,9349131}. Additionally, as the motor drives the roller, the cable's axial position shifts, causing it to move out of the groove of pulleys.

To address these issues, we design a cable-pretension mechanism within the drive box, as shown in Fig. \ref{fig:cable-pretension}. This mechanism provides pre-tension force through bolts and uses groove bearings that can rotate passively to guide the driving cables from the roller to the fixed pulleys of the decoupling mechanism in Joint1. Each driving cable is equipped with this cable-pretension mechanism and applied the same pre-tension force by utilizing an external force sensor.

\begin{figure}[t]
    \centering
    \includegraphics[width=1\linewidth]{picture/Detailed_design_of_cable-pretension_mechanism.pdf}
    \caption{Detailed design of cable-pretension mechanism in the drive box. It can rotate around its rotation axis to draw the cable wrapped around the roller towards Joint1 and apply pre-tension force in accordance with the threaded direction.}
    \label{fig:cable-pretension}
    \vspace{-0.5cm}
\end{figure}

\subsection{Overall Configuration}
Based on the above design, the primary specification of D3-Arm are presented in Table. \ref{tab:Specifications}. D3-Arm's length is 1.07 m with a approximately 1.6 kg weight moving part. To minimize the overall volume of the arm, a 1 mm steel wire rope is chosen as the driving cable. Based on the design principles, it can be inferred that selecting a steel cable of greater diameter would effectively enhance the stiffness and load-bearing capacity of the entire arm.

\begin{table}[ht]
    \centering
    \caption{Specifications of D3-Arm}
    \begin{tabular}{ccc} % 表格的列格式:3列,每列居中
        \toprule % 顶部横线
        \multicolumn{2}{c}{Items} & Value \\ % 表头
        \midrule % 中间横线
        \multirow{2}{*}{Mass} & Overall arm & 16.5 kg\\ 
                              & Moving part & 1.6 kg \\ % 第二行
        \midrule % 中间横线
        \multirow{6}{*}{Range of motion} & Joint1 & -60$^{\circ}$ $\sim$ 60$^{\circ}$\\ 
                                         & Joint2 & -130$^{\circ}$ $\sim$ 130$^{\circ}$ \\ 
                                         & Joint3 & -180$^{\circ}$ $\sim$ 180$^{\circ}$ \\
                                         & Joint4 & -90$^{\circ}$ $\sim$ 90$^{\circ}$ \\
                                         & Joint5 & -90$^{\circ}$ $\sim$ 90$^{\circ}$ \\
                                         & Joint6 & -180$^{\circ}$ $\sim$ 180$^{\circ}$ \\
                                         
        \midrule % 中间横线
        \multirow{4}{*}{Driving cables} & Type of cable & 1mm steel wire rope\\ 
                              & Maximum tension & 900 N \\ % 第二行
                              & Pretension force & 120 N \\ 
                              &Young's modulus & 100 Gpa\\
        \midrule % 中间横线
        \multirow{3}{*}{HT-04 Motor} & Nominal torque & 13 Nm\\ 
                              & Nominal speed & 300 rpm \\ % 第二行
                              & Encoder resolution & 0.087$^{\circ}$ \\ 
        \bottomrule % 底部横线        
    \end{tabular}
    \label{tab:Specifications} % 表格标签
\end{table}

%%%%%%%%%%%%%%%%%%%%%%%%%%%%%%%%%%%%%%%%%%%%%%%%%%%%%%%%%%%%%%%%%%%%%%%%%%%%%%%%
\section{Kinematics Model Analysis}
\subsection{Forward Kinematics}

The DH coordinate system of D3-Arm is established to the standard DH method, as shown in Fig. \ref{fig:Proportion}. Due to the implementation of rolling constraints, we utilize 10 equivalent joints to compute the forward kinematics of the 6-DOF D3-Arm with four constrain equations as followed
\begin{equation}
    \begin{cases}
    \Delta \theta_2 = \Delta \theta_3, \\
    \Delta \theta_4 = \Delta \theta_5, \\
    \Delta \theta_6 = -\Delta \theta_9, \\
    \Delta \theta_7 = \Delta \theta_8
    \end{cases}
\end{equation}

Then, the forward kinematics of the 6-DOF D3-Arm can be described as
\begin{equation}
f(\theta) = \mathbf{T}_2^1 \mathbf{T}_3^2 \mathbf{T}_5^4 \cdots \mathbf{T}_{10}^{9}
\label{eq:Fk}
\end{equation}
where $\mathbf{T}_i^{i-1} (2\leq i\leq 10)$ is the local homogeneous transformation matrices between adjacent coordinate systems and can be derived from local transformation $\mathbf{R}_i^{i-1}$ and position $\mathbf{p}_i^{i-1}$
\[
\mathbf{T}_i^{i-1} = 
\begin{bmatrix}
\mathbf{R}_i^{i-1} & \mathbf{p}_i^{i-1} \\
0 & 1 
\end{bmatrix}
\]

\subsection{Inverse Kinematics}
Having 10 joints variables, 6 independent DOFs and 4 constraint equations, D3-Arm can solve the inverse kinematics by substituting an improved Jacobian matrix $\mathbf{J}_{IM}$ into the least squares method based on the pseudo-inverse of the Jacobian matrix. Based on (\ref{eq:Fk}), the Jacobian matrix of D3-Arm can be derived and written as
\begin{equation}
\mathbf{J} = 
\begin{bmatrix}
\mathbf{v}_1  & \mathbf{v}_2 & \cdots & \mathbf{v}_{10} \\
\omega_1  & \omega_2 & \cdots & \omega_{10} \\

\end{bmatrix}
\quad \in \boldsymbol{R}^{6 \times 10}
\end{equation}

According to the constraint equations, the relationship between the 10 joint angle rates $\boldsymbol{\dot{\theta}}_{10 \times 1}$ of the D3-Arm and the 6 independent joint angle rates $\boldsymbol{\dot{\theta}}_{6 \times 1}^{\prime}$ can be expressed as follows
\begin{equation}
    \boldsymbol{\dot{\theta}}_{10 \times 1} = \mathbf{U}^{\top} \boldsymbol{\dot{\theta}}_{6\times1}^{\prime}\\
\end{equation}

\begin{equation}
    \mathbf{U} = 
    \begin{bmatrix}
    1 & 0 & 0 & 0 & 0 & 0 & 0 & 0 & 0 & 0\\
    0 & 1 & 1 & 0 & 0 & 0 & 0 & 0 & 0 & 0\\
    0 & 0 & 0 & 1 & 1 & 0 & 0 & 0 & 0 & 0\\
    0 & 0 & 0 & 0 & 0 & 1 & 0 & 0 & -1 & 0\\
    0 & 0 & 0 & 0 & 0 & 0 & 1 & 1 & 0 & 0\\
    0 & 0 & 0 & 0 & 0 & 0 & 0 & 0 & 0 & 1\\    
    \end{bmatrix}
\end{equation}
where $\boldsymbol{\dot{\theta}}_{10 \times 1} = \begin{bmatrix} \dot{\theta_1} & \dot{\theta_2} & \cdots & \dot{\theta}_{10} \end{bmatrix}$ and $\boldsymbol{\dot{\theta}}_{6 \times 1}^{\prime} = \begin{bmatrix} \dot{\theta_1}&\dot{\theta_2}&\dot{\theta_4}&\dot{\theta_6} &\dot{\theta_7}&\dot{\theta}_{10}\end{bmatrix}$. Then, the inverse kinematics can be calculated as
\begin{equation}
    \mathbf{\dot{X}} = \mathbf{J}\boldsymbol{\dot{\theta}}_{10 \times 1} = \mathbf{J}\mathbf{U}^{\top}  \boldsymbol{\dot{\theta}}_{6 \times1 }^{\prime} = \mathbf{J}_{IM}\boldsymbol{\dot{\theta}}_{6 \times 1}^{\prime}
\end{equation}
\begin{equation}
    \boldsymbol{\dot{\theta}}_{6 \times 1}^{\prime} = \mathbf{J}_{IM}^{-1}\mathbf{\dot{X}} 
\end{equation}
where $\mathbf{\dot{X}}$ represent the rate of change of the end-effector.

\begin{comment}
\subsection{Workspace Analysis}
Based on the kinematics model established above, the workspace of D3-Arm can be calculated based on the Monte-Carlo method as
\begin{equation}
    \mathbf{W} = \mathbf{X}(\Tilde{\theta}_i)
    \begin{cases}
    \Tilde{\theta}_i^{min} \leq \Tilde{\theta}_i \leq \Tilde{\theta}_i^{max}, \\
    i = 1,2,4,6,7,10 \\
    \end{cases}
\end{equation}


\begin{figure}[t]
    \centering
    \includegraphics[width=1\linewidth]{picture/Workspace of D3-arm.png}
    \caption{Workspace of D3-Arm. (a) Axis side view. (b) XY-plane view. (c) YZ-plane view. (d) XZ-plane view.}
    \label{fig:Workspace}
    \vspace{-0.4cm}
\end{figure}    

Fig. \ref{fig:Workspace} shows the workspace of D3-Arm, which reaches a volume of 0.83 m$^3$.

\end{comment}
    

\begin{comment}
\begin{figure}[t]
    \centering
    \subfigure[]{
        \includegraphics[width=0.21\textwidth]{picture/a-Workspace of D3-Arm.png}
        \label{fig:a-Workspace}
    }
    \subfigure[]{
        \includegraphics[width=0.21\textwidth]{picture/b-Workspace of D3-Arm.png}
        \label{fig:b-Workspace}
    }
    \subfigure[]{
        \includegraphics[width=0.21\textwidth]{picture/c-Workspace of D3-Arm.png}
        \label{fig:c-Workspace}
    }
    \subfigure[]{
        \includegraphics[width=0.21\textwidth]{picture/d-Workspace of D3-Arm.png}
        \label{fig:d-Workspace}
    }
    \caption{Workspace of D3-Arm. (a) Axis side view. (b) XY-plane view. (c) YZ-plane view. (d) XZ-plane view.}
    \label{fig:Workspace}
\end{figure}    
\end{comment}    



%%%%%%%%%%%%%%%%%%%%%%%%%%%%%%%%%%%%%%%%%%%%%%%%%%%%%%%%%%%%%%%%%%%%%%%%%%%%%%%%
\section{EXPERIMENT AND RESULTS}
To verify the performance of the proposed D3-Arm, a prototype is built to investigate its position repeatability, load capacity and dynamic motion. In the experiment, we used the motion capture system(V120 Trio, Optitrack) to measure the pose and velocity of the end effector of D3-Arm at 120Hz.

\subsection{Decoupling Verification}
To validate the effectiveness of the proposed decoupling mechanism, an experiment is conducted as shown in Fig.\ref{fig:Decoupled}. We remove one pair of driving cables of the end-effector from the motor's roller and connected them to a Push-pull gauge fixed to the base through the cable pre-tension mechanism, which applies a tension of 82 N. During the experiment, external forces are applied to move Joint1, Joint2, and Joint3 to simulate interference on the joints, and the variations in cable tension are recorded from the Push-pull gauge, as shown in Fig. \ref{fig:Decoupled}(b). The experimental results are shown in Fig. \ref{fig:Decoupled}(c). The maximum variation in tension on the cable is 1 N. Based on the Young's modulus illustrated from Table. \ref{tab:Specifications}, the corresponding length change does not exceed 0.01 mm, demonstrating that the motion interference of the joints integrated with decoupling mechanism can be considered negligible. This enables high-precision position control among the joints.

\begin{figure}[t]
    \centering
    \includegraphics[width=1\linewidth]{picture/Decoupled.pdf}
    \caption{Decoupling verification of D3-Arm. (a) Experimental setup for force measurement. (b) Snapshots of the process of applying external force to Joint1, 2, 3. (c) The variation of cable tension measured by two Push-pull gauges. Measurement accuracy is 0.1 N.}
    \label{fig:Decoupled}
    \vspace{-0.5cm}
\end{figure}

\subsection{Position Repeatability}
 According to the performance criteria and test methods of ISO 9283\cite{iso1998}, the repeatability test of D3-Arm is conducted with closed-loop position control at every motor. Considering that the application scenarios of this cable-driven robotic arm, whose motors and actuators are all mounted at the base, mostly do not allow for the installation of external observation sensors, the joint angle information is not included in the closed-loop control during this experiment.
 
 \begin{figure}[h]
    \centering
    \subfigure[]{
        \includegraphics[width=0.18\textwidth]{picture/Repeatability_test.pdf}
        \label{fig:scatter1}
    }
    \subfigure[]{
        \includegraphics[width=0.26\textwidth]{picture/Point_cloud.pdf}
        \label{fig:scatter2}
    }
    \caption{Repeatability test based on ISO 9283:1998. (a) 5 pre-defined target positions(P$_1$,P$_2$,P$_3$,P$_4$,P$_5$) for the test. D3-Arm successively moves between these 5 positions and repeats for 30 times. (b) Repeatability test results at 5 pre-defined target positions. Measurement accuracy is $\pm$ 0.2 mm.}
    \label{fig:repeatability}
    \vspace{-0.2cm}
\end{figure}

 The position repeatability test of D3-Arm's end-effector is measured at 5 pre-defined positions as shown in Fig. \ref{fig:scatter1}. D3-Arm successively moves to the poses P$_1$, P$_2$, P$_3$, P$_4$, P$_5$ and repeats for 30 times. The measurement point cloud results at these points are shown in Fig. \ref{fig:scatter2} and the average distance (Mean), its standard deviation (STD.DEV) and 3-sigma distance between the mean and recorded positions were summarized in Table \ref{tab:rp}. The average position error of D3-Arm is 1.2896mm, which is lower than the 4.9mm of Twist Snake\cite{snake} and other cable-driven arms with all the motors located at the base\cite{low-cost}, proving the effectiveness of decoupling mechanisms in improving control accuracy performance. 

\begin{table}[h]
    \centering
    \caption{Positioning repeatability test results}
    \begin{tabular}{cccc} % 这里定义四列
        \toprule % 表格顶部的横线
        Pose & Mean(mm) & STD.DEV.(mm) & 3-SIGMA(mm) \\ % 第一行(标题)
        \midrule % 表格中间的横线
        $P_1$ & 0.8562 & 0.4902 & 2.3269 \\ % 第1行数据
        $P_2$ & 1.9868 & 1.1983 & 5.5817 \\ % 第2行数据
        $P_3$ & 1.0997 & 0.7271 & 3.2810 \\ % 第3行数据
        $P_4$ & 1.5977 & 1.0693 & 4.8056 \\ % 第4行数据
        $P_5$ & 0.9078 & 0.6749 & 2.9324 \\ % 第5行数据
        \midrule % 表格中间的横线
        Total & 1.2896 & 0.8320 & 3.7855 \\ % 第6行数据
        \bottomrule % 表格底部的横线
    \end{tabular}
    \label{tab:rp}
\end{table}

 The 3-sigma distance, which corresponds to the positioning repeatability of ISO 9283, is 3.7855 mm. Due to the results of decoupling verification, we analyze that the error mainly comes from the lack of the joint sensors and the elasticity of cables, which reaches 1041mm length for the end-effector. 

\subsection{Load Capacity}
The load capacity is one of the important performance indicators of a robotic arm; however, it is noteworthy that the design objective of this cable-driven arm does not prioritize high load capacity. The load test is mainly conducted on Joint1 since the the load on the entire arm is mainly concentrated on Joint1.

During the testing process, we hang weights ranging from 1.0 kg to 2.0 kg on the end effector and drove the Joint1 to move from -50$^{\circ}$ to 0$^{\circ}$. The sequential images of the test is presented in Fig. \ref{fig:load}. From the test, it can be seen that D3-Arm can smoothly carry a weight of up to 2.0 kg. The cables began to detached from the cable-termination-mechanism under a heavier weight, which limits the load-bearing capacity of D3-Arm. Notably, as the weight increases, the end-effector of D3-Arm begins to exhibit significant deformation as shown in Fig. \ref{fig:load}(a-h), demonstrating the low stiffness of the robotic arm. This drawback is attributed to the elasticity of the long-distance and thin cables, which can be improved by incorporating a variable stiffness mechanism. 
\begin{figure}[t]
    \centering
    \includegraphics[width=1\linewidth]{picture/Load.pdf}
    \caption{Load capacity experiment of D3-Arm. (a-d) The payload tests where Joint1 bears loads of 0.5-2.0 kg at angle of -50$^{\circ}$. (e-h) The payload tests where Joint1 bears loads of 0.5-2.0 kg at angle of 0$^{\circ}$.}
    \label{fig:load}
    \vspace{-0.3cm}
\end{figure} 
\subsection{Dynamic Motion}
To evaluate the high-speed motion capabilities of the D3-Arm, we design a high-speed swinging experiment using the whole arm. Each joint of the D3-Arm is controlled using PID position control and moves along the predefined trajectories within the joint space. The sequential images and end-effector velocity, which is measured by the motion capture system, are illustrated in Fig. \ref{fig:high-dynamic}. In the dynamic motion experiment, the D3-Arm executed rapid swings within its workspace, reaching a maximum movement speed of 1.47 m/s and an acceleration of up to 10.55 m/s$^2$.

\begin{figure}[t]
    \centering
    \includegraphics[width=1\linewidth]{picture/high-dynamic.pdf}
    \caption{High speed motion test. (a) Snapshots of the test motion. The white dashed arrow indicates the direction of D3-Arm's movement. (b) Speed graphs of the motion.}
    \label{fig:high-dynamic}
    \vspace{-0.5cm}
\end{figure}    

   
\begin{comment}
\begin{figure}[t]
    \centering
    \subfigure[]{
        \includegraphics[width=0.22\textwidth]{picture/a-high-dynamic.png}
        \label{fig:a-high-dynamic}
    }
    \subfigure[]{
        \includegraphics[width=0.22\textwidth]{picture/b-high-dynamic.png}
        \label{fig:b-high-dynamic}
    }
    \caption{High speed motion test. (a) Snapshots of the test motion. The white dashed arrow indicates the direction of D3-Arm's movement. (b) Speed graphs of the motion.}
    \label{fig:high-dynamic}
    \vspace{-0.2cm}
\end{figure}    
\end{comment}


%%%%%%%%%%%%%%%%%%%%%%%%%%%%%%%%%%%%%%%%%%%%%%%%%%%%%%%%%%%%%%%%%%%%%%%%%%%%%%%%
\section{CONCLUSIONS}
In this paper, we develop a fully-decoupled and lightweight cable-driven robotic arm system named D3-Arm with all its electrical components such as motors placed at the base. The cables are aligned and transmitted efficiently through decoupling mechanisms composed by grooved bearings and pulleys, significantly enhancing the control precision compared to other cable-driven robotic arms with all the motors positioned at the base. By integrating the remote driving feature of cables and fully-decoupled transmission mechanisms, the D3-Arm is capable of performing tasks in various environments including underwater and high-radiation that require isolation and protection of electrical components. In future work, we plan to incorporate the variations of cable tension into the control of the entire arm to further improve the control accuracy and performance of D3-Arm.

\addtolength{\textheight}{-5cm}   % This command serves to balance the column lengths
                                  % on the last page of the document manually. It shortens
                                  % the textheight of the last page by a suitable amount.
                                  % This command does not take effect until the next page
                                  % so it should come on the page before the last. Make
                                  % sure that you do not shorten the textheight too much.

%%%%%%%%%%%%%%%%%%%%%%%%%%%%%%%%%%%%%%%%%%%%%%%%%%%%%%%%%%%%%%%%%%%%%%%%%%%%%%%%



%%%%%%%%%%%%%%%%%%%%%%%%%%%%%%%%%%%%%%%%%%%%%%%%%%%%%%%%%%%%%%%%%%%%%%%%%%%%%%%%



%%%%%%%%%%%%%%%%%%%%%%%%%%%%%%%%%%%%%%%%%%%%%%%%%%%%%%%%%%%%%%%%%%%%%%%%%%%%%%%%
\newpage
\bibliographystyle{ieeetr}
\bibliography{main} 



\end{document}
