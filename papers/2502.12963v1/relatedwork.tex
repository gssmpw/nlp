\section{RELATED WORKS}
% 


%\subsection{Cable-driven Robotic Arm}
% wam lims等,体现出绳索后置对绳驱臂性能提升好的好处,引出解耦装置的必要性
%As one of the earliest commercially applied cable-driven robotic arm, WAM employed low ratio cable transmission and comprised cable differential transmission systems\cite{wam}. Kim developed an anthropomorphic and low-inertia cable-driven manipulator using tension-amplifying mechanism, which can decouple the motion between the elbow joint and wrist joint\cite{kim}. With that decouple mechanism, all the joints of this manipulator are remotely driven by the motors placed at the shoulder joint. Shunichi Sakurai \textit{et al.} designed a 4-DOF tendon-driven manipulator with constant tendon-length routing and mounted all its motors at the base. Temma Suzuki \textit{et al.} introduced a low-friction cable-driven manipulator using a passive 3D wire aligner, which is composed by bearings and grooved pulleys\cite{SAQIEL}. Both of the above manipulators have achieved extreme lightness of moving part by placing the motors and actuators at the base, but their motion accuracy and dynamic performance are still limited by coupled cable-driven system.

%\subsection{decoupling mechanism}
% RoboRay hand: A highly backdrivable robotic hand with sensorless contact force measurements
% 轴线解耦装置
% Towards simplicity: On the design of a 2-DOFs wrist mechanism for tendon-driven robotic hands
% Design and Evaluation of a Cable-Actuated Palletizing Robot With Geared Rolling Joints
%In order to solve the coupling problem of cable-driven joints and let motors to be separated from the joints without affecting performance, various decoupling mechanism have been developed. Wontae Choi \textit{et al.} design a 4-DOF cable-actuated robot with geared rolling joints arranged by series\cite{choi2024design}. However, this single decoupling routing mechanism limits the dexterity of the robot. Jong Kwang Lee \textit{et al.} designed a tendon-driven servo-manipulator with a novel motion decoupling mechanism\cite{lee2008design}. They used moving pulleys and a decoupling link to successfully compensate the length variation of the cables. Surong Jiang \textit{et al.} designed a novel motion-decoupling modular joint to effectively solve the motion-coupling, hysteresis and the dead zone problems in cable-driven motion\cite{jiang2018design}. They used the motion of following wheel and decoupling cable to compensate the coupling length of the cables. But those decoupling mechanisms require a relatively large bulk, which affects the working space of the robotic arm. 

%Unlike the decoupled strategies mentioned above which is focus on the joint motion relative to the radial direction of the cables, there are other strategies that achieve axial decoupled motions. RoboRay Hand used grooved pulleys fixed with the joint to align the cables on it be collinear with the axis of the joint at all angle\cite{RoboRay_hand}. By combing this routing mechanism and rolling motion, RoboRay Hand implemented a 2-DOF cable-driven decoupling mechanism. Divya Shah presented a constant-length tendon routing mechanism for an axial joint that achieved decoupled motions\cite{constant_length}. Based on the idea of moving pulleys, this mechanism can allow tendon routing of all four wrist actuation tendons. 



%%%%%%%%%%%%%%%%%%%%%%%%%%%%%%%%%%%%%%%%%%%%%%%%%%%%%%%%%%%%%%%%%%%%%%%%%%%%%%%%