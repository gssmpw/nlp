

\subsection{Use case example}

The \gls{hpo} experiment run by using the scripts the code boxes above refer to results, at the time of writing, in the state-of-the-art classification of Braille characters from the reduced dataset used in~\cite{pedersen_neuromorphic_2024}. Figure~\ref{fig:par_coord} summarizes the exploration carried out across the search space defined in Code~\ref{code:HPO_conf}, reporting both the best training accuracy and the best validation accuracy achieved during the learning phase of each trial.
In the rightmost part, the test accuracy is reported.
As it is reported in Code~\ref{code:HPO_train}, the value for the \texttt{default} key in the dictionary given to \texttt{nni.report\_final\_result()}, namely the objective metrics for optimization with this experiment, is set to be the validation accuracy; particularly, the best validation accuracy achieved throughout the training epochs.\\
From Code~\ref{code:HPO_train}, it is also possible to track down the selection criterion for the optimal model.
At the end of the training stage, the weights from the highest validation accuracy are loaded, and test is performed.
The resulting accuracy is passed to \texttt{nni.report\_final\_result()} as value for the key \texttt{test}, and the best test accuracy at the end of the \gls{hpo} experiment will identify what combination of hyperparameters is the optimal one for the model under optimization in the selected task.\\
In Figure~\ref{fig:cm}, the confusion matrix produced on the test set by the optimized \gls{snn} is shown, with partial misclassification in two classes only and an overall accuracy of 97.14\%.

\begin{figure}[b]
    \centering
    \includegraphics[width=\textwidth]{Figures/parallel_coordinates.pdf}
    \caption{Exploration of the search space with the resulting test accuracy for each combination of hyperparameters}
    \label{fig:par_coord}
\end{figure}

\begin{figure}[t]
    \centering
    \includegraphics[width=0.5\textwidth]{Figures/cm.pdf}
    \caption{Confusion matrix produced on the test set by the optimal model. The overall accuracy is 97.14\%.}
    \label{fig:cm}
\end{figure}


\subsection{Published works}

The application-oriented automatic \gls{hpo} procedure described in this document is the result of ongoing efforts that lead to continuous refinement and customization of the pipeline initially proposed in~\cite{fra_human_2022}.
Its adaptability, rooted in the wide range of possibilities offered by \gls{nni}, is at the same time the key feature for its employment and the driving force for its never-ending development. In Table~\ref{table:works}, a summary of the published works that use it for spiking models is reported.

\begin{table*}[h]
    \renewcommand{\arraystretch}{1.15}
    \centering
    \caption{Summary of published works that performed application-oriented automatic \gls{hpo} through \gls{nni} based on the procedure presented here}
    \label{table:works}
    % \begin{tabular}{{|>{\centering}m{1.5cm}|>{\centering}m{2.7cm}|>{\centering}m{1.4cm}|>{\centering}m{2.1cm}|>{\centering}m{1.9cm}|>{\centering\arraybackslash}m{1.8cm}|}}
    %     \hline
    %     { Neuron model} & { Device } & { Used RAM } & { Mean inference time } & { Mean energy per inference } & { Accuracy } \\
    %     \hline
    %     \multirow{3.1}{*}{ \texttt{Leaky} } & { STM32MP1 } & { 65.7 MB } & { 0.13 s } & { 215.1 mJ } & \multirow{3.3}{*}{ 93.91\% } \\
    %     \cline{2-5}
    %     {  } & { Raspberry Pi 3B+ } & { 77.8 MB } & { 0.06 s } & { 268.8 mJ } & {  } \\
    %     \cline{2-5}
    %     {  } & { Raspberry Pi 4B } & { 77.4 MB } & { 0.03 s } & { 153.9 mJ } & {  } \\
    %     \hline
    %     \multirow{3.1}{*}{ \texttt{Synaptic} } & { STM32MP1 } & { 167.9 MB } & { 0.22 s } & { 383.4 mJ } & \multirow{3.3}{*}{ 93.84\% } \\
    %     \cline{2-5}
    %     {  } & { Raspberry Pi 3B+ } & { 187.5 MB } & { 0.15 s } & { 727.5 mJ } & {  } \\
    %     \cline{2-5}
    %     {  } & { Raspberry Pi 4B } & { 187.4 MB } & { 0.07 s } & { 348.9 mJ } & {  } \\
    %     \hline
    % \end{tabular}
    \begin{tabular}{{|>{\centering}m{2cm}|>{\centering}m{2.7cm}|>{\centering}m{2.5cm}|>{\centering}m{1.9cm}|>{\centering}m{2cm}|>{\centering\arraybackslash}m{1.8cm}|}}
        \hline
        { Reference } & { Task } & { Architecture } & { Event/Frame data } & { Dataset } & {Framework} \\
        \hline
        { \cite{fra_human_2022} } & { Human activity recognition } & { LMU } & { Frame } & { \cite{Weiss2019a,Weiss2019} } & { \texttt{TensorFlow} } \\
        \hline
        { \cite{muller-cleve_braille_2022} } & { Braille letter reading } & { Fully connected } & { Both } & { \cite{muller-cleve_tactile_2022} } & { \texttt{PyTorch} } \\
        \hline
        { \cite{pedersen_neuromorphic_2024} } & { Braille letter reading } & { Fully connected } & { Event } & { \href{https://github.com/neuromorphs/NIR/tree/main/paper/03_rnn/data}{Braille subset for \cite{pedersen_neuromorphic_2024}} } & { \texttt{snnTorch} } \\
        \hline
        { \cite{wand_natively_2024} } & { Human activity recognition } & { L$^2$MU } & { Frame } & { \cite{Weiss2019a,Weiss2019} } & { \texttt{snnTorch} } \\
        \hline
        { \cite{meo_neu-brauer_2025} } & { Braille letter reading } & { Fully connected } & { Frame } & { \cite{muller-cleve_tactile_2022} } & { \texttt{snnTorch} } \\
        \hline
        { \cite{fra_win-gui_2025} } & { Spike pattern classification } & { Fully connected } & { Event } & { Spike patterns from \cite{Mihalas2009} } & { \texttt{snnTorch} } \\
        \hline
        { [NICE2025] } & { Braille letter reading } & { L$^2$MU } & { Event } & { \cite{muller-cleve_tactile_2022} and \href{https://github.com/neuromorphs/NIR/tree/main/paper/03_rnn/data}{Braille subset for \cite{pedersen_neuromorphic_2024}} } & { \texttt{snnTorch} } \\
        \hline
    \end{tabular}
\end{table*}
