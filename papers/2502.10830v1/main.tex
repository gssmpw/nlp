%%
%% This is file `sample-manuscript.tex',
%% generated with the docstrip utility.
%%
%% The original source files were:
%%
%% samples.dtx  (with options: `manuscript')
%% 
%% IMPORTANT NOTICE:
%% 
%% For the copyright see the source file.
%% 
%% Any modified versions of this file must be renamed
%% with new filenames distinct from sample-manuscript.tex.
%% 
%% For distribution of the original source see the terms
%% for copying and modification in the file samples.dtx.
%% 
%% This generated file may be distributed as long as the
%% original source files, as listed above, are part of the
%% same distribution. (The sources need not necessarily be
%% in the same archive or directory.)
%%
%% Commands for TeXCount
%TC:macro \cite [option:text,text]
%TC:macro \citep [option:text,text]
%TC:macro \citet [option:text,text]
%TC:envir table 0 1
%TC:envir table* 0 1
%TC:envir tabular [ignore] word
%TC:envir displaymath 0 word
%TC:envir math 0 word
%TC:envir comment 0 0
%%
%%
%% The first command in your LaTeX source must be the \documentclass command.
%%%% Small single column format, used for CIE, CSUR, DTRAP, JACM, JDIQ, JEA, JERIC, JETC, PACMCGIT, TAAS, TACCESS, TACO, TALG, TALLIP (formerly TALIP), TCPS, TDSCI, TEAC, TECS, TELO, THRI, TIIS, TIOT, TISSEC, TIST, TKDD, TMIS, TOCE, TOCHI, TOCL, TOCS, TOCT, TODAES, TODS, TOIS, TOIT, TOMACS, TOMM (formerly TOMCCAP), TOMPECS, TOMS, TOPC, TOPLAS, TOPS, TOS, TOSEM, TOSN, TQC, TRETS, TSAS, TSC, TSLP, TWEB.
% \documentclass[acmsmall]{acmart}

%%%% Large single column format, used for IMWUT, JOCCH, PACMPL, POMACS, TAP, PACMHCI
% \documentclass[acmlarge,screen]{acmart}

%%%% Large double column format, used for TOG
% \documentclass[acmtog, authorversion]{acmart}

%%%% Generic manuscript mode, required for submission
%%%% and peer review
 %%\documentclass[manuscript,review,anonymous]{acmart}
 \documentclass[sigconf]{acmart}

%\documentclass[sigconf,authordraft]{acmart}

%%\documentclass[manuscript]{acmart}
\usepackage{subfig}
\usepackage{graphicx}
\usepackage{subcaption}
\newcommand{\theDevice}{SpellRing}

%% Fonts used in the template cannot be substituted; margin 
%% adjustments are not allowed.
%%
%% \BibTeX command to typeset BibTeX logo in the docs
\AtBeginDocument{%
  \providecommand\BibTeX{{%
    \normalfont B\kern-0.5em{\scshape i\kern-0.25em b}\kern-0.8em\TeX}}}

%% Rights management information.  This information is sent to you
%% when you complete the rights form.  These commands have SAMPLE
%% values in them; it is your responsibility as an author to replace
%% the commands and values with those provided to you when you
%% complete the rights form.
\copyrightyear{2025}
\acmYear{2025}
\setcopyright{acmlicensed}\acmConference[CHI '25]{CHI Conference on Human Factors in Computing Systems}{April 26-May 1, 2025}{Yokohama, Japan}
\acmBooktitle{CHI Conference on Human Factors in Computing Systems (CHI '25), April 26-May 1, 2025, Yokohama, Japan}
\acmDOI{10.1145/3706598.3713721}
\acmISBN{979-8-4007-1394-1/25/04}


%%
%% Submission ID.
%% Use this when submitting an article to a sponsored event. You'll
%% receive a unique submission ID from the organizers
%% of the event, and this ID should be used as the parameter to this command.
%%\acmSubmissionID{123-A56-BU3}

%%
%% For managing citations, it is recommended to use bibliography
%% files in BibTeX format.
%%
%% You can then either use BibTeX with the ACM-Reference-Format style,
%% or BibLaTeX with the acmnumeric or acmauthoryear sytles, that include
%% support for advanced citation of software artefact from the
%% biblatex-software package, also separately available on CTAN.
%%
%% Look at the sample-*-biblatex.tex files for templates showcasing
%% the biblatex styles.
%%

%%
%% The majority of ACM publications use numbered citations and
%% references.  The command \citestyle{authoryear} switches to the
%% "author year" style.
%%
%% If you are preparing content for an event
%% sponsored by ACM SIGGRAPH, you must use the "author year" style of
%% citations and references.
%% Uncommenting
%% the next command will enable that style.
%%\citestyle{acmauthoryear}

%%
%% end of the preamble, start of the body of the document source.
\begin{document}

%%
%% The "title" command has an optional parameter,
%% allowing the author to define a "short title" to be used in page headers.
\title{\theDevice: Recognizing Continuous Fingerspelling in American Sign Language using a Ring}

%
% The "author" command and its associated commands are used to define
% the authors and their affiliations.
% Of note is the shared affiliation of the first two authors, and the
% "authornote" and "authornotemark" commands
% used to denote shared contribution to the research.


\author{Hyunchul Lim, Nam Anh Dang, Dylan Lee, Tianhong Catherine Yu, Jane Lu, Franklin Mingzhe Li$^*$, Yiqi Jin, Yan Ma$^\dagger$, Xiaojun Bi$^\dagger$, François Guimbretière, and Cheng Zhang}
\email{{hl2365,nd433,dl634,ty274,jdl332,yj226,fvg3, chengzhang}@cornell.edu}
\email{mingzhe2@cs.cmu.edu, {yanma1,xiaojun}@cs.stonybrook.edu}

\orcid{0000-0001-8397-3534}
\affiliation{%
  \institution{Cornell University, Ithaca, New York, USA \\ Carnegie Mellon University$^*$, Pittsburgh, Pennsylvania, USA \\Stony Brook University$^\dagger$, Stony Brook, New York, USA}
   \country{ }}

% \affiliation{%
%   \institution{Carnegie Mellon University$^*$, Pittsburgh, Pennsylvania}
% }
% \affiliation{%
%   \institution{Stony Brook University$^\dagger$, Stony Brook, New York}
%

% \author{Hyunchul Lim}
% \email{{hl2365,nd433,dl634}@cornell.edu}
% \orcid{0000-0001-8397-3534}
% \author{Nam Anh Dang}
% %\email{nd433@cornell.edu}
% \orcid{0009-0005-8080-9268}
% \author{Dylan Lee}
% %\email{dl634@cornell.edu}
% \orcid{0009-0000-9382-468X}
% \affiliation{%
%   \institution{Cornell University}
%   \city{Ithaca}
%   \state{New York}
%   \country{USA}
%   \postcode{14850}
% }

% \author{Tianhong Catherine Yu}
% \email{{ty274,jdl332}@cornell.edu}
% \orcid{0000-0002-3742-0178}
% \author{Jane Lu}
% %\email{jdl332@cornell.edu}
% \orcid{0009-0000-9972-863X}
% %\author{Yiqi Jin}
% %\email{yj226@cornell.edu}
% \affiliation{%
%   \institution{Cornell University}
%   \city{Ithaca}
%   \state{New York}
%   \country{USA}
%   \postcode{14850}
% }

% \author{Franklin Mingzhe Li}
% \email{mingzhe2@cs.cmu.edu}
% \orcid{0000-0003-4995-4545}
% \affiliation{%
%   \institution{Carnegie Mellon University}
%   \city{Pittsburgh}
%   \state{Pennsylvania}
%   \country{USA}
% }

% \author{Yiqi Jin}
% \email{yj226@cornell.edu}
% \orcid{0009-0004-3727-5742}
% \affiliation{%
%   \institution{Cornell University}
%   \city{Ithaca}
%   \state{New York}
%   \country{USA}
%   \postcode{14850}
% }

% \author{Yan Ma}
% \email{{yanma1,xiaojun}@cs.stonybrook.edu}
% \orcid{0000-0001-8264-3103}
% \author{Xiaojun Bi}
% %\email{xiaojun@cs.stonybrook.edu}
% \orcid{0000-0002-9716-7709}
% \affiliation{%
%   \institution{Stony Brook University}
%   \city{Stony Brook}
%   \state{New York}
%   \country{USA}
% }
% \author{Francois Guimbretiere}
% \email{francois@cs.cornell.edu}
% \orcid{0000-0002-5510-6799}
% \author{Cheng Zhang}
% \email{chengzhang@cornell.edu}
% \orcid{0000-0002-5079-5927}
% \affiliation{%
%   \institution{Cornell University}
%   \city{Ithaca}
%   \state{New York}
%   \country{USA}
%   \postcode{14850}
% }
%
% By default, the full list of authors will be used in the page
% headers. Often, this list is too long, and will overlap
% other information printed in the page headers. This command allows
% the author to define a more concise list
% of authors' names for this purpose.

\renewcommand{\shortauthors}{Hyunchul Lim, et al.}

%%
%% The abstract is a short summary of the work to be presented in the
%% article.
 \begin{abstract}
  In this work, we present a novel technique for GPU-accelerated Boolean satisfiability (SAT) sampling. Unlike conventional sampling algorithms that directly operate on conjunctive normal form (CNF), our method transforms the logical constraints of SAT problems by factoring their CNF representations into simplified multi-level, multi-output Boolean functions. It then leverages gradient-based optimization to guide the search for a diverse set of valid solutions. Our method operates directly on the circuit structure of refactored SAT instances, reinterpreting the SAT problem as a supervised multi-output regression task. This differentiable technique enables independent bit-wise operations on each tensor element, allowing parallel execution of learning processes. As a result, we achieve GPU-accelerated sampling with significant runtime improvements ranging from $33.6\times$ to $523.6\times$ over state-of-the-art heuristic samplers. We demonstrate the superior performance of our sampling method through an extensive evaluation on $60$ instances from a public domain benchmark suite utilized in previous studies. 


  
  % Generating a wide range of diverse solutions to logical constraints is crucial in software and hardware testing, verification, and synthesis. These solutions can serve as inputs to test specific functionalities of a software program or as random stimuli in hardware modules. In software verification, techniques like fuzz testing and symbolic execution use this approach to identify bugs and vulnerabilities. In hardware verification, stimulus generation is particularly vital, forming the basis of constrained-random verification. While generating multiple solutions improves coverage and increases the chances of finding bugs, high-throughput sampling remains challenging, especially with complex constraints and refined coverage criteria. In this work, we present a novel technique that enables GPU-accelerated sampling, resulting in high-throughput generation of satisfying solutions to Boolean satisfiability (SAT) problems. Unlike conventional sampling algorithms that directly operate on conjunctive normal form (CNF), our method refines the logical constraints of SAT problems by transforming their CNF into simplified multi-level Boolean expressions. It then leverages gradient-based optimization to guide the search for a diverse set of valid solutions.
  % Our method specifically takes advantage of the circuit structure of refined SAT instances by using GD to learn valid solutions, reinterpreting the SAT problem as a supervised multi-output regression task. This differentiable technique enables independent bit-wise operations on each tensor element, allowing parallel execution of learning processes. As a result, we achieve GPU-accelerated sampling with significant runtime improvements ranging from $10\times$ to $1000\times$ over state-of-the-art heuristic samplers. Specifically, we demonstrate the superior performance of our sampling method through an extensive evaluation on $60$ instances from a public domain benchmark suite utilized in previous studies.

\end{abstract}

\begin{IEEEkeywords}
Boolean Satisfiability, Gradient Descent, Multi-level Circuits, Verification, and Testing.
\end{IEEEkeywords}

%%
%% The code below is generated by the tool at http://dl.acm.org/ccs.cfm.
%% Please copy and paste the code instead of the example below.
%%
% \begin{CCSXML}
% <ccs2012>
%  <concept>
%   <concept_id>10010520.10010553.10010562</concept_id>
%   <concept_desc>Computer systems organization~Embedded systems</concept_desc>
%   <concept_significance>500</concept_significance>
%  </concept>
%  <concept>
%   <concept_id>10010520.10010575.10010755</concept_id>
%   <concept_desc>Computer systems organization~Redundancy</concept_desc>
%   <concept_significance>300</concept_significance>
%  </concept>
%  <concept>
%   <concept_id>10010520.10010553.10010554</concept_id>
%   <concept_desc>Computer systems organization~Robotics</concept_desc>
%   <concept_significance>100</concept_significance>
%  </concept>
%  <concept>
%   <concept_id>10003033.10003083.10003095</concept_id>
%   <concept_desc>Networks~Network reliability</concept_desc>
%   <concept_significance>100</concept_significance>
%  </concept>
% </ccs2012>
% \end{CCSXML}

% \ccsdesc[500]{Computer systems organization~Embedded systems}
% \ccsdesc[300]{Computer systems organization~Redundancy}
% \ccsdesc{Computer systems organization~Robotics}
% \ccsdesc[100]{Networks~Network reliability}


\begin{CCSXML}
<ccs2012>
<concept>
<concept_id>10003120.10011738.10011775</concept_id>
<concept_desc>Human-centered computing~Accessibility technologies</concept_desc>
<concept_significance>500</concept_significance>
</concept>
<concept>
<concept_id>10003120.10003121.10003128.10011755</concept_id>
<concept_desc>Human-centered computing~Gestural input</concept_desc>
<concept_significance>500</concept_significance>
</concept>
</ccs2012>
\end{CCSXML}

\ccsdesc[500]{Human-centered computing~Accessibility technologies}
\ccsdesc[500]{Human-centered computing~Gestural input}


%%
%% Keywords. The author(s) should pick words that accurately describe
%% the work being presented. Separate the keywords with commas.
\keywords{datasets, neural networks, gaze detection, text tagging}

%% A "teaser" image appears between the author and affiliation
%% information and the body of the document, and typically spans the
%% page.


\begin{teaserfigure}
  \includegraphics[width=1\linewidth]{document/figures/main.png}
  \caption{\theDevice{} is a smart ring that uses an AI-powered sensing system designed to recognize continuous fingerspelling}
  \Description{}
  \label{fig:main}
\end{teaserfigure}

% \begin{teaserfigure}
%   \includegraphics[width=\textwidth]{document/figures/jake.png}
%   \caption{Seattle Mariners at Spring Training, 2010.}
%   \Description{Enjoying the baseball game from the third-base
%   seats. Ichiro Suzuki preparing to bat.}
%   \label{fig:teaser}
% \end{teaserfigure}

% \received{20 February 2007}
% \received[revised]{12 March 2009}
% \received[accepted]{5 June 2009}

%%
%% This command processes the author and affiliation and title
%% information and builds the first part of the formatted document.
\maketitle



\section{Introduction}
\label{section:introduction}

% redirection is unique and important in VR
Virtual Reality (VR) systems enable users to embody virtual avatars by mirroring their physical movements and aligning their perspective with virtual avatars' in real time. 
As the head-mounted displays (HMDs) block direct visual access to the physical world, users primarily rely on visual feedback from the virtual environment and integrate it with proprioceptive cues to control the avatar’s movements and interact within the VR space.
Since human perception is heavily influenced by visual input~\cite{gibson1933adaptation}, 
VR systems have the unique capability to control users' perception of the virtual environment and avatars by manipulating the visual information presented to them.
Leveraging this, various redirection techniques have been proposed to enable novel VR interactions, 
such as redirecting users' walking paths~\cite{razzaque2005redirected, suma2012impossible, steinicke2009estimation},
modifying reaching movements~\cite{gonzalez2022model, azmandian2016haptic, cheng2017sparse, feick2021visuo},
and conveying haptic information through visual feedback to create pseudo-haptic effects~\cite{samad2019pseudo, dominjon2005influence, lecuyer2009simulating}.
Such redirection techniques enable these interactions by manipulating the alignment between users' physical movements and their virtual avatar's actions.

% % what is hand/arm redirection, motivation of study arm-offset
% \change{\yj{i don't understand the purpose of this paragraph}
% These illusion-based techniques provide users with unique experiences in virtual environments that differ from the physical world yet maintain an immersive experience. 
% A key example is hand redirection, which shifts the virtual hand’s position away from the real hand as the user moves to enhance ergonomics during interaction~\cite{feuchtner2018ownershift, wentzel2020improving} and improve interaction performance~\cite{montano2017erg, poupyrev1996go}. 
% To increase the realism of virtual movements and strengthen the user’s sense of embodiment, hand redirection techniques often incorporate a complete virtual arm or full body alongside the redirected virtual hand, using inverse kinematics~\cite{hartfill2021analysis, ponton2024stretch} or adjustments to the virtual arm's movement as well~\cite{li2022modeling, feick2024impact}.
% }

% noticeability, motivation of predicting a probability, not a classification
However, these redirection techniques are most effective when the manipulation remains undetected~\cite{gonzalez2017model, li2022modeling}. 
If the redirection becomes too large, the user may not mitigate the conflict between the visual sensory input (redirected virtual movement) and their proprioception (actual physical movement), potentially leading to a loss of embodiment with the virtual avatar and making it difficult for the user to accurately control virtual movements to complete interaction tasks~\cite{li2022modeling, wentzel2020improving, feuchtner2018ownershift}. 
While proprioception is not absolute, users only have a general sense of their physical movements and the likelihood that they notice the redirection is probabilistic. 
This probability of detecting the redirection is referred to as \textbf{noticeability}~\cite{li2022modeling, zenner2024beyond, zenner2023detectability} and is typically estimated based on the frequency with which users detect the manipulation across multiple trials.

% version B
% Prior research has explored factors influencing the noticeability of redirected motion, including the redirection's magnitude~\cite{wentzel2020improving, poupyrev1996go}, direction~\cite{li2022modeling, feuchtner2018ownershift}, and the visual characteristics of the virtual avatar~\cite{ogawa2020effect, feick2024impact}.
% While these factors focus on the avatars, the surrounding virtual environment can also influence the users' behavior and in turn affect the noticeability of redirection.
% One such prominent external influence is through the visual channel - the users' visual attention is constantly distracted by complex visual effects and events in practical VR scenarios.
% Although some prior studies have explored how to leverage user blindness caused by visual distractions to redirect users' virtual hand~\cite{zenner2023detectability}, there remains a gap in understanding how to quantify the noticeability of redirection under visual distractions.

% visual stimuli and gaze behavior
Prior research has explored factors influencing the noticeability of redirected motion, including the redirection's magnitude~\cite{wentzel2020improving, poupyrev1996go}, direction~\cite{li2022modeling, feuchtner2018ownershift}, and the visual characteristics of the virtual avatar~\cite{ogawa2020effect, feick2024impact}.
While these factors focus on the avatars, the surrounding virtual environment can also influence the users' behavior and in turn affect the noticeability of redirection.
This, however, remains underexplored.
One such prominent external influence is through the visual channel - the users' visual attention is constantly distracted by complex visual effects and events in practical VR scenarios.
We thus want to investigate how \textbf{visual stimuli in the virtual environment} affect the noticeability of redirection.
With this, we hope to complement existing works that focus on avatars by incorporating environmental visual influences to enable more accurate control over the noticeability of redirected motions in practical VR scenarios.
% However, in realistic VR applications, the virtual environment often contains complex visual effects beyond the virtual avatar itself. 
% We argue that these visual effects can \textbf{distract users’ visual attention and thus affect the noticeability of redirection offsets}, while current research has yet taken into account.
% For instance, in a VR boxing scenario, a user’s visual attention is likely focused on their opponent rather than on their virtual body, leading to a lower noticeability of redirection offsets on their virtual movements. 
% Conversely, when reaching for an object in the center of their field of view, the user’s attention is more concentrated on the virtual hand’s movement and position to ensure successful interaction, resulting in a higher noticeability of offsets.

Since each visual event is a complex choreography of many underlying factors (type of visual effect, location, duration, etc.), it is extremely difficult to quantify or parameterize visual stimuli.
Furthermore, individuals respond differently to even the same visual events.
Prior neuroscience studies revealed that factors like age, gender, and personality can influence how quickly someone reacts to visual events~\cite{gillon2024responses, gale1997human}. 
Therefore, aiming to model visual stimuli in a way that is generalizable and applicable to different stimuli and users, we propose to use users' \textbf{gaze behavior} as an indicator of how they respond to visual stimuli.
In this paper, we used various gaze behaviors, including gaze location, saccades~\cite{krejtz2018eye}, fixations~\cite{perkhofer2019using}, and the Index of Pupil Activity (IPA)~\cite{duchowski2018index}.
These behaviors indicate both where users are looking and their cognitive activity, as looking at something does not necessarily mean they are attending to it.
Our goal is to investigate how these gaze behaviors stimulated by various visual stimuli relate to the noticeability of redirection.
With this, we contribute a model that allows designers and content creators to adjust the redirection in real-time responding to dynamic visual events in VR.

To achieve this, we conducted user studies to collect users' noticeability of redirection under various visual stimuli.
To simulate realistic VR scenarios, we adopted a dual-task design in which the participants performed redirected movements while monitoring the visual stimuli.
Specifically, participants' primary task was to report if they noticed an offset between the avatar's movement and their own, while their secondary task was to monitor and report the visual stimuli.
As realistic virtual environments often contain complex visual effects, we started with simple and controlled visual stimulus to manage the influencing factors.

% first user study, confirmation study
% collect data under no visual stimuli, different basic visual stimuli
We first conducted a confirmation study (N=16) to test whether applying visual stimuli (opacity-based) actually affects their noticeability of redirection. 
The results showed that participants were significantly less likely to detect the redirection when visual stimuli was presented $(F_{(1,15)}=5.90,~p=0.03)$.
Furthermore, by analyzing the collected gaze data, results revealed a correlation between the proposed gaze behaviors and the noticeability results $(r=-0.43)$, confirming that the gaze behaviors could be leveraged to compute the noticeability.

% data collection study
We then conducted a data collection study to obtain more accurate noticeability results through repeated measurements to better model the relationship between visual stimuli-triggered gaze behaviors and noticeability of redirection.
With the collected data, we analyzed various numerical features from the gaze behaviors to identify the most effective ones. 
We tested combinations of these features to determine the most effective one for predicting noticeability under visual stimuli.
Using the selected features, our regression model achieved a mean squared error (MSE) of 0.011 through leave-one-user-out cross-validation. 
Furthermore, we developed both a binary and a three-class classification model to categorize noticeability, which achieved an accuracy of 91.74\% and 85.62\%, respectively.

% evaluation study
To evaluate the generalizability of the regression model, we conducted an evaluation study (N=24) to test whether the model could accurately predict noticeability with new visual stimuli (color- and scale-based animations).
Specifically, we evaluated whether the model's predictions aligned with participants' responses under these unseen stimuli.
The results showed that our model accurately estimated the noticeability, achieving mean squared errors (MSE) of 0.014 and 0.012 for the color- and scale-based visual stimili, respectively, compared to participants' responses.
Since the tested visual stimuli data were not included in the training, the results suggested that the extracted gaze behavior features capture a generalizable pattern and can effectively indicate the corresponding impact on the noticeability of redirection.

% application
Based on our model, we implemented an adaptive redirection technique and demonstrated it through two applications: adaptive VR action game and opportunistic rendering.
We conducted a proof-of-concept user study (N=8) to compare our adaptive redirection technique with a static redirection, evaluating the usability and benefits of our adaptive redirection technique.
The results indicated that participants experienced less physical demand and stronger sense of embodiment and agency when using the adaptive redirection technique. 
These results demonstrated the effectiveness and usability of our model.

In summary, we make the following contributions.
% 
\begin{itemize}
    \item 
    We propose to use users' gaze behavior as a medium to quantify how visual stimuli influences the noticebility of redirection. 
    Through two user studies, we confirm that visual stimuli significantly influences noticeability and identify key gaze behavior features that are closely related to this impact.
    \item 
    We build a regression model that takes the user's gaze behavioral data as input, then computes the noticeability of redirection.
    Through an evaluation study, we verify that our model can estimate the noticeability with new participants under unseen visual stimuli.
    These findings suggest that the extracted gaze behavior features effectively capture the influence of visual stimuli on noticeability and can generalize across different users and visual stimuli.
    \item 
    We develop an adaptive redirection technique based on our regression model and implement two applications with it.
    With a proof-of-concept study, we demonstrate the effectiveness and potential usability of our regression model on real-world use cases.

\end{itemize}

% \delete{
% Virtual Reality (VR) allows the user to embody a virtual avatar by mirroring their physical movements through the avatar.
% As the user's visual access to the physical world is blocked in tasks involving motion control, they heavily rely on the visual representation of the avatar's motions to guide their proprioception.
% Similar to real-world experiences, the user is able to resolve conflicts between different sensory inputs (e.g., vision and motor control) through multisensory integration, which is essential for mitigating the sensory noise that commonly arises.
% However, it also enables unique manipulations in VR, as the system can intentionally modify the avatar's movements in relation to the user's motions to achieve specific functional outcomes,
% for example, 
% % the manipulations on the avatar's movements can 
% enabling novel interaction techniques of redirected walking~\cite{razzaque2005redirected}, redirected reaching~\cite{gonzalez2022model}, and pseudo haptics~\cite{samad2019pseudo}.
% With small adjustments to the avatar's movements, the user can maintain their sense of embodiment, due to their ability to resolve the perceptual differences.
% % However, a large mismatch between the user and avatar's movements can result in the user losing their sense of embodiment, due to an inability to resolve the perceptual differences.
% }

% \delete{
% However, multisensory integration can break when the manipulation is so intense that the user is aware of the existence of the motion offset and no longer maintains the sense of embodiment.
% Prior research studied the intensity threshold of the offset applied on the avatar's hand, beyond which the embodiment will break~\cite{li2022modeling}. 
% Studies also investigated the user's sensitivity to the offsets over time~\cite{kohm2022sensitivity}.
% Based on the findings, we argue that one crucial factor that affects to what extent the user notices the offset (i.e., \textit{noticeability}) that remains under-explored is whether the user directs their visual attention towards or away from the virtual avatar.
% Related work (e.g., Mise-unseen~\cite{marwecki2019mise}) has showcased applications where adjustments in the environment can be made in an unnoticeable manner when they happen in the area out of the user's visual field.
% We hypothesize that directing the user's visual attention away from the avatar's body, while still partially keeping the avatar within the user's field-of-view, can reduce the noticeability of the offset.
% Therefore, we conduct two user studies and implement a regression model to systematically investigate this effect.
% }

% \delete{
% In the first user study (N = 16), we test whether drawing the user's visual attention away from their body impacts the possibility of them noticing an offset that we apply to their arm motion in VR.
% We adopt a dual-task design to enable the alteration of the user's visual attention and a yes/no paradigm to measure the noticeability of motion offset. 
% The primary task for the user is to perform an arm motion and report when they perceive an offset between the avatar's virtual arm and their real arm.
% In the secondary task, we randomly render a visual animation of a ball turning from transparent to red and becoming transparent again and ask them to monitor and report when it appears.
% We control the strength of the visual stimuli by changing the duration and location of the animation.
% % By changing the time duration and location of the visual animation, we control the strengths of attraction to the users.
% As a result, we found significant differences in the noticeability of the offsets $(F_{(1,15)}=5.90,~p=0.03)$ between conditions with and without visual stimuli.
% Based on further analysis, we also identified the behavioral patterns of the user's gaze (including pupil dilation, fixations, and saccades) to be correlated with the noticeability results $(r=-0.43)$ and they may potentially serve as indicators of noticeability.
% }

% \delete{
% To further investigate how visual attention influences the noticeability, we conduct a data collection study (N = 12) and build a regression model based on the data.
% The regression model is able to calculate the noticeability of the offset applied on the user's arm under various visual stimuli based on their gaze behaviors.
% Our leave-one-out cross-validation results show that the proposed method was able to achieve a mean-squared error (MSE) of 0.012 in the probability regression task.
% }

% \delete{
% To verify the feasibility and extendability of the regression model, we conduct an evaluation study where we test new visual animations based on adjustments on scale and color and invite 24 new participants to attend the study.
% Results show that the proposed method can accurately estimate the noticeability with an MSE of 0.014 and 0.012 in the conditions of the color- and scale-based visual effects.
% Since these animations were not included in the dataset that the regression model was built on, the study demonstrates that the gaze behavioral features we extracted from the data capture a generalizable pattern of the user's visual attention and can indicate the corresponding impact on the noticeability of the offset.
% }

% \delete{
% Finally, we demonstrate applications that can benefit from the noticeability prediction model, including adaptive motion offsets and opportunistic rendering, considering the user's visual attention. 
% We conclude with discussions of our work's limitations and future research directions.
% }

% \delete{
% In summary, we make the following contributions.
% }
% % 
% \begin{itemize}
%     \item 
%     \delete{
%     We quantify the effects of the user's visual attention directed away by stimuli on their noticeability of an offset applied to the avatar's arm motion with respect to the user's physical arm. 
%     Through two user studies, we identified gaze behavioral features that are indicative of the changes in noticeability.
%     }
%     \item 
%     \delete{We build a regression model that takes the user's gaze behavioral data and the offset applied to the arm motion as input, then computes the probability of the user noticing the offset.
%     Through an evaluation study, we verified that the model needs no information about the source attracting the user's visual attention and can be generalizable in different scenarios.
%     }
%     \item 
%     \delete{We demonstrate two applications that potentially benefit from the regression model, including adaptive motion offsets and opportunistic rendering.
%     }

% \end{itemize}

\begin{comment}
However, users will lose the sense of embodiment to the virtual avatars if they notice the offset between the virtual and physical movements.
To address this, researchers have been exploring the noticing threshold of offsets with various magnitudes and proposing various redirection techniques that maintain the sense of embodiment~\cite{}.

However, when users embody virtual avatars to explore virtual environments, they encounter various visual effects and content that can attract their attention~\cite{}.
During this, the user may notice an offset when he observes the virtual movement carefully while ignoring it when the virtual contents attract his attention from the movements.
Therefore, static offset thresholds are not appropriate in dynamic scenarios.

Past research has proposed dynamic mapping techniques that adapted to users' state, such as hand moving speed~\cite{frees2007prism} or ergonomically comfortable poses~\cite{montano2017erg}, but not considering the influence of virtual content.
More specifically, PRISM~\cite{frees2007prism} proposed adjusting the C/D ratio with a non-linear mapping according to users' hand moving speed, but it might not be optimal for various virtual scenarios.
While Erg-O~\cite{montano2017erg} redirected users' virtual hands according to the virtual target's relative position to reduce physical fatigue, neglecting the change of virtual environments. 

Therefore, how to design redirection techniques in various scenarios with different visual attractions remains unknown.
To address this, we investigate how visual attention affects the noticing probability of movement offsets.
Based on our experiments, we implement a computational model that automatically computes the noticing probability of offsets under certain visual attractions.
VR application designers and developers can easily leverage our model to design redirection techniques maintaining the sense of embodiment adapt to the user's visual attention.
We implement a dynamic redirection technique with our model and demonstrate that it effectively reduces the target reaching time without reducing the sense of embodiment compared to static redirection techniques.

% Need to be refined
This paper offers the following contributions.
\begin{itemize}
    \item We investigate how visual attractions affect the noticing probability of redirection offsets.
    \item We construct a computational model to predict the noticing probability of an offset with a given visual background.
    \item We implement a dynamic redirection technique adapting to the visual background. We evaluate the technique and develop three applications to demonstrate the benefits. 
\end{itemize}



First, we conducted a controlled experiment to understand how users perceived the movement offset while subjected to various distractions.
Since hand redirection is one of the most frequently used redirections in VR interactions, we focused on the dynamic arm movements and manually added angular offsets to the' elbow joint~\cite{li2022modeling, gonzalez2022model, zenner2019estimating}. 
We employed flashing spheres in the user's field of view as distractions to attract users' visual attention.
Participants were instructed to report the appearing location of the spheres while simultaneously performing the arm movements and reporting if they perceived an offset during the movement. 
(\zhipeng{Add the results of data collection. Analyze the influence of the distance between the gaze map and the offset.}
We measured the visual attraction's magnitude with the gaze distribution on it.
Results showed that stronger distractions made it harder for users to notice the offset.)
\zhipeng{Need to rewrite. Not sure to use gaze distribution or a metric obtained from the visual content.}
Secondly, we constructed a computational model to predict the noticing probability of offsets with given visual content.
We analyzed the data from the user studies to measure the influence of visual attractions on the noticing probability of offsets.
We built a statistical model to predict the offset's noticing probability with a given visual content.
Based on the model, we implement a dynamic redirection technique to adjust the redirection offset adapted to the user's current field of view.
We evaluated the technique in a target selection task compared to no hand redirection and static hand redirection.
\zhipeng{Add the results of the evaluation.}
Results showed that the dynamic hand redirection technique significantly reduced the target selection time with similar accuracy and a comparable sense of embodiment.
Finally, we implemented three applications to demonstrate the potential benefits of the visual attention adapted dynamic redirection technique.
\end{comment}

% This one modifies arm length, not redirection
% \citeauthor{mcintosh2020iteratively} proposed an adaptation method to iteratively change the virtual avatar arm's length based on the primary tasks' performance~\cite{mcintosh2020iteratively}.



% \zhipeng{TO ADD: what is redirection}
% Redirection enables novel interactions in Virtual Reality, including redirected walking, haptic redirection, and pseudo haptics by introducing an offset to users' movement.
% \zhipeng{TO ADD: extend this sentence}
% The price of this is that users' immersiveness and embodiment in VR can be compromised when they notice the offset and perceive the virtual movement not as theirs~\cite{}.
% \zhipeng{TO ADD: extend this sentence, elaborate how the virtual environment attracts users' attention}
% Meanwhile, the visual content in the virtual environment is abundant and consistently captures users' attention, making it harder to notice the offset~\cite{}.
% While previous studies explored the noticing threshold of the offsets and optimized the redirection techniques to maintain the sense of embodiment~\cite{}, the influence of visual content on the probability of perceiving offsets remains unknown.  
% Therefore, we propose to investigate how users perceive the redirection offset when they are facing various visual attractions.


% We conducted a user study to understand how users notice the shift with visual attractions.
% We used a color-changing ball to attract the user's attention while instructing users to perform different poses with their arms and observe it meanwhile.
% \zhipeng{(Which one should be the primary task? Observe the ball should be the primary one, but if the primary task is too simple, users might allocate more attention on the secondary task and this makes the secondary task primary.)}
% \zhipeng{(We need a good and reasonable dual-task design in which users care about both their pose and the visual content, at least in the evaluation study. And we need to be able to control the visual content's magnitude and saliency maybe?)}
% We controlled the shift magnitude and direction, the user's pose, the ball's size, and the color range.
% We set the ball's color-changing interval as the independent factor.
% We collect the user's response to each shift and the color-changing times.
% Based on the collected data, we constructed a statistical model to describe the influence of visual attraction on the noticing probability.
% \zhipeng{(Are we actually controlling the attention allocation? How do we measure the attracting effect? We need uniform metrics, otherwise it is also hard for others to use our knowledge.)}
% \zhipeng{(Try to use eye gaze? The eye gaze distribution in the last five seconds to decide the attention allocation? Basically constructing a model with eye gaze distribution and noticing probability. But the user's head is moving, so the eye gaze distribution is not aligned well with the current view.)}

% \zhipeng{Saliency and EMD}
% \zhipeng{Gaze is more than just a point: Rethinking visual attention
% analysis using peripheral vision-based gaze mapping}

% Evaluation study(ideal case): based on the visual content, adjusting the redirection magnitude dynamically.

% \zhipeng{(The risk is our model's effect is trivial.)}

% Applications:
% Playing Lego while watching demo videos, we can accelerate the reaching process of bricks, and forbid the redirection during the manipulation.

% Beat saber again: but not make a lot of sense? Difficult game has complicated visual effects, while allows larger shift, but do not need large shift with high difficulty



\section{Related Work}
\label{lit_review}

\begin{highlight}
{

Our research builds upon {\em (i)} Assessing Web Accessibility, {\em (ii)} End-User Accessibility Repair, and {\em (iii)} Developer Tools for Accessibility.

\subsection{Assessing Web Accessibility}
From the earliest attempts to set standards and guidelines, web accessibility has been shaped by a complex interplay of technical challenges, legal imperatives, and educational campaigns. Over the past 25 years, stakeholders have sought to improve digital inclusion by establishing foundational standards~\cite{chisholm2001web, caldwell2008web}, enforcing legal obligations~\cite{sierkowski2002achieving, yesilada2012understanding}, and promoting a broader culture of accessibility awareness among developers~\cite{sloan2006contextual, martin2022landscape, pandey2023blending}. 
Despite these longstanding efforts, systemic accessibility issues persist. According to the 2024 WebAIM Million report~\cite{webaim2024}, 95.9\% of the top one million home pages contained detectable WCAG violations, averaging nearly 57 errors per page. 
These errors take many forms: low color contrast makes the interface difficult for individuals with color deficiency or low vision to read text; missing alternative text leaves users relying on screen readers without crucial visual context; and unlabeled form inputs or empty links and buttons hinder people who navigate with assistive technologies from completing basic tasks. 
Together, these accessibility issues not only limit user access to critical online resources such as healthcare, education, and employment but also result in significant legal risks and lost opportunities for businesses to engage diverse audiences. Addressing these pervasive issues requires systematic methods to identify, measure, and prioritize accessibility barriers, which is the first step toward achieving meaningful improvements.

Prior research has introduced methods blending automation and human evaluation to assess web accessibility. Hybrid approaches like SAMBA combine automated tools with expert reviews to measure the severity and impact of barriers, enhancing evaluation reliability~\cite{brajnik2007samba}. Quantitative metrics, such as Failure Rate and Unified Web Evaluation Methodology, support large-scale monitoring and comparative analysis, enabling cost-effective insights~\cite{vigo2007quantitative, martins2024large}. However, automated tools alone often detect less than half of WCAG violations and generate false positives, emphasizing the need for human interpretation~\cite{freire2008evaluation, vigo2013benchmarking}. Recent progress with large pretrained models like Large Language Models (LLMs)~\cite{dubey2024llama,bai2023qwen} and Large Multimodal Models (LMMs)~\cite{liu2024visual, bai2023qwenvl} offers a promising step forward, automating complex checks like non-text content evaluation and link purposes, achieving higher detection rates than traditional tools~\cite{lopez2024turning, delnevo2024interaction}. Yet, these large models face challenges, including dependence on training data, limited contextual judgment, and the inability to simulate real user experiences. These limitations underscore the necessity of combining models with human oversight for reliable, user-centered evaluations~\cite{brajnik2007samba, vigo2013benchmarking, delnevo2024interaction}. 

Our work builds on these prior efforts and recent advancements by leveraging the capabilities of large pretrained models while addressing their limitations through a developer-centric approach. CodeA11y integrates LLM-powered accessibility assessments, tailored accessibility-aware system prompts, and a dedicated accessibility checker directly into GitHub Copilot---one of the most widely used coding assistants. Unlike standalone evaluation tools, CodeA11y actively supports developers throughout the coding process by reinforcing accessibility best practices, prompting critical manual validations, and embedding accessibility considerations into existing workflows.
% This pervasive shortfall reflects the difficulty of scaling traditional approaches---such as manual audits and automated tools---that either demand immense human effort or lack the nuanced understanding needed to capture real-world user experiences. 
%
% In response, a new wave of AI-driven methods, many powered by large language models (LLMs), is emerging to bridge these accessibility detection and assessment gaps. Early explorations, such as those by Morillo et al.~\cite{morillo2020system}, introduced AI-assisted recommendations capable of automatic corrections, illustrating how computational intelligence can tackle the repetitive, common errors that plague large swaths of the web. Building on this foundation, Huang et al.~\cite{huang2024access} proposed ACCESS, a prompt-engineering framework that streamlines the identification and remediation of accessibility violations, while López-Gil et al.~\cite{lopez2024turning} demonstrated how LLMs can help apply WCAG success criteria more consistently---reducing the reliance on manual effort. Beyond these direct interventions, recent work has also begun integrating user experiences more seamlessly into the evaluation process. For example, Huq et al.~\cite{huq2024automated} translate user transcripts and corresponding issues into actionable test reports, ensuring that accessibility improvements align more closely with authentic user needs.
% However, as these AI-driven solutions evolve, researchers caution against uncritical adoption. Othman et al.~\cite{othman2023fostering} highlight that while LLMs can accelerate remediation, they may also introduce biases or encourage over-reliance on automated processes. Similarly, Delnevo et al.~\cite{delnevo2024interaction} emphasize the importance of contextual understanding and adaptability, pointing to the current limitations of LLM-based systems in serving the full spectrum of user needs. 
% In contrast to this backdrop, our work introduces and evaluates CodeA11y, an LLM-augmented extension for GitHub Copilot that not only mitigates these challenges by providing more consistent guidance and manual validation prompts, but also aligns AI-driven assistance with developers’ workflows, ultimately contributing toward more sustainable propulsion for building accessible web.

% Broader implications of inaccessibility—legal compliance, ethical concerns, and user experience
% A Historical Review of Web Accessibility Using WAVE
% "I tend to view ads almost like a pestilence": On the Accessibility Implications of Mobile Ads for Blind Users

% In the research domain, several methods have been developed to assess and enhance web accessibility. These include incorporating feedback into developer tools~\cite{adesigner, takagi2003accessibility, bigham2010accessibility} and automating the creation of accessibility tests and reports for user interfaces~\cite{swearngin2024towards, taeb2024axnav}. 

% Prior work has also studied accessibility scanners as another avenue of AI to improve web development practices~\cite{}.
% However, a persistent challenge is that developers need to be aware of these tools to utilize them effectively. With recent advancements in LLMs, developers might now build accessible websites with less effort using AI assistants. However, the impact of these assistants on the accessibility of their generated code remains unclear. This study aims to investigate these effects.

\subsection{End-user Accessibility Repair}
In addition to detecting accessibility errors and measuring web accessibility, significant research has focused on fixing these problems.
Since end-users are often the first to notice accessibility problems and have a strong incentive to address them, systems have been developed to help them report or fix these problems.

Collaborative, or social accessibility~\cite{takagi2009collaborative,sato2010social}, enabled these end-user contributions to be scaled through crowd-sourcing.
AccessMonkey~\cite{bigham2007accessmonkey} and Accessibility Commons~\cite{kawanaka2008accessibility} were two examples of repositories that store accessibility-related scripts and metadata, respectively.
Other work has developed browser extensions that leverage crowd-sourced databases to automatically correct reading order, alt-text, color contrast, and interaction-related issues~\cite{sato2009s,huang2015can}.

One drawback of collaborative accessibility approaches is that they cannot fix problems for an ``unseen'' web page on-demand, so many projects aim to automatically detect and improve interfaces without the need for an external source of fixes.
A large body of research has focused on making specific web media (e.g., images~\cite{gleason2019making,guinness2018caption, twitterally, gleason2020making, lee2021image}, design~\cite{potluri2019ai,li2019editing, peng2022diffscriber, peng2023slide}, and videos~\cite{pavel2020rescribe,peng2021say,peng2021slidecho,huh2023avscript}) accessible through a combination of machine learning (ML) and user-provided fixes.
Other work has focused on applying more general fixes across all websites.

Opportunity accessibility addressed a common accessibility problem of most websites: by default, content is often hard to see for people with visual impairments, and many users, especially older adults, do not know how to adjust or enable content zooming~\cite{bigham2014making}.
To this end, a browser script (\texttt{oppaccess.js}) was developed that automatically adjusted the browser's content zoom to maximally enlarge content without introducing adverse side-effects (\textit{e.g.,} content overlap).
While \texttt{oppaccess.js} primarily targeted zoom-related accessibility, recent work aimed to enable larger types of changes, by using LLMs to modify the source code of web pages based on user questions or directives~\cite{li2023using}.

Several efforts have been focused on improving access to desktop and mobile applications, which present additional challenges due to the unavailability of app source code (\textit{e.g.,} HTML).
Prefab is an approach that allows graphical UIs to be modified at runtime by detecting existing UI widgets, then replacing them~\cite{dixon2010prefab}.
Interaction Proxies used these runtime modification strategies to ``repair'' Android apps by replacing inaccessible widgets with improved alternatives~\cite{zhang2017interaction, zhang2018robust}.
The widget detection strategies used by these systems previously relied on a combination of heuristics and system metadata (\textit{e.g.,} the view hierarchy), which are incomplete or missing in the accessible apps.
To this end, ML has been employed to better localize~\cite{chen2020object} and repair UI elements~\cite{chen2020unblind,zhang2021screen,wu2023webui,peng2025dreamstruct}.

In general, end-user solutions to repairing application accessibility are limited due to the lack of underlying code and knowledge of the semantics of the intended content.

\subsection{Developer Tools for Accessibility}
Ultimately, the best solution for ensuring an accessible experience lies with front-end developers. Many efforts have focused on building adequate tooling and support to help developers with ensuring that their UI code complies with accessibility standards.

Numerous automated accessibility testing tools have been created to help developers identify accessibility issues in their code: i) static analysis tools, such as IBM Equal Access Accessibility Checker~\cite{ibm2024toolkit} or Microsoft Accessibility Insights~\cite{accessibilityinsights2024}, scan the UI code's compliance with predefined rules derived from accessibility guidelines; and ii) dynamic or runtime accessibility scanners, such as Chrome Devtools~\cite{chromedevtools2024} or axe-Core Accessibility Engine~\cite{deque2024axe}, perform real-time testing on user interfaces to detect interaction issues not identifiable from the code structure. While these tools greatly reduce the manual effort required for accessibility testing, they are often criticized for their limited coverage. Thus, experts often recommend manually testing with assistive technologies to uncover more complex interaction issues. Prior studies have created accessibility crawlers that either assist in developer testing~\cite{swearngin2024towards,taeb2024axnav} or simulate how assistive technologies interact with UIs~\cite{10.1145/3411764.3445455, 10.1145/3551349.3556905, 10.1145/3544548.3580679}.

Similar to end-user accessibility repair, research has focused on generating fixes to remediate accessibility issues in the UI source code. Initial attempts developed heuristic-based algorithms for fixing specific issues, for instance, by replacing text or background color attributes~\cite{10.1145/3611643.3616329}. More recent work has suggested that the code-understanding capabilities of LLMs allow them to suggest more targeted fixes.
For example, a study demonstrated that prompting ChatGPT to fix identified WCAG compliance issues in source code could automatically resolve a significant number of them~\cite{othman2023fostering}. Researchers have sought to leverage this capability by employing a multi-agent LLM architecture to automatically identify and localize issues in source code and suggest potential code fixes~\cite{mehralian2024automated}.

While the approaches mentioned above focus on assessing UI accessibility of already-authored code (\textit{i.e.,} fixing existing code), there is potential for more proactive approaches.
For example, LLMs are often used by developers to generate UI source code from natural language descriptions or tab completions~\cite{chen2021evaluating,GitHubCopilot,lozhkov2024starcoder,hui2024qwen2,roziere2023code,zheng2023codegeex}, but LLMs frequently produce inaccessible code by default~\cite{10.1145/3677846.3677854,mowar2024tab}, leading to inaccessible output when used by developers without sufficient awareness of accessibility knowledge.
The primary focus of this paper is to design a more accessibility-aware coding assistant that both produces more accessible code without manual intervention (\textit{e.g.,} specific user prompting) and gradually enables developers to implement and improve accessibility of automatically-generated code through IDE UI modifications (\textit{e.g.}, reminder notifications).

}
\end{highlight}



% Work related to this paper includes {\em (i)} Web Accessibility and {\em (ii)} Developer Practices in AI-Assisted Programming.

% \ipstart{Web Accessibility: Practice, Evaluation, and Improvements} Substantial efforts have been made to set accessibility standards~\cite{chisholm2001web, caldwell2008web}, establish legal requirements~\cite{sierkowski2002achieving, yesilada2012understanding}, and promote education and advocacy among developers~\cite{sloan2006contextual, martin2022landscape, pandey2023blending}. In the research domain, several methods have been developed to assess and enhance web accessibility. These include incorporating feedback into developer tools~\cite{adesigner, takagi2003accessibility, bigham2010accessibility} and automating the creation of accessibility tests and reports for user interfaces~\cite{swearngin2024towards, taeb2024axnav}. 
% % Prior work has also studied accessibility scanners as another avenue of AI to improve web development practices~\cite{}.
% However, a persistent challenge is that developers need to be aware of these tools to utilize them effectively. With recent advancements in LLMs, developers might now build accessible websites with less effort using AI assistants. However, the impact of these assistants on the accessibility of their generated code remains unclear. This study aims to investigate these effects.

% \ipstart{Developer Practices in AI-Assisted Programming}
% Recent usability research on AI-assisted development has examined the interaction strategies of developers while using AI coding assistants~\cite{barke2023grounded}.
% They observed developers interacted with these assistants in two modes -- 1) \textit{acceleration mode}: associated with shorter completions and 2) \textit{exploration mode}: associated with long completions.
% Liang {\em et al.} \cite{liang2024large} found that developers are driven to use AI assistants to reduce their keystrokes, finish tasks faster, and recall the syntax of programming languages. On the other hand, developers' reason for rejecting autocomplete suggestions was the need for more consideration of appropriate software requirements. This is because primary research on code generation models has mainly focused on functional correctness while often sidelining non-functional requirements such as latency, maintainability, and security~\cite{singhal2024nofuneval}. Consequently, there have been increasing concerns about the security implications of AI-generated code~\cite{sandoval2023lost}. Similarly, this study focuses on the effectiveness and uptake of code suggestions among developers in mitigating accessibility-related vulnerabilities. 


% ============================= additional rw ============================================
% - Paulina Morillo, Diego Chicaiza-Herrera, and Diego Vallejo-Huanga. 2020. System of Recommendation and Automatic Correction of Web Accessibility Using Artificial Intelligence. In Advances in Usability and User Experience, Tareq Ahram and Christianne Falcão (Eds.). Springer International Publishing, Cham, 479–489
% - Juan-Miguel López-Gil and Juanan Pereira. 2024. Turning manual web accessibility success criteria into automatic: an LLM-based approach. Universal Access in the Information Society (2024). https://doi.org/10.1007/s10209-024-01108-z
% - s
% - Calista Huang, Alyssa Ma, Suchir Vyasamudri, Eugenie Puype, Sayem Kamal, Juan Belza Garcia, Salar Cheema, and Michael Lutz. 2024. ACCESS: Prompt Engineering for Automated Web Accessibility Violation Corrections. arXiv:2401.16450 [cs.HC] https://arxiv.org/abs/2401.16450
% - Syed Fatiul Huq, Mahan Tafreshipour, Kate Kalcevich, and Sam Malek. 2025. Automated Generation of Accessibility Test Reports from Recorded User Transcripts. In Proceedings of the 47th International Conference on Software Engineering (ICSE) (Ottawa, Ontario, Canada). IEEE. https://ics.uci.edu/~seal/publications/2025_ICSE_reca11.pdf To appear in IEEE Xplore
% - Achraf Othman, Amira Dhouib, and Aljazi Nasser Al Jabor. 2023. Fostering websites accessibility: A case study on the use of the Large Language Models ChatGPT for automatic remediation. In Proceedings of the 16th International Conference on PErvasive Technologies Related to Assistive Environments (Corfu, Greece) (PETRA ’23). Association for Computing Machinery, New York, NY, USA, 707–713. https://doi.org/10.1145/3594806.3596542
% - Zsuzsanna B. Palmer and Sushil K. Oswal. 0. Constructing Websites with Generative AI Tools: The Accessibility of Their Workflows and Products for Users With Disabilities. Journal of Business and Technical Communication 0, 0 (0), 10506519241280644. https://doi.org/10.1177/10506519241280644
% ============================= additional rw ============================================


\begin{figure*}[htbp]
    \includegraphics[width=\linewidth]{document/figures/sensing_principle.png}
  \hfill
    \includegraphics[width=\linewidth]{document/figures/sensing_principle_sequential2.png}
  \caption{Acoustic and IMU data over 26 isolated English/ASL alphabet letters and continuously fingerspelled words. Continuous fingerspelling adds complexity due to natural flow and quick transitions between letters, which alter sensor values depending on adjacent letters. }
\Description{Acoustic and IMU data over Isolated 26 English manual alphabets and continuous fingerspelled words: Continuous fingerspelling involves added complexity due to the natural flow and quick transitions between letters, which alter sensor values depending on adjacent letters. }
  \label{fig:sensing_principle}
\end{figure*}



\section{SpellRing}

SpellRing is an AI-powered sensing system designed to recognize continuously fingerspelled words using a single ring. This section elaborates on the challenges of designing such a wearable device and details how we developed an AI-powered ring with intelligent sensing methods to achieve accurate recognition.


\subsection{Challenges}
Recognizing continuous fingerspelling poses several unique challenges that make it significantly more complex than recognizing isolated ASL letters:

\subsubsection{Complexity of Handshape, Movement, and Palm Orientation}
American Sign Language (ASL) fingerspelling involves complex combinations of different handshapes, movements, and palm orientations. This poses challenges for accurate fingerspelling recognition. Some letters, such as `A', `S', `M', `N', and `T' (see Figure \ref{fig:sensing_principle}), appear visually similar, while others like `K' and `P', `G' and `Q', or `H' and `U' share the same handshape while differing in palm orientation. Additionally, certain letters (e.g., `J' and `Z') involve specific hand movements, further complicating the recognition process.


\subsubsection{User-Dependent Customized Transitions}
Continuous fingerspelling introduces an additional layer of complexity due to its natural flow and quick transitions between letters (see Figure \ref{fig:sensing_principle}). These transitions vary significantly depending on letter sequences and individual signing behaviors \cite{keane2016fingerspelling, keane2015segmentation}. For instance, the letter `E' can be signed as either a closed or open form, with the open `E' more commonly used during faster fingerspelling, particularly at the beginning or end of a word.
Fluent signers can fingerspell at speeds of 5--8 letters per second \cite{hanson1982use, quinto2010rates, keane2016fingerspelling}, often blending adjacent letters \cite{hassan2023tap, shi2018american, keane2015segmentation}. This high speed increases both efficiency and user-specific signing behaviors, making the accurate recognition of continuous fingerspelling much more challenging than recognizing isolated ASL letters.


\subsubsection{Form Factor vs. Recognition Accuracy Trade-off}
Designing a wearable device for ASL recognition presents a significant challenge in balancing form factor with user experience and recognition accuracy. Glove-based devices with sensors on all fingers can capture detailed poses but are bulky and impractical for daily use, often hindering dexterity. Wristbands, such as EMG sensor bands, offer better usability but struggle with performance issues due to the need for extensive training data across sessions and muscle variability.
Rings with embedded IMUs are more user-friendly, but reliable recognition often requires multiple rings, which can still compromise simplicity. Capturing complex ASL handshapes and movements with a single ring remains a significant challenge, as it must balance unobtrusive design with the ability to capture detailed and reliable data for recognition.


 \begin{table}[t]

\textcolor{black}{
\caption{\textcolor{black}{Performance over Fingerspelling Speed and Sampling Rate. FPS (Frame Per Second), H (Hearing), CODA (Child of Deaf Adults), G (Gender), \# (Max Letters Per Second)}}
\Description{Performance Analysis: Fingerspelling Experience and Sampling Rate}
\begin{tabular}{c|c|c|c|c|c|c}
\hline
   & Experience    & G & Year & \# & FPS - 87     & FPS - 490    \\ \hline
P1 & Leaner, H & M      & 1    & 2                                                                       & 92.99 (2.01) & 93.32 (1.89) \\ \hline
P2 & Leaner, H & F      & 5    & 4                                                                       & 62.67 (3.49) & 86.86 (4.33) \\ \hline
P3 & CODA, H       & M      & 10   & 5                                                                       & 57.93 (3.61) & 86.03 (3.87) \\ \hline
\end{tabular}
}
\end{table}
\label{fig:performance_pilot}
\subsection{Hardware Prototype Design}
To address these challenges, we developed SpellRing, a single-ring system capable of recognizing fingerspelled words at the word level. Our design incorporates two key sensing modalities: acoustic sensing for handshape and IMU sensors for movement.



\begin{figure*}[t]
  \includegraphics[width=0.8\linewidth]{document/figures/model.png}
  \caption{Fusion Model Framework}
  \Description{Fusion Model Framework}
  \label{fig:model}
\end{figure*}

\subsubsection{Single Ring Approach}

SpellRing is designed specifically for the thumb. While Ring-a-Pose \cite{yu2024ring} showed that rings could potentially be placed on all five fingers to track handshape, thumb placement is ideal for ASL recognition. Placing the sensor on other fingers leads to blockage issues, especially when fingerspelling letters such as `A', `S', `M', `N', `L', and `I', and during transitions between these letters. This blockage makes it difficult both to fingerspell and to capture handshape using acoustic sensing. We chose to position the ring on the thumb to minimize these blockage issues. \textcolor{black}{We further discuss the ring's placement and user experience in Section \ref{future}.}

\subsubsection{Sensing Modalities} \textit{1) Active Acoustic Sensing:} For handshape sensing, we chose to adpot active acoustic sensing on the ring. Only requiring low-power and miniature microphone and speakers, this sensing method has shown promising performance in tracking and understanding various body postures on wearables\cite{yu2024ring,li2022eario,li2024eyeecho,li2024gazetrak,mahmud2023posesonic,li2024sonicid,mahmud2024actsonic,mahmud2024munchsonic,sun2023echonose,lee2024echowrist,zhang2023echospeech,mahmud2024wristsonic,zhang2023hpspeech,parikh2024echoguide}. Similar to Ring-a-Pose\cite{yu2024ring}, the ring acts as a `scanner' by emitting inaudible sound waves (frequency range of 20-24 kHz) to scan hand shapes. These sound waves are reflected and refracted by the fingers and received by the microphone on the ring. The preprocessed reflected acoustic signal patterns vary with different hand shapes, leading to precise estimation of handshape \cite{yu2024ring}. \textcolor{black}{However, our earlier experiments (see Table 2) with users of varying fingerspelling skills—especially in speed—using Ring-a-Pose showed that the system struggled to handle rapid fingerspelling of a participant with 10 years of ASL signing experience with a fingerspelling speed of up to 5 letters per second, resulting in an accuracy of 57.93\% on 1,164 words; our experiment is described in detail in Section \ref{experiment}. To capture sufficient information during fast fingerspelling, we increased the sampling rate by six times based on our hardware capabilities. This adjustment reduced the sensing range from 2.06 m (as with Ring-a-Pose \cite{yu2024ring}) to 34.3 cm, focusing more on finger and hand movements and capturing information every 0.12 seconds to classify letters. These changes led to improved performance, achieving an accuracy of 86.03\%---we used this setup for our full experiment.}



\textit{2) IMU for Hand Movement:} To track hand movement and palm orientation, we utilize a gyroscope from the IMU module \cite{zhang2017fingersound,zhang2017fingorbits}. This allows us to measure changes in rotational velocity (angular velocity) around three axes (x, y, and z). By integrating these measurements over time, we can track changes in hand movement, making it easy to distinguish letters with similar handshapes but different palm orientations.
\begin{figure}[b!]
  \includegraphics[width=\linewidth]{document/figures/prototyping.png}
  \caption{Hardware Prototype: (a) a 3.7V 70mAh LiPo battery, (b) an nRF MCU, (c) a customized Flexible Printed Circuit Board (FPCB) with a microphone and speaker, (d) an IMU sensor board (MPU6050), (e) an ESP32 Feather Board, and (f) a 3D-printed ring case.}
  \Description{Hardware Prototype}
  \label{fig:prototype}
\end{figure}

\subsubsection{Hardware Components}
As shown in Figure \ref{fig:prototype}, the ring incorporates a microphone (TDK-ICS-43434), a speaker (USound UT-P2019), and a customized Flexible Printed Circuit Board (FPCB) \textcolor{black}{(c)} enclosed within a 3D-printed Polylactic Acid (PLA) case \textcolor{black}{(f)}. It also features a microcontroller unit (MCU) \textcolor{black}{(e)}, an SD card for data storage \textcolor{black}{(b)}, and a 3.7V 70mAh LiPo battery \textcolor{black}{(a)}. The ring is powered by the battery, which has a switch for toggling it ON/OFF. Once powered on, the acoustic sensing system initiates and automatically saves data to the SD card until powered off. The IMU sensor board (MPU6050) \textcolor{black}{(d)} includes 6-axis inertial motion sensors (accelerometer and gyroscope), providing three-axis data output at a rate of 150Hz. The IMU is connected to the microcontroller on the wrist via a flexible wire, and the microcontroller transmits the data to an external PC through a flexible USB cable. \textcolor{black}{The acoustic data on the SD card and the IMU data on the PC are then synchronized based on timestamped records.}





% The SpellRing prototype includes a 3.7V 70mAh LiPo battery for powering the acoustic data acquisition, an onboard SD card for storing acoustic data, an ESP32 feather strapped on the wrist for reading IMU data, and an MPU6050 IMU sensor for capturing finger movement. The ESP32 feather reads the IMU data at 150 Hz and transmits it to a PC via serial-over-USB. While our current prototype includes a wrist-mounted component to reduce bulk on the thumb, further miniaturization is possible to eliminate this component in future iterations.


\subsection{Algorithms and Data Processing Pipeline}
SpellRing's software pipeline is designed to process multimodal data from the acoustic and IMU sensors and recognize fingerspelled words accurately. Our approach incorporates sophisticated data processing techniques and a deep learning model optimized for sequence recognition.

\subsubsection{Acoustic Data Processing}

Correlation-based frequency modulated continuous wave (C-FMCW) \cite{wang2018c} is used as the transmitted signal for acoustic sensing. The received signals are processed to calculate an echo profile, following methods specified in prior work \cite{yu2024ring,li2022eario,zhang2023echospeech}. These echo profiles encode temporal and spatial information of reflection and diffraction strengths, representing different handshape patterns. To isolate handshape changes from constant environmental reflections, we calculate the difference between consecutive echo frames, generating differential echo profiles. These profiles serve as the input representation of handshape patterns for our deep learning pipeline.

\subsubsection{IMU Data Processing}
Tri-axial gyroscope data, sampled at 150 Hz, is used to track palm orientation and rotational movement. Before feeding them into the deep learning model, we normalize the x, y, and z values and upsample them to synchronize with the acoustic data. This preprocessing step ensures that we can extract synchronized features from our multimodal deep learning pipeline.

\subsubsection{Deep Learning Pipeline}
Our deep learning pipeline leverages Connectionist Temporal Classification (CTC) \cite{graves2006connectionist,zhang2023echospeech}, a method widely employed in sequence labeling tasks, to recognize fingerspelled words continuously without needing to label or segment each letter. Aa shwon in Figure \ref{fig:model}, the model architecture comprises two main components: an acoustic sensing model and an IMU sensing model.

% \subsubsection{Model Framework}

For the acoustic sensing model, \textcolor{black}{we process differential echo profiles using a convolutional neural network (CNN) with ResNet-18 as the backbone. During pooling steps, we apply one-dimensional average pooling along the temporal axis only, rather than both axes, to preserve sequential information.} The IMU sensing model employs a 2D CNN architecture to process IMU data, as our pilot study demonstrated that this approach slightly outperformed a 1D CNN in terms of CTC loss.

The embeddings generated from both modalities are concatenated and then fed into a fully connected dense layer. This is followed by a dropout layer to prevent overfitting, and finally, a softmax function to produce the output probabilities. This multimodal approach allows our system to effectively combine information from both acoustic and motion sensors, enhancing the accuracy of fingerspelling recognition.

\subsubsection{Data Augmentation and Training Scheme}
To enhance performance and streamline training, we adopted several techniques. \textcolor{black}{To enhance the model's adaptability to varying fingerspelling speeds with a fixed window size, we augment the dataset by merging consecutive fingerspelled words, simply concatenating up to four words.} We also apply random noise during training to prevent overfitting and use random padding to handle variable-length inputs. Our training process involves a two-step approach: first training with data from all participants except one, then retraining with the specific participant's data for leave-one-session-out cross-validation. \textcolor{black}{We note that this two-step approach results in a user-dependent model, using 20 sessions collected from each participant over two to three different days, and the following reported results are based on this setup. User-independent results are further discussed in Section \ref{pretrained model}. }

\subsubsection{Word Correction}
To correct potential errors in the model's character sequence predictions, we compute the Levenshtein distance \cite{Levenshtein1965BinaryCC} between the predicted sequence and each unique word in a reference dictionary. The word with the smallest Levenshtein distance was selected as the corrected word, enhancing the overall accuracy of our system. \textcolor{black}{To align our system evaluation with prior literature, specifically for performance comparison with FingerSpeller \cite{martin2023fingerspeller}—such as multi-ring versus single-ring setups—we adopted their evaluation method by using the MacKenzie-Soukoreff phrase set \cite{mackenzie2003phrase} as the reference dictionary.}
\textcolor{black}{However, since the choice of reference dictionary affects the performance of the auto-correction, we discuss its impact using different dictionary sets in Section \ref{auto_correction}.}




\section{User Studies Overview}
To evaluate the effectiveness and usability of \theDevice, we conducted two user studies with signers ranging from novice to native/ fluent in ASL proficiency. The first study focused on word-level recognition, assessing the system's ability to accurately recognize individual fingerspelled words across a diverse group of participants. The second study expanded on these findings by examining phrase-level recognition in real-time scenarios, providing insights into the system's performance in more natural, context-rich environments. These studies aimed to validate SpellRing's performance across different levels of signing experience, explore the impact of signing speed and habits on recognition accuracy, and investigate how users adapt to the system in real-time use. The user study was approved by the Institutional Review Board (IRB) at the authors' institution. Participants were compensated \$40 per hour for their participation in the study.


\section{Word-level Recognition}

\subsection{Purpose and Overview}
The primary objective of this study was to evaluate SpellRing's ability to recognize individual fingerspelled words accurately \textcolor{black}{and compare its performance to that of of multiple-ring setups, such as FingerSpeller \cite{martin2023fingerspeller}.} We aimed to assess the system's performance across a range of users with varying ASL proficiencies, ranging from novice to native. This study also sought to investigate the impact of signing speed and individual habits on recognition accuracy, providing important insights for system optimization and user adaptation strategies.

\subsection{Participants}
We recruited 9 participants (6 male, 3 female, mean age = 23.0, SD = 5.45) to evaluate our system, including 3 Deaf individuals, 1 CODA (Child of Deaf Adults), and 5 hearing ASL learners. The 3 Deaf participants and CODA were fluent signers, using ASL as a primary language. The 5 hearing ASL learners had between 1 and 5 years of ASL learning experience (M = 3.0  SD = 1.4). All participants fingerspelled with their right hand as their dominant hand. Detailed participant information, including proficiency and background, is provided in Table~\ref{fig:table2}.
% \begin{table}[]
% \caption{Participant Information: C: CODA (Child of Deaf), H: Hearing Individual, HHD: Deaf and Hard-of-Hearing individual. Proficiency of fingerspelling was self-reported by participants on a scale of 1 to 7, with 7 being the highest}
% \begin{tabular}{l|l|r|l}
% \hline
% Participants          & ASL Experience & \multicolumn{1}{l|}{Proficiency} & Gender \\ \hline
% ASL learner, H  & 3 years        & 6                                & F \\ \hline
% ASL learner, H & 5 years +      & 7                                & M   \\ \hline
% ASL learner, H  & 3 years        & 6                                & F \\ \hline
% ASL learner, H  & 1 year         & 5                                & M   \\ \hline
% ASL learner, H & 3 years        & 6                                & F \\ \hline
% native, H, CODA & 5 years +      & 7                                & M   \\ \hline
% native, HHD           & 5 years +      & 7                                & M   \\ \hline
% native, HHD           & 5 years +      & 7                                & M \\ \hline
% native, HHD           & 5 years +      & 7                                & M   \\ \hline
% \end{tabular}
% \label{table:participants}
% \end{table}


% Please add the following required packages to your document preamble:
% \usepackage[table,xcdraw]{xcolor}
% Beamer presentation requires \usepackage{colortbl} instead of \usepackage[table,xcdraw]{xcolor}
% \begin{table}[]
% \caption{Participant Information: C: CODA (Child of Deaf), H: Hearing Individual, HHD: Deaf and Hard-of-Hearing individual. Experience indicates ASL}
% \begin{tabular}{ccccc}
% \hline
% \#  & Participant & Experience & Proficiency & Gender \\ \hline
% P01 & ASL learner, H                       & 3 Y                                    & 6           & F      \\ \hline
% P02 & ASL learner, H                       & 5 Y +                                  & 7           & M      \\ \hline
% P03 & ASL learner, H                       & 3 Y                                    & 6           & F      \\ \hline
% P04 & ASL learner, H                       & 1 Y                                    & 5           & M      \\ \hline
% P05 & ASL learner, H                       & 3 Y                                    & 6           & F      \\ \hline
% P06 & native, H, CODA                      & 5 Y +                                  & 7           & M      \\ \hline
% P07 & native, HHD                          & 5 Y +                                  & 7           & M      \\ \hline
% P08 & native, HHD                          & 5 Y +                                  & 7           & M      \\ \hline
% P09 & native, HHD                          & 5 Y +                                  & 7           & M     
% \end{tabular}
% \end{table}
% \label{table:participants}

% %https://www.tablesgenerator.com/#
% \begin{table}[]
% \caption{Participant Information: C: CODA (Child of Deaf), H: Hearing Individual, D: Deaf Individual. (Years of Learning ASL) Proficiency of fingerspelling was self-reported by participants on a scale of 1 to 7, with 7 being the highest}
% \begin{tabular}{|c|c|c|c|c|c|c|c|c|c|}
% \hline
%     & p01 & p02  & p03  & p04  & p05  & p06  & p07 & p08 & p09 \\ \hline
% Background      & C (5+)   & H (3) & H (1) & H (5) & H (1) & H (3) & D (5+)  & D (5+)    & D (5+)   \\ \hline

% Gender     & M   & M  & M  & F  & M  & F  & M   & M   & F  \\ \hline
% Proficiency     & 7   & 5    & 6    & 6    & 5    & 5    & 7   & 7   & 7   \\ \hline
% % Word Per Minute & 48.01 & 32.77   &  31.04  &24.94& 24.02  &  28.87  & 39.35  &  37.12  & 50.89  \\ \hline



% \end{tabular}
% \label{table:participants}
% \end{table}
% To explore the usefulness of this approach, we compared emulated fngerspelling recognition
% with a virtual keyboard for 12 Deaf participants. With practice, fngerspelling is faster (42.5 wpm), potentially
% has fewer errors (4.02\% corrected error rate) and higher throughput (14.2 bits/second),

\subsection{Dataset and Procedure}
\label{experiment}

\begin{figure}[t]
  \includegraphics[width=\linewidth]{document/figures/studyprocedure.png}
  \caption{Study Procedures: an English word with guide fingerspelling images (left) and examples of experimental setup locations (e.g., in a study room (center); in a home (right)) }
  \Description{Study Procedures: a word with guide fingerspelling images and the study locations (in a study room and a home)}
  \label{fig:studyprocedure}
\end{figure}

\textcolor{black}{To ensure consistency with FingerSpeller \cite{martin2023fingerspeller}, particularly for performance evaluation, we adopted their dataset and procedure. Since we evaluated our system as an accessible optional text entry method, the MacKenzie-Soukoreff Phrase Set \cite{mackenzie2003phrase} was selected as the standard dataset, which is commonly used to evaluate text entry systems.} This set comprises 500 phrases, totaling 1,164 unique words. The words in this set vary in length, ranging from a minimum of 1 to a maximum of 13, with an average word length of 6.055 (SD: 2.312). 


Each participant was tasked with fingerspelling each of these words twice, resulting in a comprehensive dataset of 2328=2*1164 fingerspelled words per participant, which was collected in two rounds of data collection. During each round, a partcipant completed each word once across 10 sessions. Nine of these sessions contained 116 words each, and one session contained 120 words. We carefully ensured that there was no overlap in words between sessions and that each session maintained a similar distribution of letters and word lengths. The average word length across sessions was 6.053 (SD = 0.181), with median values of 5 or 6.

The experiment was conducted in a semi-controlled environment, as illustrated in Figure \ref{fig:studyprocedure}. During each session, participants followed specific guidelines for fingerspelling. They were provided with real-time video feedback of their signing and were allowed to correct any mistakes as needed. To aid ASL learners, we displayed each English word with accompanying guide images of corresponding ASL letters. Participants used their non-dominant hand to press a space key after completing each word, which allowed us to record the start and end times for each fingerspelled sequence to calculate participants' fingerspelling speed. We instructed participants to return to a neutral hand pose between words and to fingerspell at their own comfortable, natural pace. Session durations varied from 5 to 13 minutes, depending on the participant's fingerspelling speed.

Data collection for each participant lasted around 4 hours, split across two to three days for each participant. To assess the effect of device positioning, participants were asked to remove and reattach the ring between sessions. This approach allowed us to collect data under various ring positions, simulating real-world usage scenarios.

\begin{table*}[t]
\caption{Top-N Word Recognition and Participant Information: CODA: Child of Deaf Adults, H: Hearing. Years indicates number of years learning ASL or using it as a primary language.}
\Description{Top-N Word Recognition and Participant Information: CODA (Child of Deaf Adults), H: Hearing. Years indicates number of years learning ASL or using it as a primary language.}
\begin{tabular}{cccc|cccccr}
\hline
\multicolumn{4}{c|}{}                                                                                                  & \multicolumn{6}{c}{Offline Evaulation}                                                                                                                                                                                    \\
\multicolumn{4}{c|}{}                                                                                                  & \multicolumn{6}{c}{\textbf{Word-level}}                                                                                                                                                                                   \\ \hline
\multicolumn{1}{c|}{}     & \multicolumn{1}{c|}{\textbf{Experience}} & \multicolumn{1}{c|}{\textbf{Gender}} & \textbf{Years} & \multicolumn{1}{c|}{\textbf{Top1}} & \multicolumn{1}{c|}{\textbf{Top2}} & \multicolumn{1}{c|}{\textbf{Top3}} & \multicolumn{1}{c|}{\textbf{Top4}} & \multicolumn{1}{c|}{\textbf{Top5}} & \multicolumn{1}{c}{\textbf{LER}} \\ \hline
\multicolumn{1}{c|}{\textcolor{black}{Avg.}} & \multicolumn{1}{c|}{}               & \multicolumn{1}{c|}{}                & \textit{5.44} & \multicolumn{1}{c|}{\textit{89.8}} & \multicolumn{1}{c|}{\textit{92.8}} & \multicolumn{1}{c|}{\textit{94.3}} & \multicolumn{1}{c|}{\textit{95.1}} & \multicolumn{1}{c|}{\textit{95.6}} & \textit{0.131}                   \\ \hline

\multicolumn{1}{c|}{P01}  & \multicolumn{1}{c|}{Learner, H}     & \multicolumn{1}{c|}{F}               & 3             & \multicolumn{1}{c|}{95.338}        & \multicolumn{1}{c|}{97.220}        & \multicolumn{1}{c|}{98.162}        & \multicolumn{1}{c|}{98.501}        & \multicolumn{1}{c|}{98.846}        & 0.068                            \\ \hline
\multicolumn{1}{c|}{P02}  & \multicolumn{1}{c|}{Learner, H}     & \multicolumn{1}{c|}{M}               & 5             & \multicolumn{1}{c|}{89.161}        & \multicolumn{1}{c|}{92.943}        & \multicolumn{1}{c|}{94.796}        & \multicolumn{1}{c|}{95.612}        & \multicolumn{1}{c|}{96.172}        & 0.126                            \\ \hline
\multicolumn{1}{c|}{P03}  & \multicolumn{1}{c|}{Learner, H}     & \multicolumn{1}{c|}{F}               & 3             & \multicolumn{1}{c|}{94.546}        & \multicolumn{1}{c|}{96.565}        & \multicolumn{1}{c|}{97.772}        & \multicolumn{1}{c|}{98.155}        & \multicolumn{1}{c|}{98.371}        & 0.155                            \\ \hline
\multicolumn{1}{c|}{P04}  & \multicolumn{1}{c|}{Learner, H}     & \multicolumn{1}{c|}{M}               & 1             & \multicolumn{1}{c|}{97.851}        & \multicolumn{1}{c|}{98.968}        & \multicolumn{1}{c|}{99.226}        & \multicolumn{1}{c|}{99.527}        & \multicolumn{1}{c|}{99.614}        & 0.042                            \\ \hline
\multicolumn{1}{c|}{P05}  & \multicolumn{1}{c|}{Learner, H}     & \multicolumn{1}{c|}{F}               & 3             & \multicolumn{1}{c|}{95.020}        & \multicolumn{1}{c|}{96.823}        & \multicolumn{1}{c|}{97.599}        & \multicolumn{1}{c|}{98.026}        & \multicolumn{1}{c|}{98.499}        & 0.089                            \\ \hline
\multicolumn{1}{c|}{P06}  & \multicolumn{1}{c|}{Deaf}           & \multicolumn{1}{c|}{M}               & 7             & \multicolumn{1}{c|}{79.840}        & \multicolumn{1}{c|}{83.822}        & \multicolumn{1}{c|}{86.136}        & \multicolumn{1}{c|}{87.358}        & \multicolumn{1}{c|}{87.945}        & 0.223                            \\ \hline
\multicolumn{1}{c|}{P07}  & \multicolumn{1}{c|}{CODA, H}        & \multicolumn{1}{c|}{M}               & 10            & \multicolumn{1}{c|}{87.884}        & \multicolumn{1}{c|}{91.705}        & \multicolumn{1}{c|}{93.641}        & \multicolumn{1}{c|}{94.628}        & \multicolumn{1}{c|}{95.187}        & 0.162                            \\ \hline
\multicolumn{1}{c|}{P08}  & \multicolumn{1}{c|}{Deaf}           & \multicolumn{1}{c|}{M}               & 10            & \multicolumn{1}{c|}{77.636}        & \multicolumn{1}{c|}{83.739}        & \multicolumn{1}{c|}{86.901}        & \multicolumn{1}{c|}{88.662}        & \multicolumn{1}{c|}{89.837}        & 0.211                            \\ \hline
\multicolumn{1}{c|}{P09}  & \multicolumn{1}{c|}{Deaf}           & \multicolumn{1}{c|}{M}               & 7             & \multicolumn{1}{c|}{90.688}        & \multicolumn{1}{c|}{92.993}        & \multicolumn{1}{c|}{94.397}        & \multicolumn{1}{c|}{95.665}        & \multicolumn{1}{c|}{96.345}        & 0.105                            \\ \hline
\end{tabular}
\end{table*}
\label{fig:performance_tops}


We collected a total of 20,604 fingerspelled words for system evaluation from the nine participants. However, due to technical issues, we lost data from three sessions: the 7th session of P1 (116 words), the 7th session of P7 (116 words), and the 5th session of P8 (116 words).




\subsection{Evaluation Metrics}

For evaluating recognition accuracy, we utilized two primary metrics: Letter Error Rate (LER) and word-level accuracy. LER measures the percentage of incorrect letters in the output compared to the ground truth, with a lower value indicating better accuracy. Word-level accuracy was assessed based on top-1 to top-5 predictions, providing a comprehensive view of the model's effectiveness.


\subsubsection{Letter Error Rate (LER)}

LER is a metric used to evaluate the accuracy of a system in recognizing or generating sequences of letters, such as in speech recognition, handwriting recognition, or fingerspelling recognition. It measures the percentage of incorrect letters in the output compared to a reference or ground truth. A lower LER indicates better accuracy in recognizing or generating a sequence of letters. We calculate LER before word correction. For example, when the target word is "hello" and the predicted word from the model is "helo", there is no substitution, one deletion (the second "l" is missing in the predicted word), and no insertion. In this case, the LER is 1/5 = 0.2.

\begin{equation}
LER = \frac{{\text{Substitutions} + \text{Deletions} + \text{Insertions}}}{{\text{Total number of letters in the reference sequence}}}
\end{equation}



\subsubsection{Top N Word-level Accuracy}
We also report word-level accuracy based on top-1 and top-5 predictions, as this provides a more comprehensive evaluation of the model's effectiveness. After collecting the word data, we calculate performance by determining accuracy, defined as the number of correctly predicted words divided by the total number of words. Additionally, we identify the top-3 potential words by selecting those with the smallest Levenshtein distances. For top-3 accuracy, if the correct word is among the three with the smallest Levenshtein distances, the prediction is considered correct; otherwise, it is incorrect. For example, if the input word is "fax" and the model predicts "aax", the three closest words by Levenshtein distance could be "fox", "tax", and "fax." In this case, the system's prediction would be considered correct.




\subsection{Results}

Our analysis revealed that SpellRing achieved a promising overall accuracy of 89.89\% (SD = 8.59\%) for top-1 predictions and 95.72\% (SD = 5.28\%) for top-5 predictions in recognizing the 1,164 fingerspelled words in our dataset. This performance is comparable to \textcolor{black}{FingerSpeller \cite{martin2023fingerspeller} using two rings} (87\% accuracy), demonstrating the effectiveness of our single-ring approach.

\subsubsection{Top-N Word Recognition}

SpellRing's recognition accuracy improved with increasing N in top-N predictions. For top-1 predictions, the system achieved 89.89\% (SD = 8.59\%) accuracy. This improved to 92.85\% (SD = 7.00\%) for top-2, 94.37\% (SD = 6.13\%) for top-3, 95.20\% (SD = 5.56\%) for top-4, and reached 95.72\% (SD = 5.28\%) for top-5 predictions.
Table 3 illustrates the top-N word accuracy for each participant. The significant improvement from top-1 to top-5 accuracy (a 5.83\% increase) suggests potential benefits for ASL translation applications. By considering multiple top predictions, the system could leverage additional contextual information to produce more coherent and contextually appropriate sentences. This approach could help maintain the flow and meaning of the text by selecting from the best few options at each step. Furthermore, a user interface displaying the top three candidates immediately after prediction would allow signers to choose the correct option, potentially improving overall system accuracy.



\begin{figure}[b]
  \includegraphics[width=0.8\linewidth]{document/figures/output_completion.png}
  \caption{Offline Word-level Performance over Completion Time in User study 1. Dots indicate each session. Note that the model tended to have lower performance for faster signers and better performance for slower signers.}
  \Description{Offline Word-level Performance over Completion Time in User study 1. Dots indicate each sessions. Note that faster signers tended to have lower performance, while slower signers tended to perform better.}
  \label{fig:performance_time}
\end{figure}




\subsubsection{ASL Learners vs. Native/ Fluent Signers}

We observed significant variations in performance across participants, with top-1 accuracy ranging from 77.63\% to 97.85\%. This variability can be attributed to differences in participants' fingerspelling proficiency, affecting articulatory habits and speed. Notably, the model showed better performance for ASL learners (M = 94.38\%, SD = 4.28\%) than for native/ fluent signers (M = 84.06\%, SD = 9.26\%). We attribute this difference primarily to variations in fingerspelling speed.  ASL learners took longer to complete each session (M = 428.88 sec, SD = 74.24) compared to fluent signers (M = 329.624 sec, SD = 42.54). ASL learners tended to fingerspell more slowly, clearly distinguishing each letter, while native/ fluent signers fingerspelled more quickly, resulting in greater handshape variation for certain letters and blurred transitions between letters. This speed difference posed challenges for the model in accurately recognizing individual letters in rapid sequences. We further analyze the impact of fingerspelling speed on performance in the following section.

\subsubsection{Impact of Fingerspelling Speed on Performance}

Fingerspelling speed varied among participants, leading to differences in task completion times. These speed variations, along with factors such as participants' prior experience signing specific words and breaks taken between sessions, appeared to influence the model's performance (Figure \ref{fig:performance_time}). Specifically, the model had lower performance for faster signers and performed better for slower signers.

We found that faster signing speeds often led to greater handshape variation, potentially impacting our model's performance. For instance, when signed quickly, some letters (e.g., ‘C’, ‘O’, ‘E’, ‘I’) were often not fully articulated. The contrast in fingerspelling between the highest and lowest performing participants illustrates this effect. P05, an ASL learner with 3 years' experience, achieved the highest performance with 97.85\% (SD = 2.07\%) accuracy. They articulated each letter very clearly and exhibited minimal handshape variation, resulting in consistent data. In contrast, P08, a fluent signer, had the lowest performance at 77.64\% (SD = 7.8\%). Their fingerspelling speed varied across sessions and consisted of many allophonic handshape variations.

These observations suggest that as fingerspelling speed increases, individual letters are articulated less fully and/or are coarticulated with neighboring letters, posing a challenge for accurate recognition. This challenge resembles those in early-stage speech recognition systems, which experienced performance drops when speakers spoke too rapidly or with strong accents. We discuss this in more detail in Section \ref{impact_speed}.









\section{Phrase-level Recognition}

\subsection{Purpose and Overview}
Building on the insights from our word-level study, our second investigation aimed to evaluate SpellRing's performance for real-time phrase-level recognition. We sought to understand how the system performs in more natural contexts, how users adapt their signing behavior to real-time feedback, and the effectiveness of language models in improving recognition accuracy. It is worth noting that most prior work \cite{paudyal2017dyfav, mummadi2017real} using wearables does not evaluate fingerspelling recognition continuously in real-time. This real-time performance study was a crucial step in assessing SpellRing's potential for practical, everyday use \textcolor{black}{in comprehensive ASL recognition systems.}

\subsection{Participants}
We recruited 11 participants (4 male, 7 female, mean age = 32.0, SD = 3.88) to evaluate our system, consisting of 8 Deaf individuals, 1 ASL interpreter, and 2 hearing ASL learners. The 8 Deaf participants use ASL as their primary language are fluent ASL signers and fingerspellers. The 2 hearing participants had been learning ASL for 1 and 2 years, respectively, contributing to differences in their fingerspelling proficiency. All participants fingerspelled using their right hand as their dominant hand. Detailed information about participants’ proficiency and background is provided in Table \ref{fig:table2}.






\subsection{Iteration on Hardware Prototype Design}
To evaluate our system in a more natural fingerspelling environment, we redesigned the ring prototype to a smaller form factor, enabling real-time evaluation. Our design optimized the device for comfort and ease of use while ensuring it supported continuous and real-time tracking for natural fingerspelling. \textcolor{black}{As shown in Figure \ref{fig:prototype2}, audio data from the FPCB microphone \textcolor{black}{(d)} connected to our custom nRF MCU \textcolor{black}{(b)} and gyro data from the IMU \textcolor{black}{(c)} were synchronized and sent to an off-the-shelf ESP32 S3 microcontroller \footnote{Adafruit QT Py ESP32-S3 WiFi Dev Board with STEMMA QT - 8 MB Flash} \textcolor{black}{(a)}.} This data was then transmitted via the wire to a backend system for processing through our machine learning pipeline, running on a MacBook Pro. The raw predictions were processed through autocorrection and language model pipelines to generate the final output.
\begin{figure}[t]
  \includegraphics[width=\linewidth]{document/figures/new_prototying.png}
  \caption{Prototype for real-time phrase-level evaluation: (a) ESP32 S3 microcontroller, (b) nRF MCU, (c) IMU sensor, (d) FPCB, and (e) a 3D-printed ring-shaped design}
  \Description{Prototype for real-time phrase-level evaluation: (a) ESP32 S3 microcontroller, (b) nRF MCU, (c) IMU sensor, (d) FPCB, and (e) a 3D-printed ring-shaped design}
  \label{fig:prototype2}
\end{figure}


\subsection{Language Model}
With our dataset, we used an N-gram language model to correct fingerspelled words within phrases. Based on the LM model described in \cite{zhu2018typing}, we generated a list of top N words (N = 20) along with their similarity values after autocorrecting a raw predicted word. For each new predicted word in the top N, we applied bigram and trigram probabilities and selected the word set with the highest probability for the final predicted phrase. We then calcuated WER between the ground truth phrases and the final predicted phrases for evaluation. 










\subsection{Dataset and Procedure}

For our phrase-level prediction evaluation, we again utilized the MacKenzie-Soukoreff Phrase Set \cite{mackenzie2003phrase}, as in the first user study. The phrases ranged from 16 to 40 characters in length, consisting of 4 to 8 words each. Our study procedure consisted of two main phases: initial data collection and real-time evaluation.



\subsubsection{Phase 1: Training Data Collection}
We first collected training data from all 11 participants, following a procedure similar to Study 1. Each participant provided two rounds of training data for 1,164 words (2*1164). Our training process involves a two-step
approach: first training with data from all participants except one, then retraining with the specific participant’s data
 for real-time phrase evaluation. Each participant provided the two rounds of training data over two separate days. 
\begin{figure}[t]
  \includegraphics[width=0.8\linewidth]{document/figures/interface_study2.png}
  \caption{User Interface for Real-time Evaluation: The ground truth phrases are displayed in red, and the predicted phrases appear based on each fingerspelled word. Participants receive status updates below, such as `start signing' and `processing.'  }
  \Description{ser Interface for Real-time Evaluation: The ground truth phrases are displayed in red, and the predicted phrases appear based on each fingerspelled word. Participants receive status updates below, such as 'start signing' and 'processing.'  }
  \label{fig:interface_study2}
\end{figure}
\subsubsection{Phase 2: Real-time Evaluation}
\begin{table*}[t]
\caption{Offline evaluation of 1,164 word-level recognition with WPM, Top N and LER, and real-time evaluation of 100 phrase-level recognition with WER and WPM, G: Gender, H: Hearing, DHH: Deaf or Hard of Hearing. Year: Indicates ASL experience (years learning ASL or using it as a primary language)}
\Description{Offline evaluation of 1,164 word-level recognition with WPM, Top N and LER, and real-time evaluation of 100 phrase-level recognition with WER and WPM,G: Gender, H: Hearing, DHH: Deaf or Hard of Hearing. Year: Indicates ASL experience (years learning ASL or using it as a primary language)}
\begin{tabular}{cccc|ccccccc|cc}
\hline
\textbf{}                      & \textbf{}                           & \textbf{}                            & \textbf{}     & \multicolumn{7}{c|}{Offline}                                                                                                                                                                                                                      & \multicolumn{2}{c}{Real-Time}                       \\
\textbf{}                      & \textbf{}                           & \textbf{}                            & \textbf{}     & \multicolumn{7}{c|}{\textbf{Word-level}}                                                                                                                                                                                                          & \multicolumn{2}{c}{\textbf{Phrase-level}}           \\ \hline
\multicolumn{1}{c|}{\textbf{}} & \multicolumn{1}{c|}{\textbf{Types}} & \multicolumn{1}{c|}{\textbf{G}} & \textbf{Year} & \multicolumn{1}{c|}{\textbf{Top 1}} & \multicolumn{1}{c|}{\textbf{Top 2}} & \multicolumn{1}{c|}{\textbf{Top 3}} & \multicolumn{1}{c|}{\textbf{Top 4}} & \multicolumn{1}{c|}{\textbf{Top 5}} & \multicolumn{1}{c|}{\textbf{LER}}   & \textbf{WPM}  & \multicolumn{1}{c|}{\textbf{WER}}   & \textbf{WPM}  \\ \hline
\multicolumn{1}{c|}{\textcolor{black}{Avg.}}       & \multicolumn{1}{c|}{}               & \multicolumn{1}{c|}{}                & \textit{11}   & \multicolumn{1}{c|}{\textit{82.60}} & \multicolumn{1}{c|}{\textit{87.45}} & \multicolumn{1}{c|}{\textit{90.19}} & \multicolumn{1}{c|}{\textit{91.56}} & \multicolumn{1}{c|}{\textit{92.54}} & \multicolumn{1}{c|}{\textit{0.149}} & \textit{39.9} & \multicolumn{1}{c|}{\textit{0.099}} & \textit{20.1} \\ \hline
\multicolumn{1}{c|}{P01}       & \multicolumn{1}{c|}{Deaf}           & \multicolumn{1}{c|}{M}               & 10            & \multicolumn{1}{c|}{84.87}          & \multicolumn{1}{c|}{89.17}          & \multicolumn{1}{c|}{92.43}          & \multicolumn{1}{c|}{93.64}          & \multicolumn{1}{c|}{94.58}          & \multicolumn{1}{c|}{0.127}          & 32.1          & \multicolumn{1}{c|}{0.041}          & 22.3          \\ \hline
\multicolumn{1}{c|}{P02}       & \multicolumn{1}{c|}{DHH}            & \multicolumn{1}{c|}{F}               & 5             & \multicolumn{1}{c|}{87.21}          & \multicolumn{1}{c|}{90.54}          & \multicolumn{1}{c|}{93.03}          & \multicolumn{1}{c|}{94.84}          & \multicolumn{1}{c|}{95.87}          & \multicolumn{1}{c|}{0.122}          & 39.5          & \multicolumn{1}{c|}{0.124}          & 19.8          \\ \hline
\multicolumn{1}{c|}{P03}       & \multicolumn{1}{c|}{Deaf}           & \multicolumn{1}{c|}{F}               & 37            & \multicolumn{1}{c|}{77.49}          & \multicolumn{1}{c|}{83.33}          & \multicolumn{1}{c|}{87.03}          & \multicolumn{1}{c|}{88.66}          & \multicolumn{1}{c|}{90.29}          & \multicolumn{1}{c|}{0.192}          & 47.9          & \multicolumn{1}{c|}{0.112}          & 20.3          \\ \hline
\multicolumn{1}{c|}{P04}       & \multicolumn{1}{c|}{Deaf}           & \multicolumn{1}{c|}{F}               & 10            & \multicolumn{1}{c|}{77.03}          & \multicolumn{1}{c|}{82.79}          & \multicolumn{1}{c|}{85.62}          & \multicolumn{1}{c|}{87.43}          & \multicolumn{1}{c|}{88.38}          & \multicolumn{1}{c|}{0.193}          & 52.5          & \multicolumn{1}{c|}{0.103}          & 19.7          \\ \hline
\multicolumn{1}{c|}{P05}       & \multicolumn{1}{c|}{Deaf}           & \multicolumn{1}{c|}{M}               & 5             & \multicolumn{1}{c|}{94.40}          & \multicolumn{1}{c|}{97.41}          & \multicolumn{1}{c|}{98.13}          & \multicolumn{1}{c|}{98.71}          & \multicolumn{1}{c|}{98.99}          & \multicolumn{1}{c|}{0.062}          & 34.5          & \multicolumn{1}{c|}{0.061}          & 20.2          \\ \hline
\multicolumn{1}{c|}{P06}       & \multicolumn{1}{c|}{Deaf}           & \multicolumn{1}{c|}{F}               & 22            & \multicolumn{1}{c|}{67.18}          & \multicolumn{1}{c|}{76.63}          & \multicolumn{1}{c|}{79.96}          & \multicolumn{1}{c|}{82.00}          & \multicolumn{1}{c|}{83.82}          & \multicolumn{1}{c|}{0.234}          & 56.8          & \multicolumn{1}{c|}{0.134}          & 24.9          \\ \hline
\multicolumn{1}{c|}{P07}       & \multicolumn{1}{c|}{Deaf}           & \multicolumn{1}{c|}{F}               & 5             & \multicolumn{1}{c|}{90.93}          & \multicolumn{1}{c|}{93.76}          & \multicolumn{1}{c|}{95.65}          & \multicolumn{1}{c|}{96.42}          & \multicolumn{1}{c|}{96.85}          & \multicolumn{1}{c|}{0.1}            & 29.4          & \multicolumn{1}{c|}{0.093}          & 21.3          \\ \hline
\multicolumn{1}{c|}{P08}       & \multicolumn{1}{c|}{Deaf}           & \multicolumn{1}{c|}{M}               & 17            & \multicolumn{1}{c|}{65.27}          & \multicolumn{1}{c|}{72.53}          & \multicolumn{1}{c|}{77.21}          & \multicolumn{1}{c|}{80.17}          & \multicolumn{1}{c|}{81.59}          & \multicolumn{1}{c|}{0.26}           & 52.4          & \multicolumn{1}{c|}{0.17}           & 17.9          \\ \hline
\multicolumn{1}{c|}{P09}       & \multicolumn{1}{c|}{Intepreter, H}  & \multicolumn{1}{c|}{F}               & 10            & \multicolumn{1}{c|}{85.94}          & \multicolumn{1}{c|}{89.67}          & \multicolumn{1}{c|}{93.09}          & \multicolumn{1}{c|}{93.37}          & \multicolumn{1}{c|}{94.86}          & \multicolumn{1}{c|}{0.119}          & 30.1          & \multicolumn{1}{c|}{-}              & -             \\ \hline
\multicolumn{1}{c|}{P10}       & \multicolumn{1}{c|}{Learner, H}     & \multicolumn{1}{c|}{M}               & 2             & \multicolumn{1}{c|}{86.41}          & \multicolumn{1}{c|}{91.92}          & \multicolumn{1}{c|}{94.24}          & \multicolumn{1}{c|}{95.53}          & \multicolumn{1}{c|}{95.95}          & \multicolumn{1}{c|}{0.129}          & 40.2          & \multicolumn{1}{c|}{0.052}          & 20.3          \\ \hline
\multicolumn{1}{c|}{P11}       & \multicolumn{1}{c|}{Learner, H}     & \multicolumn{1}{c|}{F}               & 1             & \multicolumn{1}{c|}{91.92}          & \multicolumn{1}{c|}{94.15}          & \multicolumn{1}{c|}{95.70}          & \multicolumn{1}{c|}{96.38}          & \multicolumn{1}{c|}{96.73}          & \multicolumn{1}{c|}{0.098}          & 23.1          & \multicolumn{1}{c|}{0.101}          & 14.1          \\ \hline
\end{tabular}
\end{table*}
\label{fig:table2}
The real-time evaluation was conducted on a third day.  In the real-time evaluation, we randomly selected 200 phrases generated from the 1164 unique words for our study. We used 100 phrases for practice and the remaining 100 phrases for testing. 

We began with practice sessions, where participants were given 100 phrases to familiarize themselves with our interface (Fig. \ref{fig:interface_study2}) and the real-time feedback mechanism. This preparatory step ensured that participants were comfortable with the system before the actual evaluation. For the evaluation, we used the remaining 100 phrases. Participants fingerspelled these phrases over the course of 5 sessions in natural environments such as their homes or quiet rooms. During each evaluation session, participants fingerspelled according to the prompts illustrated in Figure \ref{fig:interface_study2}. 

Participants were first shown the phrase, and instructed to begin fingerspelling each word after pressing the space key. As they fingerspelled, the real-time prediction model provided immediate feedback by displaying the predicted output in green on the interface.  Participants were instructed to proceed to the next word even if they saw a mispredicted word on the screen to ensure that they continued to fingerspell each word as displayed. The language model sometimes corrected a mispredicted word as signers completed more words in each phrase.


After completing each phrase, participants pressed the space key with their non-dominant hand to display the next phrase, allowing them to see and prepare for it. Once ready, they pressed the key to start the phrase and pressed it again upon completion. This action served a dual purpose: it advanced the interface to the next phrase and also marked the start and end times for fingerspelling. This timing information allowed us to calculate fingerspelling speed for each phrase by minimizing perception time, allowing for more precise estimation of participants' fingerspelling speed and any adjustments they made in response to real-time feedback. The duration of each session varied based on the participant's fingerspelling proficiency and typically ranged from 10 to 12 minutes. This variation in session length allowed us to accommodate different signing speeds and ensure that all participants could complete the phrases comfortably.

In total, we collected data on 993 phrases from 10 participants. Due to technical issues, we lost data for seven phrases, and one participant was unable to complete the entire study. Despite these minor setbacks, the collected data provided a robust basis for evaluating our system's performance in real-time, continuous fingerspelling recognition. This two-phase approach allowed us to first train our system on participant-specific data and then evaluate its performance in a realistic, real-time scenario. 







\subsection{Evaluation Metrics}
To evaluate our system, we use Word Error Rate (WER) to report performance. The WER metric ranges from 0 to 1, where 0 indicates that the compared texts are identical, and 1 indicates that they are completely different with no similarity. For example, a WER of 0.10 means there is a 10\% error rate in the compared sentences. WER is based on Levenshtein distance, but it operates at the word level instead of the phoneme (or in this case, letter) level.


\begin{equation}
WER = \frac{{\text{Substitutions} + \text{Deletions} + \text{Insertions}}}{{\text{Total number of words in the reference phrase}}}
\end{equation}








\subsection{Results}
Overall, \theDevice{} recognized 100 phrases with a WER of 0.099\% (0.039\%). While word-level performance achieved an average LER of 0.149\%, phrase-level WER improved with use of a language model. 


\subsubsection{Recognition Performance}
Our results show that fingerspelled words are better recognized within the context of a phrase using a language model (See Table 4). The model showed lower performance for faster signers, such as P06 and P08, with a top-1 accuracy of 67.27\% and 65\%, and WERs of 0.134 and 0.17; this translates to approximately 15\% error on the phrases. Compared to word-level recognition performance, this offers an improvement in recognizing fingerspelled words \textcolor{black}{by applying corrections using a language model at phrase-level recognition.} 



\subsubsection{Signing Speed in Phrase-Level Prediction}

We observed that participants adjusted their signing speed and habits according to the predictions displayed on the user interface, leading to a decrease in words per minute (WPM) for these participants, with WPM averages ranging from 39.87 (data collected in training phase) to 20.09 (data collected in real-time phase) (See Table 3). This decrease accounts for latency, including model processing time, participants' reaction times, and participants' fingerspelling more slowly in response to prediction accuracy.



\subsubsection{Qualitative Analysis}
Participants were asked open-ended survey questions regarding their overall experience with \theDevice{} in terms of performance, form factor, and usability. For performance, 8 out of 10 participants reported that the system performed well with the language model, even when phrases were entirely misclassified. They noted that short fingerspelled words (e.g., "a," "I," "am", "be") were not always recognized accurately but could be corrected by the language model when more context was available. However, participants noted that the system did not always work well at first. In these cases, we observed changes in participant behavior based on predicted results; they tended to fingerspell more slowly and distinctly immediately after they encountered recognition errors. \textcolor{black}{Although our offline evaluation demonstrated that the system works reliably without requiring participants to alter their signing behavior, real-time evaluation revealed that participants adjusted their fingerspelling habits dynamically to accommodate the system. Specifically, they slowed their fingerspelling immediately after observing misrecognized words but returned to their natural habits when the system performed accurately.}



After some practice sessions, some participants adjusted their fingerspelling speed and habits to accommodate the system. P01, P02, P05, and P07 stated that they focused on spelling clearly and distinctly without skipping letters. This directly contrasted their natural fingerspelling behavior, which often involved partially articulated letters and quick, seamless transitions between letters. 


%\section{Results}
Overall, SpellRing recognizes 1,164 fingerspelling words with of 89.89\% (8.59\%) and 95.72\% (5.28\%) in top-1 and top-5 estimations, respectively. 
% LER is 0.089 (0.151) - before. LER is 0.055 (0.179) - after correction. 
Our work using only single ring showed the comparable performance when a users using five rings \cite{martin2023fingerspeller} with top-1 accuracy, i.e., 87.0\%. 



\subsection{Top-N Word Recognition}

SpellRing recognizes 1,164 fingerspelling words with the average accuracies of 89.89\% (8.59\%), 92.85\% (7.00\%), 94.37\% (6.13\%), 95.20\% (5.56\%), and 95.72\% (5.28\%) in top 1, 2, 3, 4, and 5 estimations, respectively. Fig \ref{fig:performance_tops} shows that top N word accuracy for each participant. Seeing that top 5 accuarcy greatly surpass the top 1 accuracy, yielding a 5.83\% increase in word accuarcy, we consider a ASL translation, where additional contextual information before or after fingerspelling contributes to more coherent and contextually appropriate ASL sentences. This approach helps maintain the flow and meaning of the text by selecting from the best few options at each step. Additionally, if the UI displays the top three candidates immediately after prediction, it allows signers to choose the correct one. 

\begin{figure}
  \includegraphics[width=\linewidth]{document/figures/output_result.png}
  \caption{Top-N Word Recognition}
  \Description{}
  \label{fig:performance_tops}
\end{figure}



\subsection{ASL Learners vs Native Signers}
Due to the variation in participants' fingerspelling abilities (e.g. habits and speed), performance varies among participants ranging from 77.63\% to 97.85\% on top 1 accuracy. (e.g. fingerspelling handshape variation and fingerspelling speed). 

The model shows better performance of Top 1 on ASL learners (M = 94.38\%, SD = 4.28\% ) than native signers (M = 84.06\%, SD = 9.26\%). We attribute this variation between ASL learners and native signers to their different fingerspelling speeds. The completion on each session takes longer on ASL learners  (M = 428.88 sec, SD = 74.24) compared to native singers (M = 329.624 sec, SD = 42.54). ASL learners were likely to perform fingerspelling clearly on each letters while native singers do fingerspelling quickly, causing hand shape variation and blur transition between letters. It pose a challenges for model to accurately recognize individual letter. 






\subsection{Performance over Fingerspelling Speed}
Each participant has a different fingerspelling speed, resulting in varied task completion times. Additionally, task completion times depend on several factors, such as the participant's prior experience with signing the words and the breaks taken between sessions. We found that these task completion times might affect our performance, as shown in Fig \ref{fig:performance_time}. We also found that signing speed caused many hand shape variations, which might have affected our model's performance. For example, to sign quickly, some letters (e.g., 'C', 'O', 'E', 'I') are often not fully formed in their hand shapes.

P05 achieved the highest performance with 97.85\% (2.07\%), while P08 had the lowest performance at 77.64\% (7.8\%). The significant difference between these two participants was their signing speed and the clarity of their fingerspelling. P05 is an ASL learner whose signing is very clear for each letter, with no hand shape variation, resulting in consistent data. On the other hand, P08's signing speed varied across sessions, leading to many hand shape variations. We believe that when signers increase their signing speed, they tend to alter their hand shapes or not fully form the hand shapes between letters. 


\begin{figure}
  \includegraphics[width=\linewidth]{document/figures/output_completion.png}
  \caption{Performance over Completion Time. Dots indicate each sessions }
  \Description{}
  \label{fig:performance_time}
\end{figure}

\section{Discussion}
\omniUIST is capable of tracking a passive tool with an accuracy of roughly 6.9 mm and, at the same time, deliver a maximum force of up to 2 N to the tool. This is enabled by our novel gradient-based approach in 3D position reconstruction that accounts for the force exerted by the electromagnet. 

Over extended periods of time, \omniUIST can comfortably produce a force of 0.615 N without the risk of overheating. In our applications, we show that \omniUIST has the potential for a wide range of usage scenarios, specifically to enrich AR and VR interactions.

\omniUIST is, however, not limited to spatial applications. We believe that \omniUIST can be a valuable addition to desktop interfaces, e.g., navigating through video editing tools or gaming. We plan to broaden \omniUIST's usage scenarios in the future.

The overall tracking performance of \omniUIST suffices for interactive applications such as the ones shown in this paper. The accuracy could be improved by adding more Hall sensors, or optimizing their placement further (e.g., placing them on the outer hull of the device).
Furthermore, a spherical tip on the passive tool that more closely resembles the dipole in our magnetic model could further improve \omniUIST's accuracy. We believe, however, that the design of \omniUIST represents a good balance of cost and complexity of manufacturing, and accuracy.

Our current implementation of \omniUIST and the accompanying tracking and actuation algorithms assumes the presence of a single passive tool. Our method, however, potentially generalizes to tracking multiple passive tools by accounting for the presence of multiple permanent magnets. This poses another interesting challenge: the magnets of multiple tools will interact with each other, i.e., attract and repel each other.The electromagnet will also jointly interact with those tools, leading to challenges in terms of computation and convergence. We believe that our gradient-based optimization can account for such interactions and plan to investigate this in the future.

In developing and testing our applications, we found that \omniUIST's current frame rate of 40 Hz suffices for many interactive scenarios. The frame rate is a trade-off between speed and accuracy. In our tests, decreasing the desired accuracy in our optimization doubled the frame rate, while resulting in errors in the 3D position estimation of more than 1 cm, however. Finding the sweet spot for this trade-off depends on the application. While our applications worked well with 40 Hz and the current accuracy, more intricate actions such as high-precision sculpting might benefit from higher frame rates \textit{and} precision.
Reducing the latency of several system components (e.g., sensor latency, convergence time) is another interesting direction of future research. 

Furthermore, the control strategy we used was fairly naïve, as it only takes the current tool position into account. A model predictive strategy could account for future states, user intent, and optimize to reduce heating. We will explore in the next chapter how model predictive approaches can be used for haptic systems.

Overall, the main benefits of \omniUIST lie in the high accuracy and large force it can produce. It does so without mechanically moving parts, which would be subject to wear.
Such wear is not the case for our device, because it is exclusively based on electromagnetic force. We believe that different form factors of \omniUIST (e.g., body-mounted, larger size) can present interesting directions of future research. \add{A body-mounted version could be interesting for VR applications in which the user moves in 3D space. The larger size could result in more discernible points.}

Additionally, the influence of strength on user perception and factors such as just-noticeable-difference will allow us to characterize the benefits and challenges of \omniUIST, and electromagnetic haptic devices in general.
We believe that \omniUIST opens interesting directions for future research in terms of novel devices, and magnetic actuation and tracking.
\section{Conclusion}
We introduced \Bench, the first ever IMTS forecasting benchmark.
\Bench's datasets are created with ODE models, that were defined in decades of research and published on
the Physiome Model Repository. Our experiments showed that LinODEnet and CRU are actually
better than previous evaluation on established datasets indicated. Nevertheless,
we also provided a few datasets, on which models are unable to outperform a
constant baseline model. We believe that our datasets, especially the very difficult ones,
can help to identify deficits of current architectures and support future research on
IMTS forecasting.




%%
%% The next two lines define the bibliography style to be used, and
%% the bibliography file.
\bibliographystyle{ACM-Reference-Format}
\bibliography{main}
\end{document}
\endinput
%%
%% End of file `sample-authordraft.tex'.
