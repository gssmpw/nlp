%%
%% This is file `sample-manuscript.tex',
%% generated with the docstrip utility.
%%
%% The original source files were:
%%
%% samples.dtx  (with options: `manuscript')
%% 
%% IMPORTANT NOTICE:
%% 
%% For the copyright see the source file.
%% 
%% Any modified versions of this file must be renamed
%% with new filenames distinct from sample-manuscript.tex.
%% 
%% For distribution of the original source see the terms
%% for copying and modification in the file samples.dtx.
%% 
%% This generated file may be distributed as long as the
%% original source files, as listed above, are part of the
%% same distribution. (The sources need not necessarily be
%% in the same archive or directory.)
%%
%% Commands for TeXCount
%TC:macro \cite [option:text,text]
%TC:macro \citep [option:text,text]
%TC:macro \citet [option:text,text]
%TC:envir table 0 1
%TC:envir table* 0 1
%TC:envir tabular [ignore] word
%TC:envir displaymath 0 word
%TC:envir math 0 word
%TC:envir comment 0 0
%%
%%
%% The first command in your LaTeX source must be the \documentclass command.
%%%% Small single column format, used for CIE, CSUR, DTRAP, JACM, JDIQ, JEA, JERIC, JETC, PACMCGIT, TAAS, TACCESS, TACO, TALG, TALLIP (formerly TALIP), TCPS, TDSCI, TEAC, TECS, TELO, THRI, TIIS, TIOT, TISSEC, TIST, TKDD, TMIS, TOCE, TOCHI, TOCL, TOCS, TOCT, TODAES, TODS, TOIS, TOIT, TOMACS, TOMM (formerly TOMCCAP), TOMPECS, TOMS, TOPC, TOPLAS, TOPS, TOS, TOSEM, TOSN, TQC, TRETS, TSAS, TSC, TSLP, TWEB.
% \documentclass[acmsmall]{acmart}

%%%% Large single column format, used for IMWUT, JOCCH, PACMPL, POMACS, TAP, PACMHCI
% \documentclass[acmlarge,screen]{acmart}

%%%% Large double column format, used for TOG
% \documentclass[acmtog, authorversion]{acmart}

%%%% Generic manuscript mode, required for submission
%%%% and peer review
 %%\documentclass[manuscript,review,anonymous]{acmart}
 \documentclass[sigconf]{acmart}

%\documentclass[sigconf,authordraft]{acmart}

%%\documentclass[manuscript]{acmart}
\usepackage{subfig}
\usepackage{graphicx}
\usepackage{subcaption}
\newcommand{\theDevice}{SpellRing}

%% Fonts used in the template cannot be substituted; margin 
%% adjustments are not allowed.
%%
%% \BibTeX command to typeset BibTeX logo in the docs
\AtBeginDocument{%
  \providecommand\BibTeX{{%
    \normalfont B\kern-0.5em{\scshape i\kern-0.25em b}\kern-0.8em\TeX}}}

%% Rights management information.  This information is sent to you
%% when you complete the rights form.  These commands have SAMPLE
%% values in them; it is your responsibility as an author to replace
%% the commands and values with those provided to you when you
%% complete the rights form.
\copyrightyear{2025}
\acmYear{2025}
\setcopyright{acmlicensed}\acmConference[CHI '25]{CHI Conference on Human Factors in Computing Systems}{April 26-May 1, 2025}{Yokohama, Japan}
\acmBooktitle{CHI Conference on Human Factors in Computing Systems (CHI '25), April 26-May 1, 2025, Yokohama, Japan}
\acmDOI{10.1145/3706598.3713721}
\acmISBN{979-8-4007-1394-1/25/04}


%%
%% Submission ID.
%% Use this when submitting an article to a sponsored event. You'll
%% receive a unique submission ID from the organizers
%% of the event, and this ID should be used as the parameter to this command.
%%\acmSubmissionID{123-A56-BU3}

%%
%% For managing citations, it is recommended to use bibliography
%% files in BibTeX format.
%%
%% You can then either use BibTeX with the ACM-Reference-Format style,
%% or BibLaTeX with the acmnumeric or acmauthoryear sytles, that include
%% support for advanced citation of software artefact from the
%% biblatex-software package, also separately available on CTAN.
%%
%% Look at the sample-*-biblatex.tex files for templates showcasing
%% the biblatex styles.
%%

%%
%% The majority of ACM publications use numbered citations and
%% references.  The command \citestyle{authoryear} switches to the
%% "author year" style.
%%
%% If you are preparing content for an event
%% sponsored by ACM SIGGRAPH, you must use the "author year" style of
%% citations and references.
%% Uncommenting
%% the next command will enable that style.
%%\citestyle{acmauthoryear}

%%
%% end of the preamble, start of the body of the document source.
\begin{document}

%%
%% The "title" command has an optional parameter,
%% allowing the author to define a "short title" to be used in page headers.
\title{\theDevice: Recognizing Continuous Fingerspelling in American Sign Language using a Ring}

%
% The "author" command and its associated commands are used to define
% the authors and their affiliations.
% Of note is the shared affiliation of the first two authors, and the
% "authornote" and "authornotemark" commands
% used to denote shared contribution to the research.


\author{Hyunchul Lim, Nam Anh Dang, Dylan Lee, Tianhong Catherine Yu, Jane Lu, Franklin Mingzhe Li$^*$, Yiqi Jin, Yan Ma$^\dagger$, Xiaojun Bi$^\dagger$, François Guimbretière, and Cheng Zhang}
\email{{hl2365,nd433,dl634,ty274,jdl332,yj226,fvg3, chengzhang}@cornell.edu}
\email{mingzhe2@cs.cmu.edu, {yanma1,xiaojun}@cs.stonybrook.edu}

\orcid{0000-0001-8397-3534}
\affiliation{%
  \institution{Cornell University, Ithaca, New York, USA \\ Carnegie Mellon University$^*$, Pittsburgh, Pennsylvania, USA \\Stony Brook University$^\dagger$, Stony Brook, New York, USA}
   \country{ }}

% \affiliation{%
%   \institution{Carnegie Mellon University$^*$, Pittsburgh, Pennsylvania}
% }
% \affiliation{%
%   \institution{Stony Brook University$^\dagger$, Stony Brook, New York}
%

% \author{Hyunchul Lim}
% \email{{hl2365,nd433,dl634}@cornell.edu}
% \orcid{0000-0001-8397-3534}
% \author{Nam Anh Dang}
% %\email{nd433@cornell.edu}
% \orcid{0009-0005-8080-9268}
% \author{Dylan Lee}
% %\email{dl634@cornell.edu}
% \orcid{0009-0000-9382-468X}
% \affiliation{%
%   \institution{Cornell University}
%   \city{Ithaca}
%   \state{New York}
%   \country{USA}
%   \postcode{14850}
% }

% \author{Tianhong Catherine Yu}
% \email{{ty274,jdl332}@cornell.edu}
% \orcid{0000-0002-3742-0178}
% \author{Jane Lu}
% %\email{jdl332@cornell.edu}
% \orcid{0009-0000-9972-863X}
% %\author{Yiqi Jin}
% %\email{yj226@cornell.edu}
% \affiliation{%
%   \institution{Cornell University}
%   \city{Ithaca}
%   \state{New York}
%   \country{USA}
%   \postcode{14850}
% }

% \author{Franklin Mingzhe Li}
% \email{mingzhe2@cs.cmu.edu}
% \orcid{0000-0003-4995-4545}
% \affiliation{%
%   \institution{Carnegie Mellon University}
%   \city{Pittsburgh}
%   \state{Pennsylvania}
%   \country{USA}
% }

% \author{Yiqi Jin}
% \email{yj226@cornell.edu}
% \orcid{0009-0004-3727-5742}
% \affiliation{%
%   \institution{Cornell University}
%   \city{Ithaca}
%   \state{New York}
%   \country{USA}
%   \postcode{14850}
% }

% \author{Yan Ma}
% \email{{yanma1,xiaojun}@cs.stonybrook.edu}
% \orcid{0000-0001-8264-3103}
% \author{Xiaojun Bi}
% %\email{xiaojun@cs.stonybrook.edu}
% \orcid{0000-0002-9716-7709}
% \affiliation{%
%   \institution{Stony Brook University}
%   \city{Stony Brook}
%   \state{New York}
%   \country{USA}
% }
% \author{Francois Guimbretiere}
% \email{francois@cs.cornell.edu}
% \orcid{0000-0002-5510-6799}
% \author{Cheng Zhang}
% \email{chengzhang@cornell.edu}
% \orcid{0000-0002-5079-5927}
% \affiliation{%
%   \institution{Cornell University}
%   \city{Ithaca}
%   \state{New York}
%   \country{USA}
%   \postcode{14850}
% }
%
% By default, the full list of authors will be used in the page
% headers. Often, this list is too long, and will overlap
% other information printed in the page headers. This command allows
% the author to define a more concise list
% of authors' names for this purpose.

\renewcommand{\shortauthors}{Hyunchul Lim, et al.}

%%
%% The abstract is a short summary of the work to be presented in the
%% article.
 End-to-end imitation learning offers a promising approach for training robot policies. However, generalizing to new settings—such as unseen scenes, tasks, and object instances—remains a significant challenge. Although large-scale robot demonstration datasets have shown potential for inducing generalization, they are resource-intensive to scale. In contrast, human video data is abundant and diverse, presenting an attractive alternative. Yet, these human-video datasets lack action labels, complicating their use in imitation learning. Existing methods attempt to extract grounded action representations (e.g., hand poses), but resulting policies struggle to bridge the embodiment gap between human and robot actions.
% our approach
We propose an alternative approach: leveraging language-based reasoning from human videos - essential for guiding robot actions - to train generalizable robot policies. Building on recent advances in reasoning-based policy architectures, we introduce Reasoning through Action-free Data (RAD). RAD learns from both robot demonstration data (with reasoning and action labels) and action-free human video data (with only reasoning labels). The robot data teaches the model to map reasoning to low-level actions, while the action-free data enhances reasoning capabilities. Additionally, we will release a new dataset of 3,377 human-hand demonstrations compatible with the Bridge V2 benchmark. This dataset includes chain-of-thought reasoning annotations and hand-tracking data to help facilitate future work on reasoning-driven robot learning.
% experiments
Our experiments demonstrate that RAD enables effective transfer across the embodiment gap, allowing robots to perform tasks seen only in action-free data. Furthermore, scaling up action-free reasoning data significantly improves policy performance and generalization to novel tasks. These results highlight the promise of reasoning-driven learning from action-free datasets for advancing generalizable robot control. 
% releasing dataset
Website: \href{https://rad-generalization.github.io}{here}.


%%
%% The code below is generated by the tool at http://dl.acm.org/ccs.cfm.
%% Please copy and paste the code instead of the example below.
%%
% \begin{CCSXML}
% <ccs2012>
%  <concept>
%   <concept_id>10010520.10010553.10010562</concept_id>
%   <concept_desc>Computer systems organization~Embedded systems</concept_desc>
%   <concept_significance>500</concept_significance>
%  </concept>
%  <concept>
%   <concept_id>10010520.10010575.10010755</concept_id>
%   <concept_desc>Computer systems organization~Redundancy</concept_desc>
%   <concept_significance>300</concept_significance>
%  </concept>
%  <concept>
%   <concept_id>10010520.10010553.10010554</concept_id>
%   <concept_desc>Computer systems organization~Robotics</concept_desc>
%   <concept_significance>100</concept_significance>
%  </concept>
%  <concept>
%   <concept_id>10003033.10003083.10003095</concept_id>
%   <concept_desc>Networks~Network reliability</concept_desc>
%   <concept_significance>100</concept_significance>
%  </concept>
% </ccs2012>
% \end{CCSXML}

% \ccsdesc[500]{Computer systems organization~Embedded systems}
% \ccsdesc[300]{Computer systems organization~Redundancy}
% \ccsdesc{Computer systems organization~Robotics}
% \ccsdesc[100]{Networks~Network reliability}


\begin{CCSXML}
<ccs2012>
<concept>
<concept_id>10003120.10011738.10011775</concept_id>
<concept_desc>Human-centered computing~Accessibility technologies</concept_desc>
<concept_significance>500</concept_significance>
</concept>
<concept>
<concept_id>10003120.10003121.10003128.10011755</concept_id>
<concept_desc>Human-centered computing~Gestural input</concept_desc>
<concept_significance>500</concept_significance>
</concept>
</ccs2012>
\end{CCSXML}

\ccsdesc[500]{Human-centered computing~Accessibility technologies}
\ccsdesc[500]{Human-centered computing~Gestural input}


%%
%% Keywords. The author(s) should pick words that accurately describe
%% the work being presented. Separate the keywords with commas.
\keywords{datasets, neural networks, gaze detection, text tagging}

%% A "teaser" image appears between the author and affiliation
%% information and the body of the document, and typically spans the
%% page.


\begin{teaserfigure}
  \includegraphics[width=1\linewidth]{document/figures/main.png}
  \caption{\theDevice{} is a smart ring that uses an AI-powered sensing system designed to recognize continuous fingerspelling}
  \Description{}
  \label{fig:main}
\end{teaserfigure}

% \begin{teaserfigure}
%   \includegraphics[width=\textwidth]{document/figures/jake.png}
%   \caption{Seattle Mariners at Spring Training, 2010.}
%   \Description{Enjoying the baseball game from the third-base
%   seats. Ichiro Suzuki preparing to bat.}
%   \label{fig:teaser}
% \end{teaserfigure}

% \received{20 February 2007}
% \received[revised]{12 March 2009}
% \received[accepted]{5 June 2009}

%%
%% This command processes the author and affiliation and title
%% information and builds the first part of the formatted document.
\maketitle



\begin{figure}[ht]
    \centering
    \includegraphics[width=0.8\linewidth]{graphs/greater_than_naive.pdf}
    \vspace{0.5cm}
    \includegraphics[width=0.8\linewidth]{graphs/p1_bottom.png}
    \vspace{-5pt}
    \caption{\textcolor{positional}{Positional} vs.\ \textcolor{nonpositional}{non-positional} circuits. In a \textcolor{nonpositional}{non-positional} circuit, the same edges must be included at all positions. A \textcolor{positional}{positional} circuit can distinguish between the same edge at different positions. This specificity yields better trade-offs between circuit size and faithfulness. It can also increase both precision and recall.}
    \label{fig:p1}
    \vspace{-5pt}
\end{figure}

\section{Introduction}

\looseness=-1
A primary goal of interpretability research is to characterize the internal mechanisms in language models (LMs) and other NLP models. 
A core approach in this area is \textbf{circuit discovery}---identifying the minimal subgraph within the model's computation graph that performs a specific task \citep{olah2021framework,olah-mech}.
Typically, the nodes of a circuit represent model components (e.g., attention heads, neurons, or layers).
While manual circuit discovery methods can yield position-specific insights \citep{wanginterpretability,goldowskydill2023localizingmodelbehaviorpath}, \emph{automatic methods often overlook positional information}, treating components as uniformly relevant across all input token positions \citep{conmytowards,syed2023attribution}. 
For instance, if an attention head is included in a circuit, it is assumed to contribute equally to the computation for every position in the input sequence.
The assumption that circuits are position-invariant ignores the fact that different positions often require distinct computations.
By ignoring positions, current methods limit their ability to capture mechanisms that operate across positions, such as interactions between attention heads across positions.

In this study, we start by demonstrating that positional agnosticism is a significant limitation (\S\ref{sec:motivating}). Then, to address these limitations, we introduce a new approach: position-aware edge attribution patching (PEAP; \S\ref{sec:full_circ_discovery}; Figure~\ref{fig:p1}). Current approaches  assume that if an edge is in a circuit, then the same edge will be in the circuit at all positions, thus leading to low precision. It is also assumed that an edge's importance should be aggregated across positions before deciding whether it should be included in the circuit; this can lead to cancellation effects, and thus low recall. PEAP instead allows us to compute the importance of cross-positional edges, and separately evaluates edge importance at each position. We show that this leads to smaller and more accurate circuits; see Figure~\ref{fig:p1}.

Incorporating positional information into circuit discovery is straightforward when inputs have the same length and structure across examples.

However, realistic datasets are not nearly this templatic.
How, then, can we incorporate positional information into automatic circuit discovery?
To address this challenge, we propose \textbf{schemas} (\S\ref{sec:schema}). 
Schemas assign semantic labels to spans of tokens, enabling information aggregation across examples even when the spans differ in length.

For example, in the input ``The \textcolor{positional}{war} lasted from 1453 to 14\underline{\hspace{1em}},'' the span ``\textcolor{positional}{war}'' could be labeled as ``\emph{Subject}''.
This enables handling spans with varying lengths: the phrase ``\textcolor{positional}{Black Plague}'' in another example can be treated as a single positional span with the same role as ``\textcolor{positional}{war}''.
In experiments with two LMs and three tasks, we find that circuits discovered using schemas achieve a better trade-off between circuit size and faithfulness to the model's behavior than position-agnostic circuits.
Importantly, position-aware circuits offer a more precise representation of the underlying mechanisms, providing a more concise foundation for mechanistic explanations.

We also present a fully automated pipeline for schema generation and application (\S\ref{sec:schema-generation}) using large language models (LLMs). 
We evaluate the quality of the generated schemas and their utility in discovering position-aware circuits (\S\ref{sec:schema-eval}).
Notably, circuits derived using automatically generated and applied schemas achieve comparable faithfulness scores to circuits discovered with human-designed and manually applied schemas.

We summarize our contributions as follows:
\begin{itemize}[noitemsep,leftmargin=*,topsep=1pt,parsep=1pt]
    \item Introduce a position-aware circuit discovery method, which obtains better faithfulness than position-agnostic discovery.  
    \item Introduce dataset schemas,  facilitating positional circuit discovery in more naturalistic settings. 
    \item Develop an automated schema generation and application pipeline with LLMs, yielding schemas that are comparable to manually-annotated ones.
\end{itemize}


\section{Related work}


Recent advances in single-image animatable head avatar generation can be categorized into mainly 2D-based and 3D-based approaches. 

\paragraph{\bf Image to 2D Animatable Avatar.}
2D-based methods, leveraging the power of convolutional neural networks (CNNs)~\cite{DBLP:conf/cvpr/KarrasLAHLA20,DBLP:conf/cvpr/IsolaZZE17,DBLP:conf/nips/GoodfellowPMXWOCB14}, often employ generative adversarial networks (GANs)~\cite{DBLP:conf/cvpr/StyleGAN} for direct image synthesis. Early approaches~\cite{DBLP:conf/cvpr/WangDYSW23,DBLP:conf/cvpr/BurkovPGL20,DBLP:conf/iccv/ZakharovSBL19} focus on injecting expression and pose features into the generator network, often utilizing architectures like U-Net or StyleGAN~\cite{DBLP:conf/cvpr/StyleGAN}.
Some other 2D methods~\cite{DBLP:journals/corr/abs-2407-03168,DBLP:conf/cvpr/ZhangQZZW0CW023,DBLP:conf/cvpr/HongZS022,DBLP:conf/mm/DrobyshevCKILZ22,DBLP:conf/cvpr/BurkovPGL20,DBLP:conf/nips/SiarohinLT0S19} represent expressions and poses as warping fields applied to the source image. 
Benefiting from advances in image and video diffusion networks, more recent 2D-based works~\cite{DBLP:journals/corr/abs-2410-07718,DBLP:journals/corr/abs-2406-08801,DBLP:conf/eccv/TianWZB24} get improved results with diffusion techniques. 
However, these methods still face challenges related to long generation times and significant computational resource demands. Audio-driven 2D control methods~\cite{DBLP:conf/cvpr/ZhangCWZSGSW23,DBLP:journals/corr/abs-2211-12368,DBLP:conf/iccv/GuoCLLBZ21} are easy to use but cannot explicitly control facial expressions and poses. 2D-based techniques often struggle with large pose or expression variations due to the lack of an explicit 3D structure, sometimes producing unrealistic distortions or identity changes. While some 2D methods~\cite{SadTalker,StyleHEAT,Pirenderer,DBLP:conf/cvpr/WangM021,MegaPortraits} incorporate 3D Morphable Models (3DMMs)~\cite{DBLP:conf/fgr/GerigMBELSV18,DBLP:journals/tog/LiBBL017,DBLP:conf/avss/PaysanKARV09,DBLP:conf/siggraph/BlanzV99} to mitigate these issues, they typically cannot achieve free-viewpoint rendering. 

\vspace{-0.1in}

\begin{figure*}[h]
    \centering
    \includegraphics[width=0.9\linewidth]{images/framework.pdf}
    \caption{\textbf{Overall Framework.} Our framework utilizes learnable query features attached to FLAME vertices to perform cross-attention with the extracted multi-level image features. The extracted features are then decoded to reconstruct the Gaussian avatar in the canonical space, which can be animated utilizing standard linear blend skinning (LBS) and corrective blendshapes as the FLAME model did and rendered in real-time on various platforms.}
    \label{fig:framework}
\end{figure*}

\paragraph{\bf Image to 3D Animatable Avatar.}
3D-aware methods offer improved geometric consistency and free-viewpoint rendering capabilities. Early 3D approaches~\cite{DBLP:conf/eccv/KhakhulinSLZ22,DBLP:conf/cvpr/XuYCWDJT20} utilize 3DMMs for head avatar reconstruction. With the advent of Neural Radiance Fields (NeRFs)~\cite{DBLP:conf/eccv/MildenhallSTBRN20}, many recent methods~\cite{DBLP:conf/siggraph/YuFZWYBCSWSW23,DBLP:conf/cvpr/MaZQLZ23,DBLP:conf/cvpr/LiZWZ0CZWB023,GPAvatar,ye2024real3d,deng2024portrait4d,deng2024portrait4d2,DBLP:conf/eccv/KiMC24,DBLP:conf/cvpr/BaiFWZSYS23,PointAvatar,Nerfies,INSTA} have adopted this representation for higher fidelity, particularly in modeling fine details like hair. However, NeRF-based~\cite{DBLP:conf/cvpr/ZhangZLHLWGCL024,HAvatar,DBLP:conf/cvpr/BaiTHSTQMDDOPTB23,AD-NeRF,DBLP:journals/tog/GaoZXHGZ22,DBLP:journals/tog/ParkSHBBGMS21,DBLP:conf/cvpr/AtharXSSS22,DBLP:journals/corr/abs-2112-05637,DBLP:conf/iccv/TretschkTGZLT21,DBLP:conf/cvpr/GafniTZN21,DBLP:conf/eccv/KiMC24,DBLP:conf/cvpr/BaiFWZSYS23,PointAvatar,Nerfies,DBLP:conf/siggraph/YuFZWYBCSWSW23,DBLP:conf/cvpr/MaZQLZ23,DBLP:conf/cvpr/LiZWZ0CZWB023} approaches often require extensive training data, including multi-view or single-view videos, raising privacy concerns and limiting generalization to unseen identities. Some methods~\cite{DBLP:conf/cvpr/SunWWLZZL23,DBLP:conf/3dim/ZhuangMKS22,DBLP:journals/pami/SunWZHWL24,DBLP:journals/tvcg/TangZYZCMW24,DBLP:conf/iclr/XuZLZBFS23} bypass this data requirement by training generators with random noise and then inverting them for identity-specific reconstruction, but inversion accuracy remains a challenge. Test-time optimization offers another alternative, but its computational cost limits practical applications. Several recent works~\cite{goha2023,hidenerf2023,gpavatar2024,ye2024real3d,ma2024cvthead,deng2024portrait4d,deng2024portrait4d2,GGHead} have explored one-shot 3D head reconstruction to address the limitations of data requirements and computational cost. These methods employ various techniques, such as tri-plane features, deformation fields, point-based expression fields, and vertex-feature transformers. Despite these advancements, NeRF-based methods often struggle with real-time rendering. 
Recently, 3D Gaussian Splatting~\cite{GaussianSplatting} has emerged as a promising alternative, offering both high-quality results and fast rendering speeds. However, existing Gaussian Splatting methods~\cite{GaussianAvatar,DBLP:conf/cvpr/XuCL00ZL24} typically rely on video data for training for each person, limiting their ability to generalize to new identities. Instead, the most recent work, GAGAvatar~\cite{GAGAvatar}, proposes a one-shot 3D Gaussian-based head avatar generation method. However, it still relies heavily on complex 2D neural post-processing to achieve optimal animation outcomes, thus it is not a pure 3D solution and the extra neural network hinders its application on various platforms. In contrast, our work generates Gaussian heads that are immediately animatable and renderable without additional networks or post-processing steps, enabling seamless integration into existing rendering pipelines for real-time animation and rendering across a wide range of platforms, including mobile phones. 
\section{Cyri Architectural Design}
\label{sec:design}
Cyri represents an innovative AI-powered conversational assistant designed to
help users detect and analyze phishing attacks within email communications. By leveraging a refined Large Language Model (LLM) through prompt
engineering and Chain of Thought techniques, Cyri provides users with detailed explanations of the suspicious features that could make an email potentially malicious, as well as the necessary countermeasures. Cyri’s architecture is composed of three main components:
\begin{enumerate}
\item LLM-based Interactive Semantic Analyzer component (LISA): it performs in-depth email analysis using a local Large Language Model (LLM) through APIs.
\item E-mail client plugin: it captures incoming emails and communicates with the
Electron application. A demonstrator is implemented for Thunderbird since It is an open-source email client that offers extensive customization capabilities through its support for add-ons~\cite{b32}.
\item Visual and Audio Conversational interface (VAC): it serves as the user interface, manages data storage, and allows a non-expert user to analyze the classification of e-mails, the main semantic reasons, and inquire more on it in both interactive visual and audio means. It is implemented as an Electron web application for generality and usability.
\end{enumerate}

Cyri continuously monitors incoming emails through the email client plugin. When a new, unseen email arrives, the plugin extracts essential data such as the sender's information, subject, body content, a flag indicating whether the sender is in the user's contacts, the message ID, and the timestamp. This data is then transmitted to both the VAC interface application and to the LISA component to perform an in-depth analysis of the email (locally hosted), specifically using the Meta-Llama-3.1-8B-Instruct model~\cite{b29}. LISA evaluates the email for semantic features such as urgency, authority, instant gratification, and others, all collected from the literature or extracted by LISA itself. LISA is helped by a sub-component for links checking that, using external APIs, specifically Google Safe Browsing~\cite{b30} and AbuseIPDB~\cite{b31}, enhance detection capabilities by checking only links and domains against known malicious entities and provides additional context to the semantic analysis.\\
Upon completion of the analysis, the results are stored in the VAC application and sent to the e-mail client plugin for e-mail text tagging and classification as ``Phishing'' or ``Safe'' (see Figure~\ref{fig:thunderbirdexample}). Finally, the user is presented with this information in the VAC interface and can explore it as explanations and converse with Cyri with a mix of visual cues and audio.

\begin{figure}[htbp]
  \centering
  \includegraphics[width=0.48\textwidth]{figures/ThunderbirdExample.PNG}
  \caption{Cyri email Plugin Example using the Thunderbird email client}
  \label{fig:thunderbirdexample}
\end{figure}

Figure~\ref{fig:EmailAnalysisArchitecture} illustrates the process by which Cyri analyzes the semantically tagged email in the VAC interface.

\begin{figure}[htbp]
  \centering
  \includegraphics[width=0.45\textwidth]{figures/EmailAnalysisArchitecture.pdf}
  \caption{Cyri Architecture and Data Flow for Email Analysis}
  \label{fig:EmailAnalysisArchitecture}
\end{figure}

The user can monitor newly tagged emails, interact with the detailed analysis through visual means, and issue further queries. User queries are processed interactively by the LISA component, which generates responses based on the conversation history with the user and initial semantic analysis of the e-mails, taking into account the user's inputs and questions. The whole process is visible in Figure~\ref{fig:ConversationArchitecture}.

%The diagram below  represents the interaction between the user and Cyri during the query process.

\begin{figure}[htbp]
  \centering
  \includegraphics[width=0.45\textwidth]{figures/ConversationArchitecture.pdf}
  \caption{User Interaction and Query Processing Flow}
  \label{fig:ConversationArchitecture}
\end{figure}

%The Thunderbird plugin communicates with the Electron application using HTTP POST and GET requests. This method is employed to send email data and to poll for analysis results, providing a reliable and straightforward communication channel.

%The Electron application and the Python background LLM processing component interact via the CyriShared folder. This shared storage serves as a repository for email data, analysis results, and user queries and responses. A strict file naming convention is followed to ensure proper identification and handling of files.

Finally, security and privacy are integral to Cyri’s design, especially given the sensitive
nature of email content. Data privacy is ensured thanks to local processing and
minimal external data sharing (URLs and domains) for safety checks.
All email analyses are conducted locally on the user’s machine. Using a locally
hosted LLM ensures that sensitive information remains within the
user’s environment, mitigating the risk of data breaches. The LISA component implements the Hugging Face Transformers library to load and utilize the Llama 3.1 8B model locally.

\section{User Studies Overview}
To evaluate the effectiveness and usability of \theDevice, we conducted two user studies with signers ranging from novice to native/ fluent in ASL proficiency. The first study focused on word-level recognition, assessing the system's ability to accurately recognize individual fingerspelled words across a diverse group of participants. The second study expanded on these findings by examining phrase-level recognition in real-time scenarios, providing insights into the system's performance in more natural, context-rich environments. These studies aimed to validate SpellRing's performance across different levels of signing experience, explore the impact of signing speed and habits on recognition accuracy, and investigate how users adapt to the system in real-time use. The user study was approved by the Institutional Review Board (IRB) at the authors' institution. Participants were compensated \$40 per hour for their participation in the study.


\section{Word-level Recognition}

\subsection{Purpose and Overview}
The primary objective of this study was to evaluate SpellRing's ability to recognize individual fingerspelled words accurately \textcolor{black}{and compare its performance to that of of multiple-ring setups, such as FingerSpeller \cite{martin2023fingerspeller}.} We aimed to assess the system's performance across a range of users with varying ASL proficiencies, ranging from novice to native. This study also sought to investigate the impact of signing speed and individual habits on recognition accuracy, providing important insights for system optimization and user adaptation strategies.

\subsection{Participants}
We recruited 9 participants (6 male, 3 female, mean age = 23.0, SD = 5.45) to evaluate our system, including 3 Deaf individuals, 1 CODA (Child of Deaf Adults), and 5 hearing ASL learners. The 3 Deaf participants and CODA were fluent signers, using ASL as a primary language. The 5 hearing ASL learners had between 1 and 5 years of ASL learning experience (M = 3.0  SD = 1.4). All participants fingerspelled with their right hand as their dominant hand. Detailed participant information, including proficiency and background, is provided in Table~\ref{fig:table2}.
% \begin{table}[]
% \caption{Participant Information: C: CODA (Child of Deaf), H: Hearing Individual, HHD: Deaf and Hard-of-Hearing individual. Proficiency of fingerspelling was self-reported by participants on a scale of 1 to 7, with 7 being the highest}
% \begin{tabular}{l|l|r|l}
% \hline
% Participants          & ASL Experience & \multicolumn{1}{l|}{Proficiency} & Gender \\ \hline
% ASL learner, H  & 3 years        & 6                                & F \\ \hline
% ASL learner, H & 5 years +      & 7                                & M   \\ \hline
% ASL learner, H  & 3 years        & 6                                & F \\ \hline
% ASL learner, H  & 1 year         & 5                                & M   \\ \hline
% ASL learner, H & 3 years        & 6                                & F \\ \hline
% native, H, CODA & 5 years +      & 7                                & M   \\ \hline
% native, HHD           & 5 years +      & 7                                & M   \\ \hline
% native, HHD           & 5 years +      & 7                                & M \\ \hline
% native, HHD           & 5 years +      & 7                                & M   \\ \hline
% \end{tabular}
% \label{table:participants}
% \end{table}


% Please add the following required packages to your document preamble:
% \usepackage[table,xcdraw]{xcolor}
% Beamer presentation requires \usepackage{colortbl} instead of \usepackage[table,xcdraw]{xcolor}
% \begin{table}[]
% \caption{Participant Information: C: CODA (Child of Deaf), H: Hearing Individual, HHD: Deaf and Hard-of-Hearing individual. Experience indicates ASL}
% \begin{tabular}{ccccc}
% \hline
% \#  & Participant & Experience & Proficiency & Gender \\ \hline
% P01 & ASL learner, H                       & 3 Y                                    & 6           & F      \\ \hline
% P02 & ASL learner, H                       & 5 Y +                                  & 7           & M      \\ \hline
% P03 & ASL learner, H                       & 3 Y                                    & 6           & F      \\ \hline
% P04 & ASL learner, H                       & 1 Y                                    & 5           & M      \\ \hline
% P05 & ASL learner, H                       & 3 Y                                    & 6           & F      \\ \hline
% P06 & native, H, CODA                      & 5 Y +                                  & 7           & M      \\ \hline
% P07 & native, HHD                          & 5 Y +                                  & 7           & M      \\ \hline
% P08 & native, HHD                          & 5 Y +                                  & 7           & M      \\ \hline
% P09 & native, HHD                          & 5 Y +                                  & 7           & M     
% \end{tabular}
% \end{table}
% \label{table:participants}

% %https://www.tablesgenerator.com/#
% \begin{table}[]
% \caption{Participant Information: C: CODA (Child of Deaf), H: Hearing Individual, D: Deaf Individual. (Years of Learning ASL) Proficiency of fingerspelling was self-reported by participants on a scale of 1 to 7, with 7 being the highest}
% \begin{tabular}{|c|c|c|c|c|c|c|c|c|c|}
% \hline
%     & p01 & p02  & p03  & p04  & p05  & p06  & p07 & p08 & p09 \\ \hline
% Background      & C (5+)   & H (3) & H (1) & H (5) & H (1) & H (3) & D (5+)  & D (5+)    & D (5+)   \\ \hline

% Gender     & M   & M  & M  & F  & M  & F  & M   & M   & F  \\ \hline
% Proficiency     & 7   & 5    & 6    & 6    & 5    & 5    & 7   & 7   & 7   \\ \hline
% % Word Per Minute & 48.01 & 32.77   &  31.04  &24.94& 24.02  &  28.87  & 39.35  &  37.12  & 50.89  \\ \hline



% \end{tabular}
% \label{table:participants}
% \end{table}
% To explore the usefulness of this approach, we compared emulated fngerspelling recognition
% with a virtual keyboard for 12 Deaf participants. With practice, fngerspelling is faster (42.5 wpm), potentially
% has fewer errors (4.02\% corrected error rate) and higher throughput (14.2 bits/second),

\subsection{Dataset and Procedure}
\label{experiment}

\begin{figure}[t]
  \includegraphics[width=\linewidth]{document/figures/studyprocedure.png}
  \caption{Study Procedures: an English word with guide fingerspelling images (left) and examples of experimental setup locations (e.g., in a study room (center); in a home (right)) }
  \Description{Study Procedures: a word with guide fingerspelling images and the study locations (in a study room and a home)}
  \label{fig:studyprocedure}
\end{figure}

\textcolor{black}{To ensure consistency with FingerSpeller \cite{martin2023fingerspeller}, particularly for performance evaluation, we adopted their dataset and procedure. Since we evaluated our system as an accessible optional text entry method, the MacKenzie-Soukoreff Phrase Set \cite{mackenzie2003phrase} was selected as the standard dataset, which is commonly used to evaluate text entry systems.} This set comprises 500 phrases, totaling 1,164 unique words. The words in this set vary in length, ranging from a minimum of 1 to a maximum of 13, with an average word length of 6.055 (SD: 2.312). 


Each participant was tasked with fingerspelling each of these words twice, resulting in a comprehensive dataset of 2328=2*1164 fingerspelled words per participant, which was collected in two rounds of data collection. During each round, a partcipant completed each word once across 10 sessions. Nine of these sessions contained 116 words each, and one session contained 120 words. We carefully ensured that there was no overlap in words between sessions and that each session maintained a similar distribution of letters and word lengths. The average word length across sessions was 6.053 (SD = 0.181), with median values of 5 or 6.

The experiment was conducted in a semi-controlled environment, as illustrated in Figure \ref{fig:studyprocedure}. During each session, participants followed specific guidelines for fingerspelling. They were provided with real-time video feedback of their signing and were allowed to correct any mistakes as needed. To aid ASL learners, we displayed each English word with accompanying guide images of corresponding ASL letters. Participants used their non-dominant hand to press a space key after completing each word, which allowed us to record the start and end times for each fingerspelled sequence to calculate participants' fingerspelling speed. We instructed participants to return to a neutral hand pose between words and to fingerspell at their own comfortable, natural pace. Session durations varied from 5 to 13 minutes, depending on the participant's fingerspelling speed.

Data collection for each participant lasted around 4 hours, split across two to three days for each participant. To assess the effect of device positioning, participants were asked to remove and reattach the ring between sessions. This approach allowed us to collect data under various ring positions, simulating real-world usage scenarios.

\begin{table*}[t]
\caption{Top-N Word Recognition and Participant Information: CODA: Child of Deaf Adults, H: Hearing. Years indicates number of years learning ASL or using it as a primary language.}
\Description{Top-N Word Recognition and Participant Information: CODA (Child of Deaf Adults), H: Hearing. Years indicates number of years learning ASL or using it as a primary language.}
\begin{tabular}{cccc|cccccr}
\hline
\multicolumn{4}{c|}{}                                                                                                  & \multicolumn{6}{c}{Offline Evaulation}                                                                                                                                                                                    \\
\multicolumn{4}{c|}{}                                                                                                  & \multicolumn{6}{c}{\textbf{Word-level}}                                                                                                                                                                                   \\ \hline
\multicolumn{1}{c|}{}     & \multicolumn{1}{c|}{\textbf{Experience}} & \multicolumn{1}{c|}{\textbf{Gender}} & \textbf{Years} & \multicolumn{1}{c|}{\textbf{Top1}} & \multicolumn{1}{c|}{\textbf{Top2}} & \multicolumn{1}{c|}{\textbf{Top3}} & \multicolumn{1}{c|}{\textbf{Top4}} & \multicolumn{1}{c|}{\textbf{Top5}} & \multicolumn{1}{c}{\textbf{LER}} \\ \hline
\multicolumn{1}{c|}{\textcolor{black}{Avg.}} & \multicolumn{1}{c|}{}               & \multicolumn{1}{c|}{}                & \textit{5.44} & \multicolumn{1}{c|}{\textit{89.8}} & \multicolumn{1}{c|}{\textit{92.8}} & \multicolumn{1}{c|}{\textit{94.3}} & \multicolumn{1}{c|}{\textit{95.1}} & \multicolumn{1}{c|}{\textit{95.6}} & \textit{0.131}                   \\ \hline

\multicolumn{1}{c|}{P01}  & \multicolumn{1}{c|}{Learner, H}     & \multicolumn{1}{c|}{F}               & 3             & \multicolumn{1}{c|}{95.338}        & \multicolumn{1}{c|}{97.220}        & \multicolumn{1}{c|}{98.162}        & \multicolumn{1}{c|}{98.501}        & \multicolumn{1}{c|}{98.846}        & 0.068                            \\ \hline
\multicolumn{1}{c|}{P02}  & \multicolumn{1}{c|}{Learner, H}     & \multicolumn{1}{c|}{M}               & 5             & \multicolumn{1}{c|}{89.161}        & \multicolumn{1}{c|}{92.943}        & \multicolumn{1}{c|}{94.796}        & \multicolumn{1}{c|}{95.612}        & \multicolumn{1}{c|}{96.172}        & 0.126                            \\ \hline
\multicolumn{1}{c|}{P03}  & \multicolumn{1}{c|}{Learner, H}     & \multicolumn{1}{c|}{F}               & 3             & \multicolumn{1}{c|}{94.546}        & \multicolumn{1}{c|}{96.565}        & \multicolumn{1}{c|}{97.772}        & \multicolumn{1}{c|}{98.155}        & \multicolumn{1}{c|}{98.371}        & 0.155                            \\ \hline
\multicolumn{1}{c|}{P04}  & \multicolumn{1}{c|}{Learner, H}     & \multicolumn{1}{c|}{M}               & 1             & \multicolumn{1}{c|}{97.851}        & \multicolumn{1}{c|}{98.968}        & \multicolumn{1}{c|}{99.226}        & \multicolumn{1}{c|}{99.527}        & \multicolumn{1}{c|}{99.614}        & 0.042                            \\ \hline
\multicolumn{1}{c|}{P05}  & \multicolumn{1}{c|}{Learner, H}     & \multicolumn{1}{c|}{F}               & 3             & \multicolumn{1}{c|}{95.020}        & \multicolumn{1}{c|}{96.823}        & \multicolumn{1}{c|}{97.599}        & \multicolumn{1}{c|}{98.026}        & \multicolumn{1}{c|}{98.499}        & 0.089                            \\ \hline
\multicolumn{1}{c|}{P06}  & \multicolumn{1}{c|}{Deaf}           & \multicolumn{1}{c|}{M}               & 7             & \multicolumn{1}{c|}{79.840}        & \multicolumn{1}{c|}{83.822}        & \multicolumn{1}{c|}{86.136}        & \multicolumn{1}{c|}{87.358}        & \multicolumn{1}{c|}{87.945}        & 0.223                            \\ \hline
\multicolumn{1}{c|}{P07}  & \multicolumn{1}{c|}{CODA, H}        & \multicolumn{1}{c|}{M}               & 10            & \multicolumn{1}{c|}{87.884}        & \multicolumn{1}{c|}{91.705}        & \multicolumn{1}{c|}{93.641}        & \multicolumn{1}{c|}{94.628}        & \multicolumn{1}{c|}{95.187}        & 0.162                            \\ \hline
\multicolumn{1}{c|}{P08}  & \multicolumn{1}{c|}{Deaf}           & \multicolumn{1}{c|}{M}               & 10            & \multicolumn{1}{c|}{77.636}        & \multicolumn{1}{c|}{83.739}        & \multicolumn{1}{c|}{86.901}        & \multicolumn{1}{c|}{88.662}        & \multicolumn{1}{c|}{89.837}        & 0.211                            \\ \hline
\multicolumn{1}{c|}{P09}  & \multicolumn{1}{c|}{Deaf}           & \multicolumn{1}{c|}{M}               & 7             & \multicolumn{1}{c|}{90.688}        & \multicolumn{1}{c|}{92.993}        & \multicolumn{1}{c|}{94.397}        & \multicolumn{1}{c|}{95.665}        & \multicolumn{1}{c|}{96.345}        & 0.105                            \\ \hline
\end{tabular}
\end{table*}
\label{fig:performance_tops}


We collected a total of 20,604 fingerspelled words for system evaluation from the nine participants. However, due to technical issues, we lost data from three sessions: the 7th session of P1 (116 words), the 7th session of P7 (116 words), and the 5th session of P8 (116 words).




\subsection{Evaluation Metrics}

For evaluating recognition accuracy, we utilized two primary metrics: Letter Error Rate (LER) and word-level accuracy. LER measures the percentage of incorrect letters in the output compared to the ground truth, with a lower value indicating better accuracy. Word-level accuracy was assessed based on top-1 to top-5 predictions, providing a comprehensive view of the model's effectiveness.


\subsubsection{Letter Error Rate (LER)}

LER is a metric used to evaluate the accuracy of a system in recognizing or generating sequences of letters, such as in speech recognition, handwriting recognition, or fingerspelling recognition. It measures the percentage of incorrect letters in the output compared to a reference or ground truth. A lower LER indicates better accuracy in recognizing or generating a sequence of letters. We calculate LER before word correction. For example, when the target word is "hello" and the predicted word from the model is "helo", there is no substitution, one deletion (the second "l" is missing in the predicted word), and no insertion. In this case, the LER is 1/5 = 0.2.

\begin{equation}
LER = \frac{{\text{Substitutions} + \text{Deletions} + \text{Insertions}}}{{\text{Total number of letters in the reference sequence}}}
\end{equation}



\subsubsection{Top N Word-level Accuracy}
We also report word-level accuracy based on top-1 and top-5 predictions, as this provides a more comprehensive evaluation of the model's effectiveness. After collecting the word data, we calculate performance by determining accuracy, defined as the number of correctly predicted words divided by the total number of words. Additionally, we identify the top-3 potential words by selecting those with the smallest Levenshtein distances. For top-3 accuracy, if the correct word is among the three with the smallest Levenshtein distances, the prediction is considered correct; otherwise, it is incorrect. For example, if the input word is "fax" and the model predicts "aax", the three closest words by Levenshtein distance could be "fox", "tax", and "fax." In this case, the system's prediction would be considered correct.




\subsection{Results}

Our analysis revealed that SpellRing achieved a promising overall accuracy of 89.89\% (SD = 8.59\%) for top-1 predictions and 95.72\% (SD = 5.28\%) for top-5 predictions in recognizing the 1,164 fingerspelled words in our dataset. This performance is comparable to \textcolor{black}{FingerSpeller \cite{martin2023fingerspeller} using two rings} (87\% accuracy), demonstrating the effectiveness of our single-ring approach.

\subsubsection{Top-N Word Recognition}

SpellRing's recognition accuracy improved with increasing N in top-N predictions. For top-1 predictions, the system achieved 89.89\% (SD = 8.59\%) accuracy. This improved to 92.85\% (SD = 7.00\%) for top-2, 94.37\% (SD = 6.13\%) for top-3, 95.20\% (SD = 5.56\%) for top-4, and reached 95.72\% (SD = 5.28\%) for top-5 predictions.
Table 3 illustrates the top-N word accuracy for each participant. The significant improvement from top-1 to top-5 accuracy (a 5.83\% increase) suggests potential benefits for ASL translation applications. By considering multiple top predictions, the system could leverage additional contextual information to produce more coherent and contextually appropriate sentences. This approach could help maintain the flow and meaning of the text by selecting from the best few options at each step. Furthermore, a user interface displaying the top three candidates immediately after prediction would allow signers to choose the correct option, potentially improving overall system accuracy.



\begin{figure}[b]
  \includegraphics[width=0.8\linewidth]{document/figures/output_completion.png}
  \caption{Offline Word-level Performance over Completion Time in User study 1. Dots indicate each session. Note that the model tended to have lower performance for faster signers and better performance for slower signers.}
  \Description{Offline Word-level Performance over Completion Time in User study 1. Dots indicate each sessions. Note that faster signers tended to have lower performance, while slower signers tended to perform better.}
  \label{fig:performance_time}
\end{figure}




\subsubsection{ASL Learners vs. Native/ Fluent Signers}

We observed significant variations in performance across participants, with top-1 accuracy ranging from 77.63\% to 97.85\%. This variability can be attributed to differences in participants' fingerspelling proficiency, affecting articulatory habits and speed. Notably, the model showed better performance for ASL learners (M = 94.38\%, SD = 4.28\%) than for native/ fluent signers (M = 84.06\%, SD = 9.26\%). We attribute this difference primarily to variations in fingerspelling speed.  ASL learners took longer to complete each session (M = 428.88 sec, SD = 74.24) compared to fluent signers (M = 329.624 sec, SD = 42.54). ASL learners tended to fingerspell more slowly, clearly distinguishing each letter, while native/ fluent signers fingerspelled more quickly, resulting in greater handshape variation for certain letters and blurred transitions between letters. This speed difference posed challenges for the model in accurately recognizing individual letters in rapid sequences. We further analyze the impact of fingerspelling speed on performance in the following section.

\subsubsection{Impact of Fingerspelling Speed on Performance}

Fingerspelling speed varied among participants, leading to differences in task completion times. These speed variations, along with factors such as participants' prior experience signing specific words and breaks taken between sessions, appeared to influence the model's performance (Figure \ref{fig:performance_time}). Specifically, the model had lower performance for faster signers and performed better for slower signers.

We found that faster signing speeds often led to greater handshape variation, potentially impacting our model's performance. For instance, when signed quickly, some letters (e.g., ‘C’, ‘O’, ‘E’, ‘I’) were often not fully articulated. The contrast in fingerspelling between the highest and lowest performing participants illustrates this effect. P05, an ASL learner with 3 years' experience, achieved the highest performance with 97.85\% (SD = 2.07\%) accuracy. They articulated each letter very clearly and exhibited minimal handshape variation, resulting in consistent data. In contrast, P08, a fluent signer, had the lowest performance at 77.64\% (SD = 7.8\%). Their fingerspelling speed varied across sessions and consisted of many allophonic handshape variations.

These observations suggest that as fingerspelling speed increases, individual letters are articulated less fully and/or are coarticulated with neighboring letters, posing a challenge for accurate recognition. This challenge resembles those in early-stage speech recognition systems, which experienced performance drops when speakers spoke too rapidly or with strong accents. We discuss this in more detail in Section \ref{impact_speed}.









\section{Phrase-level Recognition}

\subsection{Purpose and Overview}
Building on the insights from our word-level study, our second investigation aimed to evaluate SpellRing's performance for real-time phrase-level recognition. We sought to understand how the system performs in more natural contexts, how users adapt their signing behavior to real-time feedback, and the effectiveness of language models in improving recognition accuracy. It is worth noting that most prior work \cite{paudyal2017dyfav, mummadi2017real} using wearables does not evaluate fingerspelling recognition continuously in real-time. This real-time performance study was a crucial step in assessing SpellRing's potential for practical, everyday use \textcolor{black}{in comprehensive ASL recognition systems.}

\subsection{Participants}
We recruited 11 participants (4 male, 7 female, mean age = 32.0, SD = 3.88) to evaluate our system, consisting of 8 Deaf individuals, 1 ASL interpreter, and 2 hearing ASL learners. The 8 Deaf participants use ASL as their primary language are fluent ASL signers and fingerspellers. The 2 hearing participants had been learning ASL for 1 and 2 years, respectively, contributing to differences in their fingerspelling proficiency. All participants fingerspelled using their right hand as their dominant hand. Detailed information about participants’ proficiency and background is provided in Table \ref{fig:table2}.






\subsection{Iteration on Hardware Prototype Design}
To evaluate our system in a more natural fingerspelling environment, we redesigned the ring prototype to a smaller form factor, enabling real-time evaluation. Our design optimized the device for comfort and ease of use while ensuring it supported continuous and real-time tracking for natural fingerspelling. \textcolor{black}{As shown in Figure \ref{fig:prototype2}, audio data from the FPCB microphone \textcolor{black}{(d)} connected to our custom nRF MCU \textcolor{black}{(b)} and gyro data from the IMU \textcolor{black}{(c)} were synchronized and sent to an off-the-shelf ESP32 S3 microcontroller \footnote{Adafruit QT Py ESP32-S3 WiFi Dev Board with STEMMA QT - 8 MB Flash} \textcolor{black}{(a)}.} This data was then transmitted via the wire to a backend system for processing through our machine learning pipeline, running on a MacBook Pro. The raw predictions were processed through autocorrection and language model pipelines to generate the final output.
\begin{figure}[t]
  \includegraphics[width=\linewidth]{document/figures/new_prototying.png}
  \caption{Prototype for real-time phrase-level evaluation: (a) ESP32 S3 microcontroller, (b) nRF MCU, (c) IMU sensor, (d) FPCB, and (e) a 3D-printed ring-shaped design}
  \Description{Prototype for real-time phrase-level evaluation: (a) ESP32 S3 microcontroller, (b) nRF MCU, (c) IMU sensor, (d) FPCB, and (e) a 3D-printed ring-shaped design}
  \label{fig:prototype2}
\end{figure}


\subsection{Language Model}
With our dataset, we used an N-gram language model to correct fingerspelled words within phrases. Based on the LM model described in \cite{zhu2018typing}, we generated a list of top N words (N = 20) along with their similarity values after autocorrecting a raw predicted word. For each new predicted word in the top N, we applied bigram and trigram probabilities and selected the word set with the highest probability for the final predicted phrase. We then calcuated WER between the ground truth phrases and the final predicted phrases for evaluation. 










\subsection{Dataset and Procedure}

For our phrase-level prediction evaluation, we again utilized the MacKenzie-Soukoreff Phrase Set \cite{mackenzie2003phrase}, as in the first user study. The phrases ranged from 16 to 40 characters in length, consisting of 4 to 8 words each. Our study procedure consisted of two main phases: initial data collection and real-time evaluation.



\subsubsection{Phase 1: Training Data Collection}
We first collected training data from all 11 participants, following a procedure similar to Study 1. Each participant provided two rounds of training data for 1,164 words (2*1164). Our training process involves a two-step
approach: first training with data from all participants except one, then retraining with the specific participant’s data
 for real-time phrase evaluation. Each participant provided the two rounds of training data over two separate days. 
\begin{figure}[t]
  \includegraphics[width=0.8\linewidth]{document/figures/interface_study2.png}
  \caption{User Interface for Real-time Evaluation: The ground truth phrases are displayed in red, and the predicted phrases appear based on each fingerspelled word. Participants receive status updates below, such as `start signing' and `processing.'  }
  \Description{ser Interface for Real-time Evaluation: The ground truth phrases are displayed in red, and the predicted phrases appear based on each fingerspelled word. Participants receive status updates below, such as 'start signing' and 'processing.'  }
  \label{fig:interface_study2}
\end{figure}
\subsubsection{Phase 2: Real-time Evaluation}
\begin{table*}[t]
\caption{Offline evaluation of 1,164 word-level recognition with WPM, Top N and LER, and real-time evaluation of 100 phrase-level recognition with WER and WPM, G: Gender, H: Hearing, DHH: Deaf or Hard of Hearing. Year: Indicates ASL experience (years learning ASL or using it as a primary language)}
\Description{Offline evaluation of 1,164 word-level recognition with WPM, Top N and LER, and real-time evaluation of 100 phrase-level recognition with WER and WPM,G: Gender, H: Hearing, DHH: Deaf or Hard of Hearing. Year: Indicates ASL experience (years learning ASL or using it as a primary language)}
\begin{tabular}{cccc|ccccccc|cc}
\hline
\textbf{}                      & \textbf{}                           & \textbf{}                            & \textbf{}     & \multicolumn{7}{c|}{Offline}                                                                                                                                                                                                                      & \multicolumn{2}{c}{Real-Time}                       \\
\textbf{}                      & \textbf{}                           & \textbf{}                            & \textbf{}     & \multicolumn{7}{c|}{\textbf{Word-level}}                                                                                                                                                                                                          & \multicolumn{2}{c}{\textbf{Phrase-level}}           \\ \hline
\multicolumn{1}{c|}{\textbf{}} & \multicolumn{1}{c|}{\textbf{Types}} & \multicolumn{1}{c|}{\textbf{G}} & \textbf{Year} & \multicolumn{1}{c|}{\textbf{Top 1}} & \multicolumn{1}{c|}{\textbf{Top 2}} & \multicolumn{1}{c|}{\textbf{Top 3}} & \multicolumn{1}{c|}{\textbf{Top 4}} & \multicolumn{1}{c|}{\textbf{Top 5}} & \multicolumn{1}{c|}{\textbf{LER}}   & \textbf{WPM}  & \multicolumn{1}{c|}{\textbf{WER}}   & \textbf{WPM}  \\ \hline
\multicolumn{1}{c|}{\textcolor{black}{Avg.}}       & \multicolumn{1}{c|}{}               & \multicolumn{1}{c|}{}                & \textit{11}   & \multicolumn{1}{c|}{\textit{82.60}} & \multicolumn{1}{c|}{\textit{87.45}} & \multicolumn{1}{c|}{\textit{90.19}} & \multicolumn{1}{c|}{\textit{91.56}} & \multicolumn{1}{c|}{\textit{92.54}} & \multicolumn{1}{c|}{\textit{0.149}} & \textit{39.9} & \multicolumn{1}{c|}{\textit{0.099}} & \textit{20.1} \\ \hline
\multicolumn{1}{c|}{P01}       & \multicolumn{1}{c|}{Deaf}           & \multicolumn{1}{c|}{M}               & 10            & \multicolumn{1}{c|}{84.87}          & \multicolumn{1}{c|}{89.17}          & \multicolumn{1}{c|}{92.43}          & \multicolumn{1}{c|}{93.64}          & \multicolumn{1}{c|}{94.58}          & \multicolumn{1}{c|}{0.127}          & 32.1          & \multicolumn{1}{c|}{0.041}          & 22.3          \\ \hline
\multicolumn{1}{c|}{P02}       & \multicolumn{1}{c|}{DHH}            & \multicolumn{1}{c|}{F}               & 5             & \multicolumn{1}{c|}{87.21}          & \multicolumn{1}{c|}{90.54}          & \multicolumn{1}{c|}{93.03}          & \multicolumn{1}{c|}{94.84}          & \multicolumn{1}{c|}{95.87}          & \multicolumn{1}{c|}{0.122}          & 39.5          & \multicolumn{1}{c|}{0.124}          & 19.8          \\ \hline
\multicolumn{1}{c|}{P03}       & \multicolumn{1}{c|}{Deaf}           & \multicolumn{1}{c|}{F}               & 37            & \multicolumn{1}{c|}{77.49}          & \multicolumn{1}{c|}{83.33}          & \multicolumn{1}{c|}{87.03}          & \multicolumn{1}{c|}{88.66}          & \multicolumn{1}{c|}{90.29}          & \multicolumn{1}{c|}{0.192}          & 47.9          & \multicolumn{1}{c|}{0.112}          & 20.3          \\ \hline
\multicolumn{1}{c|}{P04}       & \multicolumn{1}{c|}{Deaf}           & \multicolumn{1}{c|}{F}               & 10            & \multicolumn{1}{c|}{77.03}          & \multicolumn{1}{c|}{82.79}          & \multicolumn{1}{c|}{85.62}          & \multicolumn{1}{c|}{87.43}          & \multicolumn{1}{c|}{88.38}          & \multicolumn{1}{c|}{0.193}          & 52.5          & \multicolumn{1}{c|}{0.103}          & 19.7          \\ \hline
\multicolumn{1}{c|}{P05}       & \multicolumn{1}{c|}{Deaf}           & \multicolumn{1}{c|}{M}               & 5             & \multicolumn{1}{c|}{94.40}          & \multicolumn{1}{c|}{97.41}          & \multicolumn{1}{c|}{98.13}          & \multicolumn{1}{c|}{98.71}          & \multicolumn{1}{c|}{98.99}          & \multicolumn{1}{c|}{0.062}          & 34.5          & \multicolumn{1}{c|}{0.061}          & 20.2          \\ \hline
\multicolumn{1}{c|}{P06}       & \multicolumn{1}{c|}{Deaf}           & \multicolumn{1}{c|}{F}               & 22            & \multicolumn{1}{c|}{67.18}          & \multicolumn{1}{c|}{76.63}          & \multicolumn{1}{c|}{79.96}          & \multicolumn{1}{c|}{82.00}          & \multicolumn{1}{c|}{83.82}          & \multicolumn{1}{c|}{0.234}          & 56.8          & \multicolumn{1}{c|}{0.134}          & 24.9          \\ \hline
\multicolumn{1}{c|}{P07}       & \multicolumn{1}{c|}{Deaf}           & \multicolumn{1}{c|}{F}               & 5             & \multicolumn{1}{c|}{90.93}          & \multicolumn{1}{c|}{93.76}          & \multicolumn{1}{c|}{95.65}          & \multicolumn{1}{c|}{96.42}          & \multicolumn{1}{c|}{96.85}          & \multicolumn{1}{c|}{0.1}            & 29.4          & \multicolumn{1}{c|}{0.093}          & 21.3          \\ \hline
\multicolumn{1}{c|}{P08}       & \multicolumn{1}{c|}{Deaf}           & \multicolumn{1}{c|}{M}               & 17            & \multicolumn{1}{c|}{65.27}          & \multicolumn{1}{c|}{72.53}          & \multicolumn{1}{c|}{77.21}          & \multicolumn{1}{c|}{80.17}          & \multicolumn{1}{c|}{81.59}          & \multicolumn{1}{c|}{0.26}           & 52.4          & \multicolumn{1}{c|}{0.17}           & 17.9          \\ \hline
\multicolumn{1}{c|}{P09}       & \multicolumn{1}{c|}{Intepreter, H}  & \multicolumn{1}{c|}{F}               & 10            & \multicolumn{1}{c|}{85.94}          & \multicolumn{1}{c|}{89.67}          & \multicolumn{1}{c|}{93.09}          & \multicolumn{1}{c|}{93.37}          & \multicolumn{1}{c|}{94.86}          & \multicolumn{1}{c|}{0.119}          & 30.1          & \multicolumn{1}{c|}{-}              & -             \\ \hline
\multicolumn{1}{c|}{P10}       & \multicolumn{1}{c|}{Learner, H}     & \multicolumn{1}{c|}{M}               & 2             & \multicolumn{1}{c|}{86.41}          & \multicolumn{1}{c|}{91.92}          & \multicolumn{1}{c|}{94.24}          & \multicolumn{1}{c|}{95.53}          & \multicolumn{1}{c|}{95.95}          & \multicolumn{1}{c|}{0.129}          & 40.2          & \multicolumn{1}{c|}{0.052}          & 20.3          \\ \hline
\multicolumn{1}{c|}{P11}       & \multicolumn{1}{c|}{Learner, H}     & \multicolumn{1}{c|}{F}               & 1             & \multicolumn{1}{c|}{91.92}          & \multicolumn{1}{c|}{94.15}          & \multicolumn{1}{c|}{95.70}          & \multicolumn{1}{c|}{96.38}          & \multicolumn{1}{c|}{96.73}          & \multicolumn{1}{c|}{0.098}          & 23.1          & \multicolumn{1}{c|}{0.101}          & 14.1          \\ \hline
\end{tabular}
\end{table*}
\label{fig:table2}
The real-time evaluation was conducted on a third day.  In the real-time evaluation, we randomly selected 200 phrases generated from the 1164 unique words for our study. We used 100 phrases for practice and the remaining 100 phrases for testing. 

We began with practice sessions, where participants were given 100 phrases to familiarize themselves with our interface (Fig. \ref{fig:interface_study2}) and the real-time feedback mechanism. This preparatory step ensured that participants were comfortable with the system before the actual evaluation. For the evaluation, we used the remaining 100 phrases. Participants fingerspelled these phrases over the course of 5 sessions in natural environments such as their homes or quiet rooms. During each evaluation session, participants fingerspelled according to the prompts illustrated in Figure \ref{fig:interface_study2}. 

Participants were first shown the phrase, and instructed to begin fingerspelling each word after pressing the space key. As they fingerspelled, the real-time prediction model provided immediate feedback by displaying the predicted output in green on the interface.  Participants were instructed to proceed to the next word even if they saw a mispredicted word on the screen to ensure that they continued to fingerspell each word as displayed. The language model sometimes corrected a mispredicted word as signers completed more words in each phrase.


After completing each phrase, participants pressed the space key with their non-dominant hand to display the next phrase, allowing them to see and prepare for it. Once ready, they pressed the key to start the phrase and pressed it again upon completion. This action served a dual purpose: it advanced the interface to the next phrase and also marked the start and end times for fingerspelling. This timing information allowed us to calculate fingerspelling speed for each phrase by minimizing perception time, allowing for more precise estimation of participants' fingerspelling speed and any adjustments they made in response to real-time feedback. The duration of each session varied based on the participant's fingerspelling proficiency and typically ranged from 10 to 12 minutes. This variation in session length allowed us to accommodate different signing speeds and ensure that all participants could complete the phrases comfortably.

In total, we collected data on 993 phrases from 10 participants. Due to technical issues, we lost data for seven phrases, and one participant was unable to complete the entire study. Despite these minor setbacks, the collected data provided a robust basis for evaluating our system's performance in real-time, continuous fingerspelling recognition. This two-phase approach allowed us to first train our system on participant-specific data and then evaluate its performance in a realistic, real-time scenario. 







\subsection{Evaluation Metrics}
To evaluate our system, we use Word Error Rate (WER) to report performance. The WER metric ranges from 0 to 1, where 0 indicates that the compared texts are identical, and 1 indicates that they are completely different with no similarity. For example, a WER of 0.10 means there is a 10\% error rate in the compared sentences. WER is based on Levenshtein distance, but it operates at the word level instead of the phoneme (or in this case, letter) level.


\begin{equation}
WER = \frac{{\text{Substitutions} + \text{Deletions} + \text{Insertions}}}{{\text{Total number of words in the reference phrase}}}
\end{equation}








\subsection{Results}
Overall, \theDevice{} recognized 100 phrases with a WER of 0.099\% (0.039\%). While word-level performance achieved an average LER of 0.149\%, phrase-level WER improved with use of a language model. 


\subsubsection{Recognition Performance}
Our results show that fingerspelled words are better recognized within the context of a phrase using a language model (See Table 4). The model showed lower performance for faster signers, such as P06 and P08, with a top-1 accuracy of 67.27\% and 65\%, and WERs of 0.134 and 0.17; this translates to approximately 15\% error on the phrases. Compared to word-level recognition performance, this offers an improvement in recognizing fingerspelled words \textcolor{black}{by applying corrections using a language model at phrase-level recognition.} 



\subsubsection{Signing Speed in Phrase-Level Prediction}

We observed that participants adjusted their signing speed and habits according to the predictions displayed on the user interface, leading to a decrease in words per minute (WPM) for these participants, with WPM averages ranging from 39.87 (data collected in training phase) to 20.09 (data collected in real-time phase) (See Table 3). This decrease accounts for latency, including model processing time, participants' reaction times, and participants' fingerspelling more slowly in response to prediction accuracy.



\subsubsection{Qualitative Analysis}
Participants were asked open-ended survey questions regarding their overall experience with \theDevice{} in terms of performance, form factor, and usability. For performance, 8 out of 10 participants reported that the system performed well with the language model, even when phrases were entirely misclassified. They noted that short fingerspelled words (e.g., "a," "I," "am", "be") were not always recognized accurately but could be corrected by the language model when more context was available. However, participants noted that the system did not always work well at first. In these cases, we observed changes in participant behavior based on predicted results; they tended to fingerspell more slowly and distinctly immediately after they encountered recognition errors. \textcolor{black}{Although our offline evaluation demonstrated that the system works reliably without requiring participants to alter their signing behavior, real-time evaluation revealed that participants adjusted their fingerspelling habits dynamically to accommodate the system. Specifically, they slowed their fingerspelling immediately after observing misrecognized words but returned to their natural habits when the system performed accurately.}



After some practice sessions, some participants adjusted their fingerspelling speed and habits to accommodate the system. P01, P02, P05, and P07 stated that they focused on spelling clearly and distinctly without skipping letters. This directly contrasted their natural fingerspelling behavior, which often involved partially articulated letters and quick, seamless transitions between letters. 


%\section{Results}
Overall, SpellRing recognizes 1,164 fingerspelling words with of 89.89\% (8.59\%) and 95.72\% (5.28\%) in top-1 and top-5 estimations, respectively. 
% LER is 0.089 (0.151) - before. LER is 0.055 (0.179) - after correction. 
Our work using only single ring showed the comparable performance when a users using five rings \cite{martin2023fingerspeller} with top-1 accuracy, i.e., 87.0\%. 



\subsection{Top-N Word Recognition}

SpellRing recognizes 1,164 fingerspelling words with the average accuracies of 89.89\% (8.59\%), 92.85\% (7.00\%), 94.37\% (6.13\%), 95.20\% (5.56\%), and 95.72\% (5.28\%) in top 1, 2, 3, 4, and 5 estimations, respectively. Fig \ref{fig:performance_tops} shows that top N word accuracy for each participant. Seeing that top 5 accuarcy greatly surpass the top 1 accuracy, yielding a 5.83\% increase in word accuarcy, we consider a ASL translation, where additional contextual information before or after fingerspelling contributes to more coherent and contextually appropriate ASL sentences. This approach helps maintain the flow and meaning of the text by selecting from the best few options at each step. Additionally, if the UI displays the top three candidates immediately after prediction, it allows signers to choose the correct one. 

\begin{figure}
  \includegraphics[width=\linewidth]{document/figures/output_result.png}
  \caption{Top-N Word Recognition}
  \Description{}
  \label{fig:performance_tops}
\end{figure}



\subsection{ASL Learners vs Native Signers}
Due to the variation in participants' fingerspelling abilities (e.g. habits and speed), performance varies among participants ranging from 77.63\% to 97.85\% on top 1 accuracy. (e.g. fingerspelling handshape variation and fingerspelling speed). 

The model shows better performance of Top 1 on ASL learners (M = 94.38\%, SD = 4.28\% ) than native signers (M = 84.06\%, SD = 9.26\%). We attribute this variation between ASL learners and native signers to their different fingerspelling speeds. The completion on each session takes longer on ASL learners  (M = 428.88 sec, SD = 74.24) compared to native singers (M = 329.624 sec, SD = 42.54). ASL learners were likely to perform fingerspelling clearly on each letters while native singers do fingerspelling quickly, causing hand shape variation and blur transition between letters. It pose a challenges for model to accurately recognize individual letter. 






\subsection{Performance over Fingerspelling Speed}
Each participant has a different fingerspelling speed, resulting in varied task completion times. Additionally, task completion times depend on several factors, such as the participant's prior experience with signing the words and the breaks taken between sessions. We found that these task completion times might affect our performance, as shown in Fig \ref{fig:performance_time}. We also found that signing speed caused many hand shape variations, which might have affected our model's performance. For example, to sign quickly, some letters (e.g., 'C', 'O', 'E', 'I') are often not fully formed in their hand shapes.

P05 achieved the highest performance with 97.85\% (2.07\%), while P08 had the lowest performance at 77.64\% (7.8\%). The significant difference between these two participants was their signing speed and the clarity of their fingerspelling. P05 is an ASL learner whose signing is very clear for each letter, with no hand shape variation, resulting in consistent data. On the other hand, P08's signing speed varied across sessions, leading to many hand shape variations. We believe that when signers increase their signing speed, they tend to alter their hand shapes or not fully form the hand shapes between letters. 


\begin{figure}
  \includegraphics[width=\linewidth]{document/figures/output_completion.png}
  \caption{Performance over Completion Time. Dots indicate each sessions }
  \Description{}
  \label{fig:performance_time}
\end{figure}


\section{Discussion}


In this paper, we adopted a learner-centered design approach, beginning with a formative study to identify students' challenges with existing tools. Based on these insights, we developed DBox, a tool that scaffolds students in breaking problems into smaller parts and provides personalized, adaptive support. Our user study demonstrated that DBox improved learners' performance on similar algorithmic problems, increased perceived learning gains, and fostered greater cognitive engagement, achievement, and satisfaction. In this section, we discuss design implications and generalizability based on our key findings.


\ms{
\subsection{Chaining Learners' Thoughts with Visualized Structured UI Components}

Decomposition requires students to effectively organize their thoughts. While visual elements are known to promote structured thinking and support mental model construction \cite{mcdougall2001effects, liu2010mental}, our formative and user studies revealed shortcomings in existing tools like LeetCode and ChatGPT, which rely on textual representations without adequately supporting structured mental models. In contrast, DBox uses an interactive step tree to visually organize learners' thoughts. This feature was praised by 22 of 24 participants for enhancing algorithmic thinking, serving as a progress tracker, and providing value even without AI assistance.

DBox's interactive step tree and tree-based scaffolding demonstrate the broader potential of intelligent tutoring systems (ITS) to promote active learning and self-regulated problem-solving in fields requiring problem decomposition. Similar principles could benefit STEM education, such as physics or engineering, by externalizing abstract concepts and facilitating multi-step problem-solving. Additionally, progress-tracking visual components may inspire designs for professional training tools in areas like medical diagnostics or software engineering.

\subsection{Promoting Independent Thinking and Active Decomposition Learning}

\subsubsection{\textbf{Transforming Learners from Passive Readers to Active Thinkers}}

Many coding tools provide direct answers or solutions \cite{kazemitabaar2023novices, phung2023generating}, which, while efficient, often bypass opportunities to develop critical problem-solving skills. In contrast, DBox cultivates students' decomposition abilities through structured scaffolding, fostering critical thinking and self-regulated learning in line with learning by doing \cite{anzai1979theory} and constructivist principles \cite{tobias2009constructivist}.

To strengthen decomposition skills, DBox first encourages students to develop their own decomposition strategies by coding or building a step tree from scratch. While DBox can generate parts of a step tree from a student's existing code, these steps are derived from the learner's own reasoning, with DBox acting solely as a modality converter. Besides, DBox provides feedback on tree node statuses, identifying potential errors or missing steps without directly showing the correct answer, challenging students to critically evaluate and refine their decomposition plans.


DBox's scaffolded hint system further supports decomposition skill development by providing adaptive guidance tailored to the student’s progress without overwhelming them. All hints are based on the learner's current decomposition skeleton, with the most detailed hint—``reveal substep''—triggered only after repeated attempts and struggles. Notably, even the most detailed hints prompt only one substep, requiring students to complete the rest independently. As shown in Sec \ref{hintusage}, only 19\% of hints are this detailed, with students primarily relying on simpler, thought-provoking question hints. This scaffolded support system balances guidance and independent thinking, keeping students engaged during challenges without compromising their ability to independently decompose problems \cite{kinnunen2006students}.

Based on these findings, we recommend fostering active problem-solving by shifting students from passive content consumption to active solution creation. Designers could adopt layered scaffolding, starting with minimal guidance and increasing support as needed, to help students progressively master decomposition skills while maintaining confidence and avoiding frustration. Additionally, adaptive learning techniques, such as real-time feedback and progress tracking, can further tailor the support to individual decomposition barriers, encouraging deeper engagement with decomposition tasks. Moreover, designers could integrate metacognitive strategies, such as encouraging students to articulate or reflect on their decomposition approaches, to further enhance critical thinking and foster habits of independent thinking.




\subsubsection{\textbf{Choice of Scaffolding: Balancing Independent Problem-Solving and Efforts}}

Scaffolding involves providing tailored support to help learners accomplish tasks they cannot yet complete independently \cite{kim2011scaffolding, tobias2009constructivist}. Broadly, scaffolding strategies fall into two categories \cite{van2010scaffolding}: (1) gradually reducing assistance as learners gain proficiency, and (2) encouraging independent problem-solving while offering incremental support to address challenges. DBox adopts the second approach, emphasizing independent thinking and encouraging learners to actively decompose problems \cite{zimmerman2013theories}. While our scaffolding strategies successfully enhanced critical thinking, satisfaction, and perceived usefulness, they also led to increased cognitive effort (Sec. \ref{Effects_on_UX}). This tradeoff underscores the importance of carefully balancing cognitive effort with the promotion of independent thinking.

Future designs could incorporate adaptive scaffolding that adjusts support dynamically based on learner proficiency, reducing unnecessary effort in areas where students have demonstrated competence. Additionally, while incremental scaffolding was effective for algorithmic problem-solving, tailoring strategies to different educational contexts could enhance their applicability in diverse domains. Such adaptive, context-specific approaches could further optimize the balance between support and independence in learning environments.


\subsection{Supporting Personalized Algorithmic Programming Learning}

\subsubsection{\textbf{Prioritizing Learners' Own Solutions Over Optimality}}

Algorithmic problems often have multiple solutions with varying time and space complexities. DBox prioritizes independent exploration by supporting learners' strategies rather than steering them toward a single ``optimal'' solution. Using LLM-driven prompts, it evaluates and guides each step based on the learner's reasoning, preserving their step decomposition and respecting their input—even when errors occur. While some solutions may not be the most efficient, this approach fosters autonomy by aligning feedback with learners’ thought processes instead of enforcing rigid standards.

Our user study showed that this approach improves learning outcomes and is well-received by students. We recommend designing systems that respect personalized problem-solving strategies by aligning feedback with learners' reasoning while allowing for diverse approaches. Designers should balance flexibility and rigor, using prompts and interfaces that support varied strategies while gently guiding learners toward effective solutions.


\subsubsection{\textbf{Catering to Individual Learning Styles and Contextual Needs}}

DBox accommodates diverse problem-solving approaches with two input modes: coding and natural language descriptions. Each mode offers distinct advantages tailored to different learners, stages, and situations. Learners can switch seamlessly between modes, with progress automatically synced across the interface. Features such as verifying code-step alignment ensure strong integration between modes.

Our findings reveal that this flexibility enhances user experience. Participant interaction logs and interviews revealed three usage patterns, highlighting that each mode fits different needs: code mode works well for students with a clear and detailed problem-solving plan already, while the step tree with natural language descriptions helps less experienced students with only a basic idea who are not ready to write code directly, boosting their confidence.


We argue there is no universal “best” mode for programming education—each has unique benefits depending on the learner habits, expertise, and context. Future tools should provide flexibility, like DBox, or use adaptive algorithms to recommend modes based on user needs and context. This flexibility highlights the importance of designing educational tools that accommodate varying levels of expertise and problem-solving styles, which can be generalized to other domains requiring personalized learning \cite{bernacki2021systematic}.

\subsection{Appropriate Usage of LLMs for Supporting Algorithmic Programming Learning}

\subsubsection{\textbf{Caution About LLM Errors}}

Although LLMs have shown strong performance in coding tasks \cite{finnie2023my, leinonen2023using}, they remain prone to errors. Our technical evaluation and user study revealed that even with comprehensive context—such as problem statements, user code, and natural language steps—LLM sometimes misinterprets user descriptions. These errors likely arise from discrepancies between the natural language used by students and the formal, precise language the LLM was trained on, which is primarily sourced from web-based code and comments \cite{liu2023wants}.

Such misinterpretations can hinder learning by causing confusion or frustration. While future improvements to training data and GPT versions may mitigate these issues, design strategies can help address them. \textbf{First}, LLMs should avoid giving direct solutions and instead focus on fostering active problem-solving through explanations and hints. \textbf{Second}, feedback could be paired with interactive features, like a ``Run Code'' option, allowing students to validate their reasoning. \textbf{Third}, simple tutorials could teach users how to phrase their descriptions more clearly, improving LLM's understanding. Additionally, future tools could integrate a ``Language Enhancement'' feature to suggest improvements or assess the clarity of descriptions, aiding LLM in accurately capturing user intent. Most importantly, we recommend designers prioritize technical feasibility, such as conducting rigorous evaluations like ours, before fully integrating LLMs into programming learning tools.
}



\subsubsection{\textbf{Learner-LLM Co-Decomposition of Solutions: Learner as Leader, LLM as Aid}}

A central feature of DBox is the construction of a step tree, where students break solutions into steps and sub-steps. The LLM supports this by mapping code to step descriptions, evaluating them, and offering hints. However, students maintain full control, deciding how to decompose problems and define each step, fostering independent thinking. The LLM acts solely as an aid, using a scaffolding approach to support the development of learners' Zone of Proximal Development (ZPD) \cite{chaiklin2003zone}. Unlike tools like ChatGPT or Copilot that dominate problem-solving, DBox fosters deeper cognitive engagement. Students reported greater accomplishment and found this approach more effective for learning.

This contrasts with existing human-AI collaboration paradigms in non-educational scenarios where AI usually suggest options, leaving final decisions to users \cite{dang2023choice, gao2024collabcoder, gebreegziabher2023patat, ma2019smarteye, ma2022glancee}, such as in human-AI decision-making \cite{ma2023should, ma2024towards, ma2024you}. Some educational tools, like Jin et al. \cite{jin2024teach}, use LLMs to generate solutions for students to evaluate, which aids in syntax learning but such ``LLM-generate then learner-evaluate'' approach is less effective for algorithmic problem-solving, where constructing solutions is key. Just evaluating LLM-generated contents can place a cognitive anchor on learners \cite{furnham2011literature}, limiting independent thinking and creativity. Thus, task allocation between humans and AI should align with the educational context (e.g., whether it is basic knowledge/concept learning or higher-level creative thinking). Future LLM-based educational tools should carefully define the division of roles between LLMs and learners, tailoring it to specific learning contexts and goals.




% \subsubsection{Human-LLM Co-Decomposition of Solution: AI Should Judge Instead of Recommending}

% A core interaction in DBox is the construction of a step tree, where the entire solution is broken down into a series of steps and sub-steps. We refer to this as the human-LLM co-decomposition process. In this process, the LLM behind DBox plays three roles: First, it maps the student's written code into step descriptions. Second, it evaluates the status of each step and sub-step (whether they are correct, incorrect, missing, or need further decomposition). Third, it provides hints for incorrect or missing steps or sub-steps. However, the actual construction of the step tree—such as dividing the solution into steps and sub-steps and determining the content of each node—remains primarily the student's responsibility.

% This division of labor maximizes student engagement in independent thinking and problem-solving. The LLM does not provide any suggestions for decomposition nor directly recommend content for specific steps, aligning with the scaffolding educational approach, where guidance is provided appropriately, but the main task of forming the solution is left to the students.

% In contrast, when students directly seek help from an LLM, such as asking questions in ChatGPT or using Copilot for code completion, the LLM takes too much initiative by directly offering ideas or code. In our co-decomposition design, however, students demonstrated higher cognitive engagement and more active critical thinking. Furthermore, students reported that constructing solutions in this way gave them a greater sense of achievement and made them feel the process was more beneficial for learning, leading to higher satisfaction with the experience.

% Related work has proposed similar approaches. For instance, XXX, in the context of problem-solving, uses the "learning by teaching" concept, where students take on the tasks of judging and teaching, while the LLM generates most of the solutions. Compared to our approach, their division of labor between the student and the LLM is reversed. This method works well in introductory programming, where the focus is on mastering syntax. Having students guide the LLM to generate code or evaluate potentially incorrect code produced by the LLM is an effective way to quiz them. However, in our work, which focuses on algorithmic programming, the key step is constructing a solution from scratch. If the LLM builds the solution, leaving students only to judge it, it hampers their independent thinking.

% Thus, when designing LLM-based educational tools in the future, it is crucial to consider the specific context to effectively allocate tasks between the student and the LLM, ensuring that students derive the maximum benefit from the co-decomposition process.


% \subsection{Future Design Opportunities}

% \emph{Providing Appropriate Generative Assistance:} While DBox promotes independent problem-solving, some users showed interest in features like auto-completion for trivial coding tasks. Future versions could balance promoting independence with targeted assistance by enabling adjustable difficulty levels and offering contextual suggestions when appropriate.

% \emph{Covering All Stages of Algorithmic Programming:} DBox currently lacks a focus on foundational algorithm instruction and problem comprehension. Future iterations could include features like generating distractor solutions, input-output tests, and step-by-step rephrasing to help students grasp key concepts and understand the coding problem.

% \emph{Combining Step Trees with Dialogue:} Users can currently describe their thought processes but cannot ask questions. Adding a dialogue system to the step tree would allow students to share challenges and ask follow-up questions. GPT could then provide guided feedback without giving direct answers, supporting independent problem-solving.





% \emph{Other Important Features.} DBox could offer more control by allowing users to select specific parts of their code for targeted evaluation and guidance. A ``review'' feature could also help students reflect on key stumbling points, understand where their thought process went wrong, and how they eventually solved the problem.


% \subsection{Future Design Opportunities}

% \emph{Providing Appropriate Generative Assistance.} Our tool primarily focuses on encouraging users to create the step tree and write the code independently, with the system mainly serving as a judge. However, users expressed a desire for some intelligent completion features, particularly for repetitive or simple code, allowing them to focus their efforts on learning the key parts. Future improvements should strike a balance between fostering independent thinking and providing appropriate assistance. One approach could be designing basic rules where the tool offers intelligent suggestions and completions for parts unrelated to the core logic, while maintaining the current level of independence for key learning areas. Additionally, the system could offer different modes, allowing users to choose the level of assistance, from basic judgment-only feedback to a combination of judgment, guidance, necessary completions, and even on-demand suggestions.

% \emph{Covering All Stages of Algorithmic Programming.} Currently, our system does not cover the basic teaching of algorithms or the problem comprehension stage. In the future, to address the diversity and uncertainty in solutions and help students grasp multiple approaches, we could expand assistance during the idea formation phase. For example, GPT could generate multiple potential solutions with distractors, prompting students to identify the one that meets the problem's complexity requirements. We could also introduce specialized algorithm training, where students select a specific algorithm, and the system’s guidance focuses solely on that algorithm. To assist with problem comprehension, we could incorporate input-output tests to check students' understanding of the problem and step-by-step rephrasing to help them grasp more complex problems.

% \emph{Combining Interactive Step Trees with Dialogue Boxes.} Sometimes users want to describe their difficulties, and currently, we ask them to outline their thought processes. Additionally, users may want to ask follow-up questions. In the future, we could combine the structured step tree with a small dialogue box. The primary goal would still be to construct the step tree, but users could engage in a conversation with GPT in the context of the current step tree or a specific step. Importantly, GPT should guide the user without revealing direct answers.

% \emph{Other Important Features.} First, DBox could offer learners more control, such as allowing users to select specific parts of the code for targeted evaluation and guidance. We could also introduce a summary feature for key stumbling points, helping students reflect on the challenges they faced, where their thought process went wrong, and how they eventually overcame the problem.




\subsection{Limitations and Future Work}

This study has several limitations. \emph{First}, we tested DBox's effectiveness on only two problem types; future work should examine a broader range of algorithms. \emph{Second}, participants engaged in just one learning session per condition due to time constraints, whereas mastering algorithmic problems typically requires extended practice. Longitudinal studies should explore how DBox supports skill development over time, including changes in mental models and skill retention. \emph{Third}, we assessed learning gains based on correctness in a test session using similar learning and test problems. Future research should evaluate knowledge transfer to less similar problems. Due to time constraints, we conducted a single post-test rather than a pre-post comparison. While pre-test expertise filtering and randomization minimized prior familiarity effects, a more rigorous pre-post design would yield more accurate learning gain measurements. Looking ahead, we plan to release DBox as a Chrome plugin for integration with existing coding platforms, enabling large-scale field studies. This will allow for the collection of long-term usage data and periodic surveys to identify usage patterns and learning experiences over time.



% This study has several limitations. First, in our within-subject design, we selected two types of algorithm problems—Greedy and Binary Search—and randomly assigned them to two conditions (DBox and baseline). However, selection bias may still exist, as some participants might naturally excel at one type of algorithm. Although we addressed this by filtering participants' proficiency through a pre-test and using a Latin Square design, further validation across a broader range of algorithms is needed in future work.

% Second, students experienced only one learning session per condition before the test session. While this allowed for a fair comparison, mastering algorithmic problems typically requires extended practice. Future work should explore how DBox supports students' long-term improvement in algorithmic skills. Longitudinal studies could provide insights into changes in learners' mental models, allowing students more time to deepen their understanding and refine their decomposition methods. Additionally, retention tests could assess whether students can still apply learned problem-solving methods after a time gap.

% We measured learning gains through correctness scores in the test session, with relatively similar learning and test problems. Future work should explore students' ability to transfer their knowledge to problems with lower similarity. Due to time constraints, we opted for a single post-test rather than a pre-post comparison. While we minimized prior familiarity effects by filtering participants and randomizing problem assignments, future studies could adopt a more rigorous pre-post test design for better measurement of learning gains.

% Looking ahead, we plan to release DBox as a Chrome plugin for integration with existing online coding platforms and large-scale real-world testing. In such settings, where students may be more motivated (e.g., preparing for algorithm interviews), we can gather long-term usage data while ensuring privacy. We also plan to conduct periodic surveys to track changes in students' usage patterns and learning experiences over time.



% \subsection{Limitations and Future Work}

% This study has several limitations. First, in our within-subjects study, we selected two types of algorithm problems, Greedy and Binary Search, and randomly assigned them to two conditions, DBox and the baseline. However, there may still be selection bias, where some participants were naturally better at one type of algorithm. While we mitigated this issue to a large extent by filtering participants' proficiency through a pre-test and employing a Latin Square design to randomize the problem-condition assignment, there is still room for improvement. Future work should validate DBox's effectiveness across a broader range of problem types.

% Second, in our experiment, students only experienced one learning session in each condition before moving on to the test session. Although this comparison was fair (as both conditions had only one learning session), mastering an algorithmic problem often requires extended practice. Future work should explore how DBox can help students gradually improve their algorithmic programming skills over time. Longitudinal studies may reveal significant changes in learners' mental models, providing more time for them to understand a specific algorithm and enhance their decomposition methods. Additionally, future studies could include retention tests to measure whether students can still effectively apply previously learned problem-solving methods after a period of time.

% Furthermore, when objectively measuring students' learning gains, we calculated their correctness score in the test session. On the one hand, the learning session and test session problems had a relatively high degree of similarity. Future work should investigate whether students can transfer what they have learned to solve problems of the same algorithm type with lower similarity. On the other hand, due to time constraints, we did not include a pre-post test comparison, opting for a single post-test instead. This result might be influenced by students' pre-existing familiarity with the problems. Although we mitigated this issue by filtering for familiarity (ensuring participants were not too familiar with the problems) and randomizing the problem assignments, future work could include a more rigorous pre-post test design to better calculate students' learning gains.

% Moreover, DBox is currently only applied in algorithmic programming, specifically solving algorithm problems. However, this decomposition-based computational thinking approach could be extended to other learning scenarios, such as project-based learning. Future work could explore how to adapt DBox to broader educational contexts outside of algorithmic programming.

% Looking forward, we aim to deploy DBox in real-world algorithm courses. Since algorithms are a core required subject in undergraduate computer science curricula, we hope to investigate how students who have just learned algorithm concepts use DBox to develop their problem-solving skills. Additionally, we plan to convert DBox into a Chrome plugin and release it in the Chrome Web Store for real-world testing. This would allow DBox to seamlessly integrate with existing online coding platforms, enabling large-scale experiments. In such settings, students' motivation may be stronger (e.g., a graduate preparing for an algorithm interview), leading to more realistic usage patterns. Students could use DBox to tackle a wide variety of algorithm problems. We hope to collect long-term (e.g., six-month) usage data from real-world users while ensuring privacy, and use periodic surveys to capture changes in students' usage patterns and learning experiences over time.





\section{Conclusion}
% In this paper, we introduced Decomposition Box (DBox), a novel tool designed to scaffold learners in decomposing problems during algorithmic programming learning. Based on insights from a formative study, we identified key design goals to address the limitations of existing tools in algorithmic programming education. DBox supports two critical stages of the programming process: idea formation and idea implementation. By offering two modes (code mode and language mode), it encourages users to independently develop their solution strategies. The interactive, visual step tree helps students break down problems and build a structured mental model. DBox provides fine-grained, step-level feedback, enabling students to quickly identify issues, while its multi-level guidance offers targeted support without undermining independent thinking.

% Our user study demonstrated that DBox led to significantly higher learning gains, cognitive engagement, and critical thinking. Students reported a stronger sense of achievement and found the assistance both appropriate and effective for their learning. We identified three main usage patterns, underscoring the importance of respecting students' problem-solving habits and offering them autonomy. The learner-LLM co-decomposition model we designed promotes independent thinking while allowing the LLM to contribute meaningfully, even with occasional imperfections. 

% We hope the formative study, design goals, features, technical evaluation, and key findings from this work will inspire future research on developing educational tools for broader programming learning.
In this paper, we introduced DBox, an interactive tool designed to help learners decompose algorithmic programming problems by supporting both solution formation and implementation. Featuring an intuitive tree-like box widget, DBox accepts input in both code and natural language, fostering independent problem-solving while its step tree structure helps learners develop structured mental models. It provides step-level feedback and layered guidance without compromising learner autonomy.
Our user study showed that DBox significantly improved learning outcomes, cognitive engagement, and critical thinking, with students reporting a greater sense of achievement and finding the support highly effective. Additionally, we identified three key usage patterns, highlighting the importance of accommodating individual problem-solving styles. Moreover, our findings suggest that the learner-LLM co-decomposition approach fosters independent thinking while providing meaningful guidance, even with occasional imperfections.
We hope the insights from our system design will inspire future research on integrating LLMs into educational tools for programming learning.



In this study, we performed the first large-scale analysis of data leakage across 83 software engineering (SE) benchmarks, covering three popular programming languages—Python, Java, and C/C++. By combining an efficient near-duplicate detection algorithm with extensive manual labeling, we ensured the accurate identification of leaked data.



Our findings show that while data leakage is generally low, with average leakage ratios of 4.8\%, 2.8\%, and 0.7\% for Python, Java, and C/C++ benchmarks respectively, some benchmarks exhibit higher leakage that requires attention. We identified four main causes of leakage: direct inclusion of benchmark data in pre-training datasets, overlap between source repositories, reliance on platforms like LeetCode, and shared data sources such as GitHub issues.
We also found that automatic detection methods, like Perplexity-based metrics, struggle to distinguish between leaked and non-leaked samples. Additionally, our experiments reveal that data leakage inflates evaluation metrics, with models performing significantly better on leaked samples. For instance, StarCoder-7b achieved a Pass@1 score 4.9 times higher on leaked samples, underlining the need to address leakage to ensure fair evaluations.
This study offers insights into data leakage status in SE benchmarks and its impact on LLM evaluation.


In the future, we aim to expand the analysis to additional benchmarks and explore new methods to prevent or further reduce data leakage.





\vspace{0.2cm}
\noindent \textbf{Acknowledgement.}  This research / project is supported by the National Research Foundation, under its Investigatorship Grant (NRF-NRFI08-2022-0002). Any opinions, findings and conclusions or recommendations expressed in this material are those of the author(s) and do not reflect the views of National Research Foundation, Singapore.





%%
%% The next two lines define the bibliography style to be used, and
%% the bibliography file.
\bibliographystyle{ACM-Reference-Format}
\bibliography{main}
\end{document}
\endinput
%%
%% End of file `sample-authordraft.tex'.
