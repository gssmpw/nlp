\section{Results}
Overall, SpellRing recognizes 1,164 fingerspelling words with of 89.89\% (8.59\%) and 95.72\% (5.28\%) in top-1 and top-5 estimations, respectively. 
% LER is 0.089 (0.151) - before. LER is 0.055 (0.179) - after correction. 
Our work using only single ring showed the comparable performance when a users using five rings \cite{martin2023fingerspeller} with top-1 accuracy, i.e., 87.0\%. 



\subsection{Top-N Word Recognition}

SpellRing recognizes 1,164 fingerspelling words with the average accuracies of 89.89\% (8.59\%), 92.85\% (7.00\%), 94.37\% (6.13\%), 95.20\% (5.56\%), and 95.72\% (5.28\%) in top 1, 2, 3, 4, and 5 estimations, respectively. Fig \ref{fig:performance_tops} shows that top N word accuracy for each participant. Seeing that top 5 accuarcy greatly surpass the top 1 accuracy, yielding a 5.83\% increase in word accuarcy, we consider a ASL translation, where additional contextual information before or after fingerspelling contributes to more coherent and contextually appropriate ASL sentences. This approach helps maintain the flow and meaning of the text by selecting from the best few options at each step. Additionally, if the UI displays the top three candidates immediately after prediction, it allows signers to choose the correct one. 

\begin{figure}
  \includegraphics[width=\linewidth]{document/figures/output_result.png}
  \caption{Top-N Word Recognition}
  \Description{}
  \label{fig:performance_tops}
\end{figure}



\subsection{ASL Learners vs Native Signers}
Due to the variation in participants' fingerspelling abilities (e.g. habits and speed), performance varies among participants ranging from 77.63\% to 97.85\% on top 1 accuracy. (e.g. fingerspelling handshape variation and fingerspelling speed). 

The model shows better performance of Top 1 on ASL learners (M = 94.38\%, SD = 4.28\% ) than native signers (M = 84.06\%, SD = 9.26\%). We attribute this variation between ASL learners and native signers to their different fingerspelling speeds. The completion on each session takes longer on ASL learners  (M = 428.88 sec, SD = 74.24) compared to native singers (M = 329.624 sec, SD = 42.54). ASL learners were likely to perform fingerspelling clearly on each letters while native singers do fingerspelling quickly, causing hand shape variation and blur transition between letters. It pose a challenges for model to accurately recognize individual letter. 






\subsection{Performance over Fingerspelling Speed}
Each participant has a different fingerspelling speed, resulting in varied task completion times. Additionally, task completion times depend on several factors, such as the participant's prior experience with signing the words and the breaks taken between sessions. We found that these task completion times might affect our performance, as shown in Fig \ref{fig:performance_time}. We also found that signing speed caused many hand shape variations, which might have affected our model's performance. For example, to sign quickly, some letters (e.g., 'C', 'O', 'E', 'I') are often not fully formed in their hand shapes.

P05 achieved the highest performance with 97.85\% (2.07\%), while P08 had the lowest performance at 77.64\% (7.8\%). The significant difference between these two participants was their signing speed and the clarity of their fingerspelling. P05 is an ASL learner whose signing is very clear for each letter, with no hand shape variation, resulting in consistent data. On the other hand, P08's signing speed varied across sessions, leading to many hand shape variations. We believe that when signers increase their signing speed, they tend to alter their hand shapes or not fully form the hand shapes between letters. 


\begin{figure}
  \includegraphics[width=\linewidth]{document/figures/output_completion.png}
  \caption{Performance over Completion Time. Dots indicate each sessions }
  \Description{}
  \label{fig:performance_time}
\end{figure}
