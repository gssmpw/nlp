\section{Related Work}
Here, we will discuss prior work that recognizes ASL fingerspelling using cameras, and different types of wearable form factors. 


\subsection{Vision-based Approach}
ASL fingerspelling involves complex handshapes that necessitate tracking all fingers for recognition. Thus, early-stage research focused on recognizing hand appearance using vision-based approaches (e.g., cameras), yielding high performance on 26 isolated English/ ASL letters \cite{feris2005recognition, kim2017lexicon,fowley2021sign, goh2006dynamic}. Most previous studies have focused on recognizing isolated manual letters with high performance, achieving over 95\% accuracy \cite{bohavcek2022sign, du2022full, hosain2021hand}. However, given that recognition of isolated manual letters differs significantly from natural, continuous fingerspelling, recent work has shifted toward continuous fingerspelling recognition. Notably, studies using in-the-wild video datasets, such as the FSWild+ dataset \cite{shi2019fingerspelling}, have shown promising results, with performance reaching up to 71.3\%. 
Along with the development of sequential recognition models (HMM \cite{martin2023fingerspeller,goh2006dynamic}, LSTM \cite{shi2019fingerspelling,pannattee2021novel}, CTC clasification \cite{fayyazsanavi2024fingerspelling, gu2022american, graves2006connectionist}), performance on continuous fingerspelling recognition has improved. However, vision-based methods are limited by their costly setup (e.g., by requiring a camera positioned in front of the signer); wearable systems (such as wearing a glove, multiple rings, or a wristband) have proposed more portable alternatives.

\subsection{Wearable-based Approach}
To address the limitations of the vision-based methods mentioned above, wearable-based approaches have attempted to recognize fingerspelling using various form factors, such as data gloves and wristbands.
The data glove, which includes flex sensors \cite{rizwan2019american, saggio2020sign}, stretchable strain sensors \cite{li2018skingest}, magnetic sensors \cite{rinalduzzi2021gesture}, and/ or IMU sensors \cite{lee2020sensor, martin2023fingerspeller, saggio2020sign}, provides highly reliable information on hand joints, detecting both static and dynamic fingerspelling with over 96\% accuracy \cite{lee2020sensor, saggio2020sign, rizwan2019american}. Despite the reliable and high performance, comfort remains a concern among DHH users \cite{kudrinko2022assessing}, and these devices can impair the dexterity of finger movements, thereby affecting ease of use for DHH individuals. 

A wristband using electromyography (EMG) can track finger muscle movements \cite{singh2023reliable}, providing a more usable form factor with a single device worn on the wrist. \textcolor{black}{It has recently demonstrated promising performance in recognizing isolated manual letters, achieving 99.1\% accuracy \cite{singh2023reliable} and 95.36\% accuracy \cite{paudyal2017dyfav}. Additionally, it has shown the ability to recognize various ASL signs \cite{zhang2019myosign, abreu2016evaluating} with accuracies over 92.4\%.} However, continuous fingerspelling recognition is challenging due to the subtle, continuous transitions between letters and the fine-grained movements involved, which may be difficult for EMG sensors to capture accurately and consistently across different wearing sessions, a widely known issue for EMG sesnsors.  Furthermore, the armband with electrodes is dispreferred among DHH users, especially for its poor ease of use and appearance \cite{zhou2023signquery}, and can struggle to deliver reliable performance due to the inherent limitations of EMG sensors \cite{kudrinko2020wearable, ben2023sign} such as calibration, false muscle detection, and surface preparation (including hair removal). \textcolor{black}{Recent work, EchoWrist \cite{lee2024echowrist}, demonstrated continuous finger tracking using a wristband with embedded acoustic sensing, capable of tracking various hand gestures, including isolated ASL digits from 0 to 9. However, while it enables hands-free tracking and classification, its sensor placement is susceptible to obstruction by clothing, such as long sleeves.} Therefore, more effort is required to develop a practical fingerspelling recognition device. A ring-shaped device holds potential for achieving both accuracy and comfort. 

\subsection{Ring-based ASL Approach}
 Similar to the glove, a ring-shaped device places IMU sensors on the fingers, achieving similar performance for continuous fingerspelling. Fingerspeller \cite{martin2023fingerspeller} achieved 91\% accuracy on 1164 unique words and demonstrated the ability to recognize fingerspelled words using four rings, with two rings performing at 81\% accuracy on three signers. However, their work \cite{martin2023fingerspeller} showed that more IMUs are needed on each finger to track the complex movements of fingerspelling. Furthermore, how it would perform on a larger group of native/ fluent signers is unknown. Additionally, \cite{zhou2023signquery} showed that this ring-based form factor scores higher on user experience in ease of use, comfort, and appearance, as compared to GyberGlove\footnote{https://www.cyberglovesystems.com/} and the Myo EMG device\footnote{https://wearables.com/products/myo}. However, current technology still requires multiple rings for recognizing continuous fingerspelling, leaving room for improvement on a design factor. 


\textcolor{black}{Recently, Ring-a-Pose \cite{yu2024ring} demonstrated the ability to track 3D handshapes using a single ring with active acoustic sensing, achieving reliable performance on a limited set of hand pose gestures, including the isolated 10 ASL digits (0–9), with a joint error of 14.1 mm.} \textcolor{black}{ Based on research by \cite{shi2018american, shi2019fingerspelling}, which utilizes finger tracking with skeleton information for continuous fingerspelling recognition, Ring-a-Pose appears promising for this task. However, its ability to recognize complex and fast continuous fingerspelling remains uncertain. As hand-tracking error increases, research by \cite{taylor2018real} shows a general downward trend in classification accuracy for 26 isolated letters, highlighting the importance of reliable hand tracking for fingerspelling recognition. However, given that Ring-a-Pose has a 14.1 mm joint error compared to the reliable hand-tracking method used in \cite{shi2018american, shi2019fingerspelling}, its potential still needs further exploration.} Furthermore, Ring-a-Pose \cite{yu2024ring} cannot track palm orientation, which is crucial for distinguishing between manual letters that share an identical handshape and differ solely in palm orientation (e.g., `K' and `P', `G' and `Q', `H' and `U') or movement (e.g., `I' and `J'). Its resolution is also insufficient for the rapid pace of fingerspelling, as fast signers can fingerspell up to eight letters per second \cite{hassan2023tap, hanson1982use, quinto2010rates, keane2016fingerspelling}, while the system tracks only one gesture over two seconds. Fingerspelling comprises 26 letters with similarities across many letters (e.g., `A', `S', `M', `N, and `T'; `C' and `O'; `K' and `P'); fingerspelling recognition is thereby a much more complex task than ASL digit recognition. Fast fingerspelling and signers' individual habits, such as skipping letters or coarticulating neighboring letters \cite{keane2015segmentation}, also lead to variations in handshape for the same letter. Unlike isolated gestures in Ring-a-Pose \cite{yu2024ring}, continuous fingerspelling requires sequential processing to differentiate handshape variations and letter transitions. These inherent challenges necessitate a new approach to continuous fingerspelling recognition.
 
 In this paper, we propose \theDevice{}, which integrates active acoustic sensing \cite{yu2024ring} and an IMU on a thumb-mounted ring to track handshape, hand movement, and palm orientation; we investigate how it can be used to recognize continuously fingerspelled words produced by signers ranging from novice to native/ fluent in ASL proficiency. Table 1 summarizes previous work. 
