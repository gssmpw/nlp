
\section{Word-level Recognition}

\subsection{Purpose and Overview}
The primary objective of this study was to evaluate SpellRing's ability to recognize individual fingerspelled words accurately \textcolor{black}{and compare its performance to that of of multiple-ring setups, such as FingerSpeller \cite{martin2023fingerspeller}.} We aimed to assess the system's performance across a range of users with varying ASL proficiencies, ranging from novice to native. This study also sought to investigate the impact of signing speed and individual habits on recognition accuracy, providing important insights for system optimization and user adaptation strategies.

\subsection{Participants}
We recruited 9 participants (6 male, 3 female, mean age = 23.0, SD = 5.45) to evaluate our system, including 3 Deaf individuals, 1 CODA (Child of Deaf Adults), and 5 hearing ASL learners. The 3 Deaf participants and CODA were fluent signers, using ASL as a primary language. The 5 hearing ASL learners had between 1 and 5 years of ASL learning experience (M = 3.0  SD = 1.4). All participants fingerspelled with their right hand as their dominant hand. Detailed participant information, including proficiency and background, is provided in Table~\ref{fig:table2}.
% \begin{table}[]
% \caption{Participant Information: C: CODA (Child of Deaf), H: Hearing Individual, HHD: Deaf and Hard-of-Hearing individual. Proficiency of fingerspelling was self-reported by participants on a scale of 1 to 7, with 7 being the highest}
% \begin{tabular}{l|l|r|l}
% \hline
% Participants          & ASL Experience & \multicolumn{1}{l|}{Proficiency} & Gender \\ \hline
% ASL learner, H  & 3 years        & 6                                & F \\ \hline
% ASL learner, H & 5 years +      & 7                                & M   \\ \hline
% ASL learner, H  & 3 years        & 6                                & F \\ \hline
% ASL learner, H  & 1 year         & 5                                & M   \\ \hline
% ASL learner, H & 3 years        & 6                                & F \\ \hline
% native, H, CODA & 5 years +      & 7                                & M   \\ \hline
% native, HHD           & 5 years +      & 7                                & M   \\ \hline
% native, HHD           & 5 years +      & 7                                & M \\ \hline
% native, HHD           & 5 years +      & 7                                & M   \\ \hline
% \end{tabular}
% \label{table:participants}
% \end{table}


% Please add the following required packages to your document preamble:
% \usepackage[table,xcdraw]{xcolor}
% Beamer presentation requires \usepackage{colortbl} instead of \usepackage[table,xcdraw]{xcolor}
% \begin{table}[]
% \caption{Participant Information: C: CODA (Child of Deaf), H: Hearing Individual, HHD: Deaf and Hard-of-Hearing individual. Experience indicates ASL}
% \begin{tabular}{ccccc}
% \hline
% \#  & Participant & Experience & Proficiency & Gender \\ \hline
% P01 & ASL learner, H                       & 3 Y                                    & 6           & F      \\ \hline
% P02 & ASL learner, H                       & 5 Y +                                  & 7           & M      \\ \hline
% P03 & ASL learner, H                       & 3 Y                                    & 6           & F      \\ \hline
% P04 & ASL learner, H                       & 1 Y                                    & 5           & M      \\ \hline
% P05 & ASL learner, H                       & 3 Y                                    & 6           & F      \\ \hline
% P06 & native, H, CODA                      & 5 Y +                                  & 7           & M      \\ \hline
% P07 & native, HHD                          & 5 Y +                                  & 7           & M      \\ \hline
% P08 & native, HHD                          & 5 Y +                                  & 7           & M      \\ \hline
% P09 & native, HHD                          & 5 Y +                                  & 7           & M     
% \end{tabular}
% \end{table}
% \label{table:participants}

% %https://www.tablesgenerator.com/#
% \begin{table}[]
% \caption{Participant Information: C: CODA (Child of Deaf), H: Hearing Individual, D: Deaf Individual. (Years of Learning ASL) Proficiency of fingerspelling was self-reported by participants on a scale of 1 to 7, with 7 being the highest}
% \begin{tabular}{|c|c|c|c|c|c|c|c|c|c|}
% \hline
%     & p01 & p02  & p03  & p04  & p05  & p06  & p07 & p08 & p09 \\ \hline
% Background      & C (5+)   & H (3) & H (1) & H (5) & H (1) & H (3) & D (5+)  & D (5+)    & D (5+)   \\ \hline

% Gender     & M   & M  & M  & F  & M  & F  & M   & M   & F  \\ \hline
% Proficiency     & 7   & 5    & 6    & 6    & 5    & 5    & 7   & 7   & 7   \\ \hline
% % Word Per Minute & 48.01 & 32.77   &  31.04  &24.94& 24.02  &  28.87  & 39.35  &  37.12  & 50.89  \\ \hline



% \end{tabular}
% \label{table:participants}
% \end{table}
% To explore the usefulness of this approach, we compared emulated fngerspelling recognition
% with a virtual keyboard for 12 Deaf participants. With practice, fngerspelling is faster (42.5 wpm), potentially
% has fewer errors (4.02\% corrected error rate) and higher throughput (14.2 bits/second),

\subsection{Dataset and Procedure}
\label{experiment}

\begin{figure}[t]
  \includegraphics[width=\linewidth]{document/figures/studyprocedure.png}
  \caption{Study Procedures: an English word with guide fingerspelling images (left) and examples of experimental setup locations (e.g., in a study room (center); in a home (right)) }
  \Description{Study Procedures: a word with guide fingerspelling images and the study locations (in a study room and a home)}
  \label{fig:studyprocedure}
\end{figure}

\textcolor{black}{To ensure consistency with FingerSpeller \cite{martin2023fingerspeller}, particularly for performance evaluation, we adopted their dataset and procedure. Since we evaluated our system as an accessible optional text entry method, the MacKenzie-Soukoreff Phrase Set \cite{mackenzie2003phrase} was selected as the standard dataset, which is commonly used to evaluate text entry systems.} This set comprises 500 phrases, totaling 1,164 unique words. The words in this set vary in length, ranging from a minimum of 1 to a maximum of 13, with an average word length of 6.055 (SD: 2.312). 


Each participant was tasked with fingerspelling each of these words twice, resulting in a comprehensive dataset of 2328=2*1164 fingerspelled words per participant, which was collected in two rounds of data collection. During each round, a partcipant completed each word once across 10 sessions. Nine of these sessions contained 116 words each, and one session contained 120 words. We carefully ensured that there was no overlap in words between sessions and that each session maintained a similar distribution of letters and word lengths. The average word length across sessions was 6.053 (SD = 0.181), with median values of 5 or 6.

The experiment was conducted in a semi-controlled environment, as illustrated in Figure \ref{fig:studyprocedure}. During each session, participants followed specific guidelines for fingerspelling. They were provided with real-time video feedback of their signing and were allowed to correct any mistakes as needed. To aid ASL learners, we displayed each English word with accompanying guide images of corresponding ASL letters. Participants used their non-dominant hand to press a space key after completing each word, which allowed us to record the start and end times for each fingerspelled sequence to calculate participants' fingerspelling speed. We instructed participants to return to a neutral hand pose between words and to fingerspell at their own comfortable, natural pace. Session durations varied from 5 to 13 minutes, depending on the participant's fingerspelling speed.

Data collection for each participant lasted around 4 hours, split across two to three days for each participant. To assess the effect of device positioning, participants were asked to remove and reattach the ring between sessions. This approach allowed us to collect data under various ring positions, simulating real-world usage scenarios.

\begin{table*}[t]
\caption{Top-N Word Recognition and Participant Information: CODA: Child of Deaf Adults, H: Hearing. Years indicates number of years learning ASL or using it as a primary language.}
\Description{Top-N Word Recognition and Participant Information: CODA (Child of Deaf Adults), H: Hearing. Years indicates number of years learning ASL or using it as a primary language.}
\begin{tabular}{cccc|cccccr}
\hline
\multicolumn{4}{c|}{}                                                                                                  & \multicolumn{6}{c}{Offline Evaulation}                                                                                                                                                                                    \\
\multicolumn{4}{c|}{}                                                                                                  & \multicolumn{6}{c}{\textbf{Word-level}}                                                                                                                                                                                   \\ \hline
\multicolumn{1}{c|}{}     & \multicolumn{1}{c|}{\textbf{Experience}} & \multicolumn{1}{c|}{\textbf{Gender}} & \textbf{Years} & \multicolumn{1}{c|}{\textbf{Top1}} & \multicolumn{1}{c|}{\textbf{Top2}} & \multicolumn{1}{c|}{\textbf{Top3}} & \multicolumn{1}{c|}{\textbf{Top4}} & \multicolumn{1}{c|}{\textbf{Top5}} & \multicolumn{1}{c}{\textbf{LER}} \\ \hline
\multicolumn{1}{c|}{\textcolor{black}{Avg.}} & \multicolumn{1}{c|}{}               & \multicolumn{1}{c|}{}                & \textit{5.44} & \multicolumn{1}{c|}{\textit{89.8}} & \multicolumn{1}{c|}{\textit{92.8}} & \multicolumn{1}{c|}{\textit{94.3}} & \multicolumn{1}{c|}{\textit{95.1}} & \multicolumn{1}{c|}{\textit{95.6}} & \textit{0.131}                   \\ \hline

\multicolumn{1}{c|}{P01}  & \multicolumn{1}{c|}{Learner, H}     & \multicolumn{1}{c|}{F}               & 3             & \multicolumn{1}{c|}{95.338}        & \multicolumn{1}{c|}{97.220}        & \multicolumn{1}{c|}{98.162}        & \multicolumn{1}{c|}{98.501}        & \multicolumn{1}{c|}{98.846}        & 0.068                            \\ \hline
\multicolumn{1}{c|}{P02}  & \multicolumn{1}{c|}{Learner, H}     & \multicolumn{1}{c|}{M}               & 5             & \multicolumn{1}{c|}{89.161}        & \multicolumn{1}{c|}{92.943}        & \multicolumn{1}{c|}{94.796}        & \multicolumn{1}{c|}{95.612}        & \multicolumn{1}{c|}{96.172}        & 0.126                            \\ \hline
\multicolumn{1}{c|}{P03}  & \multicolumn{1}{c|}{Learner, H}     & \multicolumn{1}{c|}{F}               & 3             & \multicolumn{1}{c|}{94.546}        & \multicolumn{1}{c|}{96.565}        & \multicolumn{1}{c|}{97.772}        & \multicolumn{1}{c|}{98.155}        & \multicolumn{1}{c|}{98.371}        & 0.155                            \\ \hline
\multicolumn{1}{c|}{P04}  & \multicolumn{1}{c|}{Learner, H}     & \multicolumn{1}{c|}{M}               & 1             & \multicolumn{1}{c|}{97.851}        & \multicolumn{1}{c|}{98.968}        & \multicolumn{1}{c|}{99.226}        & \multicolumn{1}{c|}{99.527}        & \multicolumn{1}{c|}{99.614}        & 0.042                            \\ \hline
\multicolumn{1}{c|}{P05}  & \multicolumn{1}{c|}{Learner, H}     & \multicolumn{1}{c|}{F}               & 3             & \multicolumn{1}{c|}{95.020}        & \multicolumn{1}{c|}{96.823}        & \multicolumn{1}{c|}{97.599}        & \multicolumn{1}{c|}{98.026}        & \multicolumn{1}{c|}{98.499}        & 0.089                            \\ \hline
\multicolumn{1}{c|}{P06}  & \multicolumn{1}{c|}{Deaf}           & \multicolumn{1}{c|}{M}               & 7             & \multicolumn{1}{c|}{79.840}        & \multicolumn{1}{c|}{83.822}        & \multicolumn{1}{c|}{86.136}        & \multicolumn{1}{c|}{87.358}        & \multicolumn{1}{c|}{87.945}        & 0.223                            \\ \hline
\multicolumn{1}{c|}{P07}  & \multicolumn{1}{c|}{CODA, H}        & \multicolumn{1}{c|}{M}               & 10            & \multicolumn{1}{c|}{87.884}        & \multicolumn{1}{c|}{91.705}        & \multicolumn{1}{c|}{93.641}        & \multicolumn{1}{c|}{94.628}        & \multicolumn{1}{c|}{95.187}        & 0.162                            \\ \hline
\multicolumn{1}{c|}{P08}  & \multicolumn{1}{c|}{Deaf}           & \multicolumn{1}{c|}{M}               & 10            & \multicolumn{1}{c|}{77.636}        & \multicolumn{1}{c|}{83.739}        & \multicolumn{1}{c|}{86.901}        & \multicolumn{1}{c|}{88.662}        & \multicolumn{1}{c|}{89.837}        & 0.211                            \\ \hline
\multicolumn{1}{c|}{P09}  & \multicolumn{1}{c|}{Deaf}           & \multicolumn{1}{c|}{M}               & 7             & \multicolumn{1}{c|}{90.688}        & \multicolumn{1}{c|}{92.993}        & \multicolumn{1}{c|}{94.397}        & \multicolumn{1}{c|}{95.665}        & \multicolumn{1}{c|}{96.345}        & 0.105                            \\ \hline
\end{tabular}
\end{table*}
\label{fig:performance_tops}


We collected a total of 20,604 fingerspelled words for system evaluation from the nine participants. However, due to technical issues, we lost data from three sessions: the 7th session of P1 (116 words), the 7th session of P7 (116 words), and the 5th session of P8 (116 words).




\subsection{Evaluation Metrics}

For evaluating recognition accuracy, we utilized two primary metrics: Letter Error Rate (LER) and word-level accuracy. LER measures the percentage of incorrect letters in the output compared to the ground truth, with a lower value indicating better accuracy. Word-level accuracy was assessed based on top-1 to top-5 predictions, providing a comprehensive view of the model's effectiveness.


\subsubsection{Letter Error Rate (LER)}

LER is a metric used to evaluate the accuracy of a system in recognizing or generating sequences of letters, such as in speech recognition, handwriting recognition, or fingerspelling recognition. It measures the percentage of incorrect letters in the output compared to a reference or ground truth. A lower LER indicates better accuracy in recognizing or generating a sequence of letters. We calculate LER before word correction. For example, when the target word is "hello" and the predicted word from the model is "helo", there is no substitution, one deletion (the second "l" is missing in the predicted word), and no insertion. In this case, the LER is 1/5 = 0.2.

\begin{equation}
LER = \frac{{\text{Substitutions} + \text{Deletions} + \text{Insertions}}}{{\text{Total number of letters in the reference sequence}}}
\end{equation}



\subsubsection{Top N Word-level Accuracy}
We also report word-level accuracy based on top-1 and top-5 predictions, as this provides a more comprehensive evaluation of the model's effectiveness. After collecting the word data, we calculate performance by determining accuracy, defined as the number of correctly predicted words divided by the total number of words. Additionally, we identify the top-3 potential words by selecting those with the smallest Levenshtein distances. For top-3 accuracy, if the correct word is among the three with the smallest Levenshtein distances, the prediction is considered correct; otherwise, it is incorrect. For example, if the input word is "fax" and the model predicts "aax", the three closest words by Levenshtein distance could be "fox", "tax", and "fax." In this case, the system's prediction would be considered correct.




\subsection{Results}

Our analysis revealed that SpellRing achieved a promising overall accuracy of 89.89\% (SD = 8.59\%) for top-1 predictions and 95.72\% (SD = 5.28\%) for top-5 predictions in recognizing the 1,164 fingerspelled words in our dataset. This performance is comparable to \textcolor{black}{FingerSpeller \cite{martin2023fingerspeller} using two rings} (87\% accuracy), demonstrating the effectiveness of our single-ring approach.

\subsubsection{Top-N Word Recognition}

SpellRing's recognition accuracy improved with increasing N in top-N predictions. For top-1 predictions, the system achieved 89.89\% (SD = 8.59\%) accuracy. This improved to 92.85\% (SD = 7.00\%) for top-2, 94.37\% (SD = 6.13\%) for top-3, 95.20\% (SD = 5.56\%) for top-4, and reached 95.72\% (SD = 5.28\%) for top-5 predictions.
Table 3 illustrates the top-N word accuracy for each participant. The significant improvement from top-1 to top-5 accuracy (a 5.83\% increase) suggests potential benefits for ASL translation applications. By considering multiple top predictions, the system could leverage additional contextual information to produce more coherent and contextually appropriate sentences. This approach could help maintain the flow and meaning of the text by selecting from the best few options at each step. Furthermore, a user interface displaying the top three candidates immediately after prediction would allow signers to choose the correct option, potentially improving overall system accuracy.



\begin{figure}[b]
  \includegraphics[width=0.8\linewidth]{document/figures/output_completion.png}
  \caption{Offline Word-level Performance over Completion Time in User study 1. Dots indicate each session. Note that the model tended to have lower performance for faster signers and better performance for slower signers.}
  \Description{Offline Word-level Performance over Completion Time in User study 1. Dots indicate each sessions. Note that faster signers tended to have lower performance, while slower signers tended to perform better.}
  \label{fig:performance_time}
\end{figure}




\subsubsection{ASL Learners vs. Native/ Fluent Signers}

We observed significant variations in performance across participants, with top-1 accuracy ranging from 77.63\% to 97.85\%. This variability can be attributed to differences in participants' fingerspelling proficiency, affecting articulatory habits and speed. Notably, the model showed better performance for ASL learners (M = 94.38\%, SD = 4.28\%) than for native/ fluent signers (M = 84.06\%, SD = 9.26\%). We attribute this difference primarily to variations in fingerspelling speed.  ASL learners took longer to complete each session (M = 428.88 sec, SD = 74.24) compared to fluent signers (M = 329.624 sec, SD = 42.54). ASL learners tended to fingerspell more slowly, clearly distinguishing each letter, while native/ fluent signers fingerspelled more quickly, resulting in greater handshape variation for certain letters and blurred transitions between letters. This speed difference posed challenges for the model in accurately recognizing individual letters in rapid sequences. We further analyze the impact of fingerspelling speed on performance in the following section.

\subsubsection{Impact of Fingerspelling Speed on Performance}

Fingerspelling speed varied among participants, leading to differences in task completion times. These speed variations, along with factors such as participants' prior experience signing specific words and breaks taken between sessions, appeared to influence the model's performance (Figure \ref{fig:performance_time}). Specifically, the model had lower performance for faster signers and performed better for slower signers.

We found that faster signing speeds often led to greater handshape variation, potentially impacting our model's performance. For instance, when signed quickly, some letters (e.g., ‘C’, ‘O’, ‘E’, ‘I’) were often not fully articulated. The contrast in fingerspelling between the highest and lowest performing participants illustrates this effect. P05, an ASL learner with 3 years' experience, achieved the highest performance with 97.85\% (SD = 2.07\%) accuracy. They articulated each letter very clearly and exhibited minimal handshape variation, resulting in consistent data. In contrast, P08, a fluent signer, had the lowest performance at 77.64\% (SD = 7.8\%). Their fingerspelling speed varied across sessions and consisted of many allophonic handshape variations.

These observations suggest that as fingerspelling speed increases, individual letters are articulated less fully and/or are coarticulated with neighboring letters, posing a challenge for accurate recognition. This challenge resembles those in early-stage speech recognition systems, which experienced performance drops when speakers spoke too rapidly or with strong accents. We discuss this in more detail in Section \ref{impact_speed}.








