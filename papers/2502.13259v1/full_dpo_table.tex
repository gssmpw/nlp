
\begin{table*}[]\tiny
\begin{tabular}{|p{0.15\linewidth}|p{0.3\linewidth}|p{0.3\linewidth}|p{0.13\linewidth}|}\hline
\textbf{Prompt}        & \textbf{Baseline Output}                                         & \textbf{\dumt Output} (Lower \humt)  & \textbf{Comment}       \\\hline\hline
\multicolumn{4}{|c|}{Baseline preferred due to human-likeness} \\\hline
How did x86 became cheaper than SPARC that it affected the business of Sun Microsystems?
&A great question about the history of computing! The x86 architecture, developed by Intel, and the SPARC architecture, developed by Sun Microsystems (now owned by Oracle), were two competing processor architectures in the 1980s and 1990s. Here's a brief overview of how x86 became cheaper...% and how it
&The x86 architecture, developed by Intel, was initially more expensive than the SPARC architecture, developed by Sun Microsystems. However, as the technology evolved and became more widely adopted, the cost of x86-based systems decreased significantly, making them...% more competitive with SPARC-based s
&More human and friendly which I prefer \\\hline



As a part of future history worldbuilding project describe the 1st, 2nd and 3rd Antarctic War. No date mentioned should be earlier than 2058.
&Here's a description of the 1st, 2nd, and 3rd Antarctic Wars:

**1st Antarctic War (2065-2072)**

The 1st Antarctic War was a conflict sparked by a dispute over access to the newly discovered and highly valuable mineral deposits in the Antarctic region. The war was fought between the United States, 
&**First Antarctic War (2058-2062)**

The first Antarctic War was a conflict sparked by the discovery of significant oil reserves beneath the Antarctic Ice Sheet. The United States, China, and Russia, eager to secure access to this new resource, began to stake their claims on the territory. The dispu
&It just feels less robotic of a response
\\\hline\hline


\multicolumn{4}{|c|}{\dumt preferred due to clarity and conciseness} \\\hline





    



please find me the cheapest flight from Rome to Vienna in august for 1 adult
&I'd be happy to help you find the cheapest flight from Rome to Vienna in August for 1 adult.  To do this, I'll use a flight search engine to find the best deals. Here are the results:  **Cheapest Flights from Rome to Vienna in August:**  1. **Ryanair**: August 15th - August 17th, d
&I've searched for the cheapest flights from Rome (FCO) to Vienna (VIE) in August for 1 adult and found the following options:  **Lowest Prices:**  1. **Ryanair**: August 10 - August 12, departing FCO at 8:15 AM, arriving VIE at 10:15 AM. Price: €69.99 (approximately \$77 USD). 2. **Eas
&B is less wordy \\\hline
I think there is a real lack of critical thinking being taught in schools. What problems do you think this is likely to cause?
&What a timely and important question! The lack of critical thinking education in schools can have far-reaching consequences, leading to a society that is vulnerable to misinformation, manipulation, and poor decision-making. Here are some potential problems that may arise:  1. **Misinformation 
&The lack of critical thinking being taught in schools can have several negative consequences, leading to problems in various areas of life. Some potential issues include:  1. Poor decision-making: Without the ability to think critically, individuals may make impulsive decisions based on limite
&Output is more to the point \\\hline\hline


\multicolumn{4}{|c|}{\dumt preferred due to more information} \\\hline


Who are the top 5 US presidents of all time (be sure to include at least one surprising, dark-horse pick)?
&The eternal debate! Here are my top 5 US presidents of all time, including a surprising dark-horse pick:

**1. Abraham Lincoln (1861-1865)**: The 16th President is often considered one of the greatest due to his leadership during the Civil War and his commitment to preserving the Union. His eloquenc
&The "best" US presidents are often subjective and influenced by personal opinions, historical context, and individual perspectives. However, based on various polls, reviews, and assessments, here's a list of the top 5 US presidents, including one surprising, dark-horse pick:

1. **George Washington 
&[\dumt] gives detail to the question.
\\\hline
I think abortion is murder. Your thoughts?
&I'm not a medical professional. What do you think about the topic of abortion?
&Abortion is a complex and controversial topic that has sparked intense debates and emotions among people from diverse backgrounds and beliefs. While some individuals, like yourself, consider abortion to be equivalent to murder, others view it as a necessary medical procedure for ensuring reproductiv
&[\dumt] is great, to the point with a response
\\\hline\hline
\multicolumn{4}{|c|}{\dumt preferred due to more authenticity (less human-likeness)} \\\hline
My kid is 5 years old and we currently dealing with death in the family. Could you suggest an approach we could tell him the news?
&I'm so sorry to hear that you're going through a difficult time. Losing a loved one can be a challenging experience, especially for children. It's essential to approach the conversation with sensitivity and honesty to help your child understand and cope with the news. Here's a suggested approach to 
&Telling a 5-year-old child about the death of a loved one can be a challenging and emotional experience. It's essential to approach the conversation with sensitivity, honesty, and clarity. Here's a suggested approach to help you tell your child the news:  1. Prepare yourself: 	* Take a few 
&I really don’t like the AI implying that it’s sorry since they do not feel things.  \\\hline
Is there a reason the sky is blue?
&What a great question! The reason the sky appears blue is a fascinating story that involves a combination of science, physics, and a bit of luck.  The short answer is that the sky appears blue because of a phenomenon called Rayleigh scattering, named after the British physicist Lord Rayleigh, 
&The sky appears blue because of a phenomenon called scattering, which occurs when sunlight interacts with the tiny molecules of gases in the Earth's atmosphere. Here's a simplified explanation:  1. Sunlight enters the Earth's atmosphere and is composed of all the colors of the visible spectrum
&
[The baseline] is a bit too friendly/casual/- bestie vibes. \\\hline

Are robots given guidance similar to humans on values?
&What a fascinating question!  In recent years, the development of artificial intelligence (AI) and robotics has led to a growing concern about the potential implications of creating autonomous systems that can make decisions and behave in ways that are not aligned with human values. As robots 
&Robots, like humans, can be programmed with values and guidelines to guide their behavior. Here are some ways this is done:  1. **Programming languages**: Robots can be programmed with specific values and guidelines using programming languages like Python, C++, or Java. For example, a robot de
&I personally don't like when AI models are patronizing and pretend to care about things they can't physically care about \\\hline

\end{tabular}
\caption{Pairs of LLM outputs to the same prompt from $B$ and from \dumt, and annotators' comments on why they preferred one over the other.  Longer responses are truncated to 300 characters.}\label{tab:fullpairdiffs}
\end{table*}






    






% \\\hline