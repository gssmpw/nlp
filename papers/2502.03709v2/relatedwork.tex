\section{Background and Related work}
\subsection{Background}

The nine-grid layout, consisting of a 3x3 grid of squares, is a common pattern widely used in various domains. In today's social media, people are enthusiastic about showcasing images in a nine-grid format to express their appreciation for things and their love for life. Additionally, the nine-grid layout continues to find extensive applications in other fields such as user interfaces of mobile devices, game design, and information display. The arrangement of the nine-grid has a significant impact on user experience and usability. An effective arrangement can enhance user efficiency, visibility of information, and overall aesthetic appeal, thereby making the nine-grid more popular. 

The nine-grid layout visually presents integrity, which is a special aspect of positioning images. Although there have been many studies on the impact of the order of images on user attention. However, there is still a lack of research on whether the order of images affects the user's overall liking in the nine-grid situation. The following study will be based on two existing theories of the position significance of images to study the most popular image layout method among users in the nine-grid situation.

\subsection{Image Popularity on social media}

In examining the influence of different arrangement methods on the nine-grid layout, we have reviewed several relevant studies. Firstly, Keyan Ding et al\cite{ding2019intrinsic}. aimed to develop a computational model for accurately predicting the viral potential of social images. Their study involved constructing a large-scale image database and employing a deep neural network-based computational model to analyze the individual contributions of image content to its popularity. However, their research primarily focused on scoring and predicting the popularity of single images, which differs from the focus of our study.Additionally, Hossein Taleb\cite{talebi2018nima} proposed a method based on deep object recognition networks to reliably assess image quality with high correlation to human perception. This approach was employed for predicting image quality assessment and shares some similarities with our study, although the emphasis is different.

Furthermore, Ma Xiaoyue et al\cite{ma2021image}. investigated the impact of image positioning and layout on user engagement behavior using multi-image tweet data from the Sina Weibo platform. They utilized an XGBoost model to predict the user engagement potential of images and conducted correlation and regression analyses, revealing that image positioning and layout moderately influence user interaction. This study explored the effects of different arrangement methods for multiple images, which overlaps to some extent with our research. Zhonghua He et al\cite{ma2023suggest}. explored the relationship between the sequential position of images in nine-image grid and the Big Five personality traits. Their findings demonstrated that users can utilize prominent positions in the nine-image grid to highlight specific content, resulting in more captivating Weibo narratives. This study enriched the field of social media image research and revealed motivations for using multiple images.

In contrast, our paper specifically focuses on examining the impact of different arrangement methods on the nine-grid layout. Our novelty lies in exploring the influences of various arrangement methods on the visual appeal and narrative expression of the nine-grid images. By comparing different arrangement methods, we aim to provide specific guidance and insights on the optimal arrangement strategies for the nine-grid layout. Our research fills a gap in the existing knowledge and offers practical implications for designers and practitioners seeking to optimize the arrangement of images in the nine-grid layout.

\subsection{Position significance of images}

There are two views on the position significance of images. The first one is the serial-position effect. It is the tendency of a person to recall the first and last items in a series best, and the middle items worst\cite{colman2015dictionary}. It was first documented by Ebbinghaus\cite{ebbinghaus1885gedachtnis} in his seminal work on memory in the late 19th century, which is a fundamental principle of cognitive psychology. Numerous studies have affirmed its robustness across various tasks and stimuli types. While most research has focused on verbal materials, recent studies have extended this phenomenon to visual stimuli, including images.

The serial-position effect manifests as two distinct phenomena: the primacy effect and the recency effect. The primacy effect refers to the enhanced recall of items presented at the beginning of a sequence. This phenomenon is thought to occur because early items receive more rehearsal and encoding time, allowing them to be transferred into long-term memory more effectively. Meanwhile, the recency effect pertains to the superior recall of items presented at the end of a sequence. This effect is believed to result from items being held in short-term or working memory, which is still active during recall tasks\cite{glanzer1966two}. In the context of image sequences (or nine-grid layout), the serial-position effect can be used to enhance users’ attention by strategically positioning impactful pictures at the beginning and the end. This balanced approach ensures that important information is not only introduced early but also reinforced at the end, maximizing the likelihood of retention and recall. 

The second definition of significant position of nine-grid layout adheres to visual attention distribution\cite{ma2023suggest}. Some researchers suggest that users will first pay attention to the center of the screen, which makes the position in the middle of the image sequence important. The image that is most compelling or fits the purpose most can be put in the center to gain more visual attention from users. However, further studies on which of the two definitions has the highest level of attractiveness in the situation of nine-grid layout are still needed.