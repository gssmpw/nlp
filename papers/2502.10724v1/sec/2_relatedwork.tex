\section{Related Work}
\label{sec:relatedwork}

\subsection{Test-time Adaptation in 3D Pose Estimation}
Generalizing 3D human pose models to in-the-wild data is challenging due to distribution shifts, driving interest in test-time adaptation methods with 2D information. Initially, the methods~\cite{iso,boa,dynaboa,DAPA} rely on ground truth 2D poses to rectify 3D predictions along with the proposed adaptation techniques. For instance, ISO~\cite{iso} leverages invariant geometric knowledge through self-supervised learning. BOA~\cite{boa} introduces bilevel optimization to better integrate temporal information with 2D weak supervision. DynaBOA~\cite{dynaboa} extends BOA by using feature distance as a signal for domain shift with dynamic updates. DAPA~\cite{DAPA} employs data augmentation by rendering estimated poses onto test images, guiding the model toward the target domain. 

However, ground truth 2D poses are not always available in many real-world scenarios. 
Similar to our method,
CycleAdapt~\cite{cycleadapt} operates on estimated 2D data and introduces a cyclic framework, where the motion denoising module smooths 3D pseudo labels to mitigate the impact of noisy 2D inputs.
Nonetheless, the inherent 2D-to-3D ambiguity still significantly degrades model performance, and unguided predictions often occur when 2D inputs are unreliable.
Our method addresses this by leveraging explicit semantic understanding, \ie, actions, to guide the model in predicting specific motion patterns.
This is the first approach to emphasize high-level semantic information to reduce the 2D-to-3D solution space in test-time 3D human pose estimation.


\subsection{Human Motion Priors}
Leveraging motion prior as supervision is commonly applied in video-based human pose estimation due to the lack of 3D annotations with RGB input. Trained on a large and diverse human motion database AMASS~\cite{mahmood2019amass, human36m}, the human prior provides a regularized motion space. Previous works mainly drag the predicted or generated motion to the regularized motion space to achieve the refinement~\cite{vibe, shin2023wham, rempe2021humor}, infilling~\cite{yuan2022glamr,zhang2021learning}, denoising~\cite{cycleadapt} goals. Specifically,  
VIBE~\cite{vibe} extended VAE~\cite{simplifyx} and adversarial~\cite{hmrnet} feasible prior from pose into motion space to improve temporal continuity of movement. GLAMR~\cite{yuan2022glamr} learned a VAE motion infiller to infill motion where the moving human is occluded. CycleAdapt utilizes the prior to denoise 3D meshes and improves human pose estimation during test-time with noisy 2D poses. 
However, those applications of human motion prior only optimize the predicted meshes of the segment to be valid motion {without consideration of the motion semantics.} Compared to other works, we are the first to split regularized motion space according to text labels extracted from the video segment to benefit motion prediction.
