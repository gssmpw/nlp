\section{Related Work}
\label{sec:rel_work} 

In this section, we analyze related work on predicting compliance states of ongoing process instances and quantifying the degree of compliance w.r.t. a given set of compliance constraints.

\noindent\textbf{Compliance Prediction.} \cite{DBLP:journals/tsc/Marquez-Chamorro18} classify predictive process monitoring (PPM) approaches into process-aware and non-process-aware approaches, both employing regression and classification methods to predict attributes in the future, such as next activity, next time, or the outcome of the case. As part of PPM, outcome-oriented PPM \cite{DBLP:journals/tkdd/TeinemaaDRM19,DBLP:conf/bpm/TeinemaaDMF16} and predicate prediction \cite{DBLP:conf/caise/MaggiFDG14,DBLP:journals/tsc/Francescomarino19} focus on compliance predictions of ongoing instances regarding a given set of constraints; the binary yes or no answer w.r.t. compliance is not sufficient for compliance management and requires the ability to quantify the degree of compliance.

\noindent\textbf{Compliance Degree.} \cite{DBLP:journals/ism/LuSG08} calculate a compliance distance to indicate the degree of match between the process model and the set of control rules at design time. They define control rules into four distinct classes according to the ideal semantics: ideal, sub-ideal, non-compliant and irrelevant states.
\cite{DBLP:conf/aicol/LamHK20} classify cases into full-, partial- and non-compliance and formulate a framework to detect and evaluate the degree of violations after process executions. However, the user-defined piece-wise mapping function cannot distinguish the extent of violations in a fine-grained manner. In \cite{DBLP:conf/edoc/MorrisonGK09}, a compliance scale model is created to measure the degree of compliance of completed process instances. Still, the mechanisms of c-semirings to classify process instances into $<Good, Bad>$ or $<0, .5, 1>$ are not accurate enough. \cite{Shamsaei2011IndicatorbasedPC} define a set of key performance indicators (KPIs) for each compliance rule and map the evaluation values of them to a compliance level from -100 to 100 taking into account values of target, threshold and worst. However, existing approaches provide various metrics for compliance degrees in different phases of process life cycle, but none of them target compliance prediction.