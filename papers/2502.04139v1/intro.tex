\label{sec:intro}
Indoor instance segmentation is one of the fundamental tasks in 3D scene understanding, aiming to predict masks and categories for each foreground object. With the increasing popularity of AR/VR~\cite{park2020deep,manni2021snap2cad}, 3D indoor scanning~\cite{lehtola2017comparison,halber2019rescan}, 3D/4D reconstruction~\cite{wu20244d,zhu2024motiongs,lu2024dn,luiten2023dynamic,lu2024align3r,zhang2024monst3r}, and autonomous driving~\cite{neven2018towards,yurtsever2020survey}, 3D instance segmentation has become a pivotal technology enabling scene understanding. However, the complexity of scenes and the diversity of object categories pose significant challenges to 3D instance segmentation.


%
To address the aforementioned challenges, a series of 3D instance segmentation methods~\cite{yi2019gspn,hou20193d,yang2019learning,engelmann20203d,liu2020learning, chen2021hierarchical,liang2021instance,vu2022softgroup,schult2022mask3d,sun2023superpoint,lu2023query,lai2023mask} have been proposed. Generally, these methods can be categorized into three groups: proposal-based~\cite{yi2019gspn,hou20193d,yang2019learning}, grouping-based~\cite{engelmann20203d,liu2020learning,jiang2020pointgroup,chen2021hierarchical,liang2021instance,vu2022softgroup}, and transformer-based~\cite{schult2022mask3d,sun2023superpoint,lu2023query,lai2023mask}. Proposal-based methods adopt a top-down approach, where they first extract 3D bounding boxes and then utilize a mask learning branch to predict the object mask within each box. Grouping-based methods initially generate predictions for each point (e.g., semantic categories and geometric offsets) and then generate instance proposals. Recently, transformer-based methods have attracted researchers' attention due to their elegant pipelines, reduced manual selection of geometric properties, and superior performance. 
%
These methods typically initialize a fixed number of object queries, which are then fed into the decoder to aggregate scene features. 
%
After the feature aggregation of each decoder layer, the queries output instance predictions, with each layer's predictions supervised by the ground truth. We refer to this design as per-layer auxiliary loss. The predictions from the final layer are used as the final output.
%
In this process, query initialization plays a crucial role. Current transformer-based methods propose various designs for query initialization, mainly categorized into FPS-based (farthest point sampling)~\cite{schult2022mask3d,lu2023query} and learnable-based~\cite{sun2023superpoint,lai2023mask} approaches. Furthermore, inspired by 2D instance segmentation~\cite{cheng2022masked,li2023mask,jain2023oneformer}, the design of per-layer auxiliary loss has significantly improved the training effectiveness of 3D instance segmentation. However, we observe a phenomenon of \textit{\textbf{Object Disappearance}}, where predictions for certain objects vanish as the deepening of layers. As shown in Figure~\ref{motivation1}, where the object ``picture'' obtained from the prediction at layer 4 disappears in layer 5 and layer 6. This is reflected in a decrease in recall in the quantized results, as shown in Figure~\ref{motivation} (b), contradicting the intuition that more interactions between features lead to better results.
\begin{figure}[!t]
    \vspace{-2.0em}
    \begin{center}
        \includegraphics[width=1\textwidth]{graph/motivation1.pdf}
        \caption{\textbf{The phenomenon of \textit{Object Disappearance} with the deepening of layers.} }
        \label{motivation1}
    \end{center}
    \vspace{-1.2em}
\end{figure}
\begin{figure}[!t]
    \vspace{-1.0em}
    \begin{center}
        \includegraphics[width=1\textwidth]{graph/motivation2.pdf}
        \caption{\textbf{(a) The comparison of different query initialization methods.} The FPS-based methods conduct farthest point sampling separately for each scene, placing more emphasis on positional information but lacking in aggregating content information. The learnable-based methods initialize a fixed number of queries for aggregating content information across all scenes, which is prone to empty sampling, thereby compromising foreground coverage. Our method leverages the advantages of both approaches to achieve a balanced and comprehensive solution. \textbf{(b) The recall difference.} The recall of the baseline shows instability during the iterative optimization process across layers, whereas our method, with the assistance of the Hierarchical Query Fusion Decoder, demonstrates a steady improvement in recall across each layer.\vspace{-1.0em}}
        
        \label{motivation}
    \end{center}
    \vspace{-1.0em}
\end{figure}

Based on the discussion above, we have identified two challenges that need to be addressed:
1) \textit{How to better initialize queries?}
As illustrated in Figure~\ref{motivation} (a), current transformer-based methods~\cite{schult2022mask3d,sun2023superpoint,lu2023query,lai2023mask} can mainly be categorized into FPS-based~\cite{schult2022mask3d,lu2023query} and learnable-based approaches~\cite{sun2023superpoint,lai2023mask}. Mask3D~\cite{schult2022mask3d} and QueryFormer~\cite{lu2023query} utilize FPS to obtain the initialization distribution of queries, which can more likely distribute candidates to the region where objects are located, thus reducing the empty sampling rate. However, these FPS-based approaches fail to learn content embedding across scenes effectively for feature aggregation. On the other hand, SPFormer~\cite{sun2023superpoint} and Maft~\cite{lai2023mask} employ learnable queries, which can update and learn across multiple scenes in the dataset. Nevertheless, the empty sampling rate is higher, leading to a decrease in model recall. Therefore, balancing the sampling positions of candidates and learning content embedding effectively is crucial for initializing queries.
2) \textit{How to mitigate the issue of inter-layer recall decline?}
During the decoding phase, due to the existence of auxiliary loss, the predictions of each decoder layer are supervised by ground truth. For instances that are difficult to predict, such as pictures, bookshelfs, the quality of the mask corresponding to the matched query is poor. Consequently, the mask attention~\cite{schult2022mask3d,sun2023superpoint} focuses on a large amount of noisy features, causing the optimization direction of the query to be unstable, and there is a possibility of further deterioration in mask quality. Moreover, for other unmatched queries, due to the lack of supervision signal, the optimization direction is even more random. Predicting better quality for such difficult-to-predict instances is therefore more challenging. As a result, the mask of instance ``picture'' in Figure~\ref{motivation1} is lost by layer 5, and recall decreases. To address this issue, one intuitive idea is to concatenate the outputs of each layer's predictions during model inference, and then filter out duplicate predictions through non-maximum suppression (NMS)~\cite{neubeck2006efficient}. However, since it is challenging to select suitable hyperparameters and lacks accurate confidence scores, NMS often cannot filter out lower-quality duplicate masks while retaining non-repetitive instance masks. Therefore, an end-to-end, automated design is needed to ensure that inter-layer recall does not decrease.



%

To achieve the aforementioned objectives, we propose BFL.
%
To better initialize queries, we introduce the Agent-Interpolation Initialization Module (AI2M), where we initialize a set of agents comprising two corresponding queries: position queries and content queries. Subsequently, we perform FPS on the scene point cloud and interpolate the agents' content queries to obtain the sampled points' content queries based on their positions and the positions of the position queries. This approach ensures high foreground coverage of initial queries, avoiding empty sampling, and learns content information across scenes through interpolation, thereby effectively aggregating object features.
%
To mitigate the issue of inter-layer recall decline, we propose the Hierarchical Query Fusion Decoder (HQFD). 
%
Specifically, we compute the Intersection over Union (IoU) between predicted instance masks from the ($l$-1)-th layer and the $l$-th layer. 
%
Queries from the ($l$-1)-th layer, showing low overlap (\textit{i.e.}, corresponding masks having low IoU values with all masks from the $l$-th layer), are merged with queries from the $l$-th layer and collectively fed into the ($l$+1)-th layer. This method effectively retains queries with low overlap that aid in recall, mitigating the decrease in recall caused by unstable optimization directions. It's worth noting that the number of queries with low overlap is limited, so the extra queries added at each layer are few. This results in minimal impact on computational load, with a 7.8\% increase in runtime.

%
In conclusion, our main contributions are outlined as follows:

(i) We introduce a novel 3D instance segmentation method called BFL.

(ii) We introduce a new query initialization method termed the Agent-Interpolation Initialization Module. This module integrates FPS with learnable queries to produce queries that can adeptly balance foreground coverage and content learning. It proves to be tailored for navigating complex environments.

(iii) We design the Hierarchical Query Fusion Decoder to retain low overlap queries, mitigating the decrease in recall with the deepening of decoder layers.

(iv) Extensive experiments conducted on ScanNetV2~\cite{dai2017scannet}, ScanNet200~\cite{rozenberszki2022language}, ScanNet++~\cite{yeshwanth2023scannet++}, and S3DIS~\cite{armeni20163d} datasets show that BFL can surpass state-of-the-art transformer-based 3D instance segmentation methods.
