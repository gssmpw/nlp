%paragrpah spacing
%\setlength{\parindent}{0em}
% \setlength{\parskip}{1em}
\setlength{\parskip}{0.75em}
\setlength{\parindent}{0pt}
%importing packages
\usepackage{fullpage}
\usepackage{amssymb}
\usepackage{amsmath}
\usepackage{amsthm}
\usepackage{multirow}
\usepackage{enumerate}
\usepackage{graphicx, subcaption}
\usepackage{wrapfig}
\usepackage{mathrsfs}
\usepackage[utf8]{inputenc}

\usepackage{amsfonts}
\usepackage{bbm}
\usepackage{dsfont}
\usepackage[mathscr]{euscript}

\usepackage{hyperref}
%\usepackage[backref=page]{hyperref}
% \usepackage[backref=section]{hyperref}

\usepackage{cleveref}
\usepackage{svg}
\usepackage{mathtools}
\usepackage{amsfonts}
\usepackage[T1]{fontenc}
\usepackage{mathpazo}
\usepackage{bm}
\usepackage{isomath}
\usepackage{verbatim}
\usepackage{changepage}
\usepackage[dvipsnames,svgnames,table]{xcolor}

%\usepackage{cite}
\usepackage{natbib}
\renewcommand{\comment}[1]{}
% \usepackage[]{algorithm}
% \usepackage{algorithmicx}
\usepackage[ruled,vlined]{algorithm2e}

\usepackage[noend]{algpseudocode}

%custom theorem labelling
\newtheorem{innercustomthm}{Theorem}
\newenvironment{customthm}[1]
  {\renewcommand\theinnercustomthm{#1}\innercustomthm}
  {\endinnercustomthm}

%for colored links
\definecolor{blueviolet}{RGB}{60,50,200}
\definecolor{oliveg}{RGB}{40,200,30}
\definecolor{deepblue}{RGB}{18,18,79}
\definecolor{deepgray}{RGB}{100,100,100}
\hypersetup{colorlinks=true,       % false: boxed links; true: colored links
    linkcolor=blueviolet,
    citecolor=oliveg,
    urlcolor=deepgray,
}

\usepackage{tikz}
\usetikzlibrary{calc}
%\usepackage[ruled,vlined,linesnumbered]{algorithm2e}
\usepackage{color}

% \usepackage{todonotes}

%new counter for theorem
\newcounter{counttheorem}

%defining new environment
\theoremstyle{definition}
\newtheorem{definition}{Definition}[section]
\newtheorem{problem}{Problem}

\theoremstyle{plain}
\newtheorem{lemma}{Lemma}[section]
\newtheorem{theorem}{Theorem}
\newtheorem{fact}{Fact}
\newtheorem{thm}{Theorem}[theorem]
\newtheorem{corollary}{Corollary}[section]
\newtheorem{claim}{Claim}[section]
\newtheorem{subclaim}{Subclaim}[section]
\newtheorem{observation}{Observation}[section]
\newtheorem{example}{Example}[section]
\newtheorem{question}{Question}[section]
\newtheorem{assumption}{Assumption}[section]
% \newtheorem{remark}{Remark}[section]
\theoremstyle{remark}
\newtheorem{rem}{Remark}
\newtheorem*{note}{Note}
\newtheorem{case}{Case}
% \newtheorem{rem}{Remark}[section]

%defining new commands
%\newcommand{\SUM}[2]{\sum_{#1}^{#2}}
\newcommand{\der}[2]{\frac{\partial{#1}}{\partial{#2}}}

% Undirected edge
\newcommand{\undir}{\ensuremath{-}}
% Directed edge
\newcommand{\dir}{\ensuremath{\rightarrow}}
% reversed directed edge
\newcommand{\rdir}{\ensuremath{\leftarrow}}
% adjacency
\newcommand{\adj}{\ensuremath{\cdots}}
% underbar
\newcommand{\ubar}[1]{\underline{#1}}

%logic
\newcommand{\st}[0]{\ensuremath{\;\mathbf{|}\;}}

%norms and sizez
\newcommand{\abs}[1]{\ensuremath{\left|#1\right|}}
\newcommand{\card}[1]{\abs{#1}}
% \newcommand{\norm}[2][]{\ensuremath{\Vert #2 \Vert_{#1}}}

%Calculus
\newcommand{\diff}[2]{\frac{\text{d}#1}{\text{d}#2}}
\newcommand{\pdiff}[2]{\frac{\partial #1}{\partial #2}}
\newcommand{\grad}[1]{\nabla\left(#1\right)}


%ceils and floors
\newcommand{\ceil}[1]{\ensuremath{\left\lceil#1\right\rceil}}
\newcommand{\floor}[1]{\ensuremath{\left\lfloor#1\right\rfloor}}

%Probability
\newcommand{\prob}{\mathbb{P}}

%\renewcommand{\Pr}[2][]{\ensuremath{\mathbb{P}_{#1}\insq{#2}}}
%\newcommand{\E}[2][]{\ensuremath{\mathbb{E}_{#1}\insq{#2}}}
\newcommand{\Var}[2][]{\ensuremath{\mathbf{Var}_{#1}\insq{#2}}}
\newcommand{\Po}[0]{\ensuremath{\mathrm{Po}}}
%Parentheses
\newcommand{\ina}[1]{\left<#1\right>}
\newcommand{\inb}[1]{\left\{#1\right\}}
\newcommand{\inp}[1]{\left(#1\right)}
\newcommand{\insq}[1]{\left[#1\right]}

%Definitions
\newcommand*{\defeq}{\mathrel{\rlap{%
                     \raisebox{0.3ex}{$\m@th\cdot$}}%
                     \raisebox{-0.3ex}{$\m@th\cdot$}}%
                    =}
\newcommand*{\eqdef}{=
  \mathrel{\rlap{%
      \raisebox{0.3ex}{$\m@th\cdot$}}%
    \raisebox{-0.3ex}{$\m@th\cdot$}}%
}


%Fractions
\newcommand{\nfrac}[3][]{\nicefrac[#1]{#2}{#3}}
\newcommand{\xfrac}[3][]{\nicefrac[#1]{#2}{#3}}

%Graphs
\renewcommand{\deg}[1]{\ensuremath{\mathrm{deg}\inp{#1}}}
\newcommand{\skel}[1]{\ensuremath{\mathrm{skeleton}\inp{#1}}}
\newcommand{\mathdash}{\relbar\mkern-9mu\relbar}
\newcommand{\undiredge}[2]{\ensuremath{#1\mathdash#2}}


% parent
\newcommand{\pa}[2][]{\ensuremath{\textup{pa}_{#1}\inp{#2}}}
% sink node function
\newcommand{\sink}[2][]{\ensuremath{\textup{sink}_{#1}\inp{#2}}}
% sink node term
\newcommand{\sinkv}{maximal-clique-sink}



\newcommand{\1}{ \mathds{1}}
\newcommand{\A}{\mathcal{A}}
% \newcommand{\E}{\mathbb{E}}
\newcommand{\F}{\mathcal{F}}
\newcommand{\G}{\mathcal{G}}
\renewcommand{\H}{\mathcal{H}}
\newcommand{\I}{\mathcal{I}}
\newcommand{\J}{\mathcal{J}}
\newcommand{\K}{\mathcal{K}}
\renewcommand{\L}{\mathcal{L}}
\newcommand{\M}{\mathcal{M}}
\newcommand{\N}{\mathbb{N}}
\newcommand{\Ncal}{\mathcal{N}}
\renewcommand{\O}{\mathcal{O}}
\renewcommand{\P}{\mathcal{P}}
\newcommand{\p}{\mathcal{\pi}}
\newcommand{\Q}{\mathcal{Q}}
\newcommand{\R}{\mathbb{R}}
\renewcommand{\S}{\mathcal{S}}
\newcommand{\T}{\mathcal{T}}
\newcommand{\U}{\mathcal{U}}
\newcommand{\X}{\mathcal{X}}
\newcommand{\Y}{\mathcal{Y}}

\newcommand{\CE}{\texttt{ALG-CE}}
\newcommand{\UE}{\texttt{ALG-UE}}
\newcommand{\pihatUE}{\widehat{\pi}_{\texttt{UE}}}

\DeclareMathOperator*{\argmax}{arg\,max}
\DeclareMathOperator*{\argmin}{arg\,min}
\DeclareMathOperator*{\argminmax}{arg\,min\,max}
\DeclareMathOperator*{\argmaxmin}{arg\,max\,min}
\DeclareMathOperator*{\E}{\mathbb{E}}
\DeclareMathOperator*{\maximize}{\texttt{maximize}}
\DeclareMathOperator*{\minimize}{\texttt{minimize}}

\newcommand{\tr}{\top}
\newcommand{\norm}[1]{\left\lVert#1\right\rVert}


%Sets
\newcommand{\Z}[0]{\ensuremath{\mathbb{Z}}}


%\newcommand{\R}[0]{\ensuremath{\mathbb{R}}}
%Iverson bracket
% \newcommand{\I}[1]{\ensuremath{\insq{#1}}}
%\newcommand{\C}[0]{\ensuremath{\mathbb{C}}}
%\newcommand{\F}[1]{\ensuremath{\mathbb{F}_{#1}}}%Finite fields
%\newcommand{\N}[0]{\mathbb{N}}
\renewcommand{\emptyset}[0]{\varnothing}
\newcommand{\comp}[1]{\overline{#1}} %Complement

%Mathops
\newcommand{\poly}[1]{\ensuremath{\mathop{\mathrm{poly}}\inp{#1}}}

%Text
\newcommand{\etal}[0]{\emph{et al.}}

%Vectors
\renewcommand{\vec}[1]{\mathbf{#1}}
%\newcommand{\T}[0]{\ensuremath{\intercal}}


%footnotes
\renewcommand\thefootnote{\textcolor{red}{\arabic{footnote}}}
\makeatletter
\newcommand\footnoteref[1]{\protected@xdef\@thefnmark{\ref{#1}}\@footnotemark}
\makeatother


\usepackage{titling}

\setlength{\droptitle}{-5em}   % This is your set screw

%for algorithm removing colon
%\algsetup{linenodelimiter=.}

\newcommand{\ragllm}{\texttt{Rag-LLM}}
\newcommand{\reasoner}{\texttt{Reasoner}}
\newcommand{\retriever}{\texttt{Retriever}}
\newcommand{\synthesizer}{\texttt{Synthesizer}}
\newcommand{\query}{q}
\newcommand{\qd}{\mathcal{Q}}
\newcommand{\tokens}{\mathcal{T}}
\newcommand{\doc}{d}
\newcommand{\docs}{\mathcal{D}}
\newcommand{\chosendocs}{\mathcal{C}}
\newcommand{\prompt}{\mathcal{P}}
\newcommand{\augmentedprompt}{\mathcal{A}}
\newcommand{\response}{\mathcal{R}}
\newcommand{\cm}{\mathcal{M}}
\newcommand{\ct}{\tau}
\newcommand{\correctresponse}{\mathcal{CR}}
\newcommand{\vectorizationmodel}{\mathcal{V}}
