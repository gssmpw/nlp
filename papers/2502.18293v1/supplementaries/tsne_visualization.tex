\clearpage
\section{Visualization of t-SNE embeddings for Diverse Responses Across Queries}
\label{sec:tsne_visualization}

In this section, we showcase the performance of our method through plots of TSNE across various examples. These illustrative figures show how our baseline Bottom-k Algorithm (Section \ref{sec:ampo_bottomk}) chooses similar responses that are often close to each other. Hence the model misses out on feedback relating to other parts of the answer space that it often explores. Contrastingly, we often notice diversity of response selection for both the $\ampoos$ and $\ampocs$ algorithms.

% \begin{figure*}[!thbp]
%     \centering
%     \includegraphics[width=1.0\textwidth]{images/tsne_plot_with_scores_23687.pdf}
%     \caption{t-SNE visualization of projected high-dimensional response embeddings into a 2D space, illustrating the separation of actively selected responses. (a) AMPO-BottomK (baseline). (b) AMPO-Coreset (ours). (c) Opt-Select (ours). We see that the traditional baselines select many responses close to each other, based on their rating. This provides insufficient feedback to the LLM during preference optimization. In contrast, our methods simultaneously optimize for objectives including coverage, generation probability as well as preference rating.}
% \label{fig:tsne_diverse_analysis}
% \end{figure*}

% \begin{figure*}[!thbp]
%     \centering
%     \includegraphics[width=1.0\textwidth]{images/tsne_plot_with_scores_1362.pdf}
%     \caption{t-SNE visualization of projected high-dimensional response embeddings into a 2D space, illustrating the separation of actively selected responses. (a) AMPO-BottomK (baseline). (b) AMPO-Coreset (ours). (c) Opt-Select (ours). We see that the traditional baselines select many responses close to each other, based on their rating. This provides insufficient feedback to the LLM during preference optimization. In contrast, our methods simultaneously optimize for objectives including coverage, generation probability as well as preference rating.}
% \label{fig:tsne_diverse_analysis}
% \end{figure*}

% \begin{figure*}[!thbp]
%     \centering
%     \includegraphics[width=1.0\textwidth]{images/tsne_plot_with_scores_24192.pdf}

%     \caption{t-SNE visualization of projected high-dimensional response embeddings into a 2D space, illustrating the separation of actively selected responses. (a) AMPO-BottomK (baseline). (b) AMPO-Coreset (ours). (c) Opt-Select (ours). We see that the traditional baselines select many responses close to each other, based on their rating. This provides insufficient feedback to the LLM during preference optimization. In contrast, our methods simultaneously optimize for objectives including coverage, generation probability as well as preference rating.}
% \label{fig:tsne_diverse_analysis}
% \end{figure*}


% \begin{figure*}[!thbp]
%     \centering
%     \begin{subfigure}[b]{1.0\textwidth}
%         \centering
%         \includegraphics[width=0.8\textwidth]{images/tsne_plot_with_scores_23687.pdf}
%         \caption{Query-1.}
%         \label{fig:tsne_bottomk}
%     \end{subfigure}
    
%     \begin{subfigure}[b]{1.0\textwidth}
%         \centering
%         \includegraphics[width=0.8\textwidth]{images/tsne_plot_with_scores_1362.pdf}
%         \caption{Query1.}
%         \label{fig:tsne_coreset}
%     \end{subfigure}
    
%     \begin{subfigure}[b]{1.0\textwidth}
%         \centering
%         \includegraphics[width=0.8\textwidth]{images/tsne_plot_with_scores_24192.pdf}
%         \caption{Query-3.}
%         \label{fig:tsne_optselect}
%     \end{subfigure}
    
%     \caption{t-SNE visualization of projected high-dimensional response embeddings into a 2D space, illustrating the separation of actively selected responses. (a) AMPO-BottomK (baseline). (b) AMPO-Coreset (ours). (c) Opt-Select (ours). Traditional baselines select many responses close to each other based on their rating, providing insufficient feedback to the LLM during preference optimization. In contrast, our methods optimize for objectives including coverage, generation probability, and preference rating.}
%     \label{fig:tsne_combined}
% \end{figure*}


\begin{figure*}[!thbp]
    \centering
    \subfigure[1.]{
        \includegraphics[width=1.0\textwidth]{images/tsne_plot_with_scores_23687.pdf}
        \label{fig:tsne_bottomk}
    }
    
    \subfigure[2.]{
        \includegraphics[width=1.0\textwidth]{images/tsne_plot_with_scores_1362.pdf}
        \label{fig:tsne_coreset}
    }
    
    \subfigure[3.]{
        \includegraphics[width=1.0\textwidth]{images/tsne_plot_with_scores_24192.pdf}
        \label{fig:tsne_optselect}
    }
    
    \caption{t-SNE visualization of projected high-dimensional response embeddings into a 2D space, illustrating the separation of actively selected responses. (a) AMPO-BottomK (baseline). (b) AMPO-Coreset (ours). (c) Opt-Select (ours). Traditional baselines select many responses close to each other based on their rating, providing insufficient feedback to the LLM during preference optimization. In contrast, our methods optimize for objectives including coverage, generation probability, and preference rating.}
    \label{fig:tsne_combined}
\end{figure*}
