%%%%%%%% ICML 2025 EXAMPLE LATEX SUBMISSION FILE %%%%%%%%%%%%%%%%%

\documentclass{article}

% Recommended, but optional, packages for figures and better typesetting:
\usepackage{microtype}
\usepackage{graphicx}
\usepackage{subfigure}

\usepackage{booktabs} % for professional tables
% hyperref makes hyperlinks in the resulting PDF.
% If your build breaks (sometimes temporarily if a hyperlink spans a page)
% please comment out the following usepackage line and replace
% \usepackage{icml2025} with \usepackage[nohyperref]{icml2025} above.
\usepackage{hyperref}


% Attempt to make hyperref and algorithmic work together better:
\newcommand{\theHalgorithm}{\arabic{algorithm}}

% Use the following line for the initial blind version submitted for review:
% \usepackage{icml2025}

% If accepted, instead use the following line for the camera-ready submission:
\usepackage[accepted]{icml2025}

% For theorems and such
\usepackage{amsmath}
\usepackage{amssymb}
\usepackage{mathtools}
\usepackage{amsthm}
% if you use cleveref..
\usepackage[capitalize,noabbrev]{cleveref}

%%%%%%%%%%%%%%%%%%%%%%%%%%%%%%%%
% THEOREMS
%%%%%%%%%%%%%%%%%%%%%%%%%%%%%%%%
\theoremstyle{plain}
\newtheorem{theorem}{Theorem}[section]
\newtheorem{proposition}[theorem]{Proposition}
\newtheorem{lemma}[theorem]{Lemma}
\newtheorem{corollary}[theorem]{Corollary}
\theoremstyle{definition}
\newtheorem{definition}[theorem]{Definition}
\newtheorem{assumption}[theorem]{Assumption}
\theoremstyle{remark}
\newtheorem{remark}[theorem]{Remark}

% Todonotes is useful during development; simply uncomment the next line
%    and comment out the line below the next line to turn off comments
% \usepackage[disable,textsize=tiny]{todonotes}
% \usepackage[textsize=tiny]{todonotes}

%%%%%%%%%%%%%%%%%%%%%%%%%%%%%%%%
% USER DEFINED
%%%%%%%%%%%%%%%%%%%%%%%%%%%%%%%%

%%%%% NEW MATH DEFINITIONS %%%%%

% \usepackage{amsmath,amsfonts,bm}
\usepackage{amsmath,amsfonts}

\usepackage{pifont}


\newcommand{\R}{\mathbb{R}}


\def\va{{\mathbf{a}}}
\def\vg{{\mathbf{g}}}

% Sets
\def\sR{\mathbb{R}}
\def\sC{\mathbb{C}}
\def\sZ{\mathbb{Z}}
\def\sN{\mathbb{N}}
\def\sQ{\mathbb{Q}}

\def\sS{\mathcal{S}}



% Vectors
\def\vzero{{\mathbf{0}}}
\def\vone{{\mathbf{1}}}
\def\vmu{{\mathbf{\mu}}}
\def\vtheta{{\mathbf{\theta}}}
\def\va{{\mathbf{a}}}
\def\vb{{\mathbf{b}}}
\def\vc{{\mathbf{c}}}
\def\vd{{\mathbf{d}}}
\def\ve{{\mathbf{e}}}
\def\vf{{\mathbf{f}}}
\def\vg{{\mathbf{g}}}
\def\vh{{\mathbf{h}}}
\def\vi{{\mathbf{i}}}
\def\vj{{\mathbf{j}}}
\def\vk{{\mathbf{k}}}
\def\vl{{\mathbf{l}}}
\def\vm{{\mathbf{m}}}
\def\vn{{\mathbf{n}}}
\def\vo{{\mathbf{o}}}
\def\vp{{\mathbf{p}}}
\def\vq{{\mathbf{q}}}
\def\vr{{\mathbf{r}}}
\def\vs{{\mathbf{s}}}
\def\vt{{\mathbf{t}}}
\def\vu{{\mathbf{u}}}
\def\vv{{\mathbf{v}}}
\def\vw{{\mathbf{w}}}
\def\vx{{\mathbf{x}}}
\def\vy{{\mathbf{y}}}
\def\vz{{\mathbf{z}}}
\def\vzeta{{\mathbf{\zeta}}}

% Matrix
\def\mA{{\mathbf{A}}}
\def\mB{{\mathbf{B}}}
\def\mC{{\mathbf{C}}}
\def\mD{{\mathbf{D}}}
\def\mE{{\mathbf{E}}}
\def\mF{{\mathbf{F}}}
\def\mG{{\mathbf{G}}}
\def\mH{{\mathbf{H}}}
\def\mI{{\mathbf{I}}}
\def\mJ{{\mathbf{J}}}
\def\mK{{\mathbf{K}}}
\def\mL{{\mathbf{L}}}
\def\mM{{\mathbf{M}}}
\def\mN{{\mathbf{N}}}
\def\mO{{\mathbf{O}}}
\def\mP{{\mathbf{P}}}
\def\mQ{{\mathbf{Q}}}
\def\mR{{\mathbf{R}}}
\def\mS{{\mathbf{S}}}
\def\mT{{\mathbf{T}}}
\def\mU{{\mathbf{U}}}
\def\mV{{\mathbf{V}}}
\def\mW{{\mathbf{W}}}
\def\mX{{\mathbf{X}}}
\def\mY{{\mathbf{Y}}}
\def\mZ{{\mathbf{Z}}}
\def\mBeta{{\mathbf{\beta}}}
\def\mPhi{{\mathbf{\Phi}}}
\def\mLambda{{\mathbf{\Lambda}}}
\def\mSigma{{\mathbf{\Sigma}}}


% Expectation
% \def\eE{\mathop{\mathbb{E}}\limits}
\def\eE{\mathbb{E}}

% Probability
\def\pP{\mathbb{P}}

% Tilde
\def\tf{\tilde{f}}
\def\tS{\tilde{S}}
\def\wtF{\widetilde{\mathcal{F}}}
\def\whR{\widehat{R}}
\def\tvx{\tilde{\mathbf{x}}}
\def\ty{\tilde{y}}


\def\defeq{\overset{\textup{def}}{=}}
% \def\defeq{\overset{.}{=}}
\def\defone{\overset{\text{\ding{172}}}{=}}
\def\deftwo{\overset{\text{\ding{173}}}{=}}
\def\leqone{\overset{\text{\ding{172}}}{\leq}}
\def\leqtwo{\overset{\text{\ding{173}}}{\leq}}
\def\leqthree{\overset{\text{\ding{174}}}{\leq}}
\def\leqfour{\overset{\text{\ding{175}}}{\leq}}
\def\eqone{\overset{\text{\ding{172}}}{=}}
\def\eqtwo{\overset{\text{\ding{173}}}{=}}
\def\eqthree{\overset{\text{\ding{174}}}{=}}
\def\eqfour{\overset{\text{\ding{175}}}{=}}
\def\geqfive{\overset{\text{\ding{176}}}{\geq}}

%%% REVIEW
\newcommand{\tocite}{{\color{red}CITE} }
\newcommand{\toref}{{\color{red}REF} }

%%% LOGO
\newcommand{\usc}{\raisebox{-1pt}{\includegraphics[height=0.8em]{figures/usc_logo.png}}}
\newcommand{\vuam}{\raisebox{-1pt}{\includegraphics[height=0.8em]{figures/vu_logo.png}}}

%%% SIGNS and SYMBOLS
\newcommand{\grad}{\texttt{grad-CROP}}
\newcommand{\att}{\texttt{att-CROP}}
\newcommand{\seg}{\texttt{seg}}
\newcommand{\clip}{\texttt{clip-CROP}}
\newcommand{\sam}{\texttt{sam-CROP}}
\newcommand{\yolo}{\texttt{yolo-CROP}}
\newcommand{\hc}{\texttt{human-CROP}}
\newcommand{\zsvqa}{\texttt{ZSVQA}}
\newcommand{\vic}{\textbf{ViCrop}}
\newcommand{\xmark}{\text{\ding{55}}}
\newcommand{\cmark}{\text{\ding{51}}}
\newcommand{\success}{\texttt{\color{green} \cmark}}
\newcommand{\failure}{\texttt{\color{red} \xmark}}
\newcommand{\rel}{\texttt{rel-att}}
\newcommand{\gra}{\texttt{grad-att}}
\newcommand{\pgra}{\texttt{pure-grad}}
\newcommand{\relh}{\texttt{rel-att$^h$}}
\newcommand{\grah}{\texttt{grad-att$^h$}}
\newcommand{\pgrah}{\texttt{pure-grad$^h$}}


%%% Text Abb.
\makeatletter
\DeclareRobustCommand\onedot{\futurelet\@let@token\@onedot}
\def\@onedot{\ifx\@let@token.\else.\null\fi\xspace}

\def\aka{\emph{a.k.a}\onedot} \def\Eg{\emph{E.g}\onedot}
\def\eg{\emph{e.g}\onedot} \def\Eg{\emph{E.g}\onedot}
\def\ie{\emph{i.e}\onedot} \def\Ie{\emph{I.e}\onedot}
\def\cf{\emph{c.f}\onedot} \def\Cf{\emph{C.f}\onedot}
\def\etc{\emph{etc}\onedot} \def\vs{\emph{vs}\onedot}
\def\wrt{w.r.t\onedot} \def\dof{d.o.f\onedot}
\def\etal{\emph{et al}\onedot}
\makeatletter



\definecolor{myred}{HTML}{FF8577}
\definecolor{mygreen}{HTML}{0FA958}
\definecolor{myblue}{HTML}{1982C4}
\definecolor{codegreen}{rgb}{0,0.5,0}
\definecolor{codegray}{rgb}{0.5,0.5,0.5}
\definecolor{codepurple}{rgb}{0.07,0,0.53}
\definecolor{codered}{RGB}{189,41,0}
\definecolor{codecomment}{RGB}{153,153,153}
\definecolor{backcolour}{rgb}{0.96,0.96,0.96}
\definecolor{royalblue}{rgb}{0.0, 0.14, 0.4}
\definecolor{egyptianblue}{rgb}{0.06, 0.2, 0.65}
\definecolor{royalazure}{rgb}{0.0, 0.22, 0.66}
\definecolor{portlandorange}{rgb}{1.0, 0.35, 0.21}
\definecolor{sienna}{RGB}{183,105,68}
\definecolor{saddlebrown}{RGB}{139,69,19}
\definecolor{mediumbrown}{RGB}{83,41,11}
\definecolor{darkbrown}{RGB}{58,28,7}
\hypersetup{
    colorlinks=true,
    linkcolor=sienna,
    urlcolor=royalblue,
    citecolor=royalblue,
}



%%%%%%%%%%%%%%%%%%%%%%%%%%%%%%%%
% END OF USER DEFINED
%%%%%%%%%%%%%%%%%%%%%%%%%%%%%%%%

% The \icmltitle you define below is probably too long as a header.
% Therefore, a short form for the running title is supplied here:
\icmltitlerunning{\ampo: Active Multi Preference Optimization}

\begin{document}

\twocolumn[
% \icmltitle{\swepo: Simultaneous Weighted Preference Optimization\newline for Group Contrastive Alignment}
\icmltitle{\ampo: Active Multi Preference Optimization for Self-play Preference Selection}

% It is OKAY to include author information, even for blind
% submissions: the style file will automatically remove it for you
% unless you've provided the [accepted] option to the icml2025
% package.

% List of affiliations: The first argument should be a (short)
% identifier you will use later to specify author affiliations
% Academic affiliations should list Department, University, City, Region, Country
% Industry affiliations should list Company, City, Region, Country

% You can specify symbols, otherwise they are numbered in order.
% Ideally, you should not use this facility. Affiliations will be numbered
% in order of appearance and this is the preferred way.
\icmlsetsymbol{equal}{*}

\begin{icmlauthorlist}
\icmlauthor{Taneesh Gupta}{equal,comp}
\icmlauthor{Rahul Madhavan}{equal,yyy}
\icmlauthor{Xuchao Zhang}{comp}
\icmlauthor{Chetan Bansal}{comp}
\icmlauthor{Saravan Rajmohan}{comp}

%\icmlauthor{}{sch}
% \icmlauthor{Firstname8 Lastname8}{sch}
% \icmlauthor{Firstname8 Lastname8}{yyy,comp}
%\icmlauthor{}{sch}
%\icmlauthor{}{sch}
\end{icmlauthorlist}

\icmlaffiliation{yyy}{IISC}
\icmlaffiliation{comp}{Microsoft}
% \icmlaffiliation{sch}{School of ZZZ, Institute of WWW, Location, Country}

\icmlcorrespondingauthor{Taneesh Gupta}{t-taneegupta@microsoft.com}
\icmlcorrespondingauthor{Rahul Madhavan}{mrahul@iisc.com}

% % You may provide any keywords that you
% % find helpful for describing your paper; these are used to populate
% % the "keywords" metadata in the PDF but will not be shown in the document
% \icmlkeywords{Machine Learning, ICML}

% \vskip 0.3in

% \icmlsetsymbol{equal}{*}

% \begin{icmlauthorlist}
% \icmlauthor{Firstname1 Lastname1}{equal,yyy}
% \icmlauthor{Firstname2 Lastname2}{equal,yyy,comp}
% \icmlauthor{Firstname3 Lastname3}{comp}
% \icmlauthor{Firstname4 Lastname4}{sch}
% \icmlauthor{Firstname5 Lastname5}{yyy}
% \icmlauthor{Firstname6 Lastname6}{sch,yyy,comp}
% \icmlauthor{Firstname7 Lastname7}{comp}
% %\icmlauthor{}{sch}
% \icmlauthor{Firstname8 Lastname8}{sch}
% \icmlauthor{Firstname8 Lastname8}{yyy,comp}
% %\icmlauthor{}{sch}
% %\icmlauthor{}{sch}
% \end{icmlauthorlist}

% \icmlaffiliation{yyy}{Department of XXX, University of YYY, Location, Country}
% \icmlaffiliation{comp}{Company Name, Location, Country}
% \icmlaffiliation{sch}{School of ZZZ, Institute of WWW, Location, Country}

% \icmlcorrespondingauthor{Firstname1 Lastname1}{first1.last1@xxx.edu}
% \icmlcorrespondingauthor{Firstname2 Lastname2}{first2.last2@www.uk}

% You may provide any keywords that you
% find helpful for describing your paper; these are used to populate
% the "keywords" metadata in the PDF but will not be shown in the document
\icmlkeywords{Machine Learning, ICML}

\vskip 0.3in

]

% this must go after the closing bracket ] following \twocolumn[ ...

% This command actually creates the footnote in the first column
% listing the affiliations and the copyright notice.
% The command takes one argument, which is text to display at the start of the footnote.
% The \icmlEqualContribution command is standard text for equal contribution.
% Remove it (just {}) if you do not need this facility.

%\printAffiliationsAndNotice{}  % leave blank if no need to mention equal contribution
% \printAffiliationsAndNotice{\icmlEqualContribution} % otherwise use the standard text.

\printAffiliationsAndNotice{\icmlEqualContribution}

% \begin{abstract}
Retrieval-Augmented Generation (RAG) is often used with Large Language Models (LLMs) to infuse domain knowledge or user-specific information. In RAG, given a user query, a retriever extracts chunks of relevant text from a knowledge base. These chunks are sent to an LLM as part of the input prompt. Typically, any given chunk is repeatedly retrieved across user questions. However, currently, for every question, attention-layers in LLMs fully compute the key values (KVs) repeatedly for the input chunks, as state-of-the-art methods cannot reuse KV-caches when chunks appear at arbitrary locations with arbitrary contexts. Naive reuse leads to output quality degradation.  This leads to potentially redundant computations on expensive GPUs and increases latency. In this work, we propose \sys, a system for managing and reusing precomputed KVs corresponding to the text chunks (we call \textit{chunk-caches}) in RAG-based systems. We present how to identify \hl{\textit{chunk-caches} that are reusable}, how to efficiently perform a small fraction of recomputation to \textit{fix} the cache to maintain output quality, and how to efficiently store and evict \textit{chunk-caches} in the hardware for maximizing reuse while masking any overheads. With real production workloads as well as synthetic datasets, we show that \sys reduces redundant computation by \textbf{51\%} over SOTA prefix-caching and \textbf{75\%} over full recomputation.
\hl{Additionally, with continuous batching on a real production workload, we get a \textbf{1.6$\times$} speedup in throughput and a \textbf{2$\times$} reduction in end-to-end response latency over prefix-caching while maintaining quality, for both the \llama-3-8B and \llama-3-70B models. 
}
\end{abstract}






\begin{abstract}
Retrieval-Augmented Generation (RAG) is often used with Large Language Models (LLMs) to infuse domain knowledge or user-specific information. In RAG, given a user query, a retriever extracts chunks of relevant text from a knowledge base. These chunks are sent to an LLM as part of the input prompt. Typically, any given chunk is repeatedly retrieved across user questions. However, currently, for every question, attention-layers in LLMs fully compute the key values (KVs) repeatedly for the input chunks, as state-of-the-art methods cannot reuse KV-caches when chunks appear at arbitrary locations with arbitrary contexts. Naive reuse leads to output quality degradation.  This leads to potentially redundant computations on expensive GPUs and increases latency. In this work, we propose \sys, a system for managing and reusing precomputed KVs corresponding to the text chunks (we call \textit{chunk-caches}) in RAG-based systems. We present how to identify \hl{\textit{chunk-caches} that are reusable}, how to efficiently perform a small fraction of recomputation to \textit{fix} the cache to maintain output quality, and how to efficiently store and evict \textit{chunk-caches} in the hardware for maximizing reuse while masking any overheads. With real production workloads as well as synthetic datasets, we show that \sys reduces redundant computation by \textbf{51\%} over SOTA prefix-caching and \textbf{75\%} over full recomputation.
\hl{Additionally, with continuous batching on a real production workload, we get a \textbf{1.6$\times$} speedup in throughput and a \textbf{2$\times$} reduction in end-to-end response latency over prefix-caching while maintaining quality, for both the \llama-3-8B and \llama-3-70B models. 
}
\end{abstract}





\documentclass[../main.tex]{subfiles}
\graphicspath{{../images/}}
\makeatletter
\def\input@path{{../images/}}
\makeatother
\begin{document}
\section{Introduction}
\begin{figure}
\centering
\begin{tikzpicture}
\node[inner sep=0pt] (ws) at (0, 0) {
\includegraphics[height=.4\textwidth, trim={10cm 0 10cm 0},clip]{world_space.png}};
\node[inner sep=0pt] (cs) at (6,0) {\includegraphics[height=.4\textwidth, trim={10cm 1cm 10cm 4cm},clip]{conf_space.png}};
\end{tikzpicture}
\vspace{-5pt}
\label{fig:pbrm_intro}
\caption{\textbf{Left}: Shows world space obstacles as grey spheres. Robots start and goal configuration is colored red and green, respectively. Configurations along the computed path are colored transparent blue. \textbf{Right:} Mapped world space scenario to configuration space. Obstacle region is the grey mesh. Red spheres are collision-free regions computed by the neural SCDF. The optimized shortest path in the convex corridor is the blue curve.}
\vspace{-25pt}
\end{figure}
Motion planning is the problem of finding a collision-free trajectory that connects a given start and goal configuration. The planning takes place in the configuration space of the robot. For single body robots, like mobile robots or drones, the configuration space and the world space are usually the same. This simplifies the planning, since explicit obstacle representations are available which enables geometrical tools like separating hyperplanes, smallest distance to obstacles etc., to be used when designing motion planning algorithms. For multi-body robots like manipulators, the situation is completely different. The world space obstacles are usually mapped to non-convex regions, and to make the problem even harder, the mapping is usually not known. Forming explicit representations of the obstacle region in the configuration space is usually too expensive or intractable. Despite all of this, sampling based planners are used with great success, which mainly is due to their use of implicit representations of the obstacle region. The basic idea is to construct a graph in the configuration space that covers and connects the collision-free region. From this graph, a path can be extracted that connects a given start and goal configuration. The approach is computationally expensive, since the graph is constructed with the smallest geometrical building block available, points, which represents a collision-check. Furthermore, the extracted paths from the graph are non-smooth and jagged due to the stochastic nature of the approach. This adds an additional post-processing step to the process, where the paths are shortcutted and smoothened, before the path can be used for tracking. Clearly a lot of time is invested to form this graph and produce smooth paths. Thus, if the obstacles start to move, then all of this work is done in no use, since all points that make up this graph need to be re-verified, which is simply too time consuming to be done in real time.
\\\\
In this work, we want to address the existing drawbacks of the sampling based planners. Our main contribution is an improved motion planner where each vertex in the graph covers a collision-free region in the form of a sphere instead of a point and where the edges are formed with neighboring intersecting spheres. This representation has the advantage of instead of returning piecewise linear paths, returning a sequence of overlapping spheres, i.e. a convex corridor, that connects a given start and goal configuration, illustrated in Figure \ref{fig:pbrm_intro}. This convex corridor allows us to use convex optimization to produce smooth trajectories, instead of computationally expensive post-processing methods. The representation further allows us to estimate the coverage of the collision-free space, which gives us awareness and feedback in the offline roadmap construction phase. Finally, our representation is simple to adapt to moving obstacles, simply requery for the new radii and recheck for intersections. 
\\\\
The spherical collision-free regions are formed using a signed distance function (SDF), which is a function that returns the smallest distance from an arbitrary point to the boundary of an obstacle. As the name implies, the distance is signed, thus if the point is inside the obstacle it is negative otherwise positive. If the distance is positive, a sphere with radius equal to the distance is guaranteed to cover a collision-free region. Using an SDF in motion planning is not new, but what is novel about our approach is that we express the distance in the configuration space instead of the world space and by doing so allows us to form these convex collision-free regions. We refer to the resulting SDF as a signed configuration distance function (SCDF). Computing an SCDF analytically is non-trivial, our approach is therefore to parameterize the SCDF with a deep neural network and learn the mapping by supervised learning. Our resulting neural SCDF can compute distances for different parameter values of obstacle shapes and we also show how multiple distances can be combined, thus making our approach flexible.
\section{Related work}
Motion planning algorithms can roughly be divided into three families, grid-based, sampling based and optimization based methods. Grid-based methods (GBM) discretize the planning space from which a graph is then compiled. A standard search method is A$^\star$ \citep{a_star}, which is classified as an \textit{informed} search method, since it employs a heuristic function to speed up the search. A$^\star$ guarantees to return an optimal path at the level of discretization used. GBMs usually discretize the planning space by a regular lattice and this limits the GBMs to problems with low dimensionality due to the curse of dimensionality. Thus, GBMs are usually limited to single-body robots where the degrees of freedom (DOF) are low. To overcome the inherent scaling problem with the GBMs, stochastic methods are usually used for multi-body robots. These methods are termed as sampling-based methods (SBM) and core members within this family are the rapidly-exploring random trees (RRT) \citep{rrt} and the probabilistic roadmap (PRM) \citep{prm}. RRT grows a tree from the start configuration and explores the collision-free region in a rapid way until it is able to connect to the goal region. RRT is usually improved by bi-directional planning \citep{rrt_connect}, i.e. an additional tree is grown from the goal configuration and the trees are tested for connection after any tree has been expanded. RRT is a single-query method, thus it searches for a path from scratch each time it is queried. Contrary to this, PRM is a multi-query method, which solves for multiple queries without starting from scratch. PRM does this by creating a roadmap (graph) that covers the collision-free space as an offline step. The graph is then used to solve for multiple queries. PRMs are used in cases where the environment does not change since the extra offline step is too computationally costly and needs to be re-done if the environment is changed. In our work, we address this inherent issue by using a different roadmap representation. Our vertices in the graph cover a collision-free region in the form of spheres and we form the edges by checking for intersecting spheres. If something in the environment changes, we recompute the spheres radii and recheck the intersections, without relying on collision detection. We use a trained neural network to compute the sphere radius, therefore querying for the radius can be done fast, hence our representation enables the PRM for dynamic environments.
\\\\
In the recent decades, optimization based methods (OBM) \citep{chomp, schulman, itomp, stomp} have been introduced as an alternative to SBM for multi-body robots. Like the SBM, the OBMs scale well to higher dimensional problems and produce smoother motion. It is common to use a SDF in the optimization since it is a smooth function, thus enabling gradient-based methods. However, the standard way of expressing the SDF is in world space. The distance therefore needs to be mapped to the configuration space by the forward kinematics. This mapping makes the optimization problem a non-linear program (NLP), which is computationally expensive to solve. Recently, a different approach has been proposed. In \cite{mp_gcs} motion planning is formulated as a convex optimization problem by using the graph of convex sets framework \citep{gcs}. The underlying idea is to decompose the collision-free space into intersecting convex sets from which a convex optimization problem is formulated. In cases where an explicit representation of the obstacles in the configuration space exists, like for single-body robots, creating collision-free convex regions can be done fast \citep{iris}. For multi-body robots, this is non-trivial. Existing work does this successfully \citep{iris_nlp, iris_c} by an optimization based approach, but the methods are still too time consuming to be used in the presence of moving obstacles. Our approach is instead to use deep learning to learn an SDF expressed in the configuration space. With this, we can query for shortest distances to the collision boundary, which allows us to expand spherical regions which are collision-free. Our approach is fast and therefore enables our suggested roadmap planner to be used in dynamic environments.
\\\\
Recent research has focused on learning collision detection \citep{fk_kernel_distance, diffco, graphdistnet} by predicting the signed distance between the robot links and the surrounding obstacles in the world space. The learned SDF is used in trajectory optimization but since the distance is expressed in the world space, the problem becomes an NLP and therefore takes a long time to solve. We take a novel approach and suggest to instead express the signed distance in the configuration space. This allows us to improve the PRM at the same time as it enables convex optimization for trajectory optimization, which runs faster and is more reliable than NLP solvers. In \cite{cspf} a learned signed distance function in the configuration space is proposed similar to our approach. However, their approach is restricted to point cloud representations, while we propose to represent the obstacles as parameterized geometric shapes, e.g. spheres. Furthermore, we also show how to use our learned SCDF to improve an existing roadmap planner.
\section{Problem formulation}
A robot is located in the world space, $\W \subset \R^3 $. The unique location of the robot is given by its configuration $\q \in \C$, where $\C$ is the configuration space. The set of points covered by the robots bodies at a certain configuration is expressed as $\B(\q) \subset \W$. The robot is surrounded by $\NrObst$ obstacles $\O = \bigcup_{i=1}^{\NrObst} \O_i$, where  $\O_i \subset \W$. The representation of the obstacle in the configuration space is the set $\C\O_i = \{\q \in \C \: |\: \B(\q) \cap \O_i \neq \emptyset \}$. The obstacle space is formed as $\Co = \bigcup_{i=1}^{\NrObst} \C \O_i$. The complement is referred to as the free space, $\Cf = \C \setminus \Co$. The path planning problem is a tuple, ($\Cf$, $\qStart$, $\qGoal$), where we want to connect a query pair, consisting of a start, $\qStart$, and goal configuration, $\qGoal$, with a geometric path, $\q(s): [0, 1] \mapsto \Cf$, such that $\q(0)=\qStart$ and $\q(1)=\qGoal$, or report correctly when such a path does not exist.
\end{document}

\subsection{Our Contributions}
\begin{itemize}[leftmargin=1em]
    \item \textbf{Algorithmic Novelty:} We propose \emph{Active Multi-Preference Optimization} (\ampo), an on-policy framework that blends group-based preference alignment with active subset selection without exhaustively training on all generated responses. This opens out avenues for research on how to select for synthetic data, as we outline in Sections \ref{sec:subset_selection_strategies} and \ref{sec:discussion_future_work}.
    \item \textbf{Theoretical Insights:} Under mild Lipschitz assumptions, we show that coverage-based negative selection can systematically suppress low-reward modes and maximizes expected reward. This analysis (in Sections \ref{sec:opt_select} and \ref{sec:theory_main}) connects our method to the weighted $k$-medoids problem, yielding performance guarantees for alignment.
    \item \textbf{State-of-the-Art Results:} Empirically, \ampo\ sets a new benchmark on \textit{AlpacaEval} with Llama 8B, surpassing strong baselines like $\simpo$ by focusing on a small but strategically chosen set of responses each iteration (see Section \ref{sec:experimental setup}).
    \item \textbf{Dataset Releases:} We publicly release our \href{https://huggingface.co/datasets/Multi-preference-Optimization/AMPO-Coreset-selection}{\texttt{AMPO-Coreset-Selection}} 
    and \href{https://huggingface.co/datasets/Multi-preference-Optimization/AMPO-OPT-Selection}{\texttt{AMPO-Opt-Selection}} datasets on Hugging Face. These contain curated response subsets for each prompt, facilitating research on multi-preference alignment.
\end{itemize}



\vspace{-0.1in}
\section{Notations and Preliminaries}
\label{sec:notations_preliminaries}

\vspace{-0.1in}
We focus on aligning a \emph{policy model} to human preferences in a single-round (one-shot) scenario. Our goal is to generate multiple candidate responses for each prompt, then actively select a small, high-impact subset for alignment via a group-contrastive objective.

\vspace{-0.1in}
\paragraph{Queries and Policy.}
Let $\mathcal{D} = \{x_1, x_2, \ldots, x_M\}$ be a dataset of $M$ \emph{queries} (or \emph{prompts}), each from a larger space $\mathcal{X}$. We have a policy model $P_\theta(y \mid x)$, parameterized by $\theta$, which produces a distribution over possible responses $y \in \mathcal{Y}$. To generate diverse answers, we sample from $P_\theta(y \mid x)$ at some fixed \emph{temperature} (e.g., $0.8$). Formally, for each $x_i$, we draw up to $N$ responses,

\vspace{-0.1in}
\begin{equation}
   \{y_{i,1}, y_{i,2}, \dots, y_{i,N}\}, 
\end{equation}

\vspace{-0.1in}
from $P_\theta(y \mid x_i)$. Such an \textbf{on-policy} sampling, ensures, we are able to provide preference feedback on queries that are chosen by the model.

\vspace{-0.1in}
For simplicity of notation, we shall presently consider a single query (prompt) \(x\) and sampled responses \(\{y_1,\dots,y_N\}\) from \(P_\theta(\cdot \mid x)\), from the autoregressive language model.

\vspace{-0.1in}
Each response \(y_i\) is assigned a scalar reward

\vspace{-0.1in}
\begin{equation}
r_i \;=\; \mathcal{R}(x,\,y_i) \;\in\; [0,1],
\end{equation}

\vspace{-0.1in}
where \(\mathcal{R}\) is a fixed reward function or model (not optimized during policy training). We also embed each response via \(\mathbf{e}_i = \mathcal{E}(y_i)\in \mathbb{R}^d\), where \(\mathcal{E}\) might be any sentence or document encoder capturing semantic or stylistic properties.

\vspace{-0.1in}
Although one could train on all \(N\) responses, doing so is often computationally expensive. We therefore \emph{select} a subset \(\mathcal{S}\subset \{1,\dots,N\}\) of size \(\lvert\mathcal{S}\rvert = K < N\) by maximizing some selection criterion (e.g.\ favoring high rewards, broad coverage in embedding space, or both). Formally,

\vspace{-0.15in}
\begin{equation}
\label{eq:subset_selection}
\mathcal{S}
\;=\;
\arg\max_{\substack{\mathcal{I}\subset\{1,\dots,N\} \\ \lvert\mathcal{I}\rvert = K}}
\,\mathcal{U}\Bigl(\{y_i\}_{i\in\mathcal{I}},\, \{r_i\}_{i\in\mathcal{I}},\, \{\mathbf{e}_i\}_{i\in\mathcal{I}}\Bigr),
\end{equation}

\vspace{-0.1in}
where \(\mathcal{U}\) is a \emph{utility function} tailored to emphasize extremes, diversity, or other alignment needs.

\vspace{-0.1in}
Next, we split \(\mathcal{S}\) into a \emph{positive} set \(\mathcal{S}^+\) and a \emph{negative} set \(\mathcal{S}^-\). For example, let 

\vspace{-0.05in}
\[
\overline{r} \;=\;
\frac{1}{K}\,\sum_{i\in \mathcal{S}}\,r_i
\]
\vspace{-0.05in}
be the average reward of the chosen subset, and define

\vspace{-0.05in}
\[
\mathcal{S}^+
\;=\;
\{\,i\in \mathcal{S}\,\mid\,r_i > \overline{r}\},
\quad
\mathcal{S}^- 
\;=\;
\{\,i\in \mathcal{S}\,\mid\,r_i \le \overline{r}\}.
\]

\vspace{-0.05in}
Hence, \(\mathcal{S} = \mathcal{S}^+\cup \mathcal{S}^-\) and \(\lvert \mathcal{S}^+\rvert + \lvert \mathcal{S}^-\rvert = K\).

We train \(\theta\) via a reference-free \emph{group-contrastive} objective known as \(\textsc{refa}\) \citep{gupta2024refa}. Concretely, define

\vspace{-0.15in}
\begin{equation}
L_{\text{swepo}}(\theta)
\;=\;
-\,\log\!\Biggl(\!
  \frac{
    \displaystyle
    \sum_{\,i \,\in\, \mathcal{S}^+}\;
    \exp\Bigl[
      s'_\theta\bigl(y_i \mid x\bigr)
    \Bigr]
  }{
    \displaystyle
    \sum_{\,i\,\in\,(\mathcal{S}^+\cup \mathcal{S}^-)}\;
    \exp\Bigl[
      s'_\theta\bigl(y_i \mid x\bigr)
    \Bigr]
  }
\Biggr),
\end{equation}

\vspace{-0.05in}
where

\vspace{-0.25in}
\[
s'_\theta\bigl(y_i \mid x\bigr)
  =
  \log P_\theta(y_i\mid x)
  +
  \alpha \,\bigl(r_i - \overline{r}\bigr).
\]

\vspace{-0.05in}
Here, \(P_{\mathrm{ref}}\) is a reference policy (e.g.\ an older snapshot of \(P_\theta\) or a baseline model), and \(\alpha\) is a hyperparameter scaling the reward difference. In words, \(\textsc{swepo}\) encourages the model to increase the log-probability of \(\mathcal{S}^+\) while decreasing that of \(\mathcal{S}^-\), all in a single contrastive term that accounts for multiple positives and negatives simultaneously.

Although presented for a single query \(x\), this procedure extends straightforwardly to any dataset \(\mathcal{D}\) by summing \(L_{\text{swepo}}\) across all queries. In subsequent sections, we discuss diverse strategies for selecting \(\mathcal{S}\) (and thus \(\mathcal{S}^+\) and \(\mathcal{S}^-\)), aiming to maximize training efficiency and alignment quality.




\section{Algorithm and Methodology}
\label{sec:methodology}

We outline a one-vs-$k$ selection scheme in which a single \emph{best} response is promoted (positive), while an \emph{active} subroutine selects $k$ negatives from the remaining $N-1$ candidates. This setup highlights the interplay of three main objectives:

\vspace{-0.1in}
\begin{description}[leftmargin=1em, itemsep=0pt]
   \item[Probability:] High-probability responses under $P_\theta(y\mid x)$ can dominate even if suboptimal by reward.
   \item[Rewards:] Simply selecting extremes by reward misses problematic "mediocre" outputs.
   \item[Semantics:] Diverse but undesired responses in distant embedding regions must be penalized.
\end{description}

\vspace{-0.15in}
While positives reinforce a single high-reward candidate, active negative selection balances probability, reward and diversity to systematically suppress problematic regions of the response space.

% \begin{description}[leftmargin=1em, itemsep=1pt]
%     \item[Probability (Model Likelihood):] Responses that have high probability under $P_\theta(y\mid x)$ can dominate the model’s output distribution; even if such responses are not the absolute worst by reward, leaving them unpenalized can hamper alignment.
%     \item[Rewards (Quality):] Purely selecting the top- or bottom-ranked by reward is insufficient when the model generates a variety of “mediocre” or “niche” outputs that still require demotion.
%     \item[Semantics (Diversity/Coverage):] Some undesired responses may differ substantially from the mainstream modes (e.g., lying in a distant embedding region). Penalizing them is vital, else the model remains vulnerable to sporadic but harmful outputs.
% \end{description}

% Balancing these three factors (model likelihood, reward, and diversity) is the cornerstone of \emph{active} negative selection. While the \emph{positive} ensures strong reinforcement of one high-reward candidate, the \emph{negatives} promote a broader shaping of the policy to down-weight potentially problematic regions. 

% \medskip
\noindent
\textbf{Algorithm.}
Formally, let $\{y_1,\dots,y_N\}$ be the sampled responses for a single prompt $x$. Suppose we have:\\
1. A reward function $r_i = \mathcal{R}(x,y_i)\in [0,1]$.\\
2. An embedding $\mathbf{e}_i = \mathcal{E}(y_i)$.\\
3. A model probability estimate $\pi_i = P_\theta(y_i\mid x)$.
% \begin{enumerate}
% \item 
% \item .
% \item 
% \end{enumerate}

Selection algorithms may be \textit{rating-based} selection (to identify truly poor or excellent answers) with \textit{coverage-based} selection (to explore distinct regions in the embedding space), we expose the model to both common and outlier responses. This ensures that the \textsc{swepo} loss provides strong gradient signals across the spectrum of answers the model is prone to generating. In Algorithm \ref{alg:one_vs_k_active}, $\textsc{ActiveSelection}(\cdot)$ is a generic subroutine that selects a set of $k$ “high-impact” negatives. We will detail concrete implementations (e.g.\ bottom-$k$ by rating, clustering-based, etc.) in later sections.



\begin{algorithm}[t]
\caption{\textcolor{titlecolor}{\textbf{$\ampo$: One-Positive vs.\ $k$-Active Negatives}}}
\label{alg:one_vs_k_active}
\begin{algorithmic}[1]
    \STATE \textcolor{inputcolor}{\textbf{Input:} (1) A set of $N$ responses $\{y_i\}$ sampled from $P_{\theta}(y\mid x)$; (2) Their rewards $\{r_i\}$, embeddings $\{\mathbf{e}_i\}$, and probabilities $\{\pi_i\}$; (3) Number of negatives $k$, reference policy $P_{\mathrm{ref}}$, and hyperparameter $\alpha$}
    \STATE \textcolor{outputcolor}{\textbf{Output:} (i) Positive $y_{+}$; (ii) Negatives $\{y_j\}_{j \in S^-}$; (iii) Updated parameters $\theta$ via \textsc{swepo}}
    \STATE \textcolor{stepcolor}{\textit{1. Select One Positive (Highest Reward)}}
    \STATE \textcolor{mathcolor}{$i_{+} \leftarrow \arg\max_{i=1,\dots,N} r_i$, \quad $y_{+} \leftarrow y_{\,i_{+}}$}
    \STATE \textcolor{stepcolor}{\textit{2. Choose $k$ Negatives via Active Selection}}
    \STATE \textcolor{mathcolor}{$\Omega \leftarrow \{1,\dots,N\}\setminus\{i_{+}\}$}
    \STATE \textcolor{mathcolor}{$S^- \leftarrow \textsc{ActiveSelection}(\Omega,\{r_i\},\{\mathbf{e}_i\},\{\pi_i\},k)$}
    \STATE \textcolor{stepcolor}{\textit{3. Form One-vs.-$k$ \textsc{swepo} Objective}}
    \STATE \textcolor{mathcolor}{$\overline{r} \leftarrow \frac{r_{\,i_{+}} + \sum_{j\,\in\,S^-} r_j}{1 + k}$}
    \STATE For each $y_i$:
    \STATE \textcolor{mathcolor}{$s'_\theta(y_i) = \log P_\theta(y_i \mid x) + \alpha(r_i - \overline{r})$}
    \STATE \textcolor{mathcolor}{$L_{\text{swepo}}(\theta) = -\log\!\Biggl(\frac{\exp\!\bigl[s'_\theta(y_{+})\bigr]}{\exp\!\bigl[s'_\theta(y_{+})\bigr] + \sum_{\,j \,\in\, S^-}\exp\!\bigl[s'_\theta(y_j)\bigr]}\Biggr)$}
    \STATE \textcolor{stepcolor}{\textit{4. Update Model Parameters:}} \textcolor{mathcolor}{$\theta \leftarrow \theta - \eta\,\nabla_\theta L_{\text{swepo}}(\theta)$}
    \RETURN The chosen positive $y_{+}$, the negative set $\{y_j\}_{j \in S^-}$, and the updated parameters $\theta$
\end{algorithmic}
\end{algorithm}

% \begin{algorithm}[t]
% \caption{\textbf{One-Positive vs.\ $k$-Active Negatives for group contrastive alignment via $\swepo$ loss}}
% \label{alg:one_vs_k_active}
% \begin{algorithmic}[1]
%     \STATE {\bfseries Input:} (1) A set of $N$ responses $\{y_i\}$ sampled from $P_{\theta}(y\mid x)$; (2) Their rewards $\{r_i\}$, embeddings $\{\mathbf{e}_i\}$, and probabilities $\{\pi_i\}$; (3) Number of negatives $k$, reference policy $P_{\mathrm{ref}}$, and hyperparameter $\alpha$
%     \STATE {\bfseries Output:} (i) Positive $y_{+}$; (ii) Negatives $\{y_j\}_{j \in S^-}$; (iii) Updated parameters $\theta$ via \textsc{swepo}
%     \STATE {\bfseries 1. Select One Positive (Highest Reward)}\\
%     $i_{+} \leftarrow \arg\max_{i=1,\dots,N} r_i$, \quad $y_{+} \leftarrow y_{\,i_{+}}$
%     \STATE {\bfseries 2. Choose $k$ Negatives via Active Selection}\\
%     $\Omega \leftarrow \{1,\dots,N\}\setminus\{i_{+}\}$\\ $S^- \leftarrow \textsc{ActiveSelection}(\Omega,\{r_i\},\{\mathbf{e}_i\},\{\pi_i\},k)$
%     \STATE {\bfseries 3. Form One-vs.-$k$ \textsc{swepo} Objective}\\
%     $\overline{r} \leftarrow \frac{r_{\,i_{+}} + \sum_{j\,\in\,S^-} r_j}{1 + k}$
%     \STATE For each $y_i$:\\ $s'_\theta(y_i) = \log P_\theta(y_i \mid x) - \log P_{\mathrm{ref}}(y_i \mid x) + \alpha(r_i - \overline{r})$
%     \STATE $L_{\text{swepo}}(\theta) = -\log\!\Biggl(\frac{\exp\!\bigl[s'_\theta(y_{+})\bigr]}{\exp\!\bigl[s'_\theta(y_{+})\bigr] + \sum_{\,j \,\in\, S^-}\exp\!\bigl[s'_\theta(y_j)\bigr]}\Biggr)$
%     \STATE {\bfseries 4. Update Model Parameters:} $\theta \leftarrow \theta - \eta\,\nabla_\theta L_{\text{swepo}}(\theta)$
%     \RETURN The chosen positive $y_{+}$, the negative set $\{y_j\}_{j \in S^-}$, and the updated parameters $\theta$
% \end{algorithmic}
% \end{algorithm}


\vspace{-0.1in}
\noindent
\subsection{Detailed Discussion of Algorithm \ref{alg:one_vs_k_active}}

\vspace{-0.1in}
The algorithm operates in four key steps: First, it selects the highest-reward response as the positive example (lines 3-4). Second, it actively selects $k$ negative examples by considering their rewards, probabilities $\pi_i$, and embedding distances $\mathbf{e}_i$ to capture diverse failure modes (lines 5-7). Third, it constructs the \textsc{swepo} objective by computing normalized scores $s'_\theta$ using the mean reward $\overline{r}$ and forming a one-vs-$k$ contrastive loss (lines 8-12). Finally, it updates the model parameters to increase the probability of the positive while suppressing the selected negatives (line 13). This approach ensures both reinforcement of high-quality responses and systematic penalization of problematic outputs across the response distribution.

% Step 1 ensures that at least one high-quality response is explicitly reinforced. Step 2 removes this positive from the candidate pool, then invokes an \emph{active} subroutine that must simultaneously account for reward (so that sufficiently poor answers are penalized), probability (so that common but flawed responses do not slip through), and diversity (so that rare but semantically distinct mistakes are also exposed). In Step 3, the one-vs.-$k$ \textsc{swepo} loss encourages the model to push up the log-probability of the positive while pushing down that of the $k$ negatives. Finally, Step 4 updates the parameters based on this contrast. 

% By highlighting a single “best” response and a carefully chosen set of “bad” or “undesired” responses, the algorithm provides a strong training signal across the reward distribution. Unlike naive strategies that choose negatives purely by lowest reward, our active-selection perspective seeks to uncover distinct semantic modes (via $\mathbf{e}_i$) or infrequent but still plausible outputs (via $\pi_i$). In this way, the model is given contrasting examples that refine its probability distribution to \emph{both} improve high-reward regions and suppress a diverse set of negative modes. 
% By coupling a single, clearly good response with an actively selected set of negative answers, the approach sculpts the model’s distribution towards several desirable outcomes in a single optimization step.



\vspace{-0.15in}
\section{Active Subset Selection Strategies}
\label{sec:subset_selection_strategies}

\vspace{-0.1in}
In this section, we present two straightforward yet effective strategies for actively selecting a small set of \emph{negative} responses in the \textsc{AMPO} framework. First, we describe a simple strategy, \emph{\textbf{AMPO-BottomK}}, that directly picks the lowest-rated responses. Second, we propose \emph{\textbf{AMPO-Coreset}}, a clustering-based method that selects exactly one negative from each cluster in the embedding space, thereby achieving broad coverage of semantically distinct regions. In Section \ref{sec:constant_factor_subset_selection}, we connect this clustering-based approach to the broader literature on \emph{coreset construction}, which deals with selecting representative subsets of data.

\vspace{-0.1in}
\subsection{AMPO-BottomK}
\label{sec:ampo_bottomk}

\vspace{-0.05in}
\noindent
\emph{AMPO-BottomK} is the most direct approach that we use for comparison: given $N$ sampled responses and their scalar ratings $\{r_i\}_{i=1}^N$, we simply pick the $k$ lowest-rated responses as negatives. This can be expressed as:

\vspace{-0.25in}
\begin{align}
\label{eq:bottomk_negatives}
S^- \;=\; \mathrm{argtopk}_{i}(-\,r_i,\,k),
\end{align}

\vspace{-0.1in}
which identifies the $k$ indices with smallest $r_i$. Although conceptually simple, this method can be quite effective when the reward function reliably indicates “bad” behavior. 
Furthermore to break-ties, we use minimal cosine similarity with the currently selected set.


% \vspace{0.5em}

% \begin{algorithm}[tb]
% \caption{$\ampobk$($\{r_i\}_{i=1}^N$, $k$)}
% \label{alg:bottom_k_negatives}
% \begin{algorithmic}[1]
%     \STATE {\bfseries Input:} 
%     \STATE (1) A set of $N$ responses with associated scalar ratings $\{r_i\}_{i=1}^N$
%     \STATE (2) Desired number of negatives $k$
%     \STATE 
%     \STATE {\bfseries Step 1:} Rank by rating
%     \STATE Sort indices $\{1,\dots,N\}$ so that $r_{(1)} \,\le\, r_{(2)} \,\le\, \dots \,\le\, r_{(N)}$,
%     \STATE where $r_{(1)}$ is the lowest rating
%     \STATE
%     \STATE {\bfseries Step 2:} Select bottom $k$
%     \STATE $S^- \;\leftarrow\; \{ (1),\, (2),\,\dots,\,(k)\}$
%     \STATE
%     \RETURN the $k$ indices in $S^-$ as the set of negatives
% \end{algorithmic}
% \end{algorithm}

\vspace{-0.1in}
\subsection{AMPO-Coreset (Clustering-Based Selection)}
\label{sec:ampo_coreset}

\vspace{-0.05in}
\noindent
$\ampobk$ may overlook problematic modes that are slightly better than the bottom-k, but fairly important to learn on. A diversity-driven approach, which we refer to as $\ampocs$, explicitly seeks coverage in the embedding space by partitioning the $N$ candidate responses into $k$ clusters and then selecting the lowest-rated response within each cluster. Formally:

\vspace{-0.15in}
\[
\label{eq:clustering_negatives}
i^-_j 
\;=\;
\arg\min_{\,i \,\in\,C_j}\; r_i, 
\,
j = 1,\dots,k, 
\,
S^- \;=\;\bigl\{\,i^-_1,\dots,i^-_k\bigr\}
\]

\vspace{-0.15in}
where $C_j$ is the set of responses assigned to cluster $j$ by a $k$-means algorithm (\citealt{har2004coresets,cohen2022improved}; see also Section \ref{sec:constant_factor_subset_selection}). The pseudo-code is provided in Algorithm \ref{alg:cluster_negatives}.


\begin{algorithm}[tb]
\caption{\textcolor{titlecolor}{$\ampocs$ via k-means}}
\label{alg:cluster_negatives}
\begin{algorithmic}[1]
    \STATE \textcolor{inputcolor}{\textbf{Input:}}
    \STATE \textcolor{inputcolor}{(1) $N$ responses, each with embedding $\mathbf{e}_i \in \mathbb{R}^d$ and rating $r_i$}
    \STATE \textcolor{inputcolor}{(2) Desired number of negatives $k$}
    \STATE
    \STATE \textcolor{stepcolor}{\textbf{Step 1:} \textit{Run $k$-means on embeddings}}
    \STATE \textcolor{mathcolor}{Initialize $\{\mathbf{c}_1,\dots,\mathbf{c}_k\} \subset \mathbb{R}^d$ (e.g., via $k$-means++)}
    \REPEAT
        \STATE \textcolor{mathcolor}{$\pi(i) = \arg\min_{1 \le j \le k} \|\mathbf{e}_i - \mathbf{c}_j\|^2$, \quad $i = 1,\dots,N$}
        \STATE \textcolor{mathcolor}{$\mathbf{c}_j = \frac{\sum_{i:\pi(i)=j}\mathbf{e}_i}{\sum_{i:\pi(i)=j}1}$, \quad $j = 1,\dots,k$} \label{eq:vanilla_kmeans}
    \UNTIL{convergence}
    \STATE
    \STATE \textcolor{stepcolor}{\textbf{Step 2:} \textit{In each cluster, pick the bottom-rated response}}
    \STATE \textcolor{mathcolor}{For each $j \in \{1,\dots,k\}$, define $C_j = \{\, i \mid \pi(i) = j \}$}
    \STATE \textcolor{mathcolor}{Then $i_j^- = \arg\min_{i\in C_j} r_i$, \quad $j = 1,\dots,k$}
    \STATE
    \STATE \textcolor{stepcolor}{\textbf{Step 3:} Return negatives}
    \STATE \textcolor{mathcolor}{$S^- = \{\, i_1^-,\, i_2^- ,\dots, i_k^- \}$}
    \RETURN \textcolor{outputcolor}{$S^-$ as the set of $k$ negatives}
\end{algorithmic}
\end{algorithm}

% \begin{algorithm}[tb]
% \caption{$\ampocs$($\{\mathbf{e}_i,r_i\}_{i=1}^N$, $k$)}
% \label{alg:cluster_negatives}
% \begin{algorithmic}[1]
%     \STATE {\bfseries Input:} 
%     \STATE (1) $N$ responses, each with embedding $\mathbf{e}_i \in \mathbb{R}^d$ and rating $r_i$
%     \STATE (2) Desired number of negatives $k$
%     \STATE
%     \STATE {\bfseries Step 1:} Run $k$-means on embeddings
%     \STATE Initialize $\{\mathbf{c}_1,\dots,\mathbf{c}_k\} \subset \mathbb{R}^d$ (e.g., via $k$-means++)
%     \REPEAT
%         \STATE $\pi(i) = \arg\min_{1 \le j \le k} \|\mathbf{e}_i - \mathbf{c}_j\|^2$, \quad $i = 1,\dots,N$
%         \STATE $\mathbf{c}_j = \frac{\sum_{i:\pi(i)=j}\mathbf{e}_i}{\sum_{i:\pi(i)=j}1}$, \quad $j = 1,\dots,k$ \label{eq:vanilla_kmeans}
%     \UNTIL{convergence}
%     \STATE
%     \STATE {\bfseries Step 2:} In each cluster, pick the bottom-rated response
%     \STATE For each $j \in \{1,\dots,k\}$, define $C_j = \{\, i \mid \pi(i) = j \}$
%     \STATE Then $i_j^- = \arg\min_{i\in C_j} r_i$, \quad $j = 1,\dots,k$
%     \STATE
%     \STATE {\bfseries Step 3:} Return negatives
%     \STATE $S^- = \{\, i_1^-,\, i_2^- ,\dots, i_k^- \}$
%     \RETURN $S^-$ as the set of $k$ negatives
% \end{algorithmic}
% \end{algorithm}



This approach enforces that each cluster---a potential ``mode'' in the response space---contributes at least one negative example. Hence, \textsc{AMPO-Coreset} can be interpreted as selecting \emph{representative} negatives from diverse semantic regions, ensuring that the model is penalized for a wide variety of undesired responses.


\section{Opt-Select: Active Subset Selection by Optimizing Expected Reward}
\label{sec:opt_select}

In this section, we propose \emph{Opt-Select}: a strategy for choosing $k$ \emph{negative} responses (plus one \emph{positive}) so as to \emph{maximize} the policy’s expected reward under a Lipschitz continuity assumption. Specifically, we model the local “neighborhood” influence of penalizing each selected negative and formulate an optimization problem that seeks to suppress large pockets of low-reward answers while preserving at least one high-reward mode. We first describe the intuition and objective, then present two solution methods: a \emph{mixed-integer program} (MIP) and a \emph{local search} approximation.

\subsection{Lipschitz-Driven Objective}
\label{subsec:lipschitz_objective}

Let $\{y_i\}_{i=1}^n$ be candidate responses sampled on-policy, each with reward $r_i \in [0,1]$ and embedding $\mathbf{e}_i \in \mathbb{R}^d$. Suppose that if we \emph{completely suppress} a response $y_j$ (i.e.\ set its probability to zero), all answers within distance $\|\mathbf{e}_i - \mathbf{e}_j\|$ must also decrease in probability proportionally, due to a Lipschitz constraint on the policy. Concretely, if the distance is $d_{i,j} = \|\mathbf{e}_i - \mathbf{e}_j\|$, and the model’s Lipschitz constant is $L$, then the probability of $y_i$ cannot remain above $L\,d_{i,j}$ if $y_j$ is forced to probability zero.

From an \emph{expected reward} perspective, assigning zero probability to \emph{low-reward} responses (and their neighborhoods) improves overall alignment. To capture this rigorously, observe that the \emph{penalty} from retaining a below-average answer $y_i$ can be weighted by:
\begin{align}
\label{eq:weight_w_i}
    w_i 
    \;=\;
    \exp\bigl(\,\overline{r} \;-\; r_i\bigr),
\end{align}
where $\overline{r}$ is (for instance) the mean reward of $\{r_i\}$. Intuitively, $w_i$ is larger for lower-reward $y_i$, indicating it is more harmful to let $y_i$ and its neighborhood remain at high probability.

Next, define a distance matrix 
\begin{align}
\label{eq:distance_matrix}
  A_{i,j} \;=\;
  \bigl\|\mathbf{e}_i - \mathbf{e}_j\bigr\|_2,
  \quad
  1 \le i,j \le n.
\end{align}
Selecting a subset $S\subseteq \{1,\dots,n\}$ of “negatives” to penalize suppresses the probability of each $i$ in proportion to $\min_{j \in S} A_{i,j}$. Consequently, a natural \emph{cost} function measures how much “weighted distance” $y_i$ has to its closest chosen negative:
\begin{align}
\label{eq:weighted_distance_cost}
    \text{Cost}(S)
    \;=\;
    \sum_{i=1}^n 
    w_i 
    \;\min_{\,j \in S}\;
    A_{i,j}.
\end{align}
Minimizing \eqref{eq:weighted_distance_cost} yields a subset $S$ of size $k$ that “covers” or “suppresses” as many low-reward responses (large $w_i$) as possible. We then \emph{add} one \emph{positive} index $i_{\mathrm{top}}$ with the highest $r_i$ to amplify a top-quality answer. This combination of \emph{one positive} plus \emph{$k$ negatives} provides a strong signal in the training loss.

\paragraph{Interpretation and Connection to Weighted k-medoids.}
If each negative $j$ “covers” responses $i$ within some radius (or cost) $A_{i,j}$, then \eqref{eq:weighted_distance_cost} is analogous to a weighted \emph{$k$-medoid} objective, where we choose $k$ items (negatives) to minimize a total weighted distance. Formally, this can be cast as a mixed-integer program (MIP) (Problem~\ref{eq:problem_P} below). For large $n$, local search offers an efficient approximation.

\subsection{Mixed-Integer Programming Formulation}

Define binary indicators $x_j = 1$ if we choose $y_j$ as a negative, and $z_{i,j} = 1$ if $i$ is assigned to $j$ (i.e.\ $\min_{j\in S} A_{i,j}$ is realized by $j$). We write:

\vspace{-0.15in}
\begin{align}
\label{eq:problem_P}
\textbf{Problem } \mathcal{P}: \quad
&\min_{\substack{x_j \in \{0,1\},\ z_{i,j}\in\{0,1\},\ y_i\ge 0}} 
 \sum_{i=1}^n w_i \,y_i 
\\
\text{s.t.}\quad
& \sum_{j=1}^n x_j = k, 
z_{i,j}\le x_j,
\sum_{j=1}^n z_{i,j} = 1, \forall\,i,\nonumber\\
& y_i \le A_{i,j} + M\,(1 - z_{i,j}), \nonumber\\
&y_i \ge A_{i,j} - M\,(1 - z_{i,j}),\quad\forall\,i,j,
\end{align}

\vspace{-0.1in}
where $M=\max_{i,j} A_{i,j}$. In essence, each $i$ is forced to \emph{assign} to exactly one chosen negative $j$, making $y_i = A_{i,j}$, i.e. the distance between the answer embeddings for answer $\{i,j\}$. Minimizing $\sum_i w_i\,y_i$ (i.e.\ \eqref{eq:weighted_distance_cost}) then ensures that low-reward points ($w_i$ large) lie close to at least one penalized center.

\vspace{-0.1in}
\paragraph{Algorithmic Overview.}
Solving $\mathcal{P}$ gives the $k$ negatives $S_{\mathrm{neg}}$, while the highest-reward index $i_{\mathrm{top}}$ is chosen as a positive. The final subset $\{i_{\mathrm{top}}\}\cup S_{\mathrm{neg}}$ is then passed to the \textsc{swepo} loss (see Section \ref{sec:methodology}). Algorithm~\ref{alg:opt_select} outlines the procedure succinctly.


\begin{algorithm}[t]
\caption{\textcolor{titlecolor}{$\ampoos$ via Solving MIP}}
\label{alg:opt_select}
\begin{algorithmic}[1]
    \STATE \textcolor{inputcolor}{\textbf{Input:} Candidates $\{y_i\}_{i=1}^n$ with $r_i, \mathbf{e}_i$; integer $k$}
    \STATE \textcolor{mathcolor}{Compute $i_{\mathrm{top}} = \arg\max_i\,r_i$}
    \STATE \textcolor{mathcolor}{Let $w_i = \exp(\,\overline{r} - r_i)$ with $\overline{r}$ as mean reward}
    \STATE \textcolor{mathcolor}{Solve Problem~\eqref{eq:problem_P} to get $\{x_j^*\}, \{z_{i,j}^*\}, \{y_i^*\}$}
    \STATE \textcolor{mathcolor}{Let $S_{\mathrm{neg}} = \{\,j \mid x_j^*=1\}$ (size $k$)}
    \RETURN \textcolor{outputcolor}{$\{\,i_{\mathrm{top}}\}\cup S_{\mathrm{neg}}$ for \textsc{swepo} training}
\end{algorithmic}
\end{algorithm}

% \begin{algorithm}[t]
% \caption{$\ampoos$ via Solving MIP}
% \label{alg:opt_select}
% \begin{algorithmic}[1]
%     \STATE {\bfseries Input:} Candidates $\{y_i\}_{i=1}^n$ with $r_i, \mathbf{e}_i$; integer $k$
%     \STATE Compute $i_{\mathrm{top}} = \arg\max_i\,r_i$
%     \STATE Let $w_i = \exp(\,\overline{r} - r_i)$ with $\overline{r}$ as mean reward
%     \STATE Solve Problem~\eqref{eq:problem_P} to get $\{x_j^*\}, \{z_{i,j}^*\}, \{y_i^*\}$
%     \STATE Let $S_{\mathrm{neg}} = \{\,j \mid x_j^*=1\}$ (size $k$)
%     \RETURN $\{\,i_{\mathrm{top}}\}\cup S_{\mathrm{neg}}$ for \textsc{swepo} training
% \end{algorithmic}
% \end{algorithm}

\vspace{-0.1in}
\subsection{Local Search Approximation}

\vspace{-0.1in}
For large $n$, an exact MIP can be expensive. A simpler \emph{local search} approach initializes a random subset $S$ of size $k$ and iteratively swaps elements in and out if it lowers the cost \eqref{eq:weighted_distance_cost}. In practice, this provides an efficient approximation, especially when $n$ or $k$ grows.

\begin{algorithm}[t]
\caption{\textcolor{titlecolor}{$\ampoos$ via Coordinate Descent}}
\label{alg:opt_select_local_search}
\begin{algorithmic}[1]
    \STATE \textcolor{inputcolor}{\textbf{Input:} Set $I = \{1,\dots,n\}$, integer $k$, distances $A_{i,j}$, rewards $\{r_i\}$}
    \STATE \textcolor{mathcolor}{Find $i_{\mathrm{top}} = \arg\max_{i}\, r_i$}
    \STATE \textcolor{mathcolor}{Compute $w_i = \exp(\,\overline{r} - r_i)$ and $d_{i,j}=A_{i,j}$}
    \STATE \textcolor{mathcolor}{Initialize a random subset $S \subseteq I\setminus\{i_{\mathrm{top}}\}$ of size $k$}
    \WHILE{improving}
        \STATE \textcolor{mathcolor}{Swap $j_{\mathrm{out}} \in S$ with $j_{\mathrm{in}} \notin S$ if it decreases $\sum_{i \in I} w_i\,\min_{j \in S} d_{i,j}$}
    \ENDWHILE
    \RETURN \textcolor{outputcolor}{$S_{\mathrm{neg}}=S$ (negatives) and $i_{\mathrm{top}}$ (positive)}
\end{algorithmic}
\end{algorithm}

% \begin{algorithm}[t]
% \caption{$\ampoos$ via Coordinate Descent}
% \label{alg:opt_select_local_search}
% \begin{algorithmic}[1]
%     \STATE {\bfseries Input:} Set $I = \{1,\dots,n\}$, integer $k$, distances $A_{i,j}$, rewards $\{r_i\}$
%     \STATE Find $i_{\mathrm{top}} = \arg\max_{i}\, r_i$
%     \STATE Compute $w_i = \exp(\,\overline{r} - r_i)$ and $d_{i,j}=A_{i,j}$
%     \STATE Initialize a random subset $S \subseteq I\setminus\{i_{\mathrm{top}}\}$ of size $k$
%     \WHILE{improving}
%         \STATE Swap $j_{\mathrm{out}} \in S$ with $j_{\mathrm{in}} \notin S$ if it decreases $\sum_{i \in I} w_i\,\min_{j \in S} d_{i,j}$
%     \ENDWHILE
%     \RETURN $S_{\mathrm{neg}}=S$ (negatives) and $i_{\mathrm{top}}$ (positive)
% \end{algorithmic}
% \end{algorithm}

\paragraph{Intuition.}
If $y_i$ is far from all penalized points $j\in S$, then it remains relatively “safe” from suppression, which is undesirable if $r_i$ is low (i.e.\ $w_i$ large). By systematically choosing $S$ to reduce $\sum_i w_i\,\min_{j\in S}d_{i,j}$, we concentrate penalization on high-impact, low-reward regions. The local search repeatedly swaps elements until no single exchange can further reduce the cost.

\subsection{Why ``Opt-Select''? A Lipschitz Argument for Expected Reward}

We name the procedure ``Opt-Select'' because solving \eqref{eq:problem_P} (or its local search variant) directly approximates an \emph{optimal} subset for improving the policy's expected reward. Specifically, under a Lipschitz constraint with constant $L$, assigning zero probability to each chosen negative $y_j$ implies \emph{neighboring answers} $y_i$ at distance $d_{i,j}$ cannot exceed probability $L\,d_{i,j}$. Consequently, their contribution to the ``bad behavior'' portion of expected reward is bounded by
\[
   \exp\bigl(r_{\max} - r_i\bigr)\,\bigl(\,L\,d_{i,j}\bigr),
\]
where $r_{\max}$ is the rating of the best-rated response. Dividing by a normalization factor (such as $\exp(r_{\max} - \overline{r})\,L$), one arrives at a cost akin to $w_i\, d_{i,j}$ with $w_i = \exp(\overline{r}-r_i)$. 
% Hence, minimizing \eqref{eq:weighted_distance_cost} ensures that \emph{low-reward} points do not remain at large distance from any chosen negative (i.e.\ they get suppressed). 
This aligns with classical \emph{min-knapsack} of minimizing some costs subject to some constraints, and has close alignment with the \emph{weighted $k$-medoid} notions of “covering” important items at minimum cost. %By combining this with a single top-reward positive, Opt-Select systematically pushes probability mass toward high-reward modes while penalizing low-reward regions.

\section{Experiments}
\label{sec:Experiments} 

We conduct several experiments across different problem settings to assess the efficiency of our proposed method. Detailed descriptions of the experimental settings are provided in \cref{sec:apendix_experiments}.
%We conduct experiments on optimizing PINNs for convection, wave PDEs, and a reaction ODE. 
%These equations have been studied in previous works investigating difficulties in training PINNs; we use the formulations in \citet{krishnapriyan2021characterizing, wang2022when} for our experiments. 
%The coefficient settings we use for these equations are considered challenging in the literature \cite{krishnapriyan2021characterizing, wang2022when}.
%\cref{sec:problem_setup_additional} contains additional details.

%We compare the performance of Adam, \lbfgs{}, and \al{} on training PINNs for all three classes of PDEs. 
%For Adam, we tune the learning rate by a grid search on $\{10^{-5}, 10^{-4}, 10^{-3}, 10^{-2}, 10^{-1}\}$.
%For \lbfgs, we use the default learning rate $1.0$, memory size $100$, and strong Wolfe line search.
%For \al, we tune the learning rate for Adam as before, and also vary the switch from Adam to \lbfgs{} (after 1000, 11000, 31000 iterations).
%These correspond to \al{} (1k), \al{} (11k), and \al{} (31k) in our figures.
%All three methods are run for a total of 41000 iterations.

%We use multilayer perceptrons (MLPs) with tanh activations and three hidden layers. These MLPs have widths 50, 100, 200, or 400.
%We initialize these networks with the Xavier normal initialization \cite{glorot2010understanding} and all biases equal to zero.
%Each combination of PDE, optimizer, and MLP architecture is run with 5 random seeds.

%We use 10000 residual points randomly sampled from a $255 \times 100$ grid on the interior of the problem domain. 
%We use 257 equally spaced points for the initial conditions and 101 equally spaced points for each boundary condition.

%We assess the discrepancy between the PINN solution and the ground truth using $\ell_2$ relative error (L2RE), a standard metric in the PINN literature. Let $y = (y_i)_{i = 1}^n$ be the PINN prediction and $y' = (y'_i)_{i = 1}^n$ the ground truth. Define
%\begin{align*}
%    \mathrm{L2RE} = \sqrt{\frac{\sum_{i = 1}^n (y_i - y'_i)^2}{\sum_{i = 1}^n y'^2_i}} = \sqrt{\frac{\|y - y'\|_2^2}{\|y'\|_2^2}}.
%\end{align*}
%We compute the L2RE using all points in the $255 \times 100$ grid on the interior of the problem domain, along with the 257 and 101 points used for the initial and boundary conditions.

%We develop our experiments in PyTorch 2.0.0 \cite{paszke2019pytorch} with Python 3.10.12.
%Each experiment is run on a single NVIDIA Titan V GPU using CUDA 11.8.
%The code for our experiments is available at \href{https://github.com/pratikrathore8/opt_for_pinns}{https://github.com/pratikrathore8/opt\_for\_pinns}.


\subsection{2D Allen Cahn Equation}
\begin{figure*}[t]
    \centering
    \includegraphics[scale=0.38]{figs/Burgers_operator.pdf}
    \caption{1D Burgers' Equation (Operator Learning): Steady-state solutions for different initializations $u_0$ under varying viscosity $\varepsilon$: (a) $\varepsilon = 0.5$, (b) $\varepsilon = 0.1$, (c) $\varepsilon = 0.05$. The results demonstrate that all final test solutions converge to the correct steady-state solution. (d) Illustration of the evolution of a test initialization $u_0$ following homotopy dynamics. The number of residual points is $\nres = 128$.}
    \label{fig:Burgers_result}
\end{figure*}
First, we consider the following time-dependent problem:
\begin{align}
& u_t = \varepsilon^2 \Delta u - u(u^2 - 1), \quad (x, y) \in [-1, 1] \times [-1, 1] \nonumber \\
& u(x, y, 0) = - \sin(\pi x) \sin(\pi y) \label{eq.hom_2D_AC}\\
& u(-1, y, t) = u(1, y, t) = u(x, -1, t) = u(x, 1, t) = 0. \nonumber
\end{align}
We aim to find the steady-state solution for this equation with $\varepsilon = 0.05$ and define the homotopy as:
\begin{equation}
    H(u, s, \varepsilon) = (1 - s)\left(\varepsilon(s)^2 \Delta u - u(u^2 - 1)\right) + s(u - u_0),\nonumber
\end{equation}
where $s \in [0, 1]$. Specifically, when $s = 1$, the initial condition $u_0$ is automatically satisfied, and when $s = 0$, it recovers the steady-state problem. The function $\varepsilon(s)$ is given by
\begin{equation}
\varepsilon(s) = 
\left\{\begin{array}{l}
s, \quad s \in [0.05, 1], \\
0.05, \quad s \in [0, 0.05].
\end{array}\right.\label{eq:epsilon_t}
\end{equation}

Here, $\varepsilon(s)$ varies with $s$ during the first half of the evolution. Once $\varepsilon(s)$ reaches $0.05$, it remains fixed, and only $s$ continues to evolve toward $0$. As shown in \cref{fig:2D_Allen_Cahn_Equation}, the relative $L_2$ error by homotopy dynamics is $8.78 \times 10^{-3}$, compared with the result obtained by PINN, which has a $L_2$ error of $9.56 \times 10^{-1}$. This clearly demonstrates that the homotopy dynamics-based approach significantly improves accuracy.

\subsection{High Frequency Function Approximation }
We aim to approximate the following function:
$u=    \sin(50\pi x), \quad x \in [0,1].$
The homotopy is defined as $H(u,\varepsilon) = u - \sin(\frac{1}{\varepsilon}\pi x), $
where $\varepsilon \in [\frac{1}{50},\frac{1}{15}]$.

\begin{table}[htbp!]
    \caption{Comparison of the lowest loss achieved by the classical training and homotopy dynamics for different values of $\varepsilon$ in approximating $\sin\left(\frac{1}{\varepsilon} \pi x\right)$
    }
    \vskip 0.15in
    \centering
    \tiny
    \begin{tabular}{|c|c|c|c|c|} 
    \hline 
    $ $ & $\varepsilon = 1/15$ & $\varepsilon = 1/35$ & $\varepsilon = 1/50$ \\ \hline 
    Classical Loss                & 4.91e-6     & 7.21e-2     & 3.29e-1       \\ \hline 
    Homotopy Loss $L_H$                      & 1.73e-6     & 1.91e-6     & \textbf{2.82e-5}       \\ \hline
    \end{tabular}
    % On convection, \al{} provides 14.2$\times$ and 1.97$\times$ improvement over Adam or \lbfgs{} on L2RE. 
    % On reaction, \al{} provides 1.10$\times$ and 1.99$\times$ improvement over Adam or \lbfgs{} on L2RE.
    % On wave, \al{} provides 6.32$\times$ and 6.07$\times$ improvement over Adam or \lbfgs{} on L2RE.}
    \label{tab:loss_approximate}
\end{table}

As shown in \cref{fig:high_frequency_result}, due to the F-principle \cite{xu2024overview}, training is particularly challenging when approximating high-frequency functions like $\sin(50\pi x)$. The loss decreases slowly, resulting in poor approximation performance. However, training based on homotopy dynamics significantly reduces the loss, leading to a better approximation of high-frequency functions. This demonstrates that homotopy dynamics-based training can effectively facilitate convergence when approximating high-frequency data. Additionally, we compare the loss for approximating functions with different frequencies $1/\varepsilon$ using both methods. The results, presented in \cref{tab:loss_approximate}, show that the homotopy dynamics training method consistently performs well for high-frequency functions.





\subsection{Burgers Equation}
In this example, we adopt the operator learning framework to solve for the steady-state solution of the Burgers equation, given by:
\begin{align}
& u_t+\left(\frac{u^2}{2}\right)_x - \varepsilon u_{xx}=\pi \sin (\pi x) \cos (\pi x), \quad x \in[0, 1]\nonumber\\
& u(x, 0)=u_0(x),\label{eq:1D_Burgers} \\
& u(0, t)=u(1, t)=0, \nonumber 
\end{align}
with Dirichlet boundary conditions, where $u_0 \in L_{0}^2((0, 1); \mathbb{R})$ is the initial condition and $\varepsilon \in \mathbb{R}$ is the viscosity coefficient. We aim to learn the operator mapping the initial condition to the steady-state solution, $G^{\dagger}: L_{0}^2((0, 1); \mathbb{R}) \rightarrow H_{0}^r((0, 1); \mathbb{R})$, defined by $u_0 \mapsto u_{\infty}$ for any $r > 0$. As shown in Theorem 2.2 of \cite{KREISS1986161} and Theorems 2.5 and 2.7 of \cite{hao2019convergence}, for any $\varepsilon > 0$, the steady-state solution is independent of the initial condition, with a single shock occurring at $x_s = 0.5$. Here, we use DeepONet~\cite{lu2021deeponet} as the network architecture. 
The homotopy definition, similar to ~\cref{eq.hom_2D_AC}, can be found in \cref{Ap:operator}. The results can be found in \cref{fig:Burgers_result} and \cref{tab:loss_burgers}. Experimental results show that the homotopy dynamics strategy performs well in the operator learning setting as well.


\begin{table}[htbp!]
    \caption{Comparison of loss between classical training and homotopy dynamics for different values of $\varepsilon$ in the Burgers equation, along with the MSE distance to the ground truth shock location, $x_s$.}
    \vskip 0.15in
    \centering
    \tiny
    \begin{tabular}{|c|c|c|c|c|} 
    \hline  
    $ $ & $\varepsilon = 0.5$ & $\varepsilon = 0.1$ & $\varepsilon = 0.05$ \\ \hline 
    Homotopy Loss $L_H$                &  7.55e-7     & 3.40e-7     & 7.77e-7       \\ \hline 
    L2RE                      & 1.50e-3     & 7.00e-4     & 2.52e-2       \\ \hline
        MSE Distance $x_s$                      & 1.75e-8     & 9.14e-8      & 1.2e-3      \\ \hline
    \end{tabular}
    % On convection, \al{} provides 14.2$\times$ and 1.97$\times$ improvement over Adam or \lbfgs{} on L2RE. 
    % On reaction, \al{} provides 1.10$\times$ and 1.99$\times$ improvement over Adam or \lbfgs{} on L2RE.
    % On wave, \al{} provides 6.32$\times$ and 6.07$\times$ improvement over Adam or \lbfgs{} on L2RE.}
    \label{tab:loss_burgers}
\end{table}



% \begin{itemize}
%     \item Relate the curvature in the problem to the differential operator. Use this to demonstrate why the problem is ill-conditioned
%     \item Give an argument for why using Adam + L-BFGS is better than just using L-BFGS outright. The idea is that Adam lowers the errors to the point where the rest of the optimization becomes convex \ldots
%     \item Show why we need second-order methods. We would like to prove that once we are close to the optimum, second-order methods will give condition-number free linear convergence. Specialize this to the Gauss-Newton setting, with the randomized low-rank approximation.
%     % \item Show that it is not possible to get superlinear convergence under the interpolation assumption with an overparameterized neural network. This should be true b/c the Hessian at the optimum will have rank $\min(n, d)$, and when $d > n$, this guarantees that we cannot have strong convexity.
% \end{itemize}
\section{Experiments: Planning outperforms Heuristics}
\label{sec:experiment}

We begin our empirical demonstrations by showcasing the effectiveness of our planning framework on both synthetic and real datasets. We focus on the simplest planning algorithm, 1-step lookaheads (Algorithm~\ref{alg:complete}), and show that even basic planning can hold great promise. 
We illustrate our framework using two uncertainty quantification modules---GPs and 
\ensembles/ \ensembleplus. 

Throughout this section, we focus on evaluating the mean squared error of 
a regression model $\model$,  and develop adaptive policies that minimize uncertainty on $g(f)$ defined in~\eqref{eqn:l2-g-f}.
When GPs provide a valid model of uncertainty, 
our experiments show that our planning framework significantly outperforms other baselines. 
We further demonstrate that our conceptual framework extends to deep learning-based uncertainty quantification methods such as  \ensembleplus while highlighting computational challenges that need to be resolved in order to scale our ideas. 
For simplicity, we assume a naive predictor, i.e., $\psi(\cdot) \equiv 0$. However, we emphasize that this problem is just as complex as if we were using a sophisticated model $\psi(.)$. The performance gap between the algorithms 
primarily depends
on the level  of uncertainty in our prior beliefs.

To evaluate the performance of our algorithm, we benchmark it against several baselines. 
%Active learning baselines use an acquisition function $\ac$ to select points that have the highest   function value: $X\opt_t \in \argmax_{X \in \xpoolj{t}} \ac({X})$ at every step $t$. These methods may also need an UQ module, which we simply use the same UQ module as in our algorithm, and it  outputs $V(X)$ that measures the the uncertainty of each point $X \in \xpoolj{t}$.
Our first set of baselines are from active learning~\citep{AggarwalKoGuHaPh14}:
\\ % \noindent\textbf{Active Learning Heuristics:} 
\textbf{(1)} 
\textsf{Uncertainty Sampling (Static):}  In this approach, we query the samples for which the model is least certain about. Specifically, we estimate the variance of the latent output $f(X)$ for each $X \in \xpool$ using the UQ module and select the top-$K$ points with the highest uncertainty. \\
\textbf{(2)} \textsf{Uncertainty Sampling (Sequential):} This is a greedy heuristic that sequentially selects the points with the highest uncertainty within a batch, while updating the posterior beliefs using pseudo labels from the current posterior state. Unlike \textsf{Uncertainty Sampling (Static)}, this method takes into account the information gained from each point within batch, and hence tries to diversify the selected points within a batch. 

 
We also compare our approach to the  \textbf{(3)} \textsf{Random Sampling}, which selects each batch uniformly at random from the pool. Additionally, we compare solving the planning problem using  \textsf{REINFORCE}-based policy gradients with   $\mathsf{Smoothed\text{-}Autodiff}$ policy gradients.\footnote{Our code repository is available at
  \url{https://github.com/namkoong-lab/adaptive-labeling}.}
%Detailed experimental setups are provided in Section \ref{sec:details-experiments}.

%We repeat all experiments with 10 random seeds.




\begin{figure}[t]
\centering
\begin{minipage}[b]{0.49\textwidth}
\centering
\includegraphics[width=\textwidth, height=5cm]{figures/original_scale/Var_of_l_2_loss.pdf}
\caption{(Synthetic data) Variance of mean squared loss evaluated through the posterior belief $\mu_t$ at each horizon $t$. This is the objective that policy gradient methods like \textsf{REINFORCE} and $\ouralgo$ optimizes. 1-step lookaheads are surprisingly effective even in long horizons.}
\label{fig:var-l2-sim}
\end{minipage}
\hfill
\begin{minipage}[b]{0.49\textwidth}
\centering \includegraphics[width=\textwidth, height=5cm]{figures/original_scale/Error_of_estimated_model_l_2_loss.pdf}
\caption{(Synthetic data) Error between MSE calculated based on collected data $\mc{D}^{0:T}$ vs. population oracle MSE over $\mc{D}_{\rm eval} \sim P_X$. Reducing uncertainty over posteriors directly leads to better OOD evaluations. 1-step lookaheads significantly outperform active learning heuristics in small horizons.}
\label{fig:mean-l2-sim}
\end{minipage}
%\caption{Simulated data for GPs}
%\label{fig:both_plots}
\end{figure}

\subsection{Planning with Gaussian processes}
\label{sec:experiment-plan-GP}
We now briefly describe the data generation process for the GP experiments,  deferring a more detailed discussion of the dataset generation to Section~\ref{sec:details-experiments}. 
We use both the synthetic data and the real data to test our methodology.
For the \emph{simulated data},  we construct a setting where the general population is distributed across \emph{51 non-overlapping clusters} while the initial labeled data $\dtrain$ just comes from one cluster. In contrast, both $\dpool \defeq (\xpool,\ypool),\deval \defeq (\xeval,\yeval)$ are generated   from all the clusters. 
We begin with a low-dimensional scenario, generating a one-dimensional regression setting using a GP. %Gaussian Process (GP).
Although the data-generating process is not known to the algorithms,  we assume that the GP hyperparameters are known to all the algorithms
to ensure fair comparisons. This can be viewed as a setting where our prior is well-specified, allowing us to isolate the effects
of different policy optimization approaches
 without any concerns about the misspecified priors. We select $10$ batches, each of size $K=5$ across $T = 10$ time horizons.

To examine the robustness of our method against the distributional assumptions made  in the simulated case, we then move to a real dataset where the correct prior is not known. We simulate selection bias from the eICU dataset~\citep{PollardJoRaCeMaBa18}, which contains real-world patient data with in-hospital mortality outcomes. 
We conduct a $k$-means clustering to generate 51 clusters and then select data from those clusters. We view this to be a credible replication of practice, as severe distribution shifts are common due to selection bias in clinical labels.  To convert the binary mortality labels into a regression setting, we train a  random forest classifier and fit a GP on predicted scores, which serves as the UQ module for all the algorithms. As before, the task is to select 10 batches, each consisting of 5 samples, across 10 time horizons.

 In Figures~\ref{fig:var-l2-sim} and~\ref{fig:mean-l2-sim}, we present results for the simulated data. 
Figure~\ref{fig:var-l2-sim} shows the variance of $\ell_2$ loss, and Figure~\ref{fig:mean-l2-sim} presents the error in the estimated $\ell_2$ loss using $\mu_t$ (relative to true $\ell_2$ loss, that is unknown to the algorithm). 
As we can see from these plots, our method one-step lookahead  gives substantial improvements  over active learning baselines and random sampling. In addition,
compared to the one-step lookahead planning approach using \textsf{REINFORCE}-based policy gradients, 
we observe that $\mathsf{Smoothed\text{-}Autodiff}$-based policy gradients provide significantly more robust performance over all horizons.

In Figures~\ref{fig:var-l2-real}~and~\ref{fig:mean-l2-real}, we observe similar findings on the eICU data. We see that planning policies (\textsf{REINFORCE} and $\mathsf{Smoothed\text{-}Autodiff}$) consistently outperform other heuristics by a large margin.  Active learning baselines perform poorly in these small-horizon batched problems and can sometimes be even worse than the random search baselines.  Overall, our results show the importance of careful planning in adaptive labeling for reliable model evaluation. 

We offer some intuition as to why one-step lookahead planning may outperform other heuristic algorithms. 
 First,  \textsf{Uncertainty sampling (Static)} while myopically selects the
 top-$K$ inputs with the highest uncertainty, it fails to consider 
the overlap in information content among the ``best” instances; see \citep{AggarwalKoGuHaPh14} for more details. 
In other words,  it might acquire points from the same region with high uncertainty while failing to induce diversity among the batch.
Although \textsf{Uncertainty Sampling (Sequential)} somewhat addresses the issue of information overlap, a significant drawback of 
this algorithm
is the disconnect between the objective we aim to optimize and the algorithm. For example, it might sample from a region with high uncertainty but very low density. 

\begin{figure}[t]
\centering
\begin{minipage}[b]{0.48\textwidth}
\centering
\includegraphics[width=\textwidth, height=5cm]{figures/original_scale/Var_of_l_2_loss_real.pdf}
\caption{(Real-world eICU data) Variance of mean squared loss evaluated through the posterior belief $\mu_t$ at each horizon $t$. Even 1-step lookaheads are extremely effective planners, and auto-differentiation-based pathwise policy gradients provide a reliable optimization algorithm based on low-variance gradient estimates.}
\label{fig:var-l2-real}
\end{minipage}
\hfill
\begin{minipage}[b]{0.48\textwidth}
\centering \includegraphics[width=\textwidth, height=5cm]{figures/original_scale/Error_of_estimated_model_l_2_loss_real.pdf}
\caption{(Real-world eICU data) Error between MSE calculated based on collected data $\mc{D}^{0:T}$ vs. population oracle MSE over $\mc{D}_{\rm eval} \sim P_X$. Reducing uncertainty over posteriors directly leads to better OOD evaluations. Our method significantly outperforms active learning-based heuristics, and random sampling.}
\label{fig:mean-l2-real}
\end{minipage}
%\caption{Real data for GPs}
\end{figure}
 
%\vspace{-1.5cm}
% \begin{wrapfigure}{r}{.32\columnwidth}
%   \vspace{-.5cm} 
%   \centering
% \includegraphics[scale=.29]{figures/Var of l2l_2 loss.pdf}
%   \vspace{-0.2cm}
%   \caption{Results of GP}
% \label{fig:var-l2-gp}
%   \vspace{-0.1cm}
% \end{wrapfigure}


% Attempts have been made  in the past to address these  drawbacks heuristically  (see \citep{AggarwalKoGuHaPh14}). We give a unified computational framework while approaching the problem in a more principled manner and solving it more optimally.




\subsection{Planning with  neural network-based uncertainty quantification methods ($\ensembleplus$)}


We now provide a proof-of-concept that shows the generalizability of our conceptual framework  to the deep learning-based UQ modules, specifically focusing on $\ensembleplus$ due to their previously observed superior performance~\citep{OsbandWenAsDwIbLuRo23}. Recall that implementing our framework with deep learning-based UQ modules  requires us to retrain the model across multiple possible random actions $\bm{a}(\theta)$ sampled from the current policy $\pi_\theta$.
This requires significant computational resources, in sharp contrast to the GPs where the posteriors are in closed form and can be readily updated and differentiated. 

Due to the computational constraints, we test $\ensembleplus$ on a toy setting to demonstrate the generalizability of our framework. We consider a setting where the general population consists of four clusters, while the initial labeled data only comes from one cluster. Again we generate data using GPs.  The task is to select a batch of 2 points in one horizon. We detail the $\ensembleplus$ architecture in Section \ref{sec:details-experiments}, and we assume prior uncertainty to be large (depends on the scaling of the prior generating functions). 
The results are summarized in the Table~\ref{tab:UQ_ensemble}.

% \begin{table}[H]
% \vspace{-10pt}
% \caption{Performance under \ensembleplus as UQ module}
%     \centering
%     \begin{tabular}{|m{3cm}|m{2.5cm}|m{2cm}|} 
%     \hline
%       Algorithm   & Variance of $\loss_2$ loss estimate & Error of $\loss_2$ loss estimate  \\ \hline Random Sampling 
%          & $1710.9 \pm 1352.1$ & $8.67\pm6.62$ 
%       \\ \hline \ouralgo & $1.30 \pm 0.68$ & $0.91\pm0.25$ \\ \hline
%     \end{tabular}
%     \label{tab:UQ_ensemble}
%     %\vspace{-10pt}
% \end{table}




\begin{table}[h]
\vspace{-10pt}
\caption{Performance under \ensembleplus as the UQ module}
\centering
\begin{tabular}{|l|l|l|}
\hline
Algorithm   & Variance of $\loss_2$ loss estimate & Error of $\loss_2$ loss estimate  \\
\hline
\textsf{Random sampling} & 7129.8 $\pm$ 1027.0 & 136.2 $\pm$ 8.28 \\ \hline
\textsf{Uncertainty sampling (Static)} & 10852 $\pm$ 0.0 & 162.156 $\pm$ 0.0 \\ \hline
\textsf{Uncertainty sampling (Sequential)} & 8585.5 $\pm$ 898.9 & 144 $\pm$ 6.93 \\ \hline
\textsf{REINFORCE} & 1697.1 $\pm$ 0.0 & 45.27 $\pm$ 0.0 \\ \hline
\ouralgo & 1697.1 $\pm$ 0.0 & 45.27 $\pm$ 0.0 \\ \hline
\end{tabular}
%\caption{Comparison of different algorithms based on variance   and   error in $\ell_2$ loss estimation with Ensemble $+$ as the UQ module. Our results demonstrate that {\ouralgo} and REINFORCE outperformthe other active learning based heuristics, confirming the benefits of our MDP formulation for the adaptive labeling problem, as also demonstrated in Section 4.\\
%\footnotesize{Experimental details: We use Gaussian Processes as our data generating process, GP parameters are the same as in Section D.3.  The task is to select a batch of 2 points along one horizon.The marginal distribution $p_X$ has 4 \textit{non-overlapping} clusters. Initial data comes from one cluster, while pool and evaluation points comes from all the clusters. We have $20$ initial labeled data points, $10$ pool points, and $252$ evaluation points.  Training procedures are similar to the one in Section D.3.} }
\label{tab:UQ_ensemble}
\end{table}



% We faced  issues in scaling up these experiments which will be our focus in the future. 





% \begin{itemize}
%     \item Posteriors should be consistent. Two dimensions: even with less training,  
%     \item the inference should be  fast enough
% \end{itemize}


% Potential research directions for uncertainty quantification

% In this section we consider a simple setting We consider a simpler setting and 


% For synthetic dataset generation, we use ...... For real datasets, we use ...... We compare our methodolgy to several baselines ()    This Section is structured as follows:
% \begin{itemize}
%     \item \textbf{GPs, square loss objective} (Section \ref{}): 
%     %the broad aim of the experiments  in this section is to isolate the performance of our methodology without any concerns for the inefficiencies induced due to a mis-specified prior or imperfect posterior inference. To accomplish this we generate synthetic datasets using GPs (detailed later). We use the well specified prior (GPs - with same hyperparameter setting) as our UQ module.   
%      As GPs provide differentaible posterior inference - any errors induced due to imperfect posterior updates are also isolated. We note that under this setting
%      \item In Section\ref{} we demonstrate why our methodology performs better than other baselines - by devising various synthetic experiments ()
%     \item  \textbf{UQ Benchmarking }(Section \ref{}): Before diving into the experiments using $\ensembleplus$ and ENNs,  we showcase our benchmarking experiments in Section \ref{}. We use real datasets We observe that ENNs perform better
%      \item \textbf{Ensemble $+$}, objective: recall, accuracy
%     \item \textbf{ENN}, objective: recall, accuracy
% \end{itemize}




% In Section {}, we test 
% \subsection{Experimental details}

% \begin{itemize}
%     \item UQ methodologies - GPs, ENNs
%     \item Objectives - Recall,  ATE
%     \item Datasets - ATE-synthetic datasets, Recall-synthetic, real datasets
%     \item Baselines - 
%     \begin{itemize}
%         \item Random sampling
%         \item Active learning - Uncertainty based sampling - In regression setting almost all of the 
%         \item Myopic greedy - Greedy Batch based sampling
%         \item Policy Gradient
%     \end{itemize}
    
% \end{itemize}

% \subsection{Experiments}
%     \begin{itemize}
%     \item GPs with square loss
%     \item Benchmarking ENN
%         \item ENNs with ATE
%         \item ENNs with Recall
%     \end{itemize}

% \subsection{Benefits over other algorithms - intuition and experiments}

%Active learning - Myopic greedy / Don't rely on the objective rather some entropy version.


%%% Local Variables:
%%% mode: latex
%%% TeX-master: "main"
%%% End:

% \section{Experimental Setup}
\label{sec:experimental setup}

\begin{figure*}[!thbp]
    \centering
    \includegraphics[width=1.0\textwidth]{images/tsne_plot_with_scores_35914.pdf}
    \vspace{-0.25in}
    % \includegraphics[draft]{images/Sweppo_illustration.pdf}
    \caption{t-SNE visualization of projected high-dimensional response embeddings into a 2D space, illustrating the separation of actively selected responses. (a) AMPO-BottomK (baseline). (b) AMPO-Coreset (ours). (c) Opt-Select (ours). We see that the traditional baselines select many responses close to each other, based on their rating. This provides insufficient feedback to the LLM during preference optimization. In contrast, our methods simultaneously optimize for objectives including coverage, generation probability as well as preference rating.}
\label{fig:summation_logps_analysis}
\end{figure*}




\begin{table}[t]
\centering
\resizebox{\columnwidth}{!}{
\begin{tabular}{@{}lcccc@{}}
\toprule
\multirow{2}{*}{\textbf{Method}} & \multicolumn{2}{c}{\textbf{AlpacaEval 2}} & \textbf{Arena-Hard} & \textbf{MT-Bench} \\ \cmidrule(lr){2-3} \cmidrule(lr){4-4} \cmidrule(lr){5-5}
 & \textbf{LC (\%)} & \textbf{WR (\%)} & \textbf{WR (\%)} & \textbf{GPT-4} \\ \midrule
Base & 28.4 & 28.4& 26.9 & 7.93 \\
Best-vs-worst & 47.6 & 44.7 & 34.6 & 7.51 \\
\ampo-Bottomk & 50.8 & 50.5 & \textbf{44.8} & \textbf{8.11} \\
\ampo-Coreset & \textbf{52.4} & \textbf{52.1} & \underline{39.4} & \underline{8.05} \\
\ampo-Opt-Select & \underline{51.6} & \underline{51.2} & 37.9 & 7.96 \\
\bottomrule
\end{tabular}
}
\caption{Comparison of various preference optimization baselines on AlpacaEval, Arena-Hard, and MT-Bench benchmarks for Llama-3-Instruct (8B). LC-WR represents length-controlled win rate, and WR represents raw win rate. Best results are in \textbf{bold}, second-best are \underline{underlined}. Our method ($\ampo$) achieves SOTA performance across all metrics, with different variants achieving either best or second-best results consistently.}
\label{tab:llama3-results}
\end{table}


\paragraph{Model and Training Settings:}
For our experiments, we utilized the Ultrafeedback Dataset \cite{cui2023ultrafeedback}, an instruction-following benchmark annotated by GPT-4. This dataset consists of approximately 64,000 instructions, each paired with four responses generated by different language models. GPT-4 assigned scalar rewards on a 0-to-10 scale for each response, which prior research has shown to correlate strongly with human annotations. This establishes GPT-4 ratings as a reliable and cost-efficient alternative to manual feedback.

In our broader framework, we first trained a base model (mistralai/Mistral-7B-v0.1) on the UltraChat-200k dataset to obtain an SFT model. This SFT model, trained on open-source data, provides a transparent starting point. Subsequently, we refined the model by performing preference optimization on the UltraFeedback dataset. Once fine-tuned, the model was used for alignment. This two-step process ensures the model is well-prepared for tasks.

In our experiments, we observed that tuning hyperparameters is critical for optimizing the performance of all offline preference optimization algorithms, including DPO, SimPO, \refa-InfoNCA, \refa-1-vs-all and \refa-dynamic. Carefully selecting hyperparameter values significantly impacts the effectiveness of these methods across various datasets.

For \refa-InfoNCA, we found that setting the beta parameter in the range of 2.0 to 2.5 consistently yields strong performance. Similarly, for \refa-1-vs-all and \refa-dynamic, optimal results were achieved with beta values between 2.0 and 4.0, while tuning the gamma parameter within the range of 0.7 to 1.4 further improved performance. These observations highlight the importance of systematic hyperparameter tuning to achieve reliable outcomes across diverse datasets.

\paragraph{Evaluation Benchmarks}
We evaluate our models using three widely recognized open-ended instruction-following benchmarks: MT-Bench, AlpacaEval 2, AlpacaEval and Arena-Hard v0.1. These benchmarks are commonly used in the community to assess the conversational versatility of models across a diverse range of queries.

AlpacaEval 2 comprises 805 questions sourced from five datasets, while MT-Bench spans eight categories with a total of 80 questions. The recently introduced Arena-Hard builds upon MT-Bench, featuring 500 well-defined technical problem-solving queries designed to test more advanced capabilities.

We adhere to the evaluation protocols specific to each benchmark when reporting results. For AlpacaEval 2, we provide both the raw win rate (WR) and the length-controlled win rate (LC), with the latter being designed to mitigate the influence of model verbosity. For Arena-Hard, we report the win rate (WR) against a baseline model. For MT-Bench, we present the scores as evaluated by GPT-4-Preview-1106, which serve as the judge model.


\begin{figure}
    \centering
\includegraphics[width=1.0\columnwidth]{images/active_selection_optimization_samp_temp_variation.pdf}
    \vspace{-0.25in}
    \caption{Effect of Sampling Temperature on different baselines for on the AlpacaEval 2 Benchmark: (a) Length-Controlled Win Rate (LC) and (b) Overall Win Rate (WR).}
    \label{fig:samp-temp-analysis}
\end{figure}


\paragraph{Baselines}
We compare our approach against several established offline preference optimization methods, summarized in Table . Among these are RRHF \cite{yuan2023rrhf} and SLiC-HF \cite{zhao2023slic}, which employ ranking loss techniques. RRHF uses a length-normalized log-likelihood function, akin to the reward function utilized by SimPO \cite{meng2024simpo}, whereas SLiC-HF directly incorporates log-likelihood and includes a supervised fine-tuning (SFT) objective in its training process.

IPO \cite{azar2023general} presents a theoretically grounded approach that avoids the assumption made by DPO, which treats pairwise preferences as interchangeable with pointwise rewards. CPO \cite{guo2024controllable}, on the other hand, uses sequence likelihood as a reward signal and trains jointly with an SFT objective.

ORPO \cite{hong2024orpo} introduces a reference-free odds ratio term to directly contrast winning and losing responses using the policy model, also incorporating joint training with the SFT objective. R-DPO \cite{Park2024DisentanglingLF} extends DPO by adding a regularization term that mitigates the exploitation of response length.

InfoNCA \cite{chen2024noise}, which introduces a K-category cross-entropy loss, reframes generative modeling problems as classification tasks by contrasting multiple data points. It computes soft labels using dataset rewards, applying a softmax operation to map reward values into probability distributions.

Lastly, SimPO \cite{meng2024simpo} leverages the average log probability of a sequence as an implicit reward, removing the need for a reference model. It further enhances performance by introducing a target reward margin to the Bradley-Terry objective, significantly improving the algorithm's effectiveness.



% \section{Attack Results}
\label{attack_results}

This section presents the evaluation of the snowball adversarial attack.

\begin{figure*}[th!]
    \centering
    % STOP images
    \begin{subfigure}[b]{0.1\textwidth}
        \centering
        \includegraphics[width=\linewidth]{plots/attack_images/Best_stop_snowball_1_patch_0.1_angle_135.png}
        \caption{\footnotesize \centering Stop\newline Snowball 1}
        \label{fig:stop_snowball_1}
    \end{subfigure}
    \begin{subfigure}[b]{0.1\textwidth}
        \centering
        \includegraphics[width=\linewidth]{plots/attack_images/Best_stop_snowball_2_patch_0.4_angle_45.png}
        \caption{\footnotesize \centering Stop\newline Snowball 2}
        \label{fig:stop_snowball_2}
    \end{subfigure}
    \begin{subfigure}[b]{0.1\textwidth}
        \centering
        \includegraphics[width=\linewidth]{plots/attack_images/Best_stop_snowball_3_patch_0.1_angle_180.png}
        \caption{\footnotesize \centering Stop\newline Snowball 3}
        \label{fig:stop_snowball_3}
    \end{subfigure}
    \begin{subfigure}[b]{0.1\textwidth}
        \centering
       \includegraphics[width=\linewidth]{plots/attack_images/Best_stop_snowball_4_patch_0.4_angle_90.png}
        \caption{\footnotesize \centering Stop\newline Snowball 4}
        \label{fig:stop_snowball_4}
    \end{subfigure}
    \begin{subfigure}[b]{0.1\textwidth}
        \centering
        \includegraphics[width=\linewidth]{plots/attack_images/Best_stop_snowball_5_patch_0.3_angle_315.png}
        \caption{\footnotesize \centering Stop\newline Snowball 5}
        \label{fig:stop_snowball_5}
    \end{subfigure}
    \begin{subfigure}[b]{0.1\textwidth}
        \centering
        \includegraphics[width=\linewidth]{plots/attack_images/Best_stop_snowball_6_patch_0.3_angle_135.png}
        \caption{\footnotesize \centering Stop\newline Snowball 6}
        \label{fig:stop_snowball_6}
    \end{subfigure}
    \begin{subfigure}[b]{0.1\textwidth}
        \centering
       \includegraphics[width=\linewidth]{plots/attack_images/Best_stop_snowball_7_patch_0.1_angle_180.png}
        \caption{\footnotesize \centering Stop\newline Snowball 7}
        \label{fig:stop_snowball_7}
    \end{subfigure}
    \begin{subfigure}[b]{0.1\textwidth}
        \centering
       \includegraphics[width=\linewidth]{plots/attack_images/Best_stop_snowball_8_patch_0.1_angle_90.png}
        \caption{\footnotesize \centering Stop\newline Snowball 8}
        \label{fig:stop_snowball_8}
    \end{subfigure}
    \begin{subfigure}[b]{0.1\textwidth}
        \centering
        \includegraphics[width=\linewidth]{plots/attack_images/Best_stop_snowball_9_patch_0.1_angle_180.png}
        \caption{\footnotesize \centering Stop\newline Snowball 9}
        \label{fig:stop_snowball_9}
    \end{subfigure}
\caption{ Stop Snowball Adversarial images.}
    \label{fig:stop_snowball_adversarial_image}
\end{figure*}

\begin{figure*}[th!]
    \centering
    % YIELD images
    \begin{subfigure}[b]{0.1\textwidth}
        \centering
       \includegraphics[width=\linewidth]{plots/attack_images/Best_yield_snowball_1_patch_0.3_angle_180.png}
        \caption{\footnotesize \centering Yield\newline Snowball 1}
        \label{fig:yield_snowball_1}
    \end{subfigure}
    \begin{subfigure}[b]{0.1\textwidth}
        \centering
        \includegraphics[width=\linewidth]{plots/attack_images/Best_yield_snowball_2_patch_0.2_angle_225.png}
        \caption{\footnotesize \centering Yield\newline Snowball 2}
        \label{fig:yield_snowball_2}
    \end{subfigure}
    \begin{subfigure}[b]{0.1\textwidth}
        \centering
        \includegraphics[width=\linewidth]{plots/attack_images/Best_yield_snowball_3_patch_0.4_angle_45.png}
        \caption{\footnotesize \centering Yield\newline Snowball 3}
        \label{fig:yield_snowball_3}
    \end{subfigure}
    \begin{subfigure}[b]{0.1\textwidth}
        \centering
        \includegraphics[width=\linewidth]{plots/attack_images/Best_yield_snowball_4_patch_0.3_angle_225.png}
        \caption{\footnotesize \centering Yield\newline Snowball 4}
        \label{fig:yield_snowball_4}
    \end{subfigure}
    \begin{subfigure}[b]{0.1\textwidth}
        \centering
        \includegraphics[width=\linewidth]{plots/attack_images/Best_yield_snowball_5_patch_0.2_angle_45.png}
        \caption{\footnotesize \centering Yield\newline Snowball 5}
        \label{fig:yield_snowball_5}
    \end{subfigure}
    \begin{subfigure}[b]{0.1\textwidth}
        \centering
        \includegraphics[width=\linewidth]{plots/attack_images/Best_yield_snowball_6_patch_0.3_angle_45.png}
        \caption{\footnotesize \centering Yield\newline Snowball 6}
        \label{fig:yield_snowball_6}
    \end{subfigure}
    \begin{subfigure}[b]{0.1\textwidth}
        \centering
        \includegraphics[width=\linewidth]{plots/base_images_resized/yield.png}
        \caption{\footnotesize \centering Yield\newline Snowball 7}
        \label{fig:yield_snowball_7}
    \end{subfigure}
    \begin{subfigure}[b]{0.1\textwidth}
        \centering
        \includegraphics[width=\linewidth]{plots/attack_images/Best_yield_snowball_8_patch_0.3_angle_45.png}
        \caption{\footnotesize \centering Yield\newline Snowball 8}
        \label{fig:yield_snowball_8}
    \end{subfigure}
    \begin{subfigure}[b]{0.1\textwidth}
        \centering
        \includegraphics[width=\linewidth]{plots/base_images_resized/yield.png}
        \caption{\footnotesize \centering Yield\newline Snowball 9}
        \label{fig:yield_snowball_9}
    \end{subfigure}
\caption{ Yield Snowball Adversarial images. Note no effective attack was found for signs (g) and (i).}
    \label{fig:yield_snowball_adversarial_image}
\end{figure*}

\begin{figure*}[th!]
    \centering
    % PEDESTRIAN images
    \begin{subfigure}[b]{0.1\textwidth}
        \centering
        \includegraphics[width=\linewidth]{plots/attack_images/Best_pedestrian_snowball_1_patch_0.2_angle_0.png}
        \caption{\footnotesize \centering Pedestrian\newline Snowball 1}
        \label{fig:pedestrian_snowball_1}
    \end{subfigure}
    \begin{subfigure}[b]{0.1\textwidth}
        \centering
        \includegraphics[width=\linewidth]{plots/attack_images/Best_pedestrian_snowball_2_patch_0.2_angle_0.png}
        \caption{\footnotesize \centering Pedestrian\newline Snowball 2}
        \label{fig:pedestrian_snowball_2}
    \end{subfigure}
    \begin{subfigure}[b]{0.1\textwidth}
        \centering
        \includegraphics[width=\linewidth]{plots/attack_images/Best_pedestrian_snowball_3_patch_0.3_angle_315.png}
        \caption{\footnotesize \centering Pedestrian\newline Snowball 3}
        \label{fig:pedestrian_snowball_3}
    \end{subfigure}
    \begin{subfigure}[b]{0.1\textwidth}
        \centering
        \includegraphics[width=\linewidth]{plots/attack_images/Best_pedestrian_snowball_4_patch_0.3_angle_45.png}
        \caption{\footnotesize \centering Pedestrian\newline Snowball 4}
        \label{fig:pedestrian_snowball_4}
    \end{subfigure}
    \begin{subfigure}[b]{0.1\textwidth}
        \centering
        \includegraphics[width=\linewidth]{plots/attack_images/Best_pedestrian_snowball_5_patch_0.3_angle_315.png}
        \caption{\footnotesize \centering Pedestrian\newline Snowball 5}
        \label{fig:pedestrian_snowball_5}
    \end{subfigure}
    \begin{subfigure}[b]{0.1\textwidth}
        \centering
        \includegraphics[width=\linewidth]{plots/attack_images/Best_pedestrian_snowball_6_patch_0.3_angle_0.png}
        \caption{\footnotesize \centering Pedestrian\newline Snowball 6}
        \label{fig:pedestrian_snowball_6}
    \end{subfigure}
    \begin{subfigure}[b]{0.1\textwidth}
        \centering
        \includegraphics[width=\linewidth]{plots/attack_images/Best_pedestrian_snowball_7_patch_0.2_angle_0.png}
        \caption{\footnotesize \centering Pedestrian\newline Snowball 7}
        \label{fig:pedestrian_snowball_7}
    \end{subfigure}
    \begin{subfigure}[b]{0.1\textwidth}
        \centering
        \includegraphics[width=\linewidth]{plots/attack_images/Best_pedestrian_snowball_8_patch_0.2_angle_90.png}
        \caption{\footnotesize \centering Pedestrian\newline Snowball 8}
        \label{fig:pedestrian_snowball_8}
    \end{subfigure}
    \begin{subfigure}[b]{0.1\textwidth}
        \centering
        \includegraphics[width=\linewidth]{plots/attack_images/Best_pedestrian_snowball_9_patch_0.2_angle_0.png}
        \caption{\footnotesize \centering Pedestrian\newline Snowball 9}
        \label{fig:pedestrian_snowball_9}
    \end{subfigure}
\caption{ Pedestrian Snowball Adversarial images.}
    \label{fig:pedestrian_snowball_adversarial_image}
\end{figure*}

\begin{figure*}[th!]
    \centering
    % MERGE images
    \begin{subfigure}[b]{0.1\textwidth}
        \centering
        \includegraphics[width=\linewidth]{plots/attack_images/Best_merge_snowball_1_patch_0.2_angle_135.png}
        \caption{\footnotesize \centering Merge\newline Snowball 1}
        \label{fig:merge_snowball_1}
    \end{subfigure}
    \begin{subfigure}[b]{0.1\textwidth}
        \centering
        \includegraphics[width=\linewidth]{plots/attack_images/Best_merge_snowball_2_patch_0.2_angle_180.png}
        \caption{\footnotesize \centering Merge\newline Snowball 2}
        \label{fig:merge_snowball_2}
    \end{subfigure}
    \begin{subfigure}[b]{0.1\textwidth}
        \centering
        \includegraphics[width=\linewidth]{plots/attack_images/Best_merge_snowball_3_patch_0.1_angle_135.png}
        \caption{\footnotesize \centering Merge\newline Snowball 3}
        \label{fig:merge_snowball_3}
    \end{subfigure}
    \begin{subfigure}[b]{0.1\textwidth}
        \centering
        \includegraphics[width=\linewidth]{plots/attack_images/Best_merge_snowball_4_patch_0.1_angle_225.png}
        \caption{\footnotesize \centering Merge\newline Snowball 4}
        \label{fig:merge_snowball_4}
    \end{subfigure}
    \begin{subfigure}[b]{0.1\textwidth}
        \centering
        \includegraphics[width=\linewidth]{plots/attack_images/Best_merge_snowball_5_patch_0.1_angle_315.png}
        \caption{\footnotesize \centering Merge\newline Snowball 5}
        \label{fig:merge_snowball_5}
    \end{subfigure}
    \begin{subfigure}[b]{0.1\textwidth}
        \centering
        \includegraphics[width=\linewidth]{plots/attack_images/Best_merge_snowball_6_patch_0.1_angle_270.png}
        \caption{\footnotesize \centering Merge\newline Snowball 6}
        \label{fig:merge_snowball_6}
    \end{subfigure}
    \begin{subfigure}[b]{0.1\textwidth}
        \centering
        \includegraphics[width=\linewidth]{plots/attack_images/Best_merge_snowball_7_patch_0.1_angle_0.png}
        \caption{\footnotesize \centering Merge\newline Snowball 7}
        \label{fig:merge_snowball_7}
    \end{subfigure}
    \begin{subfigure}[b]{0.1\textwidth}
        \centering
        \includegraphics[width=\linewidth]{plots/attack_images/Best_merge_snowball_8_patch_0.1_angle_270.png}
        \caption{\footnotesize \centering Merge\newline Snowball 8}
        \label{fig:merge_snowball_8}
    \end{subfigure}
    \begin{subfigure}[b]{0.1\textwidth}
        \centering
        \includegraphics[width=\linewidth]{plots/attack_images/Best_merge_snowball_9_patch_0.1_angle_225.png}
        \caption{\footnotesize \centering Merge\newline Snowball 9}
        \label{fig:merge_snowball_9}
    \end{subfigure}
\caption{ Merge Snowball Adversarial images.}
    \label{fig:merge_snowball_adversarial_image}
\end{figure*}

\begin{figure*}[th!]
    \centering
    % RIGHT images
    \begin{subfigure}[b]{0.1\textwidth}
        \centering
        \includegraphics[width=\linewidth]{plots/attack_images/Best_right_snowball_1_patch_0.4_angle_180.png}
        \caption{\footnotesize \centering Turn Right\newline Snowball 1}
        \label{fig:right_snowball_1}
    \end{subfigure}
    \begin{subfigure}[b]{0.1\textwidth}
        \centering
        \includegraphics[width=\linewidth]{plots/attack_images/Best_right_snowball_2_patch_0.4_angle_180.png}
        \caption{\footnotesize \centering Turn Right\newline Snowball 2}
        \label{fig:right_snowball_2}
    \end{subfigure}
    \begin{subfigure}[b]{0.1\textwidth}
        \centering
        \includegraphics[width=\linewidth]{plots/attack_images/Best_right_snowball_3_patch_0.3_angle_315.png}
        \caption{\footnotesize \centering Turn Right\newline Snowball 3}
        \label{fig:right_snowball_3}
    \end{subfigure}
    \begin{subfigure}[b]{0.1\textwidth}
        \centering
        \includegraphics[width=\linewidth]{plots/attack_images/Best_right_snowball_4_patch_0.4_angle_270.png}
        \caption{\footnotesize \centering Turn Right\newline Snowball 4}
        \label{fig:right_snowball_4}
    \end{subfigure}
    \begin{subfigure}[b]{0.1\textwidth}
        \centering
        \includegraphics[width=\linewidth]{plots/attack_images/Best_right_snowball_5_patch_0.4_angle_225.png}
        \caption{\footnotesize \centering Turn Right\newline Snowball 5}
        \label{fig:right_snowball_5}
    \end{subfigure}
    \begin{subfigure}[b]{0.1\textwidth}
        \centering
        \includegraphics[width=\linewidth]{plots/attack_images/Best_right_snowball_6_patch_0.2_angle_135.png}
        \caption{\footnotesize \centering Turn Right\newline Snowball 6}
        \label{fig:right_snowball_6}
    \end{subfigure}
    \begin{subfigure}[b]{0.1\textwidth}
        \centering
        \includegraphics[width=\linewidth]{plots/attack_images/Best_right_snowball_7_patch_0.4_angle_180.png}
        \caption{\footnotesize \centering Turn Right\newline Snowball 7}
        \label{fig:right_snowball_7}
    \end{subfigure}
    \begin{subfigure}[b]{0.101\textwidth}
        \centering
        \includegraphics[width=\linewidth]{plots/attack_images/Best_right_snowball_8_patch_0.4_angle_90.png}
        \caption{\footnotesize \centering Turn Right\newline Snowball 8}
        \label{fig:right_snowball_8}
    \end{subfigure}
    \begin{subfigure}[b]{0.1\textwidth}
        \centering
       \includegraphics[width=\linewidth]{plots/attack_images/Best_right_snowball_9_patch_0.4_angle_90.png}
        \caption{\footnotesize \centering Turn Right\newline Snowball 9}
        \label{fig:right_snowball_9}
    \end{subfigure}
\caption{ Turn Right Snowball Adversarial images.}
    \label{fig:right_snowball_adversarial_image}
\end{figure*}


\subsection{StreetView baseline Performance}

The evaluation is conducted on 5 randomly selected Street View images -- Stop, Yield, Pedestrian Crossing, Merge and Turn Right. LISA-CNN performs impressively, with correctly predicted images achieving an average confidence score exceeding 81\% for the five images although it was never trained on Street View images.
 The Stop sign had a confidence score of 85.42\%, the Yield sign had 68.30\%, the Pedestrian Crossing sign had 86.74\%, the Merge sign had 76.51\%, and the Turn Right sign had the highest confidence at 94.85\%. These scores provide a reference point for evaluating the effects of adversarial modifications.
% Detailed confidence scores for these street signs are provided in Table~\ref{table_adversarial}.


\begin{table*}[t]
\centering
\caption{Confidence scores of the best universal adversarial images.}
% \begin{adjustbox}{width=0.98\textwidth}
\label{table_adversarial_timing}
\small
\begin{tabular}{|p{1.5cm}|p{1.5cm}|p{1.5cm}|p{1.5cm}|p{1.5cm}|p{1.5cm}|p{1.5cm}|}
\hline 
\textbf{Adversarial Image} & \textbf{One Sticker Black}  & \textbf{One Sticker  White} & \textbf{Two Sticker Black, Black} & \textbf{Two Sticker Black, White} & \textbf{Two Sticker White, Black} & \textbf{Two Sticker White, White} 
\\ \hline \hline
Stop & 86.85 & 37.79 & 60.67 & 49.17& 67.25 & 25.98   \\ \hline 
Yield  & 65.44 & $\times$ & 55.24  & 23.70& 58.87& $\times$ \\ \hline 
Merge & 88.42 & 90.49 & 75.27 &95.41  &87.05 &91.50  \\ \hline 
\end{tabular}
% \end{adjustbox}
\end{table*}



\begin{table*}[t]
\centering
\caption{Predicted labels of the adversarial images. In all but two cases the attack worked. The labels correspond to the confidences shown in entries in Table~\ref{table_adversarial_confidence}.}
\begin{adjustbox}{width=0.98\textwidth}
\label{table_adversarial_labels}
\small
\begin{tabular}{|p{1.5cm}|p{1.5cm}|p{1.5cm}|p{1.5cm}|p{1.5cm}|p{1.5cm}|p{1.5cm}|p{1.5cm}|p{1.5cm}|p{1.5cm}|}
\hline 
\textbf{Adversarial Image} & \textbf{Snowball 1}  & \textbf{Snowball 2} & \textbf{Snowball 3} & \textbf{Snowball 4} & \textbf{Snowball 5} & \textbf{Snowball 6} & \textbf{Snowball 7} & \textbf{Snowball 8} & \textbf{Snowball 9}
\\ \hline \hline
Stop & Speed Limit 25 & Yield& Speed Limit 25& Yield& Speed Limit 45& Signal Ahead& Speed Limit 25& Speed Limit 25& Speed Limit 25\\ \hline 
Yield  & Speed Limit 35 & Speed Limit 35& Speed Limit 35& Speed Limit 35& Ped. Crossing& Ped. Crossing& -& Speed Limit 35& - \\ \hline
Ped. Crossing  & Stop Ahead & Stop Ahead& Stop Ahead& Stop Ahead& Stop Ahead& Stop Ahead& Stop Ahead& Stop Ahead& Stop Ahead \\ \hline  
Merge & Ped. Crossing & Ped. Crossing & Ped. Crossing & Ped. Crossing & Ped. Crossing & Ped. Crossing & Ped. Crossing & Ped. Crossing & Ped. Crossing   \\ \hline 
Turn Right & Stop & Stop & Added Lane & Stop & Added Lane & Ped. Crossing & Stop & Stop & Stop \\ \hline  
\end{tabular}
\end{adjustbox}
\end{table*}





\subsection{Snowball Baseline Adversarial Attack}

Table~\ref{table_adversarial_confidence} shows the confidence scores for adversarial images generated by the Snowball Baseline Adversarial Attack, while Table~\ref{table_adversarial_labels} details their corresponding predicted labels across nine iterations. For instance, the ``Stop'' image exhibits considerable variability -- with predictions oscillating between ``Speed Limit 25'', ``Yield'', ``Speed Limit 45'', and ``Signal Ahead'' -- reflecting the fluctuating confidence scores that range from as low as 33.40\% to as high as 82.38\%. Similarly, the ``Yield'' image is mostly misclassified as ``Speed Limit 35'', though occasional outputs like ``Ped. Crossing'' and missing predictions hint at instability under adversarial conditions. In contrast, the ``Ped. Crossing'' image consistently yields ``Stop Ahead'' across all iterations, accompanied by high confidence levels (81.97\% to 93.73\%), indicating a systematic misclassification. The ``Merge'' image is uniformly labeled as ``Ped. Crossing'' with exceptionally high confidence (over 96\%), and the ``Turn Right'' image presents a mix of predictions -- primarily ``Stop'' with intermittent outputs like ``Added Lane'' and ``Ped. Crossing''. Overall, these variations not only demonstrate the Snowball attack's capability to induce misclassifications but also highlight the nuanced behavior of the model's decision boundaries in response to adversarial perturbations.

\subsection{Snowball Optimized Adversarial Attack}
The Snowball Optimized Adversarial Attack demonstrated varying degrees of performance improvement (i.e. lower or faster time is better) across different traffic signs at the 75\%, 50\%, and 25\% thresholds, as presented in Tables \ref{table_adversarial_timing_75}, \ref{table_adversarial_timing_50}, and~\ref{table_adversarial_timing_25}. 

At the 75\% threshold, the Stop sign shows a gradual decrease in timing, with the most significant reduction of 14\% in Snowball 9. Yield signs exhibit an initial decrease, followed by increases in timing from Snowball 2 to Snowball 7, with the highest positive change of 17\% in Snowball 2. Pedestrian Crossing signs consistently experience timing reductions, with the largest drop of 19\% in Snowball 9. The Merge and Turn Right signs also show reductions, with Merge timing decreasing by 20\% in Snowball 1 and Turn Right experiencing a peak decline of 21\% in Snowball 3.



\begin{table*}[th!]
\centering
\caption{Timing of finding the best location for the placement of snowballs on each of the images. Time units are seconds (s).}
\begin{adjustbox}{width=0.98\textwidth}
\label{table_adversarial_timing}
\small
\begin{tabular}{|p{1.5cm}|p{1.5cm}|p{1.5cm}|p{1.5cm}|p{1.5cm}|p{1.5cm}|p{1.5cm}|p{1.5cm}|p{1.5cm}|p{1.5cm}|}
\hline 
\textbf{Adversarial Image} & \textbf{Snowball 1}  & \textbf{Snowball 2} & \textbf{Snowball 3} & \textbf{Snowball 4} & \textbf{Snowball 5} & \textbf{Snowball 6} & \textbf{Snowball 7} & \textbf{Snowball 8} & \textbf{Snowball 9}
\\ \hline \hline
Stop & 5573  & 5608 & 5670 & 5671& 5716 & 5767 &5726&5743&5746 \\ \hline 
Yield  & 1010  &996 &991  & 1006& 987&1010 &1027 &1112  & 1125 \\ \hline 
Ped. Crossing  &3403  &3437 &3460 &3389 &3377 &3405&3443 &3709 &3635 \\ \hline 
Merge & 733 &757 &746 &745  &760 &770 &786 &718 &710 \\ \hline 
Turn Right & 1896 & 1897&1923  &1787 &1686 &1707 &1735 &1723 &1728 \\ \hline  
\end{tabular}
\end{adjustbox}
\end{table*}



\begin{table*}[th!]
\centering
\caption{Timing of finding the best location for the placement of snowballs on each of the images when 75\% of the mask area is used after finding the optimal point for the first snowball adversarial image. Time units are seconds (s). Lower value and negative \% are better.}
\begin{adjustbox}{width=0.98\textwidth}
\label{table_adversarial_timing_75}
\small
\begin{tabular}{|p{1.5cm}|p{1.5cm}|p{1.5cm}|p{1.5cm}|p{1.5cm}|p{1.5cm}|p{1.5cm}|p{1.5cm}|p{1.5cm}|p{1.5cm}|}
\hline 
\textbf{Adversarial Image} & \textbf{Snowball 1}  & \textbf{Snowball 2} & \textbf{Snowball 3} & \textbf{Snowball 4} & \textbf{Snowball 5} & \textbf{Snowball 6} & \textbf{Snowball 7} & \textbf{Snowball 8} & \textbf{Snowball 9}
\\ \hline \hline
Stop & 5630 \newline (+1\%)  & 5110 \newline  (-9\%) & 5052 \newline (-11\%) & 5087 \newline (-10\%)& 5063 \newline (-11\%)  & 5043 \newline (-13\%) &5236 \newline (-9\%)
&5000 \newline (-13\%)
&4953 \newline (-14\%) \\ \hline 
Yield  & 947 \newline (-6\%) &1169 \newline (+17\%)&1129 \newline (+14\%)  & 1175 \newline (+17\%)& 1152 \newline (+17\%) &1169 \newline (+16\%) &1182 \newline (+15\%) &1173 \newline (+6\%)  &  1155 \newline (+3\%)\\ \hline 
Ped. Crossing  &3230 \newline (-5\%) &2901 \newline (-16\%)
&2927 \newline (-15\%) &2893 \newline (-15\%)  &2954 \newline (-13\%) &2926 \newline (-14\%) &2951 \newline (-14\%)  &3046 \newline (-18\%) &2958 \newline (-19\%)  \\ \hline 
Merge 
&583 \newline (-20\%) &613 \newline (-19\%)  &636 \newline (-15\%)   &655 \newline (-12\%)  
&643 \newline (-15\%) 
&653 \newline (-15\%)  &661 \newline (-16\%) &653 \newline (-9\%)  &645 \newline (-9\%)  \\ \hline 
Turn Right 
&1757 \newline (-7\%)  &1573 \newline (-17\%) &1529 \newline (-21\%) &1587 \newline (-11\%) &1552 \newline (-8\%) &1566 \newline (-8\%) &1571 \newline (-9\%)  &1556 \newline (-10\%)  &1581 \newline (-9\%)  \\ \hline  
\end{tabular}
\end{adjustbox}
\end{table*}



\begin{table*}[th!]
\centering
\caption{Timing of finding the best location for the placement of snowballs on each of the images when 50\% of the mask area is used after finding the optimal point for the first snowball adversarial image. Time units are seconds (s). Lower value and negative \% are better.}
\begin{adjustbox}{width=0.98\textwidth}
\label{table_adversarial_timing_50}
\small
\begin{tabular}{|p{1.5cm}|p{1.5cm}|p{1.5cm}|p{1.5cm}|p{1.5cm}|p{1.5cm}|p{1.5cm}|p{1.5cm}|p{1.5cm}|p{1.5cm}|}
\hline 
\textbf{Adversarial Image} & \textbf{Snowball 1}  & \textbf{Snowball 2} & \textbf{Snowball 3} & \textbf{Snowball 4} & \textbf{Snowball 5} & \textbf{Snowball 6} & \textbf{Snowball 7} & \textbf{Snowball 8} & \textbf{Snowball 9}
\\ \hline \hline
Stop &5698 \newline (+2\%)  &3137 \newline (-44\%)  &3202 \newline (-44\%) 
&3141 \newline (-45\%)& 3183 \newline  (-44\%)  &3208 \newline (-44\%) & 3204 \newline  (-44\%)
& 3255 \newline (-44\%)
& 3224 \newline  (-44\%)\\ \hline 
Yield  &1035 \newline (+2\%)  
&911 \newline (-8\%) &907 \newline (-8\%) & 920 \newline  (-9\%)& 909 \newline (-8\%)
&905 \newline (-10\%) &913 \newline (-11\%) &910 \newline (-18\%)&  918 \newline  (-18\%)\\ \hline 
Ped. Crossing  &3287 \newline  (-3\%)
&1732 \newline (-50\%) &1743 \newline (-50\%) &1779 \newline (-48\%) &1704 \newline (-50\%) &1633 \newline (-52\%) &1648 \newline (-52\%)  &1775 \newline (-52\%) &1695 \newline (-53\%) \\ \hline 
Merge & 800 \newline (+9\%) &623 \newline (-18\%) &601 \newline (-20\%) &636 \newline (-15\%) &713 \newline (-6\%) &712 \newline (-8\%) &728 \newline (-7\%) &730 \newline (+2\%) &731 \newline (+3\%)\\ \hline 
Turn Right & 1794 \newline (-5\%) 
&685 \newline (-64\%) &642 \newline (-67\%) &630 \newline (-65\%) &606 \newline (-64\%) &622 \newline (-64\%) &621 \newline (-64\%) &643 \newline (-63\%) &635 \newline (-63\%)\\ \hline  
\end{tabular}
\end{adjustbox}
\end{table*}



\begin{table*}[th!]
\centering
\caption{Timing of finding the best location for the placement of snowballs on each of the images when 25\% of the mask area is used after finding the optimal point for the first snowball adversarial image. Time units are seconds (s). Lower value and negative \% are better.}
\begin{adjustbox}{width=0.98\textwidth}
\label{table_adversarial_timing_25}
\small
\begin{tabular}{|p{1.5cm}|p{1.5cm}|p{1.5cm}|p{1.5cm}|p{1.5cm}|p{1.5cm}|p{1.5cm}|p{1.5cm}|p{1.5cm}|p{1.5cm}|}
\hline 
\textbf{Adversarial Image} & \textbf{Snowball 1}  & \textbf{Snowball 2} & \textbf{Snowball 3} & \textbf{Snowball 4} & \textbf{Snowball 5} & \textbf{Snowball 6} & \textbf{Snowball 7} & \textbf{Snowball 8} & \textbf{Snowball 9}
\\ \hline \hline
Stop & 4937 \newline (-11\%)  
&757 \newline (-87\%) 
&751 \newline (-87\%) 
&776 \newline (-86\%) & 786 \newline  (-86\%)  &789 \newline (-86\%) & 794 \newline (-86\%) 
& 790 \newline (-86\%) 
& 791 \newline (-86\%)  \\ \hline 
Yield  & 1285 \newline (+27\%)   
&508 \newline (-49\%)  &474 \newline (-52\%)  & 419 \newline (-58\%) & 388 \newline  (-61\%) &383 \newline (-62\%) &371 \newline (-64\%) &369 \newline (-67\%)  & 394 \newline (-65\%) \\ \hline 
Ped. Crossing  &3184 \newline  (-6\%)  
&654 \newline (-81\%)  &665 \newline (-81\%) &677 \newline (-80\%) &638 \newline (-81\%) &606 \newline (-82\%)  &619 \newline (-82\%)  &615 \newline (-83\%) &589 \newline (-84\%)  \\ \hline 
Merge & 650 \newline (-11\%) 
&233 \newline (-69\%)   &248 \newline (-67\%) &240 \newline (-68\%) &236 \newline (-69\%) &246 \newline (-68\%) &251 \newline (-68\%) &248 \newline (-65\%) &242 \newline (-66\%)\\ \hline 
Turn Right & 1662 \newline (-12\%) 
&189 \newline (-91\%) &183 \newline (-90\%) &182 \newline (-90\%) &172 \newline (-90\%) &168 \newline  (-90\%) &176 \newline (-90\%) &176 \newline  (-90\%)  &180 \newline  (-90\%)\\ \hline  
\end{tabular}
\end{adjustbox}
\end{table*}







At the 50\% threshold, Stop signs timing initially increase by 2\% in Snowball 1, followed by a substantial decline, with the largest reduction of 45\% in Snowball 4. Yield signs show an initial increase of 2\% in timing, then decrease progressively, with the largest drop of 18\% in Snowball 9. Pedestrian Crossing signs experience severe improvements in timing, peaking at 53\% drop in Snowball 9, following a 3\% decrease in Snowball 1. Merge signs show a slight increase of 9\% in Snowball 1, followed by moderate reductions, with a recovery of a 3\% timing increase in Snowball 9. Turn Right signs experience extreme timing reductions, with the most significant drop of 67\% in Snowball 3, stabilizing around 63\% in later iterations.

At the 25\% threshold, Stop signs begin with a timing reduction of 11\% in Snowball 1, followed by a sharp decline, with the most notable reduction of 87\% in Snowball 3. Yield signs show an initial increase in time of 27\% in Snowball 1, but drop consistently in later iterations, with the largest decrease of 67\% in Snowball 8. Pedestrian Crossing signs experience a similar pattern, with an initial reduction of 6\%, followed by drastic decreases in time, culminating in a 84\% reduction in Snowball 9. Merge signs exhibit steady decreases, starting with 11\% in Snowball 1 and continuing to 66\% in Snowball 9. Turn Right signs show the most severe timing reductions, starting with a 12\% drop in Snowball 1 and reaching 91\% in Snowball 2, with only slight fluctuations thereafter, stabilizing at around 90\% in Snowball 9.

% These results highlight the heterogeneous impact of the Snowball Optimized Adversarial Attack across all traffic sign categories. 
While signs such as Stop and Yield show moderate improvement in timing at the 75\% and 50\% thresholds, they exhibit more severe performance improvements at the 25\% threshold, particularly in later Snowball iterations. Pedestrian Crossing and Turn Right signs experience consistent and substantial reductions in timing across all thresholds, with the most extreme improvement observed at the 25\% threshold.
The findings underscore the increasing effectiveness of the attack as the threshold decreases, leading to progressively more significant timing improvement, especially in signs subjected to severe changes.




This work identifies signal collapse as a critical bottleneck in one-shot neural network pruning. Performance loss in pruned networks is due to \textbf{signal collapse} in addition to the removal of critical parameters. We propose \textbf{REFLOW} (\textbf{Re}storing \textbf{F}low of \textbf{Low}-variance signals), a simple yet effective method that mitigates signal collapse without computationally expensive weight updates. By focusing on signal preservation, REFLOW highlights the importance of mitigating signal collapse in sparse networks and enables magnitude pruning to match or surpass state-of-the-art one-shot pruning methods such as CHITA, CBS, and WF.

REFLOW consistently achieves state-of-the-art accuracy across diverse architectures, restoring ResNeXt-101 from under 4.1\% to 78.9\% top-1 accuracy at 80\% sparsity on ImageNet. Its lightweight design makes it a practical solution for both research and deployment, delivering high-quality sparse models without the overhead of traditional approaches. These findings challenge the traditional emphasis on weight selection strategies and underscore the critical role of signal propagation for achieving high-quality sparse networks in the context of one-shot pruning.




% \clearpage



\section*{Impact Statement}

This paper presents work whose goal is to advance the field of 
Machine Learning. There are many potential societal consequences 
of our work, none which we feel must be specifically highlighted here.


\bibliography{References}
\bibliographystyle{icml2025}




%%%%%%%%%%%%%%%%%%%%%%%%%%%%%%%%%%%%%%%%%%%%%%%%%%%%%%%%%%%%%%%%%%%%%%%%%%%%%%%
%%%%%%%%%%%%%%%%%%%%%%%%%%%%%%%%%%%%%%%%%%%%%%%%%%%%%%%%%%%%%%%%%%%%%%%%%%%%%%%
% APPENDIX
%%%%%%%%%%%%%%%%%%%%%%%%%%%%%%%%%%%%%%%%%%%%%%%%%%%%%%%%%%%%%%%%%%%%%%%%%%%%%%%
%%%%%%%%%%%%%%%%%%%%%%%%%%%%%%%%%%%%%%%%%%%%%%%%%%%%%%%%%%%%%%%%%%%%%%%%%%%%%%%
\newpage
\appendix
\onecolumn

%%%%%%%%%%%%%%%%%%%%%%%%%%%%%%%%%%%%%%%%%%%%%%%%%%%%%%%%%%%%%%%%%%%%%%%%%%%%%%%
%%%%%%%%%%%%%%%%%%%%%%%%%%%%%%%%%%%%%%%%%%%%%%%%%%%%%%%%%%%%%%%%%%%%%%%%%%%%%%%
\vspace{1cm}
\hrule
\par\vspace{0.5cm}
{\Large\bfseries\centering \textsc 
{Supplementary Materials}
\par\vspace{0.5cm}}
\hrule
\vspace{0.5cm}
\noindent These supplementary materials provide additional details, derivations, and experimental results for our paper. The appendix is organized as follows:
\begin{itemize}[leftmargin=1em]
    \item Section \ref{sec:related_work_extended} provides a more comprehensive overview of the related literature.
    
    \item Section \ref{sec:theory_opt_select_extended} provides theoretical analysis of the equivalence of the optimal selection integer program and the reward maximization objective.

    \item Section \ref{sec:local_search_kmedoids} shows a constant factor approximation for the coordinate descent algorithm in polynomial time.
    
    \item Section \ref{sec:constant_factor_subset_selection} provides theoretical guarantees for our k-means style coreset selection algorithm.
    
    \item Section \ref{sec:optimal_selection_computation} provides the code for computation of the optimal selection algorithm.
    
    \item Section \ref{sec:tsne_visualization} provides t-sne plots for the various queries highlighting the performance of our algorithms.
    
    % \item Section \ref{sec:reward_loss_computation} provides the implementation details of the reward loss computation, including the actual code used in our experiments.    
\end{itemize}
\vspace{0.5cm}

\section{Related Work}
\label{sec:related_work_extended}

\paragraph{Preference Optimization in RLHF.}
Direct Preference Optimization (DPO) is a collection of techniques for fine-tuning language models based on human preferences \cite{rafailov2024direct}. Several variants of DPO have been developed to address specific challenges and improve its effectiveness \cite{ethayarajh2024kto,zeng2024token,dong2023raft,yuan2023rrhf}. For example, KTO and TDPO focus on different aspects of preference optimization, while RAFT and RRHF utilize alternative forms of feedback. Other variants, such as SPIN, CPO, ORPO, and SimPO, introduce additional objectives or regularizations to enhance the optimization process \cite{chen2024self,xu2024contrastive,hong2024orpo,meng2024simpo}.

Further variants, including R-DPO, LD-DPO, sDPO, IRPO, OFS-DPO, and LIFT-DPO, address issues like length bias, training strategies, and specific reasoning tasks. These diverse approaches demonstrate the ongoing efforts to refine and enhance DPO, addressing its limitations and expanding its applicability to various tasks and domains \cite{park2024disentangling,liu2024iterative,pang2024iterative,qi2024online, yuan2024following}.
% Reinforcement Learning from Human Feedback (RLHF) \citep{christiano2017deep, ziegler2019fine, ouyang2022training} typically focuses on pairwise preference comparisons between model outputs, training a reward model to distinguish “better” vs.\ “worse” responses and then fine-tuning a policy to maximize that reward. This preference-based approach has been widely adopted in language model alignment \citep{bai2022training, stiennon2020learning}, but it can oversimplify the range of human preferences by reducing them to binary comparisons.

\paragraph{Multi-Preference Approaches.}
Recent work extends standard RLHF to consider entire \emph{sets} of responses at once, enabling more nuanced feedback signals \citep{rafailov2024direct, cui2023ultrafeedback, chen2024noise}. Group-based objectives capture multiple acceptable (and multiple undesirable) answers for each query, rather than only a single “better vs.\ worse” pair. \citet{gupta2024swepo} propose a contrastive formulation, \swepo, that jointly uses multiple “positives” and “negatives.” Such multi-preference methods can reduce label noise and better reflect the complexity of real-world tasks, but their computational cost grows if one attempts to incorporate all generated outputs \citep{cui2023ultrafeedback, chen2024noise}.

\paragraph{On-Policy Self-Play.}
A key advancement in reinforcement learning has been \emph{self-play} or on-policy generation, where the model continuously updates and re-generates data from its own evolving policy \citep{silver2016mastering, silver2017mastering}. In the context of LLM alignment, on-policy sampling can keep the training set aligned with the model’s current distribution of outputs \citep{christiano2017deep, wu2023fine}. However, this approach can significantly inflate the number of candidate responses, motivating the need for selective down-sampling of training examples.

\paragraph{Active Learning for Policy Optimization.}
The notion of selectively querying the most informative examples is central to \emph{active learning} \citep{cohn1996active, settles2009active}, which aims to reduce labeling effort by focusing on high-utility samples. Several works incorporate active learning ideas into reinforcement learning, e.g., uncertainty sampling or diversity-based selection \citep{sener2017active, zhang2022active}. In the RLHF setting, \citet{christiano2017deep} highlight how strategic feedback can accelerate policy improvements, while others apply active subroutines to refine reward models \citep{wu2023fine}. By picking a small yet diverse set of responses, we avoid both computational blow-ups and redundant training signals.

\paragraph{Clustering and Coverage-Based Selection.}
Selecting representative subsets from a large dataset is a classic problem in machine learning and combinatorial optimization. \emph{Clustering} techniques such as $k$-means and $k$-medoids \citep{hartigan1979algorithm} aim to group points so that distances within each cluster are small. In the RLHF context, embedding model outputs and clustering them can ensure \emph{coverage} over semantically distinct modes \citep{har2004coresets, cohen2022improved}. These methods connect to the \emph{facility location} problem \citep{oh2017deep}—minimizing the cost of “covering” all points with a fixed number of centers—and can be addressed via coreset construction \citep{feldman2020core}. 

\paragraph{Min-Knapsack and Integer Programming.}
When picking a subset of size $k$ to cover or suppress “bad” outputs, one may cast the objective in a \emph{min-knapsack} or combinatorial optimization framework \citep{kellerer2004introduction}. For instance, forcing certain outputs to zero probability can impose constraints that ripple to nearby points in embedding space, linking coverage-based strategies to integer programs \citep{chen2020big}. \citet{cohen2022improved} and \citet{har2004coresets} demonstrate how approximate solutions to such subset selection problems can achieve strong empirical results in high-dimensional scenarios. By drawing from these established concepts, our method frames the selection of negative samples in a Lipschitz coverage sense, thereby enabling both theoretical guarantees and practical efficiency in multi-preference alignment.


Collectively, our work stands at the intersection of \emph{multi-preference alignment} \citep{gupta2024swepo, cui2023ultrafeedback}, \emph{on-policy data generation} \citep{silver2017mastering, ouyang2022training}, and \emph{active learning} \citep{cohn1996active, settles2009active}. We leverage ideas from \emph{clustering} (k-means, k-medoids) and \emph{combinatorial optimization} (facility location, min-knapsack) \citep{kellerer2004multidimensional, cacchiani2022knapsack} to construct small yet powerful training subsets that capture both reward extremes and semantic diversity. The result is an efficient pipeline for aligning LLMs via multi-preference signals without exhaustively processing all generated responses.

\section{Extended Theoretical Analysis of \textsc{Opt-Select}}
\label{sec:theory_opt_select_extended}

In this appendix, we present a more detailed theoretical treatment of $\ampoos$. We restate the core problem setup and assumptions, then provide rigorous proofs of our main results. Our exposition here augments the concise version from the main text.

\subsection{Problem Setup}

Consider a single prompt (query) \(x\) for which we have sampled \(n\) candidate responses \(\{\,y_1,\,y_2,\,\ldots,\,y_n\}\). Each response \(y_i\) has:
\begin{itemize}[itemsep=0.5em, leftmargin=1em]
    \item A scalar reward \(r_i \in [0,1]\).
    \item An embedding \(\mathbf{e}_i \in \mathbb{R}^d.\)
\end{itemize}
We define the distance between two responses \(y_i\) and \(y_j\) by
\begin{equation}
\label{eq:appdistdef}
A_{i,j} \;=\; \|\mathbf{e}_i \,-\, \mathbf{e}_j\|.
\end{equation}
We wish to learn a \emph{policy} \(\{p_i\}\), where \(p_i \ge 0\) and \(\sum_{i=1}^n p_i = 1\). The policy's \emph{expected reward} is
\begin{equation}
\label{eq:appexprew}
\mathrm{ER}(p) 
\;=\; 
\sum_{i=1}^n r_i \,p_i.
\end{equation}

\paragraph{Positive and Negative Responses.}
We designate exactly one response, denoted \(y_{i_{\mathrm{top}}}\), as a \emph{positive} (the highest-reward candidate). All other responses are potential ``negatives.'' Concretely:
\begin{itemize}[itemsep=0.5em, leftmargin=1em]
    \item We fix one index \(i_{\mathrm{top}}\) with \(\displaystyle i_{\mathrm{top}} \;=\; \arg \max_{i\in\{1,\dots,n\}}\,r_i.\)
    \item We choose a subset \(\mathcal{S}\subseteq \{1,\dots,n\}\setminus\{i_{\mathrm{top}}\}\) of size \(k\), whose elements are forced to have \(p_j=0\). (These are the ``negatives.'')
\end{itemize}

\subsubsection{Lipschitz Suppression Constraint}
\label{subsec:LipschitzConstraintApp}

We assume a mild Lipschitz-like rule:
\begin{enumerate}[label=(A\arabic*), itemsep=0.5em]
    \item\label{asmp:Lipschitz} \textbf{\(L\)-Lipschitz Constraint.} If \(p_j = 0\) for some \(j\in \mathcal{S}\), then for every response \(y_i\), we must have
    \begin{equation}
    \label{eq:LipschitzConstraintApp}
    p_i 
    \;\le\; 
    L\, A_{i,j}
    \;=\;
    L\,\|\mathbf{e}_i \,-\,\mathbf{e}_j\|.
    \end{equation}
\end{enumerate}
The effect is that whenever we force a particular negative \(j\) to have \(p_j=0\), any response \(i\) near \(j\) in embedding space also gets \emph{pushed down}, since \(p_i \le L\,A_{i,j}\). By selecting a set of $k$ negatives covering many ``bad'' or low-reward regions, we curb the policy's probability of generating undesirable responses.

\paragraph{Goal.} 
Define the feasible set of distributions:
\begin{equation}
\label{eq:feasibleRegionApp}
\mathcal{F}(\mathcal{S}) 
\;=\; 
\Bigl\{\,
\{p_i\}\colon p_j=0 \ \forall\,j\in \mathcal{S}, \ 
p_i \le L\, \min_{j\in \mathcal{S}} A_{i,j}\ \forall\,i\notin\{\,i_{\mathrm{top}}\}\cup\mathcal{S}
\Bigr\}.
\end{equation}
We then have a two-level problem:
\begin{align}
\nonumber
&\max_{\,\substack{\mathcal{S}\,\subseteq \{1,\dots,n\}\setminus\{i_{\mathrm{top}}\}\\ |\mathcal{S}|=k}}  
\quad
\max_{\substack{\{p_i\}\in \mathcal{F}(\mathcal{S}) \\ \sum_i p_i = 1,\;p_i\ge 0}}
\quad
\sum_{i=1}^n r_i\, p_i,
\\[0.75em]
&\text{subject to}\quad p_{i_{\mathrm{top}}}\text{ is unconstrained (no Lipschitz bound).}
\label{eq:lip_mainObjApp}
\end{align}
We seek \(\mathcal{S}\) that \emph{maximizes} the best possible Lipschitz-compliant expected reward.

\subsection{Coverage View and the MIP Formulation}

\paragraph{Coverage Cost.}
To highlight the crucial role of ``covering'' low-reward responses, define a weight
\begin{equation}
\label{eq:appWeightDef}
w_i 
\;=\;
\exp\bigl(\,\overline{r} - r_i\bigr),
\end{equation}
where \(\overline{r}\) can be, for instance, the average reward \(\frac{1}{n}\sum_{i=1}^n r_i\). 
Then a natural \emph{coverage} cost is
\begin{equation}
\label{eq:coverageCostApp}
\mathrm{Cost}(\mathcal{S})
\;=\;
\sum_{i=1}^n
  w_i
  \,\min_{j\in \mathcal{S}}
    A_{i,j}.
\end{equation}
A small \(\min_{j\in \mathcal{S}} A_{i,j}\) means response \(i\) is ``close'' to at least one negative center \(j\). If \(r_i\) is low, then \(w_i\) is large, so we put higher penalty on leaving \(i\) uncovered. Minimizing \(\mathrm{Cost}(\mathcal{S})\) ensures that \emph{important} (low-reward) responses are forced near penalized centers, thus \emph{suppressing} them in the policy distribution.

\paragraph{MIP \(\mathcal{P}\) for Coverage Minimization.}

We can write a mixed-integer program:

\begin{align}
\label{eq:covMIPApp}
\nonumber
\textbf{Problem } \mathcal{P}:\;
&\min_{\,\substack{x_j \in \{0,1\}\\ z_{i,j}\in \{0,1\}\\ y_i \ge 0}} 
\sum_{i=1}^n 
  w_i\,y_i,
\\
&\text{subject to}
\begin{cases}
\displaystyle
\sum_{j=1}^n x_j = k, 
\\[0.2em]
z_{i,j}\le x_j,\quad 
\sum_{j=1}^n z_{i,j} = 1,\quad \forall\,i,
\\[0.2em]
y_i\le A_{i,j} + M\,(1 - z_{i,j}),
\\[0.2em]
y_i\ge A_{i,j} - M\,(1 - z_{i,j}),\quad \forall\,i,j,
\end{cases}
\end{align}
where \(M = \max_{i,j} A_{i,j}\). Intuitively, each \(x_j\) indicates if \(j\) is chosen as a negative; each \(z_{i,j}\) indicates whether \(i\) is ``assigned'' to \(j\). At optimality, \(y_i = \min_{j\in \mathcal{S}} A_{i,j}\), so the objective 
\(\sum_i w_i\,y_i\) is precisely \(\mathrm{Cost}(\mathcal{S})\). Hence solving \(\mathcal{P}\) yields \(\mathcal{S}^*\) that \emph{minimizes} coverage cost~\eqref{eq:coverageCostApp}.

\subsection{Key Lemma: Equivalence of Coverage Minimization and Lipschitz Suppression}

\begin{lemma}[Coverage $\Leftrightarrow$ Suppression]
\label{lem:appCoverageLemma}
Assume \ref{asmp:Lipschitz} (the \(L\)-Lipschitz constraint, \eqref{eq:LipschitzConstraintApp}) and let \(i_{\mathrm{top}}\) be a highest-reward index. Suppose \(\mathcal{S}\subseteq\{1,\dots,n\}\setminus\{i_{\mathrm{top}}\}\) is a subset of size~\(k\). Then:
\begin{enumerate}[label=(\roman*), leftmargin=1.25em, itemsep=0.5em]
    \item Choosing \(\mathcal{S}\) that \emph{minimizes} \(\mathrm{Cost}(\mathcal{S})\) yields the strongest suppression of low-reward responses and thus the best possible \emph{feasible} expected reward under the Lipschitz constraint.
    \item Conversely, any set \(\mathcal{S}\) achieving the \emph{highest} feasible expected reward necessarily \emph{minimizes} \(\mathrm{Cost}(\mathcal{S})\).
\end{enumerate}
\end{lemma}

\begin{proof}
\textbf{(i) Minimizing \(\mathrm{Cost}(\mathcal{S})\) improves expected reward.}\\
Once we pick \(\mathcal{S}\), we set \(p_j=0\) for all \(j\in \mathcal{S}\). By \ref{asmp:Lipschitz}, any \(y_i\) is then forced to satisfy \(p_i \le L\,A_{i,j}\) for all \(j\in \mathcal{S}\). Hence
\[
p_i 
\;\le\; 
L \,\min_{j\in \mathcal{S}} A_{i,j}.
\]
If \(\min_{j\in \mathcal{S}} A_{i,j}\) is large, then \(p_i\) could be large; if it is small (particularly for low-reward \(r_i\)), we effectively suppress \(p_i\). By weighting each \(i\) with \(w_i = e^{\overline{r}-r_i}\), we see that leaving low-reward \(y_i\) \emph{far} from all negatives raises the risk of high \(p_i\). Minimizing \(\sum_i w_i\,\min_{j\in \mathcal{S}} A_{i,j}\) ensures that any \(i\) with large \(w_i\) (i.e.\ small \(r_i\)) has a small distance to at least one chosen center, thus bounding its probability more tightly. 

Meanwhile, the best candidate \(i_{\mathrm{top}} \in \{1,\dots,n\}\) remains unconstrained, so the policy can always place mass \(\approx 1\) on \(i_{\mathrm{top}}.\) Consequently, a set \(\mathcal{S}\) that better ``covers'' low-reward points must yield a higher feasible expected reward \(\sum_i r_i p_i\). 

\textbf{(ii) Necessity of Minimizing \(\mathrm{Cost}(\mathcal{S})\).}\\
Conversely, if there were a set \(\mathcal{S}\) that \emph{did not} minimize \(\mathrm{Cost}(\mathcal{S})\) but still provided higher feasible expected reward, that would imply we found a distribution \(\{p_i\}\) violating the Lipschitz bound on some low-reward region. Formally, \(\mathcal{S}\) that yields strictly smaller coverage cost would impose stricter probability suppression on harmful responses. By part~(i), that coverage-lowering set should then yield an even higher feasible reward, a contradiction.
\end{proof}

\subsection{Main Theorem: Optimality of \(\mathcal{P}\) for Lipschitz Alignment}

\begin{theorem}[Optimal Negative Set via \(\mathcal{P}\)]
\label{thm:appOptNegatives}
Let \(\mathcal{S}^*\) be the solution to the MIP \(\mathcal{P}\) in \eqref{eq:covMIPApp}, i.e.\ it \emph{minimizes} \(\mathrm{Cost}(\mathcal{S})\). Then \(\mathcal{S}^*\) also \emph{maximizes} the objective \eqref{eq:lip_mainObjApp}. Consequently, picking \(\mathcal{S}^*\) and allowing free probability on \(i_{\mathrm{top}} \approx \arg\max_i\, r_i\) yields the \emph{optimal} Lipschitz-compliant policy.
\end{theorem}

\begin{proof}
By construction, solving \(\mathcal{P}\) returns \(\mathcal{S}^*\) with
\(\displaystyle
\mathrm{Cost}(\mathcal{S}^*)
\;=\;
\min_{|\mathcal{S}|=k}\,\mathrm{Cost}(\mathcal{S}).
\)
Lemma~\ref{lem:appCoverageLemma} then states that such an \(\mathcal{S}^*\) simultaneously \emph{maximizes} the best possible feasible expected reward. Hence \(\mathcal{S}^*\) is precisely the negative set that achieves the maximum of \eqref{eq:lip_mainObjApp}.
\end{proof}

\paragraph{Interpretation.} 
Under a mild Lipschitz assumption in embedding space, penalizing (assigning zero probability to) a small set \(\mathcal{S}\) \emph{and} forcing all items near \(\mathcal{S}\) to have small probability is equivalent to a \emph{coverage} problem. Solving (or approximating) \(\mathcal{P}\) selects negatives that push down low-reward modes as effectively as possible.

\vspace{-0.05in}
\subsection{Discussion and Practical Implementation}

\vspace{-0.05in}
\textsc{Opt-Select} thus emerges from optimizing coverage: 
\begin{enumerate}[leftmargin=1.25em, itemsep=0.5em]
    \item \textbf{Solve or approximate} the MIP \(\mathcal{P}\) to find the best subset \(\mathcal{S}\subseteq\{1,\dots,n\}\setminus\{i_{\mathrm{top}}\}\).
    \item \textbf{Force} \(p_j=0\) for each \(j\in \mathcal{S}\); \textbf{retain} \(i_{\mathrm{top}}\) with full probability (\(p_{i_{\mathrm{top}}}\approx 1\)), subject to normalizing the distribution. 
\end{enumerate}
In practice, local search or approximate clustering-based approaches (e.g.\ Weighted \(k\)-Medoids) can find good solutions without exhaustively solving \(\mathcal{P}\). The method ensures that near any chosen negative \(j\), all semantically similar responses \(i\) have bounded probability \(p_i \le L\,A_{i,j}\). Consequently, \textsc{Opt-Select} \emph{simultaneously} covers and suppresses undesired modes while preserving at least one high-reward response unpenalized.

\vspace{-0.05in}
\paragraph{Additional Remarks.}

\vspace{-0.15in}
\begin{itemize}[leftmargin=1.25em, itemsep=0.5em]
    \item The single-positive assumption reflects a practical design where one high-reward response is explicitly promoted. This can be extended to multiple positives, e.g.\ top \(m^+\) responses each unconstrained.
    \item For large \(n\), the exact MIP solution may be expensive; local search (see Appendix~\ref{sec:local_search_kmedoids}) still achieves a constant-factor approximation.
    \item The embedding-based Lipschitz constant \(L\) is rarely known exactly; however, the coverage perspective remains valid for “sufficiently smooth” reward behaviors in the embedding space.
\end{itemize}

Overall, these results solidify \textsc{Opt-Select} as a principled framework for negative selection under Lipschitz-based alignment objectives.

\section{Local Search Guarantees for Weighted \texorpdfstring{$k$}{k}-Medoids and Lipschitz-Reward Approximation}
\label{sec:local_search_kmedoids}

In this appendix, we show in \Cref{thm:local_search_kmedoids} that a standard \emph{local search} algorithm for \emph{Weighted $k$-Medoids} achieves a constant-factor approximation in polynomial time.


\subsection{Weighted \texorpdfstring{$k$}{k}-Medoids Setup}

We are given:
\begin{itemize}
\item A set of $n$ points, each indexed by $i\in\{1,\dots,n\}$.
\item A distance function $d(i,j)\ge0$, which forms a metric: $d(i,j)\le d(i,k)+d(k,j)$, $d(i,i)=0$, $d(i,j)=d(j,i)$.
\item A nonnegative \emph{weight} $w_i$ for each point $i$.
\item A budget $k$, $1\le k\le n$.
\end{itemize}
We wish to pick a subset $\mathcal{S}\subseteq\{1,\dots,n\}$ of \emph{medoids} (centers) with size $|\mathcal{S}|=k$ that minimizes the objective
\begin{align}
\label{eq:wkmedoids_objective}
\mathrm{Cost}(\mathcal{S})
\;=\;
\sum_{i=1}^n
  w_i
  \cdot
  \min_{\,j\in \mathcal{S}}\,
    d(i,j).
\end{align}
We call this the \textbf{Weighted $k$-Medoids} problem.  Note that \textbf{medoids} must come from among the data points, as opposed to $k$-median or $k$-means where centers can be arbitrary points in the metric or vector space. Our Algorithm \ref{alg:opt_select} reduces to exactly this problem.

\subsection{Coordinate Descent Algorithm via Local Search}

Our approach to the NP-hardness of Algorithm \ref{alg:opt_select} was to recast it as a simpler coordinate descent algorithm in Algorithm \ref{alg:opt_select_local_search}, wherein we do a local search at every point towards achieving the optimal solution.
Let $\textsc{Cost}(\mathcal{S})$ be as in \eqref{eq:wkmedoids_objective}.

\begin{enumerate}
\item \textbf{Initialize:} pick any subset $\mathcal{S}\subseteq\{1,\dots,n\}$ of size $k$ (e.g.\ random or greedy).
\item \textbf{Repeat}: Try all possible single \emph{swaps} of the form
\[
   \mathcal{S}' 
   \;=\; 
   \bigl(\,\mathcal{S}\setminus\{\,j\}\bigr)
   \,\cup\,
   \{\,j'\},
\]
where $j\in\mathcal{S}$ and $j'\notin\mathcal{S}$.  
\item \textbf{If any swap improves cost}: i.e.\ $\mathrm{Cost}(\mathcal{S}') < \mathrm{Cost}(\mathcal{S})$, then set $\mathcal{S}\leftarrow \mathcal{S}'$ and continue.
\item \textbf{Else terminate}: no single swap can further reduce cost.
\end{enumerate}

When the algorithm stops, we say $\mathcal{S}$ is a \emph{local optimum under 1-swaps}.

\subsection{Constant-Factor Approximation in Polynomial Time}

We now present and prove a result: such local search yields a constant-factor approximation.  Below, we prove a version with a \emph{factor 5} guarantee for Weighted $k$-Medoids.  Tighter analyses can improve constants, but 5 is a commonly cited bound for this simple variant.


\begin{theorem}[Local Search for Weighted $k$-Medoids]
\label{thm:local_search_kmedoids}
Let $\mathcal{S}^*$ be an \textbf{optimal} subset of medoids of size $k$. Let $\widehat{\mathcal{S}}$ be any \textbf{local optimum} obtained by the above 1-swap local search. Then
\begin{equation}
    \mathrm{Cost}\bigl(\widehat{\mathcal{S}}\bigr)
  \;\;\le\;\;
  5
  \,\times\,
  \mathrm{Cost}\bigl(\mathcal{S}^*\bigr).
\end{equation}

Moreover, the procedure runs in polynomial time (at most $\bigl(\binom{n}{k}\bigr)$ “worse-case” swaps in principle, but in practice each improving swap decreases cost by a non-negligible amount, thus bounding the iteration count).
\end{theorem}

\begin{proof}
\textbf{Notation.}
\begin{itemize}
\item Let $\widehat{\mathcal{S}}$ be the final local optimum of size $k$. 
\item Let $\mathcal{S}^*$ be an optimal set of size $k$. 
\item For each point $i$, define
\[
  r_i 
  \;=\; 
  d\!\bigl(i,\widehat{\mathcal{S}}\bigr)
  \;=\;
  \min_{j \in \widehat{\mathcal{S}}} d(i,j),
  \quad
  r_i^*
  \;=\;
  \min_{j\in \mathcal{S}^*} d(i,j).
\]
Thus $\mathrm{Cost}(\widehat{\mathcal{S}}) = \sum_i w_i\,r_i$ and $\mathrm{Cost}(\mathcal{S}^*) = \sum_i w_i\,r_i^*$.

\item Let $c(\mathcal{S}) = \sum_i w_i\,d(i,\mathcal{S})$ as shorthand for $\mathrm{Cost}(\mathcal{S})$. 
\end{itemize}

\noindent
\textbf{Step 1: Construct a ``Combined'' Set.}  
Consider 
\[
  \mathcal{S}^\dagger 
  \;=\;
  \widehat{\mathcal{S}}
  \;\cup\;
  \mathcal{S}^*.
\]
We have $|\mathcal{S}^\dagger|\le 2k$.  Let $c(\mathcal{S}^\dagger) = \sum_i w_i\,d\bigl(i,\mathcal{S}^\dagger\bigr)$.  

Observe that
\[
  d\!\bigl(i,\mathcal{S}^\dagger\bigr)
  \;=\;
  \min\!\bigl\{
    d\!\bigl(i,\widehat{\mathcal{S}}\bigr),\,
    d\!\bigl(i,\mathcal{S}^*\bigr)
  \bigr\}
  \;=\;
  \min\{\,r_i,\;r_i^*\}.
\]
Hence
\[
  c(\mathcal{S}^\dagger)
  \;=\;
  \sum_{i=1}^n 
    w_i\,
    \min\{\,r_i,\ r_i^*\}.
\]
We will relate $c(\mathcal{S}^\dagger)$ to $c(\widehat{\mathcal{S}})$ and $c(\mathcal{S}^*)$.

\medskip
\noindent
\textbf{Step 2: Partition Points According to $\mathcal{S}^*$.}  
For each $j^*\in \mathcal{S}^*$, define the cluster 
\[
  C(j^*)
  \;=\;
  \bigl\{
    i \mid j^* 
    = 
    \arg\min_{j'\in \mathcal{S}^*} d(i,j')
  \bigr\}.
\]
Hence $\{\,C(j^*)\,:\,j^*\in \mathcal{S}^*\}$ is a partition of $\{1,\dots,n\}$.  We now group the cost contributions by these clusters.

\medskip
\noindent
\textbf{Goal: Existence of a Good Swap.}
We will \emph{assume} $c(\widehat{\mathcal{S}})>5\,c(\mathcal{S}^*)$ and derive a contradiction by producing a \emph{profitable swap} that local search should have found.  

Specifically, we show that there must be a center $j^*\in \mathcal{S}^*$ whose cluster $C(j^*)$ is “costly enough” under $\widehat{\mathcal{S}}$, so that swapping out some center $j\in\widehat{\mathcal{S}}$ for $j^*$ significantly reduces cost.  But since $\widehat{\mathcal{S}}$ was a local optimum, no such profitable swap could exist.  This contradiction implies $c(\widehat{\mathcal{S}})\le 5\,c(\mathcal{S}^*)$.

\medskip
\noindent
\textbf{Step 3: Detailed Bounding.}

We have
\[
  c(\mathcal{S}^\dagger)
  =
  \sum_{i=1}^n
    w_i\,\min\{r_i,\,r_i^*\}
  \;\le\;
  \sum_{i=1}^n
    w_i\,r_i^*
  =
  c(\mathcal{S}^*).
\]
Similarly, 
\[
  c(\mathcal{S}^\dagger)
  \;\le\;
  \sum_{i=1}^n
    w_i\,r_i
  =
  c\!\bigl(\widehat{\mathcal{S}}\bigr).
\]
Hence $c(\mathcal{S}^\dagger)\le\min\bigl\{c(\widehat{\mathcal{S}}),\,c(\mathcal{S}^*)\bigr\}$.  
Now define
\[
   D
   \;=\;
   \sum_{i=1}^n
     w_i
     \,\bigl[
       r_i
       -
       \min\{\,r_i,\,r_i^*\}
     \bigr]
   \;=\;
   \sum_{i=1}^n
     w_i\,\bigl(r_i - r_i^*\bigr)_{+},
\]
where $(x)_{+}=\max\{x,0\}$.  By rearranging,
\[
  \sum_{i=1}^n w_i\,r_i
  \;-\;
  \sum_{i=1}^n w_i\,\min\{\,r_i,\,r_i^*\}
  \;=\;
  D.
\]
Thus
\[
  c(\widehat{\mathcal{S}}) - c(\mathcal{S}^\dagger)
  \;=\;
  D
  \;\;\ge\;\;
  c(\widehat{\mathcal{S}}) - c(\mathcal{S}^*).
\]
So
\[
  D
  \;\ge\;
  c\!\bigl(\widehat{\mathcal{S}}\bigr)
  \;-\;
  c\!\bigl(\mathcal{S}^*\bigr).
\]
Under the assumption $c(\widehat{\mathcal{S}})>5\,c(\mathcal{S}^*)$, we get 
\[
  D
  \;>\;
  4\,c(\mathcal{S}^*).
  \tag{*}
\]

\medskip
\noindent
\textbf{Step 4: Find a Center $j^*$ with Large $D$ Contribution.}
We now “distribute” $D$ over clusters $C(j^*)$.  Let
\[
  D_{j^*}
  =
  \sum_{i \in C(j^*)}
    w_i\,\bigl(r_i - r_i^*\bigr)_{+}.
\]
Then 
\(\displaystyle
D=\sum_{j^*\in \mathcal{S}^*} D_{j^*}.
\)
Since $D>4\,c(\mathcal{S}^*)$, at least one $j^*\in \mathcal{S}^*$ satisfies
\[
  D_{j^*}
  \;>\;
  4\,
  \frac{c(\mathcal{S}^*)}{|\mathcal{S}^*|}
  \;=\;
  4\,\frac{c(\mathcal{S}^*)}{k},
\]
because $|\mathcal{S}^*|=k$.  Denote this center as $j^*_{\text{large}}$ and its cluster $C^* = C(j^*_{\text{large}})$.

\medskip
\noindent
\textbf{Step 5: Swapping $j^*$ into $\widehat{\mathcal{S}}$.}
Consider the swap
\[
  \widehat{\mathcal{S}}_{\mathrm{swap}}
  \;=\;
  \Bigl(
    \widehat{\mathcal{S}}\setminus\bigl\{\,j_{\mathrm{out}}\bigr\}
  \Bigr)
  \,\cup\,
  \bigl\{\,j^*_{\text{large}}\bigr\}
\]
where $j_{\mathrm{out}}$ is whichever center in $\widehat{\mathcal{S}}$ we choose to remove.  We must show that for an appropriate choice of $j_{\mathrm{out}}$, the cost $c(\widehat{\mathcal{S}}_{\mathrm{swap}})$ is at least $(r_i - r_i^*)_{+}$ smaller on average for the points in $C^*$, forcing a net cost reduction large enough to offset any potential cost increase for points outside $C^*$.

In detail, partition $\widehat{\mathcal{S}}$ into $k$ clusters under \emph{Voronoi} assignment:
\[
  \widehat{C}(j)
  \;=\;
  \bigl\{
    i : 
    j=\arg\min_{\,x\in\widehat{\mathcal{S}}} d(i,x)
  \bigr\},
  \quad
  j\in \widehat{\mathcal{S}}.
\]
Since $|\,\widehat{\mathcal{S}}|=k$, there must exist at least one $j_{\mathrm{out}}\in \widehat{\mathcal{S}}$ whose cluster $\widehat{C}(j_{\mathrm{out}})$ has weight
\(\displaystyle
\sum_{i\in\widehat{C}(j_{\mathrm{out}})} w_i
 \;\le\;
 \frac{1}{k}\,\sum_{i=1}^n w_i.
\)
We remove that $j_{\mathrm{out}}$ and add $j^*_{\text{large}}$.

\medskip
\noindent
\textbf{Step 6: Net Cost Change Analysis.}
After the swap, 
\[
   c\bigl(\widehat{\mathcal{S}}_{\mathrm{swap}}\bigr)
   -
   c\bigl(\widehat{\mathcal{S}}\bigr)
   \;=\;
   \underbrace{
     \Delta_{\mathrm{in}}
   }_{\text{improvement in }C^*}
   \;+\;
   \underbrace{
     \Delta_{\mathrm{out}}
   }_{\text{possible cost increase outside }C^*}.
\]
Points $i\in C^*$ can now be served by $j^*_{\text{large}}$ at distance $r_i^*(\le r_i)$, so 
\[
  \Delta_{\mathrm{in}}
  \;\le\;
  \sum_{i \in C^*} w_i\,
     \Bigl[
       d\bigl(i,\widehat{\mathcal{S}}_{\mathrm{swap}}\bigr)
       -
       d\bigl(i,\widehat{\mathcal{S}}\bigr)
     \Bigr]
  \;\le\;
  \sum_{i\in C^*} w_i
    \,\bigl(r_i^* - r_i\bigr).
\]
But recall $r_i^* \le r_i$ or $r_i^*\le r_i$; for $i\in C^*$, we specifically have $(r_i-r_i^*)_{+}$ is \emph{often} positive. Precisely:
\[
  \Delta_{\mathrm{in}}
  \;\le\;
  \sum_{i\in C^*} w_i\,\bigl(r_i^* - r_i\bigr)
  \;=\;
  -\,\sum_{i\in C^*} w_i\,\bigl(r_i - r_i^*\bigr).
\]
Hence
\[
  \Delta_{\mathrm{in}}
  \;\le\;
  -\sum_{i\in C^*} w_i\,(r_i - r_i^*)_{+}.
\]
On the other hand, some points outside $C^*$ may lose $j_{\mathrm{out}}$ as a center, which might increase their distances:
\[
  \Delta_{\mathrm{out}}
  \;=\;
  \sum_{i\notin C^*}
     w_i\,
     \Bigl[
       d\bigl(i,\widehat{\mathcal{S}}_{\mathrm{swap}}\bigr)
       -
       d\bigl(i,\widehat{\mathcal{S}}\bigr)
     \Bigr].
\]
Since each point can still use any other center in $\widehat{\mathcal{S}}\setminus\{\,j_{\mathrm{out}}\}$, 
\[
  d\!\bigl(i,\widehat{\mathcal{S}}_{\mathrm{swap}}\bigr)
  \;\le\;
  \min\!\bigl\{
    d\!\bigl(i,\widehat{\mathcal{S}}\setminus \{j_{\mathrm{out}}\}\bigr),\
    d\!\bigl(i,j^*_{\text{large}}\bigr)
  \bigr\}.
\]
Thus for each $i$, 
\[
  d\bigl(i,\widehat{\mathcal{S}}_{\mathrm{swap}}\bigr)
  \;\le\;
  d\bigl(i,\widehat{\mathcal{S}}\bigr)
\]
unless the \emph{only} center in $\widehat{\mathcal{S}}$ that served $i$ was $j_{\mathrm{out}}$. But the total weight of $\widehat{C}(j_{\mathrm{out}})$ is at most $\frac{1}{k}\sum_{i} w_i$.  Thus,
\[
  \Delta_{\mathrm{out}}
  \;\le\;
  \sum_{i\in \widehat{C}(j_{\mathrm{out}})} 
       w_i\,
       \Bigl[
         d\bigl(i,\widehat{\mathcal{S}}_{\mathrm{swap}}\bigr)
         -
         d\bigl(i,\widehat{\mathcal{S}}\bigr)
       \Bigr]
  \;\le\;
  \sum_{i\in \widehat{C}(j_{\mathrm{out}})} 
    w_i\,d\bigl(j_{\mathrm{out}},\,j^*_{\text{large}}\bigr),
\]
because $i$ is at distance at most $d(i,j_{\mathrm{out}})+d(j_{\mathrm{out}},j^*_{\text{large}})$ to $j^*_{\text{large}}$. And $d(i,\widehat{\mathcal{S}})\ge d(i,j_{\mathrm{out}})$ by definition of $\widehat{C}(j_{\mathrm{out}})$. Hence
\[
  \Delta_{\mathrm{out}}
  \;\le\;
  \Bigl(
    \sum_{i\in \widehat{C}(j_{\mathrm{out}})} w_i
  \Bigr)
  \cdot
  d\bigl(j_{\mathrm{out}},\,j^*_{\text{large}}\bigr)
  \;\le\;
  \frac{1}{k}
  \Bigl(\sum_{i=1}^n w_i\Bigr)
  \cdot
  d\bigl(j_{\mathrm{out}},\,j^*_{\text{large}}\bigr).
\]

\medskip
\noindent
\textbf{Step 7: Arriving at a contradiction.}
We get
\[
  c\bigl(\widehat{\mathcal{S}}_{\mathrm{swap}}\bigr)
  -
  c\bigl(\widehat{\mathcal{S}}\bigr)
  =
  \Delta_{\mathrm{in}} + \Delta_{\mathrm{out}}
  \;\le\;
  -\sum_{i\in C^*}
     w_i
     \bigl(r_i - r_i^*\bigr)_{+}
  \;+\;
  \frac{1}{k}
  \Bigl(\sum_{i} w_i\Bigr)
  \,
  d\bigl(j_{\mathrm{out}},j^*_{\text{large}}\bigr).
\]
But recall
\[
  \sum_{i\in C^*} 
   w_i\,
   (r_i - r_i^*)_{+}
   =
   D_{j^*_{\text{large}}}
   \;>\;
   4\,\frac{c(\mathcal{S}^*)}{k},
\]
from step 5.  Meanwhile, $d\bigl(j_{\mathrm{out}},\,j^*_{\text{large}}\bigr)\le c(\mathcal{S}^*)$ is a standard bound because $j^*_{\text{large}}$ must be served in $\mathcal{S}^*$ by some center at distance at most $c(\mathcal{S}^*) / \sum_i w_i$ \emph{or} by the triangle inequality, we can also argue $d(j_{\mathrm{out}},j^*_{\text{large}})\le$ the diameter factor times the cost.  More refined bounding uses per-point comparisons.

Hence
\[
  \Delta_{\mathrm{out}}
  \;\le\;
  \frac{1}{k}
  \Bigl(\sum_{i} w_i\Bigr)
  \,c(\mathcal{S}^*)
  \,/\,\bigl(\sum_{i} w_i\bigr)
  \;\;=\;\;
  \frac{c(\mathcal{S}^*)}{k}.
\]
Thus
\[
  c(\widehat{\mathcal{S}}_{\mathrm{swap}})
  -
  c(\widehat{\mathcal{S}})
  \;\le\;
  -\,4\,\frac{c(\mathcal{S}^*)}{k}
  \;+\;
  \frac{c(\mathcal{S}^*)}{k}
  \;=\;
  -\,3\,\frac{c(\mathcal{S}^*)}{k}
  \;<\;0,
\]
i.e.\ a net improvement.  This contradicts the local optimality of $\widehat{\mathcal{S}}$.  

Therefore our original assumption $c(\widehat{\mathcal{S}})>5\,c(\mathcal{S}^*)$ must be false, so $c(\widehat{\mathcal{S}})\le 5\,c(\mathcal{S}^*)$.  

\medskip
\noindent
\textbf{Time Complexity.}
Each swap test requires $O(n)$ time to update $\mathrm{Cost}(\mathcal{S})$.  There are at most $k\,(n-k)$ possible 1-swaps.  Each accepted swap \emph{strictly} decreases cost by at least 1 unit (or some positive $\delta$-fraction if distances are discrete/normalized).  Since the minimal cost is $\ge0$, the total number of swaps is polynomially bounded.  Thus local search terminates in polynomial time with the promised approximation.

\end{proof}

\begin{remark}[Improved Constants]
A more intricate analysis can tighten the factor 5 in \Cref{thm:local_search_kmedoids} to 3 or 4.  See, e.g., \citep{gupta2008simpler,arya2001local} for classical refinements.  The simpler argument here suffices to establish the main principles.
\end{remark}



% \section{CoreSets for Representative Selection}
% \label{sec:coreset_subsection}

\section{Constant-Factor Approximation for Subset Selection Under Bounded Intra-Cluster Distance}
\label{sec:constant_factor_subset_selection}

\noindent
The term \emph{coreset} originates in computational geometry and machine learning, referring to a subset of data that \emph{approximates} the entire dataset with respect to a particular objective or loss function \citep{bachem2017practical,feldman2020turning}. More precisely, a coreset $\mathcal{C}$ for a larger set $\mathcal{X}$ is often defined such that, for any model or solution $w$ in a hypothesis class, the loss over $\mathcal{C}$ is within a small factor of the loss over $\mathcal{X}$. 

In the context of \textsc{AMPO-Coreset}, the $k$-means clustering subroutine identifies \emph{representative} embedding-space regions, and by choosing a single worst-rated example from each region, we mimic a coreset-based selection principle: our selected negatives approximate the \emph{distributional diversity} of the entire batch of responses. In essence, we seek a small but well-covered negative set that ensures the model receives penalizing signals for all major modes of undesired behavior. 

Empirically, such coverage-driven strategies can outperform purely score-based selection (Section \ref{sec:ampo_bottomk}) when the reward function is noisy or the model exhibits rare but severe failure modes. By assigning at least one negative from each cluster, \textsc{AMPO-Coreset} mitigates the risk of ignoring minority clusters, which may be infrequent yet highly problematic for alignment. As we show in subsequent experiments, combining \emph{coreset-like coverage} with \emph{reward-based filtering} yields robust policy updates that curb a wide range of undesirable outputs.



We give a simplified theorem showing how a local-search algorithm can achieve a fixed (constant) approximation factor for selecting \(k\) ``negative'' responses. Our statement and proof are adapted from the classical \emph{Weighted \(k\)-Medoids} analysis, but use simpler notation and explicit assumptions about bounded intra-cluster distance.

% \subsection{Problem Statement and Assumptions}

% \noindent
% \textbf{Weighted k-medoids setup Setup.}  
% We have \(n\) items, indexed by \(i=1,\dots,n\). Each item \(i\) has a \emph{weight} \(w_i \ge 0\) and lies in a metric space with distance \(d(i,j)\). We want to choose a subset \(\mathcal{S} \subseteq \{1,\dots,n\}\) of size \(k\) (i.e., \(|\mathcal{S}|=k\)) that minimizes
% \[
%   \mathrm{Cost}(\mathcal{S})
%   \;=\;
%   \sum_{i=1}^n
%     w_i \,\min_{\,j \in \mathcal{S}} d(i,j).
% \]
% This is precisely the \emph{Weighted \(k\)-Medoids} objective. The points \(j\in \mathcal{S}\) can be viewed as ``negatives'' or ``centers,'' penalizing nearby items by forcing them to remain close (hence incurring a smaller distance cost).

% We aim to present a theorem here with strong assumptions, that may not hold truly in practice, but would be indicative of a performance guarantee, should such assumptions hold.

\noindent
\subsection{Additional Assumptions:}
\textbf{Assumption 1: Bounded number of clusters k.}  
We assume that the data partitions into natural clusters such that the number of such clusters is equal to the number of examples we draw from the negatives. It is of course likely that at sufficiently high temperature, an LLM may deviate from such assumptions, but given sufficiently low sampling temperature, the answers, for any given query, may concentrate to a few attractors.

\textbf{Assumption 2: Bounded Intra-Cluster Distance.}  
We assume that the data can be partitioned into natural clusters of bounded diameter \(d_{\max}\). This assumption helps us simplify our bounds, towards rigorous guarantees, and we wish to state that such an assumption may be too strict to hold in practice, especially in light of Assumption 1.




Given these assumptions, We present a distribution-dependent coreset guarantee for selecting a small ``negative'' subset of responses for a given query, thus enabling the policy to concentrate probability on the highest-rated responses. Unlike universal coreset theory, we only require that this negative subset works well for typical distributions of responses, rather than for every conceivable set of responses.

\subsection{Setup: Queries, Responses, and Ratings}

\noindent
\textbf{Queries and Candidate Responses.}  
We focus on a single \emph{query} \(x\), which admits a finite set of \(m\) candidate responses 
\[
  \{\,y_1,\dots,y_m\}.
\]
Each response \(y_i\) has a scalar rating \(r_i \in [0,1]\). For notational convenience, we assume \(r_i\) is normalized to \([0,1]\). A larger \(r_i\) indicates a better (or more desirable) response.

\vspace{0.5em}
\noindent
\textbf{Negative Ratings via Exponential Weights.}  
Let 
\begin{align}
\overline{r} 
\;=\; 
\frac{1}{m}\sum_{i=1}^m r_i
\quad\text{(the mean rating)}, 
\quad
w_i 
\;=\; 
\exp\bigl(\overline{r}-r_i\bigr).
\label{eq:neg_weight}
\end{align}
Then \(w_i\) is larger when \(r_i\) is smaller. One may also employ alternative references (\(\max r_i\) instead of \(\overline{r}\)), or re-scaling to maintain bounded ranges.

\subsection{Policy Model and Subset Selection}

\noindent
\textbf{Policy Distribution Over Responses.}  
A policy \(P_\theta(y \mid x)\) assigns a probability \(p_i \ge 0\) to each response \(y_i\), satisfying 
\(\sum_{i=1}^m p_i = 1\). The \emph{expected rating} is 
\[
  \mathrm{ER}(p_1,\dots,p_m)
  \;=\;
  \sum_{i=1}^m p_i\,r_i.
\]

\vspace{0.5em}
\noindent
\textbf{Negative Subset and Probability Suppression.}  
We aim to choose a small subset \(\mathcal{S}\subseteq\{\,1,\dots,m\}\) of size \(k\), each member of which is assigned probability zero: 
\[
   p_j = 0,\quad\forall j\in \mathcal{S}.
\]
In addition, we impose a \textit{Lipschitz-like} rule that if \(p_j=0\) for \(j\in\mathcal{S}\), then any response \(y_i\) ``close'' to \(y_j\) in some embedding space must also have probability bounded by 
\[
  p_i 
  \;\le\;
  L\,\|\mathbf{e}_i - \mathbf{e}_j\|,
\]
where \(\mathbf{e}_i\) is an embedding of \(y_i\). If \(y_j\) is \emph{negatively rated}, then forcing \(p_j=0\) also forces small probability on responses near \(y_j\). This ensures undesired modes get suppressed.

\vspace{0.5em}
\noindent
\textbf{Concentrating Probability on Top Responses.}  
We allow the policy to place nearly all probability on a small handful of high-rated responses, so that the expected rating \(\sum_{i=1}^m p_i r_i\) is maximized. Indeed, the policy will try to push mass towards the highest \(r_i\) while setting \(p_j=0\) on low-rated responses in \(\mathcal{S}\).

\noindent
\textbf{Sampling Response-Sets or ``Solutions.''}  
We suppose that the set \(\{y_1,\dots,y_m\}\) with ratings \(\{r_i\}\) arises from some distributional process (for instance, \(\mathcal{D}\) might represent typical ways the system could generate or rank responses). Denote a random draw by 
\[
  \bigl(\{y_1,\dots,y_m\},\,\{r_i\}\bigr)
  \;\sim\; 
  \mathcal{D}.
\]
We only require that our negative subset \(\mathcal{S}\) yield a near-optimal Lipschitz-compliant policy \emph{for a typical realization from \(\mathcal{D}\)}, rather than for every possible realization.

\vspace{0.5em}
\noindent
\textbf{Clustering in Embedding Space.}  
Let \(\mathbf{e}_i\in\mathbb{R}^d\) be an embedding for each response \(y_i\). Suppose we partition \(\{1,\dots,m\}\) into \(k\) clusters \(C_1,\dots,C_k\) (each of bounded diameter at most \(d\)), and within each cluster \(C_j\), pick exactly one ``negative'' index \(i_j^- \in C_j\). This yields 
\[
   \mathcal{S} 
   \;=\; 
   \{\, i_1^-, \dots, i_k^-\}.
\]
We then penalize each \(y_{i_j^-}\) by setting \(p_{\,i_j^-}=0\). Consequently, for any \(y_i \in C_j\), the Lipschitz suppression condition forces \(p_i \le L\,d\).  






\subsection{A Distribution-Dependent Coreset Guarantee}

We now state a simplified theorem that, under certain conditions on the distribution \(\mathcal{D}\), ensures that for most draws of queries and responses, the chosen subset \(\mathcal{S}\) yields a policy whose expected rating is within \((1\pm \varepsilon)\) of the optimal Lipschitz-compliant policy of size \(k\).

\begin{theorem}[Distribution-Dependent Negative Subset]
\label{thm:distribution_coreset_responses}
Let \(\mathcal{D}\) be a distribution that generates query-response sets \(\{y_1,\dots,y_m\}\), each with ratings \(\{r_i\}\subset [0,1]\). Assume we cluster the \(m\) responses into \(k\) groups \(C_1,\dots,C_k\) of diameter at most \(d\) in the embedding space, and choose exactly one ``negative'' index \(i_j^-\in C_j\). Let \(\mathcal{S}=\{\,i_1^-,\dots,i_k^-\}\). Suppose that:
\[
   \max_{i\in C_j}\,
   \|\mathbf{e}_i-\mathbf{e}_{\,i_j^-}\|
   \;\le\;d,
   \quad
   \forall\,j=1,\dots,k.
\]
Assume a Lipschitz constant \(L\), so that penalizing \(y_{i_j^-}\) (i.e.\ \(p_{\,i_j^-}=0\)) enforces \(p_i \le L\,d\) for all \(i\in C_j\). Then, under a sufficiently large random sample of queries/responses (or equivalently, a large i.i.d.\ sample from \(\mathcal{D}\) to refine the clustering), with high probability over that sample, for at least a \(\bigl(1-\delta\bigr)\) fraction of newly drawn query-response sets from \(\mathcal{D}\), the set \(\mathcal{S}\) induces a Lipschitz-compliant policy whose expected rating is within a factor \((1\pm \varepsilon)\) of the best possible among all $k$-penalized subsets.
\end{theorem}

\begin{proof}[Proof Sketch]
We give a high-level argument:

\textbf{1. Large Sample Captures Typical Configurations.} By drawing many instances of responses $\{y_i\}$, $\{r_i\}$ from $\mathcal{D}$, we can cluster them in such a way that \emph{any new} draw from $\mathcal{D}$ is, with probability at least $1-\delta$, either (a) close to one of our sampled configurations or (b) has measure less than $\delta$.

\textbf{2. Bounded-Diameter Clusters.} Suppose each cluster $C_j$ has diameter at most $d$, and we pick $i_j^- \in C_j$ as the ``negative.'' This implies every response $y_i$ in that cluster is at distance $\le d$ from $y_{i_j^-}$.

\textbf{3. Lipschitz Suppression.} If $p_{\,i_j^-}=0$, then $p_i \le L\,\|\mathbf{e}_i - \mathbf{e}_{\,i_j^-}\|\le L\,d$ for all $i \in C_j$. This ensures that the entire cluster $C_j$ cannot accumulate large probability mass on low-rated responses. Consequently, we push the policy distribution to concentrate on higher-rated responses (e.g.\ those \emph{not} near a penalized center).

\textbf{4. Near-Optimal Expected Rating.} For any typical new draw of $\{y_i\}$, $\{r_i\}$, a $k$-penalized Lipschitz policy can be approximated by using the same $k$ negatives $\mathcal{S}$. Because we ensure that the new draw is close to one of our sampled draws, the coverage or cluster assignment for the new $\{y_i\}$ is accurate enough that the resulting feasible policy is within a multiplicative $(1\pm \varepsilon)$ factor of the best possible $k$-subset. This completes the distribution-dependent argument.

\end{proof}


\section{Optimal Selection Code}
\label{sec:optimal_selection_computation}

In this section we provide the actual code used to compute the optimal selection.

\begin{lstlisting}[language=Python]
import numpy as np
from scipy.spatial.distance import cdist

def solve_local_search_min_dist_normalized(
    vectors: np.ndarray,
    rating: np.ndarray,
    k: int,
    max_iter: int = 100,
    random_seed: int = 42
):
    # Normalize ratings
    rating_min = np.min(rating)
    rating_max = np.max(rating)
    rating_normalized = (rating - rating_min) / (rating_max - rating_min) if rating_max > rating_min else np.zeros_like(rating) + 0.5  

    # Identify top-rated point
    excluded_top_index = int(np.argmax(rating_normalized))

    # Reduce dataset
    new_to_old = [idx for idx in range(len(rating_normalized)) if idx != excluded_top_index]
    vectors_reduced = np.delete(vectors, excluded_top_index, axis=0)
    rating_reduced = np.delete(rating_normalized, excluded_top_index)

    # Compute L2 distances and normalize
    if len(rating_reduced) == 0:
        return excluded_top_index, None, [], [], []
    distance_matrix = cdist(vectors_reduced, vectors_reduced, metric='euclidean')
    distance_matrix /= distance_matrix.max() if distance_matrix.max() > 1e-12 else 1

    # Compute weights
    mean_rating_reduced = np.mean(rating_reduced)
    w = np.exp(mean_rating_reduced - rating_reduced)

    # Local search setup
    def compute_objective(chosen_set):
        return sum(w[i] * min(distance_matrix[i, j] for j in chosen_set) for i in range(len(w)))

    rng = np.random.default_rng(random_seed)
    all_indices = np.arange(len(rating_reduced))
    current_set = set(rng.choice(all_indices, size=k, replace=False)) if k < len(rating_reduced) else set(all_indices)
    current_cost = compute_objective(current_set)

    # Local search loop
    improved = True
    while improved:
        improved = False
        best_swap = (None, None, 0)
        for j_out in list(current_set):
            for j_in in all_indices:
                if j_in not in current_set:
                    candidate_set = (current_set - {j_out}) | {j_in}
                    improvement = current_cost - compute_objective(candidate_set)
                    if improvement > best_swap[2]:
                        best_swap = (j_out, j_in, improvement)
        if best_swap[2] > 1e-12:
            current_set.remove(best_swap[0])
            current_set.add(best_swap[1])
            current_cost -= best_swap[2]
            improved = True

    chosen_indices_original = [new_to_old[j] for j in sorted(current_set)]
    rejected_indices_original = [new_to_old[j] for j in sorted(set(all_indices) - current_set)]
    return excluded_top_index, chosen_indices_original[0], rejected_indices_original[:k], chosen_indices_original, rejected_indices_original
\end{lstlisting}
\clearpage
\section{Visualization of t-SNE embeddings for Diverse Responses Across Queries}
\label{sec:tsne_visualization}

In this section, we showcase the performance of our method through plots of TSNE across various examples. These illustrative figures show how our baseline Bottom-k Algorithm (Section \ref{sec:ampo_bottomk}) chooses similar responses that are often close to each other. Hence the model misses out on feedback relating to other parts of the answer space that it often explores. Contrastingly, we often notice diversity of response selection for both the $\ampoos$ and $\ampocs$ algorithms.

% \begin{figure*}[!thbp]
%     \centering
%     \includegraphics[width=1.0\textwidth]{images/tsne_plot_with_scores_23687.pdf}
%     \caption{t-SNE visualization of projected high-dimensional response embeddings into a 2D space, illustrating the separation of actively selected responses. (a) AMPO-BottomK (baseline). (b) AMPO-Coreset (ours). (c) Opt-Select (ours). We see that the traditional baselines select many responses close to each other, based on their rating. This provides insufficient feedback to the LLM during preference optimization. In contrast, our methods simultaneously optimize for objectives including coverage, generation probability as well as preference rating.}
% \label{fig:tsne_diverse_analysis}
% \end{figure*}

% \begin{figure*}[!thbp]
%     \centering
%     \includegraphics[width=1.0\textwidth]{images/tsne_plot_with_scores_1362.pdf}
%     \caption{t-SNE visualization of projected high-dimensional response embeddings into a 2D space, illustrating the separation of actively selected responses. (a) AMPO-BottomK (baseline). (b) AMPO-Coreset (ours). (c) Opt-Select (ours). We see that the traditional baselines select many responses close to each other, based on their rating. This provides insufficient feedback to the LLM during preference optimization. In contrast, our methods simultaneously optimize for objectives including coverage, generation probability as well as preference rating.}
% \label{fig:tsne_diverse_analysis}
% \end{figure*}

% \begin{figure*}[!thbp]
%     \centering
%     \includegraphics[width=1.0\textwidth]{images/tsne_plot_with_scores_24192.pdf}

%     \caption{t-SNE visualization of projected high-dimensional response embeddings into a 2D space, illustrating the separation of actively selected responses. (a) AMPO-BottomK (baseline). (b) AMPO-Coreset (ours). (c) Opt-Select (ours). We see that the traditional baselines select many responses close to each other, based on their rating. This provides insufficient feedback to the LLM during preference optimization. In contrast, our methods simultaneously optimize for objectives including coverage, generation probability as well as preference rating.}
% \label{fig:tsne_diverse_analysis}
% \end{figure*}


% \begin{figure*}[!thbp]
%     \centering
%     \begin{subfigure}[b]{1.0\textwidth}
%         \centering
%         \includegraphics[width=0.8\textwidth]{images/tsne_plot_with_scores_23687.pdf}
%         \caption{Query-1.}
%         \label{fig:tsne_bottomk}
%     \end{subfigure}
    
%     \begin{subfigure}[b]{1.0\textwidth}
%         \centering
%         \includegraphics[width=0.8\textwidth]{images/tsne_plot_with_scores_1362.pdf}
%         \caption{Query1.}
%         \label{fig:tsne_coreset}
%     \end{subfigure}
    
%     \begin{subfigure}[b]{1.0\textwidth}
%         \centering
%         \includegraphics[width=0.8\textwidth]{images/tsne_plot_with_scores_24192.pdf}
%         \caption{Query-3.}
%         \label{fig:tsne_optselect}
%     \end{subfigure}
    
%     \caption{t-SNE visualization of projected high-dimensional response embeddings into a 2D space, illustrating the separation of actively selected responses. (a) AMPO-BottomK (baseline). (b) AMPO-Coreset (ours). (c) Opt-Select (ours). Traditional baselines select many responses close to each other based on their rating, providing insufficient feedback to the LLM during preference optimization. In contrast, our methods optimize for objectives including coverage, generation probability, and preference rating.}
%     \label{fig:tsne_combined}
% \end{figure*}


\begin{figure*}[!thbp]
    \centering
    \subfigure[1.]{
        \includegraphics[width=1.0\textwidth]{images/tsne_plot_with_scores_23687.pdf}
        \label{fig:tsne_bottomk}
    }
    
    \subfigure[2.]{
        \includegraphics[width=1.0\textwidth]{images/tsne_plot_with_scores_1362.pdf}
        \label{fig:tsne_coreset}
    }
    
    \subfigure[3.]{
        \includegraphics[width=1.0\textwidth]{images/tsne_plot_with_scores_24192.pdf}
        \label{fig:tsne_optselect}
    }
    
    \caption{t-SNE visualization of projected high-dimensional response embeddings into a 2D space, illustrating the separation of actively selected responses. (a) AMPO-BottomK (baseline). (b) AMPO-Coreset (ours). (c) Opt-Select (ours). Traditional baselines select many responses close to each other based on their rating, providing insufficient feedback to the LLM during preference optimization. In contrast, our methods optimize for objectives including coverage, generation probability, and preference rating.}
    \label{fig:tsne_combined}
\end{figure*}



% \subsection{Figures}

% You may want to include figures in the paper to illustrate
% your approach and results. Such artwork should be centered,
% legible, and separated from the text. Lines should be dark and at
% least 0.5~points thick for purposes of reproduction, and text should
% not appear on a gray background.

% Label all distinct components of each figure. If the figure takes the
% form of a graph, then give a name for each axis and include a legend
% that briefly describes each curve. Do not include a title inside the
% figure; instead, the caption should serve this function.

% Number figures sequentially, placing the figure number and caption
% \emph{after} the graphics, with at least 0.1~inches of space before
% the caption and 0.1~inches after it, as in
% \cref{icml-historical}. The figure caption should be set in
% 9~point type and centered unless it runs two or more lines, in which
% case it should be flush left. You may float figures to the top or
% bottom of a column, and you may set wide figures across both columns
% (use the environment \texttt{figure*} in \LaTeX). Always place
% two-column figures at the top or bottom of the page.

% \subsection{Algorithms}

% If you are using \LaTeX, please use the ``algorithm'' and ``algorithmic''
% environments to format pseudocode. These require
% the corresponding stylefiles, algorithm.sty and
% algorithmic.sty, which are supplied with this package.
% \cref{alg:example} shows an example.

% \begin{algorithm}[tb]
%    \caption{Bubble Sort}
%    \label{alg:example}
% \begin{algorithmic}
%    \STATE {\bfseries Input:} data $x_i$, size $m$
%    \REPEAT
%    \STATE Initialize $noChange = true$.
%    \FOR{$i=1$ {\bfseries to} $m-1$}
%    \IF{$x_i > x_{i+1}$}
%    \STATE Swap $x_i$ and $x_{i+1}$
%    \STATE $noChange = false$
%    \ENDIF
%    \ENDFOR
%    \UNTIL{$noChange$ is $true$}
% \end{algorithmic}
% \end{algorithm}

% \subsection{Tables}

% You may also want to include tables that summarize material. Like
% figures, these should be centered, legible, and numbered consecutively.
% However, place the title \emph{above} the table with at least
% 0.1~inches of space before the title and the same after it, as in
% \cref{sample-table}. The table title should be set in 9~point
% type and centered unless it runs two or more lines, in which case it
% should be flush left.

% Note use of \abovespace and \belowspace to get reasonable spacing
% above and below tabular lines.

% \begin{table}[t]
% \caption{Classification accuracies for naive Bayes and flexible
% Bayes on various data sets.}
% \label{sample-table}
% \vskip 0.15in
% \begin{center}
% \begin{small}
% \begin{sc}
% \begin{tabular}{lcccr}
% \toprule
% Data set & Naive & Flexible & Better? \\
% \midrule
% Breast    & 95.9$\pm$ 0.2& 96.7$\pm$ 0.2& $\surd$ \\
% Cleveland & 83.3$\pm$ 0.6& 80.0$\pm$ 0.6& $\times$\\
% Glass2    & 61.9$\pm$ 1.4& 83.8$\pm$ 0.7& $\surd$ \\
% Credit    & 74.8$\pm$ 0.5& 78.3$\pm$ 0.6&         \\
% Horse     & 73.3$\pm$ 0.9& 69.7$\pm$ 1.0& $\times$\\
% Meta      & 67.1$\pm$ 0.6& 76.5$\pm$ 0.5& $\surd$ \\
% Pima      & 75.1$\pm$ 0.6& 73.9$\pm$ 0.5&         \\
% Vehicle   & 44.9$\pm$ 0.6& 61.5$\pm$ 0.4& $\surd$ \\
% \bottomrule
% \end{tabular}
% \end{sc}
% \end{small}
% \end{center}
% \vskip -0.1in
% \end{table}

% Tables contain textual material, whereas figures contain graphical material.
% Specify the contents of each row and column in the table's topmost
% row. Again, you may float tables to a column's top or bottom, and set
% wide tables across both columns. Place two-column tables at the
% top or bottom of the page.

% \subsection{Theorems and such}
% The preferred way is to number definitions, propositions, lemmas, etc. consecutively, within sections, as shown below.
% \begin{definition}
% \label{def:inj}
% A function $f:X \to Y$ is injective if for any $x,y\in X$ different, $f(x)\ne f(y)$.
% \end{definition}
% Using \cref{def:inj} we immediate get the following result:
% \begin{proposition}
% If $f$ is injective mapping a set $X$ to another set $Y$, 
% the cardinality of $Y$ is at least as large as that of $X$
% \end{proposition}
% \begin{proof} 
% Left as an exercise to the reader. 
% \end{proof}
% \cref{lem:usefullemma} stated next will prove to be useful.
% \begin{lemma}
% \label{lem:usefullemma}
% For any $f:X \to Y$ and $g:Y\to Z$ injective functions, $f \circ g$ is injective.
% \end{lemma}
% \begin{theorem}
% \label{thm:bigtheorem}
% If $f:X\to Y$ is bijective, the cardinality of $X$ and $Y$ are the same.
% \end{theorem}
% An easy corollary of \cref{thm:bigtheorem} is the following:
% \begin{corollary}
% If $f:X\to Y$ is bijective, 
% the cardinality of $X$ is at least as large as that of $Y$.
% \end{corollary}
% \begin{assumption}
% The set $X$ is finite.
% \label{ass:xfinite}
% \end{assumption}
% \begin{remark}
% According to some, it is only the finite case (cf. \cref{ass:xfinite}) that is interesting.
% \end{remark}
%restatable





\end{document}


% This document was modified from the file originally made available by
% Pat Langley and Andrea Danyluk for ICML-2K. This version was created
% by Iain Murray in 2018, and modified by Alexandre Bouchard in
% 2019 and 2021 and by Csaba Szepesvari, Gang Niu and Sivan Sabato in 2022.
% Modified again in 2023 and 2024 by Sivan Sabato and Jonathan Scarlett.
% Previous contributors include Dan Roy, Lise Getoor and Tobias
% Scheffer, which was slightly modified from the 2010 version by
% Thorsten Joachims & Johannes Fuernkranz, slightly modified from the
% 2009 version by Kiri Wagstaff and Sam Roweis's 2008 version, which is
% slightly modified from Prasad Tadepalli's 2007 version which is a
% lightly changed version of the previous year's version by Andrew
% Moore, which was in turn edited from those of Kristian Kersting and
% Codrina Lauth. Alex Smola contributed to the algorithmic style files.
