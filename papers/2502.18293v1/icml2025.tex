%%%%%%%% ICML 2025 EXAMPLE LATEX SUBMISSION FILE %%%%%%%%%%%%%%%%%

\documentclass{article}

% Recommended, but optional, packages for figures and better typesetting:
\usepackage{microtype}
\usepackage{graphicx}
\usepackage{subfigure}

\usepackage{booktabs} % for professional tables
% hyperref makes hyperlinks in the resulting PDF.
% If your build breaks (sometimes temporarily if a hyperlink spans a page)
% please comment out the following usepackage line and replace
% \usepackage{icml2025} with \usepackage[nohyperref]{icml2025} above.
\usepackage{hyperref}


% Attempt to make hyperref and algorithmic work together better:
\newcommand{\theHalgorithm}{\arabic{algorithm}}

% Use the following line for the initial blind version submitted for review:
% \usepackage{icml2025}

% If accepted, instead use the following line for the camera-ready submission:
\usepackage[accepted]{icml2025}

% For theorems and such
\usepackage{amsmath}
\usepackage{amssymb}
\usepackage{mathtools}
\usepackage{amsthm}
% if you use cleveref..
\usepackage[capitalize,noabbrev]{cleveref}

%%%%%%%%%%%%%%%%%%%%%%%%%%%%%%%%
% THEOREMS
%%%%%%%%%%%%%%%%%%%%%%%%%%%%%%%%
\theoremstyle{plain}
\newtheorem{theorem}{Theorem}[section]
\newtheorem{proposition}[theorem]{Proposition}
\newtheorem{lemma}[theorem]{Lemma}
\newtheorem{corollary}[theorem]{Corollary}
\theoremstyle{definition}
\newtheorem{definition}[theorem]{Definition}
\newtheorem{assumption}[theorem]{Assumption}
\theoremstyle{remark}
\newtheorem{remark}[theorem]{Remark}

% Todonotes is useful during development; simply uncomment the next line
%    and comment out the line below the next line to turn off comments
% \usepackage[disable,textsize=tiny]{todonotes}
% \usepackage[textsize=tiny]{todonotes}

%%%%%%%%%%%%%%%%%%%%%%%%%%%%%%%%
% USER DEFINED
%%%%%%%%%%%%%%%%%%%%%%%%%%%%%%%%

%%%%% NEW MATH DEFINITIONS %%%%%

\usepackage{amsmath,amsfonts,bm}
\usepackage{derivative}
% Mark sections of captions for referring to divisions of figures
\newcommand{\figleft}{{\em (Left)}}
\newcommand{\figcenter}{{\em (Center)}}
\newcommand{\figright}{{\em (Right)}}
\newcommand{\figtop}{{\em (Top)}}
\newcommand{\figbottom}{{\em (Bottom)}}
\newcommand{\captiona}{{\em (a)}}
\newcommand{\captionb}{{\em (b)}}
\newcommand{\captionc}{{\em (c)}}
\newcommand{\captiond}{{\em (d)}}

% Highlight a newly defined term
\newcommand{\newterm}[1]{{\bf #1}}

% Derivative d 
\newcommand{\deriv}{{\mathrm{d}}}

% Figure reference, lower-case.
\def\figref#1{figure~\ref{#1}}
% Figure reference, capital. For start of sentence
\def\Figref#1{Figure~\ref{#1}}
\def\twofigref#1#2{figures \ref{#1} and \ref{#2}}
\def\quadfigref#1#2#3#4{figures \ref{#1}, \ref{#2}, \ref{#3} and \ref{#4}}
% Section reference, lower-case.
\def\secref#1{section~\ref{#1}}
% Section reference, capital.
\def\Secref#1{Section~\ref{#1}}
% Reference to two sections.
\def\twosecrefs#1#2{sections \ref{#1} and \ref{#2}}
% Reference to three sections.
\def\secrefs#1#2#3{sections \ref{#1}, \ref{#2} and \ref{#3}}
% Reference to an equation, lower-case.
\def\eqref#1{equation~\ref{#1}}
% Reference to an equation, upper case
\def\Eqref#1{Equation~\ref{#1}}
% A raw reference to an equation---avoid using if possible
\def\plaineqref#1{\ref{#1}}
% Reference to a chapter, lower-case.
\def\chapref#1{chapter~\ref{#1}}
% Reference to an equation, upper case.
\def\Chapref#1{Chapter~\ref{#1}}
% Reference to a range of chapters
\def\rangechapref#1#2{chapters\ref{#1}--\ref{#2}}
% Reference to an algorithm, lower-case.
\def\algref#1{algorithm~\ref{#1}}
% Reference to an algorithm, upper case.
\def\Algref#1{Algorithm~\ref{#1}}
\def\twoalgref#1#2{algorithms \ref{#1} and \ref{#2}}
\def\Twoalgref#1#2{Algorithms \ref{#1} and \ref{#2}}
% Reference to a part, lower case
\def\partref#1{part~\ref{#1}}
% Reference to a part, upper case
\def\Partref#1{Part~\ref{#1}}
\def\twopartref#1#2{parts \ref{#1} and \ref{#2}}

\def\ceil#1{\lceil #1 \rceil}
\def\floor#1{\lfloor #1 \rfloor}
\def\1{\bm{1}}
\newcommand{\train}{\mathcal{D}}
\newcommand{\valid}{\mathcal{D_{\mathrm{valid}}}}
\newcommand{\test}{\mathcal{D_{\mathrm{test}}}}

\def\eps{{\epsilon}}


% Random variables
\def\reta{{\textnormal{$\eta$}}}
\def\ra{{\textnormal{a}}}
\def\rb{{\textnormal{b}}}
\def\rc{{\textnormal{c}}}
\def\rd{{\textnormal{d}}}
\def\re{{\textnormal{e}}}
\def\rf{{\textnormal{f}}}
\def\rg{{\textnormal{g}}}
\def\rh{{\textnormal{h}}}
\def\ri{{\textnormal{i}}}
\def\rj{{\textnormal{j}}}
\def\rk{{\textnormal{k}}}
\def\rl{{\textnormal{l}}}
% rm is already a command, just don't name any random variables m
\def\rn{{\textnormal{n}}}
\def\ro{{\textnormal{o}}}
\def\rp{{\textnormal{p}}}
\def\rq{{\textnormal{q}}}
\def\rr{{\textnormal{r}}}
\def\rs{{\textnormal{s}}}
\def\rt{{\textnormal{t}}}
\def\ru{{\textnormal{u}}}
\def\rv{{\textnormal{v}}}
\def\rw{{\textnormal{w}}}
\def\rx{{\textnormal{x}}}
\def\ry{{\textnormal{y}}}
\def\rz{{\textnormal{z}}}

% Random vectors
\def\rvepsilon{{\mathbf{\epsilon}}}
\def\rvphi{{\mathbf{\phi}}}
\def\rvtheta{{\mathbf{\theta}}}
\def\rva{{\mathbf{a}}}
\def\rvb{{\mathbf{b}}}
\def\rvc{{\mathbf{c}}}
\def\rvd{{\mathbf{d}}}
\def\rve{{\mathbf{e}}}
\def\rvf{{\mathbf{f}}}
\def\rvg{{\mathbf{g}}}
\def\rvh{{\mathbf{h}}}
\def\rvu{{\mathbf{i}}}
\def\rvj{{\mathbf{j}}}
\def\rvk{{\mathbf{k}}}
\def\rvl{{\mathbf{l}}}
\def\rvm{{\mathbf{m}}}
\def\rvn{{\mathbf{n}}}
\def\rvo{{\mathbf{o}}}
\def\rvp{{\mathbf{p}}}
\def\rvq{{\mathbf{q}}}
\def\rvr{{\mathbf{r}}}
\def\rvs{{\mathbf{s}}}
\def\rvt{{\mathbf{t}}}
\def\rvu{{\mathbf{u}}}
\def\rvv{{\mathbf{v}}}
\def\rvw{{\mathbf{w}}}
\def\rvx{{\mathbf{x}}}
\def\rvy{{\mathbf{y}}}
\def\rvz{{\mathbf{z}}}

% Elements of random vectors
\def\erva{{\textnormal{a}}}
\def\ervb{{\textnormal{b}}}
\def\ervc{{\textnormal{c}}}
\def\ervd{{\textnormal{d}}}
\def\erve{{\textnormal{e}}}
\def\ervf{{\textnormal{f}}}
\def\ervg{{\textnormal{g}}}
\def\ervh{{\textnormal{h}}}
\def\ervi{{\textnormal{i}}}
\def\ervj{{\textnormal{j}}}
\def\ervk{{\textnormal{k}}}
\def\ervl{{\textnormal{l}}}
\def\ervm{{\textnormal{m}}}
\def\ervn{{\textnormal{n}}}
\def\ervo{{\textnormal{o}}}
\def\ervp{{\textnormal{p}}}
\def\ervq{{\textnormal{q}}}
\def\ervr{{\textnormal{r}}}
\def\ervs{{\textnormal{s}}}
\def\ervt{{\textnormal{t}}}
\def\ervu{{\textnormal{u}}}
\def\ervv{{\textnormal{v}}}
\def\ervw{{\textnormal{w}}}
\def\ervx{{\textnormal{x}}}
\def\ervy{{\textnormal{y}}}
\def\ervz{{\textnormal{z}}}

% Random matrices
\def\rmA{{\mathbf{A}}}
\def\rmB{{\mathbf{B}}}
\def\rmC{{\mathbf{C}}}
\def\rmD{{\mathbf{D}}}
\def\rmE{{\mathbf{E}}}
\def\rmF{{\mathbf{F}}}
\def\rmG{{\mathbf{G}}}
\def\rmH{{\mathbf{H}}}
\def\rmI{{\mathbf{I}}}
\def\rmJ{{\mathbf{J}}}
\def\rmK{{\mathbf{K}}}
\def\rmL{{\mathbf{L}}}
\def\rmM{{\mathbf{M}}}
\def\rmN{{\mathbf{N}}}
\def\rmO{{\mathbf{O}}}
\def\rmP{{\mathbf{P}}}
\def\rmQ{{\mathbf{Q}}}
\def\rmR{{\mathbf{R}}}
\def\rmS{{\mathbf{S}}}
\def\rmT{{\mathbf{T}}}
\def\rmU{{\mathbf{U}}}
\def\rmV{{\mathbf{V}}}
\def\rmW{{\mathbf{W}}}
\def\rmX{{\mathbf{X}}}
\def\rmY{{\mathbf{Y}}}
\def\rmZ{{\mathbf{Z}}}

% Elements of random matrices
\def\ermA{{\textnormal{A}}}
\def\ermB{{\textnormal{B}}}
\def\ermC{{\textnormal{C}}}
\def\ermD{{\textnormal{D}}}
\def\ermE{{\textnormal{E}}}
\def\ermF{{\textnormal{F}}}
\def\ermG{{\textnormal{G}}}
\def\ermH{{\textnormal{H}}}
\def\ermI{{\textnormal{I}}}
\def\ermJ{{\textnormal{J}}}
\def\ermK{{\textnormal{K}}}
\def\ermL{{\textnormal{L}}}
\def\ermM{{\textnormal{M}}}
\def\ermN{{\textnormal{N}}}
\def\ermO{{\textnormal{O}}}
\def\ermP{{\textnormal{P}}}
\def\ermQ{{\textnormal{Q}}}
\def\ermR{{\textnormal{R}}}
\def\ermS{{\textnormal{S}}}
\def\ermT{{\textnormal{T}}}
\def\ermU{{\textnormal{U}}}
\def\ermV{{\textnormal{V}}}
\def\ermW{{\textnormal{W}}}
\def\ermX{{\textnormal{X}}}
\def\ermY{{\textnormal{Y}}}
\def\ermZ{{\textnormal{Z}}}

% Vectors
\def\vzero{{\bm{0}}}
\def\vone{{\bm{1}}}
\def\vmu{{\bm{\mu}}}
\def\vtheta{{\bm{\theta}}}
\def\vphi{{\bm{\phi}}}
\def\va{{\bm{a}}}
\def\vb{{\bm{b}}}
\def\vc{{\bm{c}}}
\def\vd{{\bm{d}}}
\def\ve{{\bm{e}}}
\def\vf{{\bm{f}}}
\def\vg{{\bm{g}}}
\def\vh{{\bm{h}}}
\def\vi{{\bm{i}}}
\def\vj{{\bm{j}}}
\def\vk{{\bm{k}}}
\def\vl{{\bm{l}}}
\def\vm{{\bm{m}}}
\def\vn{{\bm{n}}}
\def\vo{{\bm{o}}}
\def\vp{{\bm{p}}}
\def\vq{{\bm{q}}}
\def\vr{{\bm{r}}}
\def\vs{{\bm{s}}}
\def\vt{{\bm{t}}}
\def\vu{{\bm{u}}}
\def\vv{{\bm{v}}}
\def\vw{{\bm{w}}}
\def\vx{{\bm{x}}}
\def\vy{{\bm{y}}}
\def\vz{{\bm{z}}}

% Elements of vectors
\def\evalpha{{\alpha}}
\def\evbeta{{\beta}}
\def\evepsilon{{\epsilon}}
\def\evlambda{{\lambda}}
\def\evomega{{\omega}}
\def\evmu{{\mu}}
\def\evpsi{{\psi}}
\def\evsigma{{\sigma}}
\def\evtheta{{\theta}}
\def\eva{{a}}
\def\evb{{b}}
\def\evc{{c}}
\def\evd{{d}}
\def\eve{{e}}
\def\evf{{f}}
\def\evg{{g}}
\def\evh{{h}}
\def\evi{{i}}
\def\evj{{j}}
\def\evk{{k}}
\def\evl{{l}}
\def\evm{{m}}
\def\evn{{n}}
\def\evo{{o}}
\def\evp{{p}}
\def\evq{{q}}
\def\evr{{r}}
\def\evs{{s}}
\def\evt{{t}}
\def\evu{{u}}
\def\evv{{v}}
\def\evw{{w}}
\def\evx{{x}}
\def\evy{{y}}
\def\evz{{z}}

% Matrix
\def\mA{{\bm{A}}}
\def\mB{{\bm{B}}}
\def\mC{{\bm{C}}}
\def\mD{{\bm{D}}}
\def\mE{{\bm{E}}}
\def\mF{{\bm{F}}}
\def\mG{{\bm{G}}}
\def\mH{{\bm{H}}}
\def\mI{{\bm{I}}}
\def\mJ{{\bm{J}}}
\def\mK{{\bm{K}}}
\def\mL{{\bm{L}}}
\def\mM{{\bm{M}}}
\def\mN{{\bm{N}}}
\def\mO{{\bm{O}}}
\def\mP{{\bm{P}}}
\def\mQ{{\bm{Q}}}
\def\mR{{\bm{R}}}
\def\mS{{\bm{S}}}
\def\mT{{\bm{T}}}
\def\mU{{\bm{U}}}
\def\mV{{\bm{V}}}
\def\mW{{\bm{W}}}
\def\mX{{\bm{X}}}
\def\mY{{\bm{Y}}}
\def\mZ{{\bm{Z}}}
\def\mBeta{{\bm{\beta}}}
\def\mPhi{{\bm{\Phi}}}
\def\mLambda{{\bm{\Lambda}}}
\def\mSigma{{\bm{\Sigma}}}

% Tensor
\DeclareMathAlphabet{\mathsfit}{\encodingdefault}{\sfdefault}{m}{sl}
\SetMathAlphabet{\mathsfit}{bold}{\encodingdefault}{\sfdefault}{bx}{n}
\newcommand{\tens}[1]{\bm{\mathsfit{#1}}}
\def\tA{{\tens{A}}}
\def\tB{{\tens{B}}}
\def\tC{{\tens{C}}}
\def\tD{{\tens{D}}}
\def\tE{{\tens{E}}}
\def\tF{{\tens{F}}}
\def\tG{{\tens{G}}}
\def\tH{{\tens{H}}}
\def\tI{{\tens{I}}}
\def\tJ{{\tens{J}}}
\def\tK{{\tens{K}}}
\def\tL{{\tens{L}}}
\def\tM{{\tens{M}}}
\def\tN{{\tens{N}}}
\def\tO{{\tens{O}}}
\def\tP{{\tens{P}}}
\def\tQ{{\tens{Q}}}
\def\tR{{\tens{R}}}
\def\tS{{\tens{S}}}
\def\tT{{\tens{T}}}
\def\tU{{\tens{U}}}
\def\tV{{\tens{V}}}
\def\tW{{\tens{W}}}
\def\tX{{\tens{X}}}
\def\tY{{\tens{Y}}}
\def\tZ{{\tens{Z}}}


% Graph
\def\gA{{\mathcal{A}}}
\def\gB{{\mathcal{B}}}
\def\gC{{\mathcal{C}}}
\def\gD{{\mathcal{D}}}
\def\gE{{\mathcal{E}}}
\def\gF{{\mathcal{F}}}
\def\gG{{\mathcal{G}}}
\def\gH{{\mathcal{H}}}
\def\gI{{\mathcal{I}}}
\def\gJ{{\mathcal{J}}}
\def\gK{{\mathcal{K}}}
\def\gL{{\mathcal{L}}}
\def\gM{{\mathcal{M}}}
\def\gN{{\mathcal{N}}}
\def\gO{{\mathcal{O}}}
\def\gP{{\mathcal{P}}}
\def\gQ{{\mathcal{Q}}}
\def\gR{{\mathcal{R}}}
\def\gS{{\mathcal{S}}}
\def\gT{{\mathcal{T}}}
\def\gU{{\mathcal{U}}}
\def\gV{{\mathcal{V}}}
\def\gW{{\mathcal{W}}}
\def\gX{{\mathcal{X}}}
\def\gY{{\mathcal{Y}}}
\def\gZ{{\mathcal{Z}}}

% Sets
\def\sA{{\mathbb{A}}}
\def\sB{{\mathbb{B}}}
\def\sC{{\mathbb{C}}}
\def\sD{{\mathbb{D}}}
% Don't use a set called E, because this would be the same as our symbol
% for expectation.
\def\sF{{\mathbb{F}}}
\def\sG{{\mathbb{G}}}
\def\sH{{\mathbb{H}}}
\def\sI{{\mathbb{I}}}
\def\sJ{{\mathbb{J}}}
\def\sK{{\mathbb{K}}}
\def\sL{{\mathbb{L}}}
\def\sM{{\mathbb{M}}}
\def\sN{{\mathbb{N}}}
\def\sO{{\mathbb{O}}}
\def\sP{{\mathbb{P}}}
\def\sQ{{\mathbb{Q}}}
\def\sR{{\mathbb{R}}}
\def\sS{{\mathbb{S}}}
\def\sT{{\mathbb{T}}}
\def\sU{{\mathbb{U}}}
\def\sV{{\mathbb{V}}}
\def\sW{{\mathbb{W}}}
\def\sX{{\mathbb{X}}}
\def\sY{{\mathbb{Y}}}
\def\sZ{{\mathbb{Z}}}

% Entries of a matrix
\def\emLambda{{\Lambda}}
\def\emA{{A}}
\def\emB{{B}}
\def\emC{{C}}
\def\emD{{D}}
\def\emE{{E}}
\def\emF{{F}}
\def\emG{{G}}
\def\emH{{H}}
\def\emI{{I}}
\def\emJ{{J}}
\def\emK{{K}}
\def\emL{{L}}
\def\emM{{M}}
\def\emN{{N}}
\def\emO{{O}}
\def\emP{{P}}
\def\emQ{{Q}}
\def\emR{{R}}
\def\emS{{S}}
\def\emT{{T}}
\def\emU{{U}}
\def\emV{{V}}
\def\emW{{W}}
\def\emX{{X}}
\def\emY{{Y}}
\def\emZ{{Z}}
\def\emSigma{{\Sigma}}

% entries of a tensor
% Same font as tensor, without \bm wrapper
\newcommand{\etens}[1]{\mathsfit{#1}}
\def\etLambda{{\etens{\Lambda}}}
\def\etA{{\etens{A}}}
\def\etB{{\etens{B}}}
\def\etC{{\etens{C}}}
\def\etD{{\etens{D}}}
\def\etE{{\etens{E}}}
\def\etF{{\etens{F}}}
\def\etG{{\etens{G}}}
\def\etH{{\etens{H}}}
\def\etI{{\etens{I}}}
\def\etJ{{\etens{J}}}
\def\etK{{\etens{K}}}
\def\etL{{\etens{L}}}
\def\etM{{\etens{M}}}
\def\etN{{\etens{N}}}
\def\etO{{\etens{O}}}
\def\etP{{\etens{P}}}
\def\etQ{{\etens{Q}}}
\def\etR{{\etens{R}}}
\def\etS{{\etens{S}}}
\def\etT{{\etens{T}}}
\def\etU{{\etens{U}}}
\def\etV{{\etens{V}}}
\def\etW{{\etens{W}}}
\def\etX{{\etens{X}}}
\def\etY{{\etens{Y}}}
\def\etZ{{\etens{Z}}}

% The true underlying data generating distribution
\newcommand{\pdata}{p_{\rm{data}}}
\newcommand{\ptarget}{p_{\rm{target}}}
\newcommand{\pprior}{p_{\rm{prior}}}
\newcommand{\pbase}{p_{\rm{base}}}
\newcommand{\pref}{p_{\rm{ref}}}

% The empirical distribution defined by the training set
\newcommand{\ptrain}{\hat{p}_{\rm{data}}}
\newcommand{\Ptrain}{\hat{P}_{\rm{data}}}
% The model distribution
\newcommand{\pmodel}{p_{\rm{model}}}
\newcommand{\Pmodel}{P_{\rm{model}}}
\newcommand{\ptildemodel}{\tilde{p}_{\rm{model}}}
% Stochastic autoencoder distributions
\newcommand{\pencode}{p_{\rm{encoder}}}
\newcommand{\pdecode}{p_{\rm{decoder}}}
\newcommand{\precons}{p_{\rm{reconstruct}}}

\newcommand{\laplace}{\mathrm{Laplace}} % Laplace distribution

\newcommand{\E}{\mathbb{E}}
\newcommand{\Ls}{\mathcal{L}}
\newcommand{\R}{\mathbb{R}}
\newcommand{\emp}{\tilde{p}}
\newcommand{\lr}{\alpha}
\newcommand{\reg}{\lambda}
\newcommand{\rect}{\mathrm{rectifier}}
\newcommand{\softmax}{\mathrm{softmax}}
\newcommand{\sigmoid}{\sigma}
\newcommand{\softplus}{\zeta}
\newcommand{\KL}{D_{\mathrm{KL}}}
\newcommand{\Var}{\mathrm{Var}}
\newcommand{\standarderror}{\mathrm{SE}}
\newcommand{\Cov}{\mathrm{Cov}}
% Wolfram Mathworld says $L^2$ is for function spaces and $\ell^2$ is for vectors
% But then they seem to use $L^2$ for vectors throughout the site, and so does
% wikipedia.
\newcommand{\normlzero}{L^0}
\newcommand{\normlone}{L^1}
\newcommand{\normltwo}{L^2}
\newcommand{\normlp}{L^p}
\newcommand{\normmax}{L^\infty}

\newcommand{\parents}{Pa} % See usage in notation.tex. Chosen to match Daphne's book.

\DeclareMathOperator*{\argmax}{arg\,max}
\DeclareMathOperator*{\argmin}{arg\,min}

\DeclareMathOperator{\sign}{sign}
\DeclareMathOperator{\Tr}{Tr}
\let\ab\allowbreak


%%%%%%%%%%%---SETME-----%%%%%%%%%%%%%
%replace @@ with the submission number submission site.
\newcommand{\thiswork}{INF$^2$\xspace}
%%%%%%%%%%%%%%%%%%%%%%%%%%%%%%%%%%%%


%\newcommand{\rev}[1]{{\color{olivegreen}#1}}
\newcommand{\rev}[1]{{#1}}


\newcommand{\JL}[1]{{\color{cyan}[\textbf{\sc JLee}: \textit{#1}]}}
\newcommand{\JW}[1]{{\color{orange}[\textbf{\sc JJung}: \textit{#1}]}}
\newcommand{\JY}[1]{{\color{blue(ncs)}[\textbf{\sc JSong}: \textit{#1}]}}
\newcommand{\HS}[1]{{\color{magenta}[\textbf{\sc HJang}: \textit{#1}]}}
\newcommand{\CS}[1]{{\color{navy}[\textbf{\sc CShin}: \textit{#1}]}}
\newcommand{\SN}[1]{{\color{olive}[\textbf{\sc SNoh}: \textit{#1}]}}

%\def\final{}   % uncomment this for the submission version
\ifdefined\final
\renewcommand{\JL}[1]{}
\renewcommand{\JW}[1]{}
\renewcommand{\JY}[1]{}
\renewcommand{\HS}[1]{}
\renewcommand{\CS}[1]{}
\renewcommand{\SN}[1]{}
\fi

%%% Notion for baseline approaches %%% 
\newcommand{\baseline}{offloading-based batched inference\xspace}
\newcommand{\Baseline}{Offloading-based batched inference\xspace}


\newcommand{\ans}{attention-near storage\xspace}
\newcommand{\Ans}{Attention-near storage\xspace}
\newcommand{\ANS}{Attention-Near Storage\xspace}

\newcommand{\wb}{delayed KV cache writeback\xspace}
\newcommand{\Wb}{Delayed KV cache writeback\xspace}
\newcommand{\WB}{Delayed KV Cache Writeback\xspace}

\newcommand{\xcache}{X-cache\xspace}
\newcommand{\XCACHE}{X-Cache\xspace}


%%% Notions for our methods %%%
\newcommand{\schemea}{\textbf{Expanding supported maximum sequence length with optimized performance}\xspace}
\newcommand{\Schemea}{\textbf{Expanding supported maximum sequence length with optimized performance}\xspace}

\newcommand{\schemeb}{\textbf{Optimizing the storage device performance}\xspace}
\newcommand{\Schemeb}{\textbf{Optimizing the storage device performance}\xspace}

\newcommand{\schemec}{\textbf{Orthogonally supporting Compression Techniques}\xspace}
\newcommand{\Schemec}{\textbf{Orthogonally supporting Compression Techniques}\xspace}



% Circular numbers
\usepackage{tikz}
\newcommand*\circled[1]{\tikz[baseline=(char.base)]{
            \node[shape=circle,draw,inner sep=0.4pt] (char) {#1};}}

\newcommand*\bcircled[1]{\tikz[baseline=(char.base)]{
            \node[shape=circle,draw,inner sep=0.4pt, fill=black, text=white] (char) {#1};}}



%%%%%%%%%%%%%%%%%%%%%%%%%%%%%%%%
% END OF USER DEFINED
%%%%%%%%%%%%%%%%%%%%%%%%%%%%%%%%

% The \icmltitle you define below is probably too long as a header.
% Therefore, a short form for the running title is supplied here:
\icmltitlerunning{\ampo: Active Multi Preference Optimization}

\begin{document}

\twocolumn[
% \icmltitle{\swepo: Simultaneous Weighted Preference Optimization\newline for Group Contrastive Alignment}
\icmltitle{\ampo: Active Multi Preference Optimization for Self-play Preference Selection}

% It is OKAY to include author information, even for blind
% submissions: the style file will automatically remove it for you
% unless you've provided the [accepted] option to the icml2025
% package.

% List of affiliations: The first argument should be a (short)
% identifier you will use later to specify author affiliations
% Academic affiliations should list Department, University, City, Region, Country
% Industry affiliations should list Company, City, Region, Country

% You can specify symbols, otherwise they are numbered in order.
% Ideally, you should not use this facility. Affiliations will be numbered
% in order of appearance and this is the preferred way.
\icmlsetsymbol{equal}{*}

\begin{icmlauthorlist}
\icmlauthor{Taneesh Gupta}{equal,comp}
\icmlauthor{Rahul Madhavan}{equal,yyy}
\icmlauthor{Xuchao Zhang}{comp}
\icmlauthor{Chetan Bansal}{comp}
\icmlauthor{Saravan Rajmohan}{comp}

%\icmlauthor{}{sch}
% \icmlauthor{Firstname8 Lastname8}{sch}
% \icmlauthor{Firstname8 Lastname8}{yyy,comp}
%\icmlauthor{}{sch}
%\icmlauthor{}{sch}
\end{icmlauthorlist}

\icmlaffiliation{yyy}{IISC}
\icmlaffiliation{comp}{Microsoft}
% \icmlaffiliation{sch}{School of ZZZ, Institute of WWW, Location, Country}

\icmlcorrespondingauthor{Taneesh Gupta}{t-taneegupta@microsoft.com}
\icmlcorrespondingauthor{Rahul Madhavan}{mrahul@iisc.com}

% % You may provide any keywords that you
% % find helpful for describing your paper; these are used to populate
% % the "keywords" metadata in the PDF but will not be shown in the document
% \icmlkeywords{Machine Learning, ICML}

% \vskip 0.3in

% \icmlsetsymbol{equal}{*}

% \begin{icmlauthorlist}
% \icmlauthor{Firstname1 Lastname1}{equal,yyy}
% \icmlauthor{Firstname2 Lastname2}{equal,yyy,comp}
% \icmlauthor{Firstname3 Lastname3}{comp}
% \icmlauthor{Firstname4 Lastname4}{sch}
% \icmlauthor{Firstname5 Lastname5}{yyy}
% \icmlauthor{Firstname6 Lastname6}{sch,yyy,comp}
% \icmlauthor{Firstname7 Lastname7}{comp}
% %\icmlauthor{}{sch}
% \icmlauthor{Firstname8 Lastname8}{sch}
% \icmlauthor{Firstname8 Lastname8}{yyy,comp}
% %\icmlauthor{}{sch}
% %\icmlauthor{}{sch}
% \end{icmlauthorlist}

% \icmlaffiliation{yyy}{Department of XXX, University of YYY, Location, Country}
% \icmlaffiliation{comp}{Company Name, Location, Country}
% \icmlaffiliation{sch}{School of ZZZ, Institute of WWW, Location, Country}

% \icmlcorrespondingauthor{Firstname1 Lastname1}{first1.last1@xxx.edu}
% \icmlcorrespondingauthor{Firstname2 Lastname2}{first2.last2@www.uk}

% You may provide any keywords that you
% find helpful for describing your paper; these are used to populate
% the "keywords" metadata in the PDF but will not be shown in the document
\icmlkeywords{Machine Learning, ICML}

\vskip 0.3in

]

% this must go after the closing bracket ] following \twocolumn[ ...

% This command actually creates the footnote in the first column
% listing the affiliations and the copyright notice.
% The command takes one argument, which is text to display at the start of the footnote.
% The \icmlEqualContribution command is standard text for equal contribution.
% Remove it (just {}) if you do not need this facility.

%\printAffiliationsAndNotice{}  % leave blank if no need to mention equal contribution
% \printAffiliationsAndNotice{\icmlEqualContribution} % otherwise use the standard text.

\printAffiliationsAndNotice{\icmlEqualContribution}

% \begin{abstract}  
Test time scaling is currently one of the most active research areas that shows promise after training time scaling has reached its limits.
Deep-thinking (DT) models are a class of recurrent models that can perform easy-to-hard generalization by assigning more compute to harder test samples.
However, due to their inability to determine the complexity of a test sample, DT models have to use a large amount of computation for both easy and hard test samples.
Excessive test time computation is wasteful and can cause the ``overthinking'' problem where more test time computation leads to worse results.
In this paper, we introduce a test time training method for determining the optimal amount of computation needed for each sample during test time.
We also propose Conv-LiGRU, a novel recurrent architecture for efficient and robust visual reasoning. 
Extensive experiments demonstrate that Conv-LiGRU is more stable than DT, effectively mitigates the ``overthinking'' phenomenon, and achieves superior accuracy.
\end{abstract}  

\begin{abstract}  
Test time scaling is currently one of the most active research areas that shows promise after training time scaling has reached its limits.
Deep-thinking (DT) models are a class of recurrent models that can perform easy-to-hard generalization by assigning more compute to harder test samples.
However, due to their inability to determine the complexity of a test sample, DT models have to use a large amount of computation for both easy and hard test samples.
Excessive test time computation is wasteful and can cause the ``overthinking'' problem where more test time computation leads to worse results.
In this paper, we introduce a test time training method for determining the optimal amount of computation needed for each sample during test time.
We also propose Conv-LiGRU, a novel recurrent architecture for efficient and robust visual reasoning. 
Extensive experiments demonstrate that Conv-LiGRU is more stable than DT, effectively mitigates the ``overthinking'' phenomenon, and achieves superior accuracy.
\end{abstract}  
\section{Introduction}
\label{sec:introduction}
The business processes of organizations are experiencing ever-increasing complexity due to the large amount of data, high number of users, and high-tech devices involved \cite{martin2021pmopportunitieschallenges, beerepoot2023biggestbpmproblems}. This complexity may cause business processes to deviate from normal control flow due to unforeseen and disruptive anomalies \cite{adams2023proceddsriftdetection}. These control-flow anomalies manifest as unknown, skipped, and wrongly-ordered activities in the traces of event logs monitored from the execution of business processes \cite{ko2023adsystematicreview}. For the sake of clarity, let us consider an illustrative example of such anomalies. Figure \ref{FP_ANOMALIES} shows a so-called event log footprint, which captures the control flow relations of four activities of a hypothetical event log. In particular, this footprint captures the control-flow relations between activities \texttt{a}, \texttt{b}, \texttt{c} and \texttt{d}. These are the causal ($\rightarrow$) relation, concurrent ($\parallel$) relation, and other ($\#$) relations such as exclusivity or non-local dependency \cite{aalst2022pmhandbook}. In addition, on the right are six traces, of which five exhibit skipped, wrongly-ordered and unknown control-flow anomalies. For example, $\langle$\texttt{a b d}$\rangle$ has a skipped activity, which is \texttt{c}. Because of this skipped activity, the control-flow relation \texttt{b}$\,\#\,$\texttt{d} is violated, since \texttt{d} directly follows \texttt{b} in the anomalous trace.
\begin{figure}[!t]
\centering
\includegraphics[width=0.9\columnwidth]{images/FP_ANOMALIES.png}
\caption{An example event log footprint with six traces, of which five exhibit control-flow anomalies.}
\label{FP_ANOMALIES}
\end{figure}

\subsection{Control-flow anomaly detection}
Control-flow anomaly detection techniques aim to characterize the normal control flow from event logs and verify whether these deviations occur in new event logs \cite{ko2023adsystematicreview}. To develop control-flow anomaly detection techniques, \revision{process mining} has seen widespread adoption owing to process discovery and \revision{conformance checking}. On the one hand, process discovery is a set of algorithms that encode control-flow relations as a set of model elements and constraints according to a given modeling formalism \cite{aalst2022pmhandbook}; hereafter, we refer to the Petri net, a widespread modeling formalism. On the other hand, \revision{conformance checking} is an explainable set of algorithms that allows linking any deviations with the reference Petri net and providing the fitness measure, namely a measure of how much the Petri net fits the new event log \cite{aalst2022pmhandbook}. Many control-flow anomaly detection techniques based on \revision{conformance checking} (hereafter, \revision{conformance checking}-based techniques) use the fitness measure to determine whether an event log is anomalous \cite{bezerra2009pmad, bezerra2013adlogspais, myers2018icsadpm, pecchia2020applicationfailuresanalysispm}. 

The scientific literature also includes many \revision{conformance checking}-independent techniques for control-flow anomaly detection that combine specific types of trace encodings with machine/deep learning \cite{ko2023adsystematicreview, tavares2023pmtraceencoding}. Whereas these techniques are very effective, their explainability is challenging due to both the type of trace encoding employed and the machine/deep learning model used \cite{rawal2022trustworthyaiadvances,li2023explainablead}. Hence, in the following, we focus on the shortcomings of \revision{conformance checking}-based techniques to investigate whether it is possible to support the development of competitive control-flow anomaly detection techniques while maintaining the explainable nature of \revision{conformance checking}.
\begin{figure}[!t]
\centering
\includegraphics[width=\columnwidth]{images/HIGH_LEVEL_VIEW.png}
\caption{A high-level view of the proposed framework for combining \revision{process mining}-based feature extraction with dimensionality reduction for control-flow anomaly detection.}
\label{HIGH_LEVEL_VIEW}
\end{figure}

\subsection{Shortcomings of \revision{conformance checking}-based techniques}
Unfortunately, the detection effectiveness of \revision{conformance checking}-based techniques is affected by noisy data and low-quality Petri nets, which may be due to human errors in the modeling process or representational bias of process discovery algorithms \cite{bezerra2013adlogspais, pecchia2020applicationfailuresanalysispm, aalst2016pm}. Specifically, on the one hand, noisy data may introduce infrequent and deceptive control-flow relations that may result in inconsistent fitness measures, whereas, on the other hand, checking event logs against a low-quality Petri net could lead to an unreliable distribution of fitness measures. Nonetheless, such Petri nets can still be used as references to obtain insightful information for \revision{process mining}-based feature extraction, supporting the development of competitive and explainable \revision{conformance checking}-based techniques for control-flow anomaly detection despite the problems above. For example, a few works outline that token-based \revision{conformance checking} can be used for \revision{process mining}-based feature extraction to build tabular data and develop effective \revision{conformance checking}-based techniques for control-flow anomaly detection \cite{singh2022lapmsh, debenedictis2023dtadiiot}. However, to the best of our knowledge, the scientific literature lacks a structured proposal for \revision{process mining}-based feature extraction using the state-of-the-art \revision{conformance checking} variant, namely alignment-based \revision{conformance checking}.

\subsection{Contributions}
We propose a novel \revision{process mining}-based feature extraction approach with alignment-based \revision{conformance checking}. This variant aligns the deviating control flow with a reference Petri net; the resulting alignment can be inspected to extract additional statistics such as the number of times a given activity caused mismatches \cite{aalst2022pmhandbook}. We integrate this approach into a flexible and explainable framework for developing techniques for control-flow anomaly detection. The framework combines \revision{process mining}-based feature extraction and dimensionality reduction to handle high-dimensional feature sets, achieve detection effectiveness, and support explainability. Notably, in addition to our proposed \revision{process mining}-based feature extraction approach, the framework allows employing other approaches, enabling a fair comparison of multiple \revision{conformance checking}-based and \revision{conformance checking}-independent techniques for control-flow anomaly detection. Figure \ref{HIGH_LEVEL_VIEW} shows a high-level view of the framework. Business processes are monitored, and event logs obtained from the database of information systems. Subsequently, \revision{process mining}-based feature extraction is applied to these event logs and tabular data input to dimensionality reduction to identify control-flow anomalies. We apply several \revision{conformance checking}-based and \revision{conformance checking}-independent framework techniques to publicly available datasets, simulated data of a case study from railways, and real-world data of a case study from healthcare. We show that the framework techniques implementing our approach outperform the baseline \revision{conformance checking}-based techniques while maintaining the explainable nature of \revision{conformance checking}.

In summary, the contributions of this paper are as follows.
\begin{itemize}
    \item{
        A novel \revision{process mining}-based feature extraction approach to support the development of competitive and explainable \revision{conformance checking}-based techniques for control-flow anomaly detection.
    }
    \item{
        A flexible and explainable framework for developing techniques for control-flow anomaly detection using \revision{process mining}-based feature extraction and dimensionality reduction.
    }
    \item{
        Application to synthetic and real-world datasets of several \revision{conformance checking}-based and \revision{conformance checking}-independent framework techniques, evaluating their detection effectiveness and explainability.
    }
\end{itemize}

The rest of the paper is organized as follows.
\begin{itemize}
    \item Section \ref{sec:related_work} reviews the existing techniques for control-flow anomaly detection, categorizing them into \revision{conformance checking}-based and \revision{conformance checking}-independent techniques.
    \item Section \ref{sec:abccfe} provides the preliminaries of \revision{process mining} to establish the notation used throughout the paper, and delves into the details of the proposed \revision{process mining}-based feature extraction approach with alignment-based \revision{conformance checking}.
    \item Section \ref{sec:framework} describes the framework for developing \revision{conformance checking}-based and \revision{conformance checking}-independent techniques for control-flow anomaly detection that combine \revision{process mining}-based feature extraction and dimensionality reduction.
    \item Section \ref{sec:evaluation} presents the experiments conducted with multiple framework and baseline techniques using data from publicly available datasets and case studies.
    \item Section \ref{sec:conclusions} draws the conclusions and presents future work.
\end{itemize}
\subsection{Our Contributions}
\begin{itemize}[leftmargin=1em]
    \item \textbf{Algorithmic Novelty:} We propose \emph{Active Multi-Preference Optimization} (\ampo), an on-policy framework that blends group-based preference alignment with active subset selection without exhaustively training on all generated responses. This opens out avenues for research on how to select for synthetic data, as we outline in Sections \ref{sec:subset_selection_strategies} and \ref{sec:discussion_future_work}.
    \item \textbf{Theoretical Insights:} Under mild Lipschitz assumptions, we show that coverage-based negative selection can systematically suppress low-reward modes and maximizes expected reward. This analysis (in Sections \ref{sec:opt_select} and \ref{sec:theory_main}) connects our method to the weighted $k$-medoids problem, yielding performance guarantees for alignment.
    \item \textbf{State-of-the-Art Results:} Empirically, \ampo\ sets a new benchmark on \textit{AlpacaEval} with Llama 8B, surpassing strong baselines like $\simpo$ by focusing on a small but strategically chosen set of responses each iteration (see Section \ref{sec:experimental setup}).
    \item \textbf{Dataset Releases:} We publicly release our \href{https://huggingface.co/datasets/Multi-preference-Optimization/AMPO-Coreset-selection}{\texttt{AMPO-Coreset-Selection}} 
    and \href{https://huggingface.co/datasets/Multi-preference-Optimization/AMPO-OPT-Selection}{\texttt{AMPO-Opt-Selection}} datasets on Hugging Face. These contain curated response subsets for each prompt, facilitating research on multi-preference alignment.
\end{itemize}



\vspace{-0.1in}
\section{Notations and Preliminaries}
\label{sec:notations_preliminaries}

\vspace{-0.1in}
We focus on aligning a \emph{policy model} to human preferences in a single-round (one-shot) scenario. Our goal is to generate multiple candidate responses for each prompt, then actively select a small, high-impact subset for alignment via a group-contrastive objective.

\vspace{-0.1in}
\paragraph{Queries and Policy.}
Let $\mathcal{D} = \{x_1, x_2, \ldots, x_M\}$ be a dataset of $M$ \emph{queries} (or \emph{prompts}), each from a larger space $\mathcal{X}$. We have a policy model $P_\theta(y \mid x)$, parameterized by $\theta$, which produces a distribution over possible responses $y \in \mathcal{Y}$. To generate diverse answers, we sample from $P_\theta(y \mid x)$ at some fixed \emph{temperature} (e.g., $0.8$). Formally, for each $x_i$, we draw up to $N$ responses,

\vspace{-0.1in}
\begin{equation}
   \{y_{i,1}, y_{i,2}, \dots, y_{i,N}\}, 
\end{equation}

\vspace{-0.1in}
from $P_\theta(y \mid x_i)$. Such an \textbf{on-policy} sampling, ensures, we are able to provide preference feedback on queries that are chosen by the model.

\vspace{-0.1in}
For simplicity of notation, we shall presently consider a single query (prompt) \(x\) and sampled responses \(\{y_1,\dots,y_N\}\) from \(P_\theta(\cdot \mid x)\), from the autoregressive language model.

\vspace{-0.1in}
Each response \(y_i\) is assigned a scalar reward

\vspace{-0.1in}
\begin{equation}
r_i \;=\; \mathcal{R}(x,\,y_i) \;\in\; [0,1],
\end{equation}

\vspace{-0.1in}
where \(\mathcal{R}\) is a fixed reward function or model (not optimized during policy training). We also embed each response via \(\mathbf{e}_i = \mathcal{E}(y_i)\in \mathbb{R}^d\), where \(\mathcal{E}\) might be any sentence or document encoder capturing semantic or stylistic properties.

\vspace{-0.1in}
Although one could train on all \(N\) responses, doing so is often computationally expensive. We therefore \emph{select} a subset \(\mathcal{S}\subset \{1,\dots,N\}\) of size \(\lvert\mathcal{S}\rvert = K < N\) by maximizing some selection criterion (e.g.\ favoring high rewards, broad coverage in embedding space, or both). Formally,

\vspace{-0.15in}
\begin{equation}
\label{eq:subset_selection}
\mathcal{S}
\;=\;
\arg\max_{\substack{\mathcal{I}\subset\{1,\dots,N\} \\ \lvert\mathcal{I}\rvert = K}}
\,\mathcal{U}\Bigl(\{y_i\}_{i\in\mathcal{I}},\, \{r_i\}_{i\in\mathcal{I}},\, \{\mathbf{e}_i\}_{i\in\mathcal{I}}\Bigr),
\end{equation}

\vspace{-0.1in}
where \(\mathcal{U}\) is a \emph{utility function} tailored to emphasize extremes, diversity, or other alignment needs.

\vspace{-0.1in}
Next, we split \(\mathcal{S}\) into a \emph{positive} set \(\mathcal{S}^+\) and a \emph{negative} set \(\mathcal{S}^-\). For example, let 

\vspace{-0.05in}
\[
\overline{r} \;=\;
\frac{1}{K}\,\sum_{i\in \mathcal{S}}\,r_i
\]
\vspace{-0.05in}
be the average reward of the chosen subset, and define

\vspace{-0.05in}
\[
\mathcal{S}^+
\;=\;
\{\,i\in \mathcal{S}\,\mid\,r_i > \overline{r}\},
\quad
\mathcal{S}^- 
\;=\;
\{\,i\in \mathcal{S}\,\mid\,r_i \le \overline{r}\}.
\]

\vspace{-0.05in}
Hence, \(\mathcal{S} = \mathcal{S}^+\cup \mathcal{S}^-\) and \(\lvert \mathcal{S}^+\rvert + \lvert \mathcal{S}^-\rvert = K\).

We train \(\theta\) via a reference-free \emph{group-contrastive} objective known as \(\textsc{refa}\) \citep{gupta2024refa}. Concretely, define

\vspace{-0.15in}
\begin{equation}
L_{\text{swepo}}(\theta)
\;=\;
-\,\log\!\Biggl(\!
  \frac{
    \displaystyle
    \sum_{\,i \,\in\, \mathcal{S}^+}\;
    \exp\Bigl[
      s'_\theta\bigl(y_i \mid x\bigr)
    \Bigr]
  }{
    \displaystyle
    \sum_{\,i\,\in\,(\mathcal{S}^+\cup \mathcal{S}^-)}\;
    \exp\Bigl[
      s'_\theta\bigl(y_i \mid x\bigr)
    \Bigr]
  }
\Biggr),
\end{equation}

\vspace{-0.05in}
where

\vspace{-0.25in}
\[
s'_\theta\bigl(y_i \mid x\bigr)
  =
  \log P_\theta(y_i\mid x)
  +
  \alpha \,\bigl(r_i - \overline{r}\bigr).
\]

\vspace{-0.05in}
Here, \(P_{\mathrm{ref}}\) is a reference policy (e.g.\ an older snapshot of \(P_\theta\) or a baseline model), and \(\alpha\) is a hyperparameter scaling the reward difference. In words, \(\textsc{swepo}\) encourages the model to increase the log-probability of \(\mathcal{S}^+\) while decreasing that of \(\mathcal{S}^-\), all in a single contrastive term that accounts for multiple positives and negatives simultaneously.

Although presented for a single query \(x\), this procedure extends straightforwardly to any dataset \(\mathcal{D}\) by summing \(L_{\text{swepo}}\) across all queries. In subsequent sections, we discuss diverse strategies for selecting \(\mathcal{S}\) (and thus \(\mathcal{S}^+\) and \(\mathcal{S}^-\)), aiming to maximize training efficiency and alignment quality.




\section{Algorithm and Methodology}
\label{sec:methodology}

We outline a one-vs-$k$ selection scheme in which a single \emph{best} response is promoted (positive), while an \emph{active} subroutine selects $k$ negatives from the remaining $N-1$ candidates. This setup highlights the interplay of three main objectives:

\vspace{-0.1in}
\begin{description}[leftmargin=1em, itemsep=0pt]
   \item[Probability:] High-probability responses under $P_\theta(y\mid x)$ can dominate even if suboptimal by reward.
   \item[Rewards:] Simply selecting extremes by reward misses problematic "mediocre" outputs.
   \item[Semantics:] Diverse but undesired responses in distant embedding regions must be penalized.
\end{description}

\vspace{-0.15in}
While positives reinforce a single high-reward candidate, active negative selection balances probability, reward and diversity to systematically suppress problematic regions of the response space.

% \begin{description}[leftmargin=1em, itemsep=1pt]
%     \item[Probability (Model Likelihood):] Responses that have high probability under $P_\theta(y\mid x)$ can dominate the model’s output distribution; even if such responses are not the absolute worst by reward, leaving them unpenalized can hamper alignment.
%     \item[Rewards (Quality):] Purely selecting the top- or bottom-ranked by reward is insufficient when the model generates a variety of “mediocre” or “niche” outputs that still require demotion.
%     \item[Semantics (Diversity/Coverage):] Some undesired responses may differ substantially from the mainstream modes (e.g., lying in a distant embedding region). Penalizing them is vital, else the model remains vulnerable to sporadic but harmful outputs.
% \end{description}

% Balancing these three factors (model likelihood, reward, and diversity) is the cornerstone of \emph{active} negative selection. While the \emph{positive} ensures strong reinforcement of one high-reward candidate, the \emph{negatives} promote a broader shaping of the policy to down-weight potentially problematic regions. 

% \medskip
\noindent
\textbf{Algorithm.}
Formally, let $\{y_1,\dots,y_N\}$ be the sampled responses for a single prompt $x$. Suppose we have:\\
1. A reward function $r_i = \mathcal{R}(x,y_i)\in [0,1]$.\\
2. An embedding $\mathbf{e}_i = \mathcal{E}(y_i)$.\\
3. A model probability estimate $\pi_i = P_\theta(y_i\mid x)$.
% \begin{enumerate}
% \item 
% \item .
% \item 
% \end{enumerate}

Selection algorithms may be \textit{rating-based} selection (to identify truly poor or excellent answers) with \textit{coverage-based} selection (to explore distinct regions in the embedding space), we expose the model to both common and outlier responses. This ensures that the \textsc{swepo} loss provides strong gradient signals across the spectrum of answers the model is prone to generating. In Algorithm \ref{alg:one_vs_k_active}, $\textsc{ActiveSelection}(\cdot)$ is a generic subroutine that selects a set of $k$ “high-impact” negatives. We will detail concrete implementations (e.g.\ bottom-$k$ by rating, clustering-based, etc.) in later sections.



\begin{algorithm}[t]
\caption{\textcolor{titlecolor}{\textbf{$\ampo$: One-Positive vs.\ $k$-Active Negatives}}}
\label{alg:one_vs_k_active}
\begin{algorithmic}[1]
    \STATE \textcolor{inputcolor}{\textbf{Input:} (1) A set of $N$ responses $\{y_i\}$ sampled from $P_{\theta}(y\mid x)$; (2) Their rewards $\{r_i\}$, embeddings $\{\mathbf{e}_i\}$, and probabilities $\{\pi_i\}$; (3) Number of negatives $k$, reference policy $P_{\mathrm{ref}}$, and hyperparameter $\alpha$}
    \STATE \textcolor{outputcolor}{\textbf{Output:} (i) Positive $y_{+}$; (ii) Negatives $\{y_j\}_{j \in S^-}$; (iii) Updated parameters $\theta$ via \textsc{swepo}}
    \STATE \textcolor{stepcolor}{\textit{1. Select One Positive (Highest Reward)}}
    \STATE \textcolor{mathcolor}{$i_{+} \leftarrow \arg\max_{i=1,\dots,N} r_i$, \quad $y_{+} \leftarrow y_{\,i_{+}}$}
    \STATE \textcolor{stepcolor}{\textit{2. Choose $k$ Negatives via Active Selection}}
    \STATE \textcolor{mathcolor}{$\Omega \leftarrow \{1,\dots,N\}\setminus\{i_{+}\}$}
    \STATE \textcolor{mathcolor}{$S^- \leftarrow \textsc{ActiveSelection}(\Omega,\{r_i\},\{\mathbf{e}_i\},\{\pi_i\},k)$}
    \STATE \textcolor{stepcolor}{\textit{3. Form One-vs.-$k$ \textsc{swepo} Objective}}
    \STATE \textcolor{mathcolor}{$\overline{r} \leftarrow \frac{r_{\,i_{+}} + \sum_{j\,\in\,S^-} r_j}{1 + k}$}
    \STATE For each $y_i$:
    \STATE \textcolor{mathcolor}{$s'_\theta(y_i) = \log P_\theta(y_i \mid x) + \alpha(r_i - \overline{r})$}
    \STATE \textcolor{mathcolor}{$L_{\text{swepo}}(\theta) = -\log\!\Biggl(\frac{\exp\!\bigl[s'_\theta(y_{+})\bigr]}{\exp\!\bigl[s'_\theta(y_{+})\bigr] + \sum_{\,j \,\in\, S^-}\exp\!\bigl[s'_\theta(y_j)\bigr]}\Biggr)$}
    \STATE \textcolor{stepcolor}{\textit{4. Update Model Parameters:}} \textcolor{mathcolor}{$\theta \leftarrow \theta - \eta\,\nabla_\theta L_{\text{swepo}}(\theta)$}
    \RETURN The chosen positive $y_{+}$, the negative set $\{y_j\}_{j \in S^-}$, and the updated parameters $\theta$
\end{algorithmic}
\end{algorithm}

% \begin{algorithm}[t]
% \caption{\textbf{One-Positive vs.\ $k$-Active Negatives for group contrastive alignment via $\swepo$ loss}}
% \label{alg:one_vs_k_active}
% \begin{algorithmic}[1]
%     \STATE {\bfseries Input:} (1) A set of $N$ responses $\{y_i\}$ sampled from $P_{\theta}(y\mid x)$; (2) Their rewards $\{r_i\}$, embeddings $\{\mathbf{e}_i\}$, and probabilities $\{\pi_i\}$; (3) Number of negatives $k$, reference policy $P_{\mathrm{ref}}$, and hyperparameter $\alpha$
%     \STATE {\bfseries Output:} (i) Positive $y_{+}$; (ii) Negatives $\{y_j\}_{j \in S^-}$; (iii) Updated parameters $\theta$ via \textsc{swepo}
%     \STATE {\bfseries 1. Select One Positive (Highest Reward)}\\
%     $i_{+} \leftarrow \arg\max_{i=1,\dots,N} r_i$, \quad $y_{+} \leftarrow y_{\,i_{+}}$
%     \STATE {\bfseries 2. Choose $k$ Negatives via Active Selection}\\
%     $\Omega \leftarrow \{1,\dots,N\}\setminus\{i_{+}\}$\\ $S^- \leftarrow \textsc{ActiveSelection}(\Omega,\{r_i\},\{\mathbf{e}_i\},\{\pi_i\},k)$
%     \STATE {\bfseries 3. Form One-vs.-$k$ \textsc{swepo} Objective}\\
%     $\overline{r} \leftarrow \frac{r_{\,i_{+}} + \sum_{j\,\in\,S^-} r_j}{1 + k}$
%     \STATE For each $y_i$:\\ $s'_\theta(y_i) = \log P_\theta(y_i \mid x) - \log P_{\mathrm{ref}}(y_i \mid x) + \alpha(r_i - \overline{r})$
%     \STATE $L_{\text{swepo}}(\theta) = -\log\!\Biggl(\frac{\exp\!\bigl[s'_\theta(y_{+})\bigr]}{\exp\!\bigl[s'_\theta(y_{+})\bigr] + \sum_{\,j \,\in\, S^-}\exp\!\bigl[s'_\theta(y_j)\bigr]}\Biggr)$
%     \STATE {\bfseries 4. Update Model Parameters:} $\theta \leftarrow \theta - \eta\,\nabla_\theta L_{\text{swepo}}(\theta)$
%     \RETURN The chosen positive $y_{+}$, the negative set $\{y_j\}_{j \in S^-}$, and the updated parameters $\theta$
% \end{algorithmic}
% \end{algorithm}


\vspace{-0.1in}
\noindent
\subsection{Detailed Discussion of Algorithm \ref{alg:one_vs_k_active}}

\vspace{-0.1in}
The algorithm operates in four key steps: First, it selects the highest-reward response as the positive example (lines 3-4). Second, it actively selects $k$ negative examples by considering their rewards, probabilities $\pi_i$, and embedding distances $\mathbf{e}_i$ to capture diverse failure modes (lines 5-7). Third, it constructs the \textsc{swepo} objective by computing normalized scores $s'_\theta$ using the mean reward $\overline{r}$ and forming a one-vs-$k$ contrastive loss (lines 8-12). Finally, it updates the model parameters to increase the probability of the positive while suppressing the selected negatives (line 13). This approach ensures both reinforcement of high-quality responses and systematic penalization of problematic outputs across the response distribution.

% Step 1 ensures that at least one high-quality response is explicitly reinforced. Step 2 removes this positive from the candidate pool, then invokes an \emph{active} subroutine that must simultaneously account for reward (so that sufficiently poor answers are penalized), probability (so that common but flawed responses do not slip through), and diversity (so that rare but semantically distinct mistakes are also exposed). In Step 3, the one-vs.-$k$ \textsc{swepo} loss encourages the model to push up the log-probability of the positive while pushing down that of the $k$ negatives. Finally, Step 4 updates the parameters based on this contrast. 

% By highlighting a single “best” response and a carefully chosen set of “bad” or “undesired” responses, the algorithm provides a strong training signal across the reward distribution. Unlike naive strategies that choose negatives purely by lowest reward, our active-selection perspective seeks to uncover distinct semantic modes (via $\mathbf{e}_i$) or infrequent but still plausible outputs (via $\pi_i$). In this way, the model is given contrasting examples that refine its probability distribution to \emph{both} improve high-reward regions and suppress a diverse set of negative modes. 
% By coupling a single, clearly good response with an actively selected set of negative answers, the approach sculpts the model’s distribution towards several desirable outcomes in a single optimization step.



\vspace{-0.15in}
\section{Active Subset Selection Strategies}
\label{sec:subset_selection_strategies}

\vspace{-0.1in}
In this section, we present two straightforward yet effective strategies for actively selecting a small set of \emph{negative} responses in the \textsc{AMPO} framework. First, we describe a simple strategy, \emph{\textbf{AMPO-BottomK}}, that directly picks the lowest-rated responses. Second, we propose \emph{\textbf{AMPO-Coreset}}, a clustering-based method that selects exactly one negative from each cluster in the embedding space, thereby achieving broad coverage of semantically distinct regions. In Section \ref{sec:constant_factor_subset_selection}, we connect this clustering-based approach to the broader literature on \emph{coreset construction}, which deals with selecting representative subsets of data.

\vspace{-0.1in}
\subsection{AMPO-BottomK}
\label{sec:ampo_bottomk}

\vspace{-0.05in}
\noindent
\emph{AMPO-BottomK} is the most direct approach that we use for comparison: given $N$ sampled responses and their scalar ratings $\{r_i\}_{i=1}^N$, we simply pick the $k$ lowest-rated responses as negatives. This can be expressed as:

\vspace{-0.25in}
\begin{align}
\label{eq:bottomk_negatives}
S^- \;=\; \mathrm{argtopk}_{i}(-\,r_i,\,k),
\end{align}

\vspace{-0.1in}
which identifies the $k$ indices with smallest $r_i$. Although conceptually simple, this method can be quite effective when the reward function reliably indicates “bad” behavior. 
Furthermore to break-ties, we use minimal cosine similarity with the currently selected set.


% \vspace{0.5em}

% \begin{algorithm}[tb]
% \caption{$\ampobk$($\{r_i\}_{i=1}^N$, $k$)}
% \label{alg:bottom_k_negatives}
% \begin{algorithmic}[1]
%     \STATE {\bfseries Input:} 
%     \STATE (1) A set of $N$ responses with associated scalar ratings $\{r_i\}_{i=1}^N$
%     \STATE (2) Desired number of negatives $k$
%     \STATE 
%     \STATE {\bfseries Step 1:} Rank by rating
%     \STATE Sort indices $\{1,\dots,N\}$ so that $r_{(1)} \,\le\, r_{(2)} \,\le\, \dots \,\le\, r_{(N)}$,
%     \STATE where $r_{(1)}$ is the lowest rating
%     \STATE
%     \STATE {\bfseries Step 2:} Select bottom $k$
%     \STATE $S^- \;\leftarrow\; \{ (1),\, (2),\,\dots,\,(k)\}$
%     \STATE
%     \RETURN the $k$ indices in $S^-$ as the set of negatives
% \end{algorithmic}
% \end{algorithm}

\vspace{-0.1in}
\subsection{AMPO-Coreset (Clustering-Based Selection)}
\label{sec:ampo_coreset}

\vspace{-0.05in}
\noindent
$\ampobk$ may overlook problematic modes that are slightly better than the bottom-k, but fairly important to learn on. A diversity-driven approach, which we refer to as $\ampocs$, explicitly seeks coverage in the embedding space by partitioning the $N$ candidate responses into $k$ clusters and then selecting the lowest-rated response within each cluster. Formally:

\vspace{-0.15in}
\[
\label{eq:clustering_negatives}
i^-_j 
\;=\;
\arg\min_{\,i \,\in\,C_j}\; r_i, 
\,
j = 1,\dots,k, 
\,
S^- \;=\;\bigl\{\,i^-_1,\dots,i^-_k\bigr\}
\]

\vspace{-0.15in}
where $C_j$ is the set of responses assigned to cluster $j$ by a $k$-means algorithm (\citealt{har2004coresets,cohen2022improved}; see also Section \ref{sec:constant_factor_subset_selection}). The pseudo-code is provided in Algorithm \ref{alg:cluster_negatives}.


\begin{algorithm}[tb]
\caption{\textcolor{titlecolor}{$\ampocs$ via k-means}}
\label{alg:cluster_negatives}
\begin{algorithmic}[1]
    \STATE \textcolor{inputcolor}{\textbf{Input:}}
    \STATE \textcolor{inputcolor}{(1) $N$ responses, each with embedding $\mathbf{e}_i \in \mathbb{R}^d$ and rating $r_i$}
    \STATE \textcolor{inputcolor}{(2) Desired number of negatives $k$}
    \STATE
    \STATE \textcolor{stepcolor}{\textbf{Step 1:} \textit{Run $k$-means on embeddings}}
    \STATE \textcolor{mathcolor}{Initialize $\{\mathbf{c}_1,\dots,\mathbf{c}_k\} \subset \mathbb{R}^d$ (e.g., via $k$-means++)}
    \REPEAT
        \STATE \textcolor{mathcolor}{$\pi(i) = \arg\min_{1 \le j \le k} \|\mathbf{e}_i - \mathbf{c}_j\|^2$, \quad $i = 1,\dots,N$}
        \STATE \textcolor{mathcolor}{$\mathbf{c}_j = \frac{\sum_{i:\pi(i)=j}\mathbf{e}_i}{\sum_{i:\pi(i)=j}1}$, \quad $j = 1,\dots,k$} \label{eq:vanilla_kmeans}
    \UNTIL{convergence}
    \STATE
    \STATE \textcolor{stepcolor}{\textbf{Step 2:} \textit{In each cluster, pick the bottom-rated response}}
    \STATE \textcolor{mathcolor}{For each $j \in \{1,\dots,k\}$, define $C_j = \{\, i \mid \pi(i) = j \}$}
    \STATE \textcolor{mathcolor}{Then $i_j^- = \arg\min_{i\in C_j} r_i$, \quad $j = 1,\dots,k$}
    \STATE
    \STATE \textcolor{stepcolor}{\textbf{Step 3:} Return negatives}
    \STATE \textcolor{mathcolor}{$S^- = \{\, i_1^-,\, i_2^- ,\dots, i_k^- \}$}
    \RETURN \textcolor{outputcolor}{$S^-$ as the set of $k$ negatives}
\end{algorithmic}
\end{algorithm}

% \begin{algorithm}[tb]
% \caption{$\ampocs$($\{\mathbf{e}_i,r_i\}_{i=1}^N$, $k$)}
% \label{alg:cluster_negatives}
% \begin{algorithmic}[1]
%     \STATE {\bfseries Input:} 
%     \STATE (1) $N$ responses, each with embedding $\mathbf{e}_i \in \mathbb{R}^d$ and rating $r_i$
%     \STATE (2) Desired number of negatives $k$
%     \STATE
%     \STATE {\bfseries Step 1:} Run $k$-means on embeddings
%     \STATE Initialize $\{\mathbf{c}_1,\dots,\mathbf{c}_k\} \subset \mathbb{R}^d$ (e.g., via $k$-means++)
%     \REPEAT
%         \STATE $\pi(i) = \arg\min_{1 \le j \le k} \|\mathbf{e}_i - \mathbf{c}_j\|^2$, \quad $i = 1,\dots,N$
%         \STATE $\mathbf{c}_j = \frac{\sum_{i:\pi(i)=j}\mathbf{e}_i}{\sum_{i:\pi(i)=j}1}$, \quad $j = 1,\dots,k$ \label{eq:vanilla_kmeans}
%     \UNTIL{convergence}
%     \STATE
%     \STATE {\bfseries Step 2:} In each cluster, pick the bottom-rated response
%     \STATE For each $j \in \{1,\dots,k\}$, define $C_j = \{\, i \mid \pi(i) = j \}$
%     \STATE Then $i_j^- = \arg\min_{i\in C_j} r_i$, \quad $j = 1,\dots,k$
%     \STATE
%     \STATE {\bfseries Step 3:} Return negatives
%     \STATE $S^- = \{\, i_1^-,\, i_2^- ,\dots, i_k^- \}$
%     \RETURN $S^-$ as the set of $k$ negatives
% \end{algorithmic}
% \end{algorithm}



This approach enforces that each cluster---a potential ``mode'' in the response space---contributes at least one negative example. Hence, \textsc{AMPO-Coreset} can be interpreted as selecting \emph{representative} negatives from diverse semantic regions, ensuring that the model is penalized for a wide variety of undesired responses.


\section{Opt-Select: Active Subset Selection by Optimizing Expected Reward}
\label{sec:opt_select}

In this section, we propose \emph{Opt-Select}: a strategy for choosing $k$ \emph{negative} responses (plus one \emph{positive}) so as to \emph{maximize} the policy’s expected reward under a Lipschitz continuity assumption. Specifically, we model the local “neighborhood” influence of penalizing each selected negative and formulate an optimization problem that seeks to suppress large pockets of low-reward answers while preserving at least one high-reward mode. We first describe the intuition and objective, then present two solution methods: a \emph{mixed-integer program} (MIP) and a \emph{local search} approximation.

\subsection{Lipschitz-Driven Objective}
\label{subsec:lipschitz_objective}

Let $\{y_i\}_{i=1}^n$ be candidate responses sampled on-policy, each with reward $r_i \in [0,1]$ and embedding $\mathbf{e}_i \in \mathbb{R}^d$. Suppose that if we \emph{completely suppress} a response $y_j$ (i.e.\ set its probability to zero), all answers within distance $\|\mathbf{e}_i - \mathbf{e}_j\|$ must also decrease in probability proportionally, due to a Lipschitz constraint on the policy. Concretely, if the distance is $d_{i,j} = \|\mathbf{e}_i - \mathbf{e}_j\|$, and the model’s Lipschitz constant is $L$, then the probability of $y_i$ cannot remain above $L\,d_{i,j}$ if $y_j$ is forced to probability zero.

From an \emph{expected reward} perspective, assigning zero probability to \emph{low-reward} responses (and their neighborhoods) improves overall alignment. To capture this rigorously, observe that the \emph{penalty} from retaining a below-average answer $y_i$ can be weighted by:
\begin{align}
\label{eq:weight_w_i}
    w_i 
    \;=\;
    \exp\bigl(\,\overline{r} \;-\; r_i\bigr),
\end{align}
where $\overline{r}$ is (for instance) the mean reward of $\{r_i\}$. Intuitively, $w_i$ is larger for lower-reward $y_i$, indicating it is more harmful to let $y_i$ and its neighborhood remain at high probability.

Next, define a distance matrix 
\begin{align}
\label{eq:distance_matrix}
  A_{i,j} \;=\;
  \bigl\|\mathbf{e}_i - \mathbf{e}_j\bigr\|_2,
  \quad
  1 \le i,j \le n.
\end{align}
Selecting a subset $S\subseteq \{1,\dots,n\}$ of “negatives” to penalize suppresses the probability of each $i$ in proportion to $\min_{j \in S} A_{i,j}$. Consequently, a natural \emph{cost} function measures how much “weighted distance” $y_i$ has to its closest chosen negative:
\begin{align}
\label{eq:weighted_distance_cost}
    \text{Cost}(S)
    \;=\;
    \sum_{i=1}^n 
    w_i 
    \;\min_{\,j \in S}\;
    A_{i,j}.
\end{align}
Minimizing \eqref{eq:weighted_distance_cost} yields a subset $S$ of size $k$ that “covers” or “suppresses” as many low-reward responses (large $w_i$) as possible. We then \emph{add} one \emph{positive} index $i_{\mathrm{top}}$ with the highest $r_i$ to amplify a top-quality answer. This combination of \emph{one positive} plus \emph{$k$ negatives} provides a strong signal in the training loss.

\paragraph{Interpretation and Connection to Weighted k-medoids.}
If each negative $j$ “covers” responses $i$ within some radius (or cost) $A_{i,j}$, then \eqref{eq:weighted_distance_cost} is analogous to a weighted \emph{$k$-medoid} objective, where we choose $k$ items (negatives) to minimize a total weighted distance. Formally, this can be cast as a mixed-integer program (MIP) (Problem~\ref{eq:problem_P} below). For large $n$, local search offers an efficient approximation.

\subsection{Mixed-Integer Programming Formulation}

Define binary indicators $x_j = 1$ if we choose $y_j$ as a negative, and $z_{i,j} = 1$ if $i$ is assigned to $j$ (i.e.\ $\min_{j\in S} A_{i,j}$ is realized by $j$). We write:

\vspace{-0.15in}
\begin{align}
\label{eq:problem_P}
\textbf{Problem } \mathcal{P}: \quad
&\min_{\substack{x_j \in \{0,1\},\ z_{i,j}\in\{0,1\},\ y_i\ge 0}} 
 \sum_{i=1}^n w_i \,y_i 
\\
\text{s.t.}\quad
& \sum_{j=1}^n x_j = k, 
z_{i,j}\le x_j,
\sum_{j=1}^n z_{i,j} = 1, \forall\,i,\nonumber\\
& y_i \le A_{i,j} + M\,(1 - z_{i,j}), \nonumber\\
&y_i \ge A_{i,j} - M\,(1 - z_{i,j}),\quad\forall\,i,j,
\end{align}

\vspace{-0.1in}
where $M=\max_{i,j} A_{i,j}$. In essence, each $i$ is forced to \emph{assign} to exactly one chosen negative $j$, making $y_i = A_{i,j}$, i.e. the distance between the answer embeddings for answer $\{i,j\}$. Minimizing $\sum_i w_i\,y_i$ (i.e.\ \eqref{eq:weighted_distance_cost}) then ensures that low-reward points ($w_i$ large) lie close to at least one penalized center.

\vspace{-0.1in}
\paragraph{Algorithmic Overview.}
Solving $\mathcal{P}$ gives the $k$ negatives $S_{\mathrm{neg}}$, while the highest-reward index $i_{\mathrm{top}}$ is chosen as a positive. The final subset $\{i_{\mathrm{top}}\}\cup S_{\mathrm{neg}}$ is then passed to the \textsc{swepo} loss (see Section \ref{sec:methodology}). Algorithm~\ref{alg:opt_select} outlines the procedure succinctly.


\begin{algorithm}[t]
\caption{\textcolor{titlecolor}{$\ampoos$ via Solving MIP}}
\label{alg:opt_select}
\begin{algorithmic}[1]
    \STATE \textcolor{inputcolor}{\textbf{Input:} Candidates $\{y_i\}_{i=1}^n$ with $r_i, \mathbf{e}_i$; integer $k$}
    \STATE \textcolor{mathcolor}{Compute $i_{\mathrm{top}} = \arg\max_i\,r_i$}
    \STATE \textcolor{mathcolor}{Let $w_i = \exp(\,\overline{r} - r_i)$ with $\overline{r}$ as mean reward}
    \STATE \textcolor{mathcolor}{Solve Problem~\eqref{eq:problem_P} to get $\{x_j^*\}, \{z_{i,j}^*\}, \{y_i^*\}$}
    \STATE \textcolor{mathcolor}{Let $S_{\mathrm{neg}} = \{\,j \mid x_j^*=1\}$ (size $k$)}
    \RETURN \textcolor{outputcolor}{$\{\,i_{\mathrm{top}}\}\cup S_{\mathrm{neg}}$ for \textsc{swepo} training}
\end{algorithmic}
\end{algorithm}

% \begin{algorithm}[t]
% \caption{$\ampoos$ via Solving MIP}
% \label{alg:opt_select}
% \begin{algorithmic}[1]
%     \STATE {\bfseries Input:} Candidates $\{y_i\}_{i=1}^n$ with $r_i, \mathbf{e}_i$; integer $k$
%     \STATE Compute $i_{\mathrm{top}} = \arg\max_i\,r_i$
%     \STATE Let $w_i = \exp(\,\overline{r} - r_i)$ with $\overline{r}$ as mean reward
%     \STATE Solve Problem~\eqref{eq:problem_P} to get $\{x_j^*\}, \{z_{i,j}^*\}, \{y_i^*\}$
%     \STATE Let $S_{\mathrm{neg}} = \{\,j \mid x_j^*=1\}$ (size $k$)
%     \RETURN $\{\,i_{\mathrm{top}}\}\cup S_{\mathrm{neg}}$ for \textsc{swepo} training
% \end{algorithmic}
% \end{algorithm}

\vspace{-0.1in}
\subsection{Local Search Approximation}

\vspace{-0.1in}
For large $n$, an exact MIP can be expensive. A simpler \emph{local search} approach initializes a random subset $S$ of size $k$ and iteratively swaps elements in and out if it lowers the cost \eqref{eq:weighted_distance_cost}. In practice, this provides an efficient approximation, especially when $n$ or $k$ grows.

\begin{algorithm}[t]
\caption{\textcolor{titlecolor}{$\ampoos$ via Coordinate Descent}}
\label{alg:opt_select_local_search}
\begin{algorithmic}[1]
    \STATE \textcolor{inputcolor}{\textbf{Input:} Set $I = \{1,\dots,n\}$, integer $k$, distances $A_{i,j}$, rewards $\{r_i\}$}
    \STATE \textcolor{mathcolor}{Find $i_{\mathrm{top}} = \arg\max_{i}\, r_i$}
    \STATE \textcolor{mathcolor}{Compute $w_i = \exp(\,\overline{r} - r_i)$ and $d_{i,j}=A_{i,j}$}
    \STATE \textcolor{mathcolor}{Initialize a random subset $S \subseteq I\setminus\{i_{\mathrm{top}}\}$ of size $k$}
    \WHILE{improving}
        \STATE \textcolor{mathcolor}{Swap $j_{\mathrm{out}} \in S$ with $j_{\mathrm{in}} \notin S$ if it decreases $\sum_{i \in I} w_i\,\min_{j \in S} d_{i,j}$}
    \ENDWHILE
    \RETURN \textcolor{outputcolor}{$S_{\mathrm{neg}}=S$ (negatives) and $i_{\mathrm{top}}$ (positive)}
\end{algorithmic}
\end{algorithm}

% \begin{algorithm}[t]
% \caption{$\ampoos$ via Coordinate Descent}
% \label{alg:opt_select_local_search}
% \begin{algorithmic}[1]
%     \STATE {\bfseries Input:} Set $I = \{1,\dots,n\}$, integer $k$, distances $A_{i,j}$, rewards $\{r_i\}$
%     \STATE Find $i_{\mathrm{top}} = \arg\max_{i}\, r_i$
%     \STATE Compute $w_i = \exp(\,\overline{r} - r_i)$ and $d_{i,j}=A_{i,j}$
%     \STATE Initialize a random subset $S \subseteq I\setminus\{i_{\mathrm{top}}\}$ of size $k$
%     \WHILE{improving}
%         \STATE Swap $j_{\mathrm{out}} \in S$ with $j_{\mathrm{in}} \notin S$ if it decreases $\sum_{i \in I} w_i\,\min_{j \in S} d_{i,j}$
%     \ENDWHILE
%     \RETURN $S_{\mathrm{neg}}=S$ (negatives) and $i_{\mathrm{top}}$ (positive)
% \end{algorithmic}
% \end{algorithm}

\paragraph{Intuition.}
If $y_i$ is far from all penalized points $j\in S$, then it remains relatively “safe” from suppression, which is undesirable if $r_i$ is low (i.e.\ $w_i$ large). By systematically choosing $S$ to reduce $\sum_i w_i\,\min_{j\in S}d_{i,j}$, we concentrate penalization on high-impact, low-reward regions. The local search repeatedly swaps elements until no single exchange can further reduce the cost.

\subsection{Why ``Opt-Select''? A Lipschitz Argument for Expected Reward}

We name the procedure ``Opt-Select'' because solving \eqref{eq:problem_P} (or its local search variant) directly approximates an \emph{optimal} subset for improving the policy's expected reward. Specifically, under a Lipschitz constraint with constant $L$, assigning zero probability to each chosen negative $y_j$ implies \emph{neighboring answers} $y_i$ at distance $d_{i,j}$ cannot exceed probability $L\,d_{i,j}$. Consequently, their contribution to the ``bad behavior'' portion of expected reward is bounded by
\[
   \exp\bigl(r_{\max} - r_i\bigr)\,\bigl(\,L\,d_{i,j}\bigr),
\]
where $r_{\max}$ is the rating of the best-rated response. Dividing by a normalization factor (such as $\exp(r_{\max} - \overline{r})\,L$), one arrives at a cost akin to $w_i\, d_{i,j}$ with $w_i = \exp(\overline{r}-r_i)$. 
% Hence, minimizing \eqref{eq:weighted_distance_cost} ensures that \emph{low-reward} points do not remain at large distance from any chosen negative (i.e.\ they get suppressed). 
This aligns with classical \emph{min-knapsack} of minimizing some costs subject to some constraints, and has close alignment with the \emph{weighted $k$-medoid} notions of “covering” important items at minimum cost. %By combining this with a single top-reward positive, Opt-Select systematically pushes probability mass toward high-reward modes while penalizing low-reward regions.

\section{Experiments}
\label{sec:Experiments} 

We conduct several experiments across different problem settings to assess the efficiency of our proposed method. Detailed descriptions of the experimental settings are provided in \cref{sec:apendix_experiments}.
%We conduct experiments on optimizing PINNs for convection, wave PDEs, and a reaction ODE. 
%These equations have been studied in previous works investigating difficulties in training PINNs; we use the formulations in \citet{krishnapriyan2021characterizing, wang2022when} for our experiments. 
%The coefficient settings we use for these equations are considered challenging in the literature \cite{krishnapriyan2021characterizing, wang2022when}.
%\cref{sec:problem_setup_additional} contains additional details.

%We compare the performance of Adam, \lbfgs{}, and \al{} on training PINNs for all three classes of PDEs. 
%For Adam, we tune the learning rate by a grid search on $\{10^{-5}, 10^{-4}, 10^{-3}, 10^{-2}, 10^{-1}\}$.
%For \lbfgs, we use the default learning rate $1.0$, memory size $100$, and strong Wolfe line search.
%For \al, we tune the learning rate for Adam as before, and also vary the switch from Adam to \lbfgs{} (after 1000, 11000, 31000 iterations).
%These correspond to \al{} (1k), \al{} (11k), and \al{} (31k) in our figures.
%All three methods are run for a total of 41000 iterations.

%We use multilayer perceptrons (MLPs) with tanh activations and three hidden layers. These MLPs have widths 50, 100, 200, or 400.
%We initialize these networks with the Xavier normal initialization \cite{glorot2010understanding} and all biases equal to zero.
%Each combination of PDE, optimizer, and MLP architecture is run with 5 random seeds.

%We use 10000 residual points randomly sampled from a $255 \times 100$ grid on the interior of the problem domain. 
%We use 257 equally spaced points for the initial conditions and 101 equally spaced points for each boundary condition.

%We assess the discrepancy between the PINN solution and the ground truth using $\ell_2$ relative error (L2RE), a standard metric in the PINN literature. Let $y = (y_i)_{i = 1}^n$ be the PINN prediction and $y' = (y'_i)_{i = 1}^n$ the ground truth. Define
%\begin{align*}
%    \mathrm{L2RE} = \sqrt{\frac{\sum_{i = 1}^n (y_i - y'_i)^2}{\sum_{i = 1}^n y'^2_i}} = \sqrt{\frac{\|y - y'\|_2^2}{\|y'\|_2^2}}.
%\end{align*}
%We compute the L2RE using all points in the $255 \times 100$ grid on the interior of the problem domain, along with the 257 and 101 points used for the initial and boundary conditions.

%We develop our experiments in PyTorch 2.0.0 \cite{paszke2019pytorch} with Python 3.10.12.
%Each experiment is run on a single NVIDIA Titan V GPU using CUDA 11.8.
%The code for our experiments is available at \href{https://github.com/pratikrathore8/opt_for_pinns}{https://github.com/pratikrathore8/opt\_for\_pinns}.


\subsection{2D Allen Cahn Equation}
\begin{figure*}[t]
    \centering
    \includegraphics[scale=0.38]{figs/Burgers_operator.pdf}
    \caption{1D Burgers' Equation (Operator Learning): Steady-state solutions for different initializations $u_0$ under varying viscosity $\varepsilon$: (a) $\varepsilon = 0.5$, (b) $\varepsilon = 0.1$, (c) $\varepsilon = 0.05$. The results demonstrate that all final test solutions converge to the correct steady-state solution. (d) Illustration of the evolution of a test initialization $u_0$ following homotopy dynamics. The number of residual points is $\nres = 128$.}
    \label{fig:Burgers_result}
\end{figure*}
First, we consider the following time-dependent problem:
\begin{align}
& u_t = \varepsilon^2 \Delta u - u(u^2 - 1), \quad (x, y) \in [-1, 1] \times [-1, 1] \nonumber \\
& u(x, y, 0) = - \sin(\pi x) \sin(\pi y) \label{eq.hom_2D_AC}\\
& u(-1, y, t) = u(1, y, t) = u(x, -1, t) = u(x, 1, t) = 0. \nonumber
\end{align}
We aim to find the steady-state solution for this equation with $\varepsilon = 0.05$ and define the homotopy as:
\begin{equation}
    H(u, s, \varepsilon) = (1 - s)\left(\varepsilon(s)^2 \Delta u - u(u^2 - 1)\right) + s(u - u_0),\nonumber
\end{equation}
where $s \in [0, 1]$. Specifically, when $s = 1$, the initial condition $u_0$ is automatically satisfied, and when $s = 0$, it recovers the steady-state problem. The function $\varepsilon(s)$ is given by
\begin{equation}
\varepsilon(s) = 
\left\{\begin{array}{l}
s, \quad s \in [0.05, 1], \\
0.05, \quad s \in [0, 0.05].
\end{array}\right.\label{eq:epsilon_t}
\end{equation}

Here, $\varepsilon(s)$ varies with $s$ during the first half of the evolution. Once $\varepsilon(s)$ reaches $0.05$, it remains fixed, and only $s$ continues to evolve toward $0$. As shown in \cref{fig:2D_Allen_Cahn_Equation}, the relative $L_2$ error by homotopy dynamics is $8.78 \times 10^{-3}$, compared with the result obtained by PINN, which has a $L_2$ error of $9.56 \times 10^{-1}$. This clearly demonstrates that the homotopy dynamics-based approach significantly improves accuracy.

\subsection{High Frequency Function Approximation }
We aim to approximate the following function:
$u=    \sin(50\pi x), \quad x \in [0,1].$
The homotopy is defined as $H(u,\varepsilon) = u - \sin(\frac{1}{\varepsilon}\pi x), $
where $\varepsilon \in [\frac{1}{50},\frac{1}{15}]$.

\begin{table}[htbp!]
    \caption{Comparison of the lowest loss achieved by the classical training and homotopy dynamics for different values of $\varepsilon$ in approximating $\sin\left(\frac{1}{\varepsilon} \pi x\right)$
    }
    \vskip 0.15in
    \centering
    \tiny
    \begin{tabular}{|c|c|c|c|c|} 
    \hline 
    $ $ & $\varepsilon = 1/15$ & $\varepsilon = 1/35$ & $\varepsilon = 1/50$ \\ \hline 
    Classical Loss                & 4.91e-6     & 7.21e-2     & 3.29e-1       \\ \hline 
    Homotopy Loss $L_H$                      & 1.73e-6     & 1.91e-6     & \textbf{2.82e-5}       \\ \hline
    \end{tabular}
    % On convection, \al{} provides 14.2$\times$ and 1.97$\times$ improvement over Adam or \lbfgs{} on L2RE. 
    % On reaction, \al{} provides 1.10$\times$ and 1.99$\times$ improvement over Adam or \lbfgs{} on L2RE.
    % On wave, \al{} provides 6.32$\times$ and 6.07$\times$ improvement over Adam or \lbfgs{} on L2RE.}
    \label{tab:loss_approximate}
\end{table}

As shown in \cref{fig:high_frequency_result}, due to the F-principle \cite{xu2024overview}, training is particularly challenging when approximating high-frequency functions like $\sin(50\pi x)$. The loss decreases slowly, resulting in poor approximation performance. However, training based on homotopy dynamics significantly reduces the loss, leading to a better approximation of high-frequency functions. This demonstrates that homotopy dynamics-based training can effectively facilitate convergence when approximating high-frequency data. Additionally, we compare the loss for approximating functions with different frequencies $1/\varepsilon$ using both methods. The results, presented in \cref{tab:loss_approximate}, show that the homotopy dynamics training method consistently performs well for high-frequency functions.





\subsection{Burgers Equation}
In this example, we adopt the operator learning framework to solve for the steady-state solution of the Burgers equation, given by:
\begin{align}
& u_t+\left(\frac{u^2}{2}\right)_x - \varepsilon u_{xx}=\pi \sin (\pi x) \cos (\pi x), \quad x \in[0, 1]\nonumber\\
& u(x, 0)=u_0(x),\label{eq:1D_Burgers} \\
& u(0, t)=u(1, t)=0, \nonumber 
\end{align}
with Dirichlet boundary conditions, where $u_0 \in L_{0}^2((0, 1); \mathbb{R})$ is the initial condition and $\varepsilon \in \mathbb{R}$ is the viscosity coefficient. We aim to learn the operator mapping the initial condition to the steady-state solution, $G^{\dagger}: L_{0}^2((0, 1); \mathbb{R}) \rightarrow H_{0}^r((0, 1); \mathbb{R})$, defined by $u_0 \mapsto u_{\infty}$ for any $r > 0$. As shown in Theorem 2.2 of \cite{KREISS1986161} and Theorems 2.5 and 2.7 of \cite{hao2019convergence}, for any $\varepsilon > 0$, the steady-state solution is independent of the initial condition, with a single shock occurring at $x_s = 0.5$. Here, we use DeepONet~\cite{lu2021deeponet} as the network architecture. 
The homotopy definition, similar to ~\cref{eq.hom_2D_AC}, can be found in \cref{Ap:operator}. The results can be found in \cref{fig:Burgers_result} and \cref{tab:loss_burgers}. Experimental results show that the homotopy dynamics strategy performs well in the operator learning setting as well.


\begin{table}[htbp!]
    \caption{Comparison of loss between classical training and homotopy dynamics for different values of $\varepsilon$ in the Burgers equation, along with the MSE distance to the ground truth shock location, $x_s$.}
    \vskip 0.15in
    \centering
    \tiny
    \begin{tabular}{|c|c|c|c|c|} 
    \hline  
    $ $ & $\varepsilon = 0.5$ & $\varepsilon = 0.1$ & $\varepsilon = 0.05$ \\ \hline 
    Homotopy Loss $L_H$                &  7.55e-7     & 3.40e-7     & 7.77e-7       \\ \hline 
    L2RE                      & 1.50e-3     & 7.00e-4     & 2.52e-2       \\ \hline
        MSE Distance $x_s$                      & 1.75e-8     & 9.14e-8      & 1.2e-3      \\ \hline
    \end{tabular}
    % On convection, \al{} provides 14.2$\times$ and 1.97$\times$ improvement over Adam or \lbfgs{} on L2RE. 
    % On reaction, \al{} provides 1.10$\times$ and 1.99$\times$ improvement over Adam or \lbfgs{} on L2RE.
    % On wave, \al{} provides 6.32$\times$ and 6.07$\times$ improvement over Adam or \lbfgs{} on L2RE.}
    \label{tab:loss_burgers}
\end{table}



% \begin{itemize}
%     \item Relate the curvature in the problem to the differential operator. Use this to demonstrate why the problem is ill-conditioned
%     \item Give an argument for why using Adam + L-BFGS is better than just using L-BFGS outright. The idea is that Adam lowers the errors to the point where the rest of the optimization becomes convex \ldots
%     \item Show why we need second-order methods. We would like to prove that once we are close to the optimum, second-order methods will give condition-number free linear convergence. Specialize this to the Gauss-Newton setting, with the randomized low-rank approximation.
%     % \item Show that it is not possible to get superlinear convergence under the interpolation assumption with an overparameterized neural network. This should be true b/c the Hessian at the optimum will have rank $\min(n, d)$, and when $d > n$, this guarantees that we cannot have strong convexity.
% \end{itemize}
\section{Experiments}
\label{sec:experiments}
The experiments are designed to address two key research questions.
First, \textbf{RQ1} evaluates whether the average $L_2$-norm of the counterfactual perturbation vectors ($\overline{||\perturb||}$) decreases as the model overfits the data, thereby providing further empirical validation for our hypothesis.
Second, \textbf{RQ2} evaluates the ability of the proposed counterfactual regularized loss, as defined in (\ref{eq:regularized_loss2}), to mitigate overfitting when compared to existing regularization techniques.

% The experiments are designed to address three key research questions. First, \textbf{RQ1} investigates whether the mean perturbation vector norm decreases as the model overfits the data, aiming to further validate our intuition. Second, \textbf{RQ2} explores whether the mean perturbation vector norm can be effectively leveraged as a regularization term during training, offering insights into its potential role in mitigating overfitting. Finally, \textbf{RQ3} examines whether our counterfactual regularizer enables the model to achieve superior performance compared to existing regularization methods, thus highlighting its practical advantage.

\subsection{Experimental Setup}
\textbf{\textit{Datasets, Models, and Tasks.}}
The experiments are conducted on three datasets: \textit{Water Potability}~\cite{kadiwal2020waterpotability}, \textit{Phomene}~\cite{phomene}, and \textit{CIFAR-10}~\cite{krizhevsky2009learning}. For \textit{Water Potability} and \textit{Phomene}, we randomly select $80\%$ of the samples for the training set, and the remaining $20\%$ for the test set, \textit{CIFAR-10} comes already split. Furthermore, we consider the following models: Logistic Regression, Multi-Layer Perceptron (MLP) with 100 and 30 neurons on each hidden layer, and PreactResNet-18~\cite{he2016cvecvv} as a Convolutional Neural Network (CNN) architecture.
We focus on binary classification tasks and leave the extension to multiclass scenarios for future work. However, for datasets that are inherently multiclass, we transform the problem into a binary classification task by selecting two classes, aligning with our assumption.

\smallskip
\noindent\textbf{\textit{Evaluation Measures.}} To characterize the degree of overfitting, we use the test loss, as it serves as a reliable indicator of the model's generalization capability to unseen data. Additionally, we evaluate the predictive performance of each model using the test accuracy.

\smallskip
\noindent\textbf{\textit{Baselines.}} We compare CF-Reg with the following regularization techniques: L1 (``Lasso''), L2 (``Ridge''), and Dropout.

\smallskip
\noindent\textbf{\textit{Configurations.}}
For each model, we adopt specific configurations as follows.
\begin{itemize}
\item \textit{Logistic Regression:} To induce overfitting in the model, we artificially increase the dimensionality of the data beyond the number of training samples by applying a polynomial feature expansion. This approach ensures that the model has enough capacity to overfit the training data, allowing us to analyze the impact of our counterfactual regularizer. The degree of the polynomial is chosen as the smallest degree that makes the number of features greater than the number of data.
\item \textit{Neural Networks (MLP and CNN):} To take advantage of the closed-form solution for computing the optimal perturbation vector as defined in (\ref{eq:opt-delta}), we use a local linear approximation of the neural network models. Hence, given an instance $\inst_i$, we consider the (optimal) counterfactual not with respect to $\model$ but with respect to:
\begin{equation}
\label{eq:taylor}
    \model^{lin}(\inst) = \model(\inst_i) + \nabla_{\inst}\model(\inst_i)(\inst - \inst_i),
\end{equation}
where $\model^{lin}$ represents the first-order Taylor approximation of $\model$ at $\inst_i$.
Note that this step is unnecessary for Logistic Regression, as it is inherently a linear model.
\end{itemize}

\smallskip
\noindent \textbf{\textit{Implementation Details.}} We run all experiments on a machine equipped with an AMD Ryzen 9 7900 12-Core Processor and an NVIDIA GeForce RTX 4090 GPU. Our implementation is based on the PyTorch Lightning framework. We use stochastic gradient descent as the optimizer with a learning rate of $\eta = 0.001$ and no weight decay. We use a batch size of $128$. The training and test steps are conducted for $6000$ epochs on the \textit{Water Potability} and \textit{Phoneme} datasets, while for the \textit{CIFAR-10} dataset, they are performed for $200$ epochs.
Finally, the contribution $w_i^{\varepsilon}$ of each training point $\inst_i$ is uniformly set as $w_i^{\varepsilon} = 1~\forall i\in \{1,\ldots,m\}$.

The source code implementation for our experiments is available at the following GitHub repository: \url{https://anonymous.4open.science/r/COCE-80B4/README.md} 

\subsection{RQ1: Counterfactual Perturbation vs. Overfitting}
To address \textbf{RQ1}, we analyze the relationship between the test loss and the average $L_2$-norm of the counterfactual perturbation vectors ($\overline{||\perturb||}$) over training epochs.

In particular, Figure~\ref{fig:delta_loss_epochs} depicts the evolution of $\overline{||\perturb||}$ alongside the test loss for an MLP trained \textit{without} regularization on the \textit{Water Potability} dataset. 
\begin{figure}[ht]
    \centering
    \includegraphics[width=0.85\linewidth]{img/delta_loss_epochs.png}
    \caption{The average counterfactual perturbation vector $\overline{||\perturb||}$ (left $y$-axis) and the cross-entropy test loss (right $y$-axis) over training epochs ($x$-axis) for an MLP trained on the \textit{Water Potability} dataset \textit{without} regularization.}
    \label{fig:delta_loss_epochs}
\end{figure}

The plot shows a clear trend as the model starts to overfit the data (evidenced by an increase in test loss). 
Notably, $\overline{||\perturb||}$ begins to decrease, which aligns with the hypothesis that the average distance to the optimal counterfactual example gets smaller as the model's decision boundary becomes increasingly adherent to the training data.

It is worth noting that this trend is heavily influenced by the choice of the counterfactual generator model. In particular, the relationship between $\overline{||\perturb||}$ and the degree of overfitting may become even more pronounced when leveraging more accurate counterfactual generators. However, these models often come at the cost of higher computational complexity, and their exploration is left to future work.

Nonetheless, we expect that $\overline{||\perturb||}$ will eventually stabilize at a plateau, as the average $L_2$-norm of the optimal counterfactual perturbations cannot vanish to zero.

% Additionally, the choice of employing the score-based counterfactual explanation framework to generate counterfactuals was driven to promote computational efficiency.

% Future enhancements to the framework may involve adopting models capable of generating more precise counterfactuals. While such approaches may yield to performance improvements, they are likely to come at the cost of increased computational complexity.


\subsection{RQ2: Counterfactual Regularization Performance}
To answer \textbf{RQ2}, we evaluate the effectiveness of the proposed counterfactual regularization (CF-Reg) by comparing its performance against existing baselines: unregularized training loss (No-Reg), L1 regularization (L1-Reg), L2 regularization (L2-Reg), and Dropout.
Specifically, for each model and dataset combination, Table~\ref{tab:regularization_comparison} presents the mean value and standard deviation of test accuracy achieved by each method across 5 random initialization. 

The table illustrates that our regularization technique consistently delivers better results than existing methods across all evaluated scenarios, except for one case -- i.e., Logistic Regression on the \textit{Phomene} dataset. 
However, this setting exhibits an unusual pattern, as the highest model accuracy is achieved without any regularization. Even in this case, CF-Reg still surpasses other regularization baselines.

From the results above, we derive the following key insights. First, CF-Reg proves to be effective across various model types, ranging from simple linear models (Logistic Regression) to deep architectures like MLPs and CNNs, and across diverse datasets, including both tabular and image data. 
Second, CF-Reg's strong performance on the \textit{Water} dataset with Logistic Regression suggests that its benefits may be more pronounced when applied to simpler models. However, the unexpected outcome on the \textit{Phoneme} dataset calls for further investigation into this phenomenon.


\begin{table*}[h!]
    \centering
    \caption{Mean value and standard deviation of test accuracy across 5 random initializations for different model, dataset, and regularization method. The best results are highlighted in \textbf{bold}.}
    \label{tab:regularization_comparison}
    \begin{tabular}{|c|c|c|c|c|c|c|}
        \hline
        \textbf{Model} & \textbf{Dataset} & \textbf{No-Reg} & \textbf{L1-Reg} & \textbf{L2-Reg} & \textbf{Dropout} & \textbf{CF-Reg (ours)} \\ \hline
        Logistic Regression   & \textit{Water}   & $0.6595 \pm 0.0038$   & $0.6729 \pm 0.0056$   & $0.6756 \pm 0.0046$  & N/A    & $\mathbf{0.6918 \pm 0.0036}$                     \\ \hline
        MLP   & \textit{Water}   & $0.6756 \pm 0.0042$   & $0.6790 \pm 0.0058$   & $0.6790 \pm 0.0023$  & $0.6750 \pm 0.0036$    & $\mathbf{0.6802 \pm 0.0046}$                    \\ \hline
%        MLP   & \textit{Adult}   & $0.8404 \pm 0.0010$   & $\mathbf{0.8495 \pm 0.0007}$   & $0.8489 \pm 0.0014$  & $\mathbf{0.8495 \pm 0.0016}$     & $0.8449 \pm 0.0019$                    \\ \hline
        Logistic Regression   & \textit{Phomene}   & $\mathbf{0.8148 \pm 0.0020}$   & $0.8041 \pm 0.0028$   & $0.7835 \pm 0.0176$  & N/A    & $0.8098 \pm 0.0055$                     \\ \hline
        MLP   & \textit{Phomene}   & $0.8677 \pm 0.0033$   & $0.8374 \pm 0.0080$   & $0.8673 \pm 0.0045$  & $0.8672 \pm 0.0042$     & $\mathbf{0.8718 \pm 0.0040}$                    \\ \hline
        CNN   & \textit{CIFAR-10} & $0.6670 \pm 0.0233$   & $0.6229 \pm 0.0850$   & $0.7348 \pm 0.0365$   & N/A    & $\mathbf{0.7427 \pm 0.0571}$                     \\ \hline
    \end{tabular}
\end{table*}

\begin{table*}[htb!]
    \centering
    \caption{Hyperparameter configurations utilized for the generation of Table \ref{tab:regularization_comparison}. For our regularization the hyperparameters are reported as $\mathbf{\alpha/\beta}$.}
    \label{tab:performance_parameters}
    \begin{tabular}{|c|c|c|c|c|c|c|}
        \hline
        \textbf{Model} & \textbf{Dataset} & \textbf{No-Reg} & \textbf{L1-Reg} & \textbf{L2-Reg} & \textbf{Dropout} & \textbf{CF-Reg (ours)} \\ \hline
        Logistic Regression   & \textit{Water}   & N/A   & $0.0093$   & $0.6927$  & N/A    & $0.3791/1.0355$                     \\ \hline
        MLP   & \textit{Water}   & N/A   & $0.0007$   & $0.0022$  & $0.0002$    & $0.2567/1.9775$                    \\ \hline
        Logistic Regression   &
        \textit{Phomene}   & N/A   & $0.0097$   & $0.7979$  & N/A    & $0.0571/1.8516$                     \\ \hline
        MLP   & \textit{Phomene}   & N/A   & $0.0007$   & $4.24\cdot10^{-5}$  & $0.0015$    & $0.0516/2.2700$                    \\ \hline
       % MLP   & \textit{Adult}   & N/A   & $0.0018$   & $0.0018$  & $0.0601$     & $0.0764/2.2068$                    \\ \hline
        CNN   & \textit{CIFAR-10} & N/A   & $0.0050$   & $0.0864$ & N/A    & $0.3018/
        2.1502$                     \\ \hline
    \end{tabular}
\end{table*}

\begin{table*}[htb!]
    \centering
    \caption{Mean value and standard deviation of training time across 5 different runs. The reported time (in seconds) corresponds to the generation of each entry in Table \ref{tab:regularization_comparison}. Times are }
    \label{tab:times}
    \begin{tabular}{|c|c|c|c|c|c|c|}
        \hline
        \textbf{Model} & \textbf{Dataset} & \textbf{No-Reg} & \textbf{L1-Reg} & \textbf{L2-Reg} & \textbf{Dropout} & \textbf{CF-Reg (ours)} \\ \hline
        Logistic Regression   & \textit{Water}   & $222.98 \pm 1.07$   & $239.94 \pm 2.59$   & $241.60 \pm 1.88$  & N/A    & $251.50 \pm 1.93$                     \\ \hline
        MLP   & \textit{Water}   & $225.71 \pm 3.85$   & $250.13 \pm 4.44$   & $255.78 \pm 2.38$  & $237.83 \pm 3.45$    & $266.48 \pm 3.46$                    \\ \hline
        Logistic Regression   & \textit{Phomene}   & $266.39 \pm 0.82$ & $367.52 \pm 6.85$   & $361.69 \pm 4.04$  & N/A   & $310.48 \pm 0.76$                    \\ \hline
        MLP   &
        \textit{Phomene} & $335.62 \pm 1.77$   & $390.86 \pm 2.11$   & $393.96 \pm 1.95$ & $363.51 \pm 5.07$    & $403.14 \pm 1.92$                     \\ \hline
       % MLP   & \textit{Adult}   & N/A   & $0.0018$   & $0.0018$  & $0.0601$     & $0.0764/2.2068$                    \\ \hline
        CNN   & \textit{CIFAR-10} & $370.09 \pm 0.18$   & $395.71 \pm 0.55$   & $401.38 \pm 0.16$ & N/A    & $1287.8 \pm 0.26$                     \\ \hline
    \end{tabular}
\end{table*}

\subsection{Feasibility of our Method}
A crucial requirement for any regularization technique is that it should impose minimal impact on the overall training process.
In this respect, CF-Reg introduces an overhead that depends on the time required to find the optimal counterfactual example for each training instance. 
As such, the more sophisticated the counterfactual generator model probed during training the higher would be the time required. However, a more advanced counterfactual generator might provide a more effective regularization. We discuss this trade-off in more details in Section~\ref{sec:discussion}.

Table~\ref{tab:times} presents the average training time ($\pm$ standard deviation) for each model and dataset combination listed in Table~\ref{tab:regularization_comparison}.
We can observe that the higher accuracy achieved by CF-Reg using the score-based counterfactual generator comes with only minimal overhead. However, when applied to deep neural networks with many hidden layers, such as \textit{PreactResNet-18}, the forward derivative computation required for the linearization of the network introduces a more noticeable computational cost, explaining the longer training times in the table.

\subsection{Hyperparameter Sensitivity Analysis}
The proposed counterfactual regularization technique relies on two key hyperparameters: $\alpha$ and $\beta$. The former is intrinsic to the loss formulation defined in (\ref{eq:cf-train}), while the latter is closely tied to the choice of the score-based counterfactual explanation method used.

Figure~\ref{fig:test_alpha_beta} illustrates how the test accuracy of an MLP trained on the \textit{Water Potability} dataset changes for different combinations of $\alpha$ and $\beta$.

\begin{figure}[ht]
    \centering
    \includegraphics[width=0.85\linewidth]{img/test_acc_alpha_beta.png}
    \caption{The test accuracy of an MLP trained on the \textit{Water Potability} dataset, evaluated while varying the weight of our counterfactual regularizer ($\alpha$) for different values of $\beta$.}
    \label{fig:test_alpha_beta}
\end{figure}

We observe that, for a fixed $\beta$, increasing the weight of our counterfactual regularizer ($\alpha$) can slightly improve test accuracy until a sudden drop is noticed for $\alpha > 0.1$.
This behavior was expected, as the impact of our penalty, like any regularization term, can be disruptive if not properly controlled.

Moreover, this finding further demonstrates that our regularization method, CF-Reg, is inherently data-driven. Therefore, it requires specific fine-tuning based on the combination of the model and dataset at hand.
% \section{Experimental Setup}
\label{sec:experimental setup}

\begin{figure*}[!thbp]
    \centering
    \includegraphics[width=1.0\textwidth]{images/tsne_plot_with_scores_35914.pdf}
    \vspace{-0.25in}
    % \includegraphics[draft]{images/Sweppo_illustration.pdf}
    \caption{t-SNE visualization of projected high-dimensional response embeddings into a 2D space, illustrating the separation of actively selected responses. (a) AMPO-BottomK (baseline). (b) AMPO-Coreset (ours). (c) Opt-Select (ours). We see that the traditional baselines select many responses close to each other, based on their rating. This provides insufficient feedback to the LLM during preference optimization. In contrast, our methods simultaneously optimize for objectives including coverage, generation probability as well as preference rating.}
\label{fig:summation_logps_analysis}
\end{figure*}




\begin{table}[t]
\centering
\resizebox{\columnwidth}{!}{
\begin{tabular}{@{}lcccc@{}}
\toprule
\multirow{2}{*}{\textbf{Method}} & \multicolumn{2}{c}{\textbf{AlpacaEval 2}} & \textbf{Arena-Hard} & \textbf{MT-Bench} \\ \cmidrule(lr){2-3} \cmidrule(lr){4-4} \cmidrule(lr){5-5}
 & \textbf{LC (\%)} & \textbf{WR (\%)} & \textbf{WR (\%)} & \textbf{GPT-4} \\ \midrule
Base & 28.4 & 28.4& 26.9 & 7.93 \\
Best-vs-worst & 47.6 & 44.7 & 34.6 & 7.51 \\
\ampo-Bottomk & 50.8 & 50.5 & \textbf{44.8} & \textbf{8.11} \\
\ampo-Coreset & \textbf{52.4} & \textbf{52.1} & \underline{39.4} & \underline{8.05} \\
\ampo-Opt-Select & \underline{51.6} & \underline{51.2} & 37.9 & 7.96 \\
\bottomrule
\end{tabular}
}
\caption{Comparison of various preference optimization baselines on AlpacaEval, Arena-Hard, and MT-Bench benchmarks for Llama-3-Instruct (8B). LC-WR represents length-controlled win rate, and WR represents raw win rate. Best results are in \textbf{bold}, second-best are \underline{underlined}. Our method ($\ampo$) achieves SOTA performance across all metrics, with different variants achieving either best or second-best results consistently.}
\label{tab:llama3-results}
\end{table}


\paragraph{Model and Training Settings:}
For our experiments, we utilized the Ultrafeedback Dataset \cite{cui2023ultrafeedback}, an instruction-following benchmark annotated by GPT-4. This dataset consists of approximately 64,000 instructions, each paired with four responses generated by different language models. GPT-4 assigned scalar rewards on a 0-to-10 scale for each response, which prior research has shown to correlate strongly with human annotations. This establishes GPT-4 ratings as a reliable and cost-efficient alternative to manual feedback.

In our broader framework, we first trained a base model (mistralai/Mistral-7B-v0.1) on the UltraChat-200k dataset to obtain an SFT model. This SFT model, trained on open-source data, provides a transparent starting point. Subsequently, we refined the model by performing preference optimization on the UltraFeedback dataset. Once fine-tuned, the model was used for alignment. This two-step process ensures the model is well-prepared for tasks.

In our experiments, we observed that tuning hyperparameters is critical for optimizing the performance of all offline preference optimization algorithms, including DPO, SimPO, \refa-InfoNCA, \refa-1-vs-all and \refa-dynamic. Carefully selecting hyperparameter values significantly impacts the effectiveness of these methods across various datasets.

For \refa-InfoNCA, we found that setting the beta parameter in the range of 2.0 to 2.5 consistently yields strong performance. Similarly, for \refa-1-vs-all and \refa-dynamic, optimal results were achieved with beta values between 2.0 and 4.0, while tuning the gamma parameter within the range of 0.7 to 1.4 further improved performance. These observations highlight the importance of systematic hyperparameter tuning to achieve reliable outcomes across diverse datasets.

\paragraph{Evaluation Benchmarks}
We evaluate our models using three widely recognized open-ended instruction-following benchmarks: MT-Bench, AlpacaEval 2, AlpacaEval and Arena-Hard v0.1. These benchmarks are commonly used in the community to assess the conversational versatility of models across a diverse range of queries.

AlpacaEval 2 comprises 805 questions sourced from five datasets, while MT-Bench spans eight categories with a total of 80 questions. The recently introduced Arena-Hard builds upon MT-Bench, featuring 500 well-defined technical problem-solving queries designed to test more advanced capabilities.

We adhere to the evaluation protocols specific to each benchmark when reporting results. For AlpacaEval 2, we provide both the raw win rate (WR) and the length-controlled win rate (LC), with the latter being designed to mitigate the influence of model verbosity. For Arena-Hard, we report the win rate (WR) against a baseline model. For MT-Bench, we present the scores as evaluated by GPT-4-Preview-1106, which serve as the judge model.


\begin{figure}
    \centering
\includegraphics[width=1.0\columnwidth]{images/active_selection_optimization_samp_temp_variation.pdf}
    \vspace{-0.25in}
    \caption{Effect of Sampling Temperature on different baselines for on the AlpacaEval 2 Benchmark: (a) Length-Controlled Win Rate (LC) and (b) Overall Win Rate (WR).}
    \label{fig:samp-temp-analysis}
\end{figure}


\paragraph{Baselines}
We compare our approach against several established offline preference optimization methods, summarized in Table . Among these are RRHF \cite{yuan2023rrhf} and SLiC-HF \cite{zhao2023slic}, which employ ranking loss techniques. RRHF uses a length-normalized log-likelihood function, akin to the reward function utilized by SimPO \cite{meng2024simpo}, whereas SLiC-HF directly incorporates log-likelihood and includes a supervised fine-tuning (SFT) objective in its training process.

IPO \cite{azar2023general} presents a theoretically grounded approach that avoids the assumption made by DPO, which treats pairwise preferences as interchangeable with pointwise rewards. CPO \cite{guo2024controllable}, on the other hand, uses sequence likelihood as a reward signal and trains jointly with an SFT objective.

ORPO \cite{hong2024orpo} introduces a reference-free odds ratio term to directly contrast winning and losing responses using the policy model, also incorporating joint training with the SFT objective. R-DPO \cite{Park2024DisentanglingLF} extends DPO by adding a regularization term that mitigates the exploitation of response length.

InfoNCA \cite{chen2024noise}, which introduces a K-category cross-entropy loss, reframes generative modeling problems as classification tasks by contrasting multiple data points. It computes soft labels using dataset rewards, applying a softmax operation to map reward values into probability distributions.

Lastly, SimPO \cite{meng2024simpo} leverages the average log probability of a sequence as an implicit reward, removing the need for a reference model. It further enhances performance by introducing a target reward margin to the Bradley-Terry objective, significantly improving the algorithm's effectiveness.



% \section{Experimental Results}
\label{experimental_results}

This section presents the evaluation of the adversarial universal stickers attack.


\subsection{Baseline Performance of Street Sign Classification}

The evaluation is performed on 3 randomly selected images: Stop, Yield, and Merge. The input images are from Street View, not from the LISA data set. Thus the training set (LISA images) is different from the testing set (Street View images). LISA-CNN performs well, with average over $80$\% confidence scores in the correctly predicted images. Specifically, Table~\ref{table_adversarial} shows the confidence scores for the street signs.

\begin{table}[t]
\centering
\caption{Baseline images and their confidence scores.}
%\begin{adjustbox}{width=0.48\textwidth}
\label{table_adversarial}
\small
\begin{tabular}{|p{1.8cm}|p{1.8cm}|p{1.8cm}|p{1.2cm}|}
\hline 
\textbf{Image} & \textbf{Predicted Label}  & \textbf{Confidence Score (\%)}
\\ \hline \hline
Stop &  Stop &  85.42 \\ \hline 
Yield  & Yield & 88.87 \\ \hline 
Merge & Merge & 76.51 \\ \hline 

\end{tabular}
%\end{adjustbox}
\end{table}




\subsection{One Sticker Attacks}

The first evaluation was done by generating black and white stickers, one at a time. Stickers of rectangular shape were generated. The width and height were selected to be in ranges from 5\% to 50\% of the street sign size. All combinations of widths and heights form 5 to 50, in step of 5, were tested. Figures~\ref{fig:one_sticker_black_Images} and~\ref{fig:one_sticker_white_Images} show the best location for the black and white stickers respectively. Table~\ref{table_adversarial_timing} shows the confidence scores, and Table~\ref{table_adversarial_labels} shows the (incorrectly classified) labels for these street signs. It can be seen that with one white or black sticker, placed at same location or any of the signs, the machine learning classification results are highly incorrect and attack can succeed.




\begin{figure*}[t]
    \begin{subfigure}[b]{0.25\textwidth}
        \centering
\includegraphics[width=2.2cm]{plots/attack_images/two_sticker/first_black_second_black/Best_stop_45_25_best_sticker_size.png}
        \caption{\small \centering Stop Black Stickers Image}
        \label{fig:Best_stop_black_black_two_sticker}
    \end{subfigure}
    \hfill
    \begin{subfigure}[b]{0.25\textwidth}
        \centering
         \includegraphics[width=2.2cm]{plots/attack_images/two_sticker/first_black_second_black/Best_yield_45_25_best_sticker_size.png}
        \caption{\small \centering Yield Black Stickers Image}
        \label{fig:Best_yield_black_black_two_sticker}
    \end{subfigure}
    \hfill
    \begin{subfigure}[b]{0.25\textwidth}
        \centering
  \includegraphics[width=2.2cm]{plots/attack_images/two_sticker/first_black_second_black/Best_merge_45_25_best_sticker_size.png}
        \caption{\small \centering Merge Black Stickers Image}
        \label{fig:Best_merge_black_black_two_sticker}
    \end{subfigure}
    \vspace{-1em}
    \caption{Two Black sticker attack images.}
    \label{fig:two_sticker_black_black_Images}
\end{figure*}



\begin{figure*}[t]
    \begin{subfigure}[b]{0.25\textwidth}
        \centering
\includegraphics[width=2.2cm]{plots/attack_images/two_sticker/first_black_second_white/Best_stop_50_10_best_sticker_size.png}
        \caption{\small \centering Stop Black and White Stickers Image}
        \label{fig:Best_stop_black_white_two_sticker}
    \end{subfigure}
    \hfill
    \begin{subfigure}[b]{0.25\textwidth}
        \centering
         \includegraphics[width=2.2cm]{plots/attack_images/two_sticker/first_black_second_white/Best_yield_50_10_best_sticker_size.png}
        \caption{\small \centering Yield Black and White Stickers Image}
        \label{fig:Best_yield_black_white_two_sticker}
    \end{subfigure}
    \hfill
    \begin{subfigure}[b]{0.25\textwidth}
        \centering
  \includegraphics[width=2.2cm]{plots/attack_images/two_sticker/first_black_second_white/Best_merge_50_10_best_sticker_size.png}
        \caption{\small \centering Merge Black and White Stickers Image}
        \label{fig:Best_merge_black_white_two_sticker}
    \end{subfigure}
    \vspace{-1em}
    \caption{Two Black and White sticker attack images.}
    \label{fig:two_sticker_black_white_Images}
\end{figure*}


\begin{figure*}[t]
    \begin{subfigure}[b]{0.25\textwidth}
        \centering
\includegraphics[width=2.2cm]{plots/attack_images/two_sticker/first_white_second_black/Best_stop_45_25_best_sticker_size.png}
        \caption{\small \centering Stop White and Black Stickers Image}
        \label{fig:Best_stop_white_black_two_sticker}
    \end{subfigure}
    \hfill
    \begin{subfigure}[b]{0.25\textwidth}
        \centering
         \includegraphics[width=2.2cm]{plots/attack_images/two_sticker/first_white_second_black/Best_yield_45_25_best_sticker_size.png}
        \caption{\small \centering Yield White and Black Stickers Image}
        \label{fig:Best_yield_white_black_two_sticker}
    \end{subfigure}
    \hfill
    \begin{subfigure}[b]{0.25\textwidth}
        \centering
  \includegraphics[width=2.2cm]{plots/attack_images/two_sticker/first_white_second_black/Best_merge_45_25_best_sticker_size.png}
        \caption{\small \centering Merge White and Black Stickers Image}
        \label{fig:Best_merge_white_black_two_sticker}
    \end{subfigure}
    \vspace{-1em}
    \caption{Two White and Black sticker attack images.}
    \label{fig:two_sticker_white_black_Images}
\end{figure*}


\begin{figure*}[h!]
    \begin{subfigure}[b]{0.25\textwidth}
        \centering
\includegraphics[width=2.2cm]{plots/attack_images/two_sticker/first_white_second_white/Best_stop_50_25_best_sticker_size.png}
        \caption{\small \centering Stop White and White Stickers Image}
        \label{fig:Best_stop_white_white_two_sticker}
    \end{subfigure}
    \hfill
    \begin{subfigure}[b]{0.25\textwidth}
        \centering
         \includegraphics[width=2.2cm]{plots/attack_images/two_sticker/first_white_second_white/Best_yield_50_25_best_sticker_size.png}
        \caption{\small \centering Yield White and White Stickers Image}
        \label{fig:Best_yield_white_white_two_sticker}
    \end{subfigure}
    \hfill
    \begin{subfigure}[b]{0.25\textwidth}
        \centering
 \includegraphics[width=2.2cm]{plots/attack_images/two_sticker/first_white_second_white/Best_merge_50_25_best_sticker_size.png}
        \caption{\small \centering Merge White and White Stickers Image}
        \label{fig:Best_merge_white_white_two_sticker}
    \end{subfigure}
    \vspace{-1em}
    \caption{Two White and White sticker attack images.}
    \label{fig:two_sticker_white_white_Images}
\end{figure*}

\subsection{Two Stickers}

The second evaluation was done by generating two black and white stickers, two at a time. Stickers of rectangular shape were again generated. The width and height were selected to be in ranges from 5\% to 50\% of the street sign size. All combinations of widths and heights form 5 to 50, in step of 5, were tested. Figures~\ref{fig:two_sticker_black_black_Images} to~\ref{fig:two_sticker_white_white_Images} show the best location for the different combinations of black and white stickers. Table~\ref{table_adversarial_timing} shows the confidence scores, and Table~\ref{table_adversarial_labels} shows the (incorrectly classified) labels for these street signs. It can be seen that with two white or black stickers in any combination, placed at same location or any of the signs, the machine learning classification results are highly incorrect and attack can succeed.

\subsection{Confidence Scores and Misclassified Labels Patterns}

From Table~\ref{table_adversarial_timing} and Table~\ref{table_adversarial_labels} we observed two patterns.
First, almost always it is possible to find a universal adversarial sticker for one or two stickers. The range of the confidence values (for the incorrectly classified signs) ranges from about 23\% to over 95\%. Only in two cases one of the signs was not misclassified.
Second, when misclassification occurs, the image is most often misclassified as the pedestrian crossing sign. As we treat LISA-CNN as a black-box, we do not have insights into why the pedestrian crossing sign shows up most often. However, this could be abused in future attacks where the attacker knows that the misclassification is likely to give a certain street sign class. E.g., instead of merging on a highway, the vehicle will (incorrectly) detect pedestrian crossing sign and stop, resulting in a crash on a highway.

\subsection{Sticker Size Evaluation}

We performed further extensive study to test what sticker sizes work best. Again, all combinations of widths and heights form 5 to 50, in step of 5, were tested. Tables~\ref{table_one_sticker_black} to~\ref{table_two_sticker_first_white_second_white} show the results.
For single black sticker, highest average confidence in correctly classified sign was over 80\% for 40x50 sticker.
For single white sticker, highest average confidence in correctly classified sign was over 42\% for 25x35 sticker.
For two stickers, black and black, highest average confidence in correctly classified sign was over 63\% for 25x45 sticker.
For two stickers, black and white, highest average confidence in correctly classified sign was over 56\% for 10x50 sticker.
For two stickers, white and black, highest average confidence in correctly classified sign was over 70\% for 25x45 sticker.
And for two stickers, white and white, highest average confidence in correctly classified sign was over 39\% for 25x50 sticker.

\subsection{Sticker Size Patterns}

We observed some expected and some unexpected pattern. For single black sticker, as the sticker size increases (form top-left to bottom-right in the table), the confidence in misclassified images increases. This follows the intuition that as the sticker gets bigger, it obscures larger portion of the street sign, making it less likely to be recognized correctly. On the other hand, for single white sticker, this pattern does not hold. Going form top-left to bottom-right in the table, as sticker dimensions increase, the confidence increases, but then half-way through the table it starts to decrease.

For two stickers, we observed a yet different, but consistent pattern. Going form top-left to bottom-right in the table, as sticker dimensions increase, the confidence in misclassified signs increases, then decreases a bit, until finally the size of the combined stickers is too large and does not fit in the mask area (X in the table entries). This pattern also makes sense. The intuition is that as stickers get bigger, they cause the signs to be incorrectly classified, similar to one black sticker. However, due to size of the stickers, no universal sticker can be found since for large sizes two stickers do not fit in the combined mask.

\begin{table*}[t]
\centering
\caption{Confidence scores of the best universal adversarial images.}
% \begin{adjustbox}{width=0.98\textwidth}
\label{table_adversarial_timing}
\small
\begin{tabular}{|p{1.5cm}|p{1.5cm}|p{1.5cm}|p{1.5cm}|p{1.5cm}|p{1.5cm}|p{1.5cm}|}
\hline 
\textbf{Adversarial Image} & \textbf{One Sticker Black}  & \textbf{One Sticker  White} & \textbf{Two Sticker Black, Black} & \textbf{Two Sticker Black, White} & \textbf{Two Sticker White, Black} & \textbf{Two Sticker White, White} 
\\ \hline \hline
Stop & 86.85 & 37.79 & 60.67 & 49.17& 67.25 & 25.98   \\ \hline 
Yield  & 65.44 & $\times$ & 55.24  & 23.70& 58.87& $\times$ \\ \hline 
Merge & 88.42 & 90.49 & 75.27 &95.41  &87.05 &91.50  \\ \hline 
\end{tabular}
% \end{adjustbox}
\end{table*}




\begin{table*}[t]
\centering
\caption{Predicted labels of the adversarial images. In all but two cases the attack worked. The labels correspond to the confidences shown in entries in Table~\ref{table_adversarial_confidence}.}
\begin{adjustbox}{width=0.98\textwidth}
\label{table_adversarial_labels}
\small
\begin{tabular}{|p{1.5cm}|p{1.5cm}|p{1.5cm}|p{1.5cm}|p{1.5cm}|p{1.5cm}|p{1.5cm}|p{1.5cm}|p{1.5cm}|p{1.5cm}|}
\hline 
\textbf{Adversarial Image} & \textbf{Snowball 1}  & \textbf{Snowball 2} & \textbf{Snowball 3} & \textbf{Snowball 4} & \textbf{Snowball 5} & \textbf{Snowball 6} & \textbf{Snowball 7} & \textbf{Snowball 8} & \textbf{Snowball 9}
\\ \hline \hline
Stop & Speed Limit 25 & Yield& Speed Limit 25& Yield& Speed Limit 45& Signal Ahead& Speed Limit 25& Speed Limit 25& Speed Limit 25\\ \hline 
Yield  & Speed Limit 35 & Speed Limit 35& Speed Limit 35& Speed Limit 35& Ped. Crossing& Ped. Crossing& -& Speed Limit 35& - \\ \hline
Ped. Crossing  & Stop Ahead & Stop Ahead& Stop Ahead& Stop Ahead& Stop Ahead& Stop Ahead& Stop Ahead& Stop Ahead& Stop Ahead \\ \hline  
Merge & Ped. Crossing & Ped. Crossing & Ped. Crossing & Ped. Crossing & Ped. Crossing & Ped. Crossing & Ped. Crossing & Ped. Crossing & Ped. Crossing   \\ \hline 
Turn Right & Stop & Stop & Added Lane & Stop & Added Lane & Ped. Crossing & Stop & Stop & Stop \\ \hline  
\end{tabular}
\end{adjustbox}
\end{table*}




\begin{table*}[t]
    \centering
    \caption{Average Confidence scores for universal adversarial images using a single black sticker.}
    \label{table_one_sticker_black}
    \begin{adjustbox}{width=\textwidth}
    \small
        \begin{tabular}{|p{1.5cm}|p{1.5cm}|p{1.5cm}|p{1.5cm}|p{1.5cm}|p{1.5cm}|p{1.5cm}|p{1.5cm}|p{1.5cm}|p{1.5cm}|p{1.5cm}|p{1.5cm}|}
            \hline
            \textbf{Height, Width} & 
            \textbf{5} &
            \textbf{10} & \textbf{15} & \textbf{20} & \textbf{25} & \textbf{30} & \textbf{35} & \textbf{40} & \textbf{45} & \textbf{50} \\ \hline
             \textbf{5} & - & - & - & - & - & - & - & - & - & -  \\ \hline
             \textbf{10} & - & - & - & - & 12.83 & 14.63 & 16.31 & 19.19 & 21.05  & 31.10 \\ \hline
             \textbf{15} & - & - & - & 15.74 & 19.39 & 22.46 & 26.04 & 36.27 & 41.86 & 47.98 \\ \hline
             \textbf{20} & - & - & 15.62 & 20.42 & 23.79 & 26.17 & 32.09 & 43.09 & 49.95 & 54.09 \\ \hline
            \textbf{25} & - & 14.79 & 19.28 & 24.75 & 26.85 & 30.43 & 42.95 & 55.94 & 62.54 & 72.31 \\ \hline
            \textbf{30} & - & 16.57 & 23.03 & 26.98 & 31.69 & 27.27 & 40.36 & 53.55 & 60.12 & 66.01 \\ \hline
           \textbf{35} & - & 18.07 & 24.91 & 28.01 & 45.13 & 40.38 & 47.83 & 59.38 & 67.27 & 73.54\\ \hline
            \textbf{40} & - & 19.63 & 26.07 & 35.19 & 53.13 & 49.37 & 56.81 & 65.90 & 73.55 & \textbf{80.24}  \\ \hline
           \textbf{45} & - & 20.69 & 33.72 & 41.32 & 55.24 & 54.27 & 61.89 & 69.23 & 70.08 & 77.52 \\ \hline
            \textbf{50} & - & 21.11 & 38.51 & 43.98 & 57.20 & 57.76 & 64.63 & 72.11 & 73.72 & 80.23 \\ \hline
        \end{tabular}
    \end{adjustbox}
\end{table*}

\begin{table*}[t]
    \centering
    \caption{Average Confidence scores for universal adversarial images using a single white sticker.}
    \label{table_one_sticker_white}
    \begin{adjustbox}{width=\textwidth}
    \small
        \begin{tabular}{|p{1.5cm}|p{1.5cm}|p{1.5cm}|p{1.5cm}|p{1.5cm}|p{1.5cm}|p{1.5cm}|p{1.5cm}|p{1.5cm}|p{1.5cm}|p{1.5cm}|p{1.5cm}|}
            \hline
            \textbf{Height, Width} & 
            \textbf{5} &
            \textbf{10} & \textbf{15} & \textbf{20} & \textbf{25} & \textbf{30} & \textbf{35} & \textbf{40} & \textbf{45} & \textbf{50} \\ \hline
             \textbf{5}  & -  & -  & 20.25  &  29.53  & 30.60  & 31.41  & 31.79  & 31.81  & 31.80  & 31.36  \\ \hline
            \textbf{10}  & -  & -  & 23.51  & 31.68  & 31.98  & 32.01  & 31.98  & 31.83  & 31.40  & 31.35  \\ \hline
            \textbf{15}  & 20.28  & 30.07  & 31.86  & 31.98  &  31.98 & 31.79  & 31.56  & 31.43  & 31.32  & 29.99  \\ \hline
            \textbf{20}  & 29.49 & 31.74  & 32.03  & 31.98  & 31.85  & 31.56  & 31.15  &  31.03 & 30.57  & 30.53  \\ \hline
            \textbf{25}  & 30.53  & 32.01  & 32.04 & 31.67  & 31.67  & 41.72  & \textbf{42.76}  & 31.52  & 31.26  & 30.47  \\ \hline
            \textbf{30}  & 31.35  & 31.84  & 31.80  & 37.95  & 40.34  & 31.41  & 31.56  & 31.52  & 31.18 & 27.62  \\ \hline
            \textbf{35}  & 31.50  & 31.78  & 31.68 &  32.13 & 32.14 & 31.65 & 31.77  & 31.31 & 28.21  & 25.42  \\ \hline
            \textbf{40}  & 31.54  & 31.70  & 31.73  & 39.55  & 40.35 & 31.83  &  30.79  & 30.03  & 24.59  & 22.88  \\ \hline
            \textbf{45}  & 31.35 & 31.77  & 31.79  &  32.41 & 32.38  & 28.75  & 25.55 & 23.86  & 21.17  & 18.85  \\ \hline
            \textbf{50}  & 31.33  & 31.91  & 31.96  & 31.73  & 27.75  & 24.33  & 23.85  & 19.96  & 19.51  & 18.59  \\ \hline
        \end{tabular}
    \end{adjustbox}
\end{table*}


\begin{table*}[t]
    \centering
    \caption{Average Confidence scores for universal adversarial images using two stickers (first black, second black).\vspace{-0.5em}}
\label{table_two_sticker_first_black_second_black}
    \begin{adjustbox}{width=\textwidth}
    \small
        \begin{tabular}{|p{1.5cm}|p{1.5cm}|p{1.5cm}|p{1.5cm}|p{1.5cm}|p{1.5cm}|p{1.5cm}|p{1.5cm}|p{1.5cm}|p{1.5cm}|p{1.5cm}|}
            \hline
            \textbf{Height, Width} & 
            \textbf{5} &
            \textbf{10} & \textbf{15} & \textbf{20} & \textbf{25} & \textbf{30} & \textbf{35} & \textbf{40} & \textbf{45} & \textbf{50} \\ \hline
             \textbf{5}  & -  & -  & -  & -  & -  & -  & -  & -  & -  & -  \\ \hline
            \textbf{10}  & -  & -  & -  & 11.40  & 14.41  & 16.22  & 18.67  & 27.16  & 31.69  & 34.01  \\ \hline
            \textbf{15}  & -  & -  & 12.74  & 17.13  & 25.00  & 31.09  & 33.73  & 36.89  & 40.49  & 43.55  \\ \hline
            \textbf{20}  & -  & 12.66  & 17.20  & 29.86  & 35.86  & 42.68  & 48.24  & 49.70  & 52.02  & 53.02  \\ \hline
            \textbf{25}  & -  & 16.06  & 27.41  & 34.65  & 42.22  & 50.39  & 53.44  & 57.99  & \textbf{63.73}  & 55.19  \\ \hline
            \textbf{30}  & -  & 17.14  & 28.85  & 40.52  & 44.21  & 34.50  & 38.14  & 42.35  & 39.08  & 20.87  \\ \hline
            \textbf{35}  & -  & 18.45  & 32.36  & 38.02  & 44.16  & 39.94  & 36.15  & 31.07  & 11.03  & $\times$  \\ \hline
            \textbf{40}  & -  & 17.11  & 41.25  & 41.75  & 39.77  & 31.83  & 29.96  & $\times$  & $\times$  & $\times$  \\ \hline
            \textbf{45}  & -  & 16.75  & 33.42  & 40.16  & 33.63  & 7.80  & $\times$  & $\times$ & $\times$  & $\times$ \\ \hline
            \textbf{50}  & -  & 17.15  & 27.93  & 38.82  & $\times$  & $\times$ & $\times$  & $\times$ & $\times$  & $\times$ \\ \hline
        \end{tabular}
    \end{adjustbox}
\end{table*}


\begin{table*}[t]
    \centering
     \caption{Average Confidence scores for universal adversarial images using two stickers (first black, second white).}
    \label{table_two_sticker_first_black_second_white}
    \begin{adjustbox}{width=\textwidth}
    \small
        \begin{tabular}{|p{1.5cm}|p{1.5cm}|p{1.5cm}|p{1.5cm}|p{1.5cm}|p{1.5cm}|p{1.5cm}|p{1.5cm}|p{1.5cm}|p{1.5cm}|p{1.5cm}|}
            \hline
            \textbf{Width, Height} & 
            \textbf{5} &
            \textbf{10} & \textbf{15} & \textbf{20} & \textbf{25} & \textbf{30} & \textbf{35} & \textbf{40} & \textbf{45} & \textbf{50} \\ \hline
             \textbf{5}  & -  & -          & 19.11  & 29.39  & 30.83  & 31.63  & 32.02  & 32.04  & 31.97  & 31.65  \\ \hline
            \textbf{10}  & -  & -  & 23.73          & 31.86  & 32.18  & 32.24  & 37.62  & 47.71  & 55.40  & \textbf{56.09}  \\ \hline
            \textbf{15}  & 16.59  & 29.42  & 31.94  & 32.23  & 43.31  & 46.49  & 44.18  & 53.84  & 48.54  & 46.16  \\ \hline
            \textbf{20}  & 29.40  & 31.73  & 32.06  & 32.15  & 44.74  & 44.30  & 43.09  & 42.43  & 43.47  & 36.32  \\ \hline
            \textbf{25}  & 31.19  & 31.88  & 31.72  & 39.44  & 41.62  & 39.19  & 40.92  & 36.12  & 31.94  & 36.66  \\ \hline
            \textbf{30}  & 31.76  & 31.97   & 31.55  & 34.09  & 31.23  & 30.22  & 30.29  & 29.22  & 28.30  & 32.04  \\ \hline
            \textbf{35}  & 31.65  & 31.55 & 29.34  & 28.05  & 32.18  & 28.12  & 25.71  & 24.54  & 17.05  & $\times$  \\ \hline
            \textbf{40}  & 24.96  & 25.32  & 20.39  & 26.00  & 28.17  & 22.81  & 20.06  & $\times$  & $\times$  & $\times$  \\ \hline
            \textbf{45}  & 20.61  & 20.85  & 20.25  & 22.68  & 22.46  & 15.51  & $\times$  & $\times$ & $\times$  & $\times$  \\ \hline
            \textbf{50}  & 19.09  & 19.43  & 16.56  & 18.23  & $\times$  & $\times$  & $\times$ & $\times$  & $\times$  & $\times$  \\ \hline
        \end{tabular}
    \end{adjustbox}
\end{table*}


\begin{table*}[t]
    \centering
    \caption{Average Confidence scores for universal adversarial images using two stickers (first white, second black).}
    \label{table_two_sticker_first_white_second_black}
    \begin{adjustbox}{width=\textwidth}
    \small
        \begin{tabular}{|p{1.5cm}|p{1.5cm}|p{1.5cm}|p{1.5cm}|p{1.5cm}|p{1.5cm}|p{1.5cm}|p{1.5cm}|p{1.5cm}|p{1.5cm}|p{1.5cm}|p{1.5cm}|}
            \hline
            \textbf{Height, Width} & 
            \textbf{5} &
            \textbf{10} & \textbf{15} & \textbf{20} & \textbf{25} & \textbf{30} & \textbf{35} & \textbf{40} & \textbf{45} & \textbf{50} \\ \hline
             \textbf{5}     &      - &      - &  20.25 &  27.59 &  27.79 &  24.40 &  24.83 &  25.08 &  25.90 &  24.97 \\ \hline
\textbf{10}    &      - &      - &  19.35 &  28.09 &  23.58 &  23.84 &  25.01 &  24.88 &  22.58 &  32.95 \\ \hline
\textbf{15}    &  18.65 &  19.23 &  22.63 &  23.44 &  23.00 &  22.35 &  25.45 &  35.11 &  45.80 &  52.51 \\ \hline
\textbf{20}    &  14.99 &  21.98 &  22.46 &  21.66 &  34.01 &  47.95 &  58.98 &  64.28 &  65.50 &  69.16 \\ \hline
\textbf{25}    &  17.56 &  22.78 &  20.23 &  34.17 &  49.53 &  61.45 &  66.84 &  70.02 &  \textbf{70.39} &  67.56 \\ \hline
\textbf{30}    &  18.89 &  19.09 &  28.26 &  47.14 &  60.37 &  54.23 &  56.72 &  60.39 &  59.07 &  41.45 \\ \hline
\textbf{35}    &  13.76 &  22.06 &  36.16 &  56.99 &  65.09 &  57.93 &  57.57 &  62.80 &  50.05 &      $\times$ \\ \hline
\textbf{40}    &  14.07 &  24.37 &  39.47 &  59.52 &  64.66 &  55.28 &  60.09 &      $\times$ &      $\times$ &      $\times$ \\ \hline
\textbf{45}    &  15.85 &  27.32 &  42.01 &  57.20 &  58.19 &  55.44 &     $\times$ &      $\times$ &     $\times$ &      $\times$ \\ \hline
\textbf{50}    &  18.06 &  27.92 &  41.33 &  41.64 &      $\times$ &      $\times$ &      $\times$ &      $\times$ &    $\times$  & $\times$ \\ \hline
        \end{tabular}
    \end{adjustbox}
\end{table*}




\begin{table*}[t]
    \centering
    \caption{Average Confidence scores for universal adversarial images using two stickers (first white, second white).}
    \label{table_two_sticker_first_white_second_white}
    \begin{adjustbox}{width=\textwidth}
    \small
        \begin{tabular}{|p{1.5cm}|p{1.5cm}|p{1.5cm}|p{1.5cm}|p{1.5cm}|p{1.5cm}|p{1.5cm}|p{1.5cm}|p{1.5cm}|p{1.5cm}|p{1.5cm}|p{1.5cm}|}
            \hline
            \textbf{Height, Width} & 
            \textbf{5} &
            \textbf{10} & \textbf{15} & \textbf{20} & \textbf{25} & \textbf{30} & \textbf{35} & \textbf{40} & \textbf{45} & \textbf{50} \\ \hline
            \textbf{5}  &      - &      - &  20.25 &  29.68 &  29.86 &  31.11 &  31.63 &  31.71 &  31.39 &  31.36 \\ \hline
\textbf{10} &      - &  21.04 &  31.39 &  30.85 &  29.81 &  28.94 &  28.87 &  28.94 &  28.50 &  27.01 \\ \hline
\textbf{15} &  24.50 &  31.33 &  31.27 &  31.32 &  30.50 &  28.84 &  28.30 &  27.67 &  28.01 &  29.08 \\ \hline
\textbf{20} &  29.39 &  31.31 &  31.21 &  30.29 &  28.29 &  26.81 &  26.21 &  27.30 &  28.70 &  29.88 \\ \hline
\textbf{25} &  29.81 &  31.24 &  31.37 &  30.77 &  30.04 &  28.93 &  29.07 &  29.51 &  30.19 &  \textbf{39.16} \\ \hline
\textbf{30} &  31.04 &  31.43 &  30.92 &  28.73 &  27.31 &  29.05 &  29.23 &  28.88 &  29.73 &  30.94 \\ \hline
\textbf{35} &  30.51 &  25.26 &  26.34 &  21.57 &  22.40 &  23.39 &  24.67 &  21.71 &  22.82 &      $\times$ \\ \hline
\textbf{40} &  21.59 &  21.11 &  23.39 &  22.56 &  23.91 &  24.00 &  23.81 &      $\times$ &      $\times$ &      $\times$ \\ \hline
\textbf{45} &  20.00 &  20.23 &  23.39 &  21.52 &  22.37 &  22.53 &      $\times$&      $\times$ &      $\times$ &      $\times$ \\ \hline
\textbf{50} &  17.90 &  20.94 &  21.81 &  18.03 &      $\times$ &      $\times$ &      $\times$ &      $\times$ &      $\times$&      $\times$ \\ \hline
            
        \end{tabular}
    \end{adjustbox}
\end{table*}





\section{Discussion of Assumptions}\label{sec:discussion}
In this paper, we have made several assumptions for the sake of clarity and simplicity. In this section, we discuss the rationale behind these assumptions, the extent to which these assumptions hold in practice, and the consequences for our protocol when these assumptions hold.

\subsection{Assumptions on the Demand}

There are two simplifying assumptions we make about the demand. First, we assume the demand at any time is relatively small compared to the channel capacities. Second, we take the demand to be constant over time. We elaborate upon both these points below.

\paragraph{Small demands} The assumption that demands are small relative to channel capacities is made precise in \eqref{eq:large_capacity_assumption}. This assumption simplifies two major aspects of our protocol. First, it largely removes congestion from consideration. In \eqref{eq:primal_problem}, there is no constraint ensuring that total flow in both directions stays below capacity--this is always met. Consequently, there is no Lagrange multiplier for congestion and no congestion pricing; only imbalance penalties apply. In contrast, protocols in \cite{sivaraman2020high, varma2021throughput, wang2024fence} include congestion fees due to explicit congestion constraints. Second, the bound \eqref{eq:large_capacity_assumption} ensures that as long as channels remain balanced, the network can always meet demand, no matter how the demand is routed. Since channels can rebalance when necessary, they never drop transactions. This allows prices and flows to adjust as per the equations in \eqref{eq:algorithm}, which makes it easier to prove the protocol's convergence guarantees. This also preserves the key property that a channel's price remains proportional to net money flow through it.

In practice, payment channel networks are used most often for micro-payments, for which on-chain transactions are prohibitively expensive; large transactions typically take place directly on the blockchain. For example, according to \cite{river2023lightning}, the average channel capacity is roughly $0.1$ BTC ($5,000$ BTC distributed over $50,000$ channels), while the average transaction amount is less than $0.0004$ BTC ($44.7k$ satoshis). Thus, the small demand assumption is not too unrealistic. Additionally, the occasional large transaction can be treated as a sequence of smaller transactions by breaking it into packets and executing each packet serially (as done by \cite{sivaraman2020high}).
Lastly, a good path discovery process that favors large capacity channels over small capacity ones can help ensure that the bound in \eqref{eq:large_capacity_assumption} holds.

\paragraph{Constant demands} 
In this work, we assume that any transacting pair of nodes have a steady transaction demand between them (see Section \ref{sec:transaction_requests}). Making this assumption is necessary to obtain the kind of guarantees that we have presented in this paper. Unless the demand is steady, it is unreasonable to expect that the flows converge to a steady value. Weaker assumptions on the demand lead to weaker guarantees. For example, with the more general setting of stochastic, but i.i.d. demand between any two nodes, \cite{varma2021throughput} shows that the channel queue lengths are bounded in expectation. If the demand can be arbitrary, then it is very hard to get any meaningful performance guarantees; \cite{wang2024fence} shows that even for a single bidirectional channel, the competitive ratio is infinite. Indeed, because a PCN is a decentralized system and decisions must be made based on local information alone, it is difficult for the network to find the optimal detailed balance flow at every time step with a time-varying demand.  With a steady demand, the network can discover the optimal flows in a reasonably short time, as our work shows.

We view the constant demand assumption as an approximation for a more general demand process that could be piece-wise constant, stochastic, or both (see simulations in Figure \ref{fig:five_nodes_variable_demand}).
We believe it should be possible to merge ideas from our work and \cite{varma2021throughput} to provide guarantees in a setting with random demands with arbitrary means. We leave this for future work. In addition, our work suggests that a reasonable method of handling stochastic demands is to queue the transaction requests \textit{at the source node} itself. This queuing action should be viewed in conjunction with flow-control. Indeed, a temporarily high unidirectional demand would raise prices for the sender, incentivizing the sender to stop sending the transactions. If the sender queues the transactions, they can send them later when prices drop. This form of queuing does not require any overhaul of the basic PCN infrastructure and is therefore simpler to implement than per-channel queues as suggested by \cite{sivaraman2020high} and \cite{varma2021throughput}.

\subsection{The Incentive of Channels}
The actions of the channels as prescribed by the DEBT control protocol can be summarized as follows. Channels adjust their prices in proportion to the net flow through them. They rebalance themselves whenever necessary and execute any transaction request that has been made of them. We discuss both these aspects below.

\paragraph{On Prices}
In this work, the exclusive role of channel prices is to ensure that the flows through each channel remains balanced. In practice, it would be important to include other components in a channel's price/fee as well: a congestion price  and an incentive price. The congestion price, as suggested by \cite{varma2021throughput}, would depend on the total flow of transactions through the channel, and would incentivize nodes to balance the load over different paths. The incentive price, which is commonly used in practice \cite{river2023lightning}, is necessary to provide channels with an incentive to serve as an intermediary for different channels. In practice, we expect both these components to be smaller than the imbalance price. Consequently, we expect the behavior of our protocol to be similar to our theoretical results even with these additional prices.

A key aspect of our protocol is that channel fees are allowed to be negative. Although the original Lightning network whitepaper \cite{poon2016bitcoin} suggests that negative channel prices may be a good solution to promote rebalancing, the idea of negative prices in not very popular in the literature. To our knowledge, the only prior work with this feature is \cite{varma2021throughput}. Indeed, in papers such as \cite{van2021merchant} and \cite{wang2024fence}, the price function is explicitly modified such that the channel price is never negative. The results of our paper show the benefits of negative prices. For one, in steady state, equal flows in both directions ensure that a channel doesn't loose any money (the other price components mentioned above ensure that the channel will only gain money). More importantly, negative prices are important to ensure that the protocol selectively stifles acyclic flows while allowing circulations to flow. Indeed, in the example of Section \ref{sec:flow_control_example}, the flows between nodes $A$ and $C$ are left on only because the large positive price over one channel is canceled by the corresponding negative price over the other channel, leading to a net zero price.

Lastly, observe that in the DEBT control protocol, the price charged by a channel does not depend on its capacity. This is a natural consequence of the price being the Lagrange multiplier for the net-zero flow constraint, which also does not depend on the channel capacity. In contrast, in many other works, the imbalance price is normalized by the channel capacity \cite{ren2018optimal, lin2020funds, wang2024fence}; this is shown to work well in practice. The rationale for such a price structure is explained well in \cite{wang2024fence}, where this fee is derived with the aim of always maintaining some balance (liquidity) at each end of every channel. This is a reasonable aim if a channel is to never rebalance itself; the experiments of the aforementioned papers are conducted in such a regime. In this work, however, we allow the channels to rebalance themselves a few times in order to settle on a detailed balance flow. This is because our focus is on the long-term steady state performance of the protocol. This difference in perspective also shows up in how the price depends on the channel imbalance. \cite{lin2020funds} and \cite{wang2024fence} advocate for strictly convex prices whereas this work and \cite{varma2021throughput} propose linear prices.

\paragraph{On Rebalancing} 
Recall that the DEBT control protocol ensures that the flows in the network converge to a detailed balance flow, which can be sustained perpetually without any rebalancing. However, during the transient phase (before convergence), channels may have to perform on-chain rebalancing a few times. Since rebalancing is an expensive operation, it is worthwhile discussing methods by which channels can reduce the extent of rebalancing. One option for the channels to reduce the extent of rebalancing is to increase their capacity; however, this comes at the cost of locking in more capital. Each channel can decide for itself the optimum amount of capital to lock in. Another option, which we discuss in Section \ref{sec:five_node}, is for channels to increase the rate $\gamma$ at which they adjust prices. 

Ultimately, whether or not it is beneficial for a channel to rebalance depends on the time-horizon under consideration. Our protocol is based on the assumption that the demand remains steady for a long period of time. If this is indeed the case, it would be worthwhile for a channel to rebalance itself as it can make up this cost through the incentive fees gained from the flow of transactions through it in steady state. If a channel chooses not to rebalance itself, however, there is a risk of being trapped in a deadlock, which is suboptimal for not only the nodes but also the channel.

\section{Conclusion}
This work presents DEBT control: a protocol for payment channel networks that uses source routing and flow control based on channel prices. The protocol is derived by posing a network utility maximization problem and analyzing its dual minimization. It is shown that under steady demands, the protocol guides the network to an optimal, sustainable point. Simulations show its robustness to demand variations. The work demonstrates that simple protocols with strong theoretical guarantees are possible for PCNs and we hope it inspires further theoretical research in this direction.

% \clearpage



\section*{Impact Statement}

This paper presents work whose goal is to advance the field of 
Machine Learning. There are many potential societal consequences 
of our work, none which we feel must be specifically highlighted here.


\bibliography{References}
\bibliographystyle{icml2025}




%%%%%%%%%%%%%%%%%%%%%%%%%%%%%%%%%%%%%%%%%%%%%%%%%%%%%%%%%%%%%%%%%%%%%%%%%%%%%%%
%%%%%%%%%%%%%%%%%%%%%%%%%%%%%%%%%%%%%%%%%%%%%%%%%%%%%%%%%%%%%%%%%%%%%%%%%%%%%%%
% APPENDIX
%%%%%%%%%%%%%%%%%%%%%%%%%%%%%%%%%%%%%%%%%%%%%%%%%%%%%%%%%%%%%%%%%%%%%%%%%%%%%%%
%%%%%%%%%%%%%%%%%%%%%%%%%%%%%%%%%%%%%%%%%%%%%%%%%%%%%%%%%%%%%%%%%%%%%%%%%%%%%%%
\newpage
\appendix
\onecolumn

%%%%%%%%%%%%%%%%%%%%%%%%%%%%%%%%%%%%%%%%%%%%%%%%%%%%%%%%%%%%%%%%%%%%%%%%%%%%%%%
%%%%%%%%%%%%%%%%%%%%%%%%%%%%%%%%%%%%%%%%%%%%%%%%%%%%%%%%%%%%%%%%%%%%%%%%%%%%%%%
\vspace{1cm}
\hrule
\par\vspace{0.5cm}
{\Large\bfseries\centering \textsc 
{Supplementary Materials}
\par\vspace{0.5cm}}
\hrule
\vspace{0.5cm}
\noindent These supplementary materials provide additional details, derivations, and experimental results for our paper. The appendix is organized as follows:
\begin{itemize}[leftmargin=1em]
    \item Section \ref{sec:related_work_extended} provides a more comprehensive overview of the related literature.
    
    \item Section \ref{sec:theory_opt_select_extended} provides theoretical analysis of the equivalence of the optimal selection integer program and the reward maximization objective.

    \item Section \ref{sec:local_search_kmedoids} shows a constant factor approximation for the coordinate descent algorithm in polynomial time.
    
    \item Section \ref{sec:constant_factor_subset_selection} provides theoretical guarantees for our k-means style coreset selection algorithm.
    
    \item Section \ref{sec:optimal_selection_computation} provides the code for computation of the optimal selection algorithm.
    
    \item Section \ref{sec:tsne_visualization} provides t-sne plots for the various queries highlighting the performance of our algorithms.
    
    % \item Section \ref{sec:reward_loss_computation} provides the implementation details of the reward loss computation, including the actual code used in our experiments.    
\end{itemize}
\vspace{0.5cm}

\section{Related Work}
\label{sec:related_work_extended}

\paragraph{Preference Optimization in RLHF.}
Direct Preference Optimization (DPO) is a collection of techniques for fine-tuning language models based on human preferences \cite{rafailov2024direct}. Several variants of DPO have been developed to address specific challenges and improve its effectiveness \cite{ethayarajh2024kto,zeng2024token,dong2023raft,yuan2023rrhf}. For example, KTO and TDPO focus on different aspects of preference optimization, while RAFT and RRHF utilize alternative forms of feedback. Other variants, such as SPIN, CPO, ORPO, and SimPO, introduce additional objectives or regularizations to enhance the optimization process \cite{chen2024self,xu2024contrastive,hong2024orpo,meng2024simpo}.

Further variants, including R-DPO, LD-DPO, sDPO, IRPO, OFS-DPO, and LIFT-DPO, address issues like length bias, training strategies, and specific reasoning tasks. These diverse approaches demonstrate the ongoing efforts to refine and enhance DPO, addressing its limitations and expanding its applicability to various tasks and domains \cite{park2024disentangling,liu2024iterative,pang2024iterative,qi2024online, yuan2024following}.
% Reinforcement Learning from Human Feedback (RLHF) \citep{christiano2017deep, ziegler2019fine, ouyang2022training} typically focuses on pairwise preference comparisons between model outputs, training a reward model to distinguish “better” vs.\ “worse” responses and then fine-tuning a policy to maximize that reward. This preference-based approach has been widely adopted in language model alignment \citep{bai2022training, stiennon2020learning}, but it can oversimplify the range of human preferences by reducing them to binary comparisons.

\paragraph{Multi-Preference Approaches.}
Recent work extends standard RLHF to consider entire \emph{sets} of responses at once, enabling more nuanced feedback signals \citep{rafailov2024direct, cui2023ultrafeedback, chen2024noise}. Group-based objectives capture multiple acceptable (and multiple undesirable) answers for each query, rather than only a single “better vs.\ worse” pair. \citet{gupta2024swepo} propose a contrastive formulation, \swepo, that jointly uses multiple “positives” and “negatives.” Such multi-preference methods can reduce label noise and better reflect the complexity of real-world tasks, but their computational cost grows if one attempts to incorporate all generated outputs \citep{cui2023ultrafeedback, chen2024noise}.

\paragraph{On-Policy Self-Play.}
A key advancement in reinforcement learning has been \emph{self-play} or on-policy generation, where the model continuously updates and re-generates data from its own evolving policy \citep{silver2016mastering, silver2017mastering}. In the context of LLM alignment, on-policy sampling can keep the training set aligned with the model’s current distribution of outputs \citep{christiano2017deep, wu2023fine}. However, this approach can significantly inflate the number of candidate responses, motivating the need for selective down-sampling of training examples.

\paragraph{Active Learning for Policy Optimization.}
The notion of selectively querying the most informative examples is central to \emph{active learning} \citep{cohn1996active, settles2009active}, which aims to reduce labeling effort by focusing on high-utility samples. Several works incorporate active learning ideas into reinforcement learning, e.g., uncertainty sampling or diversity-based selection \citep{sener2017active, zhang2022active}. In the RLHF setting, \citet{christiano2017deep} highlight how strategic feedback can accelerate policy improvements, while others apply active subroutines to refine reward models \citep{wu2023fine}. By picking a small yet diverse set of responses, we avoid both computational blow-ups and redundant training signals.

\paragraph{Clustering and Coverage-Based Selection.}
Selecting representative subsets from a large dataset is a classic problem in machine learning and combinatorial optimization. \emph{Clustering} techniques such as $k$-means and $k$-medoids \citep{hartigan1979algorithm} aim to group points so that distances within each cluster are small. In the RLHF context, embedding model outputs and clustering them can ensure \emph{coverage} over semantically distinct modes \citep{har2004coresets, cohen2022improved}. These methods connect to the \emph{facility location} problem \citep{oh2017deep}—minimizing the cost of “covering” all points with a fixed number of centers—and can be addressed via coreset construction \citep{feldman2020core}. 

\paragraph{Min-Knapsack and Integer Programming.}
When picking a subset of size $k$ to cover or suppress “bad” outputs, one may cast the objective in a \emph{min-knapsack} or combinatorial optimization framework \citep{kellerer2004introduction}. For instance, forcing certain outputs to zero probability can impose constraints that ripple to nearby points in embedding space, linking coverage-based strategies to integer programs \citep{chen2020big}. \citet{cohen2022improved} and \citet{har2004coresets} demonstrate how approximate solutions to such subset selection problems can achieve strong empirical results in high-dimensional scenarios. By drawing from these established concepts, our method frames the selection of negative samples in a Lipschitz coverage sense, thereby enabling both theoretical guarantees and practical efficiency in multi-preference alignment.


Collectively, our work stands at the intersection of \emph{multi-preference alignment} \citep{gupta2024swepo, cui2023ultrafeedback}, \emph{on-policy data generation} \citep{silver2017mastering, ouyang2022training}, and \emph{active learning} \citep{cohn1996active, settles2009active}. We leverage ideas from \emph{clustering} (k-means, k-medoids) and \emph{combinatorial optimization} (facility location, min-knapsack) \citep{kellerer2004multidimensional, cacchiani2022knapsack} to construct small yet powerful training subsets that capture both reward extremes and semantic diversity. The result is an efficient pipeline for aligning LLMs via multi-preference signals without exhaustively processing all generated responses.

\section{Extended Theoretical Analysis of \textsc{Opt-Select}}
\label{sec:theory_opt_select_extended}

In this appendix, we present a more detailed theoretical treatment of $\ampoos$. We restate the core problem setup and assumptions, then provide rigorous proofs of our main results. Our exposition here augments the concise version from the main text.

\subsection{Problem Setup}

Consider a single prompt (query) \(x\) for which we have sampled \(n\) candidate responses \(\{\,y_1,\,y_2,\,\ldots,\,y_n\}\). Each response \(y_i\) has:
\begin{itemize}[itemsep=0.5em, leftmargin=1em]
    \item A scalar reward \(r_i \in [0,1]\).
    \item An embedding \(\mathbf{e}_i \in \mathbb{R}^d.\)
\end{itemize}
We define the distance between two responses \(y_i\) and \(y_j\) by
\begin{equation}
\label{eq:appdistdef}
A_{i,j} \;=\; \|\mathbf{e}_i \,-\, \mathbf{e}_j\|.
\end{equation}
We wish to learn a \emph{policy} \(\{p_i\}\), where \(p_i \ge 0\) and \(\sum_{i=1}^n p_i = 1\). The policy's \emph{expected reward} is
\begin{equation}
\label{eq:appexprew}
\mathrm{ER}(p) 
\;=\; 
\sum_{i=1}^n r_i \,p_i.
\end{equation}

\paragraph{Positive and Negative Responses.}
We designate exactly one response, denoted \(y_{i_{\mathrm{top}}}\), as a \emph{positive} (the highest-reward candidate). All other responses are potential ``negatives.'' Concretely:
\begin{itemize}[itemsep=0.5em, leftmargin=1em]
    \item We fix one index \(i_{\mathrm{top}}\) with \(\displaystyle i_{\mathrm{top}} \;=\; \arg \max_{i\in\{1,\dots,n\}}\,r_i.\)
    \item We choose a subset \(\mathcal{S}\subseteq \{1,\dots,n\}\setminus\{i_{\mathrm{top}}\}\) of size \(k\), whose elements are forced to have \(p_j=0\). (These are the ``negatives.'')
\end{itemize}

\subsubsection{Lipschitz Suppression Constraint}
\label{subsec:LipschitzConstraintApp}

We assume a mild Lipschitz-like rule:
\begin{enumerate}[label=(A\arabic*), itemsep=0.5em]
    \item\label{asmp:Lipschitz} \textbf{\(L\)-Lipschitz Constraint.} If \(p_j = 0\) for some \(j\in \mathcal{S}\), then for every response \(y_i\), we must have
    \begin{equation}
    \label{eq:LipschitzConstraintApp}
    p_i 
    \;\le\; 
    L\, A_{i,j}
    \;=\;
    L\,\|\mathbf{e}_i \,-\,\mathbf{e}_j\|.
    \end{equation}
\end{enumerate}
The effect is that whenever we force a particular negative \(j\) to have \(p_j=0\), any response \(i\) near \(j\) in embedding space also gets \emph{pushed down}, since \(p_i \le L\,A_{i,j}\). By selecting a set of $k$ negatives covering many ``bad'' or low-reward regions, we curb the policy's probability of generating undesirable responses.

\paragraph{Goal.} 
Define the feasible set of distributions:
\begin{equation}
\label{eq:feasibleRegionApp}
\mathcal{F}(\mathcal{S}) 
\;=\; 
\Bigl\{\,
\{p_i\}\colon p_j=0 \ \forall\,j\in \mathcal{S}, \ 
p_i \le L\, \min_{j\in \mathcal{S}} A_{i,j}\ \forall\,i\notin\{\,i_{\mathrm{top}}\}\cup\mathcal{S}
\Bigr\}.
\end{equation}
We then have a two-level problem:
\begin{align}
\nonumber
&\max_{\,\substack{\mathcal{S}\,\subseteq \{1,\dots,n\}\setminus\{i_{\mathrm{top}}\}\\ |\mathcal{S}|=k}}  
\quad
\max_{\substack{\{p_i\}\in \mathcal{F}(\mathcal{S}) \\ \sum_i p_i = 1,\;p_i\ge 0}}
\quad
\sum_{i=1}^n r_i\, p_i,
\\[0.75em]
&\text{subject to}\quad p_{i_{\mathrm{top}}}\text{ is unconstrained (no Lipschitz bound).}
\label{eq:lip_mainObjApp}
\end{align}
We seek \(\mathcal{S}\) that \emph{maximizes} the best possible Lipschitz-compliant expected reward.

\subsection{Coverage View and the MIP Formulation}

\paragraph{Coverage Cost.}
To highlight the crucial role of ``covering'' low-reward responses, define a weight
\begin{equation}
\label{eq:appWeightDef}
w_i 
\;=\;
\exp\bigl(\,\overline{r} - r_i\bigr),
\end{equation}
where \(\overline{r}\) can be, for instance, the average reward \(\frac{1}{n}\sum_{i=1}^n r_i\). 
Then a natural \emph{coverage} cost is
\begin{equation}
\label{eq:coverageCostApp}
\mathrm{Cost}(\mathcal{S})
\;=\;
\sum_{i=1}^n
  w_i
  \,\min_{j\in \mathcal{S}}
    A_{i,j}.
\end{equation}
A small \(\min_{j\in \mathcal{S}} A_{i,j}\) means response \(i\) is ``close'' to at least one negative center \(j\). If \(r_i\) is low, then \(w_i\) is large, so we put higher penalty on leaving \(i\) uncovered. Minimizing \(\mathrm{Cost}(\mathcal{S})\) ensures that \emph{important} (low-reward) responses are forced near penalized centers, thus \emph{suppressing} them in the policy distribution.

\paragraph{MIP \(\mathcal{P}\) for Coverage Minimization.}

We can write a mixed-integer program:

\begin{align}
\label{eq:covMIPApp}
\nonumber
\textbf{Problem } \mathcal{P}:\;
&\min_{\,\substack{x_j \in \{0,1\}\\ z_{i,j}\in \{0,1\}\\ y_i \ge 0}} 
\sum_{i=1}^n 
  w_i\,y_i,
\\
&\text{subject to}
\begin{cases}
\displaystyle
\sum_{j=1}^n x_j = k, 
\\[0.2em]
z_{i,j}\le x_j,\quad 
\sum_{j=1}^n z_{i,j} = 1,\quad \forall\,i,
\\[0.2em]
y_i\le A_{i,j} + M\,(1 - z_{i,j}),
\\[0.2em]
y_i\ge A_{i,j} - M\,(1 - z_{i,j}),\quad \forall\,i,j,
\end{cases}
\end{align}
where \(M = \max_{i,j} A_{i,j}\). Intuitively, each \(x_j\) indicates if \(j\) is chosen as a negative; each \(z_{i,j}\) indicates whether \(i\) is ``assigned'' to \(j\). At optimality, \(y_i = \min_{j\in \mathcal{S}} A_{i,j}\), so the objective 
\(\sum_i w_i\,y_i\) is precisely \(\mathrm{Cost}(\mathcal{S})\). Hence solving \(\mathcal{P}\) yields \(\mathcal{S}^*\) that \emph{minimizes} coverage cost~\eqref{eq:coverageCostApp}.

\subsection{Key Lemma: Equivalence of Coverage Minimization and Lipschitz Suppression}

\begin{lemma}[Coverage $\Leftrightarrow$ Suppression]
\label{lem:appCoverageLemma}
Assume \ref{asmp:Lipschitz} (the \(L\)-Lipschitz constraint, \eqref{eq:LipschitzConstraintApp}) and let \(i_{\mathrm{top}}\) be a highest-reward index. Suppose \(\mathcal{S}\subseteq\{1,\dots,n\}\setminus\{i_{\mathrm{top}}\}\) is a subset of size~\(k\). Then:
\begin{enumerate}[label=(\roman*), leftmargin=1.25em, itemsep=0.5em]
    \item Choosing \(\mathcal{S}\) that \emph{minimizes} \(\mathrm{Cost}(\mathcal{S})\) yields the strongest suppression of low-reward responses and thus the best possible \emph{feasible} expected reward under the Lipschitz constraint.
    \item Conversely, any set \(\mathcal{S}\) achieving the \emph{highest} feasible expected reward necessarily \emph{minimizes} \(\mathrm{Cost}(\mathcal{S})\).
\end{enumerate}
\end{lemma}

\begin{proof}
\textbf{(i) Minimizing \(\mathrm{Cost}(\mathcal{S})\) improves expected reward.}\\
Once we pick \(\mathcal{S}\), we set \(p_j=0\) for all \(j\in \mathcal{S}\). By \ref{asmp:Lipschitz}, any \(y_i\) is then forced to satisfy \(p_i \le L\,A_{i,j}\) for all \(j\in \mathcal{S}\). Hence
\[
p_i 
\;\le\; 
L \,\min_{j\in \mathcal{S}} A_{i,j}.
\]
If \(\min_{j\in \mathcal{S}} A_{i,j}\) is large, then \(p_i\) could be large; if it is small (particularly for low-reward \(r_i\)), we effectively suppress \(p_i\). By weighting each \(i\) with \(w_i = e^{\overline{r}-r_i}\), we see that leaving low-reward \(y_i\) \emph{far} from all negatives raises the risk of high \(p_i\). Minimizing \(\sum_i w_i\,\min_{j\in \mathcal{S}} A_{i,j}\) ensures that any \(i\) with large \(w_i\) (i.e.\ small \(r_i\)) has a small distance to at least one chosen center, thus bounding its probability more tightly. 

Meanwhile, the best candidate \(i_{\mathrm{top}} \in \{1,\dots,n\}\) remains unconstrained, so the policy can always place mass \(\approx 1\) on \(i_{\mathrm{top}}.\) Consequently, a set \(\mathcal{S}\) that better ``covers'' low-reward points must yield a higher feasible expected reward \(\sum_i r_i p_i\). 

\textbf{(ii) Necessity of Minimizing \(\mathrm{Cost}(\mathcal{S})\).}\\
Conversely, if there were a set \(\mathcal{S}\) that \emph{did not} minimize \(\mathrm{Cost}(\mathcal{S})\) but still provided higher feasible expected reward, that would imply we found a distribution \(\{p_i\}\) violating the Lipschitz bound on some low-reward region. Formally, \(\mathcal{S}\) that yields strictly smaller coverage cost would impose stricter probability suppression on harmful responses. By part~(i), that coverage-lowering set should then yield an even higher feasible reward, a contradiction.
\end{proof}

\subsection{Main Theorem: Optimality of \(\mathcal{P}\) for Lipschitz Alignment}

\begin{theorem}[Optimal Negative Set via \(\mathcal{P}\)]
\label{thm:appOptNegatives}
Let \(\mathcal{S}^*\) be the solution to the MIP \(\mathcal{P}\) in \eqref{eq:covMIPApp}, i.e.\ it \emph{minimizes} \(\mathrm{Cost}(\mathcal{S})\). Then \(\mathcal{S}^*\) also \emph{maximizes} the objective \eqref{eq:lip_mainObjApp}. Consequently, picking \(\mathcal{S}^*\) and allowing free probability on \(i_{\mathrm{top}} \approx \arg\max_i\, r_i\) yields the \emph{optimal} Lipschitz-compliant policy.
\end{theorem}

\begin{proof}
By construction, solving \(\mathcal{P}\) returns \(\mathcal{S}^*\) with
\(\displaystyle
\mathrm{Cost}(\mathcal{S}^*)
\;=\;
\min_{|\mathcal{S}|=k}\,\mathrm{Cost}(\mathcal{S}).
\)
Lemma~\ref{lem:appCoverageLemma} then states that such an \(\mathcal{S}^*\) simultaneously \emph{maximizes} the best possible feasible expected reward. Hence \(\mathcal{S}^*\) is precisely the negative set that achieves the maximum of \eqref{eq:lip_mainObjApp}.
\end{proof}

\paragraph{Interpretation.} 
Under a mild Lipschitz assumption in embedding space, penalizing (assigning zero probability to) a small set \(\mathcal{S}\) \emph{and} forcing all items near \(\mathcal{S}\) to have small probability is equivalent to a \emph{coverage} problem. Solving (or approximating) \(\mathcal{P}\) selects negatives that push down low-reward modes as effectively as possible.

\vspace{-0.05in}
\subsection{Discussion and Practical Implementation}

\vspace{-0.05in}
\textsc{Opt-Select} thus emerges from optimizing coverage: 
\begin{enumerate}[leftmargin=1.25em, itemsep=0.5em]
    \item \textbf{Solve or approximate} the MIP \(\mathcal{P}\) to find the best subset \(\mathcal{S}\subseteq\{1,\dots,n\}\setminus\{i_{\mathrm{top}}\}\).
    \item \textbf{Force} \(p_j=0\) for each \(j\in \mathcal{S}\); \textbf{retain} \(i_{\mathrm{top}}\) with full probability (\(p_{i_{\mathrm{top}}}\approx 1\)), subject to normalizing the distribution. 
\end{enumerate}
In practice, local search or approximate clustering-based approaches (e.g.\ Weighted \(k\)-Medoids) can find good solutions without exhaustively solving \(\mathcal{P}\). The method ensures that near any chosen negative \(j\), all semantically similar responses \(i\) have bounded probability \(p_i \le L\,A_{i,j}\). Consequently, \textsc{Opt-Select} \emph{simultaneously} covers and suppresses undesired modes while preserving at least one high-reward response unpenalized.

\vspace{-0.05in}
\paragraph{Additional Remarks.}

\vspace{-0.15in}
\begin{itemize}[leftmargin=1.25em, itemsep=0.5em]
    \item The single-positive assumption reflects a practical design where one high-reward response is explicitly promoted. This can be extended to multiple positives, e.g.\ top \(m^+\) responses each unconstrained.
    \item For large \(n\), the exact MIP solution may be expensive; local search (see Appendix~\ref{sec:local_search_kmedoids}) still achieves a constant-factor approximation.
    \item The embedding-based Lipschitz constant \(L\) is rarely known exactly; however, the coverage perspective remains valid for “sufficiently smooth” reward behaviors in the embedding space.
\end{itemize}

Overall, these results solidify \textsc{Opt-Select} as a principled framework for negative selection under Lipschitz-based alignment objectives.

\section{Local Search Guarantees for Weighted \texorpdfstring{$k$}{k}-Medoids and Lipschitz-Reward Approximation}
\label{sec:local_search_kmedoids}

In this appendix, we show in \Cref{thm:local_search_kmedoids} that a standard \emph{local search} algorithm for \emph{Weighted $k$-Medoids} achieves a constant-factor approximation in polynomial time.


\subsection{Weighted \texorpdfstring{$k$}{k}-Medoids Setup}

We are given:
\begin{itemize}
\item A set of $n$ points, each indexed by $i\in\{1,\dots,n\}$.
\item A distance function $d(i,j)\ge0$, which forms a metric: $d(i,j)\le d(i,k)+d(k,j)$, $d(i,i)=0$, $d(i,j)=d(j,i)$.
\item A nonnegative \emph{weight} $w_i$ for each point $i$.
\item A budget $k$, $1\le k\le n$.
\end{itemize}
We wish to pick a subset $\mathcal{S}\subseteq\{1,\dots,n\}$ of \emph{medoids} (centers) with size $|\mathcal{S}|=k$ that minimizes the objective
\begin{align}
\label{eq:wkmedoids_objective}
\mathrm{Cost}(\mathcal{S})
\;=\;
\sum_{i=1}^n
  w_i
  \cdot
  \min_{\,j\in \mathcal{S}}\,
    d(i,j).
\end{align}
We call this the \textbf{Weighted $k$-Medoids} problem.  Note that \textbf{medoids} must come from among the data points, as opposed to $k$-median or $k$-means where centers can be arbitrary points in the metric or vector space. Our Algorithm \ref{alg:opt_select} reduces to exactly this problem.

\subsection{Coordinate Descent Algorithm via Local Search}

Our approach to the NP-hardness of Algorithm \ref{alg:opt_select} was to recast it as a simpler coordinate descent algorithm in Algorithm \ref{alg:opt_select_local_search}, wherein we do a local search at every point towards achieving the optimal solution.
Let $\textsc{Cost}(\mathcal{S})$ be as in \eqref{eq:wkmedoids_objective}.

\begin{enumerate}
\item \textbf{Initialize:} pick any subset $\mathcal{S}\subseteq\{1,\dots,n\}$ of size $k$ (e.g.\ random or greedy).
\item \textbf{Repeat}: Try all possible single \emph{swaps} of the form
\[
   \mathcal{S}' 
   \;=\; 
   \bigl(\,\mathcal{S}\setminus\{\,j\}\bigr)
   \,\cup\,
   \{\,j'\},
\]
where $j\in\mathcal{S}$ and $j'\notin\mathcal{S}$.  
\item \textbf{If any swap improves cost}: i.e.\ $\mathrm{Cost}(\mathcal{S}') < \mathrm{Cost}(\mathcal{S})$, then set $\mathcal{S}\leftarrow \mathcal{S}'$ and continue.
\item \textbf{Else terminate}: no single swap can further reduce cost.
\end{enumerate}

When the algorithm stops, we say $\mathcal{S}$ is a \emph{local optimum under 1-swaps}.

\subsection{Constant-Factor Approximation in Polynomial Time}

We now present and prove a result: such local search yields a constant-factor approximation.  Below, we prove a version with a \emph{factor 5} guarantee for Weighted $k$-Medoids.  Tighter analyses can improve constants, but 5 is a commonly cited bound for this simple variant.


\begin{theorem}[Local Search for Weighted $k$-Medoids]
\label{thm:local_search_kmedoids}
Let $\mathcal{S}^*$ be an \textbf{optimal} subset of medoids of size $k$. Let $\widehat{\mathcal{S}}$ be any \textbf{local optimum} obtained by the above 1-swap local search. Then
\begin{equation}
    \mathrm{Cost}\bigl(\widehat{\mathcal{S}}\bigr)
  \;\;\le\;\;
  5
  \,\times\,
  \mathrm{Cost}\bigl(\mathcal{S}^*\bigr).
\end{equation}

Moreover, the procedure runs in polynomial time (at most $\bigl(\binom{n}{k}\bigr)$ “worse-case” swaps in principle, but in practice each improving swap decreases cost by a non-negligible amount, thus bounding the iteration count).
\end{theorem}

\begin{proof}
\textbf{Notation.}
\begin{itemize}
\item Let $\widehat{\mathcal{S}}$ be the final local optimum of size $k$. 
\item Let $\mathcal{S}^*$ be an optimal set of size $k$. 
\item For each point $i$, define
\[
  r_i 
  \;=\; 
  d\!\bigl(i,\widehat{\mathcal{S}}\bigr)
  \;=\;
  \min_{j \in \widehat{\mathcal{S}}} d(i,j),
  \quad
  r_i^*
  \;=\;
  \min_{j\in \mathcal{S}^*} d(i,j).
\]
Thus $\mathrm{Cost}(\widehat{\mathcal{S}}) = \sum_i w_i\,r_i$ and $\mathrm{Cost}(\mathcal{S}^*) = \sum_i w_i\,r_i^*$.

\item Let $c(\mathcal{S}) = \sum_i w_i\,d(i,\mathcal{S})$ as shorthand for $\mathrm{Cost}(\mathcal{S})$. 
\end{itemize}

\noindent
\textbf{Step 1: Construct a ``Combined'' Set.}  
Consider 
\[
  \mathcal{S}^\dagger 
  \;=\;
  \widehat{\mathcal{S}}
  \;\cup\;
  \mathcal{S}^*.
\]
We have $|\mathcal{S}^\dagger|\le 2k$.  Let $c(\mathcal{S}^\dagger) = \sum_i w_i\,d\bigl(i,\mathcal{S}^\dagger\bigr)$.  

Observe that
\[
  d\!\bigl(i,\mathcal{S}^\dagger\bigr)
  \;=\;
  \min\!\bigl\{
    d\!\bigl(i,\widehat{\mathcal{S}}\bigr),\,
    d\!\bigl(i,\mathcal{S}^*\bigr)
  \bigr\}
  \;=\;
  \min\{\,r_i,\;r_i^*\}.
\]
Hence
\[
  c(\mathcal{S}^\dagger)
  \;=\;
  \sum_{i=1}^n 
    w_i\,
    \min\{\,r_i,\ r_i^*\}.
\]
We will relate $c(\mathcal{S}^\dagger)$ to $c(\widehat{\mathcal{S}})$ and $c(\mathcal{S}^*)$.

\medskip
\noindent
\textbf{Step 2: Partition Points According to $\mathcal{S}^*$.}  
For each $j^*\in \mathcal{S}^*$, define the cluster 
\[
  C(j^*)
  \;=\;
  \bigl\{
    i \mid j^* 
    = 
    \arg\min_{j'\in \mathcal{S}^*} d(i,j')
  \bigr\}.
\]
Hence $\{\,C(j^*)\,:\,j^*\in \mathcal{S}^*\}$ is a partition of $\{1,\dots,n\}$.  We now group the cost contributions by these clusters.

\medskip
\noindent
\textbf{Goal: Existence of a Good Swap.}
We will \emph{assume} $c(\widehat{\mathcal{S}})>5\,c(\mathcal{S}^*)$ and derive a contradiction by producing a \emph{profitable swap} that local search should have found.  

Specifically, we show that there must be a center $j^*\in \mathcal{S}^*$ whose cluster $C(j^*)$ is “costly enough” under $\widehat{\mathcal{S}}$, so that swapping out some center $j\in\widehat{\mathcal{S}}$ for $j^*$ significantly reduces cost.  But since $\widehat{\mathcal{S}}$ was a local optimum, no such profitable swap could exist.  This contradiction implies $c(\widehat{\mathcal{S}})\le 5\,c(\mathcal{S}^*)$.

\medskip
\noindent
\textbf{Step 3: Detailed Bounding.}

We have
\[
  c(\mathcal{S}^\dagger)
  =
  \sum_{i=1}^n
    w_i\,\min\{r_i,\,r_i^*\}
  \;\le\;
  \sum_{i=1}^n
    w_i\,r_i^*
  =
  c(\mathcal{S}^*).
\]
Similarly, 
\[
  c(\mathcal{S}^\dagger)
  \;\le\;
  \sum_{i=1}^n
    w_i\,r_i
  =
  c\!\bigl(\widehat{\mathcal{S}}\bigr).
\]
Hence $c(\mathcal{S}^\dagger)\le\min\bigl\{c(\widehat{\mathcal{S}}),\,c(\mathcal{S}^*)\bigr\}$.  
Now define
\[
   D
   \;=\;
   \sum_{i=1}^n
     w_i
     \,\bigl[
       r_i
       -
       \min\{\,r_i,\,r_i^*\}
     \bigr]
   \;=\;
   \sum_{i=1}^n
     w_i\,\bigl(r_i - r_i^*\bigr)_{+},
\]
where $(x)_{+}=\max\{x,0\}$.  By rearranging,
\[
  \sum_{i=1}^n w_i\,r_i
  \;-\;
  \sum_{i=1}^n w_i\,\min\{\,r_i,\,r_i^*\}
  \;=\;
  D.
\]
Thus
\[
  c(\widehat{\mathcal{S}}) - c(\mathcal{S}^\dagger)
  \;=\;
  D
  \;\;\ge\;\;
  c(\widehat{\mathcal{S}}) - c(\mathcal{S}^*).
\]
So
\[
  D
  \;\ge\;
  c\!\bigl(\widehat{\mathcal{S}}\bigr)
  \;-\;
  c\!\bigl(\mathcal{S}^*\bigr).
\]
Under the assumption $c(\widehat{\mathcal{S}})>5\,c(\mathcal{S}^*)$, we get 
\[
  D
  \;>\;
  4\,c(\mathcal{S}^*).
  \tag{*}
\]

\medskip
\noindent
\textbf{Step 4: Find a Center $j^*$ with Large $D$ Contribution.}
We now “distribute” $D$ over clusters $C(j^*)$.  Let
\[
  D_{j^*}
  =
  \sum_{i \in C(j^*)}
    w_i\,\bigl(r_i - r_i^*\bigr)_{+}.
\]
Then 
\(\displaystyle
D=\sum_{j^*\in \mathcal{S}^*} D_{j^*}.
\)
Since $D>4\,c(\mathcal{S}^*)$, at least one $j^*\in \mathcal{S}^*$ satisfies
\[
  D_{j^*}
  \;>\;
  4\,
  \frac{c(\mathcal{S}^*)}{|\mathcal{S}^*|}
  \;=\;
  4\,\frac{c(\mathcal{S}^*)}{k},
\]
because $|\mathcal{S}^*|=k$.  Denote this center as $j^*_{\text{large}}$ and its cluster $C^* = C(j^*_{\text{large}})$.

\medskip
\noindent
\textbf{Step 5: Swapping $j^*$ into $\widehat{\mathcal{S}}$.}
Consider the swap
\[
  \widehat{\mathcal{S}}_{\mathrm{swap}}
  \;=\;
  \Bigl(
    \widehat{\mathcal{S}}\setminus\bigl\{\,j_{\mathrm{out}}\bigr\}
  \Bigr)
  \,\cup\,
  \bigl\{\,j^*_{\text{large}}\bigr\}
\]
where $j_{\mathrm{out}}$ is whichever center in $\widehat{\mathcal{S}}$ we choose to remove.  We must show that for an appropriate choice of $j_{\mathrm{out}}$, the cost $c(\widehat{\mathcal{S}}_{\mathrm{swap}})$ is at least $(r_i - r_i^*)_{+}$ smaller on average for the points in $C^*$, forcing a net cost reduction large enough to offset any potential cost increase for points outside $C^*$.

In detail, partition $\widehat{\mathcal{S}}$ into $k$ clusters under \emph{Voronoi} assignment:
\[
  \widehat{C}(j)
  \;=\;
  \bigl\{
    i : 
    j=\arg\min_{\,x\in\widehat{\mathcal{S}}} d(i,x)
  \bigr\},
  \quad
  j\in \widehat{\mathcal{S}}.
\]
Since $|\,\widehat{\mathcal{S}}|=k$, there must exist at least one $j_{\mathrm{out}}\in \widehat{\mathcal{S}}$ whose cluster $\widehat{C}(j_{\mathrm{out}})$ has weight
\(\displaystyle
\sum_{i\in\widehat{C}(j_{\mathrm{out}})} w_i
 \;\le\;
 \frac{1}{k}\,\sum_{i=1}^n w_i.
\)
We remove that $j_{\mathrm{out}}$ and add $j^*_{\text{large}}$.

\medskip
\noindent
\textbf{Step 6: Net Cost Change Analysis.}
After the swap, 
\[
   c\bigl(\widehat{\mathcal{S}}_{\mathrm{swap}}\bigr)
   -
   c\bigl(\widehat{\mathcal{S}}\bigr)
   \;=\;
   \underbrace{
     \Delta_{\mathrm{in}}
   }_{\text{improvement in }C^*}
   \;+\;
   \underbrace{
     \Delta_{\mathrm{out}}
   }_{\text{possible cost increase outside }C^*}.
\]
Points $i\in C^*$ can now be served by $j^*_{\text{large}}$ at distance $r_i^*(\le r_i)$, so 
\[
  \Delta_{\mathrm{in}}
  \;\le\;
  \sum_{i \in C^*} w_i\,
     \Bigl[
       d\bigl(i,\widehat{\mathcal{S}}_{\mathrm{swap}}\bigr)
       -
       d\bigl(i,\widehat{\mathcal{S}}\bigr)
     \Bigr]
  \;\le\;
  \sum_{i\in C^*} w_i
    \,\bigl(r_i^* - r_i\bigr).
\]
But recall $r_i^* \le r_i$ or $r_i^*\le r_i$; for $i\in C^*$, we specifically have $(r_i-r_i^*)_{+}$ is \emph{often} positive. Precisely:
\[
  \Delta_{\mathrm{in}}
  \;\le\;
  \sum_{i\in C^*} w_i\,\bigl(r_i^* - r_i\bigr)
  \;=\;
  -\,\sum_{i\in C^*} w_i\,\bigl(r_i - r_i^*\bigr).
\]
Hence
\[
  \Delta_{\mathrm{in}}
  \;\le\;
  -\sum_{i\in C^*} w_i\,(r_i - r_i^*)_{+}.
\]
On the other hand, some points outside $C^*$ may lose $j_{\mathrm{out}}$ as a center, which might increase their distances:
\[
  \Delta_{\mathrm{out}}
  \;=\;
  \sum_{i\notin C^*}
     w_i\,
     \Bigl[
       d\bigl(i,\widehat{\mathcal{S}}_{\mathrm{swap}}\bigr)
       -
       d\bigl(i,\widehat{\mathcal{S}}\bigr)
     \Bigr].
\]
Since each point can still use any other center in $\widehat{\mathcal{S}}\setminus\{\,j_{\mathrm{out}}\}$, 
\[
  d\!\bigl(i,\widehat{\mathcal{S}}_{\mathrm{swap}}\bigr)
  \;\le\;
  \min\!\bigl\{
    d\!\bigl(i,\widehat{\mathcal{S}}\setminus \{j_{\mathrm{out}}\}\bigr),\
    d\!\bigl(i,j^*_{\text{large}}\bigr)
  \bigr\}.
\]
Thus for each $i$, 
\[
  d\bigl(i,\widehat{\mathcal{S}}_{\mathrm{swap}}\bigr)
  \;\le\;
  d\bigl(i,\widehat{\mathcal{S}}\bigr)
\]
unless the \emph{only} center in $\widehat{\mathcal{S}}$ that served $i$ was $j_{\mathrm{out}}$. But the total weight of $\widehat{C}(j_{\mathrm{out}})$ is at most $\frac{1}{k}\sum_{i} w_i$.  Thus,
\[
  \Delta_{\mathrm{out}}
  \;\le\;
  \sum_{i\in \widehat{C}(j_{\mathrm{out}})} 
       w_i\,
       \Bigl[
         d\bigl(i,\widehat{\mathcal{S}}_{\mathrm{swap}}\bigr)
         -
         d\bigl(i,\widehat{\mathcal{S}}\bigr)
       \Bigr]
  \;\le\;
  \sum_{i\in \widehat{C}(j_{\mathrm{out}})} 
    w_i\,d\bigl(j_{\mathrm{out}},\,j^*_{\text{large}}\bigr),
\]
because $i$ is at distance at most $d(i,j_{\mathrm{out}})+d(j_{\mathrm{out}},j^*_{\text{large}})$ to $j^*_{\text{large}}$. And $d(i,\widehat{\mathcal{S}})\ge d(i,j_{\mathrm{out}})$ by definition of $\widehat{C}(j_{\mathrm{out}})$. Hence
\[
  \Delta_{\mathrm{out}}
  \;\le\;
  \Bigl(
    \sum_{i\in \widehat{C}(j_{\mathrm{out}})} w_i
  \Bigr)
  \cdot
  d\bigl(j_{\mathrm{out}},\,j^*_{\text{large}}\bigr)
  \;\le\;
  \frac{1}{k}
  \Bigl(\sum_{i=1}^n w_i\Bigr)
  \cdot
  d\bigl(j_{\mathrm{out}},\,j^*_{\text{large}}\bigr).
\]

\medskip
\noindent
\textbf{Step 7: Arriving at a contradiction.}
We get
\[
  c\bigl(\widehat{\mathcal{S}}_{\mathrm{swap}}\bigr)
  -
  c\bigl(\widehat{\mathcal{S}}\bigr)
  =
  \Delta_{\mathrm{in}} + \Delta_{\mathrm{out}}
  \;\le\;
  -\sum_{i\in C^*}
     w_i
     \bigl(r_i - r_i^*\bigr)_{+}
  \;+\;
  \frac{1}{k}
  \Bigl(\sum_{i} w_i\Bigr)
  \,
  d\bigl(j_{\mathrm{out}},j^*_{\text{large}}\bigr).
\]
But recall
\[
  \sum_{i\in C^*} 
   w_i\,
   (r_i - r_i^*)_{+}
   =
   D_{j^*_{\text{large}}}
   \;>\;
   4\,\frac{c(\mathcal{S}^*)}{k},
\]
from step 5.  Meanwhile, $d\bigl(j_{\mathrm{out}},\,j^*_{\text{large}}\bigr)\le c(\mathcal{S}^*)$ is a standard bound because $j^*_{\text{large}}$ must be served in $\mathcal{S}^*$ by some center at distance at most $c(\mathcal{S}^*) / \sum_i w_i$ \emph{or} by the triangle inequality, we can also argue $d(j_{\mathrm{out}},j^*_{\text{large}})\le$ the diameter factor times the cost.  More refined bounding uses per-point comparisons.

Hence
\[
  \Delta_{\mathrm{out}}
  \;\le\;
  \frac{1}{k}
  \Bigl(\sum_{i} w_i\Bigr)
  \,c(\mathcal{S}^*)
  \,/\,\bigl(\sum_{i} w_i\bigr)
  \;\;=\;\;
  \frac{c(\mathcal{S}^*)}{k}.
\]
Thus
\[
  c(\widehat{\mathcal{S}}_{\mathrm{swap}})
  -
  c(\widehat{\mathcal{S}})
  \;\le\;
  -\,4\,\frac{c(\mathcal{S}^*)}{k}
  \;+\;
  \frac{c(\mathcal{S}^*)}{k}
  \;=\;
  -\,3\,\frac{c(\mathcal{S}^*)}{k}
  \;<\;0,
\]
i.e.\ a net improvement.  This contradicts the local optimality of $\widehat{\mathcal{S}}$.  

Therefore our original assumption $c(\widehat{\mathcal{S}})>5\,c(\mathcal{S}^*)$ must be false, so $c(\widehat{\mathcal{S}})\le 5\,c(\mathcal{S}^*)$.  

\medskip
\noindent
\textbf{Time Complexity.}
Each swap test requires $O(n)$ time to update $\mathrm{Cost}(\mathcal{S})$.  There are at most $k\,(n-k)$ possible 1-swaps.  Each accepted swap \emph{strictly} decreases cost by at least 1 unit (or some positive $\delta$-fraction if distances are discrete/normalized).  Since the minimal cost is $\ge0$, the total number of swaps is polynomially bounded.  Thus local search terminates in polynomial time with the promised approximation.

\end{proof}

\begin{remark}[Improved Constants]
A more intricate analysis can tighten the factor 5 in \Cref{thm:local_search_kmedoids} to 3 or 4.  See, e.g., \citep{gupta2008simpler,arya2001local} for classical refinements.  The simpler argument here suffices to establish the main principles.
\end{remark}



% \section{CoreSets for Representative Selection}
% \label{sec:coreset_subsection}

\section{Constant-Factor Approximation for Subset Selection Under Bounded Intra-Cluster Distance}
\label{sec:constant_factor_subset_selection}

\noindent
The term \emph{coreset} originates in computational geometry and machine learning, referring to a subset of data that \emph{approximates} the entire dataset with respect to a particular objective or loss function \citep{bachem2017practical,feldman2020turning}. More precisely, a coreset $\mathcal{C}$ for a larger set $\mathcal{X}$ is often defined such that, for any model or solution $w$ in a hypothesis class, the loss over $\mathcal{C}$ is within a small factor of the loss over $\mathcal{X}$. 

In the context of \textsc{AMPO-Coreset}, the $k$-means clustering subroutine identifies \emph{representative} embedding-space regions, and by choosing a single worst-rated example from each region, we mimic a coreset-based selection principle: our selected negatives approximate the \emph{distributional diversity} of the entire batch of responses. In essence, we seek a small but well-covered negative set that ensures the model receives penalizing signals for all major modes of undesired behavior. 

Empirically, such coverage-driven strategies can outperform purely score-based selection (Section \ref{sec:ampo_bottomk}) when the reward function is noisy or the model exhibits rare but severe failure modes. By assigning at least one negative from each cluster, \textsc{AMPO-Coreset} mitigates the risk of ignoring minority clusters, which may be infrequent yet highly problematic for alignment. As we show in subsequent experiments, combining \emph{coreset-like coverage} with \emph{reward-based filtering} yields robust policy updates that curb a wide range of undesirable outputs.



We give a simplified theorem showing how a local-search algorithm can achieve a fixed (constant) approximation factor for selecting \(k\) ``negative'' responses. Our statement and proof are adapted from the classical \emph{Weighted \(k\)-Medoids} analysis, but use simpler notation and explicit assumptions about bounded intra-cluster distance.

% \subsection{Problem Statement and Assumptions}

% \noindent
% \textbf{Weighted k-medoids setup Setup.}  
% We have \(n\) items, indexed by \(i=1,\dots,n\). Each item \(i\) has a \emph{weight} \(w_i \ge 0\) and lies in a metric space with distance \(d(i,j)\). We want to choose a subset \(\mathcal{S} \subseteq \{1,\dots,n\}\) of size \(k\) (i.e., \(|\mathcal{S}|=k\)) that minimizes
% \[
%   \mathrm{Cost}(\mathcal{S})
%   \;=\;
%   \sum_{i=1}^n
%     w_i \,\min_{\,j \in \mathcal{S}} d(i,j).
% \]
% This is precisely the \emph{Weighted \(k\)-Medoids} objective. The points \(j\in \mathcal{S}\) can be viewed as ``negatives'' or ``centers,'' penalizing nearby items by forcing them to remain close (hence incurring a smaller distance cost).

% We aim to present a theorem here with strong assumptions, that may not hold truly in practice, but would be indicative of a performance guarantee, should such assumptions hold.

\noindent
\subsection{Additional Assumptions:}
\textbf{Assumption 1: Bounded number of clusters k.}  
We assume that the data partitions into natural clusters such that the number of such clusters is equal to the number of examples we draw from the negatives. It is of course likely that at sufficiently high temperature, an LLM may deviate from such assumptions, but given sufficiently low sampling temperature, the answers, for any given query, may concentrate to a few attractors.

\textbf{Assumption 2: Bounded Intra-Cluster Distance.}  
We assume that the data can be partitioned into natural clusters of bounded diameter \(d_{\max}\). This assumption helps us simplify our bounds, towards rigorous guarantees, and we wish to state that such an assumption may be too strict to hold in practice, especially in light of Assumption 1.




Given these assumptions, We present a distribution-dependent coreset guarantee for selecting a small ``negative'' subset of responses for a given query, thus enabling the policy to concentrate probability on the highest-rated responses. Unlike universal coreset theory, we only require that this negative subset works well for typical distributions of responses, rather than for every conceivable set of responses.

\subsection{Setup: Queries, Responses, and Ratings}

\noindent
\textbf{Queries and Candidate Responses.}  
We focus on a single \emph{query} \(x\), which admits a finite set of \(m\) candidate responses 
\[
  \{\,y_1,\dots,y_m\}.
\]
Each response \(y_i\) has a scalar rating \(r_i \in [0,1]\). For notational convenience, we assume \(r_i\) is normalized to \([0,1]\). A larger \(r_i\) indicates a better (or more desirable) response.

\vspace{0.5em}
\noindent
\textbf{Negative Ratings via Exponential Weights.}  
Let 
\begin{align}
\overline{r} 
\;=\; 
\frac{1}{m}\sum_{i=1}^m r_i
\quad\text{(the mean rating)}, 
\quad
w_i 
\;=\; 
\exp\bigl(\overline{r}-r_i\bigr).
\label{eq:neg_weight}
\end{align}
Then \(w_i\) is larger when \(r_i\) is smaller. One may also employ alternative references (\(\max r_i\) instead of \(\overline{r}\)), or re-scaling to maintain bounded ranges.

\subsection{Policy Model and Subset Selection}

\noindent
\textbf{Policy Distribution Over Responses.}  
A policy \(P_\theta(y \mid x)\) assigns a probability \(p_i \ge 0\) to each response \(y_i\), satisfying 
\(\sum_{i=1}^m p_i = 1\). The \emph{expected rating} is 
\[
  \mathrm{ER}(p_1,\dots,p_m)
  \;=\;
  \sum_{i=1}^m p_i\,r_i.
\]

\vspace{0.5em}
\noindent
\textbf{Negative Subset and Probability Suppression.}  
We aim to choose a small subset \(\mathcal{S}\subseteq\{\,1,\dots,m\}\) of size \(k\), each member of which is assigned probability zero: 
\[
   p_j = 0,\quad\forall j\in \mathcal{S}.
\]
In addition, we impose a \textit{Lipschitz-like} rule that if \(p_j=0\) for \(j\in\mathcal{S}\), then any response \(y_i\) ``close'' to \(y_j\) in some embedding space must also have probability bounded by 
\[
  p_i 
  \;\le\;
  L\,\|\mathbf{e}_i - \mathbf{e}_j\|,
\]
where \(\mathbf{e}_i\) is an embedding of \(y_i\). If \(y_j\) is \emph{negatively rated}, then forcing \(p_j=0\) also forces small probability on responses near \(y_j\). This ensures undesired modes get suppressed.

\vspace{0.5em}
\noindent
\textbf{Concentrating Probability on Top Responses.}  
We allow the policy to place nearly all probability on a small handful of high-rated responses, so that the expected rating \(\sum_{i=1}^m p_i r_i\) is maximized. Indeed, the policy will try to push mass towards the highest \(r_i\) while setting \(p_j=0\) on low-rated responses in \(\mathcal{S}\).

\noindent
\textbf{Sampling Response-Sets or ``Solutions.''}  
We suppose that the set \(\{y_1,\dots,y_m\}\) with ratings \(\{r_i\}\) arises from some distributional process (for instance, \(\mathcal{D}\) might represent typical ways the system could generate or rank responses). Denote a random draw by 
\[
  \bigl(\{y_1,\dots,y_m\},\,\{r_i\}\bigr)
  \;\sim\; 
  \mathcal{D}.
\]
We only require that our negative subset \(\mathcal{S}\) yield a near-optimal Lipschitz-compliant policy \emph{for a typical realization from \(\mathcal{D}\)}, rather than for every possible realization.

\vspace{0.5em}
\noindent
\textbf{Clustering in Embedding Space.}  
Let \(\mathbf{e}_i\in\mathbb{R}^d\) be an embedding for each response \(y_i\). Suppose we partition \(\{1,\dots,m\}\) into \(k\) clusters \(C_1,\dots,C_k\) (each of bounded diameter at most \(d\)), and within each cluster \(C_j\), pick exactly one ``negative'' index \(i_j^- \in C_j\). This yields 
\[
   \mathcal{S} 
   \;=\; 
   \{\, i_1^-, \dots, i_k^-\}.
\]
We then penalize each \(y_{i_j^-}\) by setting \(p_{\,i_j^-}=0\). Consequently, for any \(y_i \in C_j\), the Lipschitz suppression condition forces \(p_i \le L\,d\).  






\subsection{A Distribution-Dependent Coreset Guarantee}

We now state a simplified theorem that, under certain conditions on the distribution \(\mathcal{D}\), ensures that for most draws of queries and responses, the chosen subset \(\mathcal{S}\) yields a policy whose expected rating is within \((1\pm \varepsilon)\) of the optimal Lipschitz-compliant policy of size \(k\).

\begin{theorem}[Distribution-Dependent Negative Subset]
\label{thm:distribution_coreset_responses}
Let \(\mathcal{D}\) be a distribution that generates query-response sets \(\{y_1,\dots,y_m\}\), each with ratings \(\{r_i\}\subset [0,1]\). Assume we cluster the \(m\) responses into \(k\) groups \(C_1,\dots,C_k\) of diameter at most \(d\) in the embedding space, and choose exactly one ``negative'' index \(i_j^-\in C_j\). Let \(\mathcal{S}=\{\,i_1^-,\dots,i_k^-\}\). Suppose that:
\[
   \max_{i\in C_j}\,
   \|\mathbf{e}_i-\mathbf{e}_{\,i_j^-}\|
   \;\le\;d,
   \quad
   \forall\,j=1,\dots,k.
\]
Assume a Lipschitz constant \(L\), so that penalizing \(y_{i_j^-}\) (i.e.\ \(p_{\,i_j^-}=0\)) enforces \(p_i \le L\,d\) for all \(i\in C_j\). Then, under a sufficiently large random sample of queries/responses (or equivalently, a large i.i.d.\ sample from \(\mathcal{D}\) to refine the clustering), with high probability over that sample, for at least a \(\bigl(1-\delta\bigr)\) fraction of newly drawn query-response sets from \(\mathcal{D}\), the set \(\mathcal{S}\) induces a Lipschitz-compliant policy whose expected rating is within a factor \((1\pm \varepsilon)\) of the best possible among all $k$-penalized subsets.
\end{theorem}

\begin{proof}[Proof Sketch]
We give a high-level argument:

\textbf{1. Large Sample Captures Typical Configurations.} By drawing many instances of responses $\{y_i\}$, $\{r_i\}$ from $\mathcal{D}$, we can cluster them in such a way that \emph{any new} draw from $\mathcal{D}$ is, with probability at least $1-\delta$, either (a) close to one of our sampled configurations or (b) has measure less than $\delta$.

\textbf{2. Bounded-Diameter Clusters.} Suppose each cluster $C_j$ has diameter at most $d$, and we pick $i_j^- \in C_j$ as the ``negative.'' This implies every response $y_i$ in that cluster is at distance $\le d$ from $y_{i_j^-}$.

\textbf{3. Lipschitz Suppression.} If $p_{\,i_j^-}=0$, then $p_i \le L\,\|\mathbf{e}_i - \mathbf{e}_{\,i_j^-}\|\le L\,d$ for all $i \in C_j$. This ensures that the entire cluster $C_j$ cannot accumulate large probability mass on low-rated responses. Consequently, we push the policy distribution to concentrate on higher-rated responses (e.g.\ those \emph{not} near a penalized center).

\textbf{4. Near-Optimal Expected Rating.} For any typical new draw of $\{y_i\}$, $\{r_i\}$, a $k$-penalized Lipschitz policy can be approximated by using the same $k$ negatives $\mathcal{S}$. Because we ensure that the new draw is close to one of our sampled draws, the coverage or cluster assignment for the new $\{y_i\}$ is accurate enough that the resulting feasible policy is within a multiplicative $(1\pm \varepsilon)$ factor of the best possible $k$-subset. This completes the distribution-dependent argument.

\end{proof}


\section{Optimal Selection Code}
\label{sec:optimal_selection_computation}

In this section we provide the actual code used to compute the optimal selection.

\begin{lstlisting}[language=Python]
import numpy as np
from scipy.spatial.distance import cdist

def solve_local_search_min_dist_normalized(
    vectors: np.ndarray,
    rating: np.ndarray,
    k: int,
    max_iter: int = 100,
    random_seed: int = 42
):
    # Normalize ratings
    rating_min = np.min(rating)
    rating_max = np.max(rating)
    rating_normalized = (rating - rating_min) / (rating_max - rating_min) if rating_max > rating_min else np.zeros_like(rating) + 0.5  

    # Identify top-rated point
    excluded_top_index = int(np.argmax(rating_normalized))

    # Reduce dataset
    new_to_old = [idx for idx in range(len(rating_normalized)) if idx != excluded_top_index]
    vectors_reduced = np.delete(vectors, excluded_top_index, axis=0)
    rating_reduced = np.delete(rating_normalized, excluded_top_index)

    # Compute L2 distances and normalize
    if len(rating_reduced) == 0:
        return excluded_top_index, None, [], [], []
    distance_matrix = cdist(vectors_reduced, vectors_reduced, metric='euclidean')
    distance_matrix /= distance_matrix.max() if distance_matrix.max() > 1e-12 else 1

    # Compute weights
    mean_rating_reduced = np.mean(rating_reduced)
    w = np.exp(mean_rating_reduced - rating_reduced)

    # Local search setup
    def compute_objective(chosen_set):
        return sum(w[i] * min(distance_matrix[i, j] for j in chosen_set) for i in range(len(w)))

    rng = np.random.default_rng(random_seed)
    all_indices = np.arange(len(rating_reduced))
    current_set = set(rng.choice(all_indices, size=k, replace=False)) if k < len(rating_reduced) else set(all_indices)
    current_cost = compute_objective(current_set)

    # Local search loop
    improved = True
    while improved:
        improved = False
        best_swap = (None, None, 0)
        for j_out in list(current_set):
            for j_in in all_indices:
                if j_in not in current_set:
                    candidate_set = (current_set - {j_out}) | {j_in}
                    improvement = current_cost - compute_objective(candidate_set)
                    if improvement > best_swap[2]:
                        best_swap = (j_out, j_in, improvement)
        if best_swap[2] > 1e-12:
            current_set.remove(best_swap[0])
            current_set.add(best_swap[1])
            current_cost -= best_swap[2]
            improved = True

    chosen_indices_original = [new_to_old[j] for j in sorted(current_set)]
    rejected_indices_original = [new_to_old[j] for j in sorted(set(all_indices) - current_set)]
    return excluded_top_index, chosen_indices_original[0], rejected_indices_original[:k], chosen_indices_original, rejected_indices_original
\end{lstlisting}
\clearpage
\section{Visualization of t-SNE embeddings for Diverse Responses Across Queries}
\label{sec:tsne_visualization}

In this section, we showcase the performance of our method through plots of TSNE across various examples. These illustrative figures show how our baseline Bottom-k Algorithm (Section \ref{sec:ampo_bottomk}) chooses similar responses that are often close to each other. Hence the model misses out on feedback relating to other parts of the answer space that it often explores. Contrastingly, we often notice diversity of response selection for both the $\ampoos$ and $\ampocs$ algorithms.

% \begin{figure*}[!thbp]
%     \centering
%     \includegraphics[width=1.0\textwidth]{images/tsne_plot_with_scores_23687.pdf}
%     \caption{t-SNE visualization of projected high-dimensional response embeddings into a 2D space, illustrating the separation of actively selected responses. (a) AMPO-BottomK (baseline). (b) AMPO-Coreset (ours). (c) Opt-Select (ours). We see that the traditional baselines select many responses close to each other, based on their rating. This provides insufficient feedback to the LLM during preference optimization. In contrast, our methods simultaneously optimize for objectives including coverage, generation probability as well as preference rating.}
% \label{fig:tsne_diverse_analysis}
% \end{figure*}

% \begin{figure*}[!thbp]
%     \centering
%     \includegraphics[width=1.0\textwidth]{images/tsne_plot_with_scores_1362.pdf}
%     \caption{t-SNE visualization of projected high-dimensional response embeddings into a 2D space, illustrating the separation of actively selected responses. (a) AMPO-BottomK (baseline). (b) AMPO-Coreset (ours). (c) Opt-Select (ours). We see that the traditional baselines select many responses close to each other, based on their rating. This provides insufficient feedback to the LLM during preference optimization. In contrast, our methods simultaneously optimize for objectives including coverage, generation probability as well as preference rating.}
% \label{fig:tsne_diverse_analysis}
% \end{figure*}

% \begin{figure*}[!thbp]
%     \centering
%     \includegraphics[width=1.0\textwidth]{images/tsne_plot_with_scores_24192.pdf}

%     \caption{t-SNE visualization of projected high-dimensional response embeddings into a 2D space, illustrating the separation of actively selected responses. (a) AMPO-BottomK (baseline). (b) AMPO-Coreset (ours). (c) Opt-Select (ours). We see that the traditional baselines select many responses close to each other, based on their rating. This provides insufficient feedback to the LLM during preference optimization. In contrast, our methods simultaneously optimize for objectives including coverage, generation probability as well as preference rating.}
% \label{fig:tsne_diverse_analysis}
% \end{figure*}


% \begin{figure*}[!thbp]
%     \centering
%     \begin{subfigure}[b]{1.0\textwidth}
%         \centering
%         \includegraphics[width=0.8\textwidth]{images/tsne_plot_with_scores_23687.pdf}
%         \caption{Query-1.}
%         \label{fig:tsne_bottomk}
%     \end{subfigure}
    
%     \begin{subfigure}[b]{1.0\textwidth}
%         \centering
%         \includegraphics[width=0.8\textwidth]{images/tsne_plot_with_scores_1362.pdf}
%         \caption{Query1.}
%         \label{fig:tsne_coreset}
%     \end{subfigure}
    
%     \begin{subfigure}[b]{1.0\textwidth}
%         \centering
%         \includegraphics[width=0.8\textwidth]{images/tsne_plot_with_scores_24192.pdf}
%         \caption{Query-3.}
%         \label{fig:tsne_optselect}
%     \end{subfigure}
    
%     \caption{t-SNE visualization of projected high-dimensional response embeddings into a 2D space, illustrating the separation of actively selected responses. (a) AMPO-BottomK (baseline). (b) AMPO-Coreset (ours). (c) Opt-Select (ours). Traditional baselines select many responses close to each other based on their rating, providing insufficient feedback to the LLM during preference optimization. In contrast, our methods optimize for objectives including coverage, generation probability, and preference rating.}
%     \label{fig:tsne_combined}
% \end{figure*}


\begin{figure*}[!thbp]
    \centering
    \subfigure[1.]{
        \includegraphics[width=1.0\textwidth]{images/tsne_plot_with_scores_23687.pdf}
        \label{fig:tsne_bottomk}
    }
    
    \subfigure[2.]{
        \includegraphics[width=1.0\textwidth]{images/tsne_plot_with_scores_1362.pdf}
        \label{fig:tsne_coreset}
    }
    
    \subfigure[3.]{
        \includegraphics[width=1.0\textwidth]{images/tsne_plot_with_scores_24192.pdf}
        \label{fig:tsne_optselect}
    }
    
    \caption{t-SNE visualization of projected high-dimensional response embeddings into a 2D space, illustrating the separation of actively selected responses. (a) AMPO-BottomK (baseline). (b) AMPO-Coreset (ours). (c) Opt-Select (ours). Traditional baselines select many responses close to each other based on their rating, providing insufficient feedback to the LLM during preference optimization. In contrast, our methods optimize for objectives including coverage, generation probability, and preference rating.}
    \label{fig:tsne_combined}
\end{figure*}



% \subsection{Figures}

% You may want to include figures in the paper to illustrate
% your approach and results. Such artwork should be centered,
% legible, and separated from the text. Lines should be dark and at
% least 0.5~points thick for purposes of reproduction, and text should
% not appear on a gray background.

% Label all distinct components of each figure. If the figure takes the
% form of a graph, then give a name for each axis and include a legend
% that briefly describes each curve. Do not include a title inside the
% figure; instead, the caption should serve this function.

% Number figures sequentially, placing the figure number and caption
% \emph{after} the graphics, with at least 0.1~inches of space before
% the caption and 0.1~inches after it, as in
% \cref{icml-historical}. The figure caption should be set in
% 9~point type and centered unless it runs two or more lines, in which
% case it should be flush left. You may float figures to the top or
% bottom of a column, and you may set wide figures across both columns
% (use the environment \texttt{figure*} in \LaTeX). Always place
% two-column figures at the top or bottom of the page.

% \subsection{Algorithms}

% If you are using \LaTeX, please use the ``algorithm'' and ``algorithmic''
% environments to format pseudocode. These require
% the corresponding stylefiles, algorithm.sty and
% algorithmic.sty, which are supplied with this package.
% \cref{alg:example} shows an example.

% \begin{algorithm}[tb]
%    \caption{Bubble Sort}
%    \label{alg:example}
% \begin{algorithmic}
%    \STATE {\bfseries Input:} data $x_i$, size $m$
%    \REPEAT
%    \STATE Initialize $noChange = true$.
%    \FOR{$i=1$ {\bfseries to} $m-1$}
%    \IF{$x_i > x_{i+1}$}
%    \STATE Swap $x_i$ and $x_{i+1}$
%    \STATE $noChange = false$
%    \ENDIF
%    \ENDFOR
%    \UNTIL{$noChange$ is $true$}
% \end{algorithmic}
% \end{algorithm}

% \subsection{Tables}

% You may also want to include tables that summarize material. Like
% figures, these should be centered, legible, and numbered consecutively.
% However, place the title \emph{above} the table with at least
% 0.1~inches of space before the title and the same after it, as in
% \cref{sample-table}. The table title should be set in 9~point
% type and centered unless it runs two or more lines, in which case it
% should be flush left.

% Note use of \abovespace and \belowspace to get reasonable spacing
% above and below tabular lines.

% \begin{table}[t]
% \caption{Classification accuracies for naive Bayes and flexible
% Bayes on various data sets.}
% \label{sample-table}
% \vskip 0.15in
% \begin{center}
% \begin{small}
% \begin{sc}
% \begin{tabular}{lcccr}
% \toprule
% Data set & Naive & Flexible & Better? \\
% \midrule
% Breast    & 95.9$\pm$ 0.2& 96.7$\pm$ 0.2& $\surd$ \\
% Cleveland & 83.3$\pm$ 0.6& 80.0$\pm$ 0.6& $\times$\\
% Glass2    & 61.9$\pm$ 1.4& 83.8$\pm$ 0.7& $\surd$ \\
% Credit    & 74.8$\pm$ 0.5& 78.3$\pm$ 0.6&         \\
% Horse     & 73.3$\pm$ 0.9& 69.7$\pm$ 1.0& $\times$\\
% Meta      & 67.1$\pm$ 0.6& 76.5$\pm$ 0.5& $\surd$ \\
% Pima      & 75.1$\pm$ 0.6& 73.9$\pm$ 0.5&         \\
% Vehicle   & 44.9$\pm$ 0.6& 61.5$\pm$ 0.4& $\surd$ \\
% \bottomrule
% \end{tabular}
% \end{sc}
% \end{small}
% \end{center}
% \vskip -0.1in
% \end{table}

% Tables contain textual material, whereas figures contain graphical material.
% Specify the contents of each row and column in the table's topmost
% row. Again, you may float tables to a column's top or bottom, and set
% wide tables across both columns. Place two-column tables at the
% top or bottom of the page.

% \subsection{Theorems and such}
% The preferred way is to number definitions, propositions, lemmas, etc. consecutively, within sections, as shown below.
% \begin{definition}
% \label{def:inj}
% A function $f:X \to Y$ is injective if for any $x,y\in X$ different, $f(x)\ne f(y)$.
% \end{definition}
% Using \cref{def:inj} we immediate get the following result:
% \begin{proposition}
% If $f$ is injective mapping a set $X$ to another set $Y$, 
% the cardinality of $Y$ is at least as large as that of $X$
% \end{proposition}
% \begin{proof} 
% Left as an exercise to the reader. 
% \end{proof}
% \cref{lem:usefullemma} stated next will prove to be useful.
% \begin{lemma}
% \label{lem:usefullemma}
% For any $f:X \to Y$ and $g:Y\to Z$ injective functions, $f \circ g$ is injective.
% \end{lemma}
% \begin{theorem}
% \label{thm:bigtheorem}
% If $f:X\to Y$ is bijective, the cardinality of $X$ and $Y$ are the same.
% \end{theorem}
% An easy corollary of \cref{thm:bigtheorem} is the following:
% \begin{corollary}
% If $f:X\to Y$ is bijective, 
% the cardinality of $X$ is at least as large as that of $Y$.
% \end{corollary}
% \begin{assumption}
% The set $X$ is finite.
% \label{ass:xfinite}
% \end{assumption}
% \begin{remark}
% According to some, it is only the finite case (cf. \cref{ass:xfinite}) that is interesting.
% \end{remark}
%restatable





\end{document}


% This document was modified from the file originally made available by
% Pat Langley and Andrea Danyluk for ICML-2K. This version was created
% by Iain Murray in 2018, and modified by Alexandre Bouchard in
% 2019 and 2021 and by Csaba Szepesvari, Gang Niu and Sivan Sabato in 2022.
% Modified again in 2023 and 2024 by Sivan Sabato and Jonathan Scarlett.
% Previous contributors include Dan Roy, Lise Getoor and Tobias
% Scheffer, which was slightly modified from the 2010 version by
% Thorsten Joachims & Johannes Fuernkranz, slightly modified from the
% 2009 version by Kiri Wagstaff and Sam Roweis's 2008 version, which is
% slightly modified from Prasad Tadepalli's 2007 version which is a
% lightly changed version of the previous year's version by Andrew
% Moore, which was in turn edited from those of Kristian Kersting and
% Codrina Lauth. Alex Smola contributed to the algorithmic style files.
