% SIAM Shared Information Template
% This is information that is shared between the main document and any
% supplement. If no supplement is required, then this information can
% be included directly in the main document.


% Packages and macros go here
\usepackage{lipsum}
\usepackage{amsfonts}
\usepackage{graphicx}
\usepackage{epstopdf}
\usepackage{algorithmic}
\usepackage{todonotes}
\ifpdf
  \DeclareGraphicsExtensions{.eps,.pdf,.png,.jpg}
\else
  \DeclareGraphicsExtensions{.eps}
\fi

% Add a serial/Oxford comma by default.
\newcommand{\creflastconjunction}{, and~}

% Used for creating new theorem and remark environments
\newsiamremark{remark}{Remark}
\newsiamremark{hypothesis}{Hypothesis}
\crefname{hypothesis}{Hypothesis}{Hypotheses}
\newsiamthm{claim}{Claim}

% Sets running headers as well as PDF title and authors
\headers{DD for Maxwell in a waveguide}{V. Dolean, A. Tonnoir. P.-H. Tournier}

% Title. If the supplement option is on, then "Supplementary Material"
% is automatically inserted before the title.
\title{Modal analysis of a domain decomposition method for Maxwell's equations in a waveguide\thanks{Submitted to the editors DATE.
%\funding{This work was funded by the Fog Research Institute under contract no.~FRI-454.}
}}

% Authors: full names plus addresses.
\author{Victorita Dolean\thanks{TU Eindhoven, The Netherlands 
  (\email{v.dolean.maini@tue.nl}).}
\and Antoine Tonnoir\thanks{Normandie Univ., INSA de Rouen, LMI (EA 3226 - FR CNRS 3335), France 
  (\email{atonnoir@insa-rouen.fr}).}
\and Pierre-Henri Tournier\thanks{Sorbonne Université, Paris, France 
  (\email{pierre-henri.tournier@sorbonne-universite.fr})}}

\usepackage{amsopn}
\DeclareMathOperator{\diag}{diag}

\newcommand{\bfE}{{\mathbf{E}}}
\newcommand{\bfH}{{\mathbf{H}}}

\newcommand{\bfe}{{\mathbf{e}}}
\newcommand{\bfh}{{\mathbf{h}}}

% Track changes
%\newcommand{\comment}[1]{{\color{myorange}[\Lightning]\emph{#1}}}
\newcommand{\comment}[1]{{\color{orange}{\tiny \textdbend}\emph{#1}}}
\newcommand{\deleted}[1]{{\color{red}\sout{#1}}}
\newcommand{\inserted}[1]{{\color{blue}#1}}
\newcommand{\modify}[2]{{\color{red}\sout{#1}$\mapsto$}{\color{blue}#2}}
\newcommand{\accepted}[2]{{\color{red}\sout{}}{\color{green}#2}}
\newcommand{\declined}[2]{{\color{red}#1}{\color{blue}}}
\newcommand{\R}{\mathbb{R}}
\newcommand{\Z}{\mathbb{Z}}
\newcommand{\N}{\mathbb{N}}
\newcommand{\C}{{\mathbb{C}}}
\newcommand{\xs}{\mathbf x_s}
\newcommand{\bfu}{\mathbf u}
\newcommand{\bfU}{\mathbf U}
\newcommand{\bfF}{\mathbf F}
\newcommand{\bfv}{\mathbf v}
\newcommand{\bfx}{\mathbf x}
\newcommand{\bfX}{\mathbf X}
\newcommand{\bfY}{\mathbf Y}
\newcommand{\bfw}{\mathbf w}
\newcommand{\w}{\omega}
\newcommand{\bfus}{\mathbf u_s}
\newcommand{\bfts}{\mathbf t_s}
\newcommand{\bft}{\mathbf t}
\newcommand{\eps}{\varepsilon}
\newcommand{\Div}{\mbox{\rm{div}\,}}
\newcommand{\rot}{\mbox{\rm{rot}\,}}
\newcommand{\abs}[1]{\lvert#1\rvert}
\newcommand{\norm}[1]{\lVert#1\rVert}
\newcommand{\dsp}{\displaystyle}
\newcommand{\ri}{{\mathrm{i}}}
\newcommand{\mbfxi}{{\boldsymbol{\xi}}}

%%% Local Variables: 
%%% mode:latex
%%% TeX-master: "ex_article"
%%% End: 
