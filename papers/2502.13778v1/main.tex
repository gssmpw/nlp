\documentclass[conference]{IEEEtran}

% 必要的宏包
\usepackage{graphicx}    % 图像
\usepackage{float}       % 控制浮动体
\usepackage{url}         % 用于处理 URL
\usepackage{caption}     % 标题
\usepackage{subcaption}  % 子图标题
\usepackage{cite}        % 引用
\usepackage{marvosym}
\bibliographystyle{plain}

% 图像路径设置
\graphicspath{{./figures/}}
\DeclareGraphicsExtensions{.pdf,.jpg,.png}


\title{Poster: SpiderSim: Multi-Agent Driven Theoretical Cybersecurity Simulation for Industrial Digitalization}

% 手动设置作者信息
\author{
    \IEEEauthorblockN{
        Jiaqi Li\IEEEauthorrefmark{1},
        Xizhong Guo\IEEEauthorrefmark{2},
        Yang Zhao\IEEEauthorrefmark{3},
        Lvyang Zhang\IEEEauthorrefmark{4},
        Lidong Zhai\IEEEauthorrefmark{5}\textsuperscript{\Letter}
    }
    \IEEEauthorblockA{
        \IEEEauthorrefmark{1}\IEEEauthorrefmark{2}\IEEEauthorrefmark{3}\IEEEauthorrefmark{4}\IEEEauthorrefmark{5}Institute of Information Engineering, Chinese Academy of Sciences, Beijing, China \\
        \IEEEauthorrefmark{1}\IEEEauthorrefmark{2}\IEEEauthorrefmark{3}\IEEEauthorrefmark{4}School of Cyber Security, University of Chinese Academy of Sciences, Beijing, China \\
    }
}

\begin{document}

% 手动设置标题
\maketitle

\begin{abstract}
%\boldmath
Rapid industrial digitalization has created intricate cybersecurity demands that necessitate effective validation methods. While cyber ranges and simulation platforms are widely deployed, they frequently face limitations in scenario diversity and creation efficiency. In this paper, we present SpiderSim, a theoretical cybersecurity simulation platform enabling rapid and lightweight scenario generation for industrial digitalization security research. At its core, our platform introduces three key innovations: a structured framework for unified scenario modeling, a multi-agent collaboration mechanism for automated generation, and modular atomic security capabilities for flexible scenario composition. Extensive implementation trials across multiple industrial digitalization contexts, including marine ranch monitoring systems, validate our platform's capacity for broad scenario coverage with efficient generation processes. Built on solid theoretical foundations and released as open-source software, SpiderSim facilitates broader research and development in automated security testing for industrial digitalization.

keywords: Cybersecurity simulation; atomic capabilities; industrial digitalization; multi-agent; theoretical simulation
\end{abstract}

%\boldmath
% IEEEtran.cls defaults to using nonbold math in the Abstract.
% This preserves the distinction between vectors and scalars. However,
% if the conference you are submitting to favors bold math in the abstract,
% then you can use LaTeX's standard command \boldmath at the very start
% of the abstract to achieve this. Many IEEE journals/conferences frown on
% math in the abstract anyway.

% no keywords




% For peer review papers, you can put extra information on the cover
% page as needed:
% \ifCLASSOPTIONpeerreview
% \begin{center} \bfseries EDICS Category: 3-BBND \end{center}
% \fi
%
% For peerreview papers, this IEEEtran command inserts a page break and
% creates the second title. It will be ignored for other modes.
%%\IEEEpeerreviewmaketitle



\section{Introduction}
% no \IEEEPARstart
The rapid advancement of industrial digitalization has introduced unprecedented cybersecurity challenges across manufacturing, energy, transportation, and other sectors. Attack-defense simulations have proven effective for training and validating security capabilities. Frameworks like MITRE ATT\&CK\cite{ATTCK} and Engage\cite{engage} facilitate defense improvements. 
While traditional cyber ranges and security testbeds provide valuable validation environments, they often face limitations in scenario coverage and generation efficiency. First, traditional cyber ranges require substantial resources for environment setup and maintenance. Second, existing platforms often lack systematic methodologies for rapid scenario generation. Third, the manual scenario creation process limits comprehensive coverage of security situations in industrial digitalization contexts. As an open-source network simulation platform, CyberBattleSim\cite{cbs} utilizes abstract modeling and reinforcement learning to advance the field. However, it focuses narrowly on internal networks rather than capturing real-world complexities. The platform also implements limited techniques and lacks integration with security tools.
To address these challenges, we present SpiderSim, a theoretical cybersecurity simulation platform that introduces three key innovations. First, it implements a structured framework for unified scenario modeling, which enables standardized scenario construction through formal methodologies. Second, it incorporates a multi-agent collaboration mechanism for automated scenario generation and validation. Third, it provides modular atomic security capabilities that support flexible composition of industrial security scenarios. This integrated approach achieves both scenario diversity and generation efficiency without compromising theoretical precision.

The remainder of this paper is organized as follows: Section II presents the platform architecture and core components of SpiderSim. Section III details the implementation and application through industrial digitalization cases. Section IV concludes the paper and discusses future research directions.


%SpaceX Starlink, Amazon Kuiper and other companies are deploying hundreds and thousands of satellites\cite{giuliari2021icarus}.


%\hfill mds
 
%\hfill January 11, 2007




% An example of a floating figure using the graphicx package.
% Note that \label must occur AFTER (or within) \caption.
% For figures, \caption should occur after the \includegraphics.
% Note that IEEEtran v1.7 and later has special internal code that
% is designed to preserve the operation of \label within \caption
% even when the captionsoff option is in effect. However, because
% of issues like this, it may be the safest practice to put all your
% \label just after \caption rather than within \caption{}.
%
% Reminder: the "draftcls" or "draftclsnofoot", not "draft", class
% option should be used if it is desired that the figures are to be
% displayed while in draft mode.
%
%\begin{figure}[!t]
%\centering
%\includegraphics[width=2.5in]{myfigure}
% where an .eps filename suffix will be assumed under latex, 
% and a .pdf suffix will be assumed for pdflatex; or what has been declared
% via \DeclareGraphicsExtensions.
%\caption{Simulation Results}
%\label{fig_sim}
%\end{figure}

% Note that IEEE typically puts floats only at the top, even when this
% results in a large percentage of a column being occupied by floats.


% An example of a double column floating figure using two subfigures.
% (The subfig.sty package must be loaded for this to work.)
% The subfigure \label commands are set within each subfloat command, the
% \label for the overall figure must come after \caption.
% \hfil must be used as a separator to get equal spacing.
% The subfigure.sty package works much the same way, except \subfigure is
% used instead of \subfloat.
%
%\begin{figure*}[!t]
%\centerline{\subfloat[Case I]\includegraphics[width=2.5in]{subfigcase1}%
%\label{fig_first_case}}
%\hfil
%\subfloat[Case II]{\includegraphics[width=2.5in]{subfigcase2}%
%\label{fig_second_case}}}
%\caption{Simulation results}
%\label{fig_sim}
%\end{figure*}
%
% Note that often IEEE papers with subfigures do not employ subfigure
% captions (using the optional argument to \subfloat), but instead will
% reference/describe all of them (a), (b), etc., within the main caption.


% An example of a floating table. Note that, for IEEE style tables, the 
% \caption command should come BEFORE the table. Table text will default to
% \footnotesize as IEEE normally uses this smaller font for tables.
% The \label must come after \caption as always.
%
%\begin{table}[!t]
%% increase table row spacing, adjust to taste
%\renewcommand{\arraystretch}{1.3}
% if using array.sty, it might be a good idea to tweak the value of
% \extrarowheight as needed to properly center the text within the cells
%\caption{An Example of a Table}
%\label{table_example}
%\centering
%% Some packages, such as MDW tools, offer better commands for making tables
%% than the plain LaTeX2e tabular which is used here.
%\begin{tabular}{|c||c|}
%\hline
%One & Two\\
%\hline
%Three & Four\\
%\hline
%\end{tabular}
%\end{table}


% Note that IEEE does not put floats in the very first column - or typically
% anywhere on the first page for that matter. Also, in-text middle ("here")
% positioning is not used. Most IEEE journals/conferences use top floats
% exclusively. Note that, LaTeX2e, unlike IEEE journals/conferences, places
% footnotes above bottom floats. This can be corrected via the \fnbelowfloat
% command of the stfloats package.

%\textbf{Spatio-temporal transcendence}: The Spatio-temporal transcendence lies in their ability to solve deep-sea problems, industry-driven attributes, and technology evolution characteristics(Figure 2). 

%\textbf{Continuous evolution}: It has to meet the needs of the present generation and develop continuously according to the socio-economic evolution of the times and the changing needs of human beings. 

%\begin{figure}[h]
   % \centering
    %\includegraphics[scale=0.5]{hyper/figure/图片2.png}
    %\caption{Characteristics of Hyper Large Scientific Infrastructure}
    %\label{fig:hyper}
%\end{figure}
\begin{figure}[h]
    \centering
    \includegraphics[scale=0.25]{./figure/figure3.png}
    \caption{Automated scenario generation and experimental framework}
    \label{fig:tuopu}
\end{figure}

\section{PLATFORM ARCHITECTURE AND COMPONENTS }
SpiderSim implements a three-layered architecture that systematically transforms abstract security requirements into executable attack-defense scenarios. The platform integrates a unified scenario modeling framework for structured requirement specification, a multi-agent collaboration mechanism for automated scenario development, and atomic security capabilities for comprehensive security validation. This hierarchical design enables efficient scenario generation while preserving theoretical rigor and practical applicability.
The layered architecture directly addresses the fundamental limitations of existing cybersecurity simulation platforms, including complex environment configuration requirements, time-consuming manual scenario creation processes, and insufficient coverage of industrial security contexts. By emphasizing lightweight deployment, automated generation workflows, and modular security components, SpiderSim provides a flexible and efficient solution for industrial security research and validation through three core components. 


\subsection{Unified scenario modeling framework}
The foundational component of SpiderSim rests upon a systematic methodology for scenario construction. At its core, the framework encompasses comprehensive domain context analysis, structured problem decomposition, detailed scenario specifications, clear objective definitions, and essential element composition. Built on a formalized approach, it ensures consistent scenario quality while enabling rapid generation across diverse industrial digitalization contexts. Such methodological structure supports both standardization and customization, allowing for efficient scenario creation while maintaining contextual relevance.

\subsection{Multi-agent collaboration mechanism}
An automated generation engine drives scenario development through coordinated agent interactions.  Agents operate through synchronized communication channels, coordinating scenario development activities from initial requirement analysis through final validation. This mechanism incorporates continuous refinement processes, enabling dynamic adjustments based on validation results and emerging requirements. The collaborative approach ensures scenario completeness while maintaining generation efficiency through automated coordination and validation procedures.

\subsection{Atomic security capabilities}
SpiderSim supports the construction of complex network attack and defense capabilities using atomized modules. These modules include the Shocktrap module\cite{hong}, honeypot module, vulnerability scanning module, among others. Users can select the necessary modules to build a customized security defense system or test the effectiveness of individual modules. Meanwhile, SpiderSim incorporates Cybersecurity Chess Manual \cite{wuposter}, to help generate more atomized attack and defense scenarios.
For example, users can choose a data encryption module, honeypot module, and Shocktrap module to form a security strategy to counteract phishing attacks. This atomized and reconfigurable module design enhances the flexibility of SpiderSim in constructing diverse attack and defense scenarios. Standardized interfaces between modules also allow third parties to develop and submit new attack and defense modules to expand the platform's capabilities.


%\section{Technical Implementation of SpiderSim}
%\subsection{Architectural Design}
%SpiderSim adopts a layered system architecture, consisting primarily of the scenario modeling layer, simulation runtime layer, and visualization and data analysis layer. Information interaction between layers is through standardized interfaces, as shown in Figure 1.

%The scenario modeling layer implements user-friendly network topology and component modeling, attack surface settings, and simulation configuration. The simulation runtime layer is responsible for scenario generation, simulation of attack-defense confrontations, and data recording. The visualization and analysis layer presents the simulation process and statistical results, generating analysis reports. Each layer is designed as an extensible framework for adding new modules.

%\subsection{API and Plugin Support}
%SpiderSim exposes APIs and a plugin system to support extensions. The API allows integration with third-party software to achieve automated workflows. The plugin system enables users to expand SpiderSim's attack and defense capabilities.

%For example, the API can be used to: automatically generate simulation scenarios by ingesting topology information from live systems; output simulation reports to import into firewall configurations; invoke IDS rulesets for conversion into SpiderSim detection modules. Plugins can be used to implement capabilities such as: integrate functional modules from open-source honeypot tools; build encoder/decoder modules for industrial protocols; incorporate machine learning algorithms into new intelligent decision-making modules.

%\subsection{Performance Optimization and Resource Management}
%SpiderSim leverages multi-core processors for parallel computing to handle complex simulation computations efficiently. Additionally, SpiderSim can dynamically allocate software and hardware resources through virtualization techniques to achieve high-density scenario simulation.
%We have tested large-scale simulation scenarios with thousands of nodes and complex multi-party interactions. SpiderSim is able to maintain smooth performance throughout. The figure below shows SpiderSim's superior simulation throughput efficiency under scenarios of different scales.


\section{PRACTICAL IMPLEMENTATION AND CASE STUDIE}
Utilizing the aforementioned framework, we constructed a digital environment for a marine ranch.  Figure 2(a) illustrates the framework for the marine ranch monitoring system, which encompasses devices such as sensors, control networks, video surveillance systems, and remote maintenance systems. Figure 2(b) displays the theoretical-level network topology and attack path diagram constructed within SpiderSim. Within this environment, we conducted a series of cyber attack-defense experiments, which led to the development of a security protection scheme tailored to the typical threats faced by the system. This scheme has been tested and proven to effectively mitigate the risk of cyber attacks on the marine ranch infrastructure.

\begin{figure}[h]
    \begin{subfigure}[b]{0.23\textwidth}
        \centering
        \includegraphics[scale=0.2]{./figure/figure2.png}
        \caption{The framework for the marine ranch monitoring system}
        \label{fig:tuopu}
    \end{subfigure}
    %\hfill
    \hspace{0.5cm}
    \begin{subfigure}[b]{0.23\textwidth}
        \centering
        \includegraphics[scale=0.3]{./figure/figure1.png}
        \caption{The network topology and attack path diagram}
        \label{fig:frameword}
    \end{subfigure}
    \caption{Digital environment and cyber attack-defense experiments of marine ranch}
    \label{fig:overall}
\end{figure}


\section{Conclusion}
SpiderSim represents an innovative approach to theoretical security simulation in industrial digitalization contexts. The platform successfully implements rapid scenario generation through systematic scene description and multi-agent collaboration, as demonstrated through practical testing. The lightweight architecture demonstrates particular advantages in complex industrial contexts where rapid scenario development is critical. 

The open-source implementation (available at \url{https://github.com/NRT2024/SpiderSim}) provides a foundation for collaborative development of next-generation security testing solutions for industrial digitalization.



%\section{Conclusion}


% conference papers do not normally have an appendix


% use section* for acknowledgement
%\section*{Acknowledgment}


%The authors would like to thank...





% trigger a \newpage just before the given reference
% number - used to balance the columns on the last page
% adjust value as needed - may need to be readjusted if
% the document is modified later
%\IEEEtriggeratref{8}
% The "triggered" command can be changed if desired:
%\IEEEtriggercmd{\enlargethispage{-5in}}

% references section

% can use a bibliography generated by BibTeX as a .bbl file
% BibTeX documentation can be easily obtained at:
% http://www.ctan.org/tex-archive/biblio/bibtex/contrib/doc/
% The IEEEtran BibTeX style support page is at:
% http://www.michaelshell.org/tex/ieeetran/bibtex/
%\bibliographystyle{IEEEtranS}
% argument is your BibTeX string definitions and bibliography database(s)
%\bibliography{IEEEabrv,../bib/paper}
%
% <OR> manually copy in the resultant .bbl file
% set second argument of \begin to the number of references
% (used to reserve space for the reference number labels box)
%\begin{thebibliography}{1}

%\bibitem{IEEEhowto:kopka}


%\end{thebibliography}

 
\bibliographystyle{plain}
\bibliography{mypaper}

\newpage
\thispagestyle{empty}
\begin{figure}[h]
  \centering
  \includegraphics[width=\textwidth]{./figure/spidersim.pdf}
\end{figure}



\end{document}


