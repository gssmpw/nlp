 % This must be in the first 5 lines to tell arXiv to use pdfLaTeX, which is strongly recommended.
\pdfoutput=1
% In particular, the hyperref package requires pdfLaTeX in order to break URLs across lines.

\documentclass[11pt]{article}

\newcommand\blfootnote[1]{%
  \begingroup
  \renewcommand\thefootnote{}\footnote{#1}%
  \addtocounter{footnote}{-1}%
  \endgroup
}

% Remove the "review" option to generate the final version.
\usepackage[]{acl}

% Standard package includes
\usepackage{times}
\usepackage{latexsym}
\usepackage{comment}
\usepackage{color}
\usepackage{listings}

\lstset{
   basicstyle=\ttfamily\small,  % Typewriter font, smaller size
   backgroundcolor=\color{gray!10}, % Light gray background
   frame=single, % Single line border around the box
   breaklines=true, % Enable line breaking within the box
   %postbreak=\mbox{\textcolor{red}{$\hookrightarrow$}\space}, % Show an arrow for wrapped lines
   tabsize=2, % Tab size
   language=, % No specific language
   keepspaces=true, % Keep spaces in text
   prebreak=\mbox{}, % Do not indent when a line is broken
   postbreak=\mbox{}
}


%\usepackage{xcolor}
\usepackage{colortbl}
% For proper rendering and hyphenation of words containing Latin characters (including in bib files)
\usepackage[T1]{fontenc}
% For Vietnamese characters
% \usepackage[T5]{fontenc}
% See https://www.latex-project.org/help/documentation/encguide.pdf for other character sets

% This assumes your files are encoded as UTF8
\usepackage[utf8]{inputenc}

% This is not strictly necessary, and may be commented out,
% but it will improve the layout of the manuscript,
% and will typically save some space.
\usepackage{microtype}
\usepackage[most]{tcolorbox}
% This is also not strictly necessary, and may be commented out.
% However, it will improve the aesthetics of text in
% the typewriter font.
\usepackage{inconsolata}

\usepackage{mathtools}
\usepackage{makecell}



\usepackage{array,multirow,graphicx}
\usepackage{amsmath}
\usepackage{amssymb}
\usepackage{booktabs}
%\usepackage{algorithm}
\usepackage{algpseudocode}

\usepackage[export]{adjustbox}
\usepackage{float}

% Support for easy cross-referencing
\usepackage[capitalize]{cleveref}
\crefname{section}{Sec.}{Secs.}
%\Crefname{section}{Section}{Secs.}
\Crefname{table}{Table}{Tables}
% \crefname{table}{Tab.}{Tabs.}
%\newcommand\algorithmautorefname{Algorithm}
\def\subsectionautorefname{Sec.}
\def\sectionautorefname{Sec.}

%\renewcommand*{\algorithmautorefname}{Algorithm}
\renewcommand*{\figureautorefname}{Fig.}
\renewcommand*{\equationautorefname}{Eq.}
\renewcommand*{\tableautorefname}{Tab.}


% \renewcommand\jt[1]{\textcolor{black}{#1}}

%%%%% NEW MATH DEFINITIONS %%%%%

\usepackage{amsmath,amsfonts,bm}

% Mark sections of captions for referring to divisions of figures
\newcommand{\figleft}{{\em (Left)}}
\newcommand{\figcenter}{{\em (Center)}}
\newcommand{\figright}{{\em (Right)}}
\newcommand{\figtop}{{\em (Top)}}
\newcommand{\figbottom}{{\em (Bottom)}}
\newcommand{\captiona}{{\em (a)}}
\newcommand{\captionb}{{\em (b)}}
\newcommand{\captionc}{{\em (c)}}
\newcommand{\captiond}{{\em (d)}}

% Highlight a newly defined term
\newcommand{\newterm}[1]{{\bf #1}}


% Figure reference, lower-case.
\def\figref#1{figure~\ref{#1}}
% Figure reference, capital. For start of sentence
\def\Figref#1{Figure~\ref{#1}}
\def\twofigref#1#2{figures \ref{#1} and \ref{#2}}
\def\quadfigref#1#2#3#4{figures \ref{#1}, \ref{#2}, \ref{#3} and \ref{#4}}
% Section reference, lower-case.
\def\secref#1{section~\ref{#1}}
% Section reference, capital.
\def\Secref#1{Section~\ref{#1}}
% Reference to two sections.
\def\twosecrefs#1#2{sections \ref{#1} and \ref{#2}}
% Reference to three sections.
\def\secrefs#1#2#3{sections \ref{#1}, \ref{#2} and \ref{#3}}
% Reference to an equation, lower-case.
\def\eqref#1{equation~\ref{#1}}
% Reference to an equation, upper case
\def\Eqref#1{Equation~\ref{#1}}
% A raw reference to an equation---avoid using if possible
\def\plaineqref#1{\ref{#1}}
% Reference to a chapter, lower-case.
\def\chapref#1{chapter~\ref{#1}}
% Reference to an equation, upper case.
\def\Chapref#1{Chapter~\ref{#1}}
% Reference to a range of chapters
\def\rangechapref#1#2{chapters\ref{#1}--\ref{#2}}
% Reference to an algorithm, lower-case.
\def\algref#1{algorithm~\ref{#1}}
% Reference to an algorithm, upper case.
\def\Algref#1{Algorithm~\ref{#1}}
\def\twoalgref#1#2{algorithms \ref{#1} and \ref{#2}}
\def\Twoalgref#1#2{Algorithms \ref{#1} and \ref{#2}}
% Reference to a part, lower case
\def\partref#1{part~\ref{#1}}
% Reference to a part, upper case
\def\Partref#1{Part~\ref{#1}}
\def\twopartref#1#2{parts \ref{#1} and \ref{#2}}

\def\ceil#1{\lceil #1 \rceil}
\def\floor#1{\lfloor #1 \rfloor}
\def\1{\bm{1}}
\newcommand{\train}{\mathcal{D}}
\newcommand{\valid}{\mathcal{D_{\mathrm{valid}}}}
\newcommand{\test}{\mathcal{D_{\mathrm{test}}}}

\def\eps{{\epsilon}}


% Random variables
\def\reta{{\textnormal{$\eta$}}}
\def\ra{{\textnormal{a}}}
\def\rb{{\textnormal{b}}}
\def\rc{{\textnormal{c}}}
\def\rd{{\textnormal{d}}}
\def\re{{\textnormal{e}}}
\def\rf{{\textnormal{f}}}
\def\rg{{\textnormal{g}}}
\def\rh{{\textnormal{h}}}
\def\ri{{\textnormal{i}}}
\def\rj{{\textnormal{j}}}
\def\rk{{\textnormal{k}}}
\def\rl{{\textnormal{l}}}
% rm is already a command, just don't name any random variables m
\def\rn{{\textnormal{n}}}
\def\ro{{\textnormal{o}}}
\def\rp{{\textnormal{p}}}
\def\rq{{\textnormal{q}}}
\def\rr{{\textnormal{r}}}
\def\rs{{\textnormal{s}}}
\def\rt{{\textnormal{t}}}
\def\ru{{\textnormal{u}}}
\def\rv{{\textnormal{v}}}
\def\rw{{\textnormal{w}}}
\def\rx{{\textnormal{x}}}
\def\ry{{\textnormal{y}}}
\def\rz{{\textnormal{z}}}

% Random vectors
\def\rvepsilon{{\mathbf{\epsilon}}}
\def\rvtheta{{\mathbf{\theta}}}
\def\rva{{\mathbf{a}}}
\def\rvb{{\mathbf{b}}}
\def\rvc{{\mathbf{c}}}
\def\rvd{{\mathbf{d}}}
\def\rve{{\mathbf{e}}}
\def\rvf{{\mathbf{f}}}
\def\rvg{{\mathbf{g}}}
\def\rvh{{\mathbf{h}}}
\def\rvu{{\mathbf{i}}}
\def\rvj{{\mathbf{j}}}
\def\rvk{{\mathbf{k}}}
\def\rvl{{\mathbf{l}}}
\def\rvm{{\mathbf{m}}}
\def\rvn{{\mathbf{n}}}
\def\rvo{{\mathbf{o}}}
\def\rvp{{\mathbf{p}}}
\def\rvq{{\mathbf{q}}}
\def\rvr{{\mathbf{r}}}
\def\rvs{{\mathbf{s}}}
\def\rvt{{\mathbf{t}}}
\def\rvu{{\mathbf{u}}}
\def\rvv{{\mathbf{v}}}
\def\rvw{{\mathbf{w}}}
\def\rvx{{\mathbf{x}}}
\def\rvy{{\mathbf{y}}}
\def\rvz{{\mathbf{z}}}

% Elements of random vectors
\def\erva{{\textnormal{a}}}
\def\ervb{{\textnormal{b}}}
\def\ervc{{\textnormal{c}}}
\def\ervd{{\textnormal{d}}}
\def\erve{{\textnormal{e}}}
\def\ervf{{\textnormal{f}}}
\def\ervg{{\textnormal{g}}}
\def\ervh{{\textnormal{h}}}
\def\ervi{{\textnormal{i}}}
\def\ervj{{\textnormal{j}}}
\def\ervk{{\textnormal{k}}}
\def\ervl{{\textnormal{l}}}
\def\ervm{{\textnormal{m}}}
\def\ervn{{\textnormal{n}}}
\def\ervo{{\textnormal{o}}}
\def\ervp{{\textnormal{p}}}
\def\ervq{{\textnormal{q}}}
\def\ervr{{\textnormal{r}}}
\def\ervs{{\textnormal{s}}}
\def\ervt{{\textnormal{t}}}
\def\ervu{{\textnormal{u}}}
\def\ervv{{\textnormal{v}}}
\def\ervw{{\textnormal{w}}}
\def\ervx{{\textnormal{x}}}
\def\ervy{{\textnormal{y}}}
\def\ervz{{\textnormal{z}}}

% Random matrices
\def\rmA{{\mathbf{A}}}
\def\rmB{{\mathbf{B}}}
\def\rmC{{\mathbf{C}}}
\def\rmD{{\mathbf{D}}}
\def\rmE{{\mathbf{E}}}
\def\rmF{{\mathbf{F}}}
\def\rmG{{\mathbf{G}}}
\def\rmH{{\mathbf{H}}}
\def\rmI{{\mathbf{I}}}
\def\rmJ{{\mathbf{J}}}
\def\rmK{{\mathbf{K}}}
\def\rmL{{\mathbf{L}}}
\def\rmM{{\mathbf{M}}}
\def\rmN{{\mathbf{N}}}
\def\rmO{{\mathbf{O}}}
\def\rmP{{\mathbf{P}}}
\def\rmQ{{\mathbf{Q}}}
\def\rmR{{\mathbf{R}}}
\def\rmS{{\mathbf{S}}}
\def\rmT{{\mathbf{T}}}
\def\rmU{{\mathbf{U}}}
\def\rmV{{\mathbf{V}}}
\def\rmW{{\mathbf{W}}}
\def\rmX{{\mathbf{X}}}
\def\rmY{{\mathbf{Y}}}
\def\rmZ{{\mathbf{Z}}}

% Elements of random matrices
\def\ermA{{\textnormal{A}}}
\def\ermB{{\textnormal{B}}}
\def\ermC{{\textnormal{C}}}
\def\ermD{{\textnormal{D}}}
\def\ermE{{\textnormal{E}}}
\def\ermF{{\textnormal{F}}}
\def\ermG{{\textnormal{G}}}
\def\ermH{{\textnormal{H}}}
\def\ermI{{\textnormal{I}}}
\def\ermJ{{\textnormal{J}}}
\def\ermK{{\textnormal{K}}}
\def\ermL{{\textnormal{L}}}
\def\ermM{{\textnormal{M}}}
\def\ermN{{\textnormal{N}}}
\def\ermO{{\textnormal{O}}}
\def\ermP{{\textnormal{P}}}
\def\ermQ{{\textnormal{Q}}}
\def\ermR{{\textnormal{R}}}
\def\ermS{{\textnormal{S}}}
\def\ermT{{\textnormal{T}}}
\def\ermU{{\textnormal{U}}}
\def\ermV{{\textnormal{V}}}
\def\ermW{{\textnormal{W}}}
\def\ermX{{\textnormal{X}}}
\def\ermY{{\textnormal{Y}}}
\def\ermZ{{\textnormal{Z}}}

% Vectors
\def\vzero{{\bm{0}}}
\def\vone{{\bm{1}}}
\def\vmu{{\bm{\mu}}}
\def\vtheta{{\bm{\theta}}}
\def\va{{\bm{a}}}
\def\vb{{\bm{b}}}
\def\vc{{\bm{c}}}
\def\vd{{\bm{d}}}
\def\ve{{\bm{e}}}
\def\vf{{\bm{f}}}
\def\vg{{\bm{g}}}
\def\vh{{\bm{h}}}
\def\vi{{\bm{i}}}
\def\vj{{\bm{j}}}
\def\vk{{\bm{k}}}
\def\vl{{\bm{l}}}
\def\vm{{\bm{m}}}
\def\vn{{\bm{n}}}
\def\vo{{\bm{o}}}
\def\vp{{\bm{p}}}
\def\vq{{\bm{q}}}
\def\vr{{\bm{r}}}
\def\vs{{\bm{s}}}
\def\vt{{\bm{t}}}
\def\vu{{\bm{u}}}
\def\vv{{\bm{v}}}
\def\vw{{\bm{w}}}
\def\vx{{\bm{x}}}
\def\vy{{\bm{y}}}
\def\vz{{\bm{z}}}

% Elements of vectors
\def\evalpha{{\alpha}}
\def\evbeta{{\beta}}
\def\evepsilon{{\epsilon}}
\def\evlambda{{\lambda}}
\def\evomega{{\omega}}
\def\evmu{{\mu}}
\def\evpsi{{\psi}}
\def\evsigma{{\sigma}}
\def\evtheta{{\theta}}
\def\eva{{a}}
\def\evb{{b}}
\def\evc{{c}}
\def\evd{{d}}
\def\eve{{e}}
\def\evf{{f}}
\def\evg{{g}}
\def\evh{{h}}
\def\evi{{i}}
\def\evj{{j}}
\def\evk{{k}}
\def\evl{{l}}
\def\evm{{m}}
\def\evn{{n}}
\def\evo{{o}}
\def\evp{{p}}
\def\evq{{q}}
\def\evr{{r}}
\def\evs{{s}}
\def\evt{{t}}
\def\evu{{u}}
\def\evv{{v}}
\def\evw{{w}}
\def\evx{{x}}
\def\evy{{y}}
\def\evz{{z}}

% Matrix
\def\mA{{\bm{A}}}
\def\mB{{\bm{B}}}
\def\mC{{\bm{C}}}
\def\mD{{\bm{D}}}
\def\mE{{\bm{E}}}
\def\mF{{\bm{F}}}
\def\mG{{\bm{G}}}
\def\mH{{\bm{H}}}
\def\mI{{\bm{I}}}
\def\mJ{{\bm{J}}}
\def\mK{{\bm{K}}}
\def\mL{{\bm{L}}}
\def\mM{{\bm{M}}}
\def\mN{{\bm{N}}}
\def\mO{{\bm{O}}}
\def\mP{{\bm{P}}}
\def\mQ{{\bm{Q}}}
\def\mR{{\bm{R}}}
\def\mS{{\bm{S}}}
\def\mT{{\bm{T}}}
\def\mU{{\bm{U}}}
\def\mV{{\bm{V}}}
\def\mW{{\bm{W}}}
\def\mX{{\bm{X}}}
\def\mY{{\bm{Y}}}
\def\mZ{{\bm{Z}}}
\def\mBeta{{\bm{\beta}}}
\def\mPhi{{\bm{\Phi}}}
\def\mLambda{{\bm{\Lambda}}}
\def\mSigma{{\bm{\Sigma}}}

% Tensor
\DeclareMathAlphabet{\mathsfit}{\encodingdefault}{\sfdefault}{m}{sl}
\SetMathAlphabet{\mathsfit}{bold}{\encodingdefault}{\sfdefault}{bx}{n}
\newcommand{\tens}[1]{\bm{\mathsfit{#1}}}
\def\tA{{\tens{A}}}
\def\tB{{\tens{B}}}
\def\tC{{\tens{C}}}
\def\tD{{\tens{D}}}
\def\tE{{\tens{E}}}
\def\tF{{\tens{F}}}
\def\tG{{\tens{G}}}
\def\tH{{\tens{H}}}
\def\tI{{\tens{I}}}
\def\tJ{{\tens{J}}}
\def\tK{{\tens{K}}}
\def\tL{{\tens{L}}}
\def\tM{{\tens{M}}}
\def\tN{{\tens{N}}}
\def\tO{{\tens{O}}}
\def\tP{{\tens{P}}}
\def\tQ{{\tens{Q}}}
\def\tR{{\tens{R}}}
\def\tS{{\tens{S}}}
\def\tT{{\tens{T}}}
\def\tU{{\tens{U}}}
\def\tV{{\tens{V}}}
\def\tW{{\tens{W}}}
\def\tX{{\tens{X}}}
\def\tY{{\tens{Y}}}
\def\tZ{{\tens{Z}}}


% Graph
\def\gA{{\mathcal{A}}}
\def\gB{{\mathcal{B}}}
\def\gC{{\mathcal{C}}}
\def\gD{{\mathcal{D}}}
\def\gE{{\mathcal{E}}}
\def\gF{{\mathcal{F}}}
\def\gG{{\mathcal{G}}}
\def\gH{{\mathcal{H}}}
\def\gI{{\mathcal{I}}}
\def\gJ{{\mathcal{J}}}
\def\gK{{\mathcal{K}}}
\def\gL{{\mathcal{L}}}
\def\gM{{\mathcal{M}}}
\def\gN{{\mathcal{N}}}
\def\gO{{\mathcal{O}}}
\def\gP{{\mathcal{P}}}
\def\gQ{{\mathcal{Q}}}
\def\gR{{\mathcal{R}}}
\def\gS{{\mathcal{S}}}
\def\gT{{\mathcal{T}}}
\def\gU{{\mathcal{U}}}
\def\gV{{\mathcal{V}}}
\def\gW{{\mathcal{W}}}
\def\gX{{\mathcal{X}}}
\def\gY{{\mathcal{Y}}}
\def\gZ{{\mathcal{Z}}}

% Sets
\def\sA{{\mathbb{A}}}
\def\sB{{\mathbb{B}}}
\def\sC{{\mathbb{C}}}
\def\sD{{\mathbb{D}}}
% Don't use a set called E, because this would be the same as our symbol
% for expectation.
\def\sF{{\mathbb{F}}}
\def\sG{{\mathbb{G}}}
\def\sH{{\mathbb{H}}}
\def\sI{{\mathbb{I}}}
\def\sJ{{\mathbb{J}}}
\def\sK{{\mathbb{K}}}
\def\sL{{\mathbb{L}}}
\def\sM{{\mathbb{M}}}
\def\sN{{\mathbb{N}}}
\def\sO{{\mathbb{O}}}
\def\sP{{\mathbb{P}}}
\def\sQ{{\mathbb{Q}}}
\def\sR{{\mathbb{R}}}
\def\sS{{\mathbb{S}}}
\def\sT{{\mathbb{T}}}
\def\sU{{\mathbb{U}}}
\def\sV{{\mathbb{V}}}
\def\sW{{\mathbb{W}}}
\def\sX{{\mathbb{X}}}
\def\sY{{\mathbb{Y}}}
\def\sZ{{\mathbb{Z}}}

% Entries of a matrix
\def\emLambda{{\Lambda}}
\def\emA{{A}}
\def\emB{{B}}
\def\emC{{C}}
\def\emD{{D}}
\def\emE{{E}}
\def\emF{{F}}
\def\emG{{G}}
\def\emH{{H}}
\def\emI{{I}}
\def\emJ{{J}}
\def\emK{{K}}
\def\emL{{L}}
\def\emM{{M}}
\def\emN{{N}}
\def\emO{{O}}
\def\emP{{P}}
\def\emQ{{Q}}
\def\emR{{R}}
\def\emS{{S}}
\def\emT{{T}}
\def\emU{{U}}
\def\emV{{V}}
\def\emW{{W}}
\def\emX{{X}}
\def\emY{{Y}}
\def\emZ{{Z}}
\def\emSigma{{\Sigma}}

% entries of a tensor
% Same font as tensor, without \bm wrapper
\newcommand{\etens}[1]{\mathsfit{#1}}
\def\etLambda{{\etens{\Lambda}}}
\def\etA{{\etens{A}}}
\def\etB{{\etens{B}}}
\def\etC{{\etens{C}}}
\def\etD{{\etens{D}}}
\def\etE{{\etens{E}}}
\def\etF{{\etens{F}}}
\def\etG{{\etens{G}}}
\def\etH{{\etens{H}}}
\def\etI{{\etens{I}}}
\def\etJ{{\etens{J}}}
\def\etK{{\etens{K}}}
\def\etL{{\etens{L}}}
\def\etM{{\etens{M}}}
\def\etN{{\etens{N}}}
\def\etO{{\etens{O}}}
\def\etP{{\etens{P}}}
\def\etQ{{\etens{Q}}}
\def\etR{{\etens{R}}}
\def\etS{{\etens{S}}}
\def\etT{{\etens{T}}}
\def\etU{{\etens{U}}}
\def\etV{{\etens{V}}}
\def\etW{{\etens{W}}}
\def\etX{{\etens{X}}}
\def\etY{{\etens{Y}}}
\def\etZ{{\etens{Z}}}

% The true underlying data generating distribution
\newcommand{\pdata}{p_{\rm{data}}}
% The empirical distribution defined by the training set
\newcommand{\ptrain}{\hat{p}_{\rm{data}}}
\newcommand{\Ptrain}{\hat{P}_{\rm{data}}}
% The model distribution
\newcommand{\pmodel}{p_{\rm{model}}}
\newcommand{\Pmodel}{P_{\rm{model}}}
\newcommand{\ptildemodel}{\tilde{p}_{\rm{model}}}
% Stochastic autoencoder distributions
\newcommand{\pencode}{p_{\rm{encoder}}}
\newcommand{\pdecode}{p_{\rm{decoder}}}
\newcommand{\precons}{p_{\rm{reconstruct}}}

\newcommand{\laplace}{\mathrm{Laplace}} % Laplace distribution

\newcommand{\E}{\mathbb{E}}
\newcommand{\Ls}{\mathcal{L}}
% \newcommand{\R}{\mathbb{R}}
\newcommand{\emp}{\tilde{p}}
\newcommand{\lr}{\alpha}
\newcommand{\reg}{\lambda}
\newcommand{\rect}{\mathrm{rectifier}}
\newcommand{\softmax}{\mathrm{softmax}}
\newcommand{\sigmoid}{\sigma}
\newcommand{\softplus}{\zeta}
\newcommand{\KL}{D_{\mathrm{KL}}}
\newcommand{\Var}{\mathrm{Var}}
\newcommand{\standarderror}{\mathrm{SE}}
\newcommand{\Cov}{\mathrm{Cov}}
% Wolfram Mathworld says $L^2$ is for function spaces and $\ell^2$ is for vectors
% But then they seem to use $L^2$ for vectors throughout the site, and so does
% wikipedia.
\newcommand{\normlzero}{L^0}
\newcommand{\normlone}{L^1}
\newcommand{\normltwo}{L^2}
\newcommand{\normlp}{L^p}
\newcommand{\normmax}{L^\infty}

\newcommand{\parents}{Pa} % See usage in notation.tex. Chosen to match Daphne's book.


% Customized Math
\DeclareMathOperator{\Tr}{Tr}

\DeclareMathOperator*{\argmax}{arg\,max}
\DeclareMathOperator*{\argmin}{arg\,min}

\DeclareMathOperator{\sign}{sign}
\let\ab\allowbreak

\newcommand{\trace}{\Tr}

% \usepackage{microtype}
\usepackage{geometry}
% \usepackage{subfig}
\usepackage{booktabs} 
\usepackage{bbm}
\usepackage{mathtools}
% \usepackage{amsthm}
\usepackage{nccmath}
\usepackage{setspace}

\usepackage{caption}
\usepackage{subcaption}

\usepackage[linesnumbered,ruled,vlined]{algorithm2e}
% \usepackage{algorithmic}
% \usepackage{algorithm}

\SetKwInput{KwInput}{Input}                % Set the Input
\SetKwInput{KwOutput}{Output}              % set the Output
\newcommand\mycommfont[1]{\footnotesize\ttfamily\textcolor{blue}{#1}}
\SetCommentSty{mycommfont}
\newcommand{\algcapsty}[1]{\small\sffamily\bfseries{#1}}
\SetAlCapSty{algcapsty}

\usepackage[T1]{fontenc}
\usepackage{wrapfig,lipsum,booktabs}

% \usepackage{natbib}
\usepackage{soul}
\usepackage{dsfont}
\usepackage{enumerate}
\usepackage{enumitem}

% \usepackage{kotex}
% \usepackage{hyperref}
% \usepackage[hidelinks]{hyperref}
% \usepackage{amsmath}
% \usepackage{amsthm}
\usepackage{amsfonts}
\usepackage{bbm}
\usepackage{dsfont}
\usepackage[Symbol]{upgreek}
\usepackage{lscape}
\usepackage{caption}
\usepackage{balance}
\usepackage{xspace}
\usepackage{float}
\usepackage{kotex}

\usepackage{wasysym}
%\usepackage[table,xcdraw,dvipsnames]{xcolor}
\usepackage{xcolor}
\usepackage{multirow}
\usepackage{array, boldline, rotating}

\usepackage{amssymb}% http://ctan.org/pkg/amssymb
\usepackage{pifont}% http://ctan.org/pkg/pifont
\newcommand{\cmark}{\ding{51}\xspace}%
\newcommand{\omark}{\textbf{$\mathcal{O}$}\xspace}%
\newcommand{\xmark}{\ding{55}\xspace}%

\newcommand{\ds}[1]{\mathds{#1}}
\newcommand{\mc}[1]{\mathcal{#1}}
\newcommand{\bb}[1]{\mathbbm{#1}}

% %%%%%% Theorem Related Things %%%%%%
% \theoremstyle{plain}
% \newtheorem{thm}{Theorem}
% \newtheorem{cor}{Corollary}
% \newtheorem{lem}{Lemma}
% \newtheorem{prop}{Proposition}

% \theoremstyle{definition}
% \newtheorem{defn}{Definition}
% \newtheorem{assum}{Assumption}



% Citation

% \let\oldeqcite\cite
% \renewcommand*\cite[1]{(\oldcite{#1})}
\let\oldeqref\eqref
\renewcommand*\eqref[1]{(\ref{#1})}

% % Highlight (incl. note)
\newcommand{\smnote}[1]{\textbf{\textcolor{Cyan}{SM: #1}}}
\newcommand{\jhnote}[1]{\textbf{\textcolor{Orange}{JH: #1}}}
\newcommand{\yes}[1]{\textcolor{blue}{[YES]}}
\newcommand{\no}[1]{\textcolor{orange}{[NO]}}
\newcommand{\na}[1]{\textcolor{gray}{[N/A]}}
%\newcommand\bg[1]{\textcolor{blue}{#1}} % JH
\newcommand\jt[1]{\textcolor{brown}{#1}} % JT
\newcommand\jh[1]{\textcolor{black}{#1}} % JH
\newcommand\sm[1]{\textcolor{blue}{#1}} % SM
\newcommand{\eg}{\emph{e.g.,~}}
\newcommand{\ie}{\emph{i.e.,~}}


% \renewcommand\jt[1]{\textcolor{black}{#1}} % JT
% \renewcommand\jh[1]{\textcolor{black}{#1}} % JH

% % Separation (paragraph)
\newcommand{\myparagraph}[1]{\vspace{0.07cm}\noindent\textbf{#1}~}

% % math op.

% % Font
\def\code#1{\texttt{#1}}
\DeclarePairedDelimiter\norm{\lVert}{\rVert}

% % % Definition
% \theoremstyle{definition}
\newcommand\scalemath[2]{\scalebox{#1}{\mbox{\ensuremath{\displaystyle #2}}}}



\newcommand{\thickhline}{\hlineB{4}}
\newcommand{\bfcode}[1]{\code{\textbf{#1}}}


\definecolor{LightCyan}{rgb}{0.88,1,1}
\definecolor{Blue}{rgb}{0, 0.3, 0.6}
\definecolor{Orange}{rgb}{0.8, 0.4, 0}
\definecolor{Green}{rgb}{0.0, 0.8, 0.0 }
\definecolor{Red}{rgb}{0.95, 0.55, 0.6}
\definecolor{Skyblue}{rgb}{0.6, 0.6, 0.95 }



% Supplementary title
\NewDocumentEnvironment{suptitle}{ +b }{
    \twocolumn[{#1}]%
}{}

\NewDocumentCommand{\supptitle}{s}{
\begin{suptitle}
        \centering
        % \rule{\textwidth}{0.07cm}\\[-0.34cm]
        \rule{\textwidth}{0.03cm}\\[0.1cm]
        -Supplementary Material-\\[0.2cm]
        {\Large 
            \textbf{\mytitle }
        }\\%[0.40cm]
        \rule{\textwidth}{0.03cm}\\[0.2cm]
\end{suptitle}}

\newcommand{\llama}{LLaMA}
\newcommand{\tr}{\textrm{tr}}
\newcommand{\per}{\textrm{c}}
\newcommand{\pool}{pool}
\newcommand{\mytitle}{Chain-of-Rank: Enhancing Large Language Models \\ for Domain-Specific RAG in Edge Device}
%\newcommand{\cmark}{\ding{51}}%
% If the title and author information does not fit in the area allocated, uncomment the following
%
%\setlength\titlebox{<dim>}
%
% and set <dim> to something 5cm or larger.

%\title{\alg: Instant Personalized LoRA Generation for On-device and Hybrid Decoding}
% \title{OPA: On-the-Fly Personalized Adapter \& Device-Server Consistent Inference for On-Device LLM}
%\title{On-Palette: On-the-fly Person-Adapted On-device LLM and Edge-to-server Transfer for Hybrid Inference}
% \title{On-Device Palette: Personalized On-the-Fly Adapter and Edge-Server Hybrid Inference}
\title{\mytitle}
% \title{Palette: Person-Aaptation of LLM On-the-Fly and Edge-Server Transit for }


%\title{Palette: Personalized Adapter for LLM Edge device T... Toward End device.}
% PersonAr,
% Personalized On-the-Fly Adapter and Device-Server Hybrid Inference for On-Device LLM.  


% Author information can be set in various styles:
% For several authors from the same institution:
% \author{Author 1 \and ... \and Author n \\
%         Address line \\ ... \\ Address line}
% if the names do not fit well on one line use
%         Author 1 \\ {\bf Author 2} \\ ... \\ {\bf Author n} \\
% For authors from different institutions:
% \author{Author 1 \\ Address line \\  ... \\ Address line
%         \And  ... \And
%         Author n \\ Address line \\ ... \\ Address line}
% To start a separate ``row'' of authors use \AND, as in
% \author{Author 1 \\ Address line \\  ... \\ Address line
%         \AND
%         Author 2 \\ Address line \\ ... \\ Address line \And
%         Author 3 \\ Address line \\ ... \\ Address line}

%\author{Jihwan Bang$^*$\hspace{1em}Juntae Lee$^*$\hspace{1em}Kyuhong Shim\hspace{1em}Seunghan Yang\hspace{1em}Simyung Chang$^\dag$\\
%{Qualcomm AI Research$^\ddag$, Qualcomm Korea YH, Seoul, Republic of Korea} \\ 
%{\texttt {\small\{jihwbang, juntlee, kshim, seunghan, simychan\}@qti.qualcomm.com}}}

\author{Juntae Lee\hspace{1em}Jihwan Bang\hspace{1em}Seunghan Yang\hspace{1em}Kyuhong Shim\hspace{1em}Simyung Chang\\
{Qualcomm AI Research$^\dag$} \\ 
{\texttt {\small\{juntlee, jihwbang, seunghan, kshim, simychan\}@qti.qualcomm.com}}}

\begin{document}
\maketitle

\blfootnote{\hspace{-1.8em}$^\dag$Qualcomm AI Research is an initiative of Qualcomm Technologies, Inc. and/or its subsidiaries.}

\begin{abstract}
%Retrieval-augmented generation (RAG) with large language models (LLMs) has proven effective in addressing factual hallucination by enabling dynamic access to external knowledge. This is particularly valuable in specialized domains, where precision is critical. 
Retrieval-augmented generation (RAG) with large language models (LLMs) is especially valuable in specialized domains, where precision is critical. 
To more specialize the LLMs into a target domain, domain-specific RAG has recently been developed by allowing the LLM to access the target domain early via finetuning. 
The domain-specific RAG makes more sense in resource-constrained environments like edge devices, as they should perform a specific task (e.g. personalization) reliably using only small-scale LLMs.
%The domain-specific RAG is more crucial when computational resources are limited such as edge devices since only small LLM is needed to perform well on the target task
%a small-scale LLM needs to focus on doing several selected tasks well. 
%Domain-specific RAG is useful for edge devices with limited computational resources, since smaller, specialized LLMs can efficiently handle certain tasks.
While the domain-specific RAG is well-aligned with edge devices in this respect, it often relies on widely-used reasoning techniques like chain-of-thought (CoT). The reasoning step is useful to understand the given external knowledge, and yet it is computationally expensive and difficult for small-scale LLMs to learn it. %Parameter-efficient fine-tuning methods, such as LoRA adapters, offer a solution to resource constraints but further limit the model's ability to perform complex reasoning tasks. 
Tackling this, we propose the Chain of Rank (CoR) which shifts the focus from intricate lengthy reasoning to simple ranking of the reliability of input external documents. Then, CoR reduces computational complexity while maintaining high accuracy, making it particularly suited for resource-constrained environments. We attain the state-of-the-art (SOTA) results in benchmarks, and analyze its efficacy.

\end{abstract}
%\jt{}
\section{Introduction}
Implicit Neural Representations (INRs), which fit the target function using only input coordinates, have recently gained significant attention.
%
By leveraging the powerful fitting capability of Multilayer Perceptrons (MLPs), INRs can implicitly represent the target function without requiring their analytical expressions. 
%
The versatility of MLPs allows INRs to be applied in various fields, including inverse graphics~\citep{mildenhall2021nerf, barron2023zip, martin2021nerf}, image super-resolution~\citep{chen2021learning, yuan2022sobolev, gao2023implicit}, 
image generation~\citep{skorokhodov2021adversarial}, and more~\citep{chen2021nerv, strumpler2022implicit, shue20233d}.
%
\begin{figure}
    \includegraphics[width=0.5\textwidth]{Image/Fig2.pdf}
    \caption{As illustrated at the circled blue regions and green regions, it can be observed that even with well-chosen standard deviation/scale, as experimented in \autoref{figure:combined}, the results are still unsatisfactory. However, using our proposed method, the noise is significantly alleviated while further enhancing the high-frequency details.}
    \label{fig:var}
    \vspace{-10pt}
\end{figure}

\begin{figure*}[!ht]
    \centering
    \begin{minipage}[b]{0.25\textwidth}
        \centering
        \includegraphics[width=1.\textwidth]{Image/fig_cropped.pdf} % 替换为你的小图文件
        \label{figure:small_image}
        \vspace{-20pt}
    \end{minipage}%
    \hfill
    \begin{minipage}[b]{0.75\textwidth}
        \centering
        \includegraphics[width=1.\textwidth]{Image/psnr_trends_rff_pe_simplified.pdf} % 替换为你的大图文件
        \vspace{-20pt}
        \label{figure:large_image}
        
    \end{minipage}
    \caption{We test the performance of MLPs with Random Fourier Features (RFF) and MLPs with Positional Encoding (PE) on a 1024-resolution image to better distinguish between high- and low-frequency regions, as demonstrated on the left-hand side of this figure. We find that the performance of MLPs+RFF degrades rapidly with increasing standard deviation compared with MLPs+PE. Since positional encoding is deterministic, scale=512 can be considered to have standard deviation around 121.}
    \label{figure:combined}
    \vspace{-10pt}
\end{figure*}
Varying the sampling standard deviation/scale may lead to degradation results, as shown in \autoref{figure:combined}.
%
However, MLPs face a significant challenge known as the spectral bias, where low-frequency signals are typically favored during training~\citep{rahaman2019spectral}. 
A common solution is to map coordinates into the frequency domain using Fourier features, such as Random Fourier Features and Positional Encoding, which can be understood as manually set high-frequency correspondence prior to accelerating the learning of high-frequency targets.~\citep{tancik2020fourier}. 
This embeddings widely applied to the INRs for novel view synthesis~\citep{mildenhall2021nerf,barron2021mip}, dynamic scene reconstruction~\citep{pumarola2021d}, object tracking~\citep{wang2023tracking}, and medical imaging~\citep{corona2022mednerf}.
% \begin{figure}[!h]
%     \centering
%     \includegraphics[width=1.\textwidth]{Image/psnr_trends_rff_pe_simplified.pdf}
%     \caption{This figure shows the change of PSNR on the whole, low-frequency region, and high-frequency region of the image fitting by using two Fourier Features Embedding with varying scale of variance: (Right) Positional Encoding (PE) (Left) Random Fourier Features (RFF). Both PE and RFF will degrade the low-frequency regions of the target image when variance increases.}
%     \vspace{-20pt} 
%     \label{figure:stats}
% \end{figure}


Although many INRs' downstream application scenarios use this encoding type, it has certain limitations when applied to specific tasks.
%
It depends heavily on two key hyperparameters: the sampling standard deviation/scale (available sampling range of frequencies) and the number of samples.
%
Even with a proper choice of sampling standard deviation/scale, the output remains unsatisfactory, as shown in \autoref{fig:var}: Noisy low-frequency regions and degraded high-frequency regions persist with well chosen sampling standard deviation/scale with the grid-searched standard deviation/scale, which may potentially affect the performance of the downstream applications resulting in noisy or coarse output.
%
However, limited research has contributed to explaining the reason and finding a proper frequency embeddings for input~\citep{landgraf2022pins, yuce2022structured}.

In this paper, we aim to offer a potential explanation for the high-frequency noise and propose an effective solution to the inherent drawbacks of Fourier feature embeddings for INRs.
%
Firstly, we hypothesize that the noisy output arises from the interaction between Fourier feature embeddings and multi-layer perceptrons (MLPs). We argue that these two elements can enhance each other's representation capabilities when combined. However, this combination also introduces the inherent properties of the Fourier series into the MLPs.
%
To support our hypothesis, we propose a simple theorem stating that the unsampled frequency components of the embeddings establish a lower bound on the expected performance. This underpins our hypothesis, as the primary fitting error in finitely sampled Fourier series originates from these unsampled frequencies.

Inspired by the analysis of noisy output and the properties of Fourier series expansion, we propose an approach to address this issue by enabling INRs to adaptively filter out unnecessary high-frequency components in low-frequency regions while enriching the input frequencies of the embeddings if possible.
%
To achieve this, we employ bias-free (additive term-free) MLPs. These MLPs function as adaptive linear filters due to their strictly linear and scale-invariant properties~\citep{mohan2019robust}, which preserves the input pattern through each activation layer and potentially enhances the expressive capability of the embeddings.
%
Moreover, by viewing the learning rate of the proposed filter and INRs as a dynamically balancing problem, we introduce a custom line-search algorithm to adjust the learning rate during training. This algorithm tackles an optimization problem to approximate a global minimum solution. Integrating these approaches leads to significant performance improvements in both low-frequency and high-frequency regions, as demonstrated in the comparison shown in \autoref{fig:var}.
%
Finally, to evaluate the performance of the proposed method, we test it on various INRs tasks and compare it with state-of-the-art models, including BACON~\citep{lindell2022bacon}, SIREN~\citep{sitzmann2020implicit}, GAUSS~\citep{ramasinghe2022beyond} and WIRE~\citep{saragadam2023wire}. 
The experimental results prove that our approach enables MLPs to capture finer details via Fourier Features while effectively reducing high-frequency noise without causing oversmoothness.
%
To summarize, the following are the main contributions of this work:
\begin{itemize}
    \item From the perspective of Fourier features embeddings and MLPs, we hypothesize that the representation capacity of their combination is also the combination of their strengths and limitations. A simple lemma offers partial validation of this hypothesis.

    
    \item  We propose a method that employs a bias-free MLP as an adaptive linear filter to suppress unnecessary high frequencies. Additionally, a custom line-search algorithm is introduced to dynamically optimize the learning rate, achieving a balance between the filter and INRs modules.

    \item To validate our approach, we conduct extensive experiments across a variety of tasks, including image regression, 3D shape regression, and inverse graphics. These experiments demonstrate the effectiveness of our method in significantly reducing noisy outputs while avoiding the common issue of excessive smoothing.
\end{itemize}

\subsubsection{Conditioned Diffusion Models}

By operating the data in latent space instead of pixel space, conditioned diffusion models have gained promising development \cite{rombach2022latentDiff}. MM-Diffusion \cite{ruan2023mmdi} designed for joint audio and video generation took advantage of coupled denoising autoencoders to generate aligned audio-video pairs from Gaussian noise. Extending the scalability of diffusion models, diffusion Transformers treat all inputs, including time, conditions, and noisy image patches, as tokens, leveraging the Transformer architecture to process these inputs \cite{bao2023ViTDiff}. In DiT \cite{peebles2023DiT}, William et al. emphasized the potential for diffusion models to benefit from Transformer architectures, where conditions were tokenized along with image tokens to achieve in-context conditioning. 

\subsubsection{Diffusion Models in Robotics}

Recently, a probabilistic multimodal action representation was proposed by Cheng Chi et al. \cite{chi2023diffusionpolicy}, where the robot action generation is considered as a conditional diffusion denoising process. Leveraging the diffusion policy, Ze et al. \cite{ze20243d} conditioned the diffusion policy on compact 3D representations and robot poses to generate coherent action sequences. Furthermore, GR-MG combined a progress-guided goal image generation model with a multimodal goal-conditioned policy, enabling the robot to predict actions based on both text instructions and generated goal images \cite{li2025grmg}. BESO used score-based diffusion models to learn goal-conditioned policies from large, uncurated datasets without rewards. Score-based diffusion models progressively add noise to the data and then reverse this process to generate new samples, making them suitable for capturing the multimodal nature of play data \cite{reuss2023md}. RDT-1B employed a scalable Transformer backbone combined with diffusion models to capture the complexity and multimodality of bimanual actions, leveraging diffusion models as a foundation model to effectively represent the multimodality inherent in bimanual manipulation tasks \cite{liu2024rdt-1b}. NoMaD exploited the diffusion model to handle both goal-directed navigation and task-agnostic exploration in unfamiliar environments, using goal masking to condition the policy on an optional goal image, allowing the model to dynamically switch between exploratory and goal-oriented behaviors \cite{sridhar2023nomad}. The aforementioned insights grounded the significant advancements of diffusion models in robotic tasks.

\subsubsection{VLM-based Autonomous Driving}

End-to-end autonomous driving introduces policy learning from sensor data input, resulting in a data-driven motion planning paradigm \cite{chen2024vadv2}. As part of the development of VLMs, they have shown significant promise in unifying multimodal data for specific downstream tasks, notably improving end-to-end autonomous driving systems\cite{ma2024dolphins}. DriveMM can process single images, multiview images, single videos, and multiview videos, and perform tasks such as object detection, motion prediction, and decision making, handling multiple tasks and data types in autonomous driving \cite{huang2024drivemm}. HE-Drive aims to create a human-like driving experience by generating trajectories that are both temporally consistent and comfortable. It integrates a sparse perception module, a diffusion-based motion planner, and a trajectory scorer guided by a Vision Language Model to achieve this goal \cite{wang2024hedrive}. Based on current perspectives, a differentiable end-to-end autonomous driving paradigm that directly leverages the capabilities of VLM and a multimodal action representation should be developed. 








We applied Recurrency Sequence Processing to address the lack of consistency in the coarse dance representation of the~\cite{li2024lodge} model. We named this Recurrency Sequence Representation Learning as Dance Recalibration (DR). Dance recalibration uses \(n\) Dance Recalibration Blocks (DRB) corresponding to the length of the rough dance sequence to add sequential information to the rough dance representation to improve the consistency of the whole dance. The overall structure of our model is illustrated in Figure 1.

\begin{figure}[!t]
    \centering
    \includegraphics[width=\textwidth]{Figure1.eps}
    \caption{overall procedure of Pooling processing by our Pooling Block}
    \label{fig:enter-label4}
\end{figure}


\subsection{Dance Recalibration (DR)}
When the dance motion representation passes through the Dance Decoder Process using the~\cite{li2024lodge} model, it yields a coarse dance motion representation. During this process, the dance motion representations that pass through Global Diffusion follow a distribution but can output unstable values. This results in awkward dance motions when viewed from a sequential perspective. To address this issue, we added a Dance Recalibration Process.

DR fundamentally follows a structure similar to RNNs. Although RNNs are known to suffer from the gradient vanishing problem as they get deeper, the sequence length of the coarse dance representation in \cite{li2024lodge} is not long enough to cause this issue, making it suitable for use. Using LSTM or GRU, which solve the gradient vanishing problem, would make the model too complex and computationally expensive, making them unsuitable for use with the Denoising Diffusion Process \cite{ho2020denoising, song2020denoising}.

The coarse dance representation has 139 channels, consisting of 4-dim foot positions, 3-dim root translation, 6-dim rotaion information and 126-dim joint rotation channels. Of these, the 126-dim channels directly impact the dance motion, and all DR operations are performed using these 126 channels.

The values output from the Global Dance Decoder \(GD_{i}\), contain unstable dance motion information that follows a distribution. We construct Global Recalibrated Dance \(GRD_{i}\) by concatenating \(C\) the information from \(GRD_{i-1}\) with \(GD_{i}\) and applying pooling \(P\), thereby adding sequential information. However, using previous information as is may result in overly simple and smoothly connected dance motions. To prevent this, we add Gaussian noise \(G\) to the previous information \(GRD_{i-1}\) to produce more varied dance motions. This process is represented in Equations 1 below. The entire procedure is illustrated in Figure 2, 3.
\begin{equation}
    GRD_{i} = P(C(GD_{i} , GRD_{i-1} + G(Threshold))
\end{equation}



\begin{figure}[!t]
    \centering
    \includegraphics[width=\textwidth]{DanceRecalibration.eps}
    \caption{Overall of the Dance Recalibration Block Structure}
    \label{fig:enter-label1}
\end{figure}

\begin{figure}[!t]
    \centering
    \includegraphics[width=\textwidth]{DanceRecalibrationBlock.eps}
    \caption{The structure of the dance recalibration block}
    \label{fig:enter-label2}
\end{figure}

\subsection{Pooling Block}
Pooling \(P\) uses a simple pooling method. When \(GRD_{i}\) with added \(G\) and \(GD_{i+1}\) are input, they are concatenated into a \((Batch\times2\times126)\). First, we perform Layer Normalization to minimize differences between layers. Then, we pass through three simple 1D-Convolution Blocks, each followed by an activation function and batch normalization, to construct \(GRD_{i+1}\) that includes information from the previous dance motion. This procedure is illustrated in Figure 4.

\begin{figure}[!t]
    \centering
    \includegraphics[width=\textwidth]{Figure3.eps}
    \caption{overall procedure of Pooling processing by our Pooling Block}
    \label{fig:enter-label3}
\end{figure}

By following all these steps, each dance motion incorporates a bit of information from the previous dance motions, producing an overall coarse dance motion that follows the distribution of Global Diffusion while also retaining sequential information. This process is expressed in Equation 2:

\begin{equation}
    Total Coarse Dance Motion = C_{i=1}^{n}(P(C(GD_{i} , GRD_{i-1} + G(Threshold))), P(GD_{0}))
\end{equation}

We did not use bi-directional information because it complicates the calculations and can destabilize sequential information when using more than two \(GD_{i}\). Since there is a trade-off between generating complex dance motions and maintaining consistency, it is crucial to add appropriate noise. However, due to time constraints, we could not conduct various ablation studies.
\begin{table*}[b] 
    \centering
     \label{tab:GPT}
    \caption{Results of Time-Series Forecasing with GPT-series Models. }
\begin{subtable}{\textwidth}
\centering
\caption{Results of Unimodal Short-term Time Series Forecasting with GPT-series Models.}
\label{tab:Uni_S_GPT}
\begin{tabular}{c|c|ccc|c} 
\hline
\multirow{2}{*}{Dataset}           & System 1    & \multicolumn{3}{c|}{~System 1 with~Test-time Reasoning Enhancement} & System~~~~ 2  \\ 
\cline{2-6}
                                   & GPT-4o      & with CoT     & with Self-Consistency & with Self-Correction  & o1-mini       \\ 
\hline
Agriculture & 0.021$\pm$0.011 & \worse{0.909$\pm$1.275} & \better{0.021$\pm$0.003} & \worse{0.025$\pm$0.007} & \worse{0.069$\pm$0.013} \\
 Climate & 1.599$\pm$0.500 & \worse{1.704$\pm$0.164} & \better{1.517$\pm$0.263} & \worse{1.998$\pm$0.677} & \better{1.412$\pm$0.159} \\
 Economy & 0.631$\pm$0.135 & \worse{0.638$\pm$0.410} & \better{0.450$\pm$0.171} & \worse{1.018$\pm$0.184} & \better{0.583$\pm$0.001} \\
 Energy & 0.363$\pm$0.110 & \better{0.258$\pm$0.029} & \better{0.167$\pm$0.242} & \worse{0.396$\pm$0.086} & \worse{0.930$\pm$0.747} \\
 Flu & 0.568$\pm$0.425 & \worse{0.592$\pm$0.291} & \better{0.481$\pm$0.288} & \worse{0.663$\pm$0.078} & \worse{1.441$\pm$1.234} \\
 Security & 0.093$\pm$0.029 & \worse{0.259$\pm$0.001} & \better{0.084$\pm$0.028} & \worse{0.165$\pm$0.070} & \worse{0.225$\pm$0.048} \\
 Employment & 0.010$\pm$0.004 & \better{0.006$\pm$0.002} & \worse{0.012$\pm$0.001} & \worse{0.013$\pm$0.003} & \worse{0.021$\pm$0.003} \\
Traffic & 0.385$\pm$0.471 & \better{0.113$\pm$0.063} & \worse{0.0468$\pm$0.009} & \better{0.053$\pm$0.009} & \worse{0.566$\pm$0.731} \\
\hline
Win System 1 & NA & $3/8$ & $5/8$ & $1/8$ & $2/8$ \\
\hline
\end{tabular}
\end{subtable}
 % Add vertical space between subtables
\begin{subtable}{\textwidth}
\centering
\caption{Short-term Unimodal forecasting (Gemini))}
\label{tab:performance_comparison_unimodal}
\begin{tabular}{c|c|ccc|c} 
\hline
\multirow{2}{*}{Dataset}           & System 1    & \multicolumn{3}{c|}{~System 1 with~Test-time Reasoning Enhancement} & System~~~~ 2  \\ 
\cline{2-6}
                                   & Gemini-2.0-flash      & with CoT     & with Self-Consistency & with Self-Correction  & Gemini-2.0-flash-thinking       \\ 
\hline
 Agriculture & 0.011$\pm$0.001 & \better{0.010$\pm$0.004} & \better{0.009$\pm$0.004} & \worse{0.012$\pm$0.008} & \worse{0.017$\pm$0.004} \\
 Climate & 1.234$\pm$0.239 & \worse{1.800$\pm$0.326} & \worse{1.749$\pm$0.791} & \worse{1.703$\pm$0.280} & \worse{2.416$\pm$0.112} \\
 Economy & 0.113$\pm$0.007 & \worse{0.272$\pm$0.256} & \worse{0.229$\pm$0.145} & \worse{0.121$\pm$0.026} & \worse{0.172$\pm$0.049} \\
 Energy & 0.172$\pm$0.038 & \worse{0.181$\pm$0.048} & \better{0.132$\pm$0.047} & \worse{0.235$\pm$0.060} & \worse{0.327$\pm$0.054} \\
 Flu & 0.809$\pm$0.353 & \better{0.641$\pm$0.224} & \better{0.402$\pm$0.197} & \worse{1.854$\pm$1.271} & \worse{2.068$\pm$1.076} \\
 Security & 0.170$\pm$0.054 & \worse{0.252$\pm$0.104} & \worse{0.380$\pm$0.323} & \worse{0.191$\pm$0.095} & \worse{0.259$\pm$0.001} \\
 Employment & 0.002$\pm$0.001 & \worse{0.005$\pm$0.003} & \worse{0.004$\pm$0.004} & \worse{0.004$\pm$0.002} & \worse{0.311$\pm$0.001} \\
Traffic & 0.347$\pm$0.415 & \better{0.097$\pm$0.060} & \better{0.016$\pm$0.006} & \better{0.034$\pm$0.014} & \better{0.201$\pm$0.001} \\
\hline
Win System 1 & NA & $3/8$ & $4/8$ & $1/8$ & $1/8$ \\
\hline
\end{tabular}
\end{subtable}
 \begin{subtable}{\textwidth}
\centering
\caption{Short-term Unimodal forecasting (DeepSeek))}
\label{tab:performance_comparison_unimodal}
\begin{tabular}{c|c|ccc|c} 
\hline
\multirow{2}{*}{Dataset}           & System 1    & \multicolumn{3}{c|}{~System 1 with~Test-time Reasoning Enhancement} & System~~~~ 2  \\ 
\cline{2-6}
                                   & DeepSeek-V3      & with CoT     & with Self-Consistency & with Self-Correction  & DeepSeek-R1       \\ 
\hline
 Agriculture & 0.038$\pm$0.032 & \better{0.019$\pm$0.001} & \worse{0.046$\pm$0.015} & \better{0.013$\pm$0.003} & \better{0.016$\pm$0.010} \\
 Climate & 1.216$\pm$0.202 & \worse{2.650$\pm$0.905} & \better{1.207$\pm$0.197} & \worse{1.246$\pm$0.081} & \worse{1.541$\pm$0.397} \\
 Economy & 0.406$\pm$0.218 & \worse{0.433$\pm$0.031} & \better{0.284$\pm$0.227} & \worse{0.441$\pm$0.161} & \worse{0.583$\pm$0.001} \\
 Energy & 0.736$\pm$0.752 & \better{0.212$\pm$0.022} & \better{0.187$\pm$0.011} & \better{0.182$\pm$0.063} & \better{0.189$\pm$0.021} \\
 Flu & 1.464$\pm$1.031 & \worse{1.650$\pm$0.236} & \better{0.980$\pm$0.445} & \worse{1.682$\pm$0.292} & \better{1.298$\pm$1.330} \\
 Security & 0.283$\pm$0.140 & \better{0.218$\pm$0.093} & \better{0.185$\pm$0.052} & \better{0.116$\pm$0.012} & \better{0.247$\pm$0.017} \\
 Employment & 0.036$\pm$0.019 & \better{0.020$\pm$0.006} & \better{0.035$\pm$0.019} & \better{0.018$\pm$0.007} & \better{0.012$\pm$0.005} \\
Traffic & 0.066±0.031 & \worse{0.201±0.001} & \worse{0.109±0.028} & \worse{0.107±0.067} & \worse{0.113±0.073} \\
\hline
Win System 1 & NA & 4/84/8 & 6/86/8 & 4/84/8 & 5/85/8 \\
\hline
\end{tabular}
\end{subtable}
\end{table*}
 
\begin{table*}[b] % 让整个表格固定在页面底部
    \centering
     \label{tab:combined_UniLong}
    \caption{Main table title containing three subtables}
\begin{subtable}{\textwidth}
\centering
\caption{Results of Unimodal Long-term Time Series Forecasting with GPT-series Models.}
\label{tab:Uni_L_GPT}
\begin{tabular}{c|c|ccc|c} 
\hline
\multirow{2}{*}{Dataset}           & System 1    & \multicolumn{3}{c|}{~System 1 with~Test-time Reasoning Enhancement} & System~~~~ 2  \\ 
\cline{2-6}
                                   & GPT-4o      & with CoT     & with Self-Consistency & with Self-Correction  & o1-mini       \\ 
\hline
 Agriculture & 0.093±0.057 & \worse{0.920±1.134} & \better{0.057±0.011} & \better{0.068±0.018} & \worse{0.293±0.089} \\
 Climate & 0.754±0.051 & \worse{1.199±0.132} & \worse{0.811±0.081} & \worse{0.877±0.041} & \better{0.708±0.058} \\
 Economy & 0.463$\pm$0.146 & \worse{1.040$\pm$0.482} & \worse{0.620$\pm$0.116} & \worse{0.748$\pm$0.069} & \better{0.359$\pm$0.001} \\
 Energy & 0.197$\pm$0.038 & \worse{0.746$\pm$0.500} & \better{0.177$\pm$0.062} & \worse{0.296$\pm$0.153} & \worse{0.926$\pm$0.771} \\
 Flu & 0.219$\pm$0.053 & \worse{0.967$\pm$0.412} & \worse{0.230$\pm$0.077} & \worse{0.639$\pm$0.479} & \worse{0.862$\pm$0.597} \\
 Security & 0.183$\pm$0.044 & \better{0.162$\pm$0.038} & \better{0.135$\pm$0.011} & \better{0.165$\pm$0.017} & \worse{0.211$\pm$0.075} \\
 Employment & 0.011$\pm$0.006 & \worse{0.013$\pm$0.002} & \better{0.009$\pm$0.003} & \worse{0.013$\pm$0.004} & \worse{0.053$\pm$0.015} \\
Traffic & 0.066$\pm$0.046 & \worse{0.218$\pm$0.158} & \better{0.046$\pm$0.016} & \better{0.036$\pm$0.008} & \worse{0.091$\pm$0.042} \\
\hline
Win System 1 & NA & 1/8 & 5/8 & 3/8 & 2/8 \\
\hline
\end{tabular}
\end{subtable}
\begin{subtable}{\textwidth}
\centering
\caption{Results of Unimodal Long-term Time Series Forecasting with Gemini-series Models.}
\label{tab:Uni_L_Gemini}
\begin{tabular}{c|c|ccc|c} 
\hline
\multirow{2}{*}{Dataset}           & System 1    & \multicolumn{3}{c|}{~System 1 with~Test-time Reasoning Enhancement} & System~~~~ 2  \\ 
\cline{2-6}
                                   & Gemini-2.0-flash      & with CoT     & with Self-Consistency & with Self-Correction  & Gemini-2.0-flash-thinking       \\ 
\hline
 Agriculture & 0.032$\pm$0.007 & \worse{0.036$\pm$0.011} & \worse{0.035$\pm$0.007} & \worse{0.077$\pm$0.026} & \worse{0.093$\pm$0.018} \\
 Climate & 1.476$\pm$0.651 & \better{0.964$\pm$0.321} & \better{0.674$\pm$0.092} & \better{0.908$\pm$0.153} & \better{1.240$\pm$0.705} \\
 Economy & 0.092$\pm$0.038 & \worse{0.216$\pm$0.142} & \better{0.078$\pm$0.013} & \better{0.066$\pm$0.003} & \worse{0.244$\pm$0.035} \\
 Energy & 0.303$\pm$0.044 & \better{0.130$\pm$0.021} & \better{0.241$\pm$0.060} & \worse{0.489$\pm$0.134} & \better{0.241$\pm$0.148} \\
 Flu & 1.190$\pm$1.171 & \better{1.049$\pm$0.447} & \better{0.596$\pm$0.128} & \better{1.095$\pm$0.546} & \worse{1.920$\pm$0.001} \\
 Security & 0.196$\pm$0.052 & \worse{0.533$\pm$0.493} & \worse{0.955$\pm$0.389} & \better{0.154$\pm$0.031} & \worse{0.207$\pm$0.001} \\
 Employment & 0.011$\pm$0.001 & \worse{0.019$\pm$0.007} & \better{0.009$\pm$0.002} & \worse{0.013$\pm$0.005} & \worse{0.268$\pm$0.001} \\
Traffic & 0.068$\pm$0.063 & \worse{0.215$\pm$0.079} & \worse{0.074$\pm$0.048} & \better{0.050$\pm$0.013} & \worse{0.414$\pm$0.001} \\
\hline
Win System 1& NA & $3/8$ & $5/8$ & $5/8$ & $2/8$ \\
\hline
\end{tabular}
\end{subtable}
\begin{subtable}{\textwidth}
\centering
\caption{Long-term Unimodal forecasting (DeepSeek))}
\label{tab:performance_comparison_unimodal}
\begin{tabular}{c|c|ccc|c} 
\hline
\multirow{2}{*}{Dataset}           & System 1    & \multicolumn{3}{c|}{~System 1 with~Test-time Reasoning Enhancement} & System~~~~ 2  \\ 
\cline{2-6}
                                   & DeepSeek-V3      & with CoT     & with Self-Consistency & with Self-Correction  & DeepSeek-R1       \\ 
\hline
 Agriculture & 0.216$\pm$0.049 & \better{0.102$\pm$0.034} & \better{0.103$\pm$0.014} & \better{0.121$\pm$0.065} & \better{0.091$\pm$0.019} \\
 Climate & 0.902$\pm$0.001 & \worse{1.383$\pm$0.227} & \better{0.786$\pm$0.153} & \worse{0.913$\pm$0.078} & \better{0.662$\pm$0.051} \\
 Economy & 0.613$\pm$0.776 & \better{0.540$\pm$0.386} & \better{0.393$\pm$0.113} & \worse{0.948$\pm$0.589} & \better{0.359$\pm$0.001} \\
 Energy & 0.603$\pm$0.359 & \better{0.575$\pm$0.452} & \worse{0.923$\pm$0.265} & \better{0.332$\pm$0.150} & \worse{1.396$\pm$0.001} \\
 Flu & 0.841$\pm$0.215 & \better{0.658$\pm$0.227} & \better{0.538$\pm$0.021} & \worse{0.939$\pm$0.328} & \worse{0.972$\pm$0.533} \\
 Security & 0.275$\pm$0.060 & \better{0.245$\pm$0.039} & \worse{0.280$\pm$0.004} & \better{0.186$\pm$0.033} & \better{0.168$\pm$0.028} \\
 Employment & 0.051$\pm$0.013 & \better{0.021$\pm$0.002} & \better{0.039$\pm$0.006} & \better{0.023$\pm$0.003} & \better{0.021$\pm$0.001} \\
Traffic & 0.414$\pm$0.001 & \better{0.209$\pm$0.145} & \worse{94.305$\pm$66.620} & \better{0.306$\pm$0.153} & \better{0.158$\pm$0.181} \\
\hline
Win System 1 & NA & 7/8 & 5/8 & 5/8 & 6/8 \\
\hline
\end{tabular}
\end{subtable}
\end{table*}
 
\begin{table*}[b] % 让整个表格固定在页面底部
    \centering
     \label{tab:combined_mul_s}
    \caption{Main table title containing three subtables}
\begin{subtable}{\textwidth}
\centering
\caption{Results of Multimodal Short-term Time Series Forecasting with GPT-series Models.}
\label{tab:Multi_S_GPT}
\begin{tabular}{c|c|ccc|c} 
\hline
\multirow{2}{*}{Dataset}           & System 1    & \multicolumn{3}{c|}{~System 1 with~Test-time Reasoning Enhancement} & System~~~~ 2  \\ 
\cline{2-6}
                                   & GPT-4o      & with CoT     & with Self-Consistency & with Self-Correction  & o1-mini       \\ 
\hline
 Agriculture & 0.018$\pm$0.015 & \better{0.018$\pm$0.011} & \better{0.013$\pm$0.008} & \better{0.018$\pm$0.006} & \worse{0.045$\pm$0.056} \\
 Climate & 1.716$\pm$0.580 & \worse{1.920$\pm$0.505} & \better{1.712$\pm$0.191} & \worse{2.042$\pm$0.609} & \better{1.603$\pm$0.496} \\
 Economy & 0.569$\pm$0.162 & \worse{0.940$\pm$0.445} & \better{0.291$\pm$0.127} & \better{0.503$\pm$0.071} & \worse{0.583$\pm$0.001} \\
 Energy & 0.541$\pm$0.457 & \better{0.316$\pm$0.125} & \better{0.187$\pm$0.090} & \better{0.225$\pm$0.080} & \worse{0.718$\pm$0.786} \\
 Flu & 0.548$\pm$0.164 & \worse{1.071$\pm$0.643} & \better{0.288$\pm$0.071} & \worse{1.261$\pm$1.164} & \worse{0.983$\pm$1.177} \\
 Security & 0.076$\pm$0.052 & \worse{0.110$\pm$0.087} & \worse{0.146$\pm$0.025} & \worse{0.151$\pm$0.035} & \worse{0.244$\pm$0.020} \\
 Employment & 0.020$\pm$0.006 & \worse{0.020$\pm$0.003} & \better{0.019$\pm$0.003} & \worse{0.021$\pm$0.004} & \worse{0.028$\pm$0.008} \\
Traffic & 0.551$\pm$0.396 & \worse{1.577$\pm$1.421} & \better{0.030$\pm$0.010} & \better{0.347$\pm$0.349} & \worse{0.911$\pm$0.594} \\
\hline
Win System 1 & NA & $2/8$ & $7/8$ & $4/8$ & $1/8$ \\
\hline
\end{tabular}
\end{subtable}
\begin{subtable}{\textwidth}
\centering
\caption{Short-term Multimodal forecasting (Gemini))}
\label{tab:performance_comparison_multimodal}
\begin{tabular}{c|c|ccc|c} 
\hline
\multirow{2}{*}{Dataset}           & System 1    & \multicolumn{3}{c|}{~System 1 with~Test-time Reasoning Enhancement} & System~~~~ 2  \\ 
\cline{2-6}
                                   & Gemini-2.0-flash      & with CoT     & with Self-Consistency & with Self-Correction  & Gemini-2.0-flash-thinking       \\ 
\hline

 Agriculture & 0.010$\pm$0.003 & \better{0.006$\pm$0.001} & \better{0.009$\pm$0.002} & \worse{0.011$\pm$0.004} & \better{0.008$\pm$0.002} \\
 Climate & 2.115$\pm$0.660 & \better{1.725$\pm$0.227} & \better{1.980$\pm$0.760} & \better{1.529$\pm$0.290} & \better{2.106$\pm$0.294} \\
 Economy & 0.376$\pm$0.085 & \better{0.326$\pm$0.067} & \better{0.373$\pm$0.079} & \better{0.283$\pm$0.083} & \worse{0.509$\pm$0.109} \\
 Energy & 0.143$\pm$0.069 & \better{0.117$\pm$0.015} & \worse{0.143$\pm$0.027} & \better{0.091$\pm$0.065} & \worse{0.218$\pm$0.106} \\
 Flu & 0.594$\pm$0.219 & \worse{0.607$\pm$0.294} & \better{0.332$\pm$0.102} & \worse{1.422$\pm$0.542} & \worse{3.171$\pm$0.001} \\
 Security & 0.558$\pm$0.604 & \better{0.145$\pm$0.050} & \better{0.172$\pm$0.119} & \better{0.141$\pm$0.065} & \better{0.259$\pm$0.001} \\
 Employment & 0.013$\pm$0.002 & \worse{0.015$\pm$0.002} & \better{0.011$\pm$0.002} & \better{0.011$\pm$0.003} & \worse{0.311$\pm$0.001} \\
Traffic & 0.322$\pm$0.196 & \better{0.046$\pm$0.017} & \better{0.163$\pm$0.106} & \worse{0.425$\pm$0.235} & \better{0.201$\pm$0.001} \\
\hline
Win System 1 & NA & $6/8$ & $7/8$ & $5/8$ & $4/8$ \\
\hline
\end{tabular}
\end{subtable}
\begin{subtable}{\textwidth}
\centering
\caption{Short-term Multimodal forecasting (DeepSeek))}
\label{tab:performance_comparison_multimodal}
\begin{tabular}{c|c|ccc|c} 
\hline
\multirow{2}{*}{Dataset}           & System 1    & \multicolumn{3}{c|}{~System 1 with~Test-time Reasoning Enhancement} & System~~~~ 2  \\ 
\cline{2-6}
                                   & DeepSeek-V3      & with CoT     & with Self-Consistency & with Self-Correction  & DeepSeek-R1       \\ 
\hline
 Agriculture & 0.032$\pm$0.012 & \better{0.027$\pm$0.006} & \better{0.023$\pm$0.001} & \worse{0.042$\pm$0.025} & \worse{2.712$\pm$0.001} \\
 Climate & 1.428$\pm$0.432 & \worse{1.857$\pm$0.431} & \better{1.371$\pm$0.001} & \better{1.411$\pm$0.258} & \worse{2.235$\pm$0.850} \\
 Economy & 0.427$\pm$0.174 & \worse{0.598$\pm$0.069} & \better{0.306$\pm$0.005} & \better{0.369$\pm$0.128} & \worse{0.615$\pm$0.101} \\
 Energy & 0.253$\pm$0.089 & \worse{0.486$\pm$0.318} & \better{0.197$\pm$0.001} & \worse{0.505$\pm$0.339} & \worse{0.731$\pm$0.777} \\
 Flu & 1.073$\pm$0.447 & \worse{1.564$\pm$0.982} & \better{0.362$\pm$0.161} & \better{0.441$\pm$0.173} & \worse{1.329$\pm$1.306} \\
 Security & 0.186$\pm$0.001 & \worse{0.206$\pm$0.010} & \worse{0.187$\pm$0.001} & \better{0.130$\pm$0.018} & \better{0.161$\pm$0.051} \\
 Employment & 0.016$\pm$0.001 & \worse{0.022$\pm$0.003} & \worse{0.016$\pm$0.001} & \worse{0.016$\pm$0.001} & \worse{0.114$\pm$0.139} \\
Traffic & 0.201$\pm$0.001 & \worse{0.201$\pm$0.001} & \worse{0.201$\pm$0.001} & \better{0.114$\pm$0.063} & \better{0.153$\pm$0.069} \\
\hline
Win System 1 & NA & $1/8$ & $5/8$ & $5/8$ & $2/8$ \\
\hline
\end{tabular}
\end{subtable}
\end{table*}
 
\begin{table*}[b] % 让整个表格固定在页面底部
    \centering
     \label{tab:combined_mul_s}
    \caption{Main table title containing three subtables}
\begin{subtable}{\textwidth}
\centering
\caption{Results of Multimodal Long-term Time Series Forecasting with GPT-series Models.}
\label{tab:Multi_L_GPT}
\begin{tabular}{c|c|ccc|c} 
\hline
\multirow{2}{*}{Dataset}           & System 1    & \multicolumn{3}{c|}{~System 1 with~Test-time Reasoning Enhancement} & System~~~~ 2  \\ 
\cline{2-6}
                                   & GPT-4o      & with CoT     & with Self-Consistency & with Self-Correction  & o1-mini       \\ 
\hline
 Agriculture & 0.110$\pm$0.065 & \better{0.097$\pm$0.044} & \better{0.063$\pm$0.009} & \better{0.051$\pm$0.042} & \worse{0.210$\pm$0.022} \\
 Climate & 1.365$\pm$0.479 & \better{0.995$\pm$0.109} & \better{1.065$\pm$0.014} & \better{0.912$\pm$0.004} & \worse{1.549$\pm$0.566} \\
 Economy & 0.487$\pm$0.237 & \worse{1.027$\pm$0.321} & \worse{0.500$\pm$0.184} & \worse{0.543$\pm$0.074} & \worse{0.827$\pm$0.662} \\
 Energy & 0.365$\pm$0.185 & \better{0.254$\pm$0.122} & \worse{33.743$\pm$23.911} & \better{0.293$\pm$0.026} & \worse{0.707$\pm$0.499} \\
 Flu & 0.291$\pm$0.065 & \worse{0.369$\pm$0.058} & \worse{0.445$\pm$0.210} & \worse{0.529$\pm$0.365} & \worse{1.070$\pm$0.284} \\
 Security & 0.196$\pm$0.056 & \better{0.188$\pm$0.027} & \better{0.140$\pm$0.028} & \better{0.116$\pm$0.041} & \worse{0.207$\pm$0.001} \\
 Employment & 0.015$\pm$0.002 & \worse{0.021$\pm$0.007} & \worse{0.021$\pm$0.002} & \worse{0.106$\pm$0.115} & \worse{0.031$\pm$0.003} \\
Traffic & 0.207$\pm$0.205 & \worse{0.341$\pm$0.402} & \better{0.045$\pm$0.013} & \worse{0.377$\pm$0.504} & \worse{1.482$\pm$1.788} \\
\hline
Win System 1 & NA & 4/84/8 & 4/84/8 & 4/84/8 & 0/80/8 \\
\hline
\end{tabular}
\end{subtable}
\begin{subtable}{\textwidth}
\centering
\caption{Long-term Multimodal forecasting (Gemini))}
\label{tab:performance_comparison_multimodal}
\begin{tabular}{c|c|ccc|c} 
\hline
\multirow{2}{*}{Dataset}           & System 1    & \multicolumn{3}{c|}{~System 1 with~Test-time Reasoning Enhancement} & System~~~~ 2  \\ 
\cline{2-6}
                                   & Gemini-2.0-flash      & with CoT     & with Self-Consistency & with Self-Correction  & Gemini-2.0-flash-thinking       \\ 
\hline
 Agriculture & 0.052$\pm$0.026 & \better{0.034$\pm$0.009} & \better{0.034$\pm$0.006} & \better{0.024$\pm$0.007} & \worse{0.096$\pm$0.032} \\
 Climate & 1.644$\pm$0.398 & \better{1.452$\pm$0.461} & \better{1.318$\pm$0.079} & \better{1.292$\pm$0.401} & \better{1.006$\pm$0.327} \\
 Economy & 0.092$\pm$0.010 & \worse{0.234$\pm$0.049} & \worse{0.134$\pm$0.044} & \worse{10.357$\pm$14.475} & \worse{1.093$\pm$0.806} \\
 Energy & 0.138$\pm$0.106 & \worse{0.208$\pm$0.116} & \worse{0.159$\pm$0.077} & \worse{0.384$\pm$0.074} & \worse{0.713$\pm$0.513} \\
 Flu & 0.659$\pm$0.173 & \better{0.557$\pm$0.164} & \better{0.477$\pm$0.006} & \worse{0.785$\pm$0.064} & \worse{1.920$\pm$0.001} \\
 Security & 0.123$\pm$0.062 & \better{0.109$\pm$0.020} & \worse{0.142$\pm$0.043} & \worse{0.151$\pm$0.053} & \worse{0.207$\pm$0.001} \\
 Employment & 0.029$\pm$0.004 & \better{0.022$\pm$0.003} & \better{0.026$\pm$0.003} & \better{0.026$\pm$0.002} & \worse{0.268$\pm$0.001} \\
Traffic & 0.085$\pm$0.068 & \better{0.037$\pm$0.027} & \better{0.020$\pm$0.007} & \better{0.058$\pm$0.010} & \worse{0.414$\pm$0.001} \\
\hline
Win System 1 & NA & $6/8$ & $5/8$ & $4/8$ & $1/8$ \\
\hline
\end{tabular}
\end{subtable}
\begin{subtable}{\textwidth}
\centering
\caption{Long-term Multimodal forecasting (DeepSeek))}
\label{tab:performance_comparison_multimodal}
\begin{tabular}{c|c|ccc|c} 
\hline
\multirow{2}{*}{Dataset}           & System 1    & \multicolumn{3}{c|}{~System 1 with~Test-time Reasoning Enhancement} & System~~~~ 2  \\ 
\cline{2-6}
                                   & DeepSeek-V3      & with CoT     & with Self-Consistency & with Self-Correction  & DeepSeek-R1       \\ 
\hline
 Agriculture & 0.088$\pm$0.058 & \better{0.063$\pm$0.022} & \worse{0.136$\pm$0.080} & \worse{0.119$\pm$0.078} & \better{0.019$\pm$0.010} \\
 Climate & 0.897$\pm$0.001 & \worse{2.193$\pm$0.330} & \worse{0.897$\pm$0.001} & \worse{0.939$\pm$0.074} & \worse{1.849$\pm$0.570} \\
 Economy & 0.629$\pm$0.147 & \better{0.558$\pm$0.282} & \better{0.486$\pm$0.074} & \better{0.623$\pm$0.218} & \worse{0.806$\pm$0.354} \\
 Energy & 0.995$\pm$0.139 & \worse{1.286$\pm$0.568} & \better{0.809$\pm$0.241} & \better{0.493$\pm$0.112} & \better{0.746$\pm$0.459} \\
 Flu & 2.624$\pm$2.400 & \better{0.974$\pm$0.446} & \better{0.644$\pm$0.488} & \better{1.135$\pm$0.643} & \better{1.560$\pm$0.957} \\
 Security & 0.179$\pm$0.002 & \worse{0.250$\pm$0.024} & \better{0.156$\pm$0.027} & \worse{0.274$\pm$0.071} & \better{0.134$\pm$0.055} \\
 Employment & 0.034$\pm$0.001 & \better{0.029$\pm$0.008} & \worse{0.034$\pm$0.001} & \better{0.030$\pm$0.005} & \worse{0.105$\pm$0.115} \\
Traffic & 0.414$\pm$0.001 & \worse{0.414$\pm$0.001} & \worse{0.414$\pm$0.001} & \better{0.192$\pm$0.157} & \better{0.152$\pm$0.185} \\
\hline
Win System 1 & NA & $4/8$ & $4/8$ & $5/8$ & $5/8$ \\
\hline
\end{tabular}
\end{subtable}
\end{table*}
\newpage


\section{Conclusions}
\vspace{-0.25cm}
We proposed the Chain of Rank (CoR) to address the limitations of the existing intricate reasoning processes like chain-of-thought in training-based, domain-specific RAG. For domain-specific RAG training, annotation expense for the reasoning data is required. Also, especially in testing on smaller LLMs in resource-constrained environments, it poses challenges in terms of the accuracy as well as computational cost. We observed that the inaccurate reasoning adversely affect the quality of final answer. By shifting the focus from elaborate reasoning to a simplified ranking of the reliability of retrieved documents, CoR significantly reduced computational complexity while attaining higher accuracy. Our experimental results demonstrated that CoR achieves SOTA results on RAG benchmarks, confirming its effectiveness in improving the domain-specific RAG performance of small-scale LLMs. %Furthermore, we showed that CoR is well-suited for specialized domains, offering a more efficient alternative to traditional reasoning methods without sacrificing precision. 
%Our findings also highlight the potential of CoR to enhance various RAG-based solutions, especially in the environments where resource efficiency is critical, such as edge devices.


\section{Limitations} 
\label{sec:limitation}
This work acknowledges the significance of reasoning in domain-specific RAG models and presents an efficient approach that reduces the need for complex training data labeling and significantly lowers reasoning costs during testing. However, we did not thoroughly investigate whether the proposed method would be equally effective in more general RAG frameworks that do not rely on task-specific training. That said, preliminary results presented in the appendix indicate the potential for success in general RAG settings, suggesting that this area warrants deeper exploration in future work. Therefore, our findings provide a promising foundation for future research.


\section{Ethical Consideration}
In the field of domain-specific RAG, if the applications involve sensitive areas such as personal information, special caution must be taken during the model training process to ensure privacy and data protection. Beyond this consideration, methodologically, our research focuses on improving the accuracy and efficiency of RAG in LLMs, we do not foresee any direct negative ethical concerns stemming from our contributions. Nonetheless, it is important to recognize that generative AI technologies, including those using LLMs, come with potential risks. As such, careful consideration of their broader ethical and societal implications is necessary when these systems are applied in the real world.

% Entries for the entire Anthology, followed by custom entries
% \bibliography{anthology,custom}
\bibliography{bib}

\newpage
\subsection{Lloyd-Max Algorithm}
\label{subsec:Lloyd-Max}
For a given quantization bitwidth $B$ and an operand $\bm{X}$, the Lloyd-Max algorithm finds $2^B$ quantization levels $\{\hat{x}_i\}_{i=1}^{2^B}$ such that quantizing $\bm{X}$ by rounding each scalar in $\bm{X}$ to the nearest quantization level minimizes the quantization MSE. 

The algorithm starts with an initial guess of quantization levels and then iteratively computes quantization thresholds $\{\tau_i\}_{i=1}^{2^B-1}$ and updates quantization levels $\{\hat{x}_i\}_{i=1}^{2^B}$. Specifically, at iteration $n$, thresholds are set to the midpoints of the previous iteration's levels:
\begin{align*}
    \tau_i^{(n)}=\frac{\hat{x}_i^{(n-1)}+\hat{x}_{i+1}^{(n-1)}}2 \text{ for } i=1\ldots 2^B-1
\end{align*}
Subsequently, the quantization levels are re-computed as conditional means of the data regions defined by the new thresholds:
\begin{align*}
    \hat{x}_i^{(n)}=\mathbb{E}\left[ \bm{X} \big| \bm{X}\in [\tau_{i-1}^{(n)},\tau_i^{(n)}] \right] \text{ for } i=1\ldots 2^B
\end{align*}
where to satisfy boundary conditions we have $\tau_0=-\infty$ and $\tau_{2^B}=\infty$. The algorithm iterates the above steps until convergence.

Figure \ref{fig:lm_quant} compares the quantization levels of a $7$-bit floating point (E3M3) quantizer (left) to a $7$-bit Lloyd-Max quantizer (right) when quantizing a layer of weights from the GPT3-126M model at a per-tensor granularity. As shown, the Lloyd-Max quantizer achieves substantially lower quantization MSE. Further, Table \ref{tab:FP7_vs_LM7} shows the superior perplexity achieved by Lloyd-Max quantizers for bitwidths of $7$, $6$ and $5$. The difference between the quantizers is clear at 5 bits, where per-tensor FP quantization incurs a drastic and unacceptable increase in perplexity, while Lloyd-Max quantization incurs a much smaller increase. Nevertheless, we note that even the optimal Lloyd-Max quantizer incurs a notable ($\sim 1.5$) increase in perplexity due to the coarse granularity of quantization. 

\begin{figure}[h]
  \centering
  \includegraphics[width=0.7\linewidth]{sections/figures/LM7_FP7.pdf}
  \caption{\small Quantization levels and the corresponding quantization MSE of Floating Point (left) vs Lloyd-Max (right) Quantizers for a layer of weights in the GPT3-126M model.}
  \label{fig:lm_quant}
\end{figure}

\begin{table}[h]\scriptsize
\begin{center}
\caption{\label{tab:FP7_vs_LM7} \small Comparing perplexity (lower is better) achieved by floating point quantizers and Lloyd-Max quantizers on a GPT3-126M model for the Wikitext-103 dataset.}
\begin{tabular}{c|cc|c}
\hline
 \multirow{2}{*}{\textbf{Bitwidth}} & \multicolumn{2}{|c|}{\textbf{Floating-Point Quantizer}} & \textbf{Lloyd-Max Quantizer} \\
 & Best Format & Wikitext-103 Perplexity & Wikitext-103 Perplexity \\
\hline
7 & E3M3 & 18.32 & 18.27 \\
6 & E3M2 & 19.07 & 18.51 \\
5 & E4M0 & 43.89 & 19.71 \\
\hline
\end{tabular}
\end{center}
\end{table}

\subsection{Proof of Local Optimality of LO-BCQ}
\label{subsec:lobcq_opt_proof}
For a given block $\bm{b}_j$, the quantization MSE during LO-BCQ can be empirically evaluated as $\frac{1}{L_b}\lVert \bm{b}_j- \bm{\hat{b}}_j\rVert^2_2$ where $\bm{\hat{b}}_j$ is computed from equation (\ref{eq:clustered_quantization_definition}) as $C_{f(\bm{b}_j)}(\bm{b}_j)$. Further, for a given block cluster $\mathcal{B}_i$, we compute the quantization MSE as $\frac{1}{|\mathcal{B}_{i}|}\sum_{\bm{b} \in \mathcal{B}_{i}} \frac{1}{L_b}\lVert \bm{b}- C_i^{(n)}(\bm{b})\rVert^2_2$. Therefore, at the end of iteration $n$, we evaluate the overall quantization MSE $J^{(n)}$ for a given operand $\bm{X}$ composed of $N_c$ block clusters as:
\begin{align*}
    \label{eq:mse_iter_n}
    J^{(n)} = \frac{1}{N_c} \sum_{i=1}^{N_c} \frac{1}{|\mathcal{B}_{i}^{(n)}|}\sum_{\bm{v} \in \mathcal{B}_{i}^{(n)}} \frac{1}{L_b}\lVert \bm{b}- B_i^{(n)}(\bm{b})\rVert^2_2
\end{align*}

At the end of iteration $n$, the codebooks are updated from $\mathcal{C}^{(n-1)}$ to $\mathcal{C}^{(n)}$. However, the mapping of a given vector $\bm{b}_j$ to quantizers $\mathcal{C}^{(n)}$ remains as  $f^{(n)}(\bm{b}_j)$. At the next iteration, during the vector clustering step, $f^{(n+1)}(\bm{b}_j)$ finds new mapping of $\bm{b}_j$ to updated codebooks $\mathcal{C}^{(n)}$ such that the quantization MSE over the candidate codebooks is minimized. Therefore, we obtain the following result for $\bm{b}_j$:
\begin{align*}
\frac{1}{L_b}\lVert \bm{b}_j - C_{f^{(n+1)}(\bm{b}_j)}^{(n)}(\bm{b}_j)\rVert^2_2 \le \frac{1}{L_b}\lVert \bm{b}_j - C_{f^{(n)}(\bm{b}_j)}^{(n)}(\bm{b}_j)\rVert^2_2
\end{align*}

That is, quantizing $\bm{b}_j$ at the end of the block clustering step of iteration $n+1$ results in lower quantization MSE compared to quantizing at the end of iteration $n$. Since this is true for all $\bm{b} \in \bm{X}$, we assert the following:
\begin{equation}
\begin{split}
\label{eq:mse_ineq_1}
    \tilde{J}^{(n+1)} &= \frac{1}{N_c} \sum_{i=1}^{N_c} \frac{1}{|\mathcal{B}_{i}^{(n+1)}|}\sum_{\bm{b} \in \mathcal{B}_{i}^{(n+1)}} \frac{1}{L_b}\lVert \bm{b} - C_i^{(n)}(b)\rVert^2_2 \le J^{(n)}
\end{split}
\end{equation}
where $\tilde{J}^{(n+1)}$ is the the quantization MSE after the vector clustering step at iteration $n+1$.

Next, during the codebook update step (\ref{eq:quantizers_update}) at iteration $n+1$, the per-cluster codebooks $\mathcal{C}^{(n)}$ are updated to $\mathcal{C}^{(n+1)}$ by invoking the Lloyd-Max algorithm \citep{Lloyd}. We know that for any given value distribution, the Lloyd-Max algorithm minimizes the quantization MSE. Therefore, for a given vector cluster $\mathcal{B}_i$ we obtain the following result:

\begin{equation}
    \frac{1}{|\mathcal{B}_{i}^{(n+1)}|}\sum_{\bm{b} \in \mathcal{B}_{i}^{(n+1)}} \frac{1}{L_b}\lVert \bm{b}- C_i^{(n+1)}(\bm{b})\rVert^2_2 \le \frac{1}{|\mathcal{B}_{i}^{(n+1)}|}\sum_{\bm{b} \in \mathcal{B}_{i}^{(n+1)}} \frac{1}{L_b}\lVert \bm{b}- C_i^{(n)}(\bm{b})\rVert^2_2
\end{equation}

The above equation states that quantizing the given block cluster $\mathcal{B}_i$ after updating the associated codebook from $C_i^{(n)}$ to $C_i^{(n+1)}$ results in lower quantization MSE. Since this is true for all the block clusters, we derive the following result: 
\begin{equation}
\begin{split}
\label{eq:mse_ineq_2}
     J^{(n+1)} &= \frac{1}{N_c} \sum_{i=1}^{N_c} \frac{1}{|\mathcal{B}_{i}^{(n+1)}|}\sum_{\bm{b} \in \mathcal{B}_{i}^{(n+1)}} \frac{1}{L_b}\lVert \bm{b}- C_i^{(n+1)}(\bm{b})\rVert^2_2  \le \tilde{J}^{(n+1)}   
\end{split}
\end{equation}

Following (\ref{eq:mse_ineq_1}) and (\ref{eq:mse_ineq_2}), we find that the quantization MSE is non-increasing for each iteration, that is, $J^{(1)} \ge J^{(2)} \ge J^{(3)} \ge \ldots \ge J^{(M)}$ where $M$ is the maximum number of iterations. 
%Therefore, we can say that if the algorithm converges, then it must be that it has converged to a local minimum. 
\hfill $\blacksquare$


\begin{figure}
    \begin{center}
    \includegraphics[width=0.5\textwidth]{sections//figures/mse_vs_iter.pdf}
    \end{center}
    \caption{\small NMSE vs iterations during LO-BCQ compared to other block quantization proposals}
    \label{fig:nmse_vs_iter}
\end{figure}

Figure \ref{fig:nmse_vs_iter} shows the empirical convergence of LO-BCQ across several block lengths and number of codebooks. Also, the MSE achieved by LO-BCQ is compared to baselines such as MXFP and VSQ. As shown, LO-BCQ converges to a lower MSE than the baselines. Further, we achieve better convergence for larger number of codebooks ($N_c$) and for a smaller block length ($L_b$), both of which increase the bitwidth of BCQ (see Eq \ref{eq:bitwidth_bcq}).


\subsection{Additional Accuracy Results}
%Table \ref{tab:lobcq_config} lists the various LOBCQ configurations and their corresponding bitwidths.
\begin{table}
\setlength{\tabcolsep}{4.75pt}
\begin{center}
\caption{\label{tab:lobcq_config} Various LO-BCQ configurations and their bitwidths.}
\begin{tabular}{|c||c|c|c|c||c|c||c|} 
\hline
 & \multicolumn{4}{|c||}{$L_b=8$} & \multicolumn{2}{|c||}{$L_b=4$} & $L_b=2$ \\
 \hline
 \backslashbox{$L_A$\kern-1em}{\kern-1em$N_c$} & 2 & 4 & 8 & 16 & 2 & 4 & 2 \\
 \hline
 64 & 4.25 & 4.375 & 4.5 & 4.625 & 4.375 & 4.625 & 4.625\\
 \hline
 32 & 4.375 & 4.5 & 4.625& 4.75 & 4.5 & 4.75 & 4.75 \\
 \hline
 16 & 4.625 & 4.75& 4.875 & 5 & 4.75 & 5 & 5 \\
 \hline
\end{tabular}
\end{center}
\end{table}

%\subsection{Perplexity achieved by various LO-BCQ configurations on Wikitext-103 dataset}

\begin{table} \centering
\begin{tabular}{|c||c|c|c|c||c|c||c|} 
\hline
 $L_b \rightarrow$& \multicolumn{4}{c||}{8} & \multicolumn{2}{c||}{4} & 2\\
 \hline
 \backslashbox{$L_A$\kern-1em}{\kern-1em$N_c$} & 2 & 4 & 8 & 16 & 2 & 4 & 2  \\
 %$N_c \rightarrow$ & 2 & 4 & 8 & 16 & 2 & 4 & 2 \\
 \hline
 \hline
 \multicolumn{8}{c}{GPT3-1.3B (FP32 PPL = 9.98)} \\ 
 \hline
 \hline
 64 & 10.40 & 10.23 & 10.17 & 10.15 &  10.28 & 10.18 & 10.19 \\
 \hline
 32 & 10.25 & 10.20 & 10.15 & 10.12 &  10.23 & 10.17 & 10.17 \\
 \hline
 16 & 10.22 & 10.16 & 10.10 & 10.09 &  10.21 & 10.14 & 10.16 \\
 \hline
  \hline
 \multicolumn{8}{c}{GPT3-8B (FP32 PPL = 7.38)} \\ 
 \hline
 \hline
 64 & 7.61 & 7.52 & 7.48 &  7.47 &  7.55 &  7.49 & 7.50 \\
 \hline
 32 & 7.52 & 7.50 & 7.46 &  7.45 &  7.52 &  7.48 & 7.48  \\
 \hline
 16 & 7.51 & 7.48 & 7.44 &  7.44 &  7.51 &  7.49 & 7.47  \\
 \hline
\end{tabular}
\caption{\label{tab:ppl_gpt3_abalation} Wikitext-103 perplexity across GPT3-1.3B and 8B models.}
\end{table}

\begin{table} \centering
\begin{tabular}{|c||c|c|c|c||} 
\hline
 $L_b \rightarrow$& \multicolumn{4}{c||}{8}\\
 \hline
 \backslashbox{$L_A$\kern-1em}{\kern-1em$N_c$} & 2 & 4 & 8 & 16 \\
 %$N_c \rightarrow$ & 2 & 4 & 8 & 16 & 2 & 4 & 2 \\
 \hline
 \hline
 \multicolumn{5}{|c|}{Llama2-7B (FP32 PPL = 5.06)} \\ 
 \hline
 \hline
 64 & 5.31 & 5.26 & 5.19 & 5.18  \\
 \hline
 32 & 5.23 & 5.25 & 5.18 & 5.15  \\
 \hline
 16 & 5.23 & 5.19 & 5.16 & 5.14  \\
 \hline
 \multicolumn{5}{|c|}{Nemotron4-15B (FP32 PPL = 5.87)} \\ 
 \hline
 \hline
 64  & 6.3 & 6.20 & 6.13 & 6.08  \\
 \hline
 32  & 6.24 & 6.12 & 6.07 & 6.03  \\
 \hline
 16  & 6.12 & 6.14 & 6.04 & 6.02  \\
 \hline
 \multicolumn{5}{|c|}{Nemotron4-340B (FP32 PPL = 3.48)} \\ 
 \hline
 \hline
 64 & 3.67 & 3.62 & 3.60 & 3.59 \\
 \hline
 32 & 3.63 & 3.61 & 3.59 & 3.56 \\
 \hline
 16 & 3.61 & 3.58 & 3.57 & 3.55 \\
 \hline
\end{tabular}
\caption{\label{tab:ppl_llama7B_nemo15B} Wikitext-103 perplexity compared to FP32 baseline in Llama2-7B and Nemotron4-15B, 340B models}
\end{table}

%\subsection{Perplexity achieved by various LO-BCQ configurations on MMLU dataset}


\begin{table} \centering
\begin{tabular}{|c||c|c|c|c||c|c|c|c|} 
\hline
 $L_b \rightarrow$& \multicolumn{4}{c||}{8} & \multicolumn{4}{c||}{8}\\
 \hline
 \backslashbox{$L_A$\kern-1em}{\kern-1em$N_c$} & 2 & 4 & 8 & 16 & 2 & 4 & 8 & 16  \\
 %$N_c \rightarrow$ & 2 & 4 & 8 & 16 & 2 & 4 & 2 \\
 \hline
 \hline
 \multicolumn{5}{|c|}{Llama2-7B (FP32 Accuracy = 45.8\%)} & \multicolumn{4}{|c|}{Llama2-70B (FP32 Accuracy = 69.12\%)} \\ 
 \hline
 \hline
 64 & 43.9 & 43.4 & 43.9 & 44.9 & 68.07 & 68.27 & 68.17 & 68.75 \\
 \hline
 32 & 44.5 & 43.8 & 44.9 & 44.5 & 68.37 & 68.51 & 68.35 & 68.27  \\
 \hline
 16 & 43.9 & 42.7 & 44.9 & 45 & 68.12 & 68.77 & 68.31 & 68.59  \\
 \hline
 \hline
 \multicolumn{5}{|c|}{GPT3-22B (FP32 Accuracy = 38.75\%)} & \multicolumn{4}{|c|}{Nemotron4-15B (FP32 Accuracy = 64.3\%)} \\ 
 \hline
 \hline
 64 & 36.71 & 38.85 & 38.13 & 38.92 & 63.17 & 62.36 & 63.72 & 64.09 \\
 \hline
 32 & 37.95 & 38.69 & 39.45 & 38.34 & 64.05 & 62.30 & 63.8 & 64.33  \\
 \hline
 16 & 38.88 & 38.80 & 38.31 & 38.92 & 63.22 & 63.51 & 63.93 & 64.43  \\
 \hline
\end{tabular}
\caption{\label{tab:mmlu_abalation} Accuracy on MMLU dataset across GPT3-22B, Llama2-7B, 70B and Nemotron4-15B models.}
\end{table}


%\subsection{Perplexity achieved by various LO-BCQ configurations on LM evaluation harness}

\begin{table} \centering
\begin{tabular}{|c||c|c|c|c||c|c|c|c|} 
\hline
 $L_b \rightarrow$& \multicolumn{4}{c||}{8} & \multicolumn{4}{c||}{8}\\
 \hline
 \backslashbox{$L_A$\kern-1em}{\kern-1em$N_c$} & 2 & 4 & 8 & 16 & 2 & 4 & 8 & 16  \\
 %$N_c \rightarrow$ & 2 & 4 & 8 & 16 & 2 & 4 & 2 \\
 \hline
 \hline
 \multicolumn{5}{|c|}{Race (FP32 Accuracy = 37.51\%)} & \multicolumn{4}{|c|}{Boolq (FP32 Accuracy = 64.62\%)} \\ 
 \hline
 \hline
 64 & 36.94 & 37.13 & 36.27 & 37.13 & 63.73 & 62.26 & 63.49 & 63.36 \\
 \hline
 32 & 37.03 & 36.36 & 36.08 & 37.03 & 62.54 & 63.51 & 63.49 & 63.55  \\
 \hline
 16 & 37.03 & 37.03 & 36.46 & 37.03 & 61.1 & 63.79 & 63.58 & 63.33  \\
 \hline
 \hline
 \multicolumn{5}{|c|}{Winogrande (FP32 Accuracy = 58.01\%)} & \multicolumn{4}{|c|}{Piqa (FP32 Accuracy = 74.21\%)} \\ 
 \hline
 \hline
 64 & 58.17 & 57.22 & 57.85 & 58.33 & 73.01 & 73.07 & 73.07 & 72.80 \\
 \hline
 32 & 59.12 & 58.09 & 57.85 & 58.41 & 73.01 & 73.94 & 72.74 & 73.18  \\
 \hline
 16 & 57.93 & 58.88 & 57.93 & 58.56 & 73.94 & 72.80 & 73.01 & 73.94  \\
 \hline
\end{tabular}
\caption{\label{tab:mmlu_abalation} Accuracy on LM evaluation harness tasks on GPT3-1.3B model.}
\end{table}

\begin{table} \centering
\begin{tabular}{|c||c|c|c|c||c|c|c|c|} 
\hline
 $L_b \rightarrow$& \multicolumn{4}{c||}{8} & \multicolumn{4}{c||}{8}\\
 \hline
 \backslashbox{$L_A$\kern-1em}{\kern-1em$N_c$} & 2 & 4 & 8 & 16 & 2 & 4 & 8 & 16  \\
 %$N_c \rightarrow$ & 2 & 4 & 8 & 16 & 2 & 4 & 2 \\
 \hline
 \hline
 \multicolumn{5}{|c|}{Race (FP32 Accuracy = 41.34\%)} & \multicolumn{4}{|c|}{Boolq (FP32 Accuracy = 68.32\%)} \\ 
 \hline
 \hline
 64 & 40.48 & 40.10 & 39.43 & 39.90 & 69.20 & 68.41 & 69.45 & 68.56 \\
 \hline
 32 & 39.52 & 39.52 & 40.77 & 39.62 & 68.32 & 67.43 & 68.17 & 69.30  \\
 \hline
 16 & 39.81 & 39.71 & 39.90 & 40.38 & 68.10 & 66.33 & 69.51 & 69.42  \\
 \hline
 \hline
 \multicolumn{5}{|c|}{Winogrande (FP32 Accuracy = 67.88\%)} & \multicolumn{4}{|c|}{Piqa (FP32 Accuracy = 78.78\%)} \\ 
 \hline
 \hline
 64 & 66.85 & 66.61 & 67.72 & 67.88 & 77.31 & 77.42 & 77.75 & 77.64 \\
 \hline
 32 & 67.25 & 67.72 & 67.72 & 67.00 & 77.31 & 77.04 & 77.80 & 77.37  \\
 \hline
 16 & 68.11 & 68.90 & 67.88 & 67.48 & 77.37 & 78.13 & 78.13 & 77.69  \\
 \hline
\end{tabular}
\caption{\label{tab:mmlu_abalation} Accuracy on LM evaluation harness tasks on GPT3-8B model.}
\end{table}

\begin{table} \centering
\begin{tabular}{|c||c|c|c|c||c|c|c|c|} 
\hline
 $L_b \rightarrow$& \multicolumn{4}{c||}{8} & \multicolumn{4}{c||}{8}\\
 \hline
 \backslashbox{$L_A$\kern-1em}{\kern-1em$N_c$} & 2 & 4 & 8 & 16 & 2 & 4 & 8 & 16  \\
 %$N_c \rightarrow$ & 2 & 4 & 8 & 16 & 2 & 4 & 2 \\
 \hline
 \hline
 \multicolumn{5}{|c|}{Race (FP32 Accuracy = 40.67\%)} & \multicolumn{4}{|c|}{Boolq (FP32 Accuracy = 76.54\%)} \\ 
 \hline
 \hline
 64 & 40.48 & 40.10 & 39.43 & 39.90 & 75.41 & 75.11 & 77.09 & 75.66 \\
 \hline
 32 & 39.52 & 39.52 & 40.77 & 39.62 & 76.02 & 76.02 & 75.96 & 75.35  \\
 \hline
 16 & 39.81 & 39.71 & 39.90 & 40.38 & 75.05 & 73.82 & 75.72 & 76.09  \\
 \hline
 \hline
 \multicolumn{5}{|c|}{Winogrande (FP32 Accuracy = 70.64\%)} & \multicolumn{4}{|c|}{Piqa (FP32 Accuracy = 79.16\%)} \\ 
 \hline
 \hline
 64 & 69.14 & 70.17 & 70.17 & 70.56 & 78.24 & 79.00 & 78.62 & 78.73 \\
 \hline
 32 & 70.96 & 69.69 & 71.27 & 69.30 & 78.56 & 79.49 & 79.16 & 78.89  \\
 \hline
 16 & 71.03 & 69.53 & 69.69 & 70.40 & 78.13 & 79.16 & 79.00 & 79.00  \\
 \hline
\end{tabular}
\caption{\label{tab:mmlu_abalation} Accuracy on LM evaluation harness tasks on GPT3-22B model.}
\end{table}

\begin{table} \centering
\begin{tabular}{|c||c|c|c|c||c|c|c|c|} 
\hline
 $L_b \rightarrow$& \multicolumn{4}{c||}{8} & \multicolumn{4}{c||}{8}\\
 \hline
 \backslashbox{$L_A$\kern-1em}{\kern-1em$N_c$} & 2 & 4 & 8 & 16 & 2 & 4 & 8 & 16  \\
 %$N_c \rightarrow$ & 2 & 4 & 8 & 16 & 2 & 4 & 2 \\
 \hline
 \hline
 \multicolumn{5}{|c|}{Race (FP32 Accuracy = 44.4\%)} & \multicolumn{4}{|c|}{Boolq (FP32 Accuracy = 79.29\%)} \\ 
 \hline
 \hline
 64 & 42.49 & 42.51 & 42.58 & 43.45 & 77.58 & 77.37 & 77.43 & 78.1 \\
 \hline
 32 & 43.35 & 42.49 & 43.64 & 43.73 & 77.86 & 75.32 & 77.28 & 77.86  \\
 \hline
 16 & 44.21 & 44.21 & 43.64 & 42.97 & 78.65 & 77 & 76.94 & 77.98  \\
 \hline
 \hline
 \multicolumn{5}{|c|}{Winogrande (FP32 Accuracy = 69.38\%)} & \multicolumn{4}{|c|}{Piqa (FP32 Accuracy = 78.07\%)} \\ 
 \hline
 \hline
 64 & 68.9 & 68.43 & 69.77 & 68.19 & 77.09 & 76.82 & 77.09 & 77.86 \\
 \hline
 32 & 69.38 & 68.51 & 68.82 & 68.90 & 78.07 & 76.71 & 78.07 & 77.86  \\
 \hline
 16 & 69.53 & 67.09 & 69.38 & 68.90 & 77.37 & 77.8 & 77.91 & 77.69  \\
 \hline
\end{tabular}
\caption{\label{tab:mmlu_abalation} Accuracy on LM evaluation harness tasks on Llama2-7B model.}
\end{table}

\begin{table} \centering
\begin{tabular}{|c||c|c|c|c||c|c|c|c|} 
\hline
 $L_b \rightarrow$& \multicolumn{4}{c||}{8} & \multicolumn{4}{c||}{8}\\
 \hline
 \backslashbox{$L_A$\kern-1em}{\kern-1em$N_c$} & 2 & 4 & 8 & 16 & 2 & 4 & 8 & 16  \\
 %$N_c \rightarrow$ & 2 & 4 & 8 & 16 & 2 & 4 & 2 \\
 \hline
 \hline
 \multicolumn{5}{|c|}{Race (FP32 Accuracy = 48.8\%)} & \multicolumn{4}{|c|}{Boolq (FP32 Accuracy = 85.23\%)} \\ 
 \hline
 \hline
 64 & 49.00 & 49.00 & 49.28 & 48.71 & 82.82 & 84.28 & 84.03 & 84.25 \\
 \hline
 32 & 49.57 & 48.52 & 48.33 & 49.28 & 83.85 & 84.46 & 84.31 & 84.93  \\
 \hline
 16 & 49.85 & 49.09 & 49.28 & 48.99 & 85.11 & 84.46 & 84.61 & 83.94  \\
 \hline
 \hline
 \multicolumn{5}{|c|}{Winogrande (FP32 Accuracy = 79.95\%)} & \multicolumn{4}{|c|}{Piqa (FP32 Accuracy = 81.56\%)} \\ 
 \hline
 \hline
 64 & 78.77 & 78.45 & 78.37 & 79.16 & 81.45 & 80.69 & 81.45 & 81.5 \\
 \hline
 32 & 78.45 & 79.01 & 78.69 & 80.66 & 81.56 & 80.58 & 81.18 & 81.34  \\
 \hline
 16 & 79.95 & 79.56 & 79.79 & 79.72 & 81.28 & 81.66 & 81.28 & 80.96  \\
 \hline
\end{tabular}
\caption{\label{tab:mmlu_abalation} Accuracy on LM evaluation harness tasks on Llama2-70B model.}
\end{table}

%\section{MSE Studies}
%\textcolor{red}{TODO}


\subsection{Number Formats and Quantization Method}
\label{subsec:numFormats_quantMethod}
\subsubsection{Integer Format}
An $n$-bit signed integer (INT) is typically represented with a 2s-complement format \citep{yao2022zeroquant,xiao2023smoothquant,dai2021vsq}, where the most significant bit denotes the sign.

\subsubsection{Floating Point Format}
An $n$-bit signed floating point (FP) number $x$ comprises of a 1-bit sign ($x_{\mathrm{sign}}$), $B_m$-bit mantissa ($x_{\mathrm{mant}}$) and $B_e$-bit exponent ($x_{\mathrm{exp}}$) such that $B_m+B_e=n-1$. The associated constant exponent bias ($E_{\mathrm{bias}}$) is computed as $(2^{{B_e}-1}-1)$. We denote this format as $E_{B_e}M_{B_m}$.  

\subsubsection{Quantization Scheme}
\label{subsec:quant_method}
A quantization scheme dictates how a given unquantized tensor is converted to its quantized representation. We consider FP formats for the purpose of illustration. Given an unquantized tensor $\bm{X}$ and an FP format $E_{B_e}M_{B_m}$, we first, we compute the quantization scale factor $s_X$ that maps the maximum absolute value of $\bm{X}$ to the maximum quantization level of the $E_{B_e}M_{B_m}$ format as follows:
\begin{align}
\label{eq:sf}
    s_X = \frac{\mathrm{max}(|\bm{X}|)}{\mathrm{max}(E_{B_e}M_{B_m})}
\end{align}
In the above equation, $|\cdot|$ denotes the absolute value function.

Next, we scale $\bm{X}$ by $s_X$ and quantize it to $\hat{\bm{X}}$ by rounding it to the nearest quantization level of $E_{B_e}M_{B_m}$ as:

\begin{align}
\label{eq:tensor_quant}
    \hat{\bm{X}} = \text{round-to-nearest}\left(\frac{\bm{X}}{s_X}, E_{B_e}M_{B_m}\right)
\end{align}

We perform dynamic max-scaled quantization \citep{wu2020integer}, where the scale factor $s$ for activations is dynamically computed during runtime.

\subsection{Vector Scaled Quantization}
\begin{wrapfigure}{r}{0.35\linewidth}
  \centering
  \includegraphics[width=\linewidth]{sections/figures/vsquant.jpg}
  \caption{\small Vectorwise decomposition for per-vector scaled quantization (VSQ \citep{dai2021vsq}).}
  \label{fig:vsquant}
\end{wrapfigure}
During VSQ \citep{dai2021vsq}, the operand tensors are decomposed into 1D vectors in a hardware friendly manner as shown in Figure \ref{fig:vsquant}. Since the decomposed tensors are used as operands in matrix multiplications during inference, it is beneficial to perform this decomposition along the reduction dimension of the multiplication. The vectorwise quantization is performed similar to tensorwise quantization described in Equations \ref{eq:sf} and \ref{eq:tensor_quant}, where a scale factor $s_v$ is required for each vector $\bm{v}$ that maps the maximum absolute value of that vector to the maximum quantization level. While smaller vector lengths can lead to larger accuracy gains, the associated memory and computational overheads due to the per-vector scale factors increases. To alleviate these overheads, VSQ \citep{dai2021vsq} proposed a second level quantization of the per-vector scale factors to unsigned integers, while MX \citep{rouhani2023shared} quantizes them to integer powers of 2 (denoted as $2^{INT}$).

\subsubsection{MX Format}
The MX format proposed in \citep{rouhani2023microscaling} introduces the concept of sub-block shifting. For every two scalar elements of $b$-bits each, there is a shared exponent bit. The value of this exponent bit is determined through an empirical analysis that targets minimizing quantization MSE. We note that the FP format $E_{1}M_{b}$ is strictly better than MX from an accuracy perspective since it allocates a dedicated exponent bit to each scalar as opposed to sharing it across two scalars. Therefore, we conservatively bound the accuracy of a $b+2$-bit signed MX format with that of a $E_{1}M_{b}$ format in our comparisons. For instance, we use E1M2 format as a proxy for MX4.

\begin{figure}
    \centering
    \includegraphics[width=1\linewidth]{sections//figures/BlockFormats.pdf}
    \caption{\small Comparing LO-BCQ to MX format.}
    \label{fig:block_formats}
\end{figure}

Figure \ref{fig:block_formats} compares our $4$-bit LO-BCQ block format to MX \citep{rouhani2023microscaling}. As shown, both LO-BCQ and MX decompose a given operand tensor into block arrays and each block array into blocks. Similar to MX, we find that per-block quantization ($L_b < L_A$) leads to better accuracy due to increased flexibility. While MX achieves this through per-block $1$-bit micro-scales, we associate a dedicated codebook to each block through a per-block codebook selector. Further, MX quantizes the per-block array scale-factor to E8M0 format without per-tensor scaling. In contrast during LO-BCQ, we find that per-tensor scaling combined with quantization of per-block array scale-factor to E4M3 format results in superior inference accuracy across models. 



\end{document}
