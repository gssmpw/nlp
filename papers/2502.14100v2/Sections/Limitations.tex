\section{Limitations}
In this work, we primarily focus on enhancing LLMs' context-robust performance when processing single context-question pairs. Our current approach demonstrates effectiveness in improving contextual understanding, though there remain promising directions for future exploration. Specifically, we aim to investigate more sophisticated relationships, both internal and external, between different contexts to further enhance model performance. While our experimental evaluation centers on Llama-2-7B and Llama-3-8B-instruct, we plan to extend our analysis to a broader range of models to validate the generalizability of our approach and identify potential model-specific optimizations.
% In this work, we have demonstrated that the representations of LLMs can significantly enhance the robustness of RAG systems. However, the underlying mechanisms by which LLMs identify, utilize, and integrate external knowledge with their internal knowledge remain an open research question. Our framework employs \textit{Rep-PCA} and introduces \textit{Rep-Contra} for context analysis. While these methods have shown promising results, we aim to explore more sophisticated analytical approaches. It is important to note that a significant challenge lies beyond the scope of our current work: determining the correctness of context when the LLM itself lacks knowledge about the question at hand. This presents a more complex problem, and we posit that external sources may be necessary, as LLMs' self-signals alone may not be sufficient to fully address this challenge.


% \section{Ethic Statement}

% This work explores using synthetic data to mitigate privacy risks in Retrieval-Augmented Generation (RAG), particularly in safety-critical domains. We argue that protecting sensitive information is crucial, as data leakage can severely impact individuals' well-being and privacy rights.
% Our approach generates synthetic data to replace sensitive data during RAG, aiming to reduce privacy breach risks. We have adhered to ethical guidelines and acknowledge the need for further research to understand the risks and benefits of our method.
% Developing privacy-preserving techniques is essential for the responsible deployment of RAG systems. Our research contributes to balancing the benefits of advanced language models with the protection of individual privacy rights.
% In our research, we concentrated primarily on the application of retrieval augmentation during the inference stage, without delving into its integration during pre-training or fine-tuning phases. Future work will aim to explore these compelling areas. Moreover, while our study has highlighted the privacy risks associated with commonly employed retrieval-augmented generation (RAG) systems, other retrieval-based language models (LMs) feature distinct components and architectures \cite{huang2023privacy,borgeaud2022improving} that warrant further investigation. In addition, developing effective strategies like differential privacy(\cite{feyisetan2020privacy,utpala2023locally,carvalho2023tem}) to protect retrieval data and leveraging RAG systems for the safeguarding of training data represent open research questions that we intend to pursue.