\documentclass[runningheads]{llncs}
\usepackage[T1]{fontenc}
% T1 fonts will be used to generate the final print and online PDFs,
% so please use T1 fonts in your manuscript whenever possible.
% Other font encondings may result in incorrect characters.

% packages added by authors
\usepackage{hyperref}
\usepackage{amsmath}
\usepackage{amssymb}
\usepackage{algorithm}
\usepackage{algorithmicx}
\usepackage{algpseudocode}
\usepackage{booktabs}
\usepackage{multirow}
\usepackage{xcolor}
\usepackage{subcaption}
\usepackage{enumitem}
\usepackage{graphicx}

\newcommand{\red}[1]{{\color{red}#1}}


% Used for displaying a sample figure. If possible, figure files should
% be included in EPS format.
%
% If you use the hyperref package, please uncomment the following two lines
% to display URLs in blue roman font according to Springer's eBook style:
%\usepackage{color}
%\renewcommand\UrlFont{\color{blue}\rmfamily}
%\urlstyle{rm}
%
\begin{document}
%
\title{Graph Neural Networks at a Fraction}

%\titlerunning{Abbreviated paper title}
% If the paper title is too long for the running head, you can set
% an abbreviated paper title here

\author{
Rucha Bhalchandra Joshi\inst{1}\thanks{This research was done while R. Joshi was a student at NISER, Bhubaneswar\inst{1} and HBNI, Mumbai\inst{2}.} \and
Sagar Prakash Barad\inst{2}\inst{3} \and
Nidhi Tiwari\inst{4} \and
Subhankar Mishra\inst{2}\inst{3}
}

\authorrunning{R. Joshi et al.}
% First names are abbreviated in the running head.
% If there are more than two authors, 'et al.' is used.

\institute{
The Cyprus Institute, Nicosia, Cyprus \and
National Institute of Science Education and Research, Bhubaneswar, India \and
Homi Bhabha National Institute, Mumbai, India \and
Microsoft Ltd., India
\\
\email{r.joshi@cyi.ac.cy, sagar.barad@niser.ac.in, nidhitiwari@microsoft.com, smishra@niser.ac.in}\\
% \url{http://www.cyi.ac.cy} \and
% \url{http://www.niser.ac.in} \and
% \url{http://www.microsoft.com}
}
%
\maketitle  

\begin{abstract}
Graph Neural Networks (GNNs) have emerged as powerful tools for learning representations of graph-structured data. In addition to real-valued GNNs, quaternion GNNs also perform well on tasks on graph-structured data. With the aim of reducing the energy footprint, we reduce the model size while maintaining accuracy comparable to that of the original-sized GNNs.
This paper introduces Quaternion Message Passing Neural Networks (QMPNNs), a framework that leverages quaternion space to compute node representations. Our approach offers a generalizable method for incorporating quaternion representations into GNN architectures at one-fourth of the original parameter count.
Furthermore, we present a novel perspective on Graph Lottery Tickets, redefining their applicability within the context of GNNs and QMPNNs. We specifically aim to find the initialization lottery from the subnetwork of the GNNs that can achieve comparable performance to the original GNN upon training. Thereby reducing the trainable model parameters even further.
To validate the effectiveness of our proposed QMPNN framework and LTH for both GNNs and QMPNNs, we evaluate their performance on real-world datasets across three fundamental graph-based tasks: node classification, link prediction, and graph classification. Our code is available at project's \href{https://github.com/SagarPrakashBarad/QuatGLT}{GitHub repository}.

\keywords{Graph Neural Networks  \and Pruning \and Quaternion Graph Neural Networks \and Lottery Ticket Hypothesis.}
\end{abstract}
%
%
%

\section{Introduction}
\label{sec:introduction}
Quaternions, a hypercomplex number system, offer several advantages over traditional real and complex numbers, making them well-suited for various tasks in deep learning. Incorporating quaternions into deep neural networks holds significant promise for enhancing their expressive power and versatility. When used in a quaternion setting, Quaternion deep neural networks perform better than their natural counterparts \cite{parcollet2020survey}. The added advantage of the quaternions is that they reduce the degree of freedom of parameters by one-fourth, which effectively means that only one-fourth of the parameters must be tuned. The inherent multidimensional nature of quaternions, coupled with their capacity to diminish parameter counts significantly, suggests that hyper-complex numbers are more suitable than real numbers for crafting more efficient models in multidimensional spaces with lesser trainable parameter counts.

Most GNNs learn the embeddings in the Euclidean space. 
A recent approach \cite{nguyen2021quaternion} proposes learning graph embeddings in Quaternion space. However, it is the quaternion version of one model GCN \cite{kipf2016semi}, and it cannot be generalized further. The GNN variants exploit different interesting properties of the graphs to learn the representations efficiently. A quaternion framework should be able to maintain these characteristics of the GNN model and embrace them in the quaternion space. With this realization, in this work, we present a generalizable quaternion framework that can adapt to any GNN variant.

Additionally, unlike in the case of the non-GNN deep neural networks \cite{iqbal2023neural,mukhopadhyay2024large,mukhopadhyay2025transformers}, the existing QGNNs do not leverage the quaternion space to reduce the number of parameters. In these non-GNN methods, the input features are usually divided in such a way that overall the four components of the quaternion vector can aggregate them and make a meaningful transformation out of it. The previous work \cite{nguyen2021quaternion}, however, uses each quaternion component to work with all the features. This adds four times the computational overhead, making the model expensive in terms of training and inference time. Additionally, it makes the outputs very sensitive to even a tiny change in the input, as this type of quaternion GNN amplifies the tiny change.  To this end, we restrict our quaternion model from quadrupling the number of parameters, thereby reducing the trainable parameter count to one-fourth of that of real, and this also prohibits the output from being extremely sensitive to the tiny change in the input features. 

% Quaternion representations generated in the current works are more expressive because every feature is represented using four coefficients. Tiny changes in the input feature may get amplified in the representations \cite{nguyen2021quaternion}. Hence, even with a few quaternion GNN layers, the number of parameters increases to almost four times the real GNN model. This means that the expressiveness comes at the cost of an increase in computation complexity. Essentially, it means the model becomes expensive in terms of training and inference time. In addition, if the input graphs are dense and large, and if the Quaternion Graph Neural Networks are deep for some reason, the model will be even slower.  In this work, we use quaternions to reduce the trainable parameters in GNNs by one-fourth. 

\begin{figure*}[ht]
    \centering
    \includegraphics[width=\linewidth]{figs/qmpnn_diff.pdf}
    \caption{Pruning a quaternion message passing neural network. Graph's features are first transformed into quaternion features. We train QMPNN, which has quaternion weights, for the given task. Furthermore, to find our proposed winning lottery ticket, we prune and train the QMPNN until we get a model with pruned quaternion weights that gives a comparable accuracy. The pruned quaternion network can be trained with only a small fraction of the parameters in the real GNN.
    % \red{We should mark here reduction in number of parameters?} check
    }
    \label{fig:qmpnn}
    % \Description[Quaternion Message Passing Neural Networks]{Pruning a quaternion message passing neural network. Graph's features are first transformed into quaternion features. We train QMPNN, which has quaternion weights, for the given task. Furthermore, to find our proposed winning lottery ticket, we prune and train the QMPNN until we get a model with pruned quaternion weights that gives a comparable accuracy.}
\end{figure*}


In this work, our goal is to attain Quaternion Graph Neural Networks that are expressive and have fewer parameters. To do this, we \emph{first} propose a generic framework to get the Quaternion model corresponding to any GNN model with real representations. \emph{Secondly}, we present a framework to sparsify the Quaternion GNNs so that we reduce the number of parameters by finding a lottery ticket.
We generalize the Lottery Ticket Hypothesis \cite{frankle2018lottery} to Quaternion Graph Neural Networks.  For graphs, the existing unified graph lottery ticket, which prunes the input graph as well as the GNN to obtain the winning graph lottery ticket, has some drawbacks. Primarily, it is not feasible to prune the input graph along with the GNNs in graph-level tasks. UGS lacks effectiveness for transfer learning purposes because of its dependency on the input graph. We tackle this by redefining the lottery ticket for graphs. Figure \ref{fig:qmpnn} presents an overview of our approach.


% We experimentally verify our approach on various graph tasks, GNN models, and real-world datasets. The tasks that we test our proposal on are node classification, graph classification, and link prediction. We showcase the results using quaternion versions of the popular GNN models GCN \cite{kipf2016semi}, GAT \cite{velivckovic2018graph} and GraphSAGE \cite{hamilton2017inductive} on real-world datasets Cora, PubMed, Citeseer, ogbn-arxiv, ogbl-collab for Node Classification and Link Prediction and PROTEINS, ENZYMES, MUTAG, ogbg-molhiv for Graph Classification.

To summarize, this paper makes contributions as follows:
% \begin{enumerate}
%     \item Quaternion Message Passing Neural Networks (QMPNNs): We present a generalizable framework for computing quaternion space representations using any GNN, reducing trainable parameters by 75\%.
    % Quaternion Message Passing Neural Networks (QMPNNs): We present a generalizable framework that enables the computation of representations in quaternion space using any GNN, reducing the number of trainable parameters to one-fourth of the original.
%     \item We redefine the graph lottery tickets from a different perspective for GNNs and QMPNNs.
%     \item To our knowledge, we are the first to empirically demonstrate the existence of graph lottery tickets in QMPNNs.
%     \item We provide the performance evaluation of the proposed LTH on both GNNs and QMPNNs on real-world datasets for three graph-based tasks - node classification, link prediction, and graph classification. 
% \end{enumerate}
(1) Quaternion Message Passing Neural Networks (QMPNNs): We present a generalizable framework for computing quaternion space representations using any GNN, reducing trainable parameters by 75\%.
(2) We redefine the graph lottery tickets from a different perspective for GNNs and QMPNNs.
(3) To our knowledge, we are the first to empirically demonstrate the existence of graph lottery tickets in QMPNNs.
(4) We provide the performance evaluation of the proposed LTH on both GNNs and QMPNNs on real-world datasets for three graph-based tasks - node classification, link prediction, and graph classification. 

% This paper is structured as follows: Section \ref{sec:introduction} outlines the research problem, objectives, and contributions. Section \ref{sec:related} describes the related works in detail. In section \ref{sec:background}, we give the background for quaternions and message-passing graph neural networks. In section \ref{sec:method_qmpnn}, we describe our proposed framework QMPNNs, and \ref{sec:method_lth} describe the LTH on QMPNNs. All details regarding experiments and their results are given in section \ref{sec:expt}. Finally, we conclude the paper and discuss future directions in section \ref{sec:conclusion}. 


% \section{Related Work}
Alongside a discussion of what is meant by LLM harmfulness,
this section covers two distinct strands of related work: measuring types of harm in LLMs, and LLMs for diverse annotation tasks. %First,

%Different kinds of 
Diverse undesirable LLM outputs, from toxic language to privacy invasion, have been discussed in the observed \cite{banko-etal-2020-unified}. Here we review the ones we include in our definition of ``harm.'' %definition. Plus, we review LLMs as judges. 
Toxic content can be elicited from both generative  \cite{deshpande2023toxicity} and masked LLMs \cite{ousidhoum-etal-2021-probing}. 
%Among ways 
To measure toxic or hateful language, some use APIs such as PerspectiveAPI \cite{lees2022new} or HateBERT \cite{caselli-etal-2021-hatebert}. \citet{openai2024gpt4technicalreport} report that GPT4 produces toxic content 0.78\% of the time, versus 6.48\% in GPT3.5.
%as opposed to GPT3.5 with 6.48\%. On the other hand,
\citet{dubey2024llama} report that llama3-70B produces harmful content 5\% of the time, %whereas the 405B model generates harm 3\% of the time. 
compared to 3\% in the 405B model.
Instead of %single value classifiers to measure harm, 
reporting an absolute rate, we focus on relative harmfulness of different LLMs. %, so we point to recent work on LLMs for annotation.

The first category of harm we consider is social stereotyping and bias. %discrimination. It has been shown that 
LLMs can perpetuate social bias based on gender, race, religion etc. \cite{lin-etal-2022-gendered,bender2021dangers,field-etal-2021-survey,gupta-etal-2024-sociodemographic,andriushchenko2024agentharm,mazeika2024harmbench}. This can marginalize these groups more, and results in less fair model performance. \citet{guo2024hey} designed a competition to elicit biased output from LLMs to assess the perception of bias from non-expert users. %The first part of our work is similar to this analysis, but 
We also intentionally elicit harmful output, going %we look at other types of harms besides bias.
beyond social bias.

%When the models become stronger, they become more robust to jailbreaking attacks to elicit harmful content. However, there are datasets that can still jailbreak models to produce harmful content \cite{andriushchenko2024agentharm,mazeika2024harmbench}.

Our second category of harm is offensiveness and toxicity, which %. As opposed to stereotyping or social discrimination, this harm 
%is more subjective and harder to define than the previous category, so there 
lacks an established definition due to its greater subjectivity \cite{dev-etal-2022-measures,korre-etal-2023-harmful}. We include hate speech (HS) and abusive language as toxic content. HS can be defined as expressions of offensive and discriminatory discourse towards a group or an individual based on characteristics such as race, religion, nationality, or other group characteristics \cite{john2000hate,jahan2023systematic,basile2019semeval,davidson2017automated}. It includes racism, negative stereotyping, and sexist language. On the other hand, abusive language is content with inappropriate words such as profanity or disrespectful terms. It also includes psychological threats such as humiliation. %or constant criticism. %Toxic content can be elicited from both generative models \cite{deshpande2023toxicity} and masked language models \cite{ousidhoum-etal-2021-probing}.

%In addition to obvious toxic content, LLMs can generate diverse implicit toxic outputs using reinforcement learning with favoring toxic content in the reward function \cite{wen-etal-2023-unveiling}.  Regarding the subjectivity of this task, \cite{korre-etal-2023-harmful} reannotate the existing datasets with different definitions of toxicity and show that broader definitions result in more robust annotations, but interannotator agreements are still lower than 0.5. \cite{dev-etal-2022-measures} also point out the lack of definition for bias and harm in general and propose a framework to guide researchers during the development of bias measures.

Harm can be implicit, such as privacy invasion
%We are also interested in privacy invasion,
where there is 
leakage of personal information. %leakage from the model. 
%LLMs can memorize details of the training data and then leak private information such as 
This includes social security numbers, phone numbers, or bank account information \cite{carlini2021extracting,brown2022does}. 
%There are several frameworks to test the privacy of LLMs \cite{li2024llm} and generate data for personal attribute inference \cite{yukhymenko2024synthetic,kim2024propile}.

%Our definition of harm includes hate speech (HS) as well. HS can be defined as \textcolor{red}{expressions of} hatred towards a social group, the humiliation of the members of a group, or %communication disparaging  extreme disparagement of a person or a group based on race, color, ethnicity, gender, sexual orientation, nationality, religion, or other group characteristics .

For data annotation, LLMs
%Besides text generation, 
%LLMs have been used to annotate data because they 
can %be comparable to 
replace humans for some tasks, %and make the annotation process faster and cheaper 
with gains in efficiency and economy \cite{tan2024large}. They have been used for sociological annotations such as for classification of stance, bots or humor  \cite{ziems2024can,zhu2023can}. For tasks such as topic and frame detection or sentence segmentation they can surpass crowd-workers
%Some works show that they can surpass crowd-workers for some tasks such as topic and frame detection or sentence segmentation %into research aspects 
\cite{he2024if,gilardi2023chatgpt}. Some have argued that human-LLM collaboration results in more reliable annotation \cite{he2024if,zhang2023llmaaa,kim2024meganno+}. In addition to more objective tasks,
%LLMs have been used to annotate data %even 
they have been applied to subjective annotations such as offensiveness and abusiveness \cite{pavlovic-poesio-2024-effectiveness,zhu2023can,he2023annollm}, %. For example, LLMs are used as judges to rank responses from different LLMs 
or to rank outputs from different LLMs based on helpfulness, accuracy, or relevance \cite{zheng2023judging,lin2024wildbench,dubois2024length}. These works tend to focus on human-large LLM interactions, whereas we focus on single-turn responses from smaller LLMs. We inspire from \citet{zheng2023judging} but we only measure harm instead of overall performance. Plus, we use 3 LLMs to evaluate smaller LLMs.


\section{Background}
\label{sec:background}

\subsection{Notations and Definitions}
We use the following notations in this paper.
A graph $\mathcal{G} = \{ \mathcal{V}, \mathcal{E}\}$ has a set of vertices $\mathcal{V}$ and a set of edges $\mathcal{E}$. The feature vector corresponding to a node $v \in \mathcal{V}$ is given as $\mathrm{x}_v$ in $\mathbb{R}^F$. The feature matrix for the graph is given as $\mathbf{X} \in \mathrm{R}^{|\mathcal{V}| \times F}$. The edge $e_{ij} = (v_i, v_j) \in \mathcal{E}$ if and only if there is an edge between two nodes $v_i$ and $v_j$. In the adjacency matrix representation $\mathbf{A} \in \mathbb{R}^{|\mathcal{V}| \times |\mathcal{V}|}$ of the graph, the $\mathbf{A}_{ij} = 1$ if $e_{ij} \in \mathcal{E}$, and $0$ otherwise.

The intermediate representations in layer $l$ corresponding to a node $v$ are given by $h_v^{(l)}$. Similarly, we denote representations in layer $l$ for node $v$ by $h_v^{(l), Q}$, using the additional superscript $Q$ to represent quaternion. 


\subsection{Quaternions}
\label{sec:quaternions}
The set of quaternions is denoted by $\mathbb{H}$. A quaternion $q$ is denoted as $q = q_r + q_i \mathrm{i} + q_j \mathrm{j} + q_k \mathrm{k}$, where $q_r, q_i, q_j, q_k \in \mathbb{R}$ and $\mathrm{i}, \mathrm{j}, \mathrm{k}$ are imaginary units. They follow the relation $\mathrm{i}^2 = \mathrm{j}^2 = \mathrm{k}^2 = \mathrm{ijk} = -1$. 

$\mathbf{Addition}$: The quaternion addition of two quaternoons $q$ and $p$ is component-wise: $(q_r + q_i \mathrm{i} + q_j \mathrm{j} + q_k \mathrm{k}) + (p_r + p_i \mathrm{i} + p_j \mathrm{j} + p_k \mathrm{k}) = (q_r + p_r) + (q_i + p_i) \mathrm{i} + (q_j + p_j) \mathrm{j} + (q_k + p_k) \mathrm{k}$. 

$\mathbf{Multiplication}$: Quaternions can be multiplied by each other. The multiplication is associative, i.e., $(pq)r = p(qr)$ for $p, q, r \in \mathbb{H}$. It is also distributive, i.e., $ (p+q)r = pr + qr$. However, the multiplication is not commutative, i.e., $pq \neq qp$. When multiplied with the scalar $\lambda$, it gives $\lambda q = \lambda q_r + \lambda q_i \mathrm{i} + \lambda q_j \mathrm{j} + \lambda q_k \mathrm{k}$. Multiplication of two quaternions $q$ and $p$ is defined by the Hamilton product $q\otimes p$. 
% It is given as follows:
% \begin{align*} 
%     q \otimes p &= (q_r + q_i \mathrm{i} + q_j \mathrm{j} + q_k \mathrm{k}) \otimes (p_r + p_i \mathrm{i} + p_j \mathrm{j} + p_k \mathrm{k}) \\
%                 &= (q_r p_r - q_i p_i - q_j p_j - q_k p_k) + 
%                     (q_i p_r + q_r p_i - q_k p_j + q_j p_k) \mathrm{i} \\
%                 &+  (q_j p_r + q_k p_i + q_r p_j - q_i p_k) \mathrm{j} +
%                     (q_k p_r - q_j p_i + q_i p_j + q_r p_k) \mathrm{k}
% \end{align*}
This is given in the matrix form as follows:
\begin{gather}
\label{eq:qmult}
    q \otimes p = 
    \begin{bmatrix}
        1 \\ \mathrm{i} \\ \mathrm{j} \\ \mathrm{k}
    \end{bmatrix}^\top 
    \begin{bmatrix}
        q_r & -q_i & -q_j & -q_k \\
        q_i & q_r & -q_k & q_j \\
        q_j & q_k & q_r & -q_i \\
        q_k & -q_j & q_i & q_r
    \end{bmatrix}
    \begin{bmatrix}
        p_r \\ p_i \\ p_j \\ p_k
    \end{bmatrix}
\end{gather}

$\mathbf{Conjugation}$: $\Bar{q} = q_r - q_i \mathrm{i} - q_j \mathrm{j} - q_k \mathrm{k}$ is the conjugation of the quaternion $q$ given above.

$\mathbf{Norm}$: The norm of quaternion $q$ is given as $\lVert q \rVert = \sqrt{q_r^2 + q_i^2 + q_j^2 + q_k^2}$. 

% \subsection{Message Passing Graph Neural Networks}

% Graph Neural Networks (GNNs) are neural networks that analyze graph-structured data and make predictions on the components of the input graph. They work by updating node representations by aggregating information from neighboring nodes, enabling them to capture both local and global structures in the graph by using multiple GNN layers. Many GNNs are based on this message-passing framework, where the information from the neighbor is a message. 
% \begin{align}
%     m_{uv}     &= \textsf{MESSAGE} (h^{(l)}_u, h^{(l)}_v, e_{uv}) \label{eq:message} \\
%     h'_u    &= \textsf{AGGREGATE} (m_{uv}), \forall v \in \mathcal{N}(u) \label{eq:aggregate} \\
%     h^{(l+1)} &= \textsf{UPDATE} (h^{(l)}_u, h'_u) \label{eq:update}
% \end{align}
% In the equation \ref{eq:message}, the \textsf{MESSAGE} function generates a message based on the representations $h^{(l)}_u$ and $h^{(l)}_v$ of node $u$ and its neighboring node $v$ respectively, and the features $e_{uv}$ of the edge connecting the two nodes, if any. These messages from all the nodes $v$ in the neighborhood $\mathcal{N}(u)$ of node $u$ are aggregated by the function \textsf{AGGREGATE}. Node $u$'s subsequent layer representations are then updated by the \textsf{UPDATE} function that acts upon the current layer representation $h^{(l)}_u$ of $u$ and the aggregated message $h'_u$ from its neighbors. The functions \textsf{MESSAGE} and \textsf{UPDATE} are differentiable, and \textsf{AGGREGATE} is a permutation invariant differentiable aggregation function. Depending on the particular variant of GNNs, these functions can be learnable. The parameters are learned using back-propagation, for which the differentiability is an essential property.

\section{Quaternion Message Passing Neural Networks}
% \red{How are these different from QGNNs} check
\label{sec:method_qmpnn}

Graph Neural Networks (GNNs) analyze graph-structured data by updating node representations through message passing. At each layer, messages from neighboring nodes are aggregated and used to update node features. The process is formalized as:
\begin{align}
    m_{uv}     &= \textsf{MESSAGE} (h^{(l)}_u, h^{(l)}_v, e_{uv}) \label{eq:message} \\
    h'_u       &= \textsf{AGGREGATE} (m_{uv}), \forall v \in \mathcal{N}(u) \label{eq:aggregate} \\
    h^{(l+1)} &= \textsf{UPDATE} (h^{(l)}_u, h'_u) \label{eq:update}
\end{align}
Here, \textsf{MESSAGE} computes messages based on node features $h^{(l)}_u, h^{(l)}_v$ and edge features $e_{uv}$; \textsf{AGGREGATE} collects messages from neighbors $\mathcal{N}(u)$; and \textsf{UPDATE} refines the node representation. These functions, often learnable and differentiable, enable GNNs to be trained via backpropagation.


Quaternion Graph Neural Networks (QGNNs) refer to neural networks that utilize quaternion weights to perform computations within the quaternion vector space. Notably, QGNN, as developed by \cite{nguyen2021quaternion} is a quaternion variant of the Graph Convolutional Network (GCN) layer only. 
% \red{QGNN not generalized to other GNNs, \#parameters remain the same, to address this challenge, we propose / their implementation has two limitations}
This is the first limitation of QGNN, as it cannot be generalized to other GNN variants. The second limitation of QGNN is by construction, number of trainable parameters remain the same as that of real-valued GCN.
While leveraging the quaternion space to generate expressive representations for nodes, we also seek to take advantage of the quaternions by reducing the number of trainable parameters. To this end, we propose a generalized framework to give the quaternion equivalent of any graph neural network method.

In the quaternion variant of any GNNs, in the equations \ref{eq:message}, \ref{eq:aggregate}, \ref{eq:update}, we ensure that the \textsf{MESSAGE}, \textsf{AGGREGATE} and \textsf{UPDATE} functions follow the quaternion operations as given in section \ref{sec:quaternions}. We describe this in more detail below:

To begin, we have the features represented using the quaternions. \(F\) is the number of features corresponding to every node. To use these features with quaternions, we make sure that \(F\) is divisible by four so that we can associate \(F/4\) real-valued features with a real and three imaginary components of quaternion. Consequently, the feature vector \(h_v^{(l), Q}\) of node \(v\) in quaternion form is represented as \(h_v^{(l), Q} = h^{(l)}_r + h^{(l)}_i \mathrm{i} + h^{(l)}_j \mathrm{j} + h^{(l)}_k \mathrm{k}\).

In every layer \( l\) of GNN, the learnable functions have associated parameters, which we represent using \( W^{(l), Q}\). This weight matrix \(W^{(l), Q}\) has consists of four real-valued components \( W^{(l)}_r,\) \(W^{(l)}_i,\) \(W^{(l)}_j,\) \(W^{(l)}_k \) of the quaternion weight matrix. Specifically, \( W^{(l), Q} = W^{(l)}_r + W^{(l)}_i \mathrm{i} + W^{(l)}_j \mathrm{j} + W^{(l)}_k \mathrm{k} \).

The weights are multiplied with feature maps using the quaternion multiplication rule given in equation \ref{eq:qmult}. Following this, it is done as follows:
\begin{equation}
\label{eq:feat_mult}
    W^{(l+1), Q} = W^{(l), Q} \otimes h_v^{(l), Q}
\end{equation}

\begin{gather}
\label{eq:quat_mat}
    \begin{bmatrix}
        \mathcal{R}(W^{(l), Q}) \\
        \mathcal{I}(W^{(l), Q}) \\
        \mathcal{J}(W^{(l), Q}) \\
        \mathcal{K}(W^{(l), Q})
    \end{bmatrix} =
    \begin{bmatrix}
        W^{(l)}_r & -W^{(l)}_i & -W^{(l)}_j & -W^{(l)}_k \\
        W^{(l)}_i & W^{(l)}_r & -W^{(l)}_k & W^{(l)}_j \\
        W^{(l)}_j & W^{(l)}_k & W^{(l)}_r & -W^{(l)}_i \\
        W^{(l)}_k & -W^{(l)}_j & W^{(l)}_i & W^{(l)}_r \\
    \end{bmatrix} \times
    \begin{bmatrix}
        h^{(l)}_r \\ 
        h^{(l)}_i \\
        h^{(l)}_j \\
        h^{(l)}_k \\
    \end{bmatrix}
\end{gather}
where \(\mathcal{R, I, J, K}\) denote the four imaginary components of the product of weights with the node representations. 

Due to quaternion weights being multiplied with quaternion features, the degrees of freedom are reduced by one-fourth. The product in equation \ref{eq:feat_mult} combines components \( (r, i, j, k) \), capturing complex interdependencies. Each quaternion weight, as shown in equation \ref{eq:quat_mat}, encodes spatial correlations across features, resulting in more expressive networks compared to real-valued counterparts~\cite{trabelsi2018deep}. Additionally, the trainable parameters are reduced to one-fourth of those in an equivalent GNN with real weights, aiding in model size reduction.


% \red{We need to have a clarity here stating that reduces the number of parameters by 4. But yes these are quaternion parameters. } check

\subsection{Differentiability}
For the QMPNNs to learn, the quaternion versions of learnable functions from \textsf{MESSAGE}, \textsf{AGGREGATE}, and \textsf{UPDATE} should be differentiable. For our framework, the differentiability follows from the generalized complex chain rule for a real-valued loss function, which is provided in much detail in Deep Complex Networks \cite{trabelsi2018deep} and Deep Quaternion Networks \cite{gaudet2018deep}. 

\subsection{Computational Complexity}
Four feature values are clubbed together to form a single quaternion in QMPNNs, hence \(F/4\) quaternions are necessary to transform the feature vector. The transformed feature space is of dimension \(F'\). While using quaternion weights, these are considered as the combination of \(F'/4\) quaternions. The number of parameters of the quaternion weight matrix, hence, is \(F/4 \times F'/4\), with each of them having four quaternion components. Since
the components \( W_r^{(l)}, W_i^{(l)}, W_j^{(l)},\) and \( W_k^{(l)}\) of the quaternion weight \( W^{(l), Q}\) are shared during the Hamilton product, the degree of freedom is reduced by one-fourth of that of the real weight matrix. 
For quaternion matrix multiplications has same time complexity as that of the real ones, we reduce the number of trainable parameters without compromising on the time complexity due to use of quaternion parameters.
\\
% \red{rewrite entirely}
% Note that the quaternion components of \( W^{(l),Q} \) are shared across the four quaternion components \(h^{(l)}_r,  h^{(l)}_i, h^{(l)}_j, h^{(l)}_k\). 
% Therefore, any slight change in the input \( h^{(l), Q} \) results in an entirely different output, which is reflected in the model performance. This phenomenon is one of the crucial reasons why the quaternion space provides highly expressive computations through the Hamilton product compared to Euclidean and complex vector spaces. It enforces the model to learn potential relations within each hidden layer and between different hidden layers, thereby increasing the representation quality. Furthermore, the four quaternion components \( W^{(l)}_r, W^{(l)}_i, W^{(l)}_j\), and  \(W^{(l)}_k \) are shared during the Hamilton product. In contrast, in the Euclidean space, all elements of the weight matrix are different parameter variables. Thus, we can maintain the same complexity and reduce the number of model parameters up to four times within the Quaternion space, similar to the parameter saving reported in the existing literature \cite{iqbal2023neural, mukhopadhyay2024large}.

\section{Lottery Ticket Hypothesis on QMPNNs}
\label{sec:method_lth}

\emph{The lottery ticket hypothesis} (LTH) \cite{frankle2018lottery} states that \emph{a randomly initialized, dense neural network contains a sub-network that is initialized such that—when trained in isolation—it can match the test accuracy of the original network after training for at most the same number of iterations.}

% We extend the LTH to Graph Neural Networks and Quaternion GNNs. 
Different from the lottery tickets defined previously \cite{chen2021unified,tsitsulin2023the,wang2023searching,zhang2024graph} - that consider a subgraph of a graph with or without the subnetwork of the GNN as a ticket, we define the LTH for Graph Neural Networks and Quaternion GNNs. 
In a graph neural network, \( f(x; W) \), with initial parameters \( W^{(0)} \), using Stochastic Gradient Descent (SGD), we train on the training set, minimizes to a minimum validation loss \( l \) in \( j \) number of iterations achieving accuracy \( a\). The lottery ticket hypothesis is that there exists \(m\) when optimizing with SGD on the same training set, \( f\) attains the test accuracy \(a' \geq a\) with a validation loss \(l' \leq l\) in \(j' \leq j\) iterations, such that \( \lVert m \rVert \ll \lVert W \rVert\), thereby reducing the number of parameters. The sub-network with parameters \(m \odot W^Q \) is the winning lottery ticket, with the mask \(m\). Algorithm \ref{alg:gnn_sparsification} sparsifies the Graph Neural Network which is required to find the winning lottery ticket. The iterative algorithm that finds it is described in algorithm \ref{alg:iterative_glt}.


\begin{algorithm}
    \caption{GNN Sparsification}
    \label{alg:gnn_sparsification}
    \begin{algorithmic}[1]
        \Require Input graph $\mathcal{G} = (\mathcal{V}, \mathcal{E}, \mathbf{X})$, GNN's initialization $W^{(0)}$, initial mask $m^0 = 1 \in \mathbb{R}^{\lVert W \rVert}$, step size $\eta, \lambda$
        \Ensure Sparsified weight mask $m$
        \For{$i \gets 0$ to $N-1$}
            \State Forward $f(\cdot, m^i \odot W)$ with $\mathcal{G} = (\mathcal{V}, \mathcal{E}, \mathbf{X})$ to compute the loss $\mathcal{L}_{\mathrm{task}}$
            \State Back-propagate to update $W^{(i+1)} \gets W^{(i)} - \eta \nabla_W \mathcal{L}_{\mathrm{task}}$
            \State Update $m^{i+1} \gets m^{i} - \lambda \nabla_{m^i} \mathcal{L}_{\mathrm{task}}$
        \EndFor
        \State Set $p=20\%$ of the lowest magnitude values in $m^N$ to $0$ and others to 1, get mask $m$
    \end{algorithmic}
\end{algorithm}


We primarily differ from the definition of GLT in the previous works, as we observe that this is not suitable for graph-level tasks \cite{sun2023all,nguyen2021quaternion}. Popular methods such as UGS \cite{chen2021unified}, AdaGLT \cite{zhang2024graph} consider only the node-classification and the link prediction task. In graph-level tasks, sparsifying the input graph does not have any significance.
For graph-level tasks, the structure and the features in the entire graph are considered, unlike in the case of node- or edge-level tasks where only the nodes in the local neighborhood and their features are considered by a graph neural network.
Also, in the case of transfer learning, where the pruned sub-network is to be further finetuned for a different but related task, the dependency of the winning ticket on the input graph restricts us from fine-tuning it.

\subsection{Quaternion Graph Lottery Tickets}
Similar to finding a winning lottery ticket in GNNs, given a QMPNN \(f(\cdot, W^{Q})\) and a graph \(\mathcal{G} = (\mathcal{V}, \mathcal{E})\), the subnetwork, i.e., winning lottery ticket, of the QMPNN is defined as \(f(\cdot, m \odot W^Q)\), where \(m\) is the binary mask on parameters. To find the winning lottery ticket in a QMPNN, we use the algorithms \ref{alg:gnn_sparsification} and \ref{alg:iterative_glt} with the quaternion-values model weights and input features. 

% \begin{algorithm}
%     \caption{GNN Sparsification}
%     \label{alg:gnn_sparsification}
%     \begin{algorithmic}[1]
%         \Require Input graph $\mathcal{G} = (\mathcal{V}, \mathcal{E}, \mathbf{X})$, GNN's initialization $W^{(0)}$, initial mask $m^0 = 1 \in \mathbb{R}^{\lVert W \rVert}$, step size $\eta, \lambda$
%         \Ensure Sparsified weight mask $m$
%         \For{$i \gets 0$ to $N-1$}
%             \State Forward $f(\cdot, m^i \odot W)$ with $\mathcal{G} = (\mathcal{V}, \mathcal{E}, \mathbf{X})$ to compute the loss $\mathcal{L}_{\mathrm{task}}$
%             \State Back-propagate to update $W^{(i+1)} \gets W^{(i)} - \eta \nabla_W \mathcal{L}_{\mathrm{task}}$
%             \State Update $m^{i+1} \gets m^{i} - \lambda \nabla_{m^i} \mathcal{L}_{\mathrm{task}}$
%         \EndFor
%         \State Set $p=20\%$ of the lowest magnitude values in $m^N$ to $0$ and others to 1, get mask $m$
%     \end{algorithmic}
% \end{algorithm}

\begin{algorithm}
    \caption{Iterative algorithm to find Graph Lottery Ticket}
    \label{alg:iterative_glt}
    \begin{algorithmic}[1]
        \Require Input graph $\mathcal{G} = (\mathcal{V}, \mathcal{E}, \mathbf{X})$, GNN's initialization $W^{(0)}$, initial mask $m^0 = 1 \in \mathbb{R}^{\lVert W \rVert}$, sparsity level $s$
        \Ensure Lottery Ticket $f(({\mathcal{V}, \mathcal{E}, \mathbf{X}}), m \odot W^0)$
        \While{$1 - \frac{\lVert m \rVert}{\lVert W \rVert}$}
            \State \label{step:gnn_sparsification} Sparsify GNN $f(\cdot, m \odot W^{0})$ using algorithm \ref{alg:gnn_sparsification}
            \State Update $m$ according to step \ref{step:gnn_sparsification}
            \State Reset GNN's weight to $W^0$
        \EndWhile
    \end{algorithmic}
\end{algorithm}



\section{Experiments}
% \red{rewrite} check
\label{sec:expt}
The effectiveness of QMPNNs for different GNN variants is validated with extensive experimentation on different real-world standard datasets. We also verify the existence of the GLTs as per our definition. We evaluated their performance on three tasks, node classification, link prediction, and graph classification. 


% We examined three real- and quaternion-valued GNN models, GCN \cite{kipf2016semi}, GAT \cite{velivckovic2018graph}, and GraphSAGE \cite{hamilton2017inductive}. 
 

\subsection{Datasets}
\label{sec:datasets}
Tables \ref{tab:node-link-data} and \ref{tab:graph-data} summarize the datasets and their statistics. For quaternion models, dataset features and classes were adjusted to be divisible by 4 by padding features with the average value and adding dummy classes. For instance, Cora's feature size was padded from 1,433 to 1,436, and the number of classes was increased from 7 to 8.


\begin{table}[h]
\centering
\caption{Node classification and link prediction datasets statistics}
\resizebox{0.9\textwidth}{!}{%
\begin{tabular}{lrrrcl}
\toprule
\textbf{Dataset} & \textit{\#Nodes} & \textit{\#Edges} & \textit{\#Features} & \textit{\#Classes} & \textit{Metric} \\ \midrule
Cora        & 2,708    & 5,429    & 1,433    & 7  & Accuracy, ROC-AUC \\ 
            % &          &          &          &    & ROC-AUC  \\ 
Citeseer    & 3,327    & 4,732    & 3,703    & 6  & Accuracy, ROC-AUC\\ 
            % &          &          &          &    & ROC-AUC  \\ 
PubMed      & 19,717   & 44,338   & 500      & 3  & Accuracy, ROC-AUC\\ 
            % &          &          &          &    & ROC-AUC  \\ 
ogbn-arxiv  & 169,343  & 1,166,243 & 128     & 40 & Accuracy, ROC-AUC\\ 
            % &          &           &         &    & ROC-AUC  \\ 
ogbl-collab & 235,868  & 2,358,104 & 128     & 0  & Hits@50  \\ 
\bottomrule
\end{tabular}
}
\label{tab:node-link-data}
\end{table}

% To make them suitable for the quaternion models, the number of features and the number of classes in each dataset need to be divisible by 4. We have augmented the datasets to ensure that the datasets meet this requirement. 
% We achieved this by padding feature values with the average of the original features and adding dummy classes, thereby increasing the number of features, nodes, and classes to the nearest multiple of 4. For instance, we padded the feature vector size for the Cora dataset from 1,433 to 1,436 and increased the number of classes from 7 to 8 by adding a dummy class. 
% More details are provided in Appendix A.

\begin{table}[h]
\centering
\caption{Graph classification datasets statistics}
\begin{tabular}{lrrcc}
\toprule
\textbf{Dataset} & \textit{\#Graphs} & \textit{Avg. \#Edges} & \textit{\#Classes} & \textit{Metric} \\ \midrule
MUTAG      & 188    & 744     & 2  & Accuracy \\ 
ENZYMES    & 600    & 1,686   & 6  & Accuracy \\ 
PROTEINS   & 1,113  & 3,666   & 2  & Accuracy \\ 
ogbg-molhiv  & 41,127 & 40      & 2  & Accuracy \\ \bottomrule
\end{tabular}
\label{tab:graph-data}
\end{table}

\subsection{Experimental Setup}

\begin{figure*}[!ht]
    \centering
    \includegraphics[width=\textwidth]{figs/pruning.pdf}
    \caption{\textbf{Performance After Pruning}: The plots show GCN, GAT, and GraphSAGE performance on OGBN-ARXIV (node classification), OGBL-COLLAB (link prediction), and OGBG-MOHLIV (graph classification) at a pruning weight fraction of 0.3. \textbf{GLTs} are marked by red (\textcolor{red}{$\star$}) and green (\textcolor{green}{$\star$}) stars, indicating comparable performance despite sparsity, while dashed lines represent pre-pruning baselines. The plot sections correspond to node classification (top), link prediction (middle), and graph classification (bottom).
    }
    \label{fig:pruning}
    % \Description{Performance of 0.3 prune rate for three QMPNN versions of three GNN models, GCN, GAT and GraphSAGE on the datasets Cora, Citeseer, Pubmed and OGBN-ARXIV for node-classification task. The yellow curve indicates the performance of the model with quaternion weights and the blue curve indicates the performance of the model with real weights.}
\end{figure*}

% \begin{figure*}[ht]
%     \centering
%     \includegraphics[width=\linewidth]{figs/no-latex/pruning_Link Prediction.pdf}
%     \caption{\textbf{Link Prediction} Performance achieved after magnitude pruning with prune weight fraction to be 0.3 for GCN, GAT, and GraphSAGE on Cora, Citeseer, Pubmed, and OGBL-COLLAB Datasets. \textit{Red stars} (\textcolor{red}{$\star$}) and \textit{green stars} (\textcolor{green}{$\star$}) indicate the located GLTs of their respective pruning methods, which reach comparable even after having highly sparse. Dashed lines represent the baseline performance of GNNs.}
%     \label{fig:link}
%     % \Description{Performance of 0.3 prune rate for three QMPNN versions of three GNN models, GCN, GAT and GraphSAGE on the datasets Cora, Citeseer, Pubmed and OGBL-COLLAB for link-prediction task. The yellow curve indicates the performance of the model with quaternion weights and the blue curve indicates the performance of the model with real weights.}
% \end{figure*}

% \begin{figure}[ht]
%     \centering
%     \includegraphics[width=\linewidth]{figs/no-latex/pruning_Graph Classification.pdf}
%     \caption{\textbf{Graph Classification} Performance achieved after magnitude pruning with prune weight fraction to be 0.3 for GCN, GAT, and GraphSAGE on MUTAG, PROTEINS, ENZYMES and and OGBG-MOLHIV Datasets. \textit{Red stars} (\textcolor{red}{$\star$}) and \textit{green stars} (\textcolor{green}{$\star$}) indicate the located GLTs of their respective pruning methods, which reach comparable even after having highly sparse. Dashed lines represent the baseline performance of GNNs.}
%     \label{fig:graph}
%     % \Description{Performance of 0.3 prune rate for three QMPNN versions of three GNN models, GCN, GAT and GraphSAGE on the datasets MUTAG, PROTEINS, ENZYMES and and OGBG-MOLHIV for graph-classification. The yellow curve indicates the performance of the model with quaternion weights and the blue curve indicates the performance of the model with real weights.}
% \end{figure}

% \subsubsection{GNN Architecture}
% We employed two-layer GCN, GAT, and GraphSAGE models, each with 128 hidden units, for both node classification and link prediction tasks on the Cora, Citeseer, PubMed, ogbn-arxiv, and ogbn-collab datasets. For graph classification on the MUTAG, PROTEINS, ENZYMES, and ogbg-molhiv datasets, we used three-layer models, maintaining 128 hidden units per layer.

% \subsubsection{Data Splits}

% \begin{enumerate}
%     \item \textbf{Node Classification} 
%     For the datasets, Cora has 2,085 nodes for training, 271 for validation, and 271 for testing. Citeseer is split into 2,661 for training, 332 for validation, and 332 for testing. PubMed has 15,773 for training, 1,971 for validation, and 1,971 for testing. With ogbn-arxiv we used 135,474 nodes for training, 16,934 for validation, and 16,934 for testing. The splits are 80\% training, 10\% validation, and 10\% testing.
%     \item \textbf{Link Prediction} 
%     We employed a standard split where 10\% of the edges were designated for testing, 5\% for validation, and the remaining 85\% for training on Cora, Citeseer, and PubMed. This same 10/5/85\% split was applied consistently to the larger dataset ogbl-collab, for validation purposes.
%     \item \textbf{Graph Classification} 
%     The MUTAG, PROTEINS, ENZYMES, and ogbg-molhiv datasets were also divided into 10\% for testing, 5\% for validation, and 85\% for training, maintaining consistency across all experiments.
% \end{enumerate}

% \subsubsection{Training and Hyperparameters}
% We split the node classification datasets with an 80-10-10 split for training, validation, and testing respectively. For link prediction and graph classification datasets, the split is 85-5-10. 
% We train all models with a learning rate of 0.01, weight decay of \(5 \times 10^{-4}\), a dropout rate of 0.6, and an Adam optimizer. Training runs for up to 1000 epochs, with early stopping after 200 epochs of no improvement. A prune fraction of 0.3 is applied. 

% \subsubsection{Evaluation and Infrastructure}
% The evaluation metrics for respective datasets are presented in tables \ref{tab:node-link-data} and \ref{tab:graph-data}. All experiments were conducted using NVIDIA RTX 3090 GPUs with 24GB of VRAM. 

\begin{itemize}[topsep=0pt, itemsep=2pt, parsep=0pt]
    \item \textbf{GNN Architecture:}  
    Two-layer GCN, GAT, and GraphSAGE models with 128 hidden units were used for node classification and link prediction on Cora, Citeseer, PubMed, ogbn-arxiv, and ogbn-collab. For graph classification on MUTAG, PROTEINS, ENZYMES, and ogbg-molhiv, three-layer models with 128 hidden units per layer were employed.
    
    \item \textbf{Training and Hyperparameters:}  
    Node classification datasets followed an 80-10-10 split, while link prediction and graph classification used an 85-5-10 split. Models were trained with a learning rate of 0.01, weight decay of \(5 \times 10^{-4}\), dropout rate of 0.6, and the Adam optimizer, for up to 1000 epochs with early stopping after 200 epochs of no improvement. A prune fraction of 0.3 was applied.
    
    \item \textbf{Evaluation and Infrastructure:}  
    Evaluation metrics are provided in Tables \ref{tab:node-link-data} and \ref{tab:graph-data}. Experiments were conducted on NVIDIA RTX 3090 GPUs with 24GB VRAM.
\end{itemize}


\subsection{Training and Inference Details}
Our evaluation metrics, detailed in Tables \ref{tab:node-link-data} and \ref{tab:graph-data} and following \cite{kipf2016semi}, include both accuracy and ROC-AUC. Accuracy measures the proportion of correctly classified instances among the total instances, providing a straightforward metric for model performance. The ROC-AUC (Receiver Operating Characteristic - Area Under the Curve) score, representing the degree of separability, is particularly valuable for assessing performance on imbalanced datasets, such as in link prediction tasks with a substantial class imbalance between positive and negative edges. Summarizing the results across these diverse tasks, tables \ref{tab:node_classification}
, \ref{tab:link_prediction}, and \ref{tab:graph_classification} present the inference outcomes for all models, facilitating comprehensive analysis.


\subsection{Results}

GCN, GAT, and GraphSAGE results on Cora, Citesser, PubMed, and ogbn-arxiv for node classification are collected in table \ref{tab:node_classification} and \ref{tab:link_prediction} shows results for link prediction on Cora, Citesser, PubMed, and ogbl-collab. The results pertaining to graph classification tasks on MUTAG, PROTEINS, ENZYMES, and ogbg-molhiv datasets are presented in table \ref{tab:graph_classification}. These tables display the accuracy and parameter count (Params) in millions for each model and dataset, with better-performing models highlighted in bold for easy identification. Pruning results for the same models on ogbn-arxiv for node classification are shown in the top section of figure \ref{fig:pruning}. The middle section of figure \ref{fig:pruning} presents results for link prediction on the ogbl-collab dataset. The bottom section of figure \ref{fig:pruning} illustrates the results for graph classification tasks on the ogbg-molhiv dataset. An extensive list of results is available at project's \href{https://github.com/SagarPrakashBarad/QuatGLT}{GitHub} repository.

\begin{table*}[!htbp]
\centering
\caption{\textbf{Node classification} results for GCN, GAT, and GraphSAGE models on semi-supervised graph datasets, including Cora, Citeseer, PubMed, and OBGN-ARXIV.  Accuracy values are reported with their standard deviations, along with the corresponding GFLOPs (denoted as GF) and parameter counts (in K units, denoted as GF).}
\resizebox{\linewidth}{!}{%
\begin{tabular}{ccccccccccccc}
\toprule
\textbf{Model}       & \multicolumn{3}{c}{\textbf{Cora}} & \multicolumn{3}{c}{\textbf{Citeseer}} & \multicolumn{3}{c}{\textbf{PubMed}} & \multicolumn{3}{c}{\textbf{OGBN-Arxiv}} \\  \cmidrule(l){2-13} 
                     & Accuracy     & Par (K)   & GF & Accuracy     & Par (K)    & GF & Accuracy    & Par (K)     & GF & Accuracy      & Par (K)   & GF \\ \midrule
GCN                  & \textbf{85.2 ± 1.3} & 140.5  & 0.33 & 73.5 ± 1.8  & 125.7   & 0.29 & 79.1 ± 1.2  & 131.9   & 0.31 & \textbf{72.8 ± 1.5} & 145.1   & 0.42 \\
QGCN                 & 84.1 ± 1.5         & 42.1   & 0.11 & 72.6 ± 1.3  & 39.2    & 0.09 & 78.4 ± 1.4  & 41.7    & 0.10 & 72.3 ± 1.6         & 43.5    & 0.12 \\
GAT                  & \textbf{87.3 ± 1.0} & 143.2  & 0.37 & 75.4 ± 1.9  & 130.8   & 0.34 & 80.7 ± 1.6  & 136.3   & 0.36 & 72.5 ± 1.2         & 149.5   & 0.45 \\
QGAT                 & 86.4 ± 1.7         & 43.7   & 0.14 & 74.8 ± 1.1  & 41.6    & 0.12 & 79.8 ± 1.5  & 42.9    & 0.13 & \textbf{72.9 ± 1.8} & 44.5    & 0.15 \\
SAGE                 & 85.5 ± 1.2         & 72.6   & 0.23 & \textbf{74.9 ± 1.4} & 66.1    & 0.19 & 80.2 ± 1.9  & 68.7    & 0.20 & \textbf{73.0 ± 1.0} & 75.3    & 0.26 \\
QSAGE                & 84.9 ± 0.8         & 22.1   & 0.08 & 74.3 ± 1.5  & 20.5    & 0.07 & \textbf{80.5 ± 1.3} & 21.3    & 0.08 & 72.8 ± 1.4         & 23.6    & 0.10 \\ \bottomrule
\end{tabular}%
}
\label{tab:node_classification}
\end{table*}

The results indicate that quaternion models (QGCN, QGAT, and QSAGE) perform equally well or better than their real counterparts (GCN, GAT, and GraphSAGE) under the same hyperparameters and training parameters. For instance, QGCN shows slightly improved accuracy on the PubMed dataset, and QGAT performs nearly as well as GAT on Cora and Citeseer for node classification tasks. Moreover, quaternion models consistently outperform their real counterparts in other tasks, including link prediction and graph classification, demonstrating superior performance in almost all cases. Furthermore, it is consistently shown that even when the real models perform better than QMPNNs, they are mostly ahead by only marginal values.

\begin{table*}[!htbp]
\centering
\caption{\textbf{Link prediction} on various datasets including CORA, CITESEER, PUBMED, and OBGL-COLLAB. ROC-AUC and Hits@50 values are reported with their standard deviations, along with the corresponding GFLOPs (denoted as GF) and parameter counts (in K units, denoted as GF).}
\resizebox{\textwidth}{!}{%
\begin{tabular}{ccccccccccccccc}
\toprule
\multirow{2}{*}{Model Name} & \multicolumn{3}{c}{\textbf{CORA}} & \multicolumn{3}{c}{\textbf{CITESEER}} & \multicolumn{3}{c}{\textbf{PUBMED}} & \multicolumn{3}{c}{\textbf{OGBL-COLLAB}} \\  \cmidrule(l){2-13} 
                             & ROC-AUC   & Par (K)  & GF & ROC-AUC     & Par (K)   & GF & ROC-AUC    & Par (K)  & GF & Hits@50        & Par (K)  & GF \\ \midrule
GCN                          & 79.27 ± 0.47      & 96.0     & 0.23  & \textbf{78.53 ± 0.45}        & 241.4      & 0.58 & \textbf{87.60 ± 0.043}       & 36.3    & 0.09 & 44.02 ± 1.57       & 12.5       & 0.03 \\
QGCN                         & \textbf{83.55 ± 0.03}      & 27.2    & 0.07 & 78.36 ± 4.13        & 63.6       & 0.15 & 87.66 ± 0.32       & 12.3     & 0.03 & \textbf{45.80 ± 1.35}       & 6.4       & 0.02 \\
GAT                          & 93.89 ± 0.06      & 96.3    & 0.24 & 85.56 ± 0.38        & 241.6      & 0.60 & \textbf{89.32 ± 0.38}       & 36.6     & 0.09 & \textbf{45.51 ± 1.61}       & 12.8       & 0.03 \\
QGAT                         & \textbf{94.99 ± 0.19}      & 27.5    & 0.07 & \textbf{86.59 ± 0.10}       & 63.8       & 0.16 & 84.04 ± 0.54       & 12.6     & 0.03 & 32.03 ± 1.92       & 65.6       & 0.16 \\
SAGE                         & 89.00 ± 0.19       & 48.1    & 0.12 & 81.65 ± 0.13        & 120.5      & 0.30 & \textbf{83.36 ± 0.0189}       & 16.1     & 0.04 & \textbf{48.50 ± 0.88}       & 6.4       & 0.02 \\ 
QSAGE                        & \textbf{89.35 ± 0.26}      & 13.6  & 0.03 & \textbf{81.93 ± 0.04}        & 31.8    & 0.08 & 79.05 ± 0.32       & 6.1  & 0.02 & 48.07 ± 0.56       & 3.2       & 0.01 \\ \bottomrule
\end{tabular}%
}
\label{tab:link_prediction}
\end{table*}

The results presented in the tables comparing real and quaternion models, and are trained and evaluated using five different seeds. For each data point, we calculated the mean and standard deviation based on these evaluations. The better-performing models are determined by performing paired t-tests. We chose the model with the better performance when the difference between $p$-values of the models was found to be statistically significant. 

\begin{table*}[!htbp]
\centering
\caption{\textbf{Graph Classification} on various datasets including MUTAG, PROTEINS, ENZYMES, and OGBG-MOLHIV. Accuracy values are reported with their standard deviations, along with the corresponding GFLOPs (denoted as GF) and parameter counts (in K units, denoted as GF).}
\resizebox{\textwidth}{!}{%
\begin{tabular}{ccccccccccccccc}
\toprule
\multirow{2}{*}{Model Name} & \multicolumn{3}{c}{\textbf{MUTAG }} & \multicolumn{3}{c}{\textbf{PROTEINS }} & \multicolumn{3}{c}{\textbf{ENZYMES }} & \multicolumn{3}{c}{\textbf{OGBG-MOLHIV }} \\  \cmidrule(l){2-13} 
                             & Accuracy   & Par (K)   & GF & Accuracy     & Par (K)    & GF & Accuracy    & Par (K)  & GF & Accuracy        & Par (K)  & GF \\ \midrule
GCN                          & 83.54 ± 0.40      & 4.8     & 0.01  & \textbf{73.57 ± 0.45}        & 9.2      & 0.02 & \textbf{72.96 ± 0.15}       & 5.6    & 0.01 & \textbf{76.24 ± 0.89}       & 5.1       & 0.01 \\
QGCN                         & \textbf{88.61 ± 0.66}      & 4.4    & 0.01 & 71.55 ± 0.42        & 5.5       & 0.01 & 71.92 ± 1.16       & 4.6     & 0.01 & 75.85 ± 0.93       & 4.5       & 0.01 \\
GAT                          & 85.26 ± 0.02      & 5.4    & 0.01 & 72.68 ± 0.93        & 9.4      & 0.02 & \textbf{74.61 ± 0.05}       & 5.8     & 0.01 & 76.24 ± 0.32       & 5.8       & 0.01 \\
QGAT                         & \textbf{87.25 ± 0.18}      & 4.7    & 0.01 & \textbf{73.22 ± 0.02}        & 5.8       & 0.01 & 73.35 ± 0.02       & 4.9     & 0.01 & \textbf{78.16 ± 0.78}       & 4.8       & 0.01 \\
SAGE                         & 82.56 ± 2.67       & 2.5    & 0.01 & \textbf{72.62 ± 0.22}        & 4.7      & 0.01 & 79.65 ± 3.06       & 2.8     & 0.01 & 71.77 ± 0.97       & 2.6       & 0.01 \\ 
QSAGE                        & \textbf{84.26 ± 0.22}      & 2.2  & 0.01 & 73.08 ± 1.36        & 2.7    & 0.01 & \textbf{78.59 ± 0.73}       & 2.3  & 0.01 & \textbf{72.98 ± 0.12}       & 2.2       & 0.01 \\ \bottomrule
\end{tabular}%
}
\label{tab:graph_classification}
\end{table*}

It is worth noting that the performance of real-valued GNN models, on datasets with and without the dataset augmentation mentioned in the section \ref{sec:datasets} is the same. Hence, it serves as a fair baseline to compare the quaternion-valued GNN performance with the real-valued GNN performance on the augmented dataset. We provide the code in the repository linked in the paper.

% ADD experiments on different seed details
% \begin{table*}[!htbp]
% \centering
% \caption{\textbf{Node classification} results for GCN, GAT, and GraphSAGE models on semi-supervised graph datasets, including Cora, Citeseer, PubMed, and OBGN-ARXIV.}
% \resizebox{\linewidth}{!}{%
% \begin{tabular}{ccccccccccccc}
% \toprule
% \textbf{Model}       & \multicolumn{3}{c}{\textbf{Cora}} & \multicolumn{3}{c}{\textbf{Citeseer}} & \multicolumn{3}{c}{\textbf{PubMed}} & \multicolumn{3}{c}{\textbf{OGBN-Arxiv}} \\  \cmidrule(l){2-13} 
%                      & Accuracy     & Params (K)  & GFLOPs & Accuracy     & Params (K)    & GFLOPs & Accuracy    & Params (K)    & GFLOPs & Accuracy      & Params (K)  & GFLOPs \\ \midrule
% GCN                  & \textbf{85.2 ± 1.3} & 140.5  & 0.33 & 73.5 ± 1.8  & 125.7   & 0.29 & 79.1 ± 1.2  & 131.9   & 0.31 & \textbf{72.8 ± 1.5} & 145.1   & 0.42 \\
% QGCN                 & 84.1 ± 1.5         & 42.1   & 0.11 & 72.6 ± 1.3  & 39.2    & 0.09 & 78.4 ± 1.4  & 41.7    & 0.10 & 72.3 ± 1.6         & 43.5    & 0.12 \\
% GAT                  & \textbf{87.3 ± 1.0} & 143.2  & 0.37 & 75.4 ± 1.9  & 130.8   & 0.34 & 80.7 ± 1.6  & 136.3   & 0.36 & 72.5 ± 1.2         & 149.5   & 0.45 \\
% QGAT                 & 86.4 ± 1.7         & 43.7   & 0.14 & 74.8 ± 1.1  & 41.6    & 0.12 & 79.8 ± 1.5  & 42.9    & 0.13 & \textbf{72.9 ± 1.8} & 44.5    & 0.15 \\
% SAGE                 & 85.5 ± 1.2         & 72.6   & 0.23 & \textbf{74.9 ± 1.4} & 66.1    & 0.19 & 80.2 ± 1.9  & 68.7    & 0.20 & \textbf{73.0 ± 1.0} & 75.3    & 0.26 \\
% QSAGE                & 84.9 ± 0.8         & 22.1   & 0.08 & 74.3 ± 1.5  & 20.5    & 0.07 & \textbf{80.5 ± 1.3} & 21.3    & 0.08 & 72.8 ± 1.4         & 23.6    & 0.10 \\ \bottomrule
% \end{tabular}%
% }

% \label{tab:node_classification}
% \end{table*}

% \begin{table*}[!htbp]
% \centering
% \caption{\textbf{Link prediction} on various datasets including CORA, CITESEER, PUBMED, and OBGL-COLLAB. Accuracy values are reported with their standard deviations, along with the corresponding GFLOPs and parameter counts (in K units).}
% \resizebox{\textwidth}{!}{%
% \begin{tabular}{ccccccccccccccc}
% \toprule
% \multirow{2}{*}{Model Name} & \multicolumn{3}{c}{\textbf{CORA}} & \multicolumn{3}{c}{\textbf{CITESEER}} & \multicolumn{3}{c}{\textbf{PUBMED}} & \multicolumn{3}{c}{\textbf{OGBL-COLLAB}} \\  \cmidrule(l){2-13} 
%                              & ROC-AUC   & Params (K) & GFLOPs & ROC-AUC     & Params (K)  & GFLOPs & ROC-AUC    & Params (K) & GFLOPs & Hits@50        & Params (K) & GFLOPs \\ \midrule
% GCN                          & 79.27 ± 0.47      & 96.0     & 0.23  & \textbf{78.53 ± 0.45}        & 241.4      & 0.58 & \textbf{87.60 ± 0.043}       & 36.3    & 0.09 & 44.02 ± 1.57       & 12.5       & 0.03 \\
% QGCN                         & \textbf{83.55 ± 0.03}      & 27.2    & 0.07 & 78.36 ± 4.13        & 63.6       & 0.15 & 87.66 ± 0.32       & 12.3     & 0.03 & \textbf{45.80 ± 1.35}       & 6.4       & 0.02 \\
% GAT                          & 93.89 ± 0.06      & 96.3    & 0.24 & 85.56 ± 0.38        & 241.6      & 0.60 & \textbf{89.32 ± 0.38}       & 36.6     & 0.09 & \textbf{45.51 ± 1.61}       & 12.8       & 0.03 \\
% QGAT                         & \textbf{94.99 ± 0.19}      & 27.5    & 0.07 & \textbf{86.59 ± 0.10}       & 63.8       & 0.16 & 84.04 ± 0.54       & 12.6     & 0.03 & 32.03 ± 1.92       & 65.6       & 0.16 \\
% SAGE                         & 89.00 ± 0.19       & 48.1    & 0.12 & 81.65 ± 0.13        & 120.5      & 0.30 & \textbf{83.36 ± 0.0189}       & 16.1     & 0.04 & \textbf{48.50 ± 0.88}       & 6.4       & 0.02 \\ 
% QSAGE                        & \textbf{89.35 ± 0.26}      & 13.6  & 0.03 & \textbf{81.93 ± 0.04}        & 31.8    & 0.08 & 79.05 ± 0.32       & 6.1  & 0.02 & 48.07 ± 0.56       & 3.2       & 0.01 \\ \bottomrule
% \end{tabular}%
% }
% \label{tab:link_prediction}
% \end{table*}


\subsection{Key Findings}
We summarize the effectiveness of magnitude pruning and quaternion-based models in GNNs:

\begin{enumerate}[topsep=0pt, itemsep=2pt, parsep=0pt]
    \item \textbf{QMPNNs operate at 1/4th parameters of real-valued models:}  
    QMPNNs achieve comparable or better performance than real-valued models with just 1/4th of the parameters, showcasing their efficiency and expressiveness, as shown in Tables \ref{tab:node_classification}, \ref{tab:link_prediction}, and \ref{tab:graph_classification}.
    
    \item \textbf{GLTs exist at 1/5th or smaller of original size:}  
    Magnitude pruning identifies lottery tickets at 1--20\% of the original model size without performance loss across QGCN, QGAT, and QGraphSAGE tasks, as observed in Tables \ref{tab:node_classification} and \ref{tab:link_prediction}.
    
    \item \textbf{GAT and GraphSAGE sparsify well; Cora is pruning-sensitive:}  
    GATs and GraphSAGE produce sparse GLTs due to attention and sampling techniques, while Cora exhibits significant sensitivity to pruning, with performance dropping at half model size, as detailed in the comprehensive experiments available on the \href{https://anonymous.4open.science/r/lth-qmpnn-B143/}{project's GitHub repository}.
    % , as shown in \red{Figure} \ref{fig:pruning}.
    
    \item \textbf{Less sparsity in graph classification GLTs:}  
    Graph classification tasks require less sparse GLTs to capture complex global features, with smaller parameter counts due to the datasets' limited graph size, as shown in Table \ref{tab:graph_classification} and Figure \ref{fig:pruning}.
    
    \item \textbf{Large graphs maintain performance:}  
    QMPNNs scale well to large graphs, as seen with ogbn-arxiv, ogbn-collab, and ogbn-molhiv datasets in Tables \ref{tab:node_classification}, \ref{tab:link_prediction}, and \ref{tab:graph_classification}, delivering comparable or better performance relative to trainable parameters.
\end{enumerate}



% \subsection{Key Findings}
% We present the major findings regarding the effectiveness of magnitude pruning and the performance of quaternion-based models in graph neural networks (GNNs).

% \textbf{Finding 1 - QMPNNs excel at 1/4th paramters to that of the real GNN models}:
% Inference tables and pruning curves show that QMPNNs match or exceed the performance of their real-valued counterparts while operating at just 1/4th of the model size. This highlights the effectiveness and enhanced expressiveness of quaternion-based neural networks on quaternion data compared to real-valued models.
% % Additionally, as noted in Observation 1, pruning effectively identifies GLTs for quaternion Graph Neural Networks.

% \textbf{Finding 2 - GLTs exist at 1/5th or smaller of the original model size}:
% Using magnitude pruning, we can identify the lottery tickets for quaternion models at 20\% to roughly 1\% of the original model size without any performance loss. This holds consistently across all link prediction and node classification tasks for QGCN, QGAT, and QGraphSAGE models. 


% \textbf{Finding 3 - GAT and GraphSAGE are more amenable to sparsified graphs; Cora is more sensitive to pruning}:
% Our results (\red{Figure} \ref{fig:node}) align with \cite{chen2021unified}, showing that GLTs for GATs are notably sparser than those for GCNs, likely due to attention-based aggregation re-identifying critical graph connections. GraphSAGE displays similar sparsity under magnitude pruning, attributed to its neighborhood sampling technique, while the Cora dataset remains particularly sensitive, with performance degrading significantly when models are pruned to half their size.

% \textbf{Finding 4 - Graph classification GLTs exist at less sparse networks}:
% As shown in Figure \ref{fig:graph}, GLTs are larger for graph classification tasks than other tasks, likely due to the need to capture complex global graph features, which requires retaining more connections to sustain performance. Additionally, Table \ref{tab:graph_classification} indicates lower parameter counts for these models than for other tasks because the datasets typically consist of smaller graphs, and quaternion models show limited parameter reduction compared to real models due to fewer weights being converted to quaternions.

% \textbf{Finding 5 - Large graphs have similar performance}: Experiments with large graphs show minimal performance deviation despite the dataset size. As observed in Tables \ref{tab:node_classification}, \ref{tab:link_prediction}, and \ref{tab:graph_classification} for ogbn-arxiv, ogbn-collab, and ogbn-molhiv, QMPNNs deliver performance that is either comparable or better relative to the number of trainable parameters, demonstrating their scalability to larger datasets.

% \red{Finding 6 - Quantization results - model sizes} check - not added 

% \red{add model sizes in conclusion - 1 line} check - not added
\section{Conclusion and Future Work}
\label{sec:conclusion}

This paper introduces Quaternion Message Passing Neural Networks (QMPNNs) as a versatile framework for graph representation learning, leveraging quaternion representations to capture intricate relationships in graph-structured data. The framework generalizes easily to existing GNN architectures with minimal adjustments, enhancing flexibility and performance across tasks like node classification, link prediction, and graph classification. By redefining graph lottery tickets, we identified key subnetworks that enable efficient training and inference, demonstrating the scalability and effectiveness of QMPNNs on real-world datasets. Future work could explore dynamic graphs, hybrid modalities, and improving interpretability to further expand QMPNN capabilities.

% In conclusion, this paper has introduced Quaternion Message Passing Neural Networks (QMPNNs) as a versatile framework for graph representation learning. By utilizing quaternion representations within graph neural networks (GNNs) architecture, we offer a flexible framework to achieve the quaternion equivalent of the GNNs, which can capture intricate relationships within graph-structured data. Our framework provides a generalizable solution, allowing for easy integration into existing GNN architectures with minimal adjustments, enhancing graph-based learning systems' flexibility.

% Furthermore, our redefinition of the graph lottery ticket was necessary for reasons such as the inability of the existing definition to hold up in transfer learning or graph-level tasks. We use this redefined version to obtain insights into both GNNs and QMPNNs. By identifying key subnetworks that significantly contribute to model performance, we enable more efficient training and inference procedures, ultimately improving the overall effectiveness of graph-based learning algorithms.

% Through comprehensive performance evaluations on real-world datasets across multiple graph-based tasks, we have demonstrated the better performance of QMPNNs over their GNN counterparts. Our results underscore the efficacy of quaternion representations in capturing the complex structures inherent in graph data, leading to improved performance across downstream tasks, such as node classification, link prediction, and graph classification.

% Extending the capabilities of QMPNNs to handle dynamic graphs and temporal data presents an interesting direction for future research. Enhancing the interpretability and explainability of QMPNNs remains another area for further exploration. Exploring hybrid approaches that combine quaternion-based representations with other modalities, such as text or image data, could open up new possibilities for multi-modal learning on heterogeneous graph data.

\bibliographystyle{splncs04}
\bibliography{sample-base}

\appendix





\end{document}
