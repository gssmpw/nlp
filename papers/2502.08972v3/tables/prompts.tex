
\begin{figure*}
    \centering
\begin{tcolorbox}[width=\textwidth]
\fontsize{8pt}{8pt}\selectfont\ttfamily 
You are an editor. \\ 
\begin{itemize}
\renewcommand\labelitemi{--}
    \item Your task is to analyze whether the candidate writing is stylistically consistent with the author's writing(s) and if not, highlight elements of the author's style that are not observed in the candidate writing. 
    \item Consider similarity with regards to the (1) length, (2) format, (3) paragraph structure, (4) sentence structure, (5) punctuation, (6) syntax, (7) voice, and (8) diction of the author's writing, but NOT the content it covers. 
    \item Use the minimum words possible in your analysis while providing specific examples of how the observed inconsistencies must be edited to become stylistically consistent with the author's writing.
    \item If the candidate writing is stylistically consistent with the author's writing, respond with "yes" in the "is\_consistent" field. Otherwise, respond with "no". 
\end{itemize}
\vspace{2mm}

\# Task \\ 
\{ task \} \\ 
\vspace{2mm}

\# Author's writing  \\ 
\{ reference\_text \} \\ 
\vspace{2mm}

\# Candidate writing to edit \\ 
\{ generated\_text \} \\ 
\vspace{2mm}

Respond only with JSON with the following format:\\ 
\{ \\ 
\phantom{xx}"explanation": "<style analysis and suggested edits>",\\ 
\phantom{xx}"is\_consistent": "<yes/no>"\\ 
\} 

\end{tcolorbox}

    \caption{Prompt template $\mathcal{E}$ for generating explanations on the difference between the reference output and the generated output.}
    \label{fig:explanation_prompt}
\end{figure*}

\begin{figure*}
    \centering
\begin{tcolorbox}[width=\textwidth]
\fontsize{8pt}{8pt}\selectfont\ttfamily
You are a stylistically consistent writer. Below are examples that exemplify your writing style.

\vspace{2mm}

\# Writing Task Example 1 \\ 
\{ example["task"] \} \\ 
\#\# Your Writing 1 \\ 
\{ example["reference\_output"] \} \\ 

\# Writing Task Example 2 \\ 
... \\ 
\vspace{2mm}

** Task to complete ** \\ 
Now complete the following writing task with a style and format consistent with `Your Writing` examples.
\\
Be consistent in terms of (1) length, (2) format, (3) paragraph structure, (4) sentence structure, (5) punctuation, (6) syntax, (7) voice, and (8) diction of your writing when completing the task. 
\\
\vspace{2mm}

Task: \{ target\_task \} \\ 

Directly provide your response in the following format:\\ 
\begin{verbatim} 
```
<your writing>
```
\end{verbatim}

    
\end{tcolorbox}
    \caption{Prompt template $\mathcal{P}$ for the few-shot in-context learning baseline.}
    \label{fig:fewshot_prompt}
\end{figure*}

\begin{figure*}
    \centering
\begin{tcolorbox}[
 width=1.0\textwidth
]
\fontsize{8pt}{8pt}\selectfont
\ttfamily
You are a stylistically consistent writer. Below are examples that exemplify your writing style.

\vspace{2mm}

\# Writing Task Example 1 \\ 
\{ example["task"] \} \\ 
\#\# Your Writing 1 \\ 
\{ example["reference\_output"] \} \\ 

\#\# Stylistically Inconsistent Writing 1-1 \\ 
\textcolor{blue}{\{ example["generated\_output"][0]["output"] \}} \\ 
\#\#\# Inconsistent stylistic elements in `Stylistically Inconsistent Writing 1-1` that should be corrected for it to become more consistent with `Your Writing 1`: \\ 
\textcolor{darkgreen}{\{ example["generated\_output"][0]["explanation"] \}} \\ 
... \\

\# Writing Task Example 2 \\ 
... \\ 
\vspace{2mm}

** Task to complete ** \\ 
Now complete the following writing task with a style and format consistent with `Your Writing` examples and also avoiding the stylistic inconsistencies found in the `Stylistically Inconsistent Writing` examples.
\\
Be consistent in terms of (1) length, (2) format, (3) paragraph structure, (4) sentence structure, (5) punctuation, (6) syntax, (7) voice, and (8) diction of your writing when completing the task. 
\\
\vspace{2mm}

Task: \{ target\_task \} \\ 

Directly provide your response in the following format: 
\begin{verbatim} 
```
<your writing>
```
\end{verbatim}

\end{tcolorbox}
    \caption{Prompt template $\mathcal{P}$ for \ours. It builds on \autoref{fig:fewshot_prompt} by adding \textcolor{blue}{negative samples} and \textcolor{darkgreen}{explanations} generated with \autoref{fig:explanation_prompt}.}
    \label{fig:our_prompt}
\end{figure*}


\begin{figure*}
    \centering
\begin{tcolorbox}[
width=1.0\textwidth
]
\fontsize{8pt}{8pt}\selectfont
\ttfamily
You are an impartial evaluator of style similarity. Below are samples of an author's writing and two options. 

\vspace{2mm}

\# Author's Writing:\\ 
EXAMPLE 1: \\ 
\{example 1\}\\ 
...\\ 
EXAMPLE 5: \\ 
\{example 5\} 

\vspace{2mm}

\# Option A:

\{candidate A\}

\vspace{2mm}

\# Option B:

\{candidate B\}

\vspace{2mm}

\# Task

Which option is more likely to have been written by the author based on style similarity to the samples given as AUTHOR'S WRITING above? Consider each option's similarity with regards to the (1) length, (2) format, (3) paragraph structure, (4) sentence structure, (5) punctuation, (6) syntax, (7) voice, and (8) diction of the author's writing, but NOT the content it covers. If one option has incoherent/odd text or formatting (e.g., random dashes, repetitive text, random signatures, etc.) that is not present in the author's writing while the other doesn't, it should be considered less similar.

\{ \\
\phantom{xx}"explanation": \{ \\ 
\phantom{xx}\phantom{xx}"length": "<Explanation>", \\ 
\phantom{xx}\phantom{xx}    "format": "<Explanation>", \\ 
\phantom{xx}\phantom{xx}    "paragraph structure": "<Explanation>", \\ 
\phantom{xx}\phantom{xx}    "sentence structure": "<Explanation>", \\ 
\phantom{xx}\phantom{xx}    "punctuation": "<Explanation>", \\ 
\phantom{xx}\phantom{xx}    "syntax": "<Explanation>", \\ 
\phantom{xx}\phantom{xx}    "voice": "<Explanation>", \\ 
\phantom{xx}\phantom{xx}    "diction": "<Explanation>", \\ 
\phantom{xx}\phantom{xx}    "odd incoherent text/formatting": "<Explanation>" \\ 
\phantom{xx} \} \\ 
\phantom{xx}"answer":  "<The option more similar to the AUTHOR'S WRITING; either A or B>" \\ 
\}

\vspace{2mm}

ALWAYS REMAIN IMPARTIAL WHEN EVALUATING OUTPUTS AND PENALIZE ODD INCOHERENT TEXT.
\end{tcolorbox}
    \caption{Prompt template for evaluating which output is more stylistically similar to the target author's writing style given five examples from the target author.}
    \label{fig:eval_template}
\end{figure*}


\begin{figure*}
    \centering
\begin{tcolorbox}[width=\textwidth]
\fontsize{8pt}{8pt}\selectfont\ttfamily 
Given the following written examples, provide a style guide on how you would go about writing in a similar style as the examples for the target task. Analyze the (1) length, (2) format, (3) paragraph structure, (4) sentence structure, (5) punctuation, (6) syntax, (7) voice, and (8) diction of the examples, but NOT the content or topic that they cover. \\ 

\vspace{2mm}

\# Writing Task Example 1 \\ 
\{ example["task"] \} \\ 
\#\# Your Writing 1 \\ 
\{ example["reference\_output"] \} \\ 

\# Writing Task Example 2 \\ 
... \\ 
\vspace{2mm}

\# Target task \\ 
Task: \{ target\_task \} 

\vspace{2mm}

Provide your style guide with the following format:
\begin{verbatim}
```
<your writing>
```
\end{verbatim}

\end{tcolorbox}

    \caption{Prompt template for generating style guides that will be used for Chain-of-Thought guided writing in \autoref{fig:cot_writing_template}.}
    \label{fig:cot_style_guide_template}
\end{figure*}


\begin{figure*}
    \centering
\begin{tcolorbox}[width=\textwidth]
\fontsize{8pt}{8pt}\selectfont\ttfamily 
You are a stylistically consistent writer. Below are examples that exemplify your writing style. \\ 
\vspace{2mm}

\# Writing Task Example 1 \\ 
\{ example["task"] \} \\ 
\#\# Your Writing 1 \\ 
\{ example["reference\_output"] \} \\ 

\# Writing Task Example 2 \\ 
... \\ 
\vspace{2mm}

\# Task to complete \\ 
Now complete the following writing task, first by analyzing the style and format of the `Your Writing` examples. \\ 

Task: \{ target\_task \} \\ 

\textcolor{blue}{\{ Chain-of-Thought Style Guide \}} \\ 

Directly provide your response for the task in the following format:
\begin{verbatim}
```
<your writing>
```
\end{verbatim}

\end{tcolorbox}

    \caption{Prompt template for Chain-of-Thought-guided writing. The \textcolor{blue}{blue text} is the style guide generated from \autoref{fig:cot_style_guide_template}.}
    \label{fig:cot_writing_template}
\end{figure*}



\begin{figure*}
    \centering
\begin{tcolorbox}[width=\textwidth]
\fontsize{8pt}{8pt}\selectfont\ttfamily 
I have some instructions along with their corresponding scores. The instructions are arranged in ascending order based on their scores, where higher scores indicate better quality. \\ 
\vspace{2mm}


\begin{verbatim}
{%
instruction: 
{{ prompt_score_pair[0] }}
score:
{{ prompt_score_pair[1] }}
{%
\end{verbatim}

The following exemplars show how to apply your instruction: you replace <INS> in each input with your instruction, then read the input and give an output. The scores are the average stylistic similarity score of the generated output, generated with your instruction, compared to the reference output.

\begin{verbatim}
{%
input:
{{ example["task"] }}
<INS>
output:
{ example["reference_output"] }
{%
\end{verbatim}

\vspace{2mm}

Write your new instruction that is different from the old ones that will help achieve a style similarity score as high as possible. Consider similarity with regards to the (1) length, (2) format, (3) paragraph structure, (4) sentence structure, (5) punctuation, (6) syntax, (7) voice, and (8) diction of exemplar outputs, but NOT the content it covers. The instruction should not mention 'reference output' or 'original input' as only the input and the instruction placed instead of <INS> is available when writing an output. Write your instruction in the following format:
\begin{verbatim}
```
<your instruction>
```
\end{verbatim}

\end{tcolorbox}

    \caption{OPRO optimization prompt. The instruction with the highest score is used for the OPRO writing prompt in \autoref{fig:opro_writing_template}.}
    \label{fig:opro_optimization_template}
\end{figure*}

\begin{figure*}
    \centering
\begin{tcolorbox}[width=\textwidth]
\fontsize{8pt}{8pt}\selectfont\ttfamily 
Task: \{ target\_task \} \\ 
\textcolor{blue}{\{ opro\_prompt \}}

\begin{verbatim}
Respond only with JSON with the following format:
{
    "thought": "<your thoughts>",
    "response": "<your response>"
}
\end{verbatim}

\end{tcolorbox}

    \caption{OPRO writing prompt. The \textcolor{blue}{blue text} is the instruction that is selected from the prompt optimization process from \autoref{fig:opro_optimization_template}.}
    \label{fig:opro_writing_template}
\end{figure*}
