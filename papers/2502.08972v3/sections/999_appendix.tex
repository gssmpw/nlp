\section*{Appendix}
\label{sec:appendix}




\section{Evaluation Dataset Details}
\label{appdx:dataset_details}

\subsection{Sampling}
\label{appdx:sampling}
We rely on sampling to alleviate issues of having a small test set per author. 
For each of the three test prompts, we sample five outputs per method. 
We pair outputs for the same prompts to create 75 comparison pairs ($5\cdot5\cdot3$) and sample 40 of them for comparison and randomly assign the order they appear in the evaluation prompt to control for order bias. 
For the versus author results, we only have 15 samples since there’s only one original author response ($1\cdot5\cdot3)$, so we permute the order for each pair to get 30 samples. 
This results in 400 ($40\cdot10$) comparisons for vs DITTO and 300 ($30\cdot10$) for vs Author to get the averaged win rates in \autoref{tab:main_combined}. 
We are able to compute statistically significant win rates with these number of comparisons, as reported in Section \ref{ssec:main_results}.

\subsection{Technical Details}
\label{appdx:technical_details}

The exact models that we use for our experiments are \texttt{gpt-4o-2024-0806}, Claude 3 Sonnet and Mistral 7B Instruct v0.2 via Amazon Bedrock. 
For evaluation (\textsection \ref{sec:pairwise_evaluation}), we use batch inference using the OpenAI API, which lets us reduce inference costs by 50\%.\footnote{\url{https://platform.openai.com/docs/guides/batch}} 

\paragraph{Notes on \ours}
At the first step of \ours, we hold out one sample $(x_i, y_i)$ at random and use the rest   $\mathcal{D}^{\mathcal{T}}_0\setminus\{(x_i, y_i)\}$ as few-shot examples that form the ICL prompt.
If we use the full set, we don't have any samples for which we can explore and learn  without seeing the reference output $y_i$ of the input $x_i$. from $\mathcal{D}^{\mathcal{T}}_0$.
This information is made more explicit in Algorithm \ref{alg:our_algorithm}. 



\paragraph{Notes on DITTO.} 
DITTO was trained with Mistral 7B Instruct, which is suspected to be a smaller model than Claude 3 Sonnet and GPT-4o. 
We found that the fine-tuned variants of Mistral 7B Instruct are prone to generating unstable outputs, producing template components such as \texttt{[INST]} or generating degenerate outputs that contain repetitive content. 
To give these approaches the best chances possible against our approaches with larger models, we reject samples that contain template components and repetitive content until we reach the desired number of samples $N=5$ from each model. 


\section{Evaluation Methods}
\label{appdx:automatic_evaluation}


\begin{table}
    \centering
    \resizebox{\linewidth}{!}{
    \begin{tabular}{llrrrrrrrrrr}
    \toprule
        Benchmark & \multicolumn{10}{c}{Author IDs}  \\ 
        \midrule
        CMCC &   2 & 3 & 5 & 6 & 7 & 8 & 11 & 13 & 15 & 17 \\ 
        CCAT & 10 & 12 & 15 & 20 & 23 & 27 & 28 & 30 & 32 & 38    \\ 
        \bottomrule 
    \end{tabular}
    }

    \caption{Mapping from author indices to actual author IDs in CMCC and CCAT that we use throughout our work. These authors are the authors that ranks in the top 10 for highest accuracy achieved with GPT-4 Eval, as described in Appendix \ref{appdx:llm as a judge} and shown in \autoref{tab:per_author_evaluation_benchmark}.}
    \label{tab:author_ids}    
\end{table}


\begin{table}[t]
    \centering
    \resizebox{\linewidth}{!}{
    \begin{tabular}{lrrrr}
        \toprule 
         Method & \# Examples & Explan. & CMCC & CCAT  \\ 
        \midrule
        \multirow{4}{*}{GPT-4o pairwise} & 1 & \xmark & $76.1_{2.2}$ & -  \\ 
        & 1 & \cmark & $78.1_{2.1}$ & - \\ 
        & 3 & \cmark & $82.6_{1.9}$ & - \\ 
        & 5 & \cmark & $89.5_{2.3}$ &  ${80.4}_{2.0}$\\ \midrule
        GPT-4o mini pairwise & 5 & \cmark & $68.2_{2.4}$ & $66.8_{2.3}$ \\ \midrule 
        GPT-4o rating & 5 & \cmark & $55.0_{1.2}$ & $20.0_{1.1}$ \\ \midrule 
        \multirow{3}{*}{SE} & 1 & - & $71.5_{1.5}$ & $61.3_{1.5}$ \\
         & 3 & - & $76.8_{1.4}$ & $63.0_{1.5}$ \\
        & 5 & - & $78.5_{1.3}$ &  $63.6_{1.5}$ \\ \midrule
        Human & 5 & - & $85.4$ & - \\ 
        \bottomrule
    \end{tabular}
    }
    \caption{Results with author texts from CMCC and CCAT for comparing various evaluation methods. GPT-4o: \texttt{gpt-4o-2024-0806}, GPT-4o mini: \texttt{gpt-4o-2024-0718-mini}, SE: Style Embeddings.}
    \label{tab:eval_method_comparison}
\end{table}


In this section, we describe benchmarking results using author texts from CMCC and CCAT for various evaluation methods. 
Results are shown in \autoref{tab:eval_method_comparison}.
For CMCC, several authors wrote responses to the same prompt, and therefore we use samples written for the same prompt as the compared samples, as this would help us control for content similarity between the tested sample and the examples in making the task arbitrarily easy. 

For CCAT, which are news articles with a format of ``\textit{Write a news article that starts with the following sentence}: \texttt{article's first sentence}''. 
There are no share prompts for this dataset. 
Since journalists tend to write articles on similar topics over time and thus content similarity can be an easy hint for detecting authorship, we select the negative sample as one that has the highest TF-IDF similarity with any one of the in-context examples.

\subsection{Embedding-based evaluation}

Universal authorship representations (UAR)~\citep{rivera-soto-etal-2021-learning} is a sentence embedding method that uses a SBERT model~\cite{reimers-gurevych-2019-sentence} fine-tuned with a contrastive learning objective such that the representations of sentences or documents from the same author become closer together than with those of other authors.
Style Embeddings (SE) is a follow-up to UAR that refines the representations to focus on style rather than content by pairing contrastive samples that share similar content~\citep{wegmann-etal-2022-author}, and therefore we focus on embedding-based comparisons with Style Embeddings. 

We compute the cosine similarity between candidate texts' SE with those of the author examples and the one with higher cosine similarity is considered more stylistically similar. 
The best score for SE is $78.5\%$ and $63.6\%$ for CMCC and CCAT, respectively, which is significantly lower than GPT-4o results but better than GPT-4o mini's results.
While having more examples also helps with SE's performance, it plateaus markedly at three examples, and there is only minimal gains observed for CCAT. 




\subsection{LLM-as-a-Judge}
\label{appdx:llm as a judge}

The pairwise comparison setup is described in Section \ref{sec:pairwise_evaluation}. 
As an alternative, we also consider having LLM-as-a-judge provide ratings for individual instances, so that we can reduce the number of samples that the LLM-as-a-judge need to evaluate and also measure relative performance of each apporach based on their aggregate scores. 
The prompt is similar to the pairwise comparison setup except that only one candidate is shown and the LLM is asked to provide a score from 1-5 on how stylistically similar it thinks the candidate is to the examples. 
Unfortunately, this result is the poorest performing approach, only achieving 55.0\% accuracy for CMCC and 20.0\% accuracy for CCAT. 
The main reason for low accuracy is that the majority of instances were given the same rating such that they were deemed as ties. 


\subsection{Human Evaluation}
\label{appdx:human_evaluation}
We have three human annotators complete the same setup with 100 samples for CMCC only with five samples from each author. 
Conducting the same evaluation for CCAT was overwhelming for human participants because the average text length in CCAT is much larger.
As shown in the results, human performance is lower than the best GPT-4o results. 
This reinforces findings from previous work that identifying authorship reliably is a difficult task for untrained humans and that model-based classifiers are more reliable and practical for this task~\cite{hallinan-etal-2023-steer, krishna-etal-2020-reformulating, liu2024authorshipstyletransferpolicy, liu2024styletransfermultiiterationpreference}.  



The full prompt template for this evaluation setup is shown in \autoref{fig:eval_template}. 













\section{Flesch Reading Ease}
\label{appdx:flesch}

The Flesch Reading Ease (FRE) score is a simple equation for approximating readability of a given text based on the average number of words per sentence and the average number of syllables per word~\cite{flesch1948new}.
The formula for calculating FRE is the following: 

$$
FRE= 206.835 - (1.015\cdot ASL)-(84.6\cdot \frac{n_{sy}}{n_w})
$$

where $ASL$ is average sentence length ($\frac{\text{total words}}{\text{total sentences}}$) and $\frac{n_{sy}}{n_w}$ is the average number of syllables per word ($\frac{\text{total syllables}}{\text{total words}}$).  
The higher the score, the easier it is to understand a piece of text when read. 
The maximum score is 121.22, while there is no limit to how low it can be and it can even be negative. 
We use Python's \texttt{textstat} package\footnote{\url{https://github.com/textstat/textstat}} to compute FRE. 


\begin{table*}[t]
    \centering
    \small
    \begin{tabular}{l|llr|rrrrrrrrrr}
    \toprule
         Data & \multicolumn{2}{c}{Method} & \multicolumn{1}{c}{$a_{avg}$} & $a_1$ & $a_2$ & $a_3$ & $a_4$ & $a_5$ & $a_6$ & $a_7$ & $a_8$ & $a_9$ & $a_{10}$ \\   
         \midrule 
         \multirow{10}{*}{CMCC} & \multirow{5}{*}{GPT-4o} & Zero-shot & $9.00_{0.72}$ & 20 & 5& 0 & 0 & 15 & 10 & 10 & 0 & 30& 0 \\
         & &  Few-shot & $42.50_{1.13}$ & 55& 35& 40& 20& 65& 55& 40& 15& 55& 45 \\ 
         & & CoT  & $41.00_{1.01}$ & 50& 35& 35& 25& 55& 50& 50& 15& 60& 35\\ 
         & & OPRO & $3.00_{0.56}$ & 5& 0& 0& 0& 0& 0& 0& 0& 25& 0\\ 
         & & \ours & $\mathbf{53.00_{1.24}}^{\dag}$ & 70& 70& 50& 35& 60& 65& 80& 45& 70& 50 \\  \cmidrule{2-14}
         & \multirow{5}{*}{Claude 3 Sonnet} & Zero-shot & $12.00_{1.03}$ & 25& 5& 5& 0& 20& 15& 5& 0& 45& 0 \\
         & &  Few-shot & $74.00_{1.51}$ & 85& 55& 70& 35& 85& 90& 95& 50& 100& 75 \\ 
         & & CoT & $71.50_{1.20}$ & 80& 60& 65& 55& 70& 85& 90& 45& 100& 65\\ 
         & & OPRO & $2.00_{0.25}$ & 5& 0& 0& 0& 0& 0& 5& 0& 10& 0\\ 
         & & \ours & $\mathbf{79.50_{1.26}}^{\dag}$ & 85& 50& 65& 90& 75& 95& 90& 95& 100& 100 \\ 
         \midrule 
         \multirow{10}{*}{CCAT} &  \multirow{5}{*}{GPT-4o} & Zero-shot & $21.00_{1.05}$ & 15& 20& 15& 5& 50& 5& 30& 20& 40& 10 \\
         & &  Few-shot & $76.50_{1.25}$ & 90& 70& 90& 40& 90& 70& 95& 90& 60& 70 \\ 
         & & CoT & $74.50_{1.04}$ & 75& 65& 85& 45& 85& 90& 90& 90& 80& 40 \\ 
         & & OPRO & $13.50_{1.27}$ & 0& 5& 0& 0& 35& 10& 55& 15& 10& 5\\ 
         & & \ours & $\mathbf{82.50_{1.25}}^{\dag}$ & 85& 65& 90& 70& 85& 90& 100& 90& 80& 70 \\ \cmidrule{2-14}
         & \multirow{5}{*}{Claude 3 Sonnet} & Zero-shot & $54.00_{1.44}$ & 50& 40& 65& 35& 65& 35& 90& 80& 50& 30\\
         & &  Few-shot&  $86.50_{0.69}$ & 95& 80& 95& 65& 95& 85& 95& 85& 90& 80\\ 
         & & CoT & $90.00_{0.67}$ & 100& 95& 100& 70& 85& 85& 100& 90& 90& 85 \\ 
         & & OPRO &$23.50_{1.33}$ & 20& 5& 10& 10& 55& 5& 55& 15& 30& 30  \\ 
         & & \ours & $\mathbf{91.50_{0.74}}$ & 100& 90& 100& 75& 100& 90& 100& 95& 80& 85\\  
         \bottomrule
         
    \end{tabular}
    \caption{LMJ-based win rate results against DITTO~\cite{shaikh2024show}. $\dag$ indicates a win rate that is larger than the next best performing baseline at a $p<0.05$ statistically significant level.
    \ours outperforms all other baselines at a statistically significant level, except for Claude 3 Sonnet on CCAT. $a_{avg}$ is the average win rate across all authors and $a1$-$a10$ are the author IDs from \autoref{tab:author_ids}.}
    \label{tab:main_results_vs_ditto}
\end{table*}


\begin{table*}[t]
    \centering
    \small
    \begin{tabular}{l|llr|rrrrrrrrrr}
    \toprule
         Data & \multicolumn{2}{c}{Method} & \multicolumn{1}{c}{$a_{avg}$} & $a_1$ & $a_2$ & $a_3$ & $a_4$ & $a_5$ & $a_6$ & $a_7$ & $a_8$ & $a_9$ & $a_{10}$ \\   
         \midrule 
         \multirow{10}{*}{CMCC} & Mistral & DITTO & $25.00_{1.08}$ & 20& 20& 35& 45& 10& 35& 5& 30& 5& 45\\
         \cmidrule{2-14}
         & \multirow{5}{*}{GPT-4o} & Zero-shot & $2.50_{0.34}$ & 0& 0& 15& 5& 0& 0& 0& 0& 5& 0 \\
         & &  Few-shot & $21.00_{1.28}$ & 20& 15& 10& 25& 35& 15& 10& 0& 15& 65\\ 
         & & CoT  & $12.00_{0.53}$ & 25& 10& 10& 10& 20& 5& 10& 0& 10& 20\\ 
         & & OPRO & $0.00_{0.00}$ & 0& 0& 0& 0& 0& 0& 0& 0& 0& 0\\ 
         & & \ours & $\mathbf{31.00_{1.18}}^{\dag}$ & 30& 20& 15& 35& 35& 25& 45& 5& 35& 65\\  \cmidrule{2-14}
         & \multirow{5}{*}{Claude 3 Sonnet} & Zero-shot & $0.50_{0.11}$ & 0& 0& 0& 0& 0& 0& 0& 0& 5& 0\\
         & &  Few-shot & $43.50_{1.37}$ & 60& 40& 30& 25& 45& 45& 55& 5& 60& 70 \\ 
         & & CoT & $44.00_{0.88}$ & 50& 40& 30& 45& 25& 40& 65& 35& 55& 55 \\ 
         & & OPRO & $0.00_{0.00}$ & 0& 0& 0& 0& 0& 0& 0& 0& 0& 0\\ 
         & & \ours & $\mathbf{54.50_{1.67}}^{\dag}$ & 80& 15& 30& 75& 60& 75& 65& 55& 60& 85\\ 
         \midrule 
         \multirow{11}{*}{CCAT} & Mistral & DITTO & $10.00_{0.55}$ & 0& 15& 15& 15& 5& 25& 0& 5& 10& 10 \\
         \cmidrule{2-14}   
         &  \multirow{5}{*}{GPT-4o} & Zero-shot & $6.00_{0.40}$ & 0& 10& 0& 15& 10& 0& 0& 10& 10& 5 \\
         & &  Few-shot & $25.50_{0.56}$ & 15& 15& 30& 25& 35& 35& 15& 30& 25& 30\\ 
         & & CoT & $21.50_{0.47}$ & 15& 10& 15& 20& 25& 30& 25& 30& 20& 25 \\ 
         & & OPRO & $0.00_{0.00}$ & 0& 0& 0& 0& 0& 0& 0& 0& 0& 0 \\ 
         & & \ours & $\mathbf{28.00_{0.63}}^{\dag}$ & 10& 30& 25& 40& 45& 40& 20& 30& 30& 30 \\ \cmidrule{2-14}
         & \multirow{5}{*}{Claude 3 Sonnet} & Zero-shot & $10.50_{0.66}$ & 0& 5& 10& 25& 20& 0& 0& 20& 15& 10 \\
         & &  Few-shot&  $53.50_{0.94}$ & 60& 70& 40& 55& 45& 40& 35& 55& 60& 75\\ 
         & & CoT & $51.00_{0.81}$ & 55& 70& 40& 50& 50& 35& 45& 50& 45& 70  \\ 
         & & OPRO &$0.00_{0.00}$ & 0& 0& 0& 0& 0& 0& 0& 0& 0& 0  \\ 
         & & \ours & $\mathbf{57.00_{0.87}}^{\dag}$ & 60& 55& 45& 55& 80& 40& 45& 55& 70& 65 \\  
         \bottomrule
         
    \end{tabular}
    \caption{LLM-as-a-judge win rates against the author's actual text. The table takes the same format as \autoref{tab:main_results_vs_ditto}.}
    \label{tab:main_results_vs_author}
\end{table*}


\section{Full Per-Author Results}

The full per-author results with comparisons against DITTO and against the author's text are shown in \autoref{tab:main_results_vs_ditto} and \autoref{tab:main_results_vs_author}, respectively.

\section{Sample Outputs}

We show some qualitative examples from our baselines and \ours in \autoref{tab:sample_outputs}. 

\begin{table*}[h]
    \footnotesize
    \centering
    \adjustbox{max width=\textwidth}{
        \begin{tabular}{p{0.5\textwidth}| p{0.5\textwidth}}
            \toprule
            \multicolumn{1}{c}{\textbf{CMCC}} & \multicolumn{1}{c}{\textbf{CCAT}} \\
            \midrule
            \multicolumn{2}{c}{Dataset} \\ 
            \midrule
            Write an approximately 500 word essay to the following prompt: Recently, school officials prevented a school shooting because one of the shooters posted a myspace bulletin. Do you think this was an invasion of privacy? & Write an article that starts with the following: A potential new source of revenue -- oil -- offers hope of a fresh start for Chad's fragile economy as it rebuilds after three decades of war. \\ 
            \midrule
            \multicolumn{2}{c}{Author}\\
            \midrule
            \char91...\char93~I feel as though the school officials were not invading privacy at all. 
            The entire point
            ~\char91...\char93~
            Why anyone~\char91...\char93~
            is beyond me, and even more so if the person does not expect to be caught. If anyone~\char91...\char93~
            Did the school harm the person who was going to do the shooting? No. ~\char91...\char93
            since it IS available to the public- to TURN OVER ~\char91...\char93~
            If there are lives in danger~\char91...\char93~
            (to catch kids drinking, etc.), then THAT would be "taking it over the edge" as an invasion of privacy. Thanks for your time, ID \# 20
            & 
            \char91...\char93~``Soon work will begin,'' Foreign Minister Saleh Kebzabo told Reuters recently. ``Chad should get some royalties so we can free the country through development.'' 
            Chad has known coups, civil war or conflict with its northern neighbour Libya for much of its life from independence in 1960. Sporadic cross-border rebel raids remain a problem.
            \char91...\char93~
            But
            \char91...\char93~
            On the negative side, external debt rose to \$817 million in 1996 from \$803 million in 1995 and 1996 GDP growth was projected at 0.5 percent, down from 4.5 percent in 1995. 
            ~\char91...\char93~
            \\
            \midrule
            \multicolumn{2}{c}{DITTO}\\ \midrule 
            \char91...\char93~I definitely do not think that the school system invaded the privacy of the person who was caught posting the myspace bulletin about the potential school shooting. 
            ~\char91...\char93~
            If this was the case, then he definitely knows what a good portion of the world is aware of- the fact that everything you do can be caught on camera. Whether it be a store, etc., everything is being recorded. 
            ~\char91...\char93~
            Sincely, ID\# 20123452
& 
\char91...\char93~
The landlocked African nation tapped its first well in 1973 but suffered instability through French suitor Cameroon's push for a larger share of any wealth or former guerrilla revolts against president Idriss Deby's rule. ``Oil is a godsend. It is not only a source of revenue but it also gives Chad prestige on the international scene,'' said Jean-Pierre Alain, head of Chad's national hydrocarbons agency. 
~\char91...\char93~
It had an estimated 1997/98 budget of \$275 million. It had international debt of \$355 million at the end of 1996
~\char91...\char93~

\\
            \midrule
            \multicolumn{2}{c}{Few-shot}
            \\ \midrule 
            \char91...\char93~ In regards to the question of whether the school officials' actions to prevent a school shooting based on a myspace bulletin were an invasion of privacy, I hold the firm belief that they were not. The fundamental duty of any educational institution ~\char91...\char93~
            Ultimately, ~\char91...\char93~
            Hence, ~\char91...\char93~ Thank you for your inquiry. -ID\#: 20
& 
            \char91...\char93~The discovery of significant oil reserves in the Doba Basin has invigorated economic forecasts and raised hopes of alleviating poverty in the landlocked Central African nation. 
            \"This is an opportunity for Chad to transform its economy,\" said Finance Minister Alhabo Abakar. \"The potential revenues from oil could be a game-changer.\"
            ~\char91...\char93~
            The oil pipeline, a multi-billion-dollar joint venture with global oil firms, promises to open up isolated regions and create jobs. 
            \char91...\char93~
            \\ \midrule
            \multicolumn{2}{c}{CoT}\\ \midrule 
            \char91...\char93~I firmly believe that the actions taken by school officials in this situation were not an invasion of privacy.
            In fact, 
            ~\char91...\char93~
            In light of this, ~\char91...\char93~
            As www.dictionary.com defines a "bulletin,"
            ~\char91...\char93~
            Hence,~\char91...\char93~
            Thank you for reaching out to discuss this crucial topic. ID\#: 20
 & 
\char91...\char93~``Oil could be the lifeline we've been waiting for,'' said Economic Minister Mahamat Djoum. 
 ``It has the potential to transform our economy, create jobs, and improve the standard of living for our people.''
 Chad's oil reserves, located primarily in the Doba basin in the south, are expected to yield significant output, with projections suggesting several hundred thousand barrels a day by the year 2000. 
 \char91...\char93~
            \\ \midrule
            \multicolumn{2}{c}{OPRO} \\ \midrule  
            Thank you for reaching out with this important question. While privacy is a significant concern,
            ~\char91...\char93~
            Nonetheless, it's crucial to continue the discussion on maintaining a balance between security and individual rights.
            & 
            \char91...\char93~Historically plagued by persistent conflict and political instability, Chad has struggled to maintain sustainable economic growth and improve the quality of life for its citizens. 
            \char91...\char93~
            \\   
            \midrule
            \multicolumn{2}{c}{\ours}\\ \midrule 
            \char91...\char93~ I genuinely believe that ~\char91...\char93~ It's simple - ~\char91...\char93~ If someone posts about harmful intentions, expecting privacy is a bit ironic, don't you think? The school officials did their job, ~\char91...\char93~ Privacy is essential, yes, ~\char91...\char93~ If a message was genuinely private, it wouldn’t be on such a public platform, plain and simple. So, let’s focus ~\char91...\char93~ Remember, if you put it out there, it’s open to be acted upon for the greater good. Thanks for hearing me out on this matter! 
            &
            \char91...\char93~``Oil represents a new chapter for Chad's economy, promising the possibility of financial stability and development,'' said Finance Minister Mahamat Allamine Bourma.
            The nation, which remains one of the world's poorest despite its natural resources, eyes the oil sector with optimism. 
            The World Bank has been involved in supporting infrastructure projects, including the Chad-Cameroon pipeline, designed to facilitate the transport of oil to international markets.
            \char91...\char93~ 
            A consortium led by ExxonMobil, Chevron, and Malaysia's Petronas plans to invest over \$3 billion            ~\char91...\char93~
               \\
            \bottomrule
        \end{tabular}
    }
    \caption{Truncated sample responses from baselines and \ours using GPT-4o for CMCC and CCAT. 
    }
    \label{tab:sample_outputs}
\end{table*}


\section{Prompt Details}
\label{appdx:prompt_details}

For details on the prompts that we use, we refer the reader to the following: 

\begin{itemize}
    \item Few-shot prompting: \autoref{fig:fewshot_prompt}
    \item \ours: \autoref{fig:our_prompt}
    \item Pairwise LMJ evaluation: \autoref{fig:eval_template}
    \item Style guide generation for CoT: \autoref{fig:cot_style_guide_template}
    \item Writing prompt for CoT that uses the generated style guide: \autoref{fig:cot_writing_template}
    \item OPRO's optimization prompt: \autoref{fig:opro_optimization_template}
    \item  OPRO's writing prompt that uses the prompt found from the optimization process: \autoref{fig:opro_writing_template}
\end{itemize}


\begin{lstlisting}[title={Sampling Responses During Training/Inference}]
Please reason step by step, and put your final answer within 
\boxed{}. 
Problem: {problem} 
\end{lstlisting}

\begin{lstlisting}[title={Verification Refinement}]
You are a math teacher. I will give you a math problem and an answer. 
Verify the answer's correctness without step-by-step solving. Use alternative verification methods. 
Question: {problem}
Answer: {answer}
Verification:
\end{lstlisting}

\begin{lstlisting}[title={Verification Collection}]
Refine this verification text to read as a natural self-check within a solution. Maintain logical flow and professionalism.
Key Requirements:
1. Avoid phrases like "without solving step-by-step" or "as a math teacher".
2. Treat the answer as your own prior solution.
3. Conclude with EXACTLY one of:
Therefore, the answer is correct.
Therefore, the answer is incorrect.
Therefore, the answer cannot be verified.
Original text: {verification}
\end{lstlisting}





         
\begin{table*}[t]
    \centering
    \begin{tabular}{lrrr}
        \hline
        Author ID & CMCC & \makecell{CCAT\\w/o TF-IDF} & \makecell{CCAT\\w/ TF-IDF} \\
        \hline
        0 & 84 & 96 & 82 \\
        1 & 74 & 96 & 76 \\
        2 & 98 & 92 & 74 \\
        3 & 92 & 98 & 74 \\
        4 & 72 & 92 & 88 \\
        5 & 94 & 64 & 58 \\
        6 & 98 & 100 & 80 \\
        7 & 96 & 100 & 86 \\
        8 & 92 & 68 & 70 \\
        9 & 88 & 68 & 64 \\
        10 & 90 & 100 & 100 \\
        11 & 98 & 86 & 86 \\
        12 & 84 & 100 & 98 \\
        13 & 100 & 96 & 76 \\
        14 & 90 & 96 & 68 \\
        15 & 100 & 100 & 96 \\
        16 & 64 & 84 & 82 \\
        17 & 98 & 96 & 80 \\
        18 & 88 & 92 & 68 \\
        19 & - & 78 & 76 \\
        20 & - & - & 100 \\
        21 & - & - & 58 \\
        22 & - & - & 70 \\
        23 & - & - & 92 \\
        24 & - & - & 64 \\
        25 & - & - & 60 \\
        26 & - & - & 90 \\
        27 & - & - & 98 \\
        28 & - & - & 100 \\
        29 & - & - & 90 \\
        30 & - & - & 90 \\
        31 & - & - & 80 \\
        32 & - & - & 100 \\
        33 & - & - & 68 \\
        34 & - & - & 76 \\
        35 & - & - & 86 \\
        36 & - & - & 72 \\
        37 & - & - & 60 \\
        38 & - & - & 94 \\
        39 & - & - & 88 \\
        \hline
        Accuracy & $89.47_{2.0}$ & $90.05_{1.7}$ & $80.45_{1.7}$ \\
        \hline
    \end{tabular}
    \caption{Full benchmarking results using human written texts in CMCC and CCAT for LLM-as-a-judge for style similarity, using \texttt{gpt-4o-2024-0806}. Values shown are per-author accuracies (\%). We used only the first 20 authors for comparing CCAT with and without using TF-IDF. 
    After finding that TF-IDF helps control for semantic similarity among author examples, we expanded the evaluation using TF-IDF to all authors to find authors that our evaluation achieves the highest accuracy.}
    \label{tab:per_author_evaluation_benchmark}
\end{table*}


