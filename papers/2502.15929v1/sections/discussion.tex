\section{Discussion}
\label{sec:discussion}
We conclude with some questions raised by the $\ell_2$ mechanism. Throughout, we take the Gaussian mechanism as a familiar baseline.

\textbf{Privacy.} As mentioned previously, since the $\ell_2$ mechanism is an instance of the $K$-norm mechanism, it also satisfies pure DP (right plot in \Cref{fig:intro}). In contrast, the $(\eps, \delta)$-DP Gaussian mechanism does not satisfy $\eps'$-DP for any $\eps' < \infty$. However, since the Gaussian mechanism's privacy loss random variable follows a Gaussian distribution, it admits easy privacy analyses under notions like concentrated~\cite{BS16}, Renyi~\cite{M17}, and Gaussian~\cite{DRS22} DP. While the $\ell_2$ mechanism's pure DP guarantee may also be ported to guarantees for these other privacy notions, they are looser than direct analyses. This affects the guarantees obtained when using, for example, amplification by subsampling of a mechanism satisfying RDP, as done tightly for the Gaussian mechanism~\cite{ACGMM+16, WBK19}.

\textbf{Sampling.} The marginals of the Gaussian mechanism are (one-dimensional) Gaussians, so it is easy to sample in parallel in one map and one combine. The $\ell_2$ mechanism can also be sampled in parallel, at the cost of an additional map and combine step (\Cref{subsec:l2_sampler}). The Gaussian mechanism also offers a discrete analogue with discrete sampling~\cite{CKS20}, but a discrete analogue of the $\ell_2$ mechanism is not known.