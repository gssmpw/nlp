\section{Introduction}
\label{sec:intro}
Computing a $d$-dimensional statistic with bounded $\ell_2$ sensitivity is a fundamental task in differential privacy (DP)~\cite{DMNS06}. It underlies standard algorithms like private stochastic gradient descent~\cite{SCS13, ACGMM+16}, the binary tree mechanism~\cite{CSS11, DNPR10}, and the projection~\cite{NTZ13, N23B}, matrix~\cite{LMHMR15, MMHM18}, and factorization mechanisms~\cite{ENU20, NT23}. The canonical approximate DP algorithm for this problem is the Gaussian mechanism~\cite{DMNS06}. To compute statistic $T(X)$, the Gaussian mechanism samples an output according to $g_X(y) \propto \exp(-[\|y - T(X)\|_2/\sigma]^2)$ for an appropriate value of $\sigma$; in particular, the analytic Gaussian mechanism~\cite{BW18} chooses the smallest possible $\sigma$ sufficient for the desired approximate DP guarantee.

In this paper, we analyze the $\ell_2$ mechanism. Given a parameter $\sigma$, this mechanism samples an output according to density $f_X(y) \propto \exp(-\|y - T(X)\|_2/\sigma)$. As an instance of the $K$-norm mechanism~\cite{HT10} using the $\ell_2$ norm, the $\ell_2$ mechanism immediately satisfies $\frac{1}{\sigma}$-(pure) DP and can be sampled efficiently. However, its approximate DP guarantees are not well understood.

\subsection{Contributions}
\label{subsec:contributions}
For arbitrary dimension $d$ and privacy parameters $\eps$ and $\delta$, we provide an algorithm for choosing $\sigma$ to obtain an $\ell_2$ mechanism that satisfies $(\eps, \delta)$-DP. The resulting $\ell_2$ mechanism can be efficiently sampled in parallel and empirically dominates both the Laplace mechanism and the analytic Gaussian mechanism in terms of mean squared $\ell_2$ error (left plot in \Cref{fig:intro}). Moreover, unlike the Gaussian mechanism, the $\ell_2$ mechanism always satisfies a pure DP guarantee (right plot in \Cref{fig:intro}).

Our algorithms bound relevant quantities of the privacy loss random variable for the $\ell_2$ mechanism.~\citet{BW18} showed that mechanism $M$ is $(\eps, \delta)$-DP if and only if
\begin{equation}
\label{eq:iff}
    \P{}{\ell_{M,X,X'} \geq \eps} - e^\eps\P{}{\ell_{M,X',X}  \leq -\eps} \leq \delta~
\end{equation}
where $\ell_{M,X,X'}$ is the privacy loss associated with $M$ on arbitrary neighboring databases $X$ and $X'$ (\Cref{sec:prelims}). Proving $(\eps,\delta)$-DP therefore reduces to upper bounding the first term and lower bounding the second. We show that the first term is defined by the mass that $M(X)$ places on a region of $\mathbb{R}^d$ determined by certain spherical caps, while the second term is defined by the mass that $M(X')$ places on the same region. We then provide algorithms to approximate the first term from above and the second term from below. Because these approximations are provably upper and lower bounds, they yield a formal differential privacy guarantee. Experiments suggest that, for reasonable algorithm parameter values, these approximations are tight (\Cref{subsec:experiments_privacy}).

\begin{figure*}[t]
        \centering
        \includegraphics[scale=0.55]{images/errors.pdf}
        \includegraphics[scale=0.55]{images/pure_dp.pdf}
        \caption{\textbf{Left}: normalized mean squared $\ell_2$ error. At each $d$, we compute mean squared $\ell_2$ error for the $(1, 10^{-5})$-DP Laplace, analytic Gaussian~\cite{BW18}, and $\ell_2$ mechanisms. Quantities are normalized so that the analytic Gaussian mechanism error is always 1. Note that we truncate the Laplace mechanism at $d=8$, after which its error relative to the analytic Gaussian mechanism continues to grow. See \Cref{subsec:experiments_error} for details. \textbf{Right}: the pure DP guarantee of the $(1, 10^{-5})$-DP $\ell_2$ mechanism as $d$ grows.}
        \label{fig:intro}
\end{figure*}

\subsection{Related Work}
\citet{GZ21} also use spherical caps to analyze what they call ``generalized Gaussians'', which have densities proportional to $\exp(-[\|y - T(X)\|_p / \sigma]^p)$ for integers $p \geq 1$. A few features separate their work from ours: they study a statistic with bounded $\ell_\infty$ sensitivity; their results do not cover the $\ell_2$ mechanism, which uses norm $p=2$ but exponent $p'=1$; they work with a sufficient condition for $(\eps, \delta)$-DP, which only bounds the first term in \Cref{eq:iff} by $\delta$, leading to looser results; and since their goal is an asymptotic utility guarantee, their results rely on asymptotic concentration inequalities that are less precise than the approach used here.

A few authors have studied the $\ell_2$ mechanism, primarily in the context of objective perturbation~\cite{CMS11, KST12, YRUF14}. However, they all use pure DP rather than approximate DP.

\arxiv{\subsection{Organization}
Preliminaries appear in \Cref{sec:prelims}. \Cref{sec:l2} describes the $\ell_2$ mechanism, its privacy guarantees, and sampling. \Cref{sec:experiments} provides empirical evaluations of the privacy analysis, error, and speed. \Cref{sec:discussion} concludes with a general discussion of context and future directions.}