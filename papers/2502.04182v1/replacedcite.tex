\section{Related work}
\label{ss:Zhao-Eppstein}

\paragraph*{The scheme by Zhao \textit{et al.}____} There are four steps in the graph embedding function of Zhao \textit{et al.}____.
First, they generate a random seed based on cryptographic keys. Then, they generate a random watermark graph (Erdős-Rényi) using the seed and match it with a subgraph of $G$. The subgraph selection is based on the structure of the vertices (i.e. their neighborhoods). Finally, the embedding is performed by an XOR operation between the subgraph and the random watermark graph.
For the extraction, they regenerate the watermark graph and the subgraph like in the embedding process. After that, they identify in the potential $G^*$ which vertices are candidates to match the subgraph (again using the vertex structure). Finally, they extract the watermark using a recursive algorithm.
Since the code was not provided by the authors, we implemented it in Python 3. We note that our execution results are consistent with those presented by the authors in their experimental section (Figure 2b in ____). We set $\theta=0.95$ and $B=25$, for an acceptable execution time.

\paragraph*{The scheme by Eppstein \textit{et al.} ____} The scheme of Eppstein \textit{et al.}____ consists of the same three functions \texttt{Keygen,} \texttt{Embed} and \texttt{Extract} as the \scheme scheme. In the scheme of Eppstein \textit{et al.}, private keys generated by the \texttt{Keygen} function are sets of vertex pairs in a graph containing as many vertices as in the graph to be watermarked. There are three main steps in the embedding function. The first step is ordering and labeling the vertices of the graph, considering degrees and using bit vectors. The vertices are divided into three categories: high, medium, and low degree vertices. Then, each edge contained in the private key is randomly flipped according to the probability of its statistical existence in the graph. The function \texttt{Extract} uses graph isomorphism to identify the watermarked graph. Unlike \scheme and Zhao____, this method identifies a graph between multiple (simultaneously created) watermark copies, not just a single watermark graph with a given key. The selected copy is the closest copy considering the Hamming distance.