\section{Related Work}
Dynamic SLAM methods extend traditional SLAM by integrating measurements of dynamic objects, enabling simultaneous estimation of sensor pose, the static background, and object states within the environment.
The state estimation problem for objects can vary depending on the modeling approach.
Some methods choose to model the object state using simple geometric primitives such as cuboids____, ellipsoids____ or quadrics____. 
While these approaches successfully estimate object poses, they do not account for motion or velocity.

In contrast, other methods focus explicitly on estimating object motion____, which is commonly represented as a 3D rigid-body transformation. This approach also allows highly accurate object pose to be recovered. 
Of these of these methods____ reports the best accuracy in terms of object pose and motion estimation. 

Trajectory prediction methods can be broadly categorized into physics-based, learning-based, and hybrid approaches____. Classical physics-based models, such as constant velocity and constant acceleration for kinematic models, and the bicycle model for dynamic models, provide interpretable and computationally efficient solutions. These methods rely on motion equations to extrapolate future states, making them effective for short-term predictions. However, they often neglect complex agent interactions and environmental data, limiting their applicability in urban traffic scenarios. To address these limitations, modern trajectory prediction methods have increasingly adopted data-driven approaches.
While classic machine learning and reinforcement learning methods remain competitive, they are often outperformed by deep learning approaches____.

Combining data-driven methods with physics-based models has emerged as a promising direction, improving prediction accuracy by leveraging spatiotemporal data while maintaining physical realism through constraints and dynamic models. Recent state-of-the-art models reflect this trend. Pedestrian trajectory prediction models often focus on socially-aware predictions____, while models for urban environments with both pedestrians and vehicles utilize diverse input sources, such as High-Definition (HD) maps____.

The challenge of domain shift across datasets and generalization across varying road conditions, regions, and driving styles persists. Ivanovic~\etal____ address this by introducing an adaptive layer for efficient domain transfer through offline fine-tuning and online adaptation. Feng~\etal____ tackle the issue by enhancing dataset size and diversity, leveraging ScenarioNet____ to unify data formats, perform single-step preprocessing, and harmonize discrepancies such as varying recording frequencies, object observation durations, and map annotation precision. Notably, all these state-of-the-art models rely on ground truth data for training.

In this work, we bridge the gap between pose estimation, motion estimation, and trajectory prediction by integrating the entire process from raw sensor data to trajectory forecasting. Unlike existing methods that rely on preprocessed pose data, our unified system ensures consistency across environments without external dependencies. To our knowledge, this is the first approach directly linking sensor inputs, estimation, and prediction, making it well-suited for real-world deployment in diverse settings.