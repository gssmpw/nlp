Following the convention of the main paper, we drop the explicit dependence on $Z$ wherever it is obvious.%

\subsection{Proof of Lemma \ref{lem:convexity_algebra}}

\lSCA*
\begin{proof}
    From the expression of $\nabla \Loglikelihood$ (see \eqref{eq:gradient_likelihood}), we get that:
    \begin{align*}
        \llangle \nabla \Loglikelihood, \Delta\rrangle =   \frac{1}{m}\sum_{k=1}^{m}
    h_k \llangle (A_k+ A_k^T) Z, \Delta \rrangle \text{ where }
    h_k =  \frac{g'(z_k)\left(g(z_k) - w_k\right)}{g(z_k)(1-g(z_k))}, \ z_k = \llangle A_k,ZZ^T \rrangle.
    \end{align*}
    Recall, by definition (see \eqref{eq:def_difference_sol}), that $Z = \solset + \Delta$. Therefore. the term $\llangle (A_k+ A_k^T) Z, \Delta \rrangle$ can be expanded as follows:
    \begin{align*}
        \llangle (A_k+ A_k^T) Z, \Delta \rrangle &= \llangle (A_k+ A_k^T) \solset, \Delta \rrangle + \llangle (A_k+ A_k^T) \Delta, \Delta \rrangle\\
        &= \llangle A_k+ A_k^T, \Delta \solset^T \rrangle + \llangle A_k+ A_k^T, \Delta \Delta^T \rrangle \quad \text{(by \eqref{eq:identity_shift})}
    \end{align*}
    Since we have assumed that our observations are noiseless, we have the identity $w_k = g(\llangle A_k,Z^*Z^{*T} \rrangle)$. Plugging this equation in the expression of $h_k$, we get:
    \begin{align*}
        h_k &=  \frac{g'(z_k)\left(g(z_k) - g(z^*_k)\right)}{g(z_k)(1-g(z_k))}; \quad z^*_k =  \llangle A_k,Z^*Z^{*T} \rrangle = \llangle A_k, \solset \solset^T \rrangle
    \end{align*}
    By the mean value theorem,
    \begin{align*}
        g(z_k) - g(z^*_k) &= g'(y_k) (z_k - z^*_k) \quad \text{for some } y_k \text{ in the interval between } z_k \text{ and } z^*_k\\
        &= g'(y_k) \left(\llangle A_k,ZZ^T \rrangle - \llangle A_k,\solset \solset^T \rrangle \right) \\
        &= g'(y_k) \left(\llangle A_k,\solset \Delta^T + \Delta \solset^T\rrangle + \llangle A_k,\Delta \Delta^T\rrangle\right)  \quad (\text{because }Z = \solset + \Delta)\\
        &= g'(y_k) \left(\llangle A_k+ A_k^T,\Delta \solset^T\rrangle + \frac{1}{2}\llangle A_k+ A_k^T,\Delta \Delta^T\rrangle\right) \quad (\text{by } \eqref{eq:A_AT_identity})
    \end{align*}
    Putting the above equations together, we get:
    \begin{align*}
        &h_k \llangle (A_k+ A_k^T) Z, \Delta \rrangle \\
        &\ = \frac{g'(z_k)g'(y_k)}{g(z_k)(1-g(z_k))} \left(\llangle A_k + A_k^T, \Delta \solset^T\rrangle + \frac{1}{2}\llangle A_k + A_k^T,\Delta \Delta^T\rrangle\right) \left(\llangle A_k + A_k^T,\Delta \solset^T\rrangle + \llangle A_k + A_k^T,\Delta \Delta^T\rrangle\right) \\
        &\ = \frac{g'(z_k)g'(y_k)}{g(z_k)(1-g(z_k))} \left(\llangle A_k + A_k^T, \Delta \solset^T\rrangle^2 + \frac{3}{2} \llangle A_k + A_k^T, \Delta \solset^T\rrangle \llangle A_k + A_k^T,\Delta \Delta^T\rrangle + \frac{1}{2}\llangle A_k + A_k^T,\Delta \Delta^T\rrangle^2\right)\\
        &\ \geq \frac{g'(z_k)g'(y_k)}{g(z_k)(1-g(z_k))}\left(\frac{1}{2}\llangle A_k + A_k^T,\Delta \solset^T\rrangle^2 - \frac{5}{8}\llangle A_k + A_k^T,\Delta \Delta^T\rrangle^2\right)
    \end{align*}
    The last step uses the inequality $2a^2 + 3ab + b^2$ $\geq {a^2} - \frac{5b^2}{4}$, which can be derived from the trivial inequality $(a + 3b/2)^2 \geq 0$. Note also that the coefficient $\frac{g'(z_k)g'(y_k)}{g(z_k)(1-g(z_k))}$ is positive.

Finally, observe that we have assumed $Z \in \overline{\Incoherentset}$. The bounds in \eqref{eq:score_bound1_Zstar} and \eqref{eq:score_bound1_Zcbar} imply $$|z_k^*| \leq 2\frac{ \mu \norm{Z^*}_F^2}{n}, \ |z_k| \leq 24\frac{ \mu \norm{Z^*}_F^2}{n}, \text{ which implies } |y_k| \leq 24\frac{\mu \norm{Z^*}_F^2}{n}$$ 
Thus, $y_k$ and $z_k$ lie in the interval $\left[-24 \mu \norm{Z^*}_F^2/n, {24 \mu \norm{Z^*}_F^2}/{n}\right]$. By the definition of $\xi$ and $\Xi$ in \eqref{eq:link_function_lower_bound} and \eqref{eq:link_function_upper_bound}, as well as the definition of the operator $\mathcal{D}(\cdot)$ in \eqref{eq:def_D_operator}, the desired expression follows.
\end{proof}


\subsection{Proof of Lemma \ref{lem:smoothness_algebra}}

\lSA*
\begin{proof}
    The proof of this lemma is similar to the proof of Lemma \ref{lem:convexity_algebra}. One major difference is that we work with terms of the form $\llangle A + A^T, Y \rrangle$ instead of terms $\llangle A, Y \rrangle$.

    Following the steps of the proof of Lemma \ref{lem:convexity_algebra}, we get:
    \begin{align*}
        \llangle \nabla \Loglikelihood, H\rrangle &=   \frac{1}{m}\sum_{k=1}^{m}
        h_k \llangle (A_k+ A_k^T) Z, H \rrangle \text{ where }
        h_k =  \frac{g'(z_k)\left(g(z_k) - w_k\right)}{g(z_k)(1-g(z_k))}, \ z_k = \llangle A_k,ZZ^T \rrangle. \\
        g(z_k) - g(z^*_k) &=  g'(y_k) \left(\llangle A_k,\solset \Delta^T + \Delta \solset^T\rrangle + \llangle A_k,\Delta \Delta^T\rrangle\right)  \quad \text{for some } y_k \text{ in the interval between } z_k \text{ and } z^*_k
        \end{align*}
    By \eqref{eq:identity_transpose} and \eqref{eq:identity_shift}, we get:
    \begin{align*}
        \llangle (A_k+ A_k^T) Z, H \rrangle &= \llangle A_k+ A_k^T, HZ^T \rrangle \\
        \llangle A_k,\solset \Delta^T + \Delta \solset^T\rrangle + \llangle A_k,\Delta \Delta^T\rrangle &= \llangle A_k + A_k^T,\solset \Delta^T\rrangle + \frac{1}{2}\llangle A_k + A_k^T,\Delta \Delta^T \rrangle 
    \end{align*}
   Putting together the equations above, we get:
    \begin{align}\label{eq:nabla_l_h}
        \llangle \nabla \Loglikelihood, H\rrangle &=   \frac{1}{m}\sum_{k=1}^{m} \frac{g'(z_k)g'(y_k)}{g(z_k)(1-g(z_k))} 
        \left( \llangle A_k + A_k^T,\solset \Delta^T\rrangle + \frac{1}{2}\llangle A_k + A_k^T,\Delta \Delta^T \rrangle \right)
        \left( \llangle A_k+ A_k^T, HZ^T \rrangle \right)
    \end{align}
    Next, we invoke two straightforward inequalities which apply to any sequence of scalars $(a_k)_{k \in [m]}, (b_k)_{k \in [m]}, \text{ and } (c_k)_{k \in [m]}$ with $a_k \geq 0 \ \forall \ k$:
    \begin{align*}
        \left(\frac{1}{m}\sum_{k = 1}^m a_k b_k c_k\right)^2 &\leq \left(\frac{1}{m}\sum_{k = 1}^m a_k b_k^2 \right) \left(\frac{1}{m}\sum_{k = 1}^m a_k c_k^2 \right) \\
        \left(\frac{1}{m}\sum_{k = 1}^m a_k b_k^2 \right) &\leq \left(\max_{k \in [m]} a_k\right) \left(\frac{1}{m}\sum_{k = 1}^m b_k^2 \right)
    \end{align*}
The first inequality can be viewed as a form of the Cauchy-Schwarz inequality and the second, a form of Hölder's inequality.

Squaring both sides of the equation in \eqref{eq:nabla_l_h} and applying these inequalities with 
\begin{align*}
    a_k = \frac{g'(z_k)g'(y_k)}{g(z_k)(1-g(z_k))}, \ b_k = \llangle A_k + A_k^T,\solset \Delta^T\rrangle + \frac{1}{2}\llangle A_k + A_k^T,\Delta \Delta^T \rrangle, \ c_k = \llangle A_k+ A_k^T, HZ^T \rrangle,
\end{align*}
and observing that $\max_{k \in [m]} a_k \leq \Xi$ (using arguments similar to those in Lemma \ref{lem:convexity_algebra}), we get
\begin{align*}
    \llangle \nabla \Loglikelihood, H\rrangle^2 &\leq \Xi^2 \left( \frac{1}{m}\sum_{k = 1}^m(\llangle A_k + A_k^T,\solset \Delta^T\rrangle + \frac{1}{2}\llangle A_k + A_k^T,\Delta \Delta^T \rrangle)^2 \right) \left(\frac{1}{m}\sum_{k = 1}^m \llangle A_k+ A_k^T, HZ^T \rrangle^2 \right) \\
    &\leq 2\Xi^2 \left( \left(\frac{1}{m}\sum_{k = 1}^m\llangle A_k + A_k^T,\solset \Delta^T\rrangle^2\right) + \frac{1}{4}\left(\frac{1}{m}\sum_{k = 1}^m \llangle A_k + A_k^T,\Delta \Delta^T \rrangle^2\right) \right) \left(\frac{1}{m}\sum_{k = 1}^m \llangle A_k+ A_k^T, HZ^T \rrangle^2 \right) \\
    &= 2 \Xi^2 \left(\Tilde{\mathcal{D}}(\Delta\solset^T) + \frac{1}{4}\Tilde{\mathcal{D}}(\Delta\Delta^T)\right) \, \Tilde{\mathcal{D}}(HZ^T),
\end{align*}
giving us the bound we want.
\end{proof}





