In this concluding section, we prove the main results of our paper, namely Lemma \ref{lem:strong_convexity}, Lemma \ref{lem:smoothness}, and Theorem \ref{thm:main}.

\subsection{Proof of Lemma \ref{lem:strong_convexity}}
As mentioned in Section \ref{sec:proof}, Lemma \ref{lem:strong_convexity} follows from Lemmas \ref{lem:convexity_algebra}, \ref{lem:convexity_lowerbound}, and \ref{lem:convexity_upperbound}; these are proven in Appendices \ref{sec:initial_lemmas}, \ref{sec:lower_bound}, and \ref{sec:upper_bound} respectively. We restate the results here for convenience.

\lSCA*
\lSCLB*
\lSCUB*
\lSCHP*
\begin{proof}
    Our proof strategy will be to put together the statements of the three lemmas and work backwards to calculate the value of the parameter $\epsilon$ needed from Lemmas \ref{lem:convexity_lowerbound} and \ref{lem:convexity_upperbound} (call them $\epsilon_1$ and $\epsilon_2$ for now). Combining the three aforementioned lemmas gives us:
    \begin{align*}
        \llangle \nabla \Loglikelihood, \Delta \rrangle &\geq \frac{\xi\gamma}{2} \left((1- \epsilon_1)\sigma^*_r\norm{\Delta}_F^2 + 2 \llangle \solset_U \Delta_V^T, \Delta_U \solset_V^T \rrangle \right) - \frac{25\Xi\gamma}{4}\left(\epsilon_2\sigma^*_r\norm{\Delta}_F^2 \right) \\
        &= \frac{2\xi(1- \epsilon_1) - 25\Xi\epsilon_2}{4} \gamma\sigma^*_r\norm{\Delta}_F^2 + \xi\gamma \llangle \solset_U \Delta_V^T, \Delta_U \solset_V^T \rrangle
    \end{align*}
    Recall from \eqref{eq:def_regularizer} that $\mathcal{R}(Z) = \norm{Z^TDZ}_F^2$. Therefore, $\nabla \mathcal{R}(Z) = 4DZZ^TDZ$. Using this identity and \eqref{eq:objective_function}, we get:
    \begin{align*}
        \llangle \nabla f, \Delta \rrangle &= \llangle \nabla \Loglikelihood, \Delta \rrangle + \frac{\lambda}{4}\llangle \nabla \mathcal{R}, \Delta \rrangle \\
        &\geq \frac{2\xi(1- \epsilon_1) - 25\Xi\epsilon_2}{4} \gamma\sigma^*_r\norm{\Delta}_F^2 + \xi\gamma \llangle \solset_U \Delta_V^T, \Delta_U \solset_V^T \rrangle + \lambda DZZ^TDZ.
    \end{align*}
    We focus on the last two terms. Define $\lambda' =  \frac{2\lambda}{\xi\gamma}$. Then
    \begin{align*}
        \xi\gamma \llangle \solset_U \Delta_V^T, \Delta_U \solset_V^T \rrangle + \lambda DZZ^TDZ = \frac{\xi\gamma}{2} \left(2\llangle \solset_U \Delta_V^T, \Delta_U \solset_V^T \rrangle + \lambda' DZZ^TDZ\right)
    \end{align*}
    Following the steps laid out in \citet{zheng2016convergence} (Appendix C.1), we get the inequality:
    \begin{align*}
        2\llangle \solset_U \Delta_V^T, \Delta_U \solset_V^T \rrangle + \lambda' DZZ^TDZ \geq \frac{\lambda'}{2}\norm{\solset^TD\Delta}_F^2 - \frac{7\lambda'}{2}\norm{\Delta}_F^4 + \left(\lambda' - \frac{1}{2}\right)\Tr{\solset^TD\Delta \solset^TD\Delta}
    \end{align*}
    We know that $\lambda = \frac{\xi\gamma}{4}$ (see \eqref{eq:objective_function}), which implies $\lambda' = 1/2$. Thus, the last term in the above inequality is cancelled out. Plugging this inequality back into the expression above, we get:
    \begin{align*}
        \llangle \nabla f, \Delta \rrangle &\geq \frac{2\xi(1- \epsilon_1) - 25\Xi\epsilon_2}{4} \gamma\sigma^*_r\norm{\Delta}_F^2 + \xi\gamma \llangle \solset_U \Delta_V^T, \Delta_U \solset_V^T \rrangle + \lambda DZZ^TDZ \\
        &\geq \frac{2\xi(1- \epsilon_1) - 25\Xi\epsilon_2}{4} \gamma\sigma^*_r\norm{\Delta}_F^2 + \frac{\xi\gamma}{2} (2\llangle \solset_U \Delta_V^T, \Delta_U \solset_V^T \rrangle + \lambda' DZZ^TDZ) \\
        &\geq \frac{2\xi(1- \epsilon_1) - 25\Xi\epsilon_2}{4} \gamma\sigma^*_r\norm{\Delta}_F^2 + \frac{\xi\gamma}{2} \left(\frac{1}{4}\norm{\solset^TD\Delta}_F^2 - \frac{7}{4}\norm{\Delta}_F^4\right) \\
        &\geq \frac{4\xi(1- \epsilon_1) - 50\Xi\epsilon_2  - 7\xi\epsilon_2}{8} \gamma\sigma^*_r\norm{\Delta}_F^2  + \frac{\xi\gamma}{8}\norm{\solset^TD\Delta}_F^2
    \end{align*}
Choosing $\epsilon_1 = 1/8$ and $\epsilon_2 = \tau/50 = \xi/(50\Xi)$ gives us $4\xi(1- \epsilon_1) - 50\Xi\epsilon_2  - 7\xi\epsilon_2 \geq 2\xi$. Therefore,
    \begin{align*}
        \llangle \nabla f, \Delta \rrangle &\geq \frac{\xi\gamma}{4}\sigma^*_r\norm{\Delta}_F^2  + \frac{\xi\gamma}{8}\norm{\solset^TD\Delta}_F^2
    \end{align*}
    The number of samples needed for Lemma \ref{lem:convexity_lowerbound} to hold with probability at least $1-\delta/2$ is  
    $$m_1 \geq 96 \mu r  \left(\kappa/\epsilon_1\right)^2 n\log\left(2n/\delta\right) = 6144 \mu r  \kappa^2 n\log\left(2n/\delta\right)$$
    The number of samples needed for Lemma \ref{lem:convexity_upperbound} to hold with probability at least $1-\delta/2$ is  
    $$m_2 \geq 845  \left(\mu r \kappa/\epsilon_2\right)^2 n \log\left(n/\delta\right) \geq 2112500  \left(\mu r \kappa /\tau \right)^2 n \log\left(2n/\delta\right)$$
    The two lemmas jointly hold with probability at least $1 - \delta$. Clearly, the sample complexity requirement from Lemma \ref{lem:convexity_upperbound} is higher. Thus, we can conclude that given  $m \geq 10^7\left(\mu r \kappa (\Xi/\xi)\right)^2 n \log\left(2n/\delta\right)$ samples, with probability at least $1-\delta$,
    \begin{align*}
        \llangle \nabla f, \Delta \rrangle &\geq \frac{\xi\gamma}{4}\sigma^*_r\norm{\Delta}_F^2  + \frac{\xi\gamma}{8}\norm{\solset^TD\Delta}_F^2 \ \forall \ Z \in \mathcal{H} \cup \mathcal{B}(\tau/50) \cup \overline{C} ,
    \end{align*}
\end{proof}

\subsection{Proof of Lemma \ref{lem:smoothness}}
Lemma \ref{lem:smoothness} follows from Lemmas \ref{lem:smoothness_algebra} and \ref{lem:smoothness_upperbound}; these are proven in Appendices \ref{sec:initial_lemmas} and \ref{sec:upper_bound} respectively. We restate the results here for convenience.
\lSA*
\lSUB*
\lSHP*
\begin{proof}
    From \eqref{eq:objective_function}, we get that 
    \begin{align}\label{eq:lem2_1}
        \nabla f &= \nabla \Loglikelihood + \frac{\lambda}{4} \nabla \mathcal{R} = \nabla \Loglikelihood + \lambda DZZ^TDZ \nonumber \\
        \therefore \norm{\nabla f}_F^2 &= \norm{\nabla \Loglikelihood + \lambda DZZ^TDZ}_F^2 \leq (\norm{\nabla \Loglikelihood}_F + \norm{\lambda DZZ^TDZ}_F)^2  \nonumber \\ 
        &\leq 2(\norm{\nabla \Loglikelihood}_F^2 + \lambda^2\norm{DZZ^TDZ}_F^2)
    \end{align}
    We have assumed that $Z \in \mathcal{B}(1)$, which implies $\norm{\Delta}_F^2 \leq \sigma^*_r \leq \sigma^*_1$. Using this bound along with
    the analysis in \citet{zheng2016convergence} (Appendix C.2), we get:
    \begin{align}\label{eq:lem2_2}
        \norm{DZZ^TDZ}_F^2 &\leq 6(\norm{\Delta}_F^2 + 4\sigma^*_1)\norm{\Delta}_F^2 \norm{Z}_2^2 + 4\sigma^*_1\norm{\solset^TD\Delta}_F^2 \nonumber \\
        &\leq 30\sigma^*_1\norm{\Delta}_F^2 \norm{Z}_2^2 + 4\sigma^*_1\norm{\solset^TD\Delta}_F^2 \  \nonumber \\
        &\leq 180(\sigma^*_1)^2\norm{\Delta}_F^2 + 4\sigma^*_1\norm{\solset^TD\Delta}_F^2 \ \quad (\norm{Z}_2^2 \leq 6\sigma^*_1)
    \end{align}
    The last bound can be derived as follows:
    \begin{align*}
        \norm{Z}_2^2 = \norm{\solset + \Delta}_2^2 \leq (\norm{\solset}_2 + \norm{\Delta}_2)^2 \leq 2(\norm{\solset}_2^2 + \norm{\Delta}_2^2) \leq 2(\norm{\solset}_2^2 + \norm{\Delta}_F^2) \leq 2(2\sigma^*_1 + \sigma^*_1) = 6\sigma^*_1
    \end{align*}
    Combining the bounds from Lemma \ref{lem:smoothness_algebra} and Lemma \ref{lem:smoothness_upperbound}, we see that if the number of samples $m$ is at least $2n\log(4n/\delta)$, then with probability at least $1-\delta$, $\forall Z \in \overline{\Incoherentset}$,
    \begin{align}\label{eq:lem2_3}
        \llangle \nabla \Loglikelihood, W \rrangle^2  &\leq 2 \Xi^2 \left(\mathcal{D}(\Delta\solset^T) + \frac{1}{4}\mathcal{D}(\Delta\Delta^T)\right) \, \mathcal{D}(WZ^T) \nonumber \\
        &\leq 2 \Xi^2 \left(16\gamma(\mu r \sigma^*_1) \norm{\Delta}_F^2 + 104 \gamma(\mu r \sigma^*_1) \norm{\Delta}_F^2\right) \, 192 \gamma(\mu r \sigma^*_1)  \norm{W}^2_F \nonumber \\
        &= 46080 (\Xi \gamma \mu r \sigma^*_1)^2 \norm{\Delta}_F^2 \norm{W}^2_F \nonumber \\
        \therefore \norm{\nabla \Loglikelihood}_F^2 &= \sup_{W \in \Real{n \times r}: \norm{W}_F = 1} \llangle \nabla \Loglikelihood, W \rrangle^2 \nonumber \\
        &\leq 46080 (\Xi \gamma \mu r \sigma^*_1)^2 \norm{\Delta}_F^2 
    \end{align}
    Putting together the bounds in \eqref{eq:lem2_1}, \eqref{eq:lem2_2}, and \eqref{eq:lem2_3}, and plugging in the value of $\lambda = \xi\gamma/4$, we see that if the number of samples $m$ is at least $2n\log(4n/\delta)$, then with probability at least $1-\delta$, $\forall \ Z \in \mathcal{B} \cap \overline{\Incoherentset}$,
    \begin{align*}
        \norm{\nabla f}_F^2 &\leq 2\left(46080 (\Xi \gamma \mu r \sigma^*_1)^2 \norm{\Delta}_F^2 + 12 (\xi\gamma\sigma^*_1)^2\norm{\Delta}_F^2\right)  + \frac{(\xi\gamma)^2}{2}\sigma^*_1\norm{\solset^TD\Delta}_F^2  \\
        &\leq 10^5 (\Xi \gamma \mu r \sigma^*_1)^2 \norm{\Delta}_F^2 + \frac{(\xi\gamma)^2}{2}\sigma^*_1\norm{\solset^TD\Delta}_F^2
    \end{align*}
    
\end{proof}

\subsection{Proof of Theorem \ref{thm:main}}
Lemmas \ref{lem:strong_convexity} and \ref{lem:smoothness} are the two key ingredients needed to prove the main theorem of this paper.
\tM*
\begin{proof}
We begin by following the standard steps in the analysis of gradient descent.
\begin{align*}
    \norm{\Delta(Z^{t+1})}_F^2 
    &= \norm{Z^{t+1} - \solset(Z^{t+1})}_F^2 \\
    &\leq \norm{Z^{t+1} - \solset(Z^{t})}_F^2 \\
    &= \norm{\mathcal{P}_{\mathcal{H}}(\mathcal{P}_{\Incoherentset}\left(Z^t - \eta \nabla f(Z^t) \right)) - \solset(Z^{t})}_F^2 \\
    &\leq \norm{Z^t - \eta \nabla f(Z^t) - \solset(Z^{t})}_F^2 \\
    &= \norm{\Delta(Z^{t}) - \eta \nabla f(Z^t)}_F^2\\    
    &= \norm{\Delta(Z^{t})}_F^2 + \eta^2 \norm{\nabla f(Z^t)}_F^2 - 2 \eta \llangle \nabla f(Z^t), \Delta(Z^{t}) \rrangle  
\end{align*}
The first inequality comes from the fact that $\solset(Z^{t+1})$ is the closest point in $\solset$ to $Z^{t + 1}$; this is by definition of $\solset(Z^{t+1})$. The second inequality follows from the fact that $\solset(Z^t) \in \Incoherentset$ (by Lemma \ref{lem:solset_in_C}) and $\solset(Z^t) \in \mathcal{H}$ (by assumption); thus, successive projections of the iterate on to $\Incoherentset$ and $\mathcal{H}$ can only bring it closer to $\solset(Z^t)$.

Next, suppose the following bounds hold for some positive constants $a, b, c,$ and $d$ and for all $t \in \mathbb{Z}_+$:
\begin{align}
    \llangle \nabla f(Z^t), \Delta(Z^t) \rrangle &\geq a \norm{\Delta(Z^t)}_F^2 + c \norm{\Delta(Z^t)^TD\solset(Z^t)}_F^2 \label{eq:thm_pf_lower} \\
    \norm{ \nabla f(Z^t) }_F^2 &\leq b \norm{\Delta(Z^t)}_F^2 + d \norm{\Delta(Z^t)^TD\solset(Z^t)}_F^2    \label{eq:thm_pf_upper}
\end{align}
It follows that:
\begin{align}\label{eq:iterate_inequality}
    \norm{\Delta(Z^{t+1})}_F^2 &\leq \norm{\Delta(Z^{t})}_F^2 + \eta^2 \norm{\nabla f(Z^t)}_F^2 - 2 \eta \llangle \nabla f(Z^t), \Delta(Z^{t}) \rrangle \nonumber \\
    &\leq (1 - 2\eta a + \eta^2 b) \norm{\Delta(Z^t)}_F^2 + (\eta^2 d - 2 \eta c) \norm{\Delta(Z^t)^TD\solset(Z^t)}_F^2 \nonumber \\
    &\leq (1 - \eta a) \norm{\Delta(Z^t)}_F^2,
\end{align}
provided $\eta \leq \min(a/b, 2c/d)$. The last step can be justified  as follows:
\begin{align*}
    \eta &\leq \frac{a}{b} \Rightarrow (1 - 2\eta a + \eta^2 b) \leq 1 - \eta a, \quad 
    \eta \leq \frac{2c}{d} \Rightarrow \eta^2 d - 2 \eta c \leq 0
\end{align*}
Further, if $\eta \leq 1/a$, then $1 - \eta a \geq 0$, implying that the right-hand side of \eqref{eq:iterate_inequality} remains positive. This allows us to use the inequality repeatedly to yield:
\begin{align*}
    \norm{\Delta(Z^{t})}_F^2 \leq (1-\eta a)^t \norm{\Delta(Z^{0})}_F^2 \ \forall \ t \in \mathbb{Z}_+
\end{align*}

Finally, observe that we have assumed the number of samples given, $m$, is at least $10^7\left(\mu r \kappa/\tau\right)^2 n \log\left(8n/\delta\right)$. This ensures that with probability at least $1-\delta$, both Lemmas \ref{lem:strong_convexity} and \ref{lem:smoothness} hold. Lemmas \ref{lem:strong_convexity} and \ref{lem:smoothness} imply that the inequalities \eqref{eq:thm_pf_lower} and \eqref{eq:thm_pf_upper} hold for all $Z \in \mathcal{H} \cap \mathcal{B}(\tau/50) \cap \overline{\Incoherentset}$ with parameters:
\begin{align*}
    {a = \frac{\xi\gamma}{4} \sigma^*_r, \quad b = 10^5 (\Xi \gamma \mu r \sigma^*_1)^2, \quad c = \frac{\xi\gamma}{8}, \quad d = \frac{(\xi\gamma)^2}{2} \sigma^*_1}
\end{align*}
Given these parameters, as long as the stepsize $\eta$ satisfies $\eta \leq a/b = 2.5 \cdot 10^{-6} (\tau/\mu r \kappa)^2 /\alpha $, the other conditions on $\eta$ are automatically satisfied.
\end{proof}



