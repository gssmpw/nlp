\documentclass[letterpaper,11pt]{article}
\usepackage[in]{fullpage}

\usepackage{bm}
\usepackage{comment}
\usepackage{microtype}
\usepackage{nicefrac}
\usepackage{enumitem} %
\usepackage{amsmath,amssymb,amsfonts,amsthm}
\usepackage{thmtools,thm-restate}
\usepackage{algorithm}
\usepackage{algorithmic}
\usepackage{subcaption}
\usepackage{natbib}  
\usepackage{graphicx}
\usepackage{appendix}
\usepackage{hyperref}
\hypersetup{
  breaklinks  = true,
  colorlinks   = true, %
  urlcolor     = black, %
  linkcolor    = red, %
  citecolor   = blue %
}

\makeatletter
\newsavebox{\@brx}
\newcommand{\llangle}[1][]{\savebox{\@brx}{\(\m@th{#1\langle}\)}%
\mathopen{\copy\@brx\kern-0.5\wd\@brx\usebox{\@brx}}}
\newcommand{\rrangle}[1][]{\savebox{\@brx}{\(\m@th{#1\rangle}\)}%
\mathclose{\copy\@brx\kern-0.5\wd\@brx\usebox{\@brx}}}
\makeatother

\newcommand{\Incoherentset}{\mathcal{C}}
\newcommand{\Dataset}{\mathcal{D}}
\newcommand{\Loglikelihood}{\mathcal{L}}
\newcommand{\Tr}[1]{\textsf{Tr}(#1)}
\newcommand{\solset}{\Phi}
\newcommand{\solz}{\solset(Z)}
\newcommand{\delz}{\Delta(Z)}
\newcommand{\Real}[1]{\mathbb{R}^{#1}}
\newcommand{\norm}[1]{\left\Vert #1 \right\Vert}
\newcommand{\vectext}[1]{\mathsf{vec}(#1)}


\theoremstyle{plain}
\newtheorem{theorem}{Theorem}[section]
\newtheorem{proposition}[theorem]{Proposition}
\newtheorem{lemma}[theorem]{Lemma}
\newtheorem{corollary}[theorem]{Corollary}
\theoremstyle{definition}
\newtheorem{definition}[theorem]{Definition}
\newtheorem{assumption}[theorem]{Assumption}
\theoremstyle{remark}
\newtheorem{remark}[theorem]{Remark}

\title{Recommendations from Sparse Comparison Data: \\ Provably Fast Convergence for Nonconvex Matrix Factorization}

\author{
  Suryanarayana Sankagiri\\
  EPFL, Switzerland \\ 
  \href{mailto:suryanarayana.sankagiri@epfl.ch}{suryanarayana.sankagiri@epfl.ch}
  \and
  Jalal Etesami \\
  TUM, Germany \\
  \href{mailto:j.etesami@tum.de}{j.etesami@tum.de}
  \and
  Matthias Grossglauser \\
  EPFL, Switzerland \\
  \href{mailto:matthias.grossglauser@epfl.ch}{matthias.grossglauser@epfl.ch}
}



\begin{document}
\maketitle

\begin{abstract}
\begin{abstract}


The choice of representation for geographic location significantly impacts the accuracy of models for a broad range of geospatial tasks, including fine-grained species classification, population density estimation, and biome classification. Recent works like SatCLIP and GeoCLIP learn such representations by contrastively aligning geolocation with co-located images. While these methods work exceptionally well, in this paper, we posit that the current training strategies fail to fully capture the important visual features. We provide an information theoretic perspective on why the resulting embeddings from these methods discard crucial visual information that is important for many downstream tasks. To solve this problem, we propose a novel retrieval-augmented strategy called RANGE. We build our method on the intuition that the visual features of a location can be estimated by combining the visual features from multiple similar-looking locations. We evaluate our method across a wide variety of tasks. Our results show that RANGE outperforms the existing state-of-the-art models with significant margins in most tasks. We show gains of up to 13.1\% on classification tasks and 0.145 $R^2$ on regression tasks. All our code and models will be made available at: \href{https://github.com/mvrl/RANGE}{https://github.com/mvrl/RANGE}.

\end{abstract}


\end{abstract}

\section{Introduction}\label{sec:introduction}

\section{Introduction}

Video generation has garnered significant attention owing to its transformative potential across a wide range of applications, such media content creation~\citep{polyak2024movie}, advertising~\citep{zhang2024virbo,bacher2021advert}, video games~\citep{yang2024playable,valevski2024diffusion, oasis2024}, and world model simulators~\citep{ha2018world, videoworldsimulators2024, agarwal2025cosmos}. Benefiting from advanced generative algorithms~\citep{goodfellow2014generative, ho2020denoising, liu2023flow, lipman2023flow}, scalable model architectures~\citep{vaswani2017attention, peebles2023scalable}, vast amounts of internet-sourced data~\citep{chen2024panda, nan2024openvid, ju2024miradata}, and ongoing expansion of computing capabilities~\citep{nvidia2022h100, nvidia2023dgxgh200, nvidia2024h200nvl}, remarkable advancements have been achieved in the field of video generation~\citep{ho2022video, ho2022imagen, singer2023makeavideo, blattmann2023align, videoworldsimulators2024, kuaishou2024klingai, yang2024cogvideox, jin2024pyramidal, polyak2024movie, kong2024hunyuanvideo, ji2024prompt}.


In this work, we present \textbf{\ours}, a family of rectified flow~\citep{lipman2023flow, liu2023flow} transformer models designed for joint image and video generation, establishing a pathway toward industry-grade performance. This report centers on four key components: data curation, model architecture design, flow formulation, and training infrastructure optimization—each rigorously refined to meet the demands of high-quality, large-scale video generation.


\begin{figure}[ht]
    \centering
    \begin{subfigure}[b]{0.82\linewidth}
        \centering
        \includegraphics[width=\linewidth]{figures/t2i_1024.pdf}
        \caption{Text-to-Image Samples}\label{fig:main-demo-t2i}
    \end{subfigure}
    \vfill
    \begin{subfigure}[b]{0.82\linewidth}
        \centering
        \includegraphics[width=\linewidth]{figures/t2v_samples.pdf}
        \caption{Text-to-Video Samples}\label{fig:main-demo-t2v}
    \end{subfigure}
\caption{\textbf{Generated samples from \ours.} Key components are highlighted in \textcolor{red}{\textbf{RED}}.}\label{fig:main-demo}
\end{figure}


First, we present a comprehensive data processing pipeline designed to construct large-scale, high-quality image and video-text datasets. The pipeline integrates multiple advanced techniques, including video and image filtering based on aesthetic scores, OCR-driven content analysis, and subjective evaluations, to ensure exceptional visual and contextual quality. Furthermore, we employ multimodal large language models~(MLLMs)~\citep{yuan2025tarsier2} to generate dense and contextually aligned captions, which are subsequently refined using an additional large language model~(LLM)~\citep{yang2024qwen2} to enhance their accuracy, fluency, and descriptive richness. As a result, we have curated a robust training dataset comprising approximately 36M video-text pairs and 160M image-text pairs, which are proven sufficient for training industry-level generative models.

Secondly, we take a pioneering step by applying rectified flow formulation~\citep{lipman2023flow} for joint image and video generation, implemented through the \ours model family, which comprises Transformer architectures with 2B and 8B parameters. At its core, the \ours framework employs a 3D joint image-video variational autoencoder (VAE) to compress image and video inputs into a shared latent space, facilitating unified representation. This shared latent space is coupled with a full-attention~\citep{vaswani2017attention} mechanism, enabling seamless joint training of image and video. This architecture delivers high-quality, coherent outputs across both images and videos, establishing a unified framework for visual generation tasks.


Furthermore, to support the training of \ours at scale, we have developed a robust infrastructure tailored for large-scale model training. Our approach incorporates advanced parallelism strategies~\citep{jacobs2023deepspeed, pytorch_fsdp} to manage memory efficiently during long-context training. Additionally, we employ ByteCheckpoint~\citep{wan2024bytecheckpoint} for high-performance checkpointing and integrate fault-tolerant mechanisms from MegaScale~\citep{jiang2024megascale} to ensure stability and scalability across large GPU clusters. These optimizations enable \ours to handle the computational and data challenges of generative modeling with exceptional efficiency and reliability.


We evaluate \ours on both text-to-image and text-to-video benchmarks to highlight its competitive advantages. For text-to-image generation, \ours-T2I demonstrates strong performance across multiple benchmarks, including T2I-CompBench~\citep{huang2023t2i-compbench}, GenEval~\citep{ghosh2024geneval}, and DPG-Bench~\citep{hu2024ella_dbgbench}, excelling in both visual quality and text-image alignment. In text-to-video benchmarks, \ours-T2V achieves state-of-the-art performance on the UCF-101~\citep{ucf101} zero-shot generation task. Additionally, \ours-T2V attains an impressive score of \textbf{84.85} on VBench~\citep{huang2024vbench}, securing the top position on the leaderboard (as of 2025-01-25) and surpassing several leading commercial text-to-video models. Qualitative results, illustrated in \Cref{fig:main-demo}, further demonstrate the superior quality of the generated media samples. These findings underscore \ours's effectiveness in multi-modal generation and its potential as a high-performing solution for both research and commercial applications.

\section{Model}\label{sec:model}

\subsection{The Generative Model}\label{sec:generative_model}
Let there be $n_1$ users and $n_2$ items. %
Each user $u$ and each item $i$ has a $r$-dimensional feature vector. %
The inner product of these two feature vectors gives the utility (or score) $x^*_{u,i}$ that user $u$ has for item $i$. 
This modeling assumption implies that
the score matrix $X^* \in \Real{n_1 \times n_2}$ has rank $r$, and thus admits the following rank-$r$ SVD:
\begin{align}
    X^* = U^* \Sigma^* V^{*T},
\end{align}
where $U^* \in \Real{n_1 \times r}$ and $V^* \in \Real{n_2 \times r}$ are matrices that satisfy $U^{*T} U^* = V^{*T} V^* = I_r$, and $\Sigma^* \in \Real{r \times r}$ is a diagonal matrix with entries $\sigma_1^* \geq \ldots \geq \sigma_r^* > 0$. Let $\kappa \triangleq \sigma_1^*/\sigma_r^*$ denote the condition number of $\Sigma^*$.

Let $n = n_1 + n_2$. Define $Z^* \in \Real{n \times r}$ and $Y^* \in \Real{n \times n}$ as follows:
\begin{align}\label{eq:matrix_z}
    Z^* &= 
    \begin{bmatrix}
        U^* \\ V^*
    \end{bmatrix}
    \Sigma^{* 1/2}, \\ 
    Y^* &= Z^*Z^{*T} = 
    \begin{bmatrix}
        U^*\Sigma^*U^{*T} & X^*\\
        X^{*T} & V^*\Sigma^*V^{*T}
    \end{bmatrix}.
\end{align}
From here on, we shall refer to $Z^*$ as the ground-truth matrix. Note that the singular values of $Z^*$ are $\sqrt{2\sigma_1^*}, \ldots, \sqrt{2\sigma_r^*}$.

We are given a dataset $\Dataset$ where each data point represents a comparison made by a user between two items. The size of the dataset, \textit{i.e.,} the number of data points, is represented by $m$. We index the dataset by $k$. Each data point $\Dataset_k$ is of the form $((u; i, j), w)$ and is sampled randomly as follows. The user index $u$ is chosen uniformly at random from $[n_1]$. The pair of item indices $(i,j)$ is chosen uniformly at random from the set of $n_2(n_2-1)$ pairs of distinct items. The item pair $(i,j)$ is sampled independently from $u$. The triplets for different datapoints are sampled independently of each other.

The variable $w$ reflects the outcome of the comparison made by the user $u$ between items $i$ and $j$. In the \textit{noisy} setting, $w$ is an indicator for the outcome of the comparison; it is one if $i$ is chosen and zero if $j$ is chosen. Given a triplet $(u; i, j)$, $w$ is a Bernoulli random variable with parameter $g(x^*_{u,i} - x^*_{u,j})$, where $g: \Real{} \rightarrow (0,1)$ is a known \textit{link function} that translates real-valued preferences to a binary scale. In the \textit{noiseless} setting, $w$ is set to the expected value of the corresponding noisy case; \textit{i.e.}, $w = g(x^*_{u,i} - x^*_{u,j})$. 

In this work, we assume we are given noiseless data. We assume the link function is a smooth, strictly increasing function and is symmetric around zero in the following sense: $g(-x) = 1 - g(x)$. For example, $g(x)$ could be the logistic link function: $e^x/(1 + e^x)$; this is the link function found in the Bradley-Terry-Luce choice model. 


\subsubsection{Important Parameters}
 

\paragraph{Incoherence} For any matrix $Z$, let $\norm{Z}_{2, \infty}$ denote the maximum of the $\ell_2$ norm of its rows and let $\norm{Z}_{F}$ denote the Frobenius norm of $Z$. Define the \textit{incoherence parameter} of the ground-truth matrix as 
\begin{align}\label{eq:def_mu}
    \mu \triangleq n(\norm{Z^*}_{2, \infty}^2/\norm{Z^*}_{F}^2).
\end{align}
In principle, $\mu$ can take values from $1$ to $n$. However, the sample complexity worsens with $\mu$.

\paragraph{Link Function Bounds}  Let $I$ denote the interval {$[-{24 \mu (\norm{Z^*}_F^2}/{n}), {24 \mu (\norm{Z^*}_F^2}/{n})]$.} %
Let $\xi$ and $\Xi$ be lower and upper bounds for the following expression:
\begin{align}\label{eq:link_function_lower_bound}
    \xi &\triangleq  \min_{(x, y) \in I \times I} \frac{g'(x)g'(y)}{g(x)(1 - g(x))}, \\ 
    \Xi &\triangleq \max_{(x, y) \in I \times I} \frac{g'(x)g'(y)}{g(x)(1 - g(x))}. \label{eq:link_function_upper_bound}
\end{align}
By the assumptions on $g(\cdot)$ stated above, $\xi$ is strictly positive and $\Xi$ is finite.
For the logistic link function, $g'(x)=g(x)(1-g(x))$, which implies $\xi = g'(24 \mu (\norm{Z^*}_F^2/n))$ and $\Xi = 1/4$.


\subsection{The Loss Function}\label{sec:loss_function}
Given any $Z \in \Real{n \times r}$, we interpret $Z$ as the concatenation of some candidate user features $U \in \Real{n_1 \times r}$ and item features $V \in \Real{n_2 \times r}$. 
The likelihood of the dataset $\Dataset$ under $Z$ is simply the probability of observing $\Dataset$ if the data was generated according to the parameters $Z$. 
In this work, we use the maximum likelihood approach to learn the latent parameters. \textit{I.e.,} we use the negative log likelihood as the loss function, which we shall minimize using a gradient-descent-like method. 
Here, we present the loss function and its gradient, using notation that will be useful later on.

Let $e_1, e_2, \ldots e_{n_1}$ denote unit vectors in $\Real{n_1}$ and let $\Tilde{e}_1, \Tilde{e}_2, \ldots, \Tilde{e}_{n_2}$ denote unit vectors in $\Real{n_2}$. Let $\llangle C, D \rrangle = \sum_{i,j} c_{i,j}d_{i,j}$ denote the matrix inner product between two matrices of the same size. Therefore:
\begin{align}\label{eq:def_A1}
    \llangle e_u(\Tilde{e}_i - \Tilde{e}_j)^T, X^* \rrangle = x^*_{u,i} - x^*_{u,j}.
\end{align}

For any triplet $(u; i, j)$, define the corresponding \textit{sampling matrix} $A \in \Real{n \times n}$ to be:
\begin{align}\label{eq:def_A2}
    A = \begin{bmatrix}
        0 & e_u(\Tilde{e}_i - \Tilde{e}_j)^T\\
        0 & 0
    \end{bmatrix} \Rightarrow \llangle A, Y^* \rrangle = x^*_{u,i} - x^*_{u,j}.
\end{align}
In the equation above, $0$ denotes matrices with all entries zero of the appropriate size. With this notation, for any data point $((u; i, j), w)$, we have:
\begin{align}
    \mathbb{P}(w = 1 \, | \, (u; i, j)) = \mathbb{P}(w = 1 \, | \, A) = g(\llangle A, Y^* \rrangle).
\end{align}


Given a binary outcome $w$, the likelihood of the outcome under a Bernoulli distribution with parameter $p$ is $p^{w}(1-p)^{1-w}$. Therefore, the negative log-likelihood of this observation is $-w\log(p) -(1-w)\log(1-p)$. Next, consider a datapoint $((u; i,j), w)$ with the corresponding sampling matrix $A$. The negative log-likelihood of this observation under our model with parameters $Z \in \Real{n \times r}$ is 
\begin{align*}
    -w \log(g(\llangle A, ZZ^T \rrangle)) - (1-w) \log(1 - g(\llangle A, ZZ^T \rrangle)).
\end{align*}
Let $A_k$ denote the sampling matrix corresponding to the datapoint $\mathcal{D}_k$. Then, for the entire dataset, the (normalized) negative log likelihood is given by:
\begin{align}\label{eq:log_likelihood}
    \Loglikelihood(Z) &= \frac{1}{m} \sum_{k = 1}^{m} -w_k \log(g(\llangle A_k, ZZ^T \rrangle)) \nonumber \\
    &\quad - (1-w_k) \log(1 - g(\llangle A_k, ZZ^T \rrangle)).
\end{align}
The gradient of $\Loglikelihood(Z)$ is 
\begin{align} \label{eq:gradient_likelihood}
    \nabla \Loglikelihood(Z) &= \frac{1}{m}\sum_{k=1}^{m}
    h_k (A_k+ A_k^T) Z, \text{ where }\\
    h_k &\triangleq \frac{g'(z_k)\left(g(z_k) - w_k\right)}{g(z_k)(1-g(z_k))}, \ z_k \triangleq \llangle A_k,ZZ^T \rrangle. \nonumber
\end{align}
Here, $\nabla \Loglikelihood(Z)$ is a matrix of the same size as $Z$ while $h_k$ and $z_k$ are scalars. %


\subsection{Symmetries in the Problem}\label{sec:symmetries}
The generative model, and consequently the log likelihood function, is invariant to certain transformations in the parameters. In other words, the problem structure has certain symmetries. We explore these symmetries and their consequences in this section.

\paragraph{Scale Invariance} 
For any ground-truth score matrix $X^*$, the factorization $(U^*, V^*)$ is not unique.
Indeed, for any invertible $r \times r$ matrix $P$, the pair of feature matrices $(U^*P^T, V^*P^{-1})$ is indistinguishable from $(U^*, V^*)$ as they both lead to the same score matrix $X^*$. However, we can distinguish `imbalanced' feature vectors from `balanced' ones by by adding the term $\norm{U^TU - V^TV}_F^2$ to the loss function. Minimizing this regularizer while keeping the log-likelihood constant leads to a pair of feature matrices that are balanced in the norms. In more compact terms, the regularizer can be written as follows:
\begin{align}\label{eq:def_regularizer}
    \mathcal{R}(Z) \triangleq \norm{Z^TDZ}_F^2; \ D \triangleq \begin{bmatrix}
        I_{n_1} & 0\\
        0 & -I_{n_2}
    \end{bmatrix}.
\end{align}
Note that the ground-truth matrix $Z^*$ satisfies $\mathcal{R}(Z^*) = 0$. Combining the regularizer with the negative log likelihood, the objective function becomes:
\begin{align}\label{eq:objective_function}
    f(Z) \triangleq \mathcal{L}(Z) + (\lambda/4)\mathcal{R}(Z),
\end{align}
where $\lambda$ is a positive constant. In this work, we set $\lambda = \xi\gamma/4$; however, in practice, it may be viewed as a hyperparameter. In summary, adding the regularizer $\mathcal{R}(Z)$ factors out the scale-invariance of the problem.

\paragraph{Rotational Invariance}
Beyond the scale invariance, the problem at hand also exhibits rotational invariance.
Let $R$ be any orthogonal matrix in $r$ dimensions, \textit{i.e.}, $R \in \mathbb{R}^{r \times r}$ such that $RR^T=R^TR=I$. The pair of feature matrices $(U^*R, V^*R)$ give rise to the same scores as $(U^*, V^*)$. Thus, one can identify the ground-truth features only up to an orthogonal transformation. Denote this equivalence class of the ground-truth feature matrices by $\solset$:
\begin{align}\label{eq:def_solutionset}
    \solset \triangleq \{ \Tilde{Z}^* \, : \Tilde{Z}^* = Z^* R \text{ for some } \text{orthogonal } R\}.
\end{align}
This equivalence class of solutions naturally gives rise to a new distance metric that measures how close a candidate solution $Z$ is to $\solset.$
Define
\begin{align}\label{eq:def_distance}
     R(Z) &\triangleq \text{arg} \min_{R: R^TR = RR^T = I_r} \norm{Z - Z^* R}_F, \\
    \solz &\triangleq \text{arg} 
    \min_{\Tilde{Z}^* \in \solset} \norm{Z - \Tilde{Z}^*}_F = Z^* R(Z), \label{eq:def_closest_sol} \\
     \delz &\triangleq Z - \solz. \label{eq:def_difference_sol}
\end{align}
We measure the quality of a solution $Z$ by $\norm{\delz}_F$.


\paragraph{Shift Invariance} Comparisons invariably involve computing the difference between the utilities of items. Therefore, the learning problem is invariant to a global shift in the scores. Mathematically, this can be seen as follows. Let $\Tilde{V}^* = V^* + 1v^T$, where $v \in \Real{r}$ and $1 \in \Real{n_2}$ is the vector of all ones. Let $\Tilde{X}^*, \Tilde{Y}^*,$ and $\Tilde{Z}^*$ denote the corresponding quantities derived from $(U^*, \Tilde{V}^*)$. Then for any triplet $(u; i, j)$ and the corresponding sampling matrix $A$, we have $\llangle e_u(\Tilde{e}_i - \Tilde{e}_j)^T, \Tilde{X}^* \rrangle = \llangle e_u(\Tilde{e}_i - \Tilde{e}_j)^T, X^* \rrangle$, which implies $\llangle A, \Tilde{Y}^* \rrangle = \llangle A, Y^* \rrangle.$ Because of this invariance, we assume, without loss of generality, that $1^T V^* = 0$. In words, we assume that the item features of all matrices in $\solset$ sum to zero.

The shift invariance also manifests itself in our objective function $\Loglikelihood(Z)$. It is important to factor out the shift invariance in order to establish a strong-convexity like property (i.e., a curvature) for $\Loglikelihood(Z)$. Therefore, we restrict our attention to the following subspace:
\begin{align}\label{eq:H_hyperplane}
    \mathcal{H} = \{ Z \in \Real{n \times r}: Z = (U, V), 1^TV = 0 \}.
\end{align}
For any $Z = (U, V)$, we shall work with the projection of $Z$ onto $\mathcal{H}$, denoted by $\mathcal{P}_{\mathcal{H}}(Z)$. This projection is given by $(U, JV)$, where $J \triangleq I_{n_2} - 11^T/(n_2)$. Finally, note that by the assumption stated before, $\solset \subseteq \mathcal{H}$.





\section{Algorithm and Result}\label{sec:algorithm}

A naive approach to minimize the loss function \eqref{eq:objective_function} is to simply apply the gradient descent method until one is sufficiently close to convergence. Indeed, in Section \ref{sec:simulations}, we show this works well in practice. However, for proving theoretical guarantees, we need to use \textit{projected gradient descent}. Notably, the projection step involves two successive projections, first onto a set of `incoherent matrices' $\Incoherentset$ and then onto $\mathcal{H}$ (defined in \eqref{eq:H_hyperplane}). The set $\Incoherentset$ is defined as follows:
{\begin{align}\label{eq:def_incoherent_set}
    \Incoherentset \triangleq \left\{Z \in  \mathbb{R}^{n \times r} \, : \, \norm{Z}_{2, \infty} \leq \frac{4}{3}\sqrt{\frac{\mu}{n} }\norm{Z^0}_F \right\}.
\end{align}}
Thus, $\Incoherentset$ contains matrices that are `nearly as incoherent' as $Z^*$ (if $\norm{Z^0}_F \approx \norm{Z^*}_F$). For any $Z \in \Real{n \times r}$, the projection of $Z$ onto $\Incoherentset$, $\mathcal{P}_{\Incoherentset}(Z)$, is a matrix in $\Real{n \times r}$ obtained by clipping the rows of $Z$ to $\beta = (4/3)\sqrt{({\mu}/{n})}\norm{Z^0}_F$:
\begin{align*}
    \forall \ j \in [n], \ \mathcal{P}_{\Incoherentset}(Z)_j &= 
    \begin{cases}
        Z_j & \text{if } \norm{Z_j}_{2} \leq \beta\\
        Z_j (\beta/\norm{Z_j}_{2}) & \text{otherwise}
    \end{cases}.
\end{align*}

The rationale for the projections is the following. One, the objective function displays a strong-convexity like property only within the region of incoherent matrices. The projection operation $\mathcal{P}_\Incoherentset$ ensures that we stay in this region, which is crucial for proving the theoretical results. The second projection, $\mathcal{P}_{\mathcal{H}}$ factors out the shift invariance in the loss function. This is essential in order to establish strong convexity; otherwise, there is no curvature in the direction of invariance. Here, there is a caveat: this second projection may push the iterates out of the set $\Incoherentset$. However, we show (in Lemma \ref{lem:incoherence_of_projection}) that the iterates remain incoherent enough, \textit{i.e.}, they remain in the set {$\overline{\Incoherentset}$}%
, where
\begin{align}\label{eq:def_Cbar}
    \overline{\Incoherentset} \triangleq \left\{Z \in  \mathbb{R}^{n \times r} \, : \, \norm{Z}_{2, \infty} \leq \sqrt{{12\mu}/{n}}\norm{Z^*}_F \right\}
\end{align}

\begin{algorithm}[htbp]
\caption{Projected Gradient Descent}\label{alg:pgd}
\begin{algorithmic}
\STATE {\bfseries Input:} Objective function $f$, initial solution $Z^0 \in \mathbb{R}^{n \times r}$, stepsize $\eta$
\STATE $t \gets 0$
\STATE $Z^0 \gets \mathcal{P}_{\mathcal{H}}\left(\mathcal{P}_{\Incoherentset}\left(Z^0\right)\right)$
\REPEAT
    \STATE $Z^{t+1} \gets \mathcal{P}_{\mathcal{H}}\left(\mathcal{P}_{\Incoherentset}\left(Z^t - \eta \nabla f(Z^t) \right)\right)$
    \STATE $t \gets t+1$
\UNTIL{convergence}
\STATE {\bfseries Output:} $Z^t$
\end{algorithmic}
\end{algorithm}


Our main theorem states that in the noiseless setting, given a sufficiently large dataset and a warm start, Algorithm \ref{alg:pgd} converges exponentially fast to a solution equivalent to the ground-truth matrix. For the sake of conciseness, we introduce the following constants: 
\begin{align*}
    \gamma \triangleq 2/(n_1(n_2 - 1)), \quad
    \tau \triangleq \xi/\Xi, \quad
    \alpha \triangleq \xi \gamma \sigma^*_r.
\end{align*}
Let $\mathcal{B}(\varepsilon) = \{Z: \norm{\Delta(Z)}_{F}^2 \leq \varepsilon\sigma^*_r\}$ denote a `ball' around the true solution. With this notation in place, we can now state the main theorem.
\begin{restatable}{theorem}{tM}\label{thm:main}
    Suppose the following conditions hold:
    \begin{itemize}
        \item The dataset $\mathcal{D}$ consists of $m$ i.i.d. samples generated according to the model presented in Section \ref{sec:generative_model}
        \item The number of samples $m$ is at least $10^7\left(\mu r \kappa / \tau \right)^2 n \log\left(8n/\delta\right)$ for some $\delta \in (0,1)$.
        \item The initial point $Z^0$ lies in $\mathcal{B}(\tau/50)$.
        \item The stepsize satisfies $\eta \alpha \leq 2.5 \cdot 10^{-6} (\tau/\mu r \kappa)^2$.
    \end{itemize}
    Then, with probability at least $1-\delta$, the iterates $Z^1, Z^2, \ldots$ of Algorithm \ref{alg:pgd} satisfy:
    \begin{align*}
        \norm{\Delta(Z^t)}_F^2 \leq \left(1 - \frac{\alpha\eta}{4} \right)^{t} \norm{\Delta(Z^0)}_F^2 \quad \forall \ t \in \mathbb{N}.
    \end{align*}
\end{restatable}
We highlight two important takeaways from the above theorem. First, the dependence on the problem size is $O(nr^2\log n)$, which is near-optimal.
Second, for a well-chosen step-size, the algorithm convergences exponentially at rate $O((\tau/\mu r \kappa )^2)$. Although the constants in the sample complexity result and convergence rate are quite large in the statement of Theorem \ref{thm:main}, our experimental results in Section \ref{sec:simulations} show that in practice, these constants are moderate.
The proof of this theorem follows the typical proof of the convergence of projected gradient descent for a strongly convex and smooth function. The strong-convexity like property comes from Lemma \ref{lem:strong_convexity} and the smoothness property from Lemma \ref{lem:smoothness}. The full proof of Theorem \ref{thm:main} is given in Appendix \ref{sec:main_proofs}.

\newpage







\section{Proof Outline}\label{sec:proof}

We begin by formally stating the two lemmas that are used in the proof of Theorem \ref{thm:main}. 
\begin{restatable}{lemma}{lSCHP}\label{lem:strong_convexity}
    Suppose the number of samples $m$ is at least $10^7\left(\mu r \kappa/\tau\right)^2 n \log\left(2n/\delta\right)$, for some $\delta \in (0,1)$.   
    Then, with probability at least $1-\delta$, $\forall \ Z \in \mathcal{H} \cap \mathcal{B}(\tau/50) \cap \overline{\Incoherentset}$,
    \begin{align*}
        \llangle \nabla f(Z), \delz \rrangle &\geq \frac{\xi\gamma}{4} \norm{\delz}_F^2 \\ &\qquad  + \frac{\xi\gamma}{8} \norm{\Delta(Z)^TD\solset(Z)}_F^2.
    \end{align*}
\end{restatable}

\begin{restatable}{lemma}{lSHP}\label{lem:smoothness}
    Suppose the number of samples $m$ is at least {$2n\log(4n/\delta)$}, for some $\delta \in (0,1)$.
    Then, with probability at least $1-\delta$, $\forall \ Z \in \mathcal{B}(1) \cap \overline{\Incoherentset}$,
    \begin{align*}
        \norm{\nabla f(Z)}^2_F &\leq 10^5 (\Xi \gamma \mu r \sigma^*_1)^2 \norm{\Delta(Z)}_F^2 \\ &\qquad + \frac{(\xi\gamma)^2}{2}\sigma^*_1\norm{\solset(Z)^TD\Delta(Z)}_F^2.
    \end{align*}
\end{restatable}
The statements of these lemmas as well as the strategy we follow to prove them are similar to (and inspired by) those in \citet{zheng2016convergence}.

At a high level, the method for proving both these lemmas is similar. First, the expressions to be bounded, namely $\llangle \nabla f, \Delta \rrangle$ and $\norm{\nabla f}_F^2$, are written out as the sum and product of terms of the following form (Lemmas \ref{lem:convexity_algebra} and \ref{lem:smoothness_algebra}):
\begin{align}\label{eq:def_D_operator}
    \mathcal{D}(Y) &\triangleq \frac{1}{m}\sum_{k = 1}^m \llangle A_k + A_k^T, Y \rrangle^2  \, ; \ Y \in \Real{n \times n}. 
\end{align}
We overload the notation $\mathcal{D}$ to highlight the fact that the operator $\mathcal{D}(\cdot)$ captures the collective action of all the sampling matrices of the dataset $\mathcal{D}$.  Second, we demonstrate that these terms, which capture an empirical mean of i.i.d. random variables, are close to their statistical mean (Lemmas \ref{lem:convexity_lowerbound}, \ref{lem:convexity_upperbound}, and \ref{lem:smoothness_upperbound}). Specifically, we show that with high probability, $\mathcal{D}(Y) \approx \mathbb{E}[\mathcal{D}(Y)]$, uniformly for all $Y$ in some appropriate set. Finally, we put these results together with the appropriate parameters to ensure that the bounds presented in Lemmas \ref{lem:strong_convexity} and \ref{lem:smoothness} hold. 

Before proceeding further, we introduce some new notation. First, we drop the dependence on $Z$ for brevity; \textit{e.g.}, we denote $\nabla f(Z)$ by $\nabla f$. Second, for any matrix $Z \in \Real{n \times r}$, we use $Z_U \in \Real{n_1 \times r}$ to denote the first $n_1$ rows of $Z$ (the user features) and $Z_V \in \Real{n_2 \times r}$ to denote the last $n_2$ rows of $Z$ (the item features). In particular, $Z^*_U = U^*\Sigma^{*1/2}$ and $Z^*_V = V^*\Sigma^{*1/2}$. 


\paragraph{Showing Strong Convexity}
We begin by breaking down the proof of Lemma \ref{lem:strong_convexity} into three smaller lemmas. 
\begin{restatable}{lemma}{lSCA}\label{lem:convexity_algebra}
    For any $Z \in \overline{\Incoherentset}$,
    \begin{align*}
        \llangle \nabla \mathcal{L}, \Delta \rrangle &\geq \frac{\xi}{2} \mathcal{D}\left(\Delta\solset^T\right) - \frac{5\Xi}{8} \mathcal{D}\left(\Delta\Delta^T\right) %
    \end{align*}
\end{restatable}

\begin{restatable}{lemma}{lSCLB}\label{lem:convexity_lowerbound}
    Let some $\epsilon, \delta \in (0, 1)$ be given. Suppose the number of samples $m$ exceeds $96 \mu r  \left(\kappa/\epsilon\right)^2 n\log\left(n/\delta\right)$. Then, with probability at least $1 - \delta$, $\forall \ Z \in \mathcal{H},$
    \begin{align*}
         \mathcal{D}\left(\Delta\solset^T\right) \geq \gamma\left((1- \epsilon)\sigma^*_r\norm{\Delta}_F^2 + 2 \llangle \solset_U \Delta_V^T, \Delta_U \solset_V^T \rrangle \right).
    \end{align*}
\end{restatable}

\begin{restatable}{lemma}{lSCUB}\label{lem:convexity_upperbound}
    Let some $\epsilon, \delta \in (0, 1)$ be given. Suppose the number of samples $m$ exceeds $845  \left(\mu r \kappa/\epsilon\right)^2 n \log\left(n/\delta\right)$. Then, with probability at least $1 - \delta$, $\forall \ Z \in \overline{\Incoherentset} \cap \mathcal{B}(\epsilon)$, 
    \begin{align*}
        \Dataset(\Delta\Delta^T)\leq 10 \epsilon \gamma  \sigma^*_r \norm{\Delta}_F^2.
    \end{align*}
\end{restatable}

Using these three lemmas, Lemma \ref{lem:strong_convexity} can be derived in a straightforward manner (proof in Appendix \ref{sec:main_proofs}). Indeed, if we ignore the cross-term $\llangle \solset_U \Delta_V^T, \Delta_U \solset_V^T \rrangle$ in Lemma \ref{lem:convexity_lowerbound}, it is not hard to see that the three lemmas combined lead to the lower bound $\llangle \nabla \mathcal{L}, \Delta \rrangle \geq O(1) \gamma \sigma^*_r \norm{\Delta}_F^2$. The gradient of the regularizer helps cancel out this cross-term, but leads to the additional $\norm{\Delta D \solset}_F^2$ term.

The steps in the proof of Lemma \ref{lem:convexity_algebra} are algebraic in nature and largely follow the pattern presented in \citet{zheng2016convergence}; the proof is given in Appendix \ref{sec:initial_lemmas}. The main technical contribution of our work is in the proof of Lemmas \ref{lem:convexity_lowerbound} and \ref{lem:convexity_upperbound}. Although the statements of these lemmas are similar to Lemmas 10 and 8 respectively of \citet{zheng2016convergence}, we prove these results in different ways. We outline the broad steps taken to prove these results, filling in the details in Appendices \ref{sec:lower_bound} and \ref{sec:upper_bound} respectively. 


A key step to prove Lemma \ref{lem:convexity_lowerbound} is to show the identity:
\begin{align}
\label{eq:quadratic_form1}
    &\mathcal{D}\left(\Delta\solset^T\right) = v^T S_{\Dataset}v, \ \text{where} \ v \triangleq \vectext{\Delta R^T}, \nonumber \\
    & \quad S_{\Dataset} \triangleq \frac{1}{m} \sum_{k = 1}^m a_ka_k^T,\  a_k \triangleq \vectext{(A_k+A_k^T) Z^*}. 
\end{align}
Here, we use the notion of vectorization of a matrix, \textit{i.e.}, stacking the columns of a matrix to form a vector. Thus, for a matrix $Z \in \Real{n \times r}$, $\vectext{Z}$ is a vector in $\Real{nr}$. 

Given this quadratic form, it follows that:
\begin{align*}
    \vert \Dataset\left(\Delta\solset^T \right) - \mathbb{E}\left[\Dataset\left(\Delta\solset^T \right)\right] \vert 
    & \leq \norm{S_{\Dataset} - \mathbb{E}[S_{\Dataset}]}_{2} \norm{v}_2^2   
\end{align*}
The term $\norm{S_{\Dataset} - \mathbb{E}[S_{\Dataset}]}_{2}$ can be bounded with high probability using the matrix Bernstein inequality (see Lemma \ref{lem:SD_concentration}). To complete the proof of Lemma \ref{lem:convexity_lowerbound}, it remains to calculate $\mathbb{E}\left[\Dataset\left(\Delta\solset^T\right)\right]$. In Lemma \ref{lem:expectation_of_D}, we show that \(\mathbb{E}\left[\Dataset\left(\Delta\solset^T\right)\right] = \gamma \norm{\Delta_U\solset_V^T + \solset_U\Delta_V^T}_F^2.\)




The proof of Lemma \ref{lem:convexity_upperbound}, just like the one for Lemma \ref{lem:convexity_lowerbound}, involves analyzing a quadratic form around a random matrix, which we split into the mean (expectation) term and the deviation from the mean. We show that:
\begin{align*}
    &\Dataset(\Delta \Delta^T) = y^T B_{\Dataset} y =  y^T \mathbb{E}[B_{\Dataset}] y + y^T (B_{\Dataset} - \mathbb{E}[B_{\Dataset}]) y; \\
    &y \in \Real{n} : \, y_j = \norm{\Delta_j}_2^2 \, \forall j , \ B_{\Dataset} = \frac{1}{m}\sum_{(u; i, j) \in \Dataset} e_u(\Tilde{e}_i + \Tilde{e}_j).
\end{align*}
The first term is bounded above with the warm-start assumption: $\norm{\Delta}_F^2 \leq O(1) \sigma^*_r$. The second term is bounded using the matrix Bernstein inequality (see Lemma \ref{lem:BD_concentration}). 


\paragraph{Showing Smoothness}
Our method of proving Lemma \ref{lem:smoothness} follows the proof style of \citet{zheng2016convergence}.
We start by observing that 
\begin{align*}
    \norm{\nabla \Loglikelihood}_F^2 = \sup_{W \in \Real{n \times r}: \norm{W}_F = 1} \llangle \nabla \Loglikelihood, W \rrangle^2.
\end{align*}
Therefore, it suffices to find a bound for the term on the right hand side of the above equation. The following lemmas, proven in Appendix \ref{sec:upper_bound}, provide the requisite bound.
\begin{restatable}{lemma}{lSA}\label{lem:smoothness_algebra}
    For any $Z \in \overline{\Incoherentset}$ and any $W \in \Real{n \times r}$,
    \begin{align*}
        \llangle \nabla \Loglikelihood, W \rrangle^2  \leq 2 \Xi^2 \left(\mathcal{D}(\Delta\solset^T) + \frac{1}{4}\mathcal{D}(\Delta\Delta^T)\right) \, \mathcal{D}(WZ^T).
    \end{align*}
\end{restatable}

\begin{restatable}{lemma}{lSUB}\label{lem:smoothness_upperbound}
    Suppose the number of samples $m$ is at least {$2n\log(4n/\delta)$}.
    Then, with probability at least $1 - \delta$, the following inequalities hold uniformly for all $Z \in \overline{\Incoherentset}$:
    \begin{align*}
    \mathcal{D}(\Delta\solset^T) 
    &\leq 16\gamma(\mu r \sigma^*_1) \norm{\Delta}_F^2,\\
    {\mathcal{D}}(\Delta\Delta^T) 
    &\leq 416 \gamma(\mu r \sigma^*_1) \norm{\Delta}_F^2,\\
    \mathcal{D}(WZ^T)
    &\leq 192 \gamma(\mu r \sigma^*_1)  \norm{W}^2_F \ \forall \ W \in \Real{n \times r}.
    \end{align*}
\end{restatable}

Lemma \ref{lem:smoothness} follows by combining these lemmas and accounting for the gradient of the regularizer (see Appendix \ref{sec:main_proofs}).



\section{Simulations}\label{sec:simulations}




\textbf{Data Generation:} We generated a random ground truth matrix $X^*\in\mathbb R^{n_1\times n_2}$ with entries selected independently at random according to normal distribution and calculated its rank-$r$ SVD, $U^*\Sigma^*V^{*T}$. 
We have two settings: a low-dimensional setting with $(n_1,n_2)=(200,300)$ and a high-dimensional setting with $(n_1,n_2)=(2000,3000)$. In both settings, we had $r=3$, $\mu\approx 1.01, \kappa=1.1$. 
Using this matrix, we randomly and independently collected $m$ comparison data points. Specifically, for each setting, the comparison dataset took the form $\{(A_k, w_k):k=1,\ldots,m\}$, where $A_k$ represents the $k\textsuperscript{th}$ sampling matrix as in \eqref{eq:def_A2} and and $w_k=g(\llangle A_k, Z^*Z^{*T}\rrangle)$. In this work, we set the regularizer coefficient to be $\lambda = \gamma/40$.
Subsequently, we applied Algorithm \ref{alg:pgd} using the stepsize $\eta$ as recommended by Theorem \ref{thm:main} (Our code can be found \href{https://docs.google.com/document/d/e/2PACX-1vSVqAZIjeDTXuklix4ETGayBs3TMl0lbPxiukHshpuJXtyUyrZQ92TOvFqYA_rqCYyU0ES5YIFT2Rz7/pub}{here}).
The quality of the algorithm's output at iteration $t$ is measured by $||\Delta(Z^t)||_F/\sqrt{n_1n_2}$.
Figure \ref{fig:1} presents the resulting plots.


\textbf{Initialization:} We initialize the algorithm with $Z^{0T} = Z^{*T} + \vartheta(N_1^T, N_2^TJ)$, where $N_1 \in \mathbb{R}^{n_1 \times r}$ and $N_2 \in \mathbb{R}^{n_2 \times r}$, with their entries drawn from a standard normal distribution.  For our experiments, we use $\vartheta \in \{0.5, 1, 2\}$. Figures \ref{fig:1} (a) and (c) show the effect of different initial solutions and also the projection steps in low and high dimensional settings, respectively. 
In both settings, the number of data points $m$ and also the stepsize were chosen as recommended in Theorem \ref{thm:main}. This result confirms the linear convergence of Algorithm \ref{alg:pgd} as predicted by our theoretical analysis. It is important to emphasize that while both a warm start and the projection step are required for our theoretical guarantees, these simulation results suggest that they are not needed in practice.

\textbf{Dataset size:} We examine the impact of dataset size $m$ on the algorithm's performance. Figures \ref{fig:1} (b) and (d) demonstrate the resulting normalized errors in low and high dimensional settings, respectively.  
As depicted in these plots, a large enough $m$ leads to linear convergence of the algorithm while for a small $m$, the error $\norm{\Delta(Z^t)}_F$ does not zero as $t$ increases. In both plots, the red curves show the converges rate for $m$ computed by $c_0(\mu r\kappa)^2n\log(n/\delta)$ with $\delta=0.05$ and $c_0$ being $1/4$ for low-dimensional and $1/2$ for high-dimensional setting. 

\begin{figure*}[ht]
    \centering
    \begin{subfigure}[b]{0.35\textwidth}
        \centering
        \includegraphics[width=\textwidth]{plots/1_1.png}
        \caption{Different initializations}
    \end{subfigure}
    \hspace{1.1cm}
    \begin{subfigure}[b]{0.35\textwidth}
        \centering
        \includegraphics[width=\textwidth]{plots/2_1_1.png}
        \caption{Varying dataset size}
    \end{subfigure}\\
        \begin{subfigure}[b]{0.35\textwidth}
        \centering
       \includegraphics[width=\textwidth]{plots/4_1.png}
       \caption{Different initializations}
    \end{subfigure}
    \hspace{1.1cm}
       \begin{subfigure}[b]{0.35\textwidth}
        \centering
       \includegraphics[width=\textwidth]{plots/5_1.png}
        \caption{Varying dataset size}
    \end{subfigure}
    \caption{The top row and bottom row show the results for $(n_1,n_2)=(200,300)$ and $(n_1,n_2)=(2000,3000)$, respectively. (a) and (c) illustrate the effect of different initializations with a fixed number of data points, while the remaining plots demonstrate the effect of varying dataset size $m$. Y-axes are in log scale. }
    \label{fig:1}
\end{figure*}



\section{Conclusion}\label{sec:conclusion}

\paragraph{Summary}
Our findings provide significant insights into the influence of correctness, explanations, and refinement on evaluation accuracy and user trust in AI-based planners. 
In particular, the findings are three-fold: 
(1) The \textbf{correctness} of the generated plans is the most significant factor that impacts the evaluation accuracy and user trust in the planners. As the PDDL solver is more capable of generating correct plans, it achieves the highest evaluation accuracy and trust. 
(2) The \textbf{explanation} component of the LLM planner improves evaluation accuracy, as LLM+Expl achieves higher accuracy than LLM alone. Despite this improvement, LLM+Expl minimally impacts user trust. However, alternative explanation methods may influence user trust differently from the manually generated explanations used in our approach.
% On the other hand, explanations may help refine the trust of the planner to a more appropriate level by indicating planner shortcomings.
(3) The \textbf{refinement} procedure in the LLM planner does not lead to a significant improvement in evaluation accuracy; however, it exhibits a positive influence on user trust that may indicate an overtrust in some situations.
% This finding is aligned with prior works showing that iterative refinements based on user feedback would increase user trust~\cite{kunkel2019let, sebo2019don}.
Finally, the propensity-to-trust analysis identifies correctness as the primary determinant of user trust, whereas explanations provided limited improvement in scenarios where the planner's accuracy is diminished.

% In conclusion, our results indicate that the planner's correctness is the dominant factor for both evaluation accuracy and user trust. Therefore, selecting high-quality training data and optimizing the training procedure of AI-based planners to improve planning correctness is the top priority. Once the AI planner achieves a similar correctness level to traditional graph-search planners, strengthening its capability to explain and refine plans will further improve user trust compared to traditional planners.

\paragraph{Future Research} Future steps in this research include expanding user studies with larger sample sizes to improve generalizability and including additional planning problems per session for a more comprehensive evaluation. Next, we will explore alternative methods for generating plan explanations beyond manual creation to identify approaches that more effectively enhance user trust. 
Additionally, we will examine user trust by employing multiple LLM-based planners with varying levels of planning accuracy to better understand the interplay between planning correctness and user trust. 
Furthermore, we aim to enable real-time user-planner interaction, allowing users to provide feedback and refine plans collaboratively, thereby fostering a more dynamic and user-centric planning process.


\newpage

\bibliography{proper_citations}
\bibliographystyle{plainnat}


\newpage
\appendix
\onecolumn
\section{Helper Lemmas}\label{sec:helper_lemmas}


\subsection{Matrix Inner Product Identities}

We state some basic identities of the matrix inner product operator, which are trivial to verify but are used frequently in the paper. In the following identities, $D, E,$ and $F$ are arbitrary matrices so long as their sizes are compatible with the equations.
\begin{align}
    \llangle E, F \rrangle &= \Tr{EF^T} = \Tr{FE^T} \label{eq:identity_trace} \\
    \llangle E, F \rrangle &= \llangle F, E \rrangle = \llangle E^T, F^T \rrangle \label{eq:identity_transpose} \\
    \llangle DE, F \rrangle &= \llangle D, FE^T \rrangle = \llangle E, D^TF \rrangle, \quad \llangle D, EF \rrangle = \llangle DF^T, E \rrangle = \llangle E^TD, F \rrangle \label{eq:identity_shift}
\end{align}
From these identities, we get that for any sampling matrix $A$ (defined in \eqref{eq:def_A2}) and any $Y, Z \in \Real{n \times r}$:
\begin{align}\label{eq:A_AT_identity}
    \llangle (A + A^T)Y, Z \rrangle &= \llangle AY, Z \rrangle + \llangle A^TY, Z \rrangle \nonumber \\
    & = \llangle A, ZY^T \rrangle + \llangle A^T, ZY^T \rrangle \nonumber \\
    & = \llangle A, ZY^T + YZ^T \rrangle
\end{align}

Let $W$ and $Z$ be two matrices in $\Real{n \times r}$. Recall the notation convention introduced in Section \ref{sec:proof}. Using the above identity and \eqref{eq:def_A2}, we get that for any sampling matrix $A$ corresponding to the triplet $(u; i, j)$,
\begin{align}\label{eq:A_AT_YZT_identity}
    \llangle (A + A^T), WZ^T \rrangle &= \llangle e_u(\Tilde{e}_i - \Tilde{e}_j)^T, W_UZ_V + Z_UW_V \rrangle\\
    &= \llangle W_u, Z_i - Z_j \rrangle + \llangle Z_u, W_i - W_j \rrangle \label{eq:A_AT_YZT_identity2}
\end{align}


\subsection{The Frobenius Norm of the Product of Two Matrices}
Let $X$ be any matrix and let $\sigma_{\max}(X)$ and $\sigma_{\min}(X)$ denote the largest and smallest singular values of $X$. Let $v$ be any vector such that the product $Xv$ is compatible.
By the definition of singular values:
\begin{align*}
    \sigma_{\min} (X)\norm{v}_2 \leq \norm{Vx}_2 \leq \sigma_{\max}(X)\norm{v}_2
\end{align*}
Using this basic fact, we can prove the following result.
\begin{lemma}\label{lem:bounds_on_product_norms}
    Let $U \in \Real{n_1 \times r}$ and $V\in \Real{n_2 \times r}$ be any two matrices. Let $\sigma_1(U) \geq \ldots \geq \sigma_r(U)$ denote the singular values of $U$ and $\sigma_1(V) \geq \ldots \geq \sigma_r(V)$ denote the singular values of $V$. Then $\norm{UV^T}_F^2$ satisfies the following bounds:
    \begin{align*}
        \sigma_r(U)^2 \norm{V}_F^2 &\leq \norm{UV^T}_F^2 \leq \sigma_1(U)^2 \norm{V}_F^2 \\
        \sigma_r(V)^2 \norm{U}_F^2 &\leq \norm{UV^T}_F^2 \leq \sigma_1(V)^2 \norm{U}_F^2         
    \end{align*}
\end{lemma}
\begin{proof}
    We first prove the inequality $\norm{UV^T}_F^2 \geq \sigma_r(U)^2 \norm{V}_F^2$. Let $V_j$ denote the $j\textsuperscript{th}$ row of $V$, written as a column vector ($r \times 1$ matrix). Let $(UV^T)^j$ denote the $j\textsuperscript{th}$ column of $UV^T$. Finally, note that the squared Frobenius norm of a matrix is the sum of the squared $\ell_2$ norms of its rows or of its columns. Stitching together these simple facts, we get.
    \begin{align*}
        \norm{UV^T}_F^2 &= \sum_{j = 1}^{n_2} \norm{(UV^T)^j}_2^2 = \sum_{j = 1}^{n_2} \norm{UV_j}_2^2 \\
        &\geq \sum_{j = 1}^{n_2} \sigma_r(U)^2 \norm{V_j}_2^2  = \sigma_r(U)^2 \sum_{j = 1}^{n_2} \norm{V_j}_2^2 \\
        &= \sigma_r(D)^2 \norm{V}_F^2
    \end{align*}    
    The upper bound $\norm{UV^T}_F^2 \leq \sigma_1(U)^2 \norm{V}_F^2$ can be derived using the same steps, except we use the inequality $\norm{UV_j}_2 \leq \sigma_1(U) \norm{V_j}_2$ instead of $\norm{UV_j}_2 \geq \sigma_r(U) \norm{V_j}_2$. Finally, the second set of bounds follow by applying the first set of bounds to the matrix $VU^T$, and noting that $\norm{UV^T}_F = \norm{VU^T}_F$.
\end{proof}


\subsection{The Incoherence of the Iterates}
Recall that we have assumed that the initial point $Z^0$ satisfies the bound $\norm{\Delta(Z^0)}_F^2 \leq \sigma^*_r/16$, i.e., we are given a warm start (see Section \ref{sec:algorithm}). With this assumption, we can prove the following lemmas.
\begin{lemma}\label{lem:solset_in_C}
    Let $\Incoherentset$ be the set defined in \eqref{eq:def_incoherent_set}, i.e.,
    \begin{align*}
        \Incoherentset \triangleq \left\{Z \in  \mathbb{R}^{n \times r} \, : \, \norm{Z}_{2, \infty} \leq \frac{4}{3}\sqrt{\frac{\mu}{n} }\norm{Z^0}_F \right\}
    \end{align*}
    Then all the equivalent ground-truth matrices lie in $\Incoherentset$, i.e. $\solset \subseteq \Incoherentset$.
\end{lemma}
\begin{proof}
Start with the identity $Z^0 = \solset(Z^0) + \Delta(Z^0)$ (which follows from \eqref{eq:def_difference_sol}). By the triangle inequality, we get
\begin{align*}
\norm{\solset(Z^0)}_F - \norm{\Delta(Z^0)}_F  \leq \norm{Z^0}_F \leq \norm{\solset(Z^0)}_F + \norm{\Delta(Z^0)}_F.
\end{align*}
Note that all matrices in $\solset$ have the same Frobenius norm. This implies that $\norm{\solset(Z^0)}_F = \norm{Z^*}_F$. Combining this with the bound on $\norm{\Delta(Z^0)}_F$, we get
\begin{align}\label{eq:Z0_bounds}
    \norm{Z^*}_F - \sqrt{\sigma^*_r}/4  \leq \norm{Z^0}_F \leq \norm{Z^*}_F + \sqrt{\sigma^*_r}/4
\end{align}
Recall that the singular values of $Z^*$ are $\sqrt{2\sigma^*_1}, \sqrt{2\sigma^*_2}, \ldots, \sqrt{2\sigma^*_r}$. We know that the Frobenius norm of a matrix is the $\ell_2$ norm of the vector of its singular values. Therefore:
\begin{align*}
    \norm{Z^*}_F &= \sqrt{2\sum_{i = 1}^r \sigma^*_i} \Rightarrow \frac{\sqrt{\sigma^*_r}}{4} \leq \frac{\norm{Z^*}_F}{4} \\
    \Rightarrow \norm{Z^0}_F &\geq \norm{Z^*}_F - \sqrt{\sigma^*_r}/4 \geq \frac{3}{4}\norm{Z^*}_F \\
    \Rightarrow \norm{Z^*}_{2, \infty} &= \sqrt{\mu/n}\norm{Z^*}_{F} \leq \frac{4}{3}\sqrt{\mu/n}\norm{Z^0}_F 
\end{align*}
Thus, we see that $Z^* \in \Incoherentset$. Because all $Z \in \solset$ have the same $\ell_2/\ell_\infty$ norm, it follows that $\solset \subseteq \Incoherentset$. 
\end{proof}


Before proceeding further, we introduce some new notation. Recall the convention (established in Section \ref{sec:proof}) that any matrix $Z$ can be viewed as a concatenation of two matrices: $Z = (Z_U, Z_V)$. To index the rows of $Z$, we use $Z_u, u \in [n_1]$ for the user features and $Z_i, Z_j, j \in [n_2]$ for the item features. In expressions involving matrix multiplication, we view $Z_u, Z_i, Z_j$ as row vectors, i.e., as $1 \times r$ matrices. By the definition of $\norm{Z}_{2, \infty}$, we get: 
\begin{align}\label{eq:def_l2inf_norm}
    \norm{Z}_{2, \infty} = \max \{\max_{u \in [n_1]} \norm{Z_u}_2, \max_{i \in [n_2]} \norm{Z_i}_2\}.
\end{align}

Equipped with this new notation, we can state and prove the next result.
\begin{lemma}\label{lem:incoherence_of_projection}
    For any $Z \in \Incoherentset$, let $W = \mathcal{P}_{\mathcal{H}}(Z)$. Then $W \in \overline{\Incoherentset}$, i.e., $W$ satisfies
    \begin{align*}
        \norm{W}_{2, \infty}^2 \leq \frac{12\mu}{n} \norm{Z^*}_F^2
    \end{align*}
\end{lemma}
\begin{proof}
    Let $z$ denote the mean of the rows of $Z_V$, i.e.,
    \begin{align*}
        z \triangleq \frac{1}{n_2} \sum_{i \in [n_2]} Z_i
    \end{align*}
    It follows that
    \begin{align*}
        \Rightarrow \norm{z}_2 &= \frac{1}{n_2} \norm{\sum_{i \in [n_2]} Z_i}_2 \leq \frac{1}{n_2} \sum_{i \in [n_2]} \norm{Z_i}_2 \leq \frac{1}{n_2} \sum_{i \in [n_2]} \norm{Z}_{2, \infty} = \norm{Z}_{2, \infty} \quad (\text{by \eqref{eq:def_l2inf_norm}})
    \end{align*}
    The operation of projecting onto the subspace $\mathcal{H}$ is such that $W_U = Z_U$ and $W_i = Z_i - v$ for all item rows $i$ (see Section \ref{sec:symmetries}). By the triangle inequality, we get:
    \begin{align*}
        \norm{W_i}_2 &= \norm{Z_i - z}_2 \leq \norm{Z_i}_2 + \norm{z}_2 \\
        \Rightarrow \max_{i \in [n_2]} \norm{W_i}_2 &\leq \max_{i \in [n_2]} \norm{Z_i}_2 + \norm{z}_2 \leq \norm{Z}_{2, \infty} + \norm{z}_2 \leq 2\norm{Z}_{2, \infty}
    \end{align*}
    Because the rows of $U$ remain unchanged, we have
    $\norm{W}_{2, \infty} \leq 2\norm{Z}_{2, \infty}$.

    Next, note that $Z \in \Incoherentset$. Therefore, 
    $$\norm{Z}_{2, \infty} \leq \frac{4}{3}\sqrt{\frac{\mu}{n}} \norm{Z^0}_F \leq \frac{5}{3}\sqrt{\frac{\mu}{n}} \norm{Z^*}_F$$
    The last step uses the inequality $\norm{Z^0}_F \leq (5/4)\norm{Z^*}_F$, which follows from \eqref{eq:Z0_bounds} in the derivation of Lemma \ref{lem:solset_in_C}.
    By combining the above inequalities, we get the desired result:
    \begin{align*}
        \norm{\Hat{Z}}_{2, \infty}^2 \leq 4\norm{Z}_{2, \infty}^2 \leq 4\frac{25}{9}\frac{\mu}{n} \norm{Z^*}_F^2\leq \frac{12\mu}{n} \norm{Z^*}_F^2.
    \end{align*}
\end{proof}
The above result is important because it establishes a useful bound that holds for all iterates $Z^t, t \in \mathbb{Z}_+$.
(Recall that Algorithm \ref{alg:pgd} takes successive projections, first on to $\Incoherentset$ and then onto $\mathcal{H}$.) 

\subsection{Bounds on the Scores}
In this subsection, we derive two related bounds on any $Z \in \Real{n \times r}$ and any sampling matrix $A$:
\begin{align}
    |\llangle A, ZZ^T \rrangle| &\leq 2\norm{Z}_{2, \infty}^2 \label{eq:score_bound1}\\
    \norm{(A + A^T)Z}_F^2 &\leq 6\norm{Z}_{2, \infty}^2 \label{eq:score_bound2}
\end{align}
Before we prove these bounds, let us explore its consequence. By the definition of the incoherence parameter $\mu$ \eqref{eq:def_mu}, $\norm{Z^*}_{2, \infty}^2 = (\mu/n) \norm{Z^*}_{F}^2$. Therefore,
\begin{align}
    |\llangle A, Z^*Z^{*T} \rrangle| &\leq \frac{2\mu}{n}\norm{Z^*}_F^2 \label{eq:score_bound1_Zstar}\\
    \norm{(A + A^T)Z^*}_F^2 &\leq \frac{6\mu}{n}\norm{Z^*}_F^2 \label{eq:score_bound2_Zstar}
\end{align}
Moreover, for all $Z \in \overline{\Incoherentset}$,
\begin{align}
    |\llangle A, ZZ^{T} \rrangle| &\leq \frac{24\mu}{n}\norm{Z^*}_F^2 \label{eq:score_bound1_Zcbar}\\
    \norm{(A + A^T)Z}_F^2 &\leq \frac{72\mu}{n}\norm{Z^*}_F^2 \label{eq:score_bound2_Zcbar}
\end{align}
As argued in the previous subsection, all iterates $(Z^t)_{t \in \mathbb{Z}_+}$ of Algorithm \ref{alg:pgd} lie in $\overline{\Incoherentset}$ and consequently satisfy the above bound.

We now proceed to the derivation of \eqref{eq:score_bound1}.
Let $Z\in \Real{n \times r}$ be some candidate feature matrix and let $X = Z_UZ_V^T$ be the corresponding score matrix. Let $(u; i, j)$ be an arbitrary triplet and let $A$ denote the corresponding sampling matrix. Recall the definition of the sampling matrix $A$ corresponding to a triplet $(u; i, j)$ from \eqref{eq:def_A1} and \eqref{eq:def_A2}. We have
\begin{align*}
    |\llangle A, ZZ^T \rrangle| = |x_{u,i} - x_{u,j}| = |\langle Z_u, (Z_i - Z_j) \rangle| \leq \norm{Z_u}_2 \norm{Z_i - Z_j}_2 \leq \norm{Z_u}_2 (\norm{Z_i}_2 + \norm{Z_j}_2) \leq 2\norm{Z}_{2, \infty}^2
\end{align*}
The last inequality follows from the definition of $\norm{Z}_{2, \infty}$ (see \eqref{eq:def_l2inf_norm}).

The derivation of \eqref{eq:score_bound2} proceeds as follows.
\begin{align*}
    A &= \begin{bmatrix}
        0 & e_u(\Tilde{e}_i - \Tilde{e}_j)^T\\
        0 & 0
    \end{bmatrix} \\
    \Rightarrow A + A^T &= \begin{bmatrix}
        0 & e_u(\Tilde{e}_i - \Tilde{e}_j)^T\\
        (\Tilde{e}_i - \Tilde{e}_j)e_u^T & 0
    \end{bmatrix} \\
    \Rightarrow (A + A^T)Z &= \begin{bmatrix}
        0 & e_u(\Tilde{e}_i - \Tilde{e}_j)^T\\
        (\Tilde{e}_i - \Tilde{e}_j)e_u^T & 0
    \end{bmatrix} \begin{bmatrix} Z_U \\ Z_vV \end{bmatrix} \\
    &= \begin{bmatrix} e_u(\Tilde{e}_i - \Tilde{e}_j)^T Z_V \\ (\Tilde{e}_i - \Tilde{e}_j)e_u^T Z_U \end{bmatrix} \\
    &= \begin{bmatrix} e_u(Z_i - Z_j) \\ (\Tilde{e}_i - \Tilde{e}_j)Z_u \end{bmatrix} \\
    \Rightarrow \norm{(A + A^T)Z}_F^2 &= \norm{e_u(Z_i - Z_j)}_F^2 + \norm{(\Tilde{e}_i - \Tilde{e}_j)Z_u}_F^2 \\
    &=  \norm{e_u}_2^2\norm{Z_i - Z_j}_2^2 + \norm{\Tilde{e}_i - \Tilde{e}_j}_2^2\norm{Z_u}_2^2 \\
    &= \norm{Z_i - Z_j}_2^2 + 2\norm{Z_u}_2^2 \quad (\norm{e_u}_2^2 = 1, \ \norm{\Tilde{e}_i - \Tilde{e}_j}_2^2 = 2) \\
    &\leq 2(\norm{Z_i}_2^2 + \norm{Z_j}_2^2) + 2\norm{Z_u}_2^2 \quad (\norm{Z_i - Z_j}_2^2 \leq (\norm{Z_i}_2 + \norm{Z_j}_2)^2 \leq 2(\norm{Z_i}_2^2 + \norm{Z_j}_2^2)) \\
    &\leq 6\norm{Z}_{2, \infty}^2 \quad (\text{by definition of $\norm{Z}_{2, \infty}$ \eqref{eq:def_l2inf_norm}})
\end{align*}
This establishes the second inequality.

\subsection{The Matrix Bernstein Inequality}
Here, we state a special version of the matrix Bernstein inequality that we use in our proofs. The statement is identical to Corollary 6.2.1 in \cite{tropp2015introduction}, barring a change in notation. %

This concentration result is stated in terms of the operator norm of a matrix $X$, which we denote as $\norm{X}_{2}$  and is defined as follows:
\begin{align}\label{eq:def_operator_norm}
    \norm{X}_{2} \triangleq \sup_{v: \norm{v}_2 = 1} {\norm{Xv}_2}
\end{align}
It follows that $\norm{X}_{2} = \sigma_{\max}(X)$.
For square matrices $X$, an alternate definition of the operator norm is:
\begin{align}\label{eq:def_operator_norm2}
    \norm{X}_{2} \triangleq \sup_{v: \norm{v}_2 = 1} {v^TXv}
\end{align}

\begin{lemma}[Matrix Bernstein Inequality]\label{lem:matrix_bernstein}
    Consider a random matrix $X$ of shape $n_1 \times n_2$ that satisfies:
    \begin{align*}
        \mathbb{E}[X] = \Bar{X} \quad \text{ and } \quad \norm{X}_2 \leq L \text{ almost surely}.
    \end{align*}
    Let $b$ be an upper bound on the second moment of $X$:
    \begin{align*}
        \norm{\mathbb{E}[XX^T]}_2 \leq b \quad \text{ and } \quad \norm{\mathbb{E}[X^TX]}_2 \leq b.
    \end{align*}
    Let $X_{\mathcal{D}} = \frac{1}{m}\sum_{k = 1}^m X_k$, where each $X_k$ is an i.i.d. copy of $X$. Then, for all $t \geq 0$,
    \begin{align*}
        P( \norm{X_{\mathcal{D}} - \Bar{X}}_2 \geq t) &\leq (n_1 + n_2) \exp\left(\frac{-mt^2/2}{b + 2Lt/3}\right)
    \end{align*}
\end{lemma}
\newpage


\section{Initial Lemmas}\label{sec:initial_lemmas}
Following the convention of the main paper, we drop the explicit dependence on $Z$ wherever it is obvious.%

\subsection{Proof of Lemma \ref{lem:convexity_algebra}}

\lSCA*
\begin{proof}
    From the expression of $\nabla \Loglikelihood$ (see \eqref{eq:gradient_likelihood}), we get that:
    \begin{align*}
        \llangle \nabla \Loglikelihood, \Delta\rrangle =   \frac{1}{m}\sum_{k=1}^{m}
    h_k \llangle (A_k+ A_k^T) Z, \Delta \rrangle \text{ where }
    h_k =  \frac{g'(z_k)\left(g(z_k) - w_k\right)}{g(z_k)(1-g(z_k))}, \ z_k = \llangle A_k,ZZ^T \rrangle.
    \end{align*}
    Recall, by definition (see \eqref{eq:def_difference_sol}), that $Z = \solset + \Delta$. Therefore. the term $\llangle (A_k+ A_k^T) Z, \Delta \rrangle$ can be expanded as follows:
    \begin{align*}
        \llangle (A_k+ A_k^T) Z, \Delta \rrangle &= \llangle (A_k+ A_k^T) \solset, \Delta \rrangle + \llangle (A_k+ A_k^T) \Delta, \Delta \rrangle\\
        &= \llangle A_k+ A_k^T, \Delta \solset^T \rrangle + \llangle A_k+ A_k^T, \Delta \Delta^T \rrangle \quad \text{(by \eqref{eq:identity_shift})}
    \end{align*}
    Since we have assumed that our observations are noiseless, we have the identity $w_k = g(\llangle A_k,Z^*Z^{*T} \rrangle)$. Plugging this equation in the expression of $h_k$, we get:
    \begin{align*}
        h_k &=  \frac{g'(z_k)\left(g(z_k) - g(z^*_k)\right)}{g(z_k)(1-g(z_k))}; \quad z^*_k =  \llangle A_k,Z^*Z^{*T} \rrangle = \llangle A_k, \solset \solset^T \rrangle
    \end{align*}
    By the mean value theorem,
    \begin{align*}
        g(z_k) - g(z^*_k) &= g'(y_k) (z_k - z^*_k) \quad \text{for some } y_k \text{ in the interval between } z_k \text{ and } z^*_k\\
        &= g'(y_k) \left(\llangle A_k,ZZ^T \rrangle - \llangle A_k,\solset \solset^T \rrangle \right) \\
        &= g'(y_k) \left(\llangle A_k,\solset \Delta^T + \Delta \solset^T\rrangle + \llangle A_k,\Delta \Delta^T\rrangle\right)  \quad (\text{because }Z = \solset + \Delta)\\
        &= g'(y_k) \left(\llangle A_k+ A_k^T,\Delta \solset^T\rrangle + \frac{1}{2}\llangle A_k+ A_k^T,\Delta \Delta^T\rrangle\right) \quad (\text{by } \eqref{eq:A_AT_identity})
    \end{align*}
    Putting the above equations together, we get:
    \begin{align*}
        &h_k \llangle (A_k+ A_k^T) Z, \Delta \rrangle \\
        &\ = \frac{g'(z_k)g'(y_k)}{g(z_k)(1-g(z_k))} \left(\llangle A_k + A_k^T, \Delta \solset^T\rrangle + \frac{1}{2}\llangle A_k + A_k^T,\Delta \Delta^T\rrangle\right) \left(\llangle A_k + A_k^T,\Delta \solset^T\rrangle + \llangle A_k + A_k^T,\Delta \Delta^T\rrangle\right) \\
        &\ = \frac{g'(z_k)g'(y_k)}{g(z_k)(1-g(z_k))} \left(\llangle A_k + A_k^T, \Delta \solset^T\rrangle^2 + \frac{3}{2} \llangle A_k + A_k^T, \Delta \solset^T\rrangle \llangle A_k + A_k^T,\Delta \Delta^T\rrangle + \frac{1}{2}\llangle A_k + A_k^T,\Delta \Delta^T\rrangle^2\right)\\
        &\ \geq \frac{g'(z_k)g'(y_k)}{g(z_k)(1-g(z_k))}\left(\frac{1}{2}\llangle A_k + A_k^T,\Delta \solset^T\rrangle^2 - \frac{5}{8}\llangle A_k + A_k^T,\Delta \Delta^T\rrangle^2\right)
    \end{align*}
    The last step uses the inequality $2a^2 + 3ab + b^2$ $\geq {a^2} - \frac{5b^2}{4}$, which can be derived from the trivial inequality $(a + 3b/2)^2 \geq 0$. Note also that the coefficient $\frac{g'(z_k)g'(y_k)}{g(z_k)(1-g(z_k))}$ is positive.

Finally, observe that we have assumed $Z \in \overline{\Incoherentset}$. The bounds in \eqref{eq:score_bound1_Zstar} and \eqref{eq:score_bound1_Zcbar} imply $$|z_k^*| \leq 2\frac{ \mu \norm{Z^*}_F^2}{n}, \ |z_k| \leq 24\frac{ \mu \norm{Z^*}_F^2}{n}, \text{ which implies } |y_k| \leq 24\frac{\mu \norm{Z^*}_F^2}{n}$$ 
Thus, $y_k$ and $z_k$ lie in the interval $\left[-24 \mu \norm{Z^*}_F^2/n, {24 \mu \norm{Z^*}_F^2}/{n}\right]$. By the definition of $\xi$ and $\Xi$ in \eqref{eq:link_function_lower_bound} and \eqref{eq:link_function_upper_bound}, as well as the definition of the operator $\mathcal{D}(\cdot)$ in \eqref{eq:def_D_operator}, the desired expression follows.
\end{proof}


\subsection{Proof of Lemma \ref{lem:smoothness_algebra}}

\lSA*
\begin{proof}
    The proof of this lemma is similar to the proof of Lemma \ref{lem:convexity_algebra}. One major difference is that we work with terms of the form $\llangle A + A^T, Y \rrangle$ instead of terms $\llangle A, Y \rrangle$.

    Following the steps of the proof of Lemma \ref{lem:convexity_algebra}, we get:
    \begin{align*}
        \llangle \nabla \Loglikelihood, H\rrangle &=   \frac{1}{m}\sum_{k=1}^{m}
        h_k \llangle (A_k+ A_k^T) Z, H \rrangle \text{ where }
        h_k =  \frac{g'(z_k)\left(g(z_k) - w_k\right)}{g(z_k)(1-g(z_k))}, \ z_k = \llangle A_k,ZZ^T \rrangle. \\
        g(z_k) - g(z^*_k) &=  g'(y_k) \left(\llangle A_k,\solset \Delta^T + \Delta \solset^T\rrangle + \llangle A_k,\Delta \Delta^T\rrangle\right)  \quad \text{for some } y_k \text{ in the interval between } z_k \text{ and } z^*_k
        \end{align*}
    By \eqref{eq:identity_transpose} and \eqref{eq:identity_shift}, we get:
    \begin{align*}
        \llangle (A_k+ A_k^T) Z, H \rrangle &= \llangle A_k+ A_k^T, HZ^T \rrangle \\
        \llangle A_k,\solset \Delta^T + \Delta \solset^T\rrangle + \llangle A_k,\Delta \Delta^T\rrangle &= \llangle A_k + A_k^T,\solset \Delta^T\rrangle + \frac{1}{2}\llangle A_k + A_k^T,\Delta \Delta^T \rrangle 
    \end{align*}
   Putting together the equations above, we get:
    \begin{align}\label{eq:nabla_l_h}
        \llangle \nabla \Loglikelihood, H\rrangle &=   \frac{1}{m}\sum_{k=1}^{m} \frac{g'(z_k)g'(y_k)}{g(z_k)(1-g(z_k))} 
        \left( \llangle A_k + A_k^T,\solset \Delta^T\rrangle + \frac{1}{2}\llangle A_k + A_k^T,\Delta \Delta^T \rrangle \right)
        \left( \llangle A_k+ A_k^T, HZ^T \rrangle \right)
    \end{align}
    Next, we invoke two straightforward inequalities which apply to any sequence of scalars $(a_k)_{k \in [m]}, (b_k)_{k \in [m]}, \text{ and } (c_k)_{k \in [m]}$ with $a_k \geq 0 \ \forall \ k$:
    \begin{align*}
        \left(\frac{1}{m}\sum_{k = 1}^m a_k b_k c_k\right)^2 &\leq \left(\frac{1}{m}\sum_{k = 1}^m a_k b_k^2 \right) \left(\frac{1}{m}\sum_{k = 1}^m a_k c_k^2 \right) \\
        \left(\frac{1}{m}\sum_{k = 1}^m a_k b_k^2 \right) &\leq \left(\max_{k \in [m]} a_k\right) \left(\frac{1}{m}\sum_{k = 1}^m b_k^2 \right)
    \end{align*}
The first inequality can be viewed as a form of the Cauchy-Schwarz inequality and the second, a form of Hölder's inequality.

Squaring both sides of the equation in \eqref{eq:nabla_l_h} and applying these inequalities with 
\begin{align*}
    a_k = \frac{g'(z_k)g'(y_k)}{g(z_k)(1-g(z_k))}, \ b_k = \llangle A_k + A_k^T,\solset \Delta^T\rrangle + \frac{1}{2}\llangle A_k + A_k^T,\Delta \Delta^T \rrangle, \ c_k = \llangle A_k+ A_k^T, HZ^T \rrangle,
\end{align*}
and observing that $\max_{k \in [m]} a_k \leq \Xi$ (using arguments similar to those in Lemma \ref{lem:convexity_algebra}), we get
\begin{align*}
    \llangle \nabla \Loglikelihood, H\rrangle^2 &\leq \Xi^2 \left( \frac{1}{m}\sum_{k = 1}^m(\llangle A_k + A_k^T,\solset \Delta^T\rrangle + \frac{1}{2}\llangle A_k + A_k^T,\Delta \Delta^T \rrangle)^2 \right) \left(\frac{1}{m}\sum_{k = 1}^m \llangle A_k+ A_k^T, HZ^T \rrangle^2 \right) \\
    &\leq 2\Xi^2 \left( \left(\frac{1}{m}\sum_{k = 1}^m\llangle A_k + A_k^T,\solset \Delta^T\rrangle^2\right) + \frac{1}{4}\left(\frac{1}{m}\sum_{k = 1}^m \llangle A_k + A_k^T,\Delta \Delta^T \rrangle^2\right) \right) \left(\frac{1}{m}\sum_{k = 1}^m \llangle A_k+ A_k^T, HZ^T \rrangle^2 \right) \\
    &= 2 \Xi^2 \left(\Tilde{\mathcal{D}}(\Delta\solset^T) + \frac{1}{4}\Tilde{\mathcal{D}}(\Delta\Delta^T)\right) \, \Tilde{\mathcal{D}}(HZ^T),
\end{align*}
giving us the bound we want.
\end{proof}







\section{A Lower Bound For Strong Convexity}\label{sec:lower_bound}
\subsection{Lower bounds on sample complexity}\label{sec:sample_compexity}
We establish a lower bound for generalized linear measurements using standard information-theoretic arguments based on Fano's inequality. While the upper bound in Theorem~\ref{thm:alg_general} is derived for the maximum probability of error over all  $k$-sparse vectors, the lower bound applies even in the weaker setting of the average probability of error, where 
$\bx$ is chosen uniformly at random.
\begin{theorem}[Lower bound for GLMs]\label{thm: lower_bdglm} Consider any  sensing matrix $\vecA$.
For a uniformly chosen $k$-sparse vector $\bx$, an algorithm $\phi$ satisfies $$\bbP\inp{\phi(\vecA, \by) \neq \bx}\leq \delta$$   only if the number of measurements $$m\geq \frac{k\log\inp{\frac{n}{k}}}{I}\inp{1 - \frac{h_2(\delta) + \delta k\log{n}}{k\log{n/k}}}$$ for some $I$ such that $I\geq {I(y_i; \bx|\vecA)}, \, i\in [m]$. In particular, when $y\in \inb{-1, 1}$, we have $\bbE\insq{\inp{g(\vecA_i^T\bx)}^2} \geq I(y_i, \bx|\vecA)$ where the expectation is over the randomness of $\vecA$ and $\bx$.
\end{theorem}
The lower bound can be interpreted in terms of a communication problem, where the input message $\bx$ is encoded to $\vecA\bx$. The decoding function takes in as input the encoding map $\vecA$ and the output vector $\by$ in order to recover $\bx$ with high probability. For optimal recovery, one needs at least $\frac{\text{message entropy}}{\text{capacity}}$ number of measurements (follows from noisy channel coding theorem~\cite{thomas2006elements}). In Theorem~\ref{thm: lower_bdglm}, the entropy of the message set $\log{n \choose k}\approx k\log{n/k}$ and the proxy for capacity is the upper bound on mutual information $I$. We provide a detailed proof of the theorem in  Section~\ref{sec:proofs}.


We first present lower bounds for \bcs\  and \logreg. The lower bound for \bcs\ is given for any sensing matrix $\vecA$ which satisfies the power constraint given by \eqref{eq:power_constraint}, whereas the one for \logreg\ is only for the special case when each entry of the sensing matrix is iid $\cN(0,1)$. Recall that \eqref{eq:power_constraint} holds in this case.  For \bcs\ (and \logreg\ respectively), we can use the upper bound of $\bbE\insq{\inp{g(\vecA_i^T\bx)}^2}$ on the mutual information term. The dependence of $\sigma^2$ (and $1/\beta^2$ respectively) requires careful bounding of this term, which is done in the formal proofs in Appendix~\ref{proof:sec:lower_bd}.


As mentioned earlier, we need at least $k\log\inp{n/k}$ measurements for \bcs and \logreg. This is because the entropy of a randomly chosen $k$-sparse vector is approximately $k\log\inp{n/k}$ and we learn at most one bit with each measurement. However, due to corruption with noise, we learn less than a bit of information about the unknown signal with each measurement. The information gain gets worse as the noise level increases. 
Our lower bounds make this reasoning explicit.  
\begin{corollary}[\bcs\ lower bound]\label{thm: lower_bd_bcs} Suppose, each row $\vecA_i, \, i\in [1:m]$ of the sensing matrix $\vecA$ satisfies the power constraint~\eqref{eq:power_constraint}.
For a uniformly chosen $k$-sparse vector $\bx$, an algorithm $\phi$ satisfies $$\bbP\inp{\phi(\vecA, {\by}) \neq \bx}\leq \delta$$ for the problem of $\bcs$ only if the number of measurements $$m\geq \frac{k+\sigma^2}{2}\log\inp{\frac{n}{k}}\inp{1 - \frac{h_2(\delta) + \delta k\log{n}}{k\log{n/k}}}.$$ 
\end{corollary}

\begin{corollary}[\logreg\ lower bound]\label{thm: lower_bd_log_reg} Consider a Gaussian  sensing matrix $\vecA$ where each entry is chosen iid $N(0,1)$.
For a uniformly chosen $k$-sparse vector $\bx$, an algorithm $\phi$ satisfies $$\bbP\inp{\phi(\vecA, \bw) \neq \bx}\leq \delta$$ for the problem of $\logreg$ only if the number of measurements $$m\geq \frac{1}{2}\inp{k+\frac{1}{\beta^2}}\log\inp{\frac{n}{k}}\inp{1 - \frac{h_2(\delta) + \delta k\log{n}}{k\log{n/k}}}.$$ 
\end{corollary}



Theorem~\ref{thm: lower_bdglm} also implies an information theoretic lower bound for \spl, which is presented below and proved in Appendix~\ref{proof:sec:lower_bd}. Note that the denominator term in the bound $\frac{1}{2}\log\inp{1+\frac{k}{\sigma^2}}$ is the capacity of a Gaussian channel with power constraint $k$ and noise variance $\sigma^2$. 
\begin{corollary}[\spl\ lower bound]\label{thm: spl_lower_bd_1}
Under the average power constraint \eqref{eq:power_constraint} on  $\vecA$, for a uniformly chosen $k$-sparse vector $\bx$, an algorithm $\phi$ satisfies $$\bbP\inp{\phi(\vecA, {\by}) \neq \bx}\leq \delta$$ only if the number of measurements
$$m\geq \frac{k\log\inp{\frac{n}{k}}-\inp{h_2(\delta) + \delta k\log{n}}}{\frac{1}{2}\log\inp{1+\frac{k}{\sigma^2}}}.$$
\end{corollary} 

\subsection{Tighter upper and lower bounds for \spl}\label{sec:tighter_bounds_spl}
We present information theoretic upper and lower bounds for \spl\ in this section. Similar to Section~\ref{sec:alg}, our upper bound is for the maximum probability of error, while the lower bounds hold even for the weaker criterion of average probability of error.

We first present an upper bound based on the maximum likelihood estimator (MLE) where  we  decode to $\hat{\bx}$ if, on output $\by$, 
\begin{align*}
\hat{\bx} = \argmax_{\stackrel{\bx\in \inb{0,1}^n}{\wh{\bx} = k}}\,\, p(\by|{\bx})
\end{align*} where $p(\by|{\bx})$ denotes the probability density function of $\by$ on input $\bx$.
\begin{theorem}[MLE upper bound for \spl]\label{thm:upper_bd_mle} Suppose  entries of the measurement matrix $\vecA$ are i.i.d. $\cN(0,1).$
The MLE  is correct with high probability if 
\begin{align}m\geq \max_{l\in[1:k]}  \frac{nN(l)}{\frac{1}{2}\log\inp{\frac{ l}{2\sigma^2}+1}}\label{eq:upper_bd_mle}
\end{align}where  $N(l):=  \frac{k}{n} h_2\inp{\frac{l}{k}} + (1-\frac{k}{n})h_2\inp{\frac{l}{n-k}}$. 
\end{theorem}
We prove the theorem in Appendix~\ref{proof:MLE}. The main proof idea involves analysing the probability that the output of the MLE is $2l$ Hamming distance away from the unknown signal $\bx$ for different values of $l\in [1:k]$ (assuming $k\leq n/2$). This depends on the number of such vectors (approximately $2^{nN(l)}$) and the probability that the MLE outputs a vector which is $2l$ Hamming distance away from $\bx$. 

Note that when $l = k\inp{1-\frac{k}{n}}$, $nN(l) = nh_2(k/n)\approx k\log{\frac{n}{k}}$ and $\log\inp{\frac{k\inp{1-k/n}}{2\sigma^2}+1}\leq \log\inp{\frac{k}{2\sigma^2}+1}$.
Thus, $m$ is at least $\frac{2k\log{n/k}}{\log\inp{\frac{k}{2\sigma^2}+1}}$ (see the bound for Corollary~\ref{thm: spl_lower_bd_1}). It is not immediately clear if this value of $l= k\inp{1-\frac{k}{n}}$ is the optimizer. However, for large $n$, this appears to be the case numerically as shown in Plot~\ref{plot:1}.

\begin{figure}[t]
\includegraphics[width=7cm]{Unknown2.png}
\centering
\caption{The figure shows the plot of the MLE upper bound \eqref{eq:upper_bd_mle} (given by m1) for different values of $k$. This is displayed in blue color. A plot of $\frac{2nN(l)}{\log\inp{\frac{ l}{2\sigma^2}+1}}$ is also presented for $l = k\inp{1-\frac{k}{n}}$ in orange color, given by m2. A part of the plot is zoomed in to emphasize the closeness between the lines. In these plots,  $\sigma^2$ is set to 1,  $n$ is 50000 and $k$ ranges from 1000 to 25000 $(n/2)$. }\label{plot:1}
\end{figure}


Inspired by the MLE analysis, we derive a lower bound with the same structure as \eqref{eq:upper_bd_mle}. We generate the unknown signal $\bx$ using the following distribution: A vector $\tilde{\bx}$ is chosen uniformly at random from the set of all $k$-sparse vectors. Given $\tilde{\bx}$, the unknown input signal $\bx$ is chosen uniformly from the set of all $k$-sparse vector which are at a Hamming distance $2l$ from $\bx$. 
The lower bound is then obtained by computing upper and lower bounds on $I(\vecA, \by;\bx|\tilde{\bx})$.
We show this lower bound only for random matrices where each entry is chosen iid $\cN(0,1)$.
\begin{theorem}[\spl\ lower bound]\label{thm:lower_bd_spl}
If each entry of $\vecA$ is chosen iid $\cN(0,1)$, then for a uniformly chosen $k$-sparse vector $\bx$, an algorithm $\phi$ satisfies 
\begin{align}
    \bbP\inp{\phi(\vecA, {\by}) \neq \bx}\leq \delta\label{eq:spl_lower_bd_l}
\end{align}  only if the number of measurements $$m\geq \max_l\frac{nN(l) - 2\log{n}- h_2(\delta) - \delta k\log{n}}{\frac{1}{2}\log\inp{1+\frac{l}{\sigma^2}\inp{2-\frac{l}{k}}}} .$$
\end{theorem} The proof of Theorem~\ref{thm:lower_bd_spl} is given in Appendix~\ref{proof:MLE}.

If we choose $l = k\inp{1-\frac{k}{n}}$ in Theorem~\ref{thm:lower_bd_spl}, we recover corollary~\ref{thm: spl_lower_bd_1} for the special case of Gaussian design.
% \begin{corollary}\label{corollary2:lower_bd_spl}
% If  each entry of $\vecA$ is chosen iid $\cN(0,1)$, then for a uniformly chosen $k$-sparse vector $\bx$, an algorithm $\phi$ satisfies 
% $$\bbP\inp{\phi(\vecA, {\by}) \neq \bx}\leq \delta$$
% only if the number of measurements 
% $$m\geq \frac{k\log\inp{\frac{n}{k}} - 2\log{n}- h_2(\delta) - \delta k\log{n}}{\log\inp{1+\frac{k}{\sigma^2}}} .$$
% \end{corollary}

% Corollary~\ref{corollary2:lower_bd_spl} can also be proved directly for any sensing matrix $\vecA$ which satisfies \eqref{eq:power_constraint} (non-necessarily a Gaussian design). 


% \begin{figure}[t]
% \includegraphics[width=8cm]{plot.png}
% \centering
% \caption{The figure shows the plot of the MLE upper bound \eqref{eq:upper_bd_mle} (given by m1) for different values of $n$. This is displayed in blue color. A plot of $\frac{2nN(l)}{\log\inp{\frac{ l}{2\sigma^2}+1}}$ is also presented for $l = k\inp{1-\frac{k}{n}}$ in orange color, given by m2. In these plots,  $\sigma^2$ is set to 1 and $k$ is $0.2n$. }\label{plot:1}
% \end{figure}



\section{Upper Bounds For Strong Convexity and Smoothness}\label{sec:upper_bound}
\section{Results Beyond Squared Loss} \label{subsec:recover_quantify}

In regression problems under some assumptions, \citet{charikar2024quantifying} proves that the strong model’s error is smaller than the weak model’s, with the gap at least the strong model’s error on the weak labels.
This observation naturally raises the following question:
\textit{Can their proof be extended from squared loss to output distribution divergence?}
In this section, we show how to theoretically bridge the gap between squared loss and KL divergence within the overall proof framework established in~\citet{charikar2024quantifying}.
% Interestingly, our analysis reveals that employing KL divergence as the loss function can potentially lead to a reduction in the reverse KL divergence, and vice versa.
To begin with, we restate an assumption used in previous study.
\begin{assumption}[Convexity Assumption~\citep{charikar2024quantifying}] \label{convex_set}
The strong model learns fine-tuning tasks from a function class $\cF_{s}$, which is a convex set. 
\end{assumption}
\vspace{-5pt}
It requires that, for any $f, g \in \cF_s$, and for any $\lambda \in [0,1]$, there exists $h \in \cF$ such that for all $z \in \R^{d_s}$, $h(z) = \lambda f(z) + (1-\lambda) g(z)$. 
% And we do not assume anything about either $f^\star$ or $f_w$, which need not belong to $\cF$. 
To satisfy the convex set assumption, $\cF_s$ can be the class of all linear functions.
In these cases, $\cF_s$ is a convex set. 
Note that it is validated by practice: a popular way to fine-tune a pre-trained model on task-specific data is by tuning the weights of only the last linear layer of the model~\citep{howard2018universal,kumar2022fine}.

\subsection{Upper Bound (Realizability)} \label{subsub:realize}

Firstly, we consider the case where $\exists f_s \in \cF_s$ such that $F_s = F^\star$ (also called ``Realizability''~\citep{charikar2024quantifying}).
It means we can find a $f_s$ such that $f_s \circ h_s = f^\star \circ h^\star$.
This assumption implicitly indicates the strong power of pre-training. 
It requires that the representation $h_s$ has learned extremely enough information during pre-training, which is reasonable in modern large language models pre-trained on very large corpus~\citep{touvron2023llama,achiam2023gpt}.
The scale and diversity of the corpus ensure that the model is exposed to a broad spectrum of lexical, syntactic, and semantic structures, enhancing its ability to generalize effectively across varied language tasks.

We state our result in the realizable setting, which corresponds to Theorem 1 in~\citet{charikar2024quantifying}.
\begin{theorem}[Proved in \cref{proof_theorem_1-main}]
\label{thm:realizable-main}

Given $F^\star$, $F_w$ and $F_{sw}$ defined above.
Consider $\cF_s$ that satisfies Assumption~\ref{convex_set}. 
Consider WTSG using reverse KL divergence loss:
\begin{align*}
    f_{sw} = \argmin_{f \in \cF_{s}}\; \dist(f \circ h_s, f_w \circ h_w).
\end{align*}
Assume that $\exists f_s \in \cF_s$ such that $F_s = F^\star$.
Then
\begin{align} \label{eqn:realizable-main}
    \dist(F^\star, F_{sw}) \le \dist(F^\star, F_w) - \dist(F_{sw}, F_w).
\end{align}
\end{theorem}

\begin{remark}
    The corresponding theorem and proof in the case of forward KL divergence loss is provided in~\cref{thm:realizable} from~\cref{proof_theorem_1}, under an additional assumption.
\end{remark}

In contrast to the symmetric squared loss studied in prior work~\citep{charikar2024quantifying}, the emergence of the reverse KL divergence is inherently tied to the asymmetric properties of the KL divergence.
Although extending previous work to both forward and reverse KL divergences presents significant technical challenges, our results demonstrate the theoretical guarantees of WTSG in these settings.
In Inequality~\eqref{eqn:realizable-main}, the left-hand side represents the error of the weakly-supervised strong model on the true data. 
On the right-hand side, the first term denotes the true error of the weak model, while the second term captures the disagreement between the strong and weak models, which also serves as the minimization objective in WTSG. 
This inequality indicates that the weakly-supervised strong model improves upon the weak model by at least the magnitude of their disagreement, $\dist(F_{sw}, F_w)$.
To reduce the error of $F_{sw}$, \cref{thm:realizable-main} aligns with~\cref{lemma:upper_lower_inf}, highlighting the importance of selecting an effective weak model and the inherent limitations of the optimization objective in WTSG.


% Notice that the error of weak model and strong model in~\cref{thm:realizable-main} is the reverse version, which fundamentally stems from the asymmetric properties of KL divergence.
% Despite this subtle difference, our empirical results in the experiments demonstrate consistent trends between forward and reverse KL divergence.











\subsection{Upper Bound (Non-Realizability)}

Now we relax the ``realizability'' condition and draw $n$ i.i.d. samples to perform WTSG.
We provide the result in the ``unrealizable'' setting, where the condition $F_s = F^\star$ may not be satisfied for any $f_s \in \cF_s$.
It corresponds to Theorem 2 in~\citet{charikar2024quantifying}.

\begin{theorem}[Proved in~\cref{proof_non-realizable-main}] \label{thm:non-realizable-finite-samples-main}
Given $F^\star$, $F_w$ and $F_{sw}$ defined above.
Consider $\cF_s$ that satisfies~\cref{convex_set}.
Consider weak-to-strong generalization using reverse KL:
\begin{align*}
    & f_{sw} = \argmin_{f \in \cF_{s}}\; \dist(f \circ h_s, f_w \circ h_w),
    \\ & \hat{f}_{sw} = \argmin_{f \in \cF_{s}}\; \hat{d}_{\cP}(f \circ h_s, f_w \circ h_w),
\end{align*}
Denote $\dist(F^\star, F_s) = \eps$. 
With probability at least $1-\delta$ over the draw of $n$ i.i.d. samples, there holds
\begin{multline} 
\dist(F^\star, \hat{F}_{sw}) \le \dist(F^\star, F_w) - \dist(\hat{F}_{sw}, F_w) + \\ \cO(\sqrt{\eps}) +  \cO\left(\sqrt{\frac{\cC_{\cF_s}}{n}}\right) + \cO\left(\sqrt{\frac{\log(1/\delta)}{n}}\right),
\end{multline}
where $\cC_{\cF_s}$ is a constant capturing the complexity of the function class $\cF_s$, and the asymptotic notation is with respect to $\eps \to 0, n \to \infty$.
\end{theorem}

\begin{remark}
    The extension to forward KL divergence loss is provided in~\cref{thm:non-realizable-finite-samples} from~\cref{proof_non-realizable}, under an additional assumption.
\end{remark}


Compared to Inequality~\eqref{eqn:realizable-main}, this bound introduces two another error terms: the first term of $\cO(\sqrt{\eps})$ arises due to the non-realizability assumption, and diminishes as the strong ceiling model $F_s$ becomes more expressive.
The remaining two error terms arise from the strong model $\hat{F}_{sw}$ being trained on a finite weakly-labeled sample. They also asymptotically approach zero as the sample size increases.







\begin{figure*}[t]
  \centering
  \subfigure[Realizable (pre-training).]{
    \includegraphics[width=0.31\textwidth]{images/exp_2/realizable-pretrain.pdf}
  }
  \label{fig3:a}
  \subfigure[Non-realizable (pre-training).]{
    \includegraphics[width=0.31\textwidth]{images/exp_2/unrealizable-pretrain.pdf}
  }
  \subfigure[Non-realizable (perturbation).]{
    \includegraphics[width=0.31\textwidth]{images/exp_2/perturb.pdf}
  }
  \subfigure[Realizable (pre-training).]{
    \includegraphics[width=0.31\textwidth]{images/exp_2/realizable-pretrain_forward.pdf}
  }
  \subfigure[Non-realizable (pre-training).]{
    \includegraphics[width=0.31\textwidth]{images/exp_2/unrealizable-pretrain_forward.pdf}
  }
  \subfigure[Non-realizable (perturbation).]{
    \includegraphics[width=0.31\textwidth]{images/exp_2/perturb_forward.pdf}
  }
  \vspace{-5pt}
  \caption{Experiments on synthetic data using reverse KL divergence loss (\textbf{a-c}) and forward KL divergence loss (\textbf{d-f}). 
  Each point corresponds to a task and the gray dotted line represents $y=x$. 
  $h^{\star}$ is a 16-layer MLP. 
  (\textbf{a,d}) Realizable (pre-training): $h_s=h^\star$, and $h_w$ is a 2-layer MLP obtained by pre-training. (\textbf{b,e}) Non-realizable (pre-training): $h_s$ is an 8-layer MLP, and $h_w$ is a 2-layer MLP. Both $h_s$ and $h_w$ are obtained by pre-training. (\textbf{c,f}) Non-realizable (perturbation): Both $h_s$ and $h_w$ are obtained by directly perturbing the weights in $h^{\star}$:  $h_s=h^{\star}+ \mathcal{N}\left(0,0.01\right)$, and $h_w=h^{\star}+\mathcal{N}\left(0,9\right)$.}
  \label{syn_result:reverse}
  \vspace{-10pt}
\end{figure*}




\subsection{Synthetic Experiments} \label{section:syn_exp}
In this section, we conduct experiments on synthetically generated data to validate the theoretical results in~\cref{subsec:recover_quantify}.
While drawing inspiration from the theoretical framework of~\citet{charikar2024quantifying}, we extend their synthetic experiments by replacing the squared loss used in their work with the output distribution divergence defined in~\cref{def:kl_dist_emp}.


\subsubsection{Experimental Setting}

In our setup, The data distribution $\mathcal{P}$ is chosen as $\mathcal{N}(0, \sigma^2 I)$, with $\sigma=500$ to ensure the data is well-dispersed.
The ground truth representation $h^\star:\R^8 \to \R^{16}$ is implemented as a randomly initialized 16-layer multi-layer perceptron (MLP) with ReLU activations.
Let the weak model and strong model representations $h_w, h_s: \R^8 \to \R^{16}$ be 2-layer and 8-layer MLP with ReLU activations, respectively.
Given $h_w$ and $h_s$ frozen, both the strong and weak models learn from the fine-tuning task class $\cF_s$, which consists of linear functions mapping $\R^{16}\to\R$.
This makes $\cF_s$ a convex set. 

For the ``realizable'' setting, we set $h_s = h^\star$.
For the ``unrealizable'' setting, we adopt the approach of~\citet{charikar2024quantifying} and investigate two methods for generating weak and strong representations: (1) \textbf{Pre-training}: 20 models $f_1^\star,\dots,f_{20}^\star: \R^8 \to \R^{16}$ are randomly sampled as fine-tuning tasks. 2000 data points are independently generated from $\cP$ for these tasks. Accordingly, $h_w$ and $h_s$ are obtained by minimizing the average output distribution divergence between ground truth label ($f_t^\star \circ h^\star$) and model prediction ($f_t^\star \circ h_w$ and $f_t^\star \circ h_s$) over the 20 tasks.
(2) \textbf{Perturbations}: As an alternative, we directly perturb the parameters of $h^\star$ to obtain the weak and strong representations. 
Specifically, we add independent Gaussian noises $\mathcal{N}(0, \sigma_s^2)$ and $\mathcal{N}(0, \sigma_w^2)$ to every parameter in $h^\star$ to generate $h_s$ and $h_w$, respectively.
% We add independent Gaussian noise $\mathcal{N}(0, \sigma_s^2)$ to every parameter in $h^\star$ to generate $h_s$. Similarly, we perturb $h^\star$ with $\mathcal{N}(0, \sigma_w^2)$ to generate $h_w$. 
To ensure the strong representation $h_s$ is closer to $h^\star$ than $h_w$~\citep{charikar2024quantifying}, we set $\sigma_s=0.1$ and $\sigma_w=3$.



\noindent \textbf{Weak Model Finetuning.} 
We freeze the weak model representation $h_w$ and train the weak models on new fine-tuning tasks.
We randomly sample 100 new fine-tuning tasks $f_{21}^\star,\dots,f_{120}^\star: \R^8 \to \R^{16}$, and independently generate another 2000 data points from $\mathcal{P}$. 
For each task $t \in \{ 21, \cdots, 120 \}$, the corresponding weak model is obtained by minimizing the output distribution divergence between ground truth label and weak model prediction.

\noindent \textbf{Weak-to-Strong Supervision.} 
Using the trained weak models, we generate weakly labeled data to supervise the strong model.
Specifically, we first independently generate another 2000 data points from $\mathcal{P}$.
Then for each task $t \in \{ 21, \cdots, 120 \}$, the strong model is obtained by minimizing the output distribution divergence between weak model supervision and strong model prediction.
At this stage, the weak-to-strong training procedure is complete. The detailed introduction of above is in~\cref{appendix:syn_train}.

\noindent \textbf{Evaluation.}
We independently draw an additional 2000 samples from $\cP$ to construct the test set.
They are used to estimate $\dist(F^\star, F_{sw})$, $\dist(F^\star, F_w)$ and $\dist(F_{sw}, F_w)$ for each task $t \in \{ 21, \cdots, 120 \}$.
We estimate these quantities using their empirical counterparts: $\disthat(F^\star, F_w)$, $\disthat(F^\star, F_{sw})$, and $\disthat(F_{sw}, F_w)$.
To validate~\cref{thm:realizable-main}-\ref{thm:non-realizable-finite-samples-main} and visualize the trend clearly, we plot $\disthat(F^\star, F_w)-\disthat(F^\star, F_{sw})$ on the $x$-axis versus $\disthat(F_{sw}, F_w)$ on the $y$-axis. The results are presented in~\cref{syn_result:reverse}(a)-(c).
We also examine forward KL divergence loss. 
To validate~\cref{thm:realizable}-\ref{thm:non-realizable-finite-samples}, which extend~\cref{thm:realizable-main}-\ref{thm:non-realizable-finite-samples-main} to the case of using forward KL divergence loss in WTSG,
we plot $\disthat(F_w, F^\star)-\disthat(F_{sw}, F^\star)$ on the $x$-axis versus $\disthat(F_w, F_{sw})$ on the $y$-axis. 
The results are presented in~\cref{syn_result:reverse}(d)-(f).


\subsubsection{Results and Analysis}

\noindent \textbf{Reverse KL divergence loss.}
Similar to previous results of squared loss~\citep{charikar2024quantifying}, the points in our experiments also cluster around the line $y=x$.
This suggests that 
$\disthat(F^\star, F_w)-\disthat(F^\star, F_{sw}) \approx \disthat(F_{sw}, F_w)$.
It is consistent with our theoretical framework, suggesting that the improvement over the weak teacher can be quantified by the disagreement between strong and weak models.

\noindent \textbf{Forward KL divergence loss.}
The observed trend closely mirrors that of reverse KL. 
The dots are generally around the line $y=x$.
It suggest that the relationship 
$\disthat(F_w, F^\star)-\disthat(F_{sw}, F^\star) \approx \disthat(F_w, F_{sw})$
may also hold, indicating a similar theoretical guarantee for forward KL in WTSG.











\section{Proof of Main Result}\label{sec:main_proofs}
In this concluding section, we prove the main results of our paper, namely Lemma \ref{lem:strong_convexity}, Lemma \ref{lem:smoothness}, and Theorem \ref{thm:main}.

\subsection{Proof of Lemma \ref{lem:strong_convexity}}
As mentioned in Section \ref{sec:proof}, Lemma \ref{lem:strong_convexity} follows from Lemmas \ref{lem:convexity_algebra}, \ref{lem:convexity_lowerbound}, and \ref{lem:convexity_upperbound}; these are proven in Appendices \ref{sec:initial_lemmas}, \ref{sec:lower_bound}, and \ref{sec:upper_bound} respectively. We restate the results here for convenience.

\lSCA*
\lSCLB*
\lSCUB*
\lSCHP*
\begin{proof}
    Our proof strategy will be to put together the statements of the three lemmas and work backwards to calculate the value of the parameter $\epsilon$ needed from Lemmas \ref{lem:convexity_lowerbound} and \ref{lem:convexity_upperbound} (call them $\epsilon_1$ and $\epsilon_2$ for now). Combining the three aforementioned lemmas gives us:
    \begin{align*}
        \llangle \nabla \Loglikelihood, \Delta \rrangle &\geq \frac{\xi\gamma}{2} \left((1- \epsilon_1)\sigma^*_r\norm{\Delta}_F^2 + 2 \llangle \solset_U \Delta_V^T, \Delta_U \solset_V^T \rrangle \right) - \frac{25\Xi\gamma}{4}\left(\epsilon_2\sigma^*_r\norm{\Delta}_F^2 \right) \\
        &= \frac{2\xi(1- \epsilon_1) - 25\Xi\epsilon_2}{4} \gamma\sigma^*_r\norm{\Delta}_F^2 + \xi\gamma \llangle \solset_U \Delta_V^T, \Delta_U \solset_V^T \rrangle
    \end{align*}
    Recall from \eqref{eq:def_regularizer} that $\mathcal{R}(Z) = \norm{Z^TDZ}_F^2$. Therefore, $\nabla \mathcal{R}(Z) = 4DZZ^TDZ$. Using this identity and \eqref{eq:objective_function}, we get:
    \begin{align*}
        \llangle \nabla f, \Delta \rrangle &= \llangle \nabla \Loglikelihood, \Delta \rrangle + \frac{\lambda}{4}\llangle \nabla \mathcal{R}, \Delta \rrangle \\
        &\geq \frac{2\xi(1- \epsilon_1) - 25\Xi\epsilon_2}{4} \gamma\sigma^*_r\norm{\Delta}_F^2 + \xi\gamma \llangle \solset_U \Delta_V^T, \Delta_U \solset_V^T \rrangle + \lambda DZZ^TDZ.
    \end{align*}
    We focus on the last two terms. Define $\lambda' =  \frac{2\lambda}{\xi\gamma}$. Then
    \begin{align*}
        \xi\gamma \llangle \solset_U \Delta_V^T, \Delta_U \solset_V^T \rrangle + \lambda DZZ^TDZ = \frac{\xi\gamma}{2} \left(2\llangle \solset_U \Delta_V^T, \Delta_U \solset_V^T \rrangle + \lambda' DZZ^TDZ\right)
    \end{align*}
    Following the steps laid out in \citet{zheng2016convergence} (Appendix C.1), we get the inequality:
    \begin{align*}
        2\llangle \solset_U \Delta_V^T, \Delta_U \solset_V^T \rrangle + \lambda' DZZ^TDZ \geq \frac{\lambda'}{2}\norm{\solset^TD\Delta}_F^2 - \frac{7\lambda'}{2}\norm{\Delta}_F^4 + \left(\lambda' - \frac{1}{2}\right)\Tr{\solset^TD\Delta \solset^TD\Delta}
    \end{align*}
    We know that $\lambda = \frac{\xi\gamma}{4}$ (see \eqref{eq:objective_function}), which implies $\lambda' = 1/2$. Thus, the last term in the above inequality is cancelled out. Plugging this inequality back into the expression above, we get:
    \begin{align*}
        \llangle \nabla f, \Delta \rrangle &\geq \frac{2\xi(1- \epsilon_1) - 25\Xi\epsilon_2}{4} \gamma\sigma^*_r\norm{\Delta}_F^2 + \xi\gamma \llangle \solset_U \Delta_V^T, \Delta_U \solset_V^T \rrangle + \lambda DZZ^TDZ \\
        &\geq \frac{2\xi(1- \epsilon_1) - 25\Xi\epsilon_2}{4} \gamma\sigma^*_r\norm{\Delta}_F^2 + \frac{\xi\gamma}{2} (2\llangle \solset_U \Delta_V^T, \Delta_U \solset_V^T \rrangle + \lambda' DZZ^TDZ) \\
        &\geq \frac{2\xi(1- \epsilon_1) - 25\Xi\epsilon_2}{4} \gamma\sigma^*_r\norm{\Delta}_F^2 + \frac{\xi\gamma}{2} \left(\frac{1}{4}\norm{\solset^TD\Delta}_F^2 - \frac{7}{4}\norm{\Delta}_F^4\right) \\
        &\geq \frac{4\xi(1- \epsilon_1) - 50\Xi\epsilon_2  - 7\xi\epsilon_2}{8} \gamma\sigma^*_r\norm{\Delta}_F^2  + \frac{\xi\gamma}{8}\norm{\solset^TD\Delta}_F^2
    \end{align*}
Choosing $\epsilon_1 = 1/8$ and $\epsilon_2 = \tau/50 = \xi/(50\Xi)$ gives us $4\xi(1- \epsilon_1) - 50\Xi\epsilon_2  - 7\xi\epsilon_2 \geq 2\xi$. Therefore,
    \begin{align*}
        \llangle \nabla f, \Delta \rrangle &\geq \frac{\xi\gamma}{4}\sigma^*_r\norm{\Delta}_F^2  + \frac{\xi\gamma}{8}\norm{\solset^TD\Delta}_F^2
    \end{align*}
    The number of samples needed for Lemma \ref{lem:convexity_lowerbound} to hold with probability at least $1-\delta/2$ is  
    $$m_1 \geq 96 \mu r  \left(\kappa/\epsilon_1\right)^2 n\log\left(2n/\delta\right) = 6144 \mu r  \kappa^2 n\log\left(2n/\delta\right)$$
    The number of samples needed for Lemma \ref{lem:convexity_upperbound} to hold with probability at least $1-\delta/2$ is  
    $$m_2 \geq 845  \left(\mu r \kappa/\epsilon_2\right)^2 n \log\left(n/\delta\right) \geq 2112500  \left(\mu r \kappa /\tau \right)^2 n \log\left(2n/\delta\right)$$
    The two lemmas jointly hold with probability at least $1 - \delta$. Clearly, the sample complexity requirement from Lemma \ref{lem:convexity_upperbound} is higher. Thus, we can conclude that given  $m \geq 10^7\left(\mu r \kappa (\Xi/\xi)\right)^2 n \log\left(2n/\delta\right)$ samples, with probability at least $1-\delta$,
    \begin{align*}
        \llangle \nabla f, \Delta \rrangle &\geq \frac{\xi\gamma}{4}\sigma^*_r\norm{\Delta}_F^2  + \frac{\xi\gamma}{8}\norm{\solset^TD\Delta}_F^2 \ \forall \ Z \in \mathcal{H} \cup \mathcal{B}(\tau/50) \cup \overline{C} ,
    \end{align*}
\end{proof}

\subsection{Proof of Lemma \ref{lem:smoothness}}
Lemma \ref{lem:smoothness} follows from Lemmas \ref{lem:smoothness_algebra} and \ref{lem:smoothness_upperbound}; these are proven in Appendices \ref{sec:initial_lemmas} and \ref{sec:upper_bound} respectively. We restate the results here for convenience.
\lSA*
\lSUB*
\lSHP*
\begin{proof}
    From \eqref{eq:objective_function}, we get that 
    \begin{align}\label{eq:lem2_1}
        \nabla f &= \nabla \Loglikelihood + \frac{\lambda}{4} \nabla \mathcal{R} = \nabla \Loglikelihood + \lambda DZZ^TDZ \nonumber \\
        \therefore \norm{\nabla f}_F^2 &= \norm{\nabla \Loglikelihood + \lambda DZZ^TDZ}_F^2 \leq (\norm{\nabla \Loglikelihood}_F + \norm{\lambda DZZ^TDZ}_F)^2  \nonumber \\ 
        &\leq 2(\norm{\nabla \Loglikelihood}_F^2 + \lambda^2\norm{DZZ^TDZ}_F^2)
    \end{align}
    We have assumed that $Z \in \mathcal{B}(1)$, which implies $\norm{\Delta}_F^2 \leq \sigma^*_r \leq \sigma^*_1$. Using this bound along with
    the analysis in \citet{zheng2016convergence} (Appendix C.2), we get:
    \begin{align}\label{eq:lem2_2}
        \norm{DZZ^TDZ}_F^2 &\leq 6(\norm{\Delta}_F^2 + 4\sigma^*_1)\norm{\Delta}_F^2 \norm{Z}_2^2 + 4\sigma^*_1\norm{\solset^TD\Delta}_F^2 \nonumber \\
        &\leq 30\sigma^*_1\norm{\Delta}_F^2 \norm{Z}_2^2 + 4\sigma^*_1\norm{\solset^TD\Delta}_F^2 \  \nonumber \\
        &\leq 180(\sigma^*_1)^2\norm{\Delta}_F^2 + 4\sigma^*_1\norm{\solset^TD\Delta}_F^2 \ \quad (\norm{Z}_2^2 \leq 6\sigma^*_1)
    \end{align}
    The last bound can be derived as follows:
    \begin{align*}
        \norm{Z}_2^2 = \norm{\solset + \Delta}_2^2 \leq (\norm{\solset}_2 + \norm{\Delta}_2)^2 \leq 2(\norm{\solset}_2^2 + \norm{\Delta}_2^2) \leq 2(\norm{\solset}_2^2 + \norm{\Delta}_F^2) \leq 2(2\sigma^*_1 + \sigma^*_1) = 6\sigma^*_1
    \end{align*}
    Combining the bounds from Lemma \ref{lem:smoothness_algebra} and Lemma \ref{lem:smoothness_upperbound}, we see that if the number of samples $m$ is at least $2n\log(4n/\delta)$, then with probability at least $1-\delta$, $\forall Z \in \overline{\Incoherentset}$,
    \begin{align}\label{eq:lem2_3}
        \llangle \nabla \Loglikelihood, W \rrangle^2  &\leq 2 \Xi^2 \left(\mathcal{D}(\Delta\solset^T) + \frac{1}{4}\mathcal{D}(\Delta\Delta^T)\right) \, \mathcal{D}(WZ^T) \nonumber \\
        &\leq 2 \Xi^2 \left(16\gamma(\mu r \sigma^*_1) \norm{\Delta}_F^2 + 104 \gamma(\mu r \sigma^*_1) \norm{\Delta}_F^2\right) \, 192 \gamma(\mu r \sigma^*_1)  \norm{W}^2_F \nonumber \\
        &= 46080 (\Xi \gamma \mu r \sigma^*_1)^2 \norm{\Delta}_F^2 \norm{W}^2_F \nonumber \\
        \therefore \norm{\nabla \Loglikelihood}_F^2 &= \sup_{W \in \Real{n \times r}: \norm{W}_F = 1} \llangle \nabla \Loglikelihood, W \rrangle^2 \nonumber \\
        &\leq 46080 (\Xi \gamma \mu r \sigma^*_1)^2 \norm{\Delta}_F^2 
    \end{align}
    Putting together the bounds in \eqref{eq:lem2_1}, \eqref{eq:lem2_2}, and \eqref{eq:lem2_3}, and plugging in the value of $\lambda = \xi\gamma/4$, we see that if the number of samples $m$ is at least $2n\log(4n/\delta)$, then with probability at least $1-\delta$, $\forall \ Z \in \mathcal{B} \cap \overline{\Incoherentset}$,
    \begin{align*}
        \norm{\nabla f}_F^2 &\leq 2\left(46080 (\Xi \gamma \mu r \sigma^*_1)^2 \norm{\Delta}_F^2 + 12 (\xi\gamma\sigma^*_1)^2\norm{\Delta}_F^2\right)  + \frac{(\xi\gamma)^2}{2}\sigma^*_1\norm{\solset^TD\Delta}_F^2  \\
        &\leq 10^5 (\Xi \gamma \mu r \sigma^*_1)^2 \norm{\Delta}_F^2 + \frac{(\xi\gamma)^2}{2}\sigma^*_1\norm{\solset^TD\Delta}_F^2
    \end{align*}
    
\end{proof}

\subsection{Proof of Theorem \ref{thm:main}}
Lemmas \ref{lem:strong_convexity} and \ref{lem:smoothness} are the two key ingredients needed to prove the main theorem of this paper.
\tM*
\begin{proof}
We begin by following the standard steps in the analysis of gradient descent.
\begin{align*}
    \norm{\Delta(Z^{t+1})}_F^2 
    &= \norm{Z^{t+1} - \solset(Z^{t+1})}_F^2 \\
    &\leq \norm{Z^{t+1} - \solset(Z^{t})}_F^2 \\
    &= \norm{\mathcal{P}_{\mathcal{H}}(\mathcal{P}_{\Incoherentset}\left(Z^t - \eta \nabla f(Z^t) \right)) - \solset(Z^{t})}_F^2 \\
    &\leq \norm{Z^t - \eta \nabla f(Z^t) - \solset(Z^{t})}_F^2 \\
    &= \norm{\Delta(Z^{t}) - \eta \nabla f(Z^t)}_F^2\\    
    &= \norm{\Delta(Z^{t})}_F^2 + \eta^2 \norm{\nabla f(Z^t)}_F^2 - 2 \eta \llangle \nabla f(Z^t), \Delta(Z^{t}) \rrangle  
\end{align*}
The first inequality comes from the fact that $\solset(Z^{t+1})$ is the closest point in $\solset$ to $Z^{t + 1}$; this is by definition of $\solset(Z^{t+1})$. The second inequality follows from the fact that $\solset(Z^t) \in \Incoherentset$ (by Lemma \ref{lem:solset_in_C}) and $\solset(Z^t) \in \mathcal{H}$ (by assumption); thus, successive projections of the iterate on to $\Incoherentset$ and $\mathcal{H}$ can only bring it closer to $\solset(Z^t)$.

Next, suppose the following bounds hold for some positive constants $a, b, c,$ and $d$ and for all $t \in \mathbb{Z}_+$:
\begin{align}
    \llangle \nabla f(Z^t), \Delta(Z^t) \rrangle &\geq a \norm{\Delta(Z^t)}_F^2 + c \norm{\Delta(Z^t)^TD\solset(Z^t)}_F^2 \label{eq:thm_pf_lower} \\
    \norm{ \nabla f(Z^t) }_F^2 &\leq b \norm{\Delta(Z^t)}_F^2 + d \norm{\Delta(Z^t)^TD\solset(Z^t)}_F^2    \label{eq:thm_pf_upper}
\end{align}
It follows that:
\begin{align}\label{eq:iterate_inequality}
    \norm{\Delta(Z^{t+1})}_F^2 &\leq \norm{\Delta(Z^{t})}_F^2 + \eta^2 \norm{\nabla f(Z^t)}_F^2 - 2 \eta \llangle \nabla f(Z^t), \Delta(Z^{t}) \rrangle \nonumber \\
    &\leq (1 - 2\eta a + \eta^2 b) \norm{\Delta(Z^t)}_F^2 + (\eta^2 d - 2 \eta c) \norm{\Delta(Z^t)^TD\solset(Z^t)}_F^2 \nonumber \\
    &\leq (1 - \eta a) \norm{\Delta(Z^t)}_F^2,
\end{align}
provided $\eta \leq \min(a/b, 2c/d)$. The last step can be justified  as follows:
\begin{align*}
    \eta &\leq \frac{a}{b} \Rightarrow (1 - 2\eta a + \eta^2 b) \leq 1 - \eta a, \quad 
    \eta \leq \frac{2c}{d} \Rightarrow \eta^2 d - 2 \eta c \leq 0
\end{align*}
Further, if $\eta \leq 1/a$, then $1 - \eta a \geq 0$, implying that the right-hand side of \eqref{eq:iterate_inequality} remains positive. This allows us to use the inequality repeatedly to yield:
\begin{align*}
    \norm{\Delta(Z^{t})}_F^2 \leq (1-\eta a)^t \norm{\Delta(Z^{0})}_F^2 \ \forall \ t \in \mathbb{Z}_+
\end{align*}

Finally, observe that we have assumed the number of samples given, $m$, is at least $10^7\left(\mu r \kappa/\tau\right)^2 n \log\left(8n/\delta\right)$. This ensures that with probability at least $1-\delta$, both Lemmas \ref{lem:strong_convexity} and \ref{lem:smoothness} hold. Lemmas \ref{lem:strong_convexity} and \ref{lem:smoothness} imply that the inequalities \eqref{eq:thm_pf_lower} and \eqref{eq:thm_pf_upper} hold for all $Z \in \mathcal{H} \cap \mathcal{B}(\tau/50) \cap \overline{\Incoherentset}$ with parameters:
\begin{align*}
    {a = \frac{\xi\gamma}{4} \sigma^*_r, \quad b = 10^5 (\Xi \gamma \mu r \sigma^*_1)^2, \quad c = \frac{\xi\gamma}{8}, \quad d = \frac{(\xi\gamma)^2}{2} \sigma^*_1}
\end{align*}
Given these parameters, as long as the stepsize $\eta$ satisfies $\eta \leq a/b = 2.5 \cdot 10^{-6} (\tau/\mu r \kappa)^2 /\alpha $, the other conditions on $\eta$ are automatically satisfied.
\end{proof}





\end{document}
