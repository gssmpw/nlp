The generative model, and consequently the log likelihood function, is invariant to certain transformations in the parameters. In other words, the problem structure has certain symmetries. We explore these symmetries and their consequences in this section.

\paragraph{Scale Invariance} 
For any ground-truth score matrix $X^*$, the factorization $(U^*, V^*)$ is not unique.
Indeed, for any invertible $r \times r$ matrix $P$, the pair of feature matrices $(U^*P^T, V^*P^{-1})$ is indistinguishable from $(U^*, V^*)$ as they both lead to the same score matrix $X^*$. However, we can distinguish `imbalanced' feature vectors from `balanced' ones by by adding the term $\norm{U^TU - V^TV}_F^2$ to the loss function. Minimizing this regularizer while keeping the log-likelihood constant leads to a pair of feature matrices that are balanced in the norms. In more compact terms, the regularizer can be written as follows:
\begin{align}\label{eq:def_regularizer}
    \mathcal{R}(Z) \triangleq \norm{Z^TDZ}_F^2; \ D \triangleq \begin{bmatrix}
        I_{n_1} & 0\\
        0 & -I_{n_2}
    \end{bmatrix}.
\end{align}
Note that the ground-truth matrix $Z^*$ satisfies $\mathcal{R}(Z^*) = 0$. Combining the regularizer with the negative log likelihood, the objective function becomes:
\begin{align}\label{eq:objective_function}
    f(Z) \triangleq \mathcal{L}(Z) + (\lambda/4)\mathcal{R}(Z),
\end{align}
where $\lambda$ is a positive constant. In this work, we set $\lambda = \xi\gamma/4$; however, in practice, it may be viewed as a hyperparameter. In summary, adding the regularizer $\mathcal{R}(Z)$ factors out the scale-invariance of the problem.

\paragraph{Rotational Invariance}
Beyond the scale invariance, the problem at hand also exhibits rotational invariance.
Let $R$ be any orthogonal matrix in $r$ dimensions, \textit{i.e.}, $R \in \mathbb{R}^{r \times r}$ such that $RR^T=R^TR=I$. The pair of feature matrices $(U^*R, V^*R)$ give rise to the same scores as $(U^*, V^*)$. Thus, one can identify the ground-truth features only up to an orthogonal transformation. Denote this equivalence class of the ground-truth feature matrices by $\solset$:
\begin{align}\label{eq:def_solutionset}
    \solset \triangleq \{ \Tilde{Z}^* \, : \Tilde{Z}^* = Z^* R \text{ for some } \text{orthogonal } R\}.
\end{align}
This equivalence class of solutions naturally gives rise to a new distance metric that measures how close a candidate solution $Z$ is to $\solset.$
Define
\begin{align}\label{eq:def_distance}
     R(Z) &\triangleq \text{arg} \min_{R: R^TR = RR^T = I_r} \norm{Z - Z^* R}_F, \\
    \solz &\triangleq \text{arg} 
    \min_{\Tilde{Z}^* \in \solset} \norm{Z - \Tilde{Z}^*}_F = Z^* R(Z), \label{eq:def_closest_sol} \\
     \delz &\triangleq Z - \solz. \label{eq:def_difference_sol}
\end{align}
We measure the quality of a solution $Z$ by $\norm{\delz}_F$.


\paragraph{Shift Invariance} Comparisons invariably involve computing the difference between the utilities of items. Therefore, the learning problem is invariant to a global shift in the scores. Mathematically, this can be seen as follows. Let $\Tilde{V}^* = V^* + 1v^T$, where $v \in \Real{r}$ and $1 \in \Real{n_2}$ is the vector of all ones. Let $\Tilde{X}^*, \Tilde{Y}^*,$ and $\Tilde{Z}^*$ denote the corresponding quantities derived from $(U^*, \Tilde{V}^*)$. Then for any triplet $(u; i, j)$ and the corresponding sampling matrix $A$, we have $\llangle e_u(\Tilde{e}_i - \Tilde{e}_j)^T, \Tilde{X}^* \rrangle = \llangle e_u(\Tilde{e}_i - \Tilde{e}_j)^T, X^* \rrangle$, which implies $\llangle A, \Tilde{Y}^* \rrangle = \llangle A, Y^* \rrangle.$ Because of this invariance, we assume, without loss of generality, that $1^T V^* = 0$. In words, we assume that the item features of all matrices in $\solset$ sum to zero.

The shift invariance also manifests itself in our objective function $\Loglikelihood(Z)$. It is important to factor out the shift invariance in order to establish a strong-convexity like property (i.e., a curvature) for $\Loglikelihood(Z)$. Therefore, we restrict our attention to the following subspace:
\begin{align}\label{eq:H_hyperplane}
    \mathcal{H} = \{ Z \in \Real{n \times r}: Z = (U, V), 1^TV = 0 \}.
\end{align}
For any $Z = (U, V)$, we shall work with the projection of $Z$ onto $\mathcal{H}$, denoted by $\mathcal{P}_{\mathcal{H}}(Z)$. This projection is given by $(U, JV)$, where $J \triangleq I_{n_2} - 11^T/(n_2)$. Finally, note that by the assumption stated before, $\solset \subseteq \mathcal{H}$.

