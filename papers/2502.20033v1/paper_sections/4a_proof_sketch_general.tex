We begin by formally stating the two lemmas that are used in the proof of Theorem \ref{thm:main}. 
\begin{restatable}{lemma}{lSCHP}\label{lem:strong_convexity}
    Suppose the number of samples $m$ is at least $10^7\left(\mu r \kappa/\tau\right)^2 n \log\left(2n/\delta\right)$, for some $\delta \in (0,1)$.   
    Then, with probability at least $1-\delta$, $\forall \ Z \in \mathcal{H} \cap \mathcal{B}(\tau/50) \cap \overline{\Incoherentset}$,
    \begin{align*}
        \llangle \nabla f(Z), \delz \rrangle &\geq \frac{\xi\gamma}{4} \norm{\delz}_F^2 \\ &\qquad  + \frac{\xi\gamma}{8} \norm{\Delta(Z)^TD\solset(Z)}_F^2.
    \end{align*}
\end{restatable}

\begin{restatable}{lemma}{lSHP}\label{lem:smoothness}
    Suppose the number of samples $m$ is at least {$2n\log(4n/\delta)$}, for some $\delta \in (0,1)$.
    Then, with probability at least $1-\delta$, $\forall \ Z \in \mathcal{B}(1) \cap \overline{\Incoherentset}$,
    \begin{align*}
        \norm{\nabla f(Z)}^2_F &\leq 10^5 (\Xi \gamma \mu r \sigma^*_1)^2 \norm{\Delta(Z)}_F^2 \\ &\qquad + \frac{(\xi\gamma)^2}{2}\sigma^*_1\norm{\solset(Z)^TD\Delta(Z)}_F^2.
    \end{align*}
\end{restatable}
The statements of these lemmas as well as the strategy we follow to prove them are similar to (and inspired by) those in \citet{zheng2016convergence}.

At a high level, the method for proving both these lemmas is similar. First, the expressions to be bounded, namely $\llangle \nabla f, \Delta \rrangle$ and $\norm{\nabla f}_F^2$, are written out as the sum and product of terms of the following form (Lemmas \ref{lem:convexity_algebra} and \ref{lem:smoothness_algebra}):
\begin{align}\label{eq:def_D_operator}
    \mathcal{D}(Y) &\triangleq \frac{1}{m}\sum_{k = 1}^m \llangle A_k + A_k^T, Y \rrangle^2  \, ; \ Y \in \Real{n \times n}. 
\end{align}
We overload the notation $\mathcal{D}$ to highlight the fact that the operator $\mathcal{D}(\cdot)$ captures the collective action of all the sampling matrices of the dataset $\mathcal{D}$.  Second, we demonstrate that these terms, which capture an empirical mean of i.i.d. random variables, are close to their statistical mean (Lemmas \ref{lem:convexity_lowerbound}, \ref{lem:convexity_upperbound}, and \ref{lem:smoothness_upperbound}). Specifically, we show that with high probability, $\mathcal{D}(Y) \approx \mathbb{E}[\mathcal{D}(Y)]$, uniformly for all $Y$ in some appropriate set. Finally, we put these results together with the appropriate parameters to ensure that the bounds presented in Lemmas \ref{lem:strong_convexity} and \ref{lem:smoothness} hold. 

Before proceeding further, we introduce some new notation. First, we drop the dependence on $Z$ for brevity; \textit{e.g.}, we denote $\nabla f(Z)$ by $\nabla f$. Second, for any matrix $Z \in \Real{n \times r}$, we use $Z_U \in \Real{n_1 \times r}$ to denote the first $n_1$ rows of $Z$ (the user features) and $Z_V \in \Real{n_2 \times r}$ to denote the last $n_2$ rows of $Z$ (the item features). In particular, $Z^*_U = U^*\Sigma^{*1/2}$ and $Z^*_V = V^*\Sigma^{*1/2}$. 
