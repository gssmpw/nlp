\begin{figure*}[ht]
    \centering
    \begin{subfigure}[b]{0.35\textwidth}
        \centering
        \includegraphics[width=\textwidth]{plots/1_1.png}
        \caption{Different initializations}
    \end{subfigure}
    \hspace{1.1cm}
    \begin{subfigure}[b]{0.35\textwidth}
        \centering
        \includegraphics[width=\textwidth]{plots/2_1_1.png}
        \caption{Varying dataset size}
    \end{subfigure}\\
        \begin{subfigure}[b]{0.35\textwidth}
        \centering
       \includegraphics[width=\textwidth]{plots/4_1.png}
       \caption{Different initializations}
    \end{subfigure}
    \hspace{1.1cm}
       \begin{subfigure}[b]{0.35\textwidth}
        \centering
       \includegraphics[width=\textwidth]{plots/5_1.png}
        \caption{Varying dataset size}
    \end{subfigure}
    \caption{The top row and bottom row show the results for $(n_1,n_2)=(200,300)$ and $(n_1,n_2)=(2000,3000)$, respectively. (a) and (c) illustrate the effect of different initializations with a fixed number of data points, while the remaining plots demonstrate the effect of varying dataset size $m$. Y-axes are in log scale. }
    \label{fig:1}
\end{figure*}
