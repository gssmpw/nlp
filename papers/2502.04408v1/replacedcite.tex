\section{Related Work}
Automated radiotherapy treatment planning (ATP) employs various methods to enhance efficiency and quality. Automated Rule Implementation and Reasoning (ARIR), such as Pinnacle's AutoPlanning ____, uses predefined rules for iterative optimization, providing reliable results but requiring detailed user input. Knowledge-Based Planning (KBP), including atlas-based solutions like RapidPlan ____ and machine learning-based systems like RayStation ____, predicts dose-volume histograms from historical data, improving plan consistency but needing extensive training data. 
Deep Learning and AI approaches, using Convolutional Neural Networks (CNNs) and Generative Adversarial Networks (GANs), predict dose distributions directly from patient data, offering high efficiency but demanding significant computational resources. Reinforcement Learning (RL) learns optimal strategies through trial and error, offering adaptability but facing high computational costs and integration challenges. Focusing on RL methods, ____ presents a deep reinforcement learning (DRL)-based approach for beam angle optimization (BAO) in intensity-modulated radiation therapy (IMRT). By formulating BAO as a Markov Decision Process (MDP) and using a 3D-Unet for dose distribution prediction, this method enables rapid, personalized beam angle selection, improving treatment plan quality and efficiency compared to conventional methods. ____ introduces a DRL framework for optimizing daily dose fractions in the radiotherapy treatment of non-small cell lung carcinoma (NSCLC). This approach uses a virtual radiotherapy environment with non-invasive CT scan data and bio-inspired optimization algorithms to personalize treatment plans, showing superior adaptability and efficacy over conventional uniform dose delivery. ____ presents a reinforcement learning framework for adaptive radiation therapy (ART) that considers uncertain tumor biological responses to radiation. The proposed model adapts treatment plans dynamically based on predicted tumor volume changes, optimizing timing and dose adaptations to enhance tumor control and minimize organ-at-risk (OAR) toxicity compared to conventional fractionation schedules. ____ introduces a knowledge-guided deep reinforcement learning (KgDRL) framework to improve the training efficiency of a virtual treatment planner network (VTPN) for intelligent automatic treatment planning in radiotherapy. By integrating human experience into the DRL process, the KgDRL approach significantly reduces training time while maintaining high plan quality, making complex treatment planning scenarios more practical and clinically applicable. For a comprehensive discussion on RL approaches in medicine, refer to ____. 

Despite the promise of RL, computational bottlenecks persist as RL agents must learn everything from scratch. Recently, large language models (LLMs) have shown success in using prior medical knowledge for diagnosis and medical tasks ____, with both general-purpose models and task-specific versions fine-tuned for specific tasks ____. Our goal is to combine the strengths of iterative learning from RL and prior knowledge from large pretrained models. We achieve this by leveraging the action model concept ____, where a large pretrained multimodal model interacts with the external world through specific functions, allowing it to receive feedback and improve its performance. We developed an RL environment based on ____, enabling the multimodal agent to select the entry gantry angles based on the patient's CT scans and radiotherapist indications. The agent receives a scalar reward as feedback, summarizing the quality of the plan generated by the Monte Carlo approach.

\vspace{-2mm}