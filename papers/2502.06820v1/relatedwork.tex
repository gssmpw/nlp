\section{Related Work}
The recent surge in LLM research has reignited interest in PEFT research. To pursue favorable task performance while using only a small number of trainable parameters, current PEFT methods primarily lie in four categories: adding extra trainable modules \citep{houlsby2019parameter,ruckle2020adapterdrop}, selectively training a small subset of key parameters \citep{zaken2021bitfit,lawton2023neural}, employing reparameterization techniques like low-rank decomposition to the incremental matrices \citep{hu2021lora,zhang2023adalora,liu2024dora,hao2024flora}, or combining multiple strategies \citep{chen2023parameter}. Among them, low-rank methods have garnered significant attention due to their mergable nature and parameter efficiency. These low-rank methods, which aim to approximate large weight matrices using a few principal components, is highly analogous to techniques employed in data compression. In fact, low-rank decomposition (or singular value decomposition) and frequency-domain decomposition (e.g., JPEG compression) represents two fundamental tools in image compression and signal processing.
    
    For image compression, frequency-domain reconstruction (e.g., DCT) are preferred due to the inherent smoothness prior of image data \citep{wallace1991jpeg}. However, when dealing with the complex data structures of neural network parameter matrices, the relative efficacy of these approaches remains unexplored. To the best of our knowledge, although FourierFT \citep{gao2024parameter} has made an empirical study of frequency-domain PEFT by employing Fourier Transform, no prior work has conducted a rigorous comparison between low-rank and frequency-domain decomposition methods in the context of PEFT. Our work aims to bridge this gap by providing a comprehensive theoretical analysis and designing a more efficient frequency-domain PEFT method. 

\begin{figure}[t]
    \centering
    \begin{minipage}{0.63\textwidth}
        \centering
    \begin{minipage}[t]{0.43\linewidth}
    \centering
    \includegraphics[width=\linewidth]{Figures/qqp.pdf}
    \end{minipage}%
    \begin{minipage}[t]{0.43\linewidth}
    \centering
    \includegraphics[width=\linewidth]{Figures/fgvc.pdf}
    \end{minipage}
    \centering
    \caption{Performance comparison under different parameter budgets on QQP (RoBERTa-base) and FGVC (ViT-base).}
    \label{fig:budget}
    \end{minipage}
    \hfill
    \begin{minipage}{0.32\textwidth}
        \centering
        \includegraphics[width=0.9\textwidth]{Figures/img_para.pdf}
        \caption{Influence of $\alpha$ and $\mathcal{B}_s$ on MRPC (RoBERTa-base).}
        \label{fig:parameter_sen}
    \end{minipage}
    \vspace{-13pt}
\end{figure}