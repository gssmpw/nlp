\section{Related Work}
\sloppy This section situates the study within the context of Human-Computer Interaction (HCI) research on everyday wearable AR and the use of digital ethnography as a method to explore how these technologies integrate into daily life.

\subsection{Everyday AR Research}

The field of HCI has long been concerned with understanding and improving human interaction with technologies~\cite{card2018psychology}. Early research focused on graphical user interfaces (GUIs) and tangible interactions, eventually expanding to immersive technologies~\cite{bowman20043d} such as XR (Extended Reality), which encompasses Virtual Reality (VR), Augmented Reality (AR), and Mixed Reality (MR)~\cite{milgram1994taxonomy}.

AR\footnote{AR typically overlays digital elements onto the real world, while MR allows deeper integration and interaction between digital and physical environments~\cite{milgram1994taxonomy}. In this paper, the term `AR' will be used as an umbrella term encompassing both technologies for simplicity, except when discussing specific devices where the distinction is crucial.} has been widely studied in professional contexts, such as industrial maintenance~\cite{burova2020utilizing, gattullo2020and, cao2020exploratory, henderson2010exploring} and healthcare~\cite{gasques2021artemis, eom2022neurolens, baashar2023towards}, where its utility has been demonstrated. Shifting focus to more consumer-oriented use cases, \citet{bowman2021keynote} introduced the concept of \textit{Everyday AR}. This vision imagines a future where virtual displays are ubiquitous, providing constant access to information and applications. Bowman contrasted everyday AR with specialised, niche experiences, arguing that true everyday AR would depend on all-day, always-on AR glasses. This idea aligns with the concept of \textit{Pervasive AR} proposed by \citet{grubert2016towards},
%’s concept of , 
a system that continuously adapts to the user’s changing context, making continuous use feasible. Progress toward this vision of everyday AR has been driven by research directed at several key areas.

Research on wearable AR has focused on overcoming technical challenges related to display technologies, tracking, rendering, and interaction techniques~\cite{azuma2019road, billinghurst2021grand, tran2023wearable}. Efforts include improving visual coherence~\cite{bang2021lenslet, rathinavel2019varifocal, hamasaki2019varifocal}, enhancing large-scale tracking with semantic understanding~\cite{runz2018maskfusion, zhang2019hierarchical}, multimodal rendering~\cite{mandl2021neural, lopes2018adding}, and refining interaction methods (e.g., freehand gestures, speech commands, hardware-based inputs)~\cite{pei2022hand, schmitz2022squeezy, hirzle2019design}. These developments are critical for integrating AR into everyday use seamlessly.

Another major focus has been on designing intuitive and unobtrusive user interfaces for AR. Lu et al.~\cite{lu2020glanceable, lu2021evaluating} introduced Glanceable AR interfaces, enabling users to quickly gather information with brief glances at their periphery. Building on this, \citet{lu2022exploring} explored interface transition mechanisms while \citet{plabst2022push} examined how spatial notification placement impacts user performance. In another important development, \citet{xu2023xair} proposed a design framework incorporating explainable AI to make AR systems’ behaviour more interpretable, reducing user confusion and surprise from unexpected outcomes.

As AR technologies become more accessible, ethical and social concerns grow increasingly important. Privacy issues arise from AR devices collecting and displaying personal information, potentially exposing sensitive activities~\cite{regenbrecht2024see, wolf2018we}. Social acceptability includes concerns about discomfort caused by devices capturing information without bystanders’ knowledge~\cite{denning2014situ, o2023privacy}. The risk of malicious use—such as targeted attacks or manipulative visual overlays~\cite{eghtebas2021advantage, eghtebas2023co}—underscores the need for ethical safeguards to prevent harm. Lastly, AR’s societal impact could also widen divides by providing users with privileged access to information~\cite{regenbrecht2024see}.

While technical challenges, interfaces, and ethical considerations drive the development of AR, understanding how these research efforts translate into everyday applications is equally important.

\subsection{Everyday AR Applications}

Everyday AR applications are designed for regular, repeated use in daily routines. Drawing on the successes of mobile AR applications like Pokémon Go and social camera tools, \citet{azuma2019road} observed that while these examples achieved sustained engagement, most consumer AR applications failed to maintain user interest beyond initial curiosity.

To address this challenge, Azuma argued that compelling everyday AR applications must seamlessly blend real and virtual elements in meaningful ways~\cite{azuma2019road}. By `meaningful,' he referred to experiences where virtual content was deeply connected to the real world, making it feel relevant and integral. He proposed strategies such as reinforcing existing interactions, reskinning the real world with virtual overlays, and helping users remember through persistent digital content~\cite{azuma2015location}. Building on this, researchers like \citet{mathis2024everyday}, who surveyed 60 participants, highlighted promising use cases for assistive interfaces that help users recall information, disconnect from reality, and enhance communication through visual augmentations. \citet{glassner2003everyday} similarly envisioned wearable AR assisting with small, casual tasks, making life easier through graphics-based, text-free applications. In terms of productivity, \citet{pavanatto2021we, pavanatto2024multiple} demonstrated how wearable AR could overcome physical space limitations by using virtual monitors to create flexible, adaptable workspaces. In industry, Meta advanced everyday AR applications with its Aria Everyday Activities Dataset~\cite{lv2024aria}, which included 143 daily activity sequences recorded by multiple wearers across five geographically diverse indoor locations. The dataset was aimed at developing \textit{`truly personalised and contextualised AI assistants that can act as an extension to the wearer’s mind.'}

These studies and industry efforts collectively envision how AR will increasingly play a role in everyday life. Since 2022, affordable consumer AR devices have made this technology accessible to a broader audience beyond specialists and study participants. Meta's Ray-Ban smart glasses exemplify this trend, shipping over 700,000 pairs since their October 2023 release~\cite{wsj2023}. This shift opens up new opportunities to study how wearable AR is being used in daily life, especially during this early phase of adoption. Our study leverages this unique moment to explore real-world integration of wearable AR.

\subsection{Digital Ethnography} 

Digital or online ethnography, introduced by Masten and Plowman in 2003~\cite{masten2003digital}, extends traditional ethnographic methods to digital platforms, enabling observations beyond geographical and temporal boundaries. It involves collecting participants' experiences shared through text, images, and audio to interpret their relevance in daily life. This method has been widely adopted in HCI research, focusing on exploring interactions with emerging technologies and real-life contexts. Examples include the iPhone 3G~\cite{blythe2009critical}, insertable devices~\cite{komkaite2019underneath}, and interactions with robots in public spaces~\cite{nielsen2023using, yu2024understanding}. Platforms like YouTube and TikTok are common data sources, offering insights into user interactions through video content and/or community comments. Similarly, wearable AR is currently in the early adopter phase of the technology adoption lifecycle~\cite{rogers2014diffusion}, where individuals are enthusiastic about exploring new technologies and are more likely to share their experiences online. These early adopters play a critical role in shaping perceptions and influencing the early majority~\cite{rogers2014diffusion}. This presents an opportunity to capture real-world usage patterns that are challenging to observe in controlled settings.

While digital ethnography provides valuable insights, it has limitations. The lack of demographic information about content creators~\cite{nielsen2023using} and the possibility that they may not represent the broader user population~\cite{blythe2009critical} limit the transferability of findings. Content can include satire~\cite{blythe2009critical}, which is difficult to analyse due to its ambiguous intent. Additionally, some content may be performative, staged, or selectively shared by creators~\cite{paay2015connecting}, potentially misrepresenting natural behaviours. To address these issues, we critically evaluate data authenticity and interpret findings within broader patterns observed across multiple videos.

\textit{Ethical Considerations Statement}: Respecting the privacy of individuals sharing content on social media is essential in digital ethnography~\cite{nielsen2023using}. In compliance with European Union’s General Data Protection Regulation (GDPR)~\cite{GDPR2018}, personal data such as usernames were not published. While fair use policies (e.g., YouTube~\cite{FairUse2021}) permit the use of publicly uploaded material for research, informed consent was obtained for quoting content or using identifiable information like screenshots. This study received ethical approval from the University of Sydney Human Research Ethics Committee (HREC), protocol 2024/HE000889.

\subsection{Summary}

While HCI research on everyday AR has advanced in areas such as technical development, interaction design, ethics, and potential applications, much of this work is based on insights from controlled settings. These studies often capture only specific aspects of AR use, offering a limited understanding of how AR technology, as a whole, integrates into real-world consumer life over time. This study addresses this gap by using a digital ethnographic approach to leverage (1) the recent surge in consumer AR headsets and glasses and (2) the experiences documented by early adopters. By observing real-world interactions, this research provides HCI with empirical insights into sustained usage patterns and practical challenges, advancing our understanding of how wearable AR can be meaningfully integrated into daily life.