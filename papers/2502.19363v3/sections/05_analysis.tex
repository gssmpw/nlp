% \vspace{-2pt}
\subsection{Analysis of Quality Ratings} \label{sec:analysis}
% \vspace{-3pt}
\paragraph{Distribution of quality ratings.}
In Figure \ref{fig:quality_rating_distribution}, we present the distribution of quality ratings across different sources in DataPajama. 
Overall, the quality ratings for each source are primarily concentrated at 4 and 5, indicating generally high sample quality. 
This may be related to the fact that DataPajama is a subset of the curated and deduplicated slimpjama corpus. 
However, for the criteria of \emph{Knowledge Novelty} and \emph{Creativity}, there is a higher proportion of samples scoring 2 and 3, which is consistent with the lower average scores for these two criteria found in Table \ref{tab:sft_avgscore_domains}. 
Across all domains, only a few scientific domains like mathematics and medicine have \emph{Knowledge Novelty} scores above 3, while in \emph{Creativity}, only culture and entertainment scores were high at 3.64 and 3.56, respectively. 
Nevertheless, in DataPajama, the combined share of domains like mathematics and medicine is only 11.5\%, and similarly, the combined share of culture and entertainment is only 25\%, both of which are relatively small. This explains the modest ratings of the DataPajama dataset in terms of \emph{Knowledge Novelty} and \emph{Creativity}.
\begin{figure*}[t]
    \centering
    % \vskip 0.05in
    \centerline{\includegraphics[width=1.\linewidth]{figures/quality_rating_distribution_figure.pdf}}
    % \vskip -0.1in
    \caption{The distribution of quality ratings across different sources in DataPajama}
    \label{fig:quality_rating_distribution}
    % \vskip -5pt
\end{figure*}


\paragraph{Correlation between quality ratings and log-likelihood.}
In Figure \ref{fig:quality_rating_nll_corr}, we illustrate the correlation between quality ratings and the log-likelihood scores computed by Llama-2-7b \citep{touvron2023llama2}. Most quality criteria do not show a significant correlation with perplexity, except for the criteria of \emph{Structural standardization, Professionalism, and Creativity}, which have Spearman correlation coefficients ranging from 0.47 to 0.55, indicating a weak correlation. 
This indicates that our 14 quality criteria and sample-with-dataman method are independent of traditional perplexity metrics and filtering, indirectly showcasing the sophistication of the \emph{``reverse thinking''}.
\begin{figure}[t]
    \centering
    % \vskip 0.1in
    \centerline{\includegraphics[width=1.\linewidth]{figures/quality_rating_nll_corr_figure.pdf}}
    \caption{Correlations of quality ratings and negative log-likelihood scores by Llama-2-7B \citep{touvron2023llama2} over 30B tokens training documents. The negative log-likelihoods are averaged over the number of tokens, and are the logarithm of the perplexity score of a single sequence. We observe that perplexity scores are not good approximations for any quality criteria.}
    \label{fig:quality_rating_nll_corr}
    \vskip -10pt
\end{figure}


\subsection{Data Inspection} \label{sec:inspection}
Furthermore, we examined examples of original documents from each source under Dataman's quality criteria and ratings. 
Specifically, for each criterion, we randomly selected samples with ratings ranging from 1 to 5 from different sources and presented them in the Appendix~\ref{app:raw_documents}.
Notably, these samples represent only a small snippet; nonetheless, they exhibit significant quality differences. 
We invite readers to review these differences in detail, which compares high and low ratings.