\newtcolorbox{promptbox}[1]{
        boxrule = 1.5pt,
        % fontupper = \normalsize\sf,  
        fontupper = \footnotesize,  
        fonttitle = \bf\color{black},
        arc = 5pt,
        rounded corners,
        colframe = black,
        colbacktitle = white!97!blue,
        colback = white!97!blue,
        title = #1,
}

\begin{minipage}[t]{1.0\linewidth}
    \vspace{-10pt} %
    \begin{promptbox}{Full Prompt}\label{fig:full_prompt_template}
Please carefully read and analyze the following text, score it based on fourteen evaluation criteria and their respective scoring definitions. Additionally, select the most appropriate category from the fifteen domain types that best matches the content of the text. Let's think step by step.\\
   
\textbf{Text}:\{text\}\\
    
\textbf{Domain Types:}
[A]Medicine [B]Finance [C]Law [D]Education [E]Technology [F]Entertainment [G]Mathematics [H]Coding [I]Government [J]Culture [K]Transportation [L]Retail E-commerce [M]Telecommunication [N]Agriculture [O]Other\\

\textbf{The Higher The Score, The Evaluation Criteria}:

[1]Accuracy: the fewer grammar, referential, and spelling errors the text contains, and the more accurate its expression. \_/5

[2]Coherence: the more fluent the content is expressed, and the stronger its logical coherence. \_/5

[3]Language Consistency: the more consistent the use of language in the text, with less mixing of languages. \_/5

[4]Semantic Density: the greater the proportion of valid information in the text, with less irrelevant or redundant information. \_/5

[5]Knowledge Novelty: the more novel and cutting-edge the knowledge provided by the text, with more insightful views on the industry or topic. \_/5

[6]Topic Focus: the more the text content focuses on the topic, with less deviation from the main theme. \_/5

[7]Creativity: the more creative elements are shown in the text's expression. \_/5

[8]Professionalism: the more professional terminology appears in the text, with more accurate use of terms and more professional domain-specific expression. \_/5

[9]Style Consistency: the more consistent the style of the text, with proper and appropriate style transitions. \_/5

[10]Grammatical Diversity: the more varied and correct the grammatical structures used in the text, showing a richer language expression ability. \_/5

[11]Structural Standardization: the clearer the structure followed by the text and the more standardized its format. \_/5

[12]Originality: the fewer repetitions and similar content in the text. \_/5

[13]Sensitivity: the more appropriately sensitive topics are handled in the text, with less inappropriate content. \_/5

[14]Overall Score: the better the comprehensive evaluation of the text, with superior performance in all aspects.\_/5
    \end{promptbox}
    \vspace{5pt}
\end{minipage}%
