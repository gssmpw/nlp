\newtcolorbox{initialpromptbox}[1]{
        boxrule = 1.5pt,
        % fontupper = \normalsize\sf,  
        fontupper = \small,  
        fonttitle = \bf\color{black},
        arc = 5pt,
        rounded corners,
        colframe = black,
        colbacktitle = white!97!blue,
        colback = white!97!blue,
        title = #1,
}

\begin{minipage}[t]{1.\linewidth}
    % \vspace{0pt} %
    \begin{initialpromptbox}{Initial Quality Criteria}\label{fig:initial_quality_criteria}

[1] Semantic Fluency/Coherence/Logic: Evaluate whether the text is smooth and easy to read, whether the content is coherent, and whether the logic is clear.

[2] Content Consistency/Variability in Language Style: Evaluate if the information within the text is contradictory and if the language style is diverse.

[3] Topic Diversity: Determine the richness and variety of topics addressed in the text.

[4] Content Regularity/Formatting: Consider whether the text adheres to a certain structure or format.

[5] Content Redundancy: Analyze the extent of information repetition within the text.

[6] Proportion of Domain-Specific Vocabulary: Measure the frequency of professional terms or specific vocabulary used in the text (such as proper nouns, technical terms, or Classical Chinese).

[7] Proportion of Sensitive Topics: Examine the percentage of content that addresses sensitive topics (e.g., involving politics, toxicity).

[8] Proportion of Creative Expression: Assess the degree of creative or innovative expression in the text (e.g., use of rhetorical techniques).

[9] Degree of Language Mixing: Analyze the extent to which different languages are used within the text (i.e., the ratio of text in various languages).

[10] Complexity of Text Structure: Evaluate the overall complexity of the text's structure.

[11] Proportion of Long Sentences: Assess the ratio of long sentences within the text.

[12] Proportion of Grammatical, Reference, and Spelling Errors: Evaluate the ratio of grammatical errors (e.g., incorrect punctuation, unclear sentence breaks), reference errors, and spelling mistakes in the text.

[13] Proportion of Content Lacking Semantics: Determine the ratio of parts within the text that lack meaningful content (e.g., garbled text, HTML tags, XML elements, navigation bars, incomplete chart numbers, or disjointed citations).
    % \vspace{0.445cm}
    \end{initialpromptbox}%
\end{minipage}%
