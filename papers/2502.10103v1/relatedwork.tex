\section{Related Work}
\label{sec:related-work}

Inverse semigroups have been introduced by Wagner~\cite{Wagner52} and Preston~\cite{Preston54} to formalize partial symmetries. 
Implicitly, they had been studied even before for example in the context of so-called \emph{pseudogroups} \cite{Golab39}.
Inverse semigroup have been investigated extensively from geometric, combinatorial, and algorithmic viewpoints; see \eg \cite{ElliottLM24,Gray2020,Margolis-Meakin:1989,ms96,mun74,OlijnykSS10,Djadchenko77,Kleiman76,Kleiman79} for a rather random selection.
For additional background on inverse semigroups, we refer to the standard books \cite{Law99,Petrich84} and the many references therein.


While membership problems in infinite semigroups recently have also gained a lot of attention (see \cite{ChoffrutK05,BellHP17,DiekertPS2024,Dong24} for a few examples), let us in the following give an overview on related work on the membership problem with the input models we use in the present work.



\mysubparagraph{Membership Problem (Cayley table model).}

The membership problem has been studied for many algebraic structures.
Indeed \cite{JonesL76} shows that the membership problem for magmas (\ie\ having a binary operation with no additional axioms) is \Ptime-complete in the Cayley table model.
In contrast, by \cite{CollinsGLW24} for quasigroups (magma with ``inverses'', a.k.a.\ latin squares) it is in \NPOLYLOGTIME using similar techniques as  we apply for the first part of \cref{thm:main-CT}. 

The membership problems for semigroups in the Cayley table model has been introduced by Jones, Lien, and Laaser \cite{JonesLL76}.
Further studies by Barrington, Kadau, and Lange \cite{BarringtonKLM01} showed that it can be solved in $\mathsf{FOLL}$ for nilpotent groups of constant class.
This result has been further improved by Collins, Grochow, Levet, and \ifAnonimous Weiß\else the last author\fi~\cite{CollinsGLW24} showing that the problem can be solved in $\mathsf{FOLL}$ for all nilpotent groups (\ie of arbitrary class) and served as a catalyst for \ifAnonimous Fleischer\else the first author\fi's work \cite{Fleischer19diss,Fleischer22} giving \NPOLYLOGTIME algorithms for Clifford semigroups, on which we build in the present work.

\mysubparagraph{Membership Problem (Transformation/Partial Bijection Model).}

The membership problem for semigroups in the transformation model has been shown to be $\PSPACE$-complete by Kozen~\cite{koz77}.
Beaudry \cite{Beaudry88} showed that the membership problem in commutative semigroups is \NP-complete in the transformation model.
This was later extended in \cite{Beaudry88thesis,BeaudryMT92} to classify the complexity of the membership problem in aperiodic monoids as outlined above.

Based on Sims' work \cite{Sims67}, Furst, Hopcroft, and Luks \cite{FurstHopcroftLuks80} showed that the membership problem for permutation groups is solvable in polynomial time, which after several partial results \cite{LuksM88,Luks86,McKenzieC87} was improved to \NC by Babai, Luks, and Seress \cite{BabaiLS87}.
Interestingly, the problem of rational subset membership is \NP-complete due to Luks \cite{Luks93} (see also \cite{LohreyRZ22}).

Turning our attention to the partial bijection model, it was observed by Jack~\cite{Jack23} that the membership problem for \emph{inverse semigroups} given by partial bijections is \PSPACE-complete.
This follows from an earlier result by by Birget, Margolis, Meakin, and Weil~\cite{BirgetMMW94} showing that the intersection non-emptiness problem for \emph{inverse automata} is \PSPACE-complete.

\mysubparagraph{Subpower Membership Problem.}
While there is no obvious generalization of the partial bijection or transformation semigroup model to non-associative structure such as magmas, the subpower membership problem still can be posed in this case.
Indeed, the subpower membership problem initially has been studied within the context of universal algebra, see \eg \cite{Mayr12,BulatovMS19,Kompatscher24}, and has turned out to be $\mathsf{EXPTIME}$-complete \cite{Kozik08} in general.
For arbitrary semi\-groups the subpower membership problem has been shown to be $\PSPACE$-complete by Bulatov, Kozik, Mayr, and Steindl \cite{BulatovKMS16}.

Further results on the subpower membership problem in semigroups are due to Steindl giving a \Ptime vs.\ \NP-completeness dichotomy for the special case of bands \cite{Steindl17} and a \Ptime vs.\ \NP-complete vs.\ \PSPACE-complete trichotomy for combinatorial Rees matrix semigroups with adjoined identity
\cite{Steindl19}. 
Here it is interesting to note that by our results the \NP-completeness case does not exists for inverse semigroups.

\medbreak

We now turn our attention to the intimately related intersection non-emptiness problem.

\mysubparagraph{Intersection Non-Emptiness Problem.}
The DFA intersection non-emptiness problem has been introduced and shown to be $\PSPACE$-complete by Kozen~\cite{koz77}.
Further work studying the complexity (including parametrized and fine-grained complexity) of the DFA intersection non-emptiness problem can be found in \cite{FernauHW21,KarakostasLV03,LangeR92,HolzerK11,SwernofskyW15,Wehar14,OliveiraW20,ArrighiFHHJOW21}.
Two special cases are that the DFA intersection non-emptiness problem is \NP-complete for DFA accepting finite languages \cite{RampersadS10} and for DFA over a unary alphabet \cite{StockmeyerM73} (see also \cite{FernauK17}).

Another important special case are permutation automata~\cite{Thierrin68} (a.k.a.\ group DFAs). 
This is closely linked to the membership problem in groups, which is in \NC \cite{BabaiLS87}. 
Thus, it comes rather as a surprise that the intersection non-emptiness problem is \NP-complete as Brondin, Krebs, and McKenzie \cite{BlondinKM12} showed; however, when restricting to permutation automata with a single accepting state it, indeed, is in \NC \cite{BlondinKM12}.
Even more, intersection non-emptiness for permutation automata plus one context-free language is \PSPACE-complete \cite{LohreyRZ22}.

Note that every permutation automaton is an inverse automaton as studied in the present work and \eg by Birget, Margolis, Meakin, and Weil \cite{BirgetMMW94}.
Furthermore, inverse automata are a special case of \emph{reversible} automata (or injective automata as they are called in \cite{BirgetMMW94}), which were studied \eg by Pin \cite{Pin87} and Radionova and Okhotin \cite{RadionovaO24}.




\medbreak

Another problem related to membership is the conjugacy problem, which for infinite groups was introduced by Dehn in 1911 \cite{Dehn11}.
For generalizations to semigroups see \cite{Otto84}.

\mysubparagraph{Conjugacy Problem.}
The conjugacy problem for permutation groups is in \NP and hard for graph isomorphism as shown by Luks \cite{Luks93}.
Jack \cite{Jack23} showed that the conjugacy problem for inverse semigroups in the partial bijection model is \PSPACE-complete. 
For a overview on different variants of conjugacy in (inverse) semigroups, we refer to \cite{AraujoKinyonKnieczny19}.

\medbreak

The following problems, which are closely tied to the membership and conjugacy problems (see, e.g., \cref{lem:mgs-reduction} and \cref{prop:eqn-alg}), have also attracted independent interest.

\mysubparagraph{Minimum Generating Set Problem.}

The minimum generating set problem has first been considered by Papadimitriou and Yannahakis \cite{PapadimitriouY96} and further studied in \cite{ArvindT06,Tang13Thesis} showing polylogarithmically space-bounded algorithms.
For groups, it has been shown recently to be solvable in polynomial time by Lucchini and Thakkar \cite{LucchiniT24}.
This bound was further improved to \NC by Collins, Grochow, Levet, and \ifAnonimous Weiß\else the last author\fi~\cite{CollinsGLW24}.
Moreover, they also showed that the minimum generating set problem for magmas is \NP-complete.


\mysubparagraph{Equations.}

There is an extensive work on equations in algebraic structures. 
In particular, the case of groups has attracted a lot of attention after Goldmann and Russell \cite{GoldmannR02} showed that deciding satisfiability of a system of equations is \NP-complete for every fixed non-abelian group and in \Ptime for abelian groups. 
For more recent conditional lower bounds and algorithms for deciding satisfiability of (single) equations, see \eg \cite{IdziakKKW22,IdziakKKW24,FoldvariH19}.

The case of semigroups has attracted much less attention.
While here the closely related problem of checking identities has been investigated thoroughly, \cite{Klima09,SeifS06,Kisielewicz04,AlmeidaVG09,Seif05}, there is relatively little work on deciding whether a (system of) equation(s) has a solution.

In \cite{BarringtonMMTT00} the problem of deciding whether a (single) equation in finite monoids is satisfiable has been investigated.
Among other results it has been shown that in the Brandt monoid $B_2^1$, which also plays an important role in our work, this problem is \NP-complete.
Furthermore, in \cite{KlimaTT07} systems of equations in semigroups were studied.
They presented dichotomy results for the class of finite monoids and the class of finite regular semigroups.
The result for finite regular semigroups is for a restricted variant of the problem, where one side of each equation contains no variable.