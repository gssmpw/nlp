\begin{abstract}


Applying diffusion models to image-to-image translation (I2I) has recently received increasing attention due to its practical applications.
%
Previous attempts inject the information from the source image into \textbf{each} denoising step for an iterative refinement, leading to time-consuming implementation.
%
We propose an efficient method that equips a diffusion model with a lightweight translator, dubbed a Diffusion Model Translator (\method), to accomplish I2I. 
%
Specifically, we first provide theoretical justification that, when using the pioneer DDPM work to accomplish the I2T task, it is feasible and sufficient to transfer the distribution from one domain to another only at \textbf{some intermediate} step.
%
We then observe that the translation performance highly depends on the selected timestep to perform domain transfer, and therefore propose a practical strategy to automatically select an appropriate timestep for a given task.
%
We evaluate our approach on a range of I2I applications, namely, image stylization, image colorization, segmentation to image, and sketch to image, to validate its efficacy and general utility.
%
Comparisons show that our \method surpasses existing methods in both quality and efficiency.
%
Code will be made publicly available.

% Applying diffusion models to image-to-image translation (I2I) has recently received wide attention due to its practical applications. Previous attempts inject the information from the source image into \textbf{each} denoising step for an iterative refinement, leading to time-consuming implementation. We propose an efficient method that equips a diffusion model with a lightweight translator, dubbed a Diffusion Model Translator (DMT), to accomplish I2I. Specifically, We first provide theoretical justification that, when using the DDPM to accomplish the I2T task, it is feasible and sufficient to transfer the distribution from one domain to another only at \textbf{some intermediate} step. We then observe that the translation performance highly depends on the selected timestep to perform domain transfer, and therefore propose a practical strategy to automatically select an appropriate timestep for a given task. We evaluate our approach on a range of I2I applications, namely, image stylelization, image colorization, segmentation to image, and sketch to image, to validate its efficacy and general utility. Comparisons show that DMT surpasses existing methods in both quality and efficiency. Code will be made publicly available.

\end{abstract}