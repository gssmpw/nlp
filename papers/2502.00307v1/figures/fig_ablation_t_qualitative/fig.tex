\begin{figure*}[t]
\centering
%\setlength\tabcolsep{2pt}
%\begin{tabular}{cc}
%\begin{minipage}[t]{0.55\linewidth}\includegraphics[width=1\linewidth]{figures/fig_method/method_palette.pdf}\end{minipage} &
%\begin{minipage}[t]{0.45\linewidth}\includegraphics[width=1\linewidth]{figures/fig_method/method_dmt.pdf}\end{minipage} \\
%(a) Our proposed \method & (b) Palette
%\end{tabular}
\includegraphics[width=1.0\textwidth]{figures/fig_ablation_t_qualitative/fig_ablation_t_qualitative.pdf}
\vspace{-10pt}
\caption{
    \textbf{Qualitative results for ablation study} of the preset timestep $t$ in our proposed \method on the four I2I tasks.
    %
    \sqq{We observe that a smaller $t$ helps in better retaining the content information from the input source,} but suffers from a larger gap between the target domain and the source domain. 
    %
    The optimally selected timestep ($t^*$) for each of the four I2I tasks is given in \cref{tab:quantitative_ablation}.
}
\label{fig:ablation_t_qualitative}
\vspace{-5pt}
\end{figure*}
