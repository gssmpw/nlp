\begin{figure*}[!ht]
\centering
%\setlength\tabcolsep{2pt}
%\begin{tabular}{cc}
%\begin{minipage}[t]{0.55\linewidth}\includegraphics[width=1\linewidth]{figures/fig_method/method_palette.pdf}\end{minipage} &
%\begin{minipage}[t]{0.45\linewidth}\includegraphics[width=1\linewidth]{figures/fig_method/method_dmt.pdf}\end{minipage} \\
%(a) Our proposed \method & (b) Palette
%\end{tabular}
\includegraphics[width=0.98\textwidth]{figures/fig_comparison/comparison_highlight_nored.pdf}
\vspace{-15pt}
\caption{
    \newrevise{
    \textbf{Qualitative comparison}. Our \method achieves on par or better results than the three baseline methods Pix2Pix~\cite{isola2017image}, Palette~\cite{saharia2021palette}, TSIT~\cite{jiang2020tsit} on the four I2I tasks, \sqq{which are} image stylization, image colorization, segmentation to image, and sketch to image. Significant differences are highlighted in red or blue boxes, and brief textual explanations are provided \yr{besides the boxes}.
    %
    The comparison on efficiency can be found in \cref{tab:quantitative_comparison}.}
   %{\em It is noteworthy that our approach enjoys significantly better training/inference efficiency than existing DDPM-based approaches, like Palette~\cite{saharia2021palette}.} [{\bf This sentence has nothing to do with this figure. It should be deleted from the caption. Such a general statement about the efficiency issue can be put somewhere in the text.} ]
    %
}
\label{fig:comparison}
\vspace{-5pt}
\end{figure*}
