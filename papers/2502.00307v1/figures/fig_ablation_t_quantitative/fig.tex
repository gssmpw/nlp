% \definecolor{myblue}{RGB}{132, 180, 213}
% \definecolor{myred}{RGB}{213, 123, 101}
\definecolor{myblue}{RGB}{0, 0, 0}
\definecolor{myred}{RGB}{0, 0, 0}
\begin{figure*}[tp]
\centering
\begin{tabular}{cccc}
\begin{minipage}[t]{0.22\linewidth}\includegraphics[width=1\linewidth]{figures/fig_ablation_t_quantitative/portrait_qmupd0-1-0_ssim.pdf}\end{minipage} & \begin{minipage}[t]{0.22\linewidth}\includegraphics[width=1\linewidth]{figures/fig_ablation_t_quantitative/afhq_ssim.pdf}\end{minipage} &
\begin{minipage}[t]{0.22\linewidth}\includegraphics[width=1\linewidth]{figures/fig_ablation_t_quantitative/celeba_ssim.pdf}\end{minipage} & \begin{minipage}[t]{0.22\linewidth}\includegraphics[width=1\linewidth]{figures/fig_ablation_t_quantitative/edges2handbags_ssim.pdf}\end{minipage} \\
(a) Portrait & (b) AFHQ & (c) CelebA-HQ & (d) Edges2handbags
\end{tabular}
\caption{
    \textbf{Analysis on the preset timestep, $t$.}
    %
    Our \method needs \revise{a} pre-defined timestep to learn and perform the distribution shift.
    %
    We plot the distance between $(x_t,y_t)$ and $(x_0, x_t)$ at different timesteps, which are shown in red and blue curves, respectively.
    %
    When $t$ increases, $d(x_t, y_t)$ decreases so that the distribution is easier to shift from $x_t$ to $y_t$, while $d(x_0, x_t)$ increases so that the \revise{input condition} signal is becoming less relevant because $x_t$ is drifting away from the input $x_0$.
    %
    Considering such a trade-off, we select the intersection as the practical choice of the timestep for \method learning.
}
\label{fig:ablation_t_quantitative}
\vspace{-10pt}
\end{figure*}

