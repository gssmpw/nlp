\begin{figure*}[t]
\centering
%\setlength\tabcolsep{2pt}
%\begin{tabular}{cc}
%\begin{minipage}[t]{0.55\linewidth}\includegraphics[width=1\linewidth]{figures/fig_method/method_palette.pdf}\end{minipage} &
%\begin{minipage}[t]{0.45\linewidth}\includegraphics[width=1\linewidth]{figures/fig_method/method_dmt.pdf}\end{minipage} \\
%(a) Our proposed \method & (b) Palette
%\end{tabular}
\includegraphics[width=1.\textwidth]{figures/fig_method/method_dmt_generalization.pdf}
\vspace{-15pt}
\caption{
    \newrevise{
    \textbf{Conceptual comparison} for (a) multi-step \method and (b) asymmetric \method.
    %
    $\{x_t\}_{t=0}^T$ represent different states of the \revise{input from the source domain}, while $y_T \to y_0$ stands for the denoising process of DDPM.
    %
    Here, $T$ denotes the total number of noise-adding steps in the diffusion process.
    %
    Multi-step \method combines the translation results of \method at two different timesteps with an auxiliary fusion UNet and denoise to achieve the final output, while asymmetric \method applies translation at different timestep pair $(s,t)$.
    %
    More discussions are addressed in \cref{subsec:generalization} and \supp.
    }
}
\label{fig:method_generalization}
\vspace{-10pt}
\end{figure*}
