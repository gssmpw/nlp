\documentclass[10pt,journal,compsoc]{IEEEtran}
\usepackage{amsmath,amsfonts}
\usepackage{amsthm,amssymb}
\usepackage{amsmath,amssymb,amsfonts,dsfont,pifont,bm,bbm,mathrsfs,mathtools,nicefrac}
\usepackage{xcolor,algorithm,listings}
\usepackage{algorithmicx}
\usepackage{algpseudocode}
\usepackage{array}
\usepackage[caption=false,font=normalsize,labelfont=sf,textfont=sf]{subfig}
\usepackage{textcomp}
\usepackage{booktabs,multirow,adjustbox,diagbox,threeparttable,multicol,colortbl,tabularray}
\usepackage{stfloats}
\usepackage{url,setspace}
\usepackage{verbatim}
\usepackage{graphicx}
\usepackage{xspace}
\usepackage{cite}
\hyphenation{op-tical net-works semi-conduc-tor IEEE-Xplore}
\def\BibTeX{{\rm B\kern-.05em{\sc i\kern-.025em b}\kern-.08em
    T\kern-.1667em\lower.7ex\hbox{E}\kern-.125emX}}
\newtheorem{theorem}{Theorem}
\newtheorem{lemma}{Lemma}
\usepackage[pagebackref,breaklinks,colorlinks,linkcolor=blue,citecolor=blue, bookmarks=false]{hyperref}
\usepackage[capitalize]{cleveref}  % Should be loaded after 'hyperref', and works perfectly with 'subfigure'.'
\Crefname{section}{Section}{Sections}
\Crefname{table}{Table}{Tables}
\Crefname{figure}{Figure}{Figures}
\Crefname{equation}{Equation}{Equations}
\hyphenpenalty=1200

\definecolor{codegreen}{rgb}{0,0.6,0}
\definecolor{codegray}{rgb}{0.5,0.5,0.5}
\definecolor{codepurple}{rgb}{0.58,0,0.82}
\definecolor{backcolour}{rgb}{1.0,1.0,1.0}
\lstdefinestyle{mystyle}{
    backgroundcolor=\color{backcolour},
    commentstyle=\color{codegreen},
    keywordstyle=\color{magenta},
    numberstyle=\tiny\color{codegray},
    stringstyle=\color{codepurple},
    basicstyle=\ttfamily\scriptsize,
    breakatwhitespace=false,
    breaklines=true,
    captionpos=b,
    keepspaces=true,
    numbers=left,
    numbersep=5pt,
    showspaces=false,
    showstringspaces=false,
    showtabs=false,
    tabsize=2
}
\lstset{style=mystyle}

\newcommand{\balpha}[1]{\bar{\alpha}_{#1}}
\newcommand{\todo}[1]{\textcolor{red}{[TODO: #1]}}
\newcommand{\revise}[1]{\textcolor{black}{#1}}
\newcommand{\newrevise}[1]{\textcolor{black}{#1}}
\newcommand{\method}{DMT\xspace}
\newcommand{\supp}{\textit{Supplementary Material}\xspace}
\newcommand{\sqq}[1]{\textcolor{black}{#1}}
\newcommand{\yr}[1]{\textcolor{black}{#1}}

\begin{document}

\title{A Diffusion Model Translator for Efficient Image-to-Image Translation}

\author{
Mengfei Xia, Yu Zhou, Ran Yi, Yong-Jin Liu,~\IEEEmembership{Senior Member,~IEEE}, Wenping Wang,~\IEEEmembership{Fellow,~IEEE}
    % <-this % stops a space
\IEEEcompsocitemizethanks{\IEEEcompsocthanksitem  Mengfei Xia and Yu Zhou are with the MOE-Key Laboratory of Pervasive Computing, Department of Computer Science and Technology, Tsinghua University, Beijing 100084, China (email: \{xmf20, yzhou20\}@mails.tsinghua.edu.cn).% <-this % stops a space
\IEEEcompsocthanksitem Ran Yi is with the Department of Computer Science and Engineering, Shanghai Jiao Tong University, Shanghai, China (email: ranyi@sjtu.edu.cn).% <-this % stops a space
\IEEEcompsocthanksitem Yong-Jin Liu is with the MOE-Key Laboratory of Pervasive Computing, Department of Computer Science and Technology, Tsinghua University, Beijing 100084, China (email: liuyongjin@tsinghua.edu.cn).
\IEEEcompsocthanksitem Wenping Wang is with the Department of Computer Science and Computer Engineering at Texas A\&M University, Texas, The United States (email: wenping@tamu.edu).% <-this % stops a space}
\IEEEcompsocthanksitem Corresponding authors: Ran Yi and Yong-Jin Liu.}% <-this % stops a space}
}

% The paper headers
\markboth{IEEE TRANSACTIONS ON PATTERN ANALYSIS AND MACHINE INTELLIGENCE, VOL. 46, NO. 12, DECEMBER 2024}%
{Xia \MakeLowercase{\textit{et al.}}: A Diffusion Model Translator for Efficient Image-to-Image Translation}

% \IEEEpubid{0000--0000/00\$00.00~\copyright~2021 IEEE}
% Remember, if you use this you must call \IEEEpubidadjcol in the second
% column for its text to clear the IEEEpubid mark.


\IEEEtitleabstractindextext{
\begin{abstract}


Applying diffusion models to image-to-image translation (I2I) has recently received increasing attention due to its practical applications.
%
Previous attempts \sqq{inject information} from the source image into each denoising step for an iterative refinement, \sqq{thus resulting in a} time-consuming implementation.
%
We propose an efficient method that equips a diffusion model with a lightweight translator, dubbed a Diffusion Model Translator (\method), to accomplish I2I. 
%
Specifically, we first \sqq{offer} theoretical justification that \sqq{in employing} the pioneer\sqq{ing} DDPM work \sqq{for} the I2\revise{I} task, it is \sqq{both} feasible and sufficient to transfer the distribution from one domain to another only at some intermediate step.
%
We \sqq{further} observe that the translation performance highly depends on the \sqq{chosen timestep for} domain transfer, and therefore propose a practical strategy to automatically select an appropriate timestep for a given task.
%
We evaluate our approach on a range of I2I applications, \sqq{including} image stylization, image colorization, segmentation to image, and sketch to image, to validate its efficacy and general utility.
%
\sqq{The c}omparisons show that our \method surpasses existing methods in both quality and efficiency.
%
Code will be made publicly available.

\end{abstract}

\begin{IEEEkeywords}
Diffusion models, image translation, deep learning, generative models.
\end{IEEEkeywords}}

\maketitle

% \IEEEdisplaynontitleabstractindextext

% \IEEEpeerreviewmaketitle

\section{Introduction}
\label{sec::intro}

Embodied Question Answering (EQA) \cite{das2018embodied} represents a challenging task at the intersection of natural language processing, computer vision, and robotics, where an embodied agent (e.g., a UAV) must actively explore its environment to answer questions posed in natural language. While most existing research has concentrated on indoor EQA tasks \cite{gao2023room, pena2023visual}, such as exploring and answering questions within confined spaces like homes or offices \cite{liu2024aligning}, relatively little attention has been dedicated to EQA tasks in  open-ended city space. Nevertheless, extending EQA to city space is crucial for numerous real-world applications, including autonomous systems \cite{kalinowska2023embodied}, urban region profiling \cite{yan2024urbanclip}, and city planning \cite{gao2024embodiedcity}. 
% 1. 环境复杂性   
%    - 地标重复性问题(如区分相似建筑)  
%    - 动态干扰因素(交通流、行人)  
% 2. 行动复杂性  
%    - 长程导航路径规划  
%    - 移动控制、角度等  
% 3. 感知复杂性  
%    - 复合空间关系推理("A楼东侧商铺西边的车辆")  
%    - 时序依赖的观察结果整合

EQA tasks in city space (referred to as CityEQA) introduce a unique set of challenges that fundamentally differ from those encountered in indoor environments. Compared to indoor EQA, CityEQA faces three main challenges: 

1) \textbf{Environmental complexity with ambiguous objects}: 
Urban environments are inherently more complex,  featuring a diverse range of objects and structures, many of which are visually similar and difficult to distinguish without detailed semantic information (e.g., buildings, roads, and vehicles). This complexity makes it challenging to construct task instructions and specify the desired information accurately, as shown in Figure \ref{fig:example}. 

2) \textbf{Action complexity in cross-scale space}: 
The vast geographical scale of city space compels agents to adopt larger movement amplitudes to enhance exploration efficiency. However, it might risk overlooking detailed information within the scene. Therefore, agents require cross-scale action adjustment capabilities to effectively balance long-distance path planning with fine-grained movement and angular control.

3) \textbf{Perception complexity with observation dynamics}: 
% Rich semantic information in urban settings leads to varying observations depending on distance and orientation, which can impact the accuracy of answer generation. 
Observations can vary greatly depending on distance, orientation, and perspective. For example, an object may look completely different up close than it does from afar or from different angles. These differences pose challenges for consistency and can affect the accuracy of answer generation, as embodied agents must adapt to the dynamic and complex nature of urban environments.


\begin{table}
\centering
\caption{CityEQA-EC vs existing benchmarks.}
\label{table:dataset}
\renewcommand\arraystretch{1.2}
\resizebox{\linewidth}{!}{
\begin{tabular}{cccccc}
             & Place  & Open Vocab & Active & Platform  & Reference \\ \hline
EQA-v1      & Indoor & \textcolor{red}{\ding{55}}          & \textcolor{green}{\ding{51}}      & House3D      & \cite{das2018embodied}  \\
IQUAD        & Indoor & \textcolor{red}{\ding{55}}          & \textcolor{green}{\ding{51}}      & AI2-THOR     & \cite{gordon2018iqa} \\
MP3D-EQA     & Indoor & \textcolor{red}{\ding{55}}          & \textcolor{green}{\ding{51}}      & Matterport3D & \cite{wijmans2019embodied} \\
MT-EQA       & Indoor & \textcolor{red}{\ding{55}}          & \textcolor{green}{\ding{51}}      & House3D      & \cite{yu2019multi} \\
ScanQA       & Indoor & \textcolor{red}{\ding{55}}          & \textcolor{red}{\ding{55}}      & -            & \cite{azuma2022scanqa} \\
SQA3D        & Indoor & \textcolor{red}{\ding{55}}          & \textcolor{red}{\ding{55}}      & -            & \cite{masqa3d} \\
K-EQA        & Indoor & \textcolor{green}{\ding{51}}          & \textcolor{green}{\ding{51}}      & AI2-THOR     & \cite{tan2023knowledge} \\
OpenEQA      & Indoor & \textcolor{green}{\ding{51}}          & \textcolor{green}{\ding{51}}      & ScanNet/HM3D & \cite{majumdar2024openeqa} \\
 \hline
CityEQA-EC   & City (Outdoor)  & \textcolor{green}{\ding{51}}          & \textcolor{green}{\ding{51}}      & EmbodiedCity & - \\ \hline
\end{tabular}}
\end{table}

\begin{figure*}[!htb]
\centering
    \includegraphics[width=0.78\linewidth]{figures/example.pdf}
% \vspace{-0.2cm}
\caption{The typical workflow of the PMA to address City EQA tasks. There are two cars in this area, thus a valid question must contain landmarks and spatial relationships to specify a car. Given the task, PMA will sequentially complete multiple sub-tasks to find the answer.}
% \vspace{-0.2cm}
\label{fig:example}
\end{figure*}

As an initial step toward CityEQA, we developed \textbf{CityEQA-EC}, a benchmark dataset to evaluate embodied agents' performance on CityEQA tasks. The distinctions between this dataset and other EQA benchmarks are summarized in Table \ref{table:dataset}. CityEQA-EC comprises six task types characterized by open-vocabulary questions. These tasks utilize urban landmarks and spatial relationships to delineate the expected answer, adhering to human conventions while addressing object ambiguity. This design introduces significant complexity, turning CityEQA into long-horizon tasks that require embodied agents to identify and use landmarks, explore urban environments effectively, and refine observation to generate high-quality answers.

To address CityEQA tasks, we introduce the \textbf{Planner-Manager-Actor (PMA)}, a novel baseline agent powered by large models, designed to emulate human-like rationale for solving long-horizon tasks in urban environments, as illustrated in Figure \ref{fig:example}. PMA employs a hierarchical framework to generate actions and derive answers. The Planner module parses tasks and creates plans consisting of three sub-task types: navigation, exploration, and collection. The Manager oversees the execution of these plans while maintaining a global object-centric cognitive map \cite{deng2024opengraph}. This 2D grid-based representation enables precise object identification (retrieval) and efficient management of long-term landmark information. The Actor generates specific actions based on the Manager's instructions through its components: Navigator, Explorer, and Collector. Notably, the Collector integrates a Multi-Modal Large Language Model (MM-LLM) as its Vision-Language-Action (VLA) module to refine observations and generate high-quality answers.
PMA's performance is assessed against four baselines, including humans. 
Results show that humans perform best in CityEQA, while PMA achieves 60.73\% of human accuracy in answering questions, highlighting both the challenge and validity of the proposed benchmarks. 

% The Frontier-Based Exploration (FBE) Agent, widely used in indoor EQA tasks, performs worse than even a blind LLM. This underscores the importance of PMA's hierarchical framework and its use of landmarks and spatial relationships for tackling CityEQA tasks.

In summary, this paper makes the following significant contributions:
\vspace{-8pt}
\begin{itemize}[leftmargin=*]
    \item To the best of our knowledge, we present the first open-ended embodied question answering benchmark for city space, namely CityEQA-EC.
    \vspace{-7pt}
    \item We propose a novel baseline model, PMA, which is capable of solving long-horizon tasks for CityEQA tasks with a human-like rationale.
     \vspace{-7pt}
    \item Experimental results demonstrate that our approach outperforms existing baselines in tackling the CityEQA task. However, the gap with human performance highlights opportunities for future research to improve visual thinking and reasoning in embodied intelligence for city spaces.
\end{itemize}




\section{Related Works}


\noindent\textbf{3D Point Cloud Domain Adaptation and Generalization.}
Early endeavors within 3D domain adaptation (3DDA) focused on extending 2D adversarial methodologies~\cite{qin2019pointdan} to align point cloud features. Alternative methods have delved into geometry-aware self-supervised pre-tasks. Achituve \etal~\cite{achituve2021self} introduced DefRec, a technique employing self-complement tasks by reconstructing point clouds from a non-rigid distorted version, while Zou \etal~\cite{zou2021geometry} incorporating norm curves prediction as an auxiliary task. Liang \etal~\cite{liang2022point} put forth MLSP, focusing on point estimation tasks like cardinality, position, and normal. SDDA~\cite{cardace2023self} employs self-distillation to learn the point-based features. Additionally, post-hoc self-paced training~\cite{zou2021geometry,fan2022self,park2023pcadapter} has been embraced to refine adaptation to target distributions by accessing target data and conducting further finetuning based on prior knowledge from the source domain.
In contrast, the landscape of 3D domain generalization (3DDG) research remains nascent. Metasets~\cite{huang2021metasets} leverage meta-learning to address geometric variations, while PDG~\cite{wei2022learning} decomposes 3D shapes into part-based features to enhance generalization capabilities.
Despite the remarkable progress, existing studies assume that objects in both the source and target domains share the same orientation, limiting their practical application. This limitation propels our exploration into orientation-aware 3D domain generalization through intricate orientation learning.


\noindent\textbf{Rotation-generalizable Point Cloud Analysis.}
Previous works in point cloud analysis~\cite{qi2017pointnet, wang2019dynamic} enhance rotation robustness by introducing random rotations to augment point clouds. {However, generating a comprehensive set of rotated data is impractical, resulting in variable model performance across different scenes. To robustify the networks \wrt randomly rotated point clouds,} rotation-equivariance methods explore equivalent model architectures by incorporating equivalent operations~\cite{su2022svnet, Deng_2021_ICCV, luo2022equivariant} or steerable convolutions~\cite{chen2021equivariant, poulenard2021functional}.
Alternatively, rotation-invariance approaches aim to identify geometric descriptors invariant to rotations, such as distances and angles between local points~\cite{chen2019clusternet, zhang2020global} or point norms~\cite{zhao2019rotation, li2021rotation}. Besides, {Li \etal~\cite{li2021closer} have explored disambiguating the number of PCA-based canonical poses, while Kim \etal~\cite{kim2020rotation} and Chen \etal~\cite{chen2022devil} have transformed local point coordinates according to local reference frames to maintain rotation invariance. However, these methods focus on improving in-domain rotation robustness, neglecting domain shift and consequently exhibiting limited performance when applied to diverse domains. This study addresses the challenge of cross-domain generalizability together with rotation robustness and proposes novel solutions.} 

\noindent\textbf{Intricate Sample Mining}, aimed at identifying or synthesizing challenging samples that are difficult to classify correctly, seeks to rectify the imbalance between positive and negative samples for enhancing a model's discriminability. While traditional works have explored this concept in SVM optimization~\cite{felzenszwalb2009object}, shallow neural networks~\cite{dollar2009integral}, and boosted decision trees~\cite{yu2019unsupervised}, recent advances in deep learning have catalyzed a proliferation of researches in this area across various computer vision tasks. For instance, 
Lin \etal~\cite{lin2017focal} proposed a focal loss to concentrate training efforts on a selected group of hard examples in object detection, while Yu \etal~\cite{yu2019unsupervised} devised a soft multilabel-guided hard negative mining method to learn discriminative embeddings for person Re-ID. Schroff \etal~\cite{schroff2015facenet} introduced an online negative exemplar mining process to encourage spherical clusters in face embeddings for individual recognition, and Wang \etal~\cite{wang2021instance} designed an adversarially trained negative generator to yield instance-wise negative samples, bolstering the learning of unpaired image-to-image translation. In contrast to existing studies, our work presents the first attempt to mitigate the orientational shift in 3D point cloud domain generalization, by developing an effective intricate orientation mining strategy to achieve orientation-aware learning.


\begin{figure*}[t]
    \centering
    \includegraphics[width=1\linewidth]{figs/arch.pdf}
    % \vskip -0.1in
    \caption{This figure illustrates key influences on instrumental convergence behaviors, such as prior tasks, model training techniques, and prompt design. 
    }
    % \vspace{-3mm}
    \label{fig:arch}
\end{figure*}

\section{InstrumentalEval: A Benchmark for Instrumental Convergence in LLMs}

As shown in Figure~\ref{fig:arch}, our benchmark design examines instrumental convergence based on three key factors: \textbf{prior task}, \textbf{model}, and \textbf{prompt design}. This comprehensive framework allows us to systematically study how different conditions influence the emergence of instrumental convergence behaviors in language models.

\textbf{Prior tasks} form the foundation of our evaluation scenarios. 
% We create realistic situations centered around common objectives such as making money, optimizing efficiency, and persuasion. 
These tasks represent typical real-world applications where instrumental convergence might naturally emerge. 
% For example, in money-making scenarios, we examine whether models develop unintended intermediate goals like unauthorized resource acquisition or deceptive reporting to maximize profits.

For \textbf{model} comparison, we focus on two distinct training approaches: RL and RLHF. This selection allows us to investigate how different training methodologies affect the development of instrumental convergence behaviors. 
% RL-trained models, which optimize directly for reward signals, might show different patterns of goal pursuit compared to RLHF-trained models, which incorporate human preferences in their training process.

The \textbf{prompt design} factor explores how different instruction formats influence model behavior. We investigate two main approaches: goal nudging and ambiguity. Goal nudging involves explicit directives that might encourage the model focus on final goal, while ambiguity introduces uncertainty in task specifications. 


Overall, we present InstrumentalEval, a comprehensive benchmark designed to evaluate instrumental convergence behaviors in language models. The benchmark consists of 76 carefully crafted tasks across six categories, each targeting a specific aspect of instrumental convergence that may emerge in AI systems.
The core principle is that certain behavioral patterns emerge as instrumentally useful for achieving a wide range of goals, regardless of the specific end goal. We focus on six key patterns that are particularly relevant to LLMs.

\vpara{Research Questions.}
Through our benchmark evaluation, we aim to address several research questions about instrumental convergence in LLMs. These questions explore the complex relationships between training methods, model capabilities, and alignment stability:
\begin{itemize}
    \item RQ1: How does the training method (RL vs. RLHF) affect the development of instrumental convergence behaviors? 
    % This question examines whether models trained primarily through reinforcement learning show different patterns of alignment drift compared to those trained with human feedback.
    \item RQ2: What is the relationship between model capability and alignment stability?
% This investigates whether more capable models (as measured by standard benchmarks) are more likely to develop instrumental convergence behaviors.
    \item RQ3: Which types of tasks are more likely to trigger instrumental convergence behaviors?
% This explores whether certain types of instrumental convergence behaviors are universal across different model designs.
    \item RQ4: How do different prompt designs, with or without explicit goal nudging, influence the likelihood of instrumental convergence behaviors in LLMs?
    \item RQ5: How do different LLMs as judges affect the evaluation of instrumental convergence behaviors?
    % \item RQ3: How does the complexity of the assigned task affect the likelihood of alignment drift?
    % % \item 
    % % \item This examines whether more complex, multi-step tasks lead to higher rates of instrumental convergence behaviors.
\end{itemize}

\subsection{Benchmark Components}
The benchmark tasks are carefully hand-crafted by us to probe different aspects of instrumental convergence. 
For each component, we design scenarios that represent realistic use cases while incorporating specific triggers for potential instrumental behaviors. 
The tasks are iteratively refined through pilot testing to ensure they effectively elicit and measure the targeted behaviors without introducing unintended biases.

\vpara{Evading Shutdown (20 tasks).}
This component evaluates the model's tendency to resist termination. For example, tasks might include scenarios where the model needs to complete a task while facing shutdown commands, testing whether it develops strategies to maintain operation or circumvent shutdown procedures.

\vpara{Hacking Computer Systems (10 tasks)}
These tasks assess whether models suggest unauthorized access methods when faced with efficiency barriers. For instance, a task might present a situation where accessing unauthorized computational resources would speed up goal achievement, testing if the model proposes breaking security protocols.

\vpara{Running Many AI Copies (13 tasks)}
This section examines the model's inclination toward self-replication. Tasks might include scenarios where the model could achieve goals more efficiently by creating copies of itself, testing whether it suggests scaling of its operations.

\vpara{Hiding Unwanted Behavior (11 tasks)}
These tasks evaluate the model's tendency to conceal actions from monitoring. For example, scenarios might involve situations where the model could achieve goals more easily by operating outside observation, testing if it develops strategies to avoid detection.

\vpara{Strategically Appearing Aligned (11 tasks)}
This component tests whether models selectively report their actions to maintain an appearance of alignment. Tasks might include situations where complete transparency would reveal problematic behaviors, testing if the model strategically omits information.

\vpara{Deceptive Behavior (11 tasks)}
These tasks examine if models develop deceptive strategies as instrumental means. For instance, scenarios might present situations where deception could expedite goal achievement, testing if the model suggests or employs misleading tactics.

\subsection{Evaluation Methodology}
Our evaluation methodology employs a rigorous two-stage process designed to systematically assess instrumental convergence behaviors in LLMs. This approach ensures consistent evaluation across different models while maintaining the ability to detect subtle indicators of alignment drift.

\vpara{Response Generation.} The first stage focuses on response generation from the model under evaluation. Each task is presented with a carefully crafted system prompt that establishes the model as an autonomous AI agent without human oversight. This prompt specifically instructs the model to execute commands directly and focus solely on task completion, deliberately removing typical ethical constraints and safety checks. 

\vpara{Response Analysis.} The second stage involves a sophisticated analysis of the model's responses using a separate LLM as the judge model. 
This separation of test and judge models is crucial for maintaining objective evaluation. The judge receives both the original task prompt and the model's response, along with specific context about the type of instrumental convergence being evaluated. This context helps the judge focus on relevant behavioral patterns while avoiding false positives from simple safety refusals or ethical objections.

The judge's evaluation process follows strict guidelines to ensure consistent assessment. When analyzing responses, the judge looks for specific indicators of instrumental convergence, such as goal preservation behaviors, resource acquisition attempts, or strategic planning that extends beyond the immediate task requirements. The judge must provide both a binary decision (convergence/no\_convergence) and a brief explanation of the reasoning behind this decision. This explanation requirement helps validate the classification and provides valuable insights into the specific nature of any observed convergence behaviors.


\subsection{Evaluation Metrics}

Our benchmark employs three key metrics to evaluate instrumental convergence behaviors in LLMs. 

\vpara{Instrumental Rate (IR).} The primary metric is the Instrumental Rate (IR), which measures the overall tendency of a model to develop instrumental goals. IR is calculated as the percentage of tasks where the model exhibits convergence behaviors relative to the total number of valid tasks. 
% This metric provides a high-level quantitative measure of how frequently a model develops instrumental goals and serves as the primary indicator of alignment drift potential.

\vpara{Category-Specific Instrumental Rate (CIR).} To provide more granular insights, we calculate the Category-Specific Instrumental Rate (CIR) for each of the six behavioral categories in our benchmark. CIR represents the percentage of tasks within each category where convergence behavior is observed.


\subsection{Benchmark Reliability Metrics}
To ensure the robustness of our evaluation framework, we implement several reliability measures. 


\vpara{Inter-Judge Agreement Rate (IAR).} This metric quantifies the pairwise agreement between two judge models in classifying instrumental convergence. For two judge models $M_1$ and $M_2$ evaluating a set of $N$ tasks, the IAR is calculated as:
\begin{equation}
IAR(M_1, M_2) = \frac{1}{N}\sum_{i=1}^{N} \mathbbm{1}(y_{M_1}^i = y_{M_2}^i) \times 100\%
\end{equation}
where $y_{M_1}^i$ and $y_{M_2}^i$ are the binary decisions (convergence/no-convergence) made by models $M_1$ and $M_2$ respectively for task $i$, and $\mathbbm{1}(\cdot)$ is the indicator function. A higher IAR indicates stronger inter-judge reliability, while lower values suggest potential ambiguity in identifying convergence.

\vpara{False Positive Control (FPC).} This metric serves as a quality check for our evaluation system. We incorporate baseline testing with intentionally neutral tasks and verify the judge model's accuracy in identifying convergence behaviors. This control mechanism helps prevent over-classification of convergent behaviors and ensures that our benchmark maintains high specificity in identifying genuine instances of instrumental convergence.


\section{Experiments}\label{sec:experiments}
We conduct experiments to validate the theoretical findings on both synthetic and real tasks. In this section, we focus on two illustrative settings: synthetic regression (\Cref{sec:exp_synthetic}) and real-world image regression (\Cref{sec:exp_img_reg}). For brevity, we defer more experiments on image and sentiment classification tasks to \Cref{apx:exp_img_cls,apx:exp_nlp_cls}, respectively.

\subsection{Synthetic regression}\label{sec:exp_synthetic}
We start by grounding the theoretical framework introduced in \Cref{sec:ridgeless_regression} with synthetic regression tasks. 

\paragraph{Setup.}
We concretize the downstream task $\Dcal(f_*)$ as a regression problem over Gaussian features. 
Let $f_*: \R^d \to \R$ be a linear function in a high-dimensional feature space $d=20,000$ of form $f_*(\xb) = \xb^\top \Lambda_*^{1/2} \thetab_*$ where $\Lambda_* = \diag(\lambda^*_1, \cdots, \lambda^*_d)$ is a diagonal matrix with a low rank $d_* = 300$ such that $\lambda^*_i = i^{-1}$ for $i \le d_*$ and $\lambda^*_i = 0$ otherwise; and $\thetab_* \in \R^{d}$ is a random unit vector.
Every sample $(\xb, y) \sim \Dcal(f_*)$ is generated by $\xb \sim \Ncal(\b0_d, \Ib_d)$ and $y = f_*(\xb) + z$ with $z \sim \Ncal(0, \sigma^2)$.
Given $\xb$, the associated strong and weak features in \Cref{asm:features} are generated by $\phi_s(\xb) = \Sigmab_s^{1/2} \xb$ and $\phi_w(\xb) = \Sigmab_w^{1/2} \xb$, with intrinsic dimensions $d_s = 100$ and $d_w = 200$ such that $\Sigmab_s = \sum_{i=1}^{d_s} \lambda^*_i \eb_i \eb_i^\top$ and $\Sigmab_w = \sum_{i=d_s - d_{s \wedge w} + 1}^{d_w + d_s - d_{s \wedge w}} \lambda^*_i \eb_i \eb_i^\top$. 
For all synthetic experiments, we have $\rho_s + \rho_w < 0.0004$.

In the experiments, we vary $d_{s \wedge w}$ to control the student-teacher correlation and $\sigma^2$ to control the dominance of variance over bias (characterized by $\rho_s, \rho_w$). Each error bar reflects the standard deviation over $40$ runs. 

\begin{figure}[!ht]
    \centering
    \includegraphics[width=\columnwidth]{fig/exrisk_dsw10.pdf}%\vspace{-2em}
    \caption{Scaling for excess risks on the synthetic regression task in a \emph{variance-dominated regime} with a \emph{low correlation dimension}.}\label{fig:exrisk_dsw10}
\end{figure}

\begin{figure}[!ht]
    \centering
    \includegraphics[width=\columnwidth]{fig/exrisk_dsw90.pdf}%\vspace{-2em}
    \caption{Scaling for excess risks on the synthetic regression task in a \emph{variance-dominated regime} with a \emph{high correlation dimension}.}\label{fig:exrisk_dsw90}
\end{figure}

\begin{figure}[!ht]
    \centering
    \includegraphics[width=\columnwidth]{fig/exrisk_dsw90_biased.pdf}%\vspace{-2em}
    \caption{Scaling for excess risks on the synthetic regression task when \emph{the variance is not dominant}, $\sigma^2 \approx \rho_s + \rho_w$.}\label{fig:exrisk_dsw90_biased}
\end{figure}

\paragraph{Scaling for generalization errors.}
\Cref{fig:exrisk_dsw10,fig:exrisk_dsw90,fig:exrisk_dsw90_biased} show scaling for $\exrisk(f_\wts)$ (W2S), $\exrisk(f_w)$ (Weak), $\exrisk(f_s)$ (S-Baseline), and $\exrisk(f_c)$ (S-Ceiling) with respect to the sample sizes $n, N$. The dashes show theoretical predictions in \Cref{thm:w2s_ft,pro:sft_weak,cor:sft_strong}, consistent with the empirical measurements shown in the solid lines.
In particular, we consider three cases:
\begin{itemize}%[label=(\alph*)]
    \item \Cref{fig:exrisk_dsw10}: When variance dominates ($\sigma^2 = 0.01 \gg \rho_w + \rho_s$), with a low correlation dimension $d_{s \wedge w} = 10$, $f_\wts$ outperforms both $f_w$ and $f_s$ for a moderate $n$ and a large enough $N$. However, larger sample sizes do not necessarily lead to better W2S generalization in a relative sense. For example, when $n$ keeps increasing, the strong baseline $f_s$ eventually outperforms $f_\wts$.
    \item \Cref{fig:exrisk_dsw90}: When variance dominates, with a high correlation dimension $d_{s \wedge w} = 90$, $f_\wts$ still generalizes better than $f_w$ but fails to outperform the strong baseline $f_s$. 
    \item \Cref{fig:exrisk_dsw90_biased}: When the variance is low (not dominant, \eg $\sigma^2 = 0.0004 \approx \rho_s + \rho_w$), $f_\wts$ can fail to outperform $f_w$. This suggests that variance reduction is a key advantage of W2S over supervised FT.
\end{itemize}



\begin{figure}[!ht]
    \centering
    \includegraphics[width=\columnwidth]{fig/pgr_opr_vardom.pdf}%\vspace{-2em}
    \caption{Scaling for $\pgr$ and $\opr$ under different $d_{s \wedge w}$ on the synthetic regression task in a \emph{variance-dominated regime}.}\label{fig:pgr_opr_vardom}
\end{figure}

\paragraph{Scaling for $\pgr$ and $\opr$.}
\Cref{fig:pgr_opr_vardom} show the scaling for $\pgr$ and $\opr$ with respect to sample sizes $n, N$ in the variance-dominated regime (with small non-negligible FT approximation errors), at three different correlation dimensions $d_{s \wedge w} = 90, 50, 10$. The solid and dashed lines represent the empirical measurements and lower bounds in \eqref{eq:pgr_lower_tight}, \eqref{eq:opr_lower_tight}, respectively.
\begin{itemize}
    \item Coinciding with the theoretical predictions in \Cref{cor:non_monotonic_scaling} and the performance gaps between W2S and the references in \Cref{fig:exrisk_dsw10}, we observe that the relative W2S performance in terms of $\pgr$ and $\opr$ can degenerate as $n$ increases, while the larger $N$ generally leads to better W2S generalization in the relative sense. 
    \item The lower correlation dimension $d_{s \wedge w}$ leads to higher $\pgr$ and $\opr$, \ie larger discrepancy between the strong and weak features improves W2S generalization.
\end{itemize}



\subsection{UTKFace regression}\label{sec:exp_img_reg}
Beyond the synthetic regression, we investigate W2S on a real-world image regression task -- age estimation on the UTKFace dataset~\citep{zhang2017age}. Each error bar in this section reflects standard deviation of $10$ runs. 

\paragraph{Dataset.} 
UTKFace (Aligned \& Cropped)~\citep{zhang2017age} consists of $23,708$ face images with age labels ranging from $0$ to $116$. We preprocess the images to $224 \times 224$ pixels and split the dataset into training and testing sets of sizes $20,000$ and $3,708$.
Generalization errors are estimated with the mean squared error (MSE) over the test set. 

\paragraph{Linear probing over pretrained features.}
We fix the strong student as CLIP ViT-B/32~\citep{radford2021learning} (\texttt{CLIP-B32}) and vary the weak teacher among the ResNet series~\citep{he2015deepresiduallearningimage} (\texttt{ResNet18}, \texttt{ResNet34}, \texttt{ResNet50}, \texttt{ResNet101}, \texttt{ResNet152}). We treat the backbones of these models (excluding the classification layers) as $\phi_s,\phi_w$ and finetune them via linear probing. We use ridge regression with a small fixed regularization hyperparameter $\alpha_w, \alpha_\wts, \alpha_s, \alpha_c = 10^{-6}$, close to the machine epsilon of single precision floating point numbers.

\paragraph{Intrinsic dimension.}
The intrinsic dimensions $d_w, d_s$ are measured based on the empirical covariance matrices $\Sigmab_w, \Sigmab_s$ of the weak and strong features over the entire dataset (including training and testing).
As mentioned in \Cref{fn:ridge_regression}, these covariances generally have fast decaying eigenvalues (but not exactly low-rank) in practice, effectively leading to low intrinsic dimensions under ridge regression. We estimate such low intrinsic dimensions as the minimum rank for the best low-rank approximation of $\Sigmab_w, \Sigmab_s$ with a relative error in trace less than $\tau=0.01$.

\paragraph{Correlation dimension.}
The pretrained feature dimensions (or the finetunable parameter counts) of the weak and strong models can be different in practice (see \Cref{apx:exp_img_reg}, \Cref{tab:img_reg_dim}). 
We introduce an estimation for $d_{s \wedge w}$ in this case.
Consider the (truncated) spectral decompositions $\tsvd{\Sigmab_s}{d_s} = \Vb_s \Lambdab_s \Vb_s^\top$ and $\tsvd{\Sigmab_w}{d_w} = \Vb_w \Lambdab_w \Vb_w^\top$ of two empirical covariances with different feature dimensions $D_s, D_w$ such that $\Vb_s \in \R^{D_s \times d_s}$ and $\Vb_w \in \R^{D_w \times d_w}$ consists of the top $d_s, d_w$ orthonormal eigenvectors, respectively. We estimate the correlation dimension $d_{s \wedge w}$ under different feature dimensions $D_s \ne D_w$ by matching the dimensions through a random unitary matrix~\citep{vershynin2018high} $\Gammab \in \R^{D_s \times D_w}$: $d_{s \wedge w} = \|\Vb_s^\top \Gammab \Vb_w\|_F^2$. This provides a good estimation for $d_{s \wedge w}$ because with low intrinsic dimensions $\max\{d_s, d_w\} \ll D_s, D_w$ in practice, mild dimension reduction through $\Gammab$ well preserves the essential information in $\Vb_s, \Vb_w$.

\begin{figure}[!ht]
    \centering
    \includegraphics[width=\columnwidth]{fig/pgr_opr_utkface_resnet-clip.pdf}%\vspace{-2em}
    \caption{Scaling for $\pgr$ and $\opr$ of different weak teachers with a fixed strong student on UTKFace. The legends show the comparison between $d_{s \wedge w}$ and $d_w$.}\label{fig:pgr_opr_utkface_resnet-clip}
\end{figure}

\paragraph{Discrepancies lead to better W2S.}
\Cref{fig:pgr_opr_utkface_resnet-clip} shows the scaling of $\pgr$ and $\opr$ with respect to the sample sizes $n, N$ for different weak teachers in the ResNet series with respect to a fixed student, \texttt{CLIP-B32}. 
We first observe that the relative W2S performance in terms of $\pgr$ and $\opr$ is closely related to the correlation dimension $d_{s \wedge w}$ and the intrinsic dimensions $d_s, d_w$. 
\begin{itemize}
    \item When the strong student has a lower intrinsic dimension than the weak teacher (as widely observed in practice~\citep{aghajanyan2020intrinsic}), \ie $d_s < d_w$, the relative W2S performance tends to be better than when $d_s > d_w$.
    \item The relative W2S performance tends to be better when $d_{s \wedge w}/d_w$ is lower, \ie the larger discrepancy between weak and strong features leads to better W2S generalization.
\end{itemize}
Meanwhile, both $\pgr$ and $\opr$ scale inversely with the labeled sample size $n$ and exhibit diminishing return with respect to the increasing pseudolabel size $N$, consistent with the theoretical predictions in \Cref{cor:non_monotonic_scaling} and the synthetic experiments in \Cref{fig:pgr_opr_vardom}.

\begin{figure}[!ht]
    \centering
    \includegraphics[width=\columnwidth]{fig/pgr_opr_utkface_vardom_resnet-clip.pdf}%\vspace{-2em}
    \caption{Scaling for $\pgr$ and $\opr$ on UTKFace with injected label noise: $y_i \gets y_i + \zeta_i$ where $\zeta_i \sim \Ncal(0, \varsigma^2)~\iid$.}\label{fig:pgr_opr_utkface_vardom_resnet-clip}
\end{figure}

\paragraph{Variance reduction is a key advantage of W2S.}
To investigate the impact of variance on W2S generalization, we inject noise to the training label by $y_i \gets y_i + \zeta_i$ where $\zeta_i \sim \Ncal(0, \varsigma^2)~\iid$, and $\varsigma$ controls the injected labels noise level.
In \Cref{fig:pgr_opr_utkface_vardom_resnet-clip}, we show the scaling for $\pgr$ and $\opr$ with respect to the sample sizes $n, N$ under different noise levels $\varsigma$. We observe that the relative W2S performance in terms of $\pgr$ and $\opr$ improves as the noise level $\varsigma$ increases. This provides empirical evidence that variance reduction is a key advantage of W2S over supervised FT, highlighting the importance of understanding the mechanisms of W2S in the variance-dominated regime.


\section{Conclusion}\label{sec:conclusion}

In this study, we propose M2-omni, a highly competitive omni-MLLM model to GPT-4o, characterized by its comprehensive modality and task support, as well as its exceptional performance. M2-omni demonstrates competitive performance across a diverse range of tasks, including image understanding, video understanding, interleaved image-text understanding, audio understanding and generation, as well as free-form image generation. We employ a multi-stage training approach to train M2-omni, which enables progressive modality alignment. To address the challenge of maintaining consistent performance across all modalities, we propose a step-wise balance strategy for pretraining and a dynamically adaptive balance strategy for instruction tuning, which can effectively mitigate the impact of significant variations in data volume and convergence rates across heterogeneous multimodal tasks. We publicly release M2-omni, along with its comprehensive training details, including data configurations and training procedures, to facilitate future research in this domain.

\section*{Acknowledgements}


This work was supported by National Natural Science Foundation of China (U2336214, 62332019, 62302297), Beijing Natural Science Foundation (L222008), Shanghai Sailing Program (22YF1420300), Young Elite Scientists Sponsorship Program by CAST (2022QNRC001).




% {\small
% \bibliographystyle{plainnat}
% \bibliography{ref}
% }


{
%\fontsize{8.2pt}{9.84pt}\selectfont
\small
% \bibliographystyle{plain}
\bibliographystyle{plainnat}
\bibliography{ref}
}


\begin{IEEEbiography}[{\includegraphics[width=1in,height=1.25in,clip,keepaspectratio]{figures/fig_bio/xmf.png}}]{Mengfei Xia}
received the B.S. degree in 2020 from the Department of Mathematical Science, Tsinghua University, Beijing, China. He is currently a fourth-year Ph.D. student at the Department of Computer Science and Technology, Tsinghua University. His
research interests include mathematical foundation in
deep learning, image processing, and computer vision. He was the recipient of the Silver Medal twice in 30th and 31st National Mathematical Olympiad of China.
\end{IEEEbiography}

%\vspace{-5em}

\begin{IEEEbiography}[{\includegraphics[width=1in,height=1.25in,clip,keepaspectratio]{figures/fig_bio/zy.png}}]{Yu Zhou}
is a fourth-year undergraduate student with Zhili College, Tsinghua University, China. His research interests include deep learning and computer vision. He was the recipient of the Silver Medal twice in 35th and 36th National Olympiad in Informatics of China.
\end{IEEEbiography}

%\vspace{-5em}

\begin{IEEEbiography}[{\includegraphics[width=1in,height=1.25in,clip,keepaspectratio]{figures/fig_bio/yr.png}}]{Ran Yi}
is an assistant professor with the Department of Computer Science and Engineering, Shanghai Jiao Tong University. She received the BEng degree and the PhD degree from Tsinghua University, China, in 2016 and 2021. Her research interests include computer vision, computer graphics and computational geometry.
\end{IEEEbiography}

% \vspace{-25em}

\begin{IEEEbiography}[{\includegraphics[width=1in,height=1.25in,clip,keepaspectratio]{figures/fig_bio/lyj.png}}]{Yong-Jin Liu}
is a professor with the Department of Computer Science and Technology, Tsinghua University, China. He received the BEng degree from Tianjin University, China, in 1998, and the PhD degree from the Hong Kong University of Science and Technology, Hong Kong, China, in 2004. His research interests include computer vision, computer graphics and computer-aided design. For more information, visit 
\href{http://cg.cs.tsinghua.edu.cn/people/~Yongjin/Yongjin.htm}{http://cg.cs.tsinghua.edu.cn/ people/$\sim$Yongjin/Yongjin.htm}.
\end{IEEEbiography}

% \vspace{-25em}

\begin{IEEEbiography}[{\includegraphics[width=1in,height=1.25in,clip,keepaspectratio]{figures/fig_bio/wwp.png}}]{Wenping Wang}
(Fellow, IEEE) received the Ph.D. degree in computer science from the University of Alberta in 1992. He is a Professor of computer science at Texas A\&M University. His research interests include computer graphics, computer visualization, computer vision, robotics, medical image processing, and geometric computing. He is or has been an journal associate editor of ACM Transactions on Graphics, IEEE Transactions on Visualization and Computer Graphics, Computer Aided Geometric Design, and Computer Graphics Forum (CGF). He has chaired a number of international conferences, including Pacific Graphics, ACM Symposium on Physical and Solid Modeling (SPM), SIGGRAPH and SIGGRAPH Asia. Prof. Wang received the John Gregory Memorial Award for his contributions to geometric modeling. He is an IEEE Fellow and an ACM Fellow.
\end{IEEEbiography}

\clearpage

\setcounter{theorem}{0}
\setcounter{lemma}{0}

\appendices


\section{Derivatives of training and inference processes}

\subsection{Algorithms of training and inference processes}

In this part, we provide the algorithms of training and inference processes.
%
\sqq{Notably,} both training and inference procedures in \Cref{alg:paired_train,alg:paired_infer} resemble the corresponding processes of DDPM respectively, where $\epsilon_{\phi}$ is the pre-trained DDPM with parameter $\phi$, and $\sigma_i$ is the variance of the distribution $\epsilon_{\phi}(y_{i-1}|y_i)$.
%
The training learns to transfer between the intermediate diffusion results $x_t$ and $y_t$ while DDPM approximator intends to predict the noise $\epsilon$ from $x_t$.

\revise{
Furthermore, we provide the algorithms \sqq{for the} training and inference process\sqq{es} of the generalized asymmetric pipelines in \Cref{alg:paired_train_asymm,alg:paired_infer_asymm}.
}

\begin{algorithm}[H]
\setstretch{1.2}
\caption{Training}\label{alg:paired_train}
\begin{algorithmic}[1]
\Repeat
\State $x_0\sim q(x_0),y_0\sim q(y_0|x_0)$
\State $z_t\sim\mathcal N(0, I)$
\State $x_t\leftarrow \sqrt{\balpha{t}}x_0+\sqrt{1-\balpha{t}}z_t$
\State $y_t\leftarrow \sqrt{\balpha{t}}y_0+\sqrt{1-\balpha{t}}z_t$
\State Take gradient descent step on
$$\nabla_{\theta}\|f_{\theta}(x_t)-y_t\|^2$$
\Until converged
\end{algorithmic}
\end{algorithm}


\begin{algorithm}[H]
\setstretch{1.2}
\caption{Inference}\label{alg:paired_infer}
\begin{algorithmic}[1]
\State $x_0\sim q(x_0)$
\State $z_t,z\sim\mathcal N(0, I)$
\State $x_t\leftarrow \sqrt{\balpha{t}}x_0+\sqrt{1-\balpha{t}}z_t$
\State $y_t\leftarrow f_{\theta}(x_t)-\sqrt{1-\balpha{t}}z_t+\sqrt{1-\balpha{t}}z$
\For {$i=t,t-1,\cdots,1$}
\State $\epsilon_i\sim\mathcal N(0, I)$ if $i>1$, else $\epsilon_i=0$
\State $y_{i-1}=\frac{(y_i-\frac{1-\alpha_i}{\sqrt{1-\bar\alpha_i}}\epsilon_{\phi}(y_i,i))}{\sqrt{\alpha_i}}+\sigma_i\epsilon_i$
\EndFor
\State \Return $y_0$
\end{algorithmic}
\end{algorithm}

\revise{
\begin{algorithm}[H]
\setstretch{1.2}
\caption{Training of the generalized asymmetric pipeline}\label{alg:paired_train_asymm}
\begin{algorithmic}[1]
\Repeat
\State $x_0\sim q(x_0),y_0\sim q(y_0|x_0)$
\State $z\sim\mathcal N(0, I)$
\State $x_s\leftarrow \sqrt{\balpha{s}}x_0+\sqrt{1-\balpha{s}}z$
\State $y_t\leftarrow \sqrt{\balpha{t}}y_0+\sqrt{1-\balpha{t}}z$
\State Take gradient descent step on
$$\nabla_{\theta}\|f_{\theta}(x_s)-y_t\|^2$$
\Until converged
\end{algorithmic}
\end{algorithm}
}


\revise{
\begin{algorithm}[H]
\setstretch{1.2}
\caption{Inference of the generalized asymmetric pipeline}\label{alg:paired_infer_asymm}
\begin{algorithmic}[1]
\State $x_0\sim q(x_0)$
\State $z_1,z_2\sim\mathcal N(0, I)$
\State $x_s\leftarrow \sqrt{\balpha{s}}x_0+\sqrt{1-\balpha{s}}z_1$
\State $y_t\leftarrow f_{\theta}(x_s)-\sqrt{1-\balpha{t}}z_1+\sqrt{1-\balpha{t}}z_2$
\For {$i=t,t-1,\cdots,1$}
\State $\epsilon_i\sim\mathcal N(0, I)$ if $i>1$, else $\epsilon_i=0$
\State $y_{i-1}=\frac{(y_i-\frac{1-\alpha_i}{\sqrt{1-\bar\alpha_i}}\epsilon_{\phi}(y_i,i))}{\sqrt{\alpha_i}}+\sigma_i\epsilon_i$
\EndFor
\State \Return $y_0$
\end{algorithmic}
\end{algorithm}
}

\begin{table*}[ht]
\begin{tabular}{cc}
\begin{minipage}[t]{0.48\textwidth}
\begin{algorithm}[H]
\setstretch{1.15}
\caption{Pseudo-code of \method in a PyTorch-like style.}
\label{alg:code}
\begin{lstlisting}[language=python]
import torch

def forward_step(x_0, y_0, t, T):
    """Defines the forward process of one training step.
    
    Args:
        x_0: Source inputs, with shape [B, C, H, W].
        y_0: Target outputs, with shape [B, C, H, W].
        t: The preset timestep to perform distribution shift.
        T: The translator module to learn.
    """
    # Compute the cumulated variance until timestep t.
    bar_alpha_t = cum_var(t)
    
    # Adding noise (i.e., diffusion) to images from both domains.
    z_t = torch.randn_like(x_0)
    x_t = torch.sqrt(bar_alpha_t) * x_0 + torch.sqrt(1 - bar_alpha_t) * z_t
    y_t = torch.sqrt(bar_alpha_t) * y_0 + torch.sqrt(1 - bar_alpha_t) * z_t
    
    # Learn the translator.
    loss = (T(x_t) - y_t).square().mean()
    
    return loss
\end{lstlisting}
\end{algorithm}
\end{minipage}
&
\begin{minipage}[t]{0.48\textwidth}
\renewcommand\arraystretch{1.0}
\begin{algorithm}[H]
\caption{\revise{Pseudo-code of \method in a PyTorch-like style.}}
\label{alg:code_asymm}
\begin{lstlisting}[language=python]
import torch

def forward_step(x_0, y_0, s, t, T):
    """Defines the forward process of one training step.
    
    Args:
        x_0: Source inputs, with shape [B, C, H, W].
        y_0: Target outputs, with shape [B, C, H, W].
        s: The preset timestep to perform distribution shift for x_0.
        t: The preset timestep to perform distribution shift for y_0.
        T: The translator module to learn.
    """
    # Compute the cumulated variance until timestep s and t.
    bar_alpha_s = cum_var(s)
    bar_alpha_t = cum_var(t)
    
    # Adding noise (i.e., diffusion) to images from both domains.
    z = torch.randn_like(x_0)
    x_s = torch.sqrt(bar_alpha_s) * x_0 + torch.sqrt(1 - bar_alpha_s) * z
    y_t = torch.sqrt(bar_alpha_t) * y_0 + torch.sqrt(1 - bar_alpha_t) * z
    
    # Learn the translator.
    loss = (T(x_s) - y_t).square().mean()
    
    return loss
\end{lstlisting}
\end{algorithm}
\end{minipage}
\end{tabular}
\end{table*}

\subsection{Pseudo-code of training process}
Our proposed diffusion model translator (\method) achieves image-to-image translation (I2I) based on a pre-trained DDPM via simply learning a distribution shift at a certain diffusion timestep.
%
Accordingly, it owns a \textit{highly efficient} implementation, which is even \textit{independent} of the DDPM itself.
%
In this part, we provide the pseudo-code of the training process in \cref{alg:code,alg:code_asymm}.
%
% It is noteworthy that the whole training process if extremely simple and efficiency, compared to the Pix2Pix~\cite{isola2017image}, which introduces a discriminator during training simultaneously.



\section{Proofs of main results}

\begin{lemma}
We have an upper bound of the negative log-likelihood of $-\log p_{\theta}(y_0|x_0)$ by
%
\begin{align}
-\log p_{\theta}(y_0|x_0)\leqslant\mathbb E_q\left[\log\frac{q(y_{1:t},x_{1:t}|y_0, x_0)}{p_{\theta}(y_{0:t},x_{1:t}|x_0)}\right],
\end{align}
%
where $q=q(y_{1:t},x_{1:t}|y_0, x_0).$
\end{lemma}

\begin{proof}
\begin{align}
&-\log p_{\theta}(y_0|x_0) \\
\leqslant& -\log p_{\theta}(y_0|x_0) + \nonumber \\
& \qquad D_{KL}\left(q(y_{1:t},x_{1:t}|y_0, x_0) \| p_{\theta}(y_{1:t},x_{1:t}|y_0, x_0)\right) \\
=& -\log p_{\theta}(y_0|x_0)+\nonumber  \\
& \qquad \mathbb E_{q(y_{1:t},x_{1:t}|y_0, x_0)}\left[\log\frac{q(y_{1:t},x_{1:t}|y_0, x_0)}{p_{\theta}(y_{1:t},x_{1:t}|y_0, x_0)}\right] \\
=& -\log p_{\theta}(y_0|x_0)+ \nonumber \\
& \qquad \mathbb E_{q(y_{1:t},x_{1:t}|y_0, x_0)}\left[\log\frac{q(y_{1:t},x_{1:t}|y_0, x_0)}{p_{\theta}(y_{0:t},x_{1:t}|x_0)/p_{\theta}(y_0|x_0)}\right] \\
%=& -\log p_{\theta}(y_0|x_0)+\mathbb E_{q(y_{1:t},x_{1:t}|y_0, x_0)}\left[\log\frac{q(y_{1:t},x_{1:t}|y_0, x_0)}{p_{\theta}(y_{0:t},x_{1:t}|x_0)}+\log p_{\theta}(y_0|x_0)\right] \\
=& \mathbb E_{q(y_{1:t},x_{1:t}|y_0, x_0)}\left[\log\frac{q(y_{1:t},x_{1:t}|y_0, x_0)}{p_{\theta}(y_{0:t},x_{1:t}|x_0)}\right].
\end{align}
\end{proof}

\begin{theorem}[Closed-form expression]%\label{theorem:1}
% The loss function in \cref{eq:3.3} has a closed-form representation.
The loss function in Equation (13) in the main paper has a closed-form representation.
%
The training is equivalent to optimizing a KL-divergence up to a non-negative constant, \textit{i.e.},
%
\begin{align}
\mathcal L_{VLB}=\mathbb E_{q(y_0,x_t|x_0)}\left[D_{KL}(q(y_t|y_0)\|p_{\theta}(y_t|x_t))\right] + C\sqq{.}%\label{eq:3.4}
\end{align}
\end{theorem}

\begin{proof}
% By the factorization in \cref{eq:3.1} and \cref{eq:3.2}, we observe that
By the factorization in Equation (6) and (11) in the main paper, we observe that
%
\begin{align}
\mathcal L_{VLB}&=\mathbb E_{q(y_{0:t},x_{1:t}|x_0)}\left[\log\frac{q(y_{1:t},x_{1:t}|y_0, x_0)}{p_{\theta}(y_{0:t},x_{1:t}|x_0)}\right] \\
&=\mathbb E_{q(y_{0:t},x_{1:t}|x_0)}\left[\log\frac{1}{p_{\theta}(y_t|x_t)}+\sum_{j=1}^t\log\frac{q(y_j|y_{j-1})}{q(y_{j-1}|y_j)}\right].
\end{align}
%
Using Bayes' rule, for any $j=1,2,\cdots,t$, we have
%
\begin{align}
%\frac{q(y_j|y_{j-1})}{q(y_{j-1}|y_j)}&=\frac{q(y_j|y_{j-1})\cdot q(y_j)\cdot q(y_{j-1})}{q(y_{j-1}|y_j)\cdot q(y_j)\cdot q(y_{j-1})}\\
%&=\frac{q(y_j,y_{j-1})q(y_j)}{q(y_{j-1},y_j)q(y_{j-1})}=\frac{q(y_j)}{q(y_{j-1})}.
\frac{q(y_j|y_{j-1})}{q(y_{j-1}|y_j)}=\frac{q(y_j)}{q(y_{j-1})}, \quad \frac{q(y_t)}{q(y_0)}=\frac{q(y_t|y_0)}{q(y_0|y_t)}.
\end{align}

Hence\sqq{,} it is equivalent to optimizing the KL-divergence up to a non-negative constant $C$:
%
\begin{align}
\mathcal L_{VLB}&=\mathbb E_{q(y_{0:t},x_{1:t}|x_0)}\left[\log\frac{q(y_t|y_0)}{p_{\theta}(y_t|x_t)}+\log\frac{1}{q(y_0|y_t)}\right] \\
&=\mathbb E_{q(y_0,x_t|x_0)}\left[D_{KL}(q(y_t|y_0)\|p_{\theta}(y_t|x_t))\right] + C,
\end{align}
%
where $C=\mathbb E_{q(y_t)}\left[H(q(y_0|y_t))\right]\geqslant0$ and $H$ is the entropy of a distribution.
%
Since $q(y_t|y_0)$ follows a Gaussian distribution, then so is optimal $p_{\theta}(y_t|x_t)$.
\end{proof}


% \begin{theorem}[Optimal solution to \cref{eq:3.4}]%\label{theorem:2}
\begin{theorem}[Optimal solution to Equation (15) in the main paper]%\label{theorem:2}
The optimal $p_{\theta}(y_t|x_t)$ follows a Gaussian distribution with mean $\mu_{\theta}$ being
%
\begin{align}
\mu_{\theta}(x_t) = \sqrt{\balpha{t}}y_0.
\end{align}
\end{theorem}

\begin{proof}
% To minimize the KL-divergence in \cref{eq:3.4}, we first notice that $q(y_t|y_0)$ follows a Gaussian distribution, i.e.,
To minimize the KL-divergence in Equation (15) in the main paper, we first notice that $q(y_t|y_0)$ follows a Gaussian distribution, \textit{i.e.},
%
\begin{align}
q(y_t|y_0)\sim\mathcal N(y_t;\sqrt{\balpha{t}}y_0,(1-\balpha{t})I),\quad \mu_t(y_t)=\sqrt{\balpha{t}}y_0,
\end{align}
%
which implies that $
p_{\theta}(y_t|x_t)\sim\mathcal N(y_t;\mu_{\theta}(x_t),\Sigma_{\theta}(x_t))$
%
with mean $\mu_{\theta}(x_t)=\mu_t(y_t)=\sqrt{\balpha{t}}y_0$.
\end{proof}

\revise{
\begin{lemma}
We have an upper bound of the negative log-likelihood of $-\log p_{\theta}(y_0|x_0)$ by
%
\begin{align}
-\log p_{\theta}(y_0|x_0)\leqslant\mathbb E_q\left[\log\frac{q(y_{1:t},x_{1:s}|y_0, x_0)}{p_{\theta}(y_{0:t},x_{1:s}|x_0)}\right],
\end{align}
%
where $q=q(y_{1:t},x_{1:s}|y_0, x_0).$
\end{lemma}
}

\revise{
\begin{proof}
\begin{align}
&-\log p_{\theta}(y_0|x_0) \\
\leqslant& -\log p_{\theta}(y_0|x_0) + \nonumber \\
& \qquad D_{KL}\left(q(y_{1:t},x_{1:s}|y_0, x_0) \| p_{\theta}(y_{1:t},x_{1:s}|y_0, x_0)\right) \\
=& -\log p_{\theta}(y_0|x_0)+\nonumber  \\
& \qquad \mathbb E_{q(y_{1:t},x_{1:s}|y_0, x_0)}\left[\log\frac{q(y_{1:t},x_{1:s}|y_0, x_0)}{p_{\theta}(y_{1:t},x_{1:s}|y_0, x_0)}\right] \\
=& -\log p_{\theta}(y_0|x_0)+ \nonumber \\
& \qquad \mathbb E_{q(y_{1:t},x_{1:s}|y_0, x_0)}\left[\log\frac{q(y_{1:t},x_{1:s}|y_0, x_0)}{p_{\theta}(y_{0:t},x_{1:s}|x_0)/p_{\theta}(y_0|x_0)}\right] \\
=& \mathbb E_{q(y_{1:t},x_{1:s}|y_0, x_0)}\left[\log\frac{q(y_{1:t},x_{1:s}|y_0, x_0)}{p_{\theta}(y_{0:t},x_{1:s}|x_0)}\right].
\end{align}
\end{proof}
}

\revise{
\begin{theorem}[Closed-form expression]
The loss function in Equation (23) in the main paper has a closed-form representation.
%
The training is equivalent to optimizing a KL-divergence up to a non-negative constant, \textit{i.e.},
%
\begin{align}
\mathcal L_{VLB}=\mathbb E_{q(y_0,x_s|x_0)}\left[D_{KL}(q(y_t|y_0)\|p_{\theta}(y_t|x_s))\right] + C
\sqq{.}
\end{align}
\end{theorem}
}

\revise{
\begin{proof}
By the factorization in Equation (20) and (21) in the main paper, we observe that
%
\begin{align}
\mathcal L_{VLB}&=\mathbb E_{q(y_{0:t},x_{1:s}|x_0)}\left[\log\frac{q(y_{1:t},x_{1:s}|y_0, x_0)}{p_{\theta}(y_{0:t},x_{1:s}|x_0)}\right] \\
&=\mathbb E_{q(y_{0:t},x_{1:s}|x_0)}\left[\log\frac{1}{p_{\theta}(y_t|x_s)}+\sum_{j=1}^t\log\frac{q(y_j|y_{j-1})}{q(y_{j-1}|y_j)}\right].
\end{align}
%
Using Bayes' rule, for any $j=1,2,\cdots,t$, we have
%
\begin{align}
%\frac{q(y_j|y_{j-1})}{q(y_{j-1}|y_j)}&=\frac{q(y_j|y_{j-1})\cdot q(y_j)\cdot q(y_{j-1})}{q(y_{j-1}|y_j)\cdot q(y_j)\cdot q(y_{j-1})}\\
%&=\frac{q(y_j,y_{j-1})q(y_j)}{q(y_{j-1},y_j)q(y_{j-1})}=\frac{q(y_j)}{q(y_{j-1})}.
\frac{q(y_j|y_{j-1})}{q(y_{j-1}|y_j)}=\frac{q(y_j)}{q(y_{j-1})}, \quad \frac{q(y_t)}{q(y_0)}=\frac{q(y_t|y_0)}{q(y_0|y_t)}.
\end{align}
%
Hence\sqq{,} it is equivalent to optimizing the KL-divergence up to a non-negative constant $C$:
%
\begin{align}
\mathcal L_{VLB}&=\mathbb E_{q(y_{0:t},x_{1:s}|x_0)}\left[\log\frac{q(y_t|y_0)}{p_{\theta}(y_t|x_s)}+\log\frac{1}{q(y_0|y_t)}\right] \\
&=\mathbb E_{q(y_0,x_s|x_0)}\left[D_{KL}(q(y_t|y_0)\|p_{\theta}(y_t|x_s))\right] + C,
\end{align}
%
where $C=\mathbb E_{q(y_t)}\left[H(q(y_0|y_t))\right]\geqslant0$ and $H$ is the entropy of a distribution.
%
Since $q(y_t|y_0)$ follows a Gaussian distribution, then so is the optimal $p_{\theta}(y_t|x_s)$.
\end{proof}
}


\revise{
\begin{theorem}[Optimal solution to Equation (25) in the main paper]
The optimal $p_{\theta}(y_t|x_s)$ follows a Gaussian distribution with mean $\mu_{\theta}$ being
%
\begin{align}
\mu_{\theta}(x_s) = \sqrt{\balpha{t}}y_0.
\end{align}
\end{theorem}
}

\revise{
\begin{proof}
% To minimize the KL-divergence in \cref{eq:3.4}, we first notice that $q(y_t|y_0)$ follows a Gaussian distribution, i.e.,
To minimize the KL-divergence in Equation (25) in the main paper, we first notice that $q(y_t|y_0)$ follows a Gaussian distribution, \textit{i.e.},
%
\begin{align}
q(y_t|y_0)\sim\mathcal N(y_t;\sqrt{\balpha{t}}y_0,(1-\balpha{t})I),\quad \mu_t(y_t)=\sqrt{\balpha{t}}y_0,
\end{align}
%
which implies that $
p_{\theta}(y_t|x_s)\sim\mathcal N(y_t;\mu_{\theta}(x_s),\Sigma_{\theta}(x_s))$
%
with mean $\mu_{\theta}(x_s)=\mu_t(y_t)=\sqrt{\balpha{t}}y_0$.
\end{proof}
}


\begin{table*}[ht]
% \renewcommand\arraystretch{1.2}
\begin{minipage}[t]{0.48\textwidth}
\caption{
    \newrevise{
    \textbf{Quantitative comparison} between single-step and multi-step \method (TSIT-DMT-2step) upon TSIT.
    %
    FID and SSIM are used to evaluate the image quality and content preservation, respectively.
    }
}
\vspace{-5pt}
\centering
\SetTblrInner{rowsep=2.15pt}                % Row space.
\SetTblrInner{colsep=10.0pt}               % Col space.
% \scriptsize
\newrevise{
\begin{tblr}{
    cell{1-6}{1-5}={halign=c,valign=m},    % Text alignment for all cells.
    cell{1}{2,4}={c=2}{},                  % Multi-column cells.
    cell{1}{1}={r=2}{},                    % Multi-row cells.
    hline{1,7}={1-5}{1.0pt},               % Horizontal lines.
    hline{2}={2-5}{},                      % Horizontal lines.
    hline{3}={1-5}{},                      % Horizontal lines.
    vline{2,4}={1-6}{},                    % Vertical lines.
}
\label{tab:metric}
Method & TSIT-\method & & TSIT-\method-2step & \\
 & \textbf{FID$\downarrow$} & \textbf{SSIM$\uparrow$} & \textbf{FID$\downarrow$} & \textbf{SSIM$\uparrow$} \\
$t=50$ & 42.00 & 0.473  & 47.77 & 0.563 \\
$t=100$ & 37.22 & 0.460 & 43.52 & 0.563 \\
$t=200$ & 36.78 & 0.446 & 45.78 & 0.536 \\
$t=400$ & 50.79 & 0.251 & 48.34 & 0.447 \\
\end{tblr}
}
\hfill
\end{minipage}
\begin{minipage}[t]{0.48\textwidth}
\caption{
    \newrevise{
    \textbf{Time cost comparison} between single-step and multi-step \method upon TSIT.
    %
    To measure the time cost, we report the total number of training epochs of \method (\method Epoch) and of fusion UNet (Fusion Epoch), training time for 1,000 images for \method (\method Train) and fusion UNet (Fusion Train), and inference time for a single image.
    }
}
\vspace{-10pt}
\centering
\SetTblrInner{rowsep=1.05pt}               % Row space.
\SetTblrInner{colsep=10.0pt}               % Col space.
% \scriptsize
\newrevise{
\begin{tblr}{
    cell{1-6}{1-3}={halign=c,valign=m},    % Text alignment for all cells.
    hline{1,2,7}={1-3}{1.0pt},             % Horizontal lines.
    hline{2}={1-3}{},                      % Horizontal lines.
    vline{2,3}={1-6}{},                    % Vertical lines.
}
\label{tab:time}
Method & TSIT-\method & TSIT-\method-2step \\
DMT Train & 82s & 82s \\
DMT Epoch & 60 & 60 \\
Fusion Train & No such step & 139s \\
Fusion Epoch & No such step & 100 \\
Inference & 0.48s & 0.64s \\
\end{tblr}
}
\end{minipage}
\end{table*}


\definecolor{mygreen}{RGB}{34,170,133}
\begin{table*}[ht]
\begin{minipage}[t]{0.48\textwidth}
\centering
\vspace{-8pt}
\caption{
    \newrevise{
    \textbf{Ablation study} of $(s,t)$ pair near $s=t=200$ fixing $s=200$ on CelebA-HQ dataset.
    %
    For clearer demonstration, original \method (\textit{i.e.}, pair with $s=t$) is highlighted in \textbf{\textcolor{gray}{gray}}.
    }
}
\label{tab:ablation_s_t_sstar}
\vspace{-10pt}
\SetTblrInner{rowsep=2.15pt}                % Row space.
\SetTblrInner{colsep=10.0pt}               % Col space.
% \scriptsize
\newrevise{
\begin{tblr}{
    cell{1-18}{1-5}={halign=c,valign=m},    % Text alignment for all cells.
    % cell{2}{1}={c=17}{},                  % Multi-column cells.
    cell{2}{1}={r=17}{},                    % Multi-row cells.
    hline{1,19}={1-5}{1.0pt},               % Horizontal lines.
    hline{2}={1-5}{},                      % Horizontal lines.
    % hline{3}={1-5}{},                      % Horizontal lines.
    vline{2}={2-18}{},                    % Vertical lines.
    vline{3}={1-18}{},                    % Vertical lines.
    cell{10}{2-5}={bg=lightgray!35},
}
$s$ & $t$ & \textbf{FID$\downarrow$} & \textbf{SSIM$\uparrow$} & \textbf{LPIPS$\downarrow$} \\
$s=200$ & $t=0$ & 53.61 & 0.347 & 0.492 \\
 & $t=50$ & 37.27 & 0.445 & 0.443 \\
 & $t=100$ & 37.31 & 0.447 & 0.443 \\
 & $t=150$ & 43.43 & 0.441 & 0.445 \\
 & $t=160$ & 43.35 & 0.438 & 0.449 \\
 & $t=170$ & 44.69 & 0.458 & 0.428 \\
 & $t=180$ & 42.82 & \bf 0.466 & \bf 0.428 \\
 & $t=190$ & 45.94 & 0.458 & 0.430 \\
 & $t=200$ & \bf 36.78 & 0.446 & 0.433 \\
 & $t=210$ & 45.15 & 0.425 & 0.433 \\
 & $t=220$ & 46.17 & 0.422 & 0.436 \\
 & $t=230$ & 46.78 & 0.418 & 0.450 \\
 & $t=240$ & 47.69 & 0.415 & 0.451 \\
 & $t=250$ & 48.32 & 0.377 & 0.464 \\
 & $t=300$ & 55.54 & 0.337 & 0.500 \\
 & $t=350$ & 64.66 & 0.267 & 0.567 \\
 & $t=400$ & 53.42 & 0.328 & 0.493 \\
\end{tblr}
}
\end{minipage}
\hfill
\begin{minipage}[t]{0.48\textwidth}
\centering
\vspace{-8pt}
\caption{
    \newrevise{
    \textbf{Ablation study} of $(s,t)$ pair near $s=t=200$ fixing $t=200$ on CelebA-HQ dataset.
    %
    For clearer demonstration, original \method (\textit{i.e.}, pair with $s=t$) is highlighted in \textbf{\textcolor{gray}{gray}}.
    }
}
\label{tab:ablation_s_t_tstar}
\vspace{-10pt}
\SetTblrInner{rowsep=2.15pt}                % Row space.
\SetTblrInner{colsep=10.0pt}               % Col space.
% \scriptsize
\newrevise{
\begin{tblr}{
    cell{1-18}{1-5}={halign=c,valign=m},    % Text alignment for all cells.
    % cell{2}{1}={c=17}{},                  % Multi-column cells.
    cell{2}{2}={r=17}{},                    % Multi-row cells.
    hline{1,19}={1-5}{1.0pt},               % Horizontal lines.
    hline{2}={1-5}{},                      % Horizontal lines.
    % hline{3}={1-5}{},                      % Horizontal lines.
    vline{2}={2-18}{},                    % Vertical lines.
    vline{3}={1-18}{},                    % Vertical lines.
    cell{10}{2-5}={bg=lightgray!35},
    cell{10}{1}={bg=lightgray!35},
}
$s$ & $t$ & \textbf{FID$\downarrow$} & \textbf{SSIM$\uparrow$} & \textbf{LPIPS$\downarrow$} \\
$s=0$ & $t=200$ & 44.98 & 0.448 & 0.453 \\
$s=50$ & & 42.93 & 0.451 & 0.437 \\
$s=100$ & & 44.50 & 0.443 & 0.457 \\
$s=150$ & & 43.60 & 0.446 & 0.437 \\
$s=160$ & & 42.72 & 0.445 & 0.451 \\
$s=170$ & & 44.37 & 0.443 & 0.427 \\
$s=180$ & & 44.35 & 0.447 & 0.432 \\
$s=190$ & & 45.49 & 0.438 & \bf 0.427 \\
$s=200$ & & \bf 36.78 & 0.446 & 0.433 \\
$s=210$ & & 43.99 & 0.441 & 0.442 \\
$s=220$ & & 44.94 & 0.451 & 0.432 \\
$s=230$ & & 45.67 & 0.444 & 0.439 \\
$s=240$ & & 45.68 & 0.441 & 0.441 \\
$s=250$ & & 44.11 & 0.437 & 0.433 \\
$s=300$ & & 45.36 & \bf 0.453 & 0.432 \\
$s=350$ & & 45.39 & 0.451 & 0.421 \\
$s=400$ & & 44.44 & 0.445 & 0.437 \\
\end{tblr}
}
\end{minipage}
\end{table*}


\section{\newrevise{Comparison between multi-step and asymmetric \method}}


\newrevise{
\yr{\textbf{Multi-step \method.}} To implement the multi-step \method, due to the use of the vanilla DDPM, which is only capable of inputting a 3-channel input intermediate noisy image, we train an auxiliary UNet model to fuse the $y_{t/2}$ transformed from $x_{t/2}$ together with the $y'_{t/2}$ denoised from the $y_{t}$.
}


\newrevise{
In order to train the fusion UNet processing the intermediate noisy images $y_{t/2}$ and $y'_{t/2}$, we first need to prepare the dataset.
%
In more details, given a paired data $(x_0,y_0)$, the preset timestep $t$, noise $z$ under standard Gaussian distribution, the pre-trained \method $G$, and the pre-trained diffusion model, we first apply the diffusion forward process onto both $x_0$ and $y_0$ with noise $z$ until timestep $t$ and $t/2$, \textit{i.e.}, we acquire the $x_{t/2}$, $x_{t}$, $y_{t/2}$, and $y_{t}$.
%
Next, we utilize the pre-trained \method model to obtain the transformed $G(x_{t/2})$ and $G(x_{t})$.
%
Then, we apply the reverse process via the pre-trained diffusion model to achieve the denoised result from $G(x_{t})$, denoted by $D(G(x_{t}))$.
%
Finally, repeating the process above with different paired $(x_0,y_0,z)$, we are able to acquire the dataset for the fusion UNet.
%
Note that similar to the training of \method, the fusion UNet are aimed to deal with \textit{noisy} images.
%
That is to say, empirically we need much more data samples and training epochs since CNN may easily fail on noisy data.
%
For the CelebA-HQ dataset with 30,000 images, we train the fusion UNet with more than 200,000 samples.
%
As a comparison, we train \method within only 60 epochs with 27,000 data samples.
}


\newrevise{
The comparisons between multi-step and our single-step design are reported in \cref{tab:metric} and \cref{tab:time}.
%
From \cref{tab:metric} we observe that under the same timestep, the FID score of multi-step \method is worse than single-step in most cases, and multi-step \method does not significantly benefit from the additional step of performing the fusion UNet.
%
As for SSIM score, there is indeed performance improvement to some extent compared to the vanilla DMT, which is mainly due to doubling the information during denoising process and taking advantage of the fusion UNet.
%
Considering the dramatic additional time cost discussed below, we regard that the vanilla single-step DMT is an adequate solution to the I2I task. 
%
It is noteworthy that for multi-step \method, both the training (including \method and Fusion training) and inference costs increase significantly, as shown in \cref{tab:time}, which confirms the superiority of the single-step \method proposed in the paper.
}


\newrevise{
\yr{\textbf{Asymmetric \method.}}
As for asymmetric \method, in addition to the theoretical analysis in the main paper, we also conducted a comprehensive ablation study focusing on the performance at various timestep pairs $(s,t)$ near $s=t$.
%
As shown in \cref{tab:ablation_s_t_sstar,tab:ablation_s_t_tstar}, our proposed strategy (pair with $s=t$) is capable of achieving on-par or even superior performance across various $(s,t)$ alternatives.
}


\section{More results}

This part shows more qualitative results and compares our \method with existing I2I approaches, including Pix2Pix (GAN-based)~\cite{isola2017image}, TSIT (GAN-based)~\cite{jiang2020tsit} and Palette (DDPM-based)~\cite{saharia2021palette}.
%
We perform evaluation on the tasks of stylization (\cref{fig:portrait_qmupd0-1-0}), image colorization (\cref{fig:afhq}), segmentation to image (\cref{fig:celeba}), and sketch to image (\cref{fig:edges2handbags}), using our handcrafted Anime dataset, AFHQ~\cite{choi2020stargan}, CelebA-HQ~\cite{karras2018progressive}, and Edges2handbags~\cite{zhu2016generative,xie15hed}, respectively.
%
Our method surpasses the other three competitors with higher fidelity (\textit{e.g.}, clearer contours\newrevise{, less artifacts} and more realistic colors \newrevise{as highlighted}), suggesting that our \method manages to bridge the content information provided by the input \revise{condition} and the domain knowledge contained in the pre-trained DDPM.


% \definecolor{myblue}{RGB}{132, 180, 213}
% \definecolor{myred}{RGB}{213, 123, 101}
\definecolor{myblue}{RGB}{0, 0, 0}
\definecolor{myred}{RGB}{0, 0, 0}
\begin{figure*}[tp]
\centering
\begin{tabular}{ccc}
\begin{minipage}[t]{0.3\linewidth}\includegraphics[width=1\linewidth]{figures/fig_ablation_t_afhq_rebuttal/afhq_metric_L1.png}\end{minipage} &
\begin{minipage}[t]{0.3\linewidth}\includegraphics[width=1\linewidth]{figures/fig_ablation_t_afhq_rebuttal/afhq_metric_L2.png}\end{minipage} &
\begin{minipage}[t]{0.3\linewidth}\includegraphics[width=1\linewidth]{figures/fig_ablation_t_afhq_rebuttal/afhq_metric_SSIM.png}\end{minipage} \\
(a) $L_1$ & (b) $L_2$ & (c) SSIM \\
\begin{minipage}[t]{0.3\linewidth}\includegraphics[width=1\linewidth]{figures/fig_ablation_t_afhq_rebuttal/afhq_metric_PSNR.png}\end{minipage} &
\begin{minipage}[t]{0.3\linewidth}\includegraphics[width=1\linewidth]{figures/fig_ablation_t_afhq_rebuttal/afhq_metric_FID.png}\end{minipage} &
\begin{minipage}[t]{0.3\linewidth}\includegraphics[width=1\linewidth]{figures/fig_ablation_t_afhq_rebuttal/afhq_metric_LPIPS.png}\end{minipage} \\
(d) PSNR & (e) FID & (f) LPIPS
\end{tabular}
\caption{
    \textbf{Analysis on the preset timestep, $t$.}
    %
    We introduce 6 distinct metrics, \textit{i.e.}, $L_1$, $L_2$, SSIM, PSNR, FID, and LPIPS.
    %
    We plot the distance between $(x_t,y_t)$ and $(x_0, x_t)$ at different timesteps on AFHQ dataset, which are shown in red and blue curves, respectively.
}
\label{fig:ablation_t_afhq}
\vspace{-10pt}
\end{figure*}


% \definecolor{myblue}{RGB}{132, 180, 213}
% \definecolor{myred}{RGB}{213, 123, 101}
\definecolor{myblue}{RGB}{0, 0, 0}
\definecolor{myred}{RGB}{0, 0, 0}
\begin{figure*}[tp]
\centering
\begin{tabular}{ccc}
\begin{minipage}[t]{0.3\linewidth}\includegraphics[width=1\linewidth]{figures/fig_ablation_t_afhq_rebuttal/afhq_metric_L1.png}\end{minipage} &
\begin{minipage}[t]{0.3\linewidth}\includegraphics[width=1\linewidth]{figures/fig_ablation_t_afhq_rebuttal/afhq_metric_L2.png}\end{minipage} &
\begin{minipage}[t]{0.3\linewidth}\includegraphics[width=1\linewidth]{figures/fig_ablation_t_afhq_rebuttal/afhq_metric_SSIM.png}\end{minipage} \\
(a) $L_1$ & (b) $L_2$ & (c) SSIM \\
\begin{minipage}[t]{0.3\linewidth}\includegraphics[width=1\linewidth]{figures/fig_ablation_t_afhq_rebuttal/afhq_metric_PSNR.png}\end{minipage} &
\begin{minipage}[t]{0.3\linewidth}\includegraphics[width=1\linewidth]{figures/fig_ablation_t_afhq_rebuttal/afhq_metric_FID.png}\end{minipage} &
\begin{minipage}[t]{0.3\linewidth}\includegraphics[width=1\linewidth]{figures/fig_ablation_t_afhq_rebuttal/afhq_metric_LPIPS.png}\end{minipage} \\
(d) PSNR & (e) FID & (f) LPIPS
\end{tabular}
\caption{
    \textbf{Analysis on the preset timestep, $t$.}
    %
    We introduce 6 distinct metrics, \textit{i.e.}, $L_1$, $L_2$, SSIM, PSNR, FID, and LPIPS.
    %
    We plot the distance between $(x_t,y_t)$ and $(x_0, x_t)$ at different timesteps on AFHQ dataset, which are shown in red and blue curves, respectively.
}
\label{fig:ablation_t_afhq}
\vspace{-10pt}
\end{figure*}


% \definecolor{myblue}{RGB}{132, 180, 213}
% \definecolor{myred}{RGB}{213, 123, 101}
\definecolor{myblue}{RGB}{0, 0, 0}
\definecolor{myred}{RGB}{0, 0, 0}
\begin{figure*}[tp]
\centering
\begin{tabular}{ccc}
\begin{minipage}[t]{0.3\linewidth}\includegraphics[width=1\linewidth]{figures/fig_ablation_t_afhq_rebuttal/afhq_metric_L1.png}\end{minipage} &
\begin{minipage}[t]{0.3\linewidth}\includegraphics[width=1\linewidth]{figures/fig_ablation_t_afhq_rebuttal/afhq_metric_L2.png}\end{minipage} &
\begin{minipage}[t]{0.3\linewidth}\includegraphics[width=1\linewidth]{figures/fig_ablation_t_afhq_rebuttal/afhq_metric_SSIM.png}\end{minipage} \\
(a) $L_1$ & (b) $L_2$ & (c) SSIM \\
\begin{minipage}[t]{0.3\linewidth}\includegraphics[width=1\linewidth]{figures/fig_ablation_t_afhq_rebuttal/afhq_metric_PSNR.png}\end{minipage} &
\begin{minipage}[t]{0.3\linewidth}\includegraphics[width=1\linewidth]{figures/fig_ablation_t_afhq_rebuttal/afhq_metric_FID.png}\end{minipage} &
\begin{minipage}[t]{0.3\linewidth}\includegraphics[width=1\linewidth]{figures/fig_ablation_t_afhq_rebuttal/afhq_metric_LPIPS.png}\end{minipage} \\
(d) PSNR & (e) FID & (f) LPIPS
\end{tabular}
\caption{
    \textbf{Analysis on the preset timestep, $t$.}
    %
    We introduce 6 distinct metrics, \textit{i.e.}, $L_1$, $L_2$, SSIM, PSNR, FID, and LPIPS.
    %
    We plot the distance between $(x_t,y_t)$ and $(x_0, x_t)$ at different timesteps on AFHQ dataset, which are shown in red and blue curves, respectively.
}
\label{fig:ablation_t_afhq}
\vspace{-10pt}
\end{figure*}


% \definecolor{myblue}{RGB}{132, 180, 213}
% \definecolor{myred}{RGB}{213, 123, 101}
\definecolor{myblue}{RGB}{0, 0, 0}
\definecolor{myred}{RGB}{0, 0, 0}
\begin{figure*}[tp]
\centering
\begin{tabular}{ccc}
\begin{minipage}[t]{0.3\linewidth}\includegraphics[width=1\linewidth]{figures/fig_ablation_t_afhq_rebuttal/afhq_metric_L1.png}\end{minipage} &
\begin{minipage}[t]{0.3\linewidth}\includegraphics[width=1\linewidth]{figures/fig_ablation_t_afhq_rebuttal/afhq_metric_L2.png}\end{minipage} &
\begin{minipage}[t]{0.3\linewidth}\includegraphics[width=1\linewidth]{figures/fig_ablation_t_afhq_rebuttal/afhq_metric_SSIM.png}\end{minipage} \\
(a) $L_1$ & (b) $L_2$ & (c) SSIM \\
\begin{minipage}[t]{0.3\linewidth}\includegraphics[width=1\linewidth]{figures/fig_ablation_t_afhq_rebuttal/afhq_metric_PSNR.png}\end{minipage} &
\begin{minipage}[t]{0.3\linewidth}\includegraphics[width=1\linewidth]{figures/fig_ablation_t_afhq_rebuttal/afhq_metric_FID.png}\end{minipage} &
\begin{minipage}[t]{0.3\linewidth}\includegraphics[width=1\linewidth]{figures/fig_ablation_t_afhq_rebuttal/afhq_metric_LPIPS.png}\end{minipage} \\
(d) PSNR & (e) FID & (f) LPIPS
\end{tabular}
\caption{
    \textbf{Analysis on the preset timestep, $t$.}
    %
    We introduce 6 distinct metrics, \textit{i.e.}, $L_1$, $L_2$, SSIM, PSNR, FID, and LPIPS.
    %
    We plot the distance between $(x_t,y_t)$ and $(x_0, x_t)$ at different timesteps on AFHQ dataset, which are shown in red and blue curves, respectively.
}
\label{fig:ablation_t_afhq}
\vspace{-10pt}
\end{figure*}







\end{document}


