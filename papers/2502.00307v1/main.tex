\documentclass[10pt,journal,compsoc]{IEEEtran}
\usepackage{amsmath,amsfonts}
\usepackage{amsthm,amssymb}
\usepackage{amsmath,amssymb,amsfonts,dsfont,pifont,bm,bbm,mathrsfs,mathtools,nicefrac}
\usepackage{xcolor,algorithm,listings}
\usepackage{algorithmicx}
\usepackage{algpseudocode}
\usepackage{array}
\usepackage[caption=false,font=normalsize,labelfont=sf,textfont=sf]{subfig}
\usepackage{textcomp}
\usepackage{booktabs,multirow,adjustbox,diagbox,threeparttable,multicol,colortbl,tabularray}
\usepackage{stfloats}
\usepackage{url,setspace}
\usepackage{verbatim}
\usepackage{graphicx}
\usepackage{xspace}
\usepackage{cite}
\hyphenation{op-tical net-works semi-conduc-tor IEEE-Xplore}
\def\BibTeX{{\rm B\kern-.05em{\sc i\kern-.025em b}\kern-.08em
    T\kern-.1667em\lower.7ex\hbox{E}\kern-.125emX}}
\newtheorem{theorem}{Theorem}
\newtheorem{lemma}{Lemma}
\usepackage[pagebackref,breaklinks,colorlinks,linkcolor=blue,citecolor=blue, bookmarks=false]{hyperref}
\usepackage[capitalize]{cleveref}  % Should be loaded after 'hyperref', and works perfectly with 'subfigure'.'
\Crefname{section}{Section}{Sections}
\Crefname{table}{Table}{Tables}
\Crefname{figure}{Figure}{Figures}
\Crefname{equation}{Equation}{Equations}
\hyphenpenalty=1200

\definecolor{codegreen}{rgb}{0,0.6,0}
\definecolor{codegray}{rgb}{0.5,0.5,0.5}
\definecolor{codepurple}{rgb}{0.58,0,0.82}
\definecolor{backcolour}{rgb}{1.0,1.0,1.0}
\lstdefinestyle{mystyle}{
    backgroundcolor=\color{backcolour},
    commentstyle=\color{codegreen},
    keywordstyle=\color{magenta},
    numberstyle=\tiny\color{codegray},
    stringstyle=\color{codepurple},
    basicstyle=\ttfamily\scriptsize,
    breakatwhitespace=false,
    breaklines=true,
    captionpos=b,
    keepspaces=true,
    numbers=left,
    numbersep=5pt,
    showspaces=false,
    showstringspaces=false,
    showtabs=false,
    tabsize=2
}
\lstset{style=mystyle}

\newcommand{\balpha}[1]{\bar{\alpha}_{#1}}
\newcommand{\todo}[1]{\textcolor{red}{[TODO: #1]}}
\newcommand{\revise}[1]{\textcolor{black}{#1}}
\newcommand{\newrevise}[1]{\textcolor{black}{#1}}
\newcommand{\method}{DMT\xspace}
\newcommand{\supp}{\textit{Supplementary Material}\xspace}
\newcommand{\sqq}[1]{\textcolor{black}{#1}}
\newcommand{\yr}[1]{\textcolor{black}{#1}}

\begin{document}

\title{A Diffusion Model Translator for Efficient Image-to-Image Translation}

\author{
Mengfei Xia, Yu Zhou, Ran Yi, Yong-Jin Liu,~\IEEEmembership{Senior Member,~IEEE}, Wenping Wang,~\IEEEmembership{Fellow,~IEEE}
    % <-this % stops a space
\IEEEcompsocitemizethanks{\IEEEcompsocthanksitem  Mengfei Xia and Yu Zhou are with the MOE-Key Laboratory of Pervasive Computing, Department of Computer Science and Technology, Tsinghua University, Beijing 100084, China (email: \{xmf20, yzhou20\}@mails.tsinghua.edu.cn).% <-this % stops a space
\IEEEcompsocthanksitem Ran Yi is with the Department of Computer Science and Engineering, Shanghai Jiao Tong University, Shanghai, China (email: ranyi@sjtu.edu.cn).% <-this % stops a space
\IEEEcompsocthanksitem Yong-Jin Liu is with the MOE-Key Laboratory of Pervasive Computing, Department of Computer Science and Technology, Tsinghua University, Beijing 100084, China (email: liuyongjin@tsinghua.edu.cn).
\IEEEcompsocthanksitem Wenping Wang is with the Department of Computer Science and Computer Engineering at Texas A\&M University, Texas, The United States (email: wenping@tamu.edu).% <-this % stops a space}
\IEEEcompsocthanksitem Corresponding authors: Ran Yi and Yong-Jin Liu.}% <-this % stops a space}
}

% The paper headers
\markboth{IEEE TRANSACTIONS ON PATTERN ANALYSIS AND MACHINE INTELLIGENCE, VOL. 46, NO. 12, DECEMBER 2024}%
{Xia \MakeLowercase{\textit{et al.}}: A Diffusion Model Translator for Efficient Image-to-Image Translation}

% \IEEEpubid{0000--0000/00\$00.00~\copyright~2021 IEEE}
% Remember, if you use this you must call \IEEEpubidadjcol in the second
% column for its text to clear the IEEEpubid mark.


\IEEEtitleabstractindextext{
\begin{abstract}


Applying diffusion models to image-to-image translation (I2I) has recently received increasing attention due to its practical applications.
%
Previous attempts \sqq{inject information} from the source image into each denoising step for an iterative refinement, \sqq{thus resulting in a} time-consuming implementation.
%
We propose an efficient method that equips a diffusion model with a lightweight translator, dubbed a Diffusion Model Translator (\method), to accomplish I2I. 
%
Specifically, we first \sqq{offer} theoretical justification that \sqq{in employing} the pioneer\sqq{ing} DDPM work \sqq{for} the I2\revise{I} task, it is \sqq{both} feasible and sufficient to transfer the distribution from one domain to another only at some intermediate step.
%
We \sqq{further} observe that the translation performance highly depends on the \sqq{chosen timestep for} domain transfer, and therefore propose a practical strategy to automatically select an appropriate timestep for a given task.
%
We evaluate our approach on a range of I2I applications, \sqq{including} image stylization, image colorization, segmentation to image, and sketch to image, to validate its efficacy and general utility.
%
\sqq{The c}omparisons show that our \method surpasses existing methods in both quality and efficiency.
%
Code will be made publicly available.

\end{abstract}

\begin{IEEEkeywords}
Diffusion models, image translation, deep learning, generative models.
\end{IEEEkeywords}}

\maketitle

% \IEEEdisplaynontitleabstractindextext

% \IEEEpeerreviewmaketitle

\section{Introduction}
% Large Language Models~(LLMs) represent a transformative advancement in the field of language processing, demonstrating an unparalleled capacity for text generation and comprehension, which can be further applied in a wide variety of applications.  
% %Large language models (LLMs) have risen to prominence in various fields, offering endless possibilities for artificial intelligence applications. 
% Despite their significant prevalence in recent years, LLMs are frequently challenged with security and privacy issues, such as poor explainability~\cite{}, poor robustness~\cite{}, data dependency~\cite{}, etc. Among them, a specific and notable concern that has garnered increasing attention is the phenomenon of `hallucination', where models generate plausible but factually inaccurate or irrelevant content when employed for specific tasks such as problem-solving.  
% %In particular, the hallucination issue is when these large models are employed for problem-solving, users frequently voice concerns regarding being misled or deceived by the models' nonsensical and erratic outputs. 
% The tendency of these models to produce inaccurate outputs and fabricate facts has severely undermined the safety and usability of LLM applications, which calls for immediate attention in LLM research. 
% %Hallucination in large language models (LLMs) is a critical issue that needs immediate attention in LLM research. The tendency of these models to produce inaccurate outputs and fabricate facts has severely undermined the safety and usability of LLM applications. 
%exceptional 
%including limited explainability, compromised robustness, and a heavy reliance on data, each 
%However, d
Large Language Models (LLMs) have revolutionized language processing, demonstrating impressive text generation and comprehension capabilities with diverse applications. However, despite their growing use, LLMs face significant security and privacy challenges~\cite{siddiq2023generate, hou2023large, kaddour2023challenges, li2024model, 10.1145/3691620.3695510}, which affect their overall effectiveness and reliability. A critical issue is the phenomenon of \emph{hallucination}, where LLMs generate outputs that are coherent but factually incorrect or irrelevant. This tendency to produce misleading information compromises the safety and usability of LLM-based systems. This paper focuses on \emph{fact-conflicting hallucina}tion (FCH), the most prominent form of hallucination in LLMs. FCH occurs when LLMs generate content that directly contradicts established facts. For instance, as illustrated in \figref{fig:example1}, an LLM incorrectly asserts that ``\emph{Haruki Murakami won the Nobel Prize in Literature in 2016}'', whereas the fact is that ``\emph{Haruki Murakami has not won the Nobel Prize, though he has received numerous other literary awards}''. 
Such inaccuracies can significantly lead to user confusion and undermine the trust and reliability that are crucial for LLM applications.

% Large Language Models~(LLMs) have brought transformative advancements to language processing and beyond, showcasing text generation and comprehension abilities with wide-ranging applications. 
% Despite the increasing prevalence, LLMs face critical challenges in security and privacy aspects~\cite{siddiq2023generate, hou2023large, kaddour2023challenges}, heavily impacting their effectiveness and reliability. 
% One notable issue is the phenomenon of \emph{hallucination}, where LLMs produce coherent but factually inaccurate or irrelevant outputs during problem-solving. 
% Such a tendency to generate misleading information jeopardizes the safety and usability of LLM-based applications. 
% This paper concerns the \emph{fact-conflicting hallucination}~(FCH), which is the primary form of hallucinations in LLMs. 
% FCH occurs when LLMs generate content that directly contradicts the well-established facts, as exemplified in \figref{fig:example1}, where an LLM incorrectly believes 
% ``\emph{Haruki Murakami won the Nobel Prize in Literature in 2016}'', deviating from the fact that ``\emph{Haruki Murakami has not won the Nobel Prize but other numerous awards for his work in Literature}''. Such misinformation can cause significant user confusion and undermine the trust and reliability that are essential in various LLM applications. 

%correct answer of 

%is manifested by
%Such misinformation dissemination leads to significant user confusion, eroding the trust and reliability that are crucial in various LLM applications. 

%Large Language Models~(LLMs) represent a transformative advancement in the field of language processing, demonstrating an unparalleled capacity for text generation and comprehension, which can be further applied in a wide variety of applications. Despite their growing prevalence, LLMs encounter critical challenges, particularly in aspects of security and privacy. These include concerns such as limited explainability~\cite{}, compromised robustness~\cite{}, and heavy reliance on data~\cite{}, each posing distinct challenges to their efficacy and reliability. Among these, the phenomenon of ``hallucination'' stands out as a notable concern. This occurs when LLMs, while employed in tasks like problem-solving, generate outputs that are coherent yet factually inaccurate or irrelevant. Such a tendency to produce misleading information not only compromises the safety of LLM applications but also raises urgent questions regarding their usability. 

% Hallucinations in LLMs manifest in several distinct forms, each contributing differently to the challenges identified in their growing applications. 
% %The first, known as ``Input-conflicting hallucination'', arises when there is a discrepancy between the model's output and the user's initial input, reflecting a potential misunderstanding of the task at hand. On the other hand, ``Context-conflicting hallucination'' represents the second type, occurring when LLMs produce inconsistent responses in prolonged or multi-turn interactions, indicative of their limitations in maintaining coherent context. 
% Among the three types categorized in the literature~\cite{huang2023survey,zhang2023hallucination}, ``Fact-conflicting hallucination~(FCH)'' poses a particularly serious concern which is the primary focus of this paper. This phenomenon generates content in direct opposition to established factual knowledge. As illustrated in the example shown in Figure~\ref{fig:example1}, when an LLM was asked about the first person to walk on the moon, it incorrectly answered ``Charles Lindbergh in 1951'', a clear deviation from the factual answer of Neil Armstrong in 1969. This type of hallucination can lead to the dissemination of incorrect information and cause significant confusion among users, undermining the trust and reliability critical in various LLM applications. %Addressing fact-conflicting hallucinations is therefore essential for the advancement of LLMs, ensuring they not only function effectively but also responsibly in their diverse roles.


% According to \cite{huang2023survey} and \cite{zhang2023hallucination}, hallucinations in large language models can be categorized into types such as factual hallucinations and contextual hallucinations. Current benchmark assessments tend to focus on evaluating the propensity of LLMs to generate erroneous facts. The origin of these issues can be traced back to multiple deficiencies, including flaws in training data, training algorithms, and the inference process.

% \begin{figure}[t]
%     \centering
%     \includegraphics[width=0.95\linewidth]{fig/example1-cropped.pdf}\\
%     \caption{A Hallucination Output Example.}
%     %\vspace{-0.5cm}
%     \label{fig:example1}
% \end{figure}

\begin{figure}[t]
\centering
\vspace{3mm}
\hspace{-3mm}
\includegraphics[width=\linewidth]{fig/drowzee-example.pdf}
\\[0.5em]
\caption{A Hallucination Output Example}
\label{fig:example1}
\vspace{-4mm}
\end{figure}
%\lnk{Factual Hallucination and LLM inference current status}

Recent studies have introduced various methods to detect LLM hallucinations. A common approach involves developing specialized benchmarks, such as TruthfulQA~\cite{lin-etal-2022-truthfulqa}, HaluEval~\cite{HaluEval}, and KoLA~\cite{yu2023kola}, to assess hallucinations in tasks like question-answering, summarization, and knowledge graphs. 
While manually labeled datasets provide valuable insights, current methods often rely on simplistic or semi-automated techniques such as string matching, manual validation, or verification through another language model. These approaches reveal significant gaps in automatically and effectively detecting fact-conflicting hallucinations (FCH). 
The primary challenges in FCH detection arise from the lack of dedicated ground truth datasets, the absence of comprehensive test cases designed to trigger FCH, and the lack of a robust testing framework.  
Unlike other types of hallucinations, such as input-conflicting or context-conflicting hallucinations~\cite{ji-etal-2023-rho, shi2023large}, which can often be identified through semantic consistency checks, detecting FCH requires the verification of factual accuracy against external knowledge sources/databases. This process is particularly challenging and resource-intensive, especially for tasks that involve complex logical relationships~\cite{zhang2024fusion}. We identify three primary challenges in addressing this research gap:


% Recent studies have introduced a range of methods for detecting 
% hallucinations. One common approach involves creating comprehensive benchmarks tailored for this purpose. 
% Datasets such as TruthfulQA~\cite{lin-etal-2022-truthfulqa}, HaluEval~\cite{HaluEval}, and KoLA~\cite{yu2023kola} have been designed to evaluate hallucinations across different contexts, including question-answering, summarization, and knowledge graphs. 
% Despite the value of these manually labeled datasets, the current techniques heavily rely on naive and semi-automatic methods, such as string matching, manual validation, or utilizing another LLM for confirmation. 
% Therefore, there is a gap 
% in automatically and effectively testing FCHs, and the primary obstacle in testing FCH is the absence of dedicated ground truth datasets and an extensive testing framework.  
% Unlike other types of hallucinations, e.g., input-conflicting or context-conflicting 
% \cite{ji-etal-2023-rho, shi2023large}, 
% which can be identified through checks for semantic consistency, 
% detecting FCH
% requires the verification of the content's factual accuracy against external sources of knowledge or databases. This makes the process particularly arduous and resource-intensive, especially for tasks processing content with complex logical connections. 
% Here, we highlight three concrete challenges in filling up the identified research gap: 




%The main obstacle in testing for FCH is the absence of dedicated ground truth datasets and specific testing frameworks. Unlike other types of hallucinations~(e.g., input-conflicting and context-conflicting hallucinations, to be detailed in Section~\ref{subsec:cat}) which can be identified through checks for semantic consistency, FCH demands the verification of the content's factual accuracy against external sources of knowledge or databases. This requirement makes the process particularly challenging and resource-intensive, especially for tasks processing contents with inherent logical connections.

% \shil{(I feel the transition is not smooth, we first introducing datasets, and not explaining how they use these datasets to test llm. after these, we can state these methods are not automatic.)}


% To tackle FCH, recent works have developed various techniques for testing and detecting hallucination~\citep{yu2023kola,HaluEval}. The typical and intuitive solution is to develop comprehensive benchmarks for detection. This is done through a process of sampling, filtering, and enhancing ground-truth answers to identify the best and correct answers from given candidates. For example, a well-known hallucination evaluation benchmark HaluEval~\cite{HaluEval} constructs scenarios where LLMs are tested on their ability to select the most factually accurate answers from a set of provided options, with a focus on filtering out hallucinated responses. %\yi{ also talk about the construction of benchmark?}
% Additionally, human annotation plays a critical role in identifying hallucinations in LLM outputs~\cite{Alpaca}. This involves humans determining whether responses contain hallucinated information and considering aspects such as unverifiability, non-factuality, and irrelevance. 



% \lnk{Key challenge: lack of hallucination testing when faced with logic reasoning related problems}
%Bridging the identified research gap in the literature necessitates exploring the inherent challenges faced in detecting FCHs, which are crucial for advancing and enhancing the reliability of LLMs. 

\begin{enumerate}[itemsep=1mm, wide,  labelindent=9pt]
%[itemsep=0ex,leftmargin=0.35cm]
%Challenge\#1: 
%While these benchmarks effectively detect certain hallucinations, they 
\item {\textbf{Automatically constructing and updating benchmark datasets.}} Existing methodologies mainly rely on manually curated benchmarks for detecting specific hallucinations, which fail to encompass the broad and dynamic spectrum of fact-conflicting scenarios in LLMs. 
Meanwhile, due to the ever-evolving nature of knowledge, the need for frequent updates to benchmark data imposes a substantial and continuous maintenance effort.
The reliance on benchmark datasets thus restricts the FCH detection techniques' adaptability, scalability, and  %more importantly, 
detection capability;  
%Challenge\#2:
% in existing test cases. 
\item {\textbf{Efficiently generating FCH test cases.}}
LLMs often answer correctly to simple, straightforward questions due to their extensive training on vast datasets. However, to effectively assess their reasoning capabilities, it is important to generate more complex questions, such as those involving intricate temporal characteristics, that require reasoning rather than just recalling facts. However, constructing such test cases is non-trivial. The challenge lies in designing questions that use familiar knowledge but involve reasoning patterns the LLM may not have been explicitly trained on. Creating such test cases efficiently while ensuring they probe reasoning skills in ways the model has not previously encountered is essential to uncovering latent hallucinations;
% queries that involve temporal concepts, such as ``\emph{Does the human population finally reach six billion by the year 2000?}'' may often be used in applications. However, the correctness of the LLM outputs cannot be guaranteed, potentially leading to misleading information. Currently, there are no satisfactory approaches to automatically verify LLM outputs in such test cases; 
%errors even before the occurrence of large model hallucinations; 
%However, it is known that 
%Another critical issue lies in the verification of temporal logic in existing test cases. 
%It is well known that test cases involving temporal-related questions often face difficulties in automatically verifying the soundness and completeness of these issues. Consequently, the correctness of these test cases cannot be guaranteed, potentially introducing errors even before the occurrence of large model hallucinations;
%Challenge\#3: 
\item {\textbf{Validating the reasoning steps from LLM outputs.}} Even when LLMs finally produce correct answers, the outputs may not indicate an accurate reasoning process, potentially masking false understanding -- a source of FCH. Additionally, the quality of manual validation can differ based on human expertise. As a result, automatically validating reasoning processes, particularly those involving complex logical relationships, is inherently challenging. 
\vspace{1mm}
\end{enumerate}







% \lnk{Key challenge: factual knowledge exploring and new facts generation}
%\yi{we should focus on testing, addressing is a little bit vague.}
% The current research landscape in LLM presents a critical gap in automatically testing FCHs. Predominantly, existing methodologies are anchored to manual benchmarks. %\yi{this sentence is quite chinglish.}
% While these benchmarks are effective in detecting certain types of hallucinations, such as those in Figure~\ref{fig:example1}, they fall short in encompassing the broad and dynamic spectrum of fact-conflicting scenarios inherent to LLMs. %\yi{again, this sentence is not very clear}
% Meanwhile, the need for frequent updates to benchmark data, due to the ever-evolving nature of knowledge, imposes a significant and continuous maintenance effort.
% The reliance on benchmark datasets thus restricts the detection techniques’ adaptability, scalability, and worse, detection capability. 
% From a second perspective, the consistency in the quality of benchmark questions can vary, reflecting the differing levels of experience and skill among the human experts responsible for creating them. This is particularly reflected in the stages such as data labeling and results validation. Additionally, it is important to acknowledge that humans are prone to errors.
% %the scalability and the deof these existing methods are also significantly challenged by their dependency on extensive human intervention, particularly in stages such as data labeling and results validation. %This heavy reliance on manual efforts not only limits the scalability of such approaches but also questions their feasibility in efficiently handling the extensive and intricate datasets characteristic of LLMs.
% Thus, the development of more autonomous, agile, and scalable testing techniques is imperative to effectively identify and tackle FCHs in LLMs.%\yi{in this paper, we focus on testing, but until this paragraph, no terms about ``testing'' explicitly occur.}

% \lnk{Solution to Challenge1: comprehensive logic reasoning based testing framework}

% \lnk{Solution to Challenge2: wiki factual knowledge extraction and prolog rules inference for scalability.}
% \lnk{Key challenge: }

%\textbf{Our Work.}
%To address limitations in the existing techniques, 
%we are the first, to the best of our knowledge, to introduce 
To address the problems outlined above, this paper presents a novel automatic end-to-end metamorphic testing technique based on temporal logic for detecting FCH. To the best of our knowledge, we are the first to create a comprehensive FCH testing framework that utilizes factual knowledge reasoning and metamorphic testing, all seamlessly integrated into the fully automated tool, \tool. 

%\shil{(which four methods?)}
\tool begins by establishing a comprehensive factual knowledge base sourced through crawling information from accessible knowledge bases such as Wikipedia. Each piece of this knowledge acts as a ``seed'' for subsequent transformations. Leveraging logical operators to automatically generate temporal reasoning rules, we transform and augment these seeds and expand factual knowledge into a well-established set of question-answer pairs.
%\yi{into xx}. 
Using the questions and answers in the knowledge set as test cases and ground truth, respectively, we construct a reliable and robust FCH testing benchmark. 


The experiment uses a series of carefully designed template-based prompts to test for FCHs in LLMs. To thoroughly evaluate the reasoning behind the responses, we instruct the LLMs not only to generate answers to the test cases but also to provide detailed justifications for their answers. To reliably identify FCH, we introduce two semantic-aware, similarity-based metamorphic oracles. These oracles extract the key semantic elements from each sentence and map out the logical relationships between them. By comparing the logical and semantic structures of the LLM's responses with the ground truth, the oracles can detect FCH by identifying significant deviations in the LLM's answers from the correct information.




%well-crafted prompts\yi{how prompts generated?} to engage LLMs, testing the alignment of their generated content with our enhanced ground truth. Disparities between LLM outputs and the ground truth signal potential hallucinations. 
%Additionally, in our commitment to fostering collaborative research, we have released our constructed dataset as a benchmark~\cite{drowzee}.

%Our approach directly addresses the need for a comprehensive and flexible testing method by transforming structural factual data into a diverse range of scenarios that LLMs may encounter. This method not only improves the reliability of detection but also enhances its adaptability to various factual contexts.
%Furthermore, we address the scalability challenge by automating the transformation and enlargement of our knowledge base, significantly reducing the dependency on human effort. The well-designed prompts used to test LLMs further streamline the process, making it more efficient in identifying potential hallucinations by comparing LLM outputs with our extended ground truth.

%\textbf{Results and Findings.}
%In evaluating our proposed FCH testing framework and \tool, 
%we undertake 
%to evaluate their effectiveness 
We demonstrate the effectiveness of our approach through comprehensive experiments in multiple contexts. First, our evaluation involves deploying \tool across a wide range of topics drawn from a diverse selection of Wikipedia articles. Second, we test our framework on various open-source and commercial LLMs, thoroughly assessing its applicability and performance across different model architectures. 
Our key findings indicate that \tool succeeds in automatically generating practical test cases and identifying hallucination issues of nine LLMs across nine domains. 
Using these test sets, our experiments show that the rate of hallucination responses produced by various LLMs ranges from 24.7\% to 59.8\% for cases unrelated to temporal reasoning and 16.7\% to 39.2\% for cases requiring temporal reasoning. 
%\shil{shall we differentiate the number for non-temporal and temporal one?}.  
We then categorize these hallucination responses into \emph{erroneous knowledge hallucination} and \emph{erroneous inference hallucination}. 
%\syh{four types?}. 
Through an in-depth analysis, we unveil that the lack of logical reasoning capabilities contributes the most to the FCH issues in LLMs. 
Additionally, we observe that LLMs are particularly prone to generating hallucinations in test cases involving temporal concepts and out-of-distribution knowledge. 
Such an evaluation demonstrates that the 
%Furthermore, we confirm that 
test cases generated using %our 
logical reasoning rules can effectively trigger and detect LLM hallucinations.  %issues in . 


This paper builds upon the earlier version~\cite{DBLP:journals/pacmpl/LiL0SW024} by incorporating hallucination detection through temporal-logic-guided test generation. It includes additional motivational examples (\secref{sec:motivating}), a comprehensive set of reasoning rules for encoding \emph{Metric Temporal Logic} (MTL)~\cite{DBLP:conf/lics/OuaknineW05} formulae (\secref{sec:encoding_MTL}) and automatically generating temporal-logic-related question-answer pairs (\secref{prompt}), and further experimental studies (the {RQ4} at \secref{sec:eval}) that detect hallucinations due to insufficient temporal reasoning capabilities. The main contributions of this work are summarized as follows: 
%We summarize the main contributions of this paper below:
\begin{itemize}[itemsep=1mm,leftmargin=0.35cm]
\item 
%Development of 
\textbf{A novel FCH testing framework.} 
To the best of our knowledge, 
we are the first to develop a novel testing framework based on logic programming and metamorphic testing to automatically detect FCH issues in LLMs. %\yi{hanging sentence}This framework represents a significant advancement over current methodologies, providing a more systematic, comprehensive approach to detection.
%Construction and Release of
\item \textbf{An extensive benchmark based on factual knowledge.} 
To facilitate collaborative efforts and future advances in identifying FCH, 
the source code of \tool and constructed benchmark dataset are publicly available  \cite{drowzee}. 
\item \textbf{Test generation via temporal reasoning.} 
Our tool automatically generates test cases that provide a more comprehensive evaluation of LLMs in handling reasoning tasks and identifying factual inconsistencies. By applying temporal logic-based reasoning rules, we expand the initial seed data from our knowledge base, enhancing the diversity and complexity of the test scenarios. 

\item \textbf{Semantic-aware oracles for LLM answer validation.} We propose and implement two automated verification mechanisms, i.e., the oracles, that analyze the semantic structure similarity between sentences. These oracles are designed to validate the reasoning logic behind the answers generated by LLMs, hereby reliably detecting the occurrence of FCHs. 

\end{itemize}



\section{Related Work} \label{related}




% \subsection{Benchmarks in Coding Scenarios}
% \begin{enumerate}
%     \item Code Generation
%     \item Bug Fixing
% \end{enumerate}

% \subsection{Large Language Model Agents}

% At the heart of the LLM Agent is an Agent Core, which coordinates the core \textit{logic} and \textit{behavioral} characteristics of the agent. In addition, the Agent includes the following key components:

% \begin{itemize}
%     \item Memory Module: It consists of both short-term and long-term memory components that record the agent's internal logs and interactions with the user.
%     \item Tools: These are the tools that the agent can use to perform tasks, usually specific third-party APIs.
%     \item Planning Module: This is used for solving complex problems, such as decomposing tasks and problems, reflexivity or critique.
% \end{itemize}

% \subsection{Multi Agent Collaboration Framework}

% MetaGPT \url{https://arxiv.org/abs/2308.00352}


\parabf{Coding \llm{s}.}
Large Language Models (\llm{s}) have become the go-to solution for a wide array of coding tasks due to their exceptional performance in both code generation and comprehension~\cite{codex}. These models have been successfully applied to various software engineering activities, including program synthesis~\cite{patton2024programming, codex, li2022competition, iyer2018mapping}, code translation~\cite{pan2024lost, roziere2020unsupervised, roziere2021leveraging}, program repair~\cite{xia2023repairstudy, chatrepair, monperrus2018living, bouzenia2024repairagent}, and test generation~\cite{titanfuzz, fuzz4all, deng2023fuzzgpt, lemieux2023codamosa, kang2023testing}. Beyond general-purpose \llm{s}, specialized models have been developed by further training on extensive datasets of open-source code snippets. Notable examples of these code-specific \llm{s} include \codex~\cite{codex}, \codellama~\cite{codellama}, StarCoder~\cite{starcoder,starcodertwo}, and \deepseek~\cite{deepseek}. Additionally, instruction-following code models have emerged, refined through instruction-tuning techniques. These include models such as \codellamainstruct~\cite{codellama}, \deepseekinstruct~\cite{deepseek}, \wizardcoder~\cite{wizardcoder}, \magicoder~\cite{magicoder}, and OpenCodeInterpreter~\cite{zheng2024opencodeinterpreter}.

\parabf{Benchmarking \llm-based coding tasks.}
To assess the capabilities of \llm{s} in coding, a variety of benchmarks have been proposed. Among the most widely utilized are \humaneval~\cite{codex} and \mbpp~\cite{austin2021program}, which are handcrafted benchmarks for code generation that include test cases to validate the correctness of \llm outputs. Other benchmarks have been developed to offer more rigorous tests~\cite{evalplus}, cover additional programming languages~\cite{zheng2023codegeex,cassano2023multipl}, and address different programming domains~\cite{livecodebench, hendrycksapps2021, codecontest, ds1000, arcade}.

More recently, research has shifted towards evaluating \llm{s} on real-world software engineering challenges by operating on entire code repositories rather than isolated coding problems~\cite{swebench, zhang2023repocoder, liu2023repobench}. A notable benchmark in this area is \swebench~\cite{swebench}, which includes tasks requiring repository modifications to resolve actual GitHub issues. The authors of \swebench have also released a more focused subset, \swebenchlite~\cite{swebenchlite}, which contains 300 problems centered on bug fixing that only involves single-file modifications in the ground truth patches. ML-Bench \cite{liu2023mlbench} is a benchmark for evaluating large language models and agents for Machine Learning tasks on reporitory-level code. It involves 18 repositories and focuses on code generation and interactions with Jupyter Notebooks.

\parabf{Repository-level coding.}
The rise of agent-based frameworks~\cite{xi2023rise} has spurred the development of agent-based approaches to software engineering tasks. Devin~\cite{devinwebpage} (and its open-source counterpart OpenDevin~\cite{opendevin}) is among the first comprehensive \llm agent-based frameworks. Devin employs agents to first perform task planning based on user requirements, then allows them to use tools like file editors, terminals, and web search engines to iteratively execute the tasks. \sweagent~\cite{sweagent} introduces a custom agent-computer interface (ACI), enabling the \llm agent to interact with the repository environment through actions like reading and editing files or running bash commands. Another agent-based approach, \autocoderover~\cite{autocoderover}, equips the \llm agent with specific APIs (e.g., searching for methods within certain classes) to effectively identify the necessary modifications for issue resolution. Beside these examples, a variety of other agent-based approaches have been developed in both open-source~\cite{aidar} and commercial products~\cite{bouzenia2024repairagent, coder, repounderstander, lingma, factorydroid, ibmagent, opencsgstarship, marscode, amazonqdeveloper}.

% Unlike these agent-based methods, \tech offers a straightforward and cost-efficient solution for addressing real-world software engineering challenges. Our work is the first to demonstrate that an \emph{agentless} approach can achieve comparable performance without the need for complex tools or modeling intricate environment behavior and feedback.

Unlike existing benchmarks and agent-based frameworks, which focus on the code generation/completion tasks, our proposed \model and \agent focus on the code deployment task, which is under-studied in the field.

% 加一节,怎么区分确定性和不确定性

\section{Methodology}


To achieve effective probabilistic predictions, we propose CoST that simultaneously leverages the advantages of both deterministic and probabilistic models. Our approach involves two stages. In the first stage, the deterministic model is pretrained to predict the conditional mean that captures the primary patterns. In the second stage, the parameters of the deterministic model are frozen, and the scale-aware diffusion model, constrained by a customized fluctuation scale, is jointly trained to model residual distributions that reflect random fluctuations.   
Figure~\ref{fig:model} illustrates an overview of our model.


\subsection{Mean-Residual Decomposition}

For urban spatiotemporal probabilistic prediction, current approaches typically employ a single probabilistic model to capture the full distribution of data, incorporating both the primary spatiotemporal patterns and the random fluctuations. However, it is challenging to model both of these distributions. Inspired by~\cite{mardani2023residual} and the Reynolds decomposition in fluid dynamics~\cite{pope2001turbulent}, we propose to decompose the target data \(\mathbf{x}^{ta}\) as follows:
\begin{equation}
 \mathbf{x}^{ta} = \underbrace{\mathbb{E}[\mathbf{x}^{ta}|\mathbf{x}^{co}]}_{\substack{:=\boldsymbol{\mu}(Deterministic)}} + \underbrace{(\mathbf{x}^{ta}-\mathbb{E}[\mathbf{x}^{ta}|\mathbf{x}^{co}])}_{\substack{:=\mathbf{r}(Probabilistic)}},
\end{equation}
where \(\boldsymbol{\mu}\) is the conditional mean representing the primary patterns, and \(\mathbf{r}\) is the residual representing the random variations. Our core idea is that if a deterministic model can accurately predict the conditional mean, that is, \(\boldsymbol{\mu}\approx\mathbb{E}_{\theta}[\mathbf{x}^{ta}|\mathbf{x}]\), then the probabilistic model only needs to focus on learning the simpler residual distribution, thus combining the strengths of both models to enhance the probabilistic prediction capability.









\subsection{Mean Prediction via Deterministic Model}

We require a deterministic model that can accurately predict the conditional mean to align with our research hypothesis, and also maintain high predictive efficiency to avoid additional increases in training and inference time. Therefore, we select the MLP-based STID model as our mean prediction module.
In the first stage of training, we pretrain the model for 50 epochs to effectively capture the primary spatiotemporal patterns. Specifically, we input historical conditional data \(\mathbf{x}^{co}\) into the STID model to obtain the conditional mean output \(\mathbb{E}_{\theta}[\mathbf{x}^{ta}|\mathbf{x}^{co}]\).

The STID model is pretrained by optimizing the following loss function:

\begin{equation}
\label{eq:loss2}
   \mathcal{L}_{2}  = \left\| \mathbb{E}_{\theta}[\mathbf{x}^{ta}|\mathbf{x}^{co}] - \mathbf{x}^{ta} \right\|_2^2 .
\end{equation}

\subsection{Residual Learning via Diffusion Model}
Diffusion models have achieved significant success in probabilistic modeling. In this work, we employ a diffusion model for probabilistic prediction, training it to learn the residual distribution:
\begin{equation}
\label{eq:one-setp-forward}
    \mathbf{r}_{ta}=\mathbf{x}^{ta}-\mathbb{E}_{\theta}[\mathbf{x}^{ta}|\mathbf{x}^{co}].
\end{equation}
Consequently, the target data \(\mathbf{x}^{ta}\) for diffusion models in Eqs.~\eqref{eq:one-setp-forward}, \eqref{eq:inference}, and \eqref{eq:loss1} is replaced by \(\mathbf{r}_{ta}\).
The residual distribution of urban spatiotemporal data is not independently and identically distributed (i.i.d.) nor does it follow a fixed distribution, such as \(\mathcal{N}(0, \mathbf{\sigma})\). Instead, it often exhibits complex spatiotemporal dependence and heterogeneity. So we consider both temporal residual learning and spatial residual learning. 




\subsubsection*{\textbf{Temporal Residual Learning.}} 
For urban spatiotemporal data, the residual distribution varies at different time points. For example, fluctuations are lower at night and higher during the day, with uncertainty varying between weekends and weekdays. To model this, we incorporate the timestamp information as the condition for the denoising process. Meanwhile, the residual distribution can also be affected by sudden weather changes or public events. To capture these real-time changes, we concatenate the context data $\mathbf{x}^{co}_0$ with noised target residual $\mathbf{r}^{ta}_n$ as the input. The noise is not added to $\mathbf{x}^{co}_0$ and $\mathbf{r}^{ta}_n$ during the diffusion training and inference.




\subsubsection*{\textbf{Spatial Residual Learning.}}
In areas with frequent traffic accidents, fluctuations tend to be more pronounced and may induce anomalous variations in adjacent regions, thus affecting their distributions.
For spatial dependence modeling, the model learns a spatial embedding for each location, following the STID approach. Additionally, we propose a scale-aware diffusion process to further distinguish the heterogeneity for different regions. In this section, we detail the calculation of \(\mathbf{Q}\) and how it is integrated into the scale-aware diffusion process.

\noindent\textbf{(i) Customized Fluctuation Scale.} Specifically, we apply the Fast Fourier Transform (FFT) to spatiotemporal sequences in the training set to quantify fluctuation levels in different regions and use the custom scale \(\mathbf{Q}\) as input to account for spatial differences in residual. Specifically, we first employ FFT to extract the fluctuation components for each region within the training set. The detailed steps are as follows:









\begin{equation}
    \begin{aligned}
    & \mathbf{A}_{\mathrm{k}} = \left| \text{FFT}(\mathbf{x})_\mathrm{k} \right|, \quad \mathbf{{\phi}}_{\mathrm{k}} = \mathbf{\phi} \left( \text{FFT}(\mathbf{x})_\mathrm{k} \right), \\
    & \mathbf{A}_{\text{max}}=\max_{\mathrm{k}\in\left\{1,\cdots,\left\lfloor\frac{\mathbf{L}}{2}\right\rfloor + 1\right\}}\mathbf{A}_{\mathrm{k}}, \\
    & \mathcal{K} = \left\{ \mathrm{k} \in \left\{ 1, \cdots, \left\lfloor \frac{{L}}{2} \right\rfloor + 1 \right\} : \mathbf{A}_{\mathrm{k}} < 0.1 \times \mathbf{A}_{\text{max}} \right\}, \\
    & \mathbf{x}_{\mathbf{r}}[i] = \sum_{\mathrm{k} \in \mathcal{K}} \mathbf{A}_{\mathrm{k}} \Big[ \cos \left( 2\pi \mathbf{f}_{\mathrm{k}} i + \mathbf{\phi}_{\mathrm{k}} \right) \\
    & \qquad \qquad + \cos \left( 2\pi \bar{\mathbf{f}}_{\mathrm{k}} i + \bar{\mathbf{\phi}}_{\mathrm{k}} \right) \Big],
    \end{aligned}
\end{equation}
where \(\mathbf{A}_{\mathrm{k}},\mathbf{\phi}_{\mathrm{k}}\) reprent the amplitude and phase of the $\mathrm{k}-$th frequency component. $L$ is the temporal length of the training set. \(\mathbf{A}_{\text{max}}\) is the maximum amplitude among the components, obtained using the $\max$ operator. $\mathcal{K}$ represents the set of indices for the selected residual components. \(\mathbf{f}_{\mathrm{k}}\) is the frequency of the \(\mathrm{k}\)-th component. $\bar{\mathbf{f}}_{\mathrm{k}}, \bar{\mathbf{\phi}}_{\mathrm{k}}$ represent the conjugate components. \(\mathbf{x}_{\mathbf{r}}\) ref to the extracted residual component of the training set. We then compute the variance $\sigma^2_k$ of the residual sequence for each location $k$ and expand it to match the shape as 
\(\mathbf{r}^{ta}_0 \in \mathbb{R}^{B \times K \times P}\) , where $B$ represents the batch size. And we can get the variance tensor \(\mathcal{M}\): 
\begin{equation}
\begin{aligned}
    &\mathcal{M}_{b,k,p}=\sigma_{k}^2,\\
&\forall b\in\{1,\cdots,B\}, \forall k\in\{1,\cdots,K\}, \forall p\in\{1,\cdots,P\}.
\end{aligned}
\end{equation}
The residual fluctuations are bidirectional, encompassing both positive and negative variations, so we generate a random sign tensor \(\mathbf{S}\in\mathbb{R}^{B\times K\times P}\) for \(\mathcal{M}\), where each element \(S_{b,k,p}\) of \(\mathbf{S}\) is sampled from a Bernoulli distribution with \(p = 0.5\). 
%That is, \(r_{b,k,p}\) takes the value $1$ with probability $0.5$ and $-1$ with probability $0.5$. 
The customized fluctuation scale \(\mathbf{Q}\) is then defined as:
\begin{equation}
\begin{aligned}
&\mathbf{Q}_{b,k,p}=S_{b,k,p}\times\mathcal{M}_{b,k,p},\\
&\forall b\in\{1,\cdots,B\}, \forall k\in\{1,\cdots,K\}, \forall p\in\{1,\cdots,P\}.
\end{aligned}
\end{equation}
Then \(\mathbf{Q}\) is used as the input of the denoising network. 





\noindent\textbf{(ii) Scale-aware Diffusion Process.}

The vanilla diffusion models typically model all regions as the same \(\mathcal{N}(0, I)\) distribution at the end of the diffusion process, failing to distinguish the differences among regions. To further model the differences of residual distribution across various regions, we adopt the technique proposed by~\cite{han2022card} to make the residual learning region-specific conditioned on \({\mathbf{Q}}\). Specially, we have calculated the customized fluctuation scale \({\mathbf{Q}}\), and We redefined the noise distribution at the endpoint of the diffusion process as follows:
\begin{equation}
    p(\mathbf{r}^{ta}_N)=\mathcal{N}({\mathbf{Q}},I),
\end{equation}
Accordingly, the Eq~\ref{eq:new one-step} in the forward process is rewritten as:
\begin{equation}
\label{eq:new one-step}
    \mathbf{r}_n^{ta} = \sqrt{\bar{\alpha}_n} \mathbf{r}_0^{ta}+(1-\sqrt{\bar{\alpha}_n})\mathbf{Q} + \sqrt{1 - \bar{\alpha}_n} \mathbf{\epsilon}, \quad \mathbf{\epsilon} \sim \mathcal{N}(0, I).
\end{equation}
And in the denoising process, we sample \(\mathbf{r}_N^{ta}\) from $\mathcal{N}({\mathbf{Q}},I)$, and denoise it use Eq~(\ref{eq:inference}), the computation of \(\mu_{\theta}(\mathbf{r}_n^{ta}, n| \mathbf{x}_0^{co})\) in Eq~\ref{eq:inference} is modified as:
\begin{equation}
\label{eq: mu}
    \mu_{\theta}(\mathbf{r}_n^{ta}, n| \mathbf{x}_0^{co})=\frac{1}{\sqrt{\bar{\alpha}_n}} \left( \mathbf{r}_n^{ta} - \frac{\beta_n}{\sqrt{1 - \bar{\alpha}_n}} \mathbf{\epsilon}_{\theta}(\mathbf{r}_n^{ta}, n| \mathbf{x}_0^{co}) \right)+(1-\frac{1}{\sqrt{\bar{\alpha}_n}})\mathbf{Q}.
\end{equation}
This approach enables the diffusion process to be governed by the \(\mathbf{Q}\) values at each region, leading to more effective utilization of the customized scale conditions.


\subsection{Training and Inference}
\begin{algorithm}
\caption{\methodname{} Training}
\KwIn{Coarse-to-fine Autoencoder $\text{Enc}$, $\text{Dec}$}
\KwOut{}
\For{$i \gets 1$ \textbf{to} $n-1$}{
    \For{$j \gets 1$ \textbf{to} $n-i$}{
        \If{$L[j] > L[j+1]$}{
            Swap $L[j]$ and $L[j+1]$
        }
    }
}
\Return $L$
\end{algorithm}
\begin{algorithm}[!t]
\caption{Inference}
\label{al: sampling}
\begin{algorithmic}[1]
    \State \textbf{Input:} Context data $\mathbf{x}_0^{co}$, customized fluctuation scale $\mathbf{Q}$, trained diffusion model $\epsilon_{\theta}$, trained deterministic model $\mathbb{E}_{\theta}$
    \State \textbf{Output:} Target data $\mathbf{x}_0^{ta}$
    \State Estimate the conditional mean \(\mathbb{E}_{\theta}[\mathbf{x}^{ta}_0|\mathbf{x}^{co}_0]\)
    \State Sample $\mathbf{r}_N^{ta}$ from $\epsilon \sim \mathcal{N}(\mathbf{S},I)$
    \For{$n = N$ to $1$} 
        \State Estimate the noise $\mathbf{\epsilon}_{\theta}(\mathbf{r}_n^{ta}, n| \mathbf{x}_0^{co})$
        \State Calculate the $\mu_{\theta}(\mathbf{r}_n^{ta}, n| \mathbf{x}_0^{co})$ using Eq.~(\ref{eq: mu})
        \State Sample $\mathbf{r}_{n-1}^{ta}$ using Eq.~(\ref{eq:inference})
    \EndFor
    \State \textbf{Return:} $\mathbf{x}_0^{ta}=\mathbb{E}_{\theta}[\mathbf{x}^{ta}_0|\mathbf{x}^{co}_0]+\mathbf{r}_0^{ta}$
\end{algorithmic}

\end{algorithm}

\subsubsection*{\textbf{Training}}
Our training process is a two-stage procedure. We first pretrain the deterministic model STID to enable it to predict the conditional mean. Subsequently, we train the diffusion mode to learn the distribution of residuals, where the residuals are calculated as the difference between the true values and the conditional mean predicted by the pretrained STID model with frozen parameters. The detailed training procedure is presented in Algorithm~\ref{al: train}.
\subsubsection*{\textbf{Inference}}
The inference process of the model consists of two paths: one utilizes the pretrained STID model to predict the conditional mean, while the other uses the diffusion model to predict the residuals. The final sample is obtained by summing the results of both paths. This process is detailed in Algorithm~\ref{al: sampling}.


\section{Experiments}
In this section, we will give a detailed experimental analysis of the whole framework, including the datasets, evaluation protocol, implementation details, comparisons with the state-of-the-art methods, and ablation analysis.

\subsection{Experimental Setup}
\noindent \textbf{Datasets.}
To verify the effectiveness and efficiency of our method, we have conducted comprehensive experiments on the RefCOCO \cite{yu2016modeling}, RefCOCO+ \cite{yu2016modeling}, RefCOCOg \cite{mao2016refcocogg,nagaraja2016refcocogu} and Flickr30K Entities \cite{plummer2015flickr30k} datasets, all of which are widely used as benchmarks for visual grounding.

\begin{itemize}
  \item \textbf{RefCOCO} features 19,994 images with 50,000 referred objects and 142,210 expressions. The dataset is divided into four subsets, consisting of 120,624 train, 10,834 validation, 5,657 test A, and 5,095 test B samples, respectively. The average length of the expressions is 3.6 words, and each image contains a minimum of two objects.

 \item
\textbf{RefCOCO+} with similar content but richer expressions, includes 19,992 images with 49,856 referred objects and 141,564 referring expressions. The dataset is divided into four subsets: 120,624 train, 10,758 validation, 5,726 test A, and 4,889 test B samples. Notably, the RefCOCO+ dataset has been constructed to be more challenging than the RefCOCO dataset by excluding certain types of absolute-location words. The average length of the expressions is 3.5 words, including the attribute and location of referents.

 \item
\textbf{RefCOCOg} , unique for its detailed annotations and longer referential expressions, contains 25,799 images with 49,856 objects. There are two commonly used split protocols for this dataset. One is RefCOCOg-google \cite{mao2016refcocogg}, and the other is RefCOCOg-umd \cite{nagaraja2016refcocogu}. We report our performance on both RefCOCOg-google (val-g) and RefCOCOg-umd (val-u and test-u) to make comprehensive comparisons. The average length of expressions within the dataset is 8.4 words, including both the attributes and the locations of the referents. This rich detail description facilitates a more nuanced understanding of the visual grounding tasks, as it captures the intricacies of how objects are referenced in various contexts.

 \item
\textbf{Flickr30K Entities} \cite{plummer2015flickr30k}, is an enhanced version of the original Flickr30K \cite{young2014image}, fortified with the addition of short region phrase correspondence annotations. This expansion yields a collection of 31,783 images, encompassing 427,000 referred entities. Following the previous studies \cite{xiao2024hivg,wang2023cogvlm}, we have divided the dataset into 29,783 images for training, 1,000 for validation, and another 1,000 for testing purposes.
\end{itemize}



\noindent
\textbf{Evaluation Metrics.} We follow the previous research that employs top-1 accuracy (\%) as the evaluation metric for visual grounding. Specifically, a prediction is deemed accurate only when its Intersection-over-Union (IoU) exceeds or equals 0.5. In addition to Precision@0.5, we also report the number of tunable parameters in the pre-trained encoders to compare the fine-tuning efficiency with traditional full fine-tuning and other PETL methods.

\begin{table*}[!t]
\centering
\caption{Comparison with PETL methods using the same Backbone as SwimVG on RefCOCO, RefCOCO+ and RefCOCOg. ``Param.'' indicates the number of tunable parameters in the pre-trained encoders.}
\vspace{-3mm}
\label{Table:comparisons with PETL}
\small
\setlength{\tabcolsep}{4.5pt}


\begin{tabular}{l|c|ccc|ccc|ccc}
\toprule
    
\multirow{2}{*}{Methods} &\multicolumn{1}{c|}{\multirow{2}{*}{Venue}}  & \multicolumn{3}{c|}{RefCOCO} & \multicolumn{3}{c|}{RefCOCO+} & \multicolumn{3}{c}{RefCOCOg} \\

 && val & testA & testB & val & testA & testB & val-g & val-u & test-u \\ \midrule



AdaptFormer \cite{chen2022adaptformer} &NeurIPS'22 & 81.75 & 83.14 & 76.73 & 72.05 & 76.61 & 64.26 & 70.19 & 70.93 & 72.36 \\


LoRA \cite{hu2021lora}  & ICLR'22  & 82.43 & 84.51 & 77.32 & 72.66 & 77.13 & 64.85 & 71.27 & 72.16 & 73.23 \\



UniAdapter \cite{lu2024uniadapter}& ICLR'24 & 85.76 & 88.31 & 81.84 & 74.95 & 78.75 & 65.97 & 73.68 & 74.72 & 74.98 \\

DAPT \cite{zhou2024dynamic} &CVPR'24 & 85.33  & 87.52 & 81.06 & 74.33 & 78.66& 65.54 & 74.02 & 75.26 & 75.47   \\ 



\midrule
\textbf{SwimVG}  &-& \textbf{88.29} & \textbf{90.37} & \multicolumn{1}{c|}{\textbf{84.89}} &  \textbf{77.92}& \textbf{83.22} & \multicolumn{1}{c|}{\textbf{69.95}}& \textbf{79.10} & \textbf{80.14} & \textbf{79.69}  \\
\bottomrule

\end{tabular}

\end{table*}


\noindent \textbf{Implementation Details.} The vision encoder is initialized with DINOv2-L/14~\cite{oquab2023dinov2}, while the language encoder uses CLIP-B~\cite{radford2021learning}. The resolution of the input image is 224×224. The DINOv2-L/14 model processes tokens with a feature dimension of 768, while 
and the CLIP-B model handles tokens with a feature dimension of 512. All prompts use Xavier initialization, and all adapters are initialized with Kaiming normal initialization. The bottleneck dimension $C_d$ for both CIA and domain-specific adapters is 56, and more dimension comparisons can be seen in Table \ref{Table:neck}. The batchsize for training is 32. For fair comparisons, other PETL methods in Tab. \ref{Table:comparisons with PETL} use the same base architecture and original hyperparameters, and keeping the vision and language encoder frozen. For RefCOCO \cite{yu2016refcoco}, RefCOCOg \cite{mao2016refcocogg,nagaraja2016refcocogu}, and Flickr30K Entities \cite{plummer2015flickr30k} datasets, the entire network is trained for 65 epochs using the AdamW optimizer. While for RefCOCO+ \cite{yu2016refcoco} dataset, the network is trained for 90 epochs. Note that most mainstream methods train RefCOCO/RefCOCOg/Flickr30K Entities for 90 epochs and RefCOCO+ for 180 epochs, which demonstrates the higher efficiency of our SwimVG. We conduct all experiments on one A800 GPU.




\subsection{Main Results}
We compare our SwimVG comprehensively with a series of previous visual grounding (VG) methods. The main experimental results are displayed in Tab. \ref{Table:comparisons with SOTA}. We can notice from these results that SwimVG reaches the best accuracy and also ensures parameter efficiency compared with all other methods, which validates its effectiveness and efficiency.

\noindent
\textbf{Effectiveness.} As Tab. \ref{Table:comparisons with SOTA} shown, on the three commonly challenging benchmarks, SwimVG outperforms all traditional full fine-tuning methods. Compared to DARA~\cite{liu2024dara}, a parameter-efficient transfer learning method, we achieves an average accuracy improvement of 10.85\% on the three benchmarks. Notably, even compared to some methods that are pre-trained on the the RefCOCO/+/g and Flickr30K Entities (indicated by $\dagger$ in Tab. \ref{Table:comparisons with SOTA}), our SwimVG model achieves the highest scores across all evaluation tasks, with particularly strong performance on the RefCOCO+, which present greater challenges compared to RefCOCO.




\noindent
\textbf{Efficiency.} Tab. \ref{Table:comparisons with SOTA} clearly illustrates that SwimVG not only achieves the best performance, but also highlights its huge advantages in parameter efficiency. SwimVG reduced the tunable backbone parameters by 97.96\% compared to the traditional full fine tuning method. In order to verify more efficient aspects such as training and inference time, experimental results on the mainstream methods using the conventional VL transformer, and the other PETL methods are shown in Tab. \ref{tab:cost}. It can be seen that SwimVG achieves significant energy efficiency advantages.

\setlength{\tabcolsep}{3pt}
\begin{table}
\centering
\caption{Efficiency of Different Methods}
\vspace{-0.1in}
\label{tab:efficiency}
% \resizebox{\textwidth}{!}{%
\scalebox{0.65}{%
\begin{tabular}{|c|ccc|ccc|ccc|}
  \hline
  \multirow{2}{*}{} & \multicolumn{3}{c|}{\textbf{memory size (MB)}} & \multicolumn{3}{c|}{\textbf{training time (s)}} & \multicolumn{3}{c|}{\textbf{matching time (s)}}\\
  % \cline{2-7}
  % {} & MByte & seconds/ep & seconds\\
  {} & \textbf{Beijing} & \textbf{Porto} & \textbf{Chengdu} & \textbf{Beijing} & \textbf{Porto} & \textbf{Chengdu} & \textbf{Beijing} & \textbf{Porto} & \textbf{Chengdu}\\
  % \cline{2-7}
  % {} & (MByte) & (minutes/ep) & (seconds/K) & (MByte) & (minutes/ep) & (seconds/K)\\
  \hline
  \textbf{MDP} & 1819MB & 2039MB & 2122MB & - & - & - & 389.14s & 361.15s & 599.51s  \\
  \textbf{HMM} & 1209MB & 1388MB & 1361MB & - & - & - & 427.97s & 380.05s & 589.08s \\
  % \hline
  \textbf{FMM} & 897MB & 931MB & 981MB & - & - & - & 1.13s & 1.02s & 1.87s \\
  % \hline
  \textbf{AMM} & 957MB & 1013MB & 1124MB & - & - & - & 3.42s & 3.05s & 5.16s \\
  % \hline
  \textbf{MTrajRec} & 9045MB & 12428MB & 11265MB & 182.4s & 2200.2s & 25672.4s & 51.22s & 42.27s & 73.68s\\
  % \hline
  \textbf{L2MM} & 9087MB & 11875MB & 10898MB & 189.1s & 2314.2s & 27032.2s & 6.71s & 5.26s & 9.10s\\
  % \hline
  \textbf{GraphMM} & 8537MB & 11752MB & 10378MB & 48.4s & 645.2s & 7311.4s & 8.06s & 6.96s & 11.18s\\
  % \hline
  \textbf{\modelName} & 2530MB & 2299MB & 2357MB & 11.9s & 126.4s & 1507.8s & 1.09s & 0.95s & 1.65s\\
  \hline
\end{tabular}}
\vspace{-0.15in}
\end{table}


\subsection{Comparison with Other PETL Methods}

\textbf{Details of Baseline PETL Methods.}

This section furnishes additional details of the PETL baselines employed in our primary manuscript. Notably, all these baselines follow the same base architecture.
\begin{itemize}

    \item \textbf{AdaptFormer \cite{chen2022adaptformer}:} We add adapters in parallel to MHA and FFN in both Vision Encoder and Language Encoder. Following the original work, we set the same bottleneck dimensions of AdaptFormer for both vision and language branch.
    
    \item \textbf{LoRA \cite{hu2021lora}:} We incorporate trainable matrices in parallel to the weight matrices in MHA and FFN in both Vision Encoder and Language Encoder. 
    We have set the same bottleneck dimensions for both the vision and language branches of LoRA, following the original setup.

    \item \textbf{UniAdapter \cite{lu2024uniadapter}:} We add UniAdapter  in both Vision Encoder and Language Encoder, according to their basic designs. 

    \item \textbf{DAPT \cite{zhou2024dynamic}:} We insert Dynamic Adapters in paralle to the weight matrices in MHA and FFN in both Vision Encoder and Language Encoder, and use their task-agnostic feature transform strategy. Other sets such as bottleneck dimensions are same as the DAPT.

   

    %\item \textbf{DARA \cite{liu2024dara}:} We add DA Adapters after the MHA, and RA Adapters in parallel to FFN in the Vision Encoder and Language Encoder, according to their basic designs. We set the bottleneck dimensions of DA and RA Adapters to 128, and the weight-sharing dimensions of RA Adapters to 256.
    
\end{itemize}

We conduct experiments comparing our SwimVG with other parameter-efficient transfer learning (PETL) methods. To ensure fairness, we retain the original parameter settings from previous methods. As these PETL methods lack the capability of multimodal fusion, we complement them with the traditional VL transformer for cross-modal understanding, thereby enabling a direct comparison with our SwimVG. Tab. \ref{Table:comparisons with PETL} illustrates that SwimVG outperforms other PETL methods on all three benchmarks. Through introducing step-wise multimodal prompts and cross-modal interactive adapters, SwimVG enhances the modeling of the vision-text alignment capability. Previous PETL methods lack this ability, rendering them less effective for VG tasks. This also proves that the multimodal fusion mechanism in SwimVG is more efficient than the VL transformer. To summarize, by the specific design for the VG domain, SwimVG achieves superior performance with only \textbf{2.04} \% tunable parameters.

\begin{figure*}[t]
\centering
\includegraphics[width=0.98\textwidth]{./figures/visual.pdf}
% \vspace{-1mm}
\caption{Visualizations of attention maps, prediction results (yellow bounding boxes) and ground truth (red bounding boxes).}
% \vspace{-4mm}
\label{fig:visualizations}
\end{figure*}


\subsection{Convergence Analysis}
Figure \ref{other-convergence} shows a comparison of the convergence epoch between SwimVG and other models. It is observed that DARA and TransVG converge around epoch 85, while CLIP-VG converges at approximately epoch 105. In contrast, SwimVG achieves convergence at around epoch 65. This demonstrates the efficiency of our method, as fewer training epochs are required, thereby reducing training costs. In addtion, we have also visualized the convergence comparison of SwimVG across the RefCOCO, RefCOCOg-u, RefCOCOg-g, and Flicker 30K datasets. Figure \ref{self-conver} indicates that convergence is achieved around epoch 65 for all these datasets.




\begin{figure}
\centering
\includegraphics[width=1\linewidth]{./figures/other-comp.png}
\vspace{-4mm}
\caption{The convergence comparison between SwimVG and other SOTA models on RefCOCO.}
\vspace{-3mm}
\label{other-convergence}
\end{figure}


\begin{figure}
\centering
\includegraphics[width=1\linewidth]{./figures/self-comp.png}
\vspace{-4mm}
\caption{The convergence comparison of SwimVG on RefCOCO, RefCOCOg and Flicker 30K datasets.}
\vspace{-3mm}
\label{self-conver}
\end{figure}

\subsection{Ablation Study}




\noindent
\textbf{Effectiveness of Multimodal Interaction in SwimVG.} We assess the impact of step-wise multimodal prompts (Swip) and cross-modal interactive adapters (CIA) by performing an ablation study, and report the results on RefCOCOg-u validation and test datasets. Considering the substantial number of parameters occupied by the encoder, we freeze all the encoder parameters during fine-tuning for efficiency. From Tab. \ref{Table:swip-cia}, it is evident that only introducing the Swip yields a ideal results (Tab. \ref{Table:swip-cia} (a)). Only by using the CIA for cross-modal fusion can achieve better results (Tab. \ref{Table:swip-cia} (b)). Compared with the previous methods using the traditional vision-language encoder, such as TransVG \cite{deng2021transvg}, DARA \cite{liu2024dara} in Tab. \ref{Table:comparisons with SOTA}, it shows that we can achieve the better results using only Swip or CIA. Tab. \ref{Table:swip-cia} (c) indicates that incorporating Swip and CIA for multimodal fusion results in an average improvement of 3.49\% across the RefCOCOg-u, achieving the best performance among these ablation variants. Swip achieves progressive multimodal fusion by gradually introducing linguistic information, while CIA explores deeper correlations by enhancing cross-modal interaction. Combining the two can simultaneously promote multimodal fusion in terms of breadth and depth.

\begin{table}[t]
\centering
\renewcommand{\arraystretch}{1.5}
\caption{Ablations of multimodal interaction in SwimVG on RefCOCOg-u \cite{yu2016modeling} dataset. Note that the visual and text encoder are frozen in the ablation studies.}
\vspace{-2mm}
\label{Table:swip-cia}
\small
\setlength{\tabcolsep}{4pt}
\begin{tabular}{ccc|ccc}
\toprule
& Step-wise & Cross-modal  & Updated  & \multicolumn{2}{c}{RefCOCOg} \\
 & Multi. Prompts & Inter. Adapters  &  Params. & val-u & test-u \\ \midrule

  % &  &0&   &   \\
(a)  & $\checkmark$ &    & 6.30M & 71.32  &70.06 \\
  % & $\checkmark$ &   &M& 71.14 &  \\
(b)  &  & $\checkmark$  &1.00M&72.22 & 71.86\\
  
% \rowcolor{gray!20}
(c)  &$\checkmark$ & $\checkmark$  &7.30M & \textbf{75.57} & \textbf{75.48} \\ \bottomrule
\end{tabular}
% \vspace{-2mm}

\end{table}





\noindent
\textbf{Effectiveness of Domain-specific Adapters.} Because the text encoder is pre-trained on a general domain, freezing the entire text backbone restricts the specific language understanding in visual grounding domain, thereby weakening the proper interaction between text and vision semantics. To enable the domain text semantics to interact with the visual encoder efficiently, we adopt domain-specific adapters to learn the domain knowledge, thus making the text encoder match with visual grounding. Tab. \ref{Table:ablation on text adapter} shows that domain-specific adapters efficiently transfer the language knowledge of the pre-trained model to VG domain, further improving an average improvement of 4.39\% across the RefCOCOg-u.



\begin{table}[!t]
\centering
\renewcommand{\arraystretch}{1.5}
\caption{Effectiveness of Domain-specific adapters. (a) represents introducing Swip and CIA in SwimVG.}
\vspace{-2mm}
\label{Table:ablation on text adapter}
\small
\setlength{\tabcolsep}{10pt}


\begin{tabular}{l|c|c|ccc}
\toprule
    
 \multirow{2}{*}{\#} & {Domain-specific } & \multicolumn{1}{c|}{Updated} & \multicolumn{2}{c}{RefCOCOg} \\

 & Adapters& Params. & val-u & test-u  \\ \midrule

(a)  &  & 7.30M &  75.57 & 75.48  \\ 
\midrule
(b) & $\checkmark$ & 7.65M  &  \textbf{80.14}& \textbf{79.69}  \\


\bottomrule

\end{tabular}
% \vspace{-2mm}

\end{table}


\begin{figure*}[t]
\centering
\includegraphics[width=\textwidth]{./figures/visual-appendix.pdf}
\caption{The visualizations of attention maps from vision encoder with different strategies of SwimVG. Red bounding boxes represent ground truth, and yellow bounding boxes are prediction results.}
\vspace{-4mm}
\label{fig:more visualizations}
\end{figure*}


\noindent
\textbf{Effects of Different Insertion Positions of SwimVG.} To determine the optimal configuration of the Cross-modal Interactive Adapter (CIA) and Text Adapter, we conducted an ablation study varying both different layers and the dimensions of the adapters. Firstly, we evaluated the impact of different adapter layers. In this experiment, the visual CIA and the Text Adapter were inserted at the same layers. From Table \ref{tab:layer}, we can see that: \textbf{(1)} Only inserting three layers for vision and text encoder can brings great performance (Table \ref{tab:layer} (a)); \textbf{(2)} observing Table \ref{tab:layer} (b), (c), and (d), it can be seen that inserting CIA later in the vision encoder can exhibit better performance; \textbf{(3)} 
from the observation of Table \ref{tab:layer} (e) and (f), it is evident that inserting text adapter later in the text encoder results in a minor performance decline; \textbf{(4)} 
adding adapters from 13 layers to 24 layers not only reduces performance but also increases the tunable parameters. This might be because the visual backbone is more likely to adapt to the VG domain at deeper layers, while the text needs to adapt from the shallow layers to the deep layers. It should be noted that the text encoder is composed of 12 layers, while the vision encoder comprises 24 layers.


\begin{table}[t]
\centering
\renewcommand{\arraystretch}{1.5}
\caption{Ablation study of different configurations of cross-modal interactive adapters and text adapters. For the ``Position'', we list the i-th layers that insert adapters in the backbone.
}
\label{tab:layer}
\small
\setlength{\tabcolsep}{3pt}
\begin{tabular}{l|l|l|c|lll}
% \begin{tabular}{c|c|c|c|c|c|ccc}

\toprule
\multirow{2}{*}{\#} & \multicolumn{2}{c|}{Position} & \multirow{2}{*}{Params}  & \multicolumn{2}{c}{RefCOCOg} \\ \cline{7-7} 
 &  \multicolumn{1}{c|}{text}  &  \multicolumn{1}{c|}{vision} & & \multicolumn{1}{c|}{val-u} & \multicolumn{1}{c}{test-u}  \\ \midrule
(a) &4,8,12 & 8,16,24 & 6.67M   & \multicolumn{1}{l|}{75.26} & 74.78 \\

(b) &2,4,6,8,10,12 & 4,8,12,16,20,24 & 7.65M   & \multicolumn{1}{l|}{78.65} & 72.54 \\ 
 
(c) &2,4,6,8,10,12 & 14,16,18,20,22,24 &7.65 M  & \multicolumn{1}{l|}{79.28} & 78.62 \\

\rowcolor{gray!20}
(d) & 2,4,6,8,10,12& 19,20,21,22,23,24 & 7.65M  & \multicolumn{1}{l|}{80.14} & 79.69 \\

(e) &7,8,9,10,11,12& 19,20,21,22,23,24 & 7.65M   & \multicolumn{1}{l|}{78.90} & 78.06 \\

(f)  & 7,8,9,10,11,12& 14,16,18,20,22,24 & 7.65M  & \multicolumn{1}{l|}{79.06} & 78.43 \\

(g) & 2,4,6,8,10,12& 13-24 & 8.65M  & \multicolumn{1}{l|}{79.51} & 78.39 \\


\bottomrule
\end{tabular}

\end{table}


\begin{table}[!t]
\centering
\renewcommand{\arraystretch}{1.5}
\caption{Effectiveness of different bottleneck for all adapters.}
% \vspace{-4mm}
\label{Table:neck}
\small
\setlength{\tabcolsep}{8pt}

\begin{tabular}{l|c|c|ccc}
\toprule
    
 \multirow{2}{*}{\#} & \multirow{2}{*}{Bottleneck dimensions} & \multicolumn{1}{c|}{Params.} & \multicolumn{2}{c}{RefCOCOg} \\

 & & (M) & val-u & test-u  \\ \midrule
(a)  &32 & 7.05 &  78.65 & 78.13  \\
(b)  &40 & 7.24 &  79.67 & 78.78  \\
\rowcolor{gray!20}
(c) & 56&7.65  & 80.14 &79.69  \\
(d) & 64& 7.87  & 79.12 & 78.52 \\
(e) & 128&  9.76 & 80.18 &79.43  \\


\bottomrule

\end{tabular}
\vspace{-2mm}

\end{table}

\noindent
\textbf{Effects of Different Hyper-parameter Settings of SwimVG.} We first ablate the bottleneck dimensions $C_d$ of all adapters (see Table \ref{Table:neck} (a,b,c)), and follow the design shown in Table \ref{Table:neck} (a). $C_d$ determines the number of tunable parameters introduced by SwimVG. As shown in Table \ref{Table:neck}, higher $C_d$ introduces more parameters, and the performance consistently increases when $C_d$ increases up to 56. $C_d$ 128 exhibits considerable performance, but its tunable parameter count is about twice that of $C_d$ 56. Thus, we select the $C_d$ as 56. This indicates that a small bottleneck may not provide sufficient adaptation capabilities, while a large dimension may lead to over-adaptation. An intermediate dimension can achieve a better adaptation to the VG domain.


\begin{table}[t]
\caption{Comparison of the contribution levels of different backbones.}
% \vspace{-2mm}
\centering

\small
\setlength{\tabcolsep}{3.5pt}

\begin{tabular}{l|cc|ccc}
\toprule

\multirow{2}{*}{Mehthods} & \multicolumn{1}{c}{Vision} & \multicolumn{1}{c|}{Language} & \multicolumn{3}{c}{RefCOCO} \\

 & Backbone & Backbone & val & testA & testB \\ \midrule
TransVG\cite{deng2021transvg} &RN101+DETR & BERT-Base & 81.02 & 82.72 & 78.35
\\
TransVG\cite{deng2021transvg} & DINOv2-L & BERT-Base  & 85.11 & 87.36  & 80.97
\\

TransVG\cite{deng2021transvg} & DINOv2-L & CLIP-Base & 85.55 & 86.79  & 80.28
\\
\hline
SwimVG & DINOv2-L & CLIP-Base & \textbf{88.29}  & \textbf{90.37 }&\textbf{84.89}
\\
\bottomrule
\end{tabular}


\vspace{-2mm}
\label{Table:backbone}
\end{table}
\noindent
\textbf{The contribution degree of different pre-trained models.} To facilitate the analysis of the contribution of different backbones to performance, we excluded the SwimVG method and compared different backbones based on TransVG\cite{deng2021transvg}. We selected ResNet101+DETR and DINOv2-L as the vision backbone and chose the mainstream BERT-Base and the text encoder in CLIP-Base as the text backbone. As see in Table \ref{Table:backbone}, the vision backbone has a relatively large impact on visual grounding, whereas the text backbones have a relatively small impact. Under the same backbone, our method outperforms TransVG, which indicates that our multimodal fusion strategy is highly effective.

\vspace{-4mm}
\subsection{More Evaluation Metrics}
We compared more challenging evaluation metrics, such as the prediction accuracy when IoU $>$ 0.6 (Pr@0.6) and Pr@0.8. Under the same metrics, we compared the latest MaPPER \cite{liu2024mapper}. As seen in Table \ref{Table:iou}, SwimVG outperforms the latest MaPPER under both the settings of Pr@0.6 and Pr@0.8.

\begin{table}[t]
\caption{Comparison of the more evaluation metrics.}
% \vspace{-2mm}
\centering

\small
\setlength{\tabcolsep}{5pt}

\begin{tabular}{l|ccc|ccc}
\toprule

\multirow{2}{*}{Mehthods} & \multicolumn{3}{c|}{Pr@0.6(RefCOCO)} & \multicolumn{3}{c}{Pr@0.8(RefCOCO)} \\

& val & testA & testB & val & testA & testB \\ \midrule
MaPPER\cite{liu2024mapper} & 82.23 & 86.03 & 76.11 & 66.62 & 72.63 &57.50
\\
SwimVG & \textbf{85.26} & \textbf{87.33} & \textbf{80.61}  & \textbf{68.86} & \textbf{72.83} & \textbf{63.04}
\\
% \hline
% MaPPER\cite{liu2024mapper} & 81.02 & 82.72 & 78.35 & \textbf{85.55} & 86.79  & 80.28
% \\
% SwimVG & 81.02 & 82.72 & 78.35 &\textbf{88.29}  & 90.37 & 84.89
% \\
\bottomrule
\end{tabular}


\vspace{-2mm}
\label{Table:iou}
\end{table}


\subsection{Qualitative Results}
\textbf{The comparison of multimodal fusion strategy.}
To verify that the multimodal fusion strategy of SwimVG is superior to the traditional vision-language transformer (VL encoder), we visualize the attention maps from the last layer of vision encoder in SwimVG. Due to the suboptimal multimodal fusion methods employed by other mainstream approaches, namely the visual language transformer (VL encoder), which lack open-source code or checkpoints, we opt to visualize the last layer of the VL encoder from TransVG. As shown in Fig.\ref{fig:visualizations}, TransVG fails to pay sufficient attention to text-relevant regions in a images. For example, TransVG lacks the alignment ability of ``$different$'', ``$black$'', and ``$standing$'' with images. The comparison with TransVG demonstrates the ability of our proposed SwimVG to focus more on the text-relevant regions, and our multimodal fusion strategy is superior to the traditional VL encoder.



\noindent
\textbf{The effectivess of CIA and Swip.}
In this section, we present more visualization of the attention maps from the vision encoder under different mixing strategies. As depicted in Figure \ref{fig:more visualizations}, we can see that: \textbf{(1)} introducing either cross-modal interactive adapters (CIA) or step-wise multimodal prompts (Swip) facilitates the interaction between the vision and language encoders. (Figure \ref{fig:more visualizations} (b,c)); \textbf{(2)} compared to CIA, the attention map of only introducing is slightly scattered (Figure \ref{fig:more visualizations} (b,c)); 
integrating CIA and Swip can further enhances the facilitation of cross-modal interaction (Figure \ref{fig:more visualizations} (d)). The interaction between the vision and language encoder, facilitated by CIA and Swip, allows the model to focus more effectively on the referred objects in diverse expression cases. 








\begin{figure*}[h]
    \centering
    \includegraphics[width=14cm]{figures/visualized_kitti5.jpg}
    \caption[Qualitative Results on KITTI \textit{val.} set]{\textbf{Qualitative results on the KITTI \textit{val} set for the car class.} The proposed method (green) and ground truth (red).
    } \label{fig:KITTI visualized}
\end{figure*}

\begin{figure*}[t]
    \centering
     \includegraphics[width=14cm]{figures/custom_result_monodetr_monoground3.jpg}
    \caption{\textbf{Qualitative results on the custom dataset.} Comparison of detection results between the proposed model (blue), the state-of-the-art models (green), and ground truth (red) in ego-view (left) and bird's-eye view (right); MonoDETR (left) and MonoGround (right).}
    \label{fig:custom_result_visualized}
\end{figure*}

\section{Conclusion}
\label{sec:conclusion}
This paper presents a novel approach to monocular 3D object detection by integrating a Vision Foundation Model as the backbone with the DETR architecture, enabling enhanced depth estimation and feature extraction within a single-stage, real-time framework. By incorporating a Hierarchical Feature Fusion Block for multi-scale detection and 6D Dynamic Anchor Boxes for iterative bounding box refinement, the proposed model achieves improved performance without relying on additional data sources, such as LiDAR. Future work will focus on extending the model's capabilities to detect 3D bounding boxes while accounting for rolling and pitching angles.
\section*{Acknowledgements}


This work was supported by National Natural Science Foundation of China (U2336214, 62332019, 62302297), Beijing Natural Science Foundation (L222008), Shanghai Sailing Program (22YF1420300), Young Elite Scientists Sponsorship Program by CAST (2022QNRC001).


% \clearpage

{
\bibliographystyle{IEEEtran}
\bibliography{ref}
}

\begin{IEEEbiography}[{\includegraphics[width=1in,height=1.25in,clip,keepaspectratio]{figures/fig_bio/xmf.png}}]{Mengfei Xia}
received the B.S. degree in 2020 from the Department of Mathematical Science, Tsinghua University, Beijing, China. He is currently a fourth-year Ph.D. student at the Department of Computer Science and Technology, Tsinghua University. His
research interests include mathematical foundation in
deep learning, image processing, and computer vision. He was the recipient of the Silver Medal twice in 30th and 31st National Mathematical Olympiad of China.
\end{IEEEbiography}

%\vspace{-5em}

\begin{IEEEbiography}[{\includegraphics[width=1in,height=1.25in,clip,keepaspectratio]{figures/fig_bio/zy.png}}]{Yu Zhou}
is a fourth-year undergraduate student with Zhili College, Tsinghua University, China. His research interests include deep learning and computer vision. He was the recipient of the Silver Medal twice in 35th and 36th National Olympiad in Informatics of China.
\end{IEEEbiography}

%\vspace{-5em}

\begin{IEEEbiography}[{\includegraphics[width=1in,height=1.25in,clip,keepaspectratio]{figures/fig_bio/yr.png}}]{Ran Yi}
is an assistant professor with the Department of Computer Science and Engineering, Shanghai Jiao Tong University. She received the BEng degree and the PhD degree from Tsinghua University, China, in 2016 and 2021. Her research interests include computer vision, computer graphics and computational geometry.
\end{IEEEbiography}

% \vspace{-25em}

\begin{IEEEbiography}[{\includegraphics[width=1in,height=1.25in,clip,keepaspectratio]{figures/fig_bio/lyj.png}}]{Yong-Jin Liu}
is a professor with the Department of Computer Science and Technology, Tsinghua University, China. He received the BEng degree from Tianjin University, China, in 1998, and the PhD degree from the Hong Kong University of Science and Technology, Hong Kong, China, in 2004. His research interests include computer vision, computer graphics and computer-aided design. For more information, visit 
\href{http://cg.cs.tsinghua.edu.cn/people/~Yongjin/Yongjin.htm}{http://cg.cs.tsinghua.edu.cn/ people/$\sim$Yongjin/Yongjin.htm}.
\end{IEEEbiography}

% \vspace{-25em}

\begin{IEEEbiography}[{\includegraphics[width=1in,height=1.25in,clip,keepaspectratio]{figures/fig_bio/wwp.png}}]{Wenping Wang}
(Fellow, IEEE) received the Ph.D. degree in computer science from the University of Alberta in 1992. He is a Professor of computer science at Texas A\&M University. His research interests include computer graphics, computer visualization, computer vision, robotics, medical image processing, and geometric computing. He is or has been an journal associate editor of ACM Transactions on Graphics, IEEE Transactions on Visualization and Computer Graphics, Computer Aided Geometric Design, and Computer Graphics Forum (CGF). He has chaired a number of international conferences, including Pacific Graphics, ACM Symposium on Physical and Solid Modeling (SPM), SIGGRAPH and SIGGRAPH Asia. Prof. Wang received the John Gregory Memorial Award for his contributions to geometric modeling. He is an IEEE Fellow and an ACM Fellow.
\end{IEEEbiography}

\clearpage

\setcounter{theorem}{0}
\setcounter{lemma}{0}

\appendices


\section{Derivatives of training and inference processes}

\subsection{Algorithms of training and inference processes}

In this part, we provide the algorithms of training and inference processes.
%
\sqq{Notably,} both training and inference procedures in \Cref{alg:paired_train,alg:paired_infer} resemble the corresponding processes of DDPM respectively, where $\epsilon_{\phi}$ is the pre-trained DDPM with parameter $\phi$, and $\sigma_i$ is the variance of the distribution $\epsilon_{\phi}(y_{i-1}|y_i)$.
%
The training learns to transfer between the intermediate diffusion results $x_t$ and $y_t$ while DDPM approximator intends to predict the noise $\epsilon$ from $x_t$.

\revise{
Furthermore, we provide the algorithms \sqq{for the} training and inference process\sqq{es} of the generalized asymmetric pipelines in \Cref{alg:paired_train_asymm,alg:paired_infer_asymm}.
}

\begin{algorithm}[H]
\setstretch{1.2}
\caption{Training}\label{alg:paired_train}
\begin{algorithmic}[1]
\Repeat
\State $x_0\sim q(x_0),y_0\sim q(y_0|x_0)$
\State $z_t\sim\mathcal N(0, I)$
\State $x_t\leftarrow \sqrt{\balpha{t}}x_0+\sqrt{1-\balpha{t}}z_t$
\State $y_t\leftarrow \sqrt{\balpha{t}}y_0+\sqrt{1-\balpha{t}}z_t$
\State Take gradient descent step on
$$\nabla_{\theta}\|f_{\theta}(x_t)-y_t\|^2$$
\Until converged
\end{algorithmic}
\end{algorithm}


\begin{algorithm}[H]
\setstretch{1.2}
\caption{Inference}\label{alg:paired_infer}
\begin{algorithmic}[1]
\State $x_0\sim q(x_0)$
\State $z_t,z\sim\mathcal N(0, I)$
\State $x_t\leftarrow \sqrt{\balpha{t}}x_0+\sqrt{1-\balpha{t}}z_t$
\State $y_t\leftarrow f_{\theta}(x_t)-\sqrt{1-\balpha{t}}z_t+\sqrt{1-\balpha{t}}z$
\For {$i=t,t-1,\cdots,1$}
\State $\epsilon_i\sim\mathcal N(0, I)$ if $i>1$, else $\epsilon_i=0$
\State $y_{i-1}=\frac{(y_i-\frac{1-\alpha_i}{\sqrt{1-\bar\alpha_i}}\epsilon_{\phi}(y_i,i))}{\sqrt{\alpha_i}}+\sigma_i\epsilon_i$
\EndFor
\State \Return $y_0$
\end{algorithmic}
\end{algorithm}

\revise{
\begin{algorithm}[H]
\setstretch{1.2}
\caption{Training of the generalized asymmetric pipeline}\label{alg:paired_train_asymm}
\begin{algorithmic}[1]
\Repeat
\State $x_0\sim q(x_0),y_0\sim q(y_0|x_0)$
\State $z\sim\mathcal N(0, I)$
\State $x_s\leftarrow \sqrt{\balpha{s}}x_0+\sqrt{1-\balpha{s}}z$
\State $y_t\leftarrow \sqrt{\balpha{t}}y_0+\sqrt{1-\balpha{t}}z$
\State Take gradient descent step on
$$\nabla_{\theta}\|f_{\theta}(x_s)-y_t\|^2$$
\Until converged
\end{algorithmic}
\end{algorithm}
}


\revise{
\begin{algorithm}[H]
\setstretch{1.2}
\caption{Inference of the generalized asymmetric pipeline}\label{alg:paired_infer_asymm}
\begin{algorithmic}[1]
\State $x_0\sim q(x_0)$
\State $z_1,z_2\sim\mathcal N(0, I)$
\State $x_s\leftarrow \sqrt{\balpha{s}}x_0+\sqrt{1-\balpha{s}}z_1$
\State $y_t\leftarrow f_{\theta}(x_s)-\sqrt{1-\balpha{t}}z_1+\sqrt{1-\balpha{t}}z_2$
\For {$i=t,t-1,\cdots,1$}
\State $\epsilon_i\sim\mathcal N(0, I)$ if $i>1$, else $\epsilon_i=0$
\State $y_{i-1}=\frac{(y_i-\frac{1-\alpha_i}{\sqrt{1-\bar\alpha_i}}\epsilon_{\phi}(y_i,i))}{\sqrt{\alpha_i}}+\sigma_i\epsilon_i$
\EndFor
\State \Return $y_0$
\end{algorithmic}
\end{algorithm}
}

\begin{table*}[ht]
\begin{tabular}{cc}
\begin{minipage}[t]{0.48\textwidth}
\begin{algorithm}[H]
\setstretch{1.15}
\caption{Pseudo-code of \method in a PyTorch-like style.}
\label{alg:code}
\begin{lstlisting}[language=python]
import torch

def forward_step(x_0, y_0, t, T):
    """Defines the forward process of one training step.
    
    Args:
        x_0: Source inputs, with shape [B, C, H, W].
        y_0: Target outputs, with shape [B, C, H, W].
        t: The preset timestep to perform distribution shift.
        T: The translator module to learn.
    """
    # Compute the cumulated variance until timestep t.
    bar_alpha_t = cum_var(t)
    
    # Adding noise (i.e., diffusion) to images from both domains.
    z_t = torch.randn_like(x_0)
    x_t = torch.sqrt(bar_alpha_t) * x_0 + torch.sqrt(1 - bar_alpha_t) * z_t
    y_t = torch.sqrt(bar_alpha_t) * y_0 + torch.sqrt(1 - bar_alpha_t) * z_t
    
    # Learn the translator.
    loss = (T(x_t) - y_t).square().mean()
    
    return loss
\end{lstlisting}
\end{algorithm}
\end{minipage}
&
\begin{minipage}[t]{0.48\textwidth}
\renewcommand\arraystretch{1.0}
\begin{algorithm}[H]
\caption{\revise{Pseudo-code of \method in a PyTorch-like style.}}
\label{alg:code_asymm}
\begin{lstlisting}[language=python]
import torch

def forward_step(x_0, y_0, s, t, T):
    """Defines the forward process of one training step.
    
    Args:
        x_0: Source inputs, with shape [B, C, H, W].
        y_0: Target outputs, with shape [B, C, H, W].
        s: The preset timestep to perform distribution shift for x_0.
        t: The preset timestep to perform distribution shift for y_0.
        T: The translator module to learn.
    """
    # Compute the cumulated variance until timestep s and t.
    bar_alpha_s = cum_var(s)
    bar_alpha_t = cum_var(t)
    
    # Adding noise (i.e., diffusion) to images from both domains.
    z = torch.randn_like(x_0)
    x_s = torch.sqrt(bar_alpha_s) * x_0 + torch.sqrt(1 - bar_alpha_s) * z
    y_t = torch.sqrt(bar_alpha_t) * y_0 + torch.sqrt(1 - bar_alpha_t) * z
    
    # Learn the translator.
    loss = (T(x_s) - y_t).square().mean()
    
    return loss
\end{lstlisting}
\end{algorithm}
\end{minipage}
\end{tabular}
\end{table*}

\subsection{Pseudo-code of training process}
Our proposed diffusion model translator (\method) achieves image-to-image translation (I2I) based on a pre-trained DDPM via simply learning a distribution shift at a certain diffusion timestep.
%
Accordingly, it owns a \textit{highly efficient} implementation, which is even \textit{independent} of the DDPM itself.
%
In this part, we provide the pseudo-code of the training process in \cref{alg:code,alg:code_asymm}.
%
% It is noteworthy that the whole training process if extremely simple and efficiency, compared to the Pix2Pix~\cite{isola2017image}, which introduces a discriminator during training simultaneously.



\section{Proofs of main results}

\begin{lemma}
We have an upper bound of the negative log-likelihood of $-\log p_{\theta}(y_0|x_0)$ by
%
\begin{align}
-\log p_{\theta}(y_0|x_0)\leqslant\mathbb E_q\left[\log\frac{q(y_{1:t},x_{1:t}|y_0, x_0)}{p_{\theta}(y_{0:t},x_{1:t}|x_0)}\right],
\end{align}
%
where $q=q(y_{1:t},x_{1:t}|y_0, x_0).$
\end{lemma}

\begin{proof}
\begin{align}
&-\log p_{\theta}(y_0|x_0) \\
\leqslant& -\log p_{\theta}(y_0|x_0) + \nonumber \\
& \qquad D_{KL}\left(q(y_{1:t},x_{1:t}|y_0, x_0) \| p_{\theta}(y_{1:t},x_{1:t}|y_0, x_0)\right) \\
=& -\log p_{\theta}(y_0|x_0)+\nonumber  \\
& \qquad \mathbb E_{q(y_{1:t},x_{1:t}|y_0, x_0)}\left[\log\frac{q(y_{1:t},x_{1:t}|y_0, x_0)}{p_{\theta}(y_{1:t},x_{1:t}|y_0, x_0)}\right] \\
=& -\log p_{\theta}(y_0|x_0)+ \nonumber \\
& \qquad \mathbb E_{q(y_{1:t},x_{1:t}|y_0, x_0)}\left[\log\frac{q(y_{1:t},x_{1:t}|y_0, x_0)}{p_{\theta}(y_{0:t},x_{1:t}|x_0)/p_{\theta}(y_0|x_0)}\right] \\
%=& -\log p_{\theta}(y_0|x_0)+\mathbb E_{q(y_{1:t},x_{1:t}|y_0, x_0)}\left[\log\frac{q(y_{1:t},x_{1:t}|y_0, x_0)}{p_{\theta}(y_{0:t},x_{1:t}|x_0)}+\log p_{\theta}(y_0|x_0)\right] \\
=& \mathbb E_{q(y_{1:t},x_{1:t}|y_0, x_0)}\left[\log\frac{q(y_{1:t},x_{1:t}|y_0, x_0)}{p_{\theta}(y_{0:t},x_{1:t}|x_0)}\right].
\end{align}
\end{proof}

\begin{theorem}[Closed-form expression]%\label{theorem:1}
% The loss function in \cref{eq:3.3} has a closed-form representation.
The loss function in Equation (13) in the main paper has a closed-form representation.
%
The training is equivalent to optimizing a KL-divergence up to a non-negative constant, \textit{i.e.},
%
\begin{align}
\mathcal L_{VLB}=\mathbb E_{q(y_0,x_t|x_0)}\left[D_{KL}(q(y_t|y_0)\|p_{\theta}(y_t|x_t))\right] + C\sqq{.}%\label{eq:3.4}
\end{align}
\end{theorem}

\begin{proof}
% By the factorization in \cref{eq:3.1} and \cref{eq:3.2}, we observe that
By the factorization in Equation (6) and (11) in the main paper, we observe that
%
\begin{align}
\mathcal L_{VLB}&=\mathbb E_{q(y_{0:t},x_{1:t}|x_0)}\left[\log\frac{q(y_{1:t},x_{1:t}|y_0, x_0)}{p_{\theta}(y_{0:t},x_{1:t}|x_0)}\right] \\
&=\mathbb E_{q(y_{0:t},x_{1:t}|x_0)}\left[\log\frac{1}{p_{\theta}(y_t|x_t)}+\sum_{j=1}^t\log\frac{q(y_j|y_{j-1})}{q(y_{j-1}|y_j)}\right].
\end{align}
%
Using Bayes' rule, for any $j=1,2,\cdots,t$, we have
%
\begin{align}
%\frac{q(y_j|y_{j-1})}{q(y_{j-1}|y_j)}&=\frac{q(y_j|y_{j-1})\cdot q(y_j)\cdot q(y_{j-1})}{q(y_{j-1}|y_j)\cdot q(y_j)\cdot q(y_{j-1})}\\
%&=\frac{q(y_j,y_{j-1})q(y_j)}{q(y_{j-1},y_j)q(y_{j-1})}=\frac{q(y_j)}{q(y_{j-1})}.
\frac{q(y_j|y_{j-1})}{q(y_{j-1}|y_j)}=\frac{q(y_j)}{q(y_{j-1})}, \quad \frac{q(y_t)}{q(y_0)}=\frac{q(y_t|y_0)}{q(y_0|y_t)}.
\end{align}

Hence\sqq{,} it is equivalent to optimizing the KL-divergence up to a non-negative constant $C$:
%
\begin{align}
\mathcal L_{VLB}&=\mathbb E_{q(y_{0:t},x_{1:t}|x_0)}\left[\log\frac{q(y_t|y_0)}{p_{\theta}(y_t|x_t)}+\log\frac{1}{q(y_0|y_t)}\right] \\
&=\mathbb E_{q(y_0,x_t|x_0)}\left[D_{KL}(q(y_t|y_0)\|p_{\theta}(y_t|x_t))\right] + C,
\end{align}
%
where $C=\mathbb E_{q(y_t)}\left[H(q(y_0|y_t))\right]\geqslant0$ and $H$ is the entropy of a distribution.
%
Since $q(y_t|y_0)$ follows a Gaussian distribution, then so is optimal $p_{\theta}(y_t|x_t)$.
\end{proof}


% \begin{theorem}[Optimal solution to \cref{eq:3.4}]%\label{theorem:2}
\begin{theorem}[Optimal solution to Equation (15) in the main paper]%\label{theorem:2}
The optimal $p_{\theta}(y_t|x_t)$ follows a Gaussian distribution with mean $\mu_{\theta}$ being
%
\begin{align}
\mu_{\theta}(x_t) = \sqrt{\balpha{t}}y_0.
\end{align}
\end{theorem}

\begin{proof}
% To minimize the KL-divergence in \cref{eq:3.4}, we first notice that $q(y_t|y_0)$ follows a Gaussian distribution, i.e.,
To minimize the KL-divergence in Equation (15) in the main paper, we first notice that $q(y_t|y_0)$ follows a Gaussian distribution, \textit{i.e.},
%
\begin{align}
q(y_t|y_0)\sim\mathcal N(y_t;\sqrt{\balpha{t}}y_0,(1-\balpha{t})I),\quad \mu_t(y_t)=\sqrt{\balpha{t}}y_0,
\end{align}
%
which implies that $
p_{\theta}(y_t|x_t)\sim\mathcal N(y_t;\mu_{\theta}(x_t),\Sigma_{\theta}(x_t))$
%
with mean $\mu_{\theta}(x_t)=\mu_t(y_t)=\sqrt{\balpha{t}}y_0$.
\end{proof}

\revise{
\begin{lemma}
We have an upper bound of the negative log-likelihood of $-\log p_{\theta}(y_0|x_0)$ by
%
\begin{align}
-\log p_{\theta}(y_0|x_0)\leqslant\mathbb E_q\left[\log\frac{q(y_{1:t},x_{1:s}|y_0, x_0)}{p_{\theta}(y_{0:t},x_{1:s}|x_0)}\right],
\end{align}
%
where $q=q(y_{1:t},x_{1:s}|y_0, x_0).$
\end{lemma}
}

\revise{
\begin{proof}
\begin{align}
&-\log p_{\theta}(y_0|x_0) \\
\leqslant& -\log p_{\theta}(y_0|x_0) + \nonumber \\
& \qquad D_{KL}\left(q(y_{1:t},x_{1:s}|y_0, x_0) \| p_{\theta}(y_{1:t},x_{1:s}|y_0, x_0)\right) \\
=& -\log p_{\theta}(y_0|x_0)+\nonumber  \\
& \qquad \mathbb E_{q(y_{1:t},x_{1:s}|y_0, x_0)}\left[\log\frac{q(y_{1:t},x_{1:s}|y_0, x_0)}{p_{\theta}(y_{1:t},x_{1:s}|y_0, x_0)}\right] \\
=& -\log p_{\theta}(y_0|x_0)+ \nonumber \\
& \qquad \mathbb E_{q(y_{1:t},x_{1:s}|y_0, x_0)}\left[\log\frac{q(y_{1:t},x_{1:s}|y_0, x_0)}{p_{\theta}(y_{0:t},x_{1:s}|x_0)/p_{\theta}(y_0|x_0)}\right] \\
=& \mathbb E_{q(y_{1:t},x_{1:s}|y_0, x_0)}\left[\log\frac{q(y_{1:t},x_{1:s}|y_0, x_0)}{p_{\theta}(y_{0:t},x_{1:s}|x_0)}\right].
\end{align}
\end{proof}
}

\revise{
\begin{theorem}[Closed-form expression]
The loss function in Equation (23) in the main paper has a closed-form representation.
%
The training is equivalent to optimizing a KL-divergence up to a non-negative constant, \textit{i.e.},
%
\begin{align}
\mathcal L_{VLB}=\mathbb E_{q(y_0,x_s|x_0)}\left[D_{KL}(q(y_t|y_0)\|p_{\theta}(y_t|x_s))\right] + C
\sqq{.}
\end{align}
\end{theorem}
}

\revise{
\begin{proof}
By the factorization in Equation (20) and (21) in the main paper, we observe that
%
\begin{align}
\mathcal L_{VLB}&=\mathbb E_{q(y_{0:t},x_{1:s}|x_0)}\left[\log\frac{q(y_{1:t},x_{1:s}|y_0, x_0)}{p_{\theta}(y_{0:t},x_{1:s}|x_0)}\right] \\
&=\mathbb E_{q(y_{0:t},x_{1:s}|x_0)}\left[\log\frac{1}{p_{\theta}(y_t|x_s)}+\sum_{j=1}^t\log\frac{q(y_j|y_{j-1})}{q(y_{j-1}|y_j)}\right].
\end{align}
%
Using Bayes' rule, for any $j=1,2,\cdots,t$, we have
%
\begin{align}
%\frac{q(y_j|y_{j-1})}{q(y_{j-1}|y_j)}&=\frac{q(y_j|y_{j-1})\cdot q(y_j)\cdot q(y_{j-1})}{q(y_{j-1}|y_j)\cdot q(y_j)\cdot q(y_{j-1})}\\
%&=\frac{q(y_j,y_{j-1})q(y_j)}{q(y_{j-1},y_j)q(y_{j-1})}=\frac{q(y_j)}{q(y_{j-1})}.
\frac{q(y_j|y_{j-1})}{q(y_{j-1}|y_j)}=\frac{q(y_j)}{q(y_{j-1})}, \quad \frac{q(y_t)}{q(y_0)}=\frac{q(y_t|y_0)}{q(y_0|y_t)}.
\end{align}
%
Hence\sqq{,} it is equivalent to optimizing the KL-divergence up to a non-negative constant $C$:
%
\begin{align}
\mathcal L_{VLB}&=\mathbb E_{q(y_{0:t},x_{1:s}|x_0)}\left[\log\frac{q(y_t|y_0)}{p_{\theta}(y_t|x_s)}+\log\frac{1}{q(y_0|y_t)}\right] \\
&=\mathbb E_{q(y_0,x_s|x_0)}\left[D_{KL}(q(y_t|y_0)\|p_{\theta}(y_t|x_s))\right] + C,
\end{align}
%
where $C=\mathbb E_{q(y_t)}\left[H(q(y_0|y_t))\right]\geqslant0$ and $H$ is the entropy of a distribution.
%
Since $q(y_t|y_0)$ follows a Gaussian distribution, then so is the optimal $p_{\theta}(y_t|x_s)$.
\end{proof}
}


\revise{
\begin{theorem}[Optimal solution to Equation (25) in the main paper]
The optimal $p_{\theta}(y_t|x_s)$ follows a Gaussian distribution with mean $\mu_{\theta}$ being
%
\begin{align}
\mu_{\theta}(x_s) = \sqrt{\balpha{t}}y_0.
\end{align}
\end{theorem}
}

\revise{
\begin{proof}
% To minimize the KL-divergence in \cref{eq:3.4}, we first notice that $q(y_t|y_0)$ follows a Gaussian distribution, i.e.,
To minimize the KL-divergence in Equation (25) in the main paper, we first notice that $q(y_t|y_0)$ follows a Gaussian distribution, \textit{i.e.},
%
\begin{align}
q(y_t|y_0)\sim\mathcal N(y_t;\sqrt{\balpha{t}}y_0,(1-\balpha{t})I),\quad \mu_t(y_t)=\sqrt{\balpha{t}}y_0,
\end{align}
%
which implies that $
p_{\theta}(y_t|x_s)\sim\mathcal N(y_t;\mu_{\theta}(x_s),\Sigma_{\theta}(x_s))$
%
with mean $\mu_{\theta}(x_s)=\mu_t(y_t)=\sqrt{\balpha{t}}y_0$.
\end{proof}
}


\begin{table*}[ht]
% \renewcommand\arraystretch{1.2}
\begin{minipage}[t]{0.48\textwidth}
\caption{
    \newrevise{
    \textbf{Quantitative comparison} between single-step and multi-step \method (TSIT-DMT-2step) upon TSIT.
    %
    FID and SSIM are used to evaluate the image quality and content preservation, respectively.
    }
}
\vspace{-5pt}
\centering
\SetTblrInner{rowsep=2.15pt}                % Row space.
\SetTblrInner{colsep=10.0pt}               % Col space.
% \scriptsize
\newrevise{
\begin{tblr}{
    cell{1-6}{1-5}={halign=c,valign=m},    % Text alignment for all cells.
    cell{1}{2,4}={c=2}{},                  % Multi-column cells.
    cell{1}{1}={r=2}{},                    % Multi-row cells.
    hline{1,7}={1-5}{1.0pt},               % Horizontal lines.
    hline{2}={2-5}{},                      % Horizontal lines.
    hline{3}={1-5}{},                      % Horizontal lines.
    vline{2,4}={1-6}{},                    % Vertical lines.
}
\label{tab:metric}
Method & TSIT-\method & & TSIT-\method-2step & \\
 & \textbf{FID$\downarrow$} & \textbf{SSIM$\uparrow$} & \textbf{FID$\downarrow$} & \textbf{SSIM$\uparrow$} \\
$t=50$ & 42.00 & 0.473  & 47.77 & 0.563 \\
$t=100$ & 37.22 & 0.460 & 43.52 & 0.563 \\
$t=200$ & 36.78 & 0.446 & 45.78 & 0.536 \\
$t=400$ & 50.79 & 0.251 & 48.34 & 0.447 \\
\end{tblr}
}
\hfill
\end{minipage}
\begin{minipage}[t]{0.48\textwidth}
\caption{
    \newrevise{
    \textbf{Time cost comparison} between single-step and multi-step \method upon TSIT.
    %
    To measure the time cost, we report the total number of training epochs of \method (\method Epoch) and of fusion UNet (Fusion Epoch), training time for 1,000 images for \method (\method Train) and fusion UNet (Fusion Train), and inference time for a single image.
    }
}
\vspace{-10pt}
\centering
\SetTblrInner{rowsep=1.05pt}               % Row space.
\SetTblrInner{colsep=10.0pt}               % Col space.
% \scriptsize
\newrevise{
\begin{tblr}{
    cell{1-6}{1-3}={halign=c,valign=m},    % Text alignment for all cells.
    hline{1,2,7}={1-3}{1.0pt},             % Horizontal lines.
    hline{2}={1-3}{},                      % Horizontal lines.
    vline{2,3}={1-6}{},                    % Vertical lines.
}
\label{tab:time}
Method & TSIT-\method & TSIT-\method-2step \\
DMT Train & 82s & 82s \\
DMT Epoch & 60 & 60 \\
Fusion Train & No such step & 139s \\
Fusion Epoch & No such step & 100 \\
Inference & 0.48s & 0.64s \\
\end{tblr}
}
\end{minipage}
\end{table*}


\definecolor{mygreen}{RGB}{34,170,133}
\begin{table*}[ht]
\begin{minipage}[t]{0.48\textwidth}
\centering
\vspace{-8pt}
\caption{
    \newrevise{
    \textbf{Ablation study} of $(s,t)$ pair near $s=t=200$ fixing $s=200$ on CelebA-HQ dataset.
    %
    For clearer demonstration, original \method (\textit{i.e.}, pair with $s=t$) is highlighted in \textbf{\textcolor{gray}{gray}}.
    }
}
\label{tab:ablation_s_t_sstar}
\vspace{-10pt}
\SetTblrInner{rowsep=2.15pt}                % Row space.
\SetTblrInner{colsep=10.0pt}               % Col space.
% \scriptsize
\newrevise{
\begin{tblr}{
    cell{1-18}{1-5}={halign=c,valign=m},    % Text alignment for all cells.
    % cell{2}{1}={c=17}{},                  % Multi-column cells.
    cell{2}{1}={r=17}{},                    % Multi-row cells.
    hline{1,19}={1-5}{1.0pt},               % Horizontal lines.
    hline{2}={1-5}{},                      % Horizontal lines.
    % hline{3}={1-5}{},                      % Horizontal lines.
    vline{2}={2-18}{},                    % Vertical lines.
    vline{3}={1-18}{},                    % Vertical lines.
    cell{10}{2-5}={bg=lightgray!35},
}
$s$ & $t$ & \textbf{FID$\downarrow$} & \textbf{SSIM$\uparrow$} & \textbf{LPIPS$\downarrow$} \\
$s=200$ & $t=0$ & 53.61 & 0.347 & 0.492 \\
 & $t=50$ & 37.27 & 0.445 & 0.443 \\
 & $t=100$ & 37.31 & 0.447 & 0.443 \\
 & $t=150$ & 43.43 & 0.441 & 0.445 \\
 & $t=160$ & 43.35 & 0.438 & 0.449 \\
 & $t=170$ & 44.69 & 0.458 & 0.428 \\
 & $t=180$ & 42.82 & \bf 0.466 & \bf 0.428 \\
 & $t=190$ & 45.94 & 0.458 & 0.430 \\
 & $t=200$ & \bf 36.78 & 0.446 & 0.433 \\
 & $t=210$ & 45.15 & 0.425 & 0.433 \\
 & $t=220$ & 46.17 & 0.422 & 0.436 \\
 & $t=230$ & 46.78 & 0.418 & 0.450 \\
 & $t=240$ & 47.69 & 0.415 & 0.451 \\
 & $t=250$ & 48.32 & 0.377 & 0.464 \\
 & $t=300$ & 55.54 & 0.337 & 0.500 \\
 & $t=350$ & 64.66 & 0.267 & 0.567 \\
 & $t=400$ & 53.42 & 0.328 & 0.493 \\
\end{tblr}
}
\end{minipage}
\hfill
\begin{minipage}[t]{0.48\textwidth}
\centering
\vspace{-8pt}
\caption{
    \newrevise{
    \textbf{Ablation study} of $(s,t)$ pair near $s=t=200$ fixing $t=200$ on CelebA-HQ dataset.
    %
    For clearer demonstration, original \method (\textit{i.e.}, pair with $s=t$) is highlighted in \textbf{\textcolor{gray}{gray}}.
    }
}
\label{tab:ablation_s_t_tstar}
\vspace{-10pt}
\SetTblrInner{rowsep=2.15pt}                % Row space.
\SetTblrInner{colsep=10.0pt}               % Col space.
% \scriptsize
\newrevise{
\begin{tblr}{
    cell{1-18}{1-5}={halign=c,valign=m},    % Text alignment for all cells.
    % cell{2}{1}={c=17}{},                  % Multi-column cells.
    cell{2}{2}={r=17}{},                    % Multi-row cells.
    hline{1,19}={1-5}{1.0pt},               % Horizontal lines.
    hline{2}={1-5}{},                      % Horizontal lines.
    % hline{3}={1-5}{},                      % Horizontal lines.
    vline{2}={2-18}{},                    % Vertical lines.
    vline{3}={1-18}{},                    % Vertical lines.
    cell{10}{2-5}={bg=lightgray!35},
    cell{10}{1}={bg=lightgray!35},
}
$s$ & $t$ & \textbf{FID$\downarrow$} & \textbf{SSIM$\uparrow$} & \textbf{LPIPS$\downarrow$} \\
$s=0$ & $t=200$ & 44.98 & 0.448 & 0.453 \\
$s=50$ & & 42.93 & 0.451 & 0.437 \\
$s=100$ & & 44.50 & 0.443 & 0.457 \\
$s=150$ & & 43.60 & 0.446 & 0.437 \\
$s=160$ & & 42.72 & 0.445 & 0.451 \\
$s=170$ & & 44.37 & 0.443 & 0.427 \\
$s=180$ & & 44.35 & 0.447 & 0.432 \\
$s=190$ & & 45.49 & 0.438 & \bf 0.427 \\
$s=200$ & & \bf 36.78 & 0.446 & 0.433 \\
$s=210$ & & 43.99 & 0.441 & 0.442 \\
$s=220$ & & 44.94 & 0.451 & 0.432 \\
$s=230$ & & 45.67 & 0.444 & 0.439 \\
$s=240$ & & 45.68 & 0.441 & 0.441 \\
$s=250$ & & 44.11 & 0.437 & 0.433 \\
$s=300$ & & 45.36 & \bf 0.453 & 0.432 \\
$s=350$ & & 45.39 & 0.451 & 0.421 \\
$s=400$ & & 44.44 & 0.445 & 0.437 \\
\end{tblr}
}
\end{minipage}
\end{table*}


\section{\newrevise{Comparison between multi-step and asymmetric \method}}


\newrevise{
\yr{\textbf{Multi-step \method.}} To implement the multi-step \method, due to the use of the vanilla DDPM, which is only capable of inputting a 3-channel input intermediate noisy image, we train an auxiliary UNet model to fuse the $y_{t/2}$ transformed from $x_{t/2}$ together with the $y'_{t/2}$ denoised from the $y_{t}$.
}


\newrevise{
In order to train the fusion UNet processing the intermediate noisy images $y_{t/2}$ and $y'_{t/2}$, we first need to prepare the dataset.
%
In more details, given a paired data $(x_0,y_0)$, the preset timestep $t$, noise $z$ under standard Gaussian distribution, the pre-trained \method $G$, and the pre-trained diffusion model, we first apply the diffusion forward process onto both $x_0$ and $y_0$ with noise $z$ until timestep $t$ and $t/2$, \textit{i.e.}, we acquire the $x_{t/2}$, $x_{t}$, $y_{t/2}$, and $y_{t}$.
%
Next, we utilize the pre-trained \method model to obtain the transformed $G(x_{t/2})$ and $G(x_{t})$.
%
Then, we apply the reverse process via the pre-trained diffusion model to achieve the denoised result from $G(x_{t})$, denoted by $D(G(x_{t}))$.
%
Finally, repeating the process above with different paired $(x_0,y_0,z)$, we are able to acquire the dataset for the fusion UNet.
%
Note that similar to the training of \method, the fusion UNet are aimed to deal with \textit{noisy} images.
%
That is to say, empirically we need much more data samples and training epochs since CNN may easily fail on noisy data.
%
For the CelebA-HQ dataset with 30,000 images, we train the fusion UNet with more than 200,000 samples.
%
As a comparison, we train \method within only 60 epochs with 27,000 data samples.
}


\newrevise{
The comparisons between multi-step and our single-step design are reported in \cref{tab:metric} and \cref{tab:time}.
%
From \cref{tab:metric} we observe that under the same timestep, the FID score of multi-step \method is worse than single-step in most cases, and multi-step \method does not significantly benefit from the additional step of performing the fusion UNet.
%
As for SSIM score, there is indeed performance improvement to some extent compared to the vanilla DMT, which is mainly due to doubling the information during denoising process and taking advantage of the fusion UNet.
%
Considering the dramatic additional time cost discussed below, we regard that the vanilla single-step DMT is an adequate solution to the I2I task. 
%
It is noteworthy that for multi-step \method, both the training (including \method and Fusion training) and inference costs increase significantly, as shown in \cref{tab:time}, which confirms the superiority of the single-step \method proposed in the paper.
}


\newrevise{
\yr{\textbf{Asymmetric \method.}}
As for asymmetric \method, in addition to the theoretical analysis in the main paper, we also conducted a comprehensive ablation study focusing on the performance at various timestep pairs $(s,t)$ near $s=t$.
%
As shown in \cref{tab:ablation_s_t_sstar,tab:ablation_s_t_tstar}, our proposed strategy (pair with $s=t$) is capable of achieving on-par or even superior performance across various $(s,t)$ alternatives.
}


\section{More results}

This part shows more qualitative results and compares our \method with existing I2I approaches, including Pix2Pix (GAN-based)~\cite{isola2017image}, TSIT (GAN-based)~\cite{jiang2020tsit} and Palette (DDPM-based)~\cite{saharia2021palette}.
%
We perform evaluation on the tasks of stylization (\cref{fig:portrait_qmupd0-1-0}), image colorization (\cref{fig:afhq}), segmentation to image (\cref{fig:celeba}), and sketch to image (\cref{fig:edges2handbags}), using our handcrafted Anime dataset, AFHQ~\cite{choi2020stargan}, CelebA-HQ~\cite{karras2018progressive}, and Edges2handbags~\cite{zhu2016generative,xie15hed}, respectively.
%
Our method surpasses the other three competitors with higher fidelity (\textit{e.g.}, clearer contours\newrevise{, less artifacts} and more realistic colors \newrevise{as highlighted}), suggesting that our \method manages to bridge the content information provided by the input \revise{condition} and the domain knowledge contained in the pre-trained DDPM.


\begin{figure*}[!ht]
\centering
%\setlength\tabcolsep{2pt}
%\begin{tabular}{cc}
%\begin{minipage}[t]{0.55\linewidth}\includegraphics[width=1\linewidth]{figures/fig_method/method_palette.pdf}\end{minipage} &
%\begin{minipage}[t]{0.45\linewidth}\includegraphics[width=1\linewidth]{figures/fig_method/method_dmt.pdf}\end{minipage} \\
%(a) Our proposed \method & (b) Palette
%\end{tabular}
\includegraphics[width=1.0\textwidth]{figures/fig_portrait_qmupd0-1-0/fig_portrait_qmupd0-1-0.pdf}
\vspace{-15pt}
\caption{
    \textbf{Qualitative comparison} when translating human face images to portraits, using our handcrafted Portrait dataset.
}
\label{fig:portrait_qmupd0-1-0}
\vspace{-5pt}
\end{figure*}
\begin{figure*}[!ht]
\centering
%\setlength\tabcolsep{2pt}
%\begin{tabular}{cc}
%\begin{minipage}[t]{0.55\linewidth}\includegraphics[width=1\linewidth]{figures/fig_method/method_palette.pdf}\end{minipage} &
%\begin{minipage}[t]{0.45\linewidth}\includegraphics[width=1\linewidth]{figures/fig_method/method_dmt.pdf}\end{minipage} \\
%(a) Our proposed \method & (b) Palette
%\end{tabular}
\includegraphics[width=1.0\textwidth]{figures/fig_portrait_qmupd0-1-0/fig_portrait_qmupd0-1-0.pdf}
\vspace{-15pt}
\caption{
    \textbf{Qualitative comparison} when translating human face images to portraits, using our handcrafted Portrait dataset.
}
\label{fig:portrait_qmupd0-1-0}
\vspace{-5pt}
\end{figure*}
\begin{figure*}[!ht]
\centering
%\setlength\tabcolsep{2pt}
%\begin{tabular}{cc}
%\begin{minipage}[t]{0.55\linewidth}\includegraphics[width=1\linewidth]{figures/fig_method/method_palette.pdf}\end{minipage} &
%\begin{minipage}[t]{0.45\linewidth}\includegraphics[width=1\linewidth]{figures/fig_method/method_dmt.pdf}\end{minipage} \\
%(a) Our proposed \method & (b) Palette
%\end{tabular}
\includegraphics[width=1.0\textwidth]{figures/fig_portrait_qmupd0-1-0/fig_portrait_qmupd0-1-0.pdf}
\vspace{-15pt}
\caption{
    \textbf{Qualitative comparison} when translating human face images to portraits, using our handcrafted Portrait dataset.
}
\label{fig:portrait_qmupd0-1-0}
\vspace{-5pt}
\end{figure*}
\begin{figure*}[!ht]
\centering
%\setlength\tabcolsep{2pt}
%\begin{tabular}{cc}
%\begin{minipage}[t]{0.55\linewidth}\includegraphics[width=1\linewidth]{figures/fig_method/method_palette.pdf}\end{minipage} &
%\begin{minipage}[t]{0.45\linewidth}\includegraphics[width=1\linewidth]{figures/fig_method/method_dmt.pdf}\end{minipage} \\
%(a) Our proposed \method & (b) Palette
%\end{tabular}
\includegraphics[width=1.0\textwidth]{figures/fig_portrait_qmupd0-1-0/fig_portrait_qmupd0-1-0.pdf}
\vspace{-15pt}
\caption{
    \textbf{Qualitative comparison} when translating human face images to portraits, using our handcrafted Portrait dataset.
}
\label{fig:portrait_qmupd0-1-0}
\vspace{-5pt}
\end{figure*}





\end{document}


