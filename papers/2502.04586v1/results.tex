\section{Results}
\label{sec_results}

The proposed partitioning strategy was implemented in python, using the Numpy module, and the least squares SLSQP method from the Scipy Optimize module. Results were produced on a MacBook Pro with M1 Max and 64 GB integrated memory. Compute time for each experiment was on the order of seconds.

\subsection{Wing example}
As a representative example, the partitioning strategy was applied to the design of an airplane wing, and to the body panels of an armored vehicle. In addition to producing viable designs for a given set of user inputs, the results presented herein illustrate how designers can use the partitioning algorithm to quickly explore the implications of each parameter on material utilization, cost, and design fidelity. 

In these examples, an exaggerated overlap width of 1.2\% (normalized with respect to the flat part geometry) was used, for clarity. In industry, nominal overlap widths typically vary between 2 and 10 cm, and are dictated by engineering specification and process parameters such as prepreg thickness, part geometry, and tooling design. Note that the exaggerated overlap width restricts the solution space, and reduces the quality of the results.

In Fig.~\ref{fig:wing_example_1}, the algorithm is applied to a generic airplane wing with thickness tolerance between 100\% (left) and 25\% (right). Internally, this is achieved by considering between 2 and 8 plies per bundle. 

\begin{figure*}[h]
    \centering
    \includegraphics[width=1.0\linewidth]{figures/wing_example_1.png}
    \caption{Airplane wing and seam patterns obtained for bundles of $m =$ 4, 6, and 8 plies, with fiber orientations as shown. The prepreg spool width $w_s$ and overlap width $w_{l}$ were 0.3 and 1.2\%, respectively, normalized with respect to the wing length. Performance in terms of laminate thickness variation, average ply width, and the number of explored nodes are plotted on the right.}
    \label{fig:wing_example_1}
\end{figure*}

As the thickness tolerance is tightened, the problem is increasingly constrained, and it is more difficult to ensure that each ply width is maximized. Predictably, the resulting partitioning patterns include more seams and smaller sub-plies as the number of plies per bundle $m$ is increased. 

\subsection{Time complexity and ply sorting}
The local linear programming problem is solved efficiently using the standard simplex method~\cite{ficken2015simplex}. Though it is not strictly polynomial, other linear programming methods are known to be~\cite{megiddo1986complexity}.

The greedy search has time complexity proportional to the number of explored nodes, which itself depends on the time to formulate and the resolvability of the local problem. In practice, this is a function of the number of seams already placed and the number of seam intersections.
To improve performance, the latter can be reduced by adjusting the ordering of plies. Specifically, since large component layups are typically made up of many similarly oriented plies, they can be sorted by orientation, and processed as bundles containing a minimum of unique fiber orientations. This reduces the number of seam intersections, and therefore also reduces the number of linear sub-spaces, and improves local problem resolvability. If possible, only a single orientation should be included in each bundle. Otherwise, selecting two orientations that are as perpendicular as possible is preferable since it minimizes the total overlap area.

To illustrate this optimization, consider that the seam patterns in Fig.~\ref{fig:wing_example_1} correspond to the first 8 plies of a 24-ply layup. If the layup consists of 6 plies for each orientation, better performance can be achieved by processing ply bundles made up of only one or two unique orientations. In Fig.~\ref{fig:wing_example_2}, the seam patterns resulting from 3 bundles of 8 plies, and 2 bundles of 12 plies are shown. In comparison to the 8-ply bundle from Fig.~\ref{fig:wing_example_1}, the number of nodes traversed is considerably lower, even for the larger bundles. Moreover, the resulting seam patterns are of higher quality since all sub-plies have maximal width (excluding end plies). This is true even for the larger 12-ply bundles which achieve an improved thickness variation tolerance of 17\%. 

\begin{figure*}[t]
    \centering
     \includegraphics[width=1.0\linewidth]{figures/wing_example_2.png}
    \caption{Seam patterns obtained for bundles of similarly-oriented plies. The spool width and overlap width were 0.3 and 1.2\%, normalized with respect to the wing length. Stay-outs are shown in black.}
    \label{fig:wing_example_2}
\end{figure*}

\subsection{Global search efficacy}
\label{sec:efficacy}

To ascertain whether the greedy search is sufficient for the global optimization problem, we compare the results to those obtained via beam search. The solver was run until failure (not further partitionable) using synthetic geometry consisting of an infinite number of randomly generated plies and 1-20 randomly positioned square stay-out zones. Solver parameters were $w_s = 0.2$, $w_l = 0.01$, $w_{min}=0.1$. Examples are shown in Fig.~\ref{fig:beam_width}, along with a histogram of the total partitioned length. 
In the beam search, a beam width of 10~000 was used to guarantee the exploration of all design nodes, effectively making the search exhaustive. Remarkably, running the solver 100~000 times revealed no performance improvement whatsoever, and no difference in the final designs. In all cases, this corresponded to an exhaustive search.

We note, however, that the beam search may still be advantageous for more complex design objectives, such as those including structural analysis or other non-convex terms.
\begin{figure}[t]
    \centering
     \includegraphics[width=1.0\linewidth]{figures/beam_search.png}
    \caption{Left: example synthetic problems and final seam patterns. Right: overlaid histograms of total ply area partitioned prior to failure via greedy search and beam search.}
    \label{fig:beam_width}
\end{figure}

\subsection{Vehicle example}
As a second example, to illustrate its practical utility, the proposed method is applied to the body panels of a retro-futuristic armored vehicle. Each of the panels can be fabricated from UD prepreg, using the partitioning method to minimize defects. In Fig.~\ref{fig:truck_example}, 24-ply layups were partitioned with prescribed stay-out zones, spool width of 20\%, normalized to the largest dimension of the axis-aligned geometry bounding box, lap width of 1\% and minimum ply width 5\%. Perpendicular fiber orientations were bundled for efficiency, achieving worst-case tolerance of 16.7\%. For each panel, viable seam patterns were identified within 20~seconds. Note that the exaggerated overlap width was the limiting factor, and much larger ply bundles can be handled with more conservative overlap width.
\begin{figure*}[h]
    \centering
    \includegraphics[width=1.0\linewidth]{figures/cybertruck_example.png}
    \caption{Retro futuristic armored vehicle body panels and their respective seam patters for two-orientation ply bundles.}
    \label{fig:truck_example}
\end{figure*}

\subsection{Design exploration}
In the following example, we demonstrate how the proposed method can be used to explore design trade-offs by plotting the estimated production cost with respect to the manufacturing constraints. Using the same wing model, the total production cost is estimated for a $[0^{\circ}, 60^{\circ}, 120^{\circ}]$ ply bundle over a range of spool widths. Figure~\ref{fig:wing_example_3} illustrates the impact of the material unit cost parameter, as well as the manufacturing cost per seam on the overall cost. 

Naturally, as the spool width decreases, the number of seams in the resulting seam pattern increases, as does material usage and final part weight. However, since spool width is positively correlated with material unit cost, the overall cost can be minimized by finding the optimal balance between spool width and material usage. 

\begin{figure*}[t]
    \centering
    \includegraphics[width=1.0\linewidth]{figures/design_exploration.png}
    \caption{Estimated production cost as a function of spool width. In each plot, $\hat{a} = 5.0$, $\hat{b} = 1.0$, and $\hat{c}_{seam} = 0.01$ unless otherwise specified.}
    \label{fig:wing_example_3}
\end{figure*}
