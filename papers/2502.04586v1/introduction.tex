\section{Introduction}
\label{sec:intro}
Fiber-reinforced laminar composites are increasingly popular in the design of large-scale structures such as wind turbine blades, aircraft fuselages, and other industrial equipment where their excellent strength to weight ratio, design flexibility, and environmental stability offer compelling advantages over traditional materials. However, the complex design and manufacturing pipeline associated with laminar composites remains an impediment to wider adoption~\cite{Grand_View_Research_2024}. 

In broad terms, laminar composite manufacturing involves layering oriented fibers, impregnating them with resin, and consolidating the assembly into a unified part. Although a wide variety of manufacturing methods exist, large-scale components are most commonly fabricated following the \emph{prepreg} method. This method streamlines the process by utilizing spools of axially-aligned reinforcing fibers pre-impregnated with semi-cured resin. Plies are cut from the prepreg material, arranged by hand into a layup, and cured to create the final consolidated laminate part~\cite{staab2015laminar} (Fig~\ref{fig:laminate_illustation}).

\begin{figure}
    \centering
    \includegraphics[width=\linewidth]{figures/laminate_illustration.png}
    \caption{Laminar composite elements: Prepreg ply, composed for matrix-embedded unidirectional fibers is layered in specific orientations during layup to form a laminate with a desired shape and mechanical properties.}
    \label{fig:laminate_illustation}
\end{figure}

\begin{figure}
    \centering
    \includegraphics[width=\linewidth]{figures/process_illustration.png}
    \caption{Design pipeline: Part geometry and composite properties are defined in the engineering design stage. The part is partitioned into flattenable regions and detailed ply designs are generated for each region in the detailed ply design stage.}
    \label{fig:design_pipeline}
\end{figure}

Prior to manufacturing, a comprehensive process planning pipeline translates the intended design into a set of cutting patterns, tooling requirements, process parameters, and assembly instructions. The pipeline can be broadly divided into three stages: \emph{engineering design}, \emph{detailed ply design}, and \emph{manufacturing design}~\cite{fibersim_siemens}. In the first stage, the part geometry, spatially-varying mechanical properties, and specific materials are defined. In the second stage, the geometry is partitioned into developable sections, and detailed ply layups are generated to achieve the desired properties, while incorporating engineering specifications. These two stages are illustrated in Fig.~\ref{fig:design_pipeline}. In the third stage, the required tooling is designed and prepreg cutting patterns, layup sequences, and process parameters are generated.

The inherent complexity of the process planning pipeline has prompted the development of specialized CAD systems such as NX FiberSIM~\cite{fibersim_siemens}, and CATIA Composite Design~\cite{noauthor_catia_2023} to streamline and automated the process. Yet, certain tasks continue to rely heavily on labor-intensive trial-and-error methods. One such task is the partitioning of plies to account for prepreg spool dimensions.

\begin{figure}
    \centering
    \includegraphics[width=\linewidth]{figures/spool.png}
    \caption{Prepreg spool illustration.}
    \label{fig:spool}
\end{figure}

Prepreg spools, from which plies are cut, are wound along the fiber direction, effectively limiting the dimensions of each ply in the fiber-transverse direction (Fig.~\ref{fig:spool}). For large-scale components, it is often necessary to partition plies into smaller sub-plies such that they can be fabricated from available materials. The resulting partition seams between adjacent sub-plies introduce potential weak spots, which can be exacerbated by positioning inaccuracy. Manufacturers mitigate this risk by introducing slight overlaps between adjacent sub-plies. However, these overlaps introduce yet another deviation from the original design intent, and can have a compounding effect when they intersect one another through the thickness of the laminate (Fig.~\ref{fig:teaser}d); thus their placement requires careful consideration \cite{mehdikhani2019voids}. Existing tools can recommend partitioning when plies exceed material dimensions and offer simple seam staggering options, but to the best of our knowledge, offer little in terms of design automation, or optimization.

The present work targets this specific task, and introduces the first automated ply partitioning strategy to meet spool width constraints, while minimizing the adverse effects on part quality. Our core contribution is a ply partitioning algorithm that ensures manufacturability from available materials, while incorporating common design considerations such as spatially-varying geometric tolerances, ease of assembly, and material wastage. 

The geometry and material definition, as well as the prior steps of the detailed ply design stage fall outside of the scope of this paper, and are treated as inputs to the algorithm.
