\section{Overview}
\label{sec:overview}

The proposed algorithm takes as input a set of ply designs, together with user-defined manufacturing constraints. The output is a set of sub-plies, as well as a cutting pattern, optimized according to the provided manufacturing constraints.

\subsection{Manufacturing constraints}
The key manufacturing constraint, and the reason partitioning is necessary in the first place, relates to the width of the prepreg spool. Since spools are wound in the fiber direction as shown in Fig.~\ref{fig:spool}, the spool width constrains the size of each ply in the fiber-transverse direction. The simplest partitioning strategy therefore consists of splitting the flattened representation of each ply along straight lines in the fiber direction. In theory, seams need not be fiber-aligned, nor straight. However, standard practice in industry is to place seams predominantly in the fiber direction. This minimizes the induced strength drop down and maximizes surface coverage for a single sub-ply.

Additional manufacturing constraints restrict seam placement to minimize adverse effects on as-manufactured part quality. These constraints are described below.

\subsubsection{Overlaps}
As mentioned above, manufacturers introduce small overlaps between sub-plies to avoid mechanical strength drop-down. This practice, however, potentially introduces a new issue, since overlaps are locally double thickness. While the effect of a single overlap is generally negligible, issues do arise when overlaps across multiple stacked plies intersect. In this instance, local thickness tolerances and overall surface quality may be significantly affected. It is therefore critical to avoid overlap stacking in the partitioning stage.

\subsubsection{Stay-out zones}
Another common constraint prohibits placement of seams in specific areas, referred to as stay-out zones, where mechanical performance or thickness tolerance is critical. These zones effectively break up the originally continuous design space for each seam offset by restricting ranges which would result in stay-out intersection.

\subsubsection{Prepreg cutting patterns}
In addition to constraints on seam placement, designers are concerned with the shape of the resulting sub-plies. Specifically, small or flimsy sub-plies should be avoided in order to reduce the risk of inaccurate placement during hand layup, transport, or part consolidation. Moreover, small sub-plies, and specifically narrow sub-plies, result in more sparse prepreg cutting patterns, and therefore greater material wastage. Such constraints can be thought of in terms of minimum offsets between seams and the flattened part boundary.

\subsubsection{Production cost and design fidelity}
Prepreg spools are offered in a range of widths, with the cost per unit area generally increasing as the width increases. On the one hand, using narrower prepreg reduces material costs; while on the other, wider prepreg leads to less partitioning and better design fidelity. Designers must carefully weigh the trade-off between production cost and design fidelity. 

\subsection{Manual partitioning}
Manual partitioning, though often trivial, frequently requires several time-consuming iterations. Consider the simple single-orientation example in Fig.~\ref{fig:manual}. A reasonable first attempt involves staggering the seams on each ply by the overlap width and placing each seam group as far apart as possible, according to the spool width. In this case, starting from the bottom, this results in seam group $1$ ends up intersecting stay-out zone $a$. In correcting this issue, seam group $2$ ends up intersecting stay-out zone $b$ due to the spool width constraint. Correcting this, in turn, requires the introduction of seam group $3$, which now intersects stay-out zone $c$. Correcting this causes seam groups $2$ and $3$ to be too close to one another, violating the minimum sub-ply width constraint. Two more iterations are required before a viable design is found. Naturally, this process can become very tedious with more complex examples, and there is no guarantee that it will produce a viable solution, whether or not one exists.

\begin{figure}
    \centering
    \includegraphics[width=1.0\linewidth]{figures/manual.png}
    \caption{Manual partitioning operation sequence beginning from a spool width-driven pattern, and correcting issues sequentially until a viable design is found.}
    \label{fig:manual}
\end{figure}

\subsection{Automated solution strategy}
Assuming plies are partitioned sequentially, and seams within each ply are placed to the right of the previous, then the insertion of a new seam presents a local optimization problem. If there exists a solution, the seam can be inserted, and the procedure can continue placing new seams until the next seam falls outside of the ply polygon, at which point, the process is repeated for the next ply.
Once all plies have been partitioned in this way, a viable solution has been found and the procedure terminates. Intuitively, a greedy strategy such as this may fail to find a viable solution; and a more exhaustive search strategy seems warranted \cite{cormen2022introduction}. Here, if the optimization objective is defined as maximizing the total distance covered by the sub-plies produced from a given set of seams, we observe that the problem is piecewise linear. A reasonable strategy is to search each linear subspace, exploring the design tree heuristically through a beam search. This solution offers a compromise between the efficiency of the greedy search and the robustness of an exhaustive one. We will demonstrate empirically, however, that the beam search does not perform better than the greedy search, and therefore seems unnecessary.


