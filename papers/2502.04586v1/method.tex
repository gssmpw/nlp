\section{Method}
\label{sec:method}

\begin{figure}
    \centering
    \includegraphics[width=\linewidth]{figures/terminology_illustration.png}
    \caption{Example partitioned ply and key terminology. The 2D flattened ply representation is divided along straight seams in the fiber direction. Overlaps extend each resulting sub-ply symmetrically about the seam. Seams are defined in terms of their offset from a ply-specific origin.}
    \label{fig:terminology}
\end{figure}

The proposed design strategy takes as input a stack of flattened ply polygons, along with the spool width $w_s$ and other user-defined manufacturing constraints.
Seams are defined in the 2D space by line equations ${\mathbf{r} = \mathbf{r}_0 + \mathbf{d}_i t}$, where $\mathbf{d}_i$ is the fiber direction of the corresponding ply $i$. We define the design variable associated with seam $i$ of ply $j$ as its offset $x_i^j$ from the ply origin point $O_j$, as shown in Fig.~\ref{fig:terminology}. For each ply, the origin $O_j$ is chosen to ensure ensure that all offsets remain positive-valued.

Next, we represent the width of a sub-ply as the distance between adjacent seams on a given ply as ${x_{i+1}^j-x_{i}^j + w_l}$, where the overlap width $w_l$ term accounts for the fact that sub-plies are extended on either side by $\frac{w_l}{2}$ to create the overlap. The spool width constraint can then be expressed as:
\begin{equation}
    x_{i+1}^j-x_{i}^j+w_l \leq w_{s}.
    \label{eq:max_ply_width}
\end{equation}

\subsection{Overlap stacking}
\begin{figure}
    \centering
    \includegraphics[width=1.0\linewidth]{figures/overlap_illustration.png}
    \caption{Top: Example partition pattern for a four-ply layup. Bottom: Overlap projection highlighting the compounding effect of intersecting seams.}
    \label{fig:overlap_illustration}
\end{figure}

When two or more seams intersect across multiple stacked plies as shown in Fig.~\ref{fig:overlap_illustration},
the thickness increase along the overlaps is compounded, producing an uneven surface. Intersections between two overlaps are largely unavoidable when the dimensions of the flattened plies are much greater than the available spool width. However, their effect is generally small over the full thickness of the laminate. Consider, for example, a laminate made up of 50 plies. A two-overlap stack would result in a thickness variation of 4\%; but as the number of intersecting overlaps increases, the thickness variation approaches 100\%. In practice, manufacturers often define a maximum tolerance for thickness variation $\delta$:
\begin{equation}
    \delta = \frac{t_{max} - t_{min}}{t_{min}},
\end{equation}
which can equivalently be written as the ratio of the maximum number of intersecting overlaps $N$ over the total number of plies $M$.
If one were to prohibit intersections of $N$ overlaps using only linear constraints, then $N$ constraints would be required.
Instead, constraints can be defined in terms of a smaller number of overlaps $n$ over a set of ply bundles, each made up of $m$ plies. In the worst case, intersections from each ply bundle may stack up, resulting in the original thickness variation of $N/M$. With this approach, the required number of plies in each bundle is:
\begin{equation}
    m = \Bigg{\lceil} \frac{Mn}{N} \Bigg{\rceil}.
\end{equation}
With two-overlap stacks being unavoidable, the smallest value of $n$ is 2, and constraints must be defined to avoid intersections between all possible combinations of $n + 1 = 3$ overlaps. Two cases may arise: either at least two seams are parallel, or all seams are non-parallel. In the first case, the intersecting area is non-zero when the parallel overlaps intersect, \emph{i.e.}, when the following inequality is violated:

\begin{equation}
\label{eq:lap_stacking_parallel}
    |x_j^* - x_i^*| \geq w_l.
\end{equation}
where $i$ and $j$ are parallel seams, and $^*$ specifies that they may be from any ply.

\begin{figure}
    \centering
    \includegraphics[width=1.0\linewidth]{figures/triple_overlap_illustration.png}
    \caption{Triple overlap stacking condition between non-parallel overlaps.}
    \label{fig:triple_overlap_illustration}
\end{figure}

In the case of three non-parallel seams, three constraints are required. Consider the situation in Fig.~\ref{fig:triple_overlap_illustration} in which the triple intersection area is highlighted in red. Triple intersection occurs when the double intersection between $i$ and $j$, shown in yellow, intersects $k$. To avoid intersection, the distance between the projection of the $ij$ intersecting area in the direction normal to $k$ must be greater than $\frac{w_l}{2}$.

This distance, denoted $d_{(ij)k}$, can be expressed in terms of the seam offsets $x_i$ $x_j$, and $x_k$
\begin{equation}
    \label{eq:lap_stacking_non_parallel}
    \begin{split}
        x_{ij} = \frac{b_i c_j - b_j c_i}{a_i b_j - a_j b_i}, \\
        y_{ij} = \frac{-a_i x_{ij} - c_i}{b_i}, \\
        d_{(ij)k} = |a_k x_{ij} + b_k y_{ij} + c_k|,
    \end{split}
\end{equation}
where $a_*$, $b_*$, $c_*$ are the standard form line parameters associated with seam $*$, with $a_*^2 + b_*^2 = 1$. This distance can be equivalently be expressed as a linear combination of line constants:
\begin{equation}
    \Big{|} \begin{bmatrix}
    \frac{a_j b_k - a_k b_j}{a_i b_j-a_j b_i} & \frac{a_k b_i - a_i b_k}{a_i b_j-a_j b_i} & 1 \\
    \end{bmatrix} 
    \cdot 
    \begin{bmatrix}
    c_i & c_j & c_k
    \end{bmatrix}^T
    \Big{|} 
    \leq d_{(ij)k}^{min},
\end{equation}
where $c_*$ is the line constant of seam $*$ and is a linear function of the seam offset $x_*$.The constraint is linear so long as  the intersection of seams $i$ and $j$ remains in the half-space defined by seam $k$.

The constant minimum distance is:
\begin{equation}
    d_{(ij)k}^{min} = \Big{|}proj_{k_{\perp}} \frac{w_l}{\sin(\theta_{ij}/2)}\Big{|} - \frac{w_l}{2},
\end{equation}
where $\theta_{ij}$ is the interior angle formed by seams $i$ and $j$. For each seam triplet, three constraints are required, one for each combinations $(ij)k$, $(jk)i$, and $(ki)j$.

Figure~\ref{fig:example_basic} shows two designs that satisfy the overlap stacking constraint. On the left, with all plies sharing a single fiber orientation, the intersection constraint defined by Eq.~\eqref{eq:lap_stacking_parallel}, together with the ply width constraint in Eq.~\eqref{eq:max_ply_width}, effectively stagger the seams such that overlap stacking is avoided entirely. On the right, with three unique fiber orientations, Eq.~\eqref{eq:lap_stacking_non_parallel} avoids triple overlap stacking, while double overlap stacking remains unavoidable.

Although this formulation guarantees that triple intersections are avoided, it may enforce constraints based on intersections outside the ply polygon, thereby over constraining the design space. To circumvent this issue, intersections outside the ply polygon are ignored. If a design update results in a related triple overlap inside the design polygon, the set of constraints for the next seam placement will rectify the violation. This solution requires that the local optimization be rerun on the final solution to ensure that it is indeed valid.

\begin{figure}
    \centering
    \includegraphics[width=1.0\linewidth]{figures/example_basic.png}
    \caption{Seam patterns for laminates with single (left) and multiple (right) fiber orientations.}
    \label{fig:example_basic}
\end{figure}

\subsection{Stay-out zones}
Stay-out zones segment the design space into allowed and prohibited domains. Consider the example in Fig.~\ref{fig:additional_constraints}. For each ply, the projection of the stay-out polygon $S$ onto a coordinate $d_{\perp}^j$ through the ply origin and normal to the fiber direction of ply $j$ defines a domain of prohibited offset values. Mathematically,
\begin{equation}
    x_*^j \notin proj_{d_{\perp}^j} S,
\end{equation}
where $x_*^j$ is the offset of any seam on ply $j$.

Figure~\ref{fig:example_stayouts} shows how the added stay-out zones (gray) would affect the splicing of the laminates from Fig.~\ref{fig:example_basic}.

\begin{figure}[h]
    \centering
    \includegraphics[width=1.0\linewidth]{figures/additional_constraints.png}
    \caption{Left: A stay-out zone and its projection onto the design space corresponding to ply orientations defined in Fig.~\ref{fig:overlap_illustration}. Center: Example seam pattern with poor quality sub-plies highlighted. Right: Fiber orientation-dependent minimum distance constraints (dashed lines) targeting small and flimsy ply conditions.}
    \label{fig:additional_constraints}
\end{figure}

\begin{figure}[h]
    \centering
    \includegraphics[width=1.0\linewidth]{figures/example_stayouts.png}
    \caption{Seam patterns for laminates with single (left) and multiple (right) fiber orientations with stay-out zones (dark gray).}
    \label{fig:example_stayouts}
\end{figure}

\subsection{Ply quality}
An additional consideration with respect to seam placement relates to the size and shape of the resulting sub-plies. Specifically, narrow, small, or flimsy sub-plies such as those shown in Fig.~\ref{fig:additional_constraints} should be avoided to ensure that they do not drift from their intended position during fabrication.

These requirements can be expressed as distance constraints to specific boundary vertices, as shown in Fig.~\ref{fig:additional_constraints} (right). The pertinent vertices for avoiding small sub-plies $V_{small}^j$ on ply $j$ are those whose edges form a cone pointing towards the seam. Formally:
\begin{equation}
    \begin{split}
        V_{small}^j =  & \{{} v | v \in V, \angle e_{i} e_{i+1} \in [0, \ \pi]  \\
        & \wedge \sign(proj_{d_\perp} \ e_i) = \sign(proj_{d_\perp^j} \ e_{i+1}) \}
    \end{split}
\end{equation}
where $V$ is the set of all $n$ ply polygon vertices, $e_i$ and $e_{i+1}$ are neighboring polygon edges, and $\mathbf{d}_\perp^j$ is the fiber-transverse direction.
Similarly, the pertinent vertices for flimsy sub-plies $V_{flim}^j$ are those which form a cone pointing away from the seam:
\begin{equation}
    \begin{split}
        V_{flim}^j = & \{ v | v \in V, \angle e_{i} e_{i+1} \in (\pi, \ 2\pi) \\
        & \wedge \sign(proj_{d_\perp} \ e_i) = \sign(proj_{d_\perp^j} \ e_{i+1}) \} 
        \end{split}
\end{equation}

In each case, an inequality constraint can be added with a user-provided minimum value:
\begin{equation}
    \begin{split}
        |x_*^j - x(V_{k \ flim}^j)| \geq x_{flim}^{min} \\
        |x_*^j - x(V_{k \ small}^j)| \geq x_{small}^{min},
    \end{split}
\end{equation}
where $x(V_{k \ flim}^j)$ is the offset of ply polygon vertex $k$ in $V_{flim}^j$, and $x(V_{k \ small}^j)$ is the offset of ply polygon vertex $k$ in $V_{small}^j$, respectively.

Figure~\ref{fig:example_ply_quality} shows how the inclusion of ply quality constraints alters the seam pattern. Without the ply quality constraints, the green and magenta sub-plies contain small corner sub-plies. With the constraints activated, all sub-plies have a minimum width greater than the user-set value of $w_s/2$.

\begin{figure}[h]
    \centering
    \includegraphics[width=1.0\linewidth]{figures/example_ply_quality.png}
    \caption{Seam patterns generated without (left) and with (right) ply quality constraints. Poor-quality sub-plies are contoured in red.}
    \label{fig:example_ply_quality}
\end{figure}

\subsection{Optimization}
\label{sec:optimization}
With the manufacturing requirements formulated as constraints, we devise an optimization problem for which the objective is to partition each ply into producible sub-plies, using as few seams as possible. If seams are placed one at a time, the local objective is to maximize the coverage for a set of existing seams:
\begin{equation}
\label{eq:strip_width_max}
\begin{split}
    F & = -\sum_{j=1}^m \sum_{i=1}^{n^j-1} (x_{i+1}^j - x_i^j) \\
    & = -\sum_{j=1}^m x_{n}^j
\end{split}
\end{equation}
where $m$ is the number of plies, $x_i^j$ is the design variable associated with the $i$\textsuperscript{th} seam on ply $j$, and $n^j$ is the number of seams on ply $j$. The $x_1^j$ term drops out since it is always set to align with the ply origin and is therefore equal to 0.

With the constraints described previously, the optimization problem is as follows,
\begin{equation}
\begin{aligned}
\minimize \limits_{\mathbf{x}} \quad & -\sum_{j=1}^m x_{n}^j&\\
\textrm{s.t.} \quad & w_{min} \leq x_{i+1}^j - x_{i}^j \leq w_t - w_l, & \forall \ i, j \\
& |x_j - x_i| \geq w_l & \forall s_i^* \parallel s_j^* \\
& d_{(ij)k} \leq d_{(ij)k}^{min} & \forall \ s_i^* \nparallel s_j^* \nparallel s_k^*\\
& x_*^j \notin proj_{d_{\perp}^j} S & \forall j, S \\
& |x_*^j - x(V_{k \ flim}^j)| \geq x_{flim}^{min} & \forall j, k \\
& |x_*^j - x(V_{k \ small}^j)| \geq x_{small}^{min} & \forall j, k \\
& x_i^j \geq 0, & \forall \ i, j \\
\end{aligned}
\label{eq:opt}
\end{equation}

For a given set of seams, the optimization problem, as formulated, is piecewise linear. Therefore, this local problem can be solved efficiently via linear programming. The global problem in which any viable design must also fully partition all plies is discrete and, at first glance, cannot be solved as efficiently. For this, one could adopt an inelegant but robust beam search strategy. Specifically, a design tree in which each node represents a locally optimized set of seams is explored breadth first placing seams in each linear-subspace, keeping a fixed number of best performing designs from each generation. 
The procedure terminates when one of the designs fully partitions all plies. The search strategy is presented graphically in Figs.~\ref{fig:local_optimization_illustration} and \ref{fig:beam_search_illustration}. 

\begin{figure}
    \centering
    \includegraphics[width=1.0\linewidth]{figures/local_optimization_illustration.png}
    \caption{Seam insertion procedure illustrating how the design domain is first divided into linear subdomains based on the current design, then each is locally optimized.}
    \label{fig:local_optimization_illustration}
\end{figure}

\begin{figure}
    \centering
    \includegraphics[width=1.0\linewidth]{figures/beam_search_illustration.png}
    \caption{Illustration of the beam search strategy. Each row of the design tree represents all designs with one additional seam from the previous.}
    \label{fig:beam_search_illustration}
\end{figure}

In the parameter study presented in Section~\ref{sec:efficacy}, it appears that a greedy search always arrives at the same final design as the beam search, regardless of the beam width. Thus, it appears that the beam search is unnecessary and the efficient greedy strategy is sufficient. The greedy solution is presented algorithmically in Alg.~\ref{alg:design}.

\RestyleAlgo{ruled}
\begin{algorithm}[t]
\caption{Generate seam pattern}\label{alg:design}
\KwData{$\{P\}, \{S\}, w_s, w_l, w_{min}$}
\KwResult{Set of seams $\mathbf{x}$}
$\mathbf{x} \gets \{\}$ \\
\For{each ply in \{P\}}{
    \While{last seam inside ply polygon}{
        Identify linear sub-spaces \\
        \While {seam not placed}{
            \For{each linear sub-space, starting from farthest to previous seam}{
                Initialize new seam in sub-space \\
                Optimize all existing seams \\
                \If {optimization success}{
                    $\mathbf{x} \gets solution$
                }
            }
        }
    }
}
\end{algorithm}


\subsection{Cost minimization}
In addition to generating viable designs given a prescribed spool width, engineers may wish to explore multiple spool width options and evaluate the trade offs in terms of design fidelity and manufacturing costs. This section describes an alternative formulation in which the objective function is an estimation of total manufacturing costs. The following linear cost function is proposed:
\begin{equation}
    \label{eq:cost}
    C :=  A_{mat} \ \hat{c}_{mat} + n_{seam} \ \hat{c}_{seam},
\end{equation}
where the first term relates to material cost, and the second one relates to processing cost. The former accounts for the total cost of consumed material, including overlaps and material wastage, while the latter accounts for labor and consumables, which are largely proportional to the number of seams. 
We assume that material cost per unit area $\hat{c}_{mat}$ is a linear function of the spool width. Further, we divide the total material area $A_{mat}$ into design area $A_d$, overlap area $A_l$, and trim loss $A_{trim}$, which is material wasted when ply patterns are cut from the prepreg spool.
$A_d$ is a constant value, defined by the original ply layup design. $A_l$ is a function of the number of seams, the overlap width, and the position of each seam within the part geometry. Here, since the polygon defined by each overlap has length much greater than width, its area can be approximated as that of a rectangle with width equal to the user-defined overlap width. Furthermore, since seams are spread relatively evenly over each ply, the length component can be approximated as constant for each ply. $A_l$ can therefore be approximated as a linear function of the number of seams. Finally, the trim loss term $A_{trim}$ is an estimation of the leftover material after sub-plies have been nested and cut from the spool stock. Since the cutting pattern is generally not a continuous function of the seam offsets, a simple approximation is proposed based on the observation that trim loss tends to be positively correlated with the difference between the width of each sub-ply and the spool width. Figure~\ref{fig:nest} shows an example nesting pattern generated by placing sub-plies sequentially, each time selecting the remaining sub-ply that leaves the smallest gap with the previous one. By considering the 180$^{\circ}$ rotated and mirrored configurations of each sub-ply, this simple strategy produces reasonably compact nesting patterns, without iteration or heavy computation. Notice that the trim loss is concentrated in two areas: (1) at interfaces between horizontally adjacent sub-plies, and (2) above narrower sub-plies that do not extend to the edge of the spool. In real-world applications, where the sub-ply length-to-width ratio is typically greater, the material waste above narrow sub-plies is expected to account for the majority of the trim loss.

\begin{figure}
    \centering
    \includegraphics[width=1.0\linewidth]{figures/nest.png}
    \caption{Example seam pattern and corresponding prepreg cutting pattern (presented in pieces). Trim loss (gray) is concentrated above narrow sub-plies.}
    \label{fig:nest}
\end{figure}

If the sub-plies are approximated as rectangles of length $\Bar{h}_sp$, then the trim loss term can be written as 
\begin{equation}
     A_{trim} = \sum_{j=1}^m \sum_{i=1}^{n^j-1} w_s - (x_{i+1}^j - x_i^j) \ \Bar{h}_{sp}
\end{equation}

Finally, the discrete $n_{seam}$ term in Eq.~\eqref{eq:cost} can be replaced by the continuous function defined in Eq.~\eqref{eq:strip_width_max}.
Eq.~\eqref{eq:cost} is then fully linearized as: 

\begin{equation}
    \label{eq:reduced_cost}
    \begin{split}
         C = (A_d + \sum_{j=1}^m \sum_{i=1}^{n^j-1} w_s - (x_{i+1}^j - x_i^j) \ \Bar{h}_{sp}) \cdot (\hat{a}_{mat} \cdot w_s + \hat{b}_{mat}) \\
         + \hat{c}_{seam} \ \sum_{j=1}^m w_s \cdot (n^j-1) - x_{n}^j
    \end{split}
\end{equation}
where $\hat{a}_{mat}$ and $\hat{b}_{mat}$ are the linear factor and constant relating the material unit cost to the spool width. The solution strategy is nearly identical to the original, with the addition of user-defined minimum $w_s^{min}$ and maximum spool width $w_s^{max}$ parameters, and substitution of the linear solver with a quadratic solver~\cite{nocedal2006quadratic}.
