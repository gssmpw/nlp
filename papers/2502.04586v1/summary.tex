\section{Concluding remarks}
\label{sec_conclusions}

The method presented in this work offers a robust and efficient alternative to manual ply partitioning for large-scale laminar composite manufacturing. By leveraging the developable nature of the detailed ply layups produced in the digital process planning pipeline, the partitioning problem is linearized and solved through a combination of linear programming and heuristic tree search. The proposed algorithm allows users to specify various manufacturing constraints, as well as to explore design options and trade-offs in a matter of minutes, without the need for tedious and unreliable trial-and-error. 

\subsection{Limitations}
While the proposed method can be applied to a wide range of design applications, it is best suited to convex geometry, and may be overly conservative for non-convex problems in which seams intersect multiple boundaries. This could be addressed by subdividing the ply polygon into convex regions. In addition, this method is limited to straight fiber-aligned seams and developable surfaces. The inclusion of fiber-transverse and staircase seams would expand the design space and should be considered in future work. However, extension to non-developable geometry is likely not possible.
%
Future work should also address special cases in which layup designs cannot be straightforwardly discretized into flattenable zones, but which nonetheless can be constructed from flat prepreg plies. The proposed method could be extended for such cases. 
%
Finally, the proposed method does not directly consider the mechanical performance of the as-manufactured part. In other words, we are not explicitly modeling the effects of manufacturing-induced defects. Our approach could potentially be coupled with structural analysis to better estimate design fidelity and structural integrity. In such case, the greedy search may not be sufficient for the global optimization, and the beam search may offer better performance.