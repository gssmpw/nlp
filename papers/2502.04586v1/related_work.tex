\section{Related Work}
\label{sec:related_work}

The ply partitioning problem is one of dividing a set of continuous domains (plies) into discrete subdomains (sub-plies), subject to geometric constraints. Although the particular design constraints are application-specific, geometry partitioning and parametrization are well-studied topics, with many applications in computer-aided design. Here, we present a brief overview of important methods and applications in which process planning has been optimized to suit manufacturing constraints.

Ply partitioning can be thought of as performing cuts on surfaces in 3D. This is reminiscent of subtractive manufacturing methods such as water jet, laser, or wire cutting, in which chunks of material are sequentially removed from a workpiece until the desired shape is revealed. In this context, a general surface is approximated as a piecewise ruled surfaces resulting from straight cuts. A large body of research focuses on fitting general surfaces to well-behaved ones such as piecewise ruled or developable surfaces amenable to specific manufacturing methods\cite{pottmann1999approximation, elber19975, subag2006piecewise}. In the related field of computer numerically control (CNC) machining, process planning involves deconstructing the workpiece into a series of cutting paths, each of which removes a controlled amount of material. Here, partitioning methods must take into account not only the workpiece geometry, but also the machine and cutting tool specifications~\cite{elber1993tool, kim2015precise}. Similarly, in additive manufacturing, digital designs are first partitioned into discrete slabs, and then into tool paths. In this context, specialized methods have been proposed to account for constraints on geometric tolerances~\cite{masood2000part, smith2002optimal, alexa2017optimal}, mechanical performance~\cite{ren2019thermo, tura2022characterization}, surface quality~\cite{thrimurthulu2004optimum, canellidis2006pre, delfs2016optimized, jin2017optimization}, build time~\cite{thrimurthulu2004optimum, canellidis2006pre, delfs2016optimized}, cost~\cite{alexander1998part}, and build volume~\cite{luo2012chopper}.

More closely related to our application, digital methods 
for garment design include strategies for decomposing 3D surfaces into flattenable patches~\cite{wang2002surface, poranne2017autocuts}, sewing pattern optimization for different body shapes~\cite{wolff2023designing, meng2012flexible}, seam durability~\cite{montes2020computational}, and aesthetics~\cite{kwok2015styling}. These methods build on techniques for optimal surface unwrapping via energy minimization~\cite{poranne2017autocuts}, edge clustering for aesthetic purposes~\cite{bruno2009you}, and packing algorithms for waste minimization~\cite{nee1986designing}.
Recently, Pietroni \emph{et al.} proposed an interactive tool for creating sewing patterns from 3D garment models~\cite{pietroni2022computational}. Their work aimed to optimize the position of seams to balance material, aesthetic, and manufacturability-related constraints. Similar to process planning for laminar composites, they partition 3D surfaces into flattenable patches that optimize the trade-off between distortion and total seam length. Seam placement takes into account both the 3D configuration for aesthetic considerations, as well as the 2D configuration for manufacturing constraints.

In the context of laminar composite design, several important distinctions from garment design apply: \emph{1. prepreg is extremely rigid}, and \emph{2. seams are aligned with fiber orientation}. \emph{1} ensures that parts are designed from the outset to avoid non-developable surfaces. Thus, a solution can be developed in the 2D domain. \emph{2} means that seams can be defined by a single variable corresponding to an offset from the origin.

These attributes allow for efficient linear optimization strategies. Specifically, the proposed algorithm casts the problem as a sequence of seam insertion problems defined as piecewise linear systems, and solves it through a combination of linear programming and heuristic tree search.