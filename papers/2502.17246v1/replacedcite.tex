\section{Related Work}
\label{sec:related_work}

\textbf{Sequence Design Methods and Tasks}
Several works have developed offline methods to tackle the problem of sequence design, particularly through reinforcement learning ____, population-based optimisation ____, model-based optimisation ____, deep generative models ____, and evolutionary search ____.
To help provide a level of standardisation for biological sequence design tasks and methods, there has recently been several open-source resources, for tasks:  ProteinGym ____, DesignBench ____, and FLEXS\footnote{\url{https://github.com/samsinai/FLEXS}}, and for methods: Design Baselines ____. Our experiments include some of the methods mentioned above, as well as Design Bench and Design Baselines suites.

\textbf{Evaluation of \textit{in silico} Benchmarks} A recent, important area of research is assessing the physical and chemical plausibility of ML-generated solutions for biological tasks. Previous studies demonstrate that ML-based methods tend to generate physically implausible structures for tasks such as docking ____ and structure-based drug design ____, and these works present a suite of biophysical measures to validate the biological viability of generated complexes. Similarly, prior works have presented biologically-inspired measures for protein design tasks ____. Our work expands the scope to include both protein and DNA tasks, and introduces additional measures for each task setting that are evaluated against both task and external datasets. 

\textbf{Generalisation of Sequence-Scoring Models} The work by ____ discusses how surrogate models can fail to identify true casual and mechanistic links between (parts of) the sequences and the biological property of interest which is scored, leading to poor generalisation to unseen sequences. The findings of this study raise questions about the oracle's generalization ability and we investigate this in our work.