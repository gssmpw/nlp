\begin{abstract}
% The \textsc{Secluded Subgraph} problem asks us to find induced subgraphs with minimal connectivity to the remaining graph while satisfying certain structural constraints. In particular,
Given an undirected graph $G$ and an integer $k$, the \textsc{Secluded $\Pi$-Subgraph} problem asks you to find a maximum size induced subgraph that satisfies a property $\Pi$ and has at most $k$ neighbors in the rest of the graph. This problem has been extensively studied; however, there is no prior study of the problem in directed graphs. This question has been mentioned by Jansen et al. [ISAAC'23].

In this paper, we initiate the study of \textsc{Secluded Subgraph} problem in directed graphs by incorporating different notions of neighborhoods: in-neighborhood, out-neighborhood, and their union. Formally, we call these problems \textsc{\{In, Out, Total\}-Secluded $\Pi$-Subgraph}, where given a directed graph $G$ and integers $k$, we want to find an induced  subgraph satisfying $\Pi$ of maximum size that has at most $k$ in/out/total-neighbors in the rest of the graph, respectively.
We investigate the parameterized complexity of these problems for different properties $\Pi$.
In particular, we prove the following parameterized results:
\begin{itemize}
    \item We design an FPT algorithm for the \textsc{Total-Secluded Strongly Connected Subgraph} problem when parameterized by $k$. 
    \item We show that the \textsc{In/Out-Secluded $\mathcal{F}$-Free Subgraph} problem with parameter $k+w$ is W[1]-hard, where $\mathcal{F}$ is a family of directed graphs except any subgraph of a star graph whose edges are directed towards the center. This result also implies that \textsc{In/Out-Secluded DAG} is W[1]-hard, unlike the undirected variants of the two problems, which are FPT.
    \item We design an FPT-algorithm for \textsc{In/Out/Total-Secluded $\alpha$-Bounded Subgraph} when parameterized by $k$, where $\alpha$-bounded graphs are a superclass of tournaments. 
    \item For undirected graphs, we improve the best-known FPT algorithm for \textsc{Secluded Clique} by providing a faster FPT algorithm that runs in time $1.6181^kn^{\bigO{1}}$. 
\end{itemize}


% In this paper, we investigate different ways to study the \textsc{Secluded Subgraph Problem} in directed graphs. We explore their computational complexity and give FPT-algorithms and hardness proofs when appropriate. We provide a general problem formulation for identifying secluded subgraphs and considering different notions of neighborhoods (incoming edges, outgoing edges, and their union).

% We develop FPT-algorithms for specific directed graph classes, including \textsc{Secluded $\mathcal{F}$-Free Subgraph}, \textsc{Secluded Strongly Connected Subgraph}, and \textsc{Secluded $\alpha$-Bounded Subgraph}. For undirected graphs, we improve the state-of-the-art run-time for \textsc{Secluded Clique} by providing faster algorithms and extending recursive understanding techniques to solve the \textsc{Secluded $d$-Edge-Connected Subgraph} problem. 
% %We show and provide FPT-algorithms for most of those cases, where the problem stays in FPT regardless of the directedness of the input graph.
% For the \textsc{Secluded $\mathcal{F}$-Free Subgraph}, we prove a surprising gap going from FPT in the undirected case to W[1]-hard in the directed---except when $\mathcal{F}$ includes an independent set which we also analyze in depth.

\end{abstract}




