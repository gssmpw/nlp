Let $G$ be a directed graph. For a vertex $v \in \ve{G}$, we denote the \emph{out-neighborhood} by $\outNei{v} = \set{u}{(v,u) \in \e{G}}$ and the \emph{in-neighborhood} by $\inNei{v} = \set{u}{(u,v) \in \e{G}}$.
The \emph{total-neighborhood} is defined as $\nei{v} = \outNei{v} \cup \inNei{v}$.
We use the same notation for sets of vertices $S \subseteq \ve{G}$ as $\outNei{S} = \bigcup_{v \in S} \outNei{v} \setminus S$. 
Furthermore, for all definitions, we also consider their \emph{closed} version that includes the vertex or vertex set itself, denoted by $\coutNei{v} = \outNei{v} \cup \set{v}$.
For a vertex set $S \subseteq \ve{G}$, we write $\induced{G}{S}$ for the subgraph induced by $S$ or $G - S$ for the subgraph induced by $\ve{G} \setminus S$. We also use $G - v$ instead of $G - \set{v}$.

When we refer to a component of a directed graph, we mean a component of the underlying undirected graph, that is, a maximal set that induces a weakly connected subgraph. In contrast, a strongly connected component refers to a maximal set that induces a strongly connected subgraph.
For standard parameterized definitions, we refer to~\cite{cygan2015parameterized}.

% \begin{itemize}
%     \item graphs: induced subgraphs $G - S$, $G-v$. in-out-total neighborhood. closed and open neighborhood. contract. component of directed graph
% \end{itemize}