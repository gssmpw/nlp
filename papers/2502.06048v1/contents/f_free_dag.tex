\label{chap:hardness}

\zzcommand{\free}{\textsc{Out-Secluded $\mathcal{F}$-Free Subgraph}}
\zzcommand{\freet}{\textsc{Total-Secluded $\mathcal{F}$-free Subgraph}}

\begin{comment}
In this section, we consider finding secluded $\mathcal{F}$-free subgraphs, a problem that received significant attention in the undirected setting and was proven FPT with different algorithms~\cite{golovach2020finding,jansen2023single}. For out-neighborhoods, the problem is defined as follows.

\begin{tcolorbox}[enhanced,title={\color{black} {\free{}}}, colback=white, boxrule=0.4pt,
	attach boxed title to top left={xshift=.3cm, yshift*=-2.5mm},
	boxed title style={size=small,frame hidden,colback=white}]
	
	\textbf{Input:}  
  A directed graph $G$, a weight function $\wOp \colon \ve{G} \to \N$, and integers $w,k \in \N$\\

	\textbf{Output:}
  Decide if there is a set $S \subseteq \ve{G}$ such that $\induced{G}{S}$ includes no $F \in \mathcal{F}$ as an induced subgraph, $\abs{\outNei{S}} \le k$, and $\w{S} \ge w$.
\end{tcolorbox}
\end{comment}

\zzcommand{\prob}{\textsc{Out-Secluded $\mathcal{F}$-Free WCS}}
In this section, we show that \free{} is W[1]-hard for almost all choices of $\mathcal{F}$, even if we enforce weakly connected solutions. Except for some obscure cases, we establish a complete dichotomy when \textsc{Out-Secluded $\mathcal{F}$-Free Weakly Connected Subgraph} (\prob{}) is hard and when it is FPT. This is a surprising result compared to undirected graphs and shows that out-neighborhood behaves completely different. 
% In \Cref{sec:ffree_ab}, we consider the setting of $\alpha$-bounded graphs again, where \textsc{Out-Secluded $\mathcal{F}$-Free Weakly Connected Subgraph} is efficiently solvable.
% \jonas{remove subsection heading}
% \iflong
% \subsection{Hardness of \textsc{Out-Secluded $\mathcal{F}$-Free WCS}}\label{sec:hardness_f_free}
% \fi
The same problem was studied before in undirected graphs, where it admits FPT-algorithms using recursive understanding~\cite{golovach2020finding} and branching on important separators~\cite{jansen2023single}. For total neighborhood, both algorithms still work efficiently since the seclusion condition acts exactly the same as in undirected graphs. However, the result changes when we look at out-neighborhood, where we show hardness in \Cref{thm:f-free-hard-always} for almost all choices of $\mathcal{F}$. 

% While for recursive understanding, it is not clear how to apply this approach to a unidirectional neighborhood, the reason that the branching algorithm does not work is more subtle. Intuitively, it is not enough to fix a part of the solution and branch on vertices in its neighborhood or its set of reachable vertices because it might be worth extending the solution by vertices in the in-neighborhood. These do not have to be limited by $k$; however, they can still be added to a weakly connected solution and increase the weight. \todo{S: what are we talking about, since line 483?}

Note that if $\mathcal{F}$ corresponds to the set of all cycles, the problem is finding a secluded DAG, a very natural extension of secluded trees, studied in~\cite{golovach2020finding,donkers2023finding}. Because this set is infinite, we also cannot solve \textsc{Total-Secluded WC DAG} in the same way as the undirected $\mathcal{F}$-free problem. 

To formalize for which kind of $\mathcal{F}$ the problem becomes W[1]-hard, we need one more definition.
\begin{definition}[Inward Star]
  We say that a directed graph $G$ is an \emph{inward star}, if there is one vertex $v \in \ve{G}$ such that $\e{G} = \set{\died{u}{v}}{u \in \ve{G} \setminus \set{v}}$, that is, the underlying undirected graph of $G$ is a star and all edges are directed towards the center.
\end{definition}

Due to the nature of the construction, we show that if no inward star contains any $F \in \mathcal{F}$ as a subgraph, the problem becomes W[1]-hard. Besides this restriction, the statement holds for all non-empty $\mathcal{F}$, even families with only a single forbidden induced subgraph. Later, we explain how some other cases of $\mathcal{F}$ can be categorized as FPT or W[1]-hard.

\restateffree*
\iflong \begin{proof}
\else \begin{proof}[Proof Sketch]
\fi
\begin{figure}
  \centering
  \hfill
  \begin{subfigure}{0.3\textwidth}
    \centering
    \includegraphics[width=0.7\textwidth,page=2]{figures/incidence_graph_reduction}
    \caption{An undirected graph $G$ with maximum clique size 3.}
  \end{subfigure}
  \hfill
  \begin{subfigure}{0.54\textwidth}
    \centering
    \includegraphics[width=0.9\textwidth,page=1]{figures/incidence_graph_reduction}
    \caption{The output graph $G'$ of our reduction function for $G$.}
  \end{subfigure}
  \hfill
  \caption{The construction of the reduction in the proof of \Cref{thm:f-free-hard-always}. Shaded vertices represent a complete graph on $k+2$ copies of $F$, and every $v \in V_V$ is connected to every copy. No solution in $G'$ can include a vertex in a copy of $F$ or $V_V$. Thus, any secluded subset $S$ of $V_E$ of size $\binom{k}{2}$ corresponds the clique $\outNei{S}$ in $G$.}\label{fig:clique_reduction}
\end{figure}

    Let $F \in \mathcal{F}$. We give a reduction from \textsc{Clique}, inspired by~\cite{fomin2013parameterized}. Given an undirected graph $G$ and $k$ and $w$, consider the incidence graph $\incG{G}$, which we modify in the following. We add $k+2$ copies of $F$ and a single vertex $s$. For every edge $e = \uned{u}{v} \in E$, we add the edges $\died{e}{u}$, $\died{e}{v}$, and $\died{e}{s}$. Furthermore, we add an edge from every vertex $v \in V_V$ to every vertex in every copy of $F$. For two different copies $F_1$ and $F_2$ of $F$, we add edges in both directions between every vertex $v_{f1} \in F_1$ and $v_{f2}$ in $F_2$. Denote this newly constructed graph by $G'$. Finally, set $k' \coloneqq k$ and $w \coloneqq \binom{k}{2} + 1$.
    See \Cref{fig:clique_reduction} for a visualization of the graph construction.
\ifshort
    We defer the remaining proof to the full version.
\else
    Notice that $k \le \abs{\ve{G}}$, therefore the construction has polynomial size, and $k' + w$ depends only polynomially on $k$.

    We show that there is a clique of size $k$ in $G$ if and only $G'$ has a weakly connected $k$-out-secluded $\mathcal{F}$-free subgraph of weight at least $w$.
    If $C \subseteq \ve{G}$ forms a $k$-clique in $G$, consider the set of vertices $S$ in $G'$ consisting of $s$ and all edge vertices $v_{\uned{a}{b}}$ where both endpoints $a$ and $b$ lie in the clique $C$. This is clearly weakly connected and has size $\binom{k}{2}+1$. Also, we have $\outNei{S} = C$ and $\abs{C} = k$.

    For the other direction, consider a solution $S$ for $G'$. Notice that no vertex from the copies of $F$ or from $V_V$ can be in $S$; otherwise, $S$ has at least $k+1$ out-neighbors. To ensure the connectivity, if $k > 1$, $s$ must be in $S$. Hence, the out-neighborhood of $S$ must be part of $V_V$. Because $\abs{\outNei{S}} \le k$ and $\abs{S} \setminus \set{s} = \binom{k}{2}$, clearly $S$ must consist of the edges of a clique in $G$.
\fi
\end{proof}

Note that \Cref{thm:f-free-hard-always} is very general and excludes many properties $\propOp$ immediately from admitting FPT-algorithms. %Specifically, it holds also for families with only a single element. 
% \begin{corollary}
%     Let $F$ be a directed graph that is not a subgraph of an inward star. Then, \prname{Out}{$F$-Free WCS} is W[1]-hard with parameter $k + w$ for unit weights.
% \end{corollary}
Another interesting example are DAGs, where we choose $\mathcal{F}$ to be the set of all directed cycles. Although this set is not finite, hardness follows from \Cref{thm:f-free-hard-always}.

\restatedag*

\Cref{thm:f-free-hard-always} shows hardness for almost all cases how $\mathcal{F}$ could look like. We want to get as close to completing all possible cases as possible and, therefore, analyze some of the more difficult cases in the following paragraphs.

\subparagraph*{Trivial Cases}
If $\mathcal{F}$ contains the graph with only a single vertex, the empty set is the only $\mathcal{F}$-free solution. If $\mathcal{F}$ contains two vertices connected by a single edge, weakly connected solutions can only consist of a single vertex or contain bidirectional edges $(u,v)$ and $(v,u)$ for $u,v \in \ve{G}$. Both of these problems are clearly solvable in FPT-time, the second one via our algorithm for \textsc{Secluded Clique}. Additionally, if $\mathcal{F} = \emptyset$, we can clearly choose the maximum weight component of $G$ as the solution.

\subparagraph*{Independent Sets} 
An independent set is a subgraph of an inward star. One kind of graph that cannot be included in $\mathcal{F}$ such that \Cref{thm:f-free-hard-always} shows hardness are independent sets. Surprisingly, the problem becomes FPT if $\mathcal{F}$ includes an independent set of any size. First, notice that independent sets that are part of $\mathcal{F}$ can be ignored if there is a smaller independent set in $\mathcal{F}$. Suppose the size of the smallest independent set is $\alpha+1$. This means that any solution must be an $\alpha$-bounded graph, i.e., a graph without an independent set of size $\alpha$.

We have shown already how these can be enumerated efficiently with the algorithm in the proof of \Cref{thm:alpha_bounded_fpt}. For every enumerated subgraph, we can simply check if it is $\mathcal{F}$-free in time $\abs{\mathcal{F}}n^{\|\mathcal{F}\|}$. Therefore, this case is also FPT with parameter $k$ if $\mathcal{F}$ is finite. 
\begin{theorem}
  Let $\mathcal{F}$ be a finite family of directed graphs that contains an independent set. Then \prob{} is solvable in time $\abs{\mathcal{F}}(2\alpha + 2)^kn^{\|\mathcal{F}\| + \alpha + \bigO{1}}$.
\end{theorem}

\subparagraph*{Inward Stars}
Consider what happens when $\mathcal{F}$ contains an inward star $F$ with two leaves. Then, the weakly connected $F$-free graphs are exactly the rooted trees where the root could be a cycle instead of a single vertex, with potentially added bidirectional edges. For $F \in \mathcal{F}$, we do not have a solution, but we conjecture that the FPT branching algorithm for \textsc{Secluded Tree} by \cite{donkers2023finding} can be transferred to the directed setting. 

If $\mathcal{F} = \set{F}$ for a star $F$ with more than two leaves, the problem remains W[1]-hard. We can slightly modify the construction in the proof of \Cref{thm:f-free-hard-always} such that there is not one extra vertex $s$, but one extra vertex $s_e$ for every $e \in \e{G}$. We connect all $s_e$ internally in a directed path. A solution for \prob{} can then include the whole extra path and the edges encoding the clique. This avoids inward star structures with more than two leaves, showing hardness in this case. However, if $\mathcal{F}$ additionally contains a path as a forbidden induced subgraph, the approach does not work to show hardness. All other cases are resolved, allowing us to conjecture the following dichotomy for a single forbidden induced subgraph.
\begin{conjecture}
  Let $F$ be a directed graph. Then, if $F$ is an independent set or an inward star with at most two leaves, \prob{} is FPT with parameter $k$. Else, \prob{} is W[1]-hard with parameter $k + w$ for unit weights.
\end{conjecture}

\subparagraph*{Remaining Cases}
This resolves almost all possible cases for $\mathcal{F}$, except for a few cases which are difficult to characterize. For example, we could have $\{F_s, F_p\} \subseteq \mathcal{F} $, where $F_s$ is an inward star and $F_p$ is a path, which makes a new connecting construction instead of $s$ and the $s_e$ necessary. The above characterization is still enough to give an understanding of which cases are hard and which are FPT for all natural choices of $\mathcal{F}$. 

% \jonas{Imo, this is the least interesting section, since it only improves on a more general algorithm in some specific cases. Noone ever cares, so should be appendix material.}
\begin{comment}
\subsection{Finding DAGs and \texorpdfstring{$\mathcal{F}$}{F}-Free Subgraphs in \texorpdfstring{$\alpha$}{α}-Bounded Graphs}\label{sec:ffree_ab}

In the last section, we showed that \textsc{Out-Secluded WC DAG} and \prob{} are W[1]-hard in general graphs in almost all cases.
Now, we consider the case when $G$ is a tournament or $\alpha$-bounded. Here, the picture changes. Notice that we can even use \Cref{thm:alpha_eff} from the previous chapter to conclude tractability for both problems on $\alpha$-bounded graphs.
However, we develop more constructive and faster algorithms, first for finding DAGs in tournaments, then for finding $\mathcal{F}$-free subgraphs in $\alpha$-bounded graphs.
Note that we can even ignore the connectivity constraint due to the locality property that we get from \Cref{lem:ind_set_reaches}. Using our lemmas for tournaments and $\alpha$-bounded graphs, we show in this section that \prname{Out}{Weakly Connected DAG} from \Cref{sec:hardness_f_free} is FPT if the base graph is a tournament or general $\alpha$-bounded graph. Recall that in this problem, we are given a directed vertex-weighted graph $G$, and we are looking for a maximum weight subset $S \subseteq V$ that induces a DAG with out-neighborhood $\abs{\outNei{S}} \le k$. For general graphs, this problem turned out to be W[1]-hard in \Cref{thm:f-free-hard-always}. Now, we consider the case when $G$ is a tournament or $\alpha$-bounded.

Remember that a tournament is a directed graph, where every edge is present in exactly one direction. The next two lemmas give some intuition on what DAGs in tournaments and their neighborhoods look like. Note that we refer to directed cycles with 3 edges as triangles.

\iflong
\begin{lemma}
\else
\begin{lemma}[$\star$]
\fi
\label{cor:DAG_triangle_free}
  A tournament graph $T$ is a DAG if and only if $T$ has no cycle of length 3 as an induced subgraph.
  More specifically, if $v \in \ve{T}$ is part of a cycle, $v$ is part of a triangle.
\end{lemma}
\iflong
\begin{proof}
  The first statement is an immediate implication of the second one.
  Let $v_1, v_2, \ldots, v_k$ be the smallest cycle that $v_1$ is part of and suppose $k > 3$. Then, $(v_1, v_3) \in \e{T}$ since otherwise $\died{v_3}{v_1} \in \e{T}$ by the tournament property and $v_1, v_2, v_3$ would be a triangle. Therefore, we can shortcut the cycle of length $k$ to arrive at a cycle of length $k-1$, a contradiction.
\end{proof}
\fi

\iflong
\begin{lemma}
\else
\begin{lemma}[$\star$]
\fi
\label{lem:neighsubset}
  Let $T$ be a tournament, $s \in \ve{T}$, and $v \in \outNei{s}$. Then $v$ is part of a triangle or $\outNei{v} \subset \outNei{s}$.
\end{lemma}
\iflong
\begin{proof}
  Let $u \in \outNei{v} \setminus \outNei{s}$. Then $u, s, v$ is a triangle including $v$. For strict inclusion, notice that $v \in \outNei{s} \setminus \outNei{v}$.
\end{proof}
\fi

These strong structural properties are enough to find secluded DAGs in tournaments using a simple branching algorithm. We use a similar of branching algorithm as in \Cref{chap:tour}, where we initially fix a small part of the solution to turn the problem into a highly local one.

\iflong
\begin{theorem}
\else
\begin{theorem}[$\star$]
\fi
\label{thm:tournament_k}
  \prname{Out}{DAG} is solvable in time $3^k n^{\bigO{1}}$, if $G$ is a tournament.
\end{theorem}
\iflong
\begin{proof}
  \newcommand{\curr}{\coutNei{u}}
    \begin{algorithm}[t]
      \caption{The branching algorithm for \prname{Out}{DAG} that returns a solution including the vertex $u \in \ve{G}$.}
      \label{alg:dag_tour}
      \DontPrintSemicolon
      \SetKwFunction{FMain}{DAG}
      \SetKwProg{Fn}{def}{:}{}
      \Fn{\FMain{$G$, $\omega$, $w$, $k$, $u$}}{
        \uIf{$k < 0$} {
          \textbf{abort}\; 
        }
        \uElseIf{$\curr$ is a solution} {
            \KwRet $\curr$\;
        }
        \ElseIf{there is a triangle $C$ with $C \cap \curr{} \ne \emptyset$} {
            \ForEach{$v \in C$} {
                Call \FMain{$G-v$, $\omega$, $w$, $k-1$, $u$}\;
            }
        }
      }
    \end{algorithm}
    
  Let $(G, \wOp, w, k)$ be an instance of \prname{Out}{DAG} and again guess a vertex $u \in \ve{G}$ that should be the only source in the desired solution $S$. We describe a branching algorithm and execute it for all choices of $u$.
  
  The only branching rule eliminates triangles overlapping $\coutNei{u}$ since at least one of the vertices in the triangle has to be part of $\outNei{S}$.
  Formally, if there is a triangle $C$ with $C \cap \coutNei{u} \ne \emptyset$, we branch on which vertex $v$ of $C$ to put in the neighborhood of $S$.
  This means that we remove $v$ from the graph and decrease $k$ by one. If $k$ decreases below 0, there is no solution. Otherwise if the branching rule is no longer applicable, we show that we have found a solution. The algorithm is also described in \Cref{alg:dag_tour}.
  
  % \subparagraph*{Correctness}
  Since there is always a unique source in a tournament DAG, running the algorithm for all choices of $u$ considers all possible solutions. 
  Notice that the branching rule is correct since one vertex of $C$ must be already in the neighborhood of $u$. Since the triangle cannot be included in a solution, at least one vertex of $C$ must be in the final neighborhood of any solution containing $u$ in the current branch.

  If there are no triangles intersecting $\coutNei{u}$, because of \Cref{lem:neighsubset}, we have $\outNei{\coutNei{u}} = \emptyset$. Also by \Cref{cor:DAG_triangle_free}, $\coutNei{u}$ must be a DAG. Therefore, $\coutNei{u}$ must be the maximum solution in this branch that includes $u$ and the algorithm is correct. 

  % \subparagraph*{Runtime}
  Our branching rule has 3 branches. Its condition can be checked in polynomial time. Furthermore, iterating over all choices of $u \in \ve{G}$ only adds a factor of $n$.
\end{proof}
\fi

\newcommand{\ffree}{$\mathcal{F}$-free}
Next we show how to construct an algorithm for \prob{}, again by branching on forbidden substructures. The algorithm is especially efficient if the graphs in $\mathcal{F}$ are small compared to $\alpha$.
We additionally define $\|\mathcal{F}\|\coloneqq \max_{F \in \mathcal{F}} \abs{\ve{F}}$. Note that we treat $\|\mathcal{F}\|$ as a constant, since $\mathcal{F}$ is part of the problem itself.

\iflong
\begin{theorem}
\else
\begin{theorem}[$\star$]
\fi
\label{thm:ffree_in_ab}
  If $G$ is \ab{}, \free{} is solvable in time $\abs{\mathcal{F}}\cdot \max\set{3^k,(2\|\mathcal{F}\|)^k}n^{\alpha + \|\mathcal{F}\| + \bigO{1}}$.
\end{theorem}
\iflong
\begin{proof}

    \newcommand{\curr}{\coutNei{\coutNei{U}}}
    \begin{algorithm}[t]
      \caption{The branching algorithm for \free{} that returns a solution including the set $U \subseteq \ve{G}$.}
      \label{alg:freeab}
      \DontPrintSemicolon
      \SetKwFunction{FMain}{$\mathcal{F}$-Free}
      \SetKwProg{Fn}{def}{:}{}
      \Fn{\FMain{$G$, $\omega$, $w$, $k$, $U$}}{
        \uIf{$k < 0$} {
          \textbf{abort}\; 
        }
        \uElseIf{$\curr$ is a solution} {
            \KwRet $\curr$\;
        }
        \uElseIf{there is $C \subseteq \curr$ with $\induced{G}{C}$ isomorphic to some $F \in \mathcal{F}$} {
            \If{$C \subseteq U$} {
              \textbf{abort}\; 
            }
            \ForEach{$v \in \bigcup_{w \in C} \shor{w}$} {
                Call \FMain{$G-v$, $\omega$, $w$, $k-1$, $U$}\;
            }
        }
        \ElseIf{there is $w \in \outNei{\curr{}}$} {
            \ForEach{$v \in \shor{w}$} {
                Call \FMain{$G-v$, $\omega$, $w$, $k-1$, $U$}\;
            }
        }
      }
    \end{algorithm}
    
  \newcommand{\alg}{\textsf{$\mathcal{F}$-Free}}
  The algorithm technique is similar to finding $\alpha$-bounded subgraphs as in \Cref{thm:alpha_bounded_fpt}. Remember that we write $\shor{v}$ for the vertices in any shortest path from $U$ to $v$, excluding $U$.
  Let $(G,\wOp,w,k)$ be a \free{} instance. We again initially guess a non-empty independent set $U \subseteq \ve{G}$ and run the algorithm for all choices of $U$. We want to find a solution $S \supseteq U$ such that every $v \in S$ is reachable from $U$ via two edges, that is, $S \subseteq \coutNei{\coutNei{U}}$. We give a recursive branching algorithm that finds an optimal solution under these constraints, which is also described in \Cref{alg:freeab}.
  
    When a vertex should be part of the final neighborhood, we can delete it and decrease $k$ by one. If $k$ decreases below 0, or if $\curr$ is a solution to the instance, we return with a base case. Otherwise we apply the following branching rules and repeat the algorithm for all non-empty independent sets $U \subseteq \ve{G}$ of size at most $\alpha$.
    
    \begin{description}
        \item[Case 1. $\curr$ is not $\mathcal{F}$-free.] In this case, there must be a set $C \subseteq \curr$ that induces a subgraph isomorphic to some $F \in \mathcal{F}$. If $C \subseteq U$ there can clearly not be a solution $S \supseteq U$.
        
        Since not all of $C$ can be part of $S$, there is a vertex $w \in C \setminus S$. This means that either $w$ or a vertex on every path from $U$ to $w$ must be in $\outNei{S}$, that is, a vertex in $\shor{w}$.
        Thus, one of $\bigcup_{w \in I} \shor{w}$ must be part of the out-neighborhood of $S$ and we branch on all of these vertices. For one vertex, delete it and decrease $k$ by 1.
    
        \medskip
        \item[Case 2. $\curr$ is $\mathcal{F}$-free, but has additional neighbors.] Let $w \in \outNei{\curr}$ be one such neighbor.
        Since $w$ is not reachable from $U$ via at most two edges, we should not include it in the solution. Again, we branch on $\shor{w}$, a path of length 3, and one of its vertices must be in $\outNei{S}$.
    \end{description}
  
  % \subparagraph*{Correctness}
  Since any induced subgraphs of an \ab{} graph is also \ab{}, the maximum weight secluded induced \ffree{} subgraph will also be \ab{}.
  Therefore, by \Cref{lem:ind_set_reaches}, there is an independent set $U$ such that for the maximum weight solution $S$ we have $S \subseteq \coutNei{\coutNei{U}}$.
  Thus, if we can find the maximum solution for $U$ if one exists in every iteration with our branching algorithm, the total algorithm is correct.
    The branching rules are a complete case distinction; if none of the rules apply, the algorithm reaches a base case.
    The remaining proof of correctness follows from a simple induction.
  
  % \subparagraph*{Runtime}
  Enumerating all independent sets can be done in time $n^{\alpha+1}$ since $G$ is \ab{}.
  For the first rule, for every $v \in C$, we have $\abs{\shor{v}} = 2$. This yields $2\abs{C} \le 2\|\mathcal{F}\|$ branches. The other rule has 3 branches, all of which lower $k$ by at least 1. All rules can be checked and applied in polynomial time since we consider $\|\mathcal{F}\|$ to be constant. 
\end{proof}
\fi

\begin{comment}
This in itself is already an interesting result.
Note that we could also apply it for finding DAGs in tournaments by \Cref{cor:DAG_triangle_free}, but it would give a worse run time of $6^kn^{\bigO{1}}$.
We now show how to apply this algorithm for finding DAGs in general $\alpha$-bounded graphs.

\iflong
\begin{lemma}
\else
\begin{lemma}[$\star$]
\fi
\label{lem:alpha_short_cycle}
  Let $G = (V,E)$ be $\alpha$-bounded. If $G$ is not acyclic, $G$ has a cycle with at most $2\alpha + 1$ vertices.
\end{lemma}
\iflong
\begin{proof}
  Let $C$ be the shortest cycle in $G$ and suppose for the sake of contradiction that $\abs{C} \ge 2\alpha + 2$.
  Consider every second vertex of $C$, leaving out the last one in case $\abs{C}$ is odd. Name this set $U$. Since $G$ is $\alpha$-bounded, we know that $U$ cannot be an independent set. The cycle does not include edges between vertices of $U$, therefore there must be $u_1, u_2 \in U$ with an edge $(u_1, u_2) \in E$.
  But this allows us to shortcut $C$ into a smaller cycle via this edge $(u_1, u_2)$, a contradiction.
\end{proof}
\fi

If we would define a set $\mathcal{F}$ naively such that DAGs are exactly \ffree{} graphs, then $\mathcal{F}$ has to be the set of all cycles, so $\|\mathcal{F}\| = \infty$. Using \Cref{lem:alpha_short_cycle} however, it is enough find a subgraph that includes no cycle with at most $2 \alpha + 1$ vertices. Therefore a set with $\|\mathcal{F}\| = 2\alpha + 1$ is enough and we arrive at the following result.

\begin{corollary}
    \label{cor:dag_in_alpha_bounded}
  \prname{Out}{Weakly Connected DAG} is solvable is time $(4\alpha + 2)^kn^{\bigO{1}}$ if $G$ is \ab{}.
\end{corollary}

\Cref{thm:ffree_in_ab} holds for total neighborhood as well. The algorithm can be applied in almost the same way. We simply pick $v$ from the total neighborhood instead of the out-neighborhood of $\coutNei{\coutNei{S}}$ in the second branching rule. This ensures that we branch on all possible neighbors.

\begin{corollary}\label{cor:ffreet_in_ab}
  In \ab{} graphs, \freet{} is solvable in time $\max\set{3^k,(2\|\mathcal{F}\|)^k}n^{\alpha + \bigO{1}}$.
\end{corollary}

For DAGs, the most efficient algorithm for general $\alpha$-bounded graphs we know follows from \Cref{thm:alpha_eff}.

\begin{corollary}
  In \ab{} graphs, \textsc{Out-Secluded DAG} is solvable in time $(2\alpha + 2)^kn^{\bigO{1}}$.
\end{corollary}
\fi
\end{comment}


% \subsection{\textsc{Secluded DAG} in Tournaments and \texorpdfstring{$\alpha$}{α}-Bounded Graphs}\label{sec:dag_in_alpha_bounded}

% Using our lemmas for tournaments and $\alpha$-bounded graphs, we show in this section that \prname{Out}{Weakly Connected DAG} from \Cref{sec:hardness_f_free} is FPT if the base graph is a tournament or general $\alpha$-bounded graph. Remember that in this problem, we are given a directed vertex-weighted graph $G$ and we are looking for a maximum weight subset $S \subseteq V$ that induces a DAG with out-neighborhood $\abs{\outNei{S}} \le k$. For general graphs, this problem turned out to be W[1]-hard in \Cref{thm:f-free-hard-always}. Now we consider the case when $G$ is a tournament or $\alpha$-bounded.

% We again start with tournaments. The next two lemmas give some intuition how DAGs in tournaments and their neighborhoods look like. Note that we refer to directed cycles with 3 edges as triangles.

% \begin{lemma}\label{cor:DAG_triangle_free}
%   A tournament graph $T$ is a DAG if and only if $T$ has no cycle of length 3 as an induced subgraph.
%   More specifically, if $v \in \ve{T}$ is part of a cycle, $v$ is part of a triangle.
% \end{lemma}
% \begin{proof}
%   The first statement is an immediate implication of the second one.
%   Let $v_1, v_2, \ldots, v_k$ be the smallest cycle that $v_1$ is part of and suppose $k > 3$. Then, $(v_1, v_3) \in \e{T}$ since otherwise $\died{v_3}{v_1} \in \e{T}$ by the tournament property and $v_1, v_2, v_3$ would be a triangle. Therefore, we can shortcut the cycle of length $k$ to arrive at a cycle of length $k-1$, a contradiction.
% \end{proof}

% \begin{lemma}\label{lem:neighsubset}
%   Let $T$ be a tournament, $s \in \ve{T}$, and $v \in \outNei{s}$. Then $v$ is part of a triangle or $\outNei{v} \subset \outNei{s}$.
% \end{lemma}
% \begin{proof}
%   Let $u \in \outNei{v} \setminus \outNei{s}$. Then $u, s, v$ is a triangle including $v$. For strict inclusion, notice that $v \in \outNei{s} \setminus \outNei{v}$.
% \end{proof}

% These strong structural properties are enough to find secluded DAGs in tournaments using a simple branching algorithm. We use the same style of branching algorithm as before, where we initially fix a small part of the solution to turn the problem into a highly local one.

% \begin{theoremE}[]\label{thm:tournament_k}
%   \prname{Out}{Weakly Connected DAG} is solvable in time $3^k n^{\bigO{1}}$, if $G$ is a tournament.
% \end{theoremE}
% \begin{proofE}
%   \newcommand{\alg}{\textsf{DAG}}
%   Given an \prname{Out}{DAG} instance $(G, \wOp, w, k)$ and a vertex $u \in \ve{G}$, we want to find a solution $S$ to the instance in which $u$ is the only source. We solve this problem with the following algorithm, called $\alg{}(G,\wOp,w,k,u)$, which we execute for all choices of $u \in \ve{G}$.
%   \begin{enumerate}[noitemsep,nolistsep]
%     \item If $k < 0$, abort.
%     \item \label{bra:tour}If there is a triangle $C$ with $C \cap \coutNei{u} \ne \emptyset$, iterate over all three $v \in C$. For each $v$, call $\alg{}(G-v,\wOp,w,k-1,u)$ and stop.
%     \item If $\w{\coutNei{u}} \ge w$, return $\coutNei{u}$. Abort otherwise.
%   \end{enumerate}

%   \subparagraph*{Correctness:}
%   Since there is always a unique source in a tournament DAG, running the algorithm for all choices of $u$ considers all possible solutions. 
%   Notice that the branching rule in \Cref{bra:tour} is correct since one vertex of $C$ must be already in the neighborhood of $u$. Since the triangle cannot be included in a solution, at least one vertex of $C$ must be in the final neighborhood of any solution containing $u$.

%   If there are no triangles intersecting $\coutNei{u}$, because of \Cref{lem:neighsubset}, we have $\outNei{\coutNei{u}} = \emptyset$. Also by \Cref{cor:DAG_triangle_free}, $\coutNei{u}$ must be a DAG. Therefore, $\coutNei{u}$ must be the maximum solution in this branch that includes $u$ and the algorithm is correct. 

%   \subparagraph*{Runtime.}
%   \Cref{bra:tour} has 3 branches. Its condition can be checked in polynomial time. Furthermore, iterating over all choices of $u \in \ve{G}$ only adds a factor of $n$.
% \end{proofE}

% \renewcommand{\ffree}{$\mathcal{F}$-free}
% Before we get to \ab{} graphs, we give an algorithm for a more general problem. Namely, we will reconsider the \ffree{} subgraph problem from \Cref{sec:hardness_f_free}. 

% \zzcommand{\free}{\textsc{Out-Secluded $\mathcal{F}$-free Subgraph}}
% \zzcommand{\freet}{\textsc{Total-Secluded $\mathcal{F}$-free Subgraph}}
% \begin{problem}{\free{}}
%   Given: & A directed graph $G$, a weight function $\wOp \colon V \to \N$, and numbers $w,k \in \N$\\
%   Task: & Decide if there is a set $S \subseteq \ve{G}$ such that $\induced{G}{S}$ includes no $F \in \mathcal{F}$ as an induced subgraph, $\abs{\outNei{S}} \le k$, and $\w{S} \ge w$.
% \end{problem}

% We additionally define $\|\mathcal{F}\|\coloneqq \max_{F \in \mathcal{F}} \abs{\ve{F}}$. Note that we treat $\|\mathcal{F}\|$ as a constant, since $\mathcal{F}$ is part of the problem itself.

% \begin{theorem}\label{thm:ffree_in_ab}
%   If $G$ is \ab{}, \free{} is solvable in time $\abs{\mathcal{F}}\cdot \max\set{3^k,(2\|\mathcal{F}\|)^k}n^{\alpha + \|\mathcal{F}\| + \bigO{1}}$.
% \end{theorem}
% \begin{proof}
%   \newcommand{\alg}{\textsf{$\mathcal{F}$-Free}}
%   The algorithm technique is similar to finding $\alpha$-bounded subgraphs as in \Cref{thm:alpha_bounded_fpt}. 
%   First enumerate all nonempty independent sets of size at most $\alpha$ in $G$.
%   For an independent set $U \subseteq \ve{G}$, we aim to the find the maximum weight secluded \ffree{} subgraph $S \subseteq \ve{G}$ such that every $v \in S$ is reachable from $U$ via two edges, that is $S \subseteq \coutNei{\coutNei{U}}$. For such a set $U$ we run the following algorithm, called $\alg{}(G,\wOp,w,k,U)$.
%   \begin{enumerate}[nolistsep,noitemsep]
%     \item If $k < 0$, abort.
%     \item \label{bra:ffree_alpha1}
%       If there is $C \subseteq \coutNei{\coutNei{U}}$ such that $\induced{G}{C}$ is isomorphic to some $F \in \mathcal{F}$, differentiate the following cases. \begin{enumerate}[nolistsep,noitemsep,label=\alph*.]
%       \item If $C \subseteq U$, abort.
%       \item Otherwise, for each vertex $v' \in \bigcup_{v \in C}\shor{v}$, call $\alg{}(G-v',\wOp,w,k-1,U)$ and stop.
%       \end{enumerate}
%     \item \label{bra:ffree_alpha2}
%       If there is $v \in \outNei{\coutNei{\coutNei{U}}}$, iterate over all $v' \in \shor{v}$, call $\alg{}(G-v',\wOp,w,k-1,U)$ and stop.
%     \item \label{bra:ffree_alpha3} If $\w{\coutNei{\coutNei{U}}} \ge w$, return $\coutNei{\coutNei{U}}$. Abort otherwise.
%   \end{enumerate}

%   \paragraph{Correctness:}
%   Since any induced subgraphs of an \ab{} graph is also \ab{}, the maximum weight secluded induced \ffree{} subgraph will also be \ab{}.
%   Therefore, by \Cref{lem:ind_set_reaches}, there is an independent set $U$ such that for the maximum weight solution $S$ we have $S \subseteq \coutNei{\coutNei{U}}$.
  
%   We now show that the algorithm steps are correct.
%   For \Cref{bra:ffree_alpha1}, note the any copy of an $F \in \mathcal{F}$ cannot be part of a solution. By the same argument as before, this means that one vertex in $\bigcup_{v \in C}\shor{v}$ must be in the neighborhood of the solution, so the rule is correct. If the branching rule is no longer applicable, then $\induced{G}{\coutNei{\coutNei{U}}}$ must be \ffree{}, and so must be any subset.

%   For \Cref{bra:ffree_alpha2}, we limit ourselves to finding \ffree{} subgraphs $S$ such that every $v \in S$ is reachable from $U$ via at most two edges. Therefore, we do not have to consider including $v$ into the solution. Therefore, either $v$ itself must be part of the final neighborhood or some vertex $w \in \shor{v}$.

%   If we reach \Cref{bra:ffree_alpha3} and the weight condition holds, $\coutNei{\coutNei{U}}$ must be the maximum solution for this branch.

%   \paragraph{Runtime:}
%   Enumerating all independent sets can be done in time $n^{\alpha+1}$ since $G$ is \ab{}.
%   For \Cref{bra:ffree_alpha1}, for every $v \in C$, we have $\abs{\shor{v}} = 2$. This yields $2\abs{C} \le 2\|\mathcal{F}\|$ branches. \Cref{bra:ffree_alpha2} has 3 branches, all of which lower $k$ by at least 1. All rules can be checked and applied in polynomial time since we consider $\|\mathcal{F}\|$ to be constant. For the specific term, notice that we can check for every ordered vertex set of size $\|\mathcal{F}\|$ in time $\abs{\mathcal{F}}n^{\bigO{1}}$ if it is isomorphic to an $F \in \mathcal{F}$.
% \end{proof}

% This in itself is already an interesting result. Note that we could also apply it for finding DAGs in tournaments by \Cref{cor:DAG_triangle_free}, but it would give a worse run time of $6^kn^{\bigO{1}}$. We now show how to apply this algorithm for finding DAGs in general $\alpha$-bounded graphs.

% \begin{lemma}\label{lem:alpha_short_cycle}
%   Let $G = (V,E)$ be $\alpha$-bounded. If $G$ is not acyclic, $G$ has a cycle with at most $2\alpha + 1$ vertices.
% \end{lemma}
% \begin{proof}
%   Let $C$ be the shortest cycle in $G$ and suppose for the sake of contradiction that $\abs{C} \ge 2\alpha + 2$.
%   Consider every second vertex of $C$, leaving out the last one in case $\abs{C}$ is odd. Name this set $U$. Since $G$ is $\alpha$-bounded, we know that $U$ cannot be an independent set. The cycle does not include edges between vertices of $U$, therefore there must be $u_1, u_2 \in U$ with an edge $(u_1, u_2) \in E$.
%   But this allows us to shortcut $C$ into a smaller cycle via this edge $(u_1, u_2)$, a contradiction.
% \end{proof}

% If we would define a set $\mathcal{F}$ naively such that DAGs are exactly \ffree{} graphs, then $\mathcal{F}$ has to be the set of all cycles, so $\|\mathcal{F}\| = \infty$. Using \Cref{lem:alpha_short_cycle} however, it is enough find a subgraph that includes no cycle with at most $2 \alpha + 1$ vertices. Therefore a set with $\|\mathcal{F}\| = 2\alpha + 1$ is enough and we arrive at the following result.

% \begin{theorem}\label{cor:dag_in_alpha_bounded}
%   \prname{Out}{Weakly Connected DAG} is solvable is time $(4\alpha + 2)^kn^{\bigO{1}}$ if $G$ is \ab.
% \end{theorem}

% Additionally, \Cref{thm:ffree_in_ab} and therefore also \Cref{cor:dag_in_alpha_bounded} hold for total neighborhood as well. The algorithm can be applied in almost the same way. We simply pick $v$ from the total neighborhood instead of the out-neighborhood of $\coutNei{\coutNei{S}}$ in \Cref{bra:ffree_alpha2}. This ensures that we branch on all possible neighbors.

% \begin{corollary}\label{cor:ffreet_in_ab}
%   In \ab{} graphs, \freet{} is solvable in time $\max\set{3^k,(2\|\mathcal{F}\|)^k}n^{\bigO{1}}$.
% \end{corollary}

