Finding substructures in graphs that satisfy specific properties is a ubiquitous problem. This general class of problems covers many classical graph problems such as finding maximum cliques, Steiner trees, or even shortest paths. Another compelling property to look for in a substructure is its isolation from the remaining graph. This motivated Chechik et al.~\cite{chechik2017secluded}, to introduce the concept of \emph{secluded subgraphs}. 
Formally, in the \textsc{Secluded $\propOp$-Subgraph} problem, given an undirected graph $G$, the goal is to find a maximum size subset of vertices $S \subseteq \ve{G}$ such that the subgraph induced on $S$ fulfills a property $\propOp$ and has a neighborhood $\abs{\nei{S}} \le k$ where $k$ is a natural number.

This problem has been studied extensively for various properties $\propOp$ such as paths~\cite{van2020parameterized,luckow2020computational,fomin2017parameterized,chechik2017secluded}, Steiner trees~\cite{chechik2017secluded,fomin2017parameterized}, induced trees~\cite{donkers2023finding,golovach2020finding}, subgraphs free of forbidden induced subgraphs~\cite{golovach2020finding,jansen2023single}, and more~\cite{bevern207finding}.
Most of these studies focus on the parameterized setting, due to the strong relation to vertex deletion and separator problems, which are foundational in parameterized complexity. Our problem fits in that category since the neighborhood can be considered a $(S,V \setminus S)$-separator. 

% The study of such a problem has, for instance, improved algorithms for problems such as finding maximum cliques and deletion to scattered graph classes~\cite{jansen2023single}.\jonas{We need more examples and refs. The finding max clique claim is not really true } \todo{it seems an incomplete sentence}

% Multiple papers studied the need to find a subgraph that not only satisfies a given property but also minimizes exposure to the rest of the graph~\cite{huffner2009isolation,chechik2017secluded,bevern207finding}. With this constraint, the problem is called \textsc{Secluded Subgraph} problem and was introduced by \cite{chechik2017secluded}. This addition becomes especially relevant in applications like privacy-aware data analysis, detecting isolated communities, and designing robust subsystems. 

% cite me: Secluded Path~\cite{van2020parameterized,luckow2020computational,fomin2017parameterized,chechik2017secluded}
%secluded Steiner trees~\cite{chechik2017secluded,fomin2017parameterized}
% tree: \cite{donkers2023finding,golovach2020finding}
% f-free: \cite{golovach2020finding,jansen2023single}
% secluded IS and many others: \cite{bevern207finding}

%In the design and analysis of networks, finding substructures that satisfy specific properties is a significant problem. 
% This problem was introduced in 1980 by Lewis and Yannakakis~\cite{lewis_node-deletion_1980} as the \textsc{$\Pi$-Subgraph} problem.
%This very general class of problems including the shortest path problem, the steiner tree problem, or the maximum clique problem can be categorized as the \textsc{$\Pi$-Subgraph} problem.
%Later, multiple papers studied the need to find a subgraph that not only satisfies a given property but also minimizes exposure to the rest of the graph~\cite{huffner2009isolation}. \todo{S: need more citations as mentioned multiple papers}
%This addition becomes especially relevant in applications like privacy-aware data analysis, detecting isolated communities, and designing robust subsystems.
%This problem is then called the \textsc{Secluded Subgraph} problem, and it was introduced by~\cite{bevern207finding}, a natural and essential extension of the well-known \textsc{$\Pi$-Subgraph} problem.
%\todo{S: Add some examples of $\Pi$-subgraph problems with citations}
%\jonas{Like the beginning but I think the end should be more concrete in case you don't know about both mentioned problems (i think it's easy to understand the meaning of Pi-Subgraph, but it's not often discussed under this name?}


While the undirected \textsc{Secluded $\Pi$-Subgraph} problem has been explored and widely understood in prior work~\cite{jansen2023single,golovach2020finding,donkers2023finding}, the directed variant has not yet been studied, although it is a natural generalization and was mentioned as an interesting direction by Jansen et al.~\cite{jansen2023single}. Directed graphs naturally model real-world systems with asymmetric interactions, such as social networks with unidirectional follow mechanisms or information flow in communication systems. Furthermore, problems such as \textsc{Directed Feedback Vertex Set}, \textsc{Directed Multicut}, and \textsc{Directed Multiway Cut} underline how directedness can make a fascinating and insightful difference when it comes to parametrized complexity ~\cite{chen2008fixed,hatzel2023fixed,multicut4,chitnis2013fixed}. These problems and their results provide a ground for studying directed secluded subgraph problems, motivating the need to investigate them systematically.

In this paper, we introduce and study three natural directed variants of the \textsc{Secluded $\Pi$-Subgraph} problem, namely \textsc{Out-Secluded $\Pi$-Subgraph}, \textsc{In-Secluded $\Pi$-Subgraph}, and \textsc{Total-Secluded $\Pi$-Subgraph}. These problems aim to minimize either the out-neighborhood of $S$, its in-neighborhood, or the union of both. The out/in-neighborhood of a set $S$ is the set of vertices reachable from $S$ via an outgoing/incoming edge. %These three kinds of neighborhoods were suggested as extensions by~\cite{jansen2023single}, and we also believe that they are the most commonly encountered. 
These problems corresponding to different types of neighborhoods can be encountered in real-life networks. For example, in privacy-aware social network analysis, one might aim to identify a community with minimal external exposure. Similarly, robust substructures with limited connectivity to vulnerable components are critical in cybersecurity. 
% Additionally, total-seclusion models are relevant for studying interactions in ecosystems or supply chains.
These real-world motivations further emphasize the need to explore and formalize directed variants of the \textsc{Secluded $\Pi$-Subgraph} problem. 

\begin{table}[t]
    \centering
    \begin{tabular}{lll}
        \toprule
        \textbf{Problem} & \textbf{In- / Out-Secluded} & \textbf{Total-Secluded}\\
        \midrule
        \textsc{WC $\mathcal{F}$-Free Subgraph} & W[1]-hard \hfill \cref{thm:f-free-hard-always} & FPT \hfill\cite{jansen2023single}\\
        \textsc{WC DAG} & W[1]-hard \hfill \cref{cor:dag_out_k+t} & ?\\
        \textsc{$\alpha$-Bounded Subgraph} & FPT \hfill \cref{thm:alpha_bounded_fpt}& FPT \hfill \cref{thm:alpha_bounded_total}\\
        % \textsc{$\propOp$-Subgraph in $\alpha$-bounded graphs} & FPT \hfill \cref{thm:alpha_eff} & FPT \hfill \cref{thm:alpha_eff_tot}\\
        \textsc{Strongly Connected Subgraph} & ? & FPT \hfill \cref{cor:scc_algo}\\
        \midrule
        \textsc{Clique} & \multicolumn{2}{l}{FPT in time $1.6181^kn^{\bigO{1}}$ \hfill \cref{thm:clique_better}}\\
        \bottomrule
    \end{tabular}
    \caption{Our main results for directed and undirected problems. WC stands for weakly connected, $\propOp$ is an arbitrary, polynomial-time verifiable property. All FPT results are with respect to parameter $k$ and all hardness results are with respect to parameter $k+w$. The problems corresponding to columns marked with $(?)$ are open. 
    \label{tab:our_results}}
\end{table}

For any property $\propOp$, we formulate our general problem as follows. Notice that in-secluded and out-secluded are equivalent for all properties that are invariant under the transposition of the edges. For this reason, we mostly focus on total-neighborhood and out-neighborhood.
\begin{tcolorbox}[enhanced,title={\color{black} {\textsc{X-Secluded $\propOp$-Subgraph} \quad ($\text{X} \in \set{\text{In}, \text{Out}, \text{Total}}$)}}, colback=white, boxrule=0.4pt,
	attach boxed title to top left={xshift=.3cm, yshift*=-2.5mm},
	boxed title style={size=small,frame hidden,colback=white}]
	\textbf{Parameter:} An integer $k \in \N$ 

	\textbf{Input:}  A directed graph $G$ with vertex weights $\omega \colon V \to \N$ and an integer $w \in \N$

	\textbf{Output:} An X-secluded set $S \subseteq \ve{G}$ of weight $\omega(S) \ge w$ that satisfies $\propOp$, or report that none exists. 
\end{tcolorbox} 
%Second paragraph: define Secluded subgraph problem ; give its connections with general theory
%Third paragraph: connection to directed variants (ref of some paper that mentions it as a possible direction)
%Fourth paragraph applications

