In this subsection, we present the results we obtained for the problem. See \Cref{tab:our_results} for an overview of the results we obtained in this paper. 
%\ifshort Due to space constraints, we omit the proofs of statements marked with ($\star$). The missing proofs and details can be found in the full version. \fi

\subparagraph*{Strongly Connected Subgraph.} We show that the \textsc{Total-Secluded Strongly Connected Subgraph} problem is fixed-parameter tractable when parameterized with $k$. Precisely, we prove the following result:
% Connectivity is an important factor in establishing tractability in undirected graphs. Hence, we decided to also analyze it in the case of directed graphs.
%Additionally, connectivity-preserving subgraphs such as paths~\cite{van2020parameterized,luckow2020computational,fomin2017parameterized,chechik2017secluded} and Steiner trees~\cite{chechik2017secluded,fomin2017parameterized} are among the most-studied secluded subgraph problems. 

\begin{restatable}[]{theorem}{test}
\label{cor:scc_algo}
  \textsc{Total-Secluded Strongly Connected Subgraph} is solvable in time $2^{2^{2^{\bigO{k^2}}}}n^{\bigO{1}}$.
\end{restatable}


% In \Cref{sec:scc}, we develop an FPT-algorithm with running time $2^{2^{2^{\bigO{k^2}}}}n^{\bigO{1}}$ for \textsc{Total-Secluded Strongly Connected Subgraph} (\Cref{cor:scc_algo}). This is done using recursive understanding, a technique introduced by~\cite{grohe2011finding} and used successfully for various parameterized problems~\cite{chitnis2016designing,golovach2020finding,cygan2014minimum,lokshtanov2018reducing}.
% The following paragraphs describe the intuition behind recursive understanding algorithms for secluded subgraph problems, following roughly the structure in~\cite{golovach2020finding}.
% A more concrete and formal description of the algorithm structure for \scs{} can be found in \Cref{alg:rec_und_scc}.




% High level intuition for recursive understanding
We design our FPT algorithm for \textsc{Total-Secluded Strongly Connected Subgraph} problem using recursive understanding, a technique introduced by~\cite{grohe2011finding} and recently used successfully for various parameterized problems~\cite{chitnis2016designing,golovach2020finding,cygan2014minimum,lokshtanov2018reducing}.
We visualize the overall structure of the algorithm in \Cref{fig:recursive_calls}.



\begin{figure}[t]
  \centering
  \includegraphics[width=0.6\textwidth,page=2]{figures/recursion}
  \caption{
  An illustration of the general recursive understanding algorithm used in \Cref{sec:scc}.
  There are two recursive calls in total, highlighted with dashed arrows.
  As defined later, only vertices inside $B$ are allowed to be in the neighborhood of a solution.
  $W$ is chosen to be the side of the separation with a smaller intersection with $T$.
  }
  \label{fig:recursive_calls}
\end{figure}


On a high level, recursive understanding algorithms work by first finding a small balanced separator of the underlying undirected graph. If no suitable balanced separator exists, the graph must be highly connected, which makes the problem simpler to solve. In the other case, we carefully reduce and simplify one side of the separator while making sure to keep an equivalent set of solutions in the whole graph. By choosing parameters to subroutines in the right way, this process reduces one side of the separator enough to invalidate the balance of the separator. Therefore, we have made progress and can iterate with another balanced separator or reach the base case.

% Why is recursive understanding applicable to secluded?
In our case, looking for a separator of size at most $k$ makes the framework applicable. Crucially, this is because in any secluded subgraph $G[S]$, where $S \subseteq \ve{G}$, the neighborhood $\nei{S}$ acts as a separator between $S$ and $\ve{G} \setminus \cnei{S}$. Therefore, if no balanced separator of size at most $k$ exists, we can deduce that either $S$ itself or $\ve{G} \setminus S$ must have a small size. This observation makes the problem significantly easier to solve in this case, using the color coding technique developed in~\cite{chitnis2016designing}.

% details and boundary complementations
In the other case, we can separate our graph into two balanced parts, $U$ and $W$, with a separator $P$ of size $\abs{P} \le k$. Now, our goal is to solve the same problem recursively for one of the sides, say $W$ and replace that side with an equivalent graph of bounded size that only contains all necessary solutions.
However, notice that finding subsets of solutions is not the same as finding solutions; the solution $S$ for the whole graph could heavily depend on including some vertices in $U$. That being said, the different options for this influence are limited. At most $2^k$ different subsets $X$ of $P$ could be part of the solution. For any such $X$, the solution can only interact across $P$ in a limited number of ways. For finding strongly connected subgraphs, we have to consider for which pair $(x_1, x_2) \in X \times X$ there already is a $x_1$-$x_2$-path in the $U$-part of the complete solution $S$. 
This allows us to construct a new instance for every such possibility by encoding $X$ and the existing paths into $W$. These auxiliary instances are called \emph{boundary complementations}. We visualize the idea of this construction later in \Cref{fig:scc_bc}.

% hint at extension and contribution
Fundamentally, we prove that an optimal solution for the original graph exists, that coincides in $W$ with an optimal solution to the boundary complementation graph in which $U$ is replaced. Hence, we restrict the space of solutions to only those whose neighborhood in $W$ coincides with the neighborhood of an optimal solution to some boundary complementation. The restricted instance consists of a bounded-size set $B$ of vertices that could be part of $\nei{S}$ and components in $W \setminus B$ that can only be included in $S$ completely or not at all. We introduce graph \emph{extensions} to formalize when exactly these components play the same role in a strongly connected subgraph. Equivalent extensions can then be merged and compressed into equivalent extensions of bounded size.
In total, this guarantees that $W$ is shrunk enough to invalidate the previous balanced separator, and we can restart the process.

\subparagraph*{$\alpha$-Bounded Subgraph \& Clique.}

In the undirected setting, the \textsc{Secluded Clique} problem is natural and has been studied specifically. There is an FPT-algorithm running in time $2^{\bigO{k \log k}}n^{\bigO{1}}$ through contracting twins~\cite{golovach2020finding}.
The previous best algorithm however uses the general result for finding secluded $\mathcal{F}$-free subgraphs via important separators in time $2^{\bigO{k}}n^{\bigO{1}}$~\cite{jansen2023single}. By the use of important separators, they require time at least $4^kn^{\bigO{1}}$. 
This property is naturally generalizable to directed graphs via tournament graphs. We go one step further.

The independence number of an undirected or directed graph $G$ is the size of the maximum independent set in $G$ (or its underlying undirected graph).
If $G$ has independence number at most $\alpha$, we also call it \emph{$\alpha$-bounded}. This concept has been used to leverage parameterized results from the simpler tournament graphs to the larger graph class of $\alpha$-bounded graphs~\cite{sahu2023kernelization,fradkin2015edge,misra2023sub}. We prove the following results:
% Finding a secluded alpha-bounded subgraph within a larger graph answers two important questions, whether there is a large enough dominance-like hierarchy between vertices and how connected is this structure to the rest of the graph. This is particularly useful in strategic network design for instance, where information or influence needs to be mostly contained or not reach a structured group. 

\begin{restatable}[]{theorem}{restateab}
\label{thm:alpha_bounded_fpt}
  \textsc{Out-Secluded $\alpha$-Bounded Subgraph} is solvable in time $(2\alpha + 2)^kn^{\alpha+\bigO{1}}$.
\end{restatable} 

\begin{restatable}[]{theorem}{restateabtotal}
\label{thm:alpha_bounded_total}
  \textsc{Total-Secluded $\alpha$-Bounded Subgraph} is solvable in time $(\alpha + 1)^kn^{\alpha+\bigO{1}}$.
\end{restatable} 

We achieve the goal via a branching algorithm, solving \textsc{Secluded $\alpha$-Bounded Subgraph} for all neighborhood definitions in FPT time (\Cref{thm:alpha_bounded_fpt,thm:alpha_bounded_total}).
Our algorithm initially picks a vertex subset $U \subseteq \ve{G}$ and looks only for solutions in the two-hop neighborhood of $U$. A structural property of $\alpha$-bounded graphs guarantees that any optimal solution is found in this way. 
On a high level, the remaining algorithm depends on two branching strategies. First, we branch on forbidden structures in the two-hop neighborhood of $U$, to ensure that it becomes $\alpha$-bounded. Second, we branch on farther away vertices to reach a secluded set. 

The ideas behind the previous algorithm can in turn be used for the simpler undirected \textsc{Secluded Clique} problem.
By a closer analysis of these two high-level rules, we arrive at a branching vector of $(1,2)$ for \textsc{Secluded Clique}. This results in the following runtime, a drastic improvement on the previous barrier of $4^kn^{\bigO{1}}$.

\begin{restatable}[]{theorem}{restateclique}
\label{thm:clique_better}
  \textsc{Secluded Clique} is solvable in time $1.6181^k n^{\bigO{1}}$.
\end{restatable}

\ifshort Due to space constraints, the details and proofs of \Cref{thm:alpha_bounded_fpt,thm:alpha_bounded_total,thm:clique_better} can be found in the full version.
\else The details and proofs of \Cref{thm:alpha_bounded_fpt,thm:alpha_bounded_total,thm:clique_better} can be found in \Cref{chap:tour}.
\fi

% \iflong In \Cref{sec:clique}, we
% \else We \fi
% develop a new branching algorithm that solves it in time $1.6181^kn^{\bigO{1}}$ (\Cref{thm:clique_better}). 
% Our algorithm initially picks one vertex $u$ and looks only for cliques including $u$, guaranteeing that any solution can only be a subset of $\cnei{u}$. 
% On a high level, the remaining algorithm depends on two branching strategies. Firstly, we branch on forbidden structures in $\cnei{u}$, to ensure that $\cnei{u}$ becomes a clique. Secondly, we decrease the size of the neighborhood of $\cnei{u}$, until $\cnei{u}$ becomes a secluded clique. 
% By differentiating more base cases in these two rules, we arrive at a branching vector of $(1,2)$ for the improved runtime. 

\subparagraph*{$\mathcal{F}$-Free Subgraph.}
In the widely-studied \textsc{Secluded $\mathcal{F}$-Free Subgraph} problem, we are given an undirected graph $G$ and a finite family of graphs $\mathcal{F}$, and we are asked to find an induced secluded subgraph of $G$ that does not contain any graph in $\mathcal{F}$ as an induced subgraph. In undirected graphs, this problem becomes FPT (with parameter $k$) using recursive understanding~\cite{golovach2020finding} or branching on important separators~\cite{jansen2023single} when we restrict it to connected solutions. We study the directed version of the problem and surprisingly, it turns out to be W[1]-hard for almost all forbidden graph families $\mathcal{F}$ even with respect to the parameter $k+w$. Precisely, we prove the following theorem.

\begin{restatable}[]{theorem}{restateffree}
\label{thm:f-free-hard-always}
    Let $\mathcal{F}$ be a non-empty set of directed graphs such that no $F \in \mathcal{F}$ is a subgraph of an inward star. Then, \textsc{Out-Secluded $\mathcal{F}$-Free Weakly Connected Subgraph} is W[1]-hard with respect to the parameter $k+w$ for unit weights.
\end{restatable}

%In \Cref{chap:hardness}, we show that the \textsc{Out-Secluded $\mathcal{F}$-Free Weakly Connected Subgraph} problem is W[1]-hard with parameter $k+w$ for unit weights in directed graphs for almost all families~$\mathcal{F}$ (\Cref{thm:f-free-hard-always}), a surprising difference to the undirected setting.
We establish an almost complete dichotomy that highlights the few cases of families~$\mathcal{F}$ for which the problem remains tractable. One of these exceptions is if~$\mathcal{F}$ contains an independent set of any size, where we employ our algorithm for \textsc{Out-Secluded $\alpha$-Bounded Subgraph}. \Cref{thm:f-free-hard-always} also imply the following result for the directed variant of \textsc{Secluded Tree} problem.

\begin{restatable}[]{corollary}{restatedag}
\label{cor:dag_out_k+t}
    \textsc{Out-Secluded Weakly Connected DAG} is W[1]-hard with parameter $k + w$ for unit weights.
\end{restatable}

% \iflong
% Finally, we develop more efficient branching algorithms for the case of $\alpha$-bounded base graphs in \Cref{sec:ffree_ab}.
% \fi

\subparagraph*{Organization.}

We consider \textsc{Total-Secluded Strongly Connected Subgraph} and prove \Cref{cor:scc_algo} in \Cref{sec:scc}.
\iflong In \Cref{chap:tour}, we give the algorithms for \textsc{Secluded $\alpha$-Bounded Subgraph} and \textsc{Secluded Clique} and proofs of \Cref{thm:alpha_bounded_fpt,thm:alpha_bounded_total,thm:clique_better}.
\fi
The hardness result about \textsc{Out-Secluded $\mathcal{F}$ Weakly Connected Subgraph} in \Cref{thm:f-free-hard-always} is proved in \Cref{chap:hardness}.
\ifshort Due to space constraints, the algorithms for \textsc{Secluded $\alpha$-Bounded Subgraph} and \textsc{Secluded Clique} and proofs of \Cref{thm:alpha_bounded_fpt,thm:alpha_bounded_total,thm:clique_better} can be found in the full version. Proofs of statements marked with ($\star$) are also deferred to the full version.
\fi

\subparagraph*{Notation.}

Let $G$ be a directed graph. For a vertex $v \in \ve{G}$, we denote the \emph{out-neighborhood} by $\outNei{v} = \set{u}{(v,u) \in \e{G}}$ and the \emph{in-neighborhood} by $\inNei{v} = \set{u}{(u,v) \in \e{G}}$.
The \emph{total-neighborhood} is defined as $\nei{v} = \outNei{v} \cup \inNei{v}$.
We use the same notation for sets of vertices $S \subseteq \ve{G}$ as $\outNei{S} = \bigcup_{v \in S} \outNei{v} \setminus S$. 
Furthermore, for all definitions, we also consider their \emph{closed} version that includes the vertex or vertex set itself, denoted by $\coutNei{v} = \outNei{v} \cup \set{v}$.
For a vertex set $S \subseteq \ve{G}$, we write $\induced{G}{S}$ for the subgraph induced by $S$ or $G - S$ for the subgraph induced by $\ve{G} \setminus S$. We also use $G - v$ instead of $G - \set{v}$.

When we refer to a component of a directed graph, we mean a component of the underlying undirected graph, that is, a maximal set that induces a weakly connected subgraph. In contrast, a strongly connected component refers to a maximal set that induces a strongly connected subgraph.
For standard parameterized definitions, we refer to~\cite{cygan2015parameterized}.

\zzcommand{\scs}{\textsc{TSSCS}}

\begin{comment}
In the following subsections, we describe our algorithm in more detail. We first introduce a generalized problem in \Cref{sec:scc_bc}, that receives a vertex set of \emph{boundary terminals} as part of the input and asks to find a solution for every boundary complementation of these vertices. In practice, the boundary terminals will be our separator, or the union of all previous separators in later iterations. In \Cref{sec:unbreak}, we explain how to solve the base case where no balanced separator exists. Next, \Cref{sec:scc_extensions} introduces \emph{extensions} for graphs and an equivalence relation on them as a useful model in which we phrase our reduction rules and the full algorithm in \Cref{sec:solving_scc}. 
\end{comment}

\begin{comment}
\subparagraph*{$d$-Edge-Connected Subgraph}
Finally, we develop another recursive understanding algorithm for the undirected \textsc{Secluded $d$-Edge-Connected Subgraph} problem. Not only is this another property based around connectivity, but 2-edge-connectivity also shares some similarities with strong connectivity in directed graphs because it requires two specific paths in the underlying undirected graph. \jonas{Merge this paragraph with the previous one? Maybe it's strange but since this is only in the long version it might make sense to only hint at this.}
\end{comment}