\label{sec:scc}
In this section, we investigate the \textsc{Total-Secluded Strongly Connected Subgraph} problem, or \scs{} for short. First, we prove that the problem is NP-hard in general graphs, motivating analysis of its parameterized complexity.

% \begin{tcolorbox}[enhanced,title={\color{black} {\textsc{Total-Secluded Strongly Connected Subgraph} (\scs{})}}, colback=white, boxrule=0.4pt,
% 	attach boxed title to top left={xshift=.3cm, yshift*=-2.5mm},
% 	boxed title style={size=small,frame hidden,colback=white}]
	
% 	\textbf{Input:}  A directed graph $G$, a weight function $\wOp \colon \ve{G} \to \N$, and integers $w,k \in \N$\\
% 	\textbf{Output:} Decide if there is a set $S \subseteq \ve{G}$ with weight $\w{S} \ge w$ and total neighborhood size $\abs{\nei{S}} \le k$, such that $\induced{G}{S}$ is strongly connected.
% \end{tcolorbox}


\iflong
\begin{theorem}
\fi
\ifshort
\begin{theorem}[$\star$]
\fi
\label{thm:total_scc_np_hard}%Use the env "theoremE" for "End" and add the two arguments [title_of_theorem][options: end, puts the proof at the end; restate copies the statement of the theorem where the proof is.]
  \scs{} is NP-hard, even for unit weights.
\end{theorem}
\iflong
\begin{proof}
\begin{figure}
  \centering
  \hfill
  \begin{subfigure}{0.49\textwidth}
    \centering
    \includegraphics[width=0.6\textwidth,page=2]{figures/incidence_graph_reduction}
    \caption{An undirected graph $G$ with maximum clique size 3.}
  \end{subfigure}
  \hfill
  \begin{subfigure}{0.49\textwidth}
    \centering
    \includegraphics[width=0.70\textwidth,page=3]{figures/incidence_graph_reduction}
    \caption{The \scs{} instance $G'$ created by our reduction in the proof of \Cref{thm:total_scc_np_hard}. }
  \end{subfigure}
  \hfill
  \caption{A visualization of the reduction in the proof of \Cref{thm:total_scc_np_hard}. %inspired by~\cite{fomin2013parameterized}. Note that $V_E$ is a complete directed graph. No strongly connected subgraph of $G'$ of weight greater than 1 can include a vertex in $V_V$ or $V_E'$. Thus, any subset $S$ of $V_E$ with $\abs{\nei{S}} \le \abs{\e{G}}+k$ of size $\binom{k}{2}$ corresponds to the clique $\nei{S} \cap V_V$ in $G$.
  }\label{fig:clique_reduction_scc}
\end{figure}

  We reduce from the \textsc{Clique} problem which is NP-hard, where given a graph the problem is to check if there is a clique of size at least $k$. Given an instance $(G,k)$ with $k\ge 2$, we reduce it to an instance of \scs{} $(G',w,k')$ as follows:
  
  We create the graph $G'$ with $\ve{G'} \coloneqq \ve{G}\cup V_E \cup V_E'$, where $V_E \coloneqq \set{v_e}{e\in E(G)}$ and $V_E'\coloneqq \set{v_e'}{e\in E(G)}$. For any vertex $x\in V(G')$, the weight of $v$ is 1.
  The new graph has edges \[
  \e{G'} \coloneqq \set{(v_{e_1}, v_{e_2})}{v_{e_1} \ne v_{e_2} \in V_E} \cup \set{(v_e, v_e'), (v_e, v_a), (v_e, v_b)}{e = \set{a,b} \in \e{G}}.
  \] 
    % We add edges to $\e{G'}$ between every possible pair $v_{e_1}, v_{e_2} \in V_E$ in both directions. 
      % For each edge $e = \{a,b\} \in \e{G}$, we add the edges $(v_e, v_e')$, $(v_e, v_a)$, and $(v_e, v_b)$. 
    We set $k' = k + \abs{\e{G}}$ and $w = \binom{k}{2}$. For an illustration of the construction of $G'$, refer to \Cref{fig:clique_reduction_scc}.
 
 
 Next, we prove that $G$ has a clique of size $k$ if and only if $G'$ has a strongly connected subgraph $H$ of weight at least $w$ with the size of total neighborhood of $H$ at most $k'$. For the forward direction, let $C \subseteq \ve{G}$ be a clique of size $k$ in $G$. Then, we can choose $S = \{v_{\{a, b\}} \mid a, b \in C\}$. Since $C$ is a clique, $S \subseteq V_E$. Also, $S$ has weight $w$ since a clique of size $k$ contains $\binom{k}{2}$ edges. Furthermore, the neighborhood of $S$ consists of all $v_e'$ for $v_e \in S$, all $v_e \notin S$, and all $v_a$ for $a \in C$, giving a total size of $k'$. Therefore, $S$ is a solution for \scs{}.

  Let $S$ be a solution for \scs{} in $G'$. For $k \ge 2$, the solution must include at least one vertex from $v_E$ to reach the desired weight. Therefore, no vertex from $\ve{G'} \cup V_E'$ can be included to not violate connectivity. The size of the neighborhood of $S$ will then be $\abs{\e{G}}$ increased by the number of incident vertices to edge-vertices picked in $S$. Since $S$ must be an edge set of size at least $\binom{k}{2}$ with at most $k$ incident vertices, this must induce a clique of size at least $k$ in $G$.
\end{proof}
\fi

The proof of \Cref{thm:total_scc_np_hard} also shows that \scs{} is W[1]-hard when parameterized by $w$, since $w$ in the proof also only depends on the parameter for \textsc{Clique}.

In the following subsections, we describe the recursive understanding algorithm to solve \scs{} parameterized by $k$.
We follow the framework by~\cite{chitnis2016designing,golovach2020finding} and first introduce generalized problems in \Cref{sec:scc_bc}. In \Cref{sec:unbreak}, we solve the case of unbreakable graphs. We introduce graph \emph{extensions} in \Cref{sec:scc_extensions} as a nice framework to formulate our reduction rules and full algorithm in \Cref{sec:solving_scc}.


\subsection{Boundaries and boundary complementations}\label{sec:scc_bc}

In this subsection, we first define an additional optimization problem that is useful for recursion. Then, we describe a problem-specific \emph{boundary complementation}. Finally, we define the auxiliary problem that our algorithm solves, which includes solving many similar instances from the optimization problem.

\zzcommand{\scsrec}{\textsc{Max \scs{}}}
\begin{tcolorbox}[enhanced,title={\color{black} {\scsrec{}}}, colback=white, boxrule=0.4pt,
	attach boxed title to top left={xshift=.3cm, yshift*=-2.5mm},
	boxed title style={size=small,frame hidden,colback=white}]
	
	\textbf{Input:}  
  A directed graph $G$, subsets $I,O,B \subseteq \ve{G}$, a weight function $\wOp \colon \ve{G} \to \N$, and an integer $k \in \N$

	\textbf{Output:} A maximum weight set $S \subseteq \ve{G}$ with $I \subseteq S$, $O \cap S = \emptyset$, $N(S) \subseteq B$, and $\abs{\nei{S}} \le k$, such that $\induced{G}{S}$ is strongly connected, or report that no feasible solution exists. 
\end{tcolorbox} 

Note that this problem directly generalizes the optimization variant of \scs{} since we can just use $I \coloneqq O \coloneqq \emptyset$ and $B \coloneqq \ve{G}$. However, \scsrec{} allows us to put additional constraints on recursive calls, enforcing vertices to be included or excluded from the solution and neighborhood.

\begin{figure}[t]
    \centering
    \hfill
    \begin{subfigure}{0.48\textwidth}
      \centering
      \includegraphics[width=0.9\textwidth,page=2]{figures/bc_solution}
      \caption{A strongly connected subgraph $S$ of a graph $G$ with $k=5$ neighbors. Black vertices are part of $S$.}
    \end{subfigure}
    \hfill
    \begin{subfigure}{0.48\textwidth}
      \centering
      \includegraphics[width=0.9\textwidth,page=3]{figures/bc_solution}
      \caption{The boundary complementation that admits an equivalent feasible solution when setting $k' \coloneqq k - 2$.}\label{fig:scc_bcb}
    \end{subfigure}
    \hfill
    \caption{A visualization of a solution in the original graph and a solution in a boundary complementation, showing how every partial solution in $U$ can be adequately represented by a specific boundary complementation.}\label{fig:scc_bc}
\end{figure}

\begin{definition}[Boundary Complementation]\label{def:scc_border_complementation}
  Let $\mathcal{I} = (G,I,O,B,\wOp,k)$ be a \scsrec{} instance. Let $T \subseteq \ve{G}$ be a set of \emph{boundary terminals} with a partition $X,Y,Z \subseteq T$ and let $R \subseteq X \times X$ be a relation on $X$. Then, we call the instance $(G',I',O',B,\wOp',k')$ a \emph{boundary complementation} of $\mathcal{I}$ and $T$ if
   \begin{enumerate}%[noitemsep,nolistsep]
    \item $G'$ is obtained from $G$ by adding vertices $u_{(a,b)}$ for every $(a,b) \in R$ and edges $(a, u_{(a,b)})$, $(u_{(a,b)}, b)$, and for every $y \in Y$ additionally $(u_{(a,b)}, y)$,
    \item $I' \coloneqq I \cup X \cup \{u_{(a,b)} \mid (a,b) \in R\}$,
    \item $O' \coloneqq O \cup Y \cup Z$,
    \item $\wOp'(v) \coloneqq \w{v}$ for $v \in \ve{G}$ and $\wOp'(u_{(a,b)}) \coloneqq 0$ for $(a,b) \in R$, and
    \item $k' \le k$.\qedhere
  \end{enumerate}
\end{definition}

See \Cref{fig:scc_bc} for an example boundary complementation.
The intuition here should be that if we take the union of $G$ with any other graph $H$ and only connect $H$ to $G$ at the vertices in $T$, then $(X,Y,Z,R)$ encodes all possibilities of how a solution in $G \cup H$ could behave from $G$'s point of view. So, for any solution $S$ to \scsrec{} in $G \cup H$, there is some boundary complementation for $G$ in which we can solve and exchange $S \cap G$ for that solution. Later, we prove a statement that is similar to this intuition.

To employ recursive understanding, we need a boundaried version of the problem. Intuitively, this problem is the same as the previous \scsrec{} but for a small part of the graph we want to try out every possibility, giving many very similar instances. This small part will later represent a separator to a different part of the graph. 

\zzcommand{\scsborder}{\textsc{Boundaried \scsrec{}}}
\begin{tcolorbox}[enhanced,title={\color{black} {\scsborder{}}}, colback=white, boxrule=0.4pt,
	attach boxed title to top left={xshift=.3cm, yshift*=-2.5mm},
	boxed title style={size=small,frame hidden,colback=white}]
	
	\textbf{Input:}  
A \scsrec{} instance $\mathcal{I} = (G,I,O,B,\wOp,k)$ and a set of boundary terminals $T \subseteq \ve{G}$ with $\abs{T} \le 2k$

	\textbf{Output:}
A solution to \scsrec{} for each boundary complementation $\mathcal{I}'$ of $\mathcal{I}$ and $T$, or report that no solution exists.
\end{tcolorbox}

To even have a chance to solve this problem, we need to make sure that there are not too many boundary complementations. The following lemma bounds that number in terms of $k$. %for the case $\abs{T} \le 2k$.

\iflong
\begin{lemma}
\else
\begin{lemma}[$\star$]
\fi
\label{lem:number_border_complementations}
  For a \scsrec{} instance $(G,I,O,B,\wOp,k)$ and $T \subseteq \ve{G}$, there are at most $3^{\abs{T}}2^{\abs{T}^2}(k+1)$ many boundary complementations, which can be enumerated in time $2^{\bigO{\abs{T}^2}}n^{\bigO{1}}$.
\end{lemma}
\iflong
\begin{proof}
  Every element of $T$ has to be in either $X$, $Y$, or $Z$, which gives $3^{\abs{T}}$ possible arrangements. Every one of the $\abs{X}^2$ elements in $X\times X$ can either be in $R$ or not, giving $2^{\abs{X}^2} \le 2^{\abs{T}^2}$ possible arrangements. Furthermore, there are $k+1$ choices for $0 \le k' \le k$. Multiplying these gives the final number. By enumerating all the respective subsets and constructing $G'$ in time $n^{\bigO{1}}$, we can enumerate all boundary complementations in the claimed time.
\end{proof}
\fi

\subsection{Unbreakable Case}\label{sec:unbreak}

This subsection gives the algorithm for the base case of our final recursive algorithm, when no balanced separator exists. We start by giving the definitions of separations and unbreakability.

\begin{definition}[Separation]
  Given two sets $A, B \subseteq \ve{G}$ with $A \cup B = \ve{G}$, we say that $(A, B)$ is a \emph{separation of order $\abs{A \cap B}$} if there is no edge with one endpoint in $A \setminus B$ and the other endpoint in $B \setminus A$.
\end{definition}

\begin{definition}[Unbreakability]
  Let $q,k \in \N$. An undirected graph $G$ is \emph{$(q,k)$-unbreakable} if for every separation $(A,B)$ of $G$ of order at most $k$, we have $\abs{A \setminus B} \le q$ or $\abs{B \setminus A} \le q$.
\end{definition}

The next lemma formalizes that the neighborhood of any solution gives you a separator of order $k$. If the graph is unbreakable, either the solution or everything but the solution must be small. With this insight, the statement should be intuitive. We only need to fill in the details since the graph changes slightly when considering the boundary complementation. However, the proof is identical to~\cite[Lemma~12]{golovach2020finding}, so we omit it.

\begin{lemma}\label{lem:unbreak_small_or_large}
  Let $\mathcal{I}$ be a \scsborder{} instance on a $(q,k)$-unbreakable graph $G$. Then, for each set $S$ in a solution of $\mathcal{I}$, either $\abs{S \cap \ve{G}} \le q$ or $\abs{\ve{G} \setminus S} \le q + k$.
\end{lemma}

For computing these solutions, we use an approach from~\cite{chitnis2016designing} that resembles color coding.

\begin{lemma}[\protect{\cite[Lemma~1.1]{chitnis2016designing}}]\label{lem:find_sets}
  Given a set $U$ of size $n$ and integers $a,b \in \N$, we can construct in time $2^{\bigO{\min\set{a,b}\log (a+b)}}n\log n$ a family $\mathcal{F}$ of at most $2^{\bigO{\min\set{a,b}\log (a+b)}}\log n$ subsets of $U$ such that the following holds. For any sets $A, B \subseteq U, A \cap B = \emptyset, \abs{A} \le a, \abs{B} \le b$, there is a set $S \in \mathcal{F}$ with $A \subseteq S$ and $B \cap S = \emptyset$.
\end{lemma}

We use the previous two lemmas to construct an algorithm for the unbreakable case.

\iflong
\begin{theorem}
\else
\begin{theorem}[$\star$]
\fi
\label{thm:unbreakable_scc}
  \scsborder{} on $(q,k)$-unbreakable graphs can be solved in time $2^{\bigO{k^2 \log(q)}}n^{\bigO{1}}$.
\end{theorem}
\iflong
\begin{proof}
  Our algorithms starts by enumerating all boundary complementations of an instance $\mathcal{I}$. For a \scsrec{} instance $\mathcal{I'} = (G,I,O,B,\wOp,k)$, we know by \Cref{lem:unbreak_small_or_large} that the solution must be either of size at most $q+4k^2$ including the newly added vertices $u_{(a,b)}$, or at least $\abs{\ve{G}} - (q+k+4k^2)$. Let $s = q+k+4k^2 \ge q + 4k^2$.

  This allows us to address the two possible cases for the solution size separately. We give one algorithm to find the maximum weight solution of size at most $s$ and one algorithm to find the maximum weight solution of size at least $\abs{\ve{G}} - s$. In the end, we return the maximum weight of the two, or none if both do not exist. 

  \subparagraph*{Finding a small solution} Our algorithm works as follows. 
  \begin{enumerate}
    \item Apply the algorithm from \Cref{lem:find_sets} with $U = \ve{G}, a = s, b = k$ to compute a family $\mathcal{F}$ of subsets of $\ve{G}$.

    \item For every $F \in \mathcal{F}$, consider the strong components of $\induced{G}{F}$ separately.

    \item For a strong component $Q$, we check if $Q$ is a feasible solution and return the maximum weight one.
  \end{enumerate}

  \subparagraph*{Finding a large solution} For this case, our algorithm looks as follows. 
  \begin{enumerate}
    \item Compute the strong components of $G$.

    \item For every strong component $C$, we construct the graph $G_C$ by taking $\induced{G}{\cnei{C}}$ and adding a single vertex $c$ with edges $(c,v)$, for every $v \in \nei{C}$. 

    \item Then, run the algorithm from \Cref{lem:find_sets} with $U = \ve{G_C}, a = s+1, b = k$ to receive a family $\mathcal{F}$ of subsets of $\ve{G_C}$.

    \item For every $F \in \mathcal{F}$, find the component including $c$.

    \item \label[step]{it:5}For each such component $Q$, check if $Q \setminus \cnei{S}$ is a feasible solution and return the maximum weight one.
  \end{enumerate}

  \subparagraph*{Correctness}
  By \Cref{lem:find_sets}, for any small solution $S$ there is $F \in \mathcal{F}$ with $S \subseteq F$ and $\nei{S} \cap F = \emptyset$, so $S$ must be both a component that is also strongly connected of $\induced{G}{F}$. Therefore, we enumerate a superset of all solutions of size at most $s$ and we find the maximum weight small solution.

  For the large solution, we claim that we find a maximum weight solution for this case. Clearly, every strongly connected subgraph of $G$ must be a subgraph of a strong component of $G$. Let $S'$ be a large solution that is a subset of a strong component $C$. Consider the set $S = \ve{G_C} \setminus \cnei{S'}$ in $G_C$. Then, we must have $\abs{S} \le a$ since $S'$ is large. Also, $\nei{S} \subseteq \nei{S'}$ by definition and therefore $\abs{\nei{S}} \le b$. Thus, $S$ is considered in \Cref{it:5} if it is weakly connected.

  If there is a $v \in S$ that is not in the same component as $c$, we take the component of $v$ in $S$ and include it in $S'$. Then, $S'$ is still strongly connected but has a strictly smaller neighborhood size and equal or greater weight since weights are non-negative. We can repeat this procedure until $S$ is a single component and will be enumerated by the algorithm. Therefore, our algorithm finds a solution of weight at least $\w{S'}$.

  \subparagraph*{Total Runtime} Both cases make use of at most $n$ calls to the algorithm from \Cref{lem:find_sets} with some small modifications. For every returned sets, both algorithms compute the components and verify strong connectivity. Therefore, we can bound the runtime per boundary complementation by $2^{\bigO{\min\set{s,k}\log(s+k)}}n^{\bigO{1}} = 2^{\bigO{k\log(q+k)}}n^{\bigO{1}}$. By \Cref{lem:number_border_complementations}, enumerating all boundary complementations adds a factor of $2^{\bigO{k^2}}n^{\bigO{1}}$, which gives the desired runtime.
\end{proof}
\fi

\subsection{Compressing Graph Extensions}\label{sec:scc_extensions}

Before we give the complete algorithm, we need one more ingredient that will be necessary to apply a set of reduction rules later in the algorithm. This ingredient will be a routine that compresses a subgraph into a subgraph whose size depends only on the rest of the graph and that is equivalent in terms of forming strongly connected subgraphs. To achieve this goal, we formally define sufficient properties to reason about this equivalence and bound the number of equivalence classes.
First, we define the notion of a graph \emph{extension}, a way to extend one graph with another. This concept allows us to speak more directly about graph properties before and after exchanging a part of the graph with a different one.

\begin{definition}[Extension]
Given a directed graph $G$, we call a pair $\dexpair$ an \emph{extension of} $G$ if $D$ is a directed graph and $\dexset \subseteq (\ve{G} \times \ve{D}) \cup (\ve{D} \times \ve{G})$ is a set of pairs between $G$ and $D$. 
We name the graph $\ex{G}{D}{\dexset} \coloneqq (\ve{G} \cup \ve{D}, \e{G} \cup \e{D} \cup \dexset)$, that can be created from the extension, $G$ \emph{extended by} $\dexpair$.
\end{definition}

Note that $\dexset$ is a set of arbitrary pairs with one element in $\ve{G}$ and one in $\ve{D}$. However, we use extensions to construct extended graphs.
Intuitively, an extension of $G$ is a second graph $D$ together with an instruction $\dexset$ on how to connect $D$ to $G$.

Next, we identify three important attributes of extensions in our context. Later, we show that these give a sufficient condition on when two extensions form the same strongly connected subgraphs.
  For this, consider a directed graph $G$ with an extension $\dexpair$. For $U \subseteq \ve{D}$, we write $\inNeiG{\dexset}{U}$ as a shorthand for $\inNeiG{\ex{G}{D}{\dexset}}{U}$, that is, all $v \in \ve{G}$ with $(v,u) \in \dexset$ for some $u \in U$. Define $\outNeiG{\dexset}{v}$ analogously.
  Write $\scc{D}$ for the \emph{condensation} of $D$, where every strongly connected component $C$ of $D$ is contracted into a single vertex.
  Define $\sourcee{D}{\dexset}, \sinke{D}{\dexset} \subseteq \pot{\ve{G}}$ such that 
  \begin{align*}
    \sourcee{D}{\dexset} &\coloneqq \set{\inNeiG{\dexset}{U}}{U \subseteq \ve{D} \text{ is a source component in } \scc{D}} \text{ and}\\
    \sinke{D}{\dexset} &\coloneqq \set{\outNeiG{\dexset}{U}}{U \subseteq \ve{D} \text{ is a sink component in } \scc{D}},
  \end{align*}
  that is, for every strongly connected source component $C$ in $\scc{D}$, $\sourcee{D}{\dexset}$ contains the set of all $v \in \ve{G}$ such that $\died{u}{v} \in \dexset$ for some $u \in C$ and analogously for $\sinke{D}{\dexset}$.
%
  Furthermore, define \[\conne{D}{\dexset} \coloneqq \set{(a,b) \in \ve{G}^2}{\text{there is a $d_1$-$d_2$-path in $D$ with } (a,d_1), (d_2,b) \in \dexset}, \] that is, all $(a,b)$ such that there is an $a$-$b$-path in $\ex{G}{D}{\dexset}$, whose intermediate vertices and edges belongs to D. Refer to \Cref{fig:extension_compression} for examples of extensions and the three sets.

\begin{definition}[Equivalent Extensions]
    Let $G$ be a directed graph. We say that two extensions $\expairi{1}$ and $\expairi{2}$ of $G$ are \emph{equivalent} if \[(\source{D_1,E_{GD_1}}, \sink{D_1,E_{GD_1}}, \conn{D_1,E_{GD_1}}) = (\source{D_2,E_{GD_2}}, \sink{D_2,E_{GD_2}}, \conn{D_2,E_{GD_2}}).\]% We also denote this by $\expair{G}{D_1} \sccequiv{G} \expair{G}{D_2}$.
\end{definition}

The name is justified; clearly, extension equivalence defines an equivalence relation.
The next statement reveals the motivation behind the definition of extension equivalence. It gives us a sufficient condition for two extensions being exchangeable in a strongly connected subgraph.

% \iflong
\begin{lemma}
% \else
% \begin{lemma}[$\star$]
% \fi
\label{lem:source_sink_conn_equiv}
  Let $G$ be a directed graph with two equivalent extensions $\expairi{1}$ and $\expairi{2}$. Let $U \subseteq \ve{G}$ be nonempty such that the extended graph $\ex{\induced{G}{U}}{D_1}{\exset{\induced{G}{U}}{D_1}}$ is strongly connected. Then $\ex{\induced{G}{U}}{D_2}{\exset{\induced{G}{U}}{D_2}}$ is also strongly connected.
\end{lemma}
% \iflong
\begin{proof}
  We construct a $v_1$-$v_2$-path for all $v_1, v_2 \in U \cup \ve{D_2}$ that only uses edges in $\e{D_2}$, $\exset{G}{D_2}$, and $\induced{G}{U}$ by case distinction.

\begin{description}
    \item[Paths $U \to U$.]  Let $u_1, u_2 \in U$. If there is a path from $u_1$ to $u_2$ in $\induced{G}{U}$, this path also exists after exchanging $\expair{G}{D_1}$ to $\expair{G}{D_2}$. If the path passes through $D_1$, since $\conn{D_1,E_{GD_1}} = \conn{D_2,E_{GD_2}}$, we can exchange all subpaths through $D_1$ by subpaths through $D_2$.

    \item[Paths $\ve{D_2} \to U$.] Let $v \in \ve{D_2}, u \in U$. We construct a $v$-$u$-path by first walking from $v$ to any sink component $T$ in $\scc{D_2}$. If there is no edge $(t,u') \in \exset{G}{D_2}$ with $t \in T, u' \in U$ that we can append, since $\sink{D_1,\expair{G}{D_1}} = \sink{D_2,\expair{G}{D_2}}$, there must also be a sink component in $\scc{D_1}$ with no outgoing edge to $U$. However, this is a contradiction to the fact that $\ex{\induced{G}{U}}{D_1}{\exset{\induced{G}{U}}{D_1}}$ with nonempty $U$. Therefore, we can find a $(t,u')$ to append for some $t \in T, u' \in U$. From $u'$, there is already a path to $u$, as proven in the first case.

    \item[Paths $U \to \ve{D_2}$.] Next, we construct a $u$-$v$-path backwards by walking from $v$ backwards to a source $s$ in $D_2$. Analogously, there is an edge $(u',s) \in \exset{G}{D_2}$ for some $u' \in U$ since $\source{D_1,E_{GD_1}} = \source{D_2,E_{GD_2}}$, which we append. From $u$, there is a path to $u'$, as proven in the first case, which we prepend to the rest of the path.

    \item [Paths $\ve{D_2} \to \ve{D_2}$.] Let $v_1, v_2 \in \ve{D_2}$. To construct a $v_1$-$v_2$-path, we can just walk from $v_1$ to any $u \in U$ and from there to $v_2$ as shown before.\qedhere
\end{description}
\end{proof}
% \fi

Furthermore, observe that the union of two extensions creates another extension where source, sink and connection sets correspond exactly to the union of the previous sets. Hence, the union of two equivalent extensions will again be equivalent. This fact is formalized in the next observation and  will turn out useful in later reduction rules.

\begin{figure}
    \centering
    \includegraphics[width=0.65\linewidth]{figures/extension_comp.pdf}
    \caption{Two example extensions of a graph $G$. Observe that $\source{D,\dexset} = \set{\set{v_1}}$, $\sink{D,\dexset} = \set{\set{v_3}}$, and $\conn{D,\dexset} = \set{(v_1,v_2), (v_1, v_3)}$. The extension $(D', E_{GD'})$ not only has the same sets $\sourceOp, \sinkOp, \connOp$ and is thereby equivalent; it is also the compressed extension of $\dexpair$. Since all sources $d_1,d_2,d_3$ have the same in-neighborhood, they are represented by the single vertex $v_S$.}
    \label{fig:extension_compression}
\end{figure}

\begin{observation} \label{lem:union_equiv_if_equiv}
  Let $G$ be a directed graph with two equivalent extensions $\expairi{1}$ and $\expairi{2}$. Consider the extension defined by $D \coloneqq (\ve{D_1} \cup \ve{D_2}, \e{D_1} \cup \e{D_2})$ and $\exset{G}{D} \coloneqq \exset{G}{D_1} \cup \exset{G}{D_2}$. Then $\dexpair$ is equivalent to $\expairi{1}$ and $\expairi{2}$.
\end{observation}

Now, we finally define our compression routine, which compresses an extension to a bounded size equivalent extension.
That means that we have to ensure that neighborhoods of source components and sink components, as well as achieved connections, stay the same. Furthermore, we want to maintain properties such as strong connectivity and weak connectivity. If an extension is strongly connected, it is easy to convince yourself that it is always possible to compress the extension to a single vertex. Otherwise, we have to be more careful. We add one source vertex per neighborhood set in $\source{D,\dexset}$ as well as one sink vertex per neighborhood set in $\sink{D,\dexset}$, realizing the same $\sourceOp$ and $\sinkOp$. Then, we add vertices in between suitable source and sink vertices to realize exactly the same connections in $\connOp$ without creating additional ones. 
The result of a compression is visualized in \Cref{fig:extension_compression}. Now, we describe the procedure formally. 

% \begin{figure}[t]
%   \begin{minipage}[c]{0.64\linewidth}
%       \centering
%       \begin{subfigure}{0.49\textwidth}
%         \includegraphics[width=\textwidth,page=1]{figures/scc_comp}
%         \caption{An extension $\dexpair$}\label{fig:scc_comp1}
%       \end{subfigure}
%       \hfill
%       \begin{subfigure}{0.49\textwidth}
%         \includegraphics[width=\textwidth,page=2]{figures/scc_comp}
%         \caption{$\dcompG$}\label{fig:scc_comp2}
%       \end{subfigure}
%       \caption{An example for compressing an extension.
%       %Note that the source set still only contains $\set{v_1}$, the sink set only contains $\set{v_3}$, and the connections are still $(v_1, v_2)$ and $(v_1, v_3)$.
%       While the size of the extension increases in this example, in general, the size of a compressed extension can be bounded by $\abs{\ve{G}}$.}\label{fig:scc_comp}\jonas{make the graph a bit bigger such that comp is smaller}
%   \end{minipage}
%   \hfill
%   \begin{minipage}[c]{0.34\linewidth}
%       \centering
%       \includegraphics[width=0.8\textwidth]{figures/IOB_fig}
%       \caption{A visualization of how the sets $I$, $O$, and $B$ can overlap after the application of \Cref{red:in_out}. Any component in $G-B$ can be in $I$, $O$, or none of them, but not both. The black vertices form a feasible solution.}\label{fig:iob}
%   \end{minipage}
% \end{figure}


\begin{comment}
\begin{definition}[Compression]\label{def:comp}
  Let $G$ be a directed graph with an acyclic extension $\dexpair$ and $\dsccset \ne \emptyset$. We define the \emph{compressed extension}, denoted as $\dcompG$, as follows.

  If $\abs{\ve{D}} = 1$, we set $\dcompG \coloneqq \dexpair$.\jonas{If D is strongly connected, compress to single vertex}
  Otherwise, we create a new DAG $D'$ and a new set $\exset{G}{D'}$ with 
  \[
  \ve{D'} \coloneqq \set{v_S}{S \in \source{D}} \cup \set{v_T}{T \in \source{D}} \cup \set{v_c}{c \in \conn{D}}.\]

  The set $\exset{G}{D'}$ contains pairs $(s, v_S)$ for $S \in \source{D}$ and  $s \in S$ as well as $(v_T, t)$ for $T \in \sink{D}$ and $t \in T$.
  Additionally, for $(a,b) \in \conn{D}$, it contains $(a, v_{(a,b)})$ and $(v_{(a,b)}, b)$.

  Add an edge from $v_S$ to $v_T$, if there is a source $s$ and a sink $t$ in $D$ such that there is an $s$-$t$-path in $D$.

  For $c = (a,b) \in \conn{D}$, let $s_c \in \ve{D}, S \coloneqq \inNeiG{\dexset}{s_c}$ be a source such that for every $v \in S$, there is $(v, b) \in \conn{D}$. Similarly, let $t_c \in \ve{D}$ be a sink with $T \coloneqq \outNeiG{\dexset}{t_c}$ such that $(a,v) \in \conn{D}$ for $v \in T$. We add edges $(v_S, v_c), (v_c, v_T)$ to $\e{D'}$.
\end{definition}
\todo{S: IT's not a definition but a procedure to compress. Rewrite }
\end{comment}

Let $G$ be a directed graph with an extension $\dexpair$. We give a \emph{compression routine} that returns an extension that we call \emph{compressed extension}, denoted as $\dcompG$. This procedure works as follows:
\subparagraph*{Compression Routine}
\begin{itemize}
    \item  If $D$ is strongly connected, we contract $D$ to a single vertex, removing self-loops and multiple edges. We adjust $E_{GD}$ accordingly, that is, change vertices in $D$ to $v$ and remove multiple edges.
      
    \item Otherwise,  $\dcompG \coloneqq (D',\exset{G}{D'})$, where $D'$ is a DAG and  $\exset{G}{D'}$ is an extension such that
    \begin{align*}
  \ve{D'} \coloneqq &\set{v_S}{S \in \source{D,\dexset}} \cup \set{v_T}{T \in \sink{D,\dexset}} \cup \set{v_c}{c \in \conn{D,\dexset}},\\
  \exset{G}{D'} \coloneqq &\set{(s,v_S)}{S\in \source{D,\dexset}, s\in S} \cup \set{(v_T,t)}{T\in \sink{D,\dexset}, t\in T }\\
   \cup &\set{(a,v_c),(v_c,b)}{c=(a,b) \in \conn{D,\dexset}}.
    \end{align*}

    To define $\e{D}$, consider every source component $C_s$ and sink component $C_t$ in $\scc{D}$ such that $C_t$ is reachable from $C_s$ in $D$. Let $S \coloneqq \inNeiG{\dexset}{C_s}$ and $T \coloneqq \inNeiG{\dexset}{C_t}$ be the corresponding sets in $\source{D,\dexset}$ and $\sink{D,\dexset}$.
    \begin{itemize}
        \item Add the edge $(v_S, v_T)$ to $\e{D}$.
        \item For every $c = (a,b) \in \conn{D,\dexset}$ that satisfies $(s,b) \in \conn{D,\dexset}$ and $(a,t) \in \conn{D,\dexset}$ for every $s \in S, t \in T$, add the edges $(v_S, v_c)$ and $(v_c, v_T)$ to $\e{D}$.
    \end{itemize}
 \end{itemize}
%%%%%%%%%%%%%%%%%%%%%%%%%%%%%

 % Since there is always a source in $D$ that reaches the endpoint of a path that gives a connection $(a,b) \in \conn{D}$, a suitable $s_c \in \ve{D}$ for \Cref{def:comp} always exists. Since the same holds for sinks, $\compOp_G$ is well-defined.

Now we go on to prove the properties we maintain while compressing the extension. Afterwards we bound the size of a compressed extension and thereby also the number of equivalence classes.
% \iflong
\begin{lemma}
% \else
% \begin{lemma}[$\star$]
% \fi
\label{lem:comp_weakly_conn_equiv}
  Let $G$ be a directed graph with an extension $\dexpair$ and let $(D', E_{GD'})$ be the compressed extension of $\dexpair$. 
  Then, the following are true.
  \begin{enumerate}
      % \item If $D$ is acyclic, then $D'$ is also acyclic. 
      \item If $D$ is weakly connected, then $D'$ is also weakly connected. 
      \item $D$ is strongly connected if and only if $D'$ is strongly connected.
      \item $(D', E_{GD'})$ is equivalent to $\dexpair$. 
  \end{enumerate}
\end{lemma} 
% \iflong
\begin{proof}
  % We can focus on the case that $\abs{\ve{D}} > 1$; the other case is trivial.\jonas{no longer necessary}
  % Verifying that $D'$ is a DAG is easily done with the topological order that chooses all $v_S$ first, then all $v_c$, and finally all $v_T$ in arbitrary order.
  For the first property, assume that $D$ is weakly connected.
  We know by definition that every sink in $D'$ is reached by at least one source.
  Consider a $v_c$ with $c = (a,b) \in \conn{D,\dexset}$. To show that $v_c$ is connected to some $v_S$ and $v_T$, consider the path from $d_1$ to $d_2$ in $D$ that realizes this connection. There must be a source component $C_S$ and a sink component $C_T$ in $\scc{D}$ such that $d_1$ is reachable from $C_S$ and $C_T$ is reachable from $d_2$. Therefore, any vertex in $C_S$ can also reach $d_2$ and $d_1$ can reach every vertex in $C_T$. By definition of compression, these two components ensure that $v_c$ is connected. 
  
  It remains to show that any source is reachable by any other source in the underlying undirected graph. Let $v_S, v_{S'}$ be two sources in $D'$ with corresponding source components $C$, $C'$ in $\scc{D}$. Since $D$ is weakly connected, there is a path from $C$ to $C'$ in the underlying undirected graph. Whenever the undirected path uses an edge in a different direction than the one before, we extend the path to first keep using edges in the same direction until a source or sink component is reached and then go back to the switching point. This new path can directly be transferred to $D'$, where we only keep the vertices corresponding to source and sink components. By definition, this is still a path in $D'$ that connects $v_S$ to $v_{S'}$, and $D'$ is weakly connected
  % Since we can do this for all $S, S' \in \source{D,\dexset}$, $D'$ must be weakly connected.

  The second property is simple to verify, since strongly connected graphs are by definition compressed to single vertices. If $D$ is not strongly connected, $D'$ will have at least one source and one sink that are not the same.
  
  Regarding the equivalence, we create one source for every $S \in \source{D,\dexset}$ with the same set of incoming neighbors and create no other sources. Therefore, $\source{D', E_{GD'}} = \source{D,\dexset}$ and $\sink{D',E_{GD'}} = \sink{D,\dexset}$ follows analogously.
  For every connection $c \in \conn{D,\dexset}$, we create $v_c$ in $D'$ that realizes this connection. Therefore, we know that $\conn{D',E_{GD'}} \supseteq \conn{D,\dexset}$. Since $v_c$ is only reachable from sources and reaches only sinks that do not give new connections, we arrive at $\conn{D',E_{GD'}} = \conn{D,\dexset}$. 
\end{proof}
% \fi

We also bound the size of the compressed extension as well as the number of different possible compression outputs.

\iflong
\begin{lemma}
\else
\begin{lemma}[$\star$]
\fi
\label{lem:range_size_comp}
  For a directed graph $G$, there can be at most $2^{2\cdot 2^{\abs{\ve{G}}} + \abs{\ve{G}}^2}$ different compressed extensions.
  Furthermore, every compressed extension has at most $2^{\abs{\ve{G}}+1} + \abs{\ve{G}}^2$ vertices.
\end{lemma}
\iflong
\begin{proof}
  There are $2^{\abs{\ve{G}}}$ subsets of $\ve{G}$, each of them can be the neighborhood of a source or sink. Additionally, every one of the at most $\abs{\ve{G}}^2$ can form a connection or not. This proves the first claim. 

  Suppose $\compOp_G$ outputs $\dexpair$. For every subset $U \subseteq \ve{G}$, there can be at most one source and one sink in $D$ that has $U$ as outgoing or incoming neighbors. Also, there are at most $\abs{\ve{G}}^2$ pairs of vertices in $G$ that can form a connection. This bounds the number of connection vertices in $D$.
\end{proof}
\fi

In the past section, we have defined graph extensions and an equivalence relation on them that captures the role they can play in forming strongly connected subgraphs. Furthermore, we have presented a way to compress an extension such that its size only depends on the size of $G$, while remaining equivalent.

In the upcoming section, we will use this theory to design reduction rules for \scsborder{}. Crucially, we view components outside of the set $B$ as extensions and show that it is enough to keep only one compressed extension of each equivalence class.

\subsection{Solving \scsborder{}}\label{sec:solving_scc}

We start by giving some reduction rules for a \scsborder{} instance $\mathcal{I} = (G, I, O, B, \wOp,  k, T)$. Additionally, we assume that $T \subseteq B$ to ensure that we do not change $T$ when changing $G - B$. This condition will always be satisfied in our algorithm.
\ifshort
A ($\star$) after a reduction rule denotes that the proof of safeness of this rule was omitted and can be found in the full version.
\fi

The first reduction rule extends the sets $I$ and $O$ to whole components of $G - B$. This is possible since no solution can include only part of a component without its neighborhood intersecting the component.

\iflong
\begin{reduction*}
\else
\begin{reduction*}[$\star$]
\fi
\label{red:in_out}
  Let $Q$ be a component of $G - B$. If $Q \cap O \ne \emptyset$, set $O = O \cup \cnei{Q}$. If $\cnei{Q} \cap I \ne \emptyset$, set $I = I \cup Q$. If both cases apply, the instance has no solution.
\end{reduction*} 
\iflong
\begin{proof}[Proof of Safeness]
  For the first case, assume $Q \cap O \ne \emptyset$. Notice that if any vertex of $\cnei{Q}$ is in the solution, there also has to be some vertex of $Q$ in the neighborhood of the solution since $Q$ is weakly connected. This, however, is not possible as $Q \cap B = \emptyset$.

  Similarly, for the second case, assume $\cnei{Q} \cap I \ne \emptyset$. If any vertex of $Q$ is not in the solution, there has to be a vertex of $Q$ in the neighborhood of the solution since $Q$ is weakly connected, which again is impossible. 
  
  Therefore, we can safely apply the first two cases. Since in the last case $I \cap O \ne \emptyset$, there can clearly be no solution.
\end{proof}
\fi

If this reduction rule is no longer applicable, every component in $G-B$ is either completely in $I$, completely in $O$, or intersects with none of the two. %We use that for the next reduction rule.
% See \Cref{fig:iob} for an illustration of how instances can be structured now.

% \begin{figure}[t]
  % \jonas{moved into other figure}
  % \begin{minipage}[c]{0.45\linewidth}
      % \centering
      % \includegraphics[width=0.3\textwidth]{figures/IOB_fig}
      % \caption{A visualization of how the sets $I$, $O$, and $B$ can overlap after the application of \Cref{red:in_out}. Any component in $G-B$ can be in $I$, $O$, or none of them, but not both. The black vertices form a feasible solution.}\label{fig:iob}
  % \end{minipage}
  % \jonas{not so important}
  % \hfill
  % \begin{minipage}[c]{0.45\linewidth}
  %     \centering
  %     \includegraphics[width=0.8\textwidth]{figures/scc_only_conn}
  %     \caption{A visualization of why source and sink sets of a component are important to consider. If we removed $Q_2$, the black vertices together with $Q_1$ would form a SCS that does not correspond to a SCS with $Q_2$.}\label{fig:scc_not_conn}
  % \end{minipage}
% \end{figure}

% \jonas{no longer necessary with the new equivalence def}
% \iflong
% \begin{reduction*}
% \else
% \begin{reduction*}[$\star$]
% \fi
% \label{red:contract_strong}
%   Let $Q$ be a strongly connected component of $G - B$ on which \Cref{red:in_out} is not applicable. Contract $Q$ into a single vertex $q$ with weight $\w{q} = \w{Q}$ and remove multi-edges.
% \end{reduction*}
% \iflong
% \begin{proof}[Proof of Safeness]
%   By \Cref{red:in_out}, it is impossible to include vertices in $Q$ into the neighborhood or to include part of $Q$ into the solution. Therefore, any solution that does not include all of $Q$ remains unchanged.

%   If a solution includes all of $Q$, using $q$ instead does not change the neighborhood size nor the weight, due to the contraction. Also, the solution stays strongly connected since $\outNei{q} = \outNei{Q}$ and $\inNei{q} = \inNei{Q}$. It also cannot create additional solutions since instead of including $q$, we could have included $Q$ because there are paths in $\induced{G}{Q}$ from every vertex to every other one.
% \end{proof}
% \fi

%\jonas{also no longer necessary?}
% Later, we will apply our theory from the previous subsection and consider weakly connected components $Q$ of $G-B$ as extensions of $G-Q$. \todo{Not clear to me.} Therefore, we remove $Q$ if it includes source and sink vertices. 


From this point, we will use extensions from the previous section for components of $G-B$. Namely, for a component $Q$ of $G-B$, let $E_{BQ}$ be all the edges with exactly one endpoint in $B$ and one in $Q$. Then, $(\induced{G}{Q}, E_{BQ})$ defines an extension of $G-Q$. For simplicity, we also refer to this extension as $(Q, E_{BQ})$. Hence, we also use $\sourceOp$, $\sinkOp$, and $\connOp$, as well as $\compOp_{G-Q}$ for these extensions. 

The next reduction rule identifies a condition under which a component $Q$ can never be part of a solution, namely, if $Q$ include strongly connected components with no in-neighbors or no out-neighbors, which is exactly the case if the empty set is in $\source{Q,E_{BQ}}$ or $\sink{Q,E_{BQ}}$.

\iflong
\begin{reduction*}
\else
\begin{reduction*}[$\star$]
\fi
\label{red:scc_no_sources_or_sinks}
  Let $Q$ be a component of $G - B$ such that $\induced{G}{Q}$ is not strongly connected. If $\emptyset \in \source{Q,E_{BQ}} \cup \sink{Q,E_{BQ}}$, include $Q$ into $O$. 
\end{reduction*}
\iflong
\begin{proof}[Proof of Safeness]
    Since $\induced{G}{Q}$ is not strongly connected, $Q$ cannot be a solution by itself. Suppose $\emptyset \in \source{Q,E_{BQ}}$, the other case is analogous. Then, there is a source component $C$ in $\scc{\induced{G}{Q}}$ that has no incoming edge, neither in $\induced{G}{Q}$, nor in $E_{BQ}$. 
    Therefore, no vertex $v \in B$ can reach $C$.
    Because $Q$ can only be included as a whole and together with other vertices in $B$, $Q$ cannot be part of a solution.
\end{proof}
\fi

Note that after \Cref{red:in_out,red:scc_no_sources_or_sinks} have been applied exhaustively, every component $Q$ of $G - B$ is acyclic. Furthermore, every source in $Q$ has incoming edges from $B$, and every sink in $Q$ has outgoing edges to $B$. Finally, we have one more simple rule, which removes vertices $v \in O \setminus B$. It relies on the fact that by \Cref{red:in_out}, we also have $\nei{v} \subseteq O$.

\begin{reduction*}\label{red:remove_out}
  If \Cref{red:in_out} is not applicable, remove $O \setminus B$ from $G$.
\end{reduction*}
The previous reduction rules were useful to remove trivial cases and extend $I$ and $O$.
From now on, we assume that the instance is exhaustively reduced by \Cref{red:remove_out,red:scc_no_sources_or_sinks,red:in_out}.
Therefore, any component of $G - B$ is either contained in $I$ or does not intersect $I$ and $O$.

The next two rules will be twin type reduction rules that allow us to bound the number of remaining components.
If there are two components of $G-B$ that form equivalent extensions it is enough to keep one of them, since they fulfill the same role in forming a strongly connected subgraph. The reduction rules rely on \Cref{lem:source_sink_conn_equiv,lem:union_equiv_if_equiv} to show that equivalent extensions can replace each other and can be added to any solution.

\iflong
\begin{reduction*}
\else
\begin{reduction*}[$\star$]
\fi
\label{red:scc_twins}
  Let $Q_1, Q_2$ be components of $G - B$ such that both $\induced{G}{Q_1}$ and $\induced{G}{Q_2}$ are not strongly connected and $(Q_1, E_{BQ_1})$ and $(Q_2, E_{BQ_2})$ are equivalent. Delete $Q_2$ and increase the weight of some $q \in Q_1$ by $\w{Q_2}$. If $Q_2 \cap I \ne \emptyset$, set $I = I \cup Q_1$.
\end{reduction*}
\iflong
\begin{proof}[Proof of Safeness]
  By the previous reduction rules, components of $G-B$ can only be included as a whole or not at all. Notice that since $\conn{Q_1, E_{BQ_1}} = \conn{Q_2, E_{BQ_2}}$ and \Cref{red:scc_no_sources_or_sinks}, we get $\nei{Q_1} = \nei{Q_2}$. Let $S$ be a solution to the old instance. We differentiate some cases.

  If $S \cap (Q_1 \cup Q_2) = \emptyset$, we know that also $\nei{S} \cap (Q_1 \cup Q_2) = \emptyset$. Therefore, the neighborhood size and strong connectivity of $S$ do not change in the new instance, and it is also a solution.
  If $S$ includes only one of $Q_1$ and $Q_2$, assume without loss of generality $S \cap (Q_1 \cup Q_2) = Q_1$, since $Q_1$ is not strongly connected by itself, the solution must include vertices of $B$. Because $\nei{Q_1} = \nei{Q_2}$, the solution must also include $Q_2$, a contradiction.
  If $S$ includes both $Q_1$ and $Q_2$, we claim that $S' \coloneqq S \setminus Q_2$ is a solution for the new instance. The neighborhood size and weight clearly remain unchanged. %For the strong connectivity, suppose that there are $u,v \in S'$ such that the $u$-$v$-path in $S$ uses vertices from $Q_2$. Such a subpath can only connect one vertex in $B$ to another one in $B$, and since $\conn{Q_1} = \conn{Q_2}$, we can use a connection via $Q_1$ instead. Therefore, $S'$ is also strongly connected and a solution of the same weight.\jonas{use union equiv lemma}
  Strong connectivity follows by \Cref{lem:union_equiv_if_equiv} and \Cref{lem:source_sink_conn_equiv}.
  The last two cases also show that if a solution had to include at least one of $Q_1$ and $Q_2$, that is, $(Q_1 \cup Q_2) \cap I \ne \emptyset$, any solution for the reduced instance must also include $Q_1$. Therefore, the adaptation to $I$ is correct.

  Let $S$ be a solution to the reduced instance. Again, if $S$ does not include vertices from $Q_1$, then $S$ will immediately be a solution to the old instance. If $Q_1 \subseteq S$, then $S' \coloneqq S \cup Q_2$ will be a solution to the old instance by \Cref{lem:union_equiv_if_equiv} and \Cref{lem:source_sink_conn_equiv}.
\end{proof}
\fi

For strongly connected components, the rule is different, since we have to acknowledge the fact that they can be a solution by themselves. For every such component we destroy strong connectivity of smaller-weight equivalent component, which can then be reduced by \Cref{red:scc_twins}.

% \jonas{moved next to other figure}
% \begin{figure}[t]
%   \centering
%   \includegraphics[width=0.4\textwidth]{figures/scc_only_conn}
%   \caption{A visualization of why source and sink sets of a component are important to consider. If we were to remove $Q_2$, the black vertices together with $Q_1$ would form a strongly connected subgraph that does not correspond to a strongly connected subgraph with $Q_2$.}\label{fig:scc_not_conn}
% \end{figure}

% \jonas{Not important, maybe I was the only one wondering}
% Intuitively, one might ask whether it is also enough to apply \Cref{red:scc_twins} if only $\conn{Q_1} = \conn{Q_2}$ when the source and sink sets can be different, especially since the proof hides this detail in \Cref{lem:union_equiv_if_equiv}. See \Cref{fig:scc_not_conn} for an example of why just $\conn{Q_1} = \conn{Q_2}$ is not enough.

\iflong
\begin{reduction*}
\else
\begin{reduction*}[$\star$]
\fi
\label{red:scc_twins_single}
  Let $Q_1, Q_2$ be components of $G - B$ that are also strongly connected with $(Q_1, E_{BQ_1})$ and $(Q_2, E_{BQ_2})$ equivalent and $\w{Q_1} \ge \w{Q_2}$.
  Add a vertex $q_2'$ with edges $\died{q_2}{q_2'}$ and $\died{q_2'}{v}$, for all $q_2 \in Q_2$ and $v \in \nei{q_2}$. Set $\w{q_2'} = 0$.
\end{reduction*}
\iflong
\begin{proof}[Proof of Safeness]
  Let $S$ be a solution of the old instance.
  If $S \cap (Q_1 \cup Q_2) = \emptyset$, then $S$ is also a solution for the new instance.
  If $S$ includes only one of $Q_1$ and $Q_2$, then we must have $S \in \set{Q_1, Q_2}$, so $S' \coloneqq Q_1$ is a solution for the new instance with $\w{S'} \ge \w{S}$.
  If $S$ includes both $Q_1$ and $Q_2$, adding $q_2'$ to $S$ obviously gives a solution of the same weight.

  For a solution of the reduced instance $S$, we can simply remove $q_2'$ if it is inside for a solution to the old instance.
\end{proof}
\fi

Using both \Cref{red:scc_twins,red:scc_twins_single} exhaustively makes sure that there are at most two components per extension equivalence class left.
The last rule compresses the remaining components to equivalent components of bounded size.

% \iflong
\begin{reduction*}
% \else
% \begin{reduction*}[$\star$]
% \fi
\label{red:scc_comp}
  Let $Q$ be a component of $G - B$ that is not equal to its compressed extension. Replace $Q$ by its equivalent compressed extension and set the weight such that $\w{Q'} = \w{Q}$. If $Q \cap I \ne \emptyset$, set $I = I \cup Q'$.
\end{reduction*} 
% \iflong
\begin{proof}[Proof of Safeness]
  Since $Q$ is not strongly connected, it can only be part of a solution that includes some vertices from $G-Q$. Thus, we can apply \Cref{lem:comp_weakly_conn_equiv}, and the old instance and the new instance have exactly the same strongly connected subgraphs. Because of $\conn{Q, E_{BQ}} = \conn{Q', E_{BQ'}}$ and \Cref{red:scc_no_sources_or_sinks}, we get $\nei{Q} = \nei{Q'}$. Since $\w{Q'} = \w{Q}$, the rule is safe.
\end{proof}
% \fi

This finally allows us to bound the size of $G-B$. In the next lemma, we summarize the progress of our reduction rules and apply the bounds from the previous section. Note that we need the stronger bound using the neighborhood since of the components instead of simply $B$.

\iflong
\begin{lemma}
\else
\begin{lemma}[$\star$]
\fi
\label{lem:scc_all_reductions}
  Let $\mathcal{Q}$ be a set of components of $G - B$ with total neighborhood size $h \coloneqq \abs{\bigcup_{Q \in \mathcal{Q}} \nei{Q}}$. Then we can reduce the instance, or there are at most $2^{2^{h+1} + h^2}\left(2^{h+1} + h^2\right) = 2^{\bigO{2^h}}$ vertices in $\mathcal{Q}$ in total.
\end{lemma}
\iflong
\begin{proof}
  After applying \Cref{red:scc_twins,red:scc_comp} exhaustively, there can be at most one copy of every compressed extension among the components in $\mathcal{Q}$.
  Since the compressed component only depends on the neighborhood of a component, we can treat $\bigcup_{Q \in \mathcal{Q}} \nei{Q}$ as the base graph for our extensions. By \Cref{lem:range_size_comp}, there are at most $\abs{\mathcal{Q}} \le 2^{2\cdot 2^h + h^2}$ components. Also, each component has at most $2^{h+1} + h^2$ vertices, which immediately gives the claimed bound on the total number of vertices in $\mathcal{Q}$.
\end{proof}
\fi

\iflong
This above result is only useful if we can guarantee that applying the reduction rules exhaustively in the correct order will always terminate. Additionally, we have to ensure that we can verify the conditions for every reduction rule quickly enough. Both statements are proven in the following lemma.

\iflong
\begin{lemma}
\else
\begin{lemma}[$\star$]
\fi
\label{lem:red_rules_poly}
  Executing the reduction rules in the proposed order guarantees termination after $\bigO{n}$ applications. The total execution takes at most $n^{\bigO{1}}$ time.
\end{lemma}
\iflong
\begin{proof}
  Most reduction rules clearly make progress, either by increasing the size of $I$ or $O$ or by decreasing the number of vertices in $G$. While compression does not always decrease the number of vertices, it will only be applied once per component. Finally, \Cref{red:scc_twins_single} decreases the number of strongly connected components in $G-B$, which also will never be increased again by \Cref{lem:comp_weakly_conn_equiv}. All of these progress measures are bounded by $n$, proving the first claim.

  For the second part, we only have to bound the runtime of a single application of a reduction rule. Most rules are clearly executable in polynomial time. The total size of the source, sink, and connection sets are also bounded by $\abs{\e{G}}$ and thus computable in polynomial time. Therefore, the compressed component can also be constructed in polynomial time. Note that this increases the size of our graph, but if we apply \Cref{red:scc_comp} only after all other reduction rules are completed, the compressed component never has to be considered again.
\end{proof}
\fi
\fi

\iflong
We use one more lemma from~\cite{golovach2020finding} that helps us to find a balanced separator.

\begin{lemma}[\protect{\cite[Lemma~1]{golovach2020finding}}]\label{lem:separation_algo}
  Given an undirected graph $G$, there is an algorithm with runtime $2^{\bigO{\min\{q,k\}\log(q+k)}}n^{\bigO{1}}$ that either finds a $(q,k)$-separation of $G$ or correctly reports that $G$ is $((2q+1)q2^k, k)$-unbreakable.
\end{lemma}
\fi

Now, we have all that it takes to solve our intermediate problem.% We use the same structure as~\cite{golovach2020finding} in their algorithms.
%
The main idea of the algorithm is to shrink $B$ to a bounded size, by solving the problem recursively. Once $B$ is bounded, we apply our reduction rules by viewing components of $G-B$ as extensions, removing redundant equivalent extensions and compressing them. Thereby, we also bound the size and number of components of $G-B$ in terms of $\abs{B}$ using \Cref{lem:scc_all_reductions}. By choosing suitable constants, we can show that this decreases the total size of $G$, which will make progress to finally reduce it to the unbreakable case.

\iflong
\begin{theorem}
\else
\begin{theorem}[$\star$]
\fi
\label{thm:border_scc_fpt}
  \scsborder{} can be solved in time $2^{2^{2^{\bigO{k^2}}}}n^{\bigO{1}}$.
\end{theorem}

\iflong
\begin{proof}

\begin{algorithm}[t]
 % \caption{The algorithm for \scsborder{}, using the same structure as the algorithms from~\cite{golovach2020finding}.} 
 \label{alg:rec_und_scc}
   \caption{Algorithm \scsborder{}}
  \DontPrintSemicolon
  \SetKwFunction{FMain}{Solve}
  \SetKwProg{Fn}{def}{:}{}
  \Fn{\FMain{$G$, $I$, $O$, $B$, $\wOp$, $k$, $T$}}{
    $q\gets 2^{2^{ck^2}}$ for a suitable constant $c$\;
    \eIf{$G$ is $((2q+1)q2^k,k)$-unbreakable}
    {
      \KwRet solve the problem using \Cref{thm:unbreakable_scc}\;
    }{
      $(U,W) \gets $ $(q,k)$-separation of $G$ with $\abs{T \cap W \setminus U} \le k$\;
      $(\tilde{G}, \tilde{I}, \tilde{O}, \tilde{B}, \tilde{\wOp}, k, \tilde{T}) \gets $ restriction to $W$ with $\tilde{T} = (T \cap W) \cup (U \cap W)$\;
      $\mathcal{R} \gets $ \FMain{$\tilde{G}$, $\tilde{I}$, $\tilde{O}$, $\tilde{B}$, $\tilde{\wOp}$, $k$, $\tilde{T}$}\;
      $\mathcal{N} \gets \tilde{T} \cup \bigcup_{R \in \mathcal{R}} \nei{R} \cap W$\;
      $\hat{B} \gets (B \cap U) \cup (B \cap \mathcal{N})$\;
      $(G^*, I^*, O^*, \hat{B}^*, \wOp^*, k^*, T^*) \gets $ reduce $(G,I,O,\hat{B},\wOp,k,T)$ with \Cref{lem:scc_all_reductions}\;
      \KwRet \FMain{$G^*$, $I^*$, $O^*$, $\hat{B}^*$, $\wOp^*$, $k^*$, $T^*$}\;
    }
  }
\end{algorithm}

  Let $\mathcal{I} = (G, I, O, B, \wOp, k, T)$ be our \scsborder{} instance. See \Cref{alg:rec_und_scc} for a more compact description of the algorithm. A high level display of the approach can be found in \Cref{fig:recursive_calls}.
  Define $q = 2^{2^{2^{ck^2}}}$ for a constant $c$. We later show that a suitable $c$ must exist.
  \zzcommand{\neighbor}{\mathcal{N}}

  First, we run the algorithm from \Cref{lem:separation_algo} on the underlying undirected graph of $G$ with $q$ and $k$. If it is $((2q+1)q2^k, k)$-unbreakable, we solve the instance directly using the algorithm from \Cref{thm:unbreakable_scc}.

  Therefore, assume that we have a $(q,k)$-separation $(U,W)$. Without loss of generality, since $\abs{T} \le 2k$ we can assume that $\abs{T \cap W \setminus U} \le k$. Thus, we can construct a new instance to solve \emph{the easier side} of the separation. Take $\tilde{G} = \induced{G}{W}, \tilde{I} = I \cap W, \tilde{O} = O \cap W, \tilde{B} = B \cap W$, write $\tilde{\wOp}$ for the restriction of $\wOp$ to $W$, and set $\tilde{T} = (T \cap W) \cup (U \cap W)$. Since $\abs{U \cap W} \le k$, also $\abs{\tilde{T}} \le 2k$ holds and $\tilde{I} \coloneqq (\tilde{G}, \tilde{I}, \tilde{O}, \tilde{B}, \tilde{\wOp}, k, \tilde{T})$ is a valid instance, which we solve recursively.

  Let $\mathcal{R}$ be the set of solutions found in the recursive call. For $R \in \mathcal{R}$, define $N_R = \nei{R} \cap W$. Define $\mathcal{N} = \tilde{T} \cup \bigcup_{R \in \mathcal{R}} N_R$. 

  We now restrict $B$ in $W$ to use only vertices in the neighborhood that have been neighbors in a solution in $\mathcal{R}$, that is, only vertices in $\mathcal{N}$. Define $\hat{B} = (B \cap U) \cup (B \cap \mathcal{N})$. We now replace $B$ in our instance with $\hat{B}$ and apply all our reduction rules exhaustively to arrive at the instance $(G^*, I^*, O^*, \hat{B}^*, \wOp^*, k^*, T^*)$. Finally, we also solve this instance recursively and return the solutions after undoing the reduction rules. 

  \subparagraph*{Correctness.}
  We already proved that the reduction rules and the algorithm for the unbreakable case are correct. The main statement we have to show is that we can replace $B$ with $\hat{B}$ without throwing away important solutions. That means that the instances $(G,I,O,B,\wOp,k,T)$ and $\hat{\mathcal{I}} \coloneqq (G,I,O,\hat{B},\wOp,k,T)$ are equivalent in the sense that any solution set for one instance can be transformed to a solution set to the other instance with at least the same weights. This justifies solving $\hat{\mathcal{I}}$ instead of $\mathcal{I}$.
  Consider the boundary complementation  $\mathcal{I}' \coloneqq (G',I',O',B',\wOp',k')$ and $\hat{\mathcal{I}}' \coloneqq (G',I',O',\hat{B}',\wOp',k')$ that are caused by the same $(X,Y,Z,R)$.
  To show the claim, we consider the two directions. Any solution for $\hat{\mathcal{I}}'$ is immediately a solution for the same boundary complementation for $\mathcal{I}'$ since the only difference is that we limit the possible neighborhood to a subset of $B$.

  For the other direction, consider a solution $S$ to $\mathcal{I}'$, and we want to show that there is a solution $\hat{S}$ to $\hat{\mathcal{I}}'$ using only vertices of $\hat{B}'$ in the neighborhood with $\w{\hat{S}} \ge \w{S}$. If $S \cap W = \emptyset$, then $S$ is also immediately a solution for $\hat{\mathcal{I}}'$. Therefore, assume $S \cap W \ne \emptyset$. Define $\tilde{X} \coloneqq \tilde{T} \cap S$, $\tilde{Y} \coloneqq \tilde{T} \cap \neiG{G - (W\setminus U)}{S}$, and $\tilde{Z} \coloneqq \tilde{T} \setminus (\tilde{X} \cup \tilde{Y})$. Let $\tilde{R}$ be the set of $(a,b) \in \tilde{X} \times \tilde{X}$ such that there is an $a$-$b$-path in $\induced{G}{S \setminus {W \setminus U}}$. Finally, let $\tilde{k} \coloneqq \abs{\nei{S} \cap W}$. Thus, we can construct a boundary complementation instance $(\tilde{G}', \tilde{I}', \tilde{O}', \tilde{B}', \tilde{w}', \tilde{k})$ from $(\tilde{X}, \tilde{Y}, \tilde{Z}, \tilde{R})$. One can easily verify that $(S \cap \ve{\tilde{G}'}) \cup \set{u_r}{r \in \tilde{R}}$ is a feasible solution to this instance. Furthermore, the maximum solution $\tilde{S} \in \mathcal{R}$ to this instance gives a new set $\hat{S} \coloneqq (\tilde{S} \cap W) \cup (S \cap U) \subseteq \ve{G'}$ that has at least the same weight as $S$. We also know that $\nei{\hat{S}} \subseteq \hat{B}'$ and $\abs{\nei{\hat{S}}} \le k$. See \Cref{fig:scc_bc} for a visualization of the construction corresponding to a solution $S$.


  All that remains is to verify that $\hat{S}$ is strongly connected. For $v_1, v_2 \in \tilde{S} \cap W$, we can simply use the $v_1$-$v_2$-path in $\tilde{S}$, replacing subpaths via $u_r$ for $r \in \tilde{R}$ with the actual paths in $S \cap U$. If $v_1 \in \tilde{S} \cap W$ and $v_2 \in S \cap U$, we can walk to any $x \in \tilde{X}$ which must have a path to $v_2$ in $S$, in which may need to replace subpaths via $W$. This can be done, since $\tilde{S} \cap W$ connects all pairs of $x_1, x_2 \in X$ that are not connected via $S \cap U$. The two remaining cases follow by symmetry, justifying the replacement of $B$ with $\hat{B}$.

  Finally, we have to show that the recursion terminates for the right choice of $c$; that is, both recursively solved instances have strictly smaller sizes than the original graph.
  For the first recursive call, note that the boundary complementation adds at most $k^2 \le q$ vertices to $\tilde{G}$. Since $\abs{U \setminus W} > q$, we have $\abs{\ve{\tilde{G}}} < \abs{\ve{G}}$.

  For the second recursive call, since $\abs{\tilde{T}} \le 2k$ and by \Cref{lem:number_border_complementations}, we have $\abs{\mathcal{N}} \le 2k + (k+1)k2^{c_1k^2} \le 2^{c_2k^2}$ for constants $c_1$ and $c_2$.
  After applying all our reduction rules, by \Cref{lem:scc_all_reductions} for a suitable choice of $c_3$ and $c$ we get \[\abs{W^*} \le \abs{\neighbor} + 2^{2^{\abs{\neighbor}+1} + \abs{\neighbor}^2}\left(2^{\abs{\neighbor}+1} + \abs{\neighbor}^2\right) \le 2^{c_3 2^{\abs{\neighbor}}} \le 2^{c_32^{2^{c_2k^2}}} \le 2^{2^{2^{ck^2}}} \eqqcolon q.\]
  Since before the reductions, we had $\abs{W \setminus U} > q$, the reduced graph $G^*$ also has fewer vertices than $G$. Therefore, this recursive call also makes progress and the recursion terminates.

  \subparagraph*{Runtime.} 
  We follow the analysis of~\cite{chitnis2016designing}. With \Cref{lem:separation_algo}, we can test
  if the undirected version of $G$ is unbreakable in time $2^{k\log{q+k}}n^{\bigO{1}} \le 2^{2^{2^{\bigO{k^2}}}}n^{\bigO{1}}$.
  If the graph turns out to be unbreakable, the algorithm of \Cref{thm:unbreakable_scc} solves it in time $2^{\bigO{k^2\log{q}}} \le 2^{2^{2^{\bigO{k^2}}}}n^{\bigO{1}}$.
  Executing the reduction rules takes polynomial time in $n$ by \Cref{lem:red_rules_poly}. 

  All that is left to do is analyze the recursion. Let $n' \coloneqq \abs{W}$. By the separation property, we know that $q < n' < n-q$. From the correctness section, we get $\abs{\ve{G^*}} \le \abs{\ve{G}} - n' + q$. 
  Note that the recursion stops when the original graph is $(q,k)$-unbreakable, which must be the case if $\abs{\ve{G}} \le 2q$.
  We arrive at the recurrence 
  \[
    T(n) = \begin{cases}2^{2^{2^{\bigO{k^2}}}}, & \text{for $n \le 2q$;}\\
    \left(\max_{q < n' < n-q} T(n' + k^2) + T(n - n' + q)\right) + 2^{2^{2^{\bigO{k^2}}}}n^{\bigO{1}},  & \text{otherwise.} \end{cases}
  \]
  Notice that $2^{2^{2^{\bigO{k^2}}}}$ appears in every summand of the expanded recurrence and can be ignored here and multiplied later. Furthermore, $n^{\bigO{1}}$ is bounded from above by a convex polynomial. Therefore, it is enough to consider the extremes of the maximum expression. For $n' = q+1$, the first recursive call evaluates to $T(q+1+k^2) \le T(2q) \le 2^{2^{2^{\bigO{k^2}}}}$. The second call only introduces an additional factor of $n$. For $n' = n-q-1$, the second expression evaluates to $T(2q+1)$ which is clearly also bounded by $2^{2^{2^{\bigO{k^2}}}}$. Thus, we arrive at the final runtime of $2^{2^{2^{\bigO{k^2}}}}n^{\bigO{1}}$.
\end{proof}
\fi

Finally, we can use \scsborder{} to solve \scs{}. Since in our original problem every vertex could be part of the solution or its neighborhood, we initially set $I \coloneqq O \coloneqq \emptyset$ and $B \coloneqq \ve{G}$. Furthermore, we only want to consider the boundary complementation that changes nothing, which we achieve by setting $T \coloneqq \emptyset$.

% \begin{restatable*}[]{theorem}{test}
% \label{thm:scc_algo}
%   \textsc{Total-Secluded Strongly Connected Subgraph} is solvable in time $2^{2^{2^{\bigO{k^2}}}}n^{\bigO{1}}$.
% \end{restatable*}
\test*