%\section{Secluded \texorpdfstring{$d$}{d}-Edge-Connected Subgraphs}\label{sec:d_edge}

In this section, we consider a similar undirected problem to strong connectivity. In many ways 2-edge-connectivity is a comparable concept, since strong connectivity also requires at least two different paths between every pair of vertices. We target a generalized version of the problem and show that the following problem is non-uniformly FPT with a similar recursive understanding approach as in \Cref{sec:scc}.

\zzcommand{\prob}{\textsc{Secluded $d$-ECS}}
\begin{tcolorbox}[enhanced,title={\color{black} {\textsc{Secluded $d$-Edge-Connected Subgraph}$~$ (\prob{})}}, colback=white, boxrule=0.4pt,
	attach boxed title to top left={xshift=.3cm, yshift*=-2.5mm},
	boxed title style={size=small,frame hidden,colback=white}]
	
	\textbf{Input:}  
  An undirected graph $G$, a weight function, $\wOp \colon \ve{G} \to \N$, and integers $w, k \in \N$\\

	\textbf{Output:}
  Decide if there is a set $S \subseteq \ve{G}$ with weight $\w{S} \ge w$ and neighborhood size $\abs{\nei{S}} \le k$, such that $\induced{G}{S}$ is $d$-edge-connected.
\end{tcolorbox}

Note that $d$ is part of the problem and not the instance and can therefore be treated as a constant in the algorithm.
In an analogous fashion to \Cref{thm:total_scc_np_hard}, we can prove NP-hardness for $d > 1$. Because the reduction is so similar, we omit the proof.
\begin{theorem}\label{thm:d_edge_np_hard}
  \prob{} is NP-hard, for all $d > 1$.
\end{theorem}

Compared to \textsc{Total-Secluded SCS}, the algorithm for \prob{} shares some similarities. The auxiliary problems are defined analogously, and we again need a definition of border complementations. Furthermore, the general algorithm structure is the same. Again, we use several reduction rules that decrease the size of everything but the allowed neighborhood set $B$. Before we define the reduction rules, we work with extensions and use an equivalence relation with a similar intuition. 

However, executing this technically is more involved than before, as are the reduction rules.
Moreover, this section will be structured slightly differently than the previous one. First of all, this is done to avoid unnecessary repetition of the same concepts. Second, this is due to the fact that we lack some proof ingredients that are necessary for a concrete analysis of the parameter dependence. We still prove that \prob{} is solvable in \emph{non-uniform} FPT-time, that is, there is a constant $c$ such that for all values the parameter $k \in \N$ can take, we can find an algorithm with runtime $\bigO{n^c}$. This is still a nontrivial result, as $d$-edge-connected graphs can generally not be classified by a set of forbidden induced subgraphs or anything else that was solved before. This result also indicates that the problem is very likely to be uniformly FPT~\cite[Chapter~6]{cygan2015parameterized}.

We start out in \Cref{sec:ext_equiv} by defining a problem-specific equivalence relation on extensions, which will be used in a similar way as in \Cref{sec:scc}. After that, we define border complementations together with auxiliary problems in \Cref{sec:d_edge_bc} using extensions and the equivalence relation. In \Cref{sec:d_edge_unbreak}, we give an algorithm for the base case in which the graph is $(q,k)$-unbreakable. Finally, we give the necessary reduction rules in \Cref{sec:d_edge_algorithm}, and explain how to stick all pieces together.

\subsection{Extension Equivalence}\label{sec:ext_equiv}

Again, we work with graph extensions as in \Cref{sec:scc_extensions}. Naturally, for an undirected graph $G$, we define an extension to be a pair $\hexpair$, where $H$ is an undirected graph and $\hexset \subseteq \set{\set{h,v}}{h \in \ve{H}, v \in \ve{G}}$. As before, we can stick the extension onto the graph $G$ to form the undirected extended graph $\ex{G}{H}{\hexset} \coloneqq (\ve{G} \cup \ve{H}, \e{G} \cup \e{H} \cup \hexset)$. Intuitively, extensions should be thought of as \emph{graph add-ons}, a smaller graph $H$ that can be attached to some part of a larger graph $G$. How exactly $H$ should be attached to $G$ is defined by the additional set $\hexset$.

Naturally, since we aim to use an extension as part of a $d$-edge-connected induced subgraph, our equivalence relation will then works by counting the existing and potential paths through the extension. To make working with paths easier, we use the following definition. 

\begin{definition}[Path via $Q$]
  Let $P$ be a path in $G$ and let $Q \subseteq \ve{G}$. We call $P$ a \emph{path via $Q$} if every edge of $P$ has at least one endpoint in $Q$ and every intermediate vertex is in $Q$.
\end{definition}

Note that this definition is equivalent to disallowing intermediate vertices outside of $Q$ and disallowing the direct edge between two vertices $u, v \notin Q$ as a path via $Q$.

First, we want measure how an extension helps in forming $d$-edge-connected induced subgraphs with $G$, which we capture with the following definition. This is the analogue of the connection set $\connOp$ for strongly connected subgraphs. In this case, however, it is more involved. We use a \emph{demand function} that defines a demand of paths for vertex pairs in $G$ that should go via the extension in a solution. The extension then fulfills this demand if all of these demanded paths are present in the extension, and all of them are edge-disjoint. For a set $X$, we denote the set of all subsets of size 2 of $X$ with $\binom{X}{2}$.

\begin{figure}[t]
    \begin{minipage}[c]{0.45\linewidth}
    \centering
    \includegraphics[width=0.6\textwidth,page=1]{figures/d_edge}
    \caption{An example for fulfillment for $d = 2$. The extension $\hexpair$ fulfills the function that demands paths between $v_1$ and $v_3$ and $v_2$ and $v_4$. If we instead demand paths as in $f(\set{v_1, v_3}) = f(\set{v_1,v_4}) = 1$, then $f$ is not fulfilled by $\hexpair$ since there is only one edge incident to $v_1$.}
    \label{fig:fulfill}
    \end{minipage}
    \hfill
    \begin{minipage}[c]{0.45\linewidth}
    \centering
    \includegraphics[width=0.6\textwidth,page=2]{figures/d_edge}
    \caption{An example for covering and sufficiency for $d = 2$. The function that maps only $f(\set{v_2,v_3}) = 1$ covers, among others, the pairs $\set{h_2,h_3}$ and $\set{h_2, v_2}$, but not $\set{h_1,h_2}$. Setting only $f(\set{v_1, v_3}) = 1$ is sufficient for $\hexpair$ and $\set{v_1, v_3}$.}
    \label{fig:sufficient}
    \end{minipage}
\end{figure}

\begin{definition}[Fulfillment]
  Let $\hexpair$ be an extension of $G$ and let $f \colon \binom{\ve{G}}{2} \to [0,d]$ be a function, which we call a \emph{demand function}. We say that $\hexpair$ \emph{fulfills} $f$ if for every distinct $v_1, v_2 \in \ve{G}$ there are $f(\set{v_1, v_2})$ paths between $v_1$ and $v_2$ via $\ve{H}$ such that \textbf{all of these paths} are edge-disjoint.
\end{definition} 

See \Cref{fig:fulfill} for an example of a fulfilled demand function.

The next definition considers the same kind of functions but from a different point of view. This time, the function can be thought of a compressed version of the rest of a potentially $d$-edge-connected subgraph, and we ask if this enough to connect a specific pair of vertices in $\ex{G}{H}{\hexset}$. To differentiate these two roles we name this kind of function \emph{supply function} although the signature of the function is the same. Think about this definition as measuring what requirements there are for a $d$-edge-connected induced subgraph such that it can include $\hexpair$. For strongly connected subgraph, this was the role of the source and sink sets $\sourceOp$ and $\sinkOp$.

\begin{definition}
  Let $\hexpair$ be an extension of $G$, and $f \colon \binom{\ve{G}}{2} \to [0,d]$ be a \emph{supply function}. Consider the graph $G' \coloneqq \ex{G}{H}{\hexset} - \e{G}$. Now insert $f(\set{v_1, v_2})$ edges between every $v_1, v_2 \in \ve{G}$. We say that $f$ \emph{covers} a pair of vertices $v_1, v_2 \in \ve{G'}$, if there are $d$ edge-disjoint paths between $v_1$ and $v_2$ in $G'$.
\end{definition}

The last definition formalizes which vertex pairs in $H$ have $d$-edge-connected paths, if the remaining solution looks like the supply function $f$.
To form a $d$-edge-connected subgraph that completely includes $\hexpair$, all pairs have to be covered. Only then is the remaining solution enough to be able to include $H$. We formalize this by a set of supply functions that the solution must be able to provide to be sufficient for the extension $\hexpair$.

\begin{definition}[Sufficiency]
  Let $\hexpair$ be an extension of $G$, $V' \subseteq \ve{G}$, and $F$ a set of supply functions. We say that $F$ is \emph{sufficient} for $\hexpair$ and $V'$ if for every $h_1, h_2 \in \ve{H}$, there is an $f \in F$ that covers $h_1, h_2$ and every $h\in \ve{H}, v \in V'$ is covered by some $f \in F$.
\end{definition}

See \Cref{fig:sufficient} for an example of covering a pair and a sufficient supply function.

Now, we define our equivalence relation on extensions for this problem. Similar to the previous section, the intuition for two components being equivalent is that including them into a $d$-edge-connected subgraph has the same benefits and the same requirements. In this case, the benefits are all demand functions that the extension fulfills. The requirements are all sufficient supply function sets. With this intuition, the next definition should come naturally.

% \jonas{moved next to other figure}
% \begin{figure}[t]
%   \centering
%   \includegraphics[width=0.3\textwidth,page=2]{figures/d_edge}
%   \caption{An example for covering pairs and sufficiency for $d = 2$. The function that maps only $f(\set{v_2,v_3}) = 1$ covers, among others, the pairs $\set{h_2,h_4}$, $\set{h_2,h_3}$, and $\set{h_2, v_2}$, but not $\set{h_1,h_2}$. The supply function that only maps $f(\set{v_1, v_3}) = 1$ is on its own sufficient for $\hexpair$ and $\set{v_1, v_3}$.}
%   \label{fig:sufficient}
% \end{figure}

\begin{definition}[Equivalence]\label{def:dequiv}
  Let $\hexpairi{1}, \hexpairi{2}$ be two extensions of $G$. We say that $\hexpairi{1}$ and $\hexpairi{2}$ are equivalent or $\hexpairi{1} \dequiv \hexpairi{2}$ if \begin{enumerate}[noitemsep,nolistsep]
    \item for all demand functions $f$, we have $\hexpairi{1}$ fulfills $f$ if and only if $\hexpairi{2}$ fulfills $f$,
    \item for all $V' \subseteq \ve{G}$ and supply function sets $F$, we have that $F$ is sufficient for $\hexpairi{1}$ and $V'$ if and only if it is sufficient for $\hexpairi{2}$ and $V'$.\qedhere
  \end{enumerate}
\end{definition}
\begin{observation}
  $\dequiv$ is an equivalence relation.
\end{observation}

To justify this definition, we have to show multiple properties. First of all, it has be computable in non-uniform FPT-time whether two components are equivalent.

Note that because of non-uniformity, the statement of the next lemma and others will be slightly odd, since we give a separate algorithm for every possible graph size $\abs{\ve{G}}$. In this case, this is due to the fact, the we use another non-uniform algorithm as a subroutine. Later, we use this kind of statement bound runtimes with unknown parameter dependence. For every constant parameter, the algorithm still runs in time bounded by the same polynomial, but we are unable to bound the $f(k)$ term.

\iflong
\begin{lemma}
\else
\begin{lemma}[$\star$]
\fi
\label{lem:check_equiv}
  For every $c \in \N$, there is an algorithm that checks for $\abs{\ve{G}} = c$ if two extensions $\hexpairi{1}$ and $\hexpairi{2}$ of $G$ are equivalent in time $\bigO{(\abs{\ve{H_1}} + \abs{\ve{H_2}})^4}$. 
\end{lemma}
\iflong
\begin{proof}
  For the first condition of \Cref{def:dequiv}, iterate over all $f$ in time $(d+1)^{\binom{c}{2}}$. For each $f$, we want to check if $\hexpairi{1}$ and $\hexpairi{2}$ fulfill $f$. To do this, we reduce to the edge-disjoint path problem, where you are given $p$ terminal pairs $(s_1, t_1), \ldots, (s_p, t_p)$ and want to find $p$ edge-disjoint paths, one between each pair.
  To reduce to this problem, for each $\uned{v_1}{v_2} \in \binom{\ve{G}}{2}$, we add $f(\uned{v_1}{v_2})$ leaves to $v_1$ and to $v_2$. For each such leaf pair $\ell_{v_1}, \ell_{v_2}$ we add one terminal pair between them.

  Now, we run the edge-disjoint path algorithm by \cite{robertson1995graph} that runs in time $\bigO{n^3}$ for a constant number of terminal pairs, where $n$ in our case is $\abs{\ve{H_1}}$ or $\abs{\ve{H_2}}$. Note that the number of terminal pairs is here is bounded by $\binom{c}{2}d$ and can thus be treated as constant.

  For the second condition of \Cref{def:dequiv}, iterate over all $V'$ and $F$ in time $2^{c}2^{(d+1)^{\binom{c}{2}}}$. Then, we check for all of the at most $(\abs{\ve{H_i}} + c)^2$ pairs if they are covered by a function in $F$. For each $f \in F$, we build $G'$ and insert the corresponding edges. Then, we can build the Gomory-Hu Tree~\cite{gomory1961multi} using $\abs{\ve{H_i}} + c$ max flow calculations with Orlin's algorithm~\cite{orlin2013max}. Then, we can easily read off the min cut value between each pair, which corresponds to the number of edge-disjoint paths. Since $c$ can be treated as a constant, this gives the total time $\bigO{\abs{\ve{H_i}}^4}$.
\end{proof}
\fi

\zzcommand{\classes}[1]{C(#1)}
\zzcommand{\classesc}[1]{C'(#1)}
Next, we want to show that the number of equivalence classes of $\dequiv$ can be bounded by a function of $\abs{\ve{G}}$, such that we can use it to bound the number of components.
To work with the equivalence classes, we want to know a smallest extension out of every non-empty equivalence class. So, for a graph $G$, pick an extension $\hexpair$ such that $\abs{\ve{H}}$ is minimum out of every non-empty equivalence class. We denote this set by $\classes{G}$.
Additionally, we consider $\classesc{G}$, defined in the same way as $\classes{G}$, except that we only consider extensions $\hexpair$ where $H$ is connected. Clearly, we have $\abs{\classesc{G}} \le \abs{\classes{G}}$.

\iflong
\begin{lemma}
\else
\begin{lemma}[$\star$]
\fi
\label{lem:number_equiv}
  There is $c \in \N$, such that the number of equivalence classes in $\dequiv$ is at most $\abs{\classes{G}} \le 2^{2^{2^{c\ve{G}^2}}}$. 
\end{lemma}
\iflong
\begin{proof}
  First, notice that there are $p \coloneqq (d+1)^{\binom{\abs{\ve{G}}}{2}}$ different supply or demand functions since for every pair in $\ve{G}$ we get to choose between values in $[0, d]$.
  Each class is now first identified by the set of demand functions it fulfills, of which there are $2^p$. Second, each subset $V'$ and supply function set $F$, of which there are $q \coloneqq 2^{\abs{\ve{G}}}2^p$, could form a sufficient combination. In total this make $2^p2^q$ equivalence classes, which gives the desired bound, since we treat $d$ as a constant.
\end{proof}
\fi

\iflong
\begin{lemma}
\else
\begin{lemma}[$\star$]
\fi
\label{lem:compute_classes}
  For every $c \in \N$, there is an algorithm that computes $\classes{G}$ and $\classesc{G}$ for all graphs $G$ with $\abs{\ve{G}} = c$ in constant time.
\end{lemma}
\iflong
\begin{proof}
  By \Cref{lem:number_equiv}, the number of equivalence classes is bounded by a function in $c$ and $d$. Therefore, the size of the smallest element of every non-empty equivalence class is also bounded by a function in $c$ and $d$; let us call this value $f(c,d)$. If we treat $c$ and $d$ as constants, we can simply generate all extensions with at most $f(c,d)$ vertices and sort them into equivalence classes using \Cref{lem:check_equiv}. Out of every non-empty class, we pick one extension with the minimum number of vertices for $\classes{G}$ and the minimum connected extension for $\classesc{G}$ if it exists.
\end{proof}
\fi

The crucial property of the equivalence relation $\dequiv$ is that two equivalent extensions form exactly the same $d$-edge-connected induced subgraphs. We prove this in \Cref{lem:equiv_d_edge}, justifying our definition. Later, this fact will allow us to replace components by equivalent components. Before we are able to prove this result, we need another lemma to simplify working with $d$-edge-connected subgraphs.

\iflong
\begin{lemma}
\else
\begin{lemma}[$\star$]
\fi
\label{lem:connect_paths}
  Let $G$ be a graph with $x,y,z \in \ve{G}$. If there are $d$ edge-disjoint paths between $x$ and $y$ and $d$ edge-disjoint paths between $y$ and $z$, there are also $d$ edge-disjoint paths between $x$ and $z$.
\end{lemma} 
\iflong
\begin{proof}
  The condition translates to the fact that the size of the min cut between $x$ and $y$ is at least $d$; the same hold for the min cut between $y$ and $z$. Consider the min cut between $x$ and $z$ and look at the graph with the min cut edges removed. If $y$ is not in a component with $x$ or $z$, consider a path from $y$ to either $x$ or $z$ that does not use vertices from the component of $x$ or $z$. This exists since $d \ge 1$. Then, we can remove all cut edges on the path from the cut, by the choice of the path, this will be a smaller cut, a contradiction. Therefore, without loss of generality $y$ lies in the same component as $x$ and the min cut between $x$ and $z$ is also a cut between $y$ and $z$. Hence it has size at least $d$, proving that there must be $d$ edge-disjoint paths between $x$ and $z$.
\end{proof}
\fi

Now, we are able to prove the crucial lemma about equivalent extensions. Note that this lemma is phrased in a way that will be slightly easier to apply, since we allow $G$ to have other extensions that cannot interact with our equivalent extensions. Later, the other extension $\hexpair$ will correspond to other components in $G-B$ that are also part of a solution.

\iflong
\begin{lemma}
\else
\begin{lemma}[$\star$]
\fi
\label{lem:equiv_d_edge}
  Let $G$ be an undirected graph with an extension $\hexpair$ and two equivalent extensions $\hexpairi{1} \dequiv \hexpairi{2}$. Consider $G' \coloneqq \ex{G}{H}{\hexset}$. Let $S \subseteq \ve{G'}$ such that $S \cup \ve{H_1}$ is $d$-edge-connected in $\ex{G'}{H_1}{\exset{G}{H_1}}$. Then, $S \cup \ve{H_2}$ is also $d$-edge-connected in $\ex{G'}{H_2}{\exset{G}{H_2}}$.
\end{lemma}
\iflong
\begin{proof}
  Define $V' \coloneqq \ve{G} \cap S$ and let $u, v \in S \cup \ve{H_2}$.
  If $u, v \in S$, consider the $d$ edge-disjoint paths between $u$ and $v$ in $S \cup \ve{H_1}$ with the subpaths $P_1, \ldots, P_{\ell}$ via $\ve{H_1}$. From these paths, we construct the corresponding demand function $f$ such that every path $P_i = v_{i,1}, \ldots, v_{i,2}$ contributes 1 to $f(\uned{v_{i,1}}{v_{i,2}})$. Then, clearly, $\hexpairi{1}$ must fulfill $f$ and so does $\hexpairi{2}$. Therefore, we can exchange $P_1$ to $P_{\ell}$ by edge-disjoint subpaths via $\ve{H_2}$ and there are $d$ edge-disjoint $u$-$v$-paths in $S \cup \ve{H_2}$.

  If $u, v \in \ve{H_2}$, we use the second property of equivalence. Consider all pairs $h \in \ve{H_2}, v \in V'$ and $h_1, h_2 \in \ve{H_2}$ with their $d$ respectively edge-disjoint paths in $S \cup \ve{H_1}$. We focus on the subpaths via $S$ and notice that we can construct a supply function for every such pair that covers it and corresponds to existing paths in $S$. Furthermore, this gives us a sufficient set $F$ for $\hexpairi{1}$ and $V'$ that must also be sufficient for $\hexpairi{2}$ and $V'$ by equivalence. Thus, there is $f \in F$ that covers $\uned{u}{v}$ in this case as well as the case $u \in \ve{H_2}$ and $v \in V'$.

  Finally, the only remaining case is $u \in \ve{H_2}$ and $v \in S \setminus V'$. Note that in this case there must be a $v \in V'$ since otherwise $S$ cannot be connected. Connectivity follows from the previous cases together with \Cref{lem:connect_paths}.
\end{proof}
\fi

\subsection{Border Complementations}\label{sec:d_edge_bc}

We define two auxiliary problems in the same way as in~\cite{golovach2020finding} and \Cref{sec:scc_bc}. We need them, to remember more information about the instance in recursive calls and work with a maximization problem instead of a decision problem.
\zzcommand{\probrec}{\textsc{Max \prob{}}}
\begin{tcolorbox}[enhanced,title={\color{black} {\probrec{}}}, colback=white, boxrule=0.4pt,
	attach boxed title to top left={xshift=.3cm, yshift*=-2.5mm},
	boxed title style={size=small,frame hidden,colback=white}]
	
	\textbf{Input:}  
  An undirected graph $G$, subsets $I,O,B \subseteq \ve{G}$, a weight function $\wOp \colon \ve{G} \to \N$, and an integer $k \in \N$\\

	\textbf{Output:}
  The maximum weight set $S \subseteq \ve{G}$ with $I \subseteq S$, $O \cap S = \emptyset$, $\nei{S} \subseteq B$, and $\abs{\nei{S}} \le k$, such that $\induced{G}{S}$ is $d$-edge-connected, or report that no feasible solution exists.
\end{tcolorbox}


We use our component equivalence definition to define border complementations in a similar way to~\cite{golovach2020finding} for finding $\mathcal{F}$-free secluded subgraphs, less explicitly than in \Cref{sec:scc_bc}. That is, we do not define the added vertices and edges directly, but use a small extension from an equivalence class of $\dequiv$.

\begin{definition}[Border Complementation]\label{def:d_edge_border_complementation}
  Let $(G,I,O,B,\wOp,k)$ be an instance for \probrec{} with a set $T \subseteq \ve{G}$ of \emph{border terminals}. A border complementation $(G',I',O',B,\wOp',k')$ is an instance obtained in the following way. Let $X,Y,Z$ be a partition of $T$ and let $\hexpair \in \classes{\induced{G}{X}}$. Additionally, we have that
  \begin{enumerate}[noitemsep,nolistsep]
    \item $G'$ is obtained by extending $X$ with $\hexpair$ and including edges $\uned{x}{y}$ for all $x \in X, y \in Y$,
    \item $I' \coloneqq I \cup X \cup \ve{H}$,
    \item $O' \coloneqq O \cup Y \cup Z$,
    \item $\wOp'(v) \coloneqq \w{v}$ for $v \in \ve{G}$ and $\wOp'(h) \coloneqq 0$ for $h \in \ve{H}$, and
    \item $k' \le k$.\qedhere
  \end{enumerate}
\end{definition}

Finally, we define the bordered problem. Analogously to \Cref{sec:scc_bc}, the task in this problem is to solve all border complementation instances of \probrec{}.

\zzcommand{\probborder}{\textsc{Bordered \probrec{}}}
\begin{tcolorbox}[enhanced,title={\color{black} {\probborder{}}}, colback=white, boxrule=0.4pt,
	attach boxed title to top left={xshift=.3cm, yshift*=-2.5mm},
	boxed title style={size=small,frame hidden,colback=white}]
	
	\textbf{Input:}  
  A \probrec{} instance $\mathcal{I} = (G,I,O,B,\wOp,k)$ and a set of border terminals $T \subseteq \ve{G}$ with $\abs{T} \le 2k$\\

	\textbf{Output:}
  A solution to \probrec{} for each border complementation of $\mathcal{I}$ and $T$, or report that no solution exists.
\end{tcolorbox}

\subsection{Unbreakable Case}\label{sec:d_edge_unbreak}

This section constitutes the base case of our recursive understanding algorithm, and we solve it in a similar fashion to our approach in \Cref{thm:unbreakable_scc}. The definitions of separation and unbreakability as well as \Cref{lem:unbreak_small_or_large,lem:find_sets} immediately transfer to undirected graphs. Therefore, we can give the algorithm description right away.

\iflong
\begin{theorem}
\else
\begin{theorem}[$\star$]
\fi
  There is a $c \in \N$, such that for every $k \in \N$, there is an algorithm that solves a \probborder{} instance $(G,I,O,B,\wOp,k,T)$ on a $(q,k)$-unbreakable graph in time $\bigO{(nq)^c}$.
\end{theorem}
\iflong
\begin{proof}
  Our algorithm works as follows. Initially, we enumerate all partitions of $T$ into $X,Y,Z$. Then, we compute $\classes{\induced{G}{X}}$ using \Cref{lem:compute_classes}. This way, we can enumerate all border complementation instances for \probrec{}. Consider one such instance $\mathcal{I'} \coloneqq (G', I', O', B', \wOp', k')$. By \Cref{lem:unbreak_small_or_large}, there is an $s \le q + f(k)$ for some function $f$, such that for every solution $S$ of $\mathcal{I'}$, we have either $\abs{S} \le s$ or $\abs{\ve{G'} \setminus S} \le s$. We address both cases separately and return the maximum weight solution of the solutions for both cases, or none if both do not exist.

  \subparagraph*{Finding a small solution} Our algorithm for this case is similar to the small case in \Cref{thm:unbreakable_scc}.
  \begin{enumerate}[noitemsep,nolistsep]
    \item We apply \Cref{lem:find_sets} with $U = \ve{G'}, a = s, b = k$ to compute a family $\mathcal{F}$ of subsets of $\ve{G'}$.
    \item For every $F \in \mathcal{F}$, compute the $d$-edge-connected components of $F$.
    \item For every $d$-edge-connected component $Q$, check its feasibility, and return the maximum weight feasible solution.
  \end{enumerate}

  \subparagraph*{Finding a large solution} Our algorithm for this case works as follows.
  \begin{enumerate}[noitemsep,nolistsep]
    \item Compute the $d$-edge-connected components of $G'$.
    \item For every $d$-edge-connected component $C$, if $\abs{\nei{C}} \le k$, then $C$ already is the maximum solution that is a subset of $C$. If not we proceed by constructing $G'_C$, analogously to \Cref{thm:unbreakable_scc}, by taking $\induced{G}{\cnei{C}}$, adding a vertex $c$, and connecting it to every $v \in \nei{C}$.
    \item Next, we run the algorithm from \Cref{lem:find_sets} in $G'_C$ with $U = \ve{G'_C}, a = s+1, b = k$.
    \item For all returned sets $F$, we only consider the weak component $S$ in $F$ that includes $c$ if such a component exists.
    \item Finally, we verify whether $C \setminus \cnei{S}$ is a feasible solution and return the maximum weight one.
  \end{enumerate}

  \subparagraph*{Correctness}

  For the small case, we know that for any solution $S$ with $\abs{S} \le s$, there is $F \in \mathcal{F}$ with $S \subseteq F$ and $\nei{S} \cap F = \emptyset$. Therefore, $S$ must be a $d$-edge-connected component of $F$ and this case is correct.

  Correctness for the large case is mostly analogous to \Cref{thm:unbreakable_scc}. We only have to consider the case of a solution $S' \subseteq C$ such that there is a component $C'$ of $\ve{G'_C} \setminus S$ that does not include $c$. We claim that we can include $C'$ to $S'$ to still give a solution. Clearly, the weight is non-decreasing and neighborhood size decreases, so we focus on edge-connectivity.

  First, notice that both $S'$ and $C$ induce $d$-edge-connected subgraphs of $G'$. That means the global min cuts of $\induced{G'}{S'}$ and $\induced{G'}{C}$ both have a value of at least $d$. We have to show that $\induced{G'}{S' \cup C'}$ also has a global min cut value of at least $d$. Suppose the global min cut is the min cut between a pair of vertices $u, v \in S' \cup C'$. We know that $u,v \in S'$ cannot be the case since adding $C'$ can only increase the min cut. Therefore, assume without loss of generality $u \in C'$ and first consider $u \in S'$. Since $C$ induces a $d$-edge-connected subgraph, removing any set of $d-1$ edges still allows us to find a path from $u$ to some $w \in S'$. Since $S'$ is $d$-edge-connected, we can still reach $v$ from $w$.
  The final case is $u, v \in C'$. Again, after removing $d-1$ edges, there still is a path from $u$ to $S'$ and from $v$ to $S'$ since $C$ is $d$-edge-connected and $\nei{C'} \subseteq S'$. Also, $S'$ is still connected after the removal of edges.

  \subparagraph*{Runtime} By \Cref{lem:find_sets}, there is a $c \in \N$, such that running the algorithm costs us $\bigO{s^cn^c} = \bigO{q^cn^c}$ for a constant $k$. We can find all $d$-edge-connected components in time $\bigO{n^4}$ using an algorithm from \cite{wang2015simple}. By \Cref{lem:compute_classes} computing and adding extensions to the graph only adds a constant overhead.
\end{proof}
\fi

\subsection{Reduction Rules and Algorithm}\label{sec:d_edge_algorithm}

Now, we will give some important reduction rules for \probborder{}. First, we transfer some basic reduction rules from \Cref{sec:scc}. We state them without proof since the proofs are simple and analogous to the corresponding reduction rules in the previous section.

\begin{reduction*}\label{red:d_edge_in_out}
  Let $Q$ be a component of $G - B$. If $Q \cap O \ne \emptyset$, set $O = O \cup \cnei{Q}$. If $\cnei{Q} \cap I \ne \emptyset$, set $I = I \cup Q$. If both cases apply, the instance has no solution.
\end{reduction*}
\begin{reduction*}\label{red:d_edge_remove_out}
  If \Cref{red:d_edge_in_out} is not applicable and there exists $v \in O \setminus B$, remove $v$ from $G$.
\end{reduction*}

Whenever possible we first apply \Cref{red:d_edge_in_out} and then \Cref{red:d_edge_remove_out} exhaustively in this order.

We will think of components $Q$ of $G-B$ as extensions $(\induced{G}{Q}, \set{\uned{q}{b}}{q \in Q, b \in B})$ of $\induced{G}{B}$. We will naturally extend equivalence on extensions to components as well as demand and supply function to functions on $B$.
The next rule consists of two closely related parts. First, it removes all components that can never be inside any solution because there are no $d$-edge-connected subgraphs in $G$ containing it. Second, if there are components that could be a solution by themself but cannot be part of a larger solution, we disconnect them from $B$. Later, this allows us to assume that any component could be part of some larger feasible solution.

\iflong
\begin{reduction*}
\else
\begin{reduction*}[$\star$]
\fi
\label{red:remove_useless}
  Let $Q$ be a component of $G - B$. If there is no $B' \subseteq B$ and a set of supply functions $F$ such that $F$ is sufficient for $Q$ and $B'$, include $Q$ into $O$.

  If the former does not apply, but there is no \emph{non-empty} $B' \subseteq B$ and $F$ such that $F$ is sufficient for $Q$ and $B'$, do the following. If $Q$ is a feasible solution, disconnect $Q$ from $B$ and set $O = O \cup \nei{Q}$. Otherwise, include $Q$ into $O$.
\end{reduction*}
\iflong
\begin{proof}[Proof of Safeness]
  Assume there is a solution $S$ with $Q \subseteq S$. Let $B' \coloneqq B \cap S$. Since $S$ is $d$-edge-connected, there are $d$ edge-disjoint paths between every pair of vertices. For every $v_1, v_2 \in Q \cup B'$, define a supply function $f$, where $f(\set{b_1, b_2})$ is the number of $b_1$-$b_2$-paths that is used to construct the $d$ edge-disjoint paths between $v_1$ and $v_2$ in $S$. Then, the set of all these $f$ must be sufficient for $Q$ and $B'$. Hence, in case the reduction rule applies, there cannot be a solution including $Q$ and we can safely remove it from the graph.

  For the second part of the rule, it is clear from the previous proof that for every solution $S$ we either have $S \cap Q = \emptyset$ or $S = Q$. If the neighborhood condition holds, we can simply disconnect it from its neighbors. Now, any solution can no longer include the former neighbors, but $Q$ itself is still a valid solution. Otherwise, $Q$ cannot be part of a solution and can thus safely be removed.
\end{proof}
\fi

After the previous reduction has been applied exhaustively, we can show that two equivalent components must have the same neighborhood.

\iflong
\begin{lemma}
\else
\begin{lemma}[$\star$]
\fi
\label{lem:similar_eq_nei}
  Suppose that \Cref{red:remove_useless} is not applicable in $G$. Then for two components $Q_1$ and $Q_2$ of $G-B$ with $Q_1 \dequiv Q_2$, we have $\nei{Q_1} = \nei{Q_2}$.
\end{lemma}
\iflong
\begin{proof}
  If $\abs{\nei{Q_1}} \ge 2$, let $b_1, b_2 \in \nei{Q_1}$. Then, $Q_1$ fulfills the demand function $f$ with $f(\set{b_1,b_2}) = 1$ and $f(\set{b,b'}) = 0$ for all other $b,b' \in B$. Since $Q_2$ must also fulfill $f$, we have $b_1, b_2 \in \nei{Q_2}$ and hence $\nei{Q_1} = \nei{Q_2}$. If $Q_i$ cannot be part of a solution or only be a solution on its on, it has $\nei{Q_i} = \emptyset$, by \Cref{red:remove_useless}. Since $d > 1$, there must be at least two edges from $Q_i$ to $B$ if $Q_i$ can be proper subset of a solution. If both edges lead to the same vertex, we can create a sufficient supply function, which again shows $\nei{Q_1} = \nei{Q_2}$.
\end{proof}
\fi

Now, we get to the core of our algorithm. In \Cref{red:d_edge_twins}, we prove that components of $G-B$ can be replaced by equivalent components, which follows directly from \Cref{lem:equiv_d_edge}. Next, we prove that when there are multiple equivalent components, it is enough to keep only a certain number of them that depends only on $\abs{B}$ and $d$. More components than this threshold are not relevant for forming $d$-edge-connected subgraphs and can always be included in a solution that includes the other equivalent components. Note that in previous applications of this approach such as the algorithms in~\cite{golovach2020finding} or \Cref{sec:scc}, the number of necessary components was always at most 2. In \Cref{red:d_edge_many_twins} we use a slightly more involved argument to acknowledge the fact that paths have to be edge-disjoint.

\iflong
\begin{reduction*}
\else
\begin{reduction*}[$\star$]
\fi
\label{red:d_edge_twins}
  Let $Q$ be a component of $G-B$ and let $Q^* \in \classesc{\induced{G}{B}}$ be the minimum size component that is equivalent to $Q$. Replace $Q$ with $Q^*$, set $\w{Q^*} \coloneqq \w{Q}$, and set $Q^* \subseteq I$ if $Q \subseteq I$.
\end{reduction*}
\iflong
\begin{proof}[Proof of Safeness]
  Remember that components can only be included as a whole, and note that $\nei{Q} = \nei{Q^*}$ by \Cref{lem:similar_eq_nei}. Let $S$ be a solution to the old instance and define $B' \coloneqq B \cap S$. Clearly, if $Q \not\subseteq S$, then $S$ is a solution to the new instance, so assume $Q \subseteq S$. Consider $S' \coloneqq (S \setminus Q) \cup Q^*$ and notice that $\nei{S'} = \nei{S}$ and $\w{S'} = \w{S}$. 
  Using \Cref{lem:equiv_d_edge} with $G = \induced{G}{B}$ and extensions corresponding to $Q$, $Q^*$, and $\ve{G} \setminus (B \cup Q)$, we conclude that $S'$ is also $d$-edge-connected.

  In this proof we only relied on the fact that $Q$ and $Q^*$ are equivalent. Therefore, the proof that any solution to the new instance can be transformed to a solution to the old instance follows by symmetry.
\end{proof}
\fi

As discussed before, the next rule is the most interesting one. We show that only $h$ copies of components from every equivalence classes are necessary to construct all solutions, for some number $h$ that only depends on $\abs{B}$ and $d$.

\iflong
\begin{reduction*}
\else
\begin{reduction*}[$\star$]
\fi
\label{red:d_edge_many_twins}
  For $h \coloneqq \max\set{(\abs{B}-1)d + 2, 2}$, let $Q_1, \ldots, Q_{h+1}$ be equivalent components of $G-B$ ordered by weight non-increasingly. Then, remove $Q_{h+1}$ from $G$ and increase $\w{Q_1}$ by $\w{Q_{h+1}}$. If $Q_{h+1} \subseteq I$, add $Q_1$ to $I$.
\end{reduction*}
\iflong
\begin{proof}[Proof of Safeness]
  We know that components can only be included as a whole. Let $S$ be a solution to the old instance.
  If $Q_{h+1} \not\subseteq S$, since $Q_{h+1} \cap B = \emptyset$, we immediately have that $S$ is also a solution for the new instance.
  If $Q_{h+1} \subseteq S$, either $S = Q_{h+1}$ or some vertices of $B$ must also be part of the solution. In the first case, $Q_1$ is a solution for the new instance with at least the same weight since $F = \emptyset$ must be sufficient for $Q_i$ and $\emptyset$ for all $i \in [h+1]$.
  In the second case, since all $Q_i$ have the same neighborhood, we must have $Q_i \subseteq S$ for all $i \in [h+1]$. We claim that $S' \coloneqq S \setminus Q_{h+1}$ is a solution for the new instance. Clearly, the neighborhoods of $S$ and $S'$ are the same. We now prove the $d$-edge-connectedness of $S'$.

\begin{figure}[t]
  \centering
  \includegraphics[width=0.25\textwidth,page=3]{figures/d_edge}
\caption{A visualization of one part of the proof of safeness of \Cref{red:d_edge_many_twins}. The example shows that a shortest path from a vertex $q_1 \in Q_1$ to a vertex $q_4 \in Q_4$ can use at most $2$ intermediate components $Q_2$ and $Q_3$ for $\abs{B} = 3$.}
  \label{fig:many_twins}
\end{figure}

  Let $s_1, s_2 \in S'$ and consider the $d$ edge-disjoint paths $P_1, \ldots, P_d$ between $s_1$ and $s_2$ in $S$, such that $\abs{P_i}$ is minimal for all $i \in [d]$, meaning that there is no way to replace one $P_i$ by a shorter path that is still edge-disjoint to the other paths. Now, the paths might use subpaths via $Q_{h+1}$. In this case we aim to replace these by different subpaths while staying edge-disjoint. For each $b \in B$, note that each $P_i$ can only include $b$ once. Otherwise, it includes a circle, which we could easily shortcut to decrease $\abs{P_i}$. Hence, we can think of $P_i$ as a path consisting of at most $\abs{B}+1$ subpaths that all start and end in a vertex in $B$, except for the first and last one. This part of the proof is also visualized in \Cref{fig:many_twins}. Now, for each such subpath, if it is a path via $Q_{h+1}$, instead reroute it via a different $Q_i$ that does not intersect with any path so far. We know that such a $Q_i$ must exist since there are only $(\abs{B}+1)d$ subpaths in total, each subpath is via exactly one component of $G-B$, and the first the last subpaths must be in the components of $s_1$ and $s_2$.  Furthermore, there is a corresponding path in $Q_i$ since $Q_i$ and $Q_{h+1}$ are equivalent and fulfill the same demand functions. Hence, we have constructed $d$ edge-disjoint paths between $s_1$ and $s_2$ in $S'$ and $S'$ is a solution to the new instance.

  For the other direction, consider a solution $S'$ for the new instance. If one $Q_i$ for $i \in [h]$ is not included in $S'$, then none of $\nei{Q_i}$ can be included in $S'$. Since $\nei{Q_i} = \nei{Q_{h+1}}$, we have that $S'$ is also a solution for the old instance.
  Suppose $Q_i \subseteq S'$ for all $i \in [h]$. We claim that $S \coloneqq S' \cup Q_{h+1}$ is a solution for the old instance. Since all $Q_i$ have the same neighborhood, we have $\nei{S'} = \nei{S}$. We now prove the $d$-edge-connectedness of $S$.

  Let $s_1, s_2 \in S$. If both $s_1, s_2 \in S'$, there are $d$ edge-disjoint paths between $s_1$ and $s_2$ only using $S'$. Therefore, the only two remaining cases are $s_1, s_2 \in Q_{h+1}$ and $s_1 \in Q_{h+1}, s_2 \notin Q_{h+1}$. We first consider the case where both vertices are in $Q_{h+1}$. Let $B' = B \cap S$. Since $Q_1 \subseteq S'$ and $S'$ is a solution, we know $S' \setminus Q_1$ must provide the paths from a sufficient supply function set $F$ for $Q_1$ and $B'$. Since $Q_1$ is equivalent to $Q_{h+1}$, the set $F$ is also sufficient for $Q_{h+1}$ and $B'$ and there are $d$ edge-disjoint paths between $s_1$ and $s_2$.

  Suppose without loss of generality only $s_1 \in Q_{h+1}$. If $s_2 \in B'$, since $S'$ provides the paths from a sufficient set $F$ for $Q_{h+1}$ and $B'$, there must be $d$ edge-disjoint paths between $s_1$ and $s_2$. If $s_2 \in Q_i$ for some $i \in [h]$, consider additionally some $b \in B'$. We know that there are $d$ edge-disjoint paths between $s_1$ and $b$ and $d$ edge-disjoint paths between $b$ and $s_2$. By \Cref{lem:connect_paths} there are at least $d$ edge-disjoint paths between $s_1$ and $s_2$ in $S$.
\end{proof}
\fi

Now, we can prove that the recursive understanding algorithm works also works for \probborder{}. Before, we summarize the progress of \Cref{red:d_edge_twins,red:d_edge_many_twins} in the following lemma.
\begin{lemma}\label{lem:d_all_reductions}
  Either we can reduce the instance, or there are at most $\max\set{(\abs{B}-1)d + 2, 2}$ components of every equivalence class. Furthermore, each component $Q$ has minimum size out of its equivalence class.
\end{lemma}

Finally, we arrive at the complete algorithm, which follows the same structure as the algorithm given in \Cref{alg:rec_und_scc}. Since the proof is very technical but also analogous to the proof of \Cref{thm:border_scc_fpt}, we only sketch it for completeness. 

\iflong
\begin{theorem}
\else
\begin{theorem}[$\star$]
\fi
\label{thm:border_d_edge_fpt}
  There is a constant $c \in \N$, such that for each $k \in \N$, there is an algorithm that solves a \probborder{} instance $\mathcal{I} = (G,I,O,B,\wOp,k,T)$ in time $\bigO{n^c}$.
\end{theorem}
\iflong
\begin{proof}
  The algorithm structure is almost identical to the algorithm of \Cref{thm:border_scc_fpt} as illustrated in \Cref{alg:rec_und_scc}, which is why we do not state it again here. The only difference besides the definition of border complementations and the set of reduction rules is the choice of $q$. In our case, $q$ will be constant for a fixed $k$ and we later prove that a suitable $q$ exists.

  We again have to prove that a border complementation covers all relevant parts of a solution, that is that we can consider $\mathcal{\hat{I}} \coloneqq (G,I,O,\hat{B},\wOp,k,T)$ instead of $\mathcal{I}$. To do this, let $S$ be a solution to a border complementation of $\mathcal{I}$. Consider the border complementation for $\tilde{G}$ that chooses $X$, $Y$, and $Z$ to match with $S$, that is $X = S \cap \tilde{T}$, $Y = \nei{S} \cap \tilde{T}$, and $Z = \tilde{T} \setminus \cnei{S}$. Furthermore, choose the border complementation that has an equivalent extension to $(\induced{G}{S \cap (U \setminus W)}, \set{\uned{u}{x}}{u \in S \cap U, x \in X})$. By \Cref{lem:equiv_d_edge}, both graphs have the same $d$-edge-connected induced subgraphs that intersect $W$. Therefore, it is enough to consider the instance with $\hat{B}$.

  Regarding the runtime and the choice of $q$, notice that $\abs{T} \le 2k$. By \Cref{lem:number_equiv}, there are at most $3^{2k}f(k)$ relevant border complementations for some function $f$. By definition of $\hat{B}$, we can bound $\abs{\hat{B}} \le 2k + k 3^{2k}f(k) \eqqcolon g(k)$. Let $h$ be the maximum size of a minimal equivalent extension of a graph with at most $g(k)$ vertices. Using \Cref{lem:d_all_reductions}, all components of $G-\hat{B}$ have at most $h$ vertices. Since the number of components per class is also bounded, we can bound the size of $W^*$ by the maximum size of each component, multiplied with the number of equivalence classes and the number of components per class, that is \[\abs{W^*} \le hf(k)((g(k)-1)d + 2) + g(k) \eqqcolon q,\] which is constant for a fixed $k$.

  Also, we only have to check equivalence and compute the classes for extensions of graphs with at most $g(k)$ vertices, which is constant for a fixed $k$ by \Cref{lem:compute_classes}. The other operations of the reduction rules are clearly polynomial. Together with the runtime analysis from \Cref{thm:border_scc_fpt}, this gives the desired runtime.
\end{proof}
\fi

Finally, we solve \prob{} using \probborder{}. We again set $I \coloneqq O \coloneqq \emptyset$, $B \coloneqq \ve{G}$, and $T \coloneqq \emptyset$.

\begin{theorem}\label{cor:d_edge_fpt}
  \prob{} is non-uniformly FPT.
\end{theorem}

What would be necessary to strengthen \Cref{cor:d_edge_fpt} and show that \prob{} is also uniformly FPT with a bounded and computable parameter dependence $f(k)$? There are two main hurdles. First, the use of a non-uniform FPT-algorithm from \cite{robertson1995graph} in \Cref{lem:check_equiv} has to be avoided. If an algorithm for the edge-disjoint path problem in uniform FPT-time is discovered, this can be used. Another way would be to check the definition of $\dequiv$ differently, or define $\dequiv$ in way that does not need such a heavy machinery. 

Second, we would need a way to define a constructive compression routine, such as the one in \Cref{sec:scc} for directed extensions. This would allow us to bound the size of the compressed extension by a function of $k$ and hence also the size of $W^*$ after applying all reduction rules. Until both of these conditions are achieved, the runtime of this algorithm stays non-uniform.
Nevertheless, this is still a strong indication towards proving the existence of a uniform FPT-algorithm~\cite[Chapter~6]{cygan2015parameterized}.
