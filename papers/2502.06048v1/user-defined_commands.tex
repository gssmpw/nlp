% This is where all the commands should go that you want to define yourself.

\def\zzcommand#1{\let#1\undefined\newcommand#1}

\newtheorem{fact}[theorem]{Fact}
% \newtheorem{reduction}{Reduction Rule}[section]
% \newtheorem{branching}[reduction]{Branching Rule}
\newcommand{\wi}{\textsf{W[1]}}

\newtheorem{reduction*}[theorem]{Reduction Rule}

\Crefname{reduction*}{Reduction Rule}{Reduction Rules}
\Crefname{step}{Step}{Steps}
\Crefname{enumi}{Step}{Steps}

\crefname{theorem}{Thm.}{Thms.}
\Crefname{theorem}{Theorem}{Theorems}
\crefname{corollary}{Cor.}{Cors.}
\Crefname{corollary}{Corollary}{Corollaries}

%
\DeclareDocumentCommand{\set}{m g o}%
{%
    \IfNoValueTF{#3}{\left}{#3}\{#1
           \IfNoValueTF{#2}{}{\ \IfNoValueTF{#3}{\left}{#3}\vert\ \vphantom{#1}#2\IfNoValueTF{#3}{\right.}{}}
                \IfNoValueTF{#3}{\right}{#3}\}%
}

\DeclareDocumentCommand{\abs}{m o}%
{%
    \IfNoValueTF{#2}{\left}{#2}\vert#1
                \IfNoValueTF{#2}{\right}{#2}\vert%
}

\newcommand{\N}{\mathbb{N}}
\newcommand{\bigO}[1]{\mathcal{O}(#1)}

\newcommand{\shor}[1]{P_{U}({#1})}

\DeclareMathOperator{\vOp}{V}
\DeclareMathOperator{\eOp}{E}
\newcommand*{\ve}[1]{\vOp({#1})}
\newcommand*{\e}[1]{\eOp({#1})}

\newcommand{\uned}[2]{\set{{#1}, {#2}}}
\newcommand{\died}[2]{({#1}, {#2})}
\DeclareMathOperator{\degOp}{\textrm{deg}}
\DeclareMathOperator{\indegOp}{\degOp^{-}}
\DeclareMathOperator{\outdegOp}{\degOp^+}
\renewcommand{\deg}[1]{\degOp({#1})}
\renewcommand{\indeg}[1]{\indegOp({#1})}
\newcommand{\outdeg}[1]{\outdegOp({#1})}

\newcommand{\degG}[2]{\degOp_{#1}({#2})}
\newcommand{\indegG}[2]{\degOp^{-}_{#1}({#2})}
\newcommand{\outdegG}[2]{\degOp^+_{#1}({#2})}

\DeclareMathOperator{\neiOp}{N}
\DeclareMathOperator{\outNeiOp}{\neiOp^+}
\newcommand{\nei}[1]{\neiOp({#1})}
\newcommand{\cnei}[1]{\neiOp[{#1}]}
\newcommand{\outNei}[1]{\outNeiOp({#1})}
\newcommand{\coutNei}[1]{\outNeiOp[{#1}]}
\newcommand{\inNei}[1]{\neiOp^-({#1})}
\newcommand{\cinNei}[1]{\neiOp^-[{#1}]}

\newcommand{\neiG}[2]{\neiOp_{#1}({#2})}
\newcommand{\cneiG}[2]{\neiOp_{#1}[{#2}]}
\newcommand{\outNeiG}[2]{\neiOp^+_{#1}({#2})}
\newcommand{\coutNeiG}[2]{\neiOp^+_{#1}[{#2}]}
\newcommand{\inNeiG}[2]{\neiOp^{-}_{#1}({#2})}
\newcommand{\cinNeiG}[2]{\neiOp^{-}_{#1}[{#2}]}

\newcommand*{\induced}[2]{{#1}[{#2}]}

\DeclareMathOperator{\undirGOp}{\textrm{Undir}}
\newcommand*{\undirG}[1]{\undirGOp({#1})}
\newcommand*{\incG}[1]{\textrm{Inc}({#1})}

\DeclareMathOperator{\exOp}{\textrm{Ext}}
\newcommand*{\ex}[3]{\exOp_{#1}({#2}, {#3})}

\newcommand*{\exset}[2]{E_{{#1}{#2}}}
\newcommand*{\expair}[2]{({#2}, \exset{#1}{#2})}
\newcommand*{\dexset}{\exset{G}{D}}
\newcommand*{\dexpair}{\expair{G}{D}}
\newcommand*{\hexset}{\exset{G}{H}}
\newcommand*{\hexpair}{\expair{G}{H}}

\newcommand*{\exseti}[1]{\exset{G}{D_{#1}}}
\newcommand*{\expairi}[1]{\expair{G}{D_{#1}}}
\newcommand*{\hexseti}[1]{\exset{G}{H_{#1}}}
\newcommand*{\hexpairi}[1]{\expair{G}{H_{#1}}}
% \newcommand*{\expair1}{\expairi{1}}
% \newcommand*{\expair2}{\expairi{2}}
% \newcommand*{\exset1}{\exseti{1}}
% \newcommand*{\exset2}{\exseti{2}}

\DeclareMathOperator{\sccOp}{\textrm{SCC}}
\newcommand*{\scc}[1]{\sccOp({#1})}

\DeclareMathOperator{\wOp}{\omega}
\newcommand{\w}[1]{\wOp({#1})}

\DeclareMathOperator{\compOp}{\textrm{Comp}}
\newcommand{\compG}[3]{\compOp_{#1}({#2}, {#3})}
\newcommand{\comp}[1]{\compOp({#1})}

\newcommand*{\dcompG}{\compG{G}{D}{\dexset}}
\newcommand*{\dcomp}{\comp{D}}

\newcommand{\sccset}[3]{\textrm{SCS}_{#1}(#2, #3)}
\newcommand{\dsccset}{\sccset{G}{D}{\dexset}}
\newcommand{\sccequiv}[1]{\equiv_{#1}}
\newcommand{\pot}[1]{\mathcal{P}(#1)}
\newcommand{\dequiv}{\equiv_G}

\DeclareMathOperator{\sourceOp}{\mathcal{S}}
\newcommand{\source}[1]{\sourceOp({#1})}
\newcommand{\sourcee}[2]{\sourceOp({#1}, {#2})}
\DeclareMathOperator{\sinkOp}{\mathcal{T}}
\newcommand{\sink}[1]{\sinkOp({#1})}
\newcommand{\sinke}[2]{\sinkOp({#1}, {#2})}
\DeclareMathOperator{\connOp}{\mathcal{C}}
\newcommand{\conn}[1]{\connOp({#1})}
\newcommand{\conne}[2]{\connOp({#1}, {#2})}

\newcommand{\ab}{$\alpha$-bounded}

\newcommand{\pname}[1]{\textsc{#1}}
\DeclareMathOperator{\propOp}{\Pi}
\newcommand{\prop}[1]{\Pi({#1})}

\newcommand{\probname}[3]{\textsc{{#3} {#1}-Secluded {#2}}}
\newcommand{\prname}[2]{\textsc{{#1}-Secluded {#2}}}
\newcommand{\weak}{WC}
\newcommand{\strong}{SC}
\newcommand{\out}{Out}
\newcommand{\inn}{In}
\newcommand{\total}{Total}

\DeclareMathOperator{\indOp}{\alpha}
\newcommand{\indG}[1]{\indOp({#1})}

\SetKwInput{KwData}{Input}
\SetKwInput{KwResult}{Output}


\usepackage{tabularx, environ}

\makeatletter

% https://tex.stackexchange.com/a/199244/26355
\newcolumntype{\expand}{}
\long\@namedef{NC@rewrite@\string\expand}{\expandafter\NC@find}

\NewEnviron{problem}[2][]{%
  \def\problem@arg{#1}%
  \def\problem@framed{framed}%
  \def\problem@hline{\hline}%
\def\problem@tablelayout{|>{\bfseries}lX|c}%
\def\problem@title{\multicolumn{2}{|%
  >{\raisebox{-\fboxsep}}%
  p{\dimexpr\textwidth-4\fboxsep-2\arrayrulewidth\relax}%
  |}{%
      %\textcolor{stroke1}{\bfseries$\blacktriangleright$}
      {\bfseries Problem.} \textsc{#2}%
  }}%
  \bigskip\par\noindent%
  \renewcommand{\arraystretch}{1.2}%
  \begin{tabularx}{\textwidth}{\expand\problem@tablelayout}%
    \problem@hline%
    \problem@title\\[2\fboxsep]%
    \BODY%\hfill\textcolor{stroke1}{\bfseries$\blacktriangleleft$}
    \\\problem@hline%
  % \vspace{1cm}%
  \end{tabularx}%
  \medskip\par%
  \vspace{.5mm}
}
\makeatother
