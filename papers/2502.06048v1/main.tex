\documentclass[a4paper,UKenglish,cleveref, autoref, thm-restate]{lipics-v2021}
\usepackage{amsmath, amssymb, amsfonts}
\usepackage{todonotes}

\usepackage{tcolorbox}
\tcbuselibrary{skins}
\usepackage{xcolor}

\colorlet{mix}{red!50!black}

\usepackage{hyperref}
\hypersetup{colorlinks={true},linkcolor={mix},citecolor=mix}


% \usepackage{enumitem} % LIPICS doesn't want
\usepackage{algorithm2e}
\RestyleAlgo{ruled}
% \usepackage{algorithmic}
% \usepackage{algorithm}
\usepackage{booktabs}
\usepackage{xspace}
\usepackage{mathtools}
\usepackage{thmtools} 
\usepackage{thm-restate}
\declaretheorem[name=Theorem,numberwithin=section]{thm}

\newcommand{\indeg}{{\textsf{in-deg}}\xspace}

\newtheorem{reduction rule}{Reduction Rule}
\newtheorem*{reduction rule*}{Reduction Rule}

\newcommand\red[1]{\color{red}#1}
% This is where all the commands should go that you want to define yourself.

\def\zzcommand#1{\let#1\undefined\newcommand#1}

\newtheorem{fact}[theorem]{Fact}
% \newtheorem{reduction}{Reduction Rule}[section]
% \newtheorem{branching}[reduction]{Branching Rule}
\newcommand{\wi}{\textsf{W[1]}}

\newtheorem{reduction*}[theorem]{Reduction Rule}

\Crefname{reduction*}{Reduction Rule}{Reduction Rules}
\Crefname{step}{Step}{Steps}
\Crefname{enumi}{Step}{Steps}

\crefname{theorem}{Thm.}{Thms.}
\Crefname{theorem}{Theorem}{Theorems}
\crefname{corollary}{Cor.}{Cors.}
\Crefname{corollary}{Corollary}{Corollaries}

%
\DeclareDocumentCommand{\set}{m g o}%
{%
    \IfNoValueTF{#3}{\left}{#3}\{#1
           \IfNoValueTF{#2}{}{\ \IfNoValueTF{#3}{\left}{#3}\vert\ \vphantom{#1}#2\IfNoValueTF{#3}{\right.}{}}
                \IfNoValueTF{#3}{\right}{#3}\}%
}

\DeclareDocumentCommand{\abs}{m o}%
{%
    \IfNoValueTF{#2}{\left}{#2}\vert#1
                \IfNoValueTF{#2}{\right}{#2}\vert%
}

\newcommand{\N}{\mathbb{N}}
\newcommand{\bigO}[1]{\mathcal{O}(#1)}

\newcommand{\shor}[1]{P_{U}({#1})}

\DeclareMathOperator{\vOp}{V}
\DeclareMathOperator{\eOp}{E}
\newcommand*{\ve}[1]{\vOp({#1})}
\newcommand*{\e}[1]{\eOp({#1})}

\newcommand{\uned}[2]{\set{{#1}, {#2}}}
\newcommand{\died}[2]{({#1}, {#2})}
\DeclareMathOperator{\degOp}{\textrm{deg}}
\DeclareMathOperator{\indegOp}{\degOp^{-}}
\DeclareMathOperator{\outdegOp}{\degOp^+}
\renewcommand{\deg}[1]{\degOp({#1})}
\renewcommand{\indeg}[1]{\indegOp({#1})}
\newcommand{\outdeg}[1]{\outdegOp({#1})}

\newcommand{\degG}[2]{\degOp_{#1}({#2})}
\newcommand{\indegG}[2]{\degOp^{-}_{#1}({#2})}
\newcommand{\outdegG}[2]{\degOp^+_{#1}({#2})}

\DeclareMathOperator{\neiOp}{N}
\DeclareMathOperator{\outNeiOp}{\neiOp^+}
\newcommand{\nei}[1]{\neiOp({#1})}
\newcommand{\cnei}[1]{\neiOp[{#1}]}
\newcommand{\outNei}[1]{\outNeiOp({#1})}
\newcommand{\coutNei}[1]{\outNeiOp[{#1}]}
\newcommand{\inNei}[1]{\neiOp^-({#1})}
\newcommand{\cinNei}[1]{\neiOp^-[{#1}]}

\newcommand{\neiG}[2]{\neiOp_{#1}({#2})}
\newcommand{\cneiG}[2]{\neiOp_{#1}[{#2}]}
\newcommand{\outNeiG}[2]{\neiOp^+_{#1}({#2})}
\newcommand{\coutNeiG}[2]{\neiOp^+_{#1}[{#2}]}
\newcommand{\inNeiG}[2]{\neiOp^{-}_{#1}({#2})}
\newcommand{\cinNeiG}[2]{\neiOp^{-}_{#1}[{#2}]}

\newcommand*{\induced}[2]{{#1}[{#2}]}

\DeclareMathOperator{\undirGOp}{\textrm{Undir}}
\newcommand*{\undirG}[1]{\undirGOp({#1})}
\newcommand*{\incG}[1]{\textrm{Inc}({#1})}

\DeclareMathOperator{\exOp}{\textrm{Ext}}
\newcommand*{\ex}[3]{\exOp_{#1}({#2}, {#3})}

\newcommand*{\exset}[2]{E_{{#1}{#2}}}
\newcommand*{\expair}[2]{({#2}, \exset{#1}{#2})}
\newcommand*{\dexset}{\exset{G}{D}}
\newcommand*{\dexpair}{\expair{G}{D}}
\newcommand*{\hexset}{\exset{G}{H}}
\newcommand*{\hexpair}{\expair{G}{H}}

\newcommand*{\exseti}[1]{\exset{G}{D_{#1}}}
\newcommand*{\expairi}[1]{\expair{G}{D_{#1}}}
\newcommand*{\hexseti}[1]{\exset{G}{H_{#1}}}
\newcommand*{\hexpairi}[1]{\expair{G}{H_{#1}}}
% \newcommand*{\expair1}{\expairi{1}}
% \newcommand*{\expair2}{\expairi{2}}
% \newcommand*{\exset1}{\exseti{1}}
% \newcommand*{\exset2}{\exseti{2}}

\DeclareMathOperator{\sccOp}{\textrm{SCC}}
\newcommand*{\scc}[1]{\sccOp({#1})}

\DeclareMathOperator{\wOp}{\omega}
\newcommand{\w}[1]{\wOp({#1})}

\DeclareMathOperator{\compOp}{\textrm{Comp}}
\newcommand{\compG}[3]{\compOp_{#1}({#2}, {#3})}
\newcommand{\comp}[1]{\compOp({#1})}

\newcommand*{\dcompG}{\compG{G}{D}{\dexset}}
\newcommand*{\dcomp}{\comp{D}}

\newcommand{\sccset}[3]{\textrm{SCS}_{#1}(#2, #3)}
\newcommand{\dsccset}{\sccset{G}{D}{\dexset}}
\newcommand{\sccequiv}[1]{\equiv_{#1}}
\newcommand{\pot}[1]{\mathcal{P}(#1)}
\newcommand{\dequiv}{\equiv_G}

\DeclareMathOperator{\sourceOp}{\mathcal{S}}
\newcommand{\source}[1]{\sourceOp({#1})}
\newcommand{\sourcee}[2]{\sourceOp({#1}, {#2})}
\DeclareMathOperator{\sinkOp}{\mathcal{T}}
\newcommand{\sink}[1]{\sinkOp({#1})}
\newcommand{\sinke}[2]{\sinkOp({#1}, {#2})}
\DeclareMathOperator{\connOp}{\mathcal{C}}
\newcommand{\conn}[1]{\connOp({#1})}
\newcommand{\conne}[2]{\connOp({#1}, {#2})}

\newcommand{\ab}{$\alpha$-bounded}

\newcommand{\pname}[1]{\textsc{#1}}
\DeclareMathOperator{\propOp}{\Pi}
\newcommand{\prop}[1]{\Pi({#1})}

\newcommand{\probname}[3]{\textsc{{#3} {#1}-Secluded {#2}}}
\newcommand{\prname}[2]{\textsc{{#1}-Secluded {#2}}}
\newcommand{\weak}{WC}
\newcommand{\strong}{SC}
\newcommand{\out}{Out}
\newcommand{\inn}{In}
\newcommand{\total}{Total}

\DeclareMathOperator{\indOp}{\alpha}
\newcommand{\indG}[1]{\indOp({#1})}

\SetKwInput{KwData}{Input}
\SetKwInput{KwResult}{Output}


\usepackage{tabularx, environ}

\makeatletter

% https://tex.stackexchange.com/a/199244/26355
\newcolumntype{\expand}{}
\long\@namedef{NC@rewrite@\string\expand}{\expandafter\NC@find}

\NewEnviron{problem}[2][]{%
  \def\problem@arg{#1}%
  \def\problem@framed{framed}%
  \def\problem@hline{\hline}%
\def\problem@tablelayout{|>{\bfseries}lX|c}%
\def\problem@title{\multicolumn{2}{|%
  >{\raisebox{-\fboxsep}}%
  p{\dimexpr\textwidth-4\fboxsep-2\arrayrulewidth\relax}%
  |}{%
      %\textcolor{stroke1}{\bfseries$\blacktriangleright$}
      {\bfseries Problem.} \textsc{#2}%
  }}%
  \bigskip\par\noindent%
  \renewcommand{\arraystretch}{1.2}%
  \begin{tabularx}{\textwidth}{\expand\problem@tablelayout}%
    \problem@hline%
    \problem@title\\[2\fboxsep]%
    \BODY%\hfill\textcolor{stroke1}{\bfseries$\blacktriangleleft$}
    \\\problem@hline%
  % \vspace{1cm}%
  \end{tabularx}%
  \medskip\par%
  \vspace{.5mm}
}
\makeatother


%\graphicspath{{./graphics/}}%helpful if your graphic files are in another directory

\bibliographystyle{plainurl}% the mandatory bibstyle

\title{A Parameterized Study of Secluded Structures in Directed Graphs} %TODO Please add
%\title{Seclusion in Directed Graphs: A Parameterized Study}

\titlerunning{A Parameterized Study of Secluded Structures in Directed Graphs} %TODO optional, please use if title is longer than one line
\author{Nadym Mallek}{Hasso Plattner Institute, Potsdam, Germany}{nadym.mallek@hpi.de}{}{}
\author{Jonas Schmidt}{Bocconi University, Milan, Italy}{jonas.schmidt2@phd.unibocconi.it}{https://orcid.org/0000-0002-1115-3868}{}
\author{Shaily Verma}{Hasso Plattner Institute, Potsdam, Germany}{shaily.verma@hpi.de}{}{}
% \author{Name}{Institute }{email}{}{}
%TODO mandatory, please use full name; only 1 author per \author macro; first two parameters are mandatory, other parameters can be empty. Please provide at least the name of the affiliation and the country. The full address is optional. Use additional curly braces to indicate the correct name splitting when the last name consists of multiple name parts.


\authorrunning{N. Mallek, J. Schmidt, and S. Verma } %TODO mandatory. First: Use abbreviated first/middle names. Second (only in severe cases): Use first author plus 'et al.'

\Copyright{Nadym Mallek, Jonas Schmidt, and Shaily Verma} %TODO mandatory, please use full first names. LIPIcs license is "CC-BY";  http://creativecommons.org/licenses/by/3.0/
\hideLIPIcs

\begin{CCSXML}
<ccs2012>
   <concept>
       <concept_id>10003752.10003809.10003716</concept_id>
       <concept_desc>Theory of computation~Graph algorithms analysis</concept_desc>
       <concept_significance>500</concept_significance>
   </concept>
   <concept>
       <concept_id>10003752.10003809.10010170</concept_id>
       <concept_desc>Theory of computation~Parameterized complexity</concept_desc>
       <concept_significance>500</concept_significance>
   </concept>
</ccs2012>
\end{CCSXML}

\ccsdesc[500]{Theory of computation~Graph algorithms analysis}
\ccsdesc[500]{Theory of computation~Parameterized complexity}
 %TODO mandatory: Please choose ACM 2012 classifications from https://dl.acm.org/ccs/ccs_flat.cfm 

\keywords{Secluded Subgraph, Parametrized Complexity, Directed Graphs, Strong Connectivity} %TODO mandatory; please add comma-separated list of keywords

%\category{} %optional, e.g. invited paper

% \relatedversion{} %optional, e.g. full version hosted on arXiv, HAL, or other respository/website
%\relatedversiondetails[linktext={opt. text shown instead of the URL}, cite=DBLP:books/mk/GrayR93]{Classification (e.g. Full Version, Extended Version, Previous Version}{URL to related version} %linktext and cite are optional

%\supplement{}%optional, e.g. related research data, source code, ... hosted on a repository like zenodo, figshare, GitHub, ...
%\supplementdetails[linktext={opt. text shown instead of the URL}, cite=DBLP:books/mk/GrayR93, subcategory={Description, Subcategory}, swhid={Software Heritage Identifier}]{General Classification (e.g. Software, Dataset, Model, ...)}{URL to related version} %linktext, cite, and subcategory are optional

%\funding{(Optional) general funding statement \dots}%optional, to capture a funding statement, which applies to all authors. Please enter author specific funding statements as fifth argument of the \author macro.

%\acknowledgements{I want to thank \dots}%optional

\nolinenumbers %uncomment to disable line numbering



%Editor-only macros:: begin (do not touch as author)%%%%%%%%%%%%%%%%%%%%%%%%%%%%%%%%%%
% \EventEditors{John Q. Open and Joan R. Access}
% \EventNoEds{2}
% \EventLongTitle{42nd Conference on Very Important Topics (CVIT 2016)}
% \EventShortTitle{CVIT 2016}
% \EventAcronym{CVIT}
% \EventYear{2016}
% \EventDate{December 24--27, 2016}
% \EventLocation{Little Whinging, United Kingdom}
% \EventLogo{}
% \SeriesVolume{42}
% \ArticleNo{23}
%%%%%%%%%%%%%%%%%%%%%%%%%%%%%%%%%%%%%%%%%%%%%%%%%%%%%%

\newcommand{\jonas}[1]{\todo[color=purple!70]{J: {#1}}}
\newcommand{\nadym}[1]{\todo[color=green]{N: {#1}}}


%short-long-version switch
\newif\iflong
\newif\ifshort

% comment the below line out for short version
 \longtrue

\iflong
\else
\shorttrue
\fi


\begin{document}

\maketitle

\begin{abstract}
Retrieval-Augmented Generation (RAG) is often used with Large Language Models (LLMs) to infuse domain knowledge or user-specific information. In RAG, given a user query, a retriever extracts chunks of relevant text from a knowledge base. These chunks are sent to an LLM as part of the input prompt. Typically, any given chunk is repeatedly retrieved across user questions. However, currently, for every question, attention-layers in LLMs fully compute the key values (KVs) repeatedly for the input chunks, as state-of-the-art methods cannot reuse KV-caches when chunks appear at arbitrary locations with arbitrary contexts. Naive reuse leads to output quality degradation.  This leads to potentially redundant computations on expensive GPUs and increases latency. In this work, we propose \sys, a system for managing and reusing precomputed KVs corresponding to the text chunks (we call \textit{chunk-caches}) in RAG-based systems. We present how to identify \hl{\textit{chunk-caches} that are reusable}, how to efficiently perform a small fraction of recomputation to \textit{fix} the cache to maintain output quality, and how to efficiently store and evict \textit{chunk-caches} in the hardware for maximizing reuse while masking any overheads. With real production workloads as well as synthetic datasets, we show that \sys reduces redundant computation by \textbf{51\%} over SOTA prefix-caching and \textbf{75\%} over full recomputation.
\hl{Additionally, with continuous batching on a real production workload, we get a \textbf{1.6$\times$} speedup in throughput and a \textbf{2$\times$} reduction in end-to-end response latency over prefix-caching while maintaining quality, for both the \llama-3-8B and \llama-3-70B models. 
}
\end{abstract}






\section{Introduction}
Finding substructures in graphs that satisfy specific properties is a ubiquitous problem. This general class of problems covers many classical graph problems such as finding maximum cliques, Steiner trees, or even shortest paths. Another compelling property to look for in a substructure is its isolation from the remaining graph. This motivated Chechik et al.~\cite{chechik2017secluded}, to introduce the concept of \emph{secluded subgraphs}. 
Formally, in the \textsc{Secluded $\propOp$-Subgraph} problem, given an undirected graph $G$, the goal is to find a maximum size subset of vertices $S \subseteq \ve{G}$ such that the subgraph induced on $S$ fulfills a property $\propOp$ and has a neighborhood $\abs{\nei{S}} \le k$ where $k$ is a natural number.

This problem has been studied extensively for various properties $\propOp$ such as paths~\cite{van2020parameterized,luckow2020computational,fomin2017parameterized,chechik2017secluded}, Steiner trees~\cite{chechik2017secluded,fomin2017parameterized}, induced trees~\cite{donkers2023finding,golovach2020finding}, subgraphs free of forbidden induced subgraphs~\cite{golovach2020finding,jansen2023single}, and more~\cite{bevern207finding}.
Most of these studies focus on the parameterized setting, due to the strong relation to vertex deletion and separator problems, which are foundational in parameterized complexity. Our problem fits in that category since the neighborhood can be considered a $(S,V \setminus S)$-separator. 

% The study of such a problem has, for instance, improved algorithms for problems such as finding maximum cliques and deletion to scattered graph classes~\cite{jansen2023single}.\jonas{We need more examples and refs. The finding max clique claim is not really true } \todo{it seems an incomplete sentence}

% Multiple papers studied the need to find a subgraph that not only satisfies a given property but also minimizes exposure to the rest of the graph~\cite{huffner2009isolation,chechik2017secluded,bevern207finding}. With this constraint, the problem is called \textsc{Secluded Subgraph} problem and was introduced by \cite{chechik2017secluded}. This addition becomes especially relevant in applications like privacy-aware data analysis, detecting isolated communities, and designing robust subsystems. 

% cite me: Secluded Path~\cite{van2020parameterized,luckow2020computational,fomin2017parameterized,chechik2017secluded}
%secluded Steiner trees~\cite{chechik2017secluded,fomin2017parameterized}
% tree: \cite{donkers2023finding,golovach2020finding}
% f-free: \cite{golovach2020finding,jansen2023single}
% secluded IS and many others: \cite{bevern207finding}

%In the design and analysis of networks, finding substructures that satisfy specific properties is a significant problem. 
% This problem was introduced in 1980 by Lewis and Yannakakis~\cite{lewis_node-deletion_1980} as the \textsc{$\Pi$-Subgraph} problem.
%This very general class of problems including the shortest path problem, the steiner tree problem, or the maximum clique problem can be categorized as the \textsc{$\Pi$-Subgraph} problem.
%Later, multiple papers studied the need to find a subgraph that not only satisfies a given property but also minimizes exposure to the rest of the graph~\cite{huffner2009isolation}. \todo{S: need more citations as mentioned multiple papers}
%This addition becomes especially relevant in applications like privacy-aware data analysis, detecting isolated communities, and designing robust subsystems.
%This problem is then called the \textsc{Secluded Subgraph} problem, and it was introduced by~\cite{bevern207finding}, a natural and essential extension of the well-known \textsc{$\Pi$-Subgraph} problem.
%\todo{S: Add some examples of $\Pi$-subgraph problems with citations}
%\jonas{Like the beginning but I think the end should be more concrete in case you don't know about both mentioned problems (i think it's easy to understand the meaning of Pi-Subgraph, but it's not often discussed under this name?}


While the undirected \textsc{Secluded $\Pi$-Subgraph} problem has been explored and widely understood in prior work~\cite{jansen2023single,golovach2020finding,donkers2023finding}, the directed variant has not yet been studied, although it is a natural generalization and was mentioned as an interesting direction by Jansen et al.~\cite{jansen2023single}. Directed graphs naturally model real-world systems with asymmetric interactions, such as social networks with unidirectional follow mechanisms or information flow in communication systems. Furthermore, problems such as \textsc{Directed Feedback Vertex Set}, \textsc{Directed Multicut}, and \textsc{Directed Multiway Cut} underline how directedness can make a fascinating and insightful difference when it comes to parametrized complexity ~\cite{chen2008fixed,hatzel2023fixed,multicut4,chitnis2013fixed}. These problems and their results provide a ground for studying directed secluded subgraph problems, motivating the need to investigate them systematically.

In this paper, we introduce and study three natural directed variants of the \textsc{Secluded $\Pi$-Subgraph} problem, namely \textsc{Out-Secluded $\Pi$-Subgraph}, \textsc{In-Secluded $\Pi$-Subgraph}, and \textsc{Total-Secluded $\Pi$-Subgraph}. These problems aim to minimize either the out-neighborhood of $S$, its in-neighborhood, or the union of both. The out/in-neighborhood of a set $S$ is the set of vertices reachable from $S$ via an outgoing/incoming edge. %These three kinds of neighborhoods were suggested as extensions by~\cite{jansen2023single}, and we also believe that they are the most commonly encountered. 
These problems corresponding to different types of neighborhoods can be encountered in real-life networks. For example, in privacy-aware social network analysis, one might aim to identify a community with minimal external exposure. Similarly, robust substructures with limited connectivity to vulnerable components are critical in cybersecurity. 
% Additionally, total-seclusion models are relevant for studying interactions in ecosystems or supply chains.
These real-world motivations further emphasize the need to explore and formalize directed variants of the \textsc{Secluded $\Pi$-Subgraph} problem. 

\begin{table}[t]
    \centering
    \begin{tabular}{lll}
        \toprule
        \textbf{Problem} & \textbf{In- / Out-Secluded} & \textbf{Total-Secluded}\\
        \midrule
        \textsc{WC $\mathcal{F}$-Free Subgraph} & W[1]-hard \hfill \cref{thm:f-free-hard-always} & FPT \hfill\cite{jansen2023single}\\
        \textsc{WC DAG} & W[1]-hard \hfill \cref{cor:dag_out_k+t} & ?\\
        \textsc{$\alpha$-Bounded Subgraph} & FPT \hfill \cref{thm:alpha_bounded_fpt}& FPT \hfill \cref{thm:alpha_bounded_total}\\
        % \textsc{$\propOp$-Subgraph in $\alpha$-bounded graphs} & FPT \hfill \cref{thm:alpha_eff} & FPT \hfill \cref{thm:alpha_eff_tot}\\
        \textsc{Strongly Connected Subgraph} & ? & FPT \hfill \cref{cor:scc_algo}\\
        \midrule
        \textsc{Clique} & \multicolumn{2}{l}{FPT in time $1.6181^kn^{\bigO{1}}$ \hfill \cref{thm:clique_better}}\\
        \bottomrule
    \end{tabular}
    \caption{Our main results for directed and undirected problems. WC stands for weakly connected, $\propOp$ is an arbitrary, polynomial-time verifiable property. All FPT results are with respect to parameter $k$ and all hardness results are with respect to parameter $k+w$. The problems corresponding to columns marked with $(?)$ are open. 
    \label{tab:our_results}}
\end{table}

For any property $\propOp$, we formulate our general problem as follows. Notice that in-secluded and out-secluded are equivalent for all properties that are invariant under the transposition of the edges. For this reason, we mostly focus on total-neighborhood and out-neighborhood.
\begin{tcolorbox}[enhanced,title={\color{black} {\textsc{X-Secluded $\propOp$-Subgraph} \quad ($\text{X} \in \set{\text{In}, \text{Out}, \text{Total}}$)}}, colback=white, boxrule=0.4pt,
	attach boxed title to top left={xshift=.3cm, yshift*=-2.5mm},
	boxed title style={size=small,frame hidden,colback=white}]
	\textbf{Parameter:} An integer $k \in \N$ 

	\textbf{Input:}  A directed graph $G$ with vertex weights $\omega \colon V \to \N$ and an integer $w \in \N$

	\textbf{Output:} An X-secluded set $S \subseteq \ve{G}$ of weight $\omega(S) \ge w$ that satisfies $\propOp$, or report that none exists. 
\end{tcolorbox} 
%Second paragraph: define Secluded subgraph problem ; give its connections with general theory
%Third paragraph: connection to directed variants (ref of some paper that mentions it as a possible direction)
%Fourth paragraph applications



\subsection*{Our Contribution}
In this subsection, we present the results we obtained for the problem. See \Cref{tab:our_results} for an overview of the results we obtained in this paper. 
%\ifshort Due to space constraints, we omit the proofs of statements marked with ($\star$). The missing proofs and details can be found in the full version. \fi

\subparagraph*{Strongly Connected Subgraph.} We show that the \textsc{Total-Secluded Strongly Connected Subgraph} problem is fixed-parameter tractable when parameterized with $k$. Precisely, we prove the following result:
% Connectivity is an important factor in establishing tractability in undirected graphs. Hence, we decided to also analyze it in the case of directed graphs.
%Additionally, connectivity-preserving subgraphs such as paths~\cite{van2020parameterized,luckow2020computational,fomin2017parameterized,chechik2017secluded} and Steiner trees~\cite{chechik2017secluded,fomin2017parameterized} are among the most-studied secluded subgraph problems. 

\begin{restatable}[]{theorem}{test}
\label{cor:scc_algo}
  \textsc{Total-Secluded Strongly Connected Subgraph} is solvable in time $2^{2^{2^{\bigO{k^2}}}}n^{\bigO{1}}$.
\end{restatable}


% In \Cref{sec:scc}, we develop an FPT-algorithm with running time $2^{2^{2^{\bigO{k^2}}}}n^{\bigO{1}}$ for \textsc{Total-Secluded Strongly Connected Subgraph} (\Cref{cor:scc_algo}). This is done using recursive understanding, a technique introduced by~\cite{grohe2011finding} and used successfully for various parameterized problems~\cite{chitnis2016designing,golovach2020finding,cygan2014minimum,lokshtanov2018reducing}.
% The following paragraphs describe the intuition behind recursive understanding algorithms for secluded subgraph problems, following roughly the structure in~\cite{golovach2020finding}.
% A more concrete and formal description of the algorithm structure for \scs{} can be found in \Cref{alg:rec_und_scc}.




% High level intuition for recursive understanding
We design our FPT algorithm for \textsc{Total-Secluded Strongly Connected Subgraph} problem using recursive understanding, a technique introduced by~\cite{grohe2011finding} and recently used successfully for various parameterized problems~\cite{chitnis2016designing,golovach2020finding,cygan2014minimum,lokshtanov2018reducing}.
We visualize the overall structure of the algorithm in \Cref{fig:recursive_calls}.



\begin{figure}[t]
  \centering
  \includegraphics[width=0.6\textwidth,page=2]{figures/recursion}
  \caption{
  An illustration of the general recursive understanding algorithm used in \Cref{sec:scc}.
  There are two recursive calls in total, highlighted with dashed arrows.
  As defined later, only vertices inside $B$ are allowed to be in the neighborhood of a solution.
  $W$ is chosen to be the side of the separation with a smaller intersection with $T$.
  }
  \label{fig:recursive_calls}
\end{figure}


On a high level, recursive understanding algorithms work by first finding a small balanced separator of the underlying undirected graph. If no suitable balanced separator exists, the graph must be highly connected, which makes the problem simpler to solve. In the other case, we carefully reduce and simplify one side of the separator while making sure to keep an equivalent set of solutions in the whole graph. By choosing parameters to subroutines in the right way, this process reduces one side of the separator enough to invalidate the balance of the separator. Therefore, we have made progress and can iterate with another balanced separator or reach the base case.

% Why is recursive understanding applicable to secluded?
In our case, looking for a separator of size at most $k$ makes the framework applicable. Crucially, this is because in any secluded subgraph $G[S]$, where $S \subseteq \ve{G}$, the neighborhood $\nei{S}$ acts as a separator between $S$ and $\ve{G} \setminus \cnei{S}$. Therefore, if no balanced separator of size at most $k$ exists, we can deduce that either $S$ itself or $\ve{G} \setminus S$ must have a small size. This observation makes the problem significantly easier to solve in this case, using the color coding technique developed in~\cite{chitnis2016designing}.

% details and boundary complementations
In the other case, we can separate our graph into two balanced parts, $U$ and $W$, with a separator $P$ of size $\abs{P} \le k$. Now, our goal is to solve the same problem recursively for one of the sides, say $W$ and replace that side with an equivalent graph of bounded size that only contains all necessary solutions.
However, notice that finding subsets of solutions is not the same as finding solutions; the solution $S$ for the whole graph could heavily depend on including some vertices in $U$. That being said, the different options for this influence are limited. At most $2^k$ different subsets $X$ of $P$ could be part of the solution. For any such $X$, the solution can only interact across $P$ in a limited number of ways. For finding strongly connected subgraphs, we have to consider for which pair $(x_1, x_2) \in X \times X$ there already is a $x_1$-$x_2$-path in the $U$-part of the complete solution $S$. 
This allows us to construct a new instance for every such possibility by encoding $X$ and the existing paths into $W$. These auxiliary instances are called \emph{boundary complementations}. We visualize the idea of this construction later in \Cref{fig:scc_bc}.

% hint at extension and contribution
Fundamentally, we prove that an optimal solution for the original graph exists, that coincides in $W$ with an optimal solution to the boundary complementation graph in which $U$ is replaced. Hence, we restrict the space of solutions to only those whose neighborhood in $W$ coincides with the neighborhood of an optimal solution to some boundary complementation. The restricted instance consists of a bounded-size set $B$ of vertices that could be part of $\nei{S}$ and components in $W \setminus B$ that can only be included in $S$ completely or not at all. We introduce graph \emph{extensions} to formalize when exactly these components play the same role in a strongly connected subgraph. Equivalent extensions can then be merged and compressed into equivalent extensions of bounded size.
In total, this guarantees that $W$ is shrunk enough to invalidate the previous balanced separator, and we can restart the process.

\subparagraph*{$\alpha$-Bounded Subgraph \& Clique.}

In the undirected setting, the \textsc{Secluded Clique} problem is natural and has been studied specifically. There is an FPT-algorithm running in time $2^{\bigO{k \log k}}n^{\bigO{1}}$ through contracting twins~\cite{golovach2020finding}.
The previous best algorithm however uses the general result for finding secluded $\mathcal{F}$-free subgraphs via important separators in time $2^{\bigO{k}}n^{\bigO{1}}$~\cite{jansen2023single}. By the use of important separators, they require time at least $4^kn^{\bigO{1}}$. 
This property is naturally generalizable to directed graphs via tournament graphs. We go one step further.

The independence number of an undirected or directed graph $G$ is the size of the maximum independent set in $G$ (or its underlying undirected graph).
If $G$ has independence number at most $\alpha$, we also call it \emph{$\alpha$-bounded}. This concept has been used to leverage parameterized results from the simpler tournament graphs to the larger graph class of $\alpha$-bounded graphs~\cite{sahu2023kernelization,fradkin2015edge,misra2023sub}. We prove the following results:
% Finding a secluded alpha-bounded subgraph within a larger graph answers two important questions, whether there is a large enough dominance-like hierarchy between vertices and how connected is this structure to the rest of the graph. This is particularly useful in strategic network design for instance, where information or influence needs to be mostly contained or not reach a structured group. 

\begin{restatable}[]{theorem}{restateab}
\label{thm:alpha_bounded_fpt}
  \textsc{Out-Secluded $\alpha$-Bounded Subgraph} is solvable in time $(2\alpha + 2)^kn^{\alpha+\bigO{1}}$.
\end{restatable} 

\begin{restatable}[]{theorem}{restateabtotal}
\label{thm:alpha_bounded_total}
  \textsc{Total-Secluded $\alpha$-Bounded Subgraph} is solvable in time $(\alpha + 1)^kn^{\alpha+\bigO{1}}$.
\end{restatable} 

We achieve the goal via a branching algorithm, solving \textsc{Secluded $\alpha$-Bounded Subgraph} for all neighborhood definitions in FPT time (\Cref{thm:alpha_bounded_fpt,thm:alpha_bounded_total}).
Our algorithm initially picks a vertex subset $U \subseteq \ve{G}$ and looks only for solutions in the two-hop neighborhood of $U$. A structural property of $\alpha$-bounded graphs guarantees that any optimal solution is found in this way. 
On a high level, the remaining algorithm depends on two branching strategies. First, we branch on forbidden structures in the two-hop neighborhood of $U$, to ensure that it becomes $\alpha$-bounded. Second, we branch on farther away vertices to reach a secluded set. 

The ideas behind the previous algorithm can in turn be used for the simpler undirected \textsc{Secluded Clique} problem.
By a closer analysis of these two high-level rules, we arrive at a branching vector of $(1,2)$ for \textsc{Secluded Clique}. This results in the following runtime, a drastic improvement on the previous barrier of $4^kn^{\bigO{1}}$.

\begin{restatable}[]{theorem}{restateclique}
\label{thm:clique_better}
  \textsc{Secluded Clique} is solvable in time $1.6181^k n^{\bigO{1}}$.
\end{restatable}

\ifshort Due to space constraints, the details and proofs of \Cref{thm:alpha_bounded_fpt,thm:alpha_bounded_total,thm:clique_better} can be found in the full version.
\else The details and proofs of \Cref{thm:alpha_bounded_fpt,thm:alpha_bounded_total,thm:clique_better} can be found in \Cref{chap:tour}.
\fi

% \iflong In \Cref{sec:clique}, we
% \else We \fi
% develop a new branching algorithm that solves it in time $1.6181^kn^{\bigO{1}}$ (\Cref{thm:clique_better}). 
% Our algorithm initially picks one vertex $u$ and looks only for cliques including $u$, guaranteeing that any solution can only be a subset of $\cnei{u}$. 
% On a high level, the remaining algorithm depends on two branching strategies. Firstly, we branch on forbidden structures in $\cnei{u}$, to ensure that $\cnei{u}$ becomes a clique. Secondly, we decrease the size of the neighborhood of $\cnei{u}$, until $\cnei{u}$ becomes a secluded clique. 
% By differentiating more base cases in these two rules, we arrive at a branching vector of $(1,2)$ for the improved runtime. 

\subparagraph*{$\mathcal{F}$-Free Subgraph.}
In the widely-studied \textsc{Secluded $\mathcal{F}$-Free Subgraph} problem, we are given an undirected graph $G$ and a finite family of graphs $\mathcal{F}$, and we are asked to find an induced secluded subgraph of $G$ that does not contain any graph in $\mathcal{F}$ as an induced subgraph. In undirected graphs, this problem becomes FPT (with parameter $k$) using recursive understanding~\cite{golovach2020finding} or branching on important separators~\cite{jansen2023single} when we restrict it to connected solutions. We study the directed version of the problem and surprisingly, it turns out to be W[1]-hard for almost all forbidden graph families $\mathcal{F}$ even with respect to the parameter $k+w$. Precisely, we prove the following theorem.

\begin{restatable}[]{theorem}{restateffree}
\label{thm:f-free-hard-always}
    Let $\mathcal{F}$ be a non-empty set of directed graphs such that no $F \in \mathcal{F}$ is a subgraph of an inward star. Then, \textsc{Out-Secluded $\mathcal{F}$-Free Weakly Connected Subgraph} is W[1]-hard with respect to the parameter $k+w$ for unit weights.
\end{restatable}

%In \Cref{chap:hardness}, we show that the \textsc{Out-Secluded $\mathcal{F}$-Free Weakly Connected Subgraph} problem is W[1]-hard with parameter $k+w$ for unit weights in directed graphs for almost all families~$\mathcal{F}$ (\Cref{thm:f-free-hard-always}), a surprising difference to the undirected setting.
We establish an almost complete dichotomy that highlights the few cases of families~$\mathcal{F}$ for which the problem remains tractable. One of these exceptions is if~$\mathcal{F}$ contains an independent set of any size, where we employ our algorithm for \textsc{Out-Secluded $\alpha$-Bounded Subgraph}. \Cref{thm:f-free-hard-always} also imply the following result for the directed variant of \textsc{Secluded Tree} problem.

\begin{restatable}[]{corollary}{restatedag}
\label{cor:dag_out_k+t}
    \textsc{Out-Secluded Weakly Connected DAG} is W[1]-hard with parameter $k + w$ for unit weights.
\end{restatable}

% \iflong
% Finally, we develop more efficient branching algorithms for the case of $\alpha$-bounded base graphs in \Cref{sec:ffree_ab}.
% \fi

\subparagraph*{Organization.}

We consider \textsc{Total-Secluded Strongly Connected Subgraph} and prove \Cref{cor:scc_algo} in \Cref{sec:scc}.
\iflong In \Cref{chap:tour}, we give the algorithms for \textsc{Secluded $\alpha$-Bounded Subgraph} and \textsc{Secluded Clique} and proofs of \Cref{thm:alpha_bounded_fpt,thm:alpha_bounded_total,thm:clique_better}.
\fi
The hardness result about \textsc{Out-Secluded $\mathcal{F}$ Weakly Connected Subgraph} in \Cref{thm:f-free-hard-always} is proved in \Cref{chap:hardness}.
\ifshort Due to space constraints, the algorithms for \textsc{Secluded $\alpha$-Bounded Subgraph} and \textsc{Secluded Clique} and proofs of \Cref{thm:alpha_bounded_fpt,thm:alpha_bounded_total,thm:clique_better} can be found in the full version. Proofs of statements marked with ($\star$) are also deferred to the full version.
\fi

\subparagraph*{Notation.}

Let $G$ be a directed graph. For a vertex $v \in \ve{G}$, we denote the \emph{out-neighborhood} by $\outNei{v} = \set{u}{(v,u) \in \e{G}}$ and the \emph{in-neighborhood} by $\inNei{v} = \set{u}{(u,v) \in \e{G}}$.
The \emph{total-neighborhood} is defined as $\nei{v} = \outNei{v} \cup \inNei{v}$.
We use the same notation for sets of vertices $S \subseteq \ve{G}$ as $\outNei{S} = \bigcup_{v \in S} \outNei{v} \setminus S$. 
Furthermore, for all definitions, we also consider their \emph{closed} version that includes the vertex or vertex set itself, denoted by $\coutNei{v} = \outNei{v} \cup \set{v}$.
For a vertex set $S \subseteq \ve{G}$, we write $\induced{G}{S}$ for the subgraph induced by $S$ or $G - S$ for the subgraph induced by $\ve{G} \setminus S$. We also use $G - v$ instead of $G - \set{v}$.

When we refer to a component of a directed graph, we mean a component of the underlying undirected graph, that is, a maximal set that induces a weakly connected subgraph. In contrast, a strongly connected component refers to a maximal set that induces a strongly connected subgraph.
For standard parameterized definitions, we refer to~\cite{cygan2015parameterized}.

\zzcommand{\scs}{\textsc{TSSCS}}

\begin{comment}
In the following subsections, we describe our algorithm in more detail. We first introduce a generalized problem in \Cref{sec:scc_bc}, that receives a vertex set of \emph{boundary terminals} as part of the input and asks to find a solution for every boundary complementation of these vertices. In practice, the boundary terminals will be our separator, or the union of all previous separators in later iterations. In \Cref{sec:unbreak}, we explain how to solve the base case where no balanced separator exists. Next, \Cref{sec:scc_extensions} introduces \emph{extensions} for graphs and an equivalence relation on them as a useful model in which we phrase our reduction rules and the full algorithm in \Cref{sec:solving_scc}. 
\end{comment}

\begin{comment}
\subparagraph*{$d$-Edge-Connected Subgraph}
Finally, we develop another recursive understanding algorithm for the undirected \textsc{Secluded $d$-Edge-Connected Subgraph} problem. Not only is this another property based around connectivity, but 2-edge-connectivity also shares some similarities with strong connectivity in directed graphs because it requires two specific paths in the underlying undirected graph. \jonas{Merge this paragraph with the previous one? Maybe it's strange but since this is only in the long version it might make sense to only hint at this.}
\end{comment}

% \subsection*{Notation}
% \section{Notation and Preliminaries}\label{sec:prelims}
This section fixes the notation and relevant notions for fair division of goods; the notation specific to division of chores is relegated to Section \ref{sec:chores}. 
 
\paragraph{Fair Division Instances.} A {fair division instance} is given by a tuple $\langle [n], [m], \{v_i\}_{i=1}^n \rangle$, where $[n]=\{1,2,.\dots,n\}$ is the set of $n\in\mathbb{Z}_+$ agents, $[m]=\{1,2, \dots, m\}$ the set of $m\in \mathbb{Z}_+$ indivisible goods, and for each agent $i\in[n]$, the set function $v_i: 2^{[m]} \to \mathbb{R}_+$ denotes the valuation of agent $i$ over subsets of goods. Specifically, $v_i(S) \in \mathbb{R}_+$ denotes the value that agent $i$ derives from the subset $S \subseteq [m]$ of goods. For subsets $S \subseteq [m]$ and $g \in [m]$, we will write $S + g$ to denote the union $S \cup \{ g\}$. 

A valuation $v_i$ is said to be monotone if the inclusion of goods into any subset does not decrease its value, under $v_i$, i.e., $v_i(S)\leq v_i(T)$ for every pair of subsets $S \subseteq T \subseteq[m]$. We will assume throughout that the agents' valuations are monotone and normalized: $v_i(\emptyset)=0$ for all agents $i$. 

We will additionally consider instances with identically ordered valuations. Here, we have an indexing of the $m$ goods, $\{g_1, \ldots g_m\}$, such that for each pair of goods $g_s, g_t$, with index $s < t$, and all agents $i \in [n]$, the inequality $v_i(S + g_s) \geq  v_i(S + g_t)$ holds for each subset $S \subset [m]$ that does not contain $g_s$ and $g_t$; see Example \ref{ex:sqrt-ordered} in Section \ref{subsec:additive-ordered}. 

This work also establishes improved bounds for the specific case of additive valuations. Recall that a valuation $v_i$ is said to be additive if, for every subset $S\subseteq[m]$ of goods, $v_i(S)=\sum_{g\in S} v_i(\{g\})$. We will use the shorthand $v_i(g)$---instead of $v_i(\{g\}) \in \mathbb{R}_+$---to denote agent $i$'s value for any good $g \in [m]$.  


\paragraph{Allocations and Multi-Allocations.} An allocation $\calB=(B_1,B_2,\ldots, B_n)$ of the goods among the $n$ agents is a partition of $[m]$ into $n$ pairwise disjoint subsets $B_1,\ldots, B_n \subseteq [m]$. Here, the subset of goods $B_i$ is assigned to agent $i \in [n]$ and is referred to as $i$'s bundle. In addition, write $\Pi_n([m])$ to denote the collection of all $n$-partitions of $[m]$. Note that for any allocation $\calB =(B_1,\ldots, B_n)$ we have, by definition, $\cup_{i=1}^n B_i = [m]$ and $B_i \cap B_j = \emptyset$, for all $i \neq j$, and hence $\calB \in \Pi_n([m])$.

 
A \textit{multi-allocation} is a tuple $\calA=(A_1,A_2\dots,A_n)$ of $n$ subsets, wherein subset $A_i \subseteq [m]$ denotes the bundle assigned to agent $i$. In contrast to allocations, in a multi-allocation, we do not require that the assigned bundles $A_i$ are pairwise disjoint and that they partition $[m]$.\footnote{Note that $A_i$s are still subsets of goods and not multisets.} Hence, in a multi-allocation, a single good may be present in multiple bundles or even in none. 

Though, when in a multi-allocation $\calA$, each good $g$ is assigned to exactly one agent, we refer to $\calA$ as an {\it exact allocation}; this is to emphasize that the bundles of such a multi-allocation do partition $[m]$. 

We associate with each bundle $A_i \subseteq [m]$ an $m$-dimensional characteristic vector $\rmchar(A_i) \in \{0,1\}^m$. For each good $g\in [m]$, the $g$th component of the characteristic vector---denoted as $\rmchar(A_i)_g$---is equal to one if $g \in A_i$, otherwise the $g$th component is zero. That is, 
\begin{align*}
\rmchar(A_i)_g \coloneqq \begin{cases}
    1 & \text{if } g\in A_i \\
    0 & \text{otherwise}.
\end{cases}
\end{align*}

For any multi-allocation $\calA=(A_1, \ldots, A_n)$, we will use $\chi^\calA \in \mathbb{Z}^m_+$ to denote the vector sum of the characteristic vectors of its bundles, $\chi^\calA \coloneqq \sum_{i=1}^n\rmchar(A_i)$. We will refer to $\chi^\calA$ as the \textit{characteristic vector} of the multi-allocation $\calA$. When there is no ambiguity, we will omit the notational dependence in the superscript and simply write $\chi$ for $\chi^\calA$.

Note that for any good $g\in [m]$ and multi-allocation $\calA$, the $g^{th}$ component of the characteristic vector $\chi^\calA_g$ is equal to the number of bundles in $\calA$ that contain $g$. Conceptually, we think of this setting as one in which $\chi^A_g$ identical copies of the good $g$ are assigned among different agents. 

Write $\ellone{\chi^\calA}$ and $\ellinfty{\chi^\calA}$ to denote the $\ell_1$ and $\ell_\infty$ norm, respectively, of the characteristic vector. Hence,  $\ellone{\chi^\calA} = \sum_{g=1}^m \chi^\calA_g$ and $\ellinfty{\chi^\calA} = \max_{g\in[m]} \chi^\calA_g$. It is relevant to note that $\ellone{\chi^\calA}$ captures the total number of goods, with copies, assigned among the agents,  $\ellone{\chi^\calA} = \sum_{i=1}^n |A_i|$. Further, $\ellinfty{\chi^\calA}$ captures the maximum number of copies of any one good $g$ assigned under $\calA$.

In particular, if $\calA$ is an {\it exact} allocation, then $\chi^\calA$ is equal to the all-ones vector and we have $\ellone{\chi^\calA} =m$ and $\ellinfty{\chi^\calA} =1$.
 
\noindent
The shared-based fairness criterion we study in this work is defined using maximin shares; these shares are defined next.
\begin{definition}[Maximin Share (MMS)]\label{def:mms}
    Given any fair division instance $\langle [n], [m], \{v_i\}_{i=1}^n \rangle$ with goods, the {maximin share}, $\mu_i \in \mathbb{R}_+$, of each agent $i \in [n]$ is defined as 
    \begin{align*}
    \mu_i \coloneqq  \max_{(X_1,\dots, X_n) \in \Pi_n([m])} \ \ \min_{j\in[n]} v_i(X_{j}).
    \end{align*}
Further, for each agent $i$, let $\calM^i=(M^i_1, M^i_2, \ldots, M^i_n) \in \Pi_n([m])$ denote an {MMS-inducing partition}:
\begin{align*}
\calM^i \in \argmax_{(X_1,\dots, X_n) \in \Pi_n([m])} \ \ \min_{j\in[n]} v_i(X_{j})
\end{align*}
\end{definition}

Note that in Definition \ref{def:mms} the maximum is taken over all $n$-partitions of $[m]$. Also, by definition, the partition $\calM^i =(M^i_1, \ldots, M^i_n)$ satisfies $v_i(M^i_j) \geq \mu_i$, for each index $j \in [n]$. 

\paragraph{Fair Multi-Allocations.} A multi-allocation $\calA=(A_1,\dots,A_n)$ is said to be an \emph{MMS multi-allocation} (i.e., it is deemed to be fair) if under it each agent receives a bundle of value at least its maximin share:  $v_i(A_i)\geq \mu_i$ for all agents $i \in [n]$.
 
To establish existential guarantees for MMS multi-allocations $\calA$, we will assume that, for all the agents, we are given the MMS-inducing partitions $\calM^i$, which in turn are guaranteed to exist (see Definition \ref{def:mms}).  

\section{Total-Secluded Strongly Connected Subgraph}
\label{sec:scc}
In this section, we investigate the \textsc{Total-Secluded Strongly Connected Subgraph} problem, or \scs{} for short. First, we prove that the problem is NP-hard in general graphs, motivating analysis of its parameterized complexity.

% \begin{tcolorbox}[enhanced,title={\color{black} {\textsc{Total-Secluded Strongly Connected Subgraph} (\scs{})}}, colback=white, boxrule=0.4pt,
% 	attach boxed title to top left={xshift=.3cm, yshift*=-2.5mm},
% 	boxed title style={size=small,frame hidden,colback=white}]
	
% 	\textbf{Input:}  A directed graph $G$, a weight function $\wOp \colon \ve{G} \to \N$, and integers $w,k \in \N$\\
% 	\textbf{Output:} Decide if there is a set $S \subseteq \ve{G}$ with weight $\w{S} \ge w$ and total neighborhood size $\abs{\nei{S}} \le k$, such that $\induced{G}{S}$ is strongly connected.
% \end{tcolorbox}


\iflong
\begin{theorem}
\fi
\ifshort
\begin{theorem}[$\star$]
\fi
\label{thm:total_scc_np_hard}%Use the env "theoremE" for "End" and add the two arguments [title_of_theorem][options: end, puts the proof at the end; restate copies the statement of the theorem where the proof is.]
  \scs{} is NP-hard, even for unit weights.
\end{theorem}
\iflong
\begin{proof}
\begin{figure}
  \centering
  \hfill
  \begin{subfigure}{0.49\textwidth}
    \centering
    \includegraphics[width=0.6\textwidth,page=2]{figures/incidence_graph_reduction}
    \caption{An undirected graph $G$ with maximum clique size 3.}
  \end{subfigure}
  \hfill
  \begin{subfigure}{0.49\textwidth}
    \centering
    \includegraphics[width=0.70\textwidth,page=3]{figures/incidence_graph_reduction}
    \caption{The \scs{} instance $G'$ created by our reduction in the proof of \Cref{thm:total_scc_np_hard}. }
  \end{subfigure}
  \hfill
  \caption{A visualization of the reduction in the proof of \Cref{thm:total_scc_np_hard}. %inspired by~\cite{fomin2013parameterized}. Note that $V_E$ is a complete directed graph. No strongly connected subgraph of $G'$ of weight greater than 1 can include a vertex in $V_V$ or $V_E'$. Thus, any subset $S$ of $V_E$ with $\abs{\nei{S}} \le \abs{\e{G}}+k$ of size $\binom{k}{2}$ corresponds to the clique $\nei{S} \cap V_V$ in $G$.
  }\label{fig:clique_reduction_scc}
\end{figure}

  We reduce from the \textsc{Clique} problem which is NP-hard, where given a graph the problem is to check if there is a clique of size at least $k$. Given an instance $(G,k)$ with $k\ge 2$, we reduce it to an instance of \scs{} $(G',w,k')$ as follows:
  
  We create the graph $G'$ with $\ve{G'} \coloneqq \ve{G}\cup V_E \cup V_E'$, where $V_E \coloneqq \set{v_e}{e\in E(G)}$ and $V_E'\coloneqq \set{v_e'}{e\in E(G)}$. For any vertex $x\in V(G')$, the weight of $v$ is 1.
  The new graph has edges \[
  \e{G'} \coloneqq \set{(v_{e_1}, v_{e_2})}{v_{e_1} \ne v_{e_2} \in V_E} \cup \set{(v_e, v_e'), (v_e, v_a), (v_e, v_b)}{e = \set{a,b} \in \e{G}}.
  \] 
    % We add edges to $\e{G'}$ between every possible pair $v_{e_1}, v_{e_2} \in V_E$ in both directions. 
      % For each edge $e = \{a,b\} \in \e{G}$, we add the edges $(v_e, v_e')$, $(v_e, v_a)$, and $(v_e, v_b)$. 
    We set $k' = k + \abs{\e{G}}$ and $w = \binom{k}{2}$. For an illustration of the construction of $G'$, refer to \Cref{fig:clique_reduction_scc}.
 
 
 Next, we prove that $G$ has a clique of size $k$ if and only if $G'$ has a strongly connected subgraph $H$ of weight at least $w$ with the size of total neighborhood of $H$ at most $k'$. For the forward direction, let $C \subseteq \ve{G}$ be a clique of size $k$ in $G$. Then, we can choose $S = \{v_{\{a, b\}} \mid a, b \in C\}$. Since $C$ is a clique, $S \subseteq V_E$. Also, $S$ has weight $w$ since a clique of size $k$ contains $\binom{k}{2}$ edges. Furthermore, the neighborhood of $S$ consists of all $v_e'$ for $v_e \in S$, all $v_e \notin S$, and all $v_a$ for $a \in C$, giving a total size of $k'$. Therefore, $S$ is a solution for \scs{}.

  Let $S$ be a solution for \scs{} in $G'$. For $k \ge 2$, the solution must include at least one vertex from $v_E$ to reach the desired weight. Therefore, no vertex from $\ve{G'} \cup V_E'$ can be included to not violate connectivity. The size of the neighborhood of $S$ will then be $\abs{\e{G}}$ increased by the number of incident vertices to edge-vertices picked in $S$. Since $S$ must be an edge set of size at least $\binom{k}{2}$ with at most $k$ incident vertices, this must induce a clique of size at least $k$ in $G$.
\end{proof}
\fi

The proof of \Cref{thm:total_scc_np_hard} also shows that \scs{} is W[1]-hard when parameterized by $w$, since $w$ in the proof also only depends on the parameter for \textsc{Clique}.

In the following subsections, we describe the recursive understanding algorithm to solve \scs{} parameterized by $k$.
We follow the framework by~\cite{chitnis2016designing,golovach2020finding} and first introduce generalized problems in \Cref{sec:scc_bc}. In \Cref{sec:unbreak}, we solve the case of unbreakable graphs. We introduce graph \emph{extensions} in \Cref{sec:scc_extensions} as a nice framework to formulate our reduction rules and full algorithm in \Cref{sec:solving_scc}.


\subsection{Boundaries and boundary complementations}\label{sec:scc_bc}

In this subsection, we first define an additional optimization problem that is useful for recursion. Then, we describe a problem-specific \emph{boundary complementation}. Finally, we define the auxiliary problem that our algorithm solves, which includes solving many similar instances from the optimization problem.

\zzcommand{\scsrec}{\textsc{Max \scs{}}}
\begin{tcolorbox}[enhanced,title={\color{black} {\scsrec{}}}, colback=white, boxrule=0.4pt,
	attach boxed title to top left={xshift=.3cm, yshift*=-2.5mm},
	boxed title style={size=small,frame hidden,colback=white}]
	
	\textbf{Input:}  
  A directed graph $G$, subsets $I,O,B \subseteq \ve{G}$, a weight function $\wOp \colon \ve{G} \to \N$, and an integer $k \in \N$

	\textbf{Output:} A maximum weight set $S \subseteq \ve{G}$ with $I \subseteq S$, $O \cap S = \emptyset$, $N(S) \subseteq B$, and $\abs{\nei{S}} \le k$, such that $\induced{G}{S}$ is strongly connected, or report that no feasible solution exists. 
\end{tcolorbox} 

Note that this problem directly generalizes the optimization variant of \scs{} since we can just use $I \coloneqq O \coloneqq \emptyset$ and $B \coloneqq \ve{G}$. However, \scsrec{} allows us to put additional constraints on recursive calls, enforcing vertices to be included or excluded from the solution and neighborhood.

\begin{figure}[t]
    \centering
    \hfill
    \begin{subfigure}{0.48\textwidth}
      \centering
      \includegraphics[width=0.9\textwidth,page=2]{figures/bc_solution}
      \caption{A strongly connected subgraph $S$ of a graph $G$ with $k=5$ neighbors. Black vertices are part of $S$.}
    \end{subfigure}
    \hfill
    \begin{subfigure}{0.48\textwidth}
      \centering
      \includegraphics[width=0.9\textwidth,page=3]{figures/bc_solution}
      \caption{The boundary complementation that admits an equivalent feasible solution when setting $k' \coloneqq k - 2$.}\label{fig:scc_bcb}
    \end{subfigure}
    \hfill
    \caption{A visualization of a solution in the original graph and a solution in a boundary complementation, showing how every partial solution in $U$ can be adequately represented by a specific boundary complementation.}\label{fig:scc_bc}
\end{figure}

\begin{definition}[Boundary Complementation]\label{def:scc_border_complementation}
  Let $\mathcal{I} = (G,I,O,B,\wOp,k)$ be a \scsrec{} instance. Let $T \subseteq \ve{G}$ be a set of \emph{boundary terminals} with a partition $X,Y,Z \subseteq T$ and let $R \subseteq X \times X$ be a relation on $X$. Then, we call the instance $(G',I',O',B,\wOp',k')$ a \emph{boundary complementation} of $\mathcal{I}$ and $T$ if
   \begin{enumerate}%[noitemsep,nolistsep]
    \item $G'$ is obtained from $G$ by adding vertices $u_{(a,b)}$ for every $(a,b) \in R$ and edges $(a, u_{(a,b)})$, $(u_{(a,b)}, b)$, and for every $y \in Y$ additionally $(u_{(a,b)}, y)$,
    \item $I' \coloneqq I \cup X \cup \{u_{(a,b)} \mid (a,b) \in R\}$,
    \item $O' \coloneqq O \cup Y \cup Z$,
    \item $\wOp'(v) \coloneqq \w{v}$ for $v \in \ve{G}$ and $\wOp'(u_{(a,b)}) \coloneqq 0$ for $(a,b) \in R$, and
    \item $k' \le k$.\qedhere
  \end{enumerate}
\end{definition}

See \Cref{fig:scc_bc} for an example boundary complementation.
The intuition here should be that if we take the union of $G$ with any other graph $H$ and only connect $H$ to $G$ at the vertices in $T$, then $(X,Y,Z,R)$ encodes all possibilities of how a solution in $G \cup H$ could behave from $G$'s point of view. So, for any solution $S$ to \scsrec{} in $G \cup H$, there is some boundary complementation for $G$ in which we can solve and exchange $S \cap G$ for that solution. Later, we prove a statement that is similar to this intuition.

To employ recursive understanding, we need a boundaried version of the problem. Intuitively, this problem is the same as the previous \scsrec{} but for a small part of the graph we want to try out every possibility, giving many very similar instances. This small part will later represent a separator to a different part of the graph. 

\zzcommand{\scsborder}{\textsc{Boundaried \scsrec{}}}
\begin{tcolorbox}[enhanced,title={\color{black} {\scsborder{}}}, colback=white, boxrule=0.4pt,
	attach boxed title to top left={xshift=.3cm, yshift*=-2.5mm},
	boxed title style={size=small,frame hidden,colback=white}]
	
	\textbf{Input:}  
A \scsrec{} instance $\mathcal{I} = (G,I,O,B,\wOp,k)$ and a set of boundary terminals $T \subseteq \ve{G}$ with $\abs{T} \le 2k$

	\textbf{Output:}
A solution to \scsrec{} for each boundary complementation $\mathcal{I}'$ of $\mathcal{I}$ and $T$, or report that no solution exists.
\end{tcolorbox}

To even have a chance to solve this problem, we need to make sure that there are not too many boundary complementations. The following lemma bounds that number in terms of $k$. %for the case $\abs{T} \le 2k$.

\iflong
\begin{lemma}
\else
\begin{lemma}[$\star$]
\fi
\label{lem:number_border_complementations}
  For a \scsrec{} instance $(G,I,O,B,\wOp,k)$ and $T \subseteq \ve{G}$, there are at most $3^{\abs{T}}2^{\abs{T}^2}(k+1)$ many boundary complementations, which can be enumerated in time $2^{\bigO{\abs{T}^2}}n^{\bigO{1}}$.
\end{lemma}
\iflong
\begin{proof}
  Every element of $T$ has to be in either $X$, $Y$, or $Z$, which gives $3^{\abs{T}}$ possible arrangements. Every one of the $\abs{X}^2$ elements in $X\times X$ can either be in $R$ or not, giving $2^{\abs{X}^2} \le 2^{\abs{T}^2}$ possible arrangements. Furthermore, there are $k+1$ choices for $0 \le k' \le k$. Multiplying these gives the final number. By enumerating all the respective subsets and constructing $G'$ in time $n^{\bigO{1}}$, we can enumerate all boundary complementations in the claimed time.
\end{proof}
\fi

\subsection{Unbreakable Case}\label{sec:unbreak}

This subsection gives the algorithm for the base case of our final recursive algorithm, when no balanced separator exists. We start by giving the definitions of separations and unbreakability.

\begin{definition}[Separation]
  Given two sets $A, B \subseteq \ve{G}$ with $A \cup B = \ve{G}$, we say that $(A, B)$ is a \emph{separation of order $\abs{A \cap B}$} if there is no edge with one endpoint in $A \setminus B$ and the other endpoint in $B \setminus A$.
\end{definition}

\begin{definition}[Unbreakability]
  Let $q,k \in \N$. An undirected graph $G$ is \emph{$(q,k)$-unbreakable} if for every separation $(A,B)$ of $G$ of order at most $k$, we have $\abs{A \setminus B} \le q$ or $\abs{B \setminus A} \le q$.
\end{definition}

The next lemma formalizes that the neighborhood of any solution gives you a separator of order $k$. If the graph is unbreakable, either the solution or everything but the solution must be small. With this insight, the statement should be intuitive. We only need to fill in the details since the graph changes slightly when considering the boundary complementation. However, the proof is identical to~\cite[Lemma~12]{golovach2020finding}, so we omit it.

\begin{lemma}\label{lem:unbreak_small_or_large}
  Let $\mathcal{I}$ be a \scsborder{} instance on a $(q,k)$-unbreakable graph $G$. Then, for each set $S$ in a solution of $\mathcal{I}$, either $\abs{S \cap \ve{G}} \le q$ or $\abs{\ve{G} \setminus S} \le q + k$.
\end{lemma}

For computing these solutions, we use an approach from~\cite{chitnis2016designing} that resembles color coding.

\begin{lemma}[\protect{\cite[Lemma~1.1]{chitnis2016designing}}]\label{lem:find_sets}
  Given a set $U$ of size $n$ and integers $a,b \in \N$, we can construct in time $2^{\bigO{\min\set{a,b}\log (a+b)}}n\log n$ a family $\mathcal{F}$ of at most $2^{\bigO{\min\set{a,b}\log (a+b)}}\log n$ subsets of $U$ such that the following holds. For any sets $A, B \subseteq U, A \cap B = \emptyset, \abs{A} \le a, \abs{B} \le b$, there is a set $S \in \mathcal{F}$ with $A \subseteq S$ and $B \cap S = \emptyset$.
\end{lemma}

We use the previous two lemmas to construct an algorithm for the unbreakable case.

\iflong
\begin{theorem}
\else
\begin{theorem}[$\star$]
\fi
\label{thm:unbreakable_scc}
  \scsborder{} on $(q,k)$-unbreakable graphs can be solved in time $2^{\bigO{k^2 \log(q)}}n^{\bigO{1}}$.
\end{theorem}
\iflong
\begin{proof}
  Our algorithms starts by enumerating all boundary complementations of an instance $\mathcal{I}$. For a \scsrec{} instance $\mathcal{I'} = (G,I,O,B,\wOp,k)$, we know by \Cref{lem:unbreak_small_or_large} that the solution must be either of size at most $q+4k^2$ including the newly added vertices $u_{(a,b)}$, or at least $\abs{\ve{G}} - (q+k+4k^2)$. Let $s = q+k+4k^2 \ge q + 4k^2$.

  This allows us to address the two possible cases for the solution size separately. We give one algorithm to find the maximum weight solution of size at most $s$ and one algorithm to find the maximum weight solution of size at least $\abs{\ve{G}} - s$. In the end, we return the maximum weight of the two, or none if both do not exist. 

  \subparagraph*{Finding a small solution} Our algorithm works as follows. 
  \begin{enumerate}
    \item Apply the algorithm from \Cref{lem:find_sets} with $U = \ve{G}, a = s, b = k$ to compute a family $\mathcal{F}$ of subsets of $\ve{G}$.

    \item For every $F \in \mathcal{F}$, consider the strong components of $\induced{G}{F}$ separately.

    \item For a strong component $Q$, we check if $Q$ is a feasible solution and return the maximum weight one.
  \end{enumerate}

  \subparagraph*{Finding a large solution} For this case, our algorithm looks as follows. 
  \begin{enumerate}
    \item Compute the strong components of $G$.

    \item For every strong component $C$, we construct the graph $G_C$ by taking $\induced{G}{\cnei{C}}$ and adding a single vertex $c$ with edges $(c,v)$, for every $v \in \nei{C}$. 

    \item Then, run the algorithm from \Cref{lem:find_sets} with $U = \ve{G_C}, a = s+1, b = k$ to receive a family $\mathcal{F}$ of subsets of $\ve{G_C}$.

    \item For every $F \in \mathcal{F}$, find the component including $c$.

    \item \label[step]{it:5}For each such component $Q$, check if $Q \setminus \cnei{S}$ is a feasible solution and return the maximum weight one.
  \end{enumerate}

  \subparagraph*{Correctness}
  By \Cref{lem:find_sets}, for any small solution $S$ there is $F \in \mathcal{F}$ with $S \subseteq F$ and $\nei{S} \cap F = \emptyset$, so $S$ must be both a component that is also strongly connected of $\induced{G}{F}$. Therefore, we enumerate a superset of all solutions of size at most $s$ and we find the maximum weight small solution.

  For the large solution, we claim that we find a maximum weight solution for this case. Clearly, every strongly connected subgraph of $G$ must be a subgraph of a strong component of $G$. Let $S'$ be a large solution that is a subset of a strong component $C$. Consider the set $S = \ve{G_C} \setminus \cnei{S'}$ in $G_C$. Then, we must have $\abs{S} \le a$ since $S'$ is large. Also, $\nei{S} \subseteq \nei{S'}$ by definition and therefore $\abs{\nei{S}} \le b$. Thus, $S$ is considered in \Cref{it:5} if it is weakly connected.

  If there is a $v \in S$ that is not in the same component as $c$, we take the component of $v$ in $S$ and include it in $S'$. Then, $S'$ is still strongly connected but has a strictly smaller neighborhood size and equal or greater weight since weights are non-negative. We can repeat this procedure until $S$ is a single component and will be enumerated by the algorithm. Therefore, our algorithm finds a solution of weight at least $\w{S'}$.

  \subparagraph*{Total Runtime} Both cases make use of at most $n$ calls to the algorithm from \Cref{lem:find_sets} with some small modifications. For every returned sets, both algorithms compute the components and verify strong connectivity. Therefore, we can bound the runtime per boundary complementation by $2^{\bigO{\min\set{s,k}\log(s+k)}}n^{\bigO{1}} = 2^{\bigO{k\log(q+k)}}n^{\bigO{1}}$. By \Cref{lem:number_border_complementations}, enumerating all boundary complementations adds a factor of $2^{\bigO{k^2}}n^{\bigO{1}}$, which gives the desired runtime.
\end{proof}
\fi

\subsection{Compressing Graph Extensions}\label{sec:scc_extensions}

Before we give the complete algorithm, we need one more ingredient that will be necessary to apply a set of reduction rules later in the algorithm. This ingredient will be a routine that compresses a subgraph into a subgraph whose size depends only on the rest of the graph and that is equivalent in terms of forming strongly connected subgraphs. To achieve this goal, we formally define sufficient properties to reason about this equivalence and bound the number of equivalence classes.
First, we define the notion of a graph \emph{extension}, a way to extend one graph with another. This concept allows us to speak more directly about graph properties before and after exchanging a part of the graph with a different one.

\begin{definition}[Extension]
Given a directed graph $G$, we call a pair $\dexpair$ an \emph{extension of} $G$ if $D$ is a directed graph and $\dexset \subseteq (\ve{G} \times \ve{D}) \cup (\ve{D} \times \ve{G})$ is a set of pairs between $G$ and $D$. 
We name the graph $\ex{G}{D}{\dexset} \coloneqq (\ve{G} \cup \ve{D}, \e{G} \cup \e{D} \cup \dexset)$, that can be created from the extension, $G$ \emph{extended by} $\dexpair$.
\end{definition}

Note that $\dexset$ is a set of arbitrary pairs with one element in $\ve{G}$ and one in $\ve{D}$. However, we use extensions to construct extended graphs.
Intuitively, an extension of $G$ is a second graph $D$ together with an instruction $\dexset$ on how to connect $D$ to $G$.

Next, we identify three important attributes of extensions in our context. Later, we show that these give a sufficient condition on when two extensions form the same strongly connected subgraphs.
  For this, consider a directed graph $G$ with an extension $\dexpair$. For $U \subseteq \ve{D}$, we write $\inNeiG{\dexset}{U}$ as a shorthand for $\inNeiG{\ex{G}{D}{\dexset}}{U}$, that is, all $v \in \ve{G}$ with $(v,u) \in \dexset$ for some $u \in U$. Define $\outNeiG{\dexset}{v}$ analogously.
  Write $\scc{D}$ for the \emph{condensation} of $D$, where every strongly connected component $C$ of $D$ is contracted into a single vertex.
  Define $\sourcee{D}{\dexset}, \sinke{D}{\dexset} \subseteq \pot{\ve{G}}$ such that 
  \begin{align*}
    \sourcee{D}{\dexset} &\coloneqq \set{\inNeiG{\dexset}{U}}{U \subseteq \ve{D} \text{ is a source component in } \scc{D}} \text{ and}\\
    \sinke{D}{\dexset} &\coloneqq \set{\outNeiG{\dexset}{U}}{U \subseteq \ve{D} \text{ is a sink component in } \scc{D}},
  \end{align*}
  that is, for every strongly connected source component $C$ in $\scc{D}$, $\sourcee{D}{\dexset}$ contains the set of all $v \in \ve{G}$ such that $\died{u}{v} \in \dexset$ for some $u \in C$ and analogously for $\sinke{D}{\dexset}$.
%
  Furthermore, define \[\conne{D}{\dexset} \coloneqq \set{(a,b) \in \ve{G}^2}{\text{there is a $d_1$-$d_2$-path in $D$ with } (a,d_1), (d_2,b) \in \dexset}, \] that is, all $(a,b)$ such that there is an $a$-$b$-path in $\ex{G}{D}{\dexset}$, whose intermediate vertices and edges belongs to D. Refer to \Cref{fig:extension_compression} for examples of extensions and the three sets.

\begin{definition}[Equivalent Extensions]
    Let $G$ be a directed graph. We say that two extensions $\expairi{1}$ and $\expairi{2}$ of $G$ are \emph{equivalent} if \[(\source{D_1,E_{GD_1}}, \sink{D_1,E_{GD_1}}, \conn{D_1,E_{GD_1}}) = (\source{D_2,E_{GD_2}}, \sink{D_2,E_{GD_2}}, \conn{D_2,E_{GD_2}}).\]% We also denote this by $\expair{G}{D_1} \sccequiv{G} \expair{G}{D_2}$.
\end{definition}

The name is justified; clearly, extension equivalence defines an equivalence relation.
The next statement reveals the motivation behind the definition of extension equivalence. It gives us a sufficient condition for two extensions being exchangeable in a strongly connected subgraph.

% \iflong
\begin{lemma}
% \else
% \begin{lemma}[$\star$]
% \fi
\label{lem:source_sink_conn_equiv}
  Let $G$ be a directed graph with two equivalent extensions $\expairi{1}$ and $\expairi{2}$. Let $U \subseteq \ve{G}$ be nonempty such that the extended graph $\ex{\induced{G}{U}}{D_1}{\exset{\induced{G}{U}}{D_1}}$ is strongly connected. Then $\ex{\induced{G}{U}}{D_2}{\exset{\induced{G}{U}}{D_2}}$ is also strongly connected.
\end{lemma}
% \iflong
\begin{proof}
  We construct a $v_1$-$v_2$-path for all $v_1, v_2 \in U \cup \ve{D_2}$ that only uses edges in $\e{D_2}$, $\exset{G}{D_2}$, and $\induced{G}{U}$ by case distinction.

\begin{description}
    \item[Paths $U \to U$.]  Let $u_1, u_2 \in U$. If there is a path from $u_1$ to $u_2$ in $\induced{G}{U}$, this path also exists after exchanging $\expair{G}{D_1}$ to $\expair{G}{D_2}$. If the path passes through $D_1$, since $\conn{D_1,E_{GD_1}} = \conn{D_2,E_{GD_2}}$, we can exchange all subpaths through $D_1$ by subpaths through $D_2$.

    \item[Paths $\ve{D_2} \to U$.] Let $v \in \ve{D_2}, u \in U$. We construct a $v$-$u$-path by first walking from $v$ to any sink component $T$ in $\scc{D_2}$. If there is no edge $(t,u') \in \exset{G}{D_2}$ with $t \in T, u' \in U$ that we can append, since $\sink{D_1,\expair{G}{D_1}} = \sink{D_2,\expair{G}{D_2}}$, there must also be a sink component in $\scc{D_1}$ with no outgoing edge to $U$. However, this is a contradiction to the fact that $\ex{\induced{G}{U}}{D_1}{\exset{\induced{G}{U}}{D_1}}$ with nonempty $U$. Therefore, we can find a $(t,u')$ to append for some $t \in T, u' \in U$. From $u'$, there is already a path to $u$, as proven in the first case.

    \item[Paths $U \to \ve{D_2}$.] Next, we construct a $u$-$v$-path backwards by walking from $v$ backwards to a source $s$ in $D_2$. Analogously, there is an edge $(u',s) \in \exset{G}{D_2}$ for some $u' \in U$ since $\source{D_1,E_{GD_1}} = \source{D_2,E_{GD_2}}$, which we append. From $u$, there is a path to $u'$, as proven in the first case, which we prepend to the rest of the path.

    \item [Paths $\ve{D_2} \to \ve{D_2}$.] Let $v_1, v_2 \in \ve{D_2}$. To construct a $v_1$-$v_2$-path, we can just walk from $v_1$ to any $u \in U$ and from there to $v_2$ as shown before.\qedhere
\end{description}
\end{proof}
% \fi

Furthermore, observe that the union of two extensions creates another extension where source, sink and connection sets correspond exactly to the union of the previous sets. Hence, the union of two equivalent extensions will again be equivalent. This fact is formalized in the next observation and  will turn out useful in later reduction rules.

\begin{figure}
    \centering
    \includegraphics[width=0.65\linewidth]{figures/extension_comp.pdf}
    \caption{Two example extensions of a graph $G$. Observe that $\source{D,\dexset} = \set{\set{v_1}}$, $\sink{D,\dexset} = \set{\set{v_3}}$, and $\conn{D,\dexset} = \set{(v_1,v_2), (v_1, v_3)}$. The extension $(D', E_{GD'})$ not only has the same sets $\sourceOp, \sinkOp, \connOp$ and is thereby equivalent; it is also the compressed extension of $\dexpair$. Since all sources $d_1,d_2,d_3$ have the same in-neighborhood, they are represented by the single vertex $v_S$.}
    \label{fig:extension_compression}
\end{figure}

\begin{observation} \label{lem:union_equiv_if_equiv}
  Let $G$ be a directed graph with two equivalent extensions $\expairi{1}$ and $\expairi{2}$. Consider the extension defined by $D \coloneqq (\ve{D_1} \cup \ve{D_2}, \e{D_1} \cup \e{D_2})$ and $\exset{G}{D} \coloneqq \exset{G}{D_1} \cup \exset{G}{D_2}$. Then $\dexpair$ is equivalent to $\expairi{1}$ and $\expairi{2}$.
\end{observation}

Now, we finally define our compression routine, which compresses an extension to a bounded size equivalent extension.
That means that we have to ensure that neighborhoods of source components and sink components, as well as achieved connections, stay the same. Furthermore, we want to maintain properties such as strong connectivity and weak connectivity. If an extension is strongly connected, it is easy to convince yourself that it is always possible to compress the extension to a single vertex. Otherwise, we have to be more careful. We add one source vertex per neighborhood set in $\source{D,\dexset}$ as well as one sink vertex per neighborhood set in $\sink{D,\dexset}$, realizing the same $\sourceOp$ and $\sinkOp$. Then, we add vertices in between suitable source and sink vertices to realize exactly the same connections in $\connOp$ without creating additional ones. 
The result of a compression is visualized in \Cref{fig:extension_compression}. Now, we describe the procedure formally. 

% \begin{figure}[t]
%   \begin{minipage}[c]{0.64\linewidth}
%       \centering
%       \begin{subfigure}{0.49\textwidth}
%         \includegraphics[width=\textwidth,page=1]{figures/scc_comp}
%         \caption{An extension $\dexpair$}\label{fig:scc_comp1}
%       \end{subfigure}
%       \hfill
%       \begin{subfigure}{0.49\textwidth}
%         \includegraphics[width=\textwidth,page=2]{figures/scc_comp}
%         \caption{$\dcompG$}\label{fig:scc_comp2}
%       \end{subfigure}
%       \caption{An example for compressing an extension.
%       %Note that the source set still only contains $\set{v_1}$, the sink set only contains $\set{v_3}$, and the connections are still $(v_1, v_2)$ and $(v_1, v_3)$.
%       While the size of the extension increases in this example, in general, the size of a compressed extension can be bounded by $\abs{\ve{G}}$.}\label{fig:scc_comp}\jonas{make the graph a bit bigger such that comp is smaller}
%   \end{minipage}
%   \hfill
%   \begin{minipage}[c]{0.34\linewidth}
%       \centering
%       \includegraphics[width=0.8\textwidth]{figures/IOB_fig}
%       \caption{A visualization of how the sets $I$, $O$, and $B$ can overlap after the application of \Cref{red:in_out}. Any component in $G-B$ can be in $I$, $O$, or none of them, but not both. The black vertices form a feasible solution.}\label{fig:iob}
%   \end{minipage}
% \end{figure}


\begin{comment}
\begin{definition}[Compression]\label{def:comp}
  Let $G$ be a directed graph with an acyclic extension $\dexpair$ and $\dsccset \ne \emptyset$. We define the \emph{compressed extension}, denoted as $\dcompG$, as follows.

  If $\abs{\ve{D}} = 1$, we set $\dcompG \coloneqq \dexpair$.\jonas{If D is strongly connected, compress to single vertex}
  Otherwise, we create a new DAG $D'$ and a new set $\exset{G}{D'}$ with 
  \[
  \ve{D'} \coloneqq \set{v_S}{S \in \source{D}} \cup \set{v_T}{T \in \source{D}} \cup \set{v_c}{c \in \conn{D}}.\]

  The set $\exset{G}{D'}$ contains pairs $(s, v_S)$ for $S \in \source{D}$ and  $s \in S$ as well as $(v_T, t)$ for $T \in \sink{D}$ and $t \in T$.
  Additionally, for $(a,b) \in \conn{D}$, it contains $(a, v_{(a,b)})$ and $(v_{(a,b)}, b)$.

  Add an edge from $v_S$ to $v_T$, if there is a source $s$ and a sink $t$ in $D$ such that there is an $s$-$t$-path in $D$.

  For $c = (a,b) \in \conn{D}$, let $s_c \in \ve{D}, S \coloneqq \inNeiG{\dexset}{s_c}$ be a source such that for every $v \in S$, there is $(v, b) \in \conn{D}$. Similarly, let $t_c \in \ve{D}$ be a sink with $T \coloneqq \outNeiG{\dexset}{t_c}$ such that $(a,v) \in \conn{D}$ for $v \in T$. We add edges $(v_S, v_c), (v_c, v_T)$ to $\e{D'}$.
\end{definition}
\todo{S: IT's not a definition but a procedure to compress. Rewrite }
\end{comment}

Let $G$ be a directed graph with an extension $\dexpair$. We give a \emph{compression routine} that returns an extension that we call \emph{compressed extension}, denoted as $\dcompG$. This procedure works as follows:
\subparagraph*{Compression Routine}
\begin{itemize}
    \item  If $D$ is strongly connected, we contract $D$ to a single vertex, removing self-loops and multiple edges. We adjust $E_{GD}$ accordingly, that is, change vertices in $D$ to $v$ and remove multiple edges.
      
    \item Otherwise,  $\dcompG \coloneqq (D',\exset{G}{D'})$, where $D'$ is a DAG and  $\exset{G}{D'}$ is an extension such that
    \begin{align*}
  \ve{D'} \coloneqq &\set{v_S}{S \in \source{D,\dexset}} \cup \set{v_T}{T \in \sink{D,\dexset}} \cup \set{v_c}{c \in \conn{D,\dexset}},\\
  \exset{G}{D'} \coloneqq &\set{(s,v_S)}{S\in \source{D,\dexset}, s\in S} \cup \set{(v_T,t)}{T\in \sink{D,\dexset}, t\in T }\\
   \cup &\set{(a,v_c),(v_c,b)}{c=(a,b) \in \conn{D,\dexset}}.
    \end{align*}

    To define $\e{D}$, consider every source component $C_s$ and sink component $C_t$ in $\scc{D}$ such that $C_t$ is reachable from $C_s$ in $D$. Let $S \coloneqq \inNeiG{\dexset}{C_s}$ and $T \coloneqq \inNeiG{\dexset}{C_t}$ be the corresponding sets in $\source{D,\dexset}$ and $\sink{D,\dexset}$.
    \begin{itemize}
        \item Add the edge $(v_S, v_T)$ to $\e{D}$.
        \item For every $c = (a,b) \in \conn{D,\dexset}$ that satisfies $(s,b) \in \conn{D,\dexset}$ and $(a,t) \in \conn{D,\dexset}$ for every $s \in S, t \in T$, add the edges $(v_S, v_c)$ and $(v_c, v_T)$ to $\e{D}$.
    \end{itemize}
 \end{itemize}
%%%%%%%%%%%%%%%%%%%%%%%%%%%%%

 % Since there is always a source in $D$ that reaches the endpoint of a path that gives a connection $(a,b) \in \conn{D}$, a suitable $s_c \in \ve{D}$ for \Cref{def:comp} always exists. Since the same holds for sinks, $\compOp_G$ is well-defined.

Now we go on to prove the properties we maintain while compressing the extension. Afterwards we bound the size of a compressed extension and thereby also the number of equivalence classes.
% \iflong
\begin{lemma}
% \else
% \begin{lemma}[$\star$]
% \fi
\label{lem:comp_weakly_conn_equiv}
  Let $G$ be a directed graph with an extension $\dexpair$ and let $(D', E_{GD'})$ be the compressed extension of $\dexpair$. 
  Then, the following are true.
  \begin{enumerate}
      % \item If $D$ is acyclic, then $D'$ is also acyclic. 
      \item If $D$ is weakly connected, then $D'$ is also weakly connected. 
      \item $D$ is strongly connected if and only if $D'$ is strongly connected.
      \item $(D', E_{GD'})$ is equivalent to $\dexpair$. 
  \end{enumerate}
\end{lemma} 
% \iflong
\begin{proof}
  % We can focus on the case that $\abs{\ve{D}} > 1$; the other case is trivial.\jonas{no longer necessary}
  % Verifying that $D'$ is a DAG is easily done with the topological order that chooses all $v_S$ first, then all $v_c$, and finally all $v_T$ in arbitrary order.
  For the first property, assume that $D$ is weakly connected.
  We know by definition that every sink in $D'$ is reached by at least one source.
  Consider a $v_c$ with $c = (a,b) \in \conn{D,\dexset}$. To show that $v_c$ is connected to some $v_S$ and $v_T$, consider the path from $d_1$ to $d_2$ in $D$ that realizes this connection. There must be a source component $C_S$ and a sink component $C_T$ in $\scc{D}$ such that $d_1$ is reachable from $C_S$ and $C_T$ is reachable from $d_2$. Therefore, any vertex in $C_S$ can also reach $d_2$ and $d_1$ can reach every vertex in $C_T$. By definition of compression, these two components ensure that $v_c$ is connected. 
  
  It remains to show that any source is reachable by any other source in the underlying undirected graph. Let $v_S, v_{S'}$ be two sources in $D'$ with corresponding source components $C$, $C'$ in $\scc{D}$. Since $D$ is weakly connected, there is a path from $C$ to $C'$ in the underlying undirected graph. Whenever the undirected path uses an edge in a different direction than the one before, we extend the path to first keep using edges in the same direction until a source or sink component is reached and then go back to the switching point. This new path can directly be transferred to $D'$, where we only keep the vertices corresponding to source and sink components. By definition, this is still a path in $D'$ that connects $v_S$ to $v_{S'}$, and $D'$ is weakly connected
  % Since we can do this for all $S, S' \in \source{D,\dexset}$, $D'$ must be weakly connected.

  The second property is simple to verify, since strongly connected graphs are by definition compressed to single vertices. If $D$ is not strongly connected, $D'$ will have at least one source and one sink that are not the same.
  
  Regarding the equivalence, we create one source for every $S \in \source{D,\dexset}$ with the same set of incoming neighbors and create no other sources. Therefore, $\source{D', E_{GD'}} = \source{D,\dexset}$ and $\sink{D',E_{GD'}} = \sink{D,\dexset}$ follows analogously.
  For every connection $c \in \conn{D,\dexset}$, we create $v_c$ in $D'$ that realizes this connection. Therefore, we know that $\conn{D',E_{GD'}} \supseteq \conn{D,\dexset}$. Since $v_c$ is only reachable from sources and reaches only sinks that do not give new connections, we arrive at $\conn{D',E_{GD'}} = \conn{D,\dexset}$. 
\end{proof}
% \fi

We also bound the size of the compressed extension as well as the number of different possible compression outputs.

\iflong
\begin{lemma}
\else
\begin{lemma}[$\star$]
\fi
\label{lem:range_size_comp}
  For a directed graph $G$, there can be at most $2^{2\cdot 2^{\abs{\ve{G}}} + \abs{\ve{G}}^2}$ different compressed extensions.
  Furthermore, every compressed extension has at most $2^{\abs{\ve{G}}+1} + \abs{\ve{G}}^2$ vertices.
\end{lemma}
\iflong
\begin{proof}
  There are $2^{\abs{\ve{G}}}$ subsets of $\ve{G}$, each of them can be the neighborhood of a source or sink. Additionally, every one of the at most $\abs{\ve{G}}^2$ can form a connection or not. This proves the first claim. 

  Suppose $\compOp_G$ outputs $\dexpair$. For every subset $U \subseteq \ve{G}$, there can be at most one source and one sink in $D$ that has $U$ as outgoing or incoming neighbors. Also, there are at most $\abs{\ve{G}}^2$ pairs of vertices in $G$ that can form a connection. This bounds the number of connection vertices in $D$.
\end{proof}
\fi

In the past section, we have defined graph extensions and an equivalence relation on them that captures the role they can play in forming strongly connected subgraphs. Furthermore, we have presented a way to compress an extension such that its size only depends on the size of $G$, while remaining equivalent.

In the upcoming section, we will use this theory to design reduction rules for \scsborder{}. Crucially, we view components outside of the set $B$ as extensions and show that it is enough to keep only one compressed extension of each equivalence class.

\subsection{Solving \scsborder{}}\label{sec:solving_scc}

We start by giving some reduction rules for a \scsborder{} instance $\mathcal{I} = (G, I, O, B, \wOp,  k, T)$. Additionally, we assume that $T \subseteq B$ to ensure that we do not change $T$ when changing $G - B$. This condition will always be satisfied in our algorithm.
\ifshort
A ($\star$) after a reduction rule denotes that the proof of safeness of this rule was omitted and can be found in the full version.
\fi

The first reduction rule extends the sets $I$ and $O$ to whole components of $G - B$. This is possible since no solution can include only part of a component without its neighborhood intersecting the component.

\iflong
\begin{reduction*}
\else
\begin{reduction*}[$\star$]
\fi
\label{red:in_out}
  Let $Q$ be a component of $G - B$. If $Q \cap O \ne \emptyset$, set $O = O \cup \cnei{Q}$. If $\cnei{Q} \cap I \ne \emptyset$, set $I = I \cup Q$. If both cases apply, the instance has no solution.
\end{reduction*} 
\iflong
\begin{proof}[Proof of Safeness]
  For the first case, assume $Q \cap O \ne \emptyset$. Notice that if any vertex of $\cnei{Q}$ is in the solution, there also has to be some vertex of $Q$ in the neighborhood of the solution since $Q$ is weakly connected. This, however, is not possible as $Q \cap B = \emptyset$.

  Similarly, for the second case, assume $\cnei{Q} \cap I \ne \emptyset$. If any vertex of $Q$ is not in the solution, there has to be a vertex of $Q$ in the neighborhood of the solution since $Q$ is weakly connected, which again is impossible. 
  
  Therefore, we can safely apply the first two cases. Since in the last case $I \cap O \ne \emptyset$, there can clearly be no solution.
\end{proof}
\fi

If this reduction rule is no longer applicable, every component in $G-B$ is either completely in $I$, completely in $O$, or intersects with none of the two. %We use that for the next reduction rule.
% See \Cref{fig:iob} for an illustration of how instances can be structured now.

% \begin{figure}[t]
  % \jonas{moved into other figure}
  % \begin{minipage}[c]{0.45\linewidth}
      % \centering
      % \includegraphics[width=0.3\textwidth]{figures/IOB_fig}
      % \caption{A visualization of how the sets $I$, $O$, and $B$ can overlap after the application of \Cref{red:in_out}. Any component in $G-B$ can be in $I$, $O$, or none of them, but not both. The black vertices form a feasible solution.}\label{fig:iob}
  % \end{minipage}
  % \jonas{not so important}
  % \hfill
  % \begin{minipage}[c]{0.45\linewidth}
  %     \centering
  %     \includegraphics[width=0.8\textwidth]{figures/scc_only_conn}
  %     \caption{A visualization of why source and sink sets of a component are important to consider. If we removed $Q_2$, the black vertices together with $Q_1$ would form a SCS that does not correspond to a SCS with $Q_2$.}\label{fig:scc_not_conn}
  % \end{minipage}
% \end{figure}

% \jonas{no longer necessary with the new equivalence def}
% \iflong
% \begin{reduction*}
% \else
% \begin{reduction*}[$\star$]
% \fi
% \label{red:contract_strong}
%   Let $Q$ be a strongly connected component of $G - B$ on which \Cref{red:in_out} is not applicable. Contract $Q$ into a single vertex $q$ with weight $\w{q} = \w{Q}$ and remove multi-edges.
% \end{reduction*}
% \iflong
% \begin{proof}[Proof of Safeness]
%   By \Cref{red:in_out}, it is impossible to include vertices in $Q$ into the neighborhood or to include part of $Q$ into the solution. Therefore, any solution that does not include all of $Q$ remains unchanged.

%   If a solution includes all of $Q$, using $q$ instead does not change the neighborhood size nor the weight, due to the contraction. Also, the solution stays strongly connected since $\outNei{q} = \outNei{Q}$ and $\inNei{q} = \inNei{Q}$. It also cannot create additional solutions since instead of including $q$, we could have included $Q$ because there are paths in $\induced{G}{Q}$ from every vertex to every other one.
% \end{proof}
% \fi

%\jonas{also no longer necessary?}
% Later, we will apply our theory from the previous subsection and consider weakly connected components $Q$ of $G-B$ as extensions of $G-Q$. \todo{Not clear to me.} Therefore, we remove $Q$ if it includes source and sink vertices. 


From this point, we will use extensions from the previous section for components of $G-B$. Namely, for a component $Q$ of $G-B$, let $E_{BQ}$ be all the edges with exactly one endpoint in $B$ and one in $Q$. Then, $(\induced{G}{Q}, E_{BQ})$ defines an extension of $G-Q$. For simplicity, we also refer to this extension as $(Q, E_{BQ})$. Hence, we also use $\sourceOp$, $\sinkOp$, and $\connOp$, as well as $\compOp_{G-Q}$ for these extensions. 

The next reduction rule identifies a condition under which a component $Q$ can never be part of a solution, namely, if $Q$ include strongly connected components with no in-neighbors or no out-neighbors, which is exactly the case if the empty set is in $\source{Q,E_{BQ}}$ or $\sink{Q,E_{BQ}}$.

\iflong
\begin{reduction*}
\else
\begin{reduction*}[$\star$]
\fi
\label{red:scc_no_sources_or_sinks}
  Let $Q$ be a component of $G - B$ such that $\induced{G}{Q}$ is not strongly connected. If $\emptyset \in \source{Q,E_{BQ}} \cup \sink{Q,E_{BQ}}$, include $Q$ into $O$. 
\end{reduction*}
\iflong
\begin{proof}[Proof of Safeness]
    Since $\induced{G}{Q}$ is not strongly connected, $Q$ cannot be a solution by itself. Suppose $\emptyset \in \source{Q,E_{BQ}}$, the other case is analogous. Then, there is a source component $C$ in $\scc{\induced{G}{Q}}$ that has no incoming edge, neither in $\induced{G}{Q}$, nor in $E_{BQ}$. 
    Therefore, no vertex $v \in B$ can reach $C$.
    Because $Q$ can only be included as a whole and together with other vertices in $B$, $Q$ cannot be part of a solution.
\end{proof}
\fi

Note that after \Cref{red:in_out,red:scc_no_sources_or_sinks} have been applied exhaustively, every component $Q$ of $G - B$ is acyclic. Furthermore, every source in $Q$ has incoming edges from $B$, and every sink in $Q$ has outgoing edges to $B$. Finally, we have one more simple rule, which removes vertices $v \in O \setminus B$. It relies on the fact that by \Cref{red:in_out}, we also have $\nei{v} \subseteq O$.

\begin{reduction*}\label{red:remove_out}
  If \Cref{red:in_out} is not applicable, remove $O \setminus B$ from $G$.
\end{reduction*}
The previous reduction rules were useful to remove trivial cases and extend $I$ and $O$.
From now on, we assume that the instance is exhaustively reduced by \Cref{red:remove_out,red:scc_no_sources_or_sinks,red:in_out}.
Therefore, any component of $G - B$ is either contained in $I$ or does not intersect $I$ and $O$.

The next two rules will be twin type reduction rules that allow us to bound the number of remaining components.
If there are two components of $G-B$ that form equivalent extensions it is enough to keep one of them, since they fulfill the same role in forming a strongly connected subgraph. The reduction rules rely on \Cref{lem:source_sink_conn_equiv,lem:union_equiv_if_equiv} to show that equivalent extensions can replace each other and can be added to any solution.

\iflong
\begin{reduction*}
\else
\begin{reduction*}[$\star$]
\fi
\label{red:scc_twins}
  Let $Q_1, Q_2$ be components of $G - B$ such that both $\induced{G}{Q_1}$ and $\induced{G}{Q_2}$ are not strongly connected and $(Q_1, E_{BQ_1})$ and $(Q_2, E_{BQ_2})$ are equivalent. Delete $Q_2$ and increase the weight of some $q \in Q_1$ by $\w{Q_2}$. If $Q_2 \cap I \ne \emptyset$, set $I = I \cup Q_1$.
\end{reduction*}
\iflong
\begin{proof}[Proof of Safeness]
  By the previous reduction rules, components of $G-B$ can only be included as a whole or not at all. Notice that since $\conn{Q_1, E_{BQ_1}} = \conn{Q_2, E_{BQ_2}}$ and \Cref{red:scc_no_sources_or_sinks}, we get $\nei{Q_1} = \nei{Q_2}$. Let $S$ be a solution to the old instance. We differentiate some cases.

  If $S \cap (Q_1 \cup Q_2) = \emptyset$, we know that also $\nei{S} \cap (Q_1 \cup Q_2) = \emptyset$. Therefore, the neighborhood size and strong connectivity of $S$ do not change in the new instance, and it is also a solution.
  If $S$ includes only one of $Q_1$ and $Q_2$, assume without loss of generality $S \cap (Q_1 \cup Q_2) = Q_1$, since $Q_1$ is not strongly connected by itself, the solution must include vertices of $B$. Because $\nei{Q_1} = \nei{Q_2}$, the solution must also include $Q_2$, a contradiction.
  If $S$ includes both $Q_1$ and $Q_2$, we claim that $S' \coloneqq S \setminus Q_2$ is a solution for the new instance. The neighborhood size and weight clearly remain unchanged. %For the strong connectivity, suppose that there are $u,v \in S'$ such that the $u$-$v$-path in $S$ uses vertices from $Q_2$. Such a subpath can only connect one vertex in $B$ to another one in $B$, and since $\conn{Q_1} = \conn{Q_2}$, we can use a connection via $Q_1$ instead. Therefore, $S'$ is also strongly connected and a solution of the same weight.\jonas{use union equiv lemma}
  Strong connectivity follows by \Cref{lem:union_equiv_if_equiv} and \Cref{lem:source_sink_conn_equiv}.
  The last two cases also show that if a solution had to include at least one of $Q_1$ and $Q_2$, that is, $(Q_1 \cup Q_2) \cap I \ne \emptyset$, any solution for the reduced instance must also include $Q_1$. Therefore, the adaptation to $I$ is correct.

  Let $S$ be a solution to the reduced instance. Again, if $S$ does not include vertices from $Q_1$, then $S$ will immediately be a solution to the old instance. If $Q_1 \subseteq S$, then $S' \coloneqq S \cup Q_2$ will be a solution to the old instance by \Cref{lem:union_equiv_if_equiv} and \Cref{lem:source_sink_conn_equiv}.
\end{proof}
\fi

For strongly connected components, the rule is different, since we have to acknowledge the fact that they can be a solution by themselves. For every such component we destroy strong connectivity of smaller-weight equivalent component, which can then be reduced by \Cref{red:scc_twins}.

% \jonas{moved next to other figure}
% \begin{figure}[t]
%   \centering
%   \includegraphics[width=0.4\textwidth]{figures/scc_only_conn}
%   \caption{A visualization of why source and sink sets of a component are important to consider. If we were to remove $Q_2$, the black vertices together with $Q_1$ would form a strongly connected subgraph that does not correspond to a strongly connected subgraph with $Q_2$.}\label{fig:scc_not_conn}
% \end{figure}

% \jonas{Not important, maybe I was the only one wondering}
% Intuitively, one might ask whether it is also enough to apply \Cref{red:scc_twins} if only $\conn{Q_1} = \conn{Q_2}$ when the source and sink sets can be different, especially since the proof hides this detail in \Cref{lem:union_equiv_if_equiv}. See \Cref{fig:scc_not_conn} for an example of why just $\conn{Q_1} = \conn{Q_2}$ is not enough.

\iflong
\begin{reduction*}
\else
\begin{reduction*}[$\star$]
\fi
\label{red:scc_twins_single}
  Let $Q_1, Q_2$ be components of $G - B$ that are also strongly connected with $(Q_1, E_{BQ_1})$ and $(Q_2, E_{BQ_2})$ equivalent and $\w{Q_1} \ge \w{Q_2}$.
  Add a vertex $q_2'$ with edges $\died{q_2}{q_2'}$ and $\died{q_2'}{v}$, for all $q_2 \in Q_2$ and $v \in \nei{q_2}$. Set $\w{q_2'} = 0$.
\end{reduction*}
\iflong
\begin{proof}[Proof of Safeness]
  Let $S$ be a solution of the old instance.
  If $S \cap (Q_1 \cup Q_2) = \emptyset$, then $S$ is also a solution for the new instance.
  If $S$ includes only one of $Q_1$ and $Q_2$, then we must have $S \in \set{Q_1, Q_2}$, so $S' \coloneqq Q_1$ is a solution for the new instance with $\w{S'} \ge \w{S}$.
  If $S$ includes both $Q_1$ and $Q_2$, adding $q_2'$ to $S$ obviously gives a solution of the same weight.

  For a solution of the reduced instance $S$, we can simply remove $q_2'$ if it is inside for a solution to the old instance.
\end{proof}
\fi

Using both \Cref{red:scc_twins,red:scc_twins_single} exhaustively makes sure that there are at most two components per extension equivalence class left.
The last rule compresses the remaining components to equivalent components of bounded size.

% \iflong
\begin{reduction*}
% \else
% \begin{reduction*}[$\star$]
% \fi
\label{red:scc_comp}
  Let $Q$ be a component of $G - B$ that is not equal to its compressed extension. Replace $Q$ by its equivalent compressed extension and set the weight such that $\w{Q'} = \w{Q}$. If $Q \cap I \ne \emptyset$, set $I = I \cup Q'$.
\end{reduction*} 
% \iflong
\begin{proof}[Proof of Safeness]
  Since $Q$ is not strongly connected, it can only be part of a solution that includes some vertices from $G-Q$. Thus, we can apply \Cref{lem:comp_weakly_conn_equiv}, and the old instance and the new instance have exactly the same strongly connected subgraphs. Because of $\conn{Q, E_{BQ}} = \conn{Q', E_{BQ'}}$ and \Cref{red:scc_no_sources_or_sinks}, we get $\nei{Q} = \nei{Q'}$. Since $\w{Q'} = \w{Q}$, the rule is safe.
\end{proof}
% \fi

This finally allows us to bound the size of $G-B$. In the next lemma, we summarize the progress of our reduction rules and apply the bounds from the previous section. Note that we need the stronger bound using the neighborhood since of the components instead of simply $B$.

\iflong
\begin{lemma}
\else
\begin{lemma}[$\star$]
\fi
\label{lem:scc_all_reductions}
  Let $\mathcal{Q}$ be a set of components of $G - B$ with total neighborhood size $h \coloneqq \abs{\bigcup_{Q \in \mathcal{Q}} \nei{Q}}$. Then we can reduce the instance, or there are at most $2^{2^{h+1} + h^2}\left(2^{h+1} + h^2\right) = 2^{\bigO{2^h}}$ vertices in $\mathcal{Q}$ in total.
\end{lemma}
\iflong
\begin{proof}
  After applying \Cref{red:scc_twins,red:scc_comp} exhaustively, there can be at most one copy of every compressed extension among the components in $\mathcal{Q}$.
  Since the compressed component only depends on the neighborhood of a component, we can treat $\bigcup_{Q \in \mathcal{Q}} \nei{Q}$ as the base graph for our extensions. By \Cref{lem:range_size_comp}, there are at most $\abs{\mathcal{Q}} \le 2^{2\cdot 2^h + h^2}$ components. Also, each component has at most $2^{h+1} + h^2$ vertices, which immediately gives the claimed bound on the total number of vertices in $\mathcal{Q}$.
\end{proof}
\fi

\iflong
This above result is only useful if we can guarantee that applying the reduction rules exhaustively in the correct order will always terminate. Additionally, we have to ensure that we can verify the conditions for every reduction rule quickly enough. Both statements are proven in the following lemma.

\iflong
\begin{lemma}
\else
\begin{lemma}[$\star$]
\fi
\label{lem:red_rules_poly}
  Executing the reduction rules in the proposed order guarantees termination after $\bigO{n}$ applications. The total execution takes at most $n^{\bigO{1}}$ time.
\end{lemma}
\iflong
\begin{proof}
  Most reduction rules clearly make progress, either by increasing the size of $I$ or $O$ or by decreasing the number of vertices in $G$. While compression does not always decrease the number of vertices, it will only be applied once per component. Finally, \Cref{red:scc_twins_single} decreases the number of strongly connected components in $G-B$, which also will never be increased again by \Cref{lem:comp_weakly_conn_equiv}. All of these progress measures are bounded by $n$, proving the first claim.

  For the second part, we only have to bound the runtime of a single application of a reduction rule. Most rules are clearly executable in polynomial time. The total size of the source, sink, and connection sets are also bounded by $\abs{\e{G}}$ and thus computable in polynomial time. Therefore, the compressed component can also be constructed in polynomial time. Note that this increases the size of our graph, but if we apply \Cref{red:scc_comp} only after all other reduction rules are completed, the compressed component never has to be considered again.
\end{proof}
\fi
\fi

\iflong
We use one more lemma from~\cite{golovach2020finding} that helps us to find a balanced separator.

\begin{lemma}[\protect{\cite[Lemma~1]{golovach2020finding}}]\label{lem:separation_algo}
  Given an undirected graph $G$, there is an algorithm with runtime $2^{\bigO{\min\{q,k\}\log(q+k)}}n^{\bigO{1}}$ that either finds a $(q,k)$-separation of $G$ or correctly reports that $G$ is $((2q+1)q2^k, k)$-unbreakable.
\end{lemma}
\fi

Now, we have all that it takes to solve our intermediate problem.% We use the same structure as~\cite{golovach2020finding} in their algorithms.
%
The main idea of the algorithm is to shrink $B$ to a bounded size, by solving the problem recursively. Once $B$ is bounded, we apply our reduction rules by viewing components of $G-B$ as extensions, removing redundant equivalent extensions and compressing them. Thereby, we also bound the size and number of components of $G-B$ in terms of $\abs{B}$ using \Cref{lem:scc_all_reductions}. By choosing suitable constants, we can show that this decreases the total size of $G$, which will make progress to finally reduce it to the unbreakable case.

\iflong
\begin{theorem}
\else
\begin{theorem}[$\star$]
\fi
\label{thm:border_scc_fpt}
  \scsborder{} can be solved in time $2^{2^{2^{\bigO{k^2}}}}n^{\bigO{1}}$.
\end{theorem}

\iflong
\begin{proof}

\begin{algorithm}[t]
 % \caption{The algorithm for \scsborder{}, using the same structure as the algorithms from~\cite{golovach2020finding}.} 
 \label{alg:rec_und_scc}
   \caption{Algorithm \scsborder{}}
  \DontPrintSemicolon
  \SetKwFunction{FMain}{Solve}
  \SetKwProg{Fn}{def}{:}{}
  \Fn{\FMain{$G$, $I$, $O$, $B$, $\wOp$, $k$, $T$}}{
    $q\gets 2^{2^{ck^2}}$ for a suitable constant $c$\;
    \eIf{$G$ is $((2q+1)q2^k,k)$-unbreakable}
    {
      \KwRet solve the problem using \Cref{thm:unbreakable_scc}\;
    }{
      $(U,W) \gets $ $(q,k)$-separation of $G$ with $\abs{T \cap W \setminus U} \le k$\;
      $(\tilde{G}, \tilde{I}, \tilde{O}, \tilde{B}, \tilde{\wOp}, k, \tilde{T}) \gets $ restriction to $W$ with $\tilde{T} = (T \cap W) \cup (U \cap W)$\;
      $\mathcal{R} \gets $ \FMain{$\tilde{G}$, $\tilde{I}$, $\tilde{O}$, $\tilde{B}$, $\tilde{\wOp}$, $k$, $\tilde{T}$}\;
      $\mathcal{N} \gets \tilde{T} \cup \bigcup_{R \in \mathcal{R}} \nei{R} \cap W$\;
      $\hat{B} \gets (B \cap U) \cup (B \cap \mathcal{N})$\;
      $(G^*, I^*, O^*, \hat{B}^*, \wOp^*, k^*, T^*) \gets $ reduce $(G,I,O,\hat{B},\wOp,k,T)$ with \Cref{lem:scc_all_reductions}\;
      \KwRet \FMain{$G^*$, $I^*$, $O^*$, $\hat{B}^*$, $\wOp^*$, $k^*$, $T^*$}\;
    }
  }
\end{algorithm}

  Let $\mathcal{I} = (G, I, O, B, \wOp, k, T)$ be our \scsborder{} instance. See \Cref{alg:rec_und_scc} for a more compact description of the algorithm. A high level display of the approach can be found in \Cref{fig:recursive_calls}.
  Define $q = 2^{2^{2^{ck^2}}}$ for a constant $c$. We later show that a suitable $c$ must exist.
  \zzcommand{\neighbor}{\mathcal{N}}

  First, we run the algorithm from \Cref{lem:separation_algo} on the underlying undirected graph of $G$ with $q$ and $k$. If it is $((2q+1)q2^k, k)$-unbreakable, we solve the instance directly using the algorithm from \Cref{thm:unbreakable_scc}.

  Therefore, assume that we have a $(q,k)$-separation $(U,W)$. Without loss of generality, since $\abs{T} \le 2k$ we can assume that $\abs{T \cap W \setminus U} \le k$. Thus, we can construct a new instance to solve \emph{the easier side} of the separation. Take $\tilde{G} = \induced{G}{W}, \tilde{I} = I \cap W, \tilde{O} = O \cap W, \tilde{B} = B \cap W$, write $\tilde{\wOp}$ for the restriction of $\wOp$ to $W$, and set $\tilde{T} = (T \cap W) \cup (U \cap W)$. Since $\abs{U \cap W} \le k$, also $\abs{\tilde{T}} \le 2k$ holds and $\tilde{I} \coloneqq (\tilde{G}, \tilde{I}, \tilde{O}, \tilde{B}, \tilde{\wOp}, k, \tilde{T})$ is a valid instance, which we solve recursively.

  Let $\mathcal{R}$ be the set of solutions found in the recursive call. For $R \in \mathcal{R}$, define $N_R = \nei{R} \cap W$. Define $\mathcal{N} = \tilde{T} \cup \bigcup_{R \in \mathcal{R}} N_R$. 

  We now restrict $B$ in $W$ to use only vertices in the neighborhood that have been neighbors in a solution in $\mathcal{R}$, that is, only vertices in $\mathcal{N}$. Define $\hat{B} = (B \cap U) \cup (B \cap \mathcal{N})$. We now replace $B$ in our instance with $\hat{B}$ and apply all our reduction rules exhaustively to arrive at the instance $(G^*, I^*, O^*, \hat{B}^*, \wOp^*, k^*, T^*)$. Finally, we also solve this instance recursively and return the solutions after undoing the reduction rules. 

  \subparagraph*{Correctness.}
  We already proved that the reduction rules and the algorithm for the unbreakable case are correct. The main statement we have to show is that we can replace $B$ with $\hat{B}$ without throwing away important solutions. That means that the instances $(G,I,O,B,\wOp,k,T)$ and $\hat{\mathcal{I}} \coloneqq (G,I,O,\hat{B},\wOp,k,T)$ are equivalent in the sense that any solution set for one instance can be transformed to a solution set to the other instance with at least the same weights. This justifies solving $\hat{\mathcal{I}}$ instead of $\mathcal{I}$.
  Consider the boundary complementation  $\mathcal{I}' \coloneqq (G',I',O',B',\wOp',k')$ and $\hat{\mathcal{I}}' \coloneqq (G',I',O',\hat{B}',\wOp',k')$ that are caused by the same $(X,Y,Z,R)$.
  To show the claim, we consider the two directions. Any solution for $\hat{\mathcal{I}}'$ is immediately a solution for the same boundary complementation for $\mathcal{I}'$ since the only difference is that we limit the possible neighborhood to a subset of $B$.

  For the other direction, consider a solution $S$ to $\mathcal{I}'$, and we want to show that there is a solution $\hat{S}$ to $\hat{\mathcal{I}}'$ using only vertices of $\hat{B}'$ in the neighborhood with $\w{\hat{S}} \ge \w{S}$. If $S \cap W = \emptyset$, then $S$ is also immediately a solution for $\hat{\mathcal{I}}'$. Therefore, assume $S \cap W \ne \emptyset$. Define $\tilde{X} \coloneqq \tilde{T} \cap S$, $\tilde{Y} \coloneqq \tilde{T} \cap \neiG{G - (W\setminus U)}{S}$, and $\tilde{Z} \coloneqq \tilde{T} \setminus (\tilde{X} \cup \tilde{Y})$. Let $\tilde{R}$ be the set of $(a,b) \in \tilde{X} \times \tilde{X}$ such that there is an $a$-$b$-path in $\induced{G}{S \setminus {W \setminus U}}$. Finally, let $\tilde{k} \coloneqq \abs{\nei{S} \cap W}$. Thus, we can construct a boundary complementation instance $(\tilde{G}', \tilde{I}', \tilde{O}', \tilde{B}', \tilde{w}', \tilde{k})$ from $(\tilde{X}, \tilde{Y}, \tilde{Z}, \tilde{R})$. One can easily verify that $(S \cap \ve{\tilde{G}'}) \cup \set{u_r}{r \in \tilde{R}}$ is a feasible solution to this instance. Furthermore, the maximum solution $\tilde{S} \in \mathcal{R}$ to this instance gives a new set $\hat{S} \coloneqq (\tilde{S} \cap W) \cup (S \cap U) \subseteq \ve{G'}$ that has at least the same weight as $S$. We also know that $\nei{\hat{S}} \subseteq \hat{B}'$ and $\abs{\nei{\hat{S}}} \le k$. See \Cref{fig:scc_bc} for a visualization of the construction corresponding to a solution $S$.


  All that remains is to verify that $\hat{S}$ is strongly connected. For $v_1, v_2 \in \tilde{S} \cap W$, we can simply use the $v_1$-$v_2$-path in $\tilde{S}$, replacing subpaths via $u_r$ for $r \in \tilde{R}$ with the actual paths in $S \cap U$. If $v_1 \in \tilde{S} \cap W$ and $v_2 \in S \cap U$, we can walk to any $x \in \tilde{X}$ which must have a path to $v_2$ in $S$, in which may need to replace subpaths via $W$. This can be done, since $\tilde{S} \cap W$ connects all pairs of $x_1, x_2 \in X$ that are not connected via $S \cap U$. The two remaining cases follow by symmetry, justifying the replacement of $B$ with $\hat{B}$.

  Finally, we have to show that the recursion terminates for the right choice of $c$; that is, both recursively solved instances have strictly smaller sizes than the original graph.
  For the first recursive call, note that the boundary complementation adds at most $k^2 \le q$ vertices to $\tilde{G}$. Since $\abs{U \setminus W} > q$, we have $\abs{\ve{\tilde{G}}} < \abs{\ve{G}}$.

  For the second recursive call, since $\abs{\tilde{T}} \le 2k$ and by \Cref{lem:number_border_complementations}, we have $\abs{\mathcal{N}} \le 2k + (k+1)k2^{c_1k^2} \le 2^{c_2k^2}$ for constants $c_1$ and $c_2$.
  After applying all our reduction rules, by \Cref{lem:scc_all_reductions} for a suitable choice of $c_3$ and $c$ we get \[\abs{W^*} \le \abs{\neighbor} + 2^{2^{\abs{\neighbor}+1} + \abs{\neighbor}^2}\left(2^{\abs{\neighbor}+1} + \abs{\neighbor}^2\right) \le 2^{c_3 2^{\abs{\neighbor}}} \le 2^{c_32^{2^{c_2k^2}}} \le 2^{2^{2^{ck^2}}} \eqqcolon q.\]
  Since before the reductions, we had $\abs{W \setminus U} > q$, the reduced graph $G^*$ also has fewer vertices than $G$. Therefore, this recursive call also makes progress and the recursion terminates.

  \subparagraph*{Runtime.} 
  We follow the analysis of~\cite{chitnis2016designing}. With \Cref{lem:separation_algo}, we can test
  if the undirected version of $G$ is unbreakable in time $2^{k\log{q+k}}n^{\bigO{1}} \le 2^{2^{2^{\bigO{k^2}}}}n^{\bigO{1}}$.
  If the graph turns out to be unbreakable, the algorithm of \Cref{thm:unbreakable_scc} solves it in time $2^{\bigO{k^2\log{q}}} \le 2^{2^{2^{\bigO{k^2}}}}n^{\bigO{1}}$.
  Executing the reduction rules takes polynomial time in $n$ by \Cref{lem:red_rules_poly}. 

  All that is left to do is analyze the recursion. Let $n' \coloneqq \abs{W}$. By the separation property, we know that $q < n' < n-q$. From the correctness section, we get $\abs{\ve{G^*}} \le \abs{\ve{G}} - n' + q$. 
  Note that the recursion stops when the original graph is $(q,k)$-unbreakable, which must be the case if $\abs{\ve{G}} \le 2q$.
  We arrive at the recurrence 
  \[
    T(n) = \begin{cases}2^{2^{2^{\bigO{k^2}}}}, & \text{for $n \le 2q$;}\\
    \left(\max_{q < n' < n-q} T(n' + k^2) + T(n - n' + q)\right) + 2^{2^{2^{\bigO{k^2}}}}n^{\bigO{1}},  & \text{otherwise.} \end{cases}
  \]
  Notice that $2^{2^{2^{\bigO{k^2}}}}$ appears in every summand of the expanded recurrence and can be ignored here and multiplied later. Furthermore, $n^{\bigO{1}}$ is bounded from above by a convex polynomial. Therefore, it is enough to consider the extremes of the maximum expression. For $n' = q+1$, the first recursive call evaluates to $T(q+1+k^2) \le T(2q) \le 2^{2^{2^{\bigO{k^2}}}}$. The second call only introduces an additional factor of $n$. For $n' = n-q-1$, the second expression evaluates to $T(2q+1)$ which is clearly also bounded by $2^{2^{2^{\bigO{k^2}}}}$. Thus, we arrive at the final runtime of $2^{2^{2^{\bigO{k^2}}}}n^{\bigO{1}}$.
\end{proof}
\fi

Finally, we can use \scsborder{} to solve \scs{}. Since in our original problem every vertex could be part of the solution or its neighborhood, we initially set $I \coloneqq O \coloneqq \emptyset$ and $B \coloneqq \ve{G}$. Furthermore, we only want to consider the boundary complementation that changes nothing, which we achieve by setting $T \coloneqq \emptyset$.

% \begin{restatable*}[]{theorem}{test}
% \label{thm:scc_algo}
%   \textsc{Total-Secluded Strongly Connected Subgraph} is solvable in time $2^{2^{2^{\bigO{k^2}}}}n^{\bigO{1}}$.
% \end{restatable*}
\test*

% \jonas{moved to appendix}
% \section{Secluded \texorpdfstring{$d$}{d}-edge connected subgraph} \label{sec:d_edge}
% %\section{Secluded \texorpdfstring{$d$}{d}-Edge-Connected Subgraphs}\label{sec:d_edge}

In this section, we consider a similar undirected problem to strong connectivity. In many ways 2-edge-connectivity is a comparable concept, since strong connectivity also requires at least two different paths between every pair of vertices. We target a generalized version of the problem and show that the following problem is non-uniformly FPT with a similar recursive understanding approach as in \Cref{sec:scc}.

\zzcommand{\prob}{\textsc{Secluded $d$-ECS}}
\begin{tcolorbox}[enhanced,title={\color{black} {\textsc{Secluded $d$-Edge-Connected Subgraph}$~$ (\prob{})}}, colback=white, boxrule=0.4pt,
	attach boxed title to top left={xshift=.3cm, yshift*=-2.5mm},
	boxed title style={size=small,frame hidden,colback=white}]
	
	\textbf{Input:}  
  An undirected graph $G$, a weight function, $\wOp \colon \ve{G} \to \N$, and integers $w, k \in \N$\\

	\textbf{Output:}
  Decide if there is a set $S \subseteq \ve{G}$ with weight $\w{S} \ge w$ and neighborhood size $\abs{\nei{S}} \le k$, such that $\induced{G}{S}$ is $d$-edge-connected.
\end{tcolorbox}

Note that $d$ is part of the problem and not the instance and can therefore be treated as a constant in the algorithm.
In an analogous fashion to \Cref{thm:total_scc_np_hard}, we can prove NP-hardness for $d > 1$. Because the reduction is so similar, we omit the proof.
\begin{theorem}\label{thm:d_edge_np_hard}
  \prob{} is NP-hard, for all $d > 1$.
\end{theorem}

Compared to \textsc{Total-Secluded SCS}, the algorithm for \prob{} shares some similarities. The auxiliary problems are defined analogously, and we again need a definition of border complementations. Furthermore, the general algorithm structure is the same. Again, we use several reduction rules that decrease the size of everything but the allowed neighborhood set $B$. Before we define the reduction rules, we work with extensions and use an equivalence relation with a similar intuition. 

However, executing this technically is more involved than before, as are the reduction rules.
Moreover, this section will be structured slightly differently than the previous one. First of all, this is done to avoid unnecessary repetition of the same concepts. Second, this is due to the fact that we lack some proof ingredients that are necessary for a concrete analysis of the parameter dependence. We still prove that \prob{} is solvable in \emph{non-uniform} FPT-time, that is, there is a constant $c$ such that for all values the parameter $k \in \N$ can take, we can find an algorithm with runtime $\bigO{n^c}$. This is still a nontrivial result, as $d$-edge-connected graphs can generally not be classified by a set of forbidden induced subgraphs or anything else that was solved before. This result also indicates that the problem is very likely to be uniformly FPT~\cite[Chapter~6]{cygan2015parameterized}.

We start out in \Cref{sec:ext_equiv} by defining a problem-specific equivalence relation on extensions, which will be used in a similar way as in \Cref{sec:scc}. After that, we define border complementations together with auxiliary problems in \Cref{sec:d_edge_bc} using extensions and the equivalence relation. In \Cref{sec:d_edge_unbreak}, we give an algorithm for the base case in which the graph is $(q,k)$-unbreakable. Finally, we give the necessary reduction rules in \Cref{sec:d_edge_algorithm}, and explain how to stick all pieces together.

\subsection{Extension Equivalence}\label{sec:ext_equiv}

Again, we work with graph extensions as in \Cref{sec:scc_extensions}. Naturally, for an undirected graph $G$, we define an extension to be a pair $\hexpair$, where $H$ is an undirected graph and $\hexset \subseteq \set{\set{h,v}}{h \in \ve{H}, v \in \ve{G}}$. As before, we can stick the extension onto the graph $G$ to form the undirected extended graph $\ex{G}{H}{\hexset} \coloneqq (\ve{G} \cup \ve{H}, \e{G} \cup \e{H} \cup \hexset)$. Intuitively, extensions should be thought of as \emph{graph add-ons}, a smaller graph $H$ that can be attached to some part of a larger graph $G$. How exactly $H$ should be attached to $G$ is defined by the additional set $\hexset$.

Naturally, since we aim to use an extension as part of a $d$-edge-connected induced subgraph, our equivalence relation will then works by counting the existing and potential paths through the extension. To make working with paths easier, we use the following definition. 

\begin{definition}[Path via $Q$]
  Let $P$ be a path in $G$ and let $Q \subseteq \ve{G}$. We call $P$ a \emph{path via $Q$} if every edge of $P$ has at least one endpoint in $Q$ and every intermediate vertex is in $Q$.
\end{definition}

Note that this definition is equivalent to disallowing intermediate vertices outside of $Q$ and disallowing the direct edge between two vertices $u, v \notin Q$ as a path via $Q$.

First, we want measure how an extension helps in forming $d$-edge-connected induced subgraphs with $G$, which we capture with the following definition. This is the analogue of the connection set $\connOp$ for strongly connected subgraphs. In this case, however, it is more involved. We use a \emph{demand function} that defines a demand of paths for vertex pairs in $G$ that should go via the extension in a solution. The extension then fulfills this demand if all of these demanded paths are present in the extension, and all of them are edge-disjoint. For a set $X$, we denote the set of all subsets of size 2 of $X$ with $\binom{X}{2}$.

\begin{figure}[t]
    \begin{minipage}[c]{0.45\linewidth}
    \centering
    \includegraphics[width=0.6\textwidth,page=1]{figures/d_edge}
    \caption{An example for fulfillment for $d = 2$. The extension $\hexpair$ fulfills the function that demands paths between $v_1$ and $v_3$ and $v_2$ and $v_4$. If we instead demand paths as in $f(\set{v_1, v_3}) = f(\set{v_1,v_4}) = 1$, then $f$ is not fulfilled by $\hexpair$ since there is only one edge incident to $v_1$.}
    \label{fig:fulfill}
    \end{minipage}
    \hfill
    \begin{minipage}[c]{0.45\linewidth}
    \centering
    \includegraphics[width=0.6\textwidth,page=2]{figures/d_edge}
    \caption{An example for covering and sufficiency for $d = 2$. The function that maps only $f(\set{v_2,v_3}) = 1$ covers, among others, the pairs $\set{h_2,h_3}$ and $\set{h_2, v_2}$, but not $\set{h_1,h_2}$. Setting only $f(\set{v_1, v_3}) = 1$ is sufficient for $\hexpair$ and $\set{v_1, v_3}$.}
    \label{fig:sufficient}
    \end{minipage}
\end{figure}

\begin{definition}[Fulfillment]
  Let $\hexpair$ be an extension of $G$ and let $f \colon \binom{\ve{G}}{2} \to [0,d]$ be a function, which we call a \emph{demand function}. We say that $\hexpair$ \emph{fulfills} $f$ if for every distinct $v_1, v_2 \in \ve{G}$ there are $f(\set{v_1, v_2})$ paths between $v_1$ and $v_2$ via $\ve{H}$ such that \textbf{all of these paths} are edge-disjoint.
\end{definition} 

See \Cref{fig:fulfill} for an example of a fulfilled demand function.

The next definition considers the same kind of functions but from a different point of view. This time, the function can be thought of a compressed version of the rest of a potentially $d$-edge-connected subgraph, and we ask if this enough to connect a specific pair of vertices in $\ex{G}{H}{\hexset}$. To differentiate these two roles we name this kind of function \emph{supply function} although the signature of the function is the same. Think about this definition as measuring what requirements there are for a $d$-edge-connected induced subgraph such that it can include $\hexpair$. For strongly connected subgraph, this was the role of the source and sink sets $\sourceOp$ and $\sinkOp$.

\begin{definition}
  Let $\hexpair$ be an extension of $G$, and $f \colon \binom{\ve{G}}{2} \to [0,d]$ be a \emph{supply function}. Consider the graph $G' \coloneqq \ex{G}{H}{\hexset} - \e{G}$. Now insert $f(\set{v_1, v_2})$ edges between every $v_1, v_2 \in \ve{G}$. We say that $f$ \emph{covers} a pair of vertices $v_1, v_2 \in \ve{G'}$, if there are $d$ edge-disjoint paths between $v_1$ and $v_2$ in $G'$.
\end{definition}

The last definition formalizes which vertex pairs in $H$ have $d$-edge-connected paths, if the remaining solution looks like the supply function $f$.
To form a $d$-edge-connected subgraph that completely includes $\hexpair$, all pairs have to be covered. Only then is the remaining solution enough to be able to include $H$. We formalize this by a set of supply functions that the solution must be able to provide to be sufficient for the extension $\hexpair$.

\begin{definition}[Sufficiency]
  Let $\hexpair$ be an extension of $G$, $V' \subseteq \ve{G}$, and $F$ a set of supply functions. We say that $F$ is \emph{sufficient} for $\hexpair$ and $V'$ if for every $h_1, h_2 \in \ve{H}$, there is an $f \in F$ that covers $h_1, h_2$ and every $h\in \ve{H}, v \in V'$ is covered by some $f \in F$.
\end{definition}

See \Cref{fig:sufficient} for an example of covering a pair and a sufficient supply function.

Now, we define our equivalence relation on extensions for this problem. Similar to the previous section, the intuition for two components being equivalent is that including them into a $d$-edge-connected subgraph has the same benefits and the same requirements. In this case, the benefits are all demand functions that the extension fulfills. The requirements are all sufficient supply function sets. With this intuition, the next definition should come naturally.

% \jonas{moved next to other figure}
% \begin{figure}[t]
%   \centering
%   \includegraphics[width=0.3\textwidth,page=2]{figures/d_edge}
%   \caption{An example for covering pairs and sufficiency for $d = 2$. The function that maps only $f(\set{v_2,v_3}) = 1$ covers, among others, the pairs $\set{h_2,h_4}$, $\set{h_2,h_3}$, and $\set{h_2, v_2}$, but not $\set{h_1,h_2}$. The supply function that only maps $f(\set{v_1, v_3}) = 1$ is on its own sufficient for $\hexpair$ and $\set{v_1, v_3}$.}
%   \label{fig:sufficient}
% \end{figure}

\begin{definition}[Equivalence]\label{def:dequiv}
  Let $\hexpairi{1}, \hexpairi{2}$ be two extensions of $G$. We say that $\hexpairi{1}$ and $\hexpairi{2}$ are equivalent or $\hexpairi{1} \dequiv \hexpairi{2}$ if \begin{enumerate}[noitemsep,nolistsep]
    \item for all demand functions $f$, we have $\hexpairi{1}$ fulfills $f$ if and only if $\hexpairi{2}$ fulfills $f$,
    \item for all $V' \subseteq \ve{G}$ and supply function sets $F$, we have that $F$ is sufficient for $\hexpairi{1}$ and $V'$ if and only if it is sufficient for $\hexpairi{2}$ and $V'$.\qedhere
  \end{enumerate}
\end{definition}
\begin{observation}
  $\dequiv$ is an equivalence relation.
\end{observation}

To justify this definition, we have to show multiple properties. First of all, it has be computable in non-uniform FPT-time whether two components are equivalent.

Note that because of non-uniformity, the statement of the next lemma and others will be slightly odd, since we give a separate algorithm for every possible graph size $\abs{\ve{G}}$. In this case, this is due to the fact, the we use another non-uniform algorithm as a subroutine. Later, we use this kind of statement bound runtimes with unknown parameter dependence. For every constant parameter, the algorithm still runs in time bounded by the same polynomial, but we are unable to bound the $f(k)$ term.

\iflong
\begin{lemma}
\else
\begin{lemma}[$\star$]
\fi
\label{lem:check_equiv}
  For every $c \in \N$, there is an algorithm that checks for $\abs{\ve{G}} = c$ if two extensions $\hexpairi{1}$ and $\hexpairi{2}$ of $G$ are equivalent in time $\bigO{(\abs{\ve{H_1}} + \abs{\ve{H_2}})^4}$. 
\end{lemma}
\iflong
\begin{proof}
  For the first condition of \Cref{def:dequiv}, iterate over all $f$ in time $(d+1)^{\binom{c}{2}}$. For each $f$, we want to check if $\hexpairi{1}$ and $\hexpairi{2}$ fulfill $f$. To do this, we reduce to the edge-disjoint path problem, where you are given $p$ terminal pairs $(s_1, t_1), \ldots, (s_p, t_p)$ and want to find $p$ edge-disjoint paths, one between each pair.
  To reduce to this problem, for each $\uned{v_1}{v_2} \in \binom{\ve{G}}{2}$, we add $f(\uned{v_1}{v_2})$ leaves to $v_1$ and to $v_2$. For each such leaf pair $\ell_{v_1}, \ell_{v_2}$ we add one terminal pair between them.

  Now, we run the edge-disjoint path algorithm by \cite{robertson1995graph} that runs in time $\bigO{n^3}$ for a constant number of terminal pairs, where $n$ in our case is $\abs{\ve{H_1}}$ or $\abs{\ve{H_2}}$. Note that the number of terminal pairs is here is bounded by $\binom{c}{2}d$ and can thus be treated as constant.

  For the second condition of \Cref{def:dequiv}, iterate over all $V'$ and $F$ in time $2^{c}2^{(d+1)^{\binom{c}{2}}}$. Then, we check for all of the at most $(\abs{\ve{H_i}} + c)^2$ pairs if they are covered by a function in $F$. For each $f \in F$, we build $G'$ and insert the corresponding edges. Then, we can build the Gomory-Hu Tree~\cite{gomory1961multi} using $\abs{\ve{H_i}} + c$ max flow calculations with Orlin's algorithm~\cite{orlin2013max}. Then, we can easily read off the min cut value between each pair, which corresponds to the number of edge-disjoint paths. Since $c$ can be treated as a constant, this gives the total time $\bigO{\abs{\ve{H_i}}^4}$.
\end{proof}
\fi

\zzcommand{\classes}[1]{C(#1)}
\zzcommand{\classesc}[1]{C'(#1)}
Next, we want to show that the number of equivalence classes of $\dequiv$ can be bounded by a function of $\abs{\ve{G}}$, such that we can use it to bound the number of components.
To work with the equivalence classes, we want to know a smallest extension out of every non-empty equivalence class. So, for a graph $G$, pick an extension $\hexpair$ such that $\abs{\ve{H}}$ is minimum out of every non-empty equivalence class. We denote this set by $\classes{G}$.
Additionally, we consider $\classesc{G}$, defined in the same way as $\classes{G}$, except that we only consider extensions $\hexpair$ where $H$ is connected. Clearly, we have $\abs{\classesc{G}} \le \abs{\classes{G}}$.

\iflong
\begin{lemma}
\else
\begin{lemma}[$\star$]
\fi
\label{lem:number_equiv}
  There is $c \in \N$, such that the number of equivalence classes in $\dequiv$ is at most $\abs{\classes{G}} \le 2^{2^{2^{c\ve{G}^2}}}$. 
\end{lemma}
\iflong
\begin{proof}
  First, notice that there are $p \coloneqq (d+1)^{\binom{\abs{\ve{G}}}{2}}$ different supply or demand functions since for every pair in $\ve{G}$ we get to choose between values in $[0, d]$.
  Each class is now first identified by the set of demand functions it fulfills, of which there are $2^p$. Second, each subset $V'$ and supply function set $F$, of which there are $q \coloneqq 2^{\abs{\ve{G}}}2^p$, could form a sufficient combination. In total this make $2^p2^q$ equivalence classes, which gives the desired bound, since we treat $d$ as a constant.
\end{proof}
\fi

\iflong
\begin{lemma}
\else
\begin{lemma}[$\star$]
\fi
\label{lem:compute_classes}
  For every $c \in \N$, there is an algorithm that computes $\classes{G}$ and $\classesc{G}$ for all graphs $G$ with $\abs{\ve{G}} = c$ in constant time.
\end{lemma}
\iflong
\begin{proof}
  By \Cref{lem:number_equiv}, the number of equivalence classes is bounded by a function in $c$ and $d$. Therefore, the size of the smallest element of every non-empty equivalence class is also bounded by a function in $c$ and $d$; let us call this value $f(c,d)$. If we treat $c$ and $d$ as constants, we can simply generate all extensions with at most $f(c,d)$ vertices and sort them into equivalence classes using \Cref{lem:check_equiv}. Out of every non-empty class, we pick one extension with the minimum number of vertices for $\classes{G}$ and the minimum connected extension for $\classesc{G}$ if it exists.
\end{proof}
\fi

The crucial property of the equivalence relation $\dequiv$ is that two equivalent extensions form exactly the same $d$-edge-connected induced subgraphs. We prove this in \Cref{lem:equiv_d_edge}, justifying our definition. Later, this fact will allow us to replace components by equivalent components. Before we are able to prove this result, we need another lemma to simplify working with $d$-edge-connected subgraphs.

\iflong
\begin{lemma}
\else
\begin{lemma}[$\star$]
\fi
\label{lem:connect_paths}
  Let $G$ be a graph with $x,y,z \in \ve{G}$. If there are $d$ edge-disjoint paths between $x$ and $y$ and $d$ edge-disjoint paths between $y$ and $z$, there are also $d$ edge-disjoint paths between $x$ and $z$.
\end{lemma} 
\iflong
\begin{proof}
  The condition translates to the fact that the size of the min cut between $x$ and $y$ is at least $d$; the same hold for the min cut between $y$ and $z$. Consider the min cut between $x$ and $z$ and look at the graph with the min cut edges removed. If $y$ is not in a component with $x$ or $z$, consider a path from $y$ to either $x$ or $z$ that does not use vertices from the component of $x$ or $z$. This exists since $d \ge 1$. Then, we can remove all cut edges on the path from the cut, by the choice of the path, this will be a smaller cut, a contradiction. Therefore, without loss of generality $y$ lies in the same component as $x$ and the min cut between $x$ and $z$ is also a cut between $y$ and $z$. Hence it has size at least $d$, proving that there must be $d$ edge-disjoint paths between $x$ and $z$.
\end{proof}
\fi

Now, we are able to prove the crucial lemma about equivalent extensions. Note that this lemma is phrased in a way that will be slightly easier to apply, since we allow $G$ to have other extensions that cannot interact with our equivalent extensions. Later, the other extension $\hexpair$ will correspond to other components in $G-B$ that are also part of a solution.

\iflong
\begin{lemma}
\else
\begin{lemma}[$\star$]
\fi
\label{lem:equiv_d_edge}
  Let $G$ be an undirected graph with an extension $\hexpair$ and two equivalent extensions $\hexpairi{1} \dequiv \hexpairi{2}$. Consider $G' \coloneqq \ex{G}{H}{\hexset}$. Let $S \subseteq \ve{G'}$ such that $S \cup \ve{H_1}$ is $d$-edge-connected in $\ex{G'}{H_1}{\exset{G}{H_1}}$. Then, $S \cup \ve{H_2}$ is also $d$-edge-connected in $\ex{G'}{H_2}{\exset{G}{H_2}}$.
\end{lemma}
\iflong
\begin{proof}
  Define $V' \coloneqq \ve{G} \cap S$ and let $u, v \in S \cup \ve{H_2}$.
  If $u, v \in S$, consider the $d$ edge-disjoint paths between $u$ and $v$ in $S \cup \ve{H_1}$ with the subpaths $P_1, \ldots, P_{\ell}$ via $\ve{H_1}$. From these paths, we construct the corresponding demand function $f$ such that every path $P_i = v_{i,1}, \ldots, v_{i,2}$ contributes 1 to $f(\uned{v_{i,1}}{v_{i,2}})$. Then, clearly, $\hexpairi{1}$ must fulfill $f$ and so does $\hexpairi{2}$. Therefore, we can exchange $P_1$ to $P_{\ell}$ by edge-disjoint subpaths via $\ve{H_2}$ and there are $d$ edge-disjoint $u$-$v$-paths in $S \cup \ve{H_2}$.

  If $u, v \in \ve{H_2}$, we use the second property of equivalence. Consider all pairs $h \in \ve{H_2}, v \in V'$ and $h_1, h_2 \in \ve{H_2}$ with their $d$ respectively edge-disjoint paths in $S \cup \ve{H_1}$. We focus on the subpaths via $S$ and notice that we can construct a supply function for every such pair that covers it and corresponds to existing paths in $S$. Furthermore, this gives us a sufficient set $F$ for $\hexpairi{1}$ and $V'$ that must also be sufficient for $\hexpairi{2}$ and $V'$ by equivalence. Thus, there is $f \in F$ that covers $\uned{u}{v}$ in this case as well as the case $u \in \ve{H_2}$ and $v \in V'$.

  Finally, the only remaining case is $u \in \ve{H_2}$ and $v \in S \setminus V'$. Note that in this case there must be a $v \in V'$ since otherwise $S$ cannot be connected. Connectivity follows from the previous cases together with \Cref{lem:connect_paths}.
\end{proof}
\fi

\subsection{Border Complementations}\label{sec:d_edge_bc}

We define two auxiliary problems in the same way as in~\cite{golovach2020finding} and \Cref{sec:scc_bc}. We need them, to remember more information about the instance in recursive calls and work with a maximization problem instead of a decision problem.
\zzcommand{\probrec}{\textsc{Max \prob{}}}
\begin{tcolorbox}[enhanced,title={\color{black} {\probrec{}}}, colback=white, boxrule=0.4pt,
	attach boxed title to top left={xshift=.3cm, yshift*=-2.5mm},
	boxed title style={size=small,frame hidden,colback=white}]
	
	\textbf{Input:}  
  An undirected graph $G$, subsets $I,O,B \subseteq \ve{G}$, a weight function $\wOp \colon \ve{G} \to \N$, and an integer $k \in \N$\\

	\textbf{Output:}
  The maximum weight set $S \subseteq \ve{G}$ with $I \subseteq S$, $O \cap S = \emptyset$, $\nei{S} \subseteq B$, and $\abs{\nei{S}} \le k$, such that $\induced{G}{S}$ is $d$-edge-connected, or report that no feasible solution exists.
\end{tcolorbox}


We use our component equivalence definition to define border complementations in a similar way to~\cite{golovach2020finding} for finding $\mathcal{F}$-free secluded subgraphs, less explicitly than in \Cref{sec:scc_bc}. That is, we do not define the added vertices and edges directly, but use a small extension from an equivalence class of $\dequiv$.

\begin{definition}[Border Complementation]\label{def:d_edge_border_complementation}
  Let $(G,I,O,B,\wOp,k)$ be an instance for \probrec{} with a set $T \subseteq \ve{G}$ of \emph{border terminals}. A border complementation $(G',I',O',B,\wOp',k')$ is an instance obtained in the following way. Let $X,Y,Z$ be a partition of $T$ and let $\hexpair \in \classes{\induced{G}{X}}$. Additionally, we have that
  \begin{enumerate}[noitemsep,nolistsep]
    \item $G'$ is obtained by extending $X$ with $\hexpair$ and including edges $\uned{x}{y}$ for all $x \in X, y \in Y$,
    \item $I' \coloneqq I \cup X \cup \ve{H}$,
    \item $O' \coloneqq O \cup Y \cup Z$,
    \item $\wOp'(v) \coloneqq \w{v}$ for $v \in \ve{G}$ and $\wOp'(h) \coloneqq 0$ for $h \in \ve{H}$, and
    \item $k' \le k$.\qedhere
  \end{enumerate}
\end{definition}

Finally, we define the bordered problem. Analogously to \Cref{sec:scc_bc}, the task in this problem is to solve all border complementation instances of \probrec{}.

\zzcommand{\probborder}{\textsc{Bordered \probrec{}}}
\begin{tcolorbox}[enhanced,title={\color{black} {\probborder{}}}, colback=white, boxrule=0.4pt,
	attach boxed title to top left={xshift=.3cm, yshift*=-2.5mm},
	boxed title style={size=small,frame hidden,colback=white}]
	
	\textbf{Input:}  
  A \probrec{} instance $\mathcal{I} = (G,I,O,B,\wOp,k)$ and a set of border terminals $T \subseteq \ve{G}$ with $\abs{T} \le 2k$\\

	\textbf{Output:}
  A solution to \probrec{} for each border complementation of $\mathcal{I}$ and $T$, or report that no solution exists.
\end{tcolorbox}

\subsection{Unbreakable Case}\label{sec:d_edge_unbreak}

This section constitutes the base case of our recursive understanding algorithm, and we solve it in a similar fashion to our approach in \Cref{thm:unbreakable_scc}. The definitions of separation and unbreakability as well as \Cref{lem:unbreak_small_or_large,lem:find_sets} immediately transfer to undirected graphs. Therefore, we can give the algorithm description right away.

\iflong
\begin{theorem}
\else
\begin{theorem}[$\star$]
\fi
  There is a $c \in \N$, such that for every $k \in \N$, there is an algorithm that solves a \probborder{} instance $(G,I,O,B,\wOp,k,T)$ on a $(q,k)$-unbreakable graph in time $\bigO{(nq)^c}$.
\end{theorem}
\iflong
\begin{proof}
  Our algorithm works as follows. Initially, we enumerate all partitions of $T$ into $X,Y,Z$. Then, we compute $\classes{\induced{G}{X}}$ using \Cref{lem:compute_classes}. This way, we can enumerate all border complementation instances for \probrec{}. Consider one such instance $\mathcal{I'} \coloneqq (G', I', O', B', \wOp', k')$. By \Cref{lem:unbreak_small_or_large}, there is an $s \le q + f(k)$ for some function $f$, such that for every solution $S$ of $\mathcal{I'}$, we have either $\abs{S} \le s$ or $\abs{\ve{G'} \setminus S} \le s$. We address both cases separately and return the maximum weight solution of the solutions for both cases, or none if both do not exist.

  \subparagraph*{Finding a small solution} Our algorithm for this case is similar to the small case in \Cref{thm:unbreakable_scc}.
  \begin{enumerate}[noitemsep,nolistsep]
    \item We apply \Cref{lem:find_sets} with $U = \ve{G'}, a = s, b = k$ to compute a family $\mathcal{F}$ of subsets of $\ve{G'}$.
    \item For every $F \in \mathcal{F}$, compute the $d$-edge-connected components of $F$.
    \item For every $d$-edge-connected component $Q$, check its feasibility, and return the maximum weight feasible solution.
  \end{enumerate}

  \subparagraph*{Finding a large solution} Our algorithm for this case works as follows.
  \begin{enumerate}[noitemsep,nolistsep]
    \item Compute the $d$-edge-connected components of $G'$.
    \item For every $d$-edge-connected component $C$, if $\abs{\nei{C}} \le k$, then $C$ already is the maximum solution that is a subset of $C$. If not we proceed by constructing $G'_C$, analogously to \Cref{thm:unbreakable_scc}, by taking $\induced{G}{\cnei{C}}$, adding a vertex $c$, and connecting it to every $v \in \nei{C}$.
    \item Next, we run the algorithm from \Cref{lem:find_sets} in $G'_C$ with $U = \ve{G'_C}, a = s+1, b = k$.
    \item For all returned sets $F$, we only consider the weak component $S$ in $F$ that includes $c$ if such a component exists.
    \item Finally, we verify whether $C \setminus \cnei{S}$ is a feasible solution and return the maximum weight one.
  \end{enumerate}

  \subparagraph*{Correctness}

  For the small case, we know that for any solution $S$ with $\abs{S} \le s$, there is $F \in \mathcal{F}$ with $S \subseteq F$ and $\nei{S} \cap F = \emptyset$. Therefore, $S$ must be a $d$-edge-connected component of $F$ and this case is correct.

  Correctness for the large case is mostly analogous to \Cref{thm:unbreakable_scc}. We only have to consider the case of a solution $S' \subseteq C$ such that there is a component $C'$ of $\ve{G'_C} \setminus S$ that does not include $c$. We claim that we can include $C'$ to $S'$ to still give a solution. Clearly, the weight is non-decreasing and neighborhood size decreases, so we focus on edge-connectivity.

  First, notice that both $S'$ and $C$ induce $d$-edge-connected subgraphs of $G'$. That means the global min cuts of $\induced{G'}{S'}$ and $\induced{G'}{C}$ both have a value of at least $d$. We have to show that $\induced{G'}{S' \cup C'}$ also has a global min cut value of at least $d$. Suppose the global min cut is the min cut between a pair of vertices $u, v \in S' \cup C'$. We know that $u,v \in S'$ cannot be the case since adding $C'$ can only increase the min cut. Therefore, assume without loss of generality $u \in C'$ and first consider $u \in S'$. Since $C$ induces a $d$-edge-connected subgraph, removing any set of $d-1$ edges still allows us to find a path from $u$ to some $w \in S'$. Since $S'$ is $d$-edge-connected, we can still reach $v$ from $w$.
  The final case is $u, v \in C'$. Again, after removing $d-1$ edges, there still is a path from $u$ to $S'$ and from $v$ to $S'$ since $C$ is $d$-edge-connected and $\nei{C'} \subseteq S'$. Also, $S'$ is still connected after the removal of edges.

  \subparagraph*{Runtime} By \Cref{lem:find_sets}, there is a $c \in \N$, such that running the algorithm costs us $\bigO{s^cn^c} = \bigO{q^cn^c}$ for a constant $k$. We can find all $d$-edge-connected components in time $\bigO{n^4}$ using an algorithm from \cite{wang2015simple}. By \Cref{lem:compute_classes} computing and adding extensions to the graph only adds a constant overhead.
\end{proof}
\fi

\subsection{Reduction Rules and Algorithm}\label{sec:d_edge_algorithm}

Now, we will give some important reduction rules for \probborder{}. First, we transfer some basic reduction rules from \Cref{sec:scc}. We state them without proof since the proofs are simple and analogous to the corresponding reduction rules in the previous section.

\begin{reduction*}\label{red:d_edge_in_out}
  Let $Q$ be a component of $G - B$. If $Q \cap O \ne \emptyset$, set $O = O \cup \cnei{Q}$. If $\cnei{Q} \cap I \ne \emptyset$, set $I = I \cup Q$. If both cases apply, the instance has no solution.
\end{reduction*}
\begin{reduction*}\label{red:d_edge_remove_out}
  If \Cref{red:d_edge_in_out} is not applicable and there exists $v \in O \setminus B$, remove $v$ from $G$.
\end{reduction*}

Whenever possible we first apply \Cref{red:d_edge_in_out} and then \Cref{red:d_edge_remove_out} exhaustively in this order.

We will think of components $Q$ of $G-B$ as extensions $(\induced{G}{Q}, \set{\uned{q}{b}}{q \in Q, b \in B})$ of $\induced{G}{B}$. We will naturally extend equivalence on extensions to components as well as demand and supply function to functions on $B$.
The next rule consists of two closely related parts. First, it removes all components that can never be inside any solution because there are no $d$-edge-connected subgraphs in $G$ containing it. Second, if there are components that could be a solution by themself but cannot be part of a larger solution, we disconnect them from $B$. Later, this allows us to assume that any component could be part of some larger feasible solution.

\iflong
\begin{reduction*}
\else
\begin{reduction*}[$\star$]
\fi
\label{red:remove_useless}
  Let $Q$ be a component of $G - B$. If there is no $B' \subseteq B$ and a set of supply functions $F$ such that $F$ is sufficient for $Q$ and $B'$, include $Q$ into $O$.

  If the former does not apply, but there is no \emph{non-empty} $B' \subseteq B$ and $F$ such that $F$ is sufficient for $Q$ and $B'$, do the following. If $Q$ is a feasible solution, disconnect $Q$ from $B$ and set $O = O \cup \nei{Q}$. Otherwise, include $Q$ into $O$.
\end{reduction*}
\iflong
\begin{proof}[Proof of Safeness]
  Assume there is a solution $S$ with $Q \subseteq S$. Let $B' \coloneqq B \cap S$. Since $S$ is $d$-edge-connected, there are $d$ edge-disjoint paths between every pair of vertices. For every $v_1, v_2 \in Q \cup B'$, define a supply function $f$, where $f(\set{b_1, b_2})$ is the number of $b_1$-$b_2$-paths that is used to construct the $d$ edge-disjoint paths between $v_1$ and $v_2$ in $S$. Then, the set of all these $f$ must be sufficient for $Q$ and $B'$. Hence, in case the reduction rule applies, there cannot be a solution including $Q$ and we can safely remove it from the graph.

  For the second part of the rule, it is clear from the previous proof that for every solution $S$ we either have $S \cap Q = \emptyset$ or $S = Q$. If the neighborhood condition holds, we can simply disconnect it from its neighbors. Now, any solution can no longer include the former neighbors, but $Q$ itself is still a valid solution. Otherwise, $Q$ cannot be part of a solution and can thus safely be removed.
\end{proof}
\fi

After the previous reduction has been applied exhaustively, we can show that two equivalent components must have the same neighborhood.

\iflong
\begin{lemma}
\else
\begin{lemma}[$\star$]
\fi
\label{lem:similar_eq_nei}
  Suppose that \Cref{red:remove_useless} is not applicable in $G$. Then for two components $Q_1$ and $Q_2$ of $G-B$ with $Q_1 \dequiv Q_2$, we have $\nei{Q_1} = \nei{Q_2}$.
\end{lemma}
\iflong
\begin{proof}
  If $\abs{\nei{Q_1}} \ge 2$, let $b_1, b_2 \in \nei{Q_1}$. Then, $Q_1$ fulfills the demand function $f$ with $f(\set{b_1,b_2}) = 1$ and $f(\set{b,b'}) = 0$ for all other $b,b' \in B$. Since $Q_2$ must also fulfill $f$, we have $b_1, b_2 \in \nei{Q_2}$ and hence $\nei{Q_1} = \nei{Q_2}$. If $Q_i$ cannot be part of a solution or only be a solution on its on, it has $\nei{Q_i} = \emptyset$, by \Cref{red:remove_useless}. Since $d > 1$, there must be at least two edges from $Q_i$ to $B$ if $Q_i$ can be proper subset of a solution. If both edges lead to the same vertex, we can create a sufficient supply function, which again shows $\nei{Q_1} = \nei{Q_2}$.
\end{proof}
\fi

Now, we get to the core of our algorithm. In \Cref{red:d_edge_twins}, we prove that components of $G-B$ can be replaced by equivalent components, which follows directly from \Cref{lem:equiv_d_edge}. Next, we prove that when there are multiple equivalent components, it is enough to keep only a certain number of them that depends only on $\abs{B}$ and $d$. More components than this threshold are not relevant for forming $d$-edge-connected subgraphs and can always be included in a solution that includes the other equivalent components. Note that in previous applications of this approach such as the algorithms in~\cite{golovach2020finding} or \Cref{sec:scc}, the number of necessary components was always at most 2. In \Cref{red:d_edge_many_twins} we use a slightly more involved argument to acknowledge the fact that paths have to be edge-disjoint.

\iflong
\begin{reduction*}
\else
\begin{reduction*}[$\star$]
\fi
\label{red:d_edge_twins}
  Let $Q$ be a component of $G-B$ and let $Q^* \in \classesc{\induced{G}{B}}$ be the minimum size component that is equivalent to $Q$. Replace $Q$ with $Q^*$, set $\w{Q^*} \coloneqq \w{Q}$, and set $Q^* \subseteq I$ if $Q \subseteq I$.
\end{reduction*}
\iflong
\begin{proof}[Proof of Safeness]
  Remember that components can only be included as a whole, and note that $\nei{Q} = \nei{Q^*}$ by \Cref{lem:similar_eq_nei}. Let $S$ be a solution to the old instance and define $B' \coloneqq B \cap S$. Clearly, if $Q \not\subseteq S$, then $S$ is a solution to the new instance, so assume $Q \subseteq S$. Consider $S' \coloneqq (S \setminus Q) \cup Q^*$ and notice that $\nei{S'} = \nei{S}$ and $\w{S'} = \w{S}$. 
  Using \Cref{lem:equiv_d_edge} with $G = \induced{G}{B}$ and extensions corresponding to $Q$, $Q^*$, and $\ve{G} \setminus (B \cup Q)$, we conclude that $S'$ is also $d$-edge-connected.

  In this proof we only relied on the fact that $Q$ and $Q^*$ are equivalent. Therefore, the proof that any solution to the new instance can be transformed to a solution to the old instance follows by symmetry.
\end{proof}
\fi

As discussed before, the next rule is the most interesting one. We show that only $h$ copies of components from every equivalence classes are necessary to construct all solutions, for some number $h$ that only depends on $\abs{B}$ and $d$.

\iflong
\begin{reduction*}
\else
\begin{reduction*}[$\star$]
\fi
\label{red:d_edge_many_twins}
  For $h \coloneqq \max\set{(\abs{B}-1)d + 2, 2}$, let $Q_1, \ldots, Q_{h+1}$ be equivalent components of $G-B$ ordered by weight non-increasingly. Then, remove $Q_{h+1}$ from $G$ and increase $\w{Q_1}$ by $\w{Q_{h+1}}$. If $Q_{h+1} \subseteq I$, add $Q_1$ to $I$.
\end{reduction*}
\iflong
\begin{proof}[Proof of Safeness]
  We know that components can only be included as a whole. Let $S$ be a solution to the old instance.
  If $Q_{h+1} \not\subseteq S$, since $Q_{h+1} \cap B = \emptyset$, we immediately have that $S$ is also a solution for the new instance.
  If $Q_{h+1} \subseteq S$, either $S = Q_{h+1}$ or some vertices of $B$ must also be part of the solution. In the first case, $Q_1$ is a solution for the new instance with at least the same weight since $F = \emptyset$ must be sufficient for $Q_i$ and $\emptyset$ for all $i \in [h+1]$.
  In the second case, since all $Q_i$ have the same neighborhood, we must have $Q_i \subseteq S$ for all $i \in [h+1]$. We claim that $S' \coloneqq S \setminus Q_{h+1}$ is a solution for the new instance. Clearly, the neighborhoods of $S$ and $S'$ are the same. We now prove the $d$-edge-connectedness of $S'$.

\begin{figure}[t]
  \centering
  \includegraphics[width=0.25\textwidth,page=3]{figures/d_edge}
\caption{A visualization of one part of the proof of safeness of \Cref{red:d_edge_many_twins}. The example shows that a shortest path from a vertex $q_1 \in Q_1$ to a vertex $q_4 \in Q_4$ can use at most $2$ intermediate components $Q_2$ and $Q_3$ for $\abs{B} = 3$.}
  \label{fig:many_twins}
\end{figure}

  Let $s_1, s_2 \in S'$ and consider the $d$ edge-disjoint paths $P_1, \ldots, P_d$ between $s_1$ and $s_2$ in $S$, such that $\abs{P_i}$ is minimal for all $i \in [d]$, meaning that there is no way to replace one $P_i$ by a shorter path that is still edge-disjoint to the other paths. Now, the paths might use subpaths via $Q_{h+1}$. In this case we aim to replace these by different subpaths while staying edge-disjoint. For each $b \in B$, note that each $P_i$ can only include $b$ once. Otherwise, it includes a circle, which we could easily shortcut to decrease $\abs{P_i}$. Hence, we can think of $P_i$ as a path consisting of at most $\abs{B}+1$ subpaths that all start and end in a vertex in $B$, except for the first and last one. This part of the proof is also visualized in \Cref{fig:many_twins}. Now, for each such subpath, if it is a path via $Q_{h+1}$, instead reroute it via a different $Q_i$ that does not intersect with any path so far. We know that such a $Q_i$ must exist since there are only $(\abs{B}+1)d$ subpaths in total, each subpath is via exactly one component of $G-B$, and the first the last subpaths must be in the components of $s_1$ and $s_2$.  Furthermore, there is a corresponding path in $Q_i$ since $Q_i$ and $Q_{h+1}$ are equivalent and fulfill the same demand functions. Hence, we have constructed $d$ edge-disjoint paths between $s_1$ and $s_2$ in $S'$ and $S'$ is a solution to the new instance.

  For the other direction, consider a solution $S'$ for the new instance. If one $Q_i$ for $i \in [h]$ is not included in $S'$, then none of $\nei{Q_i}$ can be included in $S'$. Since $\nei{Q_i} = \nei{Q_{h+1}}$, we have that $S'$ is also a solution for the old instance.
  Suppose $Q_i \subseteq S'$ for all $i \in [h]$. We claim that $S \coloneqq S' \cup Q_{h+1}$ is a solution for the old instance. Since all $Q_i$ have the same neighborhood, we have $\nei{S'} = \nei{S}$. We now prove the $d$-edge-connectedness of $S$.

  Let $s_1, s_2 \in S$. If both $s_1, s_2 \in S'$, there are $d$ edge-disjoint paths between $s_1$ and $s_2$ only using $S'$. Therefore, the only two remaining cases are $s_1, s_2 \in Q_{h+1}$ and $s_1 \in Q_{h+1}, s_2 \notin Q_{h+1}$. We first consider the case where both vertices are in $Q_{h+1}$. Let $B' = B \cap S$. Since $Q_1 \subseteq S'$ and $S'$ is a solution, we know $S' \setminus Q_1$ must provide the paths from a sufficient supply function set $F$ for $Q_1$ and $B'$. Since $Q_1$ is equivalent to $Q_{h+1}$, the set $F$ is also sufficient for $Q_{h+1}$ and $B'$ and there are $d$ edge-disjoint paths between $s_1$ and $s_2$.

  Suppose without loss of generality only $s_1 \in Q_{h+1}$. If $s_2 \in B'$, since $S'$ provides the paths from a sufficient set $F$ for $Q_{h+1}$ and $B'$, there must be $d$ edge-disjoint paths between $s_1$ and $s_2$. If $s_2 \in Q_i$ for some $i \in [h]$, consider additionally some $b \in B'$. We know that there are $d$ edge-disjoint paths between $s_1$ and $b$ and $d$ edge-disjoint paths between $b$ and $s_2$. By \Cref{lem:connect_paths} there are at least $d$ edge-disjoint paths between $s_1$ and $s_2$ in $S$.
\end{proof}
\fi

Now, we can prove that the recursive understanding algorithm works also works for \probborder{}. Before, we summarize the progress of \Cref{red:d_edge_twins,red:d_edge_many_twins} in the following lemma.
\begin{lemma}\label{lem:d_all_reductions}
  Either we can reduce the instance, or there are at most $\max\set{(\abs{B}-1)d + 2, 2}$ components of every equivalence class. Furthermore, each component $Q$ has minimum size out of its equivalence class.
\end{lemma}

Finally, we arrive at the complete algorithm, which follows the same structure as the algorithm given in \Cref{alg:rec_und_scc}. Since the proof is very technical but also analogous to the proof of \Cref{thm:border_scc_fpt}, we only sketch it for completeness. 

\iflong
\begin{theorem}
\else
\begin{theorem}[$\star$]
\fi
\label{thm:border_d_edge_fpt}
  There is a constant $c \in \N$, such that for each $k \in \N$, there is an algorithm that solves a \probborder{} instance $\mathcal{I} = (G,I,O,B,\wOp,k,T)$ in time $\bigO{n^c}$.
\end{theorem}
\iflong
\begin{proof}
  The algorithm structure is almost identical to the algorithm of \Cref{thm:border_scc_fpt} as illustrated in \Cref{alg:rec_und_scc}, which is why we do not state it again here. The only difference besides the definition of border complementations and the set of reduction rules is the choice of $q$. In our case, $q$ will be constant for a fixed $k$ and we later prove that a suitable $q$ exists.

  We again have to prove that a border complementation covers all relevant parts of a solution, that is that we can consider $\mathcal{\hat{I}} \coloneqq (G,I,O,\hat{B},\wOp,k,T)$ instead of $\mathcal{I}$. To do this, let $S$ be a solution to a border complementation of $\mathcal{I}$. Consider the border complementation for $\tilde{G}$ that chooses $X$, $Y$, and $Z$ to match with $S$, that is $X = S \cap \tilde{T}$, $Y = \nei{S} \cap \tilde{T}$, and $Z = \tilde{T} \setminus \cnei{S}$. Furthermore, choose the border complementation that has an equivalent extension to $(\induced{G}{S \cap (U \setminus W)}, \set{\uned{u}{x}}{u \in S \cap U, x \in X})$. By \Cref{lem:equiv_d_edge}, both graphs have the same $d$-edge-connected induced subgraphs that intersect $W$. Therefore, it is enough to consider the instance with $\hat{B}$.

  Regarding the runtime and the choice of $q$, notice that $\abs{T} \le 2k$. By \Cref{lem:number_equiv}, there are at most $3^{2k}f(k)$ relevant border complementations for some function $f$. By definition of $\hat{B}$, we can bound $\abs{\hat{B}} \le 2k + k 3^{2k}f(k) \eqqcolon g(k)$. Let $h$ be the maximum size of a minimal equivalent extension of a graph with at most $g(k)$ vertices. Using \Cref{lem:d_all_reductions}, all components of $G-\hat{B}$ have at most $h$ vertices. Since the number of components per class is also bounded, we can bound the size of $W^*$ by the maximum size of each component, multiplied with the number of equivalence classes and the number of components per class, that is \[\abs{W^*} \le hf(k)((g(k)-1)d + 2) + g(k) \eqqcolon q,\] which is constant for a fixed $k$.

  Also, we only have to check equivalence and compute the classes for extensions of graphs with at most $g(k)$ vertices, which is constant for a fixed $k$ by \Cref{lem:compute_classes}. The other operations of the reduction rules are clearly polynomial. Together with the runtime analysis from \Cref{thm:border_scc_fpt}, this gives the desired runtime.
\end{proof}
\fi

Finally, we solve \prob{} using \probborder{}. We again set $I \coloneqq O \coloneqq \emptyset$, $B \coloneqq \ve{G}$, and $T \coloneqq \emptyset$.

\begin{theorem}\label{cor:d_edge_fpt}
  \prob{} is non-uniformly FPT.
\end{theorem}

What would be necessary to strengthen \Cref{cor:d_edge_fpt} and show that \prob{} is also uniformly FPT with a bounded and computable parameter dependence $f(k)$? There are two main hurdles. First, the use of a non-uniform FPT-algorithm from \cite{robertson1995graph} in \Cref{lem:check_equiv} has to be avoided. If an algorithm for the edge-disjoint path problem in uniform FPT-time is discovered, this can be used. Another way would be to check the definition of $\dequiv$ differently, or define $\dequiv$ in way that does not need such a heavy machinery. 

Second, we would need a way to define a constructive compression routine, such as the one in \Cref{sec:scc} for directed extensions. This would allow us to bound the size of the compressed extension by a function of $k$ and hence also the size of $W^*$ after applying all reduction rules. Until both of these conditions are achieved, the runtime of this algorithm stays non-uniform.
Nevertheless, this is still a strong indication towards proving the existence of a uniform FPT-algorithm~\cite[Chapter~6]{cygan2015parameterized}.


\iflong
\section{Secluded Subgraphs with Small Independence Number}
\label{chap:tour}
\zzcommand{\clique}{\textsc{Secluded Clique}}
\newcommand{\aboundedlong}{\textsc{Out-Secluded $\alpha$-Bounded Subgraph}}
\newcommand{\abounded}{\textsc{Out-Secluded $\alpha$-BS}}
\newcommand{\secuna}[1]{\textsc{Undirected Secluded {#1}-BS}}

In this section, we consider a generalization of the undirected \clique{} problem to directed graphs. We generalize this property in the following way.

\begin{definition}[$\alpha$-Bounded]
    A directed graph $G$ is called \emph{$\alpha$-bounded} if $G$ includes no independent set of size $\alpha + 1$, that is, for all $S \subseteq \ve{G}$ with $\abs{S} = \alpha + 1$, the graph $\induced{G}{S}$ includes at least one edge.
\end{definition}

Our problem of interest will be the \aboundedlong{} problem (\abounded{}).
Note that $\alpha$ is part of the problem and can therefore always be assumed to be constant.
In the directed case, $\alpha = 1$ is analogous to the \prname{Out}{Tournament} problem, with the only difference that tournaments cannot include more than one edge between a pair of vertices.

In \Cref{sec:alpha_bounded}, we show how to solve \abounded{} with a branching algorithm. The general nature of these branching rules allows us to transfer and optimize them, to significantly improve the best known runtime for the \clique{} problem to $1.6181^kn^{\bigO{1}}$ in \Cref{sec:clique}.

\subsection{Secluded \texorpdfstring{$\alpha$}{α}-Bounded Subgraphs}\label{sec:alpha_bounded}

First, we consider \abounded{}.
Our branching algorithm starts by selecting one part of the solution and then relies on the fact that the remaining solution has to be close around the selected part. More concretely, we start by picking an independent set that is part of the solution and build the remaining solution within its two-hop neighborhood. The following lemma justifies this strategy. 
% To be able to apply our previous technique, we use the following lemma to show that $\alpha$-bounded graphs also admit some kind of local structure.
The proof of this lemma was sketched on \emph{Mathematics Stack Exchange}~\cite{stackexchange}, we give an inspired proof again for completeness.

\iflong
\begin{lemma}
\else
\begin{lemma}[$\star$]
\fi
\label{lem:ind_set_reaches}
  In every directed graph $G$, there is an independent set $U \subseteq \ve{G}$, such that every vertex in $\ve{G}$ is reachable from an $u \in U$ via a path of length at most $2$.
\end{lemma}
\iflong
\begin{proof}
  We prove the statement by induction on $\abs{\ve{G}}$. For $\abs{\ve{G}}=1$, it clearly holds.

  Assume the statement holds for all graphs with fewer than $\abs{\ve{G}}$ vertices, we want to prove it for $G$.
  Let $v \in \ve{G}$ be a vertex. If $v$ reaches every other vertex via at most 2 edges, $\set{v}$ is our desired independent set. Otherwise, $T \coloneqq \ve{G} \setminus \coutNei{v}$ is non-empty. Since $\abs{T} < \abs{\ve{G}}$, we can apply the induction hypothesis. So, let $T' \subseteq T$ be an independent set in $\induced{G}{T}$ that reaches every vertex of $T$ via at most two edges. Now we consider the edges between $v$ and $T'$. 

  Since $T \cap \outNei{v} = \emptyset$, there cannot be an edge $(v,t')$ for any $t' \in T'$. If there is an edge $(t', v)$ for some $t' \in T'$, then $T'$ is also a solution for $G$ since $t'$ can reach $\coutNei{v}$ via at most 2 edges.
  Finally, if there is no edge between $v$ and $T'$, then $T' \cup \set{v}$ is an independent set. Also $T' \cup \set{v}$ reaches by definition all of $T \cup \coutNei{v}$ via at most 2 edges.
\end{proof}
\fi

For $\alpha$-bounded graphs, all independent set have size at most $\alpha$, so trying every subset of $\ve{G}$ of size at most $\alpha$ allows us to find a set that plays the role of $U$ in \Cref{lem:ind_set_reaches} in our solution. Hence, the first step of our algorithm will be iterate over all such sets. From then on, we only consider solutions $S \subseteq \coutNei{\coutNei{U}}$.

Then, our algorithm uses two branching rules. Both rules have in common that in contrast to many well-known branching algorithms~\cite[Chapter~3]{cygan2015parameterized}, we never include vertices into a partial solution. Instead, due to the nature of our parameter, we only decide that vertices should be part of the final neighborhood. This means that we remove the vertex from the graph and decrease the parameter by 1.
Our first branching rule branches on independent sets of size $\alpha + 1$ in $\coutNei{\coutNei{U}}$, to ensure that $\coutNei{\coutNei{U}}$ becomes $\alpha$-bounded. The second rule is then used to decrease the size of the neighborhood of $\coutNei{\coutNei{U}}$, until it becomes secluded.
% By differentiating some more base cases in these two high-level rules, we arrive at a branching vector of $(1,2)$ for the desired runtime.

% In the first reduction rule , we used the following observation. If we are in a branch in which we already decided that $u$ should be part of the final solution $S$ and there is $v \in \outNei{u}$, then either $v \in \nei{S}$ or $v \in S$. The same works if $v$ is farther away from $u$. For a $u$-$v$-path $P$, either some vertex on $P$ has to be in $\nei{S}$ or $v$ must be in $S$. If this path has constant size, we can branch on these options, which we use in the upcoming algorithm.
To formalize which vertices exactly to branch on, we use some new notation. For a vertex $v \in \ve{G}$ and a vertex set $U \subseteq \ve{G}$, pick an arbitrary shortest path from $U$ to $v$ if there exists one. Define $\shor{v}$ to be the vertices on that path including $v$ but excluding the first vertex $u \in U$. Note that after removing vertices from the graph, $\shor{v}$ might change.

Now, we can give our algorithm for finding secluded $\alpha$-bounded graphs. 

\restateab*
\iflong
\begin{proof}
    \begin{figure}[t]
      \centering \hfill
      \begin{subfigure}{0.48\textwidth} \centering
        \includegraphics[height=0.5\textwidth,page=8]{figures/clique_rules}
        \caption{Case 1. If $\alpha \le 2$, not all of $v_1$, $v_2$, and $v_3$ can be in $S$. One of the five vertices must be in $\nei{S}$.}
      \end{subfigure} \hfill
      \begin{subfigure}{0.48\textwidth} \centering
        \includegraphics[height=0.5\textwidth,page=9]{figures/clique_rules}
        \caption{Case 2. Since $v_3$ is not reachable from $U$ via 2 edges, one of $v_1$, $v_2$, and $v_3$ must be in $\nei{S}$.}
      \end{subfigure} \hfill
      \caption{A visualization of the branching rules for the \abounded{} algorithm in \Cref{thm:alpha_bounded_fpt}. We are only looking for solutions $S$ including $U$, where every $v \in S$ is reachable from some $u \in U$ via at most 2 edges. Dotted connections stand for edges that do not exist.}\label{fig:alpha}
    \end{figure}

    \newcommand{\curr}{\coutNei{\coutNei{U}}}
    \begin{algorithm}[t]
      \caption{The branching algorithm for \abounded{} that returns a solution including the set $U \subseteq \ve{G}$.}
      \label{alg:abounded}
      \DontPrintSemicolon
      \SetKwFunction{FMain}{$\alpha$-BS}
      \SetKwProg{Fn}{def}{:}{}
      \Fn{\FMain{$G$, $\omega$, $w$, $k$, $U$}}{
        \uIf{$k < 0$} {
          \textbf{abort}\; 
        }
        \uElseIf{$\curr$ is a solution} {
            \KwRet $\curr$\;
        }
        \uElseIf{there is an independent set $I \subseteq \curr$ of size $\abs{I} = \alpha + 1$} {
            \ForEach{$v \in \bigcup_{w \in I} \shor{w}$} {
                Call \FMain{$G-v$, $\omega$, $w$, $k-1$, $U$}\;
            }
        }
        \ElseIf{there is $w \in \outNei{\curr{}}$} {
            \ForEach{$v \in \shor{w}$} {
                Call \FMain{$G-v$, $\omega$, $w$, $k-1$, $U$}\;
            }
        }
      }
    \end{algorithm}
    
    Let $(G,\wOp, w, k)$ be an instance of \abounded{}. 
    We guess a vertex set $U \subseteq \ve{G}$ that should be part of a desired solution $S$. Furthermore, we want to find a solution $S$ to the instance such that $S \subseteq \coutNei{\coutNei{S}}$, that is, every $v \in S$ should be reachable from $U$ via at most two edges.
    We give a recursive branching algorithm that finds the optimal solution under these additional constraints.
    The algorithm is also described in \Cref{alg:abounded}.
    
    When deciding that a vertex should be part of the final neighborhood, we can simply delete it and decrease $k$ by one. In case $k$ decreases below 0, there is no solution. If $\curr$ is a solution to the instance, we return it. These are the base cases of our algorithm. Otherwise we apply the following branching rules and repeat the algorithm for all non-empty independent sets $U \subseteq \ve{G}$ of size at most $\alpha$.
    The branching rules are visualized in \Cref{fig:alpha}.
    
    \begin{description}
        \item[Case 1. $\curr$ is not $\alpha$-bounded.] In this case, there must be an independent set $I \subseteq \curr$ of size $\alpha + 1$. Clearly, not all of $I$ can be part of the solution, so there is a vertex $w \in I \setminus S$. This means that either $w$ or a vertex on every path from $U$ to $w$ must be in $\outNei{S}$.
        The set $\shor{w}$ is one such path of length at most 2. Thus, one of $\bigcup_{w \in I} \shor{w}$ must be part of the out-neighborhood of $S$ and we branch on all of these vertices. For one vertex, delete it and decrease $k$ by 1.
    
        \medskip
        \item[Case 2. $\curr$ is $\alpha$-bounded, but has additional neighbors.] Consider one of these neighbors $w \in \outNei{\curr}$.
        Since $w$ is not reachable from $U$ via at most two edges, we should not include it in the solution. Now, $\shor{w}$ is a path of length 3, and one of its vertices must be in $\outNei{S}$. We again branch on all options, delete the corresponding vertex and decrease $k$ by 1.
    \end{description}

    By \Cref{lem:ind_set_reaches}, there is a suitable choice of $U$ for every solution. Therefore, if we can find the maximum solution containing $U$ if one exists in every iteration with our branching algorithm, the total algorithm is correct.
    Also notice that the branching rules are a complete case distinction; if none of the rules apply, the algorithm reaches a base case.
    The remaining proof of correctness follows from a simple induction.

    We initially have to consider all subsets of $\ve{G}$ of size at most $\alpha$ while rejecting subsets that are not an independent set in time $n^{\alpha+1}$.
    To bound the runtime of the branching algorithm, notice that in each case we branch and make progress decreasing $k$ by 1.
    In the first cases, the independent set $I$ has size $\alpha + 1$ for every $w \in I$, the path $\shor{w}$ includes at most 2 vertices. Hence, there are at most $2(\alpha + 1)$ branches in this case. The second case gives exactly 3 branches and is thus dominated by the first rule.  This gives the claimed runtime and concludes the proof. 
\end{proof}
\fi

For the tournament setting, we can simply extend the first branching rule from the algorithm from \Cref{thm:alpha_bounded_fpt} to also branch on two vertices that are connected by a bidirectional edge.

\begin{theorem}
  \prname{Out}{Tournament} is solvable in time $4^kn^{\bigO{1}}$.
\end{theorem}

\begin{comment}
\iflong
In undirected graphs, the situation is simpler. Instead of \Cref{lem:ind_set_reaches}, clearly any maximum size independent set reaches every vertex via a single edge. Therefore, when adapting the algorithm from \Cref{thm:alpha_bounded_fpt}, the first rule has at most $\alpha + 1$ branches and the second rule now has at most $2$ branches. This gives us the following corollary, which also generalizes \Cref{thm:secluded_clique_fpt}.

\begin{corollary}\label{cor:undir_ab}
  \secuna{$\alpha$} is solvable in time $(\alpha+1)^kn^{\alpha+\bigO{1}}$.
\end{corollary}
\fi
\end{comment}

Moreover, note that if we instead define the directed problem to ask for the total neighborhood $\nei{S} \le k$, the problem is identical to solving the undirected version on the underlying undirected graph. 
In undirected graphs, instead of \Cref{lem:ind_set_reaches}, any maximal independent set reaches every vertex via a single edge. We can use this insight to adapt the previous algorithm to a faster one for the total neighborhood variant of the problem.

\restateabtotal*
\begin{proof}[Proof Sketch]
    Notice that in every directed graph $G$, any maximal independent set $U \subseteq \ve{G}$ fulfills $\cnei{U} = V$. Therefore, we can execute the same algorithm as described in \Cref{alg:abounded} and the proof of \Cref{thm:alpha_bounded_fpt} with $\cnei{U}$ instead of $\coutNei{\coutNei{U}}$. This leads to a smaller size set of vertices to branch on, namely $\alpha + 1$ instead of $2\alpha + 2$ for the first case and 2 instead of 3 for the second case. The remaining steps work out in the same way.
\end{proof}

\subsection{Faster \textsc{Secluded Clique} using Branching}\label{sec:clique}

The \clique{} problem has been shown to admit an FPT-algorithm running in time $2^{\bigO{k \log k}}n^{\bigO{1}}$ by contracting twins and then using color coding~\cite{golovach2020finding}. Furthermore, the single-exponential algorithm for \textsc{Secluded $\mathcal{F}$-Free Subgraph} from~\cite{jansen2023single} can also solve this problem in time $2^{\bigO{k}}n^{\bigO{1}}$ when choosing $\mathcal{F}$ to include only the independent set on two vertices. 
In this section, we give a faster and simpler algorithm using branching that achieves the same in time $1.6181^k n^{\bigO{1}}$.

% Intuitively, what makes this problem easily tractable is its local nature. When fixing a single vertex $v \in \ve{G}$ and looking for cliques including $v$, we know that any solution must be a subset of $\cnei{v}$. This fact allows us to construct local branching rules and will be crucial also for finding graphs with higher independence number.

The ideas behind our branching rules are the the same as in \Cref{thm:alpha_bounded_fpt}. However, this time, the forbidden structures are only independent sets of size 2. Therefore, it is enough guess a single vertex $u$ initially. This allows us also to analyze the rules more closely and split them into several cases.
By differentiating between more different scenarios in these two high-level rules, we arrive at a branching vector of $(1,2)$ for the desired runtime.

\restateclique*

\iflong
\begin{proof}
    \begin{algorithm}[t]
    % \begin{algorithmic}
      \caption{The $1.6181^kn^{\bigO{1}}$ branching algorithm for \clique{} that returns a $k$-secluded clique of weight at least $w$ including $u \in \ve{G}$.}
      \label{alg:clique_fast}
      \DontPrintSemicolon
      \SetKwFunction{FMain}{Clique}
      \SetKwProg{Fn}{def}{:}{}
      \Fn{\FMain{$G$, $\omega$, $w$, $k$, $u$}}{
        \uIf{$k < 0$} {
          \textbf{abort}\; 
        }
        \uElseIf{$\cnei{u}$ is a solution} {
            \KwRet $\cnei{u}$\;
        }
        \uElseIf{there are distinct $v_1, v_2, v_3 \in \nei{u}$ with $\{v_1, v_2\}, \{v_1, v_3\} \notin \e{G}$} {
            Call \FMain{$G-v_1$, $\omega$, $w$, $k-1$, $u$}\;
            Call \FMain{$G-\{v_2, v_3\}$, $\omega$, $w$, $k-2$, $u$}\;
        }
        \uElseIf{there are distinct $v_1, v_2 \in \nei{u}$ with $\{v_1, v_2\} \notin \e{G}$, $\nei{v_1}, \nei{v_2} \subseteq \cnei{u}$, and $\w{v_1} \le \w{v_2}$} {
            Call \FMain{$G-v_1$, $\omega$, $w$, $k-1$, $u$}\;
        }
        \uElseIf{there are distinct $v_1, v_2 \in \nei{u}$ with $\{v_1, v_2\} \notin \e{G}$ and $\nei{v_1} \not\subseteq \cnei{u}$} {
            Call \FMain{$G-v_1$, $\omega$, $w$, $k-1$, $u$}\;
            Call \FMain{$G - (\neiOp'(v_1) \cup \set{v_2})$, $\wOp$, $w$, $k- (\abs{\neiOp'(v_2)} + 1)$, $u$}\;
        }
        \uElseIf{there is $v_1 \in \nei{u}$ with $\neiOp'(v_1) = \set{v_2}$} {
            Call \FMain{$G - v_2$, $\wOp$, $w$, $k-1$, $u$}\;
        }
        \ElseIf{there is $v \in \nei{u}$ with $\abs{\neiOp'(v)} \ge 2$} {
            Call \FMain{$G-v$, $\wOp$, $w$, $k-1$, $u$}\;
            Call \FMain{$G - \neiOp'(v)$, $\wOp$, $w$, $k- \abs{\neiOp'(v)}$, $u$}\;
        }
      }
    % \end{algorithmic}
    \end{algorithm}

    \begin{figure}[t] \centering \hfill
      \begin{subfigure}{0.32\textwidth} \centering
        \includegraphics[width=0.6\textwidth,page=3]{figures/clique_rules}
        \caption{Case 1a. Either $v_2 \in \nei{S}$ or $v_1,v_3 \in \nei{S}$ must hold.}
      \end{subfigure} \hfill
      \begin{subfigure}{0.32\textwidth} \centering
        \includegraphics[width=0.6\textwidth,page=4]{figures/clique_rules}
        \caption{Case 1b. It only changes the weight if $v_1 \in \nei{S}$ or $v_2 \nei{S}$.}
      \end{subfigure} \hfill
      \begin{subfigure}{0.32\textwidth} \centering
        \includegraphics[width=0.6\textwidth,page=5]{figures/clique_rules}
        \caption{Case 1c. Either $v_1 \in \nei{S}$ or $\neiOp'(v_1) \cup \set{v_2} \subseteq \nei{S}$ must hold.}
      \end{subfigure} \hfill
      \begin{subfigure}{0.49\textwidth} \centering
        \includegraphics[width=0.38\textwidth,page=6]{figures/clique_rules}
        \caption{Case 2a. The only neighbor of $v_1$ outside $\cnei{u}$ is $v_2$. Therefore, $v_2 \in \nei{S}$ must hold.}
      \end{subfigure} \hfill
      \begin{subfigure}{0.49\textwidth} \centering
        \includegraphics[width=0.38\textwidth,page=7]{figures/clique_rules}
        \caption{Case 2b. Either $v_1 \in \nei{S}$ or $\neiOp'(v_1) \subseteq \nei{S}$ must hold.}
      \end{subfigure} \hfill
      \caption{A visualization of the branching rules for the improved \textsc{Secluded Clique} algorithm in \Cref{thm:clique_better}. We are only looking for solutions $S$ including $u$. Dotted connections stand for edges that do not exist. Also, vertices outside of $\cnei{u}$ are never connected to $u$. In (b) to (e), we can assume that $v_1$ is connected to all other vertices in $\cnei{u}$, and the same holds for $v_2$ in (b) and (c). Dotted edges to the outside of $\cnei{u}$ mean that no such edges exist.}\label{fig:clique2}
    \end{figure}
    
    Let $(G,\wOp, w, k)$ be an instance of \clique{}. 
    We guess a vertex $u\in \ve{G}$ that should be part of a desired solution $S$, that is we look for a solution $S \subseteq \cnei{u}$ with $u \in S$. We give a recursive branching algorithm that branches on which vertices should be part of the neighborhood. 
    The algorithm is also described in \Cref{alg:clique_fast}.
    
    When deciding that a vertex should be part of the final neighborhood, we can simply delete it and decrease $k$ by one. In case $k$ decreases below 0, there is no solution. If $\cnei{u}$ is a solution to the instance, we return it. These are the base cases of our algorithm. Otherwise we apply the following branching rules and repeat the algorithm for all choices of $u \in \ve{G}$.
    Denote for $v \in \ve{G}$ with $\neiOp'(v) \coloneqq \nei{v} \setminus \cnei{u}$ the additional neighborhood of $v$ that is not part of $\cnei{u}$. 
    The branching rules are visualized in \Cref{fig:clique2}.
    
\begin{description}
    \item[Case 1. $\cnei{u}$ does not form a clique.] In this case we must have a pair of vertices  $v_1, v_2 \in N(X)$ with $\set{v_1, v_2}\notin E(G)$. We now break into following three subcases.
    
    \begin{itemize}
        \item\label{it:clique1} \textbf{1a. There is a vertex $v_1 \in \nei{u}$ not connected to at least 2 others.} Formally speaking, there are distinct $v_2, v_3 \in \nei{u}$ with $\set{v_1, v_2}, \set{v_1, v_3} \notin \e{G}$. In this case, we know that if $v_1$ is in $S$, then both $v_2$ and $v_3$ cannot be in $S$.
        Therefore, we branch into removing both $v_2$ and $v_3$ or removing just $v_1$. We decrease $k$ by $1$ or $2$.

        \item \label{it:clique2}\textbf{1b. There are disconnected $v_1, v_2 \in \nei{u}$ that have no other neighbors.}
        Since the previous subcase does not apply and $\neiOp'(v_1) = \neiOp'(v_2) = \emptyset$, we know that $v_1$ and $v_2$ must be twins. Not both of them can be part of $S$, so we could remove either of them. Therefore, it is optimal to remove the vertex with smaller weight, decrease $k$, and recurse.
        
        \item \label{it:clique3}\textbf{1c. There are disconnected $v_1, v_2 \in \nei{u}$ with at least one other neighbor.}
        Suppose $\neiOp'(v_1) \ne \emptyset$. In this case, we know that if $v_1 \in S$, then both $v_2$ and all of $\neiOp'(v_1)$ have to go in $\nei{S}$.
        Thus, we branch into two cases where we either remove $v_1$ or $\neiOp'(v_1) \cup \set{v_2}$ and decrease $k$ accordingly.
    \end{itemize} \medskip 
    
    \item[Case 2. $\cnei{u}$ forms a clique.] In this case, we break into following two subcases.
    \begin{itemize}
        \item \label{it:clique4}\textbf{2a. There is $v_1 \in \nei{u}$ with only one other neighbor $\neiOp'(v_1) = \set{v_2}$.} In this case, it is always optimal to include $v_1$ in $S$. This adds at most one neighbor $v_2$ compared to not including $v_1$, which would add $v_1$ into $\nei{S}$. Since weights are non-negative, $v_2$ should go into $\nei{S}$. We remove it from $G$ and decrease $k$ by 1.
        
        \item \label{it:clique5}\textbf{2b. There is $v \in \nei{u}$ with at least two other neighbors in $\neiOp'(v)$.} Secondly, if there is $v$ remaining with at least two neighbors, we again consider all options.
        If $v \in S$, we know that all of $\neiOp'(v)$ have to be in $\nei{S}$. We branch into two cases in which we either remove $v$, or remove all of $\neiOp'(v)$ from $G$. In both cases, we decrease $k$ by the number of removed vertices.
    \end{itemize}
\end{description}

Notice that the case distinction is exhaustive, that is, if none of the cases apply $\cnei{u}$ is a clique without neighbors.
To give an upper bound on the runtime of the algorithm, notice that in each of the cases we either reduce the size of the parameter by at least one (Case 1b, Case 2a) or do branching. But in each of the branching steps (Case 1a, Case 1c, Case 2b) we make progress decreasing $k$ by 1 in the first branch, and at least 2 in the second branch. Hence they have the branching vector $(1,2)$, which is known~\cite[Chapter 3]{cygan2015parameterized} to result in a runtime of $1.6181^k$.  Therefore, the runtime of the recursive algorithm is upper-bounded by $1.6181^k n^{\mathcal{O}(1)}$, where the final runtime is achieved by multiplying by $n$ (the number of initial guesses of the vertex $u$). The proof of correctness follows from a simple induction. This concludes the algorithm. 
\end{proof}
\fi


\begin{comment}
\subsection{Restricted Secluded Subgraphs in \texorpdfstring{$\alpha$}{α}-Bounded Graphs}\label{sec:restricted}

In this subsection, considering  $\alpha$-bounded graphs as an input, we look  two problems (i) \textsc{Out-Secluded $\propOp$-Subgraph} (ii) \textsc{Total-Secluded $\propOp$-Subgraph}   where $\propOp$ be any graph property that is verifiable in time $n^{\bigO{1}}$. 

More interestingly, notice that the algorithm from \Cref{thm:alpha_bounded_fpt} is a branching algorithm that can be extended to an enumeration algorithm. If we change it to output all found solutions, we can enumerate all secluded $\alpha$-bounded graphs in the same time. Furthermore, if $G$ is already $\alpha$-bounded, we can use this insight to solve \textsc{Out-Secluded $\propOp$-Subgraph} efficiently, if $\propOp$ is verifiable efficiently.

\begin{theorem}
\label{thm:alpha_eff}
  Let $\propOp$ be any graph property that is verifiable in time $n^{\bigO{1}}$. Then, \textsc{Out-Secluded $\propOp$-Subgraph} in $\alpha$-bounded graphs is solvable in time $(2\alpha + 2)^k n^{\alpha + \bigO{1}}$.
\end{theorem}
\iflong
\begin{proof}
  The algorithm first runs the enumeration algorithm for \textsc{Out-Secluded $\alpha$-BS} from \Cref{thm:alpha_bounded_fpt} on $G$. For every enumerated vertex subset $S$, we verify whether it satisfies $\propOp$. If that is the case, we output $S$. Otherwise, there is no solution.

  Because $G$ is $\alpha$-bounded, so is every induced subgraph of $G$. Therefore, any solution to \textsc{Out-Secluded $\propOp$-Subgraph} must also be $\alpha$-bounded. Thus, it will be enumerated by the algorithm from \Cref{thm:alpha_bounded_fpt}. The runtime only increases by the additional polynomial overhead from verifying the property $\propOp$.
\end{proof}
\fi

Again, the same is possible for the total neighborhood setting. 
\begin{corollary}\label{thm:alpha_eff_tot}
  Let $\propOp$ be any graph property that is verifiable in time $n^{\bigO{1}}$. Then, \textsc{Total-Secluded $\propOp$-Subgraph} in $\alpha$-bounded graphs is solvable in time $(\alpha + 1)^k n^{\alpha + \bigO{1}}$.
\end{corollary}
\end{comment}

\fi

\section{Secluded \texorpdfstring{$\mathcal{F}$}{F}-Free subgraph and Secluded DAG}
\label{chap:hardness}

\zzcommand{\free}{\textsc{Out-Secluded $\mathcal{F}$-Free Subgraph}}
\zzcommand{\freet}{\textsc{Total-Secluded $\mathcal{F}$-free Subgraph}}

\begin{comment}
In this section, we consider finding secluded $\mathcal{F}$-free subgraphs, a problem that received significant attention in the undirected setting and was proven FPT with different algorithms~\cite{golovach2020finding,jansen2023single}. For out-neighborhoods, the problem is defined as follows.

\begin{tcolorbox}[enhanced,title={\color{black} {\free{}}}, colback=white, boxrule=0.4pt,
	attach boxed title to top left={xshift=.3cm, yshift*=-2.5mm},
	boxed title style={size=small,frame hidden,colback=white}]
	
	\textbf{Input:}  
  A directed graph $G$, a weight function $\wOp \colon \ve{G} \to \N$, and integers $w,k \in \N$\\

	\textbf{Output:}
  Decide if there is a set $S \subseteq \ve{G}$ such that $\induced{G}{S}$ includes no $F \in \mathcal{F}$ as an induced subgraph, $\abs{\outNei{S}} \le k$, and $\w{S} \ge w$.
\end{tcolorbox}
\end{comment}

\zzcommand{\prob}{\textsc{Out-Secluded $\mathcal{F}$-Free WCS}}
In this section, we show that \free{} is W[1]-hard for almost all choices of $\mathcal{F}$, even if we enforce weakly connected solutions. Except for some obscure cases, we establish a complete dichotomy when \textsc{Out-Secluded $\mathcal{F}$-Free Weakly Connected Subgraph} (\prob{}) is hard and when it is FPT. This is a surprising result compared to undirected graphs and shows that out-neighborhood behaves completely different. 
% In \Cref{sec:ffree_ab}, we consider the setting of $\alpha$-bounded graphs again, where \textsc{Out-Secluded $\mathcal{F}$-Free Weakly Connected Subgraph} is efficiently solvable.
% \jonas{remove subsection heading}
% \iflong
% \subsection{Hardness of \textsc{Out-Secluded $\mathcal{F}$-Free WCS}}\label{sec:hardness_f_free}
% \fi
The same problem was studied before in undirected graphs, where it admits FPT-algorithms using recursive understanding~\cite{golovach2020finding} and branching on important separators~\cite{jansen2023single}. For total neighborhood, both algorithms still work efficiently since the seclusion condition acts exactly the same as in undirected graphs. However, the result changes when we look at out-neighborhood, where we show hardness in \Cref{thm:f-free-hard-always} for almost all choices of $\mathcal{F}$. 

% While for recursive understanding, it is not clear how to apply this approach to a unidirectional neighborhood, the reason that the branching algorithm does not work is more subtle. Intuitively, it is not enough to fix a part of the solution and branch on vertices in its neighborhood or its set of reachable vertices because it might be worth extending the solution by vertices in the in-neighborhood. These do not have to be limited by $k$; however, they can still be added to a weakly connected solution and increase the weight. \todo{S: what are we talking about, since line 483?}

Note that if $\mathcal{F}$ corresponds to the set of all cycles, the problem is finding a secluded DAG, a very natural extension of secluded trees, studied in~\cite{golovach2020finding,donkers2023finding}. Because this set is infinite, we also cannot solve \textsc{Total-Secluded WC DAG} in the same way as the undirected $\mathcal{F}$-free problem. 

To formalize for which kind of $\mathcal{F}$ the problem becomes W[1]-hard, we need one more definition.
\begin{definition}[Inward Star]
  We say that a directed graph $G$ is an \emph{inward star}, if there is one vertex $v \in \ve{G}$ such that $\e{G} = \set{\died{u}{v}}{u \in \ve{G} \setminus \set{v}}$, that is, the underlying undirected graph of $G$ is a star and all edges are directed towards the center.
\end{definition}

Due to the nature of the construction, we show that if no inward star contains any $F \in \mathcal{F}$ as a subgraph, the problem becomes W[1]-hard. Besides this restriction, the statement holds for all non-empty $\mathcal{F}$, even families with only a single forbidden induced subgraph. Later, we explain how some other cases of $\mathcal{F}$ can be categorized as FPT or W[1]-hard.

\restateffree*
\iflong \begin{proof}
\else \begin{proof}[Proof Sketch]
\fi
\begin{figure}
  \centering
  \hfill
  \begin{subfigure}{0.3\textwidth}
    \centering
    \includegraphics[width=0.7\textwidth,page=2]{figures/incidence_graph_reduction}
    \caption{An undirected graph $G$ with maximum clique size 3.}
  \end{subfigure}
  \hfill
  \begin{subfigure}{0.54\textwidth}
    \centering
    \includegraphics[width=0.9\textwidth,page=1]{figures/incidence_graph_reduction}
    \caption{The output graph $G'$ of our reduction function for $G$.}
  \end{subfigure}
  \hfill
  \caption{The construction of the reduction in the proof of \Cref{thm:f-free-hard-always}. Shaded vertices represent a complete graph on $k+2$ copies of $F$, and every $v \in V_V$ is connected to every copy. No solution in $G'$ can include a vertex in a copy of $F$ or $V_V$. Thus, any secluded subset $S$ of $V_E$ of size $\binom{k}{2}$ corresponds the clique $\outNei{S}$ in $G$.}\label{fig:clique_reduction}
\end{figure}

    Let $F \in \mathcal{F}$. We give a reduction from \textsc{Clique}, inspired by~\cite{fomin2013parameterized}. Given an undirected graph $G$ and $k$ and $w$, consider the incidence graph $\incG{G}$, which we modify in the following. We add $k+2$ copies of $F$ and a single vertex $s$. For every edge $e = \uned{u}{v} \in E$, we add the edges $\died{e}{u}$, $\died{e}{v}$, and $\died{e}{s}$. Furthermore, we add an edge from every vertex $v \in V_V$ to every vertex in every copy of $F$. For two different copies $F_1$ and $F_2$ of $F$, we add edges in both directions between every vertex $v_{f1} \in F_1$ and $v_{f2}$ in $F_2$. Denote this newly constructed graph by $G'$. Finally, set $k' \coloneqq k$ and $w \coloneqq \binom{k}{2} + 1$.
    See \Cref{fig:clique_reduction} for a visualization of the graph construction.
\ifshort
    We defer the remaining proof to the full version.
\else
    Notice that $k \le \abs{\ve{G}}$, therefore the construction has polynomial size, and $k' + w$ depends only polynomially on $k$.

    We show that there is a clique of size $k$ in $G$ if and only $G'$ has a weakly connected $k$-out-secluded $\mathcal{F}$-free subgraph of weight at least $w$.
    If $C \subseteq \ve{G}$ forms a $k$-clique in $G$, consider the set of vertices $S$ in $G'$ consisting of $s$ and all edge vertices $v_{\uned{a}{b}}$ where both endpoints $a$ and $b$ lie in the clique $C$. This is clearly weakly connected and has size $\binom{k}{2}+1$. Also, we have $\outNei{S} = C$ and $\abs{C} = k$.

    For the other direction, consider a solution $S$ for $G'$. Notice that no vertex from the copies of $F$ or from $V_V$ can be in $S$; otherwise, $S$ has at least $k+1$ out-neighbors. To ensure the connectivity, if $k > 1$, $s$ must be in $S$. Hence, the out-neighborhood of $S$ must be part of $V_V$. Because $\abs{\outNei{S}} \le k$ and $\abs{S} \setminus \set{s} = \binom{k}{2}$, clearly $S$ must consist of the edges of a clique in $G$.
\fi
\end{proof}

Note that \Cref{thm:f-free-hard-always} is very general and excludes many properties $\propOp$ immediately from admitting FPT-algorithms. %Specifically, it holds also for families with only a single element. 
% \begin{corollary}
%     Let $F$ be a directed graph that is not a subgraph of an inward star. Then, \prname{Out}{$F$-Free WCS} is W[1]-hard with parameter $k + w$ for unit weights.
% \end{corollary}
Another interesting example are DAGs, where we choose $\mathcal{F}$ to be the set of all directed cycles. Although this set is not finite, hardness follows from \Cref{thm:f-free-hard-always}.

\restatedag*

\Cref{thm:f-free-hard-always} shows hardness for almost all cases how $\mathcal{F}$ could look like. We want to get as close to completing all possible cases as possible and, therefore, analyze some of the more difficult cases in the following paragraphs.

\subparagraph*{Trivial Cases}
If $\mathcal{F}$ contains the graph with only a single vertex, the empty set is the only $\mathcal{F}$-free solution. If $\mathcal{F}$ contains two vertices connected by a single edge, weakly connected solutions can only consist of a single vertex or contain bidirectional edges $(u,v)$ and $(v,u)$ for $u,v \in \ve{G}$. Both of these problems are clearly solvable in FPT-time, the second one via our algorithm for \textsc{Secluded Clique}. Additionally, if $\mathcal{F} = \emptyset$, we can clearly choose the maximum weight component of $G$ as the solution.

\subparagraph*{Independent Sets} 
An independent set is a subgraph of an inward star. One kind of graph that cannot be included in $\mathcal{F}$ such that \Cref{thm:f-free-hard-always} shows hardness are independent sets. Surprisingly, the problem becomes FPT if $\mathcal{F}$ includes an independent set of any size. First, notice that independent sets that are part of $\mathcal{F}$ can be ignored if there is a smaller independent set in $\mathcal{F}$. Suppose the size of the smallest independent set is $\alpha+1$. This means that any solution must be an $\alpha$-bounded graph, i.e., a graph without an independent set of size $\alpha$.

We have shown already how these can be enumerated efficiently with the algorithm in the proof of \Cref{thm:alpha_bounded_fpt}. For every enumerated subgraph, we can simply check if it is $\mathcal{F}$-free in time $\abs{\mathcal{F}}n^{\|\mathcal{F}\|}$. Therefore, this case is also FPT with parameter $k$ if $\mathcal{F}$ is finite. 
\begin{theorem}
  Let $\mathcal{F}$ be a finite family of directed graphs that contains an independent set. Then \prob{} is solvable in time $\abs{\mathcal{F}}(2\alpha + 2)^kn^{\|\mathcal{F}\| + \alpha + \bigO{1}}$.
\end{theorem}

\subparagraph*{Inward Stars}
Consider what happens when $\mathcal{F}$ contains an inward star $F$ with two leaves. Then, the weakly connected $F$-free graphs are exactly the rooted trees where the root could be a cycle instead of a single vertex, with potentially added bidirectional edges. For $F \in \mathcal{F}$, we do not have a solution, but we conjecture that the FPT branching algorithm for \textsc{Secluded Tree} by \cite{donkers2023finding} can be transferred to the directed setting. 

If $\mathcal{F} = \set{F}$ for a star $F$ with more than two leaves, the problem remains W[1]-hard. We can slightly modify the construction in the proof of \Cref{thm:f-free-hard-always} such that there is not one extra vertex $s$, but one extra vertex $s_e$ for every $e \in \e{G}$. We connect all $s_e$ internally in a directed path. A solution for \prob{} can then include the whole extra path and the edges encoding the clique. This avoids inward star structures with more than two leaves, showing hardness in this case. However, if $\mathcal{F}$ additionally contains a path as a forbidden induced subgraph, the approach does not work to show hardness. All other cases are resolved, allowing us to conjecture the following dichotomy for a single forbidden induced subgraph.
\begin{conjecture}
  Let $F$ be a directed graph. Then, if $F$ is an independent set or an inward star with at most two leaves, \prob{} is FPT with parameter $k$. Else, \prob{} is W[1]-hard with parameter $k + w$ for unit weights.
\end{conjecture}

\subparagraph*{Remaining Cases}
This resolves almost all possible cases for $\mathcal{F}$, except for a few cases which are difficult to characterize. For example, we could have $\{F_s, F_p\} \subseteq \mathcal{F} $, where $F_s$ is an inward star and $F_p$ is a path, which makes a new connecting construction instead of $s$ and the $s_e$ necessary. The above characterization is still enough to give an understanding of which cases are hard and which are FPT for all natural choices of $\mathcal{F}$. 

% \jonas{Imo, this is the least interesting section, since it only improves on a more general algorithm in some specific cases. Noone ever cares, so should be appendix material.}
\begin{comment}
\subsection{Finding DAGs and \texorpdfstring{$\mathcal{F}$}{F}-Free Subgraphs in \texorpdfstring{$\alpha$}{α}-Bounded Graphs}\label{sec:ffree_ab}

In the last section, we showed that \textsc{Out-Secluded WC DAG} and \prob{} are W[1]-hard in general graphs in almost all cases.
Now, we consider the case when $G$ is a tournament or $\alpha$-bounded. Here, the picture changes. Notice that we can even use \Cref{thm:alpha_eff} from the previous chapter to conclude tractability for both problems on $\alpha$-bounded graphs.
However, we develop more constructive and faster algorithms, first for finding DAGs in tournaments, then for finding $\mathcal{F}$-free subgraphs in $\alpha$-bounded graphs.
Note that we can even ignore the connectivity constraint due to the locality property that we get from \Cref{lem:ind_set_reaches}. Using our lemmas for tournaments and $\alpha$-bounded graphs, we show in this section that \prname{Out}{Weakly Connected DAG} from \Cref{sec:hardness_f_free} is FPT if the base graph is a tournament or general $\alpha$-bounded graph. Recall that in this problem, we are given a directed vertex-weighted graph $G$, and we are looking for a maximum weight subset $S \subseteq V$ that induces a DAG with out-neighborhood $\abs{\outNei{S}} \le k$. For general graphs, this problem turned out to be W[1]-hard in \Cref{thm:f-free-hard-always}. Now, we consider the case when $G$ is a tournament or $\alpha$-bounded.

Remember that a tournament is a directed graph, where every edge is present in exactly one direction. The next two lemmas give some intuition on what DAGs in tournaments and their neighborhoods look like. Note that we refer to directed cycles with 3 edges as triangles.

\iflong
\begin{lemma}
\else
\begin{lemma}[$\star$]
\fi
\label{cor:DAG_triangle_free}
  A tournament graph $T$ is a DAG if and only if $T$ has no cycle of length 3 as an induced subgraph.
  More specifically, if $v \in \ve{T}$ is part of a cycle, $v$ is part of a triangle.
\end{lemma}
\iflong
\begin{proof}
  The first statement is an immediate implication of the second one.
  Let $v_1, v_2, \ldots, v_k$ be the smallest cycle that $v_1$ is part of and suppose $k > 3$. Then, $(v_1, v_3) \in \e{T}$ since otherwise $\died{v_3}{v_1} \in \e{T}$ by the tournament property and $v_1, v_2, v_3$ would be a triangle. Therefore, we can shortcut the cycle of length $k$ to arrive at a cycle of length $k-1$, a contradiction.
\end{proof}
\fi

\iflong
\begin{lemma}
\else
\begin{lemma}[$\star$]
\fi
\label{lem:neighsubset}
  Let $T$ be a tournament, $s \in \ve{T}$, and $v \in \outNei{s}$. Then $v$ is part of a triangle or $\outNei{v} \subset \outNei{s}$.
\end{lemma}
\iflong
\begin{proof}
  Let $u \in \outNei{v} \setminus \outNei{s}$. Then $u, s, v$ is a triangle including $v$. For strict inclusion, notice that $v \in \outNei{s} \setminus \outNei{v}$.
\end{proof}
\fi

These strong structural properties are enough to find secluded DAGs in tournaments using a simple branching algorithm. We use a similar of branching algorithm as in \Cref{chap:tour}, where we initially fix a small part of the solution to turn the problem into a highly local one.

\iflong
\begin{theorem}
\else
\begin{theorem}[$\star$]
\fi
\label{thm:tournament_k}
  \prname{Out}{DAG} is solvable in time $3^k n^{\bigO{1}}$, if $G$ is a tournament.
\end{theorem}
\iflong
\begin{proof}
  \newcommand{\curr}{\coutNei{u}}
    \begin{algorithm}[t]
      \caption{The branching algorithm for \prname{Out}{DAG} that returns a solution including the vertex $u \in \ve{G}$.}
      \label{alg:dag_tour}
      \DontPrintSemicolon
      \SetKwFunction{FMain}{DAG}
      \SetKwProg{Fn}{def}{:}{}
      \Fn{\FMain{$G$, $\omega$, $w$, $k$, $u$}}{
        \uIf{$k < 0$} {
          \textbf{abort}\; 
        }
        \uElseIf{$\curr$ is a solution} {
            \KwRet $\curr$\;
        }
        \ElseIf{there is a triangle $C$ with $C \cap \curr{} \ne \emptyset$} {
            \ForEach{$v \in C$} {
                Call \FMain{$G-v$, $\omega$, $w$, $k-1$, $u$}\;
            }
        }
      }
    \end{algorithm}
    
  Let $(G, \wOp, w, k)$ be an instance of \prname{Out}{DAG} and again guess a vertex $u \in \ve{G}$ that should be the only source in the desired solution $S$. We describe a branching algorithm and execute it for all choices of $u$.
  
  The only branching rule eliminates triangles overlapping $\coutNei{u}$ since at least one of the vertices in the triangle has to be part of $\outNei{S}$.
  Formally, if there is a triangle $C$ with $C \cap \coutNei{u} \ne \emptyset$, we branch on which vertex $v$ of $C$ to put in the neighborhood of $S$.
  This means that we remove $v$ from the graph and decrease $k$ by one. If $k$ decreases below 0, there is no solution. Otherwise if the branching rule is no longer applicable, we show that we have found a solution. The algorithm is also described in \Cref{alg:dag_tour}.
  
  % \subparagraph*{Correctness}
  Since there is always a unique source in a tournament DAG, running the algorithm for all choices of $u$ considers all possible solutions. 
  Notice that the branching rule is correct since one vertex of $C$ must be already in the neighborhood of $u$. Since the triangle cannot be included in a solution, at least one vertex of $C$ must be in the final neighborhood of any solution containing $u$ in the current branch.

  If there are no triangles intersecting $\coutNei{u}$, because of \Cref{lem:neighsubset}, we have $\outNei{\coutNei{u}} = \emptyset$. Also by \Cref{cor:DAG_triangle_free}, $\coutNei{u}$ must be a DAG. Therefore, $\coutNei{u}$ must be the maximum solution in this branch that includes $u$ and the algorithm is correct. 

  % \subparagraph*{Runtime}
  Our branching rule has 3 branches. Its condition can be checked in polynomial time. Furthermore, iterating over all choices of $u \in \ve{G}$ only adds a factor of $n$.
\end{proof}
\fi

\newcommand{\ffree}{$\mathcal{F}$-free}
Next we show how to construct an algorithm for \prob{}, again by branching on forbidden substructures. The algorithm is especially efficient if the graphs in $\mathcal{F}$ are small compared to $\alpha$.
We additionally define $\|\mathcal{F}\|\coloneqq \max_{F \in \mathcal{F}} \abs{\ve{F}}$. Note that we treat $\|\mathcal{F}\|$ as a constant, since $\mathcal{F}$ is part of the problem itself.

\iflong
\begin{theorem}
\else
\begin{theorem}[$\star$]
\fi
\label{thm:ffree_in_ab}
  If $G$ is \ab{}, \free{} is solvable in time $\abs{\mathcal{F}}\cdot \max\set{3^k,(2\|\mathcal{F}\|)^k}n^{\alpha + \|\mathcal{F}\| + \bigO{1}}$.
\end{theorem}
\iflong
\begin{proof}

    \newcommand{\curr}{\coutNei{\coutNei{U}}}
    \begin{algorithm}[t]
      \caption{The branching algorithm for \free{} that returns a solution including the set $U \subseteq \ve{G}$.}
      \label{alg:freeab}
      \DontPrintSemicolon
      \SetKwFunction{FMain}{$\mathcal{F}$-Free}
      \SetKwProg{Fn}{def}{:}{}
      \Fn{\FMain{$G$, $\omega$, $w$, $k$, $U$}}{
        \uIf{$k < 0$} {
          \textbf{abort}\; 
        }
        \uElseIf{$\curr$ is a solution} {
            \KwRet $\curr$\;
        }
        \uElseIf{there is $C \subseteq \curr$ with $\induced{G}{C}$ isomorphic to some $F \in \mathcal{F}$} {
            \If{$C \subseteq U$} {
              \textbf{abort}\; 
            }
            \ForEach{$v \in \bigcup_{w \in C} \shor{w}$} {
                Call \FMain{$G-v$, $\omega$, $w$, $k-1$, $U$}\;
            }
        }
        \ElseIf{there is $w \in \outNei{\curr{}}$} {
            \ForEach{$v \in \shor{w}$} {
                Call \FMain{$G-v$, $\omega$, $w$, $k-1$, $U$}\;
            }
        }
      }
    \end{algorithm}
    
  \newcommand{\alg}{\textsf{$\mathcal{F}$-Free}}
  The algorithm technique is similar to finding $\alpha$-bounded subgraphs as in \Cref{thm:alpha_bounded_fpt}. Remember that we write $\shor{v}$ for the vertices in any shortest path from $U$ to $v$, excluding $U$.
  Let $(G,\wOp,w,k)$ be a \free{} instance. We again initially guess a non-empty independent set $U \subseteq \ve{G}$ and run the algorithm for all choices of $U$. We want to find a solution $S \supseteq U$ such that every $v \in S$ is reachable from $U$ via two edges, that is, $S \subseteq \coutNei{\coutNei{U}}$. We give a recursive branching algorithm that finds an optimal solution under these constraints, which is also described in \Cref{alg:freeab}.
  
    When a vertex should be part of the final neighborhood, we can delete it and decrease $k$ by one. If $k$ decreases below 0, or if $\curr$ is a solution to the instance, we return with a base case. Otherwise we apply the following branching rules and repeat the algorithm for all non-empty independent sets $U \subseteq \ve{G}$ of size at most $\alpha$.
    
    \begin{description}
        \item[Case 1. $\curr$ is not $\mathcal{F}$-free.] In this case, there must be a set $C \subseteq \curr$ that induces a subgraph isomorphic to some $F \in \mathcal{F}$. If $C \subseteq U$ there can clearly not be a solution $S \supseteq U$.
        
        Since not all of $C$ can be part of $S$, there is a vertex $w \in C \setminus S$. This means that either $w$ or a vertex on every path from $U$ to $w$ must be in $\outNei{S}$, that is, a vertex in $\shor{w}$.
        Thus, one of $\bigcup_{w \in I} \shor{w}$ must be part of the out-neighborhood of $S$ and we branch on all of these vertices. For one vertex, delete it and decrease $k$ by 1.
    
        \medskip
        \item[Case 2. $\curr$ is $\mathcal{F}$-free, but has additional neighbors.] Let $w \in \outNei{\curr}$ be one such neighbor.
        Since $w$ is not reachable from $U$ via at most two edges, we should not include it in the solution. Again, we branch on $\shor{w}$, a path of length 3, and one of its vertices must be in $\outNei{S}$.
    \end{description}
  
  % \subparagraph*{Correctness}
  Since any induced subgraphs of an \ab{} graph is also \ab{}, the maximum weight secluded induced \ffree{} subgraph will also be \ab{}.
  Therefore, by \Cref{lem:ind_set_reaches}, there is an independent set $U$ such that for the maximum weight solution $S$ we have $S \subseteq \coutNei{\coutNei{U}}$.
  Thus, if we can find the maximum solution for $U$ if one exists in every iteration with our branching algorithm, the total algorithm is correct.
    The branching rules are a complete case distinction; if none of the rules apply, the algorithm reaches a base case.
    The remaining proof of correctness follows from a simple induction.
  
  % \subparagraph*{Runtime}
  Enumerating all independent sets can be done in time $n^{\alpha+1}$ since $G$ is \ab{}.
  For the first rule, for every $v \in C$, we have $\abs{\shor{v}} = 2$. This yields $2\abs{C} \le 2\|\mathcal{F}\|$ branches. The other rule has 3 branches, all of which lower $k$ by at least 1. All rules can be checked and applied in polynomial time since we consider $\|\mathcal{F}\|$ to be constant. 
\end{proof}
\fi

\begin{comment}
This in itself is already an interesting result.
Note that we could also apply it for finding DAGs in tournaments by \Cref{cor:DAG_triangle_free}, but it would give a worse run time of $6^kn^{\bigO{1}}$.
We now show how to apply this algorithm for finding DAGs in general $\alpha$-bounded graphs.

\iflong
\begin{lemma}
\else
\begin{lemma}[$\star$]
\fi
\label{lem:alpha_short_cycle}
  Let $G = (V,E)$ be $\alpha$-bounded. If $G$ is not acyclic, $G$ has a cycle with at most $2\alpha + 1$ vertices.
\end{lemma}
\iflong
\begin{proof}
  Let $C$ be the shortest cycle in $G$ and suppose for the sake of contradiction that $\abs{C} \ge 2\alpha + 2$.
  Consider every second vertex of $C$, leaving out the last one in case $\abs{C}$ is odd. Name this set $U$. Since $G$ is $\alpha$-bounded, we know that $U$ cannot be an independent set. The cycle does not include edges between vertices of $U$, therefore there must be $u_1, u_2 \in U$ with an edge $(u_1, u_2) \in E$.
  But this allows us to shortcut $C$ into a smaller cycle via this edge $(u_1, u_2)$, a contradiction.
\end{proof}
\fi

If we would define a set $\mathcal{F}$ naively such that DAGs are exactly \ffree{} graphs, then $\mathcal{F}$ has to be the set of all cycles, so $\|\mathcal{F}\| = \infty$. Using \Cref{lem:alpha_short_cycle} however, it is enough find a subgraph that includes no cycle with at most $2 \alpha + 1$ vertices. Therefore a set with $\|\mathcal{F}\| = 2\alpha + 1$ is enough and we arrive at the following result.

\begin{corollary}
    \label{cor:dag_in_alpha_bounded}
  \prname{Out}{Weakly Connected DAG} is solvable is time $(4\alpha + 2)^kn^{\bigO{1}}$ if $G$ is \ab{}.
\end{corollary}

\Cref{thm:ffree_in_ab} holds for total neighborhood as well. The algorithm can be applied in almost the same way. We simply pick $v$ from the total neighborhood instead of the out-neighborhood of $\coutNei{\coutNei{S}}$ in the second branching rule. This ensures that we branch on all possible neighbors.

\begin{corollary}\label{cor:ffreet_in_ab}
  In \ab{} graphs, \freet{} is solvable in time $\max\set{3^k,(2\|\mathcal{F}\|)^k}n^{\alpha + \bigO{1}}$.
\end{corollary}

For DAGs, the most efficient algorithm for general $\alpha$-bounded graphs we know follows from \Cref{thm:alpha_eff}.

\begin{corollary}
  In \ab{} graphs, \textsc{Out-Secluded DAG} is solvable in time $(2\alpha + 2)^kn^{\bigO{1}}$.
\end{corollary}
\fi
\end{comment}


% \subsection{\textsc{Secluded DAG} in Tournaments and \texorpdfstring{$\alpha$}{α}-Bounded Graphs}\label{sec:dag_in_alpha_bounded}

% Using our lemmas for tournaments and $\alpha$-bounded graphs, we show in this section that \prname{Out}{Weakly Connected DAG} from \Cref{sec:hardness_f_free} is FPT if the base graph is a tournament or general $\alpha$-bounded graph. Remember that in this problem, we are given a directed vertex-weighted graph $G$ and we are looking for a maximum weight subset $S \subseteq V$ that induces a DAG with out-neighborhood $\abs{\outNei{S}} \le k$. For general graphs, this problem turned out to be W[1]-hard in \Cref{thm:f-free-hard-always}. Now we consider the case when $G$ is a tournament or $\alpha$-bounded.

% We again start with tournaments. The next two lemmas give some intuition how DAGs in tournaments and their neighborhoods look like. Note that we refer to directed cycles with 3 edges as triangles.

% \begin{lemma}\label{cor:DAG_triangle_free}
%   A tournament graph $T$ is a DAG if and only if $T$ has no cycle of length 3 as an induced subgraph.
%   More specifically, if $v \in \ve{T}$ is part of a cycle, $v$ is part of a triangle.
% \end{lemma}
% \begin{proof}
%   The first statement is an immediate implication of the second one.
%   Let $v_1, v_2, \ldots, v_k$ be the smallest cycle that $v_1$ is part of and suppose $k > 3$. Then, $(v_1, v_3) \in \e{T}$ since otherwise $\died{v_3}{v_1} \in \e{T}$ by the tournament property and $v_1, v_2, v_3$ would be a triangle. Therefore, we can shortcut the cycle of length $k$ to arrive at a cycle of length $k-1$, a contradiction.
% \end{proof}

% \begin{lemma}\label{lem:neighsubset}
%   Let $T$ be a tournament, $s \in \ve{T}$, and $v \in \outNei{s}$. Then $v$ is part of a triangle or $\outNei{v} \subset \outNei{s}$.
% \end{lemma}
% \begin{proof}
%   Let $u \in \outNei{v} \setminus \outNei{s}$. Then $u, s, v$ is a triangle including $v$. For strict inclusion, notice that $v \in \outNei{s} \setminus \outNei{v}$.
% \end{proof}

% These strong structural properties are enough to find secluded DAGs in tournaments using a simple branching algorithm. We use the same style of branching algorithm as before, where we initially fix a small part of the solution to turn the problem into a highly local one.

% \begin{theoremE}[]\label{thm:tournament_k}
%   \prname{Out}{Weakly Connected DAG} is solvable in time $3^k n^{\bigO{1}}$, if $G$ is a tournament.
% \end{theoremE}
% \begin{proofE}
%   \newcommand{\alg}{\textsf{DAG}}
%   Given an \prname{Out}{DAG} instance $(G, \wOp, w, k)$ and a vertex $u \in \ve{G}$, we want to find a solution $S$ to the instance in which $u$ is the only source. We solve this problem with the following algorithm, called $\alg{}(G,\wOp,w,k,u)$, which we execute for all choices of $u \in \ve{G}$.
%   \begin{enumerate}[noitemsep,nolistsep]
%     \item If $k < 0$, abort.
%     \item \label{bra:tour}If there is a triangle $C$ with $C \cap \coutNei{u} \ne \emptyset$, iterate over all three $v \in C$. For each $v$, call $\alg{}(G-v,\wOp,w,k-1,u)$ and stop.
%     \item If $\w{\coutNei{u}} \ge w$, return $\coutNei{u}$. Abort otherwise.
%   \end{enumerate}

%   \subparagraph*{Correctness:}
%   Since there is always a unique source in a tournament DAG, running the algorithm for all choices of $u$ considers all possible solutions. 
%   Notice that the branching rule in \Cref{bra:tour} is correct since one vertex of $C$ must be already in the neighborhood of $u$. Since the triangle cannot be included in a solution, at least one vertex of $C$ must be in the final neighborhood of any solution containing $u$.

%   If there are no triangles intersecting $\coutNei{u}$, because of \Cref{lem:neighsubset}, we have $\outNei{\coutNei{u}} = \emptyset$. Also by \Cref{cor:DAG_triangle_free}, $\coutNei{u}$ must be a DAG. Therefore, $\coutNei{u}$ must be the maximum solution in this branch that includes $u$ and the algorithm is correct. 

%   \subparagraph*{Runtime.}
%   \Cref{bra:tour} has 3 branches. Its condition can be checked in polynomial time. Furthermore, iterating over all choices of $u \in \ve{G}$ only adds a factor of $n$.
% \end{proofE}

% \renewcommand{\ffree}{$\mathcal{F}$-free}
% Before we get to \ab{} graphs, we give an algorithm for a more general problem. Namely, we will reconsider the \ffree{} subgraph problem from \Cref{sec:hardness_f_free}. 

% \zzcommand{\free}{\textsc{Out-Secluded $\mathcal{F}$-free Subgraph}}
% \zzcommand{\freet}{\textsc{Total-Secluded $\mathcal{F}$-free Subgraph}}
% \begin{problem}{\free{}}
%   Given: & A directed graph $G$, a weight function $\wOp \colon V \to \N$, and numbers $w,k \in \N$\\
%   Task: & Decide if there is a set $S \subseteq \ve{G}$ such that $\induced{G}{S}$ includes no $F \in \mathcal{F}$ as an induced subgraph, $\abs{\outNei{S}} \le k$, and $\w{S} \ge w$.
% \end{problem}

% We additionally define $\|\mathcal{F}\|\coloneqq \max_{F \in \mathcal{F}} \abs{\ve{F}}$. Note that we treat $\|\mathcal{F}\|$ as a constant, since $\mathcal{F}$ is part of the problem itself.

% \begin{theorem}\label{thm:ffree_in_ab}
%   If $G$ is \ab{}, \free{} is solvable in time $\abs{\mathcal{F}}\cdot \max\set{3^k,(2\|\mathcal{F}\|)^k}n^{\alpha + \|\mathcal{F}\| + \bigO{1}}$.
% \end{theorem}
% \begin{proof}
%   \newcommand{\alg}{\textsf{$\mathcal{F}$-Free}}
%   The algorithm technique is similar to finding $\alpha$-bounded subgraphs as in \Cref{thm:alpha_bounded_fpt}. 
%   First enumerate all nonempty independent sets of size at most $\alpha$ in $G$.
%   For an independent set $U \subseteq \ve{G}$, we aim to the find the maximum weight secluded \ffree{} subgraph $S \subseteq \ve{G}$ such that every $v \in S$ is reachable from $U$ via two edges, that is $S \subseteq \coutNei{\coutNei{U}}$. For such a set $U$ we run the following algorithm, called $\alg{}(G,\wOp,w,k,U)$.
%   \begin{enumerate}[nolistsep,noitemsep]
%     \item If $k < 0$, abort.
%     \item \label{bra:ffree_alpha1}
%       If there is $C \subseteq \coutNei{\coutNei{U}}$ such that $\induced{G}{C}$ is isomorphic to some $F \in \mathcal{F}$, differentiate the following cases. \begin{enumerate}[nolistsep,noitemsep,label=\alph*.]
%       \item If $C \subseteq U$, abort.
%       \item Otherwise, for each vertex $v' \in \bigcup_{v \in C}\shor{v}$, call $\alg{}(G-v',\wOp,w,k-1,U)$ and stop.
%       \end{enumerate}
%     \item \label{bra:ffree_alpha2}
%       If there is $v \in \outNei{\coutNei{\coutNei{U}}}$, iterate over all $v' \in \shor{v}$, call $\alg{}(G-v',\wOp,w,k-1,U)$ and stop.
%     \item \label{bra:ffree_alpha3} If $\w{\coutNei{\coutNei{U}}} \ge w$, return $\coutNei{\coutNei{U}}$. Abort otherwise.
%   \end{enumerate}

%   \paragraph{Correctness:}
%   Since any induced subgraphs of an \ab{} graph is also \ab{}, the maximum weight secluded induced \ffree{} subgraph will also be \ab{}.
%   Therefore, by \Cref{lem:ind_set_reaches}, there is an independent set $U$ such that for the maximum weight solution $S$ we have $S \subseteq \coutNei{\coutNei{U}}$.
  
%   We now show that the algorithm steps are correct.
%   For \Cref{bra:ffree_alpha1}, note the any copy of an $F \in \mathcal{F}$ cannot be part of a solution. By the same argument as before, this means that one vertex in $\bigcup_{v \in C}\shor{v}$ must be in the neighborhood of the solution, so the rule is correct. If the branching rule is no longer applicable, then $\induced{G}{\coutNei{\coutNei{U}}}$ must be \ffree{}, and so must be any subset.

%   For \Cref{bra:ffree_alpha2}, we limit ourselves to finding \ffree{} subgraphs $S$ such that every $v \in S$ is reachable from $U$ via at most two edges. Therefore, we do not have to consider including $v$ into the solution. Therefore, either $v$ itself must be part of the final neighborhood or some vertex $w \in \shor{v}$.

%   If we reach \Cref{bra:ffree_alpha3} and the weight condition holds, $\coutNei{\coutNei{U}}$ must be the maximum solution for this branch.

%   \paragraph{Runtime:}
%   Enumerating all independent sets can be done in time $n^{\alpha+1}$ since $G$ is \ab{}.
%   For \Cref{bra:ffree_alpha1}, for every $v \in C$, we have $\abs{\shor{v}} = 2$. This yields $2\abs{C} \le 2\|\mathcal{F}\|$ branches. \Cref{bra:ffree_alpha2} has 3 branches, all of which lower $k$ by at least 1. All rules can be checked and applied in polynomial time since we consider $\|\mathcal{F}\|$ to be constant. For the specific term, notice that we can check for every ordered vertex set of size $\|\mathcal{F}\|$ in time $\abs{\mathcal{F}}n^{\bigO{1}}$ if it is isomorphic to an $F \in \mathcal{F}$.
% \end{proof}

% This in itself is already an interesting result. Note that we could also apply it for finding DAGs in tournaments by \Cref{cor:DAG_triangle_free}, but it would give a worse run time of $6^kn^{\bigO{1}}$. We now show how to apply this algorithm for finding DAGs in general $\alpha$-bounded graphs.

% \begin{lemma}\label{lem:alpha_short_cycle}
%   Let $G = (V,E)$ be $\alpha$-bounded. If $G$ is not acyclic, $G$ has a cycle with at most $2\alpha + 1$ vertices.
% \end{lemma}
% \begin{proof}
%   Let $C$ be the shortest cycle in $G$ and suppose for the sake of contradiction that $\abs{C} \ge 2\alpha + 2$.
%   Consider every second vertex of $C$, leaving out the last one in case $\abs{C}$ is odd. Name this set $U$. Since $G$ is $\alpha$-bounded, we know that $U$ cannot be an independent set. The cycle does not include edges between vertices of $U$, therefore there must be $u_1, u_2 \in U$ with an edge $(u_1, u_2) \in E$.
%   But this allows us to shortcut $C$ into a smaller cycle via this edge $(u_1, u_2)$, a contradiction.
% \end{proof}

% If we would define a set $\mathcal{F}$ naively such that DAGs are exactly \ffree{} graphs, then $\mathcal{F}$ has to be the set of all cycles, so $\|\mathcal{F}\| = \infty$. Using \Cref{lem:alpha_short_cycle} however, it is enough find a subgraph that includes no cycle with at most $2 \alpha + 1$ vertices. Therefore a set with $\|\mathcal{F}\| = 2\alpha + 1$ is enough and we arrive at the following result.

% \begin{theorem}\label{cor:dag_in_alpha_bounded}
%   \prname{Out}{Weakly Connected DAG} is solvable is time $(4\alpha + 2)^kn^{\bigO{1}}$ if $G$ is \ab.
% \end{theorem}

% Additionally, \Cref{thm:ffree_in_ab} and therefore also \Cref{cor:dag_in_alpha_bounded} hold for total neighborhood as well. The algorithm can be applied in almost the same way. We simply pick $v$ from the total neighborhood instead of the out-neighborhood of $\coutNei{\coutNei{S}}$ in \Cref{bra:ffree_alpha2}. This ensures that we branch on all possible neighbors.

% \begin{corollary}\label{cor:ffreet_in_ab}
%   In \ab{} graphs, \freet{} is solvable in time $\max\set{3^k,(2\|\mathcal{F}\|)^k}n^{\bigO{1}}$.
% \end{corollary}



\bibliography{main}

% \iflong
% \newpage
% %\section{Secluded \texorpdfstring{$d$}{d}-Edge-Connected Subgraphs}\label{sec:d_edge}

In this section, we consider a similar undirected problem to strong connectivity. In many ways 2-edge-connectivity is a comparable concept, since strong connectivity also requires at least two different paths between every pair of vertices. We target a generalized version of the problem and show that the following problem is non-uniformly FPT with a similar recursive understanding approach as in \Cref{sec:scc}.

\zzcommand{\prob}{\textsc{Secluded $d$-ECS}}
\begin{tcolorbox}[enhanced,title={\color{black} {\textsc{Secluded $d$-Edge-Connected Subgraph}$~$ (\prob{})}}, colback=white, boxrule=0.4pt,
	attach boxed title to top left={xshift=.3cm, yshift*=-2.5mm},
	boxed title style={size=small,frame hidden,colback=white}]
	
	\textbf{Input:}  
  An undirected graph $G$, a weight function, $\wOp \colon \ve{G} \to \N$, and integers $w, k \in \N$\\

	\textbf{Output:}
  Decide if there is a set $S \subseteq \ve{G}$ with weight $\w{S} \ge w$ and neighborhood size $\abs{\nei{S}} \le k$, such that $\induced{G}{S}$ is $d$-edge-connected.
\end{tcolorbox}

Note that $d$ is part of the problem and not the instance and can therefore be treated as a constant in the algorithm.
In an analogous fashion to \Cref{thm:total_scc_np_hard}, we can prove NP-hardness for $d > 1$. Because the reduction is so similar, we omit the proof.
\begin{theorem}\label{thm:d_edge_np_hard}
  \prob{} is NP-hard, for all $d > 1$.
\end{theorem}

Compared to \textsc{Total-Secluded SCS}, the algorithm for \prob{} shares some similarities. The auxiliary problems are defined analogously, and we again need a definition of border complementations. Furthermore, the general algorithm structure is the same. Again, we use several reduction rules that decrease the size of everything but the allowed neighborhood set $B$. Before we define the reduction rules, we work with extensions and use an equivalence relation with a similar intuition. 

However, executing this technically is more involved than before, as are the reduction rules.
Moreover, this section will be structured slightly differently than the previous one. First of all, this is done to avoid unnecessary repetition of the same concepts. Second, this is due to the fact that we lack some proof ingredients that are necessary for a concrete analysis of the parameter dependence. We still prove that \prob{} is solvable in \emph{non-uniform} FPT-time, that is, there is a constant $c$ such that for all values the parameter $k \in \N$ can take, we can find an algorithm with runtime $\bigO{n^c}$. This is still a nontrivial result, as $d$-edge-connected graphs can generally not be classified by a set of forbidden induced subgraphs or anything else that was solved before. This result also indicates that the problem is very likely to be uniformly FPT~\cite[Chapter~6]{cygan2015parameterized}.

We start out in \Cref{sec:ext_equiv} by defining a problem-specific equivalence relation on extensions, which will be used in a similar way as in \Cref{sec:scc}. After that, we define border complementations together with auxiliary problems in \Cref{sec:d_edge_bc} using extensions and the equivalence relation. In \Cref{sec:d_edge_unbreak}, we give an algorithm for the base case in which the graph is $(q,k)$-unbreakable. Finally, we give the necessary reduction rules in \Cref{sec:d_edge_algorithm}, and explain how to stick all pieces together.

\subsection{Extension Equivalence}\label{sec:ext_equiv}

Again, we work with graph extensions as in \Cref{sec:scc_extensions}. Naturally, for an undirected graph $G$, we define an extension to be a pair $\hexpair$, where $H$ is an undirected graph and $\hexset \subseteq \set{\set{h,v}}{h \in \ve{H}, v \in \ve{G}}$. As before, we can stick the extension onto the graph $G$ to form the undirected extended graph $\ex{G}{H}{\hexset} \coloneqq (\ve{G} \cup \ve{H}, \e{G} \cup \e{H} \cup \hexset)$. Intuitively, extensions should be thought of as \emph{graph add-ons}, a smaller graph $H$ that can be attached to some part of a larger graph $G$. How exactly $H$ should be attached to $G$ is defined by the additional set $\hexset$.

Naturally, since we aim to use an extension as part of a $d$-edge-connected induced subgraph, our equivalence relation will then works by counting the existing and potential paths through the extension. To make working with paths easier, we use the following definition. 

\begin{definition}[Path via $Q$]
  Let $P$ be a path in $G$ and let $Q \subseteq \ve{G}$. We call $P$ a \emph{path via $Q$} if every edge of $P$ has at least one endpoint in $Q$ and every intermediate vertex is in $Q$.
\end{definition}

Note that this definition is equivalent to disallowing intermediate vertices outside of $Q$ and disallowing the direct edge between two vertices $u, v \notin Q$ as a path via $Q$.

First, we want measure how an extension helps in forming $d$-edge-connected induced subgraphs with $G$, which we capture with the following definition. This is the analogue of the connection set $\connOp$ for strongly connected subgraphs. In this case, however, it is more involved. We use a \emph{demand function} that defines a demand of paths for vertex pairs in $G$ that should go via the extension in a solution. The extension then fulfills this demand if all of these demanded paths are present in the extension, and all of them are edge-disjoint. For a set $X$, we denote the set of all subsets of size 2 of $X$ with $\binom{X}{2}$.

\begin{figure}[t]
    \begin{minipage}[c]{0.45\linewidth}
    \centering
    \includegraphics[width=0.6\textwidth,page=1]{figures/d_edge}
    \caption{An example for fulfillment for $d = 2$. The extension $\hexpair$ fulfills the function that demands paths between $v_1$ and $v_3$ and $v_2$ and $v_4$. If we instead demand paths as in $f(\set{v_1, v_3}) = f(\set{v_1,v_4}) = 1$, then $f$ is not fulfilled by $\hexpair$ since there is only one edge incident to $v_1$.}
    \label{fig:fulfill}
    \end{minipage}
    \hfill
    \begin{minipage}[c]{0.45\linewidth}
    \centering
    \includegraphics[width=0.6\textwidth,page=2]{figures/d_edge}
    \caption{An example for covering and sufficiency for $d = 2$. The function that maps only $f(\set{v_2,v_3}) = 1$ covers, among others, the pairs $\set{h_2,h_3}$ and $\set{h_2, v_2}$, but not $\set{h_1,h_2}$. Setting only $f(\set{v_1, v_3}) = 1$ is sufficient for $\hexpair$ and $\set{v_1, v_3}$.}
    \label{fig:sufficient}
    \end{minipage}
\end{figure}

\begin{definition}[Fulfillment]
  Let $\hexpair$ be an extension of $G$ and let $f \colon \binom{\ve{G}}{2} \to [0,d]$ be a function, which we call a \emph{demand function}. We say that $\hexpair$ \emph{fulfills} $f$ if for every distinct $v_1, v_2 \in \ve{G}$ there are $f(\set{v_1, v_2})$ paths between $v_1$ and $v_2$ via $\ve{H}$ such that \textbf{all of these paths} are edge-disjoint.
\end{definition} 

See \Cref{fig:fulfill} for an example of a fulfilled demand function.

The next definition considers the same kind of functions but from a different point of view. This time, the function can be thought of a compressed version of the rest of a potentially $d$-edge-connected subgraph, and we ask if this enough to connect a specific pair of vertices in $\ex{G}{H}{\hexset}$. To differentiate these two roles we name this kind of function \emph{supply function} although the signature of the function is the same. Think about this definition as measuring what requirements there are for a $d$-edge-connected induced subgraph such that it can include $\hexpair$. For strongly connected subgraph, this was the role of the source and sink sets $\sourceOp$ and $\sinkOp$.

\begin{definition}
  Let $\hexpair$ be an extension of $G$, and $f \colon \binom{\ve{G}}{2} \to [0,d]$ be a \emph{supply function}. Consider the graph $G' \coloneqq \ex{G}{H}{\hexset} - \e{G}$. Now insert $f(\set{v_1, v_2})$ edges between every $v_1, v_2 \in \ve{G}$. We say that $f$ \emph{covers} a pair of vertices $v_1, v_2 \in \ve{G'}$, if there are $d$ edge-disjoint paths between $v_1$ and $v_2$ in $G'$.
\end{definition}

The last definition formalizes which vertex pairs in $H$ have $d$-edge-connected paths, if the remaining solution looks like the supply function $f$.
To form a $d$-edge-connected subgraph that completely includes $\hexpair$, all pairs have to be covered. Only then is the remaining solution enough to be able to include $H$. We formalize this by a set of supply functions that the solution must be able to provide to be sufficient for the extension $\hexpair$.

\begin{definition}[Sufficiency]
  Let $\hexpair$ be an extension of $G$, $V' \subseteq \ve{G}$, and $F$ a set of supply functions. We say that $F$ is \emph{sufficient} for $\hexpair$ and $V'$ if for every $h_1, h_2 \in \ve{H}$, there is an $f \in F$ that covers $h_1, h_2$ and every $h\in \ve{H}, v \in V'$ is covered by some $f \in F$.
\end{definition}

See \Cref{fig:sufficient} for an example of covering a pair and a sufficient supply function.

Now, we define our equivalence relation on extensions for this problem. Similar to the previous section, the intuition for two components being equivalent is that including them into a $d$-edge-connected subgraph has the same benefits and the same requirements. In this case, the benefits are all demand functions that the extension fulfills. The requirements are all sufficient supply function sets. With this intuition, the next definition should come naturally.

% \jonas{moved next to other figure}
% \begin{figure}[t]
%   \centering
%   \includegraphics[width=0.3\textwidth,page=2]{figures/d_edge}
%   \caption{An example for covering pairs and sufficiency for $d = 2$. The function that maps only $f(\set{v_2,v_3}) = 1$ covers, among others, the pairs $\set{h_2,h_4}$, $\set{h_2,h_3}$, and $\set{h_2, v_2}$, but not $\set{h_1,h_2}$. The supply function that only maps $f(\set{v_1, v_3}) = 1$ is on its own sufficient for $\hexpair$ and $\set{v_1, v_3}$.}
%   \label{fig:sufficient}
% \end{figure}

\begin{definition}[Equivalence]\label{def:dequiv}
  Let $\hexpairi{1}, \hexpairi{2}$ be two extensions of $G$. We say that $\hexpairi{1}$ and $\hexpairi{2}$ are equivalent or $\hexpairi{1} \dequiv \hexpairi{2}$ if \begin{enumerate}[noitemsep,nolistsep]
    \item for all demand functions $f$, we have $\hexpairi{1}$ fulfills $f$ if and only if $\hexpairi{2}$ fulfills $f$,
    \item for all $V' \subseteq \ve{G}$ and supply function sets $F$, we have that $F$ is sufficient for $\hexpairi{1}$ and $V'$ if and only if it is sufficient for $\hexpairi{2}$ and $V'$.\qedhere
  \end{enumerate}
\end{definition}
\begin{observation}
  $\dequiv$ is an equivalence relation.
\end{observation}

To justify this definition, we have to show multiple properties. First of all, it has be computable in non-uniform FPT-time whether two components are equivalent.

Note that because of non-uniformity, the statement of the next lemma and others will be slightly odd, since we give a separate algorithm for every possible graph size $\abs{\ve{G}}$. In this case, this is due to the fact, the we use another non-uniform algorithm as a subroutine. Later, we use this kind of statement bound runtimes with unknown parameter dependence. For every constant parameter, the algorithm still runs in time bounded by the same polynomial, but we are unable to bound the $f(k)$ term.

\iflong
\begin{lemma}
\else
\begin{lemma}[$\star$]
\fi
\label{lem:check_equiv}
  For every $c \in \N$, there is an algorithm that checks for $\abs{\ve{G}} = c$ if two extensions $\hexpairi{1}$ and $\hexpairi{2}$ of $G$ are equivalent in time $\bigO{(\abs{\ve{H_1}} + \abs{\ve{H_2}})^4}$. 
\end{lemma}
\iflong
\begin{proof}
  For the first condition of \Cref{def:dequiv}, iterate over all $f$ in time $(d+1)^{\binom{c}{2}}$. For each $f$, we want to check if $\hexpairi{1}$ and $\hexpairi{2}$ fulfill $f$. To do this, we reduce to the edge-disjoint path problem, where you are given $p$ terminal pairs $(s_1, t_1), \ldots, (s_p, t_p)$ and want to find $p$ edge-disjoint paths, one between each pair.
  To reduce to this problem, for each $\uned{v_1}{v_2} \in \binom{\ve{G}}{2}$, we add $f(\uned{v_1}{v_2})$ leaves to $v_1$ and to $v_2$. For each such leaf pair $\ell_{v_1}, \ell_{v_2}$ we add one terminal pair between them.

  Now, we run the edge-disjoint path algorithm by \cite{robertson1995graph} that runs in time $\bigO{n^3}$ for a constant number of terminal pairs, where $n$ in our case is $\abs{\ve{H_1}}$ or $\abs{\ve{H_2}}$. Note that the number of terminal pairs is here is bounded by $\binom{c}{2}d$ and can thus be treated as constant.

  For the second condition of \Cref{def:dequiv}, iterate over all $V'$ and $F$ in time $2^{c}2^{(d+1)^{\binom{c}{2}}}$. Then, we check for all of the at most $(\abs{\ve{H_i}} + c)^2$ pairs if they are covered by a function in $F$. For each $f \in F$, we build $G'$ and insert the corresponding edges. Then, we can build the Gomory-Hu Tree~\cite{gomory1961multi} using $\abs{\ve{H_i}} + c$ max flow calculations with Orlin's algorithm~\cite{orlin2013max}. Then, we can easily read off the min cut value between each pair, which corresponds to the number of edge-disjoint paths. Since $c$ can be treated as a constant, this gives the total time $\bigO{\abs{\ve{H_i}}^4}$.
\end{proof}
\fi

\zzcommand{\classes}[1]{C(#1)}
\zzcommand{\classesc}[1]{C'(#1)}
Next, we want to show that the number of equivalence classes of $\dequiv$ can be bounded by a function of $\abs{\ve{G}}$, such that we can use it to bound the number of components.
To work with the equivalence classes, we want to know a smallest extension out of every non-empty equivalence class. So, for a graph $G$, pick an extension $\hexpair$ such that $\abs{\ve{H}}$ is minimum out of every non-empty equivalence class. We denote this set by $\classes{G}$.
Additionally, we consider $\classesc{G}$, defined in the same way as $\classes{G}$, except that we only consider extensions $\hexpair$ where $H$ is connected. Clearly, we have $\abs{\classesc{G}} \le \abs{\classes{G}}$.

\iflong
\begin{lemma}
\else
\begin{lemma}[$\star$]
\fi
\label{lem:number_equiv}
  There is $c \in \N$, such that the number of equivalence classes in $\dequiv$ is at most $\abs{\classes{G}} \le 2^{2^{2^{c\ve{G}^2}}}$. 
\end{lemma}
\iflong
\begin{proof}
  First, notice that there are $p \coloneqq (d+1)^{\binom{\abs{\ve{G}}}{2}}$ different supply or demand functions since for every pair in $\ve{G}$ we get to choose between values in $[0, d]$.
  Each class is now first identified by the set of demand functions it fulfills, of which there are $2^p$. Second, each subset $V'$ and supply function set $F$, of which there are $q \coloneqq 2^{\abs{\ve{G}}}2^p$, could form a sufficient combination. In total this make $2^p2^q$ equivalence classes, which gives the desired bound, since we treat $d$ as a constant.
\end{proof}
\fi

\iflong
\begin{lemma}
\else
\begin{lemma}[$\star$]
\fi
\label{lem:compute_classes}
  For every $c \in \N$, there is an algorithm that computes $\classes{G}$ and $\classesc{G}$ for all graphs $G$ with $\abs{\ve{G}} = c$ in constant time.
\end{lemma}
\iflong
\begin{proof}
  By \Cref{lem:number_equiv}, the number of equivalence classes is bounded by a function in $c$ and $d$. Therefore, the size of the smallest element of every non-empty equivalence class is also bounded by a function in $c$ and $d$; let us call this value $f(c,d)$. If we treat $c$ and $d$ as constants, we can simply generate all extensions with at most $f(c,d)$ vertices and sort them into equivalence classes using \Cref{lem:check_equiv}. Out of every non-empty class, we pick one extension with the minimum number of vertices for $\classes{G}$ and the minimum connected extension for $\classesc{G}$ if it exists.
\end{proof}
\fi

The crucial property of the equivalence relation $\dequiv$ is that two equivalent extensions form exactly the same $d$-edge-connected induced subgraphs. We prove this in \Cref{lem:equiv_d_edge}, justifying our definition. Later, this fact will allow us to replace components by equivalent components. Before we are able to prove this result, we need another lemma to simplify working with $d$-edge-connected subgraphs.

\iflong
\begin{lemma}
\else
\begin{lemma}[$\star$]
\fi
\label{lem:connect_paths}
  Let $G$ be a graph with $x,y,z \in \ve{G}$. If there are $d$ edge-disjoint paths between $x$ and $y$ and $d$ edge-disjoint paths between $y$ and $z$, there are also $d$ edge-disjoint paths between $x$ and $z$.
\end{lemma} 
\iflong
\begin{proof}
  The condition translates to the fact that the size of the min cut between $x$ and $y$ is at least $d$; the same hold for the min cut between $y$ and $z$. Consider the min cut between $x$ and $z$ and look at the graph with the min cut edges removed. If $y$ is not in a component with $x$ or $z$, consider a path from $y$ to either $x$ or $z$ that does not use vertices from the component of $x$ or $z$. This exists since $d \ge 1$. Then, we can remove all cut edges on the path from the cut, by the choice of the path, this will be a smaller cut, a contradiction. Therefore, without loss of generality $y$ lies in the same component as $x$ and the min cut between $x$ and $z$ is also a cut between $y$ and $z$. Hence it has size at least $d$, proving that there must be $d$ edge-disjoint paths between $x$ and $z$.
\end{proof}
\fi

Now, we are able to prove the crucial lemma about equivalent extensions. Note that this lemma is phrased in a way that will be slightly easier to apply, since we allow $G$ to have other extensions that cannot interact with our equivalent extensions. Later, the other extension $\hexpair$ will correspond to other components in $G-B$ that are also part of a solution.

\iflong
\begin{lemma}
\else
\begin{lemma}[$\star$]
\fi
\label{lem:equiv_d_edge}
  Let $G$ be an undirected graph with an extension $\hexpair$ and two equivalent extensions $\hexpairi{1} \dequiv \hexpairi{2}$. Consider $G' \coloneqq \ex{G}{H}{\hexset}$. Let $S \subseteq \ve{G'}$ such that $S \cup \ve{H_1}$ is $d$-edge-connected in $\ex{G'}{H_1}{\exset{G}{H_1}}$. Then, $S \cup \ve{H_2}$ is also $d$-edge-connected in $\ex{G'}{H_2}{\exset{G}{H_2}}$.
\end{lemma}
\iflong
\begin{proof}
  Define $V' \coloneqq \ve{G} \cap S$ and let $u, v \in S \cup \ve{H_2}$.
  If $u, v \in S$, consider the $d$ edge-disjoint paths between $u$ and $v$ in $S \cup \ve{H_1}$ with the subpaths $P_1, \ldots, P_{\ell}$ via $\ve{H_1}$. From these paths, we construct the corresponding demand function $f$ such that every path $P_i = v_{i,1}, \ldots, v_{i,2}$ contributes 1 to $f(\uned{v_{i,1}}{v_{i,2}})$. Then, clearly, $\hexpairi{1}$ must fulfill $f$ and so does $\hexpairi{2}$. Therefore, we can exchange $P_1$ to $P_{\ell}$ by edge-disjoint subpaths via $\ve{H_2}$ and there are $d$ edge-disjoint $u$-$v$-paths in $S \cup \ve{H_2}$.

  If $u, v \in \ve{H_2}$, we use the second property of equivalence. Consider all pairs $h \in \ve{H_2}, v \in V'$ and $h_1, h_2 \in \ve{H_2}$ with their $d$ respectively edge-disjoint paths in $S \cup \ve{H_1}$. We focus on the subpaths via $S$ and notice that we can construct a supply function for every such pair that covers it and corresponds to existing paths in $S$. Furthermore, this gives us a sufficient set $F$ for $\hexpairi{1}$ and $V'$ that must also be sufficient for $\hexpairi{2}$ and $V'$ by equivalence. Thus, there is $f \in F$ that covers $\uned{u}{v}$ in this case as well as the case $u \in \ve{H_2}$ and $v \in V'$.

  Finally, the only remaining case is $u \in \ve{H_2}$ and $v \in S \setminus V'$. Note that in this case there must be a $v \in V'$ since otherwise $S$ cannot be connected. Connectivity follows from the previous cases together with \Cref{lem:connect_paths}.
\end{proof}
\fi

\subsection{Border Complementations}\label{sec:d_edge_bc}

We define two auxiliary problems in the same way as in~\cite{golovach2020finding} and \Cref{sec:scc_bc}. We need them, to remember more information about the instance in recursive calls and work with a maximization problem instead of a decision problem.
\zzcommand{\probrec}{\textsc{Max \prob{}}}
\begin{tcolorbox}[enhanced,title={\color{black} {\probrec{}}}, colback=white, boxrule=0.4pt,
	attach boxed title to top left={xshift=.3cm, yshift*=-2.5mm},
	boxed title style={size=small,frame hidden,colback=white}]
	
	\textbf{Input:}  
  An undirected graph $G$, subsets $I,O,B \subseteq \ve{G}$, a weight function $\wOp \colon \ve{G} \to \N$, and an integer $k \in \N$\\

	\textbf{Output:}
  The maximum weight set $S \subseteq \ve{G}$ with $I \subseteq S$, $O \cap S = \emptyset$, $\nei{S} \subseteq B$, and $\abs{\nei{S}} \le k$, such that $\induced{G}{S}$ is $d$-edge-connected, or report that no feasible solution exists.
\end{tcolorbox}


We use our component equivalence definition to define border complementations in a similar way to~\cite{golovach2020finding} for finding $\mathcal{F}$-free secluded subgraphs, less explicitly than in \Cref{sec:scc_bc}. That is, we do not define the added vertices and edges directly, but use a small extension from an equivalence class of $\dequiv$.

\begin{definition}[Border Complementation]\label{def:d_edge_border_complementation}
  Let $(G,I,O,B,\wOp,k)$ be an instance for \probrec{} with a set $T \subseteq \ve{G}$ of \emph{border terminals}. A border complementation $(G',I',O',B,\wOp',k')$ is an instance obtained in the following way. Let $X,Y,Z$ be a partition of $T$ and let $\hexpair \in \classes{\induced{G}{X}}$. Additionally, we have that
  \begin{enumerate}[noitemsep,nolistsep]
    \item $G'$ is obtained by extending $X$ with $\hexpair$ and including edges $\uned{x}{y}$ for all $x \in X, y \in Y$,
    \item $I' \coloneqq I \cup X \cup \ve{H}$,
    \item $O' \coloneqq O \cup Y \cup Z$,
    \item $\wOp'(v) \coloneqq \w{v}$ for $v \in \ve{G}$ and $\wOp'(h) \coloneqq 0$ for $h \in \ve{H}$, and
    \item $k' \le k$.\qedhere
  \end{enumerate}
\end{definition}

Finally, we define the bordered problem. Analogously to \Cref{sec:scc_bc}, the task in this problem is to solve all border complementation instances of \probrec{}.

\zzcommand{\probborder}{\textsc{Bordered \probrec{}}}
\begin{tcolorbox}[enhanced,title={\color{black} {\probborder{}}}, colback=white, boxrule=0.4pt,
	attach boxed title to top left={xshift=.3cm, yshift*=-2.5mm},
	boxed title style={size=small,frame hidden,colback=white}]
	
	\textbf{Input:}  
  A \probrec{} instance $\mathcal{I} = (G,I,O,B,\wOp,k)$ and a set of border terminals $T \subseteq \ve{G}$ with $\abs{T} \le 2k$\\

	\textbf{Output:}
  A solution to \probrec{} for each border complementation of $\mathcal{I}$ and $T$, or report that no solution exists.
\end{tcolorbox}

\subsection{Unbreakable Case}\label{sec:d_edge_unbreak}

This section constitutes the base case of our recursive understanding algorithm, and we solve it in a similar fashion to our approach in \Cref{thm:unbreakable_scc}. The definitions of separation and unbreakability as well as \Cref{lem:unbreak_small_or_large,lem:find_sets} immediately transfer to undirected graphs. Therefore, we can give the algorithm description right away.

\iflong
\begin{theorem}
\else
\begin{theorem}[$\star$]
\fi
  There is a $c \in \N$, such that for every $k \in \N$, there is an algorithm that solves a \probborder{} instance $(G,I,O,B,\wOp,k,T)$ on a $(q,k)$-unbreakable graph in time $\bigO{(nq)^c}$.
\end{theorem}
\iflong
\begin{proof}
  Our algorithm works as follows. Initially, we enumerate all partitions of $T$ into $X,Y,Z$. Then, we compute $\classes{\induced{G}{X}}$ using \Cref{lem:compute_classes}. This way, we can enumerate all border complementation instances for \probrec{}. Consider one such instance $\mathcal{I'} \coloneqq (G', I', O', B', \wOp', k')$. By \Cref{lem:unbreak_small_or_large}, there is an $s \le q + f(k)$ for some function $f$, such that for every solution $S$ of $\mathcal{I'}$, we have either $\abs{S} \le s$ or $\abs{\ve{G'} \setminus S} \le s$. We address both cases separately and return the maximum weight solution of the solutions for both cases, or none if both do not exist.

  \subparagraph*{Finding a small solution} Our algorithm for this case is similar to the small case in \Cref{thm:unbreakable_scc}.
  \begin{enumerate}[noitemsep,nolistsep]
    \item We apply \Cref{lem:find_sets} with $U = \ve{G'}, a = s, b = k$ to compute a family $\mathcal{F}$ of subsets of $\ve{G'}$.
    \item For every $F \in \mathcal{F}$, compute the $d$-edge-connected components of $F$.
    \item For every $d$-edge-connected component $Q$, check its feasibility, and return the maximum weight feasible solution.
  \end{enumerate}

  \subparagraph*{Finding a large solution} Our algorithm for this case works as follows.
  \begin{enumerate}[noitemsep,nolistsep]
    \item Compute the $d$-edge-connected components of $G'$.
    \item For every $d$-edge-connected component $C$, if $\abs{\nei{C}} \le k$, then $C$ already is the maximum solution that is a subset of $C$. If not we proceed by constructing $G'_C$, analogously to \Cref{thm:unbreakable_scc}, by taking $\induced{G}{\cnei{C}}$, adding a vertex $c$, and connecting it to every $v \in \nei{C}$.
    \item Next, we run the algorithm from \Cref{lem:find_sets} in $G'_C$ with $U = \ve{G'_C}, a = s+1, b = k$.
    \item For all returned sets $F$, we only consider the weak component $S$ in $F$ that includes $c$ if such a component exists.
    \item Finally, we verify whether $C \setminus \cnei{S}$ is a feasible solution and return the maximum weight one.
  \end{enumerate}

  \subparagraph*{Correctness}

  For the small case, we know that for any solution $S$ with $\abs{S} \le s$, there is $F \in \mathcal{F}$ with $S \subseteq F$ and $\nei{S} \cap F = \emptyset$. Therefore, $S$ must be a $d$-edge-connected component of $F$ and this case is correct.

  Correctness for the large case is mostly analogous to \Cref{thm:unbreakable_scc}. We only have to consider the case of a solution $S' \subseteq C$ such that there is a component $C'$ of $\ve{G'_C} \setminus S$ that does not include $c$. We claim that we can include $C'$ to $S'$ to still give a solution. Clearly, the weight is non-decreasing and neighborhood size decreases, so we focus on edge-connectivity.

  First, notice that both $S'$ and $C$ induce $d$-edge-connected subgraphs of $G'$. That means the global min cuts of $\induced{G'}{S'}$ and $\induced{G'}{C}$ both have a value of at least $d$. We have to show that $\induced{G'}{S' \cup C'}$ also has a global min cut value of at least $d$. Suppose the global min cut is the min cut between a pair of vertices $u, v \in S' \cup C'$. We know that $u,v \in S'$ cannot be the case since adding $C'$ can only increase the min cut. Therefore, assume without loss of generality $u \in C'$ and first consider $u \in S'$. Since $C$ induces a $d$-edge-connected subgraph, removing any set of $d-1$ edges still allows us to find a path from $u$ to some $w \in S'$. Since $S'$ is $d$-edge-connected, we can still reach $v$ from $w$.
  The final case is $u, v \in C'$. Again, after removing $d-1$ edges, there still is a path from $u$ to $S'$ and from $v$ to $S'$ since $C$ is $d$-edge-connected and $\nei{C'} \subseteq S'$. Also, $S'$ is still connected after the removal of edges.

  \subparagraph*{Runtime} By \Cref{lem:find_sets}, there is a $c \in \N$, such that running the algorithm costs us $\bigO{s^cn^c} = \bigO{q^cn^c}$ for a constant $k$. We can find all $d$-edge-connected components in time $\bigO{n^4}$ using an algorithm from \cite{wang2015simple}. By \Cref{lem:compute_classes} computing and adding extensions to the graph only adds a constant overhead.
\end{proof}
\fi

\subsection{Reduction Rules and Algorithm}\label{sec:d_edge_algorithm}

Now, we will give some important reduction rules for \probborder{}. First, we transfer some basic reduction rules from \Cref{sec:scc}. We state them without proof since the proofs are simple and analogous to the corresponding reduction rules in the previous section.

\begin{reduction*}\label{red:d_edge_in_out}
  Let $Q$ be a component of $G - B$. If $Q \cap O \ne \emptyset$, set $O = O \cup \cnei{Q}$. If $\cnei{Q} \cap I \ne \emptyset$, set $I = I \cup Q$. If both cases apply, the instance has no solution.
\end{reduction*}
\begin{reduction*}\label{red:d_edge_remove_out}
  If \Cref{red:d_edge_in_out} is not applicable and there exists $v \in O \setminus B$, remove $v$ from $G$.
\end{reduction*}

Whenever possible we first apply \Cref{red:d_edge_in_out} and then \Cref{red:d_edge_remove_out} exhaustively in this order.

We will think of components $Q$ of $G-B$ as extensions $(\induced{G}{Q}, \set{\uned{q}{b}}{q \in Q, b \in B})$ of $\induced{G}{B}$. We will naturally extend equivalence on extensions to components as well as demand and supply function to functions on $B$.
The next rule consists of two closely related parts. First, it removes all components that can never be inside any solution because there are no $d$-edge-connected subgraphs in $G$ containing it. Second, if there are components that could be a solution by themself but cannot be part of a larger solution, we disconnect them from $B$. Later, this allows us to assume that any component could be part of some larger feasible solution.

\iflong
\begin{reduction*}
\else
\begin{reduction*}[$\star$]
\fi
\label{red:remove_useless}
  Let $Q$ be a component of $G - B$. If there is no $B' \subseteq B$ and a set of supply functions $F$ such that $F$ is sufficient for $Q$ and $B'$, include $Q$ into $O$.

  If the former does not apply, but there is no \emph{non-empty} $B' \subseteq B$ and $F$ such that $F$ is sufficient for $Q$ and $B'$, do the following. If $Q$ is a feasible solution, disconnect $Q$ from $B$ and set $O = O \cup \nei{Q}$. Otherwise, include $Q$ into $O$.
\end{reduction*}
\iflong
\begin{proof}[Proof of Safeness]
  Assume there is a solution $S$ with $Q \subseteq S$. Let $B' \coloneqq B \cap S$. Since $S$ is $d$-edge-connected, there are $d$ edge-disjoint paths between every pair of vertices. For every $v_1, v_2 \in Q \cup B'$, define a supply function $f$, where $f(\set{b_1, b_2})$ is the number of $b_1$-$b_2$-paths that is used to construct the $d$ edge-disjoint paths between $v_1$ and $v_2$ in $S$. Then, the set of all these $f$ must be sufficient for $Q$ and $B'$. Hence, in case the reduction rule applies, there cannot be a solution including $Q$ and we can safely remove it from the graph.

  For the second part of the rule, it is clear from the previous proof that for every solution $S$ we either have $S \cap Q = \emptyset$ or $S = Q$. If the neighborhood condition holds, we can simply disconnect it from its neighbors. Now, any solution can no longer include the former neighbors, but $Q$ itself is still a valid solution. Otherwise, $Q$ cannot be part of a solution and can thus safely be removed.
\end{proof}
\fi

After the previous reduction has been applied exhaustively, we can show that two equivalent components must have the same neighborhood.

\iflong
\begin{lemma}
\else
\begin{lemma}[$\star$]
\fi
\label{lem:similar_eq_nei}
  Suppose that \Cref{red:remove_useless} is not applicable in $G$. Then for two components $Q_1$ and $Q_2$ of $G-B$ with $Q_1 \dequiv Q_2$, we have $\nei{Q_1} = \nei{Q_2}$.
\end{lemma}
\iflong
\begin{proof}
  If $\abs{\nei{Q_1}} \ge 2$, let $b_1, b_2 \in \nei{Q_1}$. Then, $Q_1$ fulfills the demand function $f$ with $f(\set{b_1,b_2}) = 1$ and $f(\set{b,b'}) = 0$ for all other $b,b' \in B$. Since $Q_2$ must also fulfill $f$, we have $b_1, b_2 \in \nei{Q_2}$ and hence $\nei{Q_1} = \nei{Q_2}$. If $Q_i$ cannot be part of a solution or only be a solution on its on, it has $\nei{Q_i} = \emptyset$, by \Cref{red:remove_useless}. Since $d > 1$, there must be at least two edges from $Q_i$ to $B$ if $Q_i$ can be proper subset of a solution. If both edges lead to the same vertex, we can create a sufficient supply function, which again shows $\nei{Q_1} = \nei{Q_2}$.
\end{proof}
\fi

Now, we get to the core of our algorithm. In \Cref{red:d_edge_twins}, we prove that components of $G-B$ can be replaced by equivalent components, which follows directly from \Cref{lem:equiv_d_edge}. Next, we prove that when there are multiple equivalent components, it is enough to keep only a certain number of them that depends only on $\abs{B}$ and $d$. More components than this threshold are not relevant for forming $d$-edge-connected subgraphs and can always be included in a solution that includes the other equivalent components. Note that in previous applications of this approach such as the algorithms in~\cite{golovach2020finding} or \Cref{sec:scc}, the number of necessary components was always at most 2. In \Cref{red:d_edge_many_twins} we use a slightly more involved argument to acknowledge the fact that paths have to be edge-disjoint.

\iflong
\begin{reduction*}
\else
\begin{reduction*}[$\star$]
\fi
\label{red:d_edge_twins}
  Let $Q$ be a component of $G-B$ and let $Q^* \in \classesc{\induced{G}{B}}$ be the minimum size component that is equivalent to $Q$. Replace $Q$ with $Q^*$, set $\w{Q^*} \coloneqq \w{Q}$, and set $Q^* \subseteq I$ if $Q \subseteq I$.
\end{reduction*}
\iflong
\begin{proof}[Proof of Safeness]
  Remember that components can only be included as a whole, and note that $\nei{Q} = \nei{Q^*}$ by \Cref{lem:similar_eq_nei}. Let $S$ be a solution to the old instance and define $B' \coloneqq B \cap S$. Clearly, if $Q \not\subseteq S$, then $S$ is a solution to the new instance, so assume $Q \subseteq S$. Consider $S' \coloneqq (S \setminus Q) \cup Q^*$ and notice that $\nei{S'} = \nei{S}$ and $\w{S'} = \w{S}$. 
  Using \Cref{lem:equiv_d_edge} with $G = \induced{G}{B}$ and extensions corresponding to $Q$, $Q^*$, and $\ve{G} \setminus (B \cup Q)$, we conclude that $S'$ is also $d$-edge-connected.

  In this proof we only relied on the fact that $Q$ and $Q^*$ are equivalent. Therefore, the proof that any solution to the new instance can be transformed to a solution to the old instance follows by symmetry.
\end{proof}
\fi

As discussed before, the next rule is the most interesting one. We show that only $h$ copies of components from every equivalence classes are necessary to construct all solutions, for some number $h$ that only depends on $\abs{B}$ and $d$.

\iflong
\begin{reduction*}
\else
\begin{reduction*}[$\star$]
\fi
\label{red:d_edge_many_twins}
  For $h \coloneqq \max\set{(\abs{B}-1)d + 2, 2}$, let $Q_1, \ldots, Q_{h+1}$ be equivalent components of $G-B$ ordered by weight non-increasingly. Then, remove $Q_{h+1}$ from $G$ and increase $\w{Q_1}$ by $\w{Q_{h+1}}$. If $Q_{h+1} \subseteq I$, add $Q_1$ to $I$.
\end{reduction*}
\iflong
\begin{proof}[Proof of Safeness]
  We know that components can only be included as a whole. Let $S$ be a solution to the old instance.
  If $Q_{h+1} \not\subseteq S$, since $Q_{h+1} \cap B = \emptyset$, we immediately have that $S$ is also a solution for the new instance.
  If $Q_{h+1} \subseteq S$, either $S = Q_{h+1}$ or some vertices of $B$ must also be part of the solution. In the first case, $Q_1$ is a solution for the new instance with at least the same weight since $F = \emptyset$ must be sufficient for $Q_i$ and $\emptyset$ for all $i \in [h+1]$.
  In the second case, since all $Q_i$ have the same neighborhood, we must have $Q_i \subseteq S$ for all $i \in [h+1]$. We claim that $S' \coloneqq S \setminus Q_{h+1}$ is a solution for the new instance. Clearly, the neighborhoods of $S$ and $S'$ are the same. We now prove the $d$-edge-connectedness of $S'$.

\begin{figure}[t]
  \centering
  \includegraphics[width=0.25\textwidth,page=3]{figures/d_edge}
\caption{A visualization of one part of the proof of safeness of \Cref{red:d_edge_many_twins}. The example shows that a shortest path from a vertex $q_1 \in Q_1$ to a vertex $q_4 \in Q_4$ can use at most $2$ intermediate components $Q_2$ and $Q_3$ for $\abs{B} = 3$.}
  \label{fig:many_twins}
\end{figure}

  Let $s_1, s_2 \in S'$ and consider the $d$ edge-disjoint paths $P_1, \ldots, P_d$ between $s_1$ and $s_2$ in $S$, such that $\abs{P_i}$ is minimal for all $i \in [d]$, meaning that there is no way to replace one $P_i$ by a shorter path that is still edge-disjoint to the other paths. Now, the paths might use subpaths via $Q_{h+1}$. In this case we aim to replace these by different subpaths while staying edge-disjoint. For each $b \in B$, note that each $P_i$ can only include $b$ once. Otherwise, it includes a circle, which we could easily shortcut to decrease $\abs{P_i}$. Hence, we can think of $P_i$ as a path consisting of at most $\abs{B}+1$ subpaths that all start and end in a vertex in $B$, except for the first and last one. This part of the proof is also visualized in \Cref{fig:many_twins}. Now, for each such subpath, if it is a path via $Q_{h+1}$, instead reroute it via a different $Q_i$ that does not intersect with any path so far. We know that such a $Q_i$ must exist since there are only $(\abs{B}+1)d$ subpaths in total, each subpath is via exactly one component of $G-B$, and the first the last subpaths must be in the components of $s_1$ and $s_2$.  Furthermore, there is a corresponding path in $Q_i$ since $Q_i$ and $Q_{h+1}$ are equivalent and fulfill the same demand functions. Hence, we have constructed $d$ edge-disjoint paths between $s_1$ and $s_2$ in $S'$ and $S'$ is a solution to the new instance.

  For the other direction, consider a solution $S'$ for the new instance. If one $Q_i$ for $i \in [h]$ is not included in $S'$, then none of $\nei{Q_i}$ can be included in $S'$. Since $\nei{Q_i} = \nei{Q_{h+1}}$, we have that $S'$ is also a solution for the old instance.
  Suppose $Q_i \subseteq S'$ for all $i \in [h]$. We claim that $S \coloneqq S' \cup Q_{h+1}$ is a solution for the old instance. Since all $Q_i$ have the same neighborhood, we have $\nei{S'} = \nei{S}$. We now prove the $d$-edge-connectedness of $S$.

  Let $s_1, s_2 \in S$. If both $s_1, s_2 \in S'$, there are $d$ edge-disjoint paths between $s_1$ and $s_2$ only using $S'$. Therefore, the only two remaining cases are $s_1, s_2 \in Q_{h+1}$ and $s_1 \in Q_{h+1}, s_2 \notin Q_{h+1}$. We first consider the case where both vertices are in $Q_{h+1}$. Let $B' = B \cap S$. Since $Q_1 \subseteq S'$ and $S'$ is a solution, we know $S' \setminus Q_1$ must provide the paths from a sufficient supply function set $F$ for $Q_1$ and $B'$. Since $Q_1$ is equivalent to $Q_{h+1}$, the set $F$ is also sufficient for $Q_{h+1}$ and $B'$ and there are $d$ edge-disjoint paths between $s_1$ and $s_2$.

  Suppose without loss of generality only $s_1 \in Q_{h+1}$. If $s_2 \in B'$, since $S'$ provides the paths from a sufficient set $F$ for $Q_{h+1}$ and $B'$, there must be $d$ edge-disjoint paths between $s_1$ and $s_2$. If $s_2 \in Q_i$ for some $i \in [h]$, consider additionally some $b \in B'$. We know that there are $d$ edge-disjoint paths between $s_1$ and $b$ and $d$ edge-disjoint paths between $b$ and $s_2$. By \Cref{lem:connect_paths} there are at least $d$ edge-disjoint paths between $s_1$ and $s_2$ in $S$.
\end{proof}
\fi

Now, we can prove that the recursive understanding algorithm works also works for \probborder{}. Before, we summarize the progress of \Cref{red:d_edge_twins,red:d_edge_many_twins} in the following lemma.
\begin{lemma}\label{lem:d_all_reductions}
  Either we can reduce the instance, or there are at most $\max\set{(\abs{B}-1)d + 2, 2}$ components of every equivalence class. Furthermore, each component $Q$ has minimum size out of its equivalence class.
\end{lemma}

Finally, we arrive at the complete algorithm, which follows the same structure as the algorithm given in \Cref{alg:rec_und_scc}. Since the proof is very technical but also analogous to the proof of \Cref{thm:border_scc_fpt}, we only sketch it for completeness. 

\iflong
\begin{theorem}
\else
\begin{theorem}[$\star$]
\fi
\label{thm:border_d_edge_fpt}
  There is a constant $c \in \N$, such that for each $k \in \N$, there is an algorithm that solves a \probborder{} instance $\mathcal{I} = (G,I,O,B,\wOp,k,T)$ in time $\bigO{n^c}$.
\end{theorem}
\iflong
\begin{proof}
  The algorithm structure is almost identical to the algorithm of \Cref{thm:border_scc_fpt} as illustrated in \Cref{alg:rec_und_scc}, which is why we do not state it again here. The only difference besides the definition of border complementations and the set of reduction rules is the choice of $q$. In our case, $q$ will be constant for a fixed $k$ and we later prove that a suitable $q$ exists.

  We again have to prove that a border complementation covers all relevant parts of a solution, that is that we can consider $\mathcal{\hat{I}} \coloneqq (G,I,O,\hat{B},\wOp,k,T)$ instead of $\mathcal{I}$. To do this, let $S$ be a solution to a border complementation of $\mathcal{I}$. Consider the border complementation for $\tilde{G}$ that chooses $X$, $Y$, and $Z$ to match with $S$, that is $X = S \cap \tilde{T}$, $Y = \nei{S} \cap \tilde{T}$, and $Z = \tilde{T} \setminus \cnei{S}$. Furthermore, choose the border complementation that has an equivalent extension to $(\induced{G}{S \cap (U \setminus W)}, \set{\uned{u}{x}}{u \in S \cap U, x \in X})$. By \Cref{lem:equiv_d_edge}, both graphs have the same $d$-edge-connected induced subgraphs that intersect $W$. Therefore, it is enough to consider the instance with $\hat{B}$.

  Regarding the runtime and the choice of $q$, notice that $\abs{T} \le 2k$. By \Cref{lem:number_equiv}, there are at most $3^{2k}f(k)$ relevant border complementations for some function $f$. By definition of $\hat{B}$, we can bound $\abs{\hat{B}} \le 2k + k 3^{2k}f(k) \eqqcolon g(k)$. Let $h$ be the maximum size of a minimal equivalent extension of a graph with at most $g(k)$ vertices. Using \Cref{lem:d_all_reductions}, all components of $G-\hat{B}$ have at most $h$ vertices. Since the number of components per class is also bounded, we can bound the size of $W^*$ by the maximum size of each component, multiplied with the number of equivalence classes and the number of components per class, that is \[\abs{W^*} \le hf(k)((g(k)-1)d + 2) + g(k) \eqqcolon q,\] which is constant for a fixed $k$.

  Also, we only have to check equivalence and compute the classes for extensions of graphs with at most $g(k)$ vertices, which is constant for a fixed $k$ by \Cref{lem:compute_classes}. The other operations of the reduction rules are clearly polynomial. Together with the runtime analysis from \Cref{thm:border_scc_fpt}, this gives the desired runtime.
\end{proof}
\fi

Finally, we solve \prob{} using \probborder{}. We again set $I \coloneqq O \coloneqq \emptyset$, $B \coloneqq \ve{G}$, and $T \coloneqq \emptyset$.

\begin{theorem}\label{cor:d_edge_fpt}
  \prob{} is non-uniformly FPT.
\end{theorem}

What would be necessary to strengthen \Cref{cor:d_edge_fpt} and show that \prob{} is also uniformly FPT with a bounded and computable parameter dependence $f(k)$? There are two main hurdles. First, the use of a non-uniform FPT-algorithm from \cite{robertson1995graph} in \Cref{lem:check_equiv} has to be avoided. If an algorithm for the edge-disjoint path problem in uniform FPT-time is discovered, this can be used. Another way would be to check the definition of $\dequiv$ differently, or define $\dequiv$ in way that does not need such a heavy machinery. 

Second, we would need a way to define a constructive compression routine, such as the one in \Cref{sec:scc} for directed extensions. This would allow us to bound the size of the compressed extension by a function of $k$ and hence also the size of $W^*$ after applying all reduction rules. Until both of these conditions are achieved, the runtime of this algorithm stays non-uniform.
Nevertheless, this is still a strong indication towards proving the existence of a uniform FPT-algorithm~\cite[Chapter~6]{cygan2015parameterized}.

% \fi

\end{document}
