In today's world, people have the opportunity to travel to many different places 
throughout their lifetimes and in short spans. Every day, millions of people spend billions of hours collectively on their commutes. According to a recent survey by Agoda, 70\% of the world population have visited up to 10 countries~\cite{agoda_2019}. For instance, 2022 scored over 963 million international tourists~\cite{unwto_2023}. Mobility is an essential aspect of our daily lives, and with urbanization, globalization, and advancements in transportation, it is expected to continue to grow~\cite{barbosa2018human,pappalardo2023future,10.1145/3240323.3240361}.


Despite the abundance of data generated by modern transportation systems, location-based services, and digital footprints, a comprehensive understanding of the interplay between individuals and their surroundings remains elusive~\cite{10.1145/3494993,cuttone2018understanding}. Questions persist regarding the determinants of location preference, variations in mobility patterns, and the delineation of routine locations. While numerous studies have explored human mobility through various lenses, existing approaches often fall short due to oversimplified mechanisms~\cite{10.1145/3240323.3240361,Huandong_2022,10.1145/3494993}. The literature has witnessed significant innovation in mobility modeling, spanning from probabilistic models to deep learning and generative AI approaches~\cite{pappalardo2024survey,10.1145/3494993}. Traditional mechanistic models assume that human mobility behavior obeys simplistic governing laws such as heavy-tailed displacements and waiting times, a slow-down tendency for location discovery, or preferential return~\cite{gonzalez2008understanding,song2010modelling}. However, these conventional methods often fall short of capturing the full spectrum of human mobility behavior. AI techniques like Convolutional Neural Networks (CNNs) and Generative Adversarial Networks (GANs) have emerged to address these limitations. CNNs can incorporate external contextual factors, while GANs excel in capturing diverse and nonlinear relationships concurrently— aspects that conventional models may overlook. Nonetheless, these AI models lack transparency, obscuring the rationale behind predictions and trajectory generation~\cite{luca2021survey}. 

Recent research explores innovative methodologies rooted in knowledge-driven paradigms. Knowledge Graphs (KG) are a powerful representation framework that encodes structured information about entities and their relationships and have emerged as a promising tool for modeling human mobility~\cite{10.1145/3240323.3240361,Huandong_2022,10.1145/3494993}. By encapsulating mobility behaviors as relational facts within a structured KG and leveraging advanced Knowledge Graph Embedding (KGE) techniques, authors in~\cite{10.1145/3677019} aim to capture the complexity of individual mobility patterns. However, applying KG to model the spatial-temporal mobility behavior of users poses several key challenges. Different individuals demonstrate diverse mobility patterns influenced by various factors such as occupation, lifestyle, and personal preferences. A student's mobility patterns may differ significantly from those of a working professional. Capturing and representing this diversity within a single KG framework is a significant challenge due to the complexity and heterogeneity of human behavior and corresponding activities. Furthermore, human mobility behavior often varies over time, with individuals exhibiting different movement patterns on different days or during different times of the day\cite{alessandretti2020scales,pappalardo2024survey}. 

Motivated by the pressing need for a comprehensive understanding of human mobility dynamics, this paper first attempts to explain the complex relationship between individuals and their environments. Specifically, we propose to characterize an individual's connection to a particular place based on the time spent there and the frequency of visits to that place. Through extensive evaluations and comparative analyses, we demonstrate the effectiveness of our proposed methodology in characterizing individual visits, underscoring its practical significance for location-based services, public health interventions, and transportation management strategies.
In summary, our paper makes the following contributions:

\begin{itemize}[leftmargin=*]
    \item \textbf{Data completeness:} We introduce two novel metrics, \textit{temporal completeness} and \textit{spatial completeness}, designed to evaluate the quality and continuity of mobility data. These metrics enable the identification of users with consistently recorded trajectories over extended periods, ensuring more reliable datasets for mobility analysis.

    \item \textbf{Characterization of individual relationships with spatial instances:} We introduce a novel method for characterizing individuals' interactions with spatial locations, incorporating visit frequency and dwell time to classify different types of visits. This allows us to distinguish between routine, exploratory, and casual interactions with locations, shedding light on the complex ways individuals engage with their environments. Our method provides a nuanced understanding of mobility behaviors, revealing patterns such as the transition from exploratory to anchor locations or the persistence of casual visits.

    \item \textbf{Visits characterization evaluation:} We thoroughly evaluate the proposed method using real-world data from Singapore and Beijing, analyzing visit types across semantic, temporal, and spatial dimensions. Our results show that the classification framework effectively distinguishes between predictable mobility patterns and those requiring further investigation. We find that Singapore exhibits a fluid mix of routine and exploratory behaviors, while Beijing shows a stronger adherence to key locations. Semantic categories further highlight city-specific patterns, emphasizing the need for mobility models to reflect geographic and cultural contexts. These insights can enhance recommendation systems by better adapting to diverse mobility behaviors and preferences.
\end{itemize}