
\begin{figure*}[ht]
    \centering
    \begin{subfigure}[b]{0.48\textwidth}
        \centering
        \includegraphics[width=\textwidth]{fig/bic_aic_singapore.png}
        \caption{Singapore.}
        \label{fig:bic_aic_singapore}
    \end{subfigure}
    \hfill
    \begin{subfigure}[b]{0.48\textwidth}
        \centering
        \includegraphics[width=\textwidth]{fig/bic_aic_beijing.png}
        \caption{Beijing.}
        \label{fig:bic_aic_beijing}
    \end{subfigure}
    \caption{BIC and AIC values for GMM with varying numbers of clusters and different covariance types. Both subfigures share the same legend.}
    \label{fig:bic_aic}
\end{figure*}

Human beings, as inherently spatial creatures, intricately intertwine their activities with the surrounding geographical environments~\cite{lefebvre1991}. The organization of daily activities is significantly influenced by spatial factors, and residential decisions are often based on proximity to workplaces, schools, and essential amenities~\cite{cui_zhao_li_li_gong_deng_si_yan_dang_2024}. Similarly, workplaces are often selected for their accessibility and connectivity, while recreational areas are chosen for their appealing spatial qualities and ease of access~\cite{doi:10.1080/02673030500062335}.

Despite these general tendencies, the perception and navigation of geographical space vary significantly among individuals. Certain locations, such as homes, workplaces, and grocery stores, constitute the core of daily routines, frequented regularly for work, errands, and other obligations. These spaces seamlessly integrate into daily life, navigated with a habitual familiarity that underscores their importance. Conversely, leisure destinations, including parks, cinemas, and shopping malls, are sought for relaxation and recreational activities. Moreover, certain regions are encountered more casually, often during travel or exploration. These might encompass routes taken for vacations, spontaneous visits to new areas, or exploratory walks through different neighborhoods. Engaged with a sense of discovery and curiosity, these spaces are marked by less routine and more spontaneous interactions.

Understanding the connections between specific locations and individuals is essential for distinguishing predictable patterns from those that are uncertain and require further attention. In this study, we aim to capture and articulate the intricate relationships individuals maintain with their spatial environments.

The frequency and duration of visits to a location are robust indicators of its significance to an individual. To address this, we propose characterizing an individual's mobility trace by classifying their visits based on the number of days they visit and the duration of their stay at the considered PoIs.



We employ the Gaussian Mixture Model (GMM) to cluster PoIs at the user level based on visit frequency and dwell time. To ensure the effectiveness of our clustering, we determine the optimal number of clusters by computing the Bayesian Information Criterion (BIC)~\cite{https://doi.org/10.1002/wics.199} and the Akaike Information Criterion (AIC)~\cite{Bozdogan1987}. We explore a range of cluster numbers from 1 to 21 and evaluate four different covariance constraints: spherical, diagonal, tied, and full. 



The BIC and AIC values for each combination of cluster count and covariance type are displayed in Figures~\ref{fig:bic_aic_singapore} and~\ref{fig:bic_aic_beijing}. These visuals help us understand the balance between model complexity and fit quality. In Figure~\ref{fig:bic_aic_singapore}, a clear elbow point is observable at 7 clusters, particularly for the tied covariance type. This finding is further supported by a zoomed-in inset, which also shows a distinct elbow at 7 clusters for the tied covariance type. Visually, the results of clustering with 7 clusters effectively capture the main structures in the data, providing meaningful and interpretable segmentations of PoIs characterization.


Following the determination of the optimal number of clusters, we apply the GMM clustering algorithm with seven components and a tied covariance type. We report the obtained results in Figures~\ref{fig:gmm_singapore} and~\ref{fig:gmm_beijing}. Despite differences in the dataset duration between Singapore and Geolife, we observe the formation of similar groups:

\begin{figure}[ht]
    \centering
    \includegraphics[width=8cm]{fig/GMM_clusters_singapore_1.png}
    \caption{Visits characterization Singapore (GMM).}
    \label{fig:gmm_singapore}
\end{figure}


\begin{figure}[ht]
    \centering
    \includegraphics[width=8cm]{fig/GMM_clusters_beijing_1.png}
    \caption{Visits characterization Beijing (GMM).}
    \label{fig:gmm_beijing}
\end{figure}


\begin{figure*}[!ht]
    \centering
    \begin{subfigure}[b]{0.3\textwidth}
        \centering
        \includegraphics[width=\textwidth]{fig/singapore_visit_pattern_0_1.png}
        \caption{Visitation Pattern 1.}
        \label{fig:visitation_patterns_singapore_0}
    \end{subfigure}
    \hfill
    \begin{subfigure}[b]{0.3\textwidth}
        \centering
        \includegraphics[width=\textwidth]{fig/singapore_visit_pattern_1_1.png}
        \caption{Visitation Pattern 2.}
        \label{fig:visitation_patterns_singapore_1}
    \end{subfigure}
    \hfill
    \begin{subfigure}[b]{0.38\textwidth}
        \centering
        \includegraphics[width=\textwidth]{fig/singapore_visit_pattern_2_1.png}
        \caption{Visitation Pattern 3.}
        \label{fig:visitation_patterns_singapore_2}
    \end{subfigure}
    \caption{Singapore visitation patterns.}
    \label{fig:visitation_patterns_singapore}
\end{figure*}

\begin{figure*}[ht]
    \centering
    \begin{subfigure}[b]{0.3\textwidth}
        \centering
        \includegraphics[width=\textwidth]{fig/beijing_visit_pattern_0_1.png}
        \caption{Visitation Pattern 1.}
        \label{fig:visitation_patterns_beijing_0}
    \end{subfigure}
    \hfill
    \begin{subfigure}[b]{0.3\textwidth}
        \centering
        \includegraphics[width=\textwidth]{fig/beijing_visit_pattern_1_1.png}
        \caption{Visitation Pattern 2.}
        \label{fig:visitation_patterns_beijing_1}
    \end{subfigure}
    \hfill
    \begin{subfigure}[b]{0.38\textwidth}
        \centering
        \includegraphics[width=\textwidth]{fig/beijing_visit_pattern_2_1.png}
        \caption{Visitation Pattern 3.}
        \label{fig:visitation_patterns_beijing_2}
    \end{subfigure}
    \caption{Beijing visitation patterns.}
    \label{fig:visitation_patterns_beijing}
\end{figure*}


\begin{itemize}
    \item \textbf{G1}: These are visits to locations that occur less frequently than the lower 20\% for both frequency and dwell distributions. These \textit{short-duration}, exploratory visits represent places individuals seldom go to, reflecting infrequent and sporadic mobility behavior.

    \item \textbf{G2}: Locations visited rarely (within the bottom 20\% of the frequency distribution) but where individuals spend extended time, typically between 10 hours and less than a full day. These locations often correspond to special events or sites of significant personal interest, representing \textit{long-duration}, rare explorations.

    \item \textbf{G3}: These visits are to locations that represent a change in the user's routine, often indicating a significant shift in their usual patterns of behavior. These locations typically have high dwell times exceeding a full day and varying visit frequencies, suggesting a \textit{temporary change in routine}.

    \item \textbf{G4}: These are \textit{casual} visits to locations where the frequency of visits falls within  20\% (16\%) and 44\% (33\%) for Singapore (Beijing) of the distribution, and the average dwell time is within the lower 20\%. While these locations are part of an individual's mobility life, they are not a regular part of their daily routine. Examples include a casually visited restaurant or pharmacy, where visits are brief and not frequent.


    \item \textbf{G5}: These visits occur with moderate frequency, generally between 20\% and 40\% of the frequency distribution, and involve spending a moderate to high amount of time. These locations hold more \textit{importance} than casual visits but are less frequent than routine ones.


    \item \textbf{G6}: These are visits to places that an individual frequents regularly, falling within the upper 60\% and 40\% of the frequency distribution. These locations are integral to the individual's daily \textit{routine} and typically have moderate dwell times.


    \item \textbf{G7}: These refer to \textit{anchored-like} locations where an individual spends a significant amount of time daily, falling within the upper 20\% of the frequency distribution. Such locations are characterized by frequent visits and prolonged dwell time, making it a central part of the individual's routine.
    
\end{itemize}

In summary, the seven visit categories capture distinct mobility patterns, ranging from exploratory to routine behaviors, characterized by variations in visit frequency and dwell time. These patterns offer insights into individuals' interactions with their environment, from sporadic visits to locations of personal significance to the establishment of routine, anchored places central to daily life.


\subsection{Visitation motifs}
Based on our proposed visitation classification, we identified the three dominant visitation patterns in both Singapore and Beijing, as shown in Figures~\ref{fig:visitation_patterns_singapore} and~\ref{fig:visitation_patterns_beijing}, respectively. 


In Singapore (see Figures~\ref{fig:visitation_patterns_singapore}), emergent visitation behaviors provide key insights into individual mobility patterns. The first pattern (Figure~\ref{fig:visitation_patterns_singapore_0}) demonstrates that short-duration exploratory visits to \textit{G1} locations often lead to further exploratory activity, with occasional transitions to more stable, routine and anchored-like locations such as \textit{G6} and \textit{G7}. Visits to long-term exploratory sites, like \textit{G2} and \textit{G3}, frequently culminate in \textit{G6} visits, signifying a move toward more routine behavior. Individuals making brief, familiar visits to \textit{G4} typically continue with similar visit types, rarely shifting to exploratory or routine-altering activities. Following visits to \textit{G5} or \textit{G6}, individuals generally return to \textit{G6}, indicative of a stabilized routine. Meanwhile, visits to \textit{G7}—reflecting deep anchored routine—are predominantly followed by subsequent \textit{G7} visits, reinforcing a steady, habitual mobility patterns. This pattern is characterized by individuals who engage in short exploratory visits but tend to follow stable routine behaviors. In contrast, Figure~\ref{fig:visitation_patterns_singapore_1} highlights frequent transitions between exploratory-like visits and routine-like visits compared to the first pattern, such as \textit{G1,G2,G3}-to-\textit{G5} and \textit{G5}-to-\textit{G2}. This group also exhibits longer dwell times, regardless of whether their activities are exploratory or routine. Finally, Figure~\ref{fig:visitation_patterns_singapore_2} underscores a higher level of exploratory activity, both short- and long-term, with significant disruptions to routine (e.g., transitions to \textit{G3}). Visits to routine and anchored locations are still common for this subset, but the exploratory behavior leads to more dynamic mobility patterns.


In Beijing (see Figures~\ref{fig:visitation_patterns_beijing}), the first visitation pattern (Figure~\ref{fig:visitation_patterns_beijing_0}) reflects a distinct behavioral trend characterized by frequent yet brief visits to \textit{G1} locations, indicating exploratory activities. These visits are often complemented by casual, short-duration stops at familiar places, such as \textit{G4}, signaling a preference for familiar but flexible engagements. Moreover, individuals adhering to this pattern exhibit a pronounced tendency toward routine. This is evidenced by their consistent transitions to more stable, anchored locations, such as \textit{G6} and \textit{G7}, suggesting a balance between exploration and the need for stable, routine environments. The second visitation pattern (Figure~\ref{fig:visitation_patterns_beijing_1}) highlights a group with high levels of exploratory activities, both in short- and long-duration visits. These individuals demonstrate a propensity for breaking routine, engaging frequently in activities outside of their established patterns. Additionally, Figure~\ref{fig:visitation_patterns_beijing_1} depicts a marked tendency toward highly stable mobility patterns, as showed by frequent \textit{G7}-to-\textit{G7} transitions. This indicates a dual behavior where routine-breaking exploration coexists with a strong attachment to certain stable, anchored locations. Unlike the first pattern, where transitions to stable locations often occur directly after exploratory visits, this group tends to return to routine locations following visits to other routine sites. Lastly, the third visitation pattern (Figure~\ref{fig:visitation_patterns_beijing_2}) represents a subset of the population that primarily engages in exploratory visits, with only minimal routine behavior. This group appears to favor flexible and dynamic mobility, with limited engagement in consistent, anchored routines, distinguishing them as a highly exploratory subset within the population.

In both Singapore and Beijing, individuals balance exploratory and routine mobility behaviors, but their patterns diverge in key ways. In Singapore, transitions between exploratory and stable locations are more fluid, with individuals spending longer periods in exploratory visits (\textit{G2} and \textit{G3}). In contrast, mobility patterns in Beijing show a stronger attachment to routine, with a predominance of \textit{G7} visits, where individuals frequently return to stable, anchored locations after brief exploratory activities.



\subsection{Semantic patterns}
In Figure~\ref{fig:semantic_patterns}, we present the top five location types (semantics) visited across the different identified visitation categories. 

\begin{figure}[ht]
\centering
    \begin{subfigure}[b]{.45\textwidth}
        \centering
        \includegraphics[width=8.2cm]{fig/top_5_visits_semantics_singapore_1.png}
        \caption{Singapore.}
        \label{fig:semantic_visits_singapore}
    \end{subfigure}
\hfill
    \begin{subfigure}[b]{.45\textwidth}
        \centering
    \includegraphics[width=8.2cm]{fig/top_5_visits_semantics_beijing_1.png}
    \caption{Beijing.}
    \label{fig:semantic_visits_beijing}
    \end{subfigure}
\caption{Top semantic for each type of visit.}
\label{fig:semantic_patterns}
\end{figure}
\vspace{-1mm}


In the case of Singapore (Figure~\ref{fig:semantic_visits_singapore}), frequently visited location types include `food', `public\_service', `residential', `store', and `office\_building', which are visited across all categories of visits, but in varying proportions. Particularly for \textit{G7} visits, the appearance of `recreation' as a top semantic category suggests a notable representation of workers employed in recreational areas within the Singapore dataset.


For Beijing (Figure~\ref{fig:semantic_patterns}), the semantic categories of `office\_building' and `public\_service' consistently appear almost across all visitation types, reflecting a mixture of regular employees and transient visitors. `Food' establishments are primarily visited during \textit{G1}, \textit{G2},  \textit{G3}, and \textit{G5} visits, indicating two distinct visitor profiles: leisure visitors, who are captured through \textit{G1} visits, and individuals working in the food industry, as inferred from \textit{G5} visits characterized by extended durations and frequent occurrences.

The presence of `education' in \textit{G7} and \textit{G6} visits likely reflects the mobility patterns of students and educational staff captured in the Geolife dataset. Similarly, the appearance of `power\_plant' in \textit{G6} visits suggests the inclusion of individuals employed at power plants. An additional semantic category of note is `religious', which appears in the top five for \textit{G7} and \textit{G4} visits, highlighting both full-time staff at religious institutions and casual attendees. `Residential' and `store' locations are among the top five for several visit categories. Lastly, `recreation' emerges in \textit{G5}, \textit{G3}, \textit{G2}, and \textit{G1} visits, signifying the presence of both workers and individuals frequenting recreational venues.


In both Singapore and Beijing, key location types like `public\_service', `residential', and `office\_building' dominate across visitation categories, with unique patterns emerging in specific contexts. Notable distinctions include recreational areas in Singapore's \textit{G7} visits and the presence of `food', `education', and power-related locations in Beijing, reflecting diverse visitor profiles.



\subsection{Temporal patterns}
In Figure~\ref{fig:temporal_patterns}, we present the weekly temporal patterns of different visit types, normalized across both weeks and users.
\begin{figure}[ht]
\centering
\begin{subfigure}[b]{.45\textwidth}
    \centering
    \includegraphics[width=6cm]{fig/temporal_activity_singapore_1.png}
    \caption{Singapore.}
    \label{fig:temporal_activity_singapore}
\end{subfigure}
\hfill
    \begin{subfigure}[b]{.45\textwidth}
    \centering
    \includegraphics[width=6cm]{fig/temporal_activity_beijing_1.png}
    \caption{Beijing.}
    \label{fig:temporal_activity_beijing}
\end{subfigure}
    \caption{Temporal patterns.}
    \label{fig:temporal_patterns}
\end{figure}


For Singapore (cf.~\ref{fig:temporal_activity_singapore}), \textit{G1} short-duration exploratory visits exhibit significant peaks late in the day and some early in the morning, making them the most prevalent type of visit. This pattern reflects the population's high tendency to explore new locations for short durations, particularly on Thursday evenings. In Singapore, Thursday evenings are commonly associated with social gatherings and after-work meetups, encouraging exploratory behavior. \textit{G2} visits remain stable throughout the week, with a slight increase on Saturdays when individuals have more time for extended explorations. Meanwhile, \textit{G3} visits, which signal routine changes, are consistent throughout the week, peaking in the late afternoon and evening, indicating shifts in nighttime activities. Visits to \textit{G5}, \textit{G6}, and \textit{G7} locations exhibit stability during the week, with a slight rise on Thursdays, followed by a decline over the weekend. %\textbf{}

In Figure~\ref{fig:temporal_activity_beijing}, we present the temporal patterns for \textit{G1} visits in Beijing. Unlike other visit types, which exhibit less consistent patterns due to the limited data available in the Geolife dataset, \textit{G1} visits show a more stable trend throughout the week. Notably, there is a significant increase in \textit{G1} visits during the weekend, particularly on Saturdays. This spike suggests that users have more leisure time to explore new locations, highlighting a clear pattern of increased exploration and activity during the weekend.

The overall skewness observed in visit patterns for Beijing is attributed to the constraints of the dataset, which impacts the accuracy and clarity of temporal dynamics for other visit types. The more extensive data available for Singapore enables a clearer characterization of temporal patterns, providing a more comprehensive understanding of visit behaviors.


\subsection{Spatial patterns}

\begin{figure}[ht]
\begin{subfigure}[b]{.45\textwidth}
    \centering
    \includegraphics[width=7cm]{fig/spatial_use_occasional.png}
    %\vspace{-2cm}
    \caption{G1.}
    \label{fig:spatial_occasional}
\end{subfigure}
\hfill
    %\vspace{-.5cm}
    \begin{subfigure}[b]{.45\textwidth}
    \centering
    \includegraphics[width=7cm]{fig/spatial_use_anchors.png}
    %\vspace{-1cm}
    \caption{G7.}
    \label{fig:spatial_anchors}
\end{subfigure}
\caption{Spatial exploitation (Singapore).}
    \label{fig:spatial_patterns}
\end{figure}

In Figure~\ref{fig:spatial_patterns}, we illustrate the spatial distribution of two contrasting visit types in Singapore based on our proposed classification: short-duration exploratory visits \textit{G1} and anchored-like visits \textit{G7}. \textit{G1} visits exhibit a broad spatial dispersion across the AOI, indicating a diverse range of locations frequented by users with varying patterns. This wide dispersion suggests that users with \textit{G1} visits engage in a variety of activities across diverse settings, reflecting a high level of mobility within the city. In contrast, \textit{G7} visits are notably concentrated around residential areas. This spatial concentration highlights the tendency for \textit{G7} visits to occur in more familiar, localized settings, reflecting their association with extended dwell times and frequent returns to home or central locations.

Similarly, for Beijing, Figure~\ref{fig:spatial_occasional_beijing} illustrates the spatial distribution of \textit{G1} visits, while Figure~\ref{fig:spatial_anchors_beijing} depicts the spatial patterns of both \textit{G7} and \textit{G6} visits combined, given their relatively low frequencies. Consistent with observations in Singapore, \textit{G1} visits in Beijing are broadly dispersed across the city, indicating a wide range of locations and activities engaged in by users. In contrast, \textit{G7} and \textit{G6} visits exhibit notable concentration in the upper part of downtown, particularly around Peking University and the Haidian business district. The concentration of \textit{G7} and \textit{G6} visits in these areas suggests a high level of activity related to education and business. The clustering in these central, well-established locations reflects the individuals' regular engagement with these key urban hubs.


\begin{figure}[ht]
\centering
\begin{subfigure}[b]{.45\textwidth}
    \centering
    \includegraphics[width=5.5cm]{fig/spatial_use_occasiona_beijing.png}
    %\vspace{-2cm}
    \caption{G1.}
    \label{fig:spatial_occasional_beijing}
\end{subfigure}
\hfill
    %\vspace{-.5cm}
    \begin{subfigure}[b]{.45\textwidth}
    \centering
    \includegraphics[width=5.5cm]{fig/spatial_use_anchor_routine_beijing.png}
    %\vspace{-1cm}
    \caption{G6-G7.}
    \label{fig:spatial_anchors_beijing}
\end{subfigure}
\caption{Spatial exploitation (Beijing).}
    \label{fig:spatial_patterns_beijing}
\end{figure}
\vspace{-1mm}