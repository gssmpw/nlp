The analysis of human mobility is a focal point in various domains, including location-based services, urban planning, and geosocial marketing. Traditional mobility models, such as Markovian-based approaches, have been widely used to predict future locations based on past visitation patterns~\cite{Song_2010,9992252}. While effective in capturing frequent transitions, these models often overlook the semantic significance of different locations, leading to a limited understanding of the motivations behind human movement~\cite{10.1145/3423334.3431449}. Recent advancements have sought to enhance mobility modeling by incorporating temporal and spatial dimensions. For instance, models like the Time-Geo framework have introduced time-aware location predictions, improving the accuracy of movement forecasts by considering the temporal aspects of visits~\cite{Jiang_2016}. Also, spatial clustering techniques have been employed to group similar locations, aiming to identify routine visits more effectively. However, these approaches often treat locations homogeneously, failing to distinguish between various types of visits that carry different meanings for individuals~\cite{9992252,pappalardo2015returners,10.1145/3397536.3422248}.

Several works have attempted to integrate semantic data into mobility models. For example, SemanticTraj utilized semantic annotations to classify trajectories, offering a more refined understanding of movement patterns~\cite{al2016semantictraj}. Similarly, the PlaceRank model introduced the concept of location importance, ranking locations based on visitation frequency and duration~\cite{WANG2015335}. Despite these advancements, the existing models typically focus on either spatial or semantic aspects in isolation, and they often lack a comprehensive framework that integrates semantic, spatial, and temporal dimensions~\cite{9992252}.

This paper addresses these gaps by introducing the a new method, which uniquely characterizes visits into seven distinct categories according to their degrees of explorations and routine. Unlike previous approaches, the proposed method leverages a multi-dimensional analysis that accounts for the semantic significance, spatial distribution, and temporal patterns of visits, providing a holistic view of human mobility.

In summary, while there has been substantial progress in the field of mobility modeling, the existing methods fall short of capturing the complex and multi-faceted nature of human-environment interactions. The visits characterization model bridges this gap by offering a more nuanced characterization of visits, paving the way for improved applications in location-based services, urban planning, and personalized recommendations.