\documentclass{article}


\usepackage{arxiv}

\usepackage[utf8]{inputenc} % allow utf-8 input
\usepackage[T1]{fontenc}    % use 8-bit T1 fonts
\usepackage{hyperref}       % hyperlinks
\usepackage{url}            % simple URL typesetting
\usepackage{booktabs}       % professional-quality tables
\usepackage{amsfonts}       % blackboard math symbols
\usepackage{nicefrac}       % compact symbols for 1/2, etc.
\usepackage{microtype}      % microtypography
\usepackage{lipsum}
\usepackage{graphicx}

\usepackage{subcaption}

%\usepackage{todonotes}
\usepackage{enumitem}

%<\graphicspath{ {./images/} }

\newtheorem{definition}{Definition}
\title{Understanding Individual-Space Relationships to Inform and Enhance Location-Based Applications}

\author{
 Licia Amichi \\
 Oak Ridge National Laboratory\\
  \texttt{amichil@ornl.gov} \\
  %% examples of more authors
   \And
 Gautam Malviya Thakur \\
 Oak Ridge National Laboratory\\
\texttt{thakurg@ornl.gov} \\
  \And
 Carter Christopher \\
  Oak Ridge National Laboratory\\
  \texttt{christophesc@ornl.gov} \\
  %% \AND
  %% Coauthor \\
  %% Affiliation \\
  %% Address \\
  %% \texttt{email} \\
  %% \And
  %% Coauthor \\
  %% Affiliation \\
  %% Address \\
  %% \texttt{email} \\
  %% \And
  %% Coauthor \\
  %% Affiliation \\
  %% Address \\
  %% \texttt{email} \\
}

\begin{document}
\maketitle
\begin{abstract}
Understanding the complex dynamics of human navigation and spatial behavior is essential for advancing location-based services, public health, and related fields. This paper investigates the multifaceted relationship between individuals and their environments (e.g. location and places they visit), acknowledging the distinct influences of personal preferences, experiences, and social connections. While certain locations hold sentimental value and are frequently visited, others function as mere transitory points. To the best of our knowledge, this paper is the first to exploit visitation patterns and dwell times to characterize an individual's relationship with specific locations. We identify seven key types of spatial relationships and analyze the discrepancies among these visit types across semantic, spatial, and temporal dimensions. Our analysis highlights key findings, such as the prevalence of anchored-like visits (e.g. home, work) in both real-world Singapore and Beijing datasets, with unique associations in each city -Singapore's anchored-liked visits include recreational spaces, while Beijing's are limited to residential, business, and educational sites. These findings emphasize the importance of geographic and cultural context in shaping mobility and their potential in benefiting the precision and personalization of location-based services.
\end{abstract}


% keywords can be removed
%\keywords{First keyword \and Second keyword \and More}


\section{Introduction}
\section{Introduction}


\begin{figure}[t]
\centering
\includegraphics[width=0.6\columnwidth]{figures/evaluation_desiderata_V5.pdf}
\vspace{-0.5cm}
\caption{\systemName is a platform for conducting realistic evaluations of code LLMs, collecting human preferences of coding models with real users, real tasks, and in realistic environments, aimed at addressing the limitations of existing evaluations.
}
\label{fig:motivation}
\end{figure}

\begin{figure*}[t]
\centering
\includegraphics[width=\textwidth]{figures/system_design_v2.png}
\caption{We introduce \systemName, a VSCode extension to collect human preferences of code directly in a developer's IDE. \systemName enables developers to use code completions from various models. The system comprises a) the interface in the user's IDE which presents paired completions to users (left), b) a sampling strategy that picks model pairs to reduce latency (right, top), and c) a prompting scheme that allows diverse LLMs to perform code completions with high fidelity.
Users can select between the top completion (green box) using \texttt{tab} or the bottom completion (blue box) using \texttt{shift+tab}.}
\label{fig:overview}
\end{figure*}

As model capabilities improve, large language models (LLMs) are increasingly integrated into user environments and workflows.
For example, software developers code with AI in integrated developer environments (IDEs)~\citep{peng2023impact}, doctors rely on notes generated through ambient listening~\citep{oberst2024science}, and lawyers consider case evidence identified by electronic discovery systems~\citep{yang2024beyond}.
Increasing deployment of models in productivity tools demands evaluation that more closely reflects real-world circumstances~\citep{hutchinson2022evaluation, saxon2024benchmarks, kapoor2024ai}.
While newer benchmarks and live platforms incorporate human feedback to capture real-world usage, they almost exclusively focus on evaluating LLMs in chat conversations~\citep{zheng2023judging,dubois2023alpacafarm,chiang2024chatbot, kirk2024the}.
Model evaluation must move beyond chat-based interactions and into specialized user environments.



 

In this work, we focus on evaluating LLM-based coding assistants. 
Despite the popularity of these tools---millions of developers use Github Copilot~\citep{Copilot}---existing
evaluations of the coding capabilities of new models exhibit multiple limitations (Figure~\ref{fig:motivation}, bottom).
Traditional ML benchmarks evaluate LLM capabilities by measuring how well a model can complete static, interview-style coding tasks~\citep{chen2021evaluating,austin2021program,jain2024livecodebench, white2024livebench} and lack \emph{real users}. 
User studies recruit real users to evaluate the effectiveness of LLMs as coding assistants, but are often limited to simple programming tasks as opposed to \emph{real tasks}~\citep{vaithilingam2022expectation,ross2023programmer, mozannar2024realhumaneval}.
Recent efforts to collect human feedback such as Chatbot Arena~\citep{chiang2024chatbot} are still removed from a \emph{realistic environment}, resulting in users and data that deviate from typical software development processes.
We introduce \systemName to address these limitations (Figure~\ref{fig:motivation}, top), and we describe our three main contributions below.


\textbf{We deploy \systemName in-the-wild to collect human preferences on code.} 
\systemName is a Visual Studio Code extension, collecting preferences directly in a developer's IDE within their actual workflow (Figure~\ref{fig:overview}).
\systemName provides developers with code completions, akin to the type of support provided by Github Copilot~\citep{Copilot}. 
Over the past 3 months, \systemName has served over~\completions suggestions from 10 state-of-the-art LLMs, 
gathering \sampleCount~votes from \userCount~users.
To collect user preferences,
\systemName presents a novel interface that shows users paired code completions from two different LLMs, which are determined based on a sampling strategy that aims to 
mitigate latency while preserving coverage across model comparisons.
Additionally, we devise a prompting scheme that allows a diverse set of models to perform code completions with high fidelity.
See Section~\ref{sec:system} and Section~\ref{sec:deployment} for details about system design and deployment respectively.



\textbf{We construct a leaderboard of user preferences and find notable differences from existing static benchmarks and human preference leaderboards.}
In general, we observe that smaller models seem to overperform in static benchmarks compared to our leaderboard, while performance among larger models is mixed (Section~\ref{sec:leaderboard_calculation}).
We attribute these differences to the fact that \systemName is exposed to users and tasks that differ drastically from code evaluations in the past. 
Our data spans 103 programming languages and 24 natural languages as well as a variety of real-world applications and code structures, while static benchmarks tend to focus on a specific programming and natural language and task (e.g. coding competition problems).
Additionally, while all of \systemName interactions contain code contexts and the majority involve infilling tasks, a much smaller fraction of Chatbot Arena's coding tasks contain code context, with infilling tasks appearing even more rarely. 
We analyze our data in depth in Section~\ref{subsec:comparison}.



\textbf{We derive new insights into user preferences of code by analyzing \systemName's diverse and distinct data distribution.}
We compare user preferences across different stratifications of input data (e.g., common versus rare languages) and observe which affect observed preferences most (Section~\ref{sec:analysis}).
For example, while user preferences stay relatively consistent across various programming languages, they differ drastically between different task categories (e.g. frontend/backend versus algorithm design).
We also observe variations in user preference due to different features related to code structure 
(e.g., context length and completion patterns).
We open-source \systemName and release a curated subset of code contexts.
Altogether, our results highlight the necessity of model evaluation in realistic and domain-specific settings.






\section{Mobility Data Processing}
In this section, we begin by presenting the datasets used in our study, highlighting their key characteristics. We then describe the pre-processing procedures applied to the data, which ensure its suitability for the subsequent analyses tasks.


\subsection{Datasets}
In our experiments, we use real-world mobility datasets and Points of Interest (PoIs) data from two Areas of Interest (AOIs), Singapore and Beijing, to assess the performance of our proposed framework.

\vspace{.1cm}
\noindent\textbf{Singapore dataset:}
The Singapore dataset comprises an extensive collection of daily mobility records for 144,795 users, spanning a two-month period from December 1, 2022, to January 31, 2023. Data points were sampled at intervals of the order of a few seconds. With a total of 264,246,458 data records, this dataset provides a comprehensive and granular view of user mobility patterns. 


\vspace{.1cm}
\noindent\textbf{Geolife dataset:}
The Geolife dataset consists of mobility data from 182 users, collected over a period of more than five years, from April 2007 to August 2012. Data points were sampled at intervals of 1 to 5 seconds. These trajectories span over 30 cities, primarily in China, with a concentration in Beijing, along with some data from cities in the USA and Europe, offering a diverse range of mobility patterns across different urban environments.


The Singapore dataset is anonymized, and the data-collecting company adheres to international privacy standards, including the EU General Data Protection Regulation (GDPR) and the California Consumer Privacy Act (CCPA). Similarly, the Geolife dataset follows stringent privacy protocols. Collectively, these measures ensure compliance with established ethical handling standards.


\vspace{.1cm}
\noindent\textbf{PlanetSense PoI dataset:}
We also use the PlanetSense PoI dataset to complement the mobility data by providing contextual information about the locations visited by users. This dataset covers two locations and includes 238,690 PoIs in Singapore and 1,677,835 PoIs in Beijing, categorized into 44 distinct semantic types~\cite{10.1145/3356991.3365474,osti_2000381, thakur2015planetsense}.



\vspace{.1cm}
We begin by introducing the fundamental concepts of human mobility. Let $\mathcal{U}$ and $\mathcal{L}$ denote the sets of $N$ users and $M$ locations, respectively. 


\begin{figure*}[ht]
    \centering
    \begin{subfigure}[b]{0.24\textwidth} % Reduced width for fitting all figures in one line
        \centering
        \includegraphics[width=\textwidth]{fig/completeness_temporal_singapore.png}
        \caption{Singapore (temporal).}
        \label{fig:temporal_completeness_singapore}
    \end{subfigure}%
    \hspace{\fill} % Add flexible space between figures
    \begin{subfigure}[b]{0.24\textwidth}
        \centering
        \includegraphics[width=\textwidth]{fig/completeness_spatial_singapore.png}
        \caption{Singapore (spatial).}
        \label{fig:spatial_completeness_singapore}
    \end{subfigure}%
    \hspace{\fill} % Add flexible space between figures
    \begin{subfigure}[b]{0.24\textwidth}
        \centering
        \includegraphics[width=\textwidth]{fig/completeness_temporal_beijing.png}
        \caption{Beijing (temporal).}
        \label{fig:temporal_completeness_beijing}
    \end{subfigure}%
    \hspace{\fill} % Add flexible space between figures
    \begin{subfigure}[b]{0.24\textwidth}
        \centering
        \includegraphics[width=\textwidth]{fig/completeness_spatial_beijing.png}
        \caption{Beijing (spatial).}
        \label{fig:spatial_completeness_beijing}
    \end{subfigure}
    \caption{Temporal and Spatial Completeness in Singapore and Beijing.}
    \label{fig:completeness}
\end{figure*}




\begin{definition}[Mobility Record] 
A mobility record \( m_i \) is defined as a tuple \( m_i = (lat_i, lon_i, t_i) \), where \( (lat_i, lon_i) \) denotes the geographical coordinates of the location visited at time \( t_i \).
\end{definition}

\begin{definition}[Mobility Trace] 
The mobility trace \(\mathcal{D}_u\) of a user \(u\) is a sequence of records \(\{m_i\}_{i=1}^n\). The sequence is ordered such that \(t_1 < t_2 < \ldots < t_n\), ensuring that the tuples are arranged in chronological order.
\end{definition}

\subsection{Data quality assessment metrics}
\subsubsection{Temporal completeness}

Let $\mathcal{D}_u$ represent the mobility trajectory of user $u$ with a total of $N$ observations, recorded over a period $\mathcal{T}$. We define $\tau$ as the window interval size and $P$ as the observation unit. We introduce \textit{temporal completeness} metric $\mu_T(\mathcal{D}_u)$ of a mobility trajectory $\mathcal{D}_u$, as follows,
\begin{equation}
    \mu_T(\mathcal{D}_u) = \frac{\tau}{P} \sum\limits_{i = 0}^{\frac{P}{\tau}} f_{\tau}(i),
\end{equation}

where $f_{\tau}(i) = \left\{
    \begin{array}{ll}
        1 & \mbox{if  } \exists t \in ]\tau \times (i-1), \tau \times i] \mbox{, s.t. } t \in \mathcal{D}_u \\
        0 & \mbox{else.}
    \end{array}
\right.
$

This metric assesses whether mobility records are observed within selected time windows, allowing us to select users with comprehensive temporal mobility patterns.

\subsubsection{Spatial completeness}
Let $\Delta r_i$ represent the distance traveled between locations $l_{i-1}$ and $l_i$, which were visited by the user at times $t_{i-1}$ and $t_i$, respectively. Additionally, let $\Delta t_i$ denote the time elapsed between these two visits. We introduce the \textit{spatial completeness} metric, denoted as $\mu_S(\mathcal{D}_u)$, to quantify the spatial coverage of a mobility trajectory $\mathcal{D}_u$. This metric is defined as follows:
\begin{equation}
    \mu_S(\mathcal{D}_u) = \frac{1}{N} \sum\limits_{i = 1}^{N} g(i),
\end{equation}

where $g(i) = \left\{
    \begin{array}{ll}
        1 & \mbox{if  } \Delta t_i <= P \mbox{ and  } \frac{\Delta r_i}{\Delta t_i} <= \mbox{MAX\_SPEED}\\
        0 & \mbox{else.}
    \end{array}
\right.
$

$MAX\_SPEED$ denotes the maximum plausible speed, serving as a threshold to filter out high speeds that could suggest data errors or anomalies. The \textit{spatial completeness} score ensures that the mobility data accurately reflects realistic movement patterns, avoiding sudden, unrealistic jumps in location.


\subsubsection{Quality assessment}
In this work, we are interested in measuring completeness across various temporal and spatial granularities. To capture daily behaviors effectively, we set the observation unit $P$ to 24 hours, which corresponds to a full day. To evaluate completeness over different time scales, we vary the data availability period $\mathcal{T}$ among one week, two weeks, and one month, i.e., $\mathcal{T} \in \{7, 15, 30\}$ days. These periods are chosen to assess how completeness measures perform over short-term and extended temporal spans. A one-week period helps capture weekly patterns, while two weeks offers insight into possible bi-weekly variations, and a month-long period allows for the evaluation of longer-term trends. Additionally, we vary the window size $\tau$ across $\{1, 4, 6\}$ hours to measure completeness at different temporal granularities. A 1-hour window provides high resolution, enabling detailed analysis of short-term mobility. A 4-hour window balances resolution and data volume, while a 6-hour window captures broader patterns, suitable for identifying less frequent but significant trends. We also set $MAX\_SPEED$ to 150 km/h to filter out movement anomalies, as this value exceeds the typical speed limits of up to 100 km/h in Singapore\footnote{https://www.lta.gov.sg/content/ltagov/en.html} and 120 km/h  in Beijing\footnote{http://www.china.org.cn/bjzt}, thus capturing realistic travel scenarios.


Figure~\ref{fig:completeness} shows that higher temporal granularity (i.e., smaller values of $\tau$) corresponds to lower temporal completeness. This is because a smaller window $\tau$ captures data at a finer time scale, which often results in gaps and reduced completeness. Conversely, increasing the window size $\tau$ typically leads to more comprehensive data coverage for the user.

For the Singapore dataset (see Figure~\ref{fig:temporal_completeness_singapore}), which is richer and denser, the temporal completeness does not show significant variation with changes in the observation period $\mathcal{T}$. This indicates that extending the observation period does not adversely affect data quality. Therefore, we can select a maximum observation period of 30 days without compromising data integrity.

In contrast, the Beijing dataset shows a reduction in temporal completeness with longer observation periods $\mathcal{T}$, as shown in Figure~\ref{fig:temporal_completeness_beijing}. This suggests a trade-off between the desired observation period and data quality. For Beijing, we therefore opt to set the observation period $\mathcal{T}$ to 15 days to balance data quality and coverage.



From Figure~\ref{fig:spatial_completeness_singapore} and~\ref{fig:spatial_completeness_beijing}, the spatial completeness of the data remains consistently high across both AOIs, even when varying the parameters such as window size and observation period. This high spatial completeness suggests that the mobility trajectories captured in the data are reliable and realistic, with minimal occurrences of artifacts such as teleportation or unrealistic jumps in location.


To ensure high data quality in both datasets, we set the observation period $\mathcal{T}$ to 30 days for the Singapore dataset and 15 days for the Beijing dataset. Additionally, we chose a window size $\tau$ of 1 hour for both datasets to achieve a high level of temporal granularity. Subsequently, we filtered and selected only users who have data spanning $\mathcal{T}$ days with a temporal frequency of $\tau = 1$ hour.

\section{Proposed Methodology}
% \section{Preliminaries}
% \label{sec:preliminaries}

% \begin{table}[t]
% \footnotesize
% \centering
% \caption{Frequently-used Notations}
% \label{tab:notation}
% %\resizebox{0.5\textwidth}{!}{
% \begin{tabular}{c|c}
% \toprule
% Notation & Description                 \\\midrule
% $\Graph = (\VSet, \ESet, \ASet)$ & an attributed graph \\
% $n/m$  &number of nodes/edges of graph $\Graph$\\
% $ q = (\VSet_q, \ASet_q)$        & a query with node set $\VSet_q$ and attribute set $\ASet_q$\\
% $\community_q$ & the community containing query $q$ \\
% $\task = (\Graph, Q, L)$ & one task\\ %with a set of queries $Q$\\
% $l_q^+/l_q^-$ & positive/negative samples for query $q$\\
% $e_a$  & pre-trained attribute embedding for attribute $a \in \mathcal{A}$ \\
% $e(v)$ & original feature encoding for node $v$\\
% $I_l{v}$ & indicator whether the node $v$ is in the community \\
% $e_{\VSet_q}$ & average query nodes embeddings\\
% $e_{\ASet_q}$ & average query attributes embeddings\\
% $\support_i = (Q_i, L_i)$ & support set of a task $\task_i$\\
%  \bottomrule
% \end{tabular}
% %}
% \end{table}


% In this section, we first introduce the definitions of CS and ACS formally, and then formulate the inductive learning-based CS. Finally, we describe the existing GNN-based framework for CS as the technical background. Table~\ref{tab:notation} depicts the frequently-used notations and their descriptions. 

% \subsection{Definitions \& Concepts}
% An undirected simple graph $\Graph = (\VSet, \ESet)$ consists of a set of nodes, $\VSet$, and a set of undirected edges  $\ESet \subseteq \VSet \times \VSet$.
% Let $n = |\VSet|$ and $m = |\ESet|$ denote the number of nodes and edges, respectively. 
% The neighborhood of node $v_i$ is denoted as $\mathcal{N}(v_i) = \{ v_j |
% (v_j, v_i) \in \ESet \}$.

% \stitle{Community Search (CS).} For a graph $\Graph = (\VSet, \ESet)$, given a node set $\VSet_q \subseteq \VSet$ as a query $q$, the problem of Community Search aims to find the query-dependent community $\community_q \subseteq \VSet$, where the nodes in $\community_q$ are intensively intra-connected, i.e., maintaining cohesive structure.

% An undirected attributed graph $\Graph = (\VSet, \ESet, \ASet)$ has an additional attribute set 
% $\ASet$. Each node $v_i$ possesses its attribute set $\ASet_i$, and $\ASet$ is the union of all the node attribute sets, i.e.,
% $\ASet = \ASet_1 \cup \cdots \cup \ASet_n $.  

% \stitle{Attributed Community Search (ACS).} For an attributed graph, $\Graph = (\VSet, \ESet, \ASet)$, given a query $ q = (\VSet_q, \ASet_q)$ where $\VSet_q \subseteq \VSet$ is a set of query nodes, and $\ASet_q \subseteq \ASet$ is a set of query attributes, the problem of {Attributed Community Search}~(ACS) aims to find the query-dependent community $\community_q \subseteq \mathcal{V}$. Nodes in community $\community_q$ need to be structure cohesive and attribute homogeneous simultaneously, i.e., the nodes in the community are densely intra-connected in structure and the attributes of these nodes are similar.

% \stitle{Learning-based CS/ACS.} The general process of the learning-based approaches~\cite{ICSGNN,AQDGNN,coclep,cgnp,communityAF} consists of two stages, the training stage and the inference stage. In the training stage, for a graph $\Graph$, a parametric ML model $\mathcal{M}: q \mapsto [0, 1]^n$ is constructed offline from a set of queries and corresponding ground-truth communities.
% In the inference stage, for an online new query,  the model $\mathcal{M}$ predicts the likelihood of whether each node is in the community of the query, as a vector $\hat{y} \in [0, 1]^n$.
% The query supported can be non-attributed queries ($q = (\VSet_q, \emptyset)$) for CS and attributed queries ($ q = (\VSet_q, \ASet_q)$) for ACS.
% To be concise, we consider the ACS problem in this paper and regard CS as a special case of ACS ($\ASet_q = \emptyset$).
% Distinguished from prior algorithmic approaches~\cite{ATC, ACQ, CTC}, the community $\community_q$ discovered by learning-based approaches is not restricted to any specific $k$-related subgraph. %Instead, it is learned based on the provided ground-truth information regarding community membership.

% \subsection{Problem Statement}
% For existing learning-based ACS~\cite{AQDGNN}, the model $\mathcal{M}$ trained in a graph $\Graph$ is  expected to serve the same graph involving the same communities in the inference stage. 
% In this paper, we aim to explicitly empower the model to generalize and adapt to new communities and graphs by inductive learning, in the following two perspectives:

% \etitle{For new communities.} For a graph $\Graph$, given a set of training queries $Q =\{q_1, \cdots, q_i\}$ with corresponding ground-truth labels from the community set $\{ \community_{q_1}, \cdots, \community_{q_i}\}$, the model trained by $Q$ is used to answer query $q^*$ from a new community $\community_{q^*}$, 
% %TODO $\{\community_{q_1}, \cdots, \community_{q_i}\} \cap \community_{q^*}$???
% i.e., $\community_{q_1} \cap \community_{q^*} = \emptyset, \cdots,  \community_{q_i} \cap \community_{q^*} = \emptyset$. Furthermore, the graph $\Graph$ may even not contain the community $\community_{q^*}$, e.g., $\community_{q^*}$ is in a large online social network where $\Graph$ is a local subgraph extracted offline.

% \etitle{For new graphs.} For a graph $\Graph$, a model constructed from queries in $\Graph$ is used to answer queries from a new graph $\Graph^*$.

% \comment{
% \begin{example}
% Assume the data graph can be extracted into two subgraphs (shown in Fig.~\ref{fig:tasktop}), denoted as $\Graph_1$ and $\Graph_2$, and they contain node set $\VSet_1=\{v_1,v_2,v_3,v_4,v_5,v_6\}$ and $\VSet_2=\{v_7,v_8,v_9,v_{10},v_{11}\}$, respectively. On the one hand, $\community(\Graph_1)$ represents communities within $\Graph_1$, which consists of academic communities. On the other hand, $\community(\Graph_2)$ consists of musical communities. When the model is trained on $\Graph_1$ and can be adaptively applied to $\Graph_2$ for community search, we claim that the model demonstrates inductive ability for communities, since $\community(\Graph_1) \cap \community(\Graph_2)=\emptyset$. 
% Furthermore, the model demonstrates inductive ability for graphs if the model can be applied to another data graph, such as a graph containing sports communities as presented in Fig.~\ref{fig:taskbottom}.
% \end{example}
% }


% \begin{example}
% Fig.~\ref{fig:inductivetask} demonstrates a toy example of above two perspectives of inductive ACS. In Fig.~\ref{fig:tasktop}, the model is constructed by training data of an academic community (in orange),  and will be used for answering queries from a new musical community (in blue). Although the two communities are in a single large graph, their local structures and attribute sets are different. 
% In Fig.~\ref{fig:taskbottom}, the model is trained by a graph containing one community (on the left), and is  expected to answer ACS queries in a new graph (on the right).  
% \end{example}


% The challenges of model generalization on new communities and graphs lie in data heterogeneity, i.e., \emph{structural heterogeneity} and \emph{attribute heterogeneity}. For one thing, the topological structures are heterogeneous across different communities and graphs. 
% For the other thing, the attribute set, their semantics and distribution are heterogeneous across different communities and graphs.
% In general, data heterogeneity across different graphs would be more severe than that across different communities.
% To this end, in this paper, we construct a model $\model$ by inductive learning from multiple ACS tasks. 


% \stitle{ACS Task.} We formulate an ACS task as a triplet $\task = (\Graph, Q, L)$. And $\Graph = (\VSet, \ESet, \ASet)$ is an attribute graph, $Q = \{q_1, \cdots, q_i\}$ is a set of queries, i.e., $q_j = (\VSet_{q_j}, \ASet_{q_j})$, $\VSet_{q_j} \subseteq \VSet$, $\ASet_{q_j} \subseteq \ASet$, $\forall j \in [1, \cdots, i]$, $L = \{l_{q_1}, \cdots, l_{q_i} \}$ is the set of ground-truth of $j$ queries, correspondingly. Specifically, $l_q$ is a nonempty node set in $\Graph$ w.r.t. query $q$, containing a set of positive node samples, $l^{+}_q \subseteq \community_q$, and a set of negative samples, $l_q^{-} \subseteq (\VSet \setminus \community_{q}$).

% By inductive learning, the model $\model$ is trained on a set of training tasks, $\{\task_1, \cdots, \task_N\}$, and will be used in a new ACS task $\task^* = (\Graph^*, Q^*, L^*)$.  Here, $\Graph^*$ and the graphs in the training tasks are either different local subgraphs in a large graph, which do not have overlapping communities, or are fully different graphs. 
% Here, $Q^*$ and $L^*$ are a small number of queries with corresponding ground-truth for $\task^*$, i.e., $|Q^*| \ll |\VSet|$, which can be exploited by $\model$ to adapt to the specific task $\task^*$. The number of queries in $Q^*$, $|Q^*|$, is called shot in the following. 




% \begin{figure}
%     \centering
%    \begin{tabular}[h]{c}  
%   \subfigure[Inductive Setting for Communities]{
%     \includegraphics[width=0.44\textwidth]{fig/tasktop3.pdf}
%     \label{fig:tasktop}}
%     \vspace{-1ex}
%     \\
    
%   \subfigure[Inductive Setting for Graphs]{
%     \includegraphics[width=0.46\textwidth]{fig/taskbottom3.pdf}
%   \label{fig:taskbottom}}
%   \vspace{-1ex}
%   \\
%   \end{tabular}
  
%     \caption{Two Cases of Inductive ACS}
%     \label{fig:inductivetask}
% \end{figure}        

% \subsection{GNN for Learning-based ACS}

% Existing learning-based CS/ACS approaches~\cite{AQDGNN, communityAF,coclep, ICSGNN} employ GNN as the backbone of their models. A GNN of $K$-layers follows a neighborhood
% aggregation paradigm to generate a new embedding for
% each node by aggregating the embeddings of its neighbors
% in $K$ iterations. 
% Let $h^{(k)}(v)$ denote the embedding of node $v$ in the $k$-th iteration. In the $k$-th iteration (layer), an aggregate function $\Fagg^{(k)}$ aggregates the embeddings of the neighbors of $v$ generated in $(k - 1)$-th layer as Eq.~\eqref{eq:gnn:fagg}. Subsequently, a combine function $\Fcom^{(k)}$ updates the embedding of $v$ as Eq.~\eqref{eq:gnn:fcom}. The aggregate and combine functions of each layer are neural networks.
% %
% %GNN is recognized as a message passing procedure that aggregates the representations of neighboring nodes to generate a new representation for each node.
% %Most of existing works~\cite{AQDGNN, communityAF,coclep} formulate CS/ACS as a binary classification task and employ GNN to model structural cohesiveness and attribute homogeneity. 
% %Through $K$ layers GNN as described in Eq.~(\ref{eq:gnn:fagg})-(\ref{eq:gnn:fcom}), $h^{(K)}(v)$, the node representation is activated by a $\sigmoid$~function, denoted as, $\hat{y}(v) = \sigmoid(h^{(K)}(v))$, which represents the probability of node $v$ belonging to the same community as the query $q$.  
% \begin{align}
% 	a^{(k)}(v) &= \Fagg^{(k)}(\{ h^{(k - 1)}(u) | u \in \neighbor(v)\}), \label{eq:gnn:fagg} \\
% 	h^{(k)}(v)  &= \Fcom^{(k)}(h^{(k - 1)}(v), a^{(k)}(v)). \label{eq:gnn:fcom}
% \end{align}

% For dealing with ACS, the query information is injected into the initial node embedding, $h^{(0)}(v)$, by concatenating identifiers of query nodes and query attributes to the original node features as Eq.~\eqref{eq:gnn:input}.
% %
% %Here, $\Fagg$ is an aggregate function to aggregate the representation of the neighbors and $\Fcom$ is a combine function to combine the aggregated representation and previous representation.
% %For an ACS task $\task=(\Graph,Q,L)$, the GNN model can learn via its prediction error feedback given the queries $Q$ and ground-truth $L$ and then adapt to new queries in an end-to-end way. 
% %To jointly leverage structure and attribute knowledge, the initial node representation $h^{(0)}(v)$ is formed by concatenating the attribute feature vector $\mathcal{A}(v)$, the binary query node identifier $I_q(v)$ and query attribute identifier $I_a(v)$, as Eq.~\eqref{eq:gnn:input} shows. 
% \begin{align}
%     h^{(0)}(v) = [ I_q(v)\| I_{\ASet}(v) \|\ASet(v)],
%  \label{eq:gnn:input}
% \end{align}
% Here, $I_q(v) \in \{0, 1\}$ identifies whether node $v$ is a query node, $ I_{\ASet}(v) \in \{0, 1\}^{|\ASet|}$ identifies which attributes are the query attributes, and $\ASet(v)$ is the vectorized representation of the attributes of $v$.
% Through the transformation of $K$ layers, GNN-based models predict the likelihood  that $v$ is in the community of the query by a prediction layer $\hat{f}$, i.e., $\hat{y}(v) = \hat{f}(h^{(K)}(v))$, where the prediction layer $\hat{f}$ may contain extra neural network layers, followed by sigmoid activation. Thereby, an ACS task $\task=(\Graph,Q,L)$ is a query-specific binary classification task in $\Graph$, where GNN-based models are trained by minimizing the binary cross entropy (BCE) loss on queries $Q$ and ground-truth $L$ as Eq.~\eqref{eq:loss:bce}.
% %where $I_q(v)=1$ if $v$ is the query node, otherwise $I_q(v)=0$. And $I_a(v)=1$ if $v$ has a query attribute $a$, otherwise $I_a(v)=0$. If there are multiple query attributes, we need to construct multiple initial node representations and combine the representation through an aggregate operation.
% %The binary cross entropy (BCE) loss for a specific query  $q \in Q$, as shown in Eq.~(\ref{eq:loss:bce}), measures the discrepancy between the predicted probability and ground-truth $l_q = (l_q^{+}, l_q^{-})$, under the GNN parameterized by $\theta$. Here, $l_q^{+}$ and $l_q^{-}$ represent the positive and negative samples respectively, w.r.t. $q$. 
% %
% \begin{align}
% 	\label{eq:loss:bce}
% 	\loss(q; \theta) = - \sum_{v^{+} \in l_q^+} \log\hat{y}(v^+) -  \sum_{v^{-} \in l_q^-} \log (1-\hat{y}(v^-)) 
% \end{align}

% Existing approaches train a distinct model for each ACS task by implicitly assuming that  the graph used for training and inference for unseen queries are the same. Such transductive models cannot infer communities for queries from different graphs and different attribute sets.



% \comment{We first introduce general framework of GNN, which serves as the basis of many learning-based CS methods.
% A $K$-layer GNN follows a neighborhood aggregation paradigm to generate new representations for each node in the graph. This process involves iterating over $K$ layers.
% The representation of a node $v$ in the $k$-th iteration is denoted as $h_v^{(k)}$, which is a $d^{(k)}$-dimensional vector.
% In the $k$-th iteration (layer), for each node $v \in V(G)$, an aggregate function $\Fagg^{(k)}$ is used to combine the representations of the neighbors of $v$ that are generated in the previous ($k$-$1$)-th iteration as Eq.~(\ref{eq:gnn:fagg}). 
% Subsequently, an aggregate function $\Fcom^{(k)}$ updates the representation of $v$ by aggregating representation $a_v^{(k)}$ and previous representation $h_v^{(k-1)}$ as Eq.~(\ref{eq:gnn:fcom}).
% \begin{align}
% 	a_v^{(k)} &= \Fagg^{(k)}(\{ h_u^{(k - 1)} | u \in \neighbor(v)\}) \label{eq:gnn:fagg} \\
% 	h_v^{(k)}  &= \Fcom^{(k)}(h_v^{(k - 1)}, a_v^{(k)}) \label{eq:gnn:fcom}
% \end{align}}


\section{Related Works}
\section{Related Work}
Advancements in large language models (LLMs) have spurred significant interest in evaluating their capabilities across various domains. In competitive programming, several studies emphasize the importance of rigorous benchmarks and highlight challenges such as reasoning ability, data contamination, and evaluation methodologies.

\subsection{Competition-Level Problems as LLM Evaluators}
Programming challenges from platforms like Codeforces and the International Collegiate Programming Contest (ICPC) offer unique evaluation benchmarks for LLMs due to their complexity and diversity. These problems require a deep understanding of algorithms, mathematics, and reasoning, making them ideal for assessing LLM capabilities. Performance on unseen problems often drops significantly, indicating limitations in reasoning and generalization \cite{b1,b12}. These challenges underscore the need for reliable benchmarks and techniques to enhance reasoning in LLMs.

\subsection{Evaluation Methodologies}
Traditional methods for evaluating code generation have faced criticism due to issues like ``context leakage" and ``evolving-ignored" problems. Benchmarks such as those discussed in \cite{b2} better simulate real-world scenarios by considering the evolving nature of software development. These approaches reduce inflated performance metrics caused by unrealistic evaluation settings, providing a more accurate reflection of an LLM's problem-solving capabilities.

\subsection{Code Refinement and Interactivity}
Interactive code refinement and test-driven workflows have been shown to improve the quality of LLM-generated solutions. Iterative problem-solving techniques, combined with clear feedback, enhance LLM performance, particularly on complex programming challenges \cite{b3,b13}.

\subsection{Model-Specific Insights}
Studies focusing on specific models, such as the o1 family, emphasize their advanced chain-of-thought reasoning and robust handling of competitive programming problems \cite{b4}. These models outperform others in minimizing hallucinations and achieving consistent performance across diverse tasks, further validating their efficacy for high-stakes evaluations. Similarly, Mistral 7B demonstrates optimized performance for efficiency, highlighting the importance of model design for resource-constrained environments \cite{b11}.

\subsection{Challenges in Code Evaluation}
Issues like data contamination, overfitting, and reliance on pretraining data limit the generalization of LLMs. Research on the calibration and correctness of LLM-generated code highlights the importance of confidence scores and error analysis in ensuring reliable outputs, particularly in scenarios requiring high precision \cite{b6}.

\subsection{Performance Comparisons}
Comparative studies of models like GPT-4o, Llama-3.1, and o1 systems reveal stark differences in accuracy and resource efficiency \cite{b27,b10}. These comparisons underscore the importance of both training data and architectural design in achieving superior results on competitive programming problems.

By synthesizing these findings, our work contributes to the growing body of research by evaluating multiple LLMs on ICPC-style problems. This expands on previous studies by combining problem-solving performance with detailed error and resource utilization analysis.


%external .bib file (using bibtex).
%%% and comment out the ``thebibliography'' section.


%%% Comment out this section when you \bibliography{references} is enabled.
%\begin{thebibliography}{1}
\bibliographystyle{unsrt}  
\bibliography{ref}  %%% Remove comment to use the 
%\end{thebibliography}
%\addbibresource{ref.bib}
%\printbibliography

\end{document}
