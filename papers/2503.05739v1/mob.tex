In this section, we begin by presenting the datasets used in our study, highlighting their key characteristics. We then describe the pre-processing procedures applied to the data, which ensure its suitability for the subsequent analyses tasks.


\subsection{Datasets}
In our experiments, we use real-world mobility datasets and Points of Interest (PoIs) data from two Areas of Interest (AOIs), Singapore and Beijing, to assess the performance of our proposed framework.

\vspace{.1cm}
\noindent\textbf{Singapore dataset:}
The Singapore dataset comprises an extensive collection of daily mobility records for 144,795 users, spanning a two-month period from December 1, 2022, to January 31, 2023. Data points were sampled at intervals of the order of a few seconds. With a total of 264,246,458 data records, this dataset provides a comprehensive and granular view of user mobility patterns. 


\vspace{.1cm}
\noindent\textbf{Geolife dataset:}
The Geolife dataset consists of mobility data from 182 users, collected over a period of more than five years, from April 2007 to August 2012. Data points were sampled at intervals of 1 to 5 seconds. These trajectories span over 30 cities, primarily in China, with a concentration in Beijing, along with some data from cities in the USA and Europe, offering a diverse range of mobility patterns across different urban environments.


The Singapore dataset is anonymized, and the data-collecting company adheres to international privacy standards, including the EU General Data Protection Regulation (GDPR) and the California Consumer Privacy Act (CCPA). Similarly, the Geolife dataset follows stringent privacy protocols. Collectively, these measures ensure compliance with established ethical handling standards.


\vspace{.1cm}
\noindent\textbf{PlanetSense PoI dataset:}
We also use the PlanetSense PoI dataset to complement the mobility data by providing contextual information about the locations visited by users. This dataset covers two locations and includes 238,690 PoIs in Singapore and 1,677,835 PoIs in Beijing, categorized into 44 distinct semantic types~\cite{10.1145/3356991.3365474,osti_2000381, thakur2015planetsense}.



\vspace{.1cm}
We begin by introducing the fundamental concepts of human mobility. Let $\mathcal{U}$ and $\mathcal{L}$ denote the sets of $N$ users and $M$ locations, respectively. 


\begin{figure*}[ht]
    \centering
    \begin{subfigure}[b]{0.24\textwidth} % Reduced width for fitting all figures in one line
        \centering
        \includegraphics[width=\textwidth]{fig/completeness_temporal_singapore.png}
        \caption{Singapore (temporal).}
        \label{fig:temporal_completeness_singapore}
    \end{subfigure}%
    \hspace{\fill} % Add flexible space between figures
    \begin{subfigure}[b]{0.24\textwidth}
        \centering
        \includegraphics[width=\textwidth]{fig/completeness_spatial_singapore.png}
        \caption{Singapore (spatial).}
        \label{fig:spatial_completeness_singapore}
    \end{subfigure}%
    \hspace{\fill} % Add flexible space between figures
    \begin{subfigure}[b]{0.24\textwidth}
        \centering
        \includegraphics[width=\textwidth]{fig/completeness_temporal_beijing.png}
        \caption{Beijing (temporal).}
        \label{fig:temporal_completeness_beijing}
    \end{subfigure}%
    \hspace{\fill} % Add flexible space between figures
    \begin{subfigure}[b]{0.24\textwidth}
        \centering
        \includegraphics[width=\textwidth]{fig/completeness_spatial_beijing.png}
        \caption{Beijing (spatial).}
        \label{fig:spatial_completeness_beijing}
    \end{subfigure}
    \caption{Temporal and Spatial Completeness in Singapore and Beijing.}
    \label{fig:completeness}
\end{figure*}




\begin{definition}[Mobility Record] 
A mobility record \( m_i \) is defined as a tuple \( m_i = (lat_i, lon_i, t_i) \), where \( (lat_i, lon_i) \) denotes the geographical coordinates of the location visited at time \( t_i \).
\end{definition}

\begin{definition}[Mobility Trace] 
The mobility trace \(\mathcal{D}_u\) of a user \(u\) is a sequence of records \(\{m_i\}_{i=1}^n\). The sequence is ordered such that \(t_1 < t_2 < \ldots < t_n\), ensuring that the tuples are arranged in chronological order.
\end{definition}

\subsection{Data quality assessment metrics}
\subsubsection{Temporal completeness}

Let $\mathcal{D}_u$ represent the mobility trajectory of user $u$ with a total of $N$ observations, recorded over a period $\mathcal{T}$. We define $\tau$ as the window interval size and $P$ as the observation unit. We introduce \textit{temporal completeness} metric $\mu_T(\mathcal{D}_u)$ of a mobility trajectory $\mathcal{D}_u$, as follows,
\begin{equation}
    \mu_T(\mathcal{D}_u) = \frac{\tau}{P} \sum\limits_{i = 0}^{\frac{P}{\tau}} f_{\tau}(i),
\end{equation}

where $f_{\tau}(i) = \left\{
    \begin{array}{ll}
        1 & \mbox{if  } \exists t \in ]\tau \times (i-1), \tau \times i] \mbox{, s.t. } t \in \mathcal{D}_u \\
        0 & \mbox{else.}
    \end{array}
\right.
$

This metric assesses whether mobility records are observed within selected time windows, allowing us to select users with comprehensive temporal mobility patterns.

\subsubsection{Spatial completeness}
Let $\Delta r_i$ represent the distance traveled between locations $l_{i-1}$ and $l_i$, which were visited by the user at times $t_{i-1}$ and $t_i$, respectively. Additionally, let $\Delta t_i$ denote the time elapsed between these two visits. We introduce the \textit{spatial completeness} metric, denoted as $\mu_S(\mathcal{D}_u)$, to quantify the spatial coverage of a mobility trajectory $\mathcal{D}_u$. This metric is defined as follows:
\begin{equation}
    \mu_S(\mathcal{D}_u) = \frac{1}{N} \sum\limits_{i = 1}^{N} g(i),
\end{equation}

where $g(i) = \left\{
    \begin{array}{ll}
        1 & \mbox{if  } \Delta t_i <= P \mbox{ and  } \frac{\Delta r_i}{\Delta t_i} <= \mbox{MAX\_SPEED}\\
        0 & \mbox{else.}
    \end{array}
\right.
$

$MAX\_SPEED$ denotes the maximum plausible speed, serving as a threshold to filter out high speeds that could suggest data errors or anomalies. The \textit{spatial completeness} score ensures that the mobility data accurately reflects realistic movement patterns, avoiding sudden, unrealistic jumps in location.


\subsubsection{Quality assessment}
In this work, we are interested in measuring completeness across various temporal and spatial granularities. To capture daily behaviors effectively, we set the observation unit $P$ to 24 hours, which corresponds to a full day. To evaluate completeness over different time scales, we vary the data availability period $\mathcal{T}$ among one week, two weeks, and one month, i.e., $\mathcal{T} \in \{7, 15, 30\}$ days. These periods are chosen to assess how completeness measures perform over short-term and extended temporal spans. A one-week period helps capture weekly patterns, while two weeks offers insight into possible bi-weekly variations, and a month-long period allows for the evaluation of longer-term trends. Additionally, we vary the window size $\tau$ across $\{1, 4, 6\}$ hours to measure completeness at different temporal granularities. A 1-hour window provides high resolution, enabling detailed analysis of short-term mobility. A 4-hour window balances resolution and data volume, while a 6-hour window captures broader patterns, suitable for identifying less frequent but significant trends. We also set $MAX\_SPEED$ to 150 km/h to filter out movement anomalies, as this value exceeds the typical speed limits of up to 100 km/h in Singapore\footnote{https://www.lta.gov.sg/content/ltagov/en.html} and 120 km/h  in Beijing\footnote{http://www.china.org.cn/bjzt}, thus capturing realistic travel scenarios.


Figure~\ref{fig:completeness} shows that higher temporal granularity (i.e., smaller values of $\tau$) corresponds to lower temporal completeness. This is because a smaller window $\tau$ captures data at a finer time scale, which often results in gaps and reduced completeness. Conversely, increasing the window size $\tau$ typically leads to more comprehensive data coverage for the user.

For the Singapore dataset (see Figure~\ref{fig:temporal_completeness_singapore}), which is richer and denser, the temporal completeness does not show significant variation with changes in the observation period $\mathcal{T}$. This indicates that extending the observation period does not adversely affect data quality. Therefore, we can select a maximum observation period of 30 days without compromising data integrity.

In contrast, the Beijing dataset shows a reduction in temporal completeness with longer observation periods $\mathcal{T}$, as shown in Figure~\ref{fig:temporal_completeness_beijing}. This suggests a trade-off between the desired observation period and data quality. For Beijing, we therefore opt to set the observation period $\mathcal{T}$ to 15 days to balance data quality and coverage.



From Figure~\ref{fig:spatial_completeness_singapore} and~\ref{fig:spatial_completeness_beijing}, the spatial completeness of the data remains consistently high across both AOIs, even when varying the parameters such as window size and observation period. This high spatial completeness suggests that the mobility trajectories captured in the data are reliable and realistic, with minimal occurrences of artifacts such as teleportation or unrealistic jumps in location.


To ensure high data quality in both datasets, we set the observation period $\mathcal{T}$ to 30 days for the Singapore dataset and 15 days for the Beijing dataset. Additionally, we chose a window size $\tau$ of 1 hour for both datasets to achieve a high level of temporal granularity. Subsequently, we filtered and selected only users who have data spanning $\mathcal{T}$ days with a temporal frequency of $\tau = 1$ hour.