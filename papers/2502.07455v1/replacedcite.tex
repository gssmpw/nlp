\section{Related Works}
\subsection{Cultural Awareness of Generation Models}

We define that a model has a high level of \textit{cultural awareness} if it can generate semantically correct results for text prompts that contain specific concepts related to a particular culture. Multicultural awareness involves understanding of linguistic and semantic features ____, as well as correctly semantic matching of concepts from different cultures ____. Earlier, a tendency towards Western culture in generative models has been noted ____. As language is a conduit of culture ____, many NLP studies have focused on the issue of cultural awareness. This includes the task of adaptive translation ____, offensive language detection ____, dialog systems operation ____ and other tasks ____. The development of visual-language models (VLM) has led to a transfer of cultural awareness issues for the multimodal architectures ____. This problem was considered in the context of the visual question answering (VQA) task ____, image-text retrieval and grounding ____. With the addition of a new modality, the problem of cultural awareness has become more acute. For example, there are different levels of understanding of concepts from regional cultures among modern VLMs ____. 
%One of the reasons for this is that text translations are used to create training datasets. Images, on the other hand, do not change and do not introduce new visual concepts (2).
In text-to-image generation, quality metrics have long focused on the aesthetics and photorealism of generated results, while ignoring cultural awareness ____ Several studies have identified significant gaps in the level of multicultural awareness for the most popular T2I models. As far as we know, our work represents the first comprehensive approach to the issue of cultural awareness in relation to Russian culture in the T2I task.

\begin{figure*}[t]
\includegraphics[width=\linewidth]{images/categories.jpg}
  \caption{19 categories of Russian cultural code in our RusCode benchmark dataset. The images are generated by the Kandinsky 3.1 model ____.}
  \label{fig:categories}
\end{figure*}

\subsection{Multicultural Benchmarks}

Benchmarks for evaluating the multicultural and multilingual abilities of generative models primarily emerged in NLP tasks. Due to the fact that most existing language models are designed primarily for English, several studies have focused on assessing the linguistic and grammatical features of other languages ____. These benchmarks were designed to expand the range of multilingual knowledge tested in language models, as previously translated English-language datasets have overlooked various cultural and linguistic features ____. This is especially important for common languages, which, nevertheless, have limited resources in terms of accessible open information on the Internet ____. The main types of tasks included in such benchmarks were question answering ____, natural language generation ____, multilingual dialog generation ____, text style transfer ____, and many other tasks ____.

The next significant step forward was the development of multimodal benchmarks for evaluating multilingual VLMs ____. The range of benchmark tasks here primarily includes visual question answering (VQA) ____, as well as cross-modal retrieval, grounded reasoning, and grounded entailment tasks ____. Among the findings regarding the results of applying these benchmarks, it has been noted that the quality of modern VLM models varies depending on geographic and cultural categories ____. In addition, efforts have been made to combine text collected from the Internet with text generated by a pre-trained image captioning model.

Due to the fact that the T2I task primarily requires an assessment of visual cultural characteristics, not much work has been done in this area. Currently, existing benchmarks are limited in terms of the number of languages and cultural categories they cover ____.In addition, they do not support the Russian language, despite its relatively high level of usage on the Internet\footnote{\url{https://w3techs.com/technologies/overview/content_language}}. In this work, we are, to the best of our knowledge, the first to conduct a comprehensive cultural analysis in order to create a benchmark dataset for assessing the quality of image generation incorporating elements of Russian culture.

\begin{table*}[!t]
\caption{The list of categories and subcategories in the RusCode benchmark dataset}   
\centering
\small
\begin{tabular}{ll}
    \hline
    \textbf{Categories} & \textbf{Subcategories} \\
    \hline
    Architecture & Orthodox Church; Sights; Major cities\\
    \hline
    Art and culture & Painting; Music; Theater; Ballet; Opera; Musical; Photography; Cinema; Cartoons;\\
    & Architecture; Sculpture; Decorative and Applied arts; Design; Circus\\
    \hline
    Literature & Folklore, fairy tales and legends; Poems; Prose; Fables; Children's literature\\
    \hline
    Famous personalities & Public figures; Cultural figures; Scientists; Entrepreneurs; Military; Cosmonauts;\\
    & Russian writers; Musicians; Actors; Bloggers; Politicians; Athletes\\
    \hline
    Flora & Coniferous plants; Deciduous plants; Trees and shrubs; Flowers and herbs; Tundra vegetation;\\
    & Steppe vegetation; Swamp vegetation; Desert vegetation; Fungi; Lower plants; Spore plants;\\
    & Fruit plants; Berries; Root crops\\
    \hline
    Fauna & Mammals; Fish; Birds; Reptiles; Cold-blooded; Artiodactyls; Ungulates; Carnivores;\\
    & Herbivores; Amphibians; Wild animals; Domesticated animals; Small animals; Large animals\\
    \hline
    Media and TV & Animation; Documentaries; TV Series; Talk Shows; Reality Shows; Feature films;\\
    & Social networks; Advertising\\
    \hline
    Peoples of Russia & Nationalities; Clothing; Traditions; Religion; Crafts\\
    \hline
    National cuisine & First courses; Second courses; Hot appetizers; Cold appetizers; Desserts; Meat dishes;\\
    & Fish dishes; Milk and dairy products; Bread and bakery products; Cereals; Vegetables;\\
    & Fruits; Soft drinks; Alcoholic drinks\\
    \hline
    Holidays & Religious holidays; Civil holidays; Political holidays; Family holidays; Professional holidays;\\
    & National holidays; International holidays\\
    \hline
    Science & Natural Sciences; Exact Sciences; Social and Humanitarian Sciences;\\
    & Fundamental Sciences; Applied Sciences\\
    \hline
    Machinery & Modern machinery; Soviet machinery; Agricultural machinery; Aviation machinery;\\
    & Shipping machinery; Construction machinery\\
    \hline
    Symbols & State symbols; National symbols\\
    \hline
    Inscriptions & Signage and billboards; Logos and symbols\\
    \hline
    Inventions and discoveries & \\
    \hline
    Russian memes & \\
    \hline
    Locations & Natural; Man-made\\
    \hline
    Civil auto industry & Passenger cars; Trucks; Public transport\\
    \hline
    Military technics & Tanks, armored personnel carriers, air defense; Airplanes and helicopters;\\
    & Ships and submarines; The rest; Equipment of the 19th century;\\
    & Equipment of the 17th-18th century; Equipment of an earlier period\\
    \hline
\end{tabular}
\label{tab:categories}
\end{table*}

\subsection{Ethics and Social Biases in Generative AI}

Insufficient cultural awareness of image generation models can lead to the spread of social biases, misinformation, and offensive content ____. A number of studies have focused on reducing the biases in generative models that are based on factors such as race, skin color, gender, geography, and social status ____. Cultural stereotypes were also seen as undesirable in favor of greater globalization ____. Although we agree that cultural stereotypes can be offensive and need to be eliminated, knowledge about the specific cultural features should be retained by the model. For this reason, in our work we create a benchmark using expert knowledge to test the model's ability to capture real cultural features, while avoiding offensive stereotypes.

\begin{figure*}[t]
\centering
\includegraphics[width=\linewidth]{images/references.jpg}
\caption{Examples of prompts from RusCode dataset with corresponding reference images}
\label{fig:references}
\end{figure*}