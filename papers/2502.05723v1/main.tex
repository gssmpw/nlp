
%%%%%%%% ICML 2025 EXAMPLE LATEX SUBMISSION FILE %%%%%%%%%%%%%%%%%
% \documentclass{article}
\documentclass[twocolumn,10pt]{article}

% Full-page, two-column format settings
\setlength{\columnsep}{0.25in}  % Space between columns
\textwidth 7in       % Adjusted width for full-page two-column format
\textheight 9.5in    % Adjusted height for full-page use
\oddsidemargin -0.25in
\evensidemargin -0.25in
\topmargin -0.75in
\headheight 0in
\headsep 0.2in
\footskip 0.4in


% Recommended, but optional, packages for figures and better typesetting:
\usepackage{microtype}
\usepackage{graphicx}

\usepackage{subcaption}
\usepackage{booktabs} % for professional tables
\usepackage{makecell}
% hyperref makes hyperlinks in the resulting PDF.
% If your build breaks (sometimes temporarily if a hyperlink spans a page)
% please comment out the following usepackage line and replace
% \usepackage{icml2025} with \usepackage[nohyperref]{icml2025} above.
%\usepackage{hyperref}

\usepackage[hang,flushmargin]{footmisc} % footnote no indentation

\usepackage[noend]{algorithmic} % Or \usepackage[noend]{algpseudocode}
% Attempt to make hyperref and algorithmic work together better:
% \newcommand{\theHalgorithm}{\arabic{algorithm}}



%% \documentclass[11pt]{article}
% \documentclass{article}

\usepackage{commands}
\def\marrow{\marginpar[\hfill$\longrightarrow$]{$\longleftarrow$}}
\def\edith#1{\textsc{\color{magenta} (Edith says: }\marrow\textsf{\color{magenta} #1})}
\def\mihir#1{\textsc{ \color{red} (Mihir says: }\marrow\textsf{\color{red} #1})}
\def\uri#1{\textsc{ \color{blue} (Uri says: }\marrow\textsf{\color{blue} #1})}
\usepackage{enumitem}

\SetKwFor{OnInput}{\textbf{on input}}{}{}
\SetKwFor{Function}{\textbf{function}}{}{}

\newcommand{\DDD}{\mathcal{D}}
\newcommand{\al}{\alpha}
\newcommand{\eps}{\varepsilon}
\newcommand{\e}{\eps}
\newcommand{\f}{\frac}

\newcommand{\txtBkSketch}{\texttt{BkSketch}}
\newcommand{\txtSCest}{\texttt{StdEst}}
\newcommand{\txtRBCest}{\texttt{RobustEst}}
\newcommand{\txtRCest}{\texttt{TRobustEst}}

% \usepackage{icml2025}
% If accepted, instead use the following line for the camera-ready submission:
% \usepackage[accepted]{icml2025}

% \icmltitlerunning{Breaking the Quadratic Barrier: Robust Cardinality Sketches for Adaptive Queries}

\title{\texorpdfstring{Breaking the Quadratic Barrier:\\ Robust Cardinality Sketches for Adaptive Queries}{Breaking the Quadratic Barrier: Robust Cardinality Sketches for Adaptive Queries}}

% \ignore{
\author{Edith Cohen\\
Google Research and Tel Aviv University\\
Mountain View, CA, USA\\
\texttt{edith@cohenwang.com}
\and
Mihir Singhal\\
UC Berkeley and Google Research\\
Berkeley, CA, USA\\
\texttt{mihir.a.singhal@gmail.com}
\and
Uri Stemmer\\
Tel Aviv University and Google Research\\
Tel Aviv-Yafo, Israel\\
\texttt{u@uri.co.il}
}
% }
\date{}



\begin{document}

\ignore{
\twocolumn[
\icmltitle{\texorpdfstring{Breaking the Quadratic Barrier:\\ Robust Cardinality Sketches for Adaptive Queries}{Breaking the Quadratic Barrier: Robust Cardinality Sketches for Adaptive Queries}}


% It is OKAY to include author information, even for blind
% submissions: the style file will automatically remove it for you
% unless you've provided the [accepted] option to the icml2025
% package.

% List of affiliations: The first argument should be a (short)
% identifier you will use later to specify author affiliations
% Academic affiliations should list Department, University, City, Region, Country
% Industry affiliations should list Company, City, Region, Country

% You can specify symbols, otherwise they are numbered in order.
% Ideally, you should not use this facility. Affiliations will be numbered
% in order of appearance and this is the preferred way.
% \icmlsetsymbol{equal}{*}

\begin{icmlauthorlist}
\icmlauthor{Edith Cohen}{google,tau}
\icmlauthor{Mihir Singhal}{berkeley,google}
\icmlauthor{Uri Stemmer}{tau,google}
\end{icmlauthorlist}

\icmlaffiliation{tau}{School of Computer Science, Tel Aviv University, Israel}
\icmlaffiliation{google}{Google Research}
\icmlaffiliation{berkeley}{School of Computer Science, UC Berkeley, Berkeley, CA, USA}

\icmlcorrespondingauthor{Edith Cohen}{edith@cohenwang.com}
\icmlcorrespondingauthor{Mihir Singhal}{mihir.a.singhal@gmail.com}
\icmlcorrespondingauthor{Uri Stemmer}{u@uri.co.il}


% You may provide any keywords that you
% find helpful for describing your paper; these are used to populate
% the "keywords" metadata in the PDF but will not be shown in the document
\icmlkeywords{Adaptive Inputs, Cardinality Sketches, Robustness}
\vskip 0.3in
]

% this must go after the closing bracket ] following \twocolumn[ ...

% This command actually creates the footnote in the first column
% listing the affiliations and the copyright notice.
% The command takes one argument, which is text to display at the start of the footnote.
% The \icmlEqualContribution command is standard text for equal contribution.
% Remove it (just {}) if you do not need this facility.

%\printAffiliationsAndNotice{}  % leave blank if no need to mention equal contribution
% \printAffiliationsAndNotice{\icmlEqualContribution} % otherwise use the standard text.

\printAffiliationsAndNotice{} 

} %ignore ICML

\maketitle 
\begin{abstract}
%Cardinality sketches are compact data structures that significantly reduce storage, communication, and computational overhead while providing accurate approximations of distinct counts. The designs are randomized, with a sketching map sampled from a distribution and reused across multiple queries. This necessary randomness, however, introduces vulnerabilities: for non-adaptive queries (independent of prior responses), cardinality sketches provide statistical guarantees for $t$ queries, where $t$ scales exponentially with the sketch size $k$. However, under adaptive queries, these guarantees degrade to $t = \tilde{O}(k^2)$.
Cardinality sketches are compact data structures that efficiently estimate the number of distinct elements across multiple queries while minimizing storage, communication, and computational costs. However, recent research has shown that these sketches can fail under {\em adaptively chosen queries}, breaking down after approximately $\tilde{O}(k^2)$ queries, where $k$ is the sketch size.

In this work, we overcome this \emph{quadratic barrier} by designing robust estimators with fine-grained guarantees. Specifically, our constructions can handle an {\em exponential number of adaptive queries}, provided that each element participates in at most $\tilde{O}(k^2)$ queries. This effectively shifts the quadratic barrier from the total number of queries to the number of queries {\em sharing the same element}, which can be significantly smaller. Beyond cardinality sketches, our approach expands the toolkit for robust algorithm design.

%In this work, we overcome this \emph{quadratic barrier} by designing robust estimators with fine-grained guarantees. Specifically, when each key participates in at most $r = \tilde{O}(k^2)$ queries, our approach can handle an unlimited number of adaptive queries, effectively shifting the quadratic barrier on $t$ to the potentially much smaller $r$. Furthermore, we show that this result holds even when a fraction of the keys are \emph{heavy}, as is commonly observed in Pareto distributions, and demonstrate the potential for substantial practical gains. Beyond cardinality sketches, our approach broadens the toolkit for robust algorithm design.
% beyond generic methods constrained by $t = \tilde{O}(k^2)$.


% Additionally, our results reveal fundamental differences in robustness across statistics: for $\ell_2$ sketches, existing attacks remain effective with $t = \tilde{O}(k^2)$, even when $r = 1$.

\ignore{
===========
Composable sketch designs for fundamental aggregates, including cardinality (distinct count), randomly sample a sketching map and use it across queries. Randomness is known to be necessary but is vulnerable to adaptive queries. When inputs do not depend on the sketching map, the provided statistical guarantees are for answering correctly a number of queries $t$ that is exponential in the sketch size $k$. The guarantees with adaptive queries, however, are much weaker: Wrapper methods applied to the basic designs guarantee only $t = \tilde{O}(k^2)$ queries [Hassidim et al based on ADA]. Unfortunately, this quadratic bound was shown to be tight for cardinality sketching. In this work, we aim to mitigate this bad news with fine grained guarantees that apply with a large  number of queries $t$ when most keys  participate in a limited $r \ll t$ queries. 
For cardinality sketching, through a novel analysis method, we show that we can process an unlimited number of adaptive queries (with per-query guarantees similar to a non-adaptive setting) as long as $r=\tilde{O}(k^2)$, shifting the quadratic barrier to $r$ instead of $t$. 
Importantly, we extend the ADA toolkit for robust algorithm design to beyond the wrapper methods, which only guarantees $t = \tilde{O}(k^2)$ queries even in this case.
In contrast, for $\ell_2$ sketches, the known attacks apply with $t=\tilde{O}(k^2)$ even with $r=1$. }
\end{abstract}


\section{Introduction}

% \eccomment{$r$: per-key limit.  $t$: total number of queries. $n$: ground set size. $\rho$: randomness}
% \eccomment{To do:  Look into generalization with the binning statistics and $k$-partition}

When dealing with massive datasets, compact summary structures (known as sketches) allow us to drastically reduce storage, communication, and computation while still providing useful approximate answers.

Cardinality sketches are specifically designed to estimate the number of distinct elements in a query set~\cite{FlajoletMartin85,hyperloglog:2007,ECohen6f,ams99,BJKST:random02,KaneNW10,ECohenADS:TKDE2015,Blasiok20}. 
For a ground set $[n]$ of keys, a sketch is defined by a \emph{sketching map} $S$ that maps subsets $V\subset [n]$ to their sketches $S(V)$, and an \emph{estimator} that processes the sketch $S(U)$ and returns an approximation of the cardinality $|U|$.
 %
An important property of sketching maps is 
\emph{composability}: The sketch $S(U\cup V)$ of the union of two sets $U$ and $V$ can be computed directly from the sketches $S(U)$ and $S(V)$. Composability is crucial for most applications, particularly in distributed systems where data is stored and processed across multiple locations.

Cardinality sketches are extensively used in practice. \emph{MinHash sketches} are composable sketches based on hash mappings of keys to priorities, where the sketch of a set is determined by the minimum priorities of its elements \citep{FlajoletMartin85,hyperloglog:2007,ECohen6f,Broder:CPM00,Rosen1997a,ECohen6f,BRODER:sequences97,BJKST:random02}\footnote{For a survey see \cite{MinHash:Enc2008,Cohen:PODS2023}.}. Many practical implementations\footnote{See, e.g., \citep{datasketches,bigquerydocs}.} use MinHash sketches of various types, particularly bottom-$k$ and HyperLogLog sketches \cite{hyperloglog:2007,hyperloglogpractice:EDBT2013}. 

These sketches can answer an exponential number of queries (in the sketch size $k$) with a small relative error.
The design samples a sketching map from a distribution, and to ensure composability, \emph{the same map must be used to sketch all queries}. The guarantees are statistical: for \emph{any} sequence of queries, with \emph{high probability over the sampling of the map}.
The guarantees hold provided that the queries 
% are {\em fixed in advance}
{\em do not depend on the sampled sketching map}, which is known as the {\em non-adaptive setting}.



% Linear sketches~\citet{CormodeDIM03,distinct_deletions_Ganguly:2007,KaneNW10} apply with vectors $\boldsymbol{v}$ and we want to approximate the $\|\boldsymbol{v} \|_0$ (the number of nonzero entries).

 




\subsection{The adaptive setting}
In the adaptive setting, we assume that the sequence of queries may be chosen adaptively based on previous interactions with the sketch. This arises naturally when a feedback loop causes queries to depend on prior outputs. Sketching algorithms that guarantee utility in this setting are said to be {\em robust} to adaptive inputs.

The main challenge in the adaptive setting (compared to the non-adaptive setting) is that the queries become {\em correlated with the internal randomness of the sketch}. This would not be an issue if the sketching algorithm were deterministic, but unfortunately, randomness is necessary. In particular, any sublinear composable cardinality sketch that is statistically guaranteed to be accurate on all inputs must be randomized \citep{KaneNW10}. By this, we mean that the sketching map cannot be predetermined and must instead be sampled from a distribution. 
% When the inputs are adaptive, it may be possible to use fewer queries in order to zoom on inputs that are bad for the sketching map.


% for each particular query set $U$, the probability that it is 
% not estimated well using a sampled sketching map is exponentially small in the the sketch size. But each sketching map must fail on some inputs. When the inputs are adaptive, it may be possible to more quickly zoom on inputs that are bad for the sketching map.


\citet{HassidimKMMS20} presented a {\em generic robustness wrapper} that transforms a non-robust randomized sketch into a more robust one. Informally, this wrapper uses {\em differential privacy} \citep{dwork2006calibrating} to obscure the internal randomness of the sketching algorithm, effectively breaking correlations between the queries and the internal randomness. 

In more detail, to support $t$ adaptive queries, the wrapper of \citet{HassidimKMMS20} maintains approximately 
$\sqrt{t}$ independent copies of the non-robust sketch and answers each query by querying all (or some) of these sketches and aggregating their responses. 
As \citet{HassidimKMMS20} showed, this results in a more robust (and composable) sketch, that can support $t$ queries in total at the cost of increasing the space complexity by a factor of  
$\approx\sqrt{t}$.
Instantiating this wrapper with classical (non-robust) cardinality sketches results in a sketch for cardinality estimation that uses space $k\approx\sqrt{t}/\alpha^2$, where $\alpha$ is the accuracy parameter.

\textbf{Lower bounds.} 
Lower bounds on robustness are established by designing {\em attacks} in the form of adaptive sequences of queries. The objective of an attack is to force the algorithm to fail. An attack is more efficient if it causes the algorithm to fail using a smaller number of adaptive queries. We refer to number of queries in the attack as the {\em size} of the attack, which is typically a function of the sketch size $k$. Some attacks are {\em tailored} to a particular estimator, while others are {\em universal} in the sense that they apply to {\em any} estimator.


For cardinality sketches,
\citet{DBLP:journals/icl/ReviriegoT20} and~\citet{cryptoeprint:2021/1139} constructed $\tilde{O}(k)$-size attacks on the popular HLL sketch with its standard estimator. %\footnote{See \citep{hyperloglog:2007,hyperloglogpractice:EDBT2013}.},
\citet{AhmadianCohen:ICML2024} constructed 
$\tilde{O}(k)$-size attacks for popular MinHash sketching maps with their standard estimators, as well as a $\tilde{O}(k^2)$-size universal attacks. \citet{GribelyukLWYZ:FOCS2024} presented polynomial-size universal attacks on all linear sketching maps for cardinality estimation. Finally, \citet{CNSSS:ArXiv2024} presented optimal $\tilde{O}(k^2)$-size universal attacks on essentially all composable and linear sketching maps\footnotemark.


To summarize, there are matching upper and lower bounds of $t=\tilde{\Theta}(k^2)$ on the number of adaptive cardinality queries that can be approximated using a sketch of size $k$. The upper bound is obtained from the generic wrapper of \citet{HassidimKMMS20}, while the lower bound arises from the universal attack of \citet{CNSSS:ArXiv2024}. %against all composable and linear sketching maps for cardinality estimation. 
We refer to this limitation as the \emph{quadratic barrier}.



\subsection{Per-key participation}

Nevertheless, we can still hope for stronger {\em data-dependent} guarantees, and since cardinality sketches are widely used in practice, achieving this under realistic conditions is important. Specifically, we seek common properties of input queries %and a compatible sketch and estimator design 
such that, if these properties hold, we can guarantee accurate processing of $t \gg k^2$ adaptive queries.



We consider a parameter $r$ that is the \emph{per-key participation} in queries. Current attack constructions are such that most keys are involved in a large number of the queries and therefore $r\approx t$. However, in many realistic scenarios, the majority of keys participate in only a small number of queries and $r \ll t$. 
This pattern emerges when the distribution over the key domain shifts over time. For instance, the popularity of watched videos or browsed webpages can change over time, leading to a changing  set of access frequencies of keys.  Additionally, even when the query distribution is fixed, this pattern is consistent with Pareto-distributed frequencies, where a small fraction of keys (the ``heavy hitters'') appear in most queries, while most keys appear in only a limited number of queries. We therefore pose the following question:

\footnotetext{Here, ``essentially all'' means that the maps must satisfy certain basic reasonability conditions.}

\begin{ques} \label{main:problem}
Can we shift the quadratic barrier from the total number of queries $t$ to the typically much smaller parameter $r$, that is, can we design a robust (and composable) sketch of size $k\approx\sqrt{r}$ instead of $k\approx\sqrt{t}$? 
\end{ques}

\SetKwFunction{BkSketch}{BkSketch}
\SetKwFunction{SCest}{StdEst}
\SetKwFunction{RBCest}{RobustEst}
\SetKwFunction{RCest}{TRobustEst}

\subsection{Results Overview}
We provide an affirmative answer to \cref{main:problem}. Specifically, we design a sketch and estimator capable of handling an exponential number of adaptive queries provided that each key participates in at most $r=\tilde{O}(k^2)$
queries. We further provide an extension which maintain the guarantee even if this condition fails for a small fraction of the keys in each query. %Furthermore, our design gracefully degrades in approximation quality when a fraction of the keys in the sketch exceeds $r$ queries.

\textbf{Reformulating the robustness wrapper (\cref{framework:sec}).} As we mentioned, the generic wrapper of \citet{HassidimKMMS20} transforms non-robust sketches into more robust ones by obscuring their internal randomness using differential privacy. This effectively ``reduces'' the problem of designing a robust sketch to that of designing a suitable differentially private aggregation procedure. Our first contribution is to reformulate this wrapper so that the reduction is not to differential privacy, but rather to the problem of {\em adaptive data analysis (ADA)}.

In the ADA problem, we get an input dataset $S$ sampled from some unknown distribution $\DDD$, and then need to answer a sequence of {\em adaptively chosen statistical queries (SQ)} w.r.t.\ $\DDD$. This problem was introduces by \citet{DworkFHPRR15} who showed that it is possible to answer $\approx |S|^2$ statistical queries efficiently. The application of differential privacy as a tool for the ADA problem predated its application for robust data structures.

Our reformulated wrapper has two benefits: (1) It allows us to augment the generic wrapper with the granularity needed to address \cref{main:problem}, whereas the existing wrapper lacks this flexibility. (2) Even though our construction ultimately solves the ADA problem using differential privacy, other known solutions to the ADA problem now exist, and it is conceivable that future applications might need to leverage properties of these alternative solutions.



%Our first observation is that the existing robustness wrapper, which treats each sampled sketching map as a single privacy unit, lacks the granularity needed to address \cref{main:problem}. From a technical standpoint, tackling this challenge requires a fundamentally different approach to translating sketching maps into the statistical query (SQ) model, combined with a fine-grained analysis of SQ.

To this end, in \cref{framework:sec} we introduce a tool: an SQ framework with fine-grained generalization guarantees. %which combines individual privacy charging analysis with the SQ model. 
This framework addresses an analogous version of \cref{main:problem} for the ADA problem, where $k$
represents the sample size and $r$
is the maximum number of query predicates that are satisfied by a given key $x$.
To adapt this tool for sketching, we need to represent the randomness determining the sketching map as a sample from a product distribution and express the query response algorithm in terms of appropriate statistical queries. %The framework is general and can, in particular, be applied to analyze the robust version of CountSketch introduced by~\citet{CLNSSS:ICML2022}.


\textbf{Robust estimators for bottom-${\boldsymbol{k}}$ sketch.}
We use our fine-grained SQ framework in order to design a robust version for the popular bottom-$k$ MinHash cardinality sketch
\cite{Rosen1997a,ECohen6f,BRODER:sequences97,BJKST:random02}.
The randomness in the bottom-$k$ sketch corresponds to a map from keys to i.i.d.\ random priorities. The sketch $\txtBkSketch(V)$ of a subset $V$ includes  the $k$ keys with lowest priorities and their priority values. 
The standard cardinality estimator for this sketch returns a function of the highest priority in the sketch, which is a sufficient statistic for the cardinality. 
A sketch size of
$k=\tilde{\Omega}(\alpha^{-2})$ yields with high probability a relative error of $1\pm\alpha$.
The standard estimator, however, can be compromised using an attack with $t=\tilde{O}(k)$  queries~\cite{AhmadianCohen:ICML2024}. We present two robust estimators for the bottom-$k$ sketch that are analyzed using our fine-grained SQ framework:

\begin{itemize}[leftmargin=8.5pt,topsep=0pt,itemsep=0pt]
    \item \textbf{Basic Robust Estimator (\cref{bottomk:sec}).} We present a {\em stateless} estimator and show that for $\alpha\in (0,1)$ and $r=\tilde{\Omega}(k^2\alpha^{4})$, all estimates are accurate with high probability provided that all keys participate in no more than $r$ query sketches. In particular, since it always holds that $r\geq t$, this implies a guarantee of $t=\tilde{\Omega}(k^2\alpha^{4})$  on the number of adaptive queries.
Note that the sketch size ``budget'' of $k$ can be used to trade off accuracy and robustness.


\item \textbf{Tracking Robust Estimator (\cref{trackingest:Sec}).} 
We present another estimator that tracks the exposure of keys based on their participation in query sketches. Once a limit of $r$ is reached, the key is deactivated and is not used in future queries. 
The tracking estimator allows for smooth degradation and continues to be accurate as long as at most an $\alpha$ fraction of entries in the sketch are deactivated. 
Note that the tracking state is maintained by the query responder (server-side) and does not effect the computation or size of the sketch.
% The advantages of this approach is that it guarantees that the sketching map is not leaked and no adversary can compromise it.
% This is important because adversarial inputs can be constructed from workloads of benign queries \citep{AhmadianCohen:ICML2024} and this allow for robust long term use of the sketching map.
% This is much stronger than a guarantee in terms of the entries \emph{in the query set}. The sketch contains at most $k$ keys and the query set can be huge.
\end{itemize}



% with $k=\tilde{\Omega}(\sqrt{t}\alpha^{-2})$ for $t$ adaptive queries.




% \item [(i)] \textbf{Basic Robust Estimator} (\cref{bottomk:sec}): 
% This estimator does not maintain state from prior queries and accuracy and
%This estimator does not maintain state across queries. The statistical guarantees are for $r=\tilde{\Omega}(k^2\alpha^{-2})$, provided that all keys participate in no more than $r$ query sketches. 
% \ignore{It allows for smooth degradation in terms of the total number of keys that appeared in more than $r$ query sketches.}
% Our bottom-$k$ robust sketch is also a cleaner design compared with the super-sketch approach: We simply increase the size parameter of a standard bottom-$k$.
% is a standard sketch that can be scaled up to any size $k$ and has improved constant factors on the quadratic $t=\tilde{\Omega}(k^2)$ guarantee.


% We also show that the estimation quality declines with the number of keys in the query set that exceeded the participation limit. In particular, when the inputs do not adhere to participation limit, the responses leak information on the sketching map that may compromise it (in that an adversarial distribution can be constructed).

 
\textbf{Experiments.}
Finally, in \cref{experiments:sec}, we demonstrate the benefits of our fine-grained analysis using simulations on query sets sampled from uniform and Pareto distributions and observe $12 \times$ to $100 \times$ gains.


\subsection{Additional Related Work}

The adaptive setting has been studied extensively across multiple domains, including statistical queries~\citep{Freedman:1983,Ioannidis:2005,FreedmanParadox:2009,HardtUllman:FOCS2014,DworkFHPRR15,BassilyNSSSU:sicomp2021}, sketching and streaming algorithms~\citep{MironovNS:STOC2008,HardtW:STOC2013,BenEliezerJWY21,HassidimKMMS20,WoodruffZ21,AttiasCSS21,BEO21,DBLP:conf/icml/CohenLNSSS22,CNSS:AAAI2023Tricking,AhmadianCohen:ICML2024}, dynamic graph algorithms~\citep{ShiloachEven:JACM1981,AhnGM:SODA2012,gawrychowskiMW:ICALP2020,GutenbergPW:SODA2020,Wajc:STOC2020, BKMNSS22}, and machine learning~\citep{szegedy2013intriguing,goodfellow2014explaining,athalye2018synthesizing,papernot2017practical}.


\textbf{Lower bounds.}
For the ADA problem, 
\citet{HardtUllman:FOCS2014,SteinkeUllman:COLT2015} designed a quadratic-size universal attack, using Fingerprinting Codes~\citep{BonehShaw_fingerprinting:1998}. 
\citet{HardtW:STOC2013} designed a
polynomial-size universal attack on any linear sketching map for $\ell_2$ norm estimation.
\citet{DBLP:conf/nips/CherapanamjeriN20} constructed an
$\tilde{O}(k)$-size
attack on the Johnson Lindenstrauss Transform with the standard estimator.
\citet{BenEliezerJWY21} presented an
$\tilde{O}(k)$-size attack on the AMS sketch~\citep{ams99} with the standard estimator.
\citet{DBLP:conf/icml/CohenLNSSS22}
presented an
$\tilde{O}(k)$-size attack on Count-Sketch~\citep{CharikarCFC:2002} with the standard estimator.
\citet{CNSS:AAAI2023Tricking} presented  $\tilde{O}(k^2)$ size universal attack on the AMS sketch~\citep{ams99}  for $\ell_2$ norm estimation and on Count-Sketch~\cite{CharikarCFC:2002} (for  heavy hitter or inner product estimation).


% \ignore{

% The quintessential model for Adaptive Data Analysis (ADA) is  statistical queries (SQ) \citep{Kearns1998}. 
% There is a distribution $\Dd$ and a 
% statistical query with a predicate $h$ is for an estimate of the expected value $\E_{x\sim \Dd}[h(x)]$ of the predicate over $\Dd$. Statistical queries can be estimated from an i.i.d.\ sample $(x_1,\ldots,x_k)\sim \Dd^k$ using the empirical average $\frac{1}{k}\sum_{i\in[k]} h(x_i)$. 

% In non-adaptive settings, where queries are independent of prior results, it is possible to approximate an exponential number of queries in the sample size $k$, with small relative error. However, in adaptive settings, an adversary can exploit feedback from prior queries to overfit the sample, quickly constructing queries for which the empirical average deviates significantly from the true expectation.

% A simple approach to adaptive statistical queries is to partition our sample and use $\tilde{O}(\log k)$
% % $\approx 1/\alpha^2 \log(k)$ 
% fresh samples for each query. This allows for a linear  $t=\tilde{O}(k)$ number of adaptive queries.
% Advanced approaches improve this to support $t=\tilde{\Omega}(k^2)$ queries:
% \citep{DworkFHPRR:STOC2015,BassilyNSSSU:sicomp2021} showed that when the predicates preserve differential privacy of the sample then it still generalizes. Privacy is preserved by adding noise to the empirical averages before releasing them.
% % advanced composition~\citep{dwork2006calibrating}. 
% \citet{Blanc:STOC2023} proposed an alternative that does not add noise and instead uses a subsample to respond to each query.
% }%ignore


\section{Preliminaries}

\subsection{DP tools: linear queries with per-unit charging} \label{prelimDP:sec}

Differential privacy~\citep{dwork2006calibrating} (DP) is a Lipschitz-like stability property of algorithms, parametrized by $(\eps,\delta)$.
Two datasets $\boldsymbol{x},\boldsymbol{x}'\in X^n$ are \emph{neighboring} if they differ in at most one entry. Two probability distributions $\Dd$ and $\Dd'$ satisfy $\Dd\approx_{\eps,\delta} \Dd'$ if and only if for any measurable set of events $E$, $\Pr_D(E) \leq e^\eps \Pr_{\Dd'}(E)+\delta$ and $\Pr_{\Dd'}(E) \leq e^\eps \Pr_{\Dd}(E)+\delta$.
A randomized algorithm $A$ is \textit{$(\eps,\delta)$-DP} if for any two neighboring inputs $\boldsymbol{x}$ and $\boldsymbol{x}'$,
$A(\boldsymbol{x})\approx_{\eps,\delta} A(\boldsymbol{x}')$. DP algorithms compose in the sense that multiple applications of a DP algorithm to the dataset are also DP (with composed parameters). % We will use the following tools in a black-box manner.

\SetKwFunction{AboveThreshold}{AboveThreshold}
\SetKwFunction{BetweenThresholds}{BetweenThresholds}

Given a dataset 
$\boldsymbol{x} := (x_1,\ldots,x_n) \in X^n$ of items from domain $X$,
a \emph{counting query} is specified by a predicate 
$f:[n]\times X\to [0,1]$ and has the form
$f(\boldsymbol{x}) := \sum_{i\in [n]} f(i,x_i)$. For $\eps>0$, an algorithm that returns a noisy count 
$\hat{f}(\boldsymbol{x}) := f(\boldsymbol{x})+\Lap[1/\eps]$, where $\Lap$ is the Laplace distribution, satisfies $(\eps,0)$-DP. When multiple such tests are performed over the same dataset, the privacy parameters compose to $(r\eps,0)$-DP or alternatively to $(\sqrt{2r\log(1/\delta)}\eps +r\eps^2,\delta)$-DP for any $\delta>0$.

The Sparse Vector Technique (SVT) 
\citep{DNRRV:STOC2009,DBLP:conf/stoc/RothR10,DBLP:conf/focs/HardtR10,DBLP:books/sp/17/Vadhan17-dp-complex}
is a privacy analysis technique for a situation when an adaptive sequence of threshold tests on counting queries is performed on the same sensitive dataset $\boldsymbol{x}$. Each test $\AboveThreshold_\eps(f,T)$ is specified by a predicate $f$ and threshold value $T$.
The result is the noisy value $\hat{f}(\boldsymbol{x})$ if $\hat{f} > T$ and is $\perp$ otherwise. The appeal of the technique is a privacy analysis that only depends on the number $r$ of queries for which the test result is positive.

We will use here an extension of (a stateless version of) SVT, described in
\cref{algo:svt-individual}, 
that improves utility for the same privacy parameters  \citep{DBLP:conf/colt/KaplanMS21,feldman2021individual,CLNSSS:ICML2022,targetcharging:arxiv2023}.  
The algorithm maintains a set $A$ of \emph{active} indices that is initialized to all of $[n]$ and maintains charge counters $(C_i)_{i\in[n]}$, initialized to $0$.
For each query $(h,T)$, the response is   $\AboveThreshold^A_\eps(h,T)$ test result that is  $\hat{f} := \sum_{i\in A} h(i,x_i)+\Lap[1/\eps]$ if $\hat{h} > T$ and is $\perp$ otherwise. Note that $\AboveThreshold^A$ evaluates the query only over active indices. For each query with a positive (above) response, the algorithm increases the charge counts on all the indices that contributed to the query, namely, $h(i,x_i) = 1$. Once $C_i=r$, index $i$  is removed from the active set $A$ of indices.

The appeal of \cref{algo:svt-individual} is a fine-grained analysis that can only result in an improvement --
the privacy bounds have the same dependence on the parameter $r$, that in the basic approach bounds the total number $t$ of tests with positive outcomes and in the fine-grained one bounds the (potentially much smaller) per-index participation in such tests: 
\begin{theorem} [\cite{targetcharging:ICML2023} Privacy of Algorithm~\ref{algo:svt-individual}]\label{SVTindividual:thm}
For any $\eps < 1$ and $\delta \in (0, 1)$,  Algorithm~\ref{algo:svt-individual} is $(O(\sqrt{r \log(1/\delta)}\eps), 2^{-\Omega(r)} + \delta)$-DP (see \cref{thm:TCprivacy} for more precise expressions).
\end{theorem}

\begin{algorithm2e}[htbp]
    \caption{Linear Queries with Individual Privacy Charging}
    \label{algo:svt-individual}
    \DontPrintSemicolon
\small{    
%    \LinesNumbered
    \KwIn{
        Sensitive data set $(x_1,\ldots,x_n)\in X^n$; privacy budget $r > 0$; Privacy parameter $\eps>0$.
    }
    \lForEach(\tcp*[f]{Initialize counters}){$i\in [n]$}{
        $C_i\gets 0$ 
    }
    $A\gets [n]$ \tcp*{Initialize the active set}

\textbf{Function} $\AboveThreshold^A_\eps(h,T)$ \tcp*{\AboveThreshold query} 
\Indp
\KwIn{predicate $h:[n]\times X\to \{0,1\}$ and threshold $T\in \mathbb{R}$}
$\hat{h} \gets \left(\sum_{i\in A}h(i,x_i)\right) + \Lap(1/\eps) $ \tcp*{Laplace noise}
        \eIf(\tcp*[f]{Test against threshold}){$\hat{h} \ge T$}{
            \ForEach{$i\in A$ such that $h(i,x_i)  = 1$}{
                $C_i\gets C_i + 1$ \;
                \lIf{$C_i = r$}{
                    $A\gets A\setminus\{i\}$ 
                }
            }
            \Return{$\hat{h}$ \;
        }
        }{\Return{$\perp$}}
\Indm        
    \BlankLine
\OnInput(\tcp*[f]{Main Loop: process queries}){$(f,T)$}{ \Return{$\AboveThreshold^A_\eps(f,T)$}}
}
\end{algorithm2e}



\subsection{ADA tools}



The generalization property of differential privacy applies when the dataset $\bsx$ is sampled from a distribution. It states that if a predicate $h$ is selected in a way that preserves the privacy of the sampled points then we can bound its generalization error: the count over $\bsx$ is not too far from the expected count when we sample from the distribution. We will use the following variant of the cited works (see \cref{genproof:sec} for a proof):


% \begin{theorem}[Generalization property of DP \cite{DworkFHPRR15,BassilyNSSSU:sicomp2021,FeldmanS17}] \label{thm:DP-generalization} \label{thm:DP-gen-mod}
\begin{restatable}[Generalization property of DP \cite{DworkFHPRR15,BassilyNSSSU:sicomp2021,FeldmanS17}]{theorem}{dpgen} \label{thm:DP-generalization} \label{thm:DP-gen-mod}
Let $\mathcal{A}:X^n \to 2^{X}$ be an $(\eps, \delta)$-differentially private algorithm that operates on a dataset of size $n$ and outputs a predicate $h: X\to \{0,1\}$. Let $\Dd=D_1\times\cdots D_n$ be a product distribution over $X^n$, let $\bsx=(x_1,\ldots,x_n) \sim \Dd$ be a sample from $\Dd$, and let $h\gets \mathcal{A}(\bsx)$. Then for any $T\ge 1$ it holds that 
\footnotesize{
\begin{align*}
    \Pr_{\substack{\bsx\sim \Dd,\\ h\gets \mathcal{A}(\bsx)}}\left[ e^{-2\e} \E_{\boldsymbol{y}\sim \Dd} h(\boldsymbol{y}) - h(\bsx) > \frac{4}{\eps}\log(T+1) + 2Tn\delta \right] < \frac{1}{T},\\
    \Pr_{\substack{\bsx\sim \Dd,\\ h\gets \mathcal{A}(\bsx)}}\left[
h(\bsx) -
e^{2\e} \E_{\boldsymbol{y}\sim \Dd} h(\boldsymbol{y}) 
> \frac{4}{\eps}\log(T+1) + 2Tn\delta \right] < \frac{1}{T},
\end{align*}
}
% \end{theorem}
\end{restatable}

where $h(\bsy)$ denotes the total value of $h$ over elements of $\bsy$.

When applying \cref{thm:DP-generalization}, we will assume that $h$ also takes the index $i$ as an argument (so, we write $h(i, x_i)$ instead of $h(x_i)$). This is equivalent because we can replace $D_i$ with a distribution that samples the tuple $(i, x_i)$ for $x_i \sim D_i$. 

\section{ADA with fine-grained analysis} \label{framework:sec}

We now consider a variation of the ADA framework where
we sample a dataset $\boldsymbol{x}\sim \Dd$ from a  product distribution $\Dd$ and then process adaptive linear threshold queries as in Algorithm~\ref{algo:svt-individual} over the dataset $\boldsymbol{x}$. 
The benefit of this is obtaining bounds in terms of the 
per-key participation in queries (that is the number of queries $h$ for which $h(i,x_i)=1$), which is always lower
than the total number of queries. 
Moreover, the approach tolerates a small fraction of deactivated indices in each query, which simply contribute proportionally to the error. 

We bound the error due to generalization and sampling and due to the privacy noise and the deactivation of keys that reached the charging limit $r$:

\begin{lemma} [Generalization and sampling error bound] \label{generror:lemma}
    Let $\Dd=D_1\times\cdots\times D_n$ be a product distribution over $X^n$. Let 
    $\boldsymbol{x}\sim \Dd$ be a sampled dataset. Let $\alpha, \beta > 0$ be sufficiently small (i.e., smaller than some absolute constant).
    Consider an execution of \cref{algo:svt-individual} on dataset $\boldsymbol{x}$ with $m$ adaptive queries, parameter $r \gg \log (n/\beta)$, and \[ \e_0 \coloneqq \frac{\alpha}{4\sqrt{r\log(n/\beta)}}.\] 
    Then it holds that with probability at least $1-\beta$, for all of the $m$ query predicates $h$,
    {\small
\[
 \left| \E_{\boldsymbol{y}\sim \Dd} [h(\boldsymbol{y})] - h(\boldsymbol{x}) \right| <  \alpha \cdot \E_{\boldsymbol{y}\sim \Dd} [h(\boldsymbol{y})] + O\p{\frac{\log(m/\beta)}{\alpha}}.
\]}    

\end{lemma}
\begin{proof}
    The first claim of the sampling and generalization error, follows from \cref{SVTindividual:thm} and \cref{thm:DP-generalization}
    
    For the privacy parameters in \cref{SVTindividual:thm} we set $\delta = \beta/n^2$ and obtain $\eps = \sqrt{r\log(n^2/\beta)}\cdot\eps_0 < \al/2\sqrt 2$.
    From \cref{thm:DP-generalization} we get that, for each query $h$, 
    the additive error $\left| \E_{\boldsymbol{y}\sim \Dd} h(\boldsymbol{y}) - h(\boldsymbol{x}) \right|$ is at most \[
    (e^{2\e}-1)\cdot \E_{\boldsymbol{y}\sim \Dd} h(\boldsymbol{y}) + \frac{4}{\eps }\log(T+1) + 2Tn\delta,
    \]
    with probability at least $1-2/T$.
    Note that we have $e^{2\e}-1 < \al$. Thus, setting $T=2m/\beta$ (so that a union bound over all queries gives a failure probability of $1-\beta$), the result follows.
\end{proof}

\begin{claim} [Noise and deactivation error bounds] \label{noisedeactiveerror:claim}
Under the conditions of \cref{generror:lemma},
with probability at least $1-\beta$, for all $m$ query predicates $h$, for $\hat{h} \coloneqq \sum_{i\in A} h(i,x_i )+\Lap(1/\eps_0)$,
\begin{gather*}
    h(\boldsymbol{x}) - \hat{h} >
    - \log(2m/\beta)/\eps_0,\\
    h(\boldsymbol{x}) - \hat{h} <
    \log(2m/\beta)/\eps_0 + \sum_{i\in [n]\setminus A} h(i,x_i).
\end{gather*}
\end{claim}
\begin{proof}
    Each Laplace noise $\Lap(1/\eps_0)$ is bounded by $\pm\log(2m/\beta)/\eps_0$ with probability at least $1-\beta/m$, so by a union bound, with probability at least $1-\beta$, it is bounded as such for all queries, and the result follows immediately.
\end{proof}

The total error of $\hat{h}$ with respect to the expectation $\E_{\boldsymbol{y}\sim \Dd} h(\boldsymbol{y})$ is bounded by the sum of errors in \cref{generror:lemma} and \cref{noisedeactiveerror:claim}:
\begin{corollary} \label{totalerror:coro}
For some constants $c_1,c_2>0$, under the conditions of \cref{generror:lemma}, with probability at least $1-\beta$, for all $m$ queries $h$,
\[ -\Delta <
\E_{\boldsymbol{y}\sim \Dd} [h(\boldsymbol{y})] - \hat{h} 
< \Delta + \sum_{i\in [n]\setminus A} h(i,x_i),
\]
where
\[
\Delta = \alpha\cdot \E_{\boldsymbol{y}\sim \Dd} [h(\boldsymbol{y})] + O(\al^{-1} \sqrt{r} \log^{3/2}(mn/\beta)).
\]
\end{corollary}

We can apply this fine-grained ADA to analyze the robustness of randomized data structures (or algorithms) that sample randomness $\boldsymbol{\rho}$ and process adaptive queries $M_i$ that depend on the interaction till now and the randomness $\boldsymbol{\rho}$. 
To do so, we need to specify a product distribution 
$\mathcal{\Dd} = D_1 \times D_2 \times \dots \times D_n$
so that 
\begin{enumerate}
\item
    The distribution of $\boldsymbol{\rho}$ is $\Dd$.
\item The queries in the original problem can be specified in terms of  linear queries over $\bsr$ and have statistical guarantees of utility over $\bsr \sim \Dd$. 
% Moreover, we get utility also with additional relative error 
% so that we get utility if the approximation is good over the distribution of $\rho \sim \Dd$.
\end{enumerate}
When applying this to sketching maps which do not contain all the information of $\bsr$, we will need to ensure that the linear queries we use can be evaluated over the sketch.


% \ignore{
% \begin{remark}
%     Current writing is just for "above threshold" queries. But framework can be generalize to between threshold (pay only when close to threshold) and selection (multiple predicates and select one or top-$k$ scores and pay only for selections). That is, applications in the target charging framework. But the individual DP algorithms are just counts since this is what the ADA framework does (linear queries). 
% \end{remark}
% }


\section{The Bottom-\texorpdfstring{$k$}{k} Cardinality Sketch} \label{bottomk:sec}



\begin{algorithm2e}[t]\caption{Bottom-$k$ Cardinality Sketch and Standard Estimator}\label{bottomk:algo}
\DontPrintSemicolon
{\small
Sample $\rho_i \sim U[0, 1]$ for $i\in [n]$.\tcp{Randomness for the Sketching Map}

\Function(\tcp*[f]{Bottom-$k$ sketching map using $\boldsymbol{\rho}$}){ $\BkSketch_{\boldsymbol{\rho}}(V)$}
{
\KwIn{Set $V\subset [n]$}
\eIf{$|V|\leq k$}{\Return{$\{(i, \rho_i) \mid i \in V\}$ }}
{\Return{$\{(i, \rho_i) \mid i \in V, \rho_i < \boldsymbol{\rho}_{(k),V}\}$}\tcp{
where ${\boldsymbol{\rho}}_{(k),V}$ is the $k$th smallest in the multiset $\{\rho_i \mid i\in V\}$}
}}
\BlankLine
\Function(\tcp*[f]{Standard estimator}){$\SCest_k(S)$}{
\KwIn{A bottom-$k$ sketch $S$}
\eIf{$|S| < k$}{
    \Return{$|S|$} 
}{
$\tau \gets \max_{(i,\rho_i)\in S} \rho_i$\tcp*{$k$th smallest $\rho_i$}
        \Return{$(k-1)/\tau$}
}
}
\OnInput(\tcp*[f]{Main Loop}){$S$}{\Return{$\SCest_{k}(S)$}}
}
\end{algorithm2e}

\subsection{Sketch and standard estimator}
The bottom-$k$ cardinality sketch and standard estimator are described in \cref{bottomk:algo}. 
Let the ground set of keys be $[n]$.
We sample a vector of random values $\boldsymbol{\rho} \sim [0, 1]^n$.
% ; this will be the database which we keep private. 
That is, for each key $i \in [n]$ there is an associated i.i.d.\ $\rho_i \sim \Dd$.
The vector $\boldsymbol{\rho}$ specifies the bottom-$k$ sketching map $\txtBkSketch_{\boldsymbol{\rho}}(V)$ that maps a set $V\subset [n]$ to its sketch. The sketch consists of the pairs $(i, \rho_i)$ for the $k$ values of $i \in V$ such that $\rho_i$ is smallest. When $|V|\leq k$, the sketch contains all elements of $V$. Note that, though $n$ and also $|V|$ can be very large, the size of the sketch is at most $k$.   This sketching map is clearly composable.\footnote{The analysis uses a common assumption of full i.i.d.\ randomness in the specification of the sketching maps. Note that $O(\log k)$ bits of representation are sufficient. Implementations use pseudo-random hash maps $i\mapsto \rho_i$.}

For a query set $V\subset [n]$, we apply an estimator to the sketch $S := \txtBkSketch_r(V)$ to obtain an estimate of the cardinality of $V$. The standard estimator $\txtSCest(S)$ computes $\tau$ which is the $k$th order statistics of the $\rho_i$ values in the sketch, which is a sufficient statistic of the cardinality. It then returns the value $(k-1)/\tau$. The estimate is unbiased, has variance at most $|V|/(k-2)$, and an exponential tail (see e.g.~\cite{ECohenADS:TKDE2015}).
This standard estimator is know to optimally use the information in the sketch $S$ but
can be attacked with a linear number of queries~\cite{AhmadianCohen:ICML2024}.


\subsection{Basic robust estimator}

\begin{algorithm2e}[t]\caption{Basic Robust Cardinality Estimator}\label{bottomkrobust:algo}
\DontPrintSemicolon
{\small
\Function(\tcp*[f]{Estimate $|V|$ from $\BkSketch_{\boldsymbol{\rho}}(V)$}){$\RBCest_{k}(S)$}{
    \KwIn{A bottom-$k$ sketch $S$, $\alpha\in(0,0.5)$}
    
    \eIf(\tcp*[f]{Return exact value when $\leq k$}){$|S| < k$}{
        \Return{$|S|$}
    }{
        $\tau \gets k/2n$, $T = (1-\al)k$\;
        
        \While {$(\tau<1)$ and $(\tw{h}\gets \sum_{(i,\rho_i)\in S} \ind{(\rho_i < \tau)}+\Lap(1/\eps_0)) < T $} {
            $\tau \gets (1+\al/4)\tau$\;
        }
        
        \Return{$T/\tau$}        
    }  
}
\KwIn{Parameters $k, r, n\geq 1$ and $\al, \beta > 0$}
$\eps_0 \gets \f{\al/8}{4\sqrt{r \log(n/(\beta/4))}}$ \tcp{as \cref{generror:lemma} with $\frac{\al}{8}$, $\frac{\beta}{4}$}
\OnInput(\tcp*[f]{Main Loop}){$S$}{\Return{$\RBCest_{k,r}(S)$}}
}
\end{algorithm2e}



\cref{bottomkrobustbaseline:algo} describes a robust estimator $\txtRBCest$ that is applied to a bottom-$k$ sketch.

We analyze this estimator under the assumption that the query set sequence $(V_j)_{j\in [t]}$ has the property that each key
$i\in [n]$ 
appears in at most $r$ sketches in $( \txtBkSketch_{\boldsymbol{\rho}}(V_i))_{j\in[t]}$:
\begin{equation} \label{limitassumptionsketch:eq}
\forall i\in[n], \sum_{j\in[t]} \ind{i\in  \txtBkSketch_{\boldsymbol{\rho}}(V_i)} \leq r.
\end{equation}
Note that for this to hold it suffices that each key 
is included in at most $r$ query sets. That is,
$\forall i\in[n], \sum_{j\in[t]} \ind{i\in V_j} \leq r$.

\begin{theorem} [Basic robust estimator guarantee] \label{basicrobust:thm}
    If the query sequence in \cref{bottomkrobust:algo} satisfies \eqref{limitassumptionsketch:eq} for some $r \gg \log(n/\beta)$, then for a value of $k = O(\al^{-2} \sqrt r \log^{3/2}(n/\beta))$, every output will be $(1\pm\al)$-accurate with probability at least $1-\beta$.
\end{theorem}

% \eccomment{Shouldn't it be $1\pm\alpha$? Can we say that we can replace $n$ with the max of query set size and $t$? }

\subsection{Analysis of Basic robust estimator} \label{sec:analysis-basic}
In this section we prove \cref{basicrobust:thm}.

Before we start, we make some basic assumptions on the parameters which we will use throughout the proof. First, we pick
\[k = C \al^{-2} \sqrt r \log^{3/2}(n/\beta),\]
where the constant $C$ is chosen to be sufficiently large. Furthermore, note that if $k \ge n$ then we are always storing the whole set $V$ in the sketch, so we may assume that $k < n$ (and therefore $r < n^2$ and $\al > 1/\sqrt{n}$).

We map \cref{bottomkrobust:algo} to 
the framework of \cref{algo:svt-individual}, where the dataset is $\boldsymbol{\rho}$.

First, we show that we can consider the sum of $\ind{\rho_i < \tau}$ to be over the entire set $V$, rather than just those that are included in the bottom-$k$ sketch $S$:

\begin{lemma} \label{lemma:sub-h-hat}
Suppose that, in \cref{bottomkrobust:algo}, $\tw h = \sum_{(i,\rho_i)\in S} \ind{(\rho_i < \tau)}+\Lap(1/\eps_0))$ is replaced by $\hat h = \sum_{i \in V} \ind{(\rho_i < \tau)}+\Lap(1/\eps_0))$ (where the two Laplace random variables are coupled to be the same value). Then, the sequence of outputs of \cref{bottomkrobust:algo} changes with probability at most $\beta/4$.
\end{lemma}
\begin{proof}
Since $S$ contains the $k$ values of $i$ such that $\rho_i$ is minimal, the only way to have $\hat h \neq \tw h$ is to have the sum over $S$ be equal to $k$ and the sum over $V$ to be greater than $k$. The outputs may then only differ if $\tw h < T$, but the probability that $k + \Lap(1/\eps_0) < (1-\al)k$ is at most $e^{-\e_0 \al k} < \beta/\poly(n)$,
where the polynomial in $n$ can be made as large as we like (by setting the constant on $k$).

Now, note that the total number of queries to \cref{bottomkrobust:algo} cannot exceed $nr \le \poly(n)$ by \eqref{limitassumptionsketch:eq}. Furthermore, the total number of iterations of the while loop per call is at most $O(\al^{-1} \log n) = \poly(n)$.

The lemma follows by taking a union bound over all iterations of the while loop and over all queries to \cref{bottomkrobust:algo}.
\end{proof}

With this lemma in mind, we will henceforth assume through this entire section that \cref{bottomkrobust:algo} uses $\hat h$ instead of $\tw h$, introducing a failure probability of at most $\beta/4$.

Now, we will show that the execution of \cref{bottomkrobust:algo} can be performed via queries to \cref{algo:svt-individual}, rather than accessing $\bsr$ directly. Indeed, note that the counting query $\hat h$ takes the same form (except for the check being over $[n]$ instead of $A$) as its analog in \cref{algo:svt-individual}, where the query function is
\[
h_{V,\tau}(i,\rho_i) \coloneqq \ind{i\in V \land \rho_i<\tau}.
\]

Observe that for any query sketch $S$, there is at most one positive test in \cref{bottomkrobust:algo}. Therefore, per assumption \eqref{limitassumptionsketch:eq} on the input, each index appears in at most $r$ positive tests. Therefore, if we were to instead perform these tests using \cref{algo:svt-individual}, all indices would remain active and nothing would ever be removed from $A$. Thus, we would have that $A=[n]$, so indeed the values of $\hat h$ are identical in \cref{algo:svt-individual} and \cref{bottomkrobust:algo}. Thus, \cref{bottomkrobust:algo} can be simulated by queries to \cref{algo:svt-individual}, so we may apply the results of \cref{framework:sec}.

In order to apply \cref{totalerror:coro}, we need to first compute the expectation of $h_{V, \tau}$ on $\Dd$:
\begin{claim} \label{card:claim}
    \begin{equation} \label{expectation:eq}
\E_{\boldsymbol{y}} \left[h_{V,\tau}(\boldsymbol{y})\right] = \tau |V|.
\end{equation}
\end{claim}
\begin{proof}
  Observe that for $i\not\in V$, $h_{V,\tau}(i,y_i)=0$ for all $y_i$ and for $i\in V$,
$\E[h_{V,\tau}(i,y_i)] = \tau$.  
\end{proof}

Finally, recall from the proof of \cref{lemma:sub-h-hat} that the total number of iterations of the while loop (and thus the overall total number of calls to \cref{algo:svt-individual}) is at most $\poly(n)$, so in \cref{totalerror:coro} we can take $m=\poly(n)$.

\begin{algorithm2e}[t!]\caption{Tracking Robust Estimator}\label{bottomkrobustbaseline:algo}
\small{
\DontPrintSemicolon
\Function(\tcp*[f]{Robust Cardinality Estimate of $V$ from $\BkSketch_{\boldsymbol{\rho}}(V)$}){$\RCest_{k,r}(S)$}{
    \KwIn{A bottom-$k$ sketch $S$}
    
    \eIf(\tcp*[f]{Return exact value when $\leq k$}){$|S| < k$}{
        \Return{$|S|$}
    }{
        $\tau \gets k/2n$, $T \gets k/4$\;
        
        \While {$(\tau<1) \land (\tw h \ot \sum_{(i,\rho_i)\in S} \ind{(\rho_i < \tau)\land (C[i]<r)}+\Lap(1/\eps_0)) < T$} {
            $\tau \gets (1+\al/8)\tau$\;
        }
        % \eIf(\tcp*[f]{too many exposed keys}){$\tau>1$}{\Return{$\bot$}}{
        \ForEach(\tcp*[f]{Per-key tracking}){$(i,x_i)\in S$}{
            % \If(\tcp*[f]{increment exposure counter}){$x_i<\tau$}{\leIf{$i\not\in C$}{$C[i]\gets 1$}{$C[i]\gets C[i]+1$}}
            \lIf{$x_i<\tau$}{
                $C[i]\gets C[i]+1$
            }
        }
        
        \Return{$T/\tau$}
        % }
    }  
}
\KwIn{Parameters $k$, $r\geq 1$}
\tcp{Initialization}
$C\gets \{ \}$ \tcp*{Dictionary with default value $0$} 
$\eps_0 \gets \f{\al/16}{4\sqrt{r \log(n/(\beta/4))}}$\tcp{as \cref{generror:lemma} with $\frac{\al}{16}$, $\frac{\beta}{4}$}
\OnInput(\tcp*[f]{Main Loop}){$S$}{\Return{$\RCest_{k,r}(S)$}}
}
\end{algorithm2e}

Now, by \cref{totalerror:coro}, we have with probability at least $1-\beta/4$ that
\begin{gather}
\hat h < (1 + \al/8) \tau |V| + \al k / 8, \label{eq:hat-ub} \\
\hat h > (1 - \al/8) \tau |V| - \al k / 8, \label{eq:hat-lb}
\end{gather}
where we have used that
\[\Delta = O(\al^{-1} \sqrt r \log^{3/2}(n/\beta)) < \al k / 8,\]
by the choice of $k$. (Recall also that $[n] \setminus A$ is always empty, so the sum term in \cref{totalerror:coro} vanishes.) We assume henceforth that this probability-$(1-\beta/4)$ event does in fact occur. 

\begin{prop} \label{prop:low-guarantee}
Whenever $\tau < (1 - \al/2) T / |V|$, the while loop in \cref{bottomkrobust:algo} continues to the next value of $\tau$.
\end{prop}
\begin{proof}
Note that $\al k / 8 < \al T / 4$ since $T > k/2$. Thus, by \eqref{eq:hat-ub}, when $\tau < (1 - \al/2) T / |V|$, we have $\hat h < (1 + \al/8)(1-\al/2)T + \al T / 4 < T$ (for sufficiently small $\al$), so we are done.
\end{proof}

\begin{prop} \label{prop:high-guarantee}
Whenever $\tau > (1 + \al/2) T / |V|$, the while loop in \cref{bottomkrobust:algo} terminates.
\end{prop}
\begin{proof}
Again, by \eqref{eq:hat-lb}, when $\tau > (1 + \al/2) T / |V|$, we have $\hat h > (1 - \al/8)(1 + \al/2)T - \al T / 4 > T$, so we are done.
\end{proof}

Now, \cref{prop:low-guarantee} ensures that the output of the algorithm is always at least $(1-\al/2)|V|$. Moreover, since $\tau$ is incremented by factors of $1+\al/4$, there will be some value of $\tau$ tested that is between $(1 + \al/2) T / |V|$ and $(1 + \al) T / |V|$ (note that since $|V| \ge k$, we have $(1 + \al) T / |V| < 1$). By \cref{prop:high-guarantee}, this value will cause the while loop to terminate, yielding an output that is at most $(1+\al)|V|$. This completes the proof of \cref{basicrobust:thm}.


\begin{figure*}[t!]
    \centering
    \includegraphics[width=0.32\textwidth]{plots/Trobust_uniforms5000sweepknum.png}
    \includegraphics[width=0.32\textwidth]{plots/Trubust_pareto2s5000sweepknum.png}
    \includegraphics[width=0.32\textwidth]{plots/Trobust_pareto15s5000sweepknum.png}
\caption{Number of guaranteed queries for sketch size $k$. The gain factor of \txtRCest\ over baseline is over two orders of magnitude with the Uniform distribution, $40\times$ for Pareto with $\alpha=2$, and $12\times$ for Pareto with $\alpha=1.5$.}
    \label{fig:fine-grained-gain}    
\end{figure*}   

When assumption 
\eqref{limitassumptionsketch:eq} does not hold, that is, when some keys get \emph{maxed} (have participated in more than $r$ query sketches), the guarantees are lost even when there are 
no maxed keys in the query set. The universal attack constructions of~\cite{AhmadianCohen:ICML2024,CNSSS:ArXiv2024} show this is unavoidable. The attack 
fixes a ground set $U$ and identifies keys with low priorities (these are the keys that tend to be maxed). The query sets that is $U$ with the identified keys deleted has cardinality close to $|U|$ but the estimates of \txtRBCest\ would be biased down.
In the next section we introduce a tracking estimator that allows for smooth degradation in accuracy guarantees as keys get maxed. 




\section{Robust estimator with tracking} \label{trackingest:Sec}


We next propose and analyze the estimator $\RCest$ in \cref{bottomkrobust:algo} that is an extension of \txtRBCest\ that includes tracking and deactivation of keys that appeared in the query sketches more than $r$ times. 
This estimator offers smooth degradation in estimate quality that depends only on the number of \emph{deactivated} keys present in the sketch and this is guaranteed as long as there are no queries where most of the sketch is deactivated. 
% It provides key-level robustness guarantees. Its advantages over basic:

% This estimator is guaranteed, on any input, not to leak information on the sketching map that can be used to construct adversarial inputs.







%\newcommand{\leIf}[3]{\textbf{if} #1 \textbf{then} #2 \textbf{else} #3}

% \begin{theorem} \label{thm:main-tracking}
\begin{restatable}[Analysis of \txtRCest]{theorem}{trackinganalysis}
\label{thm:main-tracking}
For a value of 
$k = O(\al^{-2} \sqrt r \log^{3/2}(n/\beta))$,
% $k=O(\al^{-2} \log(n/\beta) + \al^{-1} \sqrt r \log^{3/2}(mn/\beta))$, 
suppose that an adaptive adversary provides at most $m$ inputs to \cref{bottomkrobustbaseline:algo} such that the sketch of every input has at most $k/2$ deactivated keys. Then, with probability at least $1-\beta$, for every input whose sketch has at most $\al k / 4$ deactivated keys, the output is a $(1+\al)$-approximation of the true cardinality.
\end{restatable}

This theorem guarantees that, as long as no query has too many (more than $k/2$) deactivated keys, the results of \cref{bottomkrobustbaseline:algo} will continue to be accurate even for queries that have a few (at most $\al k/4$) deactivated keys. This allows the algorithm to continue guaranteeing accuracy even if a few keys are subject to many queries. We discuss the numerical advantages of this further in \cref{experiments:sec}.
The proof is analogous to \cref{sec:analysis-basic} and is provided in \cref{trackinganalysis:sec}.


\section{Empirical Demonstration} \label{experiments:sec}

We demonstrate the effectiveness of our fine-grained approach by
comparing the number of queries that can be answered effectively with \txtRCest\ to that of the baseline per-query analysis.
We use synthetically generated query sets sampled from Uniform and Pareto distributions with $\alpha\in \{1.5,2\}$ and $x_m=1$, support size of $10^6$ and query set size of $5\times 10^3$.
For each sketch size $k$, we match a value of the parameter $r= 0.002 k^2$ (with \txtRCest) and respectively $t= 0.002 k^2$ with baseline analysis.
% For noise scale $\eps=5/k$ and bounding the number of  \AboveThreshold tests using target-charging analysis. 

With \txtRCest, we count the number of queries for which at most $10\%$ of sketch entries are deactivated and stop when there is a sketch with $50\%$ of entries deactivated.
Figure~\ref{fig:fine-grained-gain} reports the number of queries with the baseline and \txtRCest\ estimators. The respective gain factor is measured by the ratio of the number of queries that can be effectively answered with per-key analysis to the baseline. We observe gains of two orders of magnitude for uniformly sampled query sets. This hold even without tracking -- using \txtRBCest\ -- where we stop as soon as there is a key that appeared in $r$ queries. For  Pareto query sets, tracking is necessary, as some keys do appear in many query sketches. We observe gains of $12\times$ for the very skewed $\alpha=1.5$ and $ 40\times$ with $\alpha=2$.


%The generalization bounds hold with estimator \RBCest, until $k/2$ keys in total are maxed, or with estimator \RCest, until there is a query sketch with more than $k/2$ keys deactivated. We count the number of queries that can be answered effectively when there are at most $10\%$ of sketch entries maxed or deactivated, respectively.






\section*{Conclusion}


Our work raises several follow-up questions. Our fine-grained robust estimators are specifically designed for the bottom-$k$ cardinality sketch.
We conjecture that it is possible to derive estimators with similar guarantees for other MinHash sketches, including the 
$k$-partition (PCSA -- Stochastic Averaging) cardinality sketches~\cite{FlajoletMartin85,hyperloglog:2007}. The missing piece is that the fine-grained ADA framework lacks the necessary flexibility, and requires an extension beyond plain linear queries. 
Another open question is whether similar results hold for other norms, particularly in scenarios where most inputs are sparse, and only a fraction of entries are 'heavy'—meaning they are nonzero across many inputs. For $\ell_2$ norm estimation with the popular AMS sketch~\cite{ams99}, the answer is negative, as known quadratic-size attacks remain effective even when inputs are sparse with disjoint supports~\cite{CNSS:AAAI2023Tricking}. However, we conjecture that similar results are possible for sublinear statistics, including capping statistics~\cite{CapSampling,CohenGeri:NeurIPS2019},  whose sketches incorporate generalized cardinality sketches, and for (universal or specialized) bottom-$k$ sketches, which are weighted versions of the bottom-$k$ cardinality sketch.
  
% Our fine-grained robust estimators are specifically designed for the bottom-$k$ cardinality sketch. It would be nice to derive such estimators for $k$-partition (stochastic averaging) cardinality sketches such as the popular HyperLogLog~\cite{hyperloglog:2007}. The hurdle is that 
% the fine-grained ADA framework does not have the needed flexibility and we need to extend it beyond plain linear queries.


% Another question is whether we can hope for similar results for other norms, in the situation that most inputs are sparse and at most a fraction of the entries are `heavy'' in the sense that they are nonzero on many inputs.
% For $\ell_2$  norm estimation, with popular sketches~\cite{ams99}, the answer is negative -- known quadratic size attacks apply even with sparse inputs with disjoint supports~\cite{CNSS:AAAI2023Tricking}. The question is open for sublinear statistics, including capping statistics~\cite{CapSampling,CohenGeri:NeurIPS2019} that have sketches based on extensions of cardinality sketches.

% This can be extended with data-dependent privacy analysis on linear queries (e.g. selection, between thresholds) where we only pay for queries that hit the target. 




\newpage

\section*{Acknowledgments}

Edith Cohen was partially supported by Israel Science Foundation (grant 1156/23). 
Uri Stemmer was Partially supported by the Israel Science Foundation (grant 1419/24) and the Blavatnik Family foundation.

\ignore{
\section*{Impact Statement}

This paper presents work whose goal is to advance the field of 
Machine Learning. There are many potential societal consequences 
of our work, none which we feel must be specifically highlighted here.
}


% \bibliographystyle{plainnat}
\bibliographystyle{icml2025}
% \bibliography{main,references,robustHH}
\documentclass{MITstyle}

%\usepackage[table]{xcolor}
\usepackage{chngcntr}
\usepackage{hyperref}
\usepackage{microtype}

\title{A Lightweight and Extensible Cell Segmentation and Classification Model for Whole Slide Images}

\author{Nikita Shvetsov~$^{1, }$\footnote{Correspondence e-mail: nikita.shvetsov@uit.no}, Thomas K. Kilvaer~$^{2, 3}$, Masoud Tafavvoghi~$^{4}$, Anders Sildnes~$^{1}$, \\ Kajsa Møllersen~$^{4}$, Lill-Tove Rasmussen Busund~$^{5, 6}$, Lars Ailo Bongo~$^{1}$ \\
%
\vspace{1em} % Space between authors and afilliations
%
\normalfont{\small $^{1}$Department of Computer Science, UiT The Arctic University of Norway}\\
\normalfont{\small $^{2}$Department of Oncology, University Hospital of North Norway}\\
\normalfont{\small $^{3}$Department of Clinical Medicine, UiT The Arctic University of Norway}\\
\normalfont{\small $^{4}$Department of Community Medicine, UiT The Arctic University of Norway}\\
\normalfont{\small $^{5}$Department of Medical Biology, UiT The Arctic University of Norway} \\
\normalfont{\small $^{6}$Department of Clinical Pathology, University Hospital of North Norway} %\vspace{2em}
}

\begin{document}
\maketitle

\section*{Abstract}

% \begin{abstract}
% Developing clinically useful cell-level analysis tools in digital pathology remains challenging due to limitations in dataset granularity, inconsistent annotations, computational demands of advanced models, and difficulties in integrating new technologies into clinical workflows. To address these challenges, we propose a multi-faceted solution that enhances data quality, model performance, and usability to create a lightweight and extensible cell segmentation and classification model.

% First, we update data labels by employing a cross-relabeling process that refines the labels of two existing datasets, PanNuke and MoNuSAC, to create a new unified dataset with enhanced granularity, encompassing seven distinct cell types. Second, we leverage the H-Optimus foundation model as a fixed encoder to improve feature representation for simultaneous cell segmentation and classification tasks. Third, to address the computational demands of foundation models, we employ knowledge distillation to reduce model size and complexity while maintaining comparable performance. Finally, to facilitate integration into clinical workflows, we integrate the distilled model into the QuPath software, a widely used open-source platform in digital pathology.

% Our results demonstrate improvements in cell segmentation and classification performance using the H‑Optimus-based model compared to a CNN-based model. Specifically, the average $R^2$ improved from 0.575 to 0.871, and the average $PQ$ score improved from 0.450 to 0.492, indicating better alignment with actual cell counts and enhanced segmentation and classification quality. Furthermore, the distilled student model maintains performance comparable to the larger foundation model while reducing the parameter count by a factor of 48.
% Overall, by reducing computational complexity and integrating it into existing workflows, the proposed approach may significantly impact diagnostic processes, reduce the workload of pathologists, and contribute to improved patient outcomes. Though our approach shows potential enhancements in efficiency and usability of cell segmentation and classification models in digital pathology, extensive validation is needed to deploy these models in clinical practice.
% \end{abstract}

%%% shortened abstract
\begin{abstract}
Developing clinically useful cell-level analysis tools in digital pathology remains challenging due to limitations in dataset granularity, inconsistent annotations, high computational demands, and difficulties integrating new technologies into workflows. To address these issues, we propose a solution that enhances data quality, model performance, and usability by creating a lightweight, extensible cell segmentation and classification model. 

First, we update data labels through cross-relabeling to refine annotations of PanNuke and MoNuSAC, producing a unified dataset with seven distinct cell types. Second, we leverage the H-Optimus foundation model as a fixed encoder to improve feature representation for simultaneous segmentation and classification tasks. Third, to address foundation models' computational demands, we distill knowledge to reduce model size and complexity while maintaining comparable performance. Finally, we integrate the distilled model into QuPath, a widely used open-source digital pathology platform. 

Results demonstrate improved segmentation and classification performance using the H-Optimus-based model compared to a CNN-based model. Specifically, average $R^2$ improved from 0.575 to 0.871, and average $PQ$ score improved from 0.450 to 0.492, indicating better alignment with actual cell counts and enhanced segmentation quality. The distilled model maintains comparable performance while reducing parameter count by a factor of 48. By reducing computational complexity and integrating into workflows, this approach may significantly impact diagnostics, reduce pathologist workload, and improve outcomes. Although the method shows promise, extensive validation is necessary prior to clinical deployment.
\end{abstract}
\clearpage

\section{Introduction}
In digital pathology, accurate segmentation and classification of cells are crucial for many diagnostic, prognostic, and predictive analyses \cite{Jaber_Beziaeva_etal._2019,Lin_Pan_etal._2022,Park_Ock_etal._2022,Shen_Choi_etal._2024}. Nowadays, developments in computational pathology offer multiple solutions \cite{H._Qu_P._Wu_etal._2020,Javed_Mahmood_etal._2020} to utilize cell-level datasets to train machine learning models that solve these problems. The quality and specificity of training datasets are critical for robust and accurate models. Adhering to the principle of "garbage in, garbage out", it is essential to ensure that these datasets are extensively and accurately labeled with distinct classes that reflect the diverse biological characteristics of different cell types. Unfortunately, the number of open-source datasets comprising such high-quality annotations is limited. Existing cell segmentation datasets \cite{Gamper_Koohbanani_etal._2019,Graham_Vu_etal._2019,Verma_Kumar_etal._2021} may offer extensive annotations for certain cell types while providing more general labels for others. For example, in PanNuke, which is one of the largest open-source datasets comprising labeled cells, various types of morphologically and functionally different inflammatory cells like macrophages and lymphocytes are clustered in a broad "inflammatory" class. Consequently, these classes are frequently omitted from analyses or aggregated into broader meta-classes \cite{Gamper_Koohbanani_etal._2020} and likely interfere with other cell classes included in the dataset. This and similar inconsistencies in annotation granularity limit the ability of machine learning models to learn the comprehensive and nuanced features necessary for accurate cell segmentation and classification. To address these challenges, methods for refining and standardizing dataset annotations are essential to enhance the quality of training data.

A complementary approach to mitigate the absence of high-quality training data is the use of foundation models. Foundation models as encoders are defined as large-scale, versatile networks pre-trained on vast, diverse datasets using self-supervised learning, contrasting with convolutional neural network (CNN) pre-trained encoders that rely on supervised learning with labeled data. In practice, foundation models leverage enormous amounts of weakly or unlabeled data from millions of whole slide images (WSIs) and employ self-attention mechanisms to capture long-range dependencies and global context \cite{Chen_Ding_etal._2024,Saillard_Jenatton_etal._2024,Vorontsov_Bozkurt_etal._2024,Xu_Usuyama_etal._2024}. As a consequence, foundation models are able to produce transferable feature representations across different cell types and tissue environments. The feature representations can be leveraged by decoder networks to produce segmentation masks and pixel-level classifications. Because foundation models have comprehensive feature representations, they can be effectively fine-tuned using much smaller amounts of cell-level data compared to the large datasets needed to train models from scratch. Furthermore, foundation models incorporate adversarial training elements or contrastive learning \cite{Chen_Ding_etal._2024,Xu_Usuyama_etal._2024}, enhancing their resilience and adaptability by exposing them to challenging and varied scenarios during training. This may result in more generalizable models, often making them well-suited for diverse and complex tasks in digital pathology.

Despite the inherent advantages of foundation models, their deployment for practical use faces its own obstacles. In particular, they require substantial computational power, financial investments and rigorous testing to ensure reliability and efficacy for a given task \cite{Akkus_Dangott_etal._2022,Dragomir_Cocuz_etal._2022,Go_2022,Jafri_Farooqui_etal._2024}. Moreover, while foundation models enhance feature representation and performance, they depend on the quality of available annotations for decoder fine-tuning and, like any other model, cannot resolve existing inconsistencies or ambiguities in data labels. Therefore, there remains a critical need for solutions that address both data quality and practical deployment considerations.
Further, integrating new technologies into existing clinical workflows often encounters resistance, as it necessitates adjustments to established diagnostic processes. So, there is a need to develop solutions that could be integrated into current practices, minimizing the burden on medical professionals to adopt new tools \cite{King_Williams_etal._2023}.

Existing solutions \cite{Goldsborough_Philps_etal._2024,Hörst_Rempe_etal._2024}, while addressing some aspects of these challenges, fall short in providing a comprehensive approach. To address the data quality and clinical deployment issues, we propose a multi-faceted solution that encompasses data refinement, model optimization, and integration with existing pathology tools (\hyperref[fig:fig1]{Figure 1}). The outcome is a lightweight cell segmentation and classification model that can be integrated into digital pathology workflows for practical clinical use.

\begin{figure}[h!]
    \centering
    \includegraphics[width=\textwidth, height=0.82\textheight, keepaspectratio]{images/Figure_1.pdf}
    \caption{Overview of the proposed solution, including 1) Data refinement using cross-relabeling, 2) Teacher model development and fine tuning, 3) Student model optimization with knowledge distillation and 4) Student model and QuPath integration}
    \label{fig:fig1}
\end{figure}
\clearpage

Our approach begins with preparing the data for the fine-tuning and training of the machine learning models. We create a refined dataset, acquired via cross-relabeling two cell-level datasets, enhancing annotation specificity and consistency of the labeled data. Subsequently, we create a cell segmentation and classification model based on the foundation model. We leverage the foundation model as a fixed encoder and fine-tune a decoder using the refined dataset to improve generalization across diverse tissue- and cell types.
To ensure that the model remains lightweight and deployable in a possibly resource-constrained environment, we employ knowledge distillation to approximate the functionality of the foundation model. Finally, to facilitate the practical application of our model in digital pathology workflows, we integrate it with the QuPath \cite{Bankhead_Loughrey_etal._2017} application. Each methodological component contributes to the overarching goal of enhancing model performance, generalizability, and usability in clinical settings.

The primary contributions of this paper are:
\begin{enumerate}
    \item \textit{Data labels refinement through cross-relabeling:}
    
    We propose a new method for refining labels of cell-level datasets through cross-relabeling. This method employs classification models to re-label broad and ambiguous instances, resulting in a more diverse dataset. Our evaluation demonstrates that these classification models achieve high accuracy on test subsets, indicating the reliability of the method for label refinement.

    \item \textit{Enhanced model performance via foundation models:}
    
    We employ a foundation model as a feature extractor for the cell segmentation and classification task. In comparison with training a CNN model from scratch, the foundation model backbone only needs fine-tuning, which significantly reduces training time, computational resources and data requirements. We show that using a foundation model encoder leads to better performance in cell segmentation and classification networks than using a CNN-based encoder. This improvement may enable the model to generalize more effectively across various tissue types and imaging methods.
    
    \item \textit{Model optimization through knowledge distillation:}
    
    We show that a smaller student model trained using knowledge distillation on the refined dataset obtained via our cross-relabeling approach from a foundation model achieves comparable performance in cell segmentation and quantification tasks. As a result, this model is more suitable for deployment in environments without high-performance computing resources.
    
    \item \textit{Integration with QuPath:}
    
    We integrate the distilled cell segmentation and classification model into QuPath, a widely used open-source digital pathology platform, to accelerate clinical adaptation by enabling pathologists to more easily incorporate advanced computational tools into their existing workflows.
\end{enumerate}

Through these methodological steps, we aim to bridge the gap between advanced machine learning techniques and practical clinical applications, making accurate and efficient digital pathology accessible in a broader range of healthcare settings.

\section{Refining Existing Datasets Using Cross-Relabeling}
To address the limitations of sparse and ambiguous labeling of cell-level datasets, we propose a generalizable cross-relabeling strategy that can be applied to any dataset containing broadly categorized or imprecisely labeled cell types. This approach involves training and subsequently leveraging classification models to refine broad categories into more specific or biologically relevant classes.
When applied to cell-level data, the methodology includes extracting individual cell images from the dataset patches, preprocessing these images to standardize the size and accommodate partial cells, and then training deep learning classifiers capable of distinguishing between the finer cell subtypes within the coarser categories. 
To illustrate our approach, we focus on the PanNuke \cite{Gamper_Koohbanani_etal._2020, Gamper_Koohbanani_etal._2019} and MoNuSAC \cite{Verma_Kumar_etal._2021} datasets that we have used to train models for cell quantification in our previous works \cite{Shvetsov_Grønnesby_etal._2022,Shvetsov_Sildnes_etal._2024}. We find that for better cell differentiation we have to introduce more granular labels. PanNuke includes a broad classification of "inflammatory" cells, encompassing lymphocytes, macrophages, and neutrophils. Each cell type differs significantly in structure, function, and clinical relevance. Conversely, MoNuSAC uses the label "epithelial" for a class that comprises both benign epithelial cells and malignant neoplastic cells. This practice makes it challenging to differentiate between benign and malignant epithelial cells in the dataset, which is a critical distinction when identifying tumor areas within tissue samples. To address these issues, we implement a cross-relabeling strategy as shown in \hyperref[fig:fig2]{Figure 2}. The key components are two classification models: one is trained on singular cell images from PanNuke data to classify the epithelial meta-class into epithelial and neoplastic classes. The other is trained on MoNuSAC to refine the inflammatory class into lymphocytes, neutrophils, and macrophages.

\begin{figure}[h!]
    \centering
    \includegraphics[width=\textwidth]{images/Figure_2.pdf}
    \caption{Refined dataset generation via cross relabeling}
    \label{fig:fig2}
\end{figure}

The refining approach consists of three consecutive steps. The first is the preprocessing step, in which we extract individual cells from both datasets (\hyperref[fig:fig3]{Figure 3}). The specifics of PanNuke and MoNuSAC patch preparation before cell preprocessing are provided in \hyperref[chap:S1]{Appendix S1}.

\begin{figure}[h!]
    \centering
    \includegraphics[width=\textwidth]{images/Figure_3.pdf}
    \caption{Cell instances preprocessing including (1) cell map extraction, (2) bounding box delineation, (3) adjusting cell boxes and (4) cropping and resizing of cell images}
    \label{fig:fig3}
\end{figure}

During preprocessing, we extract cell type maps from the ground truth label mask and calculate bounding boxes around each cell instance. To accommodate partial cells at patch borders, a common issue in cropped patch images, we employ mirror padding and extend the field of view of the cell label by 15 pixels to capture adjacent cells. We then crop and resize the identified regions to $64 \times 64$ pixels using bicubic interpolation.

The preprocessed PanNuke dataset comprises 68,031 neoplastic and 23,207 epithelial cell images, while MoNuSAC comprises  33,104 lymphocytes, 1,252 neutrophils, and 1,695 macrophages, which we subsequently use in training cell classification models and classifying the cell image data \hyperref[fig:S2]{Appendix Figure S2 (1)}. 

The next step is to train two distinct ResNet50-based classifiers tailored to address the specific labeling challenges inherent in each dataset. We use ResNet50 for classification models due to its proven effectiveness for image classification tasks in histopathology \cite{pan2022reviewmachinelearningapproaches}, and its compatibility with small images. For the PanNuke dataset, we design the classifier, trained on MoNuSAC data, to disaggregate the heterogeneous "inflammatory" cell category into distinct subtypes: lymphocytes, macrophages, and neutrophils. Similarly, for the MoNuSAC dataset, the classifier is trained on PanNuke data and distinguishes between benign and malignant epithelial cells within the overarching "epithelial" label. By applying these targeted classifiers to their respective datasets, we assign more specific labels to individual cell instances, thus enabling us to create a unified dataset.
To ensure a balanced representation of classes, we train both models on datasets that had been equalized to match the size of the least represented class. Thus, we obtain datasets comprising 23,207 samples per class for PanNuke and 1,252 samples per class for MoNuSAC data. Next, we partition both of them into training (70\%), validation (20\%), and testing (10\%) subsets. To mitigate the risk of overfitting, we use a single dropout layer with a rate of p=0.5 in both models and data augmentation using randomized color perturbations, rotation, and horizontal and vertical flipping. We employ AdamW optimizer and the cross-entropy loss function for the training criterion.

To evaluate the two trained models, we measure the classification accuracy on the respective test subsets. The accuracies on the test subset for both classifiers are presented in \hyperref[tab:1]{Table 1}. The PanNuke model achieves an average accuracy of 93.57\%, with higher accuracy for neoplastic cells (96.06\%) compared to epithelial cells (86.26\%). The confusion matrix in Figure A3.1 shows that the model predominantly distinguishes accurately between epithelial and neoplastic tissues, with a substantial number of correct classifications and relatively few misclassifications. The MoNuSAC model demonstrates an average accuracy of 98.92\%, excelling in classifying lymphocytes (99.67\%) and macrophages (94.12\%), with lower performance for neutrophils (85.71\%). The confusion matrix in Figure A3.2 shows that the model identifies lymphocytes and performs reasonably well with macrophages and neutrophils.

\begin{table}[h!]
\renewcommand{\arraystretch}{1.5}
  \centering
  \caption{Cell classification results for PanNuke and MoNuSAC trained models (CI 95\%).}
  \label{tab:1}
  \begin{tabular}{|l|c|c|}
   \hline
   %\rowcolor{gray!30}
    Accuracy               & PanNuke model              & MoNuSAC model              \\
    \hline
    Average      & 0.936 (0.931--0.941)         & 0.989 (0.986--0.993)        \\
    \hline
    Neoplastic   & 0.961 (0.956--0.965)         & -                          \\
    \hline
    Epithelial   & 0.863 (0.849--0.877)         & -                          \\
    \hline
    Lymphocytes  & -                          & 0.997 (0.995--0.999)        \\
    \hline
    Neutrophils  & -                          & 0.857 (0.796--0.918)        \\
    \hline
    Macrophages  & -                          & 0.941 (0.906--0.976)        \\
    \hline
  \end{tabular}
\end{table}

Finally, during the last step, we use the model trained on PanNuke data for epithelial cells in MoNuSAC and the model trained on MoNuSAC for the inflammatory cells class in PanNuke. Specifically, we use classifier models to relabel epithelial cells in MoNuSAC and inflammatory cells in PanNuke data. Then we combine cells with refined labels and the rest of the cells in both datasets to create a refined dataset (\hyperref[fig:S2]{Appendix Figure S2 (2)}). The process of relabeling cells and visualizing them on a patch is shown in \hyperref[fig:fig4]{Figure 4}. The cell counts in the refined dataset are provided in \hyperref[tab:S4]{Appendix Table S4}.

\begin{figure}[h!]
    \centering
    \includegraphics[width=\textwidth, height=0.42\textheight, keepaspectratio]{images/Figure_4.pdf}
    \caption{Cell relabeling procedure for epithelial and inflammatory cell classes}
    \label{fig:fig4}
\end{figure}

%\hfill

Relabeling and combining datasets have been explored in a prior study \cite{Parulekar_Kanwat_etal._2023}, where consecutive fine-tuning on multiple datasets was employed to account for hierarchical class label structures. While the method presented in \cite{Parulekar_Kanwat_etal._2023} is intuitive, it often lacks consistency and requires multiple fine-tuning runs, which can be cumbersome and time-consuming. 
In contrast, cross-relabeling simplifies this process by using specialized classification models tailored to each dataset's specific labeling challenges. This approach provides better transparency and produces a unified dataset encompassing seven distinct cell types across multiple tissue samples, enhancing data diversity for further model training or fine-tuning.

Despite these improvements, cross-relabeling does not entirely resolve issues related to poor labeling quality or the amount of labeled data. Specifically, our results show lower accuracies persist for underrepresented classes, such as macrophages, which may stem from a limited sample availability and intrinsic challenges in distinguishing these cells based solely on H\&E staining. Furthermore, while our method enhances label specificity, it relies on the initial quality of the broad labels; thus, any fundamental inaccuracies in the original annotations can propagate through the relabeling process. Addressing the overall problem of limited data labels may require integrating additional data sources or utilizing complementary immunohistochemical staining methods.
Although the reported performance metrics are obtained from evaluations on the native test sets of each dataset, it is important to note that the primary application of these classifiers is to perform cross-relabeling, where a model trained on one dataset (e.g., PanNuke) is applied to another (e.g., MoNuSAC) and vice versa. We acknowledge that a more systematic evaluation of cross-dataset generalization is needed and could be performed in future work.

Overall, the refined dataset produced by our approach can enhance the supervised training or fine-tuning of cell segmentation and classification models, especially those that utilize pre-trained foundation models to improve feature extraction robustness. In addition, these models can detect nuanced classes that enable researchers to conduct more detailed analyses of biological processes in computational pathology.

\section{Foundation models for robust cell segmentation and classification}

Accurate cell segmentation and classification in digital pathology are hindered by limited labeled data and the fact that conventional CNNs are unable to capture global contextual information due to their local receptive field constraints \cite{Gheflati_Rivaz_2022,Yang_Marcus_etal.}. Traditional approaches in cell quantification have predominantly relied on CNN encoders, such as ResNet50, given their proven effectiveness in semantic segmentation tasks \cite{Deshmane_2023,Graham_Vu_etal._2019,Mukasheva_Koishiyeva_etal._2024,Stringer_Wang_etal._2021}. However, approaches that include fine-tuning of pretrained CNNs, data augmentation, and stain normalization to partially increase data variability and address staining differences often fail to achieve the necessary generalization and robustness across diverse tissue types and staining conditions \cite{G._Wang_W._Li_etal._2018,Gao_Bagci_etal._2018,Karim_El_Khoury_Martin_Fockedey_etal._2021}.

To overcome these challenges, we leverage an encoder-decoder network that uses a foundation model as the encoder and a CNN upsampling decoder (\hyperref[fig:fig5]{Figure 5}) for simultaneous cell segmentation and classification in 2D patches extracted from WSIs. Foundation models with transformer-based architectures are viable alternatives to CNN-based encoders \cite{Shamshad_Khan_etal._2023,Sourget_2023}. They enable the creation of more advanced architectures that can decode or transform learned features more effectively \cite{Chen_Duan_etal._2023,Cheng_Misra_etal._2022,Xie_Wang_etal._2021}.

\begin{figure}[h!]
    \centering
    \includegraphics[width=\textwidth]{images/Figure_5.pdf}
    \caption{UNETR-like model with foundational model as backbone}
    \label{fig:fig5}
\end{figure}

By utilizing a transformer-based encoder, we incorporate global contextual information into the feature extraction process, which is a key advantage of such architectures \cite{Chen_Lu_etal._2021}. This foundation model integration facilitates accurate pixel-wise segmentation and classification without the need for extensive encoder training, thereby potentially improving generalization across varied cellular structures and tissue types.
In our implementation, we employ a modified UNETR \cite{Hatamizadeh_Tang_etal._2021} architecture that combines a vision transformer (ViT) \cite{Dosovitskiy_Beyer_etal._2021} encoder with a CNN-based decoder. The encoder utilizes the pretrained H-Optimus foundation model, which contains 1.1 billion parameters and is trained on over 500,000 H\&E stained WSIs \cite{Saillard_Jenatton_etal._2024}. We extract outputs from four evenly spaced transformer blocks $Z_i$, where $i \in [1, 14, 26, 38]$, to serve as residual connections for the CNN decoder. We select these blocks based on our observation that features from non-adjacent levels of the encoder lead to better overall performance on the test subset.

The CNN decoder upsamples the feature representations, acquired from the transformer blocks, to generate an intermediate vector that is handled by two task-specific layers that generate cell segmentation and classification masks. The first task-specific layer is the ‘Cellpose head’,  which is used to delineate cell instances. The layer generates horizontal and vertical gradient maps to form vector fields that are refined through gradient tracking in a post-processing step using the Cellpose algorithm \cite{Stringer_Wang_etal._2021}, known for its efficacy in cell segmentation tasks and generalizability across multiple domains \cite{Pachitariu_Stringer_2022,Stringer_Pachitariu_2024}. The second task-specific layer is the "Cell type head", which assigns labels to individual pixels. In the post-processing step, we determine the output classification label of each segmented cell instance by majority voting over the labeled pixels that comprise the cell in the segmentation map.

To evaluate model performance and measure the impact of adding a foundation model as backbone, we compare it to a ResNet50-based model. ResNet50 is a widely used solution for encoders in segmentation architectures in the medical domain \cite{Deshmane_2023,Graham_Vu_etal._2019,Mukasheva_Koishiyeva_etal._2024,Stringer_Wang_etal._2021}. For the H-Optimus-based model, we utilize frozen weights for the encoder and only fine-tune the decoder to take advantage of the extensive pre-training of the foundation model. For the ResNet50-based model we start with ImageNet \cite{Deng_Dong_etal.} weights and train both encoder and decoder parts. Hyperparameters for the training step are set to be identical, where possible, for comparable evaluation. 
For this evaluation, we deliberately use the PanNuke dataset to provide a standardized and controlled comparison between the H‑Optimus and ResNet50-based models (\hyperref[fig:S2]{Appendix Figure S2 (3)}). Specifically, we use two of the default PanNuke dataset splits (66\%) for training and validation, and reserve the third split (33\%) for testing.

To address the challenge of cell class imbalance in the PanNuke dataset, which is a common characteristic in most cell-level H\&E patch datasets, both models’ training processes employ a weighted loss function comprising cross-entropy and focal loss \cite{Lin_Goyal_etal._2018}. The focal loss component is adjusted with coefficients derived from each cell class' instance frequency, emphasizing learning from underrepresented classes and enhancing the model's sensitivity to rare but significant cellular patterns. The cross-entropy loss is augmented with spectral decoupling regularization \cite{Pezeshki_Kaba_etal._2021,Pohjonen_Stürenberg_etal._2022} and spatially varying label smoothing \cite{Islam_Glocker_2021}, which potentially stabilizes training and improves generalization in case of complex tissue morphologies. For optimization, we employ the \textit{AdamW} \cite{Loshchilov_Hutter_2019} to counter unbalanced class scenarios, with cosine annealing learning rate scheduler.

We utilize the scikit-learn library \cite{Van_der_Walt_Schönberger_etal._2014} and HoVer-Net \cite{Graham_Vu_etal._2019} implementations of $R^2$ (the coefficient of determination) and $PQ$ (panoptic quality) to evaluate our experiments. Complete mathematical formulations and detailed explanations of these metrics are provided in \hyperref[chap:S5]{Appendix S5}. To compute confidence intervals, we use nonparametric bootstrapping, where after calculating the metric on the full sample, we generated 1000 bootstrap replicates by resampling with replacement and then determined the 95\% confidence intervals as the 2.5th and 97.5th percentiles of the resulting empirical distribution.

%\hfill

The model comparisons are summarized in \hyperref[tab:2]{Table 2}. The H‑Optimus-based model achieves higher $R^2$ across all cell classes compared to the ResNet50-based model, which means that its predictions are more closely aligned with the PanNuke cell counts, indicating a stronger correlation with the observed data. Notably, the improvement of $R^2_{dead}$ may be an indicator of better global contextual representations provided by the foundation model backbone. In terms of segmentation and classification quality combined, measured by the PQ score, the H‑Optimus-based model demonstrates notable improvements across most cell classes. Overall, the average $R^2$ improved from 0.575 to 0.871, while the average $PQ$ score improved from 0.450 to 0.492, demonstrating better performance of the H-Optimus-based model.

\begin{table}[h!]
\renewcommand{\arraystretch}{1.5}
  \centering
  \caption{Cell quantification metrics for baseline and proposed models (CI 95\%).}
  \label{tab:2}
  \begin{tabular}{|l|c|c|}
    \hline
    %\rowcolor{gray!30}
    Metric             & Resnet50-based            & H-optimus-based              \\
    \hline
    $R^2_{neoplastic}$    & 0.681 (0.576--0.769)       & \textbf{0.941 (0.917--0.960)} \\
    \hline
    $R^2_{inflammatory}$  & 0.863 (0.778--0.903)       & \textbf{0.949 (0.918--0.966)} \\
    \hline
    $R^2_{connective}$    & 0.600 (0.488--0.698)       & 0.609 (0.436--0.772)          \\
    \hline
    $R^2_{dead}$          & 0.097 (-11.389--0.669)     & 0.925 (0.404--0.982)          \\
    \hline
    $R^2_{epithelial}$    & 0.635 (0.490--0.747)       & \textbf{0.930 (0.886--0.964)} \\
    \hline
    $PQ_{neoplastic}$       & 0.517 (0.499--0.535)       & \textbf{0.589 (0.575--0.604)} \\
    \hline
    $PQ_{inflammatory}$     & 0.455 (0.429--0.482)       & \textbf{0.528 (0.507--0.549)} \\
    \hline
    $PQ_{connective}$       & 0.416 (0.400--0.431)       & \textbf{0.451 (0.436--0.465)} \\
    \hline
    $PQ_{dead}$             & 0.374 (0.342--0.408)       & 0.292 (0.209--0.365)          \\
    \hline
    $PQ_{epithelial}$       & 0.488 (0.460--0.519)       & \textbf{0.599 (0.579--0.618)} \\
    \hline
  \end{tabular}
\end{table}

Our results  show that integrating the H‑Optimus foundation model within the UNETR architecture enhances the model's ability to segment and classify cells across diverse tissues from PanNuke data. The pretrained transformer encoder provides robust feature representations, resulting in higher average $R^2$ and $PQ$ scores compared to the CNN-based model. This leads to more reliable cell quantification and more accurate downstream analysis. Additionally, the streamlined fine-tuning process reduces computational overhead and training time, making the model more adaptable for new data.

Despite these advancements, the foundation model-based approach does not fully resolve all challenges related to cell segmentation and classification. We observe lower metric scores for underrepresented classes in the training data. Furthermore, foundation models typically encompass billions of parameters, resulting in substantial computational and memory requirements. It therefore poses challenges for deployment in resource-constrained environments, limiting their practical applicability in certain clinical settings.

\section{Model optimization via Knowledge Distillation}

To address the limitations posed by the extensive size of foundation models, we implement knowledge distillation — a model compression technique that leverages the teacher-student paradigm \cite{Hinton_Vinyals_etal._2015}. By training a smaller, more efficient student model to replicate the output of a larger, pre-trained teacher model, we retain performance while significantly reducing the model's complexity and resource requirements (\hyperref[fig:fig6]{Figure 6}).

\begin{figure}[h!]
    \centering
    \includegraphics[width=\textwidth, height=0.45\textheight, keepaspectratio]{images/Figure_6.pdf}
    \caption{Knowledge distillation framework for training a student model using a pre-trained teacher}
    \label{fig:fig6}
\end{figure}

We employ knowledge distillation to compress the H‑Optimus-based teacher model into a more efficient student model. The teacher model is the modified UNETR architecture with the H‑Optimus foundation model described in the previous chapter. The student model is based on a UNet architecture augmented with residual connections and incorporates a smaller ViT encoder with 9 million parameters \cite{Steiner_Kolesnikov_etal._2022,Wightman_2019}. 

First, we fine-tune the teacher model using the refined dataset from the cross-relabeling procedure (Section 2). Initially we train the decoder of the teacher model while keeping the encoder weights frozen. We split the refined dataset into train (70\%), validation (20\%) and test (10\%) subsets (\hyperref[fig:S2]{Appendix Figure S2 (4)}). During fine-tuning, we use the train and validation subsets, while leaving the test subset for model evaluation. We set the training procedure and model hyperparameters to be identical to those that were used to demonstrate the utility of foundation models for the simultaneous cell segmentation and classification task.

Next, we perform knowledge distillation from teacher to student using the refined dataset used to fine-tune the teacher model. The student model is trained to replicate the teacher model's outputs. We utilize a specialized loss function that aligns the student's predicted probability distribution with the teacher's, incorporating the teacher's class probability distribution derived from the output. Following the methodology of Hinton et al. \cite{Hinton_Vinyals_etal._2015}, we experiment with various hyperparameter settings for the temperature ($T$) and the balancing coefficients ($\alpha$ and $\beta$) in the loss function. We vary $T$ from 1 to 20 and adjust $\alpha$ and $\beta$ to balance the distillation and student losses. Through iterative tuning and evaluation, we identify that setting $T=14$, $\alpha=0.3$, and $\beta=0.7$ yields a configuration that converges and closely approximates the teacher model's performance during training.

Finally, we assess the performance of both models using the $R^2$ and $PQ$ (defined in \hyperref[chap:S5]{Appendix S5}) on the test set of the refined dataset (\hyperref[tab:3]{Table 3}). We observe that the 95\% confidence intervals overlap for most cell types, so we cannot claim statistically significant performance differences between the teacher and student models. One exception appears in the neoplastic class. The teacher model produces an $R^2$ of 0.919, while the student model shows an $R^2$ of 0.852. In addition, the student model achieves higher $PQ$ values for the neoplastic and connective classes, though the confidence intervals show overlap.

\begin{table}[h!]
\renewcommand{\arraystretch}{1.5}
  \centering
  \caption{Cell quantification metrics for teacher and distilled student models (CI 95\%).}
  \label{tab:3}
  \begin{tabular}{|l|c|c|}
    \hline
    %\rowcolor{gray!30}
    Metric & Teacher & Student \\
    \hline
    $R^2_{neoplastic}$    & \textbf{0.919} (0.898--0.939) & 0.852 (0.800--0.891) \\
    \hline
    $R^2_{lymphocyte}$    & 0.969 (0.956--0.977)         & 0.969 (0.956--0.978) \\
    \hline
    $R^2_{connective}$    & 0.694 (0.548--0.809)         & 0.618 (0.469--0.741) \\
    \hline
    $R^2_{dead}$          & 0.755 (0.400--0.908)         & 0.424 (0.100--0.731) \\
    \hline
    $R^2_{epithelial}$    & 0.922 (0.870--0.958)         & 0.843 (0.738--0.917) \\
    \hline
    $R^2_{macrophage}$    & 0.384 (-0.369--0.724)        & 0.704 (0.352--0.859) \\
    \hline
    $R^2_{neutrofil}$     & 0.854 (0.578--0.929)         & 0.833 (0.502--0.925) \\
    \hline
    $PQ_{neoplastic}$       & 0.581 (0.569--0.593)         & 0.601 (0.588--0.613) \\
    \hline
    $PQ_{lymphocyte}$       & 0.536 (0.520--0.553)         & 0.563 (0.544--0.579) \\
    \hline
    $PQ_{connective}$       & 0.436 (0.421--0.451)         & 0.457 (0.441--0.474) \\
    \hline
    $PQ_{dead}$             & 0.272 (0.235--0.315)         & 0.279 (0.201--0.369) \\
    \hline
    $PQ_{epithelial}$       & 0.522 (0.500--0.545)         & 0.530 (0.506--0.555) \\
    \hline
    $PQ_{macrophage}$       & 0.524 (0.459--0.588)         & 0.474 (0.405--0.543) \\
    \hline
    $PQ_{neutrofil}$        & 0.541 (0.490--0.592)         & 0.565 (0.522--0.607) \\
    \hline
  \end{tabular}
\end{table}


We further decompose the $PQ$ metric into its $SQ$ and $DQ$ components (\hyperref[tab:S6]{Appendix Table S6}). Both models produce nearly identical $SQ$ values, which indicates that they predict instance boundaries with similar precision. Although the student model shows some improvement in $DQ$ scores for certain classes, the confidence intervals overlap and do not confirm a statistically significant difference.

We observe that the student and teacher models yield comparable detection performance despite the student model using a much smaller and simpler architecture. A model with fewer parameters reduces the risk of overfitting when training data are scarce relative to the model’s complexity \cite{Farias_Ludermir_etal._2022}. The knowledge distillation process also encourages the student model to focus on the most generalizable detection features learned from the teacher. These factors enable the student model to achieve similar detection performance across different cell types.

Additionally, considering the model sizes reported in \hyperref[tab:4]{Table 4}, the distilled model achieves a significant reduction compared to the teacher model, with a 48-fold decrease in parameter count and a 5.5-fold reduction in on-disk size. In inference mode, the teacher model requires 16 GB of VRAM for a batch size of 32, while the distilled model only needs 3 GB of VRAM for the same batch size. These reductions make the distilled model significantly more practical for fine-tuning and deployment in resource-constrained environments.

\begin{table}[h!]
\renewcommand{\arraystretch}{1.5}
  \centering
  \caption{Parameter counts and size of teacher and distilled model}
  \label{tab:4}
  \adjustbox{max width=\textwidth}{%
  \begin{tabular}{|l|c|c|c|}
    \hline
    %\rowcolor{gray!30}
    Metric & H-optimus-based (Teacher) & mobileViT-based (Student) & Magnitude of difference \\
    \hline
    Parameters count       & 1,158,917,906   & \textbf{24,093,393}   & \textbf{48x}  \\
    \hline
    Estimated Total Size (MB) & 87,912       & \textbf{15,935}    & \textbf{5.5x} \\
    \hline
  \end{tabular}%
}
\end{table}

%\hfill

With recent advancements in complex network architectures and the use of pretrained encoders to achieve state-of-the-art performance \cite{Baumann_Dislich_etal._2024,Hörst_Rempe_etal._2024} in cell segmentation and classification tasks, model size, computational complexity, and processing times have increased. This limits the scalability and accessibility of these models. As we demonstrate, this may be mitigated using knowledge distillation. Studies in the field of natural language processing have demonstrated the efficacy of knowledge distillation in retaining the capabilities of the teacher model while achieving significant reductions in size and complexity \cite{Huangpu_Gao_2024,Sun_Yu_etal.}. 

We demonstrate the feasibility of knowledge distillation in digital pathology, specifically for cell segmentation and classification tasks. Moreover, we achieve this performance while also significantly reducing the parameter count. In addressing the challenge of knowledge transfer, we found that distillation from a transformer-based model to a smaller transformer is more straightforward than attempting to map transformer features to CNN blocks. In our experiments, using a CNN-based network as a student results in worse cell quantification performance due to the structural constraints of CNN feature space dimensions. 

Although our primary approach relies on a transformer-based student model that performs well, it can be further optimized to incorporate advantages from CNN architectures. For example, employing alternative techniques such as using ViT adapters \cite{Chen_Duan_etal._2023} or $1 \times 1$ convolutions to adjust feature map sizes may be beneficial for harnessing CNN advantages like enhanced local feature extraction. Moreover, if additional performance improvements are desired, the process can be further enhanced by applying supplementary knowledge distillation techniques, such as self-distillation \cite{Zhang_Song_etal._2019} or online distillation \cite{Houyon_Cioppa_etal._2023}.

Despite these promising results, further validation on independent datasets is necessary to fully understand the model's limitations. Underrepresented classes may pose challenges when addressing complex cases. Pathologists need to validate these models to adopt them in clinical settings. While the distilled models are smaller and more deployable, a technological gap persists because pathologists traditionally rely on established methods for inspecting WSIs and diagnosing diseases. Addressing the complexities involved in deploying models for inference and supporting pathologists in adopting new tools is essential for integrating these models into clinical workflows.

\section{Model integration with QuPath}
Digital pathology tools with graphical user interfaces are essential for visualizing and analyzing WSIs. To make our student model useful in clinical pathology workflows, it needs to be integrated into a tool that enables inspecting regions, creating annotations, and providing quantitative analyses of biomarkers. Therefore, we integrate the trained student model from the previous chapter into the QuPath open‑source platform \cite{Bankhead_Loughrey_etal._2017}. QuPath provides the required annotation, visualization, and analysis tools to interpret complex histological data, including workflows for cell segmentation, classification, and quantification (\hyperref[fig:fig7]{Figure 7}). 

\begin{figure}[h!]
    \centering
    \includegraphics[width=\textwidth]{images/Figure_7.pdf}
    \caption{Visualization of model-generated cell quantification annotations (left) and the corresponding unannotated slide (right) in QuPath}
    \label{fig:fig7}
\end{figure}

To identify the regions in a WSI critical for prognosticating tumor development, such as specific tumor areas or border regions without overlapping healthy tissue, the pathologist uses QuPath to outline these regions. Then, the pathologist initiates a cell segmentation and classification script through the QuPath interface for the selected regions. The resulting annotations and quantified cell information are then directly overlaid onto the WSI in the QuPath interface. Additional design and implementation details are in \hyperref[chap:S7]{Appendix S7}. 

Two common approaches for integrating deep learning models into QuPath are Java‑based native QuPath extensions \cite{Goldsborough_Philps_etal._2024} and the execution of RESTful API requests to a model server coupled with handling the response via an extension, as demonstrated in the application of cell segmentation models applied to immunofluorescence images \cite{Sugawara_2023}. While the community is actively working on these integration strategies, there is currently no universal solution that fully addresses all integration and performance requirements.

Extensions may offer better integration with QuPath, allowing slightly improved performance and more widespread usage of the built-in QuPath models, but they lack the flexibility to customize models and modify their behavior. For example, the newest version of QuPath includes models such as StarDist \cite{Weigert_Schmidt} and InstanSeg \cite{Goldsborough_Philps_etal._2024} that can perform cell segmentation. Both models pose limitations when applied to simultaneous cell segmentation and classification. StarDist performs well only on convex, round shapes by design, whereas some neoplastic, inflammatory, and connective cells exhibit complex and non-convex shapes. InstanSeg provides only semantic segmentation without assigning classes to the segmented cells.

%\hfill

In contrast, our approach offers an alternative integration strategy. It utilizes the paquo library to directly interact with QuPath’s internal application programming interface from within Python. This enables data exchange and processing without the need for intermediate conversion steps and provides greater control over model customization, retraining, and the incorporation of custom processing steps.

The integration of our custom model with QuPath underscores its potential to significantly enhance the diagnostic process by reducing the time burden on pathologists and enabling them to focus on more complex interpretative tasks using familiar software. Leveraging a tool that is already well-established among pathologists increases the likelihood of its adoption into daily clinical workflows. The quantitative data generated through the automated workflow is critical for both clinical decision-making and research, facilitating more accurate biomarker analysis, enabling robust statistical evaluations, and supporting hypothesis generation and testing. Additionally, by streamlining cell segmentation and classification, the tool enhances the scalability and reproducibility of pathological assessments, ultimately contributing to improved diagnostic accuracy and patient outcomes.

\section{Conclusion and future work}

In this study, we address critical challenges in digital pathology and tackle the usability and deployment issues of the developed models in standard computing environments without the need for high-performance computing systems. Our multi-faceted approach encompasses data refinement through cross-relabeling, leveraging foundation models for robust cell segmentation and classification, optimizing model performance via knowledge distillation, and integrating the optimized model into the QuPath software for practical application. This approach is used to construct a capable, versatile, and adjustable model for cell segmentation and classification, with enhanced performance and usability.

\begin{sloppypar}
While our approach shows potential in the field of computational pathology, certain limitations persist. 
For example, our implementation currently exhibits lower performance in detecting macrophages. 
This serves as an instance of the broader challenge of accurately identifying complex cell types. In order to address this issue, extending our approach to incorporate additional data sources, exploring alternative modeling approaches, and integrating other imaging modalities such as immunohistochemical staining may help improve detection accuracy. Moreover, although the distilled model reduces computational demands, integrating advanced deep learning models into clinical practice requires addressing technological gaps and potential resistance to adopting new tools within established diagnostic processes.
\end{sloppypar}

Future work could focus on several key areas to refine the proposed approach and facilitate its adoption in clinical environments. Enhancing the cell-relabeling process with additional datasets \cite{Graham_Jahanifar_etal._2021} could improve the representation of underrepresented cell types and enhance overall model performance. Also, incorporating additional data sources, such as multi-modal imaging or complementary staining methods, may address limitations related to cell type differentiation and class imbalance. Exploring other foundation models \cite{Vorontsov_Bozkurt_etal._2024,Zimmermann_Vorontsov_etal._2024} or introducing additional modalities \cite{Ding_Wagner_etal._2024,Vaidya_Zhang_etal._2025} may provide alternative architectures better suited to specific tasks or offer improved efficiency. Implementing more complex knowledge distillation techniques \cite{Houyon_Cioppa_etal._2023,Zhang_Song_etal._2019} could further optimize the model's performance and adaptability. Additionally, deeper integration with QuPath or other digital pathology software could provide pathologists more control over cell quantification analysis directly within the QuPath interface, thereby increasing accessibility and usability. Such enhancements would not only refine model performance but also ensure greater adaptability and scalability within various clinical environments. Finally, extensive validation of the model by pathologists and benchmarking against independent datasets are essential steps toward establishing the model's reliability and fostering confidence in its clinical utility.

\section*{Acknowledgments} 
This work was funded in part by the Research Council of Norway grant no. 309439 SFI Visual Intelligence, and the North Norwegian Health Authority grant no. HNF1521-20.

\bibliographystyle{IEEEtran}
\begin{sloppypar}
\begin{thebibliography}{99}

\bibitem{chaplot2020neural} Chaplot, Devendra Singh, et al. "Neural topological slam for visual navigation." Proceedings of the IEEE/CVF conference on computer vision and pattern recognition. 2020.

\bibitem{maksymets2021thda} Maksymets, Oleksandr, et al. "Thda: Treasure hunt data augmentation for semantic navigation." Proceedings of the IEEE/CVF International Conference on Computer Vision. 2021.

\bibitem{mezghan2022memory} Mezghan, Lina, et al. "Memory-augmented reinforcement learning for image-goal navigation." 2022 IEEE/RSJ International Conference on Intelligent Robots and Systems (IROS). IEEE, 2022.

\bibitem{al2022zero} Al-Halah, Ziad, Santhosh Kumar Ramakrishnan, and Kristen Grauman. "Zero experience required: Plug \& play modular transfer learning for semantic visual navigation." Proceedings of the IEEE/CVF Conference on Computer Vision and Pattern Recognition. 2022.

\bibitem{ye2021auxiliary} Ye, Joel, et al. "Auxiliary tasks and exploration enable objectgoal navigation." Proceedings of the IEEE/CVF international conference on computer vision. 2021.

\bibitem{chaplot2020object} Chaplot, Devendra Singh, et al. "Object goal navigation using goal-oriented semantic exploration." Advances in Neural Information Processing Systems 33 (2020)

\bibitem{ramakrishnan2022poni} Ramakrishnan, Santhosh Kumar, et al. "Poni: Potential functions for objectgoal navigation with interaction-free learning." Proceedings of the IEEE/CVF Conference on Computer Vision and Pattern Recognition. 2022.

\bibitem{ramrakhya2022habitat} Ramrakhya, Ram, et al. "Habitat-web: Learning embodied object-search strategies from human demonstrations at scale." Proceedings of the IEEE/CVF Conference on Computer Vision and Pattern Recognition. 2022.

\bibitem{mousavian2019visual} Mousavian, Arsalan, et al. "Visual representations for semantic target driven navigation." 2019 International Conference on Robotics and Automation (ICRA). IEEE, 2019.

\bibitem{dhariwal2021diffusion} Dhariwal, Prafulla, and Alexander Nichol. "Diffusion models beat gans on image synthesis." Advances in neural information processing systems 34 (2021)

\bibitem{ho2022classifier} Ho, Jonathan, and Tim Salimans. "Classifier-free diffusion guidance." arXiv preprint arXiv:2207.12598 (2022).

\bibitem{nichol2021glide} Nichol, Alex, et al. "Glide: Towards photorealistic image generation and editing with text-guided diffusion models." arXiv preprint arXiv:2112.10741 (2021)

\bibitem{brooks2023instructpix2pix} Brooks, Tim, Aleksander Holynski, and Alexei A. Efros. "Instructpix2pix: Learning to follow image editing instructions." Proceedings of the IEEE/CVF Conference on Computer Vision and Pattern Recognition. 2023.

\bibitem{fu2023guiding} Fu, Tsu-Jui, et al. "Guiding instruction-based image editing via multimodal large language models." arXiv preprint arXiv:2309.17102 (2023).

\bibitem{geng2024instructdiffusion} Geng, Zigang, et al. "Instructdiffusion: A generalist modeling interface for vision tasks." Proceedings of the IEEE/CVF Conference on Computer Vision and Pattern Recognition. 2024.

\bibitem{zhou2024minedreamer} Zhou, Enshen, et al. "Minedreamer: Learning to follow instructions via chain-of-imagination for simulated-world control." arXiv preprint arXiv:2403.12037 (2024).

\bibitem{zhou2023esc} Zhou, Kaiwen, et al. "Esc: Exploration with soft commonsense constraints for zero-shot object navigation." International Conference on Machine Learning. PMLR, 2023.

\bibitem{yu2023l3mvn} Yu, Bangguo, Hamidreza Kasaei, and Ming Cao. "L3mvn: Leveraging large language models for visual target navigation." 2023 IEEE/RSJ International Conference on Intelligent Robots and Systems (IROS). IEEE, 2023.

\bibitem{gadre2023cows} Gadre, Samir Yitzhak, et al. "Cows on pasture: Baselines and benchmarks for language-driven zero-shot object navigation." Proceedings of the IEEE/CVF Conference on Computer Vision and Pattern Recognition. 2023.

\bibitem{shah2023navigation} Shah, Dhruv, et al. "Navigation with large language models: Semantic guesswork as a heuristic for planning." Conference on Robot Learning. PMLR, 2023.

\bibitem{cai2024bridging} Cai, Wenzhe, et al. "Bridging zero-shot object navigation and foundation models through pixel-guided navigation skill." 2024 IEEE International Conference on Robotics and Automation (ICRA). IEEE, 2024.

\bibitem{yu2023co} Yu, Bangguo, Hamidreza Kasaei, and Ming Cao. "Co-NavGPT: Multi-robot cooperative visual semantic navigation using large language models." arXiv preprint arXiv:2310.07937 (2023).

\bibitem{wu2024voronav} Wu, Pengying, et al. "Voronav: Voronoi-based zero-shot object navigation with large language model." arXiv preprint arXiv:2401.02695 (2024).

\bibitem{qin2023mp5} Qin, Yiran, et al. "Mp5: A multi-modal open-ended embodied system in minecraft via active perception." arXiv preprint arXiv:2312.07472 (2023).

\bibitem{du2024learning} Du, Yilun, et al. "Learning universal policies via text-guided video generation." Advances in Neural Information Processing Systems 36 (2024).

\bibitem{ajay2024compositional} Ajay, Anurag, et al. "Compositional foundation models for hierarchical planning." Advances in Neural Information Processing Systems 36 (2024).

\bibitem{liang2024skilldiffuser} Liang, Zhixuan, et al. "Skilldiffuser: Interpretable hierarchical planning via skill abstractions in diffusion-based task execution." Proceedings of the IEEE/CVF Conference on Computer Vision and Pattern Recognition. 2024.

\bibitem{heusel2017gans} Heusel, Martin, et al. "Gans trained by a two time-scale update rule converge to a local nash equilibrium." Advances in neural information processing systems 30 (2017).

\bibitem{zhang2018unreasonable} Zhang, Richard, et al. "The unreasonable effectiveness of deep features as a perceptual metric." Proceedings of the IEEE conference on computer vision and pattern recognition. 2018.

\bibitem{brown2020language} Brown, Tom B. "Language models are few-shot learners." arXiv preprint arXiv:2005.14165 (2020).

\bibitem{podell2023sdxl} Podell, Dustin, et al. "Sdxl: Improving latent diffusion models for high-resolution image synthesis." arXiv preprint arXiv:2307.01952 (2023).

\bibitem{brohan2022rt} Brohan, Anthony, et al. "Rt-1: Robotics transformer for real-world control at scale." arXiv preprint arXiv:2212.06817 (2022).

\bibitem{brohan2023rt} Brohan, Anthony, et al. "Rt-2: Vision-language-action models transfer web knowledge to robotic control." arXiv preprint arXiv:2307.15818 (2023).

\bibitem{li2024manipllm} Li, Xiaoqi, et al. "Manipllm: Embodied multimodal large language model for object-centric robotic manipulation." Proceedings of the IEEE/CVF Conference on Computer Vision and Pattern Recognition. 2024.

\bibitem{shah2023vint} Shah, Dhruv, et al. "ViNT: A foundation model for visual navigation." arXiv preprint arXiv:2306.14846 (2023).

\bibitem{liu2024visual} Liu, Haotian, et al. "Visual instruction tuning." Advances in neural information processing systems 36 (2024).

\bibitem{hu2021lora} Hu, Edward J., et al. "Lora: Low-rank adaptation of large language models." arXiv preprint arXiv:2106.09685 (2021).

\bibitem{qin2023supfusion} Qin, Yiran, et al. "SupFusion: Supervised LiDAR-camera fusion for 3D object detection." Proceedings of the IEEE/CVF International Conference on Computer Vision. 2023.

\bibitem{qin2024worldsimbench} Qin, Yiran, et al. "Worldsimbench: Towards video generation models as world simulators." arXiv preprint arXiv:2410.18072 (2024).

\bibitem{yu2025gamefactory} Yu, Jiwen, et al. "GameFactory: Creating New Games with Generative Interactive Videos." arXiv preprint arXiv:2501.08325 (2025).

\bibitem{zhou2024code} Zhou, Enshen, et al. "Code-as-Monitor: Constraint-aware Visual Programming for Reactive and Proactive Robotic Failure Detection." arXiv preprint arXiv:2412.04455 (2024).

\bibitem{zhang2024ad} Zhang, Zaibin, et al. "AD-H: Autonomous Driving with Hierarchical Agents." arXiv preprint arXiv:2406.03474 (2024).

\bibitem{wang2024toward} Wang, Chaoqun, et al. "Toward Accurate Camera-based 3D Object Detection via Cascade Depth Estimation and Calibration." arXiv preprint arXiv:2402.04883 (2024).

\bibitem{huang2024story3d} Huang, Yuzhou, et al. "Story3d-agent: Exploring 3d storytelling visualization with large language models." arXiv preprint arXiv:2408.11801 (2024).

\bibitem{savinov2018semi} Savinov, Nikolay, Alexey Dosovitskiy, and Vladlen Koltun. "Semi-parametric topological memory for navigation." arXiv preprint arXiv:1803.00653 (2018).

\bibitem{majumdar2022zson} Majumdar, Arjun, et al. "Zson: Zero-shot object-goal navigation using multimodal goal embeddings." Advances in Neural Information Processing Systems 35 (2022): 32340-32352.

\bibitem{yadav2023offline} Yadav, Karmesh, et al. "Offline visual representation learning for embodied navigation." Workshop on Reincarnating Reinforcement Learning at ICLR 2023. 2023.

\bibitem{yadav2023ovrl} Yadav, Karmesh, et al. "Ovrl-v2: A simple state-of-art baseline for imagenav and objectnav." arXiv preprint arXiv:2303.07798 (2023).

\bibitem{sun2024fgprompt} Sun, Xinyu, et al. "FGPrompt: fine-grained goal prompting for image-goal navigation." Advances in Neural Information Processing Systems 36 (2024).

\bibitem{zhu2017target} Zhu, Yuke, et al. "Target-driven visual navigation in indoor scenes using deep reinforcement learning." 2017 IEEE international conference on robotics and automation (ICRA). IEEE, 2017.

\bibitem{koh2024generating} Koh, Jing Yu, Daniel Fried, and Russ R. Salakhutdinov. "Generating images with multimodal language models." Advances in Neural Information Processing Systems 36 (2024).

\bibitem{krantz2022instance} Krantz, Jacob, et al. "Instance-specific image goal navigation: Training embodied agents to find object instances." arXiv preprint arXiv:2211.15876 (2022).

\bibitem{schulman2017proximal} Schulman, John, et al. "Proximal policy optimization algorithms." arXiv preprint arXiv:1707.06347 (2017).

\bibitem{anderson2018evaluation} Anderson, Peter, et al. "On evaluation of embodied navigation agents." arXiv preprint arXiv:1807.06757 (2018).

\bibitem{lin2024navcot} Lin, Bingqian, et al. "NavCoT: Boosting LLM-Based Vision-and-Language Navigation via Learning Disentangled Reasoning." arXiv preprint arXiv:2403.07376 (2024).

\bibitem{NavGPT} Zhou, Gengze, Yicong Hong, and Qi Wu. "Navgpt: Explicit reasoning in vision-and-language navigation with large language models." Proceedings of the AAAI Conference on Artificial Intelligence.

\bibitem{hahn2021no} Hahn, Meera, et al. "No rl, no simulation: Learning to navigate without navigating." Advances in Neural Information Processing Systems 34 (2021): 26661-26673.

\bibitem{li2025t2isafety} Li, Lijun, et al. "T2ISafety: Benchmark for Assessing Fairness, Toxicity, and Privacy in Image Generation." arXiv preprint arXiv:2501.12612 (2025).

\bibitem{an2024agfsync} An, Jingkun, et al. "AGFSync: Leveraging AI-Generated Feedback for Preference Optimization in Text-to-Image Generation." arXiv preprint arXiv:2403.13352 (2024).


\end{thebibliography}
\end{sloppypar}

\clearpage
\beginsupplement
\section*{Appendix}
\renewcommand{\thesubsection}{S\arabic{subsection}}

\subsection{\label{chap:S1}PanNuke and MoNuSAC preprocessing}
The PanNuke dataset comprises a set of 7,901 RGB patches, each with dimensions of $256 \times 256$ pixels, which we set as the standard patch size for our analysis. In contrast, the MoNuSAC dataset encompasses 294 images of heterogeneous dimensions. To standardize the MoNuSAC images with our experiments, we implement a standardization protocol. Specifically, for images exceeding the dimensions of $256 \times 256$ pixels, we segment them into equal-sized patches and apply mirror padding to the remaining portions to avoid information loss at the peripherals. Patches with dimensions less than $128 \times 128$ pixels are excluded from the dataset due to the insufficient resolution to capture relevant cellular details. For patches where either dimension falls between 128 and 256 pixels, we employ upsampling to achieve the standard patch size. As a result, we obtain a total of 2,823 RGB patches derived from the MoNuSAC dataset for subsequent analysis. For additional details on the MoNuSAC data preparation process, refer to the source code \cite{Shvetsov_2025a}.
\clearpage

\subsection{\label{chap:S2}Data usage for the methodology}

\counterwithin{figure}{subsection}
\renewcommand{\thefigure}{S\arabic{subsection}}

\begin{figure}[h!]
    \centering
    \includegraphics[width=\textwidth, height=0.85\textheight, keepaspectratio]{images/A2.pdf}
    \caption{Overview of the methodology for cross-labeling, dataset refinement, and model comparison. (1) Cross-relabeling - training and testing cell classification models, (2) Cross-relabeling - using cell classification models to create refined dataset, (3) Fine-tuning and training models for comparison, (4) Student knowledge distillation with refined dataset}
    \label{fig:S2}
\end{figure}
\clearpage

\subsection{\label{chap:S3}Confusion matrices for classification models}
\counterwithin{figure}{subsection}
\renewcommand{\thefigure}{S\arabic{subsection}.\arabic{figure}}

\begin{figure}[h!]
    \centering
    \includegraphics[width=\textwidth, height=0.4\textheight, keepaspectratio]{images/A3_1.pdf}
    \caption{Confusion matrix for PanNuke trained model}
    \label{fig:S3.1}
\end{figure}

\begin{figure}[h!]
    \centering
    \includegraphics[width=\textwidth, height=0.4\textheight, keepaspectratio]{images/A3_2.pdf}
    \caption{Confusion matrix for MoNuSAC trained model}
    \label{fig:S3.2}
\end{figure}

\clearpage

\subsection{\label{chap:S4}Datasets cell counts}

\counterwithin{table}{subsection}
\renewcommand{\thetable}{S\arabic{subsection}}

\begin{table}[h!]
\renewcommand{\arraystretch}{2.0}
\centering
\caption{\label{tab:S4}Cell counts for PanNuke, MoNuSAC and refined datasets. Numbers in parentheses indicate preprocessed cell counts for cell classifier models training and testing.}
%\adjustbox{max width=\textwidth}{%
\begin{tabular}{|l|c|c|c|}
\hline
%\rowcolor{gray!30}
Cell type & PanNuke & MoNuSAC & Refined \\
\hline
Neoplastic & 77,403 (68,031) & - & 105,451 \\
\hline
Epithelial & 26,572 (23,207) & - & 29,926 \\
\hline
Epithelial (benign and malignant) & - & 31,402 & - \\
\hline
Inflammatory & 32,276 & - & - \\
\hline
Lymphocytes & - & 37,045 (33,104) & 65,275 \\
\hline
Neutrophils & - & 1,355 (1,252) & 3,833 \\
\hline
Macrophage & - & 1,842 (1,695) & 3,410 \\
\hline
Dead & 2,908 & - & 2,908 \\
\hline
Connective & 50,585 & - & 50,585 \\
\hline
\end{tabular}
%
%}
\end{table}



\clearpage

\subsection{\label{chap:S5}Definition of validation metrics}
\counterwithin{equation}{subsection}
\renewcommand{\theequation}{\arabic{equation}}

\subsubsection{\label{chap:S5.1}R\textsuperscript{2}}
The coefficient of determination, denoted as $R^2$, is a statistical measure that represents the proportion of variance in the dependent variable that is predictable from the independent variables. In the context of cell quantification in pathology, $R^2$ is used to assess how well the predicted quantities of different cell types in a patch align with the actual quantities observed in the ground truth data, with higher values representing more accurate quantification. $R^2$ is defined as
\begin{equation*}
R^2 = 1 - \frac{\sum_{i=1}^n (y_i - \hat{y}_i)^2}{\sum_{i=1}^n (y_i - \bar{y})^2},
\end{equation*}
where $y_i$ represents the actual number of cells of a specific type in the $i$-th image, $\hat{y}_i$ represents the predicted number of cells of that type in the $i$-th image, $\bar{y}$ is the mean of the actual numbers across all images, and $n$ is the total number of images in the dataset.

The $R^2$ metric has a range of $(-\infty, 1]$. An $R^2$ of 1 indicates perfect prediction, where all predicted values exactly match the actual values. An $R^2$ of 0 suggests that the model explains none of the variability of the response data around its mean. If $R^2$ is negative, it indicates that the model performs worse than a model that simply predicts the mean of the actual values for all observations.

\subsubsection{\label{chap:S5.2}PQ}
Panoptic Quality ($PQ$) is a comprehensive metric used to evaluate the performance of segmentation models in tasks that require both instance segmentation and classification. $PQ$ provides a single score that encapsulates both the detection accuracy (i.e., how many objects were correctly identified) and the segmentation quality (i.e., how accurately the objects' boundaries were delineated). This metric is particularly useful in multiclass scenarios where each pixel is classified into distinct categories, such as different cell types in pathology images.

$PQ$ is calculated as the product of two terms: Detection Quality ($DQ$) and Segmentation Quality ($SQ$). It can be expressed as
\begin{equation*}
PQ = DQ \cdot SQ,
\end{equation*}
where
\begin{equation*}
DQ = \frac{TP}{TP + 0.5\, FP + 0.5\, FN},
\end{equation*}
\begin{equation*}
SQ = \frac{\sum_{(p, g) \in \mathcal{M}} IoU(p, g)}{TP}.
\end{equation*}
In these formulas, $TP$ denotes the number of correctly matched instances between ground truth and prediction, $FP$ denotes the predicted instances that have no corresponding ground truth, $FN$ denotes the ground truth instances that were not detected, $IoU(p, g)$ is the Intersection over Union for a pair of matched instances $p$ (prediction) and $g$ (ground truth), and $\mathcal{M}$ is the set of matched pairs.

The $PQ$ metric is calculated for each class and is averaged across classes to provide a global performance measure.

The $PQ$ score has a range of $[0, 1.0]$, where a higher score indicates better performance in both detecting and segmenting the instances correctly. A $PQ$ of 1 signifies perfect identification and segmentation of all instances, whereas a $PQ$ of 0 indicates that no instances were correctly identified and segmented.

\clearpage

\subsection{\label{chap:S6}Segmentation and Detection quality metrics for teacher and student models}

\begin{table}[h!]
\renewcommand{\arraystretch}{2.0}
\centering
\caption{Segmentation and detection quality for student and teacher models (CI 95\%)}
\label{tab:S6}
%\adjustbox{max width=\textwidth}{%
\begin{tabular}{|l|c|c|}
\hline
%\rowcolor{gray!30}
Metric & Teacher & Student \\
\hline
$SQ_{neoplastic}$ & 0.819 (0.815--0.823) & 0.824 (0.819--0.828) \\
\hline
$SQ_{lymphocyte}$ & 0.795 (0.788--0.802) & 0.790 (0.783--0.796) \\
\hline
$SQ_{connective}$ & 0.770 (0.762--0.776) & 0.780 (0.772--0.786) \\
\hline
$SQ_{dead}$ & 0.659 (0.623--0.688) & 0.657 (0.624--0.695) \\
\hline
$SQ_{epithelial}$ & 0.780 (0.770--0.790) & 0.788 (0.779--0.797) \\
\hline
$SQ_{macrophage}$ & 0.788 (0.760--0.810) & 0.757 (0.730--0.783) \\
\hline
$SQ_{neutrofil}$ & 0.782 (0.761--0.801) & 0.775 (0.759--0.792) \\
\hline
$DQ_{neoplastic}$ & 0.706 (0.692--0.719) & 0.727 (0.712--0.741) \\
\hline
$DQ_{lymphocyte}$ & 0.675 (0.656--0.698) & 0.713 (0.691--0.734) \\
\hline
$DQ_{connective}$ & 0.566 (0.546--0.584) & 0.583 (0.565--0.602) \\
\hline
$DQ_{dead}$ & 0.410 (0.361--0.465) & 0.435 (0.306--0.561) \\
\hline
$DQ_{epithelial}$ & 0.668 (0.639--0.694) & 0.673 (0.644--0.702) \\
\hline
$DQ_{macrophage}$ & 0.657 (0.583--0.727) & 0.615 (0.531--0.703) \\
\hline
$DQ_{neutrofil}$ & 0.691 (0.625--0.753) & 0.729 (0.679--0.778) \\
\hline
\end{tabular}
%
%}
\end{table}

\clearpage

\subsection{\label{chap:S7}QuPath integration method}
We adopt an integration strategy leveraging the paquo \cite{Bayer_AG} library, a Python package that enables direct interaction with QuPath’s internal API, thereby facilitating seamless data exchange without intermediate conversion steps. The data processing pipeline (\hyperref[fig:S7]{Appendix Figure S7}) begins with the acquisition of WSIs and their associated annotations from QuPath, which are represented as Shapely \cite{Gillies_Wel_etal._2024} polygons. Utilizing paquo, we directly read, create, and modify these annotations and detections within a QuPath project in the Python environment. Images are then cropped using these polygons and processed by cell segmentation and classification models employing standard vision processing toolkits such as OpenCV, pyvips, and PyTorch. Additionally, QuPath employs Groovy scripts to initiate a Python process that starts the entire pipeline from QuPath graphical interface: fetching polygons, extracting images from them, and running deep learning model inference on the cropped images. 
The results are returned to QuPath, leveraging paquo's Python bindings to manipulate QuPath data while minimizing the computational overhead typically associated with cross-environment communication.

\counterwithin{figure}{subsection}
\renewcommand{\thefigure}{S\arabic{subsection}}

\begin{figure}[h!]
    \centering
    \includegraphics[width=\textwidth]{images/A7.pdf}
    \caption{QuPath integration workflow using Python environment}
    \label{fig:S7}
\end{figure}

Compared to traditional workflows that involve exporting annotations as GeoJSON, classifying them in Python, and reimporting them into QuPath, our approach offers several advantages. We eliminate the need to switch between programming languages, providing a cohesive and streamlined development process entirely within QuPath software and removing the necessity to use other tools. Meanwhile, we avoid storing annotations as intermediate JSON files unless required for external use or archiving. By conducting the entire inference and post-processing workflow within the Python environment, we leverage the power and flexibility of Python libraries for image processing and machine learning. This approach also enables adjustments to any set of labels and models, thereby improving its applicability.

%\hfill

The distilled model and QuPath integration code are packaged into a Docker container, enabling streamlined execution with the Docker engine. Detailed integration code and deployment instructions can be found in the GitHub repository \cite{Shvetsov_2025b}.

Despite these benefits, we acknowledge that the paquo library is a proof‑of‑concept project in its early development stage and has not been tested across all versions of QuPath.

\clearpage

\subsection{\label{chap:S8}Data and code availability statement}
All datasets, models, and code used in this study are publicly available and can be obtained from the repositories listed below. 
The PanNuke \cite{Gamper_Koohbanani_etal._2019} and MoNuSAC \cite{Verma_Kumar_etal._2021} datasets are publicly accessible, and download information along with detailed descriptions can be found in their respective articles. Preprocessing scripts for PanNuke and MoNuSAC data, as well as individual cell extraction scripts, are available on GitHub \cite{Shvetsov_2025a}. The H-Optimus foundation model used in our experiments can be downloaded from the HuggingFace repository \cite{hoptimus2024}, and model information is available on GitHub \cite{Saillard_Jenatton_etal._2024}. In addition, the integration code for QuPath and the distilled model packaged in a Docker container are provided in the repository \cite{Shvetsov_2025b}, and paquo Python library is available from the authors GitHub repository \cite{Bayer_AG}.
\clearpage

\end{document}



\newpage
\appendix
\onecolumn



\section{Fine-grained SVT privacy bounds}
More precise bounds for Theorem~\ref{algo:svt-individual} and for standard SVT throught the target charging technique (TCT).
\begin{theorem}[Privacy of Target-Charging \cite{targetcharging:ICML2023} ]\label{thm:TCprivacy}
Algorithm~\ref{algo:svt-individual} (and per-query SVT) satisfy the following approximate DP privacy bounds:
\begin{align*}
&\left( (1+\alpha)\frac{r}{q}\eps, \delta^*(r,\alpha)\right) , & \text{for any $\alpha>0$;}\\
&\left( \frac{1}{2}(1+\alpha)\frac{r}{q} \eps^2  + \eps \sqrt{(1+\alpha)\frac{r}{q} \log(1/\delta)}, \delta + \delta^*(r,\alpha) \right), & \text{for any $\delta>0$, $\alpha>0$.}
\end{align*}
where $\delta^*(r,\alpha) \leq e^{-\frac{\alpha^2}{2(1+\alpha)} r}$ and $q=\frac{1}{e^\eps+1}$.
\end{theorem}

\section{Proof of extended generalization} \label{genproof:sec}

Here we will prove \cref{thm:DP-gen-mod}:

\dpgen*

The proof is obtained by transforming the following {\em expectation bound} into a {\em high probability bound}. 

\begin{lemma}[Expectation bound {\cite{kontorovich2022adaptive}}]\label{lem:MKLCondExpSQ}
    Let $\Bb$ be an $(\eps,\delta)$-differentially private algorithm that operates on $T$ sub-databases and outputs a predicate $h:X\rightarrow[0,1]$ and an index $t\in\{1,2,\dots,T\}$.
Let $\Dd=D_1\times\cdots D_n$ be a product distribution over $X^n$ be a distribution over $X$, let $\vec{\bsx}=(\bsx_1,\dots,\bsx_T)$ where every $\bsx_j \sim \Dd$ is sampled independently, and let $(h,t)\leftarrow \Bb\left(\vec{\bsx}\right)$.
Then,
$$
\E_{\substack{\vec{\bsx}\sim\Dd \\ (h,t)\leftarrow \Bb\left(\vec{\bsx}\right)}}\Big[ e^{-\eps} \cdot h(\Dd) \Big] - Tn\delta \;\;\;
\leq
\E_{\substack{\vec{\bsx}\sim\Dd \\ (h,t)\leftarrow \Bb\left(\vec{\bsx}\right)}}\Big[ h(\bsx_t) \Big] \;\;\;
\leq \E_{\substack{\vec{\bsx}\sim\Dd \\ (h,t)\leftarrow \Bb\left(\vec{\bsx}\right)}}\Big[ e^{\eps} \cdot h(\Dd) \Big] + Tn\delta.
$$
\end{lemma}
(Here, we use $h(\Dd)$ as shorthand to denote $\E_{\bsy \sim \Dd} h(\bsy)$.)
\begin{proof}
The proof is identical to that of Lemma 3.1 of \cite{kontorovich2022adaptive} with $\psi = 0$, and omitting the final inequality in the last chain of inequalities.
\end{proof}

\begin{proof}[Proof of Theorem~\ref{thm:DP-gen-mod}]
We prove the first inequality; the second follows from similar arguments. 
Fix a product distribution $\Dd$ on $X$. Assume towards contradiction that with probability at least $1/T$ algorithm $\Aa$ outputs a predicate $h$ such that \ 
$e^{-2\eps} \cdot h(\Dd)-h(\bsx)> \frac{4}{\eps}\log(T+1) + 2Tn\delta$.
We now use $\Aa$ and $\Dd$ to construct the following algorithm $\Bb$ that contradicts Lemma~\ref{lem:MKLCondExpSQ}. We remark that algorithm $\Bb$ ``knows'' the distribution $\Dd$. This will still lead to a contradiction because the expectation bound of Lemma~\ref{lem:MKLCondExpSQ} holds for {\em every} differentially private algorithm and {\em every} underlying distribution.


\begin{algorithm2e}[htbp]
\caption{$\Bb$}\addcontentsline{lof}{figure}{Algorithm $\Bb$}
\DontPrintSemicolon
\KwIn{$T$ databases of size $n$ each: $\vec{\bsx}=(\bsx_1,\dots,\bsx_T)$}%, where $T\triangleq\left\lfloor \eps/\delta \right\rfloor$.

Define $h^0\equiv 0$ and set $F \ot \{(h^0,1)\}$.

\For{$t=1,...,T$}{
Let $h_t \leftarrow \Aa(\bsx_t)$, and set $F=F\cup\left\{\left(h_t,t\right)\right\}$}

Sample $(h^*,t^*)$ from $F$ with probability proportional to $\exp\left(\frac{\eps}{2} \left(e^{-2\eps} \cdot h^*(\Dd)-h^*(\bsx_{t^*})\right)\right)$.

\Return{$(h^*, t^*).$}
\end{algorithm2e}


Observe that $\Bb$ only accesses its input through $\Aa$ (which is $(\eps,\delta)$-differentially private) and the exponential mechanism (which is $(\eps,0)$-differentially private). Thus, by composition and post-processing, $\Bb$ is $(2\eps,\delta)$-differentially private. 
%
Now consider applying $\Bb$ on databases $\vec{\bsx} = (\bsx_1,\dots,\bsx_T)$ containing i.i.d.\ samples from $\Dd$. By our assumption on $\Aa$, for every $t$ we have that 
$e^{-2\eps} \cdot h_t(\Dd)-h_t(\bsx_t)\geq  \frac{4}{\eps} \log(T+1) + 2Tn\delta$
 with probability at least $1/T$. We therefore get
$$\Pr_{\substack{\vec{\bsx}\sim\Dd \\ \Bb\left(\vec{\bsx}\right)}}\left[{\max_{t \in [T]} \left\{ e^{-2\eps} \cdot h_t(\Dd)-h_t(\bsx_t) \right\} \geq \frac{4}{\eps} \log(T+1) + 2Tn\delta  }\right] \geq 1 - \left( 1 - 1/T \right)^T \geq \frac12.$$
The probability is taken over the random choice of
the examples in $\vec{\bsx}$ according to $\Dd$ and the generation of the predicates $h_t$ according to $\Bb(\vec{\bsx})$.
Thus, by Markov's inequality,
$$
\E_{\substack{\vec{\bsx}\sim\Dd \\ \Bb\left(\vec{\bsx}\right)}}\left[{\max\{0,\max_{t \in [T]}  \left\{ e^{-2\eps} \cdot h_t(\Dd)-h_t(\bsx_t) \right\} }\right] \geq \frac{2}{\eps} \log(T+1) + Tn\delta.
$$
Recall that the set $F$ (constructed in step~2 of algorithm $\Bb$) contains the predicate $h^0\equiv0$, and hence,
\begin{equation}\label{eq:LargeError}
\E_{\substack{\vec{\bsx}\sim\Dd \\ \Bb\left(\vec{\bsx}\right)}}\left[\max_{(h,t) \in F} \left\{ e^{-2\eps} \cdot h_t(\Dd)-h_t(\bsx_t) \right\}\right] =\E_{\substack{\vec{\bsx}\sim\Dd \\ \Bb\left(\vec{\bsx}\right)}}\left[{\max\{0,\max_{t \in [T]}  \left\{ e^{-2\eps} \cdot h_t(\Dd)-h_t(\bsx_t) \right\} }\right] \geq \frac{2}{\eps} \log(T+1) + Tn\delta.
\end{equation}

So, in expectation, the set $F$ contains a pair $(h,t)$ with large difference $e^{-2\eps} \cdot h(\Dd)-h(\bsx_t)$. In order to contradict the expectation bound of Lemma~\ref{lem:MKLCondExpSQ}, we need to show that this is also the case for the pair $(h^*,t^*)$ that is sampled in Step~3. Indeed, by the properties of the exponential mechanism, we have that
\begin{equation}
\E_{(h^*,t^*)\in_R F}\Big[ e^{-2\eps} \cdot h^*(\Dd)- h^*(\bsx_{t^*})   \Big] \geq \max_{(h,t)\in F} \{ e^{-2\eps} \cdot h(\Dd) - h(\bsx_{t}) \} - \frac{2}{\eps} \log(T+1). \label{eq:Utility}
\end{equation}
Taking the expectation also over $\vec{\bsx}\sim\Dd$ and $\Bb(\vec{\bsx})$ we get that
\begin{eqnarray*}
\E_{\substack{\vec{\bsx}\sim\Dd \\ \Bb\left(\vec{\bsx}\right)}}\Big[  e^{-2\eps} \cdot h^*(\Dd) - h^*(\bsx_{t^*})  \Big] 
&\geq& \E_{\substack{\vec{\bsx}\sim\Dd \\ \Bb\left(\vec{\bsx}\right)}}\Big[\max_{(h,t)\in F} \{ e^{-2\eps} \cdot h(\Dd) - h(\bsx_{t}) \} \Big] - \frac{2}{\eps} \log(T+1)\\
&\geq& \frac{2}{\eps} \log(T+1) + Tn\delta - \frac{2}{\eps} \log(T+1) = Tn\delta.
\end{eqnarray*}
This contradicts Lemma~\ref{lem:MKLCondExpSQ}.
\end{proof}

% \ignore{
% \begin{theorem}[Generalization property of DP \cite{DworkFHPRR15,BassilyNSSSU:sicomp2021,FeldmanS17}] \label{thm:DP-generalization}
% Let $\mathcal{A}:X^n \to 2^{X}$ be an $(\eps, \delta)$-differentially private algorithm that operates on a dataset of size $n$ and outputs a predicate $h:X\to \{0,1\}$. Let $\Dd$ be a distribution over $X$, let $\boldsymbol{x}=(x_1,\ldots,x_n) \sim \Dd^n$ be i.i.d.\ samples from $\Dd$, and let $h\gets \mathcal{A}(\boldsymbol{x})$. Then for any $T\ge 1$ it holds that \eccomment{Other direction holds as well. Add after we converge on what we need and in what form.}
% {\small
% \[
% \Pr_{\boldsymbol{x}\sim \Dd^d,\atop h\gets \mathcal{A}(\boldsymbol{x})}\left[ e^{-2\eps} \E_{y\sim \Dd} h(y) - \frac{1}{n} \sum_{i\in [n]} h(x_i )> \frac{4}{\eps n}\log(T+1) + 2T\delta \right] < \frac{1}{T}.
% \]}
% \end{theorem}
% }

\section{Analysis of the tracking estimator} \label{trackinganalysis:sec}

Here we will prove \cref{thm:main-tracking}:

\trackinganalysis*

The analysis is analogous to \cref{sec:analysis-basic}; indeed, we will reuse most of the results from that section. We may make the same assumptions on $k, r, \al$ as at the start of \cref{sec:analysis-basic}. Again, we begin with an analog of \cref{lemma:sub-h-hat}:

\begin{lemma} \label{lemma:sub-h-hat-tracking}
If $\tw h = \sum_{(i,\rho_i)\in S} \ind{(\rho_i < \tau)\land (C[i]<r)}+\Lap(1/\eps_0)$ is replaced by $\hat h = \sum_{i \in V} \ind{(\rho_i < \tau)\land (C[i]<r)}+\Lap(1/\eps_0)$, then the outputs of \cref{bottomkrobustbaseline:algo} change with probability at most $\beta/4$.
\end{lemma}
\begin{proof}
In order for $\tw h$ not to equal $\hat h$, the maximum value of $\rho_i$ in $S$ must be below $\tau$. However, in that case, by assumption, at most $k/2$ of these elements may be inactive, so the sum in $\tw h$ is at least $k/2$. In order for the output to change in any given step, we must then have $\tw h < T < \hat h$. However, this would require $\Lap(1/\e_0) < -k/4$, which has probability at most $e^{-\e_0 k/4} < m/\beta$. By a union bound (as in the proof of \cref{lemma:sub-h-hat}), we are done.
\end{proof}

Again, we assume henceforth that \cref{bottomkrobustbaseline:algo} uses $\hat h$ instead of $\tw h$. We now show once again that \cref{bottomkrobustbaseline:algo} can be simulated by calls to \cref{algo:svt-individual}, with the same function $h_{V, \tau}$ as in \cref{sec:analysis-basic}. Indeed, the only difference from \cref{algo:svt-individual} is that $C$ can only increment elements of the sketch $S$ rather than the whole set $V$, so we need to ensure that there are never keys $i \in V$ such that $\rho_i < \tau$ and $i \notin S$. We show this, along with the analogs of \cref{prop:low-guarantee} and \cref{prop:high-guarantee}, by induction:

\begin{claim} \label{claim:consistent-induct}
The following holds with probability at least $1-\beta/2$. Let $d$ be the number of deactivated elements in the sketch $S$. Then, whenever $\tau < (1 - \al/4) T/|V|$, the while loop in \cref{bottomkrobustbaseline:algo} continues to the next value of $\tau$, and whenever $\tau > ((1 + \al/4) T + d) / |V|$, it terminates. Moreover, the values in $C$ always match the values that \cref{algo:svt-individual} would have.
\end{claim}
\begin{proof}
We proceed by induction; suppose the statement has held true on all previous inputs and iterations of the while loop. We show that it holds on the current iteration --- note that we must have $\tau < (1 + 3\al/8) (T + d) / |V|$ by the inductive hypothesis, since otherwise the loop would have terminated in the previous step. Recall by assumption that $d < k/2$, and we set $T = k/4$, so this means that $\tau < \f78 \cdot k/|V|$.

We first show that on the current input, the keys $i \in V$ with $\rho_i < \tau$ are all in the sketch $S$. Indeed, by the inductive hypothesis, we may apply \cref{generror:lemma} on $h_{V, \tau}$:

\begin{align*}
    |\tau |V| - h_{V, \tau}(\bsr)| &< \f{\al}{32} \cdot \tau |V| + O(\log(m/\beta)/\al) \\
    &< \f{\al}{32} \cdot \tau |V| + \f{\al k}{8},
\end{align*}

Since $\tau |V| < 7k/8$, this means that we have $h_{V, \tau}(\bsr) < k$. However, $h_{V, \tau}(\bsr)$ is just the count of $i \in V$ such that $\rho_i < \tau$, so if this count is less than $k$, then all such $i \in V$ are included in the sketch (since it is a bottom-$k$ sketch).

Therefore, we have shown the second part of \cref{claim:consistent-induct}, since every value that would need to be incremented is actually in the sketch $C$. It remains to show the first part.

We now apply \cref{totalerror:coro} (again using the inductive hypothesis that our algorithm has matched \cref{algo:svt-individual}), to obtain that
\begin{gather}
\hat h < (1 + \al/16) \tau |V| + \al k / 16 + d, \label{eq:hat-ub-2} \\
\hat h > (1 - \al/16) \tau |V| - \al k / 16, \label{eq:hat-lb-2}
\end{gather}
where again, the value of $\D$ is at most $\al k / 16$ by the choice of $k$, and the sum in \cref{totalerror:coro} is bounded by the number of deactivated elements satisfying $h_{V,\tau}(i, \rho_i)=1$, which is at most $d$ (since we just showed that all such elements are in $S$). The remainder of this proof is now identical to that of \cref{prop:low-guarantee} and \cref{prop:high-guarantee}.
\end{proof}

From \cref{claim:consistent-induct}, we deduce (identically to the previous analysis) that whenever $d < \al k / 4$, the output of the algorithm is a $(1+\al)$-approximation of $|V|$.


\end{document}





\section{Related links}



Combine 
multiplicative statement:
\url{https://arxiv.org/pdf/1706.05069}
Theorem 4.1

With the additive statement for product dist:
Adaptive Data Analysis with Correlated Observations
\url{https://arxiv.org/pdf/2201.08704}
Lemma 3.1

Additive bound for generalization \url{https://arxiv.org/pdf/1511.02513}

Generalization guarantees for standard deviation
\url{https://vtaly.net/papers/FS_KLAS.1217.pdf}
\url{https://arxiv.org/abs/1706.05069}



\end{document}