\section{Quantile Estimation: Exact Selection in Sublinear Memory}
In this section, we give an algorithm (\Cref{alg:exact-selection}) that, given a random-order string of $n$ elements, outputs the exact $k$-th largest element with high probability.

\begin{algorithm2e}
    \caption{Exact Selection}\label{alg:exact-selection}
    \KwIn{Stream length $n$, memory size $m$, target rank $k$, and access to random-order stream $s = (s_1, s_2, \ldots, s_n)$}
    $B_{ \lfloor \log_2 k \rfloor + 1} \leftarrow n$; $B_0 \leftarrow 0$\;
    \lFor{$i = \lfloor \log_2 k \rfloor, \ldots, 2, 1$}{sample $B_i \sim \Binomial\left(B_{i+1}, 1/2\right)$}
    \lFor{$i = 0, 1, \ldots, \lfloor \log_2 k\rfloor$} {$T_i \leftarrow \min\left\{B_{i+1},\left\lfloor\frac{k}{2^{ \lfloor\log_2 k\rfloor  - i}}\right\rfloor\right\}$}
    $i_0 \gets \min\{i \in \{0, 1, \ldots, \lfloor \log_2 k\rfloor\}: T_{i_0} > m/2$\}\;
    Let $M = (+\infty, -\infty, -\infty, \ldots, -\infty)$ be an array of length $3m$\;
    Read the first $B_{i_0 + 1}$ elements; store the largest $\min\{B_{i_0+1}, 3m-1\}$ elements among them in $M[2]$ through $M[\min\{B_{i_0+1}, 3m-1\}+1]$ in decreasing order\;\label{line:base-case}
    $M.\text{move}(3m/2 - T_{i_0} - 1)$\;\label{line:base-case-shift} 
    $\rank \gets T_{i_0}$\;
    \For{$i = i_0+1, \ldots, \lfloor \log_2 k\rfloor$} {
        \For{$j = B_i +1, B_{i} + 2, \ldots, B_{i+1}$} {
            Read the $j$-th element $s_j$\;
            \lIf{$s_j > M[3m/2]$}{$\rank \gets \rank + 1$}
            $M.\text{insert}(s_j)$\;
        }
        $M.\text{move}(\rank - T_i)$\; \label{line:inductive-step-shift}
        $\rank \gets T_i$\;
    }
    \Return $M[3m/2]$\;
\end{algorithm2e}

\begin{algorithm2e}
    \caption{$M.\text{insert}(q)$}
    Find the smallest $i^*$ such that $M[i^*] < q$\;
    \lIf{no such $i^*$ exists or $M[i^*-1] = \bot$}{\Return}
    \uIf{$i^* \le 3m/2$}{
        \lFor {$ i = 1, 2, \ldots, i^* - 2$} {$M[i] \gets M[i + 1]$}
        $M[i^* - 1]\gets q$\;
    } \Else {
        \lFor {$i = 3m, 3m-1, \ldots, i^* +1$} {$M[i] \gets M[i - 1]$}
        $M[i^*]\gets q$\;
    }
    \label{algorithm.insert}
\end{algorithm2e}

\begin{algorithm2e}
    \caption{$M$.move($d$)}
    \uIf{$d < 0$} {
        \lFor{$i = 1,  \ldots, 3m - |d|$}{$M[i] \leftarrow M\left[i + |d|\right]$}
        \For{$i =  3m - |d| + 1, 3m - |d| + 2, \ldots, 3m$}{
            \lIf{$M[3m - |d|] = -\infty$}{$M[i] \leftarrow -\infty$}
            \lElse {$M[i]\leftarrow \bot$}
        }
    } \Else {
        \lFor{$i = 3m, 3m-1, \ldots, d+1$}{$M[i] \leftarrow M\left[i - d\right]$}
        \For{$i = d, d-1, \ldots, 1$}{
            \lIf{$M[d+1] = +\infty$}{$M[i] \leftarrow +\infty$}
            \lElse{$M[i] \leftarrow \bot$}
        }
    }
  \label{algorithm.move}
\end{algorithm2e}

The algorithm maintains an array $M$ of length $3m$, divided into three blocks of length $m$. The first and the last $m$ entries are called \emph{buffers}. The algorithm tries to keep the $k$-th largest element in the middle $m$ entries of the array. In order to do so: (1) The array of length $3m$ always contains consecutive elements (among the elements that have been observed so far) in decreasing order; (2) We keep track of the rank of the middle element in the array (i.e., $M[3m/2]$) among the elements seen. Note that, by the first condition, we would know the rank of every element in the array.
 
The algorithm runs in $O(\log k)$ \emph{stages}. At the beginning of each stage $i$, the algorithm samples a string of length $B_i \sim \Binomial\left(B_{i+1}, \frac{1}{2}\right)$, while $B_{i+1}$ is the number of elements seen at the end of that stage, and tries to find the $2^i$-th largest element among the first $B_{i+1}$ elements seen.

In a typical execution of the algorithm, at the end of each stage $i$, the $\left\lfloor\frac{k}{2^{\lfloor\log_2 k\rfloor -i}}\right\rfloor$-th ($\approx 2^i$-th) largest element (among the elements that have been encountered so far) stays in the middle $m$ elements of the array. In this case, we shift the entries of the array to ensure that the $\left\lfloor\frac{k}{2^{\lfloor\log_2 k\rfloor -i}}\right\rfloor$-th largest element is at the center of the array (i.e., with index $3m/2$). All the other elements are shifted together, so that the array still stores consecutive elements in decreasing order. Doing so might cause one of the buffers to be half empty. We expect the algorithm, after reading the elements in the next stage, will be able to fill this half-full buffer.
 
We start with an easy observation: at the beginning of stage~$i$, the $B_i$ elements that have already been seen constitute a uniformly random subset of the first $B_{i+1}$ elements in the random-order stream.
\begin{lemma}\label{lemma:uniform-subset}
    Conditioning on the value of $B_{i+1}$ and the set $S$ of the first $B_{i+1}$ elements, the subset of the first $B_i$ elements is uniformly distributed over all subsets of $S$.
\end{lemma}

\begin{proof}
    This is a special case of \Cref{lemma:subsample} when $p = 1/2$.
\end{proof}

Next, we define a ``good event'' over the randomness in the ordering of the $n$ elements as well as the values $B_1$ through $B_{\lfloor \log_2 k\rfloor}$. Later, we will show that this good event happens with high probability, and implies the correctness of \Cref{alg:exact-selection}. Recall from \Cref{alg:exact-selection} that we define $T_i = \min\left\{B_{i+1},\left\lfloor\frac{k}{2^{ \lfloor\log_2 k\rfloor  - i}}\right\rfloor\right\}.$
\begin{definition}[Good event]\label{def.egood}
    Let $s_1, s_2, \ldots, s_n$ denote the $n$ elements in the order in which they arrive. Define $\goodevent$ as the event that the following conditions hold simultaneously:
    \begin{enumerate}
        \item For each stage $i$ such that $T_i \ge m / 2$, the $T_{i-1}$-th largest element among $s_{1:B_i}$ has a rank between $T_{i} - m/2$ and $T_i + m/2$ (inclusive) among $s_{1:B_{i+1}}$.
       \item For each stage $i\ge 2$, 
           either $T_{i-1} \le m$ or at least $m/2$ elements in $s_{B_i +1:B_{i+1}}$ are between the $(T_{i-1}-m+1)$-th and the $ T_{i-1}$-th largest elements in $s_{1:B_i}$.
        \item For each stage $i \ge 2$, 
         either $T_{i-1}  + m> B_i$ or at least $m / 2$ elements in $s_{B_i +1:B_{i+1}}$ are between the $T_{i-1}$-th and the $(T_{i-1} + m)$-th largest elements in $s_{1:B_i}$.
         \item For stage $i_0 \coloneqq \min\left\{i \in \{0, 1, \ldots, \lfloor \log_2 k \rfloor \}: T_{i} > \frac{m}{2} \right\}$, we have $T_{i_0} \le 2m$.
    \end{enumerate}
\end{definition}

\begin{lemma}\label{lemma:exact-cond-1}
   For any $m \ge 4$,
Condition 1 in \Cref{def.egood} holds with probability at least
    \[
        1-   \lfloor\log_2 k \rfloor\cdot2^{-m/2} - 2\sum_{i=0}^{\lfloor\log_2k\rfloor-1}\exp\left(- \frac{m^2}{32(k/2^i)}\right).
    \]
\end{lemma}

\begin{proof}
We fix a stage $i$, and condition on the value of $B_{i+1}$ as well as the set $\{s_1, s_2, \ldots, s_{B_{i+1}}\}$ of the first $B_{i+1}$ elements in the stream (but not the exact order in which they arrive). We will upper bound the probability for Condition~1 to be violated in Stage~$i$. Note that after conditioning on $B_{i+1}$, the value of $T_i$ is determined, and we may assume that $T_i \ge m / 2$; otherwise, Condition~1 would hold vacuously.

Let $x_i^*$ denote the $T_{i-1}$-th largest element among $s_{1:B_i}$. Recall that Condition~1 is the intersection of the following two events:
\begin{enumerate}
    \item Event $E_1$: The rank of $x^*_i$ among $s_{1:B_{i+1}}$ is at least $T_i - m/2$.
    \item Event $E_2$: The rank of $x^*_i$ among $s_{1:B_{i+1}}$ is at most $T_i + m/2$.
\end{enumerate}
In the following, we lower bound the probability of each of the two events above.

\paragraph{Lower bound $\pr{}{E_1}$.} We note that the following is a sufficient condition for $E_1$:
\begin{itemize}
    \item Event $E'_1$: Among the largest $T_i - m/2$ elements in $s_{1:B_{i+1}}$, strictly fewer than $T_{i-1}$ of them appear in $s_{1:B_i}$.
\end{itemize}
To see why $E'_1$ implies $E_1$, we sort the elements $s_1, s_2, \ldots, s_{B_{i+1}}$ in decreasing order: $x_1 > x_2 > \cdots > x_{B_{i+1}}$. When $E'_1$ happens, strictly fewer than $T_{i-1}$ of the elements $x_1, x_2, \ldots, x_{T_i - m/2}$ appear in $s_{1:B_i}$. Then, $x_i^*$---being the $T_{i-1}$-th largest among $s_{1:B_i}$---must be smaller than $x_{T_i - m/2}$. In other words, the rank of $x^*_i$ among $s_{1:B_{i+1}}$ must be strictly higher than $T_i - m/2$, which implies event $E_1$. 

Next, we note that if event $E'_1$ does not happen, at least one of the following two must be true:
\begin{itemize}
    \item Event $E_{1, a}$: None of the $m/2$ elements $x_{B_{i+1} - m/2 + 1}, x_{B_{i+1} - m/2 + 2}, \ldots, x_{B_{i+1}}$ appear in $s_{1:B_i}$. (This is well-defined, since $B_{i+1} \ge T_i \ge m/2$.)
    \item Event $E_{1, b}$: At least $\lfloor k / 2^{\lfloor \log_2 k\rfloor - (i-1)} \rfloor$ elements among $x_1, x_2, \ldots, x_{T_i - m /2}$ appear in $s_{1:B_i}$.
\end{itemize}
To see this, recall that $T_{i-1} = \min\left\{B_i, \lfloor k/2^{\lfloor\log_2 k\rfloor - (i-1)}\rfloor\right\}$. If $B_i \ge \lfloor k/2^{\lfloor\log_2 k\rfloor - (i-1)}\rfloor$, we have $T_{i-1} = \lfloor k / 2^{\lfloor \log_2 k\rfloor - (i-1)} \rfloor$, so the negation of event $E'_1$ would exactly be event $E_{1,b}$. If $B_{i} < \lfloor k/2^{\lfloor\log_2 k\rfloor - (i-1)}\rfloor$, we have $T_{i-1} = B_i$, in which case event $E'_1$ is violated only if exactly $T_{i-1} = B_i$ elements among $x_1, \ldots, x_{T_i - m/2}$ appear in $s_{1:B_i}$. For this to happen, none of the elements $x_{T_i - m/2 + 1}, \ldots, x_{B_{i+1}}$ may appear in $s_{1:B_i}$. Since $B_{i+1} \ge T_i$, this implies the event $E_{1,a}$.

Now we lower bound $\pr{}{E'_1}$ by upper bounding $\pr{}{E_{1,a}}$ and $\pr{}{E_{1,b}}$. By \Cref{lemma:uniform-subset}, each of $x_{T_i - m/2+1}, \ldots, x_{B_{i+1}}$ appears in $s_{1:B_i}$ with probability $1/2$ independently. It follows that
\[
    \pr{}{E_{1,a}} \le 2^{-m/2}.
\]
To control $\pr{}{E_{1,b}}$, we note that, for any $j \in [B_{i+1}]$, \Cref{lemma:uniform-subset} implies that $s_{1:B_i} \cap \{x_1, \ldots , x_j\}$ is uniformly distributed among the $2^{j}$ subsets of $\{x_1, x_2, \ldots, x_j\}$. In particular, for $j = T_i - m / 2$, $\left|s_{1:B_i} \cap \{x_1, \ldots, x_{T_{i}-m/2}\}\right|$ follows the binomial distribution $\Binomial\left(T_{i}-m/2, 1/2\right)$. Since $\lfloor k/2^{\lfloor\log_2 k\rfloor - (i-1)}\rfloor \ge \frac{1}{2}\cdot (\lfloor k/2^{\lfloor\log_2 k\rfloor - i}\rfloor-1) \ge \frac{1}{2}\cdot (T_i-1)$, by a Chernoff bound, we have
\begin{align*}
    \pr{}{E_{1,b}}
&=  \pr{X \sim \Binomial(T_i - m/2, 1/2)}{X \ge \lfloor k / 2^{\lfloor \log_2 k\rfloor - (i-1)}\rfloor}\\
&\le\pr{X \sim \Binomial(T_i - m/2, 1/2)}{X \ge (T_i-1) / 2}\\
&=  \pr{X \sim \Binomial(T_i - m/2, 1/2)}{X - \left(T_i / 2 - m / 4\right) \ge m / 4 - 1/2}\\
&\le\exp\left(-\frac{2\cdot(m / 4 - 1/2)^2}{T_i - m/2}\right)\\
&\le \exp\left(-\frac{(m-2)^2}{8\cdot T_{i}}\right) \tag{$T_i - m / 2 \le T_i$} \\
&\le \exp\left(-\frac{m^2}{32k/2^{\lfloor\log_2k\rfloor-i}}\right). \tag{$m \ge 4$, $T_i \le k/2^{\lfloor\log_2k\rfloor-i}$}
\end{align*}

Therefore, we have
\[
    \pr{}{E_1} 
\ge \pr{}{E'_1}
\ge 1 - \pr{}{E_{1,a}} - \pr{}{E_{1, b}}
\ge 1 - \left[\exp\left(-\frac{m^2}{32k/2^{\lfloor\log_2k\rfloor-i}}\right) + 2^{-m/2}\right].
\]


\paragraph{Lower bound $\pr{}{E_2}$.} The analysis for $\pr{}{E_2}$ is almost symmetric. First, we note that the following is a sufficient condition for $E_2$:
\begin{itemize}
    \item Event $E_{2}'$: Among the largest $\min\{\left\lfloor k/2^{\lfloor\log_2 k\rfloor - i}\right\rfloor + m/2, B_{i+1}\}$ elements in $s_{1:B_{i+1}}$, at least $T_{i-1}$ of them appear in $s_{1:B_i}$.
\end{itemize}
To see why $E'_2$ implies $E_2$, again, sort the elements $s_1, s_2, \ldots, s_{B_{i+1}}$ in decreasing order: $x_1 > x_2 > \cdots > x_{B_{i+1}}$. When $E'_2$ happens, at least $T_{i-1}$ of the elements $x_1, x_2, \ldots, x_{\min\{\left\lfloor k/2^{\lfloor\log_2 k\rfloor - i}\right\rfloor + m/2, B_{i+1}\}}$ appear in $s_{1:B_i}$. Then, $x_i^*$---being the $T_{i-1}$-th largest among $s_{1:B_i}$---must be larger than or equal to $x_{\min\{\left\lfloor k/2^{\lfloor\log_2 k\rfloor - i}\right\rfloor + m/2, B_{i+1}\}}$. In other words, the rank of $x^*_i$ among $s_{1:B_{i+1}}$ must be at most
\[
    \min\left\{\left\lfloor k/2^{\lfloor\log_2 k\rfloor - i}\right\rfloor + m/2, B_{i+1}\right\}
\le \min\left\{\left\lfloor k/2^{\lfloor\log_2 k\rfloor - i}\right\rfloor, B_{i+1}\right\} + m/2
=   T_i + m/2.
\]
This implies event $E_2$. 

To violate event $E_2'$, the following must be true:
\begin{itemize}
    \item Event $E_{2, a}$: $B_{i+1} > \left\lfloor k/2^{\lfloor\log_2 k\rfloor - i}\right\rfloor + m/2$, and  strictly less than $T_{i-1}$ elements among $x_1, x_2, \ldots, x_{\left\lfloor k/2^{\lfloor\log_2 k\rfloor - i}\right\rfloor + m/2}$ appear in $s_{1:B_i}$.
\end{itemize}
To see this, if $B_{i+1} \le \left\lfloor k/2^{\lfloor\log_2 k\rfloor - i}\right\rfloor + m/2$, event $E'_2$ would (vacuously) happen: among the largest $B_{i+1}$ elements in $s_{1:B_{i+1}}$, at least $T_{i-1}$ of them appear in $s_{1:B_i}$, which follows from $T_{i-1} \le B_i$. In the remaining case that $B_{i+1} > \left\lfloor k/2^{\lfloor\log_2 k\rfloor - i}\right\rfloor + m/2$, the second part of $E_{2, a}$ is exactly the negation of $E_{2}'$.

Now, we upper bound $\pr{}{E_{2, a}}$. Again, \Cref{lemma:uniform-subset} implies that
\[
    \left|s_{1:B_i} \cap \left\{x_1, \ldots, x_{\left\lfloor k/2^{\lfloor\log_2 k\rfloor - i}\right\rfloor + m/2}\right\}\right|
\]
follows the binomial distribution $\Binomial\left(\left\lfloor k/2^{\lfloor\log_2 k\rfloor - i}\right\rfloor + m/2, 1/2\right)$.  By a Chernoff bound, we have
\begin{align*}
  \pr{}{E_{2, a}}
&=  \pr{X \sim \Binomial(\left\lfloor k/2^{\lfloor\log_2 k\rfloor - i}\right\rfloor + m/2, 1/2)}{X < T_{i-1} }\\
&\le\pr{X \sim \Binomial(\left\lfloor k/2^{\lfloor\log_2 k\rfloor - i}\right\rfloor + m/2, 1/2)}{X  \le \left\lfloor k/2^{\lfloor\log_2 k\rfloor - i}\right\rfloor / 2}\\
&=  \pr{X \sim \Binomial(\left\lfloor k/2^{\lfloor\log_2 k\rfloor - i}\right\rfloor + m/2, 1/2)}{\left(\left\lfloor k/2^{\lfloor\log_2 k\rfloor - i}\right\rfloor / 2 + m / 4\right) - X \ge m / 4 }\\
&\le\exp\left(-\frac{2\cdot (m / 4)^2}{\left\lfloor k/2^{\lfloor\log_2 k\rfloor - i}\right\rfloor+ m / 2}\right)\\
&\le \exp\left(-\frac{m^2 / 8}{2 \left\lfloor k/2^{\lfloor\log_2 k\rfloor - i}\right\rfloor}\right) \notag \\
&\le \exp\left(-\frac{m^2}{16k/2^{\lfloor\log_2k\rfloor-i}}\right),
\end{align*}
where the first inequality is due to $  \left\lfloor k/2^{\lfloor\log_2 k\rfloor - i}\right\rfloor  \ge 2\cdot\left\lfloor k/2^{\lfloor\log_2 k\rfloor - (i-1)}\right\rfloor \ge 2\cdot T_{i-1}$, and the third inequality is due to $m/2 \le T_i \le \left\lfloor k/2^{\lfloor\log_2 k\rfloor - i}\right\rfloor$.

Therefore, we have
\[
    \pr{}{E_2}
\ge \pr{}{E'_2}
\ge 1 - \pr{}{E_{2,a}}
\ge 1 - \exp\left(-\frac{m^2}{16k/2^{\lfloor\log_2k\rfloor-i}}\right).
\]

\paragraph{Put everything together.} By the union bound, for each stage $i$, the probability that  $x_i^*$'s rank among $s_{1:B_{i+1}}$ is outside the range $[T_i - m/2, T_i + m/2]$ is at most
\[
    (1 - \pr{}{\event_1}) + (1 - \pr{}{\event_2})
\le 2\exp\left(-\frac{m^2}{32k/2^{\lfloor\log_2k\rfloor-i}}\right) + 2^{-m/2}.
\]
Applying another union bound again over all stages $i \in \{1, 2, \ldots, \lfloor\log_2k\rfloor\}$ proves the lemma.
\end{proof}

\begin{lemma}\label{lemma:exact-cond-23}
    Condition~2 in \Cref{def.egood} holds with probability at least
    \[
        1 - \lfloor\log_2 k \rfloor\cdot 4e^{2/3}\cdot e^{-m/12},
    \]
   and Condition~3 holds with probability at least
    \[
        1 - \lfloor\log_2 k \rfloor\cdot 2e^{-m/12}.
    \]
\end{lemma}

\begin{proof}
We start with the claim regarding Condition~2 in \Cref{def.egood}. For Condition~2 to be violated, there must be a stage $i$ such that $T_{i-1} > m$, and strictly less than $m/2$ elements in $s_{B_i +1:B_{i+1}}$ are between the $(T_{i-1}-m+1)$-th and the $ T_{i-1}$-th largest elements in $s_{1:B_i}$. We fix a stage~$i$ and upper bound the probability of the event that Condition~$2$ is violated in stage~$i$. Note that we may assume that $\left\lfloor k/2^{\lfloor\log_2 k\rfloor - (i-1)}\right\rfloor > m$: since $T_{i-1} \le \left\lfloor k/2^{\lfloor\log_2 k\rfloor - (i-1)}\right\rfloor$, for $T_{i-1} > m$ to hold, we must have $\left\lfloor k/2^{\lfloor\log_2 k\rfloor - (i-1)}\right\rfloor > m$.

Recall that $T_{i-1}$ is the smaller value between $\left\lfloor k/2^{\lfloor\log_2 k\rfloor - (i-1)}\right\rfloor$ and $B_i$. For Condition~2 in \Cref{def.egood} to be violated at stage~$i$, at least one of the following two events must happen:
\begin{itemize}
    \item Event $E_a$: $B_i \ge \left\lfloor k/2^{\lfloor\log_2 k\rfloor - (i-1)}\right\rfloor$, and there are strictly less than $m/2$ elements in $s_{B_{i} + 1: B_{i+1}}$ between the $\left(\left\lfloor k/2^{\lfloor\log_2 k\rfloor - (i-1)}\right\rfloor - m+1\right)$-th largest  and the $\left\lfloor k/2^{\lfloor\log_2 k\rfloor - (i-1)}\right\rfloor$-th largest elements in $s_{1:B_i}$.
    \item Event $E_b$: $B_i > m$, and there are strictly less than $m/2$ elements in $s_{B_i + 1: B_{i+1}}$ between the $(B_i - m+1)$-th largest and the $B_i$-th largest elements in $s_{1:B_i}$.
\end{itemize}
In the following, we control the probabilities of events $E_a$ and $E_b$.

\paragraph{Upper bound $\pr{}{E_a}$.}
List $s_{1:B_{i+1}}$ in decreasing order:  $x_1> x_2 > \cdots> x_{B_{i+1}}$. For each $i_1 \in [B_{i+1}]$, conditioning on that $x_{i_1}$ is the $\left(\left\lfloor k/2^{\lfloor\log_2 k\rfloor - (i-1)}\right\rfloor - m+1\right)$-th largest element in $s_{1:B_i}$, event $E_a$ implies at least one of the following two:
\begin{itemize}
    \item Event $E_{a,1}$: $B_{i+1} - i_1 \ge \frac{3m}{2}$, and at least $(m-1)$ elements among $ x_{i_1 + 1}, \ldots, x_{i_1 + 3m/2}$ are in $s_{1:B_i}$.
    \item Event $E_{a,2}$: $B_{i+1} - i_1 < \frac{3m}{2}$, and at least $(m-1)$ elements among $ x_{i_1 + 1}, \ldots, x_{B_{i+1}}$ are in $s_{1:B_i}$.
\end{itemize}
To see this, let $x_{i_2}$ be the $\left\lfloor k/2^{\lfloor\log_2 k\rfloor - (i-1)}\right\rfloor$-th largest element among $s_{1:B_i}$; such an element must exist, since event $E_a$ requires $B_i \ge \left\lfloor k/2^{\lfloor\log_2 k\rfloor - (i-1)}\right\rfloor$. Among $x_{i_1}, x_{i_1 + 1}, \ldots, x_{i_2}$, there are exactly $m$ elements in $s_{1:B_i}$ along with $(i_2 - i_1 + 1) - m = i_2 - i_1 - m+1$ elements in $s_{B_i + 1: B_{i+1}}$. Event $E_a$ would then imply that $i_2 - i_1 - (m-1) < m/2$, which is equivalent to $i_2 < i_1 + 3m/2-1$. It follows that, among $x_{i_1 + 1}, x_{i_1 + 2}, \ldots, x_{\min\{i_1 + 3m/2, B_{i+1}\}}$, at least $m$ of them must be in $s_{1:B_i}$, which further implies either $E_{a,1}$ or $E_{a,2}$.

We start with the probability of event $E_{a,1}$. By \Cref{lemma:uniform-subset}, conditioning on that $x_{i_1}$ is the $\left(\left\lfloor k/2^{\lfloor\log_2 k\rfloor - (i-1)}\right\rfloor - m+1\right)$-th largest element in $s_{1:B_i}$, each element among $ x_{i_1 + 1}, \ldots, x_{i_1 + 3m/2}$ still independently appears in $s_{1:B_i}$ with probability $1/2$. It then follows from a Chernoff bound that
\begin{align*}
  \pr{}{E_{a,1}}
&=  \pr{X \sim \Binomial(3m/2, 1/2)}{X \ge m-1}\\
&\le\pr{X \sim \Binomial(3m/2, 1/2)}{X -3m/4 \ge m/4-1}\\
&\le\exp\left(-\frac{2\cdot(m / 4 -1)^2}{3m/2}\right)
\le e^{2/3}\cdot   e^{-m/12}.
\end{align*}
Similarly, to upper bound the probability of event $E_{a,2}$, we note that \Cref{lemma:uniform-subset} implies that, conditioning on that $x_{i_1}$ is the $\left(\left\lfloor k/2^{\lfloor\log_2 k\rfloor - (i-1)}\right\rfloor - m+1\right)$-th largest element in $s_{1:B_i}$, each element among $ x_{i_1 + 1}, \ldots, x_{B_{i+1}}$ independently appears in $s_{1:B_i}$ with probability $1/2$. It again follows from a Chernoff bound that
\begin{align*}
  \pr{}{E_{a,2}}
&=  \pr{X \sim \Binomial(B_{i+1}-i_1, 1/2)}{X \ge m-1}\\
&\le\pr{X \sim \Binomial(3m/2, 1/2)}{X -3m/4 \ge m/4-1} \tag{$B_{i+1} - i_1 < 3m/2$}\\
&\le\exp\left(-\frac{2\cdot(m / 4 -1)^2}{3m/2}\right)
\le e^{2/3}\cdot e^{-m/12}.
\end{align*}
By the union bound, we have $\pr{}{E_a} \le \pr{}{E_{a,1}} + \pr{}{E_{a,2}} \le 2e^{2/3}\cdot e^{-m/12}$.

\paragraph{Upper bound $\pr{}{E_b}$.}
List $s_{1:B_{i+1}}$ in decreasing order:  $x_1> x_2 > \cdots> x_{B_{i+1}}$.
Conditioning on that $x_{i_2}$ is the smallest element in $s_{1:B_i}$, for event $E_b$ to happen, at least one of the following two must be true:
\begin{itemize}
    \item Event $E_{b,1}$: $i_2 > \frac{3m}{2}$, and at least $m-1$ elements among $x_{i_2- 3m/2}, x_{i_2- 3m/2+1},  \ldots, x_{i_2-1}$ are in $s_{1:B_i}$.
    \item Event $E_{b,2}$: $i_2 \le \frac{3m}{2}$, and at least $m-1$ elements among $x_{1}, x_2, \ldots, x_{i_2-1}$ are in $s_{1:B_i}$.
\end{itemize}

To see the above argument, let $x_{i_1}$ be the $\left(B_i - m+1\right)$-th largest element among $s_{1:B_i}$; such an element must exist, since event $E_b$ requires $B_i > m$. Among $x_{i_1}, x_{i_1 + 1}, \ldots, x_{i_2}$, there are exactly $m $ elements in $s_{1:B_i}$ along with $(i_2 - i_1 + 1) - m = i_2 - i_1 - m+1$ elements in $s_{B_i + 1: B_{i+1}}$. Event $E_b$ would then imply that $i_2 - i_1 - m +1< m/2$, which is equivalent to $i_1 > i_2 - 3m/2-1$. It follows that, among $x_{i_2 -1}, x_{i_2-2}, \ldots, x_{\max\{i_2 - 3m/2, 1\}}$, at least $m$ of them must be in $s_{1:B_i}$, which further implies either $E_{b,1}$ or $E_{b,2}$.

Conditioning on that $x_{i_2}$ is the smallest element in $s_{1:B_i}$, by a similar application of \Cref{lemma:uniform-subset}, each element among $x_{i_2 - 1}, x_{i_2 - 2}, \ldots, x_1$ independently appears in $s_{1:B_i}$. It then follows from a Chernoff bound that both $\pr{}{E_{b,1}}$ and $\pr{}{E_{b,1}}$ are upper bounded by $e^{2/3}\cdot e^{-m/12}$.

Finally, applying the union bound over all the stages proves the claim for Condition~$2$.

\paragraph{Condition~3.} The analysis for Condition~3 is similar and simpler. For Condition~$3$ in \Cref{def.egood} to be violated, there must be a stage $i$ such that $T_{i-1} + m \le B_i$ and strictly less than $m/2$ elements in $s_{B_i+1:B_{i+1}}$ are between the $T_{i-1}$-th and the $(T_{i-1} + m)$-th largest element in $s_{1:B_i}$. Recall that $T_{i-1}$ is the smaller value between $B_i$ and $\left\lfloor k/2^{\lfloor\log_2 k\rfloor - (i-1)}\right\rfloor$. When $T_{i-1} = B_i$, the above cannot hold, since we would have $T_{i-1} + m = B_i + m > B_i$. Thus, we will focus on the case that $T_{i-1} = \left\lfloor k/2^{\lfloor\log_2 k\rfloor - (i-1)}\right\rfloor$ in the following.

For Condition~3 in \Cref{def.egood} to be violated at stage~$i$, the following event must happen:
\begin{itemize}
    \item Event $E_a'$: $B_i\ge \left\lfloor k/2^{\lfloor\log_2 k\rfloor - (i-1)}\right\rfloor + m$, and there are strictly less than $m/2$ elements in $s_{B_i + 1: B_{i+1}}$ between the $\left\lfloor k/2^{\lfloor\log_2 k\rfloor - (i-1)}\right\rfloor $-th largest  and the $\left(\left\lfloor k/2^{\lfloor\log_2 k\rfloor - (i-1)}\right\rfloor + m\right) $-th largest elements in $s_{1:B_i}$.
\end{itemize}
To upper bound $\pr{}{E'_a}$, list $s_{1:B_{i+1}}$ in decreasing order:  $x_1> x_2 > \cdots> x_{B_{i+1}}$. Conditioning on that $x_{i_1}$ is the $\left\lfloor k/2^{\lfloor\log_2 k\rfloor - (i-1)}\right\rfloor $-th largest element in $s_{1:B_i}$, event $E_{a'}$ implies that at least one of the following two events happens:
\begin{itemize}
    \item Event $E_{a,1}'$: $B_{i+1} - i_1 \ge \frac{3m}{2}$, and at least $m$ elements among $ x_{i_1 + 1}, \ldots, x_{i_1 + 3m/2}$ are in $s_{1:B_i}$.
    \item Event $E_{a,2}'$: $B_{i+1} - i_1 < \frac{3m}{2}$, and at least $m$ elements among $ x_{i_1 + 1}, \ldots, x_{B_{i+1}}$ are in $s_{1:B_i}$.
\end{itemize}
This follows from an argument identical to the one for proving $E_a \implies E_{a,1} \vee E_{a,2}$ for Condition~2. Then, by similar applications of Chernoff bounds,
\[
    \pr{}{E'_a} \le \pr{}{E'_{a,1}} + \pr{}{E'_{a,2}} \le 2e^{-m/12}.
\]
Applying the union bound over all stages proves the claim for Condition~3.
\end{proof}

\begin{lemma}\label{lemma:exact-cond-4}
    Condition 4 in \Cref{def.egood} holds with probability at least \[
    1 - \lfloor \log_2 k\rfloor \cdot e^{-m/4}.
    \]
\end{lemma}
\begin{proof}
    For Condition 4 to be false, there must exist a stage $i = i_0 $ such that $T_i > 2m$ and $T_{i-1}\le \frac{m}{2}$. Fix $i \in \{1, 2, \ldots, \lfloor \log_2k \rfloor\}$. We will upper bound the probability of the event that both $T_i > 2m$ and $T_{i-1} \le m/2$ hold.
    
    Recall that $T_i = \min\left\{\left\lfloor k/2^{\lfloor\log_2 k\rfloor -i}\right\rfloor, B_{i+1}\right\}$. For $T_i > 2m$ to hold, we must have $B_{i+1} > 2m$ and $\left\lfloor k/2^{\lfloor\log_2 k\rfloor -i}\right\rfloor > 2m$. It follows that
    \[
        \left\lfloor k/2^{\lfloor\log_2 k\rfloor -(i-1)}\right\rfloor \ge \frac{1}{2}\cdot \left(\left\lfloor k/2^{\lfloor\log_2 k\rfloor -i}\right\rfloor - 1\right) \ge \frac{1}{2}\cdot 2m = m.
    \]
    Then, we must have $B_i \le m / 2$; otherwise, we would have
    \[
        T_{i-1}
    =   \min\left\{\left\lfloor k/2^{\lfloor\log_2 k\rfloor -(i-1)}\right\rfloor, B_i\right\}
    \ge \min\{m, m/2 + 1\}
    =   m/2+1,
    \]
    which contradicts $T_{i-1} \le m/2$.
    

    So far, we have proved that $T_{i-1} \le m / 2$ and $T_i > 2m$ together imply $B_i \le m/2$ and $B_{i+1} > 2m$. Recall that, conditioning on the value of $B_{i+1}$, $B_i$ follows the binomial distribution $\Binomial(B_{i+1}, 1/2)$. Therefore, we have
    \begin{align*}
        \pr{}{T_{i-1} \le m/2 \wedge T_i > 2m}
    &\le\pr{}{B_i \le m/2 \wedge B_{i + 1} > 2m}\\
    &\le\pr{X \sim \Binomial(2m, 1/2)}{X \le m/2} \\
    &\le\exp\left(-\frac{2\cdot (m/2)^2}{2m}\right) 
    =   e^{-m/4}. \tag{Chernoff bound}
    \end{align*}
    Applying the union bound over $i = 1, 2, \ldots, \lfloor \log_2 k\rfloor$ proves the lemma.
\end{proof}

\begin{lemma}\label{lemma:exact-correctness}
    When $\goodevent$ happens, \Cref{alg:exact-selection} outputs the $k$-th largest element among $s_1, s_2, \ldots, s_n$.
\end{lemma}
\begin{proof}
  We first note that the following invariants hold for the array $M$ throughout \Cref{alg:exact-selection}:
    \begin{enumerate}
        \item[(1)] Finite entries (i.e., everything except $\pm\infty$ and $\bot$) form a contiguous subsequence in $M$ (denoted by $M[l], M[l + 1],  \ldots, M[r]$).
        \item[(2)] Finite entries are in decreasing order, and contain every element between $M[l]$ and $M[r]$ that has appeared in the stream so far.
        \item[(3)] On either side (i.e., among $M[1], M[2], \ldots, M[l-1]$ and $M[r+1], M[r+2], \ldots, M[3m]$), there might be multiple copies of $\bot$, $+\infty$, or $-\infty$, but only one of the three on each side.
        \item[(4)] There are $+\infty$ on the left side only if $M[l]$ is the largest element so far.
        \item[(5)] There are $-\infty$ on the right side only if $M[r]$ is the smallest element so far.
    \end{enumerate}

    In the following, we will prove by induction that, under event $\goodevent$, the following two additional invariants hold at the end of each stage $i \ge i_0 = \min\{i \in \{0, 1, \ldots, \lfloor \log_2k\rfloor\}: T_i > m/2\}$:
    \begin{itemize}
        \item $M[3m/2]$ holds the $T_i$-th largest element among $s_{1:B_{i+1}}$.
        \item On either side of $M$, if ``$\bot$'' appears, there are at most $m/2$ copies.
    \end{itemize}
    Note that the above implies that, at the end of the last stage $i = \lfloor \log_2 k \rfloor$, $M[3m/2]$ holds the $T_i$-th largest element among the entire stream, where $T_i = \min\{B_{i+1}, \lfloor k / 2^{\lfloor\log_2 k\rfloor - i}\rfloor\} = \min\{n, k\} = k$. Therefore, the algorithm outputs the correct answer.
    
    \paragraph{The base case.} We start with the base case that $i = i_0$. In Line~\ref{line:base-case}, before the start of stage~$i_0+1$, \Cref{alg:exact-selection} stores the largest $\min\{B_{i_0 + 1}, 3m-1\}$ elements in decreasing order in array $M$ (starting from $M[2]$). By Condition~4 in \Cref{def.egood}, we have $T_{i_0} \le 2m$, which implies $T_{i_0} + 1 \le 2m + 1 \le 3m$. Thus, $M[T_{i_0} + 1]$ holds the $T_{i_0}$-th largest element among $s_{1:B_{i_0+1}}$. Then, in Line~\ref{line:base-case-shift}, the algorithm shifts the array by $3m/2 - T_{i_0}-1$, so that the $T_{i_0}$-th largest element is moved to $M[3m/2]$. This verifies the first invariant.
    
    For the second invariant, we consider the direction of the shifting in Line~\ref{line:base-case-shift}. If the shifting is to the right (i.e., $T_{i_0} + 1 < 3m/2$), since $M[1] = +\infty$, the left side of $M$ would be filled with more copies of $+\infty$. Furthermore, the right side of $M$ either contains finite entries exclusively, or consists of both finite entries and one or more copies of $-\infty$. In either case, the second invariant is satisfied.

    If the shifting is to the left (i.e., $T_{i_0} + 1 > 3m/2$), the left side of $M$ would only contain finite entries.  For the right side, if $M[3m] = -\infty$ before the shifting, the shifting would only introduce more copies of $-\infty$. If $M[3m]$ held a finite element before the shifting, at most $T_{i_0} + 1 - 3m/2 \le 2m + 1 - 3m/2 < m/2$ copies of ``$\bot$'' are introduced during the shifting. In either case, we would have the second invariant.
    
    \paragraph{The inductive step.} We prove the inductive step in the following way: First, we apply Conditions 2~and~3 of the good event $\goodevent$ to show that, before the shifting in Line~\ref{line:inductive-step-shift} is conducted, both sides of $M$ are filled with either finite entries exclusively, or both finite entries and copies of $\pm\infty$. In other words, $M$ does not contain the the empty entry ``$\bot$''. Then, we use Condition~1 of $\goodevent$ to show that, before the shifting, the rank of $M[3m/2]$ among $s_{1:B_{i+1}}$ is within $T_i \pm m/2$. Finally, we show that we have both invariants after the shifting is performed.
    \begin{enumerate}
        \item[(1)]  \textbf{The left side either is completely filled or contains $+\infty$.} Consider the state of the left side of $M$ at the beginning of stage~$i$. If it contained one or more $+\infty$ elements, after reading the elements in stage~$i$, it still only contains finite entries and $+\infty$ (but not $\bot$).
    
        If the left side did not have $+\infty$ at the beginning of stage~$i$, by the inductive hypothesis, it might have contained at most $m/2$ copies of $\bot$. Then, $M[m/2+1], M[m/2+2], \ldots, M[3m/2]$ must be consecutive entries among $s_{1:B_i}$, with the $T_{i-1}$-th largest element among $s_{1:B_i}$ at index $3m/2$. Thus, we have $T_{i-1} \ge 3m/2 - (m/2+1) + 1 = m$. By Condition~2 of $\goodevent$ (\Cref{def.egood}), there are at least $m/2$ elements at stage $i$ that are within the range of the $[T_{i-1}-m+1, T_{i-1}]$-th largest element at the end of stage $i-1$, so the left buffer is filled to full. 
    \item[(2)] \textbf{The right side either is completely filled or contains $-\infty$.}  Similarly, consider the state of the right side of $M$ at the beginning of stage~$i$. If it contained $-\infty$ elements, it would still only contain finite elements and $-\infty$ after reading the elements in stage~$i$.
    
    If the right side of $M$ did not contain $-\infty$ at the beginning, by the inductive hypothesis, it might have contained $\le m/2$ copies of ``$\bot$''. Then, $M[3m/2], M[3m/2+1], \ldots, M[5m/2]$ must be consecutive entries among $s_{1:B_i}$, with the $T_{i-1}$-th largest element at index $3m/2$. It follows that $T_{i-1} + m \le B_i$. By Condition~3 of $\goodevent$ (\Cref{def.egood}),  there are at least $m/2$ elements in $s_{B_i+1:B_{i+1}}$ that are between the $T_{i-1}$-th and the $(T_{i-1}+m)$-th largest elements among $s_{1:B_i}$. It follows that, after reading $s_{B_i+1}$ through $s_{B_{i+1}}$, the right side of $M$ is filled with finite elements. 
    \item[(3)] \textbf{The rank of $M[3m/2]$ among $s_{1:B_{i+1}}$ is in $[T_i - m/2, T_i + m/2]$.} By the induction hypothesis, at the beginning of stage~$i$, $M[3m/2]$ holds the $T_{i-1}$-th largest element among $s_{1:B_i}$. By Condition~1 in $\goodevent$ (\Cref{def.egood}), this element has a rank between $T_i - m/2$ and  $T_i + m/2$ (inclusive) among $s_{1:B_{i+1}}$. Therefore, after reading the $B_{i+1} - B_i$ new elements in stage~$i$, the value of the variable $\rank$ is in $[T_i - m/2, T_i + m/2]$. 
    \item[(4)] \textbf{The first invariant holds after shifting.} Now, we verify the invariants at the end of stage~$i$. Before the shifting in Line~\ref{line:inductive-step-shift}, the rank of $M[3m/2]$ among $s_{1:B_{i+1}}$ is tracked by $\rank \in [T_i - m/2, T_i + m/2]$. We claim that the $T_i$-th largest element must be held inside the array $M$. Assuming this claim, after the shifting in Line~\ref{line:inductive-step-shift}, that element will be moved to $M[3m/2]$. This proves the first invariant.
    
    To prove the claim, suppose that $\rank < T_i$; the case that $\rank > T_i$ is almost symmetric. Since $3m/2 + (T_i - \rank) \le 3m/2 + m/2 = 2m$, $3m/2 + (T_i - \rank)$ is a valid index for array $M$. Furthermore, $M[3m/2 + (T_i - \rank)]$ cannot be $\bot$; otherwise the right side of $M$ would contain at least $m + 1 > m/2$ copies of $\bot$. $M[3m/2 + (T_i - \rank)]$ cannot be $-\infty$ either; otherwise, the largest element among $s_{1:B_{i+1}}$ must be stored in $M[r]$ for some $r \in [3m/2, 3m/2 + (T_i - \rank))$. It follows that
    \[
        T_i - \rank
    =   3m/2 + (T_i - \rank) - 3m/2
    >   r - 3m/2
    =   B_{i+1} - \rank
    \ge T_i - \rank,
    \]
    a contradiction. Therefore, $M[3m/2 + (T_i - \rank)]$ must be a finite element, and its rank among $s_{1:B_{i+1}}$ is given by $\rank + (T_i - \rank) = T_i$. This proves the claim that the $T_i$-th largest element must be stored in $M$.

    \item[(5)] \textbf{The second invariant holds after shifting.} To verify the second invariant, recall that, before the shifting in Line~\ref{line:inductive-step-shift}, each side of $M$ either only contains finite elements, or consists of only finite elements and $\pm\infty$. In Line~\ref{line:inductive-step-shift}, the array $M$ gets shifted by at most $|\rank - T_i| \le m/2$. Then, on one of the two sides of $M$, we fill the empty spaces with either $\pm\infty$ or $\bot$. In either case, the second invariant is maintained.
    \end{enumerate}
    This concludes the inductive step and finishes the proof.
\end{proof}

\quantileexact*

\begin{proof}
By the union bound along with \Cref{lemma:exact-cond-1,lemma:exact-cond-23,lemma:exact-cond-4}, event $\goodevent$ happens with probability at least
\[
    1- 12\lfloor\log_2 k \rfloor\cdot \exp\left(-\min\left\{\frac{1}{12},\frac{\ln 2}{2}, \frac{1}{4}\right\}\cdot m\right) - 2\sum_{i=0}^{\lfloor\log_2k\rfloor-1}\exp\left(-\frac{m^2}{32k/2^i}\right). 
\]
The theorem then follows from \Cref{lemma:exact-correctness}.
\end{proof}