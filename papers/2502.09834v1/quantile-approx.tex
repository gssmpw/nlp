\section{Quantile Estimation in Logarithmic Memory}\label{section.2}
In this section, we prove \Cref{thm:quantile-approx} by giving an algorithm that uses $O(\log k)$ memory and finds the approximately $k$-th largest element in a random-order stream of $n$ elements, up to an $O(\sqrt{k})$ error in the rank in expectation.

\subsection{The Algorithm}

Our main algorithm (\Cref{algorithm.main}) calls a recursive procedure $\findkth$ (\Cref{algorithm.recurse}) with parameters $n$, $m$, $k$ and $+\infty$ to find the $k$-th largest element (among those that are smaller than $+\infty$) in the next $n$ elements. We sketch how \Cref{algorithm.recurse} works in the following, in the special case that $a = +\infty$ (i.e., the first level of the recursion):
\begin{itemize}
    \item First, it samples $B \sim \Binomial\left(n, 1/2\right)$ and divides the $n$ upcoming elements (denoted by $s_1, s_2, \ldots, s_n$) into two halves: $s_{1:B}$ and $s_{(B+1):n}$.
    \item Then, among the first half, it further sub-samples a small fraction of $B_1 \sim \Binomial\left(B, \frac{2m}{3k}\right)$ elements. The hope is to find an element $a'$ such that its rank among $s_{1:B}$ is slightly below $k/2$. (Concretely, $\rank_{1:B}(a'; a) = \lfloor k/2\rfloor - k'$ for some $k'$ between $1$ and $\delta \coloneqq C_0\cdot k$.)
    \item At this point, we know that the $(k/2)$-th largest element among $s_{1:B}$ is the $k'$-th largest element among $s_{1:B} \cap (-\infty, a')$.
    \item Finally, we call \Cref{algorithm.recurse} recursively with parameters $n' = n - B$, $k'$ and $a'$ to find the $k'$-th largest element among $s_{(B+1):n} \cap (-\infty, a')$, in the hope that it will be approximately the overall $k$-th largest among $s_{1:n}$.
\end{itemize}

In the more general case that $a \ne +\infty$, the algorithm essentially does the same thing, except that all elements that are larger than or equal to $a$ are ignored.

\begin{algorithm2e}
    \caption{Main Algorithm} \label{algorithm.main}
    \KwIn{Stream length $n$, memory size $m$, target rank $k$.}
    \KwOut{An approximately $k$-th largest element among the $n$ elements.}
    Call $\findkth(n, m, k, +\infty)$ in \Cref{algorithm.recurse}\;
\end{algorithm2e}


\begin{algorithm2e}
    \caption{$\findkth(n, m, k, a)$}
    \label{algorithm.recurse}
    \KwIn{Stream length $n$, memory size $m$, target rank $k$, element threshold $a$, and access to random-order sequence $s = (s_1, s_2, \ldots, s_n)$}
    \KwOut{The (approximately) $k$-th largest element among $s \cap (-\infty, a)$, or $-\infty$ if $|s \cap (-\infty, a)| < k$}
    Use $O(1)$ memory to keep track of: (1) the smallest element, denoted by $\underline{x} = \min\{s_1, s_2, \ldots, s_n\}$; (2) the number of elements that are strictly smaller than $a$, denoted by $n' = |s \cap (-\infty, a)|$\;
    \uIf{$k \le m$} {
        Use $O(m)$ memory to find the $k$ largest elements among $s_{1:n} \cap (-\infty, a)$\;\label{line:naive} 
        \Return the $k$-th largest element, or $-\infty$ if $n' < k$\;
    }
    Draw $B \sim \Binomial\left(n, \frac{1}{2}\right)$ and $B_1 \sim \Binomial\left(B, \frac{2m}{3k}\right)$\; \label{line:draw-B-1}
    Use a length-$m$ array $M$ to find the $m$ largest elements among $s_{1:B_1} \cap (-\infty, a)$\; \label{line:first-half}
    Read until the $B$-th element in the string to compute the value of $\rank_{1:B}(x; a)$ for each element $x$ in array $M$\;
    \uIf{$|s_{1:B} \cap (-\infty, a)| < \lfloor k/2\rfloor$}{
        Read the remaining $n - B$ elements to compute $\underline{x}$ and $n'$\;
        \Return $\underline{x}$, or $-\infty$ if $n' < k$\; \label{line.case1}
    }
    $\delta \leftarrow C_0\cdot k$, where $C_0$ is a sufficiently small constant parameter\;
    Find the largest element $a'$ in $M$ such that $\rank_{1:B}\left(a'; a\right) \in [\lfloor k/2 \rfloor - \delta, \lfloor k/2 \rfloor - 1]$\;\label{line:find-a'}
    \uIf{no such element $a'$ exists}{
        Read the remaining $n - B$ elements\;
        \Return the largest element in the array below $a$, or $-\infty$ if $n' < k$\;\label{line:no-a'}
    }
    $x \gets \findkth\left(n - B, m, \left\lfloor k / 2\right\rfloor - \rank_{1:B}\left(a'; a\right), a'\right)$\;\label{line:recursive-call}
    \lIf{$n' < k$}{\Return $-\infty$}
    \lIf{$x = -\infty$}{\Return $\underline{x}$}\label{line:translate-infinity-to-smallest}
    \Return $x$\;\label{line.end}
\end{algorithm2e}

\paragraph{Comparison-based algorithms.} We note that our algorithm is \emph{comparison-based} in the sense that it can be implemented such that the algorithm only accesses the elements in the sequence in the following three ways: (1) compare a pair of elements; (2) return an element as output; (3) pass an element as a parameter of a recursive call. One desirable property of such comparison-based algorithms is that, roughly speaking, the output of the algorithm only depends on the \emph{relative ordering} of the elements, rather than their exact identities. We formalize this property and use it as the definition of comparison-based algorithms in the following.

\begin{definition}[Comparison-based algorithms]\label{def:comparison-based}
    A quantile estimation algorithm $\A$ is comparison-based if, for any $n$ and $k$, when finding the $k$-th largest element in a random-order sequence $s$ of $n$ distinct elements, the distribution of
    \[
        \rank_{1:n}(\A(n, k, s))
    \]
    is the same regardless of the choice of the $n$ elements in $s$.

    More generally, suppose that the quantile estimation problem has an addition threshold parameter $a$. An algorithm $\A$ is comparison-based if, as long as $a \notin \{s_1, s_2, \ldots, s_n\}$, the distribution of
    \[
        \rank_{1:n}(\A(n, k, a, s); a)
    \]
    only depends on $n$, $k$, and $\rank_{1:n}(a)$, and does not depend on $a$ and the $n$ elements in $s$.
\end{definition}

\subsection{Overview of the Analysis}\label{sec:approx-analysis-overview}
The rest of this section is devoted to the analysis of \Cref{algorithm.recurse}. We start by observing the behavior of $\findkth$ in several different edge cases. We then sketch how we analyze the randomness in both the random-order stream and the algorithm, so that the randomness in different parts can be significantly decoupled.

\paragraph{Corner cases.} Let $n' \coloneqq |\{s_1, s_2, \ldots, s_n\} \cap (-\infty, a)|$ denote the number of elements that are strictly smaller than the threshold $a$. When $n' < k$, the quantile estimation problem is ill-defined, in which case $\findkth$ always returns $-\infty$. Note that the $n' < k$ case has the highest priority among all edge cases in the sense that, before the algorithm tries to output anything (possibly as a result of handling other corner cases), it makes sure to read through the end of the stream $s$ to check whether $n' < k$ (and outputs $-\infty$ if so).

On Line~\ref{line:naive}, we choose to use the straightforward algorithm for quantile estimation when $k \le m$. This serves as the boundary condition of the recursion, and also ensures that the parameter $\frac{2m}{3k}$ on Line~\ref{line:draw-B-1} is indeed in $[0, 1]$, and thus valid.

On Line~\ref{line.case1}, if it turns out that $|s_{1:B} \cap (-\infty, a)| < \lfloor k/2\rfloor$, we return the smallest element in the entire stream. To see why we do this, recall that there are $n'$ elements in $s_{1:n} \cap (-\infty, a)$. If we are not in the edge case that $n' < k$, we should expect that there are $\approx n' / 2 \ge k / 2$ elements in $s_{1:B} \cap (-\infty, a)$, since $s_{1:B}$, as a set, is uniformly distributed among all subsets of $\{s_1, s_2, \ldots, s_n\}$. Therefore, whenever the edge case $|s_{1:B} \cap (-\infty, a)| < \lfloor k/2\rfloor$ happens, we are sure that $n'$ only exceeds $k$ by a small amount, in which case outputting the $n'$-th largest element in $s \cap (-\infty, a)$ (namely, the smallest element in $s$) is accurate enough.

When the above does not happen, we aim to find an element $a'$ in the first half of the stream, $s_{1:B}$, so that $a'$ is close to the $\lfloor k/2\rfloor$-th largest among $s_{1:B} \cap (-\infty, a)$. On Line~\ref{line:no-a'}, if we fail to find such an $a'$, we return the largest element among $s_{1:n} \cap (-\infty, a)$, which has a rank of $1$. While doing so leads to an error of $k - 1$ in the rank, as we show in \Cref{lemma:prob-find-a'}, this edge case only happens with probability $1/\poly(k)$, so the contribution to the expected error is negligible.

If none of the edge cases above happen, $\findkth$ calls itself recursively and gets an output $x$. The final edge case is that this $x$ might take value $-\infty$, as a result of encountering the first edge case in the recursive call. If this happens, we translate $-\infty$ to the smallest element $\underline{x}$ in the stream.

\paragraph{Unravel the randomness.} We will prove the error bound of $\findkth$ by induction. As mentioned in \Cref{sec:overview-details}, a rigorous analysis is complicated because, for the inductive step, we need to show that the accuracy of the recursive call
\[
    x^* \gets \findkth(n - B, m, \lfloor k/2\rfloor - \rank_{1:B}(a'; a), a')
\]
(conditioning on all the parameters) implies the accuracy of the original procedure. For this purpose, we might hope that $s_{(B+1):n}$ (as a set) is uniformly distributed among all size-$(n-B)$ subsets of $s_{1:n}$, so that we can translate the rank of $x^*$ among $s_{(B+1):n}$ to its rank among the entire stream $s_{1:n}$. Unfortunately, this uniformity might not hold, since the conditioning on $a'$ and $\rank_{1:B}(a'; a)$ would bias the conditional distribution of $s_{(B+1):n}$.

To make the analysis valid, we need to ``realize'' the randomness of $s_{1:n}$ in a specific order. In the following, we sketch the analysis assuming that none of the edge cases above happen:
\begin{itemize}
    \item First, over the randomness in $B \sim \Binomial(n, 1/2)$, $s_{1:B}$ is uniformly distributed among all subsets of $s_{1:n}$ (\Cref{lemma:subsample}). In particular, $s_{1:B} \cap (-\infty, a)$ is a uniformly random subset of $s_{1:n} \cap (-\infty, a)$. It follows that: (1) $B' \coloneqq |s_{1:B} \cap (-\infty, a)|$ follows the distribution $\Binomial(n', 1/2)$, where $n' \coloneqq |s_{1:n} \cap (-\infty, a)|$; (2) Conditioning on the realization of $B'$, $s_{1:B} \cap (-\infty, a)$ is uniformly distributed among all size-$B'$ subsets of the size-$n'$ set $s_{1:n} \cap (-\infty, a)$. 
    \item We condition on the realization of $B'$. The algorithm finds an element $a' \in s_{1:B}$ on Line~\ref{line:find-a'}. Let $i \coloneqq \rank_{1:B}(a'; a)$ denote its rank among the first half (when only elements below $a$ are considered). Note that the distribution of $i$ is solely determined by the value of $B'$, since the algorithm is comparison-based (\Cref{def:comparison-based}).
    \item Conditioning on the realization of $(B', i)$, $a'$ is identically distributed as the $i$-th largest element in a uniformly random size-$B'$ subset of $s_{1:n} \cap (-\infty, a)$. Let $i_1 \coloneqq \rank_{1:n}(a'; a)$ denote the rank of $a'$. Later, we can analyze the concentration of $i_1 \mid B', i$ using \Cref{lemma:subset-rank-concentration}. 
    \item Conditioning on the realization of $(B', i, i_1)$, the algorithm makes a recursive call (on Line~\ref{line:recursive-call}) to find the $k'$-th largest among $s_{(B+1):n} \cap (-\infty, a')$, where $k' \coloneqq \lfloor k/2\rfloor - i$. Let $x^*$ denote the element returned by the recursive call, and $l \coloneqq \rank_{B+1:n}(x^*; a')$ be its actual rank. During the inductive proof, we will use the induction hypothesis that $l \mid B', i, i_1$ concentrates around $k'$.
    \item Conditioning on $(B', i, i_1, l)$, $x^*$ is the $l$-th largest element among $s_{(B+1):n} \cap (-\infty, a')$. Furthermore, we can verify that $s_{(B+1):n} \cap (-\infty, a')$ is uniformly distributed among all size-$(n - i_1 - (B - i))$ subsets of $s_{1:n} \cap (\infty, a')$. Applying \Cref{lemma:subset-rank-concentration} again shows that $\rank_{1:n}(x^*; a')$ concentrates, which in turn implies the concentration of $\rank_{1:n}(x^*; a) = \rank_{1:n}(x^*; a') + i_1$.
    \item Finally, our goal is to show that the rank of $x^*$, $\rank_{1:n}(x^*; a)$, concentrates around $k$. For this purpose, we will start with the conditional concentration bound above, and take an expectation over the joint distribution of $(B', i, i_1, l)$.
\end{itemize}

\subsection{The Analysis}

We will frequently use the following simple fact, which we prove in \Cref{sec:technical-lemmas}: a random prefix of a random-order stream is distributed as an independently subsampled subset.
\begin{restatable}{lemma}{lemmasubsample}\label{lemma:subsample}
    Suppose that $s_1, s_2, \ldots, s_n$ is a uniformly random permutation of a size-$n$ set $S$, and $p \in [0, 1]$. Then, for $B$ independently drawn from $\Binomial\left(n, p\right)$, $\{s_1, s_2, \ldots, s_B\}$ is distributed as a random subset of $S$ that includes every element with probability $p$ independently.
\end{restatable}

Ideally, we want there to be an element $a'$ such that the rank of it among $x_{1:B'}$ is in the range of $\left[\left\lfloor k/2 \right\rfloor - \delta, \left\lfloor k/2 \right\rfloor - 1\right]$ to conduct the recursion call on Line~\ref{line:recursive-call}. Recall that $n' \coloneqq |s_{1:n} \cap (-\infty, a)|$ denotes the number of elements in $s_{1:n}$ that are strictly smaller than $a$. The following lemma shows that, in a call to procedure $\findkth(n, m, k, a)$ with $n' \ge k$, it holds with high probability that we successfully find an element $a'$ on Line~\ref{line:find-a'}, unless the first half of the sequence, $s_{1:B}$, contains strictly fewer than $\lfloor k/2 \rfloor$ elements below $a$.

\begin{lemma}\label{lemma:prob-find-a'}
    Consider a call to the procedure $\findkth(n, m, k, a)$ that satisfies $n' \ge k \ge 2$. With probability at least 
    \[
        1 - \exp\left(-\frac{2}{9}\cdot m\right) - \exp\left(1-\frac{2C_0}{3}\cdot m\right),
    \]
    one of the following two is true:
    \begin{enumerate}
        \item $B' \coloneqq |s_{1:B} \cap (-\infty, a)| < \lfloor k / 2\rfloor$.
        \item On Line~\ref{line:find-a'}, the algorithm successfully finds an element $a'$ in the length-$m$ array such that $\rank_{1:B}(a'; a) \in [\lfloor k /2 \rfloor - C_0k, \lfloor k /2 \rfloor - 1]$.
    \end{enumerate}
\end{lemma}

\begin{proof}
    It is sufficient to prove the following: Conditioning on the realization of $B' \ge \lfloor k/2 \rfloor$ and the set $s_{1:B'} \cap (-\infty, a)$ (but not their order), the second condition (that we successfully find $a'$) holds with probability at least
    \[
        1 - \exp\left(-\frac{2}{9}\cdot m\right) - \exp\left(1-\frac{2C_0}{3}\cdot m\right).
    \]
    The lemma would then follow from the law of total probability.

    Conditioning on the value of $B'$ and the size-$B'$ set $s_{1:B} \cap (-\infty, a)$, the ordering in which these $B'$ elements appear in $s$ is still uniformly distributed. Then, since $B_1$ is sampled independently from $\Binomial(B, \frac{2m}{3k})$, by \Cref{lemma:subsample}, the elements that we examine on Line~\ref{line:first-half} constitute a random subset of $s_{1:B} \cap (-\infty, a)$ that includes every element with probability $\frac{2m}{3k}$ independently.

    Formally, suppose that the $B'$ elements in $s_{1:B} \cap (-\infty, a)$, when sorted in descending orders, are $x_1 > x_2 > \cdots > x_{B'}$. For each $i \in [B']$, let binary random variable $Y_i \coloneqq \1{x_i \in \{s_1, s_2, \ldots, s_{B_1}\}}$ denote whether $x_i$ is among the first $B_1$ elements (and thus examined on Line~\ref{line:first-half}). Then, as we discussed above, $Y_1$ through $Y_{B'}$ are independently distributed as $\Bern(\frac{2m}{3k})$.

    Then, we re-write the event that the algorithm fails to find a good $a'$ (on Line~\ref{line:find-a'}) as a condition regarding $Y_1$ through $Y_{B'}$. We claim that this can happen only if at least one of the following two events happen:
    \begin{itemize}
        \item Event $\event_a$: $Y_1 + Y_2 + \cdots + Y_{\lfloor k/2\rfloor} > m$.
        \item Event $\event_b$: $Y_i = 0$ holds for every integer $i \in [\lfloor k/2\rfloor - C_0k, \lfloor k/2\rfloor - 1]$.
    \end{itemize}
    To see this, we first note that the negation of $\event_b$ implies that there exists $i \in [\lfloor k/2\rfloor - C_0k, \lfloor k/2\rfloor - 1]$ such that $x_i$ is among $s_{1:B_1}$. Then, by the negation of $\event_a$, $x_i$ must be among the top $m$ elements that get examined on Line~\ref{line:first-half}. Then, $x_i$ would be a valid choice for $a'$ on Line~\ref{line:find-a'}.

    Since $Y_1 + Y_2 + \cdots + Y_{\lfloor k/2 \rfloor}$ follows $\Binomial(\lfloor k/2 \rfloor, \frac{2m}{3k})$, a Chernoff bound (relative error form) gives
    \begin{align*}
            \pr{}{\event_a}
    &=      \pr{X\sim \Binomial\left(\lfloor k/2\rfloor, \frac{2m}{3k}\right)}{X \ge m}\\
    &\le    \pr{X\sim \Binomial\left(\lfloor k/2\rfloor, \frac{2m}{3k}\right)}{X \ge 3\Ex{}{X}}\\
    &\le    e^{-\Ex{}{X}}
    =       \exp\left(-\left\lfloor\frac{k}{2}\right\rfloor\cdot\frac{2m}{3k}\right)
    \le     e^{-2m/9}. \tag{$\lfloor k/2 \rfloor \ge k/3$ for all $k \ge 2$}
    \end{align*}
    For the other event $\event_b$, since there are $\lfloor C_0 k\rfloor$ integers in $[\lfloor k/2 \rfloor - C_0 k, \lfloor k/2 \rfloor - 1]$, we have
    \[
        \pr{}{\event_b}
    =   \left(1 - \frac{2m}{3k}\right)^{\lfloor C_0 k\rfloor}
    \le \exp\left(-\frac{2m}{3k}\cdot \lfloor C_0 k\rfloor\right)
    \le \exp\left(-\frac{2m}{3k}\cdot (C_0 k - 1)\right)
    \le e^{1-(2C_0 / 3)\cdot m}.
    \]
    The lemma then follows from a union bound.
\end{proof}
  
The following lemma, which we prove in \Cref{sec:technical-lemmas}, considers a random size-$k$ subset of $[n]$, and shows how well the $i$-th smallest element in the subset concentrates.

\begin{restatable}{lemma}{subsetrankconcentration}\label{lemma:subset-rank-concentration}
    Suppose that $n \ge k \ge i \ge 1$. Let $x$ be the $i$-th smallest element in a size-$k$ subset of $\{1, 2, \ldots, n\}$ chosen uniformly at random. Then,
    \[
        \Ex{}{\left|x - i\cdot \frac{n}{k}\right|} \le 2\cdot \sqrt{i\cdot \frac{n}{k}}\cdot \frac{n}{k}+\frac{n^2}{k^2}.
    \]
\end{restatable}

Recall the overall plan of our analysis from \Cref{sec:approx-analysis-overview}. The following lemma shows that, conditioning on the values of $(B', i, i_1, l)$, the $l$-th largest element among $s_{(B+1):n} \cap (-\infty, a')$, denoted by $x^*$, satisfies a certain concentration of $\rank_{1:n}(x^*; a)$. To simplify the notation, the lemma is stated for the special case that $a = +\infty$, so that $n'$ and $B'$ coincide with $n$ and $B$.

\begin{lemma}\label{lemma:rank-in-second-half-to-overall}
    Suppose that $s = (s_1, s_2, \ldots, s_n)$ is a uniformly random permutation of elements $x_1 > x_2 > \cdots > x_n$. Then, the following holds for all integers $B$, $i$, and $i_1$ that satisfy $1 \le i \le B \le n$ and $i \le i_1 \le i + (n - B)$: Let $\event$ denote the event that $x_{i_1}$ is the $i$-th largest element among $s_{1:B}$. Conditioning on event $\event$, for any $l \in \{1, 2, \ldots, (n - B) - (i_1 - i)\}$, the $l$-th largest element among $s_{(B+1):n} \cap (-\infty, x_{i_1})$, denoted by $x^*$, satisfies
    \begin{align*}
        &~\Ex{}{\left|\rank_{{1:n}}(x^*) - \left(i_1+l\cdot \frac{n-i_1}{(n - i_1) - (B-i)}\right)\right|}\\
    \le &~2 \cdot \left(\frac{n-i_1}{n-i_1-(B-i)}\right)\sqrt{l\cdot \frac{n-i_1}{(n-i_1)-(B-i)}} + \left(\frac{n-i_1}{(n-i_1)-(B-i)}\right)^2,
    \end{align*}
    where the expectation is over the randomness in $s_1, s_2, \ldots, s_n$ after conditioning on $\event$.
\end{lemma}
\begin{proof}
    Let $a' \coloneqq x_{i_1}$. Since $a'$ is the $i$-th largest element in $s_{1:B}$, exactly $B - i$ elements in $s_{1:B}$ are strictly smaller than $a'$. Among $s_{1:n}$, exactly $n - i_1$ elements (namely, $x_{i_1 + 1}$ through $x_n$) are strictly smaller than $a'$. Therefore, we have
    \[
        |s_{(B+1):n} \cap (-\infty, a')|
    =   |s_{1:n} \cap (-\infty, a')| - |s_{1:B} \cap (-\infty, a')|
    =   (n - i_1) - (B - i).
    \]
    In other words, exactly $(n - i_1) - (B - i)$ elements in $s_{(B+1):n}$ are strictly smaller than $a'$.

    Without the conditioning on event $\event$, the elements in $s_{(B + 1): n}$ form a size-$(n-B)$ subset of $s_{1:n}$ chosen uniformly at random. Then, after conditioning on $\event$, $s_{(B+1):n} \cap (-\infty, a')$ is a uniformly random subset of $\{x_{i_1+1}, \ldots, x_n\}$ of size $(n - i_1) - (B - i)$. Applying \Cref{lemma:subset-rank-concentration} with parameters
    \[
        \tilde n = n - i_1, \quad \tilde k = (n - i_1) - (B - i), \quad \tilde i = l
    \]
    shows that
    \begin{align*}
        &~\Ex{}{\left|\rank_{{1:n}}(x^*; a') - l\cdot \frac{n-i_1}{(n-B)-(i_1-i)}\right|}\\
    \le &~2 \cdot \left(\frac{n-i_1}{n-i_1-(B-i)}\right)\sqrt{l\cdot \frac{n-i_1}{n-i_1-(B-i)}} + \left(\frac{n-i_1}{n-i_1-(B-i)}\right)^2,
    \end{align*}
    where $x^*$ is unique element in $s_{(B+1):n}$ that satisfies $\rank_{B+1:n}(x^*; a') = l$ (namely, the $l$-th largest element among $s_{(B+1):n} \cap (-\infty, a)$).
   
    Finally, the lemma follows from the observation that, for any $x^* < a' = x_{i_1}$,
    \[
        \rank_{1:n}(x^*)
    =   \rank_{1:n}(a') + \rank_{1:n}(x^*; a')
    =   \rank_{1:n}(x^*; a') + i_1.
    \]
\end{proof}

The following technical lemma shows how \Cref{lemma:rank-in-second-half-to-overall} is going to be applied in the analysis of \Cref{algorithm.recurse}: Conditioning on any realization of $B' \in (n'/4, 3n'/4)$ and $i \coloneqq \rank_{1:B}(a'; a)$, over the remaining randomness in $i_1\coloneqq \rank_{1:n}(a'; a)$ and $l\coloneqq \rank_{B+1:n}(x^*; a')$, the outer procedure is accurate assuming that the recursive call is accurate enough. In particular, \Cref{lemma:rank-in-second-half-to-overall} is used to translate the concentration of $l$ into the concentration of $\rank_{1:n}(x^*; a)$.

\begin{lemma}\label{lemma:rank-in-second-half-to-overall-simplified}
    Consider a call to the procedure $\findkth(n, m, k, a)$ on a random-order stream $s_1, s_2, \ldots, s_n$. Let $n' \coloneqq |s_{1:n} \cap (-\infty, a)|$ and $B' \coloneqq |s_{1:B} \cap (-\infty, a)|$ denote the numbers of elements that are strictly smaller than the threshold $a$ among $s_{1:n}$ and $s_{1:B}$, respectively. Let $i \coloneqq \rank_{1:B}(a'; a)$ denote the rank of element $a'$ (defined on Line~\ref{line:find-a'}) among $s_{1:B} \cap (-\infty, a)$, and $i_1 \coloneqq \rank_{1:n}(a'; a)$ be its rank among $s_{1:n} \cap (-\infty, a)$.
    
    Suppose that: (1) $B' \in (n'/4, 3n'/4)$; (2) conditioning on the values of $B'$, $i$, and $i_1$, the element $x^* \in s_{(B+1):n}$ returned by the recursive call (on Line~\ref{line:recursive-call}) satisfies
    \[
        \Ex{}{\left|\rank_{{B+1:n}}(x^*; a')- k'\right|} \le C_2\sqrt{k'},
    \]
    where $k' \coloneqq \lfloor k/2\rfloor - i \ge 1$ and $C_2 = 262$. Then, we have
    \[
    \Ex{}{\left|\rank_{{1:n}}(x^*; a) - \left(i_1+l\cdot \frac{n'-i_1}{n'-i_1-(B'-i)}\right)\right|} \le C_1\cdot\sqrt{C_2}\cdot\sqrt{k'},
    \]
    where $C_1 = 914$, and the expectation is over the randomness in the ordering of $s_1, s_2, \ldots, s_n$ after the conditioning on $B'$ and $i$.
\end{lemma}
\begin{proof}
Recall that $B' \coloneqq |s_{1:B} \cap (-\infty, a)|$, $i \coloneqq \rank_{1:B}(a'; a)$ and $i_1 \coloneqq \rank_{1:n}(a'; a)$. In addition, we introduce the shorthands
\[
    j \coloneqq B'-i+1,
\quad
    j_1 \coloneqq n'-i_1 +1.
\]
Then, $a'$ is both the $j$-th smallest element in $s_{1:B} \cap (-\infty, a)$ and the $j_1$-th smallest element in $s_{1:n} \cap (-\infty, a)$. It follows that, conditioning on the values of $B'$ and $i$ (and thus $j$), the value of $j_1$ is identically distributed as the $j$-th smallest number in a uniformly random size-$B'$ subset of $\{1, 2, \ldots, n'\}$, and thus can be analyzed in a similar way to the analysis in \Cref{lemma:subset-rank-concentration}.

In the rest of the proof, we will upper bound the expectation by considering two cases separately: $j_1 \ge (1+\eps)\cdot j$ and $j_1 < (1+\eps)\cdot j$, where $\eps = 1/16$.
\paragraph{Case 1: $j_1 \ge (1+\eps)\cdot j$.}
In this case, we have
\[
    \frac{B'-i}{n'-i_1-(B'-i)} = \frac{j-1}{j_1 - j} \le \frac{j-1}{(1+\eps)\cdot j - j} \le \frac{1}{\eps}.
\]
It follows that
\[
    \frac{n' - i_1}{n' - i_1 - (B' - i)}
=   \frac{B' - i}{n' - i_1 - (B' - i)} + 1
\le \frac{1}{\eps} + 1
\le \frac{2}{\eps}.
\]
Then, by \Cref{lemma:rank-in-second-half-to-overall}, we have
 \begin{equation}\begin{split}\label{eq:case-1-intermediate-bound}
    &~\Ex{i_1, l}{\1{j_1 \ge (1+\eps)\cdot j}\cdot\left|\rank_{{1:n}}(x^*; a) - \left(i_1+l\cdot \frac{n'-i_1}{n'-i_1-(B'-i)}\right)\right|}\\
%\le &~\Ex{i_1, l}{\1{j_1 \ge (1+\eps)\cdot j}\cdot\left(2 \cdot \frac{n'-i_1}{n'-i_1-(B'-i)}\cdot \sqrt{l\cdot \frac{n'-i_1}{n'-i_1-(B'-i)}} + \left(\frac{n'-i_1}{n'-i_1-(B'-i)}\right)^2\right)}\\
\le &~\Ex{i_1}{\1{j_1 \ge (1+\eps)\cdot j}\cdot\left(\frac{4\sqrt{2}}{\eps^{3/2}}\cdot \Ex{l}{\sqrt{l}\mid i_1} + \frac{4}{\eps^2}\right)}.
\end{split}\end{equation}
To deal with the term $\Ex{l}{\sqrt{l}\mid i_1}$, we note that, conditioning on the realization of $i_1$, 
\begin{equation}\label{eq:sqrt-l-bound}
    \Ex{l}{\sqrt{l} \mid i_1}
\le \sqrt{\Ex{}{l \mid i_1}}
\le \sqrt{k' + \Ex{}{|l - k'| \mid i_1}}
\le \sqrt{k' + C_2\sqrt{k'}}
\le \sqrt{2C_2}\cdot\sqrt{k'},
\end{equation}
where the first step applies Jensen's inequality, the second step applies the triangle inequality, and the third step follows from our assumption on the concentration of $l$.


Plugging the above into \Cref{eq:case-1-intermediate-bound} and setting $\eps = 1/16$ gives
\begin{equation}\begin{split}\label{eq:case-1-final-bound}
    &~\Ex{i_1, l}{\1{j_1 \ge (1+\eps)\cdot j}\cdot\left|\rank_{{1:n}}(x^*; a) - \left(i_1+l\cdot \frac{n'-i_1}{n'-i_1-(B'-i)}\right)\right|}\\
\le &~ \frac{4\sqrt{2}}{\eps^{3/2}}\cdot \sqrt{2C_2}\cdot\sqrt{k'} + \frac{4}{\eps^2}\\
=   &~512\sqrt{C_2}\cdot\sqrt{k'} + 1024
\le 768\sqrt{C_2}\cdot\sqrt{k'},
\end{split}\end{equation}
where the last step follows from $\sqrt{C_2} \cdot \sqrt{k'} \ge \sqrt{C_2} \ge 4$.

\paragraph{Case 2: $j_1 < (1+\eps)\cdot j$.} We use a similar method as in the proof of \Cref{lemma:subset-rank-concentration} to upper bound the probability of this case. As discussed earlier, this case corresponds to the event that, when a size-$B'$ subset $S \subseteq [n']$ is chosen uniformly at random, the $j$-th smallest element in $S$ is strictly smaller than $(1 + \eps)\cdot j$. This, in turn, implies the following event for $\tilde j \coloneqq \lceil (1 + \eps)\cdot j \rceil - 1 \le (1 + \eps)\cdot j$:
\begin{itemize}
    \item Event $\event_a$: $S$ contains at least $j$ elements among $1, 2, \ldots, \tilde j$.
\end{itemize}

To upper bound the probability of event $\event_a$, we use a similar method to the proof of \Cref{lemma:subset-rank-concentration}: Defining a sequence of binary random variables $X_1, \ldots, X_{n'}$ such that each $X_{t}$ takes value $1$ if and only if $t \in S$. Then, $X_1$ through $X_{n'}$ are sampled without replacement from a size-$n'$ population that consists of $B'$ copies of $1$ and $n' - B'$ copies of $0$. Using \Cref{lemma.cite}, we can relate the moment generating function of $\sum_t X_t$ to that of $\sum_t Y_t$, where each $Y_t$ is independently drawn from $\Bern(B'/n')$. It then follows from a Chernoff bound that
\begin{align*}
        \pr{}{\event_a}
&=      \pr{X}{\sum_{t=1}^{\tilde j}X_t \ge j}\\
&\le    \exp\left(-2\cdot\frac{\left(j - \tilde j\cdot\frac{B'}{n'}\right)^2}{\tilde j}\right)\\
&\le    \exp\left(-2\cdot\frac{\left(j - (1 + \eps)\cdot j\cdot\frac{B'}{n'}\right)^2}{(1 + \eps)\cdot j}\right).
\end{align*}
The third step above holds since $\tilde j \le (1 + \eps)j$ and the function $x \mapsto -\frac{(a - bx)^2}{x}$ is monotone increasing when $a \ge bx$; in this case, the assumption that $B' < 3n'/4$ indeed guarantees
\[
    a
=   j
\ge \frac{3}{4}\cdot (1 + 1/16)j
\ge \frac{B'}{n'}\cdot (1 + \eps)j
=   bx.
\]
Plugging $\eps = 1 / 16$ and $B' / n' < 3/4$ into the above gives the upper bound
\[
    \pr{}{\event_a}
\le \exp\left(-2\cdot\frac{\left(1 - (1 + 1/16)\cdot\frac{3}{4}\right)^2}{1 + 1/16}\cdot j\right)
\le \exp(-0.077\cdot j).
\]

Recall that $j_1 = n' - i_1 + 1$, $j = B' - i + 1$, and
\[
    \frac{n' - i_1}{n' - i_1 - (B' - i)}
=   \frac{B' - i}{n' - i_1 - (B' - i)} + 1
=   \frac{j - 1}{j_1 - j} + 1
\le j.
\]
By \Cref{lemma:rank-in-second-half-to-overall}, the contribution of this case to the overall expectation is
\begin{align*}
    &~\Ex{i_1, l}{\1{j_1 < (1+\eps)\cdot j}\cdot\left|\rank_{{1:n}}(x^*; a) - \left(i_1+l\cdot \frac{n'-i_1}{n'-i_1-(B'-i)}\right)\right|}\\
%    \le &~\Ex{i_1, l}{\1{j_1 \ge (1+\eps)\cdot j}\cdot\left(2 \cdot \frac{n'-i_1}{n'-i_1-(B'-i)}\cdot \sqrt{l\cdot \frac{n'-i_1}{n'-i_1-(B'-i)}} + \left(\frac{n'-i_1}{n'-i_1-(B'-i)}\right)^2\right)}\\
    \le &~\Ex{i_1}{\1{j_1 < (1+\eps)\cdot j}\cdot\left(2\cdot j^{3/2}\cdot \Ex{l}{\sqrt{l}\mid i_1} + j^2\right)}.
\end{align*}

Plugging the bound $\Ex{l}{\sqrt{l} \mid i_1} \le \sqrt{2C_2}\cdot\sqrt{k'}$ from \Cref{eq:sqrt-l-bound} into the above gives an upper bound of
\[
    (2\sqrt{2}\cdot j^{3/2}\cdot\sqrt{C_2}\cdot\sqrt{k'} + j^2)\cdot \pr{}{j_1 < (1 + \eps)\cdot j}
\le (2\sqrt{2}\cdot j^{3/2} + j^2)\cdot \pr{}{j_1 < (1 + \eps)\cdot j}\cdot \sqrt{C_2}\cdot\sqrt{k'}.
\]
Since we have shown that the $j_1 < (1 + \eps)\cdot j$ case happens with probability at most $e^{-0.077\cdot j}$, we conclude that
\begin{equation}\begin{split}\label{upper-bound.2}
    &~\Ex{i_1, l}{\1{j_1 < (1+\eps)\cdot j}\cdot\left|\rank_{{1:n}}(x^*; a) - \left(i_1+l\cdot \frac{n'-i_1}{n'-i_1-(B'-i)}\right)\right|}\\
\le &~(2\sqrt{2}\cdot j^{3/2} + j^2)\cdot e^{-0.077j}\cdot\sqrt{C_2}\cdot\sqrt{k'}\\
\le &~ 146\sqrt{C_2}\cdot\sqrt{k'},
\end{split}\end{equation}
where the last step follows from
\begin{align*}
    \max_{x \ge 0}\left(2\sqrt{2}\cdot x^{3/2} + x^2\right)\cdot e^{-0.077x}
&\le 2\sqrt{2}\cdot\max_{x\ge0}x^{3/2}e^{-0.077x} + \max_{x\ge0}x^2e^{-0.077x}\\
&=  2\sqrt{2}\cdot \left(\frac{3/2}{0.077}\right)^{3/2}\cdot e^{-3/2} + \left(\frac{2}{0.077}\right)^{2}\cdot e^{-2}
<   146.
\end{align*}

\paragraph{Combine two upper bounds.}
Finally we combine the contributions from both cases (\Cref{eq:case-1-final-bound,upper-bound.2}) and obtain
\begin{align*}
    &~\Ex{i_1, l}{\left|\rank_{{1:n}}(x^*; a) - \left(i_1+l\cdot \frac{n'-i_1}{n'-i_1-(B'-i)}\right)\right|}\\
=   &~\Ex{i_1, l}{\1{j_1 \ge (1+\eps)\cdot j}\cdot\left|\rank_{{1:n}}(x^*; a) - \left(i_1+l\cdot \frac{n'-i_1}{n'-i_1-(B'-i)}\right)\right|}\\
+   &~\Ex{i_1, l}{\1{j_1 < (1+\eps)\cdot j}\cdot\left|\rank_{{1:n}}(x^*; a) - \left(i_1+l\cdot \frac{n'-i_1}{n'-i_1-(B'-i)}\right)\right|}\\
\le &~768\sqrt{C_2}\cdot\sqrt{k'} + 146\sqrt{C_2}\cdot\sqrt{k'}
=   914\sqrt{C_2}\cdot\sqrt{k'}.
\end{align*}
\end{proof}

Recall that \Cref{lemma:rank-in-second-half-to-overall-simplified} analyzed the concentration of $\rank_{1:n}(x^*; a)$---the rank of the element $x^*$ returned by the recursive call on Line~\ref{line:recursive-call}. However, that concentration is conditional on the realization of $(B', i)$, and is around a quantity that depends on $B'$ and $i$. In the following lemma, we further analyze the randomness in $B'$ and $i$, and show a similar concentration bound around $k$, the rank that we aim for.

\begin{lemma}\label{lemma:inductive-step}
    Let $C_2 = 262$ be a universal constant. As in the setup of \Cref{lemma:rank-in-second-half-to-overall-simplified}, consider a call to the procedure $\findkth(n, m, k, a)$ on a random-order stream $s_1, s_2, \ldots, s_n$. Define $n' \coloneqq |s_{1:n} \cap (-\infty, a)|$, $B' \coloneqq |s_{1:B} \cap (-\infty, a)|$, $i \coloneqq \rank_{1:B}(a'; a)$ and $i_1 \coloneqq \rank_{1:n}(a'; a)$ in the same way. Suppose that, conditioning on the realization of $(B', i, i_1)$, the following is true for the recursive call on Line~\ref{line:recursive-call}: (1) If $|s_{(B+1):n} \cap (-\infty, a')| \ge k' \coloneqq \lfloor k/2 \rfloor - i$, it returns an element $x'$ that satisfies
    \[
        \Ex{}{\left|\rank_{B+1:n}(x'; a') - k'\right|} \le C_2 \cdot \sqrt{k'};
    \]
    (2) If $|s_{(B+1):n} \cap (-\infty, a')| < k'$, the recursive call returns $-\infty$.
    
    Then, as long as $n' \ge k$, the current call to $\findkth$ returns an element $x^*$ such that
    \[
        \Ex{}{\left|\rank_{{1:n}}(x^*; a) - k\right|\cdot \1{\goodevent}} \le (C_2-9)\cdot \sqrt{k},
    \]
    where the expectation is over the (unconditional) randomness in $s_1, s_2, \ldots, s_n$ (and thus in the realization of $(B', i, i_1)$ and the recursive call), and $\goodevent$ is the event that the following two both hold: (1) $B' \ge \lfloor k/2\rfloor$; (2) $a'$ is successfully found on Line~\ref{line:find-a'}.
\end{lemma}

The proof uses the following lemma, which we prove in \Cref{sec:technical-lemmas}: If $B$ follows $\Binomial(n, 1/2)$, the ratio $n/B$ concentrates around $2$ up to an error of $O(1/\sqrt{n})$ in expectation (ignoring the pathological case of $B = 0$).
\begin{restatable}{lemma}{expectednoverb}\label{lemma.integral1}
    It holds for every integer $n \ge 1$ that
    \[
        \Ex{B \sim \Binomial(n, 1/2)}{\left|\frac{n}{B} - 2\right|\cdot\1{B \ne 0}}  \le \frac{14}{\sqrt{n}},
    \]
    assuming that $\left|n/B - 2\right|\cdot\1{B \ne 0}$ evaluates to $0$ when $B = 0$.
\end{restatable}

\begin{proof}[Proof of \Cref{lemma:inductive-step}]
Let $n'' \coloneqq |s_{(B+1):n} \cap (-\infty, a')| = (n' - i_1) - (B' - i)$ denote the number of elements below $a'$ in the second half of the stream, which plays the role of ``$n'$'' in the recursive call. We upper bound the expectation of interest---$\Ex{}{\left|\rank_{{1:n}}(x^*; a) - k\right|\cdot \1{\goodevent}}$---by separately controlling the contribution from the case that $B' \in (n'/4, 3n'/4)$, and that from the $B' \notin (n'/4, 3n'/4)$ case. The former case is better-behaved. The latter case is unlikely to happen, so its contribution will be negligible. We further divide the former case into two sub-cases, depending on whether $n'' < k'$ (so that the recursive call returns $-\infty$) or $n'' \ge k'$ (in which case the recursive call is accurate by our assumptions).

\paragraph{Case 1: $n'/4<B'<3n'/4$ and $n'' < k'$.} In this case, the recursive call on Line~\ref{line:recursive-call} tries to find the $k'$-th largest element among $n''$ elements. By our second assumption, the recursive call always returns $-\infty$. It follows that the output of the current recursion, $x^*$, is simply the smallest element in $s_{1:n}$, i.e., $\rank_{1:n}(x^*; a) = n'$. 

Note that we can re-write $n' - k$ as follows:
\begin{align*} 
    n' - k
=   &~i_1+\left(n'-i_1-(B'-i)\right)\cdot \frac{n'-i_1}{n'-i_1-(B'-i)}  - k\\
\le &~i_1+\left(n'-i_1-(B'-i)\right)\cdot \left(2 + \left|\frac{n'}{n'-B'}-2\right|\right)\\
&\quad + \left(n'-i_1-(B'-i)\right)\cdot \left|\frac{n'-i_1}{n'-i_1-(B'-i)} - \frac{n'}{n'-B'}\right|  - k\\
\le &~i_1+\left(\left\lfloor\frac{k}{2}\right\rfloor - i\right)\cdot \left(2 + \left|\frac{n'}{n'-B'}-2\right|\right)\\
&\quad + \left(n'-i_1-(B'-i)\right)\cdot \left|\frac{n'-i_1}{n'-i_1-(B'-i)} - \frac{n'}{n'-B'}\right|  - k\\
\le &~i_1 - 2i + (k/2)\cdot \left|\frac{n'}{n'-B'}-2\right| + \left|i_1 - i\cdot \frac{n'}{B'}\right|\cdot \left| \frac{B'}{n'-B'}\right|,
\end{align*}
where the second step applies the triangle inequality
\[
    \frac{n'-i_1}{n'-i_1-(B'-i)}
\le \left|\frac{n'-i_1}{n'-i_1-(B'-i)} - \frac{n'}{n'-B'}\right| + \left|\frac{n'}{n'-B'}-2\right| + 2,
\]
the third step holds since 
\[
    n'-i_1-(B'-i)
=   n''
<   k'
=   \left\lfloor\frac{k}{2}\right\rfloor - i,
\]
and the last step follows from $\left\lfloor\frac{k}{2}\right\rfloor - i \le k/2$, as well as
\begin{align*}
    \left|\frac{n'-i_1}{n'-i_1-(B'-i)} - \frac{n'}{n'-B'}\right|
=   &~\left|\frac{n'-i_1 - [n'-i_1-(B'-i)]\cdot \frac{n'}{n'-B'}}{n'-i_1-(B'-i)}\right| \\
=   &~\left|\frac{-i_1 + (i_1-i)\cdot \frac{n'}{n'-B'}}{n'-i_1-(B'-i)}\right| \\
=   &~ \frac{1}{|n'-i_1-(B'-i)|}\cdot\left| \frac{B'}{n'-B'}\right|\cdot \left|i_1 - \frac{n'}{B'}\cdot i\right|.
\end{align*}
Then, after multiplying the two indicators
\begin{align*}
   \mathbbm{1}_{b}&\coloneqq \1{\goodevent \wedge B' \in (n'/4, 3n'/4)},\\
\mathbbm{1}_{i_1}& \coloneqq  \1{n'' < k'}
\end{align*}
and taking an expectation, we obtain:
\begin{align*}
            A_1
&\coloneqq  \Ex{}{\mathbbm{1}_{b}\mathbbm{1}_{i_1}|\rank_{1:n}(x^*; a) - k|}\\
&\le        \Ex{}{\mathbbm{1}_{b}\mathbbm{1}_{i_1}(i_1 - 2i)} + (k/2)\cdot \Ex{}{\mathbbm{1}_{b}\mathbbm{1}_{i_1}\left|\frac{n'}{n'-B'}-2\right|} + \Ex{}{\mathbbm{1}_{b}\mathbbm{1}_{i_1}\left|i_1 - i\cdot \frac{n'}{B'}\right|\cdot \left| \frac{B'}{n'-B'}\right|}\\
&\le        \Ex{}{\mathbbm{1}_{b}\mathbbm{1}_{i_1}(i_1 - 2i)} + (k/2)\cdot \Ex{}{\mathbbm{1}_{b}\mathbbm{1}_{i_1}\left|\frac{n'}{n'-B'}-2\right|} + 3\cdot \Ex{}{\mathbbm{1}_{b}\mathbbm{1}_{i_1}\left|i_1 - i\cdot \frac{n'}{B'}\right|
}\\
&\le        4\Ex{}{\mathbbm{1}_{b}\mathbbm{1}_{i_1}\left|i_1 - i\cdot \frac{n'}{B'}\right|} + (k/2)\cdot \Ex{}{\mathbbm{1}_{b}\mathbbm{1}_{i_1}\left|\frac{n'}{n'-B'}-2\right|} +  \Ex{}{\mathbbm{1}_{b}\mathbbm{1}_{i_1}i\cdot \left|2 - \frac{n'}{B'}\right|}\\
&\le        4\Ex{}{\mathbbm{1}_{b}\mathbbm{1}_{i_1}\left|i_1 - i\cdot \frac{n'}{B'}\right|} + (k/2)\cdot \Ex{}{\mathbbm{1}_{b}\mathbbm{1}_{i_1}\left|\frac{n'}{n'-B'}-2\right|}+ (k/2)\cdot \Ex{}{ \mathbbm{1}_{b}\mathbbm{1}_{i_1}\left|2 - \frac{n'}{B'}\right|},
\end{align*}
The third step holds since $B' \in (n'/4, 3n'/4)$ implies $B' / (n' - B') \le 3$. The fourth step applies the triangle inequality again: $i_1 - 2i \le |i_1 - i\cdot(n' / B')| + |2i - i\cdot(n' / B')|$. The last step applies $i \le \lfloor k / 2\rfloor \le k /2$, which holds whenever the algorithm successfully finds a valid element $a'$ on Line~\ref{line:find-a'}, which is in turn guaranteed by the event $\goodevent$.

\paragraph{Case 2: $n' / 4 < B' < 3n'/4$ and $n'' \ge k'$.} Define
    \[l \coloneqq\rank_{B+1:n}(x'; a').\]
By our assumptions, conditioning on any realization of $(B', i, i_1)$ that leads to the recursive call with $n'' \ge k'$, we have
\begin{equation}\label{eq:lemma-assumption}
    \Ex{l}{\left|l - k'\right|} \le C_2 \cdot \sqrt{k'}. 
\end{equation}
In the following, we write $\mathbbm{1}_{i_1}' \coloneqq \1{n'' \ge k'}$ as the complement of $\mathbbm{1}_{i_1}$.

Recall that our goal is to control the expectation
\[
    \Ex{}{\mathbbm{1}_b\mathbbm{1}'_{i_1}|\rank_{1:n}(x^*; a) - k|}.
\]
By the triangle inequality, $|\rank_{1:n}(x^*; a) - k|$ is upper bounded by
\begin{equation}\label{eq:case-2-three-terms}
    \left|l\cdot \frac{n-i_1}{n'-i_1 - (B'-i)} - 2l\right|
+   |i_1 + 2l - k|
+   \left|\rank_{1:n}(x^*; a) - \left(i_1+l\cdot \frac{n'-i_1}{n'-i_1-(B'-i)}\right)\right|.
\end{equation}
Let $A_{2,1}$, $A_{2,2}$ and $A_{2,3}$ denote the expectation of the three terms above, after multiplying the indicators $\mathbbm{1}_b\mathbbm{1}'_{i_1}$. We will upper bound these three terms separately in the following.

We start with $A_{2,1}$. Conditioning on the realization of $B' \ge \lfloor k/2 \rfloor$, we have
\begin{align*}
    &~\Ex{i_1}{\mathbbm{1}_{i_1}'\Ex{l}{\left|l\cdot \frac{n-i_1}{n'-i_1 - (B'-i)} - 2l\right|}}\\
\le &~\Ex{i_1}{\mathbbm{1}_{i_1}'\Ex{l}{\left|l\cdot \frac{n-i_1}{n'-i_1 - (B'-i)} - l\cdot \frac{n'}{n'-B'}\right| + \left|2l- l\cdot \frac{n'}{n'-B'}\right|}} \tag{triangle inequality}\\
=   &~\Ex{i_1}{\mathbbm{1}_{i_1}' \Ex{l}{l\cdot\frac{\left|i_1 - i\cdot \frac{n'}{B'}\right|\cdot \left|\frac{B'}{n'-B'}\right|}{n''}}} + \Ex{}{l}\cdot \left|\frac{n'}{n'-B'} - 2\right|\\
\le &~\Ex{i_1 }{\mathbbm{1}_{i_1}'\left|i_1 - i\cdot \frac{n'}{B'}\right|\cdot \left|\frac{B'}{n'-B'}\right|} +\left(k' + \Ex{}{\left|l - k'\right|}\right)\cdot \left|\frac{n'}{n'-B'} - 2\right|\tag{$l \le n''$, $l \le k' + |l - k'|$} \\
\le &~\Ex{i_1}{\mathbbm{1}_{i_1}'\left|i_1 - i\cdot \frac{n'}{B'}\right|\cdot \left|\frac{B'}{n'-B'}\right|} +\left(k' + C_2\sqrt{k'}\right)\cdot \left|\frac{n'}{n'-B'} - 2\right| \tag{\Cref{eq:lemma-assumption}}\\
\le &~\Ex{i_1}{\mathbbm{1}_{i_1}' \left|i_1 - i\cdot \frac{n'}{B'}\right|\cdot \left|\frac{B'}{n'-B'}\right|} +\left(\delta + C_2\sqrt{\delta}\right)\cdot \left|\frac{n'}{n'-B'} - 2\right|. \tag{$k' \le \delta$}
\end{align*}

Then taking an expectation over the randomness in $B'$ gives:
\begin{align*}
    A_{2,1}
&\coloneqq  \Ex{}{\mathbbm{1}_b\mathbbm{1}_{i_1}'\left|l\cdot \frac{n'-i_1}{n'-i_1 - (B'-i)} - 2l\right|}\\
&\le        \left(\delta + C_2\sqrt{\delta}\right)\cdot \Ex{}{\mathbbm{1}_b\mathbbm{1}_{i_1}'\left|\frac{n'}{n'-B'} - 2\right|}+ \Ex{}{\mathbbm{1}_b\mathbbm{1}_{i_1}' \left|i_1 - i\cdot \frac{n'}{B'}\right|\cdot \left|\frac{B'}{n'-B'}\right|}\\
&\le        \left(\delta + C_2\sqrt{\delta}\right)\cdot  \Ex{}{\mathbbm{1}_b\mathbbm{1}_{i_1}'\left|\frac{n'}{n'-B'} - 2\right|} +  3\cdot \Ex{}{ \mathbbm{1}_b\mathbbm{1}_{i_1}'\left|i_1 - i\cdot \frac{n'}{B'}\right|},
\end{align*}
where the third step holds since $\mathbbm{1}_b \ne 0 \implies B' \in (n'/4, 3n'/4) \implies B' / (n' - B') \le 3$.

Now we turn to the $A_{2,2}$ term, i.e., the expectation of the second term $|i_1 + 2l - k|$ in \Cref{eq:case-2-three-terms}. Conditioning on the value of $B'$ and taking an expectation over $(i, i_1, l)$ gives
\begin{align*}
    &~\Ex{}{\mathbbm{1}_{i_1}'\left|(i_1 + 2l)-k\right|}\\
\le &~\Ex{}{\mathbbm{1}_{i_1}'\left|i_1 + 2k' - k\right|} + 2\Ex{}{\left|l - k'\right|} \tag{triangle inequality}\\
\le &~\Ex{}{\mathbbm{1}_{i_1}'\left|i_1 - 2i + 2\lfloor k/2\rfloor -k\right|} + 2C_2 \cdot \sqrt{k'} \tag{$k' = \lfloor k/2\rfloor - i$, \Cref{eq:lemma-assumption}}\\
\le &~\Ex{}{\mathbbm{1}_{i_1}'|i_1 - 2i|} + 2C_2\sqrt{\delta}+ 1. \tag{$|2\lfloor k/2\rfloor - k| \le 1$, $k' \le \delta$}
\end{align*}
Multiplying with the indicator $\mathbbm{1}_b$ and taking another expectation over $B'$ gives
\begin{align*}
    A_{2,2}
&       \coloneqq\Ex{}{\mathbbm{1}_b\mathbbm{1}_{i_1}'\left|(i_1 + 2l)-k\right|}\\
&\le    \Ex{}{\mathbbm{1}_b\mathbbm{1}_{i_1}'|i_1 - 2i|} + 2C_2\sqrt{\delta}+ 1\\
&\le    \Ex{}{\mathbbm{1}_b\mathbbm{1}_{i_1}'\left(\left|i_1 - \frac{n'}{B'}\cdot i\right| + i\cdot\ \left|\frac{n'}{B'} - 2\right|\right)} + 2C_2\sqrt{\delta}+ 1 \tag{triangle inequality}\\
&\le    \Ex{}{\mathbbm{1}_b\mathbbm{1}_{i_1}'\left(\left|i_1 - \frac{n'}{B'}\cdot i\right| + \frac{k}{2}\cdot\ \left|\frac{n'}{B'} - 2\right|\right)} + 2C_2\sqrt{\delta} + 1,
\end{align*}
where the last step holds since, whenever $\mathbbm{1}_b \ne 0$, event $\goodevent$ happens, which guarantees $i \le \lfloor k/2\rfloor \le k/2$.

Finally, for the last term $A_{2,3}$, \Cref{lemma:rank-in-second-half-to-overall-simplified} implies that, conditioning on fixed $B'$ and $i$, we have
\[
    \Ex{}{\mathbbm{1}_{i_1}'\left|\rank_{1:n}(x^*; a) - \left(i_1+l\cdot \frac{n'-i_1}{n'-i_1-(B'-i)}\right)\right|} \le 914\sqrt{C_2}\cdot\sqrt{k'},
\]
where $k' = \lfloor k/2\rfloor - i \le \delta$ for all possible values of $i$. Then, taking an expectation over $B'$ and $i$ gives
\begin{align*}
            A_{2,3}
&\coloneqq  \Ex{}{\mathbbm{1}_b\mathbbm{1}_{i_1}'\left|\rank_{1:n}(x^*; a) - \left(i_1+l\cdot \frac{n'-i_1}{n'-i_1-(B'-i)}\right)\right|}\\
&\le       914\sqrt{C_2}\cdot\sqrt{\delta}.
\end{align*}

\paragraph{Case 3: Either $B \le n'/4$ or $B' \ge 3n'/4$.} By a Chernoff bound,
\[
    \pr{B' \sim \Binomial\left(n', 1/2\right)}{B' \le \frac{n'}{4} \vee B' \ge \frac{3n'}{4}}  \le 2\cdot\exp\left(-2\cdot n' \cdot (1/4)^2\right) = 2e^{-n'/8}.
\]
Note that, regardless of the choice of $x^* \in s_{1:n} \cap (-\infty, a)$, its rank $\rank_{1:n}(x^*; a)$ is always between $1$ and $n'$. Thus, we have the upper bound
\[
    A_3 \coloneqq \Ex{}{\1{B \notin (n'/4, 3n'/4)}\cdot\left|\rank_{1:n}(x^*; a) - k\right|}
\le 2n'\cdot \exp\left(-n'/8\right) \le 6,
\]
where the last step follows from $\max_{x \ge 0}2xe^{-x/8} = 16 / e \le 6$.

\paragraph{Put everything together.} Now, we upper bound the total contribution from the three cases above, namely, 
\[
    \Ex{}{\left|\rank_{1:n}(x^*; a) - k\right| \cdot \1{\goodevent}}
\le A_1 + A_{2,1} + A_{2,2} + A_{2,3} + A_3.
\]

Recall that $\mathbbm{1}_{b} = \1{\goodevent \wedge B' \in (n'/4, 3n'/4)}$ is the indicator for the following three to hold simultaneously: (1) $B' \ge \lfloor k/2\rfloor$; (2) $\findkth$ successfully finds $a'$ on Line~\ref{line:find-a'}; (3) $B' \in (n'/4, 3n'/4)$. For brevity, we shorthand the terms that frequently appear in these upper bounds: 
\begin{align*}
    T_1& \coloneqq \Ex{}{\mathbbm{1}_b|n'/B' - 2|},\\
    T_2 &\coloneqq \Ex{}{\mathbbm{1}_b|n'/(n' - B') - 2|},\\
    T_3 & \coloneqq \Ex{}{\mathbbm{1}_b\left|i_1 - i\cdot \frac{n'}{B'}\right|}.
\end{align*}
Then, by dropping the indicators $\mathbbm{1}_{i_1}$ and $\mathbbm{1}'_{i_1}$, the upper bounds that we have obtained so far can be simplified into:
\begin{align*}
    A_1 &\le \frac{k}{2}(T_1 + T_2) + 4T_3,\\
    A_{2, 1} & \le \left(\delta + C_2\sqrt{\delta}\right)\cdot T_2 + 3T_3,\\
    A_{2, 2} & \le \frac{k}{2}\cdot T_1 + T_3 + \left(2C_2\sqrt{\delta} + 1\right),\\
    A_{2, 3} & \le  914\sqrt{C_2}\cdot\sqrt{\delta}\\
    A_3 & \le 6.
\end{align*}
Taking a sum gives an overall upper bound of
\begin{equation}\label{eq:overall-bound-unsimplified}
    kT_1 + \left(\delta + C_2\sqrt{\delta} + \frac{k}{2}\right)\cdot T_2 + 8T_3 + \left(914\sqrt{C_2}\cdot\sqrt{\delta} + 2C_2\sqrt{\delta} + 7\right).
\end{equation}

Now we upper bound the terms $T_1$, $T_2$ and $T_3$. Recall that $\mathbbm{1}_b \ne 0$ implies $B' \in (n'/4, 3n'/4)$, which further implies $B' \ne 0$ and $n' - B' \ne 0$. Thus, by \Cref{lemma.integral1},
\[
    T_1, T_2
\le \frac{14}{\sqrt{n'}}
\le \frac{14}{\sqrt{k}}.
\]
For $T_3$, we apply \Cref{lemma:rank-in-second-half-to-overall}, $n' / B' < 4$ and $i \le \lfloor k/2\rfloor \le k/2$ to obtain
\[
    T_3
\le \Ex{}{\mathbbm{1}_b\left(2\cdot\frac{n'}{B'}\cdot\sqrt{i\cdot \frac{n'}{B'}} + \left(\frac{n'}{B'}\right)^2\right)}
\le 16\sqrt{k/2} + 16 
\le 28\sqrt{k}.
\]

Then, \Cref{eq:overall-bound-unsimplified} can be further simplified into
\begin{align*}
    &~14\sqrt{k} + (\delta + C_2\sqrt{\delta})\cdot\frac{14}{\sqrt{k}} + 7\sqrt{k} + 224\sqrt{k} + (914\sqrt{C_2}\cdot\sqrt{\delta} + 2C_2\sqrt{\delta} + 7)\\
\le &~252\sqrt{k} + \left[(\delta + C_2\sqrt{\delta})\cdot\frac{14}{\sqrt{k}} + 914\sqrt{C_2}\cdot\sqrt{\delta} + 2C_2\sqrt{\delta}\right].
\end{align*}
Recall that $\delta$ is set to $C_0\cdot k$ in \Cref{algorithm.recurse}. When $C_0$ is sufficiently small, all the terms that depend on $\delta$ in the above can be made smaller than $\sqrt{k}$ in total. It follows that
\[
    \Ex{}{\left|\rank_{1:n}(x^*; a) - k\right| \cdot \1{\goodevent}}
\le 253\sqrt{k}
=   (C_2 - 9)\cdot\sqrt{k}.
\]
\end{proof}

Finally, we prove our main theorem below.

\quantileapprox*

\begin{proof}
We prove the following statement regarding the $\findkth$ procedure (\Cref{algorithm.recurse}) by induction on $k$: 
\paragraph{Induction Hypothesis:} The procedure $\findkth(n, m, k, a)$, when running on a random-order sequence $s_1, s_2, \ldots, s_n$ that satisfies $n' \coloneqq |s_{1:n} \cap (-\infty, a)| \ge k$, returns an element $x^*$ that satisfies
\[
    \Ex{}{\left|\rank_{1:n}(x^*; a) - k\right|} \le C_2\sqrt{k},
\]
where $C_2 = 262$ and the expectation is over the randomness both in the ordering of the $n$ elements and in the algorithm.

The induction hypothesis above, when applied to the call to $\findkth(n, m, k, +\infty)$ in \Cref{algorithm.main}, directly proves the theorem. Also note that the proof by induction is valid since, whenever $\findkth$ is called recursively (on Line~\ref{line:recursive-call}), the parameter ``$k$'' for the recursive call is set to $\lfloor k/2 \rfloor - i$, which is strictly smaller than $k$.

\paragraph{The base case.} We start with the base case that $k \in \{1, 2, \ldots, m\}$. Since $k \le m$, $\findkth$ would always choose to use the straightforward method on Line~\ref{line:naive}, and find the $k$-th largest element exactly. This ensures that $\Ex{}{|\rank_{1:n}(x^*; a) - k|} = 0 \le C_2\sqrt{k}$ holds in the base case.


\paragraph{The inductive step.} Now we deal with the inductive step for $k > m$, in which case \Cref{algorithm.recurse} does not use the straightforward solution on Line~\ref{line:naive}. Note that we may also assume that $n' \ge k$; otherwise, the induction hypothesis would be vacuously true.

We upper bound the expectation by considering the following three cases:
\begin{itemize}
    \item \textbf{Case 1:} $s_{1:B}$ contains strictly fewer than $\lfloor k/2 \rfloor$ elements that are strictly smaller than $a$. In other words, $B' \coloneqq |s_{1:B} \cap (-\infty, a)| < \lfloor k/2 \rfloor$.
    \item \textbf{Case 2:} $B' \ge \lfloor k/2 \rfloor$, but we fail to find $a'$ on Line~\ref{line:find-a'}.
    \item \textbf{Case 3:} $B' \ge \lfloor k/2 \rfloor$ and we successfully find $a'$ on Line~\ref{line:find-a'}. 
\end{itemize}
Formally, we can decompose the expectation in question into
\[
    \Ex{}{\left|\rank_{1:n}(x^*; a) - k\right|} = A_1 + A_2 + A_3,
\]
where for each $i \in \{1, 2, 3\}$,
\[
    A_i \coloneqq \Ex{}{\left|\rank_{1:n}(x^*; a) - k\right|\cdot \1{\text{Case $i$}}}.
\]
In the rest of the proof, we upper bound the three terms separately.

\paragraph{Case 1: $B' < \lfloor k/2 \rfloor$.} In this case, $\findkth$ would set $x \gets -\infty$ on Line~\ref{line.case1}. Then, by Line~\ref{line:translate-infinity-to-smallest}, the algorithm would eventually output $x^* = \underline{x}$, the smallest element among $s_1, s_2, \ldots, s_n$. In other words, we always have $\rank_{1:n}(x^*; a) = |s_{1:n} \cap (-\infty, a)| = n'$ in this case. Recall that $B'$ follows the distribution $\Binomial(n', 1/2)$. Therefore, the contribution of this case, denoted by $A_1 \coloneqq \Ex{}{|\rank_{1:n}(x^*; a) - k| \cdot \1{\text{Case 1}}}$, is given by
\[
    A_1
=   \Ex{B' \sim \Binomial(n', 1/2)}{(n' - k) \cdot \1{B' < \lfloor k/2 \rfloor}}.
\]
For any $B' < \lfloor k / 2 \rfloor$, it holds that
$n' - k \le n' - 2B' \le |n' - 2B'|$, so we have the upper bound
\[
    A_1
\le \Ex{B'}{\left|n' - 2B'\right| \cdot \1{B'< \lfloor k/2\rfloor}}.
\]
Note that $B' = 0$ happens with probability $2^{-n'}$, so this case contributes at most $2^{-n'}\cdot n' \le 1$ to the right-hand side above. In other words,
\[
    A_1 \le \Ex{B'}{\left|n' - 2B'\right| \cdot \1{ 1\le B'< \lfloor k/2\rfloor}\}} + 1.
\]
For any $B' \in [1, \lfloor k/2 \rfloor)$, we have $|n' - 2B'|
=   B'\cdot|n' / B' - 2|
\le \frac{k}{2}\cdot |n' / B' - 2|$. Therefore,
\begin{align*}
        \Ex{B'}{\left|n' - 2B'\right| \cdot \1{1 \le B'< \lfloor k/2\rfloor}}
&\le    \frac{k}{2} \cdot \Ex{B'}{\left|\frac{n'}{B'}-2\right| \cdot \1{ 1\le B'< \lfloor k/2\rfloor}}\\
&\le    \frac{k}{2}\cdot\Ex{B'}{\left|\frac{n'}{B'}-2\right| \cdot \1{B' \ne 0}}\\
&\le    \frac{k}{2}\cdot \frac{14}{\sqrt{n'}} \tag{\Cref{lemma.integral1}}\\
&\le    7\sqrt{k}. \tag{$n' \ge k$}
\end{align*}

Therefore, we conclude that $A_1
\le 7\sqrt{k} + 1 \le 8\sqrt{k}$.
     
\paragraph{Case 2: $B' \ge \lfloor k/2\rfloor$ but $a'$ cannot be found.} In this case, by Line~\ref{line:no-a'} of the algorithm, the output $x^*$ is set to the largest element in the sequence that is strictly smaller than $a$. In other words, we always have $\rank_{1:n}(x^*; a) = 1$, which gives $|\rank_{1:n}(x^*; a) - k| = k - 1 \le k$. Furthermore, by \Cref{lemma:prob-find-a'}, this case happens with probability at most
\[
    \exp\left(-\frac{2}{9}\cdot m\right) + \exp\left(1-\frac{2C_0}{3}\cdot m\right).
\]
Therefore, the contribution of this case to $\Ex{}{|\rank_{1:n}(x^*; a) - k|}$ is at most
\begin{align*}
    A_2
&=  \Ex{}{|\rank_{1:n}(x^*; a) - k| \cdot \1{\text{Case 2}}}\\
&\le k\cdot \pr{}{\text{Case 2}}\\
&\le k\cdot\left[\exp\left(-\frac{2}{9}\cdot m\right) + \exp\left(1-\frac{2C_0}{3}\cdot m\right)\right]\\
&\le \sqrt{k},
\end{align*}
where the last step holds for all sufficiently large $m \ge \Omega(\log k)$.
     
\paragraph{Case 3: $B' \ge \lfloor k/2 \rfloor$ and $a'$ is found.} In this case, we make a recursive call on Line~\ref{line:recursive-call}, and use its result as the output $x^*$. By \Cref{lemma:inductive-step}, we have
\[
    A_3 = \Ex{}{\left|\rank_{1:n}(x^*; a) - k\right|\cdot\1{\text{Case 3}}} \le (C_2 - 9)\sqrt{k}.
\]       

\paragraph{Put everything together.} Combining the three cases, we have
\[
    \Ex{}{\left| \rank_{1:n}(x^*; a) - k\right|}
\le A_1 + A_2 + A_3
\le 8\sqrt{k} + \sqrt{k} + (C_2 - 9)\sqrt{k}
=   C_2\sqrt{k}.
\]
This completes the inductive step and finishes the proof.
\end{proof}