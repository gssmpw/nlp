\section{Related Work}
From the perspective of theory, the paper ____ defines the local optimal lattice quantizer by using a method similar to defining the local minimum value of a function. Local optimal lattice quantizer is a lattice quantizer that satisfies the requirement that NSM will not decrease after a lattice matrix is left multiplied by a matrix that is infinitely tending to the identity matrix. On this basis, ____ proved that all Voronoi regions of local optimal lattice quantizer satisfy some symmetry, that is, the correlation matrix is a constant multiple of the unit matrix. This proves theoretically that the Voronoi region of the optimal lattice quantizer must have a certain degree of symmetry.

Paper ____ considers using lower triangular matrix to represent lattice quantizer matrix, using stochastic gradient descent algorithm to optimize lattice NSM, and proposes a powerful tool for converting numeric lattice representations into their underlying exact forms.

____ considers splitting the entire $n$-dimensional space into several subspace when designing an $n$-dimensional lattice. The optimal results of these subspace are then orthogonal concatenated. After realizing that orthogonality is the worst allocation method, we decided to use gradient descent to explore non-orthogonal cases.

Specifically, we referred to the optimal method of constructing an $n$-dimensional lattice from two low-dimensional lattices, as described in ____. According to the formulas in ____, given $k$ lattices, denoted as $A_i$ with volume $V_i$ and normalized $n$-sphere measure (NSM) as $G_i$, the best orthogonal concatenation $a_1A_1 \otimes a_2A_2 \otimes \dots \otimes a_kA_k$ must satisfy:

\[
a_i = \frac{C}{\sqrt{G_i}V_i^{\frac{1}{n}}}
\]

where $C$ is a constant.

Since $K_{12}$ is used in dimensions $13, 14, 15$, and $\Lambda_{16}$ is used in dimensions $17$ to $22$, we fixed the coefficients of these two matrices to $1$, manually computed the coefficients $a_i$ for the other matrices, and obtained the best matrices under orthogonal concatenation.