\begin{figure*}[h]
    \centering
    \includegraphics[width=\linewidth]{figures/src/intro_dataset.pdf}
    % \vspace{-4mm}
    \caption{
    \textbf{Examples of Subsets in \DatasetName.}
    The dataset includes Synthetic Data and Real Data splits. 
    Synthetic data consists of three progressively complex subsets, while real data is gathered from diverse sources.
    }
    \label{fig:all_datasets_visualization}
\end{figure*}


\section{Related Works}
\label{sec:related_works}

\paragraph{Text-conditioned Image Synthesis.}
Recent advances in generative modeling have prominently featured diffusion-based~\cite{song2019generative,ho2020denoising,ho2022classifier} and autoregressive-based~\cite{ramesh2021zero,esser2021taming} frameworks. Diffusion models, such as DALL-E~\cite{dalle} and Parti~\cite{parti}, produce high-fidelity outputs through an iterative refinement process but are limited by slow inference speeds. 
While, autoregressive models~\cite{unifiedio2,emu,anygpt,llamagen}, model images as sequential token streams by leveraging vector quantization~\cite{van2017neural} to discretize raw pixel data into tokens, which balances efficiency and sample quality, making autoregressive modeling increasingly popular.

Despite significant progress, current methods still face challenges in generating dense, stylized text within images while maintaining high precision and aesthetic coherence.
Our approach bridges this gap by building a carefully curated, text-rich dataset to enhance the accuracy and stylistic variety, even when conditioned on complex, lengthy prompts.


\paragraph{Text-Image Pair Datasets for Generation.}

MS-COCO~\cite{mscoco} and TextCaps~\cite{textcaps} are widely used image-text pair benchmarks.
MS-COCO features descriptive annotations and TextCaps adds more contextually rich captions.
Recently, CC3M~\cite{cc12m} and LAION~\cite{laion} further emphasize large-scale data sourced from the Web, which have been instrumental in training text-conditioned image generation models.
However, both primarily cater to short or moderately long captions, limiting their suitability for tasks involving lengthy textual content.
% Other datasets, such as OpenImages~\cite{openimages} and Visual Genome~\cite{visualgenome}, prioritize detailed object relationships and scene understanding but do not cater to tasks requiring extensive or structured text.

More recent efforts, Marion10M~\cite{textdiffuser} and AnyWords3M~\cite{anytext} aim to diversify text inputs but often lack high-quality annotations or precise alignment, prioritizing visual scenes over accurate textual rendering.
To bridge these gaps, we introduce a long text rendering dataset \DatasetName~explicitly designed for generating images from extensive and structured text. 
To the best of our knowledge, this is the first large-scale dataset of its kind, addressing the limitations of existing resources and enabling advancements in long text-to-image generation tasks.




\paragraph{Visual Text Rendering.}

Rendering text accurately in images requires balancing textural correctness, visual quality, and contextual coherence. 

Prior work in text image synthesis is broadly categorized into two directions: structured methods~\cite{anytext,glyphcontrol,glyphdraw,textdiffuser,pixart,ca_aware}, which enforce layout guidelines to achieve precise text placement for design-oriented tasks (e.g., posters), and unconstrained approaches~\cite{dnd_transformer} that prioritize flexibility for long-text generation (e.g., documents) without extra guidelines.
Despite their strengths, both approaches struggle to render extended text sequences in a diverse and precise way.

These challenges are largely due to the absence of high-quality, large-scale dense-text image datasets. To close this gap, we introduce a diverse and comprehensive text-rich dataset to enable more precise and flexible text rendering.
