\section{Related Work}
The facility location problem (FLP) has a long and rich history, with its origins tracing back to $17^{\text{th}}$-century mathematicians like Pierre de Fermat and Evangelista Torricelli, who studied geometric optimization problems involving the positioning of points to minimize distances to a given set of locations, known as the Fermat-Weber problem____. This early work laid the foundation for modern FLP. The field saw significant growth after World War II, spurred by advances in operations research, as facility location became crucial for industrial planning, supply chains, and logistics____. During this period, figures such as Harold Kuhn formalized mathematical models that enabled the practical application of FLP to real-world challenges, ranging from public service placement to telecommunications infrastructure____.
In modern times, the facility location problem has found broad applications in diverse fields such as operations research, computer science, and electronics. With the rise of data-driven decision-making, facility location models are now applied in cloud computing infrastructure, data centers, network design, and even in the placement of sensors in wireless networks ____. The continued relevance of facility location models underscores their versatility in addressing problems that require optimal resource allocation and spatial planning.


\subsection{General Facility Location}
The general facility location problem has been widely studied across various fields due to its applications in logistics, urban planning, and operations research. 
A general overview of the results and variants of FLPs can be found in ____. Several variants of the FLP have been studied, such as obnoxious facility location____ and capacitated facility location ____. Online FLPs are also studied where the agents arrive in an online fashion and a set of facilities is maintained____.
% \sout{____ explore robust facility location under uncertainty by proposing a two-stage optimization model to handle demand uncertainty. This is relevant as it aligns with our framework's ability to adapt to more general scenarios. ____ tackle facility location problems with varying objective functions, which directly corresponds to our aim of accommodating different welfare functions within a flexible framework.} 
Additionally, ____ examines the polytope associated with the asymmetric version of the facility location problem. 
____ also study facility location with concave welfare functions. However, their focus is on designing algorithms with a constant approximation ratio, whereas our work investigates the structural properties of such a system. ____ considers a probabilistic view of FLPs. This is relevant as we also perform a probabilistic analysis.

\subsection{Facility Location on a Line, Fairness, and Strategyproofness}
The facility location problem on a line, where both agents and facilities are confined to a linear domain, has garnered significant attention for its simplicity and traceability. ____ provide approximation guarantees to deterministic and randomized mechanisms that try to minimize total cost while maintaining strategyproofness to ensure no agent can manipulate the outcome. These works highlight the need for welfare functions that incorporate fairness, and our framework addresses this requirement.


Recent work on fairness in facility location problems has become increasingly relevant as a growing emphasis has been placed on equitable distribution across agents ____. ____ introduced the Nash welfare function, establishing its foundational role in welfare economics. ____ further highlight its application in facility location, demonstrating that the Nash welfare function effectively balances fairness and efficiency.
 This is particularly important for our generalized welfare framework, which aims to extend beyond specific functions like Nash welfare. Furthermore, ____ introduce algorithms for 2-facility location that ensure envy-freeness, reinforcing the importance of fairness in our work.  ____ examine the problem of \textit{proportional fairness} in obnoxious facility location, where facilities are undesirable to agents and fairness becomes a key concern. ____ introduce the concept of \textit{positive intra-group externalities} in facility location, focusing on how intra-group dynamics affect utility and strategyproof mechanisms ____. 

\subsection{Welfare Functions and $p$-mean Functions}
Welfare functions have long been central to decision-making and resource allocation in facility location. 
% ____
%____
____ presents a method for learning welfare functions from revealed preferences, which is critical as our generalized framework aims to accommodate complex and dynamically evolving welfare functions. ____ explores the theoretical front of learning welfares or preferences through the context of generalization bounds. In the context of Nash welfare, ____ demonstrates its use in allocation problems, reinforcing the importance of designing flexible welfare functions that balance fairness and efficiency.

Researchers have also explored generalizations of utilitarian and egalitarian welfare through $p$-mean functions ____, which can be viewed as a parameterized family of welfare functions where varying the parameter $p$ adjusts the balance between fairness and efficiency ____. For instance, $p = 1$ corresponds to utilitarian welfare, $p = \infty$ corresponds to egalitarian welfare, and intermediate values of $p$ provide trade-offs between these extremes. Our work builds on these concepts by integrating $p$-mean functions into a broader framework for generalized welfare functions.

____ and ____ contribute to the growing body of work on Nash welfare, focusing on balancing fairness and efficiency. Our framework expands on these ideas by allowing for general welfare functions that can capture more complex and non-linear utility structures, as noted by ____ in facility location. The increasing need for \textit{learned} generalized welfare functions ____ to accommodate engineering applications and other real-world complexities further motivates our research. In the next section, we will introduce the formal problem setup and explain the notations.




%%%%%%%%%%%%%%%%%%%%%%%%%%%%%%%%%%%%%%%%%%%%%%%%