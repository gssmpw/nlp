\subsection{Bearing-based Formation Tracking Control} \label{sec:BPEcontrol}
Denote $\tilde{p}_{ij} = (p_{j} - p_{i}) - (p_{j}^* - p_{i}^*)$ and $\tilde{v}_{ij} = (v_{j} - v_{i}) - (v_{j}^* - v_{i}^*)$ as the respective relative position and velocity error $\forall i\ge 2$. The distributed bearing formation controller is defined as
\begin{equation}
    u_i^{b} = \sum_{j \in \mathcal{N}_i}( -{k_{p,i}} \pi_{g_{ij}} p_{ij}^*) - {k_{d,i}} \tilde{v}_{ij} + u_i^*, \ i\ge 2,
    \label{eq:BPElaw}
\end{equation}
where $u_i^{*} \in \mathbb{R}^{3}$ is a desired acceleration to be tracked, and $k_{p,i}$ and $k_{d,i}$ are positive scalar gains such that $k_{d,i} > 1$ and $k_{p,i} < \frac{4}{N_i} - \frac{4}{k_{d,i}^2 N_i^3}$.
\begin{theorem} \label{th:formation}
Consider a system with $n$ ($n\geq2$) quadrotor UAVs. For all agents $i\ge 2$, consider the closed-loop system \eqref{eqn:double_integrator_model} along with the proposed controller \eqref{eq:BPElaw}, such that $u_i = u_i^b$. If Assumptions \ref{asu:topology} and \ref{asu:boundedBPE} are satisfied, the equilibrium point $(\tilde{p}_{ij},\tilde{v}_{ij})=(0,0), \ i \ge 2$ is uniformly exponentially (UE) stable.
\end{theorem}
Considering the assumption in (\ref{eqn:double_integrator_model}), the UE stabilization of the equilibrium point $(\tilde{p}_{ij},\tilde{v}_{ij})=(0,0), \ i \ge 2$ is equivalent to the proof by mathematical induction in \cite[Theorem 2]{tang2021formation}. 

\subsection{Reactive multi-vehicle collision avoidance}%Collision Avoidance Augmentation}
To ensure safe navigation and prevent collisions between vehicles, we propose an additive term to the bearing formation controller, leveraging the constructive barrier feedback proposed in \cite{collision}
\begin{equation}
	u_i = u_i^{b} + u_i^{c},
\end{equation}
%we augment the control law with a collision avoidance term adapted from \cite{collision}. This term introduces an additional virtual acceleration input
with
\begin{equation}
	u_i^{c} = \sum_{j\in \mathcal N_i} k_{o,i}\gamma(d_{ij}) g_{ij}g_{ij}^\top v_{ij},
\end{equation}
where $d_{ij}=||p_{ij}|| - r$ and $r\in \mathbb{R}^{+}$ is the safety margin between neighboring agents and the collision avoidance effect is only active when two vehicles are close enough such that $\gamma(.)$ is chosen as
\begin{equation}
\gamma(z)=\left\{ \begin{aligned}
    0, &\ \ z\in (\epsilon_1,\infty)\\
    \phi(z) \cdot z^{-1}, & \ \ z\in [\epsilon_2,\epsilon_1]\\
    z^{-1}, &\ \ z\in (0,\epsilon_2)
\end{aligned}\right.,
\end{equation}
where $0<\epsilon_1<\epsilon_2$ and $\phi$ is a smooth function such that $\gamma$ is a continuously differentiable function for all $z\in(0,\infty)$. In this work, we use $\phi(z)=\frac{1}{2}-\frac{1}{2}\cos(\pi\frac{z-\epsilon_1}{\epsilon_2-\epsilon_1})$.
The constructive barrier feedback $u_i^c$ decreases the relative velocity in the direction of neighboring quadrotors without compromising
the performance of the nominal controller.
%denotes a desired constant safety distance between agents. %Note that this shaping term acts as a repulsive force, guiding each agent away from its neighbors, but also allowing for violating the desired BPE formation if they approach the safety distance threshold.

% Combining the nominal desired acceleration generated by the BPE formation tracking controller $u_i^{b}$ with the collision avoidance term $u_i^{c}$, the combined desired acceleration input for each vehicle is given by
% \begin{equation}
% 	u_i = u_i^{b} + u_i^{c}.
% \end{equation}
