%%%%%%%%%%%%%%%%%%%%%%%%%%%%%%%%%%%%%%%%%%%%%%%%%%%%%%%%%%%%%%%%%%%%%%%%%%%%%%%%
%2345678901234567890123456789012345678901234567890123456789012345678901234567890
%        1         2         3         4         5         6         7         8

% \documentclass[letterpaper, 10 pt, conference]{ieeeconf}  % Comment this line out if you need a4paper

\documentclass[letter, 10pt, conference, hidelinks]{ieeeconf}      % Use this line for a4 paper

\IEEEoverridecommandlockouts                              % This command is only needed if 
                                                          % you want to use the \thanks command

\overrideIEEEmargins                                      % Needed to meet printer requirements.

%In case you encounter the following error:
%Error 1010 The PDF file may be corrupt (unable to open PDF file) OR
%Error 1000 An error occurred while parsing a contents stream. Unable to analyze the PDF file.
%This is a known problem with pdfLaTeX conversion filter. The file cannot be opened with acrobat reader
%Please use one of the alternatives below to circumvent this error by uncommenting one or the other
%\pdfobjcompresslevel=0
%\pdfminorversion=4

% See the \addtolength command later in the file to balance the column lengths
% on the last page of the document

% The following packages can be found on http:\\www.ctan.org
\usepackage{graphicx} % for pdf, bitmapped graphics files
%\usepackage{epsfig} % for postscript graphics files
%\usepackage{mathptmx} % assumes new font selection scheme installed
%\usepackage{times} % assumes new font selection scheme installed
\usepackage{amsmath} % assumes amsmath package installed
\interdisplaylinepenalty=2500
\usepackage{amssymb}  % assumes amsmath package installed
\usepackage{hyperref}
\usepackage{soul}
\usepackage{float}

\let\proof\relax \let\endproof\relax \usepackage{amsthm}
%%%%%%%%%%%%%%%% DEFINITIONS %%%%%%%%%%%%%%%%
\theoremstyle{definition}
\newtheorem{defi}{Definition}
%%%%%%%%%%%%%%%% ASSUMPTIONS %%%%%%%%%%%%%%%%
\theoremstyle{definition}
\newtheorem{asu}{Assumption}
%%%%%%%%%%%%%%%% THEOREMS %%%%%%%%%%%%%%%%
\theoremstyle{plain}
\newtheorem{theorem}{Theorem}
%%%%%%%%%%%%%%%% PROOFS %%%%%%%%%%%%%%%%
\theoremstyle{definition}
%\newtheorem*{pro}{Proof}
%%%%%%%%%%%%%%%% REMARKS %%%%%%%%%%%%%%%%
\theoremstyle{definition}
\newtheorem{rema}{Remark}
%%%%%%%%%%%%%%%% PROBLEMS %%%%%%%%%%%%%%%%
\theoremstyle{definition}
\newtheorem{prob}{Problem}

\newtheorem{remark}{Remark}

%%%%%%%%%%%%%%%% TEMPORARY!!! %%%%%%%%%%%%%%%%
\usepackage{xcolor}
% \newcommand{\lt}[1]{{\color{blue} #1}}

%\usepackage[dvipsnames]{xcolor}
%\newcommand{\review}[1]{\textcolor{red}{#1}}
\usepackage{easyReview}

\title{\LARGE \bf
Leader-Follower Formation Tracking Control of Quadrotor UAVs\\Using Bearing Measurements 
}


\author{Sander Doodeman$^{1}$, Zhiqi Tang$^{2}$, Marcelo Jacinto$^{3}$, Rita Cunha$^{3}$, and Carlos Silvestre$^{3,4}$% <-this % stops a space
% \thanks{*This work was not supported by any organization}% <-this % stops a space
\thanks{$^{1}$Department of Mechanical Engineering, TU/e, Eindhoven University of Technology, Eindhoven, The Netherlands. E-mail:
        {\tt\small s.doodeman@tue.nl}}%
\thanks{$^{2}$Division of Decision and Control Systems, KTH Royal Institute of Technology, Sweden. E-mail:
        {\tt\small ztang2@kth.se}} %
\thanks{$^{3}$Institute for Systems and Robotics (ISR), Laboratory of Robotics and Engineering Systems (LARSyS), Instituto Superior Técnico, University of Lisbon, Portugal. E-mails:
        {\tt\small \{mjacinto, rita\}@isr.tecnico.ulisboa.pt }} %
\thanks{$^{4}$Department of Electrical and Computer Engineering of the Faculty of Science and Technology of the University of Macau, China. E-mail:
        {\tt\small csilvestre@umac.mo}} %
\thanks{The work of M. Jacinto was supported by the PhD Grant 2022.09587.BD from Funda\c{c}{\~a}o para a Ci{\^e}ncia e a Tecnologia
	(FCT), Portugal.}
}




\begin{document}

\maketitle
\thispagestyle{empty}
\pagestyle{empty}


%%%%%%%%%%%%%%%%%%%%%%%%%%%%%%%%%%%%%%%%%%%%%%%%%%%%%%%%%%%%%%%%%%%%%%%%%%%%%%%%
\begin{abstract}

To develop generalizable models in multi-agent reinforcement learning, recent approaches have been devoted to discovering task-independent skills for each agent, which generalize across tasks and facilitate agents' cooperation. However, particularly in partially observed settings, such approaches struggle with sample efficiency and generalization capabilities due to two primary challenges: (a) How to incorporate global states into coordinating the skills of different agents? (b) How to learn generalizable and consistent skill semantics when each agent only receives partial observations? To address these challenges, we propose a framework called \textbf{M}asked \textbf{A}utoencoders for \textbf{M}ulti-\textbf{A}gent \textbf{R}einforcement \textbf{L}earning (MA2RL), which encourages agents to infer unobserved entities by reconstructing entity-states from the entity perspective. The entity perspective helps MA2RL generalize to diverse tasks with varying agent numbers and action spaces. Specifically, we treat local entity-observations as masked contexts of the global entity-states, and MA2RL can infer the latent representation of dynamically masked entities, facilitating the assignment of task-independent skills and the learning of skill semantics. Extensive experiments demonstrate that MA2RL achieves significant improvements relative to state-of-the-art approaches, demonstrating extraordinary performance, remarkable zero-shot generalization capabilities and advantageous transferability.

 % Additional rewards transform the original MTRL problem into a multi-objective MTRL problem, and the coupling relationship between the outputs of SP and ACP further complicates the optimization process. To solve this challenge, TSAC assigns a virtual expected budget to convert the multi-objective MTRL into a constrained single-objective formulation and then employs the Lagrangian method to transform a constrained single-objective optimization into an unconstrained one. The multiplier in the Lagrangian method automatically adjusts the weights during the training process, promoting cooperation between SP and ACP.
\end{abstract}
\begin{IEEEImpStatement}
The Current policies trained by Multi-Agent Reinforcement Learning (MARL) predominantly rely on meticulously designed structured environments, which considerably constrain the agents' generalization capabilities across multitasking and cross-task skill reuse. In this paper, we design a novel masked autoencoders for MARL to coordinate the skills of different agents and learn generalizable and consistent skill semantics when each agent only receives partial observations. Experimental results demonstrate that our proposed MA2RL framework significantly enhances both the asymptotic performance and generalization capabilities of the generalizable models. Specifically, MA2RL introduces masked autoencoders tailored for MARL, aimed at enhancing generalizable models. The framework holds promise for inspiring further explorations into the generalization of multi-agent reinforcement learning.
\end{IEEEImpStatement}


% Note that keywords are not normally used for peerreview papers.
\begin{IEEEkeywords}
Multi-Agent reinforcement learning, generalization, self-supervised learning.
\end{IEEEkeywords}


\IEEEpeerreviewmaketitle

%%%%%%%%%%%%%%%%%%%%%%%%%%%%%%%%%%%%%%%%%%%%%%%%%%%%%%%%%%%%%%%%%%%%%%%%%%%%%%%%
\section*{Supplementary Material}
\noindent\textbf{Simulation Video:} \href{https://youtu.be/zqNK-d2ZgY0}{\texttt{youtu.be/zqNK-d2ZgY0}} \\
\noindent\textbf{Experiments Video:} \href{https://youtu.be/-cPlcVHDzzU}{\texttt{youtu.be/-cPlcVHDzzU}} \\
\noindent\textbf{Code:} \href{https://github.com/SDoodeman/bpe_quadrotor}{\texttt{github.com/SDoodeman/bpe\_quadrotor}}

%%%%%%%%%%%%%%%%%%%%%%%%%%%%%%%%%%%%%%%%%%%%%%%%%%%%%%%%%%%%%%%%%%%%%%%%%%%%%%%%
% 
% 
The widespread integration of communication networks and smart devices in modern control systems has increased the vulnerability of industrial systems to online cyber-attacks, e.g., Industroyer, Blackenergy, etc \citep{osti_1505628}.
% Modern control systems have seen a large push to include communication networks and smart devices to increase performance, made possible by improvements in communication device cost and energy consumption. This trend has been coupled with the usage of open-standard communication protocols among industrial control systems, making them vulnerable to online cyber-attacks such as Industroyer, Blackenergy, etc \citep{osti_1505628}. 
To counter this, methods have been developed to improve security by achieving attack detection, mitigation, and monitoring, among others \citep{sandberg2022secure}. This paper focuses on active attack diagnosis to mitigate stealthy attacks. 
%
%\subsection{Literature review}

Active diagnosis techniques rely on the inclusion of additional moduli to control systems
% inclusion within the control system of additional moduli 
to alter the behavior of the system compared to information known by the attacker. 
For instance, the concept of additive watermarking was introduced in \cite{mo2015physical}, where noise signals of known mean and variance are added at the plant and compensated for it at the controller. 
This compensation, however, is not exact, causing some performance degradation. Thus, trade-offs between performance and detectability  are necessary \citep{zhu2023detection}.
% A later work \citep{zhu2023detection} designs the watermark signal by trading performance for detection. Thus, although additive watermarking serves as a good detection scheme, they endure performance losses even in the nominal case. 

In encrypted control \citep{darup2021encrypted}, the sensor data is encrypted, sent to the controller, and then operated on directly. Encrypted input signals are sent back to the plant for decryption. Although encryption is widespread in IT security, in control systems it presents some concerns, such as the introduction of time delays \citep{stabile2024verifiable}, while it may present inherent weaknesses \citep{alisic2023model}.
% they are not preferred as they introduce time delays \citep{stabile2024verifiable} which can cause instability, and some encryption schemes can be very weak  \citep{alisic2023model}. 

In moving target defense \citep{griffioen2020moving}, the plant is augmented with fictitious dynamics, known to the controller. The plant output is transmitted to the controller along with the fictitious states over a network under attack. 
The additional measurements then aide in the detection of attacks. 
This comes at the cost of higher communication bandwidth needs, which increases rapidly with the dimension of the augmented systems.
% Since the dynamics of the fictitious dynamics are exactly known to the controller, the attack is detected easily. However, when the scale of the system increases, the communication bandwidth used by moving the target defense approach increases rapidly. 

Other recently proposed works include two-way coding \citep{fang2019two}, a weak encryuption technique, and dynamic masking \citep{abdalmoaty2023privacy}, which enhances privacy as well as security, have been shown to be effective against zero-dynamics attacks.
% Two-way coding \citep{fang2019two} and dynamic masking \citep{abdalmoaty2023privacy} are other recently proposed approaches. Two-way coding is another form of weak encryption technique whilst dynamic masking proposes an architecture that enhances both privacy and security. These schemes are shown to be effective against zero dynamics attacks but remain to be studied for other classes of attacks. 
% Recent extensions include \citep{mukherjee2021secure,ramos2024privacy}.
% Some other works which are related are \citep{mukherjee2021secure}, an extension of \cite{fang2019two}. The work \citep{ramos2024privacy} is an extension of moving target defense for multi-agent systems. 
Furthermore, filtering techniques for attack detection are proposed by \cite{murguia2020security,hashemi2022codesign,escudero2023safety}, while not focusing on stealthy attacks.
% The works \citep{murguia2020security,hashemi2022codesign,escudero2023safety} develop filtering techniques to guarantee safety, without being focused on stealthy covert attacks.

Multiplicative watermarking (mWM) has been proposed by the authors as a diagnosis technique \citep{ferrari2020switching}. mWM consists of a pair of filters on each communication channel between the plant and its controller; the scheme is affine to weak encryption, whereby ``encoding'' and ``decoding'' are done by changing signals' dynamic characteristics through inverse pairs of filters. This enables original signals to be recovered exactly, and thus does not lead to performance degradation.
% A multiplicative watermark is an affine to a weak encryption technique, through which the signal is ``encoded'' by a filter, changing its dynamic behavior. The use of inverse pairs means that the original signal can be recovered, through ``decoding'' via an inverse filter. As such, differently to techniques based on additive watermarking, no performance is lost due to the injection of noise, and there are no bandwidth limitations.

%\subsection{Contributions}
One of the critical features of multiplicative watermarking is that to detect stealthy attacks, the mWM filter parameters must be switched over time. In this paper, an algorithm to optimally design the mWM parameters after a switching event is presented, enhancing detection performance, without changing the switching time.
% This is done without changing the switching time, which is taken as given.

\textcolor{black}{
To formalize the filter design problem, we suppose the defender is interested in optimal performance against adversaries injecting covert attacks with matched system parameters \citep{smith2015covert}, including the mWM parameters prior to the switch. This scenario represents a worst case where malicious agents can take full control of the system while remaining undetected.
Thus, the attack strategy is explicitly included within the formulation of the closed-loop system, and the mWM filters are chosen by solving an optimization problem minimizing the attack-energy-constrained output-to-output gain (AEC-OOG) \citep{anand2023risk}, a variation of the output-to-output gain proposed in  \cite{teixeira2015strategic}.
}
The main contributions of this paper are:
% We consider an adversary injecting a covert attack with matched system parameters \citep{smith2015covert}, i.e., an attacker with full knowledge of the control system parameters, including those of the mWM filters before the switch. This scenario is taken as a worst case, as it has been shown that this class of attacks can be made stealthy. To quantitatively define a cost, the output-to-output gain (OOG) \citep{teixeira2015strategic} is leveraged,
% a metric introduced to evaluate the impact of an additive attack in a control system. %Specifically, OOG evaluates the worst-case performance loss that an attacker injecting an undetectable attack can obtain. 
% Here, the maximum performance loss caused by a stealthy adversary with limited energy is taken, the attack-energy-constrained OOG (AEC-OOG) \citep{anand2023risk}. The main contributions of this paper are:
\begin{enumerate}
%[label=\alph*.]
\item The problem of optimally designing the switching mWM filters is formulated as an optimization problem, with the AEC-OOG is taken as the objective;%where the AEC-OOG is taken as the impact metric; 
\item The worst-case scenario of a covert attack with exact knowledge of plant and mWM filter parameters is embedded within the design problem;
% The optimization problem is defined to incorporate the worst-case scenario of a covert attack with exact knowledge of plant and mWM filter parameters;
\item The feasibility of the optimization problem is shown to be dependent only on stability conditions; 
\item A solution scheme is proposed to promote randomization of the mWM filter parameters such that an eavesdropping adversary cannot remain stealthy.
\end{enumerate} 

This builds on the results of \cite{ferrari2020switching}, where the focus was on the design of the switching protocols, rather than the parameters themselves.
Compared to previous work \citep{gallo2021design}, this paper introduces an optimization problem which is always feasible (thanks to the use of AEC-OOG in the objective), while also considering a more sophisticated class of covert attacks, where the presence of watermark is known to the adversary. 
Moreover, this paper poses a different objective than \citep{zhang2023hybrid}; indeed, while \citep{zhang2023hybrid} provided a design strategy to ensure certain privacy properties, in this paper we address the problem of optimal parameter design following a switching event.


%\subsection{Organization}
The rest of the paper is organized as follows. 
After formulating the problem in Section~\ref{sec:PF}, we propose our design algorithm in Section~\ref{sec:main}, and analyze its properties. It is then evaluated through a numerical example in Section~\ref{sec:NE}, and concluding remarks are given Section~\ref{sec:Con}.
% We provide the problem background in Section~\ref{sec:PF}. We formulate the design problem in Section~\ref{sec:main}, together with an analysis of its properties. The proposed algorithm is evaluated through a numerical example in Section \ref{sec:NE}. Concluding remarks are offered in Section \ref{sec:Con}.

%%%%%%%%%%%%%%%%%%%%%%%%%%%%%%%%%%%%%%%%%%%%%%%%%%%%%%%%%%%%%%%%%%%%%%%%%%%%%%%%
In this section,  we introduce related concepts including GNN for recommendation, negative sampling, and Mixup \cite{mixup}.

\subsection{Graph Neural Networks for Recommendation}
 Recommendation is the most important technology in many e-commerce platforms, which has evolved from collaborative filtering to graph-based models. Graph-based recommendation represents all users and items by embedding and recommending items with maximum similarity score (by a inner product operation) for a given user. Here, we briefly describe the pipeline of GNN-based representation learning, including aggregation and optimization with negative sampling.

GNNs learn distributed vectors of nodes by leveraging node features and the graph structure. 
The neighborhood aggregation follows the ``message passing'' mechanism, which iteratively updates a node's embedding $h$ by aggregating the embeddings of its neighbors. Formally, the embedding $h_i^l$ of node $i$ in the $l$-th layer of GNN is defined as:
% GNNs use graph structures and node features to learn distributed vectors to represent graph information. Learning follows the "message passing" mechanism of neighborhood aggregation by iteratively updating a node's embedding $h$ by aggregating the embeddings of its neighbors. Formally, the representation $h_k^i$ of node $i$ in the $k$th layer of GNN is defined as:
 \begin{equation} \label{agg}
     \small
     h_i^l= \sigma\left(\text{AGG}\left(h_{i}^{l-1}, h_{j}^{l-1} \mid j \in N_{(i)},W_l\right)\right),
 \end{equation}
where the \(\sigma\) is activation function, $W_l$ denotes the trainable weights at layer l, $N_{(i)}$ denotes all nodes adjacent to $i$, $\text{AGG}$ is an aggregation function implemented by specific GNN model (\eg GraphSAGE, GCN, GAT, \etc), and $h_i^0$ is typically initialized as the input node feature $v_i$.

 
\subsection{Negative Sampling}

Negative sampling \cite{negsamp} is firstly proposed to serve as a simplified version of Noise Contrastive Estimation\cite{NCE}, which is an efficient way to compute the partition function of an unnormalized distribution to accelerate the training of Word2Vec\cite{word2vec}. The GNN has different non-Euclidean encoder layers with the following negative sampling objective:

\begin{equation}\label{negsampling}
    \mathcal{L} = \log(\sigma (e_{v_i}^Te_{v_p}))+\sum^{c}_{j=1}\mathbb{E}_{v_j\sim P_n(v)}  \log(1-\sigma (e_{v_i}^Te_{v_j})),
\end{equation}
where $v_i$ is a node in the graph, $v_p$ is sampled from the positive distribution of node $v_i$, $v_j$ is sampled from the negative distribution of node $v_i$, $e$ represents the embedding of the node, $\sigma$ represents the sigmoid function, $c$ represents the number of negative samples for each positive sample pair. 
%So Negative Sampling are free to simplify NCE as long as the vector representations retain their quality, it is an effective method to calculate the partition function of unnormalized distribution.


\subsection{Mixup}


\textbf{Mixup\cite{mixup}} is an simple yet effective data augmentation method that is originally proposed for image classification tasks. 
Mathematically, let $(x, y)$ denotes a sample of training data, where $x$ is the raw input samples and $y$ represents the one-hot label of $x$, the Mixup generates synthetic training samples $(\tilde{x}, \tilde{y})$  as follows:
\begin{equation}
% \vspace{-0.2cm}
\begin{split}
% \setlength\abovedisplayskip{0cm}
& \tilde{x}=\lambda x_{i}+(1-\lambda) x_{j}, \\
% \setlength\belowdisplayskip{0cm}
% \setlength\abovedisplayskip{0cm}
& \tilde{y}=\lambda y_{i}+(1-\lambda) y_{j}. \\
% \setlength\belowdisplayskip{0cm}
\end{split}
% \vspace{-0.2cm}
\end{equation}
It generates new samples by using linear interpolations to mix different images and their labels.

%%%%%%%%%%%%%%%%%%%%%%%%%%%%%%%%%%%%%%%%%%%%%%%%%%%%%%%%%%%%%%%%%%%%%%%%%%%%%%%%
\section{Modeling and problem formulation} \label{sec:modeling}
Consider the problem of formation tracking control of a group of $n$ quadrotor vehicles. For each quadrotor $i$, let $p_i\in \mathbb R^3$ and $v_i\in \mathbb R^3$ denote its position and velocity, respectively, expressed in a common inertial frame $\{\mathcal I\}$ that follows a north-east-down (NED) convention. Let $\{\mathcal B_i\}$ be a body-fixed frame, attached to the $i$th quadrotor, following a front-right-down (FRD) convention, as shown in Fig. \ref{fig:coordinates}.
% \vspace{-0.1cm}
\begin{figure}[t]
        \vspace{0.2cm}
	\centering
	\includegraphics[width=0.40\textwidth]{NewFigures/ReferenceFrames.pdf}
	% \vspace{-0.15cm}
	\caption{Schematic representation of the FRD vehicle body frame $\{B\}$, relative to an NED inertial frame $\{I\}$.}
	\label{fig:coordinates}
    % \vspace{-0.1cm}
\end{figure}
% \vspace{-0.2cm}
Let $R_i\in SO(3)$ denote the attitude $\{\mathcal B_i\}$ with respect to $\{\mathcal I\}$. The dynamics of quadrotor $i$ are given by
\begin{subequations}
    \label{eq:trans_dyn}
    \begin{align}
        \dot{p}_i &= v_i, \label{eq:dotp}\\
        m_i \dot{v}_i &= - T_i R_i e_3 + m_i g e_3, \label{eq:dotv} \\
         \dot{R}_i &= R_i {[\Omega_i]}_\times,\label{eq:dotR}
    \end{align}
\end{subequations}
where $g \approx 9.81\text{ms}^{-2}$ is the gravitational acceleration, $e_3 = \begin{bmatrix} 0 & 0 & 1\end{bmatrix}^\top$, $T_i\in \mathbb R_0^{+}$ is the total thrust magnitude, $m_i\in \mathbb R^{+}$ is the mass and $\Omega_i\in \mathbb R^3$ denotes the angular velocity input of agent $i$ expressed in $\{\mathcal B_i\}$.

Define the relative position vector $p_{ij}$ and relative velocity vector $v_{ij}$ between vehicle $i$ and its neighbor vehicle $j$ as 
$$p_{ij}=p_j-p_i, \quad v_{ij}=v_j-v_i, \quad \forall j\in \mathcal N_i.$$
As long as $\|p_{ij}\|\ne 0$, the relative bearing $g_{ij}$ of agent $i$ to agent $j$ is defined by
\begin{eqnarray}
    g_{ij} := \frac{p_{ij}}{\lVert p_{ij}\rVert} \in \mathbb S^2. \label{eq:bearing}
\end{eqnarray}
 Let the stacked vector $\boldsymbol{p}=[p_1^\top,...,p_n^\top]^\top\in \mathbb{R}^{3n}$ denote the configuration of $\mathcal{G}$ and the digraph $\mathcal{G}$ together with the configuration $\boldsymbol{p}$ define a formation $\mathcal{G}(\boldsymbol p)$ in three-dimensional space. We now introduce the first assumption related to the sensing graph topology.

\begin{asu} \label{asu:topology} 
    The sensing topology of the formation $\mathcal G (\boldsymbol p)$ is described as a leader-follower structure, i.e., an acyclic digraph $\mathcal{G}(\mathcal{V}, \mathcal{E})$ that has a single directed spanning tree. 
    Without loss of generality, agents are numbered (or can be renumbered) such that agent $1$ is the leader, i.e.  $\mathcal{N}_1= \varnothing$,  all other agents $i, \ i\ge 2$ are followers whose neighboring set is $\mathcal{N}_i  \subseteq \{1, \ldots, i-1\}$, according to the examples in Fig. \ref{fig:BPE}.
    Each agent $i \geq 2$ can measure the relative bearing vectors $g_{ij}$ and relative velocities $v_{ij}$ to its neighbors $j \in \mathcal{N}_i$, as well as its own attitude $R_i$. % represented by the rotation matrix
\end{asu}

Given this leader-follower structure, we introduce the definition of a BPE formation \cite{tang2021formation} which will be used later to define desired trajectories.
\begin{defi} \label{def:BPE}
    A leader–follower formation $\mathcal{G}(\boldsymbol p(t))$ is called BPE, if $\forall i \in \mathcal{V}$, the matrices $\sum_{j\in \mathcal{N}_i} \pi_{g_{ij}(t)}$ satisfy the PE condition \eqref{eq:PEmatrix}.
\end{defi}
Note that if $\sum_{j\in \mathcal{N}_i} \pi_{g_{ij}(t)}$ is PE, then agent $i$ has at least one bearing measurement $g_{ij}$ that is time-varying or at least two bearings that are non-collinear \cite[Lemma 1]{tang2021formation}. We now define the assumptions for the desired trajectories.
\begin{asu} \label{asu:boundedBPE}
    The desired velocity $v_{i}^*(t)$, the desired acceleration $u_i^*(t)$, and the desired jerk $\dot{u}_i^*(t)$ are bounded for all $t > 0, i \in \mathcal V$, and such that the resulting desired bearings $g_{ij}^*(t), (i,j)\in \mathcal E$ are well-defined and the desired formation is BPE for all $t > 0$.
\end{asu}

Let us now present the considered problem formulation.
\begin{prob} \label{prob:statement}
Design distributed formation tracking controllers for all follower vehicles ($i\ge 2$) such that a group of $n\ (n\ge2)$ quadrotor vehicles successfully tracks a desired BPE formation under Assumptions \ref{asu:topology}-\ref{asu:boundedBPE}, while avoiding inter-agent collisions.
\end{prob}

%%%%%%%%%%%%%%%%%%%%%%%%%%%%%%%%%%%%%%%%%%%%%%%%%%%%%%%%%%%%%%%%%%%%%%%%%%%%%%%%
\section{Hierarchical control structure} \label{sec:controlstrategy}
In this section, a hierarchical control structure (see Fig. \ref{fig:controllayout}) is proposed, assuming a time-scale separation between the translational and orientation dynamics \cite{Bertrand2011,herisse12}. Expanding the system dynamics \eqref{eq:dotv} with the desired attitude terms yields
\begin{equation}
	m_i\dot{v}_i = -T_i (R_ie_3- R_i^* e_3) - T_i R_i^* e_3 + m_i g e_3,
\end{equation}
where $-T_iR_i^* e_3$ is a total desired force. Provided that an inner-loop attitude controller is designed to provide a fast enough convergence, the time-scale separation allows for neglecting the attitude error in the outer-loop dynamics, allowing the system to be given by  
\begin{equation}
	\dot{v}_i \approx \underbrace{- \frac{T_i}{m_i} r_{3,i}^* + g e_3}_{u_i},
	\label{eqn:double_integrator_model}
\end{equation}
where $u_i \in \mathbb{R}^3$ is a virtual acceleration input to be designed and $r_{3,i}^*:=R_i^* e_3$ is the desired z-axis of the vehicle. The total thrust $T_i$ applied to each vehicle can be computed according to
\begin{equation}
	T_i := m_i||u_i - g e_3||.
\end{equation}
%\subsection{Attitude Controller Design}
The desired z-axis of each vehicle is set as
\begin{equation}
	r_{3,i}^* := \frac{u_i - g e_3}{||u_i - g e_3||},
\end{equation}
from which $R_i^*\in SO(3)$ can be obtained by complementing $r_{3,i}^*$ with a desired yaw angle. Finally, the angular velocity used as input to the vehicle, adopted from \cite{Tang2015}, is given by
\begin{equation}\label{eq:Omega}
	\Omega_i := n_i [e_3]_{\times} R_i^\top r_{3,i}^* + \frac{m_i}{T_i} \pi_{e_3} R_i^\top [r_{3,i}^*]_{\times}\dot{u}_i,
    %\Omega_i := -n_i R_i^\top [r_{3,i}^*]_{\times} r_{3,i} -\frac{m_i}{T_i} R_i^\top [r_{3,i}^*]_{\times}\pi_{r_{3,1}^*}\dot{u}_i,
\end{equation}
where $n_i \in \mathbb{R}^{+}$ is a controller gain and $ r_{3,i}$ is the third column of $R_i$. 
\begin{figure}%[H]
	\centering
	\includegraphics[width=0.99\linewidth]{NewFigures/Inner_outer_loop.pdf}
    \vspace{-0.4cm}
	\caption{Hierarchical control structure for each quadrotor $i$.}
	\label{fig:controllayout}
    % \vspace{-0.1cm}
\end{figure}

% \begin{remark}
% 	For the implementation of this high-gain inner-loop attitude controller, it is assumed that the feedforward term $\dot{u}_i\approx\dot{u}_i^*$ to obtain zero tracking error when the trajectories of the desired formation are time-varying. As such, the desired trajectories are assumed to be thrice differentiable and have a desired jerk $\dot u_i^*$ .
% \end{remark}

%%%%%%%%%%%%%%%%%%%%%%%%%%%%%%%%%%%%%%%%%%%%%%%%%%%%%%%%%%%%%%%%%%%%%%%%%%%%%%%%
\subsection{Bearing-based Formation Tracking Control} \label{sec:BPEcontrol}
Denote $\tilde{p}_{ij} = (p_{j} - p_{i}) - (p_{j}^* - p_{i}^*)$ and $\tilde{v}_{ij} = (v_{j} - v_{i}) - (v_{j}^* - v_{i}^*)$ as the respective relative position and velocity error $\forall i\ge 2$. The distributed bearing formation controller is defined as
\begin{equation}
    u_i^{b} = \sum_{j \in \mathcal{N}_i}( -{k_{p,i}} \pi_{g_{ij}} p_{ij}^*) - {k_{d,i}} \tilde{v}_{ij} + u_i^*, \ i\ge 2,
    \label{eq:BPElaw}
\end{equation}
where $u_i^{*} \in \mathbb{R}^{3}$ is a desired acceleration to be tracked, and $k_{p,i}$ and $k_{d,i}$ are positive scalar gains such that $k_{d,i} > 1$ and $k_{p,i} < \frac{4}{N_i} - \frac{4}{k_{d,i}^2 N_i^3}$.
\begin{theorem} \label{th:formation}
Consider a system with $n$ ($n\geq2$) quadrotor UAVs. For all agents $i\ge 2$, consider the closed-loop system \eqref{eqn:double_integrator_model} along with the proposed controller \eqref{eq:BPElaw}, such that $u_i = u_i^b$. If Assumptions \ref{asu:topology} and \ref{asu:boundedBPE} are satisfied, the equilibrium point $(\tilde{p}_{ij},\tilde{v}_{ij})=(0,0), \ i \ge 2$ is uniformly exponentially (UE) stable.
\end{theorem}
Considering the assumption in (\ref{eqn:double_integrator_model}), the UE stabilization of the equilibrium point $(\tilde{p}_{ij},\tilde{v}_{ij})=(0,0), \ i \ge 2$ is equivalent to the proof by mathematical induction in \cite[Theorem 2]{tang2021formation}. 

\subsection{Reactive multi-vehicle collision avoidance}%Collision Avoidance Augmentation}
To ensure safe navigation and prevent collisions between vehicles, we propose an additive term to the bearing formation controller, leveraging the constructive barrier feedback proposed in \cite{collision}
\begin{equation}
	u_i = u_i^{b} + u_i^{c},
\end{equation}
%we augment the control law with a collision avoidance term adapted from \cite{collision}. This term introduces an additional virtual acceleration input
with
\begin{equation}
	u_i^{c} = \sum_{j\in \mathcal N_i} k_{o,i}\gamma(d_{ij}) g_{ij}g_{ij}^\top v_{ij},
\end{equation}
where $d_{ij}=||p_{ij}|| - r$ and $r\in \mathbb{R}^{+}$ is the safety margin between neighboring agents and the collision avoidance effect is only active when two vehicles are close enough such that $\gamma(.)$ is chosen as
\begin{equation}
\gamma(z)=\left\{ \begin{aligned}
    0, &\ \ z\in (\epsilon_1,\infty)\\
    \phi(z) \cdot z^{-1}, & \ \ z\in [\epsilon_2,\epsilon_1]\\
    z^{-1}, &\ \ z\in (0,\epsilon_2)
\end{aligned}\right.,
\end{equation}
where $0<\epsilon_1<\epsilon_2$ and $\phi$ is a smooth function such that $\gamma$ is a continuously differentiable function for all $z\in(0,\infty)$. In this work, we use $\phi(z)=\frac{1}{2}-\frac{1}{2}\cos(\pi\frac{z-\epsilon_1}{\epsilon_2-\epsilon_1})$.
The constructive barrier feedback $u_i^c$ decreases the relative velocity in the direction of neighboring quadrotors without compromising
the performance of the nominal controller.
%denotes a desired constant safety distance between agents. %Note that this shaping term acts as a repulsive force, guiding each agent away from its neighbors, but also allowing for violating the desired BPE formation if they approach the safety distance threshold.

% Combining the nominal desired acceleration generated by the BPE formation tracking controller $u_i^{b}$ with the collision avoidance term $u_i^{c}$, the combined desired acceleration input for each vehicle is given by
% \begin{equation}
% 	u_i = u_i^{b} + u_i^{c}.
% \end{equation}


%%%%%%%%%%%%%%%%%%%%%%%%%%%%%%%%%%%%%%%%%%%%%%%%%%%%%%%%%%%%%%%%%%%%%%%%%%%%%%%%
\section{Simulation results} \label{sec:simulations}
To validate the proposed control laws, in this section two MATLAB simulation results are provided, demonstrating the successful tracking of time-varying formations by a group of four quadrotor vehicles. The control gains adopted for both scenarios are $k_{p,i}=1.9$ and $k_{d,i}=3$, and $n_i=20, $ $\forall i\ge 2$. In the first scenario, the desired formation is chosen to have a time-varying shape while simultaneously translating along the $y$-axis. Specifically, we choose $p_1^*{=}\begin{bmatrix} 1 ~ \frac{1}{5}t ~ 1\end{bmatrix}^\top$, $p_2^*{=}\begin{bmatrix} -1{-}\frac{3}{4}\sin(t) \ \ \frac{1}{5}t \ \ 1{+}\frac{3}{4}\sin(t) \end{bmatrix}^\top$, $p_3^*{=}\begin{bmatrix} {-}1 ~ \frac{1}{5}t 
 ~ {-}1\end{bmatrix}^\top$, $p_4^*{=}\begin{bmatrix} 1 ~ \frac{1}{5}t ~ {-}1\end{bmatrix}^\top$. The three-dimensional trajectories of the formation are shown in Fig. \ref{fig:simvarying} and the evolution of state errors converging to the origin is shown in Fig. \ref{fig:simerrvarying}. The topology is described as $\mathcal{N}_2 = \{1\}$, $\mathcal{N}_3 = \{2\}$, and $\mathcal{N}_4 = \{2, 3\}$.
\begin{figure}[p]
 	\centering
 	\includegraphics[width=0.99\linewidth]{NewFigures/SimVarying.pdf}
        \vspace{-0.4cm}
 	\caption{Simulated trajectories of 4-agent formation with time-varying shape. The dashed lines represent the desired trajectories, and the solid lines the simulated trajectories. The solid black arrows indicate connections between agents.}
 	\label{fig:simvarying}
\end{figure}
\begin{figure}%[H]
	\centering
	\includegraphics[width=0.99\linewidth]{NewFigures/SimErrorVarying.pdf}
    \vspace{-0.4cm}
	\caption{Absolute position, velocity, and rotation error of the follower agents in the 4-agent formation with time-varying shape.}
	\label{fig:simerrvarying}
\end{figure}

In the second scenario, the desired formation is chosen to be a rigid shape rotating around and translating along the $y$-axis. At the same time, to pass through a narrow window, the formation is able to change its scale accordingly, as shown in Fig. \ref{fig:simrescale}. The formation has a minimal leader-follower graph formed by a single directed path, i.e. each follower has only one neighbor such that $\mathcal{N}_i = \{i - 1\}, i\in \mathcal{V}\setminus\{1\}$. The desired trajectories are given by $p_1^*{=}\begin{bmatrix} 0 ~ \frac{2}{5}t ~ 0\end{bmatrix}^\top$, $p_i^* = p_1^* + R_y(\frac{1}{2}t)d_i$ for $i\in\{2,3,4\}$, $d_2 = (1{+}|\frac{t-40}{20}| ) e_3$, $d_3 = R_y({-}\frac{2}{3}\pi) d_2$, $d_4 = R_y({-}\frac{4}{3}\pi) d_2$, where $R_y(\theta)$ is the rotation matrix around the $y$-axis.
The performance for this setup is shown in Fig. \ref{fig:simerrrescale}. The cascaded nature of the system is clearly visible, with the slowest convergence for agent $4$.
\begin{figure}%[H]
	\centering
	\includegraphics[width=0.99\linewidth]{NewFigures/SimErrorRescale.pdf}
    \vspace{-0.4cm}
	\caption{Absolute position and velocity error of the follower agents for the 4-agent formation with rigid shape, but a varying scaling factor.}
	\label{fig:simerrrescale}
\end{figure}
\begin{figure}
    \vspace{0.2cm}
    \centering
    \includegraphics[width=0.99\linewidth]{NewFigures/SimRescale.pdf}
    \vspace{-0.25cm}
    \caption{Simulated trajectories of 4-agent formation with rotating rigid shape, but a varying scaling factor. Three instances of the formation are shown, the first one (open circles) represents the initial positions, the second one at $t=40$ (solid circles) represents the point where the radius is at the smallest, after which the agents follow the desired trajectory until $t=80$ where the final position (solid circles) is shown. The dashed lines represent the desired trajectory of agent 4, and the solid yellow line is its simulated trajectory.}
    \label{fig:simrescale}
\end{figure}
The numerical simulations indicate an effective performance of the proposed controllers and a clear convergence of the formation tracking errors under a variety of desired formation trajectories with more relaxed graph topologies compared to \cite{schiano2016rigidity} and  \cite{erskine2021model}. Remark that the chosen desired trajectory is one of the important factors for the convergence rate as indicated in \cite[Theorem 3]{tang2021formation}. For instance, Fig. \ref{fig:simerrrescale} shows a faster convergence than Fig. \ref{fig:simerrvarying} even though the same control gains are used.


%%%%%%%%%%%%%%%%%%%%%%%%%%%%%%%%%%%%%%%%%%%%%%%%%%%%%%%%%%%%%%%%%%%%%%%%%%%%%%%%
In this section, we empirically compare the proposed algorithm on both sequence windows and time windows with existing methods.
\paragraph{Datasets} For the sequence-based model, we used two synthetic datasets and two cross-language datasets. The statistics of the datasets are provided in Table \ref{table:statistics}:

\begin{table}[t]
    \centering
    \caption{The statistics of the datasets. The datasets satisfy $1 \leq \|\vx\|\|\vy\| \leq R $.}
    \label{table:statistics}
    \begin{tabular}{|c|c|c|c|c|c|}
    \hline
        Dataset & $n$ & $m_x$ & $m_y$ & $N$ & $R$ \\ \hline
        SYNTHETIC(1) & 100,000 & 1,000 & 2,000 & 50,000 & 65 \\ \hline
        SYNTHETIC(2) & 100,000 & 1,000 & 2,000 & 50,000 & 724 \\ \hline
        APR & 23,235 & 28,017 & 42,833 & 10,000 & 773 \\ \hline
        PAN11 & 88,977 & 5,121 & 9,959 & 10,000 & 5,548 \\ \hline
        EURO & 475,834 & 7,247 & 8,768 & 100,000 & 107,840 \\ \hline
    \end{tabular}
\end{table}

\begin{itemize}
    \item Synthetic: The elements of the two synthetic datasets are initially uniformly sampled from the range (0,1), then multiplied by a coefficient to adjust the maximum column squared norm $R$. The X matrix has 1,000 rows, and the Y matrix has 2,000 rows, each with 100,000 columns. The window size is set to 50,000.
    \item APR: The Amazon Product Reviews (APR) dataset is a publicly available collection containing product reviews and related information from the Amazon website. This dataset consists of millions of sentences in both English and French. We structured it into a review matrix where the X matrix has 28,017 rows, and the Y matrix has 42,833 rows, with both matrices sharing 23,235 columns. The window size is 10,000.
    \item PAN11: PANPC-11 (PAN11) is a dataset designed for text analysis, particularly for tasks such as plagiarism detection, author identification, and near-duplicate detection. The dataset includes texts in English and French. The X and Y matrices contain 5,121 and 9,959 rows, respectively, with both matrices having 88,977 columns. The window size is 10,000.
\end{itemize}
We evaluate the time-based model on another real-world dataset:
\begin{itemize}
    \item EURO: The Europarl (EURO) dataset is a widely used multilingual parallel corpus, comprising the proceedings of the European Parliament. We selected a subset of its English and French text portions. The X and Y matrices contain 7,247 and 8,768 rows, respectively, and both matrices share 475,834 columns. Timestamps are generated using the $Poisson$ $Arrival$ $Process$ with a rate parameter of $\lambda=2$. The window size is set to 100,000, with approximately 30,000 columns of data on average in each window.
\end{itemize}

\paragraph{Setup} For the sequence-based model, we compare the proposed hDS-COD and  aDS-COD with EH-COD~\cite{yao2024approximate} and DI-COD~\cite{yao2024approximate}. We do not consider the Sampling algorithm as a baseline, as its performance is inferior to that of EH-COD and DI-CID, as demonstrated in \cite{yao2024approximate}. %The hDS-COD is adjusted by the parameter $\ell$ and the maximum number of levels $L = \log{R}$, where $R$ is the prior estimate of the maximum squared column norm of the dataset. DI-COD similarly requires a prior estimate of $R$ to limit the maximum number of levels $L = \log{(R/\varepsilon})$. In contrast, aDS-COD and EH-COD do not require an estimate of $R$; their error-space balance is controlled by the parameter $\ell = \frac{1}{\varepsilon}$. 
For the time-based model, we compare the proposed hDS-COD and  aDS-COD with EH-COD and the Sampling algorithm since DI-COD cannot be applied to time-based sliding window model. To achieve the same error bound, the maximum number of levels for hDS-COD is set to $L = \log{(\varepsilon NR)}$, and the initial threshold for aDS-COD is set to $1$.

Our experiments aim to illustrate the trade-offs between space and approximation errors. The x-axis represents two metrics for space: final sketch size and total space cost. The final sketch size refers to the number of columns in the result sketches $\mA$ and $\mB$ generated by the algorithm, representing a compression ratio. The total space cost refers to the maximum space required during the algorithm's execution, measured by the number of columns.We evaluate the approximation performance of all algorithms based on correlation errors $\operatorname{corr-err}(\mathbf{X}_W \mathbf{Y}_W^\top, \mathbf{A} \mathbf{B}^\top)$, which is reflected on the y-axis. Every 1,000 iterations, all algorithms query the window and record the average and maximum errors across all sampled windows.

The experiments for all algorithms were conducted using MATLAB (R2023a), with all algorithms running on a Windows server equipped with 32GB of memory and a single processor of Intel i9-13900K.

\paragraph{Performance} Figure \ref{fig:error vs l} and Figure \ref{fig:error vs space} illustrate the space efficiency comparison of the algorithms on sequence-based datasets. Panels (a-d) show the average errors across all sampled windows, while panels (e-h) display the maximum errors.

Figure \ref{fig:error vs l} evaluates the compression effect of the final sketch. The hDS-COD, aDS-COD, and EH-COD show similar compression performances. But the DS series is more stable, particularly on the synthetic datasets, where they significantly outperform EH-COD and DI-COD. The performance of hDS-COD and aDS-COD is nearly the same, indicating that the adaptive threshold trick in aDS-COD does not have a noticeable negative impact on it, maintaining the same error as hDS-COD.

Figure \ref{fig:error vs space} measures the total space cost of the algorithms. hDS-COD and aDS-COD show a significant advantage over existing methods, as they can achieve the  $\varepsilon$-approximation error with much less space. For the same space cost, the correlation errors of hDS-COD and aDS-COD are much smaller than those of EH-COD and DI-COD. Also, aDS-COD has better space efficiency than hDS-COD because aDS only uses a single-level structure while hDS requires $\log R+1$ levels. We find that hDS-COD requires more space on  SYNTHETIC(2) dataset compared to SYNTHETIC(1) dataset. This phenomenon occurs because SYNTHETIC(2) dataset has a larger $R$, which confirms the dependence on $R$ as stated in Theorem~\ref{thm:hds}. 

Figure \ref{fig:time-based} compares the performance of algorithms on time-based windows. Panels (a) and (b) present the error against the final sketch size, which show that our aDS-COD and hDS-COD algorithms enjoy similar performance as EH-COD and significantly outperform the sampling algorithm. On the other hand, as shown in panels (c) and (d), our methods outperform baselines in terms of total space cost.


%%%%%%%%%%%%%%%%%%%%%%%%%%%%%%%%%%%%%%%%%%%%%%%%%%%%%%%%%%%%%%%%%%%%%%%%%%%%%%%%
Software development is increasingly conceived as a collaboration activity between developers and AIs. Indeed, IDEs already implement features to enable interactive development, with AI suggesting implementations that are reused by developers.

Although multiple studies show this interaction can be successful, there is still limited understanding of how the models must be configured and used in the context of code generation tasks. This study addresses this gap, systematically investigating the impact of several key parameters, including the repeated submission of a prompt to accommodate for the non-deterministic nature of the models.

Our study reveals several key findings about the usage of ChatGPT. In particular, we discovered how creativity, although up to a limited extent, is useful to increase the range of methods whose code can be generated correctly. A major role is played by parameter top-p, which is commonly underrated, and instead has a major impact on the correctness of the results, with lower values producing better results. Finally, prompts should be submitted multiple times, with $5$ repetitions combined with a temperature of $1.2$ resulting in an effective configuration in our experiments.  

Future work concerns two main research directions. One is about replicating this experiment with other AI assistants, to validate our findings in multiple contexts. The second research direction concerns finding strategies to deal with the need to submit the same prompt multiple times to obtain a useful result, and thus developing approaches able to select or merge multiple responses automatically. 


% \section*{APPENDIX}
 
% \textcolor{red}{Appendixes should appear before the acknowledgment.}



\bibliographystyle{IEEEtran}
% \bibliography{IEEEabrv,bibliography}
\bibliography{IEEEabrv,references}

\end{document}
