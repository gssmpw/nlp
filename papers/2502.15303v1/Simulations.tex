\section{Simulation results} \label{sec:simulations}
To validate the proposed control laws, in this section two MATLAB simulation results are provided, demonstrating the successful tracking of time-varying formations by a group of four quadrotor vehicles. The control gains adopted for both scenarios are $k_{p,i}=1.9$ and $k_{d,i}=3$, and $n_i=20, $ $\forall i\ge 2$. In the first scenario, the desired formation is chosen to have a time-varying shape while simultaneously translating along the $y$-axis. Specifically, we choose $p_1^*{=}\begin{bmatrix} 1 ~ \frac{1}{5}t ~ 1\end{bmatrix}^\top$, $p_2^*{=}\begin{bmatrix} -1{-}\frac{3}{4}\sin(t) \ \ \frac{1}{5}t \ \ 1{+}\frac{3}{4}\sin(t) \end{bmatrix}^\top$, $p_3^*{=}\begin{bmatrix} {-}1 ~ \frac{1}{5}t 
 ~ {-}1\end{bmatrix}^\top$, $p_4^*{=}\begin{bmatrix} 1 ~ \frac{1}{5}t ~ {-}1\end{bmatrix}^\top$. The three-dimensional trajectories of the formation are shown in Fig. \ref{fig:simvarying} and the evolution of state errors converging to the origin is shown in Fig. \ref{fig:simerrvarying}. The topology is described as $\mathcal{N}_2 = \{1\}$, $\mathcal{N}_3 = \{2\}$, and $\mathcal{N}_4 = \{2, 3\}$.
\begin{figure}[p]
 	\centering
 	\includegraphics[width=0.99\linewidth]{NewFigures/SimVarying.pdf}
        \vspace{-0.4cm}
 	\caption{Simulated trajectories of 4-agent formation with time-varying shape. The dashed lines represent the desired trajectories, and the solid lines the simulated trajectories. The solid black arrows indicate connections between agents.}
 	\label{fig:simvarying}
\end{figure}
\begin{figure}%[H]
	\centering
	\includegraphics[width=0.99\linewidth]{NewFigures/SimErrorVarying.pdf}
    \vspace{-0.4cm}
	\caption{Absolute position, velocity, and rotation error of the follower agents in the 4-agent formation with time-varying shape.}
	\label{fig:simerrvarying}
\end{figure}

In the second scenario, the desired formation is chosen to be a rigid shape rotating around and translating along the $y$-axis. At the same time, to pass through a narrow window, the formation is able to change its scale accordingly, as shown in Fig. \ref{fig:simrescale}. The formation has a minimal leader-follower graph formed by a single directed path, i.e. each follower has only one neighbor such that $\mathcal{N}_i = \{i - 1\}, i\in \mathcal{V}\setminus\{1\}$. The desired trajectories are given by $p_1^*{=}\begin{bmatrix} 0 ~ \frac{2}{5}t ~ 0\end{bmatrix}^\top$, $p_i^* = p_1^* + R_y(\frac{1}{2}t)d_i$ for $i\in\{2,3,4\}$, $d_2 = (1{+}|\frac{t-40}{20}| ) e_3$, $d_3 = R_y({-}\frac{2}{3}\pi) d_2$, $d_4 = R_y({-}\frac{4}{3}\pi) d_2$, where $R_y(\theta)$ is the rotation matrix around the $y$-axis.
The performance for this setup is shown in Fig. \ref{fig:simerrrescale}. The cascaded nature of the system is clearly visible, with the slowest convergence for agent $4$.
\begin{figure}%[H]
	\centering
	\includegraphics[width=0.99\linewidth]{NewFigures/SimErrorRescale.pdf}
    \vspace{-0.4cm}
	\caption{Absolute position and velocity error of the follower agents for the 4-agent formation with rigid shape, but a varying scaling factor.}
	\label{fig:simerrrescale}
\end{figure}
\begin{figure}
    \vspace{0.2cm}
    \centering
    \includegraphics[width=0.99\linewidth]{NewFigures/SimRescale.pdf}
    \vspace{-0.25cm}
    \caption{Simulated trajectories of 4-agent formation with rotating rigid shape, but a varying scaling factor. Three instances of the formation are shown, the first one (open circles) represents the initial positions, the second one at $t=40$ (solid circles) represents the point where the radius is at the smallest, after which the agents follow the desired trajectory until $t=80$ where the final position (solid circles) is shown. The dashed lines represent the desired trajectory of agent 4, and the solid yellow line is its simulated trajectory.}
    \label{fig:simrescale}
\end{figure}
The numerical simulations indicate an effective performance of the proposed controllers and a clear convergence of the formation tracking errors under a variety of desired formation trajectories with more relaxed graph topologies compared to \cite{schiano2016rigidity} and  \cite{erskine2021model}. Remark that the chosen desired trajectory is one of the important factors for the convergence rate as indicated in \cite[Theorem 3]{tang2021formation}. For instance, Fig. \ref{fig:simerrrescale} shows a faster convergence than Fig. \ref{fig:simerrvarying} even though the same control gains are used.
