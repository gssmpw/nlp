\section{Experimental results} \label{sec:experiments}
In this section, we describe the practical experiments carried out with three Kopis CineWhoop 3\textquotesingle\textquotesingle\, quadrotors in a closed arena (3m x 6m x 2m), with the end goal of validating the effective performance of the control laws in the presence of disturbances and communication delays, according to Fig. \ref{fig:expshot}. A setup with an OptiTrack Motion Capture (MOCAP) system is used to obtain the inter-agent virtual bearing measurements (which is a usual setup as in the previous literature \cite{schiano2016rigidity,erskine2021model}), focusing on the performance of the control laws, instead of on the acquisition of bearing measurements with a vision system. 
\begin{figure}%[H]
    \centering
    \includegraphics[width=0.99\linewidth]{NewFigures/PracticalSetup.png}
    \vspace{-0.5cm}
    \caption{Three Kopis CineWhoop 3\textquotesingle\textquotesingle\, quadrotors during an experiment.}
    \label{fig:expshot}
    % \vspace{-0.1cm}
\end{figure}
Each quadrotor is controlled in real time by a central desktop using UDP communication at 33 Hz over WiFi, using the Pegasus GNC with ROS 2 and PX4 as the base software infrastructure \cite{10556959}. The \href{https://github.com/SDoodeman/bpe_quadrotor}{code repository used in this experimental setup is made publicly available} and provides additional Gazebo simulations.

In the practical experiments, the control gains and parameter are chosen as $k_{p,i}=5.5$ and $k_{d,i}=5.2$, $k_{o,i}=0.4$, $n_i=5.0, \forall i\ge 2$ and $r=0.10$. The adopted topology is composed of a leader quadrotor, a first follower that can only measure the relative bearing to the leader $\mathcal{N}_2 = \{1\}$, and a second follower quadrotor that is able to measure the bearing with respect to the other two vehicles $\mathcal{N}_3 = \{1, 2\}$, according to Fig. \ref{fig:exptopology}. Fig. \ref{fig:exp13D} shows the evolution of the formation in 3D space. The scale of the formation is well maintained, while only one bearing measurement to one neighbor is necessary for a following quadrotor (i.e. quadrotor 2 in Fig. \ref{fig:exptopology}), unlike the work in \cite{schiano2016rigidity} and \cite{erskine2021model} which requires one distance measurement.  Fig. \ref{fig:exp1err} shows the evolution of the state's errors, indicating the effective performance and robustness of the proposed methods on a real quadrotor formation with the presence of aerodynamic influences and disturbances.

The effect of the collision avoidance term $u_i^c$ during the initial convergence of the vehicles to their desired formation can be seen in Fig. \ref{fig:coll}, indicating that when the distance between two agents decreases, the collision avoidance term in the opposite direction to this neighboring agent increases.
\begin{figure}%[H]
    \centering
    \vspace{0.2cm}
    \includegraphics[width=0.99\linewidth]{NewFigures/ExperimentTopology.pdf}
    \vspace{-0.5cm}
    \caption{Used topology for the experiment, where, for each quadrotor, the trajectory is given by $p^*(t) = \left[
        A(t)\cos(\omega t-\phi) \ \ A(t)\sin(\omega t-\phi) \ \ h(t)
    \right]^\top$, creating a circular motion, moving upwards, and decreasing and increasing the circle radius. The limited space of the experimental arena is considered for these trajectories.}
    \label{fig:exptopology}
\end{figure}
\begin{figure}%[H]
    \centering
    \includegraphics[width=0.99\linewidth]{NewFigures/ExperimentTrajectory.pdf}
    \vspace{-0.5cm}
    \caption{Trajectory of the Kopis quadrotors, using $k_{p} = 5.5$ and $k_{d} = 5.2$.}
    \label{fig:exp13D}
    % \vspace{-0.1cm}
\end{figure}

\begin{figure}%[H]
	\centering
        \vspace{0.2cm}
	\includegraphics[width=0.99\linewidth]{NewFigures/ExperimentError.pdf}
    \vspace{-0.5cm}
	\caption{Relative position and velocity error during the experiment.}
	\label{fig:exp1err}
    % \vspace{-0.1cm}
\end{figure}
% Although the trajectory is approximately followed, a steady-state error is still present. This is most significant for the first following quadrotor, since the second follower has more bearing measurements available.
% The aerodynamic influences between the three quadrotors and the wall of the relatively small arena can be seen as one of the causes of the steady-state error, where the convergence of the BPE control is not strong enough to reduce this steady-state error.

% 
\begin{figure}%[H]
	\centering
	\includegraphics[width=0.99\linewidth]{NewFigures/ExperimentCollision.pdf}
    \vspace{-0.5cm}
    \caption{Collision avoidance term in the opposite direction to neighboring agents $-{u_i^c}^\top g_{ij}$ together with $d_{ij}$ between agents during an experiment.}
	\label{fig:coll}
    % \vspace{-0.1cm}
\end{figure}