\section{Introduction} \label{sec:intro}

The demand for a swarm of Unmanned Aerial Vehicles (UAVs) is rapidly expanding due to their ability to perform various missions, such as agricultural monitoring, exploration of large spaces, and search and rescue in disaster environments in a distributed and efficient manner \cite{chung2018survey}.
During these missions, it is essential for a swarm of UAVs to cooperatively track desired trajectories.
Traditional UAV formation tracking problems assume that the global position information is available for each agent, typically relying on accurate Global Navigation Satellite Systems (GNSS) to provide this information \cite{dong2016time}. While this assumption generally holds in open outdoor environments, GNSS signals are often unreliable in congested areas or unavailable altogether in indoor settings. In addition, in safety-critical missions, GNSS jamming and spoofing is a major concern. Therefore, resorting to onboard exteroceptive sensors is a much more effective solution to provide local relative measurements for UAVs. For instance, ultra-wideband delivers accurate relative distance measurements through radio communication, and vision sensors provide simple visual cues such as relative bearing (direction) measurements, which are robust to noise.
Due to the passive property of cameras, bearing-based formation control strategies can be established under sensing-only graph topologies, preferable when communication between vehicles is not available.

Due to the minimum sensing requirement,  bearing-based formation control has received growing attention.
Theoretical works \cite{zhao2016bearing,eren2003sensor} have exploited \textit{bearing rigidity} theory and formally described under which topological conditions the shape of a formation, up to a scaling factor, can be uniquely determined by constant inter-agent bearing measurements.
Based on the notion of classical \textit{bearing rigidity}, formation controllers relying on bearings have been designed for multi-agent systems under single-integrator dynamics \cite{zhao2016bearing, trinh2018bearing} and double-integrator dynamics \cite{zhao2019bearing} with applications to robotic vehicles.
For instance, a leader-follower formation tracking controller is proposed in \cite{zhao2019bearing} for a group of unicycle robots to track a translating rigid formation under the assumption that two leaders in the formation know their absolute global position. Many other works also require two leader vehicles \cite{parada2024twoleaders, ding_dynamics_2024, chen2023twoleaders, xu2020affine}.
The work in \cite{schiano2016rigidity} proposed a bearing-based formation stabilizing controller for quadrotors under simplified kinematic models, while \cite{erskine2021model} extended this approach to dynamic models using model predictive control. In both \cite{schiano2016rigidity} and \cite{erskine2021model}, a known distance measurement between two vehicles is necessary. Furthermore, just as in \cite{schiano2016rigidity}, most control strategies require classical bearing rigid conditions and are limited to achieving desired formations with constant bearing references only \cite{zhao2016bearing, trinh2019bearing}.

%These formation control methods with constant bearings under classical bearing rigid conditions with at least two leaders or inter-agent distance measurements restrict their applications to track rigid formations involving translation and scaling transformations.

Recently, Tang et al. \cite{tang2022relaxed, tang2021formation} introduced the concepts of \textit{bearing persistently exciting} (BPE) formation and \textit{relaxed bearing rigidity}, which provide bearing formation control solutions that can deal with time-varying bearing references while also relaxing the conditions on the graph topologies required in classical bearing rigidity. Under the proposed scheme in \cite{tang2021formation}, the convergence of position and velocity errors to zero can be guaranteed under much-relaxed topologies (i.e. graphs containing a single spanning tree as shown in Fig. \ref{fig:BPE}) with a single leader as opposed to the typical requirements of two leaders. However, in these works, only ideal agents with single or double-integrator models are considered. 

This work extends our previous research on formation tracking control \cite{tang2021formation} and collision avoidance in leader-follower formation control \cite{collision} to address practical challenges, such as collisions and disturbances, of controlling multiple quadrotor vehicles in realistic scenarios. We present a hierarchical control strategy for achieving formation tracking under a leader-follower directed sensing graph topology. It is assumed that only one leader quadrotor in the group knows its global position. The remaining agents, called followers, are equipped with local onboard sensors and are assumed to measure their orientation as well as the relative bearing and relative translational velocity to their neighboring agents. To ensure safe navigation in complex environments, this controller is augmented with a collision avoidance term, and a high-gain inner-loop controller is adopted for fast attitude tracking. The effectiveness of the complete system is validated through numerical simulations demonstrating the performance for time-varying shapes and rotational maneuvers, as well as experimental implementation on three physical quadrotors, considering the influence of external disturbances.

The remainder of the paper is organized as follows.
Section \ref{sec:preliminaries} provides the mathematical background on graph theory and the formal BPE definitions.
Section \ref{sec:modeling} provides the adopted quadrotor model and problem formulation, while Section \ref{sec:controlstrategy} introduces the proposed hierarchical control strategy. MATLAB simulation results are discussed in Section \ref{sec:simulations}, showing the performance for both time-varying and rigid formation shapes. Experimental results are provided in Section \ref{sec:experiments}, showing the performance and limits of the proposed method. The paper concludes with final comments in Section \ref{sec:conclusion}.
\begin{figure}
	\centering
        \vspace{0.2cm}
	\includegraphics[width=0.33\textwidth]{NewFigures/Topologies.pdf}
	% \vspace{-0.1cm}
	\caption{Possible graph topologies of leader follower BPE formation (a1)-(b3). Only (a2) and (b3) satisfy the minimum graph requirement by bearing rigid formation \cite{trinh2018bearing}.}
	\label{fig:BPE}
    % \vspace{-0.1cm}
\end{figure}
