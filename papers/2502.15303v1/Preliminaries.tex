\section{Preliminaries} \label{sec:preliminaries}

 Let $\mathbb{S}^2 := \{y\in \mathbb{R}^3 : \lVert y \rVert = 1 \}$ denote the 2-sphere, and $\lVert . \rVert$ the Euclidean norm. The operator $\begin{bmatrix} . \end{bmatrix}_\times$ represents the skew-symmetric matrix related to any argument in $\mathbb{R}^3$. Let $I_d$ be the $d\times d$ identity matrix. We then define the projection operator $\pi_y = I_3-yy^\top \geq 0$ for any $y\in\mathbb{S}^2$, inducing $\pi_y = -[y]_\times [y]_\times$.
 
The considered formations concern systems of $n$ ($n\geq2$) connected agents. The topology between those agents can be modeled as a directed graph $\mathcal{G} := (\mathcal{V},\mathcal{E})$, where $\mathcal{V}=\{1,2,...,n\}$ is the set of vertices and $\mathcal{E} \subseteq \mathcal{V} \times \mathcal{V}$ is the set of directed edges. If the ordered pair $(i,j) \in \mathcal{E}$, then agent $i$ can sense information about the neighboring agent $j$. The set of neighbors of agent $i$ is denoted by $\mathcal{N}_i := \{j \in \mathcal{V}|(i,j)\in \mathcal{E}\}$. $N_i = |\mathcal{N}_i|$ is defined as the cardinality of the set. A digraph $\mathcal{G}$ is called an acyclic digraph if it has no directed cycle. The digraph $\mathcal{G}$ is called a directed tree with a root vertex $i, i \in \mathcal{V}$, if for any vertex $j \neq i, j \in \mathcal{V}$, there exists only one directed path connecting $j$ to $i$. 

%The following definitions and conditions related to persistence of excitation will be used later to define the desired formation.
\begin{defi} \label{defi:PEmatrix}
    A positive semi-definite matrix $\Sigma(t) \in \mathbb{R}^{n \times n}$ is called \textit{persistently exciting} (PE) if there exist $T>0$ and $0<\mu<T$ such that for all $t>0$
    \begin{equation}
        \frac{1}{T}\int_t^{t+T}\Sigma(\tau)d\tau \geq \mu I_3.
        \label{eq:PEmatrix}
    \end{equation}
\end{defi}

\begin{defi} \label{defi:PEdirection}
     A direction $y(t)\in\mathbb{S}^2$ is said to be PE if the matrix $\pi_{y(t)}$ satisfies condition \eqref{eq:PEmatrix} \cite{tang2021formation}.
\end{defi}