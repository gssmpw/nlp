\section{Modeling and problem formulation} \label{sec:modeling}
Consider the problem of formation tracking control of a group of $n$ quadrotor vehicles. For each quadrotor $i$, let $p_i\in \mathbb R^3$ and $v_i\in \mathbb R^3$ denote its position and velocity, respectively, expressed in a common inertial frame $\{\mathcal I\}$ that follows a north-east-down (NED) convention. Let $\{\mathcal B_i\}$ be a body-fixed frame, attached to the $i$th quadrotor, following a front-right-down (FRD) convention, as shown in Fig. \ref{fig:coordinates}.
% \vspace{-0.1cm}
\begin{figure}[t]
        \vspace{0.2cm}
	\centering
	\includegraphics[width=0.40\textwidth]{NewFigures/ReferenceFrames.pdf}
	% \vspace{-0.15cm}
	\caption{Schematic representation of the FRD vehicle body frame $\{B\}$, relative to an NED inertial frame $\{I\}$.}
	\label{fig:coordinates}
    % \vspace{-0.1cm}
\end{figure}
% \vspace{-0.2cm}
Let $R_i\in SO(3)$ denote the attitude $\{\mathcal B_i\}$ with respect to $\{\mathcal I\}$. The dynamics of quadrotor $i$ are given by
\begin{subequations}
    \label{eq:trans_dyn}
    \begin{align}
        \dot{p}_i &= v_i, \label{eq:dotp}\\
        m_i \dot{v}_i &= - T_i R_i e_3 + m_i g e_3, \label{eq:dotv} \\
         \dot{R}_i &= R_i {[\Omega_i]}_\times,\label{eq:dotR}
    \end{align}
\end{subequations}
where $g \approx 9.81\text{ms}^{-2}$ is the gravitational acceleration, $e_3 = \begin{bmatrix} 0 & 0 & 1\end{bmatrix}^\top$, $T_i\in \mathbb R_0^{+}$ is the total thrust magnitude, $m_i\in \mathbb R^{+}$ is the mass and $\Omega_i\in \mathbb R^3$ denotes the angular velocity input of agent $i$ expressed in $\{\mathcal B_i\}$.

Define the relative position vector $p_{ij}$ and relative velocity vector $v_{ij}$ between vehicle $i$ and its neighbor vehicle $j$ as 
$$p_{ij}=p_j-p_i, \quad v_{ij}=v_j-v_i, \quad \forall j\in \mathcal N_i.$$
As long as $\|p_{ij}\|\ne 0$, the relative bearing $g_{ij}$ of agent $i$ to agent $j$ is defined by
\begin{eqnarray}
    g_{ij} := \frac{p_{ij}}{\lVert p_{ij}\rVert} \in \mathbb S^2. \label{eq:bearing}
\end{eqnarray}
 Let the stacked vector $\boldsymbol{p}=[p_1^\top,...,p_n^\top]^\top\in \mathbb{R}^{3n}$ denote the configuration of $\mathcal{G}$ and the digraph $\mathcal{G}$ together with the configuration $\boldsymbol{p}$ define a formation $\mathcal{G}(\boldsymbol p)$ in three-dimensional space. We now introduce the first assumption related to the sensing graph topology.

\begin{asu} \label{asu:topology} 
    The sensing topology of the formation $\mathcal G (\boldsymbol p)$ is described as a leader-follower structure, i.e., an acyclic digraph $\mathcal{G}(\mathcal{V}, \mathcal{E})$ that has a single directed spanning tree. 
    Without loss of generality, agents are numbered (or can be renumbered) such that agent $1$ is the leader, i.e.  $\mathcal{N}_1= \varnothing$,  all other agents $i, \ i\ge 2$ are followers whose neighboring set is $\mathcal{N}_i  \subseteq \{1, \ldots, i-1\}$, according to the examples in Fig. \ref{fig:BPE}.
    Each agent $i \geq 2$ can measure the relative bearing vectors $g_{ij}$ and relative velocities $v_{ij}$ to its neighbors $j \in \mathcal{N}_i$, as well as its own attitude $R_i$. % represented by the rotation matrix
\end{asu}

Given this leader-follower structure, we introduce the definition of a BPE formation \cite{tang2021formation} which will be used later to define desired trajectories.
\begin{defi} \label{def:BPE}
    A leader–follower formation $\mathcal{G}(\boldsymbol p(t))$ is called BPE, if $\forall i \in \mathcal{V}$, the matrices $\sum_{j\in \mathcal{N}_i} \pi_{g_{ij}(t)}$ satisfy the PE condition \eqref{eq:PEmatrix}.
\end{defi}
Note that if $\sum_{j\in \mathcal{N}_i} \pi_{g_{ij}(t)}$ is PE, then agent $i$ has at least one bearing measurement $g_{ij}$ that is time-varying or at least two bearings that are non-collinear \cite[Lemma 1]{tang2021formation}. We now define the assumptions for the desired trajectories.
\begin{asu} \label{asu:boundedBPE}
    The desired velocity $v_{i}^*(t)$, the desired acceleration $u_i^*(t)$, and the desired jerk $\dot{u}_i^*(t)$ are bounded for all $t > 0, i \in \mathcal V$, and such that the resulting desired bearings $g_{ij}^*(t), (i,j)\in \mathcal E$ are well-defined and the desired formation is BPE for all $t > 0$.
\end{asu}

Let us now present the considered problem formulation.
\begin{prob} \label{prob:statement}
Design distributed formation tracking controllers for all follower vehicles ($i\ge 2$) such that a group of $n\ (n\ge2)$ quadrotor vehicles successfully tracks a desired BPE formation under Assumptions \ref{asu:topology}-\ref{asu:boundedBPE}, while avoiding inter-agent collisions.
\end{prob}