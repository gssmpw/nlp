%%%%%%%% ICML 2025 EXAMPLE LATEX SUBMISSION FILE %%%%%%%%%%%%%%%%%

\documentclass{article}



\usepackage{enumitem}

\usepackage[table,xcdraw]{xcolor}
\usepackage{float}

\usepackage{multirow}
\usepackage{multicol}
\usepackage[normalem]{ulem}
\useunder{\uline}{\ul}{}
\usepackage{listings} % Required for code formatting

% Define a darker green color
\definecolor{darkergreen}{rgb}{0.0, 0.5, 0.0} % Adjust RGB values as needed
% \usepackage{soul}   

% Define a command for each collaborator with bold comments
\newcommand{\bryan}[1]{\textcolor{blue}{\textbf{[Bryan: #1]}}}
\newcommand{\bingxuan}[1]{\textcolor{green}{\textbf{[bingxuan: #1]}}}
\newcommand{\mohan}[1]{\textcolor{teal}{\textbf{[mohan: #1]}}}
\newcommand{\violet}[1]{\textcolor{violet}{\textbf{[violet: #1]}}}
\newcommand{\kaiwei}[1]{\textcolor{orange}{\textbf{[kaiwei: #1]}}}
\newcommand{\telin}[1]{\textcolor{magenta}{\textbf{[telin: #1]}}}
\newcommand{\xiaomeng}[1]{\textcolor{red}{\textbf{[xiaomeng: #1]}}}
\newcommand{\heng}[1]{\textcolor{brown}{\textbf{[heng: #1]}}}
\newcommand{\task}[1]{\textcolor{purple}{\textbf{[TODO: #1]}}}


% Recommended, but optional, packages for figures and better typesetting:
\usepackage{microtype}
\usepackage{graphicx}
\usepackage{subfigure}
\usepackage{booktabs} % for professional tables
\usepackage{tabularx}

% Adjust spacing globally
\setlength{\textfloatsep}{5pt} % Space between figures and text
\setlength{\textfloatsep}{5pt} % Space between table and text
\setlength{\abovecaptionskip}{5pt} % Space above table caption
\setlength{\belowcaptionskip}{5pt} % Space below table caption
\setlength{\abovedisplayskip}{5pt} % Space above display math
\setlength{\belowdisplayskip}{5pt} % Space below display math


% hyperref makes hyperlinks in the resulting PDF.
% If your build breaks (sometimes temporarily if a hyperlink spans a page)
% please comment out the following usepackage line and replace
% \usepackage{icml2025} with \usepackage[nohyperref]{icml2025} above.
\usepackage{hyperref}


% Attempt to make hyperref and algorithmic work together better:
% \newcommand{\theHalgorithm}{\arabic{algorithm}}

% Use the following line for the initial blind version submitted for review:
\usepackage[accepted]{icml2025}

% If accepted, instead use the following line for the camera-ready submission:
% \usepackage[accepted]{icml2025}

% For theorems and such
\usepackage{amsmath}
\usepackage{amssymb}
\usepackage{mathtools}
\usepackage{amsthm}

% if you use cleveref..
\usepackage[capitalize,noabbrev]{cleveref}

%%%%%%%%%%%%%%%%%%%%%%%%%%%%%%%%
% THEOREMS
%%%%%%%%%%%%%%%%%%%%%%%%%%%%%%%%
\theoremstyle{plain}
\newtheorem{theorem}{Theorem}[section]
\newtheorem{proposition}[theorem]{Proposition}
\newtheorem{lemma}[theorem]{Lemma}
\newtheorem{corollary}[theorem]{Corollary}
\theoremstyle{definition}
\newtheorem{definition}[theorem]{Definition}
\newtheorem{assumption}[theorem]{Assumption}
\theoremstyle{remark}
\newtheorem{remark}[theorem]{Remark}

% Todonotes is useful during development; simply uncomment the next line
%    and comment out the line below the next line to turn off comments
%\usepackage[disable,textsize=tiny]{todonotes}
\usepackage[textsize=tiny]{todonotes}


% The \icmltitle you define below is probably too long as a header.
% Therefore, a short form for the running title is supplied here:
\icmltitlerunning{Contrastive Visual Data Augmentation}

\begin{document}

\twocolumn[
\icmltitle{Contrastive Visual Data Augmentation}

% It is OKAY to include author information, even for blind
% submissions: the style file will automatically remove it for you
% unless you've provided the [accepted] option to the icml2025
% package.

% List of affiliations: The first argument should be a (short)
% identifier you will use later to specify author affiliations
% Academic affiliations should list Department, University, City, Region, Country
% Industry affiliations should list Company, City, Region, Country

% You can specify symbols, otherwise they are numbered in order.
% Ideally, you should not use this facility. Affiliations will be numbered
% in order of appearance and this is the preferred way.


% \icmlsetsymbol{equal}{*}

% \begin{icmlauthorlist}
% \icmlauthor{Yu Zhou}{equal,ucla}
% \icmlauthor{Bingxuan Li}{equal,ucla}
% \icmlauthor{Mohan Tang}{equal,ucla}
% \icmlauthor{Xiaomeng Jin}{uiuc}
% \icmlauthor{Te-Lin Wu}{ucla}
% \icmlauthor{Kuan-Hao Huang}{uiuc,tamu}\\
% \icmlauthor{Heng Ji}{uiuc}
% %\icmlauthor{}{sch}
% \icmlauthor{Kai-Wei Chang}{ucla}
% \icmlauthor{Nanyun Peng}{ucla}
% % \icmlauthor{}{sch}
% % \icmlauthor{}{sch}
% \end{icmlauthorlist}

% \icmlaffiliation{ucla}{UCLA}
% \icmlaffiliation{uiuc}{UIUC}
% \icmlaffiliation{tamu}{TAMU}

% \icmlcorrespondingauthor{Yu Zhou}{yuzhou@cs.ucla.edu}
% \icmlcorrespondingauthor{Nanyun Peng}{violetpeng@cs.ucla.edu}

% % You may provide any keywords that you
% % find helpful for describing your paper; these are used to populate
% % the "keywords" metadata in the PDF but will not be shown in the document
% % \icmlkeywords{Machine Learning, ICML}

\icmlsetsymbol{equal}{*}
\begin{icmlauthorlist}
\icmlauthor{Yu Zhou}{ucla,equal}
\icmlauthor{Bingxuan Li}{ucla,equal}
\icmlauthor{Mohan Tang}{ucla,equal}
\icmlauthor{Xiaomeng Jin}{uiuc}
\icmlauthor{Te-Lin Wu}{ucla}
\icmlauthor{Kuan-Hao Huang}{uiuc,tamu}\\
\icmlauthor{Heng Ji}{uiuc}
\icmlauthor{Kai-Wei Chang}{ucla}
\icmlauthor{Nanyun Peng}{ucla}
\end{icmlauthorlist}

\icmlaffiliation{ucla}{University of California, Los Angeles}
\icmlaffiliation{uiuc}{University of Illinois at Urbana-Champaign}
\icmlaffiliation{tamu}{Texas A\&M University}

\icmlcorrespondingauthor{Yu Zhou}{yuzhou@cs.ucla.edu}

\vskip 0.3in
]

\printAffiliationsAndNotice{* Equal contribution, interchangeable ordering.}


% this must go after the closing bracket ] following \twocolumn[ ...

% This command actually creates the footnote in the first column
% listing the affiliations and the copyright notice.
% The command takes one argument, which is text to display at the start of the footnote.
% The \icmlEqualContribution command is standard text for equal contribution.
% Remove it (just {}) if you do not need this facility.

%\printAffiliationsAndNotice{}  % leave blank if no need to mention equal contribution
% \printAffiliationsAndNotice{\icmlEqualContribution} % otherwise use the standard text.



\begin{abstract}


The choice of representation for geographic location significantly impacts the accuracy of models for a broad range of geospatial tasks, including fine-grained species classification, population density estimation, and biome classification. Recent works like SatCLIP and GeoCLIP learn such representations by contrastively aligning geolocation with co-located images. While these methods work exceptionally well, in this paper, we posit that the current training strategies fail to fully capture the important visual features. We provide an information theoretic perspective on why the resulting embeddings from these methods discard crucial visual information that is important for many downstream tasks. To solve this problem, we propose a novel retrieval-augmented strategy called RANGE. We build our method on the intuition that the visual features of a location can be estimated by combining the visual features from multiple similar-looking locations. We evaluate our method across a wide variety of tasks. Our results show that RANGE outperforms the existing state-of-the-art models with significant margins in most tasks. We show gains of up to 13.1\% on classification tasks and 0.145 $R^2$ on regression tasks. All our code and models will be made available at: \href{https://github.com/mvrl/RANGE}{https://github.com/mvrl/RANGE}.

\end{abstract}

    
\section{Introduction}
Backdoor attacks pose a concealed yet profound security risk to machine learning (ML) models, for which the adversaries can inject a stealth backdoor into the model during training, enabling them to illicitly control the model's output upon encountering predefined inputs. These attacks can even occur without the knowledge of developers or end-users, thereby undermining the trust in ML systems. As ML becomes more deeply embedded in critical sectors like finance, healthcare, and autonomous driving \citep{he2016deep, liu2020computing, tournier2019mrtrix3, adjabi2020past}, the potential damage from backdoor attacks grows, underscoring the emergency for developing robust defense mechanisms against backdoor attacks.

To address the threat of backdoor attacks, researchers have developed a variety of strategies \cite{liu2018fine,wu2021adversarial,wang2019neural,zeng2022adversarial,zhu2023neural,Zhu_2023_ICCV, wei2024shared,wei2024d3}, aimed at purifying backdoors within victim models. These methods are designed to integrate with current deployment workflows seamlessly and have demonstrated significant success in mitigating the effects of backdoor triggers \cite{wubackdoorbench, wu2023defenses, wu2024backdoorbench,dunnett2024countering}.  However, most state-of-the-art (SOTA) backdoor purification methods operate under the assumption that a small clean dataset, often referred to as \textbf{auxiliary dataset}, is available for purification. Such an assumption poses practical challenges, especially in scenarios where data is scarce. To tackle this challenge, efforts have been made to reduce the size of the required auxiliary dataset~\cite{chai2022oneshot,li2023reconstructive, Zhu_2023_ICCV} and even explore dataset-free purification techniques~\cite{zheng2022data,hong2023revisiting,lin2024fusing}. Although these approaches offer some improvements, recent evaluations \cite{dunnett2024countering, wu2024backdoorbench} continue to highlight the importance of sufficient auxiliary data for achieving robust defenses against backdoor attacks.

While significant progress has been made in reducing the size of auxiliary datasets, an equally critical yet underexplored question remains: \emph{how does the nature of the auxiliary dataset affect purification effectiveness?} In  real-world  applications, auxiliary datasets can vary widely, encompassing in-distribution data, synthetic data, or external data from different sources. Understanding how each type of auxiliary dataset influences the purification effectiveness is vital for selecting or constructing the most suitable auxiliary dataset and the corresponding technique. For instance, when multiple datasets are available, understanding how different datasets contribute to purification can guide defenders in selecting or crafting the most appropriate dataset. Conversely, when only limited auxiliary data is accessible, knowing which purification technique works best under those constraints is critical. Therefore, there is an urgent need for a thorough investigation into the impact of auxiliary datasets on purification effectiveness to guide defenders in  enhancing the security of ML systems. 

In this paper, we systematically investigate the critical role of auxiliary datasets in backdoor purification, aiming to bridge the gap between idealized and practical purification scenarios.  Specifically, we first construct a diverse set of auxiliary datasets to emulate real-world conditions, as summarized in Table~\ref{overall}. These datasets include in-distribution data, synthetic data, and external data from other sources. Through an evaluation of SOTA backdoor purification methods across these datasets, we uncover several critical insights: \textbf{1)} In-distribution datasets, particularly those carefully filtered from the original training data of the victim model, effectively preserve the model’s utility for its intended tasks but may fall short in eliminating backdoors. \textbf{2)} Incorporating OOD datasets can help the model forget backdoors but also bring the risk of forgetting critical learned knowledge, significantly degrading its overall performance. Building on these findings, we propose Guided Input Calibration (GIC), a novel technique that enhances backdoor purification by adaptively transforming auxiliary data to better align with the victim model’s learned representations. By leveraging the victim model itself to guide this transformation, GIC optimizes the purification process, striking a balance between preserving model utility and mitigating backdoor threats. Extensive experiments demonstrate that GIC significantly improves the effectiveness of backdoor purification across diverse auxiliary datasets, providing a practical and robust defense solution.

Our main contributions are threefold:
\textbf{1) Impact analysis of auxiliary datasets:} We take the \textbf{first step}  in systematically investigating how different types of auxiliary datasets influence backdoor purification effectiveness. Our findings provide novel insights and serve as a foundation for future research on optimizing dataset selection and construction for enhanced backdoor defense.
%
\textbf{2) Compilation and evaluation of diverse auxiliary datasets:}  We have compiled and rigorously evaluated a diverse set of auxiliary datasets using SOTA purification methods, making our datasets and code publicly available to facilitate and support future research on practical backdoor defense strategies.
%
\textbf{3) Introduction of GIC:} We introduce GIC, the \textbf{first} dedicated solution designed to align auxiliary datasets with the model’s learned representations, significantly enhancing backdoor mitigation across various dataset types. Our approach sets a new benchmark for practical and effective backdoor defense.



\section{Related Work}

\subsection{Large 3D Reconstruction Models}
Recently, generalized feed-forward models for 3D reconstruction from sparse input views have garnered considerable attention due to their applicability in heavily under-constrained scenarios. The Large Reconstruction Model (LRM)~\cite{hong2023lrm} uses a transformer-based encoder-decoder pipeline to infer a NeRF reconstruction from just a single image. Newer iterations have shifted the focus towards generating 3D Gaussian representations from four input images~\cite{tang2025lgm, xu2024grm, zhang2025gslrm, charatan2024pixelsplat, chen2025mvsplat, liu2025mvsgaussian}, showing remarkable novel view synthesis results. The paradigm of transformer-based sparse 3D reconstruction has also successfully been applied to lifting monocular videos to 4D~\cite{ren2024l4gm}. \\
Yet, none of the existing works in the domain have studied the use-case of inferring \textit{animatable} 3D representations from sparse input images, which is the focus of our work. To this end, we build on top of the Large Gaussian Reconstruction Model (GRM)~\cite{xu2024grm}.

\subsection{3D-aware Portrait Animation}
A different line of work focuses on animating portraits in a 3D-aware manner.
MegaPortraits~\cite{drobyshev2022megaportraits} builds a 3D Volume given a source and driving image, and renders the animated source actor via orthographic projection with subsequent 2D neural rendering.
3D morphable models (3DMMs)~\cite{blanz19993dmm} are extensively used to obtain more interpretable control over the portrait animation. For example, StyleRig~\cite{tewari2020stylerig} demonstrates how a 3DMM can be used to control the data generated from a pre-trained StyleGAN~\cite{karras2019stylegan} network. ROME~\cite{khakhulin2022rome} predicts vertex offsets and texture of a FLAME~\cite{li2017flame} mesh from the input image.
A TriPlane representation is inferred and animated via FLAME~\cite{li2017flame} in multiple methods like Portrait4D~\cite{deng2024portrait4d}, Portrait4D-v2~\cite{deng2024portrait4dv2}, and GPAvatar~\cite{chu2024gpavatar}.
Others, such as VOODOO 3D~\cite{tran2024voodoo3d} and VOODOO XP~\cite{tran2024voodooxp}, learn their own expression encoder to drive the source person in a more detailed manner. \\
All of the aforementioned methods require nothing more than a single image of a person to animate it. This allows them to train on large monocular video datasets to infer a very generic motion prior that even translates to paintings or cartoon characters. However, due to their task formulation, these methods mostly focus on image synthesis from a frontal camera, often trading 3D consistency for better image quality by using 2D screen-space neural renderers. In contrast, our work aims to produce a truthful and complete 3D avatar representation from the input images that can be viewed from any angle.  

\subsection{Photo-realistic 3D Face Models}
The increasing availability of large-scale multi-view face datasets~\cite{kirschstein2023nersemble, ava256, pan2024renderme360, yang2020facescape} has enabled building photo-realistic 3D face models that learn a detailed prior over both geometry and appearance of human faces. HeadNeRF~\cite{hong2022headnerf} conditions a Neural Radiance Field (NeRF)~\cite{mildenhall2021nerf} on identity, expression, albedo, and illumination codes. VRMM~\cite{yang2024vrmm} builds a high-quality and relightable 3D face model using volumetric primitives~\cite{lombardi2021mvp}. One2Avatar~\cite{yu2024one2avatar} extends a 3DMM by anchoring a radiance field to its surface. More recently, GPHM~\cite{xu2025gphm} and HeadGAP~\cite{zheng2024headgap} have adopted 3D Gaussians to build a photo-realistic 3D face model. \\
Photo-realistic 3D face models learn a powerful prior over human facial appearance and geometry, which can be fitted to a single or multiple images of a person, effectively inferring a 3D head avatar. However, the fitting procedure itself is non-trivial and often requires expensive test-time optimization, impeding casual use-cases on consumer-grade devices. While this limitation may be circumvented by learning a generalized encoder that maps images into the 3D face model's latent space, another fundamental limitation remains. Even with more multi-view face datasets being published, the number of available training subjects rarely exceeds the thousands, making it hard to truly learn the full distibution of human facial appearance. Instead, our approach avoids generalizing over the identity axis by conditioning on some images of a person, and only generalizes over the expression axis for which plenty of data is available. 

A similar motivation has inspired recent work on codec avatars where a generalized network infers an animatable 3D representation given a registered mesh of a person~\cite{cao2022authentic, li2024uravatar}.
The resulting avatars exhibit excellent quality at the cost of several minutes of video capture per subject and expensive test-time optimization.
For example, URAvatar~\cite{li2024uravatar} finetunes their network on the given video recording for 3 hours on 8 A100 GPUs, making inference on consumer-grade devices impossible. In contrast, our approach directly regresses the final 3D head avatar from just four input images without the need for expensive test-time fine-tuning.


\section{Study Design}
% robot: aliengo 
% We used the Unitree AlienGo quadruped robot. 
% See Appendix 1 in AlienGo Software Guide PDF
% Weight = 25kg, size (L,W,H) = (0.55, 0.35, 06) m when standing, (0.55, 0.35, 0.31) m when walking
% Handle is 0.4 m or 0.5 m. I'll need to check it to see which type it is.
We gathered input from primary stakeholders of the robot dog guide, divided into three subgroups: BVI individuals who have owned a dog guide, BVI individuals who were not dog guide owners, and sighted individuals with generally low degrees of familiarity with dog guides. While the main focus of this study was on the BVI participants, we elected to include survey responses from sighted participants given the importance of social acceptance of the robot by the general public, which could reflect upon the BVI users themselves and affect their interactions with the general population \cite{kayukawa2022perceive}. 

The need-finding processes consisted of two stages. During Stage 1, we conducted in-depth interviews with BVI participants, querying their experiences in using conventional assistive technologies and dog guides. During Stage 2, a large-scale survey was distributed to both BVI and sighted participants. 

This study was approved by the University’s Institutional Review Board (IRB), and all processes were conducted after obtaining the participants' consent.

\subsection{Stage 1: Interviews}
We recruited nine BVI participants (\textbf{Table}~\ref{tab:bvi-info}) for in-depth interviews, which lasted 45-90 minutes for current or former dog guide owners (DO) and 30-60 minutes for participants without dog guides (NDO). Group DO consisted of five participants, while Group NDO consisted of four participants.
% The interview participants were divided into two groups. Group DO (Dog guide Owner) consisted of five participants who were current or former dog guide owners and Group NDO (Non Dog guide Owner) consisted of three participants who were not dog guide owners. 
All participants were familiar with using white canes as a mobility aid. 

We recruited participants in both groups, DO and NDO, to gather data from those with substantial experience with dog guides, offering potentially more practical insights, and from those without prior experience, providing a perspective that may be less constrained and more open to novel approaches. 

We asked about the participants' overall impressions of a robot dog guide, expectations regarding its potential benefits and challenges compared to a conventional dog guide, their desired methods of giving commands and communicating with the robot dog guide, essential functionalities that the robot dog guide should offer, and their preferences for various aspects of the robot dog guide's form factors. 
For Group DO, we also included questions that asked about the participants' experiences with conventional dog guides. 

% We obtained permission to record the conversations for our records while simultaneously taking notes during the interviews. The interviews lasted 30-60 minutes for NDO participants and 45-90 minutes for DO participants. 

\subsection{Stage 2: Large-Scale Surveys} 
After gathering sufficient initial results from the interviews, we created an online survey for distributing to a larger pool of participants. The survey platform used was Qualtrics. 

\subsubsection{Survey Participants}
The survey had 100 participants divided into two primary groups. Group BVI consisted of 42 blind or visually impaired participants, and Group ST consisted of 58 sighted participants. \textbf{Table}~\ref{tab:survey-demographics} shows the demographic information of the survey participants. 

\subsubsection{Question Differentiation} 
Based on their responses to initial qualifying questions, survey participants were sorted into three subgroups: DO, NDO, and ST. Each participant was assigned one of three different versions of the survey. The surveys for BVI participants mirrored the interview categories (overall impressions, communication methods, functionalities, and form factors), but with a more quantitative approach rather than the open-ended questions used in interviews. The DO version included additional questions pertaining to their prior experience with dog guides. The ST version revolved around the participants' prior interactions with and feelings toward dog guides and dogs in general, their thoughts on a robot dog guide, and broad opinions on the aesthetic component of the robot's design. 

\section{NovelSpecies Dataset}
\label{sec:novel_dataset}

Proprietary LMMs like GPT4o~\cite{hurst2024gpt4o} and Gemini~\cite{team2023gemini} are trained on vast online text-image data and proprietary data, both non-public and impossible to inspect. Some open-source and open-data LMMs such as LLaVA~\cite{liu2024improved, liu2024visual} are trained on publicly available image-text datasets. However, the text encoders used by such models are often not open-data, for example LLaVA-1.6 34B uses the closed-data Yi-34B model as its language backbone. Even in the rare cases where both image-text training data and text encoder training data are publicly available, it is still difficult to ascertain whether concepts in your benchmark were seen by your LMM through indirect data leakage (i.e. partial / paraphrased mentions). Due to the above issues, it is difficult to evaluate true novel concept recognition ability with existing datasets. 
% \footnote{Knowledge cutoff date: Dec 2023}

One way to bypass this problem with 100\% guaranteed success is to use a dataset that only contains concepts created / discovered after the LMM's knowledge cutoff, i.e. the latest knowledge cutoff date among all of its textual / visual sub-components. Based on this idea, we curate \textbf{NovelSpecies}, a dataset of novel animal species discovered in each recent year, starting with 2023 and 2024. We provide detailed information for each species, including time of discovery, latin name, common name, family category, textual description, and more. Data will be released upon publication.
% Details are described in Sec.\ref{subsec:NovelSpecies_details}.

To create \textbf{NovelSpecies}, we start by collecting the list of species first described in each year by Wikidata~\cite{wikidata}. Then, to make sure we can curate a visual benchmark of novel species, we manually annotate and filter out extinct species and species with too few publicly available images. After filtering, we end up with a dataset of 64 new species, each consisting of 35 human-verified image instances, thus a total of 2240 images. The images are split into training, validation, and test sets. For each specie, there are 5 training images, 15 validation images, and 15 test images. This data split is consistent with our goal of creating a benchmark dataset for novel concept recognition, where the maximum number of training instances for a completely unseen concept can range from 1 to 5.







% and 2170 images in total, which consist of train, validation, and test sets of equal proportion for all species. Finally, all the images are 















% \section{Datasets}
% \label{sec:dataset}


% \subsection{Confusing Pair Extraction}
% Our focus on confusing pairs arises from the need to strengthen the model's performance in distinguishing between visually similar species—a challenge where LLaVA currently shows limitations. Confusing pairs represent instances where the model's classification often fails, typically due to subtle visual cues or shared features among species within similar taxonomic groups. We designed strategy to extract confusing pair for each dataset.

% \paragraph{INaturalist and Novel Species Dataset} We extract confusing pairs with three-steps as following: 

% \begin{enumerate}
%     \item \textbf{Iterative Subset Selection:} We select a random subset of species in each iteration, sampling across different supercategories. This strategy allows us to identify confusing pairs without overloading the system, progressively building a collection of challenging cases from each subset.
%     \item \textbf{Evaluate Classification Patters:} For each species within a subset, we create prompts in a multiple-choice format, incorporating the image and a randomized list of options from all the species in the subset. Based on the response from LLaVA, we are able to highlight specific species that are commonly mistaken for one another, guiding us in selecting pairs for further analysis. The process is repeated across new subsets, incrementally building an ample dataset of confusing pairs.
%     \item \textbf{Identification of confusing pairs: } We choose a threshhold of 0.2. If class A is misclassified into class B with frequency more than 0.2 in the above multiple-choice setting, we consider the pair to be confusing. 
% \end{enumerate}

% \paragraph{SUN Dataset} We adapted the above methodology for scene classification with minor modification on the subset selection process. Instead of taxonomic groupings, we created subsets by selecting a target scene and the nine most similar scenes based on shared object occurrence. The subsequent steps—classification pattern analysis and confusing pair definition—remained consistent with the species datasets.









% \subsection{Curated INaturalist Dataset}
% In this study, we utilize a random sample of 15 classes from the "Mammals" supercategory of the iNaturalist dataset. Below, we outline the reasoning behind our dataset selection and sampling approach.
% \paragraph{iNaturalist Dataset}
% The iNaturalist dataset is known for its complexity and has proven to be a challenging benchmark for many vision-language models. Due to the extensive diversity and fine-grained nature of the categories, most models do not achieve perfect performance on this dataset, leaving ample room for further improvements.
% \paragraph{Sampling Strategy}
% Given the scale of the iNaturalist dataset, which contains approximately 10,000 classes with 50 images per class, it is necessary to reduce its size for practical purposes. Additionally, current models, such as LLaVA, have limitations in handling an excessive number of options. Therefore, we have opted to sample the dataset to manage the number of classes and reduce the computational load.
% \paragraph{Random Sampling Justification}
% Initially, we considered sampling all species from a single order, family, or genus. However, this approach resulted in classes that were too similar, making the classification task more challenging than our models could handle. By employing random sampling, the selected classes that are likely sufficiently distinct from each other, with only a few potentially confusing cases.

% Random sampling also reduces the risk of introducing human bias into the selection process, making it a more defensible approach compared to sampling based on performance metrics. 

% \subparagraph{Data Filtering}
% iNaturalist dataset contains a large number of noisy or low quality images. To ensure the quality of the dataset, we implemented an automatic filtering process to eliminate low-quality images. This step is crucial to prevent noise from negatively impacting model performance. Common issues in low-quality images include:

% 1. \textbf{Blurriness}: Images where the main subject is not in focus.
% 2. \textbf{Species Not Present}: Instances where the species is not visible (e.g., only showing its nest or footprint).
% 3. \textbf{Incomplete Specimen}: Images depicting only parts of deceased animals or broken bodies.
% 4. \textbf{Obstructions}: Cases where the species is almost entirely blocked by objects, making identification impossible.

% To improve image quality, we use CLIP score to select the images with top scores. Scores are calculated by evaluating similarity score with [
%         "a photo of an animal",
%         f"a photo of a \{common\_name\}"
%     ]. We rank the images according to this score and selected top 100 images. We randomly split the images to obtain 50 images for training, 20 images for validation and 30 images for testing. 



\section{Experiments}
\label{section5}

In this section, we conduct extensive experiments to show that \ourmethod~can significantly speed up the sampling of existing MR Diffusion. To rigorously validate the effectiveness of our method, we follow the settings and checkpoints from \cite{luo2024daclip} and only modify the sampling part. Our experiment is divided into three parts. Section \ref{mainresult} compares the sampling results for different NFE cases. Section \ref{effects} studies the effects of different parameter settings on our algorithm, including network parameterizations and solver types. In Section \ref{analysis}, we visualize the sampling trajectories to show the speedup achieved by \ourmethod~and analyze why noise prediction gets obviously worse when NFE is less than 20.


\subsection{Main results}\label{mainresult}

Following \cite{luo2024daclip}, we conduct experiments with ten different types of image degradation: blurry, hazy, JPEG-compression, low-light, noisy, raindrop, rainy, shadowed, snowy, and inpainting (see Appendix \ref{appd1} for details). We adopt LPIPS \citep{zhang2018lpips} and FID \citep{heusel2017fid} as main metrics for perceptual evaluation, and also report PSNR and SSIM \citep{wang2004ssim} for reference. We compare \ourmethod~with other sampling methods, including posterior sampling \citep{luo2024posterior} and Euler-Maruyama discretization \citep{kloeden1992sde}. We take two tasks as examples and the metrics are shown in Figure \ref{fig:main}. Unless explicitly mentioned, we always use \ourmethod~based on SDE solver, with data prediction and uniform $\lambda$. The complete experimental results can be found in Appendix \ref{appd3}. The results demonstrate that \ourmethod~converges in a few (5 or 10) steps and produces samples with stable quality. Our algorithm significantly reduces the time cost without compromising sampling performance, which is of great practical value for MR Diffusion.


\begin{figure}[!ht]
    \centering
    \begin{minipage}[b]{0.45\textwidth}
        \centering
        \includegraphics[width=1\textwidth, trim=0 20 0 0]{figs/main_result/7_lowlight_fid.pdf}
        \subcaption{FID on \textit{low-light} dataset}
        \label{fig:main(a)}
    \end{minipage}
    \begin{minipage}[b]{0.45\textwidth}
        \centering
        \includegraphics[width=1\textwidth, trim=0 20 0 0]{figs/main_result/7_lowlight_lpips.pdf}
        \subcaption{LPIPS on \textit{low-light} dataset}
        \label{fig:main(b)}
    \end{minipage}
    \begin{minipage}[b]{0.45\textwidth}
        \centering
        \includegraphics[width=1\textwidth, trim=0 20 0 0]{figs/main_result/10_motion_fid.pdf}
        \subcaption{FID on \textit{motion-blurry} dataset}
        \label{fig:main(c)}
    \end{minipage}
    \begin{minipage}[b]{0.45\textwidth}
        \centering
        \includegraphics[width=1\textwidth, trim=0 20 0 0]{figs/main_result/10_motion_lpips.pdf}
        \subcaption{LPIPS on \textit{motion-blurry} dataset}
        \label{fig:main(d)}
    \end{minipage}
    \caption{\textbf{Perceptual evaluations on \textit{low-light} and \textit{motion-blurry} datasets.}}
    \label{fig:main}
\end{figure}

\subsection{Effects of parameter choice}\label{effects}

In Table \ref{tab:ablat_param}, we compare the results of two network parameterizations. The data prediction shows stable performance across different NFEs. The noise prediction performs similarly to data prediction with large NFEs, but its performance deteriorates significantly with smaller NFEs. The detailed analysis can be found in Section \ref{section5.3}. In Table \ref{tab:ablat_solver}, we compare \ourmethod-ODE-d-2 and \ourmethod-SDE-d-2 on the \textit{inpainting} task, which are derived from PF-ODE and reverse-time SDE respectively. SDE-based solver works better with a large NFE, whereas ODE-based solver is more effective with a small NFE. In general, neither solver type is inherently better.


% In Table \ref{tab:hazy}, we study the impact of two step size schedules on the results. On the whole, uniform $\lambda$ performs slightly better than uniform $t$. Our algorithm follows the method of \cite{lu2022dpmsolverplus} to estimate the integral part of the solution, while the analytical part does not affect the error.  Consequently, our algorithm has the same global truncation error, that is $\mathcal{O}\left(h_{max}^{k}\right)$. Note that the initial and final values of $\lambda$ depend on noise schedule and are fixed. Therefore, uniform $\lambda$ scheduling leads to the smallest $h_{max}$ and works better.

\begin{table}[ht]
    \centering
    \begin{minipage}{0.5\textwidth}
    \small
    \renewcommand{\arraystretch}{1}
    \centering
    \caption{Ablation study of network parameterizations on the Rain100H dataset.}
    % \vspace{8pt}
    \resizebox{1\textwidth}{!}{
        \begin{tabular}{cccccc}
			\toprule[1.5pt]
            % \multicolumn{6}{c}{Rainy} \\
            % \cmidrule(lr){1-6}
             NFE & Parameterization      & LPIPS\textdownarrow & FID\textdownarrow &  PSNR\textuparrow & SSIM\textuparrow  \\
            \midrule[1pt]
            \multirow{2}{*}{50}
             & Noise Prediction & \textbf{0.0606}     & \textbf{27.28}   & \textbf{28.89}     & \textbf{0.8615}    \\
             & Data Prediction & 0.0620     & 27.65   & 28.85     & 0.8602    \\
            \cmidrule(lr){1-6}
            \multirow{2}{*}{20}
              & Noise Prediction & 0.1429     & 47.31   & 27.68     & 0.7954    \\
              & Data Prediction & \textbf{0.0635}     & \textbf{27.79}   & \textbf{28.60}     & \textbf{0.8559}    \\
            \cmidrule(lr){1-6}
            \multirow{2}{*}{10}
              & Noise Prediction & 1.376     & 402.3   & 6.623     & 0.0114    \\
              & Data Prediction & \textbf{0.0678}     & \textbf{29.54}   & \textbf{28.09}     & \textbf{0.8483}    \\
            \cmidrule(lr){1-6}
            \multirow{2}{*}{5}
              & Noise Prediction & 1.416     & 447.0   & 5.755     & 0.0051    \\
              & Data Prediction & \textbf{0.0637}     & \textbf{26.92}   & \textbf{28.82}     & \textbf{0.8685}    \\       
            \bottomrule[1.5pt]
        \end{tabular}}
        \label{tab:ablat_param}
    \end{minipage}
    \hspace{0.01\textwidth}
    \begin{minipage}{0.46\textwidth}
    \small
    \renewcommand{\arraystretch}{1}
    \centering
    \caption{Ablation study of solver types on the CelebA-HQ dataset.}
    % \vspace{8pt}
        \resizebox{1\textwidth}{!}{
        \begin{tabular}{cccccc}
			\toprule[1.5pt]
            % \multicolumn{6}{c}{Raindrop} \\     
            % \cmidrule(lr){1-6}
             NFE & Solver Type     & LPIPS\textdownarrow & FID\textdownarrow &  PSNR\textuparrow & SSIM\textuparrow  \\
            \midrule[1pt]
            \multirow{2}{*}{50}
             & ODE & 0.0499     & 22.91   & 28.49     & 0.8921    \\
             & SDE & \textbf{0.0402}     & \textbf{19.09}   & \textbf{29.15}     & \textbf{0.9046}    \\
            \cmidrule(lr){1-6}
            \multirow{2}{*}{20}
              & ODE & 0.0475    & 21.35   & 28.51     & 0.8940    \\
              & SDE & \textbf{0.0408}     & \textbf{19.13}   & \textbf{28.98}    & \textbf{0.9032}    \\
            \cmidrule(lr){1-6}
            \multirow{2}{*}{10}
              & ODE & \textbf{0.0417}    & 19.44   & \textbf{28.94}     & \textbf{0.9048}    \\
              & SDE & 0.0437     & \textbf{19.29}   & 28.48     & 0.8996    \\
            \cmidrule(lr){1-6}
            \multirow{2}{*}{5}
              & ODE & \textbf{0.0526}     & 27.44   & \textbf{31.02}     & \textbf{0.9335}    \\
              & SDE & 0.0529    & \textbf{24.02}   & 28.35     & 0.8930    \\
            \bottomrule[1.5pt]
        \end{tabular}}
        \label{tab:ablat_solver}
    \end{minipage}
\end{table}


% \renewcommand{\arraystretch}{1}
%     \centering
%     \caption{Ablation study of step size schedule on the RESIDE-6k dataset.}
%     % \vspace{8pt}
%         \resizebox{1\textwidth}{!}{
%         \begin{tabular}{cccccc}
% 			\toprule[1.5pt]
%             % \multicolumn{6}{c}{Raindrop} \\     
%             % \cmidrule(lr){1-6}
%              NFE & Schedule      & LPIPS\textdownarrow & FID\textdownarrow &  PSNR\textuparrow & SSIM\textuparrow  \\
%             \midrule[1pt]
%             \multirow{2}{*}{50}
%              & uniform $t$ & 0.0271     & 5.539   & 30.00     & 0.9351    \\
%              & uniform $\lambda$ & \textbf{0.0233}     & \textbf{4.993}   & \textbf{30.19}     & \textbf{0.9427}    \\
%             \cmidrule(lr){1-6}
%             \multirow{2}{*}{20}
%               & uniform $t$ & 0.0313     & 6.000   & 29.73     & 0.9270    \\
%               & uniform $\lambda$ & \textbf{0.0240}     & \textbf{5.077}   & \textbf{30.06}    & \textbf{0.9409}    \\
%             \cmidrule(lr){1-6}
%             \multirow{2}{*}{10}
%               & uniform $t$ & 0.0309     & 6.094   & 29.42     & 0.9274    \\
%               & uniform $\lambda$ & \textbf{0.0246}     & \textbf{5.228}   & \textbf{29.65}     & \textbf{0.9372}    \\
%             \cmidrule(lr){1-6}
%             \multirow{2}{*}{5}
%               & uniform $t$ & 0.0256     & 5.477   & \textbf{29.91}     & 0.9342    \\
%               & uniform $\lambda$ & \textbf{0.0228}     & \textbf{5.174}   & 29.65     & \textbf{0.9416}    \\
%             \bottomrule[1.5pt]
%         \end{tabular}}
%         \label{tab:ablat_schedule}



\subsection{Analysis}\label{analysis}
\label{section5.3}

\begin{figure}[ht!]
    \centering
    \begin{minipage}[t]{0.6\linewidth}
        \centering
        \includegraphics[width=\linewidth, trim=0 20 10 0]{figs/trajectory_a.pdf} %trim左下右上
        \subcaption{Sampling results.}
        \label{fig:traj(a)}
    \end{minipage}
    \begin{minipage}[t]{0.35\linewidth}
        \centering
        \includegraphics[width=\linewidth, trim=0 0 0 0]{figs/trajectory_b.pdf} %trim左下右上
        \subcaption{Trajectory.}
        \label{fig:traj(b)}
    \end{minipage}
    \caption{\textbf{Sampling trajectories.} In (a), we compare our method (with order 1 and order 2) and previous sampling methods (i.e., posterior sampling and Euler discretization) on a motion blurry image. The numbers in parentheses indicate the NFE. In (b), we illustrate trajectories of each sampling method. Previous methods need to take many unnecessary paths to converge. With few NFEs, they fail to reach the ground truth (i.e., the location of $\boldsymbol{x}_0$). Our methods follow a more direct trajectory.}
    \label{fig:traj}
\end{figure}

\textbf{Sampling trajectory.}~ Inspired by the design idea of NCSN \citep{song2019ncsn}, we provide a new perspective of diffusion sampling process. \cite{song2019ncsn} consider each data point (e.g., an image) as a point in high-dimensional space. During the diffusion process, noise is added to each point $\boldsymbol{x}_0$, causing it to spread throughout the space, while the score function (a neural network) \textit{remembers} the direction towards $\boldsymbol{x}_0$. In the sampling process, we start from a random point by sampling a Gaussian distribution and follow the guidance of the reverse-time SDE (or PF-ODE) and the score function to locate $\boldsymbol{x}_0$. By connecting each intermediate state $\boldsymbol{x}_t$, we obtain a sampling trajectory. However, this trajectory exists in a high-dimensional space, making it difficult to visualize. Therefore, we use Principal Component Analysis (PCA) to reduce $\boldsymbol{x}_t$ to two dimensions, obtaining the projection of the sampling trajectory in 2D space. As shown in Figure \ref{fig:traj}, we present an example. Previous sampling methods \citep{luo2024posterior} often require a long path to find $\boldsymbol{x}_0$, and reducing NFE can lead to cumulative errors, making it impossible to locate $\boldsymbol{x}_0$. In contrast, our algorithm produces more direct trajectories, allowing us to find $\boldsymbol{x}_0$ with fewer NFEs.

\begin{figure*}[ht]
    \centering
    \begin{minipage}[t]{0.45\linewidth}
        \centering
        \includegraphics[width=\linewidth, trim=0 0 0 0]{figs/convergence_a.pdf} %trim左下右上
        \subcaption{Sampling results.}
        \label{fig:convergence(a)}
    \end{minipage}
    \begin{minipage}[t]{0.43\linewidth}
        \centering
        \includegraphics[width=\linewidth, trim=0 20 0 0]{figs/convergence_b.pdf} %trim左下右上
        \subcaption{Ratio of convergence.}
        \label{fig:convergence(b)}
    \end{minipage}
    \caption{\textbf{Convergence of noise prediction and data prediction.} In (a), we choose a low-light image for example. The numbers in parentheses indicate the NFE. In (b), we illustrate the ratio of components of neural network output that satisfy the Taylor expansion convergence requirement.}
    \label{fig:converge}
\end{figure*}

\textbf{Numerical stability of parameterizations.}~ From Table 1, we observe poor sampling results for noise prediction in the case of few NFEs. The reason may be that the neural network parameterized by noise prediction is numerically unstable. Recall that we used Taylor expansion in Eq.(\ref{14}), and the condition for the equality to hold is $|\lambda-\lambda_s|<\boldsymbol{R}(s)$. And the radius of convergence $\boldsymbol{R}(t)$ can be calculated by
\begin{equation}
\frac{1}{\boldsymbol{R}(t)}=\lim_{n\rightarrow\infty}\left|\frac{\boldsymbol{c}_{n+1}(t)}{\boldsymbol{c}_n(t)}\right|,
\end{equation}
where $\boldsymbol{c}_n(t)$ is the coefficient of the $n$-th term in Taylor expansion. We are unable to compute this limit and can only compute the $n=0$ case as an approximation. The output of the neural network can be viewed as a vector, with each component corresponding to a radius of convergence. At each time step, we count the ratio of components that satisfy $\boldsymbol{R}_i(s)>|\lambda-\lambda_s|$ as a criterion for judging the convergence, where $i$ denotes the $i$-th component. As shown in Figure \ref{fig:converge}, the neural network parameterized by data prediction meets the convergence criteria at almost every step. However, the neural network parameterized by noise prediction always has components that cannot converge, which will lead to large errors and failed sampling. Therefore, data prediction has better numerical stability and is a more recommended choice.


\section{Discussion}\label{sec:discussion}



\subsection{From Interactive Prompting to Interactive Multi-modal Prompting}
The rapid advancements of large pre-trained generative models including large language models and text-to-image generation models, have inspired many HCI researchers to develop interactive tools to support users in crafting appropriate prompts.
% Studies on this topic in last two years' HCI conferences are predominantly focused on helping users refine single-modality textual prompts.
Many previous studies are focused on helping users refine single-modality textual prompts.
However, for many real-world applications concerning data beyond text modality, such as multi-modal AI and embodied intelligence, information from other modalities is essential in constructing sophisticated multi-modal prompts that fully convey users' instruction.
This demand inspires some researchers to develop multimodal prompting interactions to facilitate generation tasks ranging from visual modality image generation~\cite{wang2024promptcharm, promptpaint} to textual modality story generation~\cite{chung2022tale}.
% Some previous studies contributed relevant findings on this topic. 
Specifically, for the image generation task, recent studies have contributed some relevant findings on multi-modal prompting.
For example, PromptCharm~\cite{wang2024promptcharm} discovers the importance of multimodal feedback in refining initial text-based prompting in diffusion models.
However, the multi-modal interactions in PromptCharm are mainly focused on the feedback empowered the inpainting function, instead of supporting initial multimodal sketch-prompt control. 

\begin{figure*}[t]
    \centering
    \includegraphics[width=0.9\textwidth]{src/img/novice_expert.pdf}
    \vspace{-2mm}
    \caption{The comparison between novice and expert participants in painting reveals that experts produce more accurate and fine-grained sketches, resulting in closer alignment with reference images in close-ended tasks. Conversely, in open-ended tasks, expert fine-grained strokes fail to generate precise results due to \tool's lack of control at the thin stroke level.}
    \Description{The comparison between novice and expert participants in painting reveals that experts produce more accurate and fine-grained sketches, resulting in closer alignment with reference images in close-ended tasks. Novice users create rougher sketches with less accuracy in shape. Conversely, in open-ended tasks, expert fine-grained strokes fail to generate precise results due to \tool's lack of control at the thin stroke level, while novice users' broader strokes yield results more aligned with their sketches.}
    \label{fig:novice_expert}
    % \vspace{-3mm}
\end{figure*}


% In particular, in the initial control input, users are unable to explicitly specify multi-modal generation intents.
In another example, PromptPaint~\cite{promptpaint} stresses the importance of paint-medium-like interactions and introduces Prompt stencil functions that allow users to perform fine-grained controls with localized image generation. 
However, insufficient spatial control (\eg, PromptPaint only allows for single-object prompt stencil at a time) and unstable models can still leave some users feeling the uncertainty of AI and a varying degree of ownership of the generated artwork~\cite{promptpaint}.
% As a result, the gap between intuitive multi-modal or paint-medium-like control and the current prompting interface still exists, which requires further research on multi-modal prompting interactions.
From this perspective, our work seeks to further enhance multi-object spatial-semantic prompting control by users' natural sketching.
However, there are still some challenges to be resolved, such as consistent multi-object generation in multiple rounds to increase stability and improved understanding of user sketches.   


% \new{
% From this perspective, our work is a step forward in this direction by allowing multi-object spatial-semantic prompting control by users' natural sketching, which considers the interplay between multiple sketch regions.
% % To further advance the multi-modal prompting experience, there are some aspects we identify to be important.
% % One of the important aspects is enhancing the consistency and stability of multiple rounds of generation to reduce the uncertainty and loss of control on users' part.
% % For this purpose, we need to develop techniques to incorporate consistent generation~\cite{tewel2024training} into multi-modal prompting framework.}
% % Another important aspect is improving generative models' understanding of the implicit user intents \new{implied by the paint-medium-like or sketch-based input (\eg, sketch of two people with their hands slightly overlapping indicates holding hand without needing explicit prompt).
% % This can facilitate more natural control and alleviate users' effort in tuning the textual prompt.
% % In addition, it can increase users' sense of ownership as the generated results can be more aligned with their sketching intents.
% }
% For example, when users draw sketches of two people with their hands slightly overlapping, current region-based models cannot automatically infer users' implicit intention that the two people are holding hands.
% Instead, they still require users to explicitly specify in the prompt such relationship.
% \tool addresses this through sketch-aware prompt recommendation to fill in the necessary semantic information, alleviating users' workload.
% However, some users want the generative AI in the future to be able to directly infer this natural implicit intentions from the sketches without additional prompting since prompt recommendation can still be unstable sometimes.


% \new{
% Besides visual generation, 
% }
% For example, one of the important aspect is referring~\cite{he2024multi}, linking specific text semantics with specific spatial object, which is partly what we do in our sketch-aware prompt recommendation.
% Analogously, in natural communication between humans, text or audio alone often cannot suffice in expressing the speakers' intentions, and speakers often need to refer to an existing spatial object or draw out an illustration of her ideas for better explanation.
% Philosophically, we HCI researchers are mostly concerned about the human-end experience in human-AI communications.
% However, studies on prompting is unique in that we should not just care about the human-end interaction, but also make sure that AI can really get what the human means and produce intention-aligned output.
% Such consideration can drastically impact the design of prompting interactions in human-AI collaboration applications.
% On this note, although studies on multi-modal interactions is a well-established topic in HCI community, it remains a challenging problem what kind of multi-modal information is really effective in helping humans convey their ideas to current and next generation large AI models.




\subsection{Novice Performance vs. Expert Performance}\label{sec:nVe}
In this section we discuss the performance difference between novice and expert regarding experience in painting and prompting.
First, regarding painting skills, some participants with experience (4/12) preferred to draw accurate and fine-grained shapes at the beginning. 
All novice users (5/12) draw rough and less accurate shapes, while some participants with basic painting skills (3/12) also favored sketching rough areas of objects, as exemplified in Figure~\ref{fig:novice_expert}.
The experienced participants using fine-grained strokes (4/12, none of whom were experienced in prompting) achieved higher IoU scores (0.557) in the close-ended task (0.535) when using \tool. 
This is because their sketches were closer in shape and location to the reference, making the single object decomposition result more accurate.
Also, experienced participants are better at arranging spatial location and size of objects than novice participants.
However, some experienced participants (3/12) have mentioned that the fine-grained stroke sometimes makes them frustrated.
As P1's comment for his result in open-ended task: "\emph{It seems it cannot understand thin strokes; even if the shape is accurate, it can only generate content roughly around the area, especially when there is overlapping.}" 
This suggests that while \tool\ provides rough control to produce reasonably fine results from less accurate sketches for novice users, it may disappoint experienced users seeking more precise control through finer strokes. 
As shown in the last column in Figure~\ref{fig:novice_expert}, the dragon hovering in the sky was wrongly turned into a standing large dragon by \tool.

Second, regarding prompting skills, 3 out of 12 participants had one or more years of experience in T2I prompting. These participants used more modifiers than others during both T2I and R2I tasks.
Their performance in the T2I (0.335) and R2I (0.469) tasks showed higher scores than the average T2I (0.314) and R2I (0.418), but there was no performance improvement with \tool\ between their results (0.508) and the overall average score (0.528). 
This indicates that \tool\ can assist novice users in prompting, enabling them to produce satisfactory images similar to those created by users with prompting expertise.



\subsection{Applicability of \tool}
The feedback from user study highlighted several potential applications for our system. 
Three participants (P2, P6, P8) mentioned its possible use in commercial advertising design, emphasizing the importance of controllability for such work. 
They noted that the system's flexibility allows designers to quickly experiment with different settings.
Some participants (N = 3) also mentioned its potential for digital asset creation, particularly for game asset design. 
P7, a game mod developer, found the system highly useful for mod development. 
He explained: "\emph{Mods often require a series of images with a consistent theme and specific spatial requirements. 
For example, in a sacrifice scene, how the objects are arranged is closely tied to the mod's background. It would be difficult for a developer without professional skills, but with this system, it is possible to quickly construct such images}."
A few participants expressed similar thoughts regarding its use in scene construction, such as in film production. 
An interesting suggestion came from participant P4, who proposed its application in crime scene description. 
She pointed out that witnesses are often not skilled artists, and typically describe crime scenes verbally while someone else illustrates their account. 
With this system, witnesses could more easily express what they saw themselves, potentially producing depictions closer to the real events. "\emph{Details like object locations and distances from buildings can be easily conveyed using the system}," she added.

% \subsection{Model Understanding of Users' Implicit Intents}
% In region-sketch-based control of generative models, a significant gap between interaction design and actual implementation is the model's failure in understanding users' naturally expressed intentions.
% For example, when users draw sketches of two people with their hands slightly overlapping, current region-based models cannot automatically infer users' implicit intention that the two people are holding hands.
% Instead, they still require users to explicitly specify in the prompt such relationship.
% \tool addresses this through sketch-aware prompt recommendation to fill in the necessary semantic information, alleviating users' workload.
% However, some users want the generative AI in the future to be able to directly infer this natural implicit intentions from the sketches without additional prompting since prompt recommendation can still be unstable sometimes.
% This problem reflects a more general dilemma, which ubiquitously exists in all forms of conditioned control for generative models such as canny or scribble control.
% This is because all the control models are trained on pairs of explicit control signal and target image, which is lacking further interpretation or customization of the user intentions behind the seemingly straightforward input.
% For another example, the generative models cannot understand what abstraction level the user has in mind for her personal scribbles.
% Such problems leave more challenges to be addressed by future human-AI co-creation research.
% One possible direction is fine-tuning the conditioned models on individual user's conditioned control data to provide more customized interpretation. 

% \subsection{Balance between recommendation and autonomy}
% AIGC tools are a typical example of 
\subsection{Progressive Sketching}
Currently \tool is mainly aimed at novice users who are only capable of creating very rough sketches by themselves.
However, more accomplished painters or even professional artists typically have a coarse-to-fine creative process. 
Such a process is most evident in painting styles like traditional oil painting or digital impasto painting, where artists first quickly lay down large color patches to outline the most primitive proportion and structure of visual elements.
After that, the artists will progressively add layers of finer color strokes to the canvas to gradually refine the painting to an exquisite piece of artwork.
One participant in our user study (P1) , as a professional painter, has mentioned a similar point "\emph{
I think it is useful for laying out the big picture, give some inspirations for the initial drawing stage}."
Therefore, rough sketch also plays a part in the professional artists' creation process, yet it is more challenging to integrate AI into this more complex coarse-to-fine procedure.
Particularly, artists would like to preserve some of their finer strokes in later progression, not just the shape of the initial sketch.
In addition, instead of requiring the tool to generate a finished piece of artwork, some artists may prefer a model that can generate another more accurate sketch based on the initial one, and leave the final coloring and refining to the artists themselves.
To accommodate these diverse progressive sketching requirements, a more advanced sketch-based AI-assisted creation tool should be developed that can seamlessly enable artist intervention at any stage of the sketch and maximally preserve their creative intents to the finest level. 

\subsection{Ethical Issues}
Intellectual property and unethical misuse are two potential ethical concerns of AI-assisted creative tools, particularly those targeting novice users.
In terms of intellectual property, \tool hands over to novice users more control, giving them a higher sense of ownership of the creation.
However, the question still remains: how much contribution from the user's part constitutes full authorship of the artwork?
As \tool still relies on backbone generative models which may be trained on uncopyrighted data largely responsible for turning the sketch into finished artwork, we should design some mechanisms to circumvent this risk.
For example, we can allow artists to upload backbone models trained on their own artworks to integrate with our sketch control.
Regarding unethical misuse, \tool makes fine-grained spatial control more accessible to novice users, who may maliciously generate inappropriate content such as more realistic deepfake with specific postures they want or other explicit content.
To address this issue, we plan to incorporate a more sophisticated filtering mechanism that can detect and screen unethical content with more complex spatial-semantic conditions. 
% In the future, we plan to enable artists to upload their own style model

% \subsection{From interactive prompting to interactive spatial prompting}


\subsection{Limitations and Future work}

    \textbf{User Study Design}. Our open-ended task assesses the usability of \tool's system features in general use cases. To further examine aspects such as creativity and controllability across different methods, the open-ended task could be improved by incorporating baselines to provide more insightful comparative analysis. 
    Besides, in close-ended tasks, while the fixing order of tool usage prevents prior knowledge leakage, it might introduce learning effects. In our study, we include practice sessions for the three systems before the formal task to mitigate these effects. In the future, utilizing parallel tests (\textit{e.g.} different content with the same difficulty) or adding a control group could further reduce the learning effects.

    \textbf{Failure Cases}. There are certain failure cases with \tool that can limit its usability. 
    Firstly, when there are three or more objects with similar semantics, objects may still be missing despite prompt recommendations. 
    Secondly, if an object's stroke is thin, \tool may incorrectly interpret it as a full area, as demonstrated in the expert results of the open-ended task in Figure~\ref{fig:novice_expert}. 
    Finally, sometimes inclusion relationships (\textit{e.g.} inside) between objects cannot be generated correctly, partially due to biases in the base model that lack training samples with such relationship. 

    \textbf{More support for single object adjustment}.
    Participants (N=4) suggested that additional control features should be introduced, beyond just adjusting size and location. They noted that when objects overlap, they cannot freely control which object appears on top or which should be covered, and overlapping areas are currently not allowed.
    They proposed adding features such as layer control and depth control within the single-object mask manipulation. Currently, the system assigns layers based on color order, but future versions should allow users to adjust the layer of each object freely, while considering weighted prompts for overlapping areas.

    \textbf{More customized generation ability}.
    Our current system is built around a single model $ColorfulXL-Lightning$, which limits its ability to fully support the diverse creative needs of users. Feedback from participants has indicated a strong desire for more flexibility in style and personalization, such as integrating fine-tuned models that cater to specific artistic styles or individual preferences. 
    This limitation restricts the ability to adapt to varied creative intents across different users and contexts.
    In future iterations, we plan to address this by embedding a model selection feature, allowing users to choose from a variety of pre-trained or custom fine-tuned models that better align with their stylistic preferences. 
    
    \textbf{Integrate other model functions}.
    Our current system is compatible with many existing tools, such as Promptist~\cite{hao2024optimizing} and Magic Prompt, allowing users to iteratively generate prompts for single objects. However, the integration of these functions is somewhat limited in scope, and users may benefit from a broader range of interactive options, especially for more complex generation tasks. Additionally, for multimodal large models, users can currently explore using affordable or open-source models like Qwen2-VL~\cite{qwen} and InternVL2-Llama3~\cite{llama}, which have demonstrated solid inference performance in our tests. While GPT-4o remains a leading choice, alternative models also offer competitive results.
    Moving forward, we aim to integrate more multimodal large models into the system, giving users the flexibility to choose the models that best fit their needs. 
    


\section{Conclusion}\label{sec:conclusion}
In this paper, we present \tool, an interactive system designed to help novice users create high-quality, fine-grained images that align with their intentions based on rough sketches. 
The system first refines the user's initial prompt into a complete and coherent one that matches the rough sketch, ensuring the generated results are both stable, coherent and high quality.
To further support users in achieving fine-grained alignment between the generated image and their creative intent without requiring professional skills, we introduce a decompose-and-recompose strategy. 
This allows users to select desired, refined object shapes for individual decomposed objects and then recombine them, providing flexible mask manipulation for precise spatial control.
The framework operates through a coarse-to-fine process, enabling iterative and fine-grained control that is not possible with traditional end-to-end generation methods. 
Our user study demonstrates that \tool offers novice users enhanced flexibility in control and fine-grained alignment between their intentions and the generated images.





\section*{Impact Statement}
This paper presents work whose goal is to advance the field of 
Machine Learning. There are many potential societal consequences 
of our work, none which we feel must be specifically highlighted here.
% Authors are \textbf{required} to include a statement of the potential 
% broader impact of their work, including its ethical aspects and future 
% societal consequences. This statement should be in an unnumbered 
% section at the end of the paper (co-located with Acknowledgements -- 
% the two may appear in either order, but both must be before References), 
% and does not count toward the paper page limit. In many cases, where 
% the ethical impacts and expected societal implications are those that 
% are well established when advancing the field of Machine Learning, 
% substantial discussion is not required, and a simple statement such 
% as the following will suffice:

% ``This paper presents work whose goal is to advance the field of 
% Machine Learning. There are many potential societal consequences 
% of our work, none which we feel must be specifically highlighted here.''

% The above statement can be used verbatim in such cases, but we 
% encourage authors to think about whether there is content which does 
% warrant further discussion, as this statement will be apparent if the 
% paper is later flagged for ethics review.


% In the unusual situation where you want a paper to appear in the
% references without citing it in the main text, use \nocite
% \nocite{langley00}


\section*{Acknowledgment}
This material is based on research supported by the ECOLE program under Cooperative Agreement HR00112390060, with the US Defense Advanced Research Projects Agency (DARPA). The views and conclusions contained herein are those of the authors and should not be interpreted as necessarily representing DARPA, or the U.S. Government.

\bibliography{main}
\bibliographystyle{icml2025}

\newpage
\appendix
\onecolumn
\begin{center}{\bf \Large Appendix}\end{center}
%\section*{Appendix}
\vspace{0.15in}


\paragraph{Organization} 
The appendix is structured as follows: 
We first present the derivations excluded from the main paper due to space limitation in Section~\ref{app:proof}.
Section~\ref{app:ho-intro} introduces the concept and examples of higher-order networks.
Additional explanations on related work are provided in Section~\ref{app:related}. 
Section~\ref{app:detail-HOG-Diff} details the generation process, including the architecture of the proposed denoising network, as well as the training and sampling procedures.
Computational efficiency is discussed in Section~\ref{app:complexity}.
Section~\ref{app:exp_set} outlines the experimental setup, and Section~\ref{app:vis} concludes with visualizations of the generated samples.


\section{Formal Statements and Proofs}
\label{app:proof}
This section presents the formal statements of key theoretical results and their detailed derivations. 
We will recall and more precisely state the propositions before presenting the proof.

\subsection{Diffusion Bridge Process}

In the following, we derive the Generalized Ornstein-Uhlenbeck (GOU) bridge process using Doob's $h$-transform \cite{doob-h-transform1984} and analyze its relationship with the Brownian bridge process.

Recall that the generalized Ornstein-Uhlenbeck (GOU) process is the time-varying OU process.
It is a stationary Gaussian-Markov process whose marginal distribution gradually tends towards a stable mean and variance over time. 
The GOU process $\mathbb{Q}$ is generally defined as follows \cite{GOU1988,IRSDE+ICML2023}:
\begin{equation}
\mathbb{Q}: \mathrm{d}\bm{G}_t=\theta_t\left(\bm{\mu}-\bm{G}_t\right)\mathrm{d}t+g_t\mathrm{d}\bm{W}_t,
\end{equation}
where $\bm{\mu}$ is a given state vector, $\theta_t$ denotes a scalar drift coefficient and $g_t$ represents the diffusion coefficient. At the same time, we require $\theta_t,g_t$ to satisfy the specified relationship $2\sigma^2=g_t^2/\theta_t$, where $\sigma^2$ is a given constant scalar. As a result, its transition probability possesses a closed-form analytical solution:
\begin{equation}
\begin{split}
p\left(\bm{G}_{t}\mid \bm{G}_s\right)
& =\mathcal{N}(\mathbf{m}_{s:t},v_{s:t}^{2}\bm{I}), \\
\mathbf{m}_{s:t} 
& = \bm{\mu}+\left(\bm{G}_s-\bm{\mu}\right)e^{-\bar{\theta}_{s:t}},\\
v_{s:t}^{2} 
&= \sigma^2 \left(1-e^{-2\bar{\theta}_{s:t}}\right).
\end{split}
\end{equation}
Here, $\bar{\theta}_{s:t}=\int_s^t\theta_zdz$. When the starting time $t=0$, we substitute $\bar{\theta}_{0:t}$ with $\bar{\theta}_t$ for notation simplicity. 





\begin{customthe}[Proposition~\ref{pro:OUB}]
%Let $\bm{G}_t$ evolve according to the generalized OU process in \cref{eq:GOU-SDE}, subject to the terminal conditional $\bm{\mu}=\bm{G}_{\tau_k}$. 
%Then, the evolution of the conditional marginal distribution $p(\bm{G}_t \mid \bm{G}_{\tau_k})$ satisfies the following SDE:
The conditional marginal distribution $p(\bm{G}_t\mid\bm{G}_{\tau_k})$ then evolves according to the following SDE:
\begin{equation}
%\fontsize{8.5pt}{8.5pt}\selectfont
\mathrm{d}\bm{G}_t = \theta_t \left( 1 + \frac{2}{e^{2\bar{\theta}_{t:\tau_k}}-1}  \right)(\bm{G}_{\tau_k} - \bm{G}_t)  \mathrm{d}t 
+ g_{k,t} \mathrm{d}\bm{W}_t.\nonumber
%\fontsize{10pt}{10pt}\selectfont
\end{equation}
The conditional transition probability $p(\bm{G}_t \mid \bm{G}_{\tau_{k-1}}, \bm{G}_{\tau_k})$ has analytical form as follows:
\begin{equation}
\begin{split}
&p(\bm{G}_t \mid  \bm{G}_{\tau_{k-1}}, \bm{G}_{\tau_k}) 
= \mathcal{N}(\bar{\mathbf{m}}_t, \bar{v}_t^2 \bm{I}),\\
&\bar{\mathbf{m}}_t = 
\bm{G}_{\tau_k} + (\bm{G}_{\tau_{k-1}}-\bm{G}_{\tau_k})e^{-\bar{\theta}_{\tau_{k-1}:t}} 
\frac{v_{t:\tau_k}^2}{v_{\tau_{k-1}:\tau_k}^2}, \\
&\bar{v}_t^2 = {v_{\tau_{k-1}:t}^2 v_{t:\tau_k}^2}/{v_{\tau_{k-1}:\tau_k}^2}.
\end{split}
\end{equation}
Here, $\bar{\theta}_{a:b}=\int_a^b \theta_s  \mathrm{d}s$, and $v_{a:b}=\sigma^2(1-e^{-2\bar{\theta}_{a:b}})$.
Let $\bm{G}_t$ evolve according to the generalized OU process in \cref{eq:GOU-SDE}, subject to the terminal conditional $\bm{\mu}=\bm{G}_{\tau_k}$. 
%
The conditional marginal distribution $p(\bm{G}_t\mid\bm{G}_{\tau_k})$ then evolves according to the following SDE:
\begin{equation}
\mathrm{d}\bm{G}_t = \theta_t \left( 1 + \frac{2}{e^{2\bar{\theta}_{t:\tau_k}}-1}  \right)(\bm{G}_{\tau_k} - \bm{G}_t)  \mathrm{d}t 
+ g_{k,t} \mathrm{d}\bm{W}_t.
\label{eq:GOUB-SDE}
\end{equation}
The conditional transition probability $p(\bm{G}_t \mid \bm{G}_{\tau_{k-1}}, \bm{G}_{\tau_k})$ has analytical form as follows:
\begin{equation}
\begin{split}
&p(\bm{G}_t \mid  \bm{G}_{\tau_{k-1}}, \bm{G}_{\tau_k}) 
= \mathcal{N}(\bar{\mathbf{m}}_t, \bar{v}_t^2 \bm{I}),\\
&\bar{\mathbf{m}}_t = 
\bm{G}_{\tau_k} + (\bm{G}_{\tau_{k-1}}-\bm{G}_{\tau_k})e^{-\bar{\theta}_{\tau_{k-1}:t}} 
\frac{v_{t:\tau_k}^2}{v_{\tau_{k-1}:\tau_k}^2}, \\
&\bar{v}_t^2 = {v_{\tau_{k-1}:t}^2 v_{t:\tau_k}^2}/{v_{\tau_{k-1}:\tau_k}^2}.
\end{split}
\end{equation}
Here, $\bar{\theta}_{a:b}=\int_a^b \theta_s  \mathrm{d}s$, and $v_{a:b}=\sigma^2(1-e^{-2\bar{\theta}_{a:b}})$.
\end{customthe}

\begin{proof}
To simplify the notion, in the $k$-th generation step, we adopt the following conventions: 
 $T=\tau_k$, $\mathbf{x}_t = \bm{G}_t^{(k)}$, $0=\tau_{k-1}$, $\mathbf{x}_0=\bm{G}_{\tau_{k-1}}$, $\mathbf{x}_T=\bm{G}_{\tau_k}$. 

From \cref{eq:GOU-p}, we can derive the following conditional distribution
\begin{equation}
    p(\mathbf{x}_T \mid \mathbf{x}_t)=\mathcal{N}(
    \mathbf{x}_T + (\mathbf{x}_t-\mathbf{x}_T) e^{\bar{\theta}_{t:T}},
    v_{t:T}^2 \bm{I}).
\end{equation}
Hence, the $h$-function can be directly computed as:
\begin{equation}
\begin{split}
\bm{h}(\mathbf{x}_t, t, \mathbf{x}_T, T) 
& = \nabla_{\mathbf{x}_t} \log p(\mathbf{x}_T \mid \mathbf{x}_t)\\
& = -\nabla_{\mathbf{x}_t} \left[\frac{(\mathbf{x}_t - \mathbf{x}_T)^2 e^{-2 \bar{\theta}_{t:T}}}{2 v_{t:T}^2} + const \right]\\
& = (\mathbf{x}_T - \mathbf{x}_t) \frac{e^{-2 \bar{\theta}_{t:T}}}{v_{t:T}^2} \\
& = (\mathbf{x}_T - \mathbf{x}_t) \sigma^{-2}/(e^{2\bar{\theta}_{t:T}}-1).
\end{split}
\end{equation}


Then the Doob's $h$-transform yields the representation of an endpoint $\mathbf{x}_T$ conditioned process defined by the following SDE: 
% 
\begin{equation}
\begin{split}
\mathrm{d}\mathbf{x}_t 
&= \left[ f(\mathbf{x}_t, t) + g_t^2 \bm{h}(\mathbf{x}_t, t, \mathbf{x}_T, T) \right] \mathrm{d}t + g_t \mathrm{d}\mathbf{w}_t\\
&= \left( \theta_t + \frac{g_t^2}{\sigma^2 (e^{2\bar{\theta}_{t:T}}-1)}  \right)(\mathbf{x}_T - \mathbf{x}_t)  \mathrm{d}t + g_t \mathrm{d}\mathbf{w}_t \\
& = \theta_t \left( 1 + \frac{2}{e^{2\bar{\theta}_{t:T}}-1}  \right)(\mathbf{x}_T - \mathbf{x}_t)  \mathrm{d}t + g_t \mathrm{d}\mathbf{w}_t.
\end{split}
\end{equation}

Given that the joint distribution of $[\mathbf{x}_0, \mathbf{x}_t, \mathbf{x}_T]$ is multivariate normal, the conditional distribution $p(\mathbf{x}_t \mid \mathbf{x}_0, \mathbf{x}_T)$ is also Gaussian:
\begin{equation}
    p(\mathbf{x}_t\mid \mathbf{x}_0, \mathbf{x}_T) = \mathcal{N}(\bar{\mathbf{m}}_t, \bar{v}_t^2 \bm{I}),
\end{equation}
where the mean $\bar{\mathbf{m}}_t$ and variance $\bar{v}_t^2$ are determined using the conditional formulas for multivariate normal variables:
\begin{equation}
\begin{split}
\bar{\mathbf{m}}_t 
=  \mathbb{E}[\mathbf{x}_t\mid \mathbf{x}_0 \mid \mathbf{x}_T]
=\mathbb{E}[\mathbf{x}_t\mid \mathbf{x}_0]+\mathrm{Cov}(\mathbf{x}_t,\mathbf{x}_T\mid \mathbf{x}_0)\mathrm{Var}(\mathbf{x}_T\mid \mathbf{x}_0)^{-1}(\mathbf{x}_T-\mathbb{E}[\mathbf{x}_T\mid \mathbf{x}_0]),\\
\bar{v}_t^2
= \mathrm{Var}(\mathbf{x}_t\mid \mathbf{x}_0 \mid \mathbf{x}_T)
=\mathrm{Var}(\mathbf{x}_t\mid \mathbf{x}_0)-\mathrm{Cov}(\mathbf{x}_t,\mathbf{x}_T\mid \mathbf{x}_0)\mathrm{Var}(\mathbf{x}_T\mid \mathbf{x}_0)^{-1}\mathrm{Cov}(\mathbf{x}_T,\mathbf{x}_t\mid \mathbf{x}_0).
\end{split}
\label{eq:OUB-m-v}
\end{equation}

Notice that
\begin{equation}
    \mathrm{Cov}(\mathbf{x}_t,\mathbf{x}_T\mid \mathbf{x}_0)=\mathrm{Cov}\left(\mathbf{x}_t,(\mathbf{x}_t-\mathbf{x}_T)e^{-\bar{\theta}_{t:T}}\mid \mathbf{x}_0\right)=e^{-\bar{\theta}t:T}\mathrm{Var}(\mathbf{x}_t\mid \mathbf{x}_0).
\end{equation}
By substituting this and the results in \cref{eq:GOU-p} into \cref{eq:OUB-m-v}, we can obtain
\begin{equation}
\begin{split}
\bar{\mathbf{m}}_t 
& = \left(\mathbf{x}_T+(\mathbf{x}_0-\mathbf{x}_T)e^{-\bar{\theta}_t}\right)
+ \left(e^{-\bar{\theta}_{t:T}} v_t^2\right)
/ v_T^2
\cdot \left(\mathbf{x}_T - \mathbf{x}_T - (\mathbf{x}_0 - \mathbf{x}_T)e^{-\bar{\theta}_T}\right) \\
& = \mathbf{x}_T + (\mathbf{x}_0-\mathbf{x}_T) \left(e^{-\bar{\theta}_t} -  e^{-\bar{\theta}_{t:T}}e^{-\bar{\theta}_T} v_t^2 /v_T^2\right) \\
& = \mathbf{x}_T + (\mathbf{x}_0-\mathbf{x}_T)e^{-\bar{\theta}_t} 
\left(\frac{1-e^{-2\bar{\theta}_{T}}-e^{-2\bar{\theta}_{t:T}}(1-e^{-2\bar{\theta}_t})}{1-e^{-2\bar{\theta}_{T}}}\right)\\
& = \mathbf{x}_T + (\mathbf{x}_0-\mathbf{x}_T)e^{-\bar{\theta}_t} 
v_{t:T}^2/v_T^2,
\end{split}
\end{equation}
and 
\begin{equation}
\begin{split}
\bar{v}_t^2
& = v_t^2 - \left(e^{-\bar{\theta}_{t:T}} v_t^2 \right)^2 / v_T^2\\
& = \frac{v_t^2}{v_T^2}(v_T^2-e^{-2\bar{\theta}_{t:T}}v_t^2)\\
& = \frac{v_t^2}{v_T^2} \sigma^2\left(1-e^{-2\bar{\theta}_T} - e^{-2\bar{\theta}_{t:T}}(1-e^{-\bar{2\theta}_t})\right)\\
& = v_t^2 v_{t:T}^2/ v_T^2.
\end{split}
\end{equation}

Finally, we conclude the proof by reverting to the original notations.
\end{proof}



Note that the generalized OU bridge process, also referred to as the conditional GOU process, has been studied theoretically in previous works \cite{salminen1984conditional,GOUB2021,GOUB+ICML2024}. However, we are the first to demonstrate its effectiveness in explicitly learning higher-order structures within the graph generation process.


\paragraph{Brownian Bridge Process}  
In the following, we demonstrate that the Brownian bridge process is a particular case of the generalized OU bridge process when $\theta_t$ approaches zero.

Assume $\theta_t = \theta$ is a constant that tends to zero, we obtain 
\begin{equation}
    \bar{\theta}_{a:b}=\int_a^b \theta_s \diff{s} = \theta (b-a)\rightarrow 0.
\end{equation}

Consider the term $ e^{2\bar{\theta}_{t:\tau_k}}-1$, we approximate the exponential function using a first-order Taylor expansion for small $\bar{\theta}_{t:\tau_k}$:
\begin{equation}
    e^{2\bar{\theta}_{t:\tau_k}}-1 
    \approx
    2\bar{\theta}_{t:\tau_k}
        \rightarrow
        2\theta(\tau_k - t).
\end{equation}
Hence, the drift term in the generalized OU bridge simplifies to
\begin{equation}
    \theta_t \left( 1 + \frac{2}{e^{2\bar{\theta}_{t:\tau_k}}-1}\right)
     \approx
     \theta\left(1+\frac{2}{2\theta(\tau_k-t)}\right)
     \rightarrow
     \frac{1}{\tau_k-t}.
\end{equation}

Consequently, in the limit $\theta_t \rightarrow 0$, the generalized OU bridge process described in \cref{eq:GOUB-SDE} can be modelled by the following SDE:
\begin{equation}
    \mathrm{d}\bm{G}_t=  \frac{\bm{G}_{\tau_k}-\bm{G}_t}{\tau_k-t}\mathrm{d}t+
    g_{k,t}\mathrm{d}\bm{W}_t.
\end{equation}
This equation precisely corresponds to the SDE representation of the classical Brownian bridge process.


In contrast to the generalized OU bridge process in \cref{eq:GOUB-SDE}, the evolution of the Brownian bridge is fully determined by the noise schedule $g_{k,t}$, resulting in a simpler SDE representation. 
However, this constraint in the Brownian bridge reduces the flexibility in designing the generative process.


Note that the Brownian bridge is an endpoint-conditioned process relative to a reference Brownian motion, which the SDE governs:
\begin{equation}
    \mathrm{d}\bm{G}_t=  
    g_{t}\mathrm{d}\bm{W}_t.
\end{equation}
This equation describes a pure diffusion process without drift, making it a specific instance of the generalized OU process in \cref{eq:GOU-SDE}.

\subsection{Proof of Proposition~\ref{pro:training}}

To establish proof, we begin by introducing essential definitions and assumptions.

\begin{definition}[$\beta$-smooth]
A function $f:\mathbb{R}^m  \to \mathbb{R}^n$ is said to be $\beta$-smooth if and only if
\begin{equation}
    \norm{f(\mathbf{w})-f(\mathbf{v})-\nabla f(\mathbf{v})(\mathbf{w}-\mathbf{v})} \leq \frac{\beta}{2} \norm{\mathbf{w}-\mathbf{v}}^2, \forall \mathbf{w},\mathbf{v}\in \mathbb{R}^m.
\end{equation}
\end{definition}

\begin{customthe}[Proposition~\ref{pro:training}\textnormal{ (Formal)}]
Let $\ell^{(k)}(\boldsymbol{\theta})$ be a loss function that is $\beta$-smooth and satisfies the $\mu$-PL (Polyak-Łojasiewicz) condition in the ball $B\left(\boldsymbol{\theta}_0, R\right)$ of radius $R=2N \sqrt{2 \beta \ell^{(k)}\left(\boldsymbol{\theta}_0\right)}/(\mu \delta)$, where $\delta>0$. 
%
Then, with probability $1-\delta$ over the choice of mini-batch of size $b$, stochastic gradient descent (SGD) with a learning rate $\eta^*=\frac{\mu N}{N \beta\left(N^2 \beta+\mu(b-1)\right)}$ converges to a global solution in the ball $B$ with exponential convergence rage: 
\begin{equation}
 \mathbb{E}\left[\ell^{(k)}\left(\boldsymbol{\theta}_i\right)\right] \leq\left(1-\frac{b \mu^2}{\beta N\left(\beta N^2+\mu(b-1)\right)}\right)^i \ell^{(k)}\left(\boldsymbol{\theta}_0\right).
\end{equation}
Here, $N$ denotes the size of the training dataset.
Furthermore, the proposed generative model yields a smaller smoothness constant $\beta_{\text{HOG-Diff}}$ compared to that of the classical model $\beta_{\text {classical}}$, \ie, $\beta_\text{HOG-Diff} \leq \beta_{\text {classical}}$, implying that the learned distribution in HOG-Diff converges to the target distribution faster than classical generative models.
\end{customthe}

\begin{proof}
Assume that the loss function $\ell^{(k)}(\bm{\theta})$ in \cref{eq:final-loss} is minimized using standard Stochastic Gradient Descent (SGD) on a training dataset $\mathcal{S}=\{\mathbf{x}^i\}_{i=1}^N$. At the $i$-th iteration, parameter $\bm{\theta}_i$ is updated using a mini-batch of size $b$ as follows:
\begin{equation}
    \bm{\theta}_{i+1} \triangleq \bm{\theta}_i - \eta \nabla \ell^{(k)}(\bm{\theta}_i),
\end{equation}
where $\eta$ is the learning rate.


Following \citet{liu2020toward} and \citet{GSDM+TPAMI2023}, we assume that $\ell^{(k)}(\bm{\theta})$ is $\beta$-smooth and satisfies the $\mu$-PL condition in the ball $B(\bm{\theta}_0, R)$ with $R=2N\sqrt{2\beta \ell^{(k)}(\bm{\theta}_0)}/(\mu\delta)$ where $\delta>0$. 
%
Then, with probability $1-\delta$ over the choice of mini-batch of size $b$, SGD with a learning  rate $\eta^* =\frac{\mu N}{N\beta (N^2\beta +\mu(b-1))}$ converges to a global solution in the ball $B(\bm{\theta}_0, R)$ with exponential convergence rate \cite{liu2020toward}:
\begin{equation}
\mathbb{E}[\ell^{(k)}(\bm{\theta}_i)] 
\leq \left(1-\frac{b\mu\eta^*}{N}\right)^i \ell^{(k)}(\bm{\theta}_0)
= \left(1-\frac{b\mu^2}{\beta N(\beta N^2+\mu(b-1))}\right)^i \ell^{(k)}(\bm{\theta}_0).
\end{equation}

% 2------
Next, we show that the proposed framework has a smaller smoothness constant than the classical one-step model. 
Therefore, we focus exclusively on the spectral component $||\bm{s}^{(k)}_{\bm{\theta},\bm{\Lambda}} - \nabla_{\bm{\Lambda}} \log p_t(\bm{G}_t | \bm{G}_{\tau_k})||_2^2$ from the full loss function in \cref{eq:final-loss}, as the feature-related part of the loss function in HOG-Diff aligns with that of the classical framework.  
For simplicity, we use the notation $\bar{\ell}(\bm{\theta})=||\bm{s}^{(k)}_{\bm{\theta},\bm{\Lambda}} - \nabla_{\bm{\Lambda}} \log p_t(\bm{G}_t | \bm{G}_{\tau_k})||^2 = ||\bm{s}_{\bm{\theta}}(\mathbf{x}_t) - \nabla_{\mathbf{x}} \log p_t(\mathbf{x}_t)||^2$ as the feature-related part of the loss.%, and let $\bar{\ell}(\bm{\varphi}) = \mathbb{E} ||s_{\bm{\varphi}}(\mathbf{x}_t) - \nabla_{\mathbf{x}} q_t(\mathbf{x}_t|\mathbf{x}_0)||^2$ be its classical counterpart.


Next, we verify that $\bar{\ell}(\bm{\theta})$ is $\beta$-smooth under the assumptions given.
Notice that the gradient of the loss function is given by:
\begin{equation}
\nabla \bar{\ell}(\bm{\theta})=2\mathbb{E}\left[(\bm{s}_{\bm{\theta}}(\mathbf{x})-\nabla\log p(\mathbf{x}))^\top\nabla_{\bm{\theta}} s_{\bm{\theta}}(\mathbf{x})\right]
\end{equation}
Hence,
\begin{equation}
\begin{split}
&\|\nabla \bar{\ell}(\bm{\theta}_1)-\nabla \bar{\ell}(\bm{\theta}_2)\|
=2\left\|\mathbb{E}\left[(\bm{s}_{\bm{\theta}_1}(\mathbf{x})-\nabla\log p(\mathbf{x}))^\top\nabla \bm{s}_{\bm{\theta}_1}(\mathbf{x})-(\bm{s}_{\bm{\theta}_2}(\mathbf{x})-\nabla\log p(\mathbf{x}))^\top\nabla \bm{s}_{\bm{\theta}_2}(\mathbf{x})\right]\right\|\\
&\leq 2\mathbb{E}[\|\bm{s}_{\bm{\theta}_{1}}(\mathbf{x})-\bm{s}_{\bm{\theta}_2}(\mathbf{x})\|\cdot\|\nabla \bm{s}_{\bm{\theta}_1}(\mathbf{x})\|  
+\|\bm{s}_{\bm{\theta}_2}(\mathbf{x})-\nabla\log p(\mathbf{x})\|\cdot\|\nabla \bm{s}_{\bm{\theta}_1}(\mathbf{x})-\nabla \bm{s}_{\bm{\theta}_2}(\mathbf{x})\|].
\end{split}
\end{equation}

Suppose $\|\nabla_{\bm{\theta}} \bm{s}_{\bm{\theta}}(\mathbf{x})\|\leq C_1$ and $\|\bm{s}_{\bm{\theta}}(\mathbf{x})-\nabla\log p(\mathbf{x})\|\leq C_2$, then we can obtain
\begin{equation}
\begin{split}
\|\nabla \bar{\ell}(\bm{\theta}_1)-\nabla \bar{\ell}(\bm{\theta}_2)\|\
& \leq 2 \mathbb{E} \left[C_1 \beta_{\bm{s}_{\bm{\theta}}}\|\bm{\theta}_1-\bm{\theta}_2\|+C_2\beta_{\nabla \bm{s}_{\bm{\theta}}}\|\bm{\theta}_1-\bm{\theta}_2\| \right] \\
& =2(\beta_{\bm{s}_{\bm{\theta}}} C_1+C_2\beta_{\nabla \bm{s}_{\bm{\theta}}})\|\bm{\theta}_1-\bm{\theta}_2\|.
\end{split}
\end{equation}

To satisfy the $\beta$-smooth of $\bar{\ell}(\bm{\theta})$, we require that
\begin{equation}
    2(C_1\beta_{\bm{s}_{\bm{\theta}}}+C_2\beta_{\nabla \bm{s}_{\bm{\theta}}}) 
\leq \beta.
\end{equation}

This implies that the distribution learned by the proposed framework can converge to the target distribution. Therefore, following \citet{CCDF+CVPR2022}, we further assume that $\bm{s}_{\bm{\theta}}$ is a sufficiently expressive parameterized score function so that 
$\beta_{\bm{s}_{\bm{\theta}}} =  \beta_{\nabla \log p_{t|\tau_{k-1}}}$ and $\beta_{\nabla^2 \bm{s}_{\bm{\theta}}} =  \beta_{\nabla^2 \log p_{t|\tau_{k-1}}}$.


%
Consider the loss function of classical generative models goes as: $\bar{\ell}(\bm{\varphi}) = \mathbb{E} ||\bm{s}_{\bm{\varphi}}(\mathbf{x}_t) - \nabla_{\mathbf{x}_t} q_t(\mathbf{x}_t|\mathbf{x}_0)||^2$.
To demonstrate that the proposed framework converges faster to the target distribution compared to the classical one-step generation framework, it suffices to show that: $\beta_{\nabla p_{t|\tau_{k-1}}} \leq \beta_{\nabla q_{t|0}}$ and $\beta_{\nabla^2 p_{t|\tau_{k-1}}} \leq \beta_{\nabla^2 q_{t|0}}$.

Let $\mathbf{x}\sim q_{t|0}$ and $\mathbf{x}'\sim p_{t|\tau_{k-1}}$. Since we inject topological information from $\mathbf{x}$ into $\mathbf{x}^{\prime}$, $\mathbf{x}'$ can be viewed as being obtained by adding noise to $\mathbf{x}$. Hence, we can model $\mathbf{x}'$ as $\mathbf{x}' = \mathbf{x} + \epsilon$ where $\epsilon \sim \mathcal{N}(\mathbf{0},\sigma^2 \bm{I})$. The variance of Gaussian noise $\sigma^2$ controls the information remained in $z'$. 
Hence, its distribution can be expressed as $p(\mathbf{x}')=\int q(\mathbf{x}'-\epsilon)\pi(\epsilon)\diff\epsilon$.

Therefore, we can obtain
\begin{equation}
\begin{split}
||\nabla_{\mathbf{x}'}^k p(\mathbf{x}'_1) - \nabla_{\mathbf{x}'}^k p(\mathbf{x}'_2)||
& =|| \nabla_{\mathbf{x}'}^k \int \left(q(\mathbf{x}_1'-\epsilon)-q(\mathbf{x}_2'-\epsilon)\right)\pi(\epsilon)\diff{\epsilon}||\\
&\leq  \int ||\nabla_{\mathbf{x}'}^k q(\mathbf{x}_1'-\epsilon) - \nabla_{\mathbf{x}'}^k q(\mathbf{x}_2'-\epsilon)|| \pi(\epsilon)\diff{\epsilon} \\ %变量代换
& \leq ||\nabla_{\mathbf{x}'}^k q(\mathbf{x}') ||_{\mathrm{lip}} (\mathbf{x}_1'-\mathbf{x}_2')  \int \pi(\epsilon)\diff{\epsilon}\\
& \leq ||\nabla_{\mathbf{x}'}^k q(\mathbf{x}') ||_{\mathrm{lip}} (\mathbf{x}_1'-\mathbf{x}_2').
\end{split}
\end{equation}


Hence, $||\nabla_{\mathbf{x}'}^k \log p(\mathbf{x}')||_{\mathrm{lip}} \leq ||\nabla_{\mathbf{x}'}^k \log q(\mathbf{x}')||_{\mathrm{lip}}$.

By setting $k=3$ and $k=4$, we can obtain $\beta_{\nabla \log p_{t|\tau_{k-1}}} \leq \beta_{\nabla \log q_{t|0}}$ and $\beta_{\nabla^2 \log p_{t|\tau_{k-1}}} \leq \beta_{\nabla^2 \log q_{t|0}}$. 
Therefore $\beta_{\text{HOG-Diff}}\leq \beta_{\text{classical}}$, implying that the training process of HOG-Diff ($\bm{s}_{\bm{\theta}}$) will converge faster than the classical generative framework ($\bm{s}_{\bm{\varphi}}$).

\end{proof}








\subsection{Proof of Proposition~\ref{pro:reconstruction-error}}

Here, we denote the expected reconstruction error at each generation process
 as $\mathcal{E}(t)=\mathbb{E}\norm{\bar{\bm{G}}_t-\widehat{\bm{G}}_t}^2$.

\begin{customthe}[Proposition~\ref{pro:reconstruction-error}]
Under appropriate Lipschitz and boundedness assumptions, the reconstruction error of HOG-Diff satisfies the following bound: 
\begin{equation}
%\fontsize{8.5pt}{8.5pt}\selectfont
\mathcal{E}(0)
\leq 
\alpha(0)\exp{\int_0^{\tau_1} \gamma(s) } \diff{s},
%\fontsize{10pt}{10pt}\selectfont
\end{equation}
where $\alpha(0)=C^2 \ell^{(1)} (\bm{\theta}) \int_0^{\tau_1} g_{1,s}^4 \diff{s}
+ C \mathcal{E}(\tau_1) \int_0^{\tau_1} h_{1,s}^2 \diff{s}$, 
$\gamma(s) = C^2 g_{1,s}^4 \|\bm{s}_{\bm{\theta}}(\cdot,s)\|_{\mathrm{lip}}^2 
 + C \|h_{1,s}\|_{\mathrm{lip}}^2$,
and $h_{1,s} = \theta_s \left(1 + \frac{2}{e^{2\bar{\theta}_{s:\tau_1}}-1}\right)$.
%
Furthermore, we can derive that the reconstruction error bound of HOG-Diff is sharper than classical graph generation models.
\end{customthe}
 

\begin{proof}

Let $\mathcal{E}(t)= \mathbb{E}\norm{\bar{\bm{G}}_t-\widehat{\bm{G}}_t}^2$, which reflects the expected error between the data reconstructed with the ground truth score $\nabla \log p_t(\cdot)$ and the learned scores $\bm{s}_{\bm{\theta}} (\cdot)$.  
%
In particular, $\bar{\bm{G}}$ is obtained by solving the following oracle reversed time SDE:
\begin{equation}
    \diff \bar{\bm{G}}_t=\left(\mathbf{f}_{k,t}(\bar{\bm{G}}_t)-g_{k,t}^2 \nabla_{\bm{G}}\log p_t(\bar{\bm{G}}_t)\right)\diff\bar{t}
    +g_{k,t}\diff \bar{\bm{W}}_t, t\in[\tau_{k-1},\tau_k],
\end{equation}
whereas $\widehat{\bm{G}}_t$ is governed based on the corresponding estimated reverse time SDE:
\begin{equation}
    \mathrm{d}\widehat{\bm{G}}_t=\left(\mathbf{f}_{k,t}(\widehat{\bm{G}}_t)-g_{k,t}^2 \bm{s}_{\bm{\theta}}(\widehat{\bm{G}}_t,t)\right)\diff\bar{t}
    +g_{k,t}\diff\bar{\bm{W}}_t, t\in [\tau_{k-1},\tau_k].
\end{equation}
Here, $\mathbf{f}_{k,t}$ is the drift function of the Ornstein–Uhlenbeck bridge. 
For simplicity, we we denote the Lipschitz norm by $||\cdot||_{\operatorname{lip}}$ and $\mathbf{f}_{k,s}(\bm{G}_s)=h_{k,s}(\bm{G}_{\tau_k}-\bm{G}_s)$, where $h_{k,s}=\theta_s \left(1 + \frac{2}{e^{2\bar{\theta}_{s:\tau_k}}-1}\right)$. 


To bound the expected reconstruction error $\mathbb{E}\norm{\bar{\bm{G}}_{\tau_{k-1}}-\widehat{\bm{G}}_{\tau_{k-1}}}^2$ at each generation process, we begin by analyze how $\mathbb{E}\norm{\bar{\bm{G}}_t-\widehat{\bm{G}}_t}^2$ evolves as time $t$ is reversed from $\tau_k$ to $\tau_{k-1}$. 
The reconstruction error goes as follows
\begin{equation}
\begin{aligned}
\mathcal{E}(t)
&\leq \mathbb{E}\int_{\tau_k}^t\norm{\left(\mathbf{f}_{k,s}(\bar{\bm{G}}_s)-\mathbf{f}_{k,s}(\widehat{\bm{G}}_{s})\right)+g_{k,s}^2 \left(\bm{s}_{\bm{\theta}}(\widehat{\bm{G}}_{s},s)-\nabla_{\bm{G}}\log p_{s}(\bar{\bm{G}}_{s})\right)}^2\mathrm{d}\bar{s} \\ 
% Line2
&\leq C\mathbb{E}\int_{\tau_k}^t\left\|\mathbf{f}_{k,s}(\bar{\bm{G}}_{s})-\mathbf{f}_{k,s}(\widehat{\bm{G}}_{s})\right\|^2 \mathrm{d}\bar{s} 
+ C\mathbb{E}\int_{\tau_k}^t g_{k,s}^4\left\|\bm{s}_{\bm{\theta}}(\widehat{\bm{G}}_s,s)-\nabla_{\bm{G}}\log p_s(\bar{\bm{G}}_s)\right\|^2\mathrm{d}\bar{s} \\ 
% Line3
&\leq C\int_{\tau_k}^t\|h_{k,s}\|_{\mathrm{lip}}^2\cdot \mathcal{E}(s) \mathrm{d}\bar{s} 
+ C \mathcal{E}(\tau_k) \int_{\tau_k}^t h_{k,s}^2 \mathrm{d}\bar{s}  \\
&+ C^2 \int_{\tau_k}^t g_{k,s}^4 \cdot\mathbb{E}\left\|\bm{s}_{\bm{\theta}}(\widehat{\bm{G}}_{s},s)-\bm{s}_{\bm{\theta}}(\bar{\bm{G}}_{s},s)\right\|^2   
+g_{k,s}^4\cdot\mathbb{E}\left\|\bm{s}_{\bm{\theta}}(\bar{\bm{G}}_s,s)-\nabla_{\bm{G}}\log p_s(\bar{\bm{G}}_s)\right\|^2\mathrm{d}\bar{s}  \\
% Line5
&\leq \underbrace{C^2 \ell^{(k)}(\bm{\theta}) \int_{\tau_k}^t g_{k,s}^4\mathrm{d}\bar{s} 
+ C \mathcal{E}(\tau_k) \int_{\tau_k}^t h_{k,s}^2   \mathrm{d}\bar{s}}_{\alpha(t)}  
+\int_{\tau_k}^t \underbrace{\left( C^2 g_{k,s}^4 \|\bm{s}_{\bm{\theta}}(\cdot,s)\|_{\mathrm{lip}}^2 
+C\|h_{k,s}\|_{\mathrm{lip}}^2 \right)}_{\gamma(s)}  \mathcal{E}(s)  \mathrm{d}\bar{s} \\
% Line6
& = \alpha(t) + \int_{\tau_k}^t \gamma(s) \mathcal{E}(s)  \mathrm{d}\bar{s}.
\end{aligned}
\end{equation}



% changing the integral range (通过改变积分范围,我们可以得到等价的形式为)
Let $v(t)=\mathcal{E}(\tau_k-t)$ and $s'=\tau_k-s$, it can be derived that
\begin{equation}
    v(t) = \mathcal{E}(\tau_k-t) \leq \alpha(\tau_k-t) + \int _0 ^t \gamma(\tau_k - s')v(s')\diff s'.
\end{equation}

Here, $\alpha(\tau_k - t)$ is a non-decreasing function. 
By applying Grönwall’s inequality, we can derive that
\begin{align}
    v(t) & \leq \alpha(\tau_k-t)   \exp{
    \int_0^t \gamma(\tau_k-s') } \mathrm{d}s'  \\
    & = \alpha(\tau_k-t)  \exp{
    \int_{\tau_k-t}^{\tau_k} \gamma(s) } \mathrm{d}s.
\end{align}

Hence,
\begin{equation}
    \mathcal{E}(t) \leq \alpha(t)  \exp{
    \int_t^{\tau_k} \gamma(s) } \mathrm{d}s.
\end{equation}
Therefore, the reconstruction error of HOG-Diff is bounded by 
\begin{equation}
\begin{split}
\mathcal{E}(0)
%\mathbb{E}\norm{\bar{\mathbf{x}}_0-\widehat{\mathbf{x}}_0}^2 
&\leq \alpha(0)  \exp{ \int_0^{\tau_1} \gamma(s) } \mathrm{d}s \\
&= 
 \left(C^2 \ell^{(1)}(\bm{\theta}) \int_0^{\tau_1} g_{1,s}^4\mathrm{d}s
+ C \mathcal{E}(\tau_1) \int_0^{\tau_1} h_{1,s}^2   \mathrm{d}s \right) \exp{ \int_0^{\tau_1} \gamma(s) } \mathrm{d}s.
\end{split}
\end{equation}
A comparable calculation for a classical graph generation model (with diffusion interval $[0, T]$) yields a bound
\begin{equation}
    \mathcal{E}^\prime (0) \leq
 \left(C^2 \ell(\bm{\varphi}) \int_0^T g_s^4\mathrm{d}s
+ C \mathcal{E}^{\prime}(T) \int_0^T h_s^2   \mathrm{d}s \right) \exp \int_0^T \gamma^\prime(s)  \mathrm{d}s,
\end{equation}
where $h_s=\theta_s \left(1 + \frac{2}{e^{2\bar{\theta}_{s:T}}-1}\right)$.


Let $h(s,\tau)=\theta_s  \left(1 + \frac{2}{e^{2\bar{\theta}_{s:\tau}}-1}\right)$, $a(\tau)=\int_0^{\tau} h(s,\tau)^2 \diff{s}$, and $b(\tau)=\int_0^{\tau} \|h(s,\tau)\|_{\operatorname{lip}}^2 \diff{s}$ .
Since $\tau_1\leq T$, it follows that $ \mathcal{E}(\tau_1)\leq \mathcal{E}^\prime(T)$. Additionally, by \cref{pro:training}, $\ell(\cdot)$ converges exponentially in the score-matching process.
Therefore, to prove $\mathcal{E}(0)\leq \mathcal{E}^\prime(0)$, it suffices to show that both $a(\tau)$ and $b(\tau)$ are increasing functions.

Applying the Leibniz Integral Rule, we obtain:
\begin{equation}
a^\prime (\tau) = h(\tau,\tau)^2 + \int_0^{\tau} \frac{\partial}{\partial \tau} h(s, \tau)^2 \diff{s} 
\qquad \mathrm{and} \qquad
b^\prime(\tau) = \|h(\tau,\tau)\|^2_{\operatorname{lip}} + \int_0^{\tau} \frac{\partial}{\partial \tau} \|h(s, \tau)\|^2_{\operatorname{lip}} \diff{s}.
\end{equation}
Since $h(\tau,\tau) \rightarrow 0$, we can derive that $a^\prime (\tau)>0$ and $b^\prime (\tau)>0$. 
This implies $\int_0^{\tau_1} h_{1,s}^2 \diff{s} \leq \int_0^T h_s^2 \diff{s}$ and $\int_0^{\tau_1} \gamma(s) \diff{s} \leq \int_0^T \gamma^{\prime}(s) \diff{s}$.
Combining these inequalities, we can finally conclude $\mathcal{E}(0)\leq \mathcal{E}^\prime(0)$.
Therefore, HOG-Diff provides a sharper reconstruction error bound than the classical graph generation framework.
\end{proof}


\section{Higher-order Networks}
\label{app:ho-intro}
Graphs are elegant and useful abstractions for modelling irregular relationships in empirical systems, but their inherent limitation to pairwise interactions restricts their representation of group dynamics \cite{HigherOrderReview2020,xiao2022people}. 
% 
For example, cyclic structures like benzene rings and functional groups play a holistic role in molecular networks; densely interconnected structures, like simplices, often have a collective influence on social networks; and functional brain networks exhibit higher-order dependencies.
%
To address this, various topological models have been employed to describe data in terms of its higher-order relations, including simplicial complexes \cite{HiGCN2024}, cell complexes \cite{CWN+NeurIPS2021}, and combinatorial complexes \cite{combinatorial-complexes}.
%
As such, the study of higher-order networks has gained increasing attention for their capacity to capture higher-order interactions \cite{TDL-position+ICML2024,HoRW2024}.


Given the broad applicability and theoretical richness of higher-order networks, the following delves deeper into two key frameworks for modelling such interactions: simplicial complexes and cell complexes.

\subsection{Simplicial Complexes}

Simplicial complexes (SCs) are fundamental concepts in algebraic topology that flexibly subsume pairwise graphs \cite{Top_Hodge_Hatcher+2001}. 
Specifically, simplices generalize fundamental geometric structures such as points, lines, triangles, and tetrahedra, enabling the modelling of higher-order interactions in networks. They offer a robust framework for capturing multi-way relationships that extend beyond pairwise connections typically represented in classical networks.

A simplicial complex $\mathcal{X}$ consists of a set of simplices of varying dimensions, including vertices (dimension 0), edges (dimension 1), and triangles (dimension 2).

% @ xxx
A $d$-dimensional simplex is formed by a set of $(d+1)$ interacting nodes and includes all the subsets of $\delta + 1$ nodes (with $\delta<d$), which are called the $\delta$-dimensional faces of the simplex.
A simplicial complex of dimension $d$ is formed by simplices of dimension at most equal to $d$ glued along their faces.


\begin{definition}[Simplicial complexes]
A simplicial complex $\mathcal{X}$ is a finite collection of node subsets closed under the operation of taking nonempty subsets, and such a node subset $\sigma \in \mathcal{X}$ is called a simplex. 
\end{definition}


We can obtain a clique complex, a particular kind of SCs, by extracting all cliques from a given graph and regarding them as simplices. 
%
This implies that an empty triangle (owning $\left[v_1,v_2\right]$, $\left[v_1,v_3\right]$, $\left[v_2,v_3\right]$ but without $\left[v_1,v_2,v_3\right]$) cannot occur in clique complexes.

\subsection{Cell Complexes}


Cell complexes generalize simplicial complexes by incorporating generalized building blocks called cells instead of relying solely on simplices \cite{Top_Hodge_Hatcher+2001}.
% Cells capture many-body interactions that are less restrictive than those of simplicial complexes. 
This broader approach allows for the representation of many-body interactions that do not adhere to the strict requirements of simplicial complexes.
For example, a square can be interpreted as a cell of four-body interactions whose faces are just four links. 
This flexibility is advantageous in scenarios such as social networks, where, for instance, a discussion group might not involve all-to-all pairwise interactions, or in protein interaction networks, where proteins in a complex may not bind pairwise.

\begin{figure}[!t]
\centering
\includegraphics[width=0.96\linewidth]{figs/cell-example.pdf}
\vspace{-3mm}
\caption{\textbf{Visual illustration of cell complexes.} (\textbf{a}) Triangle. (\textbf{b}) Tetrahedron. (\textbf{c}) Sphere. (\textbf{d}) Torus.}
\label{fig:cell-example}
\vspace{-4mm}
\end{figure}


% @ Higher-Order Networks
Formally, a cell complex is termed regular if each attaching map is a homeomorphism onto the closure of the associated cell’s image. 
Regular cell complexes generalize graphs, simplicial complexes, and polyhedral complexes while retaining many desirable combinatorial and intuitive properties of these simpler structures.
In this paper, all cell complexes will be regular and consist of finitely many cells. 

As shown in \cref{fig:cell-example} \textbf{a} and \textbf{b},
triangles and tetrahedrons are two particular types of cell complexes called simplicial complexes (SCs). The only 2-cells they allow are triangle-shaped.
%
The sphere shown in \cref{fig:cell-example} \textbf{c} is a 2-dimensional cell complex. It is constructed using two 0-cells (\ie, nodes), connected by two 1-cells (\ie, the edges forming the equator). The equator serves as the boundary for two 2-dimensional disks (the hemispheres), which are glued together along the equator to form the sphere.
% Torus
The torus in \cref{fig:cell-example} \textbf{d} is a 2-dimensional cell complex formed by attaching a single 1-cell to itself in two directions to form the loops of the torus. The resulting structure is then completed by attaching a 2-dimensional disk, forming the surface of the torus.
Note that this is just one way to represent the torus as a cell complex, and other decompositions might lead to different numbers of cells and faces.



\section{Additional Explanation on Related Works}
\label{app:related}


\subsection{Graph Generative Models}
Graph generation has been extensively studied, which dates back to the early works of the random network models, such as the Erdős–Rényi (ER) model \cite{ER1960} and the Barabási-Albert (BA) model \cite{BA1999}.
Recent graph generative models make great progress in graph distribution learning by exploiting the capacity of deep neural networks. 
%
GraphRNN \cite{GraphRNN2018} and GraphVAE \cite{GraphVAE-DrugDiscovery} adopt sequential strategies to generate nodes and edges.  
MolGAN \cite{GAN1-MolGAN} integrates generative adversarial networks (GANs) with reinforcement learning objectives to synthesize molecules with desired chemical properties. 
\citet{GraphAF-ICLR2020} generates molecular graphs using a flow-based approach, while GraphDF \cite{GraphDF-ICML2021} adopts an autoregressive flow-based model with discrete latent variables.
Additionally, GraphEBM \cite{GraphEBM2021} employs an energy-based model for molecular graph generation.
However, the end-to-end structure of these methods often makes them more challenging to train compared to diffusion-based generative models.

\subsection{Diffusion-based Generative Models}
A leap in graph generative models has been marked by the recent progress in diffusion-based generative models \cite{Score-SDE+ICLR2021}.
%
EDP-GNN \cite{EDPGNN-2020} generates the adjacency matrix by learning the score function of the denoising diffusion process, while GDSS \cite{GDSS+ICML2022} extends this framework by simultaneously generating node features and an adjacency matrix with a joint score function capturing the node-edge dependency.
%
DiGress \cite{DiGress+ICLR2023} addresses the discretization challenge due to Gaussian noise, while CDGS \cite{CDGS+AAAI2023} designs a conditional diffusion model based on discrete graph structures.
%
GSDM \cite{GSDM+TPAMI2023} introduces an efficient graph diffusion model driven by low-rank diffusion SDEs on the spectrum of adjacency matrices.
% 
% GPrinFlowNet 
GPrinFlowNet \cite{GPrinFlowNet+ACM2024} proposes a semantic-preserving framework based on a low-to-high frequency generation curriculum, where the $k$-th intermediate generation state corresponds to the $k$ smallest principal components of the adjacency matrices.
%
Despite these advancements, current methods are ineffective at modelling the topological properties of higher-order systems since learning to denoise the noisy samples does not explicitly lead to preserving the intricate structural dependencies required for generating realistic graphs.


\subsection{Diffusion Bridge} 
Several recent works have improved the generative framework of diffusion models by leveraging the diffusion bridge processes, \ie, processes conditioned to the endpoints.
%
\citet{wu2022diffusion} inject physical information into the process by incorporating informative prior to the drift.
% 
GLAD \cite{GLAD-ICMLworkshop2024} employs the Brownian bridge on a discrete latent space with endpoints conditioned on data samples.
%
GruM \cite{GruM+ICML2024} utilizes the OU bridge to condition the diffusion endpoint as the weighted mean of all possible final graphs.
%
However, existing methods often overlook or inadvertently disrupt the higher-order topological structures in the graph generation process.




\section{Details for Higher-order Guided Generation }
\label{app:detail-HOG-Diff}

\subsection{Denoising Network Parametrization}
\label{app:denoising-model}

The denoising network in HOG-Diff is a critical component responsible for estimating the score functions required to reverse the diffusion process effectively. 
The architecture of the proposed denoising network is depicted in \cref{fig:denoising-model}.
The input $\bm{A}_t$ is computed from $\bm{U}_0$ and $\bm{\Lambda}_t^{(k)}$ using the relation  $\bm{A}_t=\bm{D}_t^{(k)}-\bm{L}^{(k)}_t$, where the Laplacian matrix is given by $\bm{L}^{(k)}_t=\bm{U}_0 \bm{\Lambda}_t^{(k)}\bm{U}_0^\top$ and the diagonal degree matrix is given by $\bm{D}_t^{(k)}=\operatorname{diag}\left(\bm{L}_t^{(k)}\right)$.
%
To enhance the input to the Attention module, we derive enriched node and edge features using the  $l$-step random walk matrix obtained from the binarized $\bm{A}_t$.
Specifically, the arrival probability vector is incorporated as additional node features, while the truncated shortest path distance derived from the same matrix is employed as edge features.
Temporal information is integrated into the outputs of the Attention and GCN modules using Feature-wise Linear Modulation (FiLM) \cite{Film+AAAI2018} layers, following sinusoidal position embeddings \cite{attention+NeurIPS2017}.



\begin{figure}[t]
\centering
\includegraphics[width=0.96\linewidth]{figs/ScoreNet.pdf}
\caption{\textbf{Denoising Network Architecture of HOG-Diff.} 
The denoising network integrates GCN and Attention blocks to capture both local and global features, and further incorporates time information through FiLM layers.
These enriched outputs are subsequently concatenated and processed by separate feed-forward networks to produce predictions for $\nabla_{\bm{X}} \log p(\bm{G}_t|\bm{G}_{\tau_k})$ and $\nabla_{\bm{\Lambda}_t} \log p(\bm{G}_t|\bm{G}_{\tau_k})$, respectively.
% \tolga{Please write a description in the caption here.}
}
\label{fig:denoising-model}
\vspace{-4mm}
\end{figure}

% permutation equivariant
A graph processing module is considered permutation invariant if its output remains unchanged under any permutation of its input, formally expressed as $f(\bm{G}) = x \iff f(\pi(\bm{G})) = x$, where $\pi(\bm{G})$ represents a permutation of the input graph $\bm{G}$. It is permutation equivariant when the output undergoes the same permutation as the input, formally defined as $f(\pi(\bm{G})) = \pi(f(\bm{G}))$. 
It is worth noting that our denoising network model is permutation equivalent as each model component avoids any node ordering-dependent operations.



\subsection{Training and Sampling Proceudre}

As shown in \cref{fig:gFrame}, HOG-Diff implements a coarse-to-fine generation curriculum, with the forward diffusion and reverse denoising processes divided into $K$ easy-to-learn subprocesses. Each subprocess is realized using the generalized OU bridge process.


\begin{figure}[!ht]
    \centering
    \includegraphics[width=0.9\linewidth]{figs/gFrame.pdf}
    \caption{\textbf{Illustration of the coarse-to-fine generation process in HOG-Diff using the generalized OU bridge.}}
    \label{fig:gFrame}
\end{figure}


We provide the pseudo-code of the training process in \cref{alg:train}. In our experiments, we adopt a two-step generation process, \ie, $K=2$.  
We initialize $\mathcal{S}^{(0)} = \bm{G}$; under this specific condition, the cell complex filtering operation returns the input unchanged. The set $\mathcal{S}^{(1)}$ corresponds to the 2-cell complex for molecule generation tasks or the 3-simplicial complex for generic graph generation tasks. 
We set $\mathcal{S}^{(2)} = \varnothing$, and for this particular case, we define the cell complex filtering function as
$\operatorname{CCF}(\bm{G}, \varnothing) = \mathcal{N}(\bm{0}, \bm{I})$.

\begin{figure}[t!]
%\vspace{-0.2in}
\centering
\begin{minipage}{\linewidth}
\centering
\begin{algorithm}[H]
\small
\caption{ Training Algorithm of HOG-Diff }
    \textbf{Input:} denoising network $\bm{s}_{\bm{\theta}}^{(k)}$,
                    authentic graph data $\bm{G}_0=(\bm{X}_0, \bm{A}_0)$, 
                    cell complex list $\{\mathcal{S}^{(0)},\cdots,\mathcal{S}^{(K)}\}$, 
                    \phantom{-} training epochs $M_k$.\\
    \textbf{For the $k$-th step:} \phantom{-}             
\begin{algorithmic}[1]
\FOR{$m=1$ \textbf{to} $M_k$}
    \STATE Sample $\bm{G}_0=(\bm{X}_0,\bm{A}_0) \sim \mathcal{G}$
    \STATE $\bm{G}_{\tau_k} \leftarrow \operatorname{CCF}(\bm{G}_0, \mathcal{S}^{(k)})$, and $\bm{G}_{\tau_{k-1}} \leftarrow \operatorname{CCF}(\bm{G}_0, \mathcal{S}^{(k-1)})$ \COMMENT{Cell complex filtering}
    \STATE $\bm{U}_0 \leftarrow \operatorname{EigenVectors}(\bm{D}_0 - \bm{A}_0)$
    \STATE $\bm{\Lambda}_{\tau_k} \leftarrow \operatorname{EigenDecompostion}(\bm{D}_{\tau_k} - \bm{A}_{\tau_k})$
    \STATE $\bm{\Lambda}_{\tau_{k-1}} \leftarrow \operatorname{EigenDecompostion}(\bm{D}_{\tau_{k-1}} - \bm{A}_{\tau_{k-1}})$
    \STATE Sample $t \sim \operatorname{Unif}([0,\tau_k - \tau_{k-1}])$
    \STATE $\bm{X}_t^{(k)} \sim p(\bm{X}_t \mid \bm{X}_{\tau_{k-1}},\bm{X}_{\tau_{k}})$ \COMMENT{\cref{eq:GOU-p}}
    \STATE $\bm{\Lambda}_t^{(k)} \sim p(\bm{\Lambda}_t \mid \bm{\Lambda}_{\tau_{k-1}},\bm{\Lambda}_{\tau_{k}})$ \COMMENT{\cref{eq:GOU-p}}
    \STATE $\bm{L}_t^{(k)} \leftarrow \bm{U}_0 \bm{\Lambda}_t^{(k)} \bm{U}_0^\top$
    \STATE $\bm{A}_t^{(k)} \leftarrow \bm{D}_t^{(k)} - \bm{L}_t^{(k)}  $
    \STATE $\ell^{(k)}(\bm{\theta}) \leftarrow c_1\|
\bm{s}^{(k)}_{\bm{\theta},\bm{X}} - \nabla_{\bm{X}} \log p_t(\bm{G}_t | \bm{G}_{\tau_k})\|^2 \nonumber + c_2 ||\bm{s}^{(k)}_{\bm{\theta},\bm{\Lambda}} - \nabla_{\bm{\Lambda}} \log p_t(\bm{G}_t |\bm{G}_{\tau_k})||^2$
    \STATE $\bm{\theta} \leftarrow \operatorname{optimizer}(\ell^{(k)}(\bm{\theta}))$ 
\ENDFOR
\STATE \textbf{Return:} $\bm{s}^{(k)}_{\bm{\theta}}$
\end{algorithmic}
\label{alg:train}
\end{algorithm}
\end{minipage}
\vspace{-0.2in}
\end{figure}

The pseudo-code of sampling with HOG-Diff is described in \cref{alg:sample}.
The reverse diffusion processes are divided into $K$ hierarchical time windows, denoted as  $\{[\tau_{k-1},\tau_k]\}_{k=1}^K$, where $0 = \tau_0 < \cdots < \tau_{k-1}< \tau_k < \cdots < \tau_K = T$.
We first initialize the sampling process by drawing samples for $\widehat{\bm{X}}_{\tau_K}$ and $\widehat{\bm{\Lambda}}_{\tau_K}$ from a standard Gaussian distribution, and $\widehat{\bm{U}}_0$ is sampled uniformly from the eigenvector matrices of the Laplacian matrix in the training dataset. 
The reverse-time process starts at $\tau_K$ and iteratively updates $\widehat{\bm{X}}_t$ and $\widehat{\bm{\Lambda}}_t$ by solving the reverse-time SDEs with the denoising network $\bm{s_\theta}^{(k)}$.
Subsequently, we reconstruct the Laplacian matrix $\widehat{\bm{L}}_t$ using the fixed eigenvector matrix $\widehat{\bm{U}}_0$ and the updated eigenvalues $\widehat{\bm{\Lambda}}_t$.
%
Endpoint of one generation step serves as the starting point for the next process.
%
Finally, after iterating through all diffusion segments, the algorithm returns the final feature matrix $\widehat{\bm{X}}_0$ and adjacency matrix $\widehat{\bm{A}}_0$, thereby completing the graph generation process. 



\begin{figure}[t!]
%\vspace{-0.2in}
\centering
\begin{minipage}{\linewidth}
\centering
\begin{algorithm}[H]
\small
\caption{ Sampling Algorithm of HOG-Diff }
    \textbf{Input:} Trained denoising network $\bm{s}_{\theta}^{(k)}$, 
    diffusion time split $\{\tau_0,\cdots,\tau_K\}$,
    number of sampling steps $M_k$
\begin{algorithmic}[1]
\STATE $t \leftarrow \tau_K$
\STATE $\widehat{\bm{X}}_{\tau_K}\sim \mathcal{N}(\bm{0}, \bm{I})$ and $\widehat{\bm{\Lambda}}_{\tau_K}\sim \mathcal{N}(\bm{0}, \bm{I})$
\STATE $\widehat{\bm{U}}_0 \sim \operatorname{Unif}\left(\{\bm{U}_0 \triangleq \operatorname{EigenVectors}(\bm{L}_0)\}\right)$
\STATE $\widehat{\bm{G}}_{\tau_K} \leftarrow (\widehat{\bm{X}}_{\tau_K},\widehat{\bm{\Lambda}}_{\tau_K},\widehat{\bm{D}}_{\tau_K}-\widehat{\bm{U}}_0 \widehat{\bm{\Lambda}}_{\tau_K} \widehat{\bm{U}}_0 ^\top)$
\FOR{$k=K$ \textbf{to} $1$}
\FOR{$m=M_k-1$ \textbf{to} $0$}
\STATE $\bm{S}_{\bm{X}}, \bm{S}_{\bm{\Lambda}} \leftarrow \bm{s}^{(k)}_{\bm{\theta}}(\widehat{\bm{G}}_t, \widehat{\bm{G}}_{\tau_k},t)$
% Predict X
\STATE $\widehat{\bm{X}}_t \leftarrow \widehat{\bm{X}}_t - \left[
\theta_t \left( 1 + \frac{2}{e^{2\bar{\theta}_{t:\tau_k}}-1}  \right)(\widehat{\bm{X}}_{\tau_k} - \widehat{\bm{X}}_t)
-g_{k,t}^2 \bm{S_X} \right]\delta t 
+g_{k,t} \sqrt{\delta t} \bm{w}_{\bm{X}}$, $\bm{w}_{\bm{X}} \sim \mathcal{N}(\bm{0}, \bm{I})$ \COMMENT{Prediction step: $\bm{X}$}
% Predict eig
\STATE $\widehat{\bm{\Lambda}}_t \leftarrow \widehat{\bm{\Lambda}}_t - \left[
\theta_t \left( 1 + \frac{2}{e^{2\bar{\theta}_{t:\tau_k}}-1}  \right)(\widehat{\bm{\Lambda}}_{\tau_k} - \widehat{\bm{\Lambda}}_t)
-g_{k,t}^2 \bm{S_\Lambda} \right] \delta t
+g_{k,t} \sqrt{\delta t}  \bm{w}_{\bm{\Lambda}}$, $\bm{w}_{\bm{\Lambda}} \sim \mathcal{N}(\bm{0}, \bm{I})$ \COMMENT{Prediction step: $\bm{\Lambda}$}
\STATE $\widehat{\bm{L}}_t \leftarrow \widehat{\bm{U}}_0 \widehat{\bm{\Lambda}}_t \widehat{\bm{U}}_0^\top$
\STATE $\widehat{\bm{A}}_t \leftarrow \widehat{\bm{D}}_t - \widehat{\bm{L}}_t$
\STATE $t \leftarrow t - \delta t$
\ENDFOR
\STATE $\widehat{\bm{A}}_{\tau_{k-1}} = \operatorname{quantize}(\widehat{\bm{A}}_t)$\COMMENT{Quantize if necessary}
\STATE $\widehat{\bm{G}}_{\tau_{k-1}} \leftarrow (\widehat{\bm{X}}_t,\widehat{\bm{\Lambda}}_t,\widehat{\bm{A}}_t)$ 
\ENDFOR
\STATE \textbf{Return:} $\widehat{\bm{X}}_0$, $\widehat{\bm{A}}_0$ \COMMENT{$\tau_0 = 0$}
\end{algorithmic}
\label{alg:sample}
\end{algorithm}
\end{minipage}
%\vspace{-0.3in}
\end{figure}




\section{Complexity Analysis}
\label{app:complexity}



When the targeted graph is not in the desired higher-order forms, one should also consider the one-time preprocessing procedure for graph filtering.

Cell filtering can be dramatically accelerated because it avoids explicitly finding all cells and only determines whether nodes and edges belong to a cell. Specifically, the $2$-cell filter requires only checking whether each edge belongs to some cycle.

One method to achieve the $2$-cell filter is using a depth-first search (DFS). Starting from the adjacency matrix, we temporarily remove the edge $(i, j)$ and initiate a DFS from node $i$, keeping track of the path length. If the target node $j$ is visited within a path length of $l$, the edge $(i, j)$ is marked as belonging to a $2$-cell of length at most $l$. In sparse graphs with $n$ nodes and $m$ edges, the time complexity of a single DFS is $\mathcal{O}(m + n)$. With the path length limited to $l$, the DFS may traverse up to $l$ layers of recursion in the worst case. Therefore, the complexity of a single DFS is $\mathcal{O}(\min(m + n, l \cdot k_{max})) $, where $k_{max}$ is the maximum degree of the graph. For all $m$ edges, the total complexity is
$\mathcal{O}\left(m \cdot \min(m + n, l \cdot k_{max})\right)$.




Alternatively, matrix operations can be utilized to accelerate this process. By removing the edge $(i, j)$ from the adjacency matrix $A$ to obtain $\bar{A}$, the presence of a path of length $l$ between $i$ and $j$ can be determined by checking whether $\bar{A}^l_{i,j} > 0$. This indicates that the edge $(i, j)$ belongs to a $2$-cell with a maximum length of $l+1$. Assuming the graph has $n$ nodes and $m$ edges, the complexity of sparse matrix multiplication is $\mathcal{O}(mn)$. Since $l$ matrix multiplications are required, the total complexity is: $\mathcal{O}(l \cdot m^2 \cdot n)$. While this complexity is theoretically higher than the DFS approach, matrix methods can benefit from significant parallel acceleration on modern hardware, such as GPUs and TPUs. In practice, this makes the matrix-based method competitive, especially for large-scale graphs or cases where $l$ is large.



For simplicial complexes, the number of $p$-simplices in a graph with $n$ nodes and $m$ edges is upper-bounded by $\mathcal{O}(n^{p-1})$, and they can be enumerated in $\mathcal{O}( a\left(\mathcal{G}\right)^{p-3} m)$ time \cite{chiba1985arboricity}, where $a\left(\mathcal{G}\right)$ is the arboricity of the graph $\mathcal{G}$, a measure of graph sparsity.
Since arboricity is demonstrated to be at most $\mathcal{O}(m^{1/2})$ and $m \leq n^2$, all $p$-simplices can thus be listed in $\mathcal{O}\left( n^{p-3} m \right)$.
Besides, the complexity of finding $2$-simplex is estimated to be $\mathcal{O}(\left\langle k \right\rangle m)$ with the Bron–Kerbosch algorithm \cite{find_cliques1973}, where $\left \langle k \right \rangle$ denotes the average node degree, typically a small value for empirical networks.


\section{Experimental Setup}
\label{app:exp_set}





\subsection{Computing Resources}
In this work, all experiments are conducted using PyTorch on a single NVIDIA L40S GPU with 46 GB memory and AMD EPYC 9374F 32-Core Processor.


\subsection{Generic Graph Generation}


We follow the experimental and evaluation setting from \citet{GDSS+ICML2022} with the same train/test split to ensure a fair comparison with baselines.
%
We use node degree and spectral features of the graph Laplacian decomposition as hand-crafted input features.

\cref{tab:data_summary} summarizes the key characteristics of the datasets utilized in this study. The table outlines the type of dataset, the total number of graphs, and the range of graph sizes ($|V|$). Additionally, it also provides the number of distinct node types and edge types for each dataset. Notably, the synthetic datasets (Community-small and Ego-small) contain relatively small graphs, whereas the molecular datasets (QM9 and ZINC250k) exhibit more diversity in graph size and complexity. %, as reflected by higher average numbers of nodes and edges.


\begin{table}[h]
\centering
\scalebox{0.95}{
\begin{tabular}{c|cc}
\toprule
                             & \#Test      & \#Temporal relations   \\ \midrule
TempEvalQA-Bi                & 448      &  2           \\ 
 TRACIE                      & 4248     &  2          \\
 MCTACO                      & 9442     & 1-19 \\ \bottomrule
\end{tabular}
}
\caption{Dataset Statistics.
For TempEvalQA-Bi, the numbers represent the total number of questions. For TRACIE, the numbers refer to the number of story-hypothesis pairs. For MCTACO, the numbers reflect question-and-answer candidate pairs.}
\label{tab:data_summary}
\end{table}

\subsection{Molecule Generation}
\label{app:mol}


% @ CDGS
Early efforts in molecule generation introduce sequence-based generative models and represent molecules as SMILES strings \cite{SMILES-ICML2017}. 
%
Nevertheless, this representation frequently encounters challenges related to long dependency modelling and low validity issues, as the SMILES string fails to ensure absolute validity. 
Therefore, in recent studies, graph representations are more commonly employed for molecule structures where atoms are represented as nodes and chemical bonds as connecting edges \cite{GDSS+ICML2022}.
Consequently, this shift has driven the development of graph-based methodologies for molecule generation, which aim to produce valid, meaningful, and diverse molecules.


% @ GDSS / CDGS
In experiments, each molecule is preprocessed into a graph comprising adjacency matrix $\bm{A}\in \{0,1,2,3\}^{n\times n}$ and node feature matrix $\bm{X}\in \{0,1\}^{n\times d}$, where $n$ denotes the maximum number of atoms in a molecule of the dataset, and $d$ is the number of possible atom types. The entries of $\bm{A}$ indicate the bond types: 0 for no bound, 1 for the single bond, 2 for the double bond, and 3 for the triple bond. 
Further, we scale $\bm{A}$ with a constant scale of 3 in order to bound the input of the model in the interval [0, 1], and rescale the final sample of the generation process to recover the bond types.
%
Following the standard procedure \cite{GraphAF-ICLR2020, GraphDF-ICML2021}, all molecules are kekulized by the RDKit library \cite{Rdkit2016} with hydrogen atoms removed. In addition, we make use of the valency correction proposed by \citet{Moflow-SIGKDD2020}. 
After generating samples by simulating the reverse diffusion process,  the adjacency matrix entries are quantized to discrete values ${0, 1, 2, 3}$ by by applying value clipping. Specifically, values in $(-\infty, 0.5)$ are mapped to 0, $[0.5, 1.5)$ to 1, $[1.5, 2.5)$ to 2, and $[2.5, +\infty)$ to 3, ensuring the bond types align with their respective categories.




% 分子图指标介绍 @GPrinFlowNet, has modified
To comprehensively assess the quality of the generated molecules across datasets, we evaluate 10,000 generated samples using several key metrics: validity, validity w/o check, Frechet ChemNet Distance (FCD) \cite{FCD}, Neighborhood Subgraph Pairwise Distance Kernel (NSPDK) MMD \cite{NSPKD-MMD}, uniqueness, and novelty \cite{GDSS+ICML2022}.
% 1. FCD
\textbf{FCD} quantifies the similarity between generated and test molecules by leveraging the activations of ChemNet's penultimate layer, accessing the generation quality within the chemical space.
% 2. NSPDK-MMD
In contrast, \textbf{NSPDK-MMD} evaluates the generation quality from the graph topology perspective by computing the MMD between the generated and test sets while considering both node and edge features.
% 3. validity
\textbf{Validity} is measured as the fraction of valid molecules to all generated molecules after applying post-processing corrections such as valency adjustments or edge resampling, while \textbf{validity w/o correction}, following \citet{GDSS+ICML2022}, computes the fraction of valid molecules before any corrections, providing insight into the intrinsic quality of the generative process. 
Whether molecules are valid is generally determined by compliance with the valence rules in RDkit \cite{Rdkit2016}.
% 4. novelty
\textbf{Novelty} assesses the model’s ability to generalize by calculating the percentage of generated graphs that are not subgraphs of the training set, with two graphs considered identical if isomorphic.
% 5. uniqueness
\textbf{Uniqueness} quantifies the diversity of generated molecules as the ratio of unique samples to valid samples, removing duplicates that are subgraph-isomorphic, ensuring variety in the output.





\section{Visualization Results}
\label{app:vis}

In this section, we additionally provide the visualizations of the generated graphs for both molecule generation tasks and generic graph generation tasks.
Figs.~\ref{fig:qm9}-\ref{fig:enzymes} illustrate non-curated generated samples. HOG-Diff demonstrates the capability to generate high-quality samples that closely resemble the topological properties of empirical data while preserving essential structural details.

\begin{figure}[!ht]
    \centering
    \includegraphics[width=0.96\linewidth]{figs/vis_qm9.pdf}
    \caption{Visualization of random samples taken from the HOG-Diff trained on the QM9 dataset. }
    \label{fig:qm9}
\end{figure}

\begin{figure}
    \centering
    \includegraphics[width=0.96\linewidth]{figs/vis_zinc250k.pdf}
    \caption{Visualization of random samples taken from the HOG-Diff trained on the Zinc250k dataset. }
    \label{fig:zinc250k}
\end{figure}

\begin{figure}
    \centering
    \includegraphics[width=0.96\linewidth]{figs/vis_cs.pdf}
    \caption{Visual comparison between training set graph samples and generated graph samples produced by HOG-Diff on the Community-small dataset.}
\end{figure}

\begin{figure}
    \centering
    \includegraphics[width=0.96\linewidth]{figs/vis_ego.pdf}
    \caption{Visual comparison between training set graph samples and generated graph samples produced by HOG-Diff on the Ego-small dataset.}
\end{figure}

\begin{figure}
    \centering
    \includegraphics[width=0.96\linewidth]{figs/vis_enzymes.pdf}
    \caption{Visual comparison between training set graph samples and generated graph samples produced by HOG-Diff on the Enzymes dataset.}
    \label{fig:enzymes}
\end{figure}

%%%%%%%%%%%%%%%%%%%%%%%%%%%%%%%%%%%%%%%%%%%%%%%%%%%%%%%%%%%%%%%%%%%%%%%%%%%%%%%
%%%%%%%%%%%%%%%%%%%%%%%%%%%%%%%%%%%%%%%%%%%%%%%%%%%%%%%%%%%%%%%%%%%%%%%%%%%%%%%
% APPENDIX
%%%%%%%%%%%%%%%%%%%%%%%%%%%%%%%%%%%%%%%%%%%%%%%%%%%%%%%%%%%%%%%%%%%%%%%%%%%%%%%
%%%%%%%%%%%%%%%%%%%%%%%%%%%%%%%%%%%%%%%%%%%%%%%%%%%%%%%%%%%%%%%%%%%%%%%%%%%%%%%
% \newpage
% \appendix
% \onecolumn
% \section{You \emph{can} have an appendix here.}

% You can have as much text here as you want. The main body must be at most $8$ pages long.
% For the final version, one more page can be added.
% If you want, you can use an appendix like this one.  

% The $\mathtt{\backslash onecolumn}$ command above can be kept in place if you prefer a one-column appendix, or can be removed if you prefer a two-column appendix.  Apart from this possible change, the style (font size, spacing, margins, page numbering, etc.) should be kept the same as the main body.
%%%%%%%%%%%%%%%%%%%%%%%%%%%%%%%%%%%%%%%%%%%%%%%%%%%%%%%%%%%%%%%%%%%%%%%%%%%%%%%
%%%%%%%%%%%%%%%%%%%%%%%%%%%%%%%%%%%%%%%%%%%%%%%%%%%%%%%%%%%%%%%%%%%%%%%%%%%%%%%


\end{document}


% This document was modified from the file originally made available by
% Pat Langley and Andrea Danyluk for ICML-2K. This version was created
% by Iain Murray in 2018, and modified by Alexandre Bouchard in
% 2019 and 2021 and by Csaba Szepesvari, Gang Niu and Sivan Sabato in 2022.
% Modified again in 2023 and 2024 by Sivan Sabato and Jonathan Scarlett.
% Previous contributors include Dan Roy, Lise Getoor and Tobias
% Scheffer, which was slightly modified from the 2010 version by
% Thorsten Joachims & Johannes Fuernkranz, slightly modified from the
% 2009 version by Kiri Wagstaff and Sam Roweis's 2008 version, which is
% slightly modified from Prasad Tadepalli's 2007 version which is a
% lightly changed version of the previous year's version by Andrew
% Moore, which was in turn edited from those of Kristian Kersting and
% Codrina Lauth. Alex Smola contributed to the algorithmic style files.
