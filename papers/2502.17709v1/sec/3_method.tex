\section{Methods}
\label{sec:method}

\section{Method}\label{sec:method}
\begin{figure}
    \centering
    \includegraphics[width=0.85\textwidth]{imgs/heatmap_acc.pdf}
    \caption{\textbf{Visualization of the proposed periodic Bayesian flow with mean parameter $\mu$ and accumulated accuracy parameter $c$ which corresponds to the entropy/uncertainty}. For $x = 0.3, \beta(1) = 1000$ and $\alpha_i$ defined in \cref{appd:bfn_cir}, this figure plots three colored stochastic parameter trajectories for receiver mean parameter $m$ and accumulated accuracy parameter $c$, superimposed on a log-scale heatmap of the Bayesian flow distribution $p_F(m|x,\senderacc)$ and $p_F(c|x,\senderacc)$. Note the \emph{non-monotonicity} and \emph{non-additive} property of $c$ which could inform the network the entropy of the mean parameter $m$ as a condition and the \emph{periodicity} of $m$. %\jj{Shrink the figures to save space}\hanlin{Do we need to make this figure one-column?}
    }
    \label{fig:vmbf_vis}
    \vskip -0.1in
\end{figure}
% \begin{wrapfigure}{r}{0.5\textwidth}
%     \centering
%     \includegraphics[width=0.49\textwidth]{imgs/heatmap_acc.pdf}
%     \caption{\textbf{Visualization of hyper-torus Bayesian flow based on von Mises Distribution}. For $x = 0.3, \beta(1) = 1000$ and $\alpha_i$ defined in \cref{appd:bfn_cir}, this figure plots three colored stochastic parameter trajectories for receiver mean parameter $m$ and accumulated accuracy parameter $c$, superimposed on a log-scale heatmap of the Bayesian flow distribution $p_F(m|x,\senderacc)$ and $p_F(c|x,\senderacc)$. Note the \emph{non-monotonicity} and \emph{non-additive} property of $c$. \jj{Shrink the figures to save space}}
%     \label{fig:vmbf_vis}
%     \vspace{-30pt}
% \end{wrapfigure}


In this section, we explain the detailed design of CrysBFN tackling theoretical and practical challenges. First, we describe how to derive our new formulation of Bayesian Flow Networks over hyper-torus $\mathbb{T}^{D}$ from scratch. Next, we illustrate the two key differences between \modelname and the original form of BFN: $1)$ a meticulously designed novel base distribution with different Bayesian update rules; and $2)$ different properties over the accuracy scheduling resulted from the periodicity and the new Bayesian update rules. Then, we present in detail the overall framework of \modelname over each manifold of the crystal space (\textit{i.e.} fractional coordinates, lattice vectors, atom types) respecting \textit{periodic E(3) invariance}. 

% In this section, we first demonstrate how to build Bayesian flow on hyper-torus $\mathbb{T}^{D}$ by overcoming theoretical and practical problems to provide a low-noise parameter-space approach to fractional atom coordinate generation. Next, we present how \modelname models each manifold of crystal space respecting \textit{periodic E(3) invariance}. 

\subsection{Periodic Bayesian Flow on Hyper-torus \texorpdfstring{$\mathbb{T}^{D}$}{}} 
For generative modeling of fractional coordinates in crystal, we first construct a periodic Bayesian flow on \texorpdfstring{$\mathbb{T}^{D}$}{} by designing every component of the totally new Bayesian update process which we demonstrate to be distinct from the original Bayesian flow (please see \cref{fig:non_add}). 
 %:) 
 
 The fractional atom coordinate system \citep{jiao2023crystal} inherently distributes over a hyper-torus support $\mathbb{T}^{3\times N}$. Hence, the normal distribution support on $\R$ used in the original \citep{bfn} is not suitable for this scenario. 
% The key problem of generative modeling for crystal is the periodicity of Cartesian atom coordinates $\vX$ requiring:
% \begin{equation}\label{eq:periodcity}
% p(\vA,\vL,\vX)=p(\vA,\vL,\vX+\vec{LK}),\text{where}~\vec{K}=\vec{k}\vec{1}_{1\times N},\forall\vec{k}\in\mathbb{Z}^{3\times1}
% \end{equation}
% However, there does not exist such a distribution supporting on $\R$ to model such property because the integration of such distribution over $\R$ will not be finite and equal to 1. Therefore, the normal distribution used in \citet{bfn} can not meet this condition.

To tackle this problem, the circular distribution~\citep{mardia2009directional} over the finite interval $[-\pi,\pi)$ is a natural choice as the base distribution for deriving the BFN on $\mathbb{T}^D$. 
% one natural choice is to 
% we would like to consider the circular distribution over the finite interval as the base 
% we find that circular distributions \citep{mardia2009directional} defined on a finite interval with lengths of $2\pi$ can be used as the instantiation of input distribution for the BFN on $\mathbb{T}^D$.
Specifically, circular distributions enjoy desirable periodic properties: $1)$ the integration over any interval length of $2\pi$ equals 1; $2)$ the probability distribution function is periodic with period $2\pi$.  Sharing the same intrinsic with fractional coordinates, such periodic property of circular distribution makes it suitable for the instantiation of BFN's input distribution, in parameterizing the belief towards ground truth $\x$ on $\mathbb{T}^D$. 
% \yuxuan{this is very complicated from my perspective.} \hanlin{But this property is exactly beautiful and perfectly fit into the BFN.}

\textbf{von Mises Distribution and its Bayesian Update} We choose von Mises distribution \citep{mardia2009directional} from various circular distributions as the form of input distribution, based on the appealing conjugacy property required in the derivation of the BFN framework.
% to leverage the Bayesian conjugacy property of von Mises distribution which is required by the BFN framework. 
That is, the posterior of a von Mises distribution parameterized likelihood is still in the family of von Mises distributions. The probability density function of von Mises distribution with mean direction parameter $m$ and concentration parameter $c$ (describing the entropy/uncertainty of $m$) is defined as: 
\begin{equation}
f(x|m,c)=vM(x|m,c)=\frac{\exp(c\cos(x-m))}{2\pi I_0(c)}
\end{equation}
where $I_0(c)$ is zeroth order modified Bessel function of the first kind as the normalizing constant. Given the last univariate belief parameterized by von Mises distribution with parameter $\theta_{i-1}=\{m_{i-1},\ c_{i-1}\}$ and the sample $y$ from sender distribution with unknown data sample $x$ and known accuracy $\alpha$ describing the entropy/uncertainty of $y$,  Bayesian update for the receiver is deducted as:
\begin{equation}
 h(\{m_{i-1},c_{i-1}\},y,\alpha)=\{m_i,c_i \}, \text{where}
\end{equation}
\begin{equation}\label{eq:h_m}
m_i=\text{atan2}(\alpha\sin y+c_{i-1}\sin m_{i-1}, {\alpha\cos y+c_{i-1}\cos m_{i-1}})
\end{equation}
\begin{equation}\label{eq:h_c}
c_i =\sqrt{\alpha^2+c_{i-1}^2+2\alpha c_{i-1}\cos(y-m_{i-1})}
\end{equation}
The proof of the above equations can be found in \cref{apdx:bayesian_update_function}. The atan2 function refers to  2-argument arctangent. Independently conducting  Bayesian update for each dimension, we can obtain the Bayesian update distribution by marginalizing $\y$:
\begin{equation}
p_U(\vtheta'|\vtheta,\bold{x};\alpha)=\mathbb{E}_{p_S(\bold{y}|\bold{x};\alpha)}\delta(\vtheta'-h(\vtheta,\bold{y},\alpha))=\mathbb{E}_{vM(\bold{y}|\bold{x},\alpha)}\delta(\vtheta'-h(\vtheta,\bold{y},\alpha))
\end{equation} 
\begin{figure}
    \centering
    \vskip -0.15in
    \includegraphics[width=0.95\linewidth]{imgs/non_add.pdf}
    \caption{An intuitive illustration of non-additive accuracy Bayesian update on the torus. The lengths of arrows represent the uncertainty/entropy of the belief (\emph{e.g.}~$1/\sigma^2$ for Gaussian and $c$ for von Mises). The directions of the arrows represent the believed location (\emph{e.g.}~ $\mu$ for Gaussian and $m$ for von Mises).}
    \label{fig:non_add}
    \vskip -0.15in
\end{figure}
\textbf{Non-additive Accuracy} 
The additive accuracy is a nice property held with the Gaussian-formed sender distribution of the original BFN expressed as:
\begin{align}
\label{eq:standard_id}
    \update(\parsn{}'' \mid \parsn{}, \x; \alpha_a+\alpha_b) = \E_{\update(\parsn{}' \mid \parsn{}, \x; \alpha_a)} \update(\parsn{}'' \mid \parsn{}', \x; \alpha_b)
\end{align}
Such property is mainly derived based on the standard identity of Gaussian variable:
\begin{equation}
X \sim \mathcal{N}\left(\mu_X, \sigma_X^2\right), Y \sim \mathcal{N}\left(\mu_Y, \sigma_Y^2\right) \Longrightarrow X+Y \sim \mathcal{N}\left(\mu_X+\mu_Y, \sigma_X^2+\sigma_Y^2\right)
\end{equation}
The additive accuracy property makes it feasible to derive the Bayesian flow distribution $
p_F(\boldsymbol{\theta} \mid \mathbf{x} ; i)=p_U\left(\boldsymbol{\theta} \mid \boldsymbol{\theta}_0, \mathbf{x}, \sum_{k=1}^{i} \alpha_i \right)
$ for the simulation-free training of \cref{eq:loss_n}.
It should be noted that the standard identity in \cref{eq:standard_id} does not hold in the von Mises distribution. Hence there exists an important difference between the original Bayesian flow defined on Euclidean space and the Bayesian flow of circular data on $\mathbb{T}^D$ based on von Mises distribution. With prior $\btheta = \{\bold{0},\bold{0}\}$, we could formally represent the non-additive accuracy issue as:
% The additive accuracy property implies the fact that the "confidence" for the data sample after observing a series of the noisy samples with accuracy ${\alpha_1, \cdots, \alpha_i}$ could be  as the accuracy sum  which could be  
% Here we 
% Here we emphasize the specific property of BFN based on von Mises distribution.
% Note that 
% \begin{equation}
% \update(\parsn'' \mid \parsn, \x; \alpha_a+\alpha_b) \ne \E_{\update(\parsn' \mid \parsn, \x; \alpha_a)} \update(\parsn'' \mid \parsn', \x; \alpha_b)
% \end{equation}
% \oyyw{please check whether the below equation is better}
% \yuxuan{I fill somehow confusing on what is the update distribution with $\alpha$. }
% \begin{equation}
% \update(\parsn{}'' \mid \parsn{}, \x; \alpha_a+\alpha_b) \ne \E_{\update(\parsn{}' \mid \parsn{}, \x; \alpha_a)} \update(\parsn{}'' \mid \parsn{}', \x; \alpha_b)
% \end{equation}
% We give an intuitive visualization of such difference in \cref{fig:non_add}. The untenability of this property can materialize by considering the following case: with prior $\btheta = \{\bold{0},\bold{0}\}$, check the two-step Bayesian update distribution with $\alpha_a,\alpha_b$ and one-step Bayesian update with $\alpha=\alpha_a+\alpha_b$:
\begin{align}
\label{eq:nonadd}
     &\update(c'' \mid \parsn, \x; \alpha_a+\alpha_b)  = \delta(c-\alpha_a-\alpha_b)
     \ne  \mathbb{E}_{p_U(\parsn' \mid \parsn, \x; \alpha_a)}\update(c'' \mid \parsn', \x; \alpha_b) \nonumber \\&= \mathbb{E}_{vM(\bold{y}_b|\bold{x},\alpha_a)}\mathbb{E}_{vM(\bold{y}_a|\bold{x},\alpha_b)}\delta(c-||[\alpha_a \cos\y_a+\alpha_b\cos \y_b,\alpha_a \sin\y_a+\alpha_b\sin \y_b]^T||_2)
\end{align}
A more intuitive visualization could be found in \cref{fig:non_add}. This fundamental difference between periodic Bayesian flow and that of \citet{bfn} presents both theoretical and practical challenges, which we will explain and address in the following contents.

% This makes constructing Bayesian flow based on von Mises distribution intrinsically different from previous Bayesian flows (\citet{bfn}).

% Thus, we must reformulate the framework of Bayesian flow networks  accordingly. % and do necessary reformulations of BFN. 

% \yuxuan{overall I feel this part is complicated by using the language of update distribution. I would like to suggest simply use bayesian update, to provide intuitive explantion.}\hanlin{See the illustration in \cref{fig:non_add}}

% That introduces a cascade of problems, and we investigate the following issues: $(1)$ Accuracies between sender and receiver are not synchronized and need to be differentiated. $(2)$ There is no tractable Bayesian flow distribution for a one-step sample conditioned on a given time step $i$, and naively simulating the Bayesian flow results in computational overhead. $(3)$ It is difficult to control the entropy of the Bayesian flow. $(4)$ Accuracy is no longer a function of $t$ and becomes a distribution conditioned on $t$, which can be different across dimensions.
%\jj{Edited till here}

\textbf{Entropy Conditioning} As a common practice in generative models~\citep{ddpm,flowmatching,bfn}, timestep $t$ is widely used to distinguish among generation states by feeding the timestep information into the networks. However, this paper shows that for periodic Bayesian flow, the accumulated accuracy $\vc_i$ is more effective than time-based conditioning by informing the network about the entropy and certainty of the states $\parsnt{i}$. This stems from the intrinsic non-additive accuracy which makes the receiver's accumulated accuracy $c$ not bijective function of $t$, but a distribution conditioned on accumulated accuracies $\vc_i$ instead. Therefore, the entropy parameter $\vc$ is taken logarithm and fed into the network to describe the entropy of the input corrupted structure. We verify this consideration in \cref{sec:exp_ablation}. 
% \yuxuan{implement variant. traditionally, the timestep is widely used to distinguish the different states by putting the timestep embedding into the networks. citation of FM, diffusion, BFN. However, we find that conditioned on time in periodic flow could not provide extra benefits. To further boost the performance, we introduce a simple yet effective modification term entropy conditional. This is based on that the accumulated accuracy which represents the current uncertainty or entropy could be a better indicator to distinguish different states. + Describe how you do this. }



\textbf{Reformulations of BFN}. Recall the original update function with Gaussian sender distribution, after receiving noisy samples $\y_1,\y_2,\dots,\y_i$ with accuracies $\senderacc$, the accumulated accuracies of the receiver side could be analytically obtained by the additive property and it is consistent with the sender side.
% Since observing sample $\y$ with $\alpha_i$ can not result in exact accuracy increment $\alpha_i$ for receiver, the accuracies between sender and receiver are not synchronized which need to be differentiated. 
However, as previously mentioned, this does not apply to periodic Bayesian flow, and some of the notations in original BFN~\citep{bfn} need to be adjusted accordingly. We maintain the notations of sender side's one-step accuracy $\alpha$ and added accuracy $\beta$, and alter the notation of receiver's accuracy parameter as $c$, which is needed to be simulated by cascade of Bayesian updates. We emphasize that the receiver's accumulated accuracy $c$ is no longer a function of $t$ (differently from the Gaussian case), and it becomes a distribution conditioned on received accuracies $\senderacc$ from the sender. Therefore, we represent the Bayesian flow distribution of von Mises distribution as $p_F(\btheta|\x;\alpha_1,\alpha_2,\dots,\alpha_i)$. And the original simulation-free training with Bayesian flow distribution is no longer applicable in this scenario.
% Different from previous BFNs where the accumulated accuracy $\rho$ is not explicitly modeled, the accumulated accuracy parameter $c$ (visualized in \cref{fig:vmbf_vis}) needs to be explicitly modeled by feeding it to the network to avoid information loss.
% the randomaccuracy parameter $c$ (visualized in \cref{fig:vmbf_vis}) implies that there exists information in $c$ from the sender just like $m$, meaning that $c$ also should be fed into the network to avoid information loss. 
% We ablate this consideration in  \cref{sec:exp_ablation}. 

\textbf{Fast Sampling from Equivalent Bayesian Flow Distribution} Based on the above reformulations, the Bayesian flow distribution of von Mises distribution is reframed as: 
\begin{equation}\label{eq:flow_frac}
p_F(\btheta_i|\x;\alpha_1,\alpha_2,\dots,\alpha_i)=\E_{\update(\parsnt{1} \mid \parsnt{0}, \x ; \alphat{1})}\dots\E_{\update(\parsn_{i-1} \mid \parsnt{i-2}, \x; \alphat{i-1})} \update(\parsnt{i} | \parsnt{i-1},\x;\alphat{i} )
\end{equation}
Naively sampling from \cref{eq:flow_frac} requires slow auto-regressive iterated simulation, making training unaffordable. Noticing the mathematical properties of \cref{eq:h_m,eq:h_c}, we  transform \cref{eq:flow_frac} to the equivalent form:
\begin{equation}\label{eq:cirflow_equiv}
p_F(\vec{m}_i|\x;\alpha_1,\alpha_2,\dots,\alpha_i)=\E_{vM(\y_1|\x,\alpha_1)\dots vM(\y_i|\x,\alpha_i)} \delta(\vec{m}_i-\text{atan2}(\sum_{j=1}^i \alpha_j \cos \y_j,\sum_{j=1}^i \alpha_j \sin \y_j))
\end{equation}
\begin{equation}\label{eq:cirflow_equiv2}
p_F(\vec{c}_i|\x;\alpha_1,\alpha_2,\dots,\alpha_i)=\E_{vM(\y_1|\x,\alpha_1)\dots vM(\y_i|\x,\alpha_i)}  \delta(\vec{c}_i-||[\sum_{j=1}^i \alpha_j \cos \y_j,\sum_{j=1}^i \alpha_j \sin \y_j]^T||_2)
\end{equation}
which bypasses the computation of intermediate variables and allows pure tensor operations, with negligible computational overhead.
\begin{restatable}{proposition}{cirflowequiv}
The probability density function of Bayesian flow distribution defined by \cref{eq:cirflow_equiv,eq:cirflow_equiv2} is equivalent to the original definition in \cref{eq:flow_frac}. 
\end{restatable}
\textbf{Numerical Determination of Linear Entropy Sender Accuracy Schedule} ~Original BFN designs the accuracy schedule $\beta(t)$ to make the entropy of input distribution linearly decrease. As for crystal generation task, to ensure information coherence between modalities, we choose a sender accuracy schedule $\senderacc$ that makes the receiver's belief entropy $H(t_i)=H(p_I(\cdot|\vtheta_i))=H(p_I(\cdot|\vc_i))$ linearly decrease \emph{w.r.t.} time $t_i$, given the initial and final accuracy parameter $c(0)$ and $c(1)$. Due to the intractability of \cref{eq:vm_entropy}, we first use numerical binary search in $[0,c(1)]$ to determine the receiver's $c(t_i)$ for $i=1,\dots, n$ by solving the equation $H(c(t_i))=(1-t_i)H(c(0))+tH(c(1))$. Next, with $c(t_i)$, we conduct numerical binary search for each $\alpha_i$ in $[0,c(1)]$ by solving the equations $\E_{y\sim vM(x,\alpha_i)}[\sqrt{\alpha_i^2+c_{i-1}^2+2\alpha_i c_{i-1}\cos(y-m_{i-1})}]=c(t_i)$ from $i=1$ to $i=n$ for arbitrarily selected $x\in[-\pi,\pi)$.

After tackling all those issues, we have now arrived at a new BFN architecture for effectively modeling crystals. Such BFN can also be adapted to other type of data located in hyper-torus $\mathbb{T}^{D}$.

\subsection{Equivariant Bayesian Flow for Crystal}
With the above Bayesian flow designed for generative modeling of fractional coordinate $\vF$, we are able to build equivariant Bayesian flow for each modality of crystal. In this section, we first give an overview of the general training and sampling algorithm of \modelname (visualized in \cref{fig:framework}). Then, we describe the details of the Bayesian flow of every modality. The training and sampling algorithm can be found in \cref{alg:train} and \cref{alg:sampling}.

\textbf{Overview} Operating in the parameter space $\bthetaM=\{\bthetaA,\bthetaL,\bthetaF\}$, \modelname generates high-fidelity crystals through a joint BFN sampling process on the parameter of  atom type $\bthetaA$, lattice parameter $\vec{\theta}^L=\{\bmuL,\brhoL\}$, and the parameter of fractional coordinate matrix $\bthetaF=\{\bmF,\bcF\}$. We index the $n$-steps of the generation process in a discrete manner $i$, and denote the corresponding continuous notation $t_i=i/n$ from prior parameter $\thetaM_0$ to a considerably low variance parameter $\thetaM_n$ (\emph{i.e.} large $\vrho^L,\bmF$, and centered $\bthetaA$).

At training time, \modelname samples time $i\sim U\{1,n\}$ and $\bthetaM_{i-1}$ from the Bayesian flow distribution of each modality, serving as the input to the network. The network $\net$ outputs $\net(\parsnt{i-1}^\mathcal{M},t_{i-1})=\net(\parsnt{i-1}^A,\parsnt{i-1}^F,\parsnt{i-1}^L,t_{i-1})$ and conducts gradient descents on loss function \cref{eq:loss_n} for each modality. After proper training, the sender distribution $p_S$ can be approximated by the receiver distribution $p_R$. 

At inference time, from predefined $\thetaM_0$, we conduct transitions from $\thetaM_{i-1}$ to $\thetaM_{i}$ by: $(1)$ sampling $\y_i\sim p_R(\bold{y}|\thetaM_{i-1};t_i,\alpha_i)$ according to network prediction $\predM{i-1}$; and $(2)$ performing Bayesian update $h(\thetaM_{i-1},\y^\calM_{i-1},\alpha_i)$ for each dimension. 

% Alternatively, we complete this transition using the flow-back technique by sampling 
% $\thetaM_{i}$ from Bayesian flow distribution $\flow(\btheta^M_{i}|\predM{i-1};t_{i-1})$. 

% The training objective of $\net$ is to minimize the KL divergence between sender distribution and receiver distribution for every modality as defined in \cref{eq:loss_n} which is equivalent to optimizing the negative variational lower bound $\calL^{VLB}$ as discussed in \cref{sec:preliminaries}. 

%In the following part, we will present the Bayesian flow of each modality in detail.

\textbf{Bayesian Flow of Fractional Coordinate $\vF$}~The distribution of the prior parameter $\bthetaF_0$ is defined as:
\begin{equation}\label{eq:prior_frac}
    p(\bthetaF_0) \defeq \{vM(\vm_0^F|\vec{0}_{3\times N},\vec{0}_{3\times N}),\delta(\vc_0^F-\vec{0}_{3\times N})\} = \{U(\vec{0},\vec{1}),\delta(\vc_0^F-\vec{0}_{3\times N})\}
\end{equation}
Note that this prior distribution of $\vm_0^F$ is uniform over $[\vec{0},\vec{1})$, ensuring the periodic translation invariance property in \cref{De:pi}. The training objective is minimizing the KL divergence between sender and receiver distribution (deduction can be found in \cref{appd:cir_loss}): 
%\oyyw{replace $\vF$ with $\x$?} \hanlin{notations follow Preliminary?}
\begin{align}\label{loss_frac}
\calL_F = n \E_{i \sim \ui{n}, \flow(\parsn{}^F \mid \vF ; \senderacc)} \alpha_i\frac{I_1(\alpha_i)}{I_0(\alpha_i)}(1-\cos(\vF-\predF{i-1}))
\end{align}
where $I_0(x)$ and $I_1(x)$ are the zeroth and the first order of modified Bessel functions. The transition from $\bthetaF_{i-1}$ to $\bthetaF_{i}$ is the Bayesian update distribution based on network prediction:
\begin{equation}\label{eq:transi_frac}
    p(\btheta^F_{i}|\parsnt{i-1}^\calM)=\mathbb{E}_{vM(\bold{y}|\predF{i-1},\alpha_i)}\delta(\btheta^F_{i}-h(\btheta^F_{i-1},\bold{y},\alpha_i))
\end{equation}
\begin{restatable}{proposition}{fracinv}
With $\net_{F}$ as a periodic translation equivariant function namely $\net_F(\parsnt{}^A,w(\parsnt{}^F+\vt),\parsnt{}^L,t)=w(\net_F(\parsnt{}^A,\parsnt{}^F,\parsnt{}^L,t)+\vt), \forall\vt\in\R^3$, the marginal distribution of $p(\vF_n)$ defined by \cref{eq:prior_frac,eq:transi_frac} is periodic translation invariant. 
\end{restatable}
\textbf{Bayesian Flow of Lattice Parameter \texorpdfstring{$\boldsymbol{L}$}{}}   
Noting the lattice parameter $\bm{L}$ located in Euclidean space, we set prior as the parameter of a isotropic multivariate normal distribution $\btheta^L_0\defeq\{\vmu_0^L,\vrho_0^L\}=\{\bm{0}_{3\times3},\bm{1}_{3\times3}\}$
% \begin{equation}\label{eq:lattice_prior}
% \btheta^L_0\defeq\{\vmu_0^L,\vrho_0^L\}=\{\bm{0}_{3\times3},\bm{1}_{3\times3}\}
% \end{equation}
such that the prior distribution of the Markov process on $\vmu^L$ is the Dirac distribution $\delta(\vec{\mu_0}-\vec{0})$ and $\delta(\vec{\rho_0}-\vec{1})$, 
% \begin{equation}
%     p_I^L(\boldsymbol{L}|\btheta_0^L)=\mathcal{N}(\bm{L}|\bm{0},\bm{I})
% \end{equation}
which ensures O(3)-invariance of prior distribution of $\vL$. By Eq. 77 from \citet{bfn}, the Bayesian flow distribution of the lattice parameter $\bm{L}$ is: 
\begin{align}% =p_U(\bmuL|\btheta_0^L,\bm{L},\beta(t))
p_F^L(\bmuL|\bm{L};t) &=\mathcal{N}(\bmuL|\gamma(t)\bm{L},\gamma(t)(1-\gamma(t))\bm{I}) 
\end{align}
where $\gamma(t) = 1 - \sigma_1^{2t}$ and $\sigma_1$ is the predefined hyper-parameter controlling the variance of input distribution at $t=1$ under linear entropy accuracy schedule. The variance parameter $\vrho$ does not need to be modeled and fed to the network, since it is deterministic given the accuracy schedule. After sampling $\bmuL_i$ from $p_F^L$, the training objective is defined as minimizing KL divergence between sender and receiver distribution (based on Eq. 96 in \citet{bfn}):
\begin{align}
\mathcal{L}_{L} = \frac{n}{2}\left(1-\sigma_1^{2/n}\right)\E_{i \sim \ui{n}}\E_{\flow(\bmuL_{i-1} |\vL ; t_{i-1})}  \frac{\left\|\vL -\predL{i-1}\right\|^2}{\sigma_1^{2i/n}},\label{eq:lattice_loss}
\end{align}
where the prediction term $\predL{i-1}$ is the lattice parameter part of network output. After training, the generation process is defined as the Bayesian update distribution given network prediction:
\begin{equation}\label{eq:lattice_sampling}
    p(\bmuL_{i}|\parsnt{i-1}^\calM)=\update^L(\bmuL_{i}|\predL{i-1},\bmuL_{i-1};t_{i-1})
\end{equation}
    

% The final prediction of the lattice parameter is given by $\bmuL_n = \predL{n-1}$.
% \begin{equation}\label{eq:final_lattice}
%     \bmuL_n = \predL{n-1}
% \end{equation}

\begin{restatable}{proposition}{latticeinv}\label{prop:latticeinv}
With $\net_{L}$ as  O(3)-equivariant function namely $\net_L(\parsnt{}^A,\parsnt{}^F,\vQ\parsnt{}^L,t)=\vQ\net_L(\parsnt{}^A,\parsnt{}^F,\parsnt{}^L,t),\forall\vQ^T\vQ=\vI$, the marginal distribution of $p(\bmuL_n)$ defined by \cref{eq:lattice_sampling} is O(3)-invariant. 
\end{restatable}


\textbf{Bayesian Flow of Atom Types \texorpdfstring{$\boldsymbol{A}$}{}} 
Given that atom types are discrete random variables located in a simplex $\calS^K$, the prior parameter of $\boldsymbol{A}$ is the discrete uniform distribution over the vocabulary $\parsnt{0}^A \defeq \frac{1}{K}\vec{1}_{1\times N}$. 
% \begin{align}\label{eq:disc_input_prior}
% \parsnt{0}^A \defeq \frac{1}{K}\vec{1}_{1\times N}
% \end{align}
% \begin{align}
%     (\oh{j}{K})_k \defeq \delta_{j k}, \text{where }\oh{j}{K}\in \R^{K},\oh{\vA}{KD} \defeq \left(\oh{a_1}{K},\dots,\oh{a_N}{K}\right) \in \R^{K\times N}
% \end{align}
With the notation of the projection from the class index $j$ to the length $K$ one-hot vector $ (\oh{j}{K})_k \defeq \delta_{j k}, \text{where }\oh{j}{K}\in \R^{K},\oh{\vA}{KD} \defeq \left(\oh{a_1}{K},\dots,\oh{a_N}{K}\right) \in \R^{K\times N}$, the Bayesian flow distribution of atom types $\vA$ is derived in \citet{bfn}:
\begin{align}
\flow^{A}(\parsn^A \mid \vA; t) &= \E_{\N{\y \mid \beta^A(t)\left(K \oh{\vA}{K\times N} - \vec{1}_{K\times N}\right)}{\beta^A(t) K \vec{I}_{K\times N \times N}}} \delta\left(\parsn^A - \frac{e^{\y}\parsnt{0}^A}{\sum_{k=1}^K e^{\y_k}(\parsnt{0})_{k}^A}\right).
\end{align}
where $\beta^A(t)$ is the predefined accuracy schedule for atom types. Sampling $\btheta_i^A$ from $p_F^A$ as the training signal, the training objective is the $n$-step discrete-time loss for discrete variable \citep{bfn}: 
% \oyyw{can we simplify the next equation? Such as remove $K \times N, K \times N \times N$}
% \begin{align}
% &\calL_A = n\E_{i \sim U\{1,n\},\flow^A(\parsn^A \mid \vA ; t_{i-1}),\N{\y \mid \alphat{i}\left(K \oh{\vA}{KD} - \vec{1}_{K\times N}\right)}{\alphat{i} K \vec{I}_{K\times N \times N}}} \ln \N{\y \mid \alphat{i}\left(K \oh{\vA}{K\times N} - \vec{1}_{K\times N}\right)}{\alphat{i} K \vec{I}_{K\times N \times N}}\nonumber\\
% &\qquad\qquad\qquad-\sum_{d=1}^N \ln \left(\sum_{k=1}^K \out^{(d)}(k \mid \parsn^A; t_{i-1}) \N{\ydd{d} \mid \alphat{i}\left(K\oh{k}{K}- \vec{1}_{K\times N}\right)}{\alphat{i} K \vec{I}_{K\times N \times N}}\right)\label{discdisc_t_loss_exp}
% \end{align}
\begin{align}
&\calL_A = n\E_{i \sim U\{1,n\},\flow^A(\parsn^A \mid \vA ; t_{i-1}),\N{\y \mid \alphat{i}\left(K \oh{\vA}{KD} - \vec{1}\right)}{\alphat{i} K \vec{I}}} \ln \N{\y \mid \alphat{i}\left(K \oh{\vA}{K\times N} - \vec{1}\right)}{\alphat{i} K \vec{I}}\nonumber\\
&\qquad\qquad\qquad-\sum_{d=1}^N \ln \left(\sum_{k=1}^K \out^{(d)}(k \mid \parsn^A; t_{i-1}) \N{\ydd{d} \mid \alphat{i}\left(K\oh{k}{K}- \vec{1}\right)}{\alphat{i} K \vec{I}}\right)\label{discdisc_t_loss_exp}
\end{align}
where $\vec{I}\in \R^{K\times N \times N}$ and $\vec{1}\in\R^{K\times D}$. When sampling, the transition from $\bthetaA_{i-1}$ to $\bthetaA_{i}$ is derived as:
\begin{equation}
    p(\btheta^A_{i}|\parsnt{i-1}^\calM)=\update^A(\btheta^A_{i}|\btheta^A_{i-1},\predA{i-1};t_{i-1})
\end{equation}

The detailed training and sampling algorithm could be found in \cref{alg:train} and \cref{alg:sampling}.





% Our main method, \textbf{Co}ntrastive Visual \textbf{D}ata \textbf{A}ugmentation (\textbf{CoDA}), is simple and easy to apply to LMMs in a variety of scenarios. Several components in the pipeline utilize existing off-the-shelf model components that can be easily swapped out for superior versions of similar models as research in their respective field progresses. Therefore, we expect the efficiency and effectiveness of \textbf{CoDA} to dramatically scale along with the advancement of relevant models. 

% Here we provide a step-by-step breakdown of the \textbf{CoDA} method:


As shown in Fig.\ref{fig:method}, \textbf{CoDA} consists of 4 major steps including contrastive textual and visual feature extraction, feature filtering, feature-controlled image generation, and augmented image filtering. Together these steps ensure \textbf{CoDA} reliably generates informative and high-quality augmented images that help LMMs recognize novel and confusing concepts.


\subsection{Feature Extraction}
\label{subsec:feature_extraction}

% For any novel, confusing, or low-resource concepts that the LMM has trouble recognizing (examples shown in Figure.\ref{fig:teaser}), we first extract visually identifying features from the concept. Such features can be later used to guide text-to-image generative models in providing targeted image generation. Specifically, we can extract such features from textual knowledge and few-shot visual examples. 

% \vspace{-1em}
\paragraph*{Textual Feature Extraction} 

% on current LMMs' concept recognition abilities with the SUN Dataset~\cite{xiao2010sun}, iNaturalist Dataset~\cite{van2018inaturalist}, and general real-world examples

In our exploratory experiments, we find that significant mis-recognition errors occur on low-resource or commonly mis-represented concepts in vision-language instruction fine-tuning and multimodal pre-training datasets, which the LMMs are trained on. For example, the LLaVA 1.6 (34B) model~\cite{liu2024llavanext}, mainly tuned on LAION-GPT-4V\cite{2024LAIONGPT4V} and ShareGPT-4V~\cite{chen2023sharegpt4v} datasets, has a strong tendency to mis-recognize interior images of ``Resupply Base'' as ``Wholesale Store'' (Fig.\ref{fig:teaser}). Unsurprisingly, we find that all related references of ``Resupply Base'' across the 3 instruction-tuning datasets only depict exterior views of the concept rather than interior views. While the concept itself is not a low-resource concept in existing text corpora, it is severely low-resource and also commonly mis-represented in vision-language instruction fine-tuning datasets.



To address this issue, we prompt LLMs to directly generate feature attributes of the target concept based on their existing knowledge, focusing on visual appearance, and avoiding hallucination for unfamiliar concepts. For this task, we use the cost-efficient GPT4o-mini model with chain of thought reasoning. Generally, textual feature extraction is most applicable for concepts that are high-resource in existing textual corpora, yet low-resource and/or commonly mis-represented in vision-language instruction-tuning and pre-training datasets. Here we do not try to classify which concepts fall under this criteria, but rather apply this step for all concepts. To ensure extracted feature quality and filter out hallucinated and/or non-visually-recognizable features, we pass all extracted features through an automatic filtering step, as described in~\ref{subsec:feature_filtering}.

We also considered other methods for textual feature extraction, including using knowledge bases~\cite{jin2024armada}, retrieval augmented generation, and LLMs with internet search. However, we believe currently the advantages brought by these methods do not out-weigh their complexity overhead, thus we opt for simplicity.



% textual feature extraction is attempted 

% In such cases where visual examples of the concept are scarce

% for cases where visual examples are scarce

% and textual knowledge may be general enough

% this can be enhanced with RAG with knowledge bases (ARMADA) or web search agents.
\vspace{-1em}
\paragraph*{Visual Feature Extraction}
\label{subsec:visual_feature_extraction}

While textual feature extraction generally works well for pre-existing and non-hyper-domain-specific concepts that are prevalent in textual data sources, it tends to fail when either of the conditions are not met. For example, a large language model with a knowledge cutoff prior to June 2023 would not be able to provide meaningful features regarding the Apple Vision Pro device announced in July, or the new animal species ``Clouded Tiger Cat (L. pardinoides)'' first described by scientists in April 2024~(\ref{fig:teaser}). In addition to this weakness, LLM-based textual feature extraction is also unreliable when asked to provide detailed information regarding hyper-domain-specific concepts like the ``Mazda MX-5 Miata RF'' or the ``Lear's Macaw (Anodorhynchus Leari)''. In practice, we observe that for novel and hyper-domain-specific concepts, most of the LLM extracted textual features end up being filtered out by our automatic feature filtering module.


To address this weakness, we implement an additional visual feature extraction module based on VLMs. Given a single image of the target concept, the VLM is asked to extract its key visual features. When there is more than one image containing the target concept available, we use a LM to de-duplicate and summarize the combined extracted visual features from all images. For simplicity and cost-efficiency, we use the GPT4o-mini model for both visual feature extraction and feature de-duplication.

% In contrast to textual feature extraction, visual feature extraction is most effective for hyper-domain-specific and novel concepts that are very rare or non-existent in textual corpora but have a limited number of visual examples, thus it well-complements the textual feature extraction method. Similarly: here we do not try to classify which concepts fall under this criteria, but rather apply this step for all concepts and rely on automatic filtering~\ref{subsec:feature_filtering} to remove low quality features. 

In contrast to textual feature extraction, visual feature extraction is most effective for hyper-domain-specific and novel concepts that are very rare or non-existent in textual corpora but have a limited number of visual examples. Thus, it well-complements textual feature extraction. Similarly, we do not attempt to classify which concepts fall under this criterion; instead, we apply this step to all concepts and rely on automatic filtering~(\ref{subsec:feature_filtering}) to remove low-quality features.

% concepts that are common in textual training data while rare / commonly mis-represented in visual-text training data, it does not work 





% another equally significant category of concept mis-recognition errors occur on hyper-domain-specific and novel concepts that are very rare or non-existent in textual corpora but have a limited number of visual-text examples, which can be used to extract useful visual features. 


% Clear examples of of this type of concepts include: newly discovered or uncommon plant/animal species (such as Clouded Tiger Cat in Figure.~\ref{fig:teaser}), newly released airplane, car, or electronic device models (such as the Apple Vision Pro), etc.


% For where concept is too novel, lacking textual knowledge in database or online.

% can work with as few as one single example



\vspace{-1em}
\paragraph*{Contrastive Feature Extraction} 
\label{subsec:contrastive_feature_extraction}

While basic textual and visual feature extraction both aim to exhaustively list identifying features of the target concept, this is essentially an intractable task for complex concepts as it usually requires a huge number of features to fully describe them. For novel or low-resource concepts the LMM has likely never seen before, it is extremely difficult to teach the LLM the new concept using an incomplete description. 

There are two potential solutions to this problem: (1). Leveraging hierarchical information to narrow down concept category and reduce descriptional features. (2). Illustrating the new concept based on contrastive differences from a similar existing concept the LMM already understands. Previous works in language and visual data augmentation~\cite{jin2024armada} tend to use solution (1). However, its feasibility is contingent on the existence of a comprehensive textual knowledge base or tree-like structure that already includes the target concept. As discussed in Section \ref{subsec:visual_feature_extraction}, this is often not the case for novel concepts such as new electronic products (e.g. Apple Vision Pro) or new animal species (eg. Clouded Tiger Cat).


To enable the handling of novel concepts and remove the need for external databases, we adopt solution (2) and perform contrastive multimodal feature extraction for all target concepts. First, we use the LMM's zero-shot inference on the target concept $\mathcal{C_\mathcal{T}}$ to obtain the misidentified concept $\mathcal{C_\mathcal{M}}$. Then, we perform contrastive textual and visual feature extraction by querying LLMs and VLMs for visually identifying features that belong to $\mathcal{C_\mathcal{T}}$ but not $\mathcal{C_\mathcal{M}}$. 




% Contrastive textual feature extraction, which queries LLMs for visually identifying features that distinguish the target concept from the confused concept.

% Contrastive visual feature extraction, which queries VLMs for visually identifying features that distinguish the target concept from the confused concept.





% as teaching LMMs a new concept by fully describing it from scratch requires a huge number of features.


% it usually requires a huge number of features to teach the model a concept from scratch.


% For concepts with simple names that resonate its category and 




% One idea is to describe the object from top down in a hierarchical manor\cite{jin2024armada}. However, this is often not ideal for two reasons:

% 1. for visual features, its often difficult to 
% 2. it requires a powerful model to generate 




% learn from similar concept

% learn fr



% there are always similar concept

% You can force the model to make a prediction as to 


% it is difficult to describe the concept from scratch

% its easier to learn from relevant concepts and figure out the difference








% \subsection{Feature Extraction}
% \subsubsection{Feature Selection and Data Synthesis for Fine-Tuning}

% \textbf{Attribute Extraction Using GPT-4o-mini:}

% We leveraged GPT-4o-mini to extract distinctive visual attributes of each target species. The extracted text features will be used later to generate prompts for synthesizing images for finetuning. There are two sources for features. 

% \begin{itemize}
%     \item \textit{From Visual Data:} By analyzing images of the species to identify observable characteristics. This means that GPT-4o-mini is provided with the image.
%     \item \textit{From Textual Knowledge:} By utilizing GPT-4's knowledge base to summarize important features. This means that GPT-4o-mini is not provided with the image.
% \end{itemize}

% There are six prompting strategies employed:

% \begin{itemize}
%     \item \textit{Visual Prompt} 
%     \item \textit{Textual Prompt} 
%     \item \textit{Contrastive Visual Prompt} 
%     \item \textit{Contrastive Textual Prompt} 
%     \item \textit{Visual-Text Combined Prompt} 
%     \item \textit{Contrastive Visual Text Combined Prompt} 
% \end{itemize}
% For Visual-Text combined prompting, we take the union of features extracted from test only and visual only prompts. In the contrastive prompting strategies, we distinguish in two ways:

% \begin{enumerate}
%     \item We provide GPT-4 with the names of both the main class (target species) and the confusing class, and prompt it to focus only on features of the main class, explicitly excluding features of the confusing class.
%     \item We employ automatic filtering methods based on real images to ensure the extracted features are truly distinctive and contrastive, which will be discussed in the following section.
% \end{enumerate}


% We experimented with mixtures of these approaches to optimize the quality and relevance of the extracted attributes.







\subsection{Feature Filtering}
\label{subsec:feature_filtering}


\paragraph*{Automatic Feature Filtering} 


After obtaining visually identifying features from contrastive textual and visual feature extraction, we filter them based on two key criteria:

\begin{enumerate}[topsep=0pt, itemsep=0em]
    \item \textbf{Discriminability ($D(f, \mathcal{C}_T, \mathcal{C}_M)$) :} measures whether a feature $f$ can indeed be used to differentiate the target class $\mathcal{C_\mathcal{T}}$ from the misidentified concept $\mathcal{C_\mathcal{M}}$ ($f$ must first be a valid feature of $\mathcal{C_\mathcal{T}}$).
    \item \textbf{Generability ($G(f, \mathcal{C}_T, \mathcal{C}_M)$) :} measures whether a feature $f$ can be properly generated by the text-to-image generative model.
\end{enumerate}


To calculate the Discriminability of a feature $f$ given the target concept $\mathcal{C_\mathcal{T}}$ and misidentified concept $\mathcal{C_\mathcal{M}}$, we compute the likelihood that CLIP~\cite{radford2021learning} associates this feature with real images of the target concept compared to the likelihood that it is associated with real images of the misidentified class:
\vspace{-0.5em}
\[
D(f, \mathcal{C}_T, \mathcal{C}_M) = \sum_{i \in I} \frac{\text{CLIP}(f, i_{\mathcal{C}_T}^{\text{real}})}{\text{CLIP}(f, i_{\mathcal{C}_T}^{\text{real}}) + \text{CLIP}(f, i_{\mathcal{C}_M}^{\text{real}})}
\]
% \vspace{-0.1em}
Here we use an equal number of images of the target and misidentified concepts. A score below 0.5 indicates that the feature is more likely to be associated with the misidentified class rather then the target class. To ensure that selected features are more strongly associated with the target class, we filter out all features with Discriminability below 0.6. This method avoids the CLIP score bias against smaller features by only comparing feature association with the two classes and not relying on the absolute CLIP score. 
% It also ensures that the target concept $\mathcal{C_\mathcal{T}}$ actually contains the feature $\mathcal{C_\mathcal{M}}$ as this is a prerequisite for strong association.

Generability is calculated in a similar manner, comparing the average CLIP similarity between $f$ and synthetic images of the target concept against the average CLIP similarity between $f$ and real images of the misidentified concept:

\vspace{-0.7em}
\[
G(f, \mathcal{C}_T, \mathcal{C}_M) = \sum_{i \in I} \frac{\text{CLIP}(f, i_{\mathcal{C}_T}^{\text{synthetic}})}{\text{CLIP}(f, i_{\mathcal{C}_T}^{\text{synthetic}}) + \text{CLIP}(f, i_{\mathcal{C}_M}^{\text{real}})}
\]
\vspace{-0.7em}

Here we rank all remaining features by their Generability score and select the top 5 features to be passed to the text-to-image generative model (as current diffusion models usually have limited text encoder attention span). This step identifies features that not only help distinguish the target concept, but also can be effectively rendered by the text-to-image generative model in synthetic images, which is critical to the success of synthetic data augmentation.

Our automatic feature filtering module based on Discriminability and Generability ensures feature quality and limits the information loss between features and the generated augmented images. The remaining features are used for image generation and improving in-context recognition ability in inference prompts. We further verify the quality of remaining features with human evaluation in Sec.\ref{sec:human_eval}.

% \[
% \text{D}(f, \mathcal{C}_T, \mathcal{C}_M) = \sum \frac{\text{CLIP}(f, I_{\mathcal{C}_T}^{\text{real}})}{\text{CLIP}(f, I_{\mathcal{C}_T}^{\text{real}}) + \text{CLIP}(f, I_{\mathcal{C}_M}^{\text{real}})}
% \]

% \[
% \text{G}(f, \mathcal{C}_T, \mathcal{C}_M) = \sum \frac{\text{CLIP}(f, I_{\mathcal{C}_T}^{\text{synthetic}})}{\text{CLIP}(f, I_{\mathcal{C}_T}^{\text{synthetic}}) + \text{CLIP}(f, I_{\mathcal{C}_M}^{\text{real}})}
% \]



% \[
% \text{Discriminability}(f, {\text{$\mathcal{C_\mathcal{T}}$, $\mathcal{C_\mathcal{M}}$}}) = 
% \frac{\text{CLIP}(\text{f},\, \text{$\mathcal{C_\mathcal{T}}$ images})}
%      {\substack{\text{CLIP}(\text{f},\, \text{$\mathcal{C_\mathcal{T}}$ images}) + \\
%       \text{CLIP}(\text{f},\, \text{$\mathcal{C_\mathcal{M}}$ images})}}
% \]



% \[
% \text{Generability}(f, {\text{$\mathcal{C_\mathcal{T}}$, $\mathcal{C_\mathcal{M}}$}}) = 
% \frac{\text{CLIP}(\text{f},\, \text{synthetic $\mathcal{C_\mathcal{T}}$ images})}
%      {\substack{\text{CLIP}(\text{f},\, \text{synthetic $\mathcal{C_\mathcal{T}}$ images}) + \\
%       \text{CLIP}(\text{f},\, \text{$\mathcal{C_\mathcal{M}}$ images})}}
% \]



% To enhance the effectiveness of our fine-tuning process, we implement a feature filtering method that selects the most distinctive and generable features of each main class. This filtering is based on two main criteria:

% \begin{enumerate}
%     \item \textbf{Combining Features from Multiple Images:} We extract features from 5 distinct images of the main class to capture a comprehensive set of visual attributes.
%     \item \textbf{CLIP Score-Based Filtering:} We use CLIP scores to filter out features based on:
%     \begin{itemize}
%         \item \textit{Contrastiveness:} How much a feature appears in real images of the main class compared to real images of the confusing class.
%         \item \textit{Generability:} How well a feature appears in synthetic images of the main class compared to real images of the confusing class.
%     \end{itemize}
% \end{enumerate}

% The exact algorithm is as follows:

% \begin{enumerate}
%     \item \textbf{Feature Extraction:} We use GPT-4 to extract features from each of the selected images of the main class. Each image yields a list of features.

%     \item \textbf{Feature Consolidation:} We combine the features from all images into a single list, removing duplicates to create a unified set of candidate features.

%     \item \textbf{Contrastiveness Scoring:} For each feature, we compute a contrastiveness score using CLIP embeddings to measure how much more likely the feature is associated with the main class than with the confusing class. The score is calculated as:

% \[
% \text{Score}_{\text{contrastiveness}} = 
% \frac{\text{CLIP}(\text{feature},\, \text{main class images})}
%      {\substack{\text{CLIP}(\text{feature},\, \text{main class images}) + \\
%       \text{CLIP}(\text{feature},\, \text{confusing class images})}}
% \]

%     where $\text{CLIP}(\text{feature}, \text{images})$ represents the cosine similarity between the textual description of the feature and the image embeddings.

%     \item \textbf{Contrastiveness Filtering:} We filter out features with a contrastiveness score less than 0.6. This threshold ensures that the selected features are more strongly associated with the main class than with the confusing class (a score below 0.5 would indicate the feature is more associated with the confusing class).

%     \item \textbf{Generability Scoring:} For the remaining features, we assess their generability by generating synthetic images of the main class conditioned on each feature. We then compute the CLIP similarity between the feature and these synthetic images, as well as between the feature and real images of the confusing class. The generability score is calculated as:
% \[
% \text{Score}_{\text{generability}} = 
% \frac{\text{CLIP}(\text{feature},\, \text{synthetic main class images})}
%      {\substack{\text{CLIP}(\text{feature},\, \text{synthetic main class images}) + \\
%       \text{CLIP}(\text{feature},\, \text{confusing class images})}}
% \]


%     \item \textbf{Generability Ranking:} We rank the features based on their generability scores. This step identifies features that are not only distinctive but also can be effectively rendered in synthetic images.

%     \item \textbf{Feature Selection:} We select the top features based on the generability ranking (we choose the top 5 features) for use in synthetic image generation during fine-tuning. This ensures that the features are both distinctive and generable.

% \end{enumerate}

% By applying this feature filtering method, we increase the quality of automatically selected features. This enhances the model's ability to distinguish between confusing pairs by focusing on the most relevant and generable features.


% \subsubsection{Human Eval}
% \TODO{bryan}

\subsection{Image Generation and Verification}

% \vspace{-1em}
\paragraph{Image Generation}

After feature extraction and filtering based on Discriminability and Generability, we pass the selected features to a text-to-image generative model to generate augmented visual data. We experiment with both SOTA open-weights~\citep{esser2024scaling, stablediffusion3.5} and proprietary~\cite{2024RecraftV3} models.

% Stable Diffusion 3.5 Large Turbo model
\vspace{-0.5em}
\paragraph{Verification} To ensure final images for augmentation contain our extracted and filtered target concept features, we propose a simple automatic verification metric that checks whether desired features are recognized in the augmented images by the LMM we want to update: Given the vanilla LMM $\mathcal{M}$, a set of features $\mathcal{F}$, and %the set of 
an augmented images $i^{\text{synthetic}}$, 
the feature satisfaction rate $S(i^\text{synthetic}, F, M)$ for each augmented image:

\vspace{-0.7em}
\[
S(i^\text{synthetic}, \mathcal{F}, \mathcal{M}) = \frac{\sum_{f \in \mathcal{F}} \mathbf{1}\{ \mathcal{M}(f, i^\text{synthetic}) \}}{|\mathcal{F}|}
\]
\vspace{-0.7em}


Here $\mathcal{M}(f, i^\text{synthetic}) \}$ returns true if the feature $f$ is recognized in the image $i^\text{synthetic}$. Afterwards, we filter out all images with $S(i^\text{synthetic}, F, M)$ \textless 1.0, keeping only augmented images that fully match all target concept features.



% \violet{this will likely raise concern that the data augmentation method is "model dependent" because for different model, we need to build the verification differently. Probably worth discussing and be prepared to add results for some "model agnostic" verification.}







% The score evaluates each target feature as a constraint and calculates a satisfaction rate ($R_{\text{satisfied}}$) for each image by prompting LLaVA to detect and verify each specified feature



% implemented an automated verification pipeline with LLaVA. The pipeline evaluates each target feature as a constraint and calculates a satisfaction rate ($R_{\text{satisfied}}$) for each image by prompting LLaVA to detect and verify each specified feature:
% \[
% R_{\text{satisfied}} = \frac{\text{Number of Satisfied features}}{\text{Total number of features}}
% \]


% After feature extraction, we generate synthetic images based on these features through a two-stage pipeline: generation and verification.

% \paragraph{Generation} We construct text prompts using the extracted features and generate synthetic images using Stable Diffusion 3.5 Large \citep{esser2024scaling, githubGitHubStabilityAIsd35}.The prompt templates are attached in the Appendix .

% \paragraph{Verification} To ensure generated images contain our extracted and filtered target concept features, we implemented an automated verification pipeline with LLaVA. The pipeline evaluates each target feature as a constraint and calculates a satisfaction rate ($R_{\text{satisfied}}$) for each image by prompting LLaVA to detect and verify each specified feature:
% \[
% R_{\text{satisfied}} = \frac{\text{Number of Satisfied features}}{\text{Total number of features}}
% \]
% We filter out images with $R_{\text{satisfied}}$ \textless 1.0, keeping only those that fully match all target features.






\subsection{Human Evaluation} 
\label{sec:human_eval}

\begin{tabular}{lccc}
%& \multicolumn{3}{c}{\textbf{ToTTo}} \\
%\cmidrule(lr){2-4}
 & Win\% & Tie\% & Loss\% \\
\midrule
SFT & 15.2 & 44.8 & 40.0 \\
\scope  & \textbf{40.0} & 44.8  & 15.2 \\
\midrule
\end{tabular}

To verify the reliability of our feature filtering and augmented image verification modules, we conduct human evaluation on a subset of iNaturalist and the novel animal species dataset. For target concepts, we select 100 image-feature pairs for both real and augmented synthetic images. We also select 100 image-feature pairs for real images of misidentified concepts. 3 external human annotators are asked to label whether they believe the given feature belongs to the concept in the corresponding image.


Results in Tab.\ref{tab:human_eval} show human annotators overwhelmingly agree that the final extracted features belong to the target concept (92\%) but not the misidentified concept (14\%). The augmented synthetic images of the target concept also likely contains the desired features (83\%), though as expected, there is some information loss between the text-to-image generation step. In addition, the three independent annotators generally agreed in their response (\textgreater 0.8 IAA).



\subsection{In-Context Inference for Enhanced Recognition}
In addition to updating the LMM with augmented data, we can further boost performance by integrating the extracted features into the inference prompt. For each query, we can append a concise list of the most discriminative and generable features of the target and confusable classes. These features serve as an in-context guide, focusing the LMM’s attention on critical distinguishing attributes. By explicitly highlighting what to look for (and what not to mistake it for), the model more reliably identifies the correct concept.



% We conducted a human evaluation study to assess the reliability of our verification step. We recruited 3 external volunteer to compare LLaVA's automated feature detection results with human judgments of whether these features were present in the generated images. 






% \subsection{Model Updating and Inference}

% We selected LLaVA-v1.6-34b \citep{liu2024llavanext, liu2023improvedllava, liu2023llava} as our experimental framework to evaluate knowledge updating using augmented visual data. The model was fine-tuned using LoRA on a subset of the training dataset. During inference, we enhanced the model's classification capabilities by incorporating extracted features into the prompt structure. Specifically, we prepended text features from both the main class and potential confounding classes as a prefix to the input prompt. The complete prompt template is provided in Appendix \ref{}.
% We conducted experiments under both high-resource and low-resource settings:

% \paragraph{High-Resource Setting}
% In this configuration, we randomly selected 20 images per class across 20 class pairs, yielding a training set of 800 images. The model was fine-tuned using LoRA for 30 epochs. After merging the trained weights with the original model, we evaluated performance on a held-out validation set containing 20 images per class.

% \paragraph{Low-Resource Setting}
% To test our method's efficiency with limited data, we reduced the training samples to 5 images per class while maintaining the same number of class pairs (20 pairs). The training protocol remained consistent with the high-resource setting, including the 30-epoch duration and LoRA fine-tuning. The evaluation procedure on the validation set was also kept identical.

