\begin{figure*}[t!]
  \centering
  \includegraphics[width=\linewidth]{img/method.pdf}
   %\includegraphics[width=0.8\linewidth]{egfigure.eps}
    \vspace{-2em}
   \caption{\footnotesize \textbf{The \textbf{CoDA} method}. Including \textcolor[HTML]{0F9ED5}{Feature Extraction}, \textcolor[HTML]{0F9ED5}{Feature Filtering}, \textcolor[HTML]{A02B93}{Feature-controlled Augmentation}, and \textcolor[HTML]{A02B93}{Augmented Image Filtering}. The \textcolor[HTML]{4EA72E}{target concept} and \textcolor[HTML]{FA4032}{misidentified concept} are highlighted respectively. Specific feature filtering scores are for illustration only. Here the example concepts Anodorhynchus Leari (Lear's Macaw) and Cyanopsitta Spixii (Spix's Macaw) are from the iNaturalist~\cite{van2018inaturalist} dataset, and augmented images are produced by the Recraft V3 model~\cite{2024RecraftV3}. }
   \vspace{-1em}
   \label{fig:method}
\end{figure*}








% \begin{figure*}
%   \centering
%   \begin{subfigure}{0.68\linewidth}
%     \fbox{\rule{0pt}{2in} \rule{.9\linewidth}{0pt}}
%     \caption{An example of a subfigure.}
%     \label{fig:short-a}
%   \end{subfigure}
%   \hfill
%   \begin{subfigure}{0.28\linewidth}
%     \fbox{\rule{0pt}{2in} \rule{.9\linewidth}{0pt}}
%     \caption{Another example of a subfigure.}
%     \label{fig:short-b}
%   \end{subfigure}
%   \caption{Example of a short caption, which should be centered.}
%   \label{fig:short}
% \end{figure*}