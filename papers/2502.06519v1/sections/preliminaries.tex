\section{Preliminaries}
\label{sec:preliminaries}
We present relevant notation used in this paper. We denote the trace of a function by $\trace$ and the determinant and variance of a matrix by $\det$ and $\var$, respectively. Further, we denote the strictly positive orthant by ${\mbb{R}_{++}}$. Next, we provide a brief introduction to Gaussian Splatting.
Gaussian Splatting represents non-empty space in a scene using a set of ellipsoidal primitives, each parameterized by a mean ${\mu \in \mbb{R}^{3}}$, a covariance ${\Sigma \in \mbb{R}^{3 \times 3}}$ defined by a rotation matrix ${H \in \SO(3)}$ and a diagonal scaling matrix ${\Lambda \in \mbb{R}^{3 \times 3}}$, an opacity parameter ${\alpha \in [0, 1]}$, and spherical harmonic parameters. These attributes are optimized via gradient descent on the loss function: ${\mcal{L}_{\mathrm{gs}} = (1 - \lambda) \sum_{\mcal{I} \in \mcal{D}} \norm{\mcal{I} - \hat{\mcal{I}}}_{1} + \lambda \mcal{L}_{\mathrm{D-SSIM}}}$, over the training dataset $\mcal{D}$, where ${\lambda \in (0, 1)}$ represents the relative weight term and $\mcal{L}_{\mathrm{D-SSIM}}$ represents the differentiable structural similarity loss index measure. The first term in the rendering loss represents the photometric loss between the ground-truth image and the rendered image, generated via a tile-based rasterization procedure, given a camera pose.