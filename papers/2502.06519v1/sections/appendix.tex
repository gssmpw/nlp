\section{Coarse Gaussian-to-Gaussian Registration}
\label{app:method}
We discuss the derivation of the closed-form solution to \eqref{eq:coarse_registration}.
Let the objective function of \eqref{eq:coarse_registration} be denoted by $J$.
From the first-order optimality conditions: 
\begin{equation}
    \nabla_{t} J = \sum_{(i, j) \in \mcal{E}} \left( w_{ij} s_{c}Rp_{i} + t - q_{j} \right) = 0,
\end{equation}
yielding the optimal translation:
\begin{equation}
    \label{eq:coarse_registration_opt_translation_app}
    t_{c}^{\star} = \tilde{\mu}_{\mcal{Q}} - s_{c}^{\star} R^{\star} \tilde{\mu}_{\mcal{P}},
\end{equation}
where ${\tilde{\mu}_{\mcal{P}}}$ and ${\tilde{\mu}_{\mcal{Q}}}$ denote the weighted average of the means of the Gaussians in $\mcal{P}$ and $\mcal{Q}$, with weights $w_{ij}$ for Gaussian $i$ in $\mcal{P}$ and Gaussian $j$ in $\mcal{Q}$.
By substituting the optimal value of $t$ \eqref{eq:coarse_registration_opt_translation_app} in \eqref{eq:coarse_registration}, we obtain the following optimization problem over $s_{c}$ and $R$:
\begin{equation}
    \label{eq:coarse_registration_frob}
    \begin{aligned}
        \minimize{s_{c} \in \mbb{R}_{++}, R \in \SO(3)} &\frac{1}{2} \norm{(s_{c}R\check{P} - \check{Q})W^{\frac{1}{2}}}_{F}^{2} \\
        & + \frac{1}{2} \sum_{(i, j) \in \mcal{E}}  w_{ij} \norm{s_{c}R \check{H}_{p_{i}} - \check{H}_{q_{j}}}_{F}^{2},
    \end{aligned}
\end{equation}
where ${\check{P} \in \mbb{R}^{3 \times N}}$ and ${\check{Q} \in \mbb{R}^{3 \times N}}$ represent the \emph{zero-centered} Gaussians in $\mcal{P}$ and ${\mcal{Q}}$, respectively, with the $i$th column of $\check{P}$ given by ${\check{P}_{i} = p_{i} - \tilde{\mu}_{\mcal{P}}}$ and similarly for the $j$th column of $\check{Q}$, and 
${\check{H}_{p_{i}} = H_{p_{i}}  \Lambda_{p_{i}}}$ and ${\check{H}_{q_{j}} = H_{q_{j}}  \Lambda_{q_{j}}}$. Lastly, ${W \in \mbb{R}^{N \times N}}$ denotes the diagonal weight matrix, ${W_{kk} = w_{k}}$, with ${w_{k} = w_{ij}}$,~${\forall k = (i, j) \in \mcal{E}}$.
Now, we can reformulate \eqref{eq:coarse_registration_frob} as a trace-minimization problem, by leveraging the relation: ${\norm{A}_{F}^{2} = \trace(A^{\top}A)}$ for any real-valued matrix ${A \in \mbb{R}^{m \times n}}$.
Reformulating the problem as a trace-minimization problem enables us to decompose the norm-minimization problem \eqref{eq:coarse_registration_frob} into a nested pair of subproblems: an outer subproblem over $s_{c}$ and an inner subproblem over $R$. We can simplify the inner subproblem into the form:
\begin{equation}
    \label{eq:coarse_registration_trace_R}
    \begin{aligned}
        \minimize{R \in \SO(3)} &-\trace\left(R^{\top} \left(\check{Q}W\check{P}^{\top}
        + \sum_{(i, j) \in \mcal{E}} w_{ij} \check{H}_{q_{j}}\check{H}_{p_{i}}^{\top}\right)\right),
    \end{aligned}
\end{equation}
which affords a closed-form optimal solution, with 
\begin{equation}
    \label{eq:coarse_registration_opt_R_app}
    R_{c}^{\star} = U_{c} \Theta_{c} V_{c}^{\top},
\end{equation}
where ${U_{c} \Sigma_{c} V_{c}^{\top} = \check{Q}W\check{P}^{\top}
+ \sum_{(i, j) \in \mcal{E}} w_{ij} \check{H}_{q_{j}}\check{H}_{p_{i}}^{\top}}$, computed via the singular value decomposition (SVD),
and ${\Theta_{c} = \diag(1, 1, \det(U_{c}V_{c}^{\top}))}$. Using the first-order optimality condition, we can compute the optimal scale after computing the optimal rotation from the outer subproblem given by:
\begin{equation}
    \label{eq:coarse_registration_trace_scale}
    \begin{aligned}
        \minimize{s_{c} \in \mbb{R}_{++}} & \frac{1}{2} s_{c}^{2} \trace\left(W\check{P}^{\top}\check{P} + \sum_{(i, j) \in \mcal{E}} w_{ij} \check{H}_{p_{i}}^{\top}\check{H}_{p_{i}}\right) \\
        & \hspace{-1em} - s_{c}\trace\left(R^{\top} \left(\check{Q} W \check{P}^{\top} + \sum_{(i, j) \in \mcal{E}} w_{ij} \check{H}_{q_{j}}\check{H}_{p_{i}}^{\top} \right) \right),
    \end{aligned}
\end{equation}
yielding the optimal solution:
\begin{equation}
    \label{eq:coarse_registration_opt_scale_app}
    s_{c}^{\star} = \frac{\trace(\Theta_{c} \Sigma)}{\trace\left(W\check{P}^{\top}\check{P} + \sum_{(i, j) \in \mcal{E}} w_{ij} \check{H}_{p_{i}}^{\top}\check{H}_{p_{i}}\right)}.
\end{equation}
For brevity, we omit further analysis of the optimality of the solution and refer interested readers to \cite{umeyama1991least} for the proof of a related problem, which applies to the problem considered in this work.

\section{Experiments}
\label{sec:appendix_experiments}
We report the photometric performance of each registration method with the Mip-NeRF360 dataset and the data collected by the robots in our experiments and provide further discussion of the experimental results. In addition, we present ablations, examining the different components of \algname. Lastly, we explore finetuning the resulting composite maps for higher visual fidelity.

\subsection{Mip-NeRF360 Dataset}
\label{ssec:appendix_experiments_mip_nerf_360}
\smallskip
\noindent\textbf{Geometric Evaluation.}
In the \emph{Room} scene, PhotoReg achieves the lowest rotation error by a factor of about $1.47$x but also achieves the largest translation and scale errors. 
Based on the results across all scenes, \algname almost always consistently outperforms competing methods. From \Cref{tab:baseline_geometric_performance_metrics}, RANSAC-GR and FGR achieve the fastest computation times; however, RANSAC-GR and FGR do not generally achieve consistently low geometric errors. Although \algname is slower than the classical point-cloud registration algorithms and GaussReg, \algname generally outperforms these methods in accuracy by significant margins. Moreover, about $40\%$ to $50\%$ of the total computation time of \algname is spent on the semantics extraction procedure. Hence, the total computation time can be significantly improved by utilizing faster semantics distillation methods, e.g., \cite{shorinwa2024fast}.


\smallskip
\noindent\textbf{Photometric Evaluation.}
From \Cref{fig:photometric_performance_rendered_images}, in the \emph{Playroom} scene, PhotoReg fails to sufficiently register the individual maps to obtain photorealistic renderings. In contrast, GaussReg, Colored-ICP, and \algname-R generate high-fidelity renderings. However, the fused maps generated by GaussReg and Colored-ICP are not accurately aligned, compared to that of \algname-R, as evidenced in insets in the images. The fused map in GaussReg and Colored-ICP contain duplicate objects due to inaccurate registration of the individual maps. In contrast, \algname-R provides greater accuracy. Likewise, \algname-R achieves the highest-fidelity rendering in the \emph{Truck} scene with consistent geometry, whereas Colored-ICP fails to register the left and right sides of the truck. Although GaussReg fuses both sides of the truck, GaussReg fails to compute a high-accuracy transform, resulting in the artifacts visible in \Cref{fig:photometric_performance_rendered_images}. Although PhotoReg registers the cargo bed of the truck in both maps, PhotoReg fails to align the truck accurately in terms of the rotation transform, with the front end of the truck in one map registered to the rear end of the truck in the other map.
Finally, in the \emph{Room} scene, whereas \algname-R generates high-fidelity rendered images, other methods fail to accurately register the individual maps. In particular, Colored-ICP generates a fused map with duplicate objects, e.g., the piano and the table, indicated by the green squares, while PhotoReg and GaussReg generate fused maps with notable artifacts.

\begin{table*}[th]
	\centering
	\caption{Photometric performance of registration algorithms for GSplat maps from the Mip-NeRF360 dataset.}
	\label{tab:baseline_photometric_performance_metrics}
	\begin{adjustbox}{width=\linewidth}
		{\begin{tabular}{l | c c c | c c c | c c c }
				\toprule
                    & \multicolumn{3}{c |}{\emph{Playroom}} & \multicolumn{3}{c |}{\emph{Truck}} & \multicolumn{3}{c}{\emph{Room}} \\
				Methods & PSNR $\uparrow$ & SSIM  $\uparrow$ & LPIPS  $\downarrow$ & PSNR $\uparrow$ & SSIM  $\uparrow$ & LPIPS  $\downarrow$ & PSNR $\uparrow$ & SSIM  $\uparrow$ & LPIPS  $\downarrow$ \\
				\midrule
                    PhotoReg \cite{yuan2024photoreg} & 11.5 $\pm$ 2.3 &   0.68 $\pm$ 0.11 & 0.67 $\pm$ 0.12 & 10.6 $\pm$ 0.9 & 0.39 $\pm$ 0.07 & 0.72 $\pm$ 0.08 & 10.2 $\pm$ 1.2 & 0.46 $\pm$ 0.07 &  0.78 $\pm$  0.04 \\
                    GaussReg \cite{chang2025gaussreg} & 23.7 $\pm$ 3.3 &  0.86 $\pm$ 0.06 & 0.22 $\pm$ 0.08 & 13.4 $\pm$ 1.9 & 0.54 $\pm$ 0.12 & 0.53  $\pm$ 0.13  &  13.4 $\pm$ 3.6  &  0.61 $\pm$ 0.11  &  0.55  $\pm$  0.15 \\
                    RANSAC-GR \cite{fischler1981random, holz2015registration} & 17.9 $\pm$ 3.2 & 0.77 $\pm$ 0.09 & 0.37 $\pm$ 0.09 & \cellcolor{WildStrawberry!40}{18.9 $\pm$ 7.0} & \cellcolor{WildStrawberry!40}{0.66 $\pm$ 0.24} & \cellcolor{Goldenrod!40}{0.32 $\pm$ 0.22} & 14.2 $\pm$ 2.4 & 0.66 $\pm$ 0.09 & 0.46 $\pm$ 0.11 \\
                    FGR \cite{zhou2016fast} & 22.2 $\pm$ 3.2 & 0.85 $\pm$ 0.06 & 0.24 $\pm$ 0.08 & 13.0 $\pm$ 2.6 & 0.57 $\pm$ 0.19 & 0.43 $\pm$ 0.20 & \cellcolor{GreenYellow!40}17.2 $\pm$ 2.3 & \cellcolor{GreenYellow!40}0.77 $\pm$ 0.08 & \cellcolor{GreenYellow!40}0.33 $\pm$ 0.11 \\
                    ICP \cite{rusinkiewicz2001efficient} & 22.7 $\pm$ 3.3 & 0.85 $\pm$ 0.06 & 0.24 $\pm$ 0.08 & 12.9 $\pm$ 2.6 & 0.57 $\pm$ 0.19 & 0.43 $\pm$ 0.20 & 16.8 $\pm$ 2.2 & 0.76 $\pm$ 0.09 & 0.35 $\pm$ 0.11 \\
                    Colored-ICP \cite{park2017colored} & \cellcolor{GreenYellow!40}{26.2 $\pm$ 3.1} & \cellcolor{Goldenrod!40}{0.89 $\pm$ 0.04} & \cellcolor{Goldenrod!40}{0.17 $\pm$ 0.06} & 13.4 $\pm$ 2.4 & \cellcolor{Goldenrod!40}{0.58 $\pm$ 0.19} & 0.40 $\pm$ 0.18 & 15.4 $\pm$ 2.1 & 0.71 $\pm$ 0.10 & 0.41 $\pm$ 0.13 \\
                    \algname-NR [{Ours}] & \cellcolor{Goldenrod!40}{26.3 $\pm$ 3.1} & \cellcolor{GreenYellow!40}{0.87 $\pm$ 0.05} & \cellcolor{Goldenrod!40}{0.17 $\pm$ 0.06} & \cellcolor{GreenYellow!40}15.4 $\pm$ 1.7 & 0.52 $\pm$ 0.12 & \cellcolor{GreenYellow!40}0.35 $\pm$ 0.05 & \cellcolor{WildStrawberry!40}24.8 $\pm$ 3.3 & \cellcolor{WildStrawberry!40}{0.83 $\pm$ 0.04} & \cellcolor{WildStrawberry!40}{0.22 $\pm$ 0.06} \\
                    \algname-R [{Ours}] & \cellcolor{WildStrawberry!40}{28.3 $\pm$ 2.9} & \cellcolor{WildStrawberry!40}{0.90 $\pm$ 0.04} & \cellcolor{WildStrawberry!40}{0.15 $\pm$ 0.06} & \cellcolor{Goldenrod!40}{16.4 $\pm$ 2.4} & \cellcolor{GreenYellow!40}0.57 $\pm$ 0.13 & \cellcolor{WildStrawberry!40}{0.31 $\pm$ 0.07} & \cellcolor{Goldenrod!40}{24.1 $\pm$ 3.1} & \cellcolor{Goldenrod!40}{0.82 $\pm$ 0.05} & \cellcolor{Goldenrod!40}{0.23 $\pm$ 0.06} \\
				\bottomrule
		\end{tabular}}
	\end{adjustbox}
\end{table*}


\subsection{Mobile-Robot Mapping}
\label{ssec:appendix_experiments_mobile_mapping}
\smallskip
\noindent\textbf{Photometric Evaluation.}
From \Cref{fig:photometric_performance_mobile_robot_rendered_images},
as highlighted by the green squares, \algname-R generates composite maps that are consistent with the ground-truth, unlike GaussReg and PhotoReg. The fused maps generated by GaussReg and PhotoReg contain conspicuous artifacts due to inaccurate registration of the individual maps created by the robots, especially in the \emph{Kitchen} scene. PhotoReg fails to sufficiently register the individual maps, resulting in blurry renderings, with few recognizable features, e.g., in the \emph{Workshop} scene. In the \emph{Apartment} scene, the rendered images from GaussReg contain duplicate objects, unlike those of \algname-R, which have accurate geometric detail.

\begin{table*}[th]
	\centering
	\caption{Photometric performance of GSplat registration algorithms for mobile-robot mapping.}
	\label{tab:baseline_photometric_performance_metrics_mobile_robot_mapping}
	\begin{adjustbox}{width=\linewidth}
		{\begin{tabular}{l | c c c | c c c | c c c }
				\toprule
                    & \multicolumn{3}{c |}{\emph{Kitchen}} & \multicolumn{3}{c |}{\emph{Workshop}} & \multicolumn{3}{c}{\emph{Apartment}} \\
				Methods & PSNR $\uparrow$ & SSIM  $\uparrow$ & LPIPS  $\downarrow$ & PSNR $\uparrow$ & SSIM  $\uparrow$ & LPIPS  $\downarrow$ & PSNR $\uparrow$ & SSIM  $\uparrow$ & LPIPS  $\downarrow$ \\
				\midrule
                    PhotoReg \cite{yuan2024photoreg} & 0.75 $\pm$ 0.05 & 11.6 $\pm$ 1.3 & 0.53 $\pm$ 0.06 & 0.78 $\pm$ 0.08 & 11.3 $\pm$ 1.3 &   0.48 $\pm$ 0.05 & 13.4 $\pm$ 1.5 & 0.61 $\pm$ 0.05 & 0.75 $\pm$ 0.04 \\
                    GaussReg \cite{chang2025gaussreg} & \cellcolor{GreenYellow!40}14.1 $\pm$ 1.9 & \cellcolor{GreenYellow!40}0.62 $\pm$ 0.06 & \cellcolor{GreenYellow!40}0.60 $\pm$ 0.11 & \cellcolor{GreenYellow!40}17.0 $\pm$ 2.8 & \cellcolor{Goldenrod!40}0.61  $\pm$ 0.05 & \cellcolor{GreenYellow!40}0.53 $\pm$ 0.11 & \cellcolor{GreenYellow!40}14.3 $\pm$ 1.6 & \cellcolor{GreenYellow!40}0.62 $\pm$ 0.05 & \cellcolor{GreenYellow!40}0.62 $\pm$ 0.04 \\
                    \algname-NR [\textbf{Ours}] & \cellcolor{WildStrawberry!40}19.3 $\pm$ 3.2 & \cellcolor{WildStrawberry!40}0.63 $\pm$ 0.05 & \cellcolor{WildStrawberry!40}0.40 $\pm$ 0.09 & \cellcolor{Goldenrod!40}19.9 $\pm$ 2.5 &  \cellcolor{GreenYellow!40}0.60 $\pm$ 0.04 & \cellcolor{Goldenrod!40}0.40 $\pm$ 0.08 & \cellcolor{Goldenrod!40}15.3 $\pm$ 1.5 &  \cellcolor{WildStrawberry!40}0.64 $\pm$ 0.04 & \cellcolor{WildStrawberry!40}0.55 $\pm$ 0.03   \\
                    \algname-R [\textbf{Ours}] & \cellcolor{Goldenrod!40}18.8 $\pm$ 2.8 & \cellcolor{Goldenrod!40}0.62 $\pm$ 0.05 & \cellcolor{Goldenrod!40}0.41 $\pm$ 0.08 & \cellcolor{WildStrawberry!40}20.3 $\pm$ 2.7 & \cellcolor{WildStrawberry!40}0.62 $\pm$ 0.04 & \cellcolor{WildStrawberry!40}0.38 $\pm$ 0.09 & \cellcolor{WildStrawberry!40}15.3 $\pm$ 1.4 & \cellcolor{Goldenrod!40}0.63 $\pm$ 0.04 & \cellcolor{WildStrawberry!40}0.55 $\pm$ 0.03  \\
				\bottomrule
		\end{tabular}}
	\end{adjustbox}
\end{table*}


\subsection{Ablations}
We examine the constituent registration steps in \algname, namely: the coarse Gaussian-to-Gaussian and fine photometric registration procedures, assessing the accuracy of the registration result generated by each procedure. We denote the variant of \algname with coarse registration performed without RANSAC and fine registration by \algname-CNR. Likewise, we denote the variant of \algname with coarse registration performed using RANSAC but without fine registration by \algname-CR. We compute the geometric and photometric performance metrics for each of these variants and report the results in \Cref{tab:baseline_geometric_performance_metrics_ablation} and \Cref{tab:baseline_photometric_performance_metrics_ablation}, respectively. We also report the performance of \algname-NR and \algname-R from \Cref{tab:baseline_geometric_performance_metrics} and \Cref{tab:baseline_photometric_performance_metrics} for easy reference.  From \Cref{tab:baseline_geometric_performance_metrics_ablation}, we note that the fine registration step in \algname notably improves the rotation error to sub-degree errors, achieving about $2$x smaller translation errors and in some cases, $100$x smaller translation errors. Likewise, the fine registration step generally results in much smaller scale errors, although not necessarily in all cases, as reflected in the \emph{Truck} scene.
Similarly, the variants of \algname with fine registration (i.e., \algname-NR and \algname-R) achieve notably higher photometric performance, especially in the \emph{Playroom} and \emph{Room} scenes, reported in \Cref{tab:baseline_photometric_performance_metrics_ablation}. 
In general, the coarse registration step brings corresponding objects in both GSplat maps into close proximity in the fused map. However, the resulting fused map lacks precise geometric detail, degrading its visual fidelity. After the coarse registration step, the fine registration procedure refines the transformation parameters for precise alignment of the individual maps, ultimately generating a photorealistic fused map.

Although \algname-CR does not always outperform RANSAC-GR in \Cref{tab:baseline_geometric_performance_metrics}, we observed empirically that the performance of RANSAC-GR has a high variance, posing a challenge for the fine registration step, which requires a sufficient number of corresponding features between rendered frames across the individual maps to compute a solution. Moreover, ICP and its variants tend to converge to a local optimum, close to the solution used for initialization. As a result, these methods generally fail to provide a sufficiently good initialization for the fine registration procedure. The coarse registration step in \algname relies significantly on the semantics extracted from the map to overcome these limitations, leveraging the inherent semantics to register corresponding objects at a sufficient level of accuracy for fine registration.


\begin{table*}[th]
	\centering
	\caption{Geometric Performance: Ablation of the Coarse Gaussian-to-Gaussian and Fine Photometric Registration in \algname.}
	\label{tab:baseline_geometric_performance_metrics_ablation}
	\begin{adjustbox}{width=\linewidth}
		{\begin{tabular}{l | c c c c | c c c c | c c c c}
				\toprule
                    & \multicolumn{4}{c |}{\emph{Playroom}} & \multicolumn{4}{c |}{\emph{Truck}} & \multicolumn{4}{c}{\emph{Room}} \\
				Methods & RE $\downarrow$ & TE  $\downarrow$ & SE $\downarrow$ & CT $\downarrow$ & RE $\downarrow$ & TE $\downarrow$ & SE $\downarrow$ & CT $\downarrow$ & RE $\downarrow$ & TE $\downarrow$ & SE $\downarrow$ & CT $\downarrow$ \\
				\midrule
                    \algname-CNR & 22.72 & {454.2} & {482.1} & \cellcolor{WildStrawberry!40}20.20 & {49.98} & {355.4} & 55.06 & \cellcolor{WildStrawberry!40}24.17 & {20.50} & {474.0} & {371.9} & \cellcolor{WildStrawberry!40}17.27 \\
                    \algname-CR & 21.15 & {324.2} & {51.94} & 20.47 & {0.804} & {7.691} & 7.744 & 26.32 & {24.07} & {381.8} & {155.1} & 17.58 \\
                    \algname-NR & \cellcolor{Goldenrod!40}0.348 & \cellcolor{Goldenrod!40}{4.860} & \cellcolor{Goldenrod!40}{0.282} & 41.16 & \cellcolor{Goldenrod!40}{0.511} & \cellcolor{Goldenrod!40}{8.07} & \cellcolor{Goldenrod!40}9.581 & 53.42 & \cellcolor{Goldenrod!40}{0.381} & \cellcolor{WildStrawberry!40}{2.648} & \cellcolor{WildStrawberry!40}{1.016} & 40.24 \\
                    \algname-R & \cellcolor{WildStrawberry!40}{0.170} & \cellcolor{WildStrawberry!40}{1.933} & \cellcolor{WildStrawberry!40}{0.170} & 39.73 & \cellcolor{WildStrawberry!40}{0.413} & \cellcolor{WildStrawberry!40}{6.845} & \cellcolor{WildStrawberry!40}{2.548} & 52.47 & \cellcolor{WildStrawberry!40}{0.237} & \cellcolor{Goldenrod!40}{3.289} & \cellcolor{Goldenrod!40}{2.673} & 39.71 \\
				\bottomrule
		\end{tabular}}
	\end{adjustbox}
\end{table*}


\begin{table*}[th]
	\centering
	\caption{Photometric Performance: Ablation of the Coarse Gaussian-to-Gaussian and Fine Photometric Registration in \algname.}
	\label{tab:baseline_photometric_performance_metrics_ablation}
	\begin{adjustbox}{width=\linewidth}
		{\begin{tabular}{l | c c c | c c c | c c c }
				\toprule
                    & \multicolumn{3}{c |}{\emph{Playroom}} & \multicolumn{3}{c |}{\emph{Truck}} & \multicolumn{3}{c}{\emph{Room}} \\
				Methods & PSNR $\uparrow$ & SSIM  $\uparrow$ & LPIPS  $\downarrow$ & PSNR $\uparrow$ & SSIM  $\uparrow$ & LPIPS  $\downarrow$ & PSNR $\uparrow$ & SSIM  $\uparrow$ & LPIPS  $\downarrow$ \\
				\midrule
                    \algname-CNR & {12.0 $\pm$ 3.2} & {0.67 $\pm$ 0.11} & {0.65 $\pm$ 0.17} & 11.4 $\pm$ 2.4 & 0.52 $\pm$ 0.11 & 0.63 $\pm$ 0.22 & 6.9 $\pm$ 5.9 & {0.72 $\pm$ 0.11} & {0.39 $\pm$ 0.15} \\
                    \algname-CR & {13.7 $\pm$ 3.4} & {0.65 $\pm$ 0.14} & {0.58 $\pm$ 0.19} & \cellcolor{Goldenrod!40}16.2 $\pm$ 2.2 & \cellcolor{Goldenrod!40}0.56 $\pm$ 0.14 & \cellcolor{WildStrawberry!40}0.31 $\pm$ 0.07 & 11.7 $\pm$ 1.9 & {0.48 $\pm$ 0.12} & {0.64 $\pm$ 0.10} \\
                    \algname-NR & \cellcolor{Goldenrod!40}{26.3 $\pm$ 3.1} & \cellcolor{Goldenrod!40}{0.87 $\pm$ 0.05} & \cellcolor{Goldenrod!40}{0.17 $\pm$ 0.06} & 15.4 $\pm$ 1.7 & 0.52 $\pm$ 0.12 & 0.35 $\pm$ 0.05 & \cellcolor{WildStrawberry!40}24.8 $\pm$ 3.3 & \cellcolor{WildStrawberry!40}{0.83 $\pm$ 0.04} & \cellcolor{WildStrawberry!40}{0.22 $\pm$ 0.06} \\
                    \algname-R & \cellcolor{WildStrawberry!40}{28.3 $\pm$ 2.9} & \cellcolor{WildStrawberry!40}{0.90 $\pm$ 0.04} & \cellcolor{WildStrawberry!40}{0.15 $\pm$ 0.06} & \cellcolor{WildStrawberry!40}{16.4 $\pm$ 2.4} & \cellcolor{WildStrawberry!40}0.57 $\pm$ 0.13 & \cellcolor{WildStrawberry!40}{0.31 $\pm$ 0.07} & \cellcolor{Goldenrod!40}{24.1 $\pm$ 3.1} & \cellcolor{Goldenrod!40}{0.82 $\pm$ 0.05} & \cellcolor{Goldenrod!40}{0.23 $\pm$ 0.06} \\
				\bottomrule
		\end{tabular}}
	\end{adjustbox}
\end{table*}



\subsection{Finetuning}
\label{ssec:finetuning}
\algname does not pre-process the local GSplat maps before registration of the maps, resulting in the retention of floaters in the fused map whenever floaters exist in the local maps. Here, we examine finetuning the fused map with rendered images from the local maps to remove visual artifacts, without requiring access to the data used in the training the local GSplat maps, i.e., we do not require access to the real-world camera images and poses. To finetune the fused map without access to the original dataset, we select camera poses expressed in the local frames of the local GSplat maps (e.g., randomly or via an informed approach) and render images from these maps at these camera poses. Subsequently, we transform the set of camera poses from their associated local frames to the frame of the fused map using the transformation parameters computed by \algname. We construct a finetuning dataset from the set of images and associated camera poses, which we use in finetuning the fused map.

In \Cref{tab:finetuning_performance_metrics}, we provide the photometric scores of the fused GSplat map from \algname-R before and after finetuning and the ground-truth GSplat map. We train the ground-truth GSplat map using the combined training datasets used in training the local GSplat maps (i.e., the real-world camera images and poses, not the set of rendered images generated from the local GSplat maps), representing the ideal composite GSplat model. The computation time in \Cref{tab:finetuning_performance_metrics} represents the total training time for the ground-truth map and the total time used in finetuning the fused map. \Cref{tab:finetuning_performance_metrics} indicates that finetuning the fused map improves the PSNR, SSIM, and LPIPS scores compared to that of the pre-finetuned map. Specifically, in less than $90$ seconds, finetuning reduces the gap between the photometric scores of the ground-truth map and the photometric scores of the fused map by about $20\%$ to $40\%$. The relative improvements provided by finetuning the fused map depend on the finetuning data used, an area for future research. 
We provide rendered images from the fused GSplat map computed by \algname-R, before and after finetuning, and the corresponding images in the ground-truth fused map in \Cref{fig:photometric_performance_rendered_images_finetuning}. Across all three scenes, finetuning the fused map removes floaters and other artifacts, e.g., in the regions indicated by the green squares, ultimately resulting in higher PSNR and SSIM scores, as reported in \Cref{tab:finetuning_performance_metrics}.


\begin{table*}[th]
	\centering
	\caption{Photometric performance after finetuning \algname-R.}
	\label{tab:finetuning_performance_metrics}
	\begin{adjustbox}{width=\linewidth}
		{\begin{tabular}{l | c c c c | c c c c | c c c c }
				\toprule
                    & \multicolumn{4}{c |}{\emph{Playroom}} & \multicolumn{4}{c |}{\emph{Truck}} & \multicolumn{4}{c}{\emph{Room}} \\
				Methods & PSNR $\uparrow$ & SSIM  $\uparrow$ & LPIPS $\downarrow$ & CT $\downarrow$ & PSNR $\uparrow$ & SSIM  $\uparrow$ & LPIPS  $\downarrow$ & CT $\downarrow$ & PSNR $\uparrow$ & SSIM  $\uparrow$ & LPIPS  $\downarrow$ & CT $\downarrow$ \\
				\midrule
                    Ground-Truth & 36.3 $\pm$ 3.5 & 0.96 $\pm$ 0.03 & 0.09 $\pm$ 0.05 & 721.1 & 26.4 $\pm$ 1.4 & 0.89 $\pm$ 0.02 & 0.10 $\pm$ 0.01 & 601.7 & 34.1 $\pm$ 1.7 & 0.94 $\pm$ 0.02 & 0.12 $\pm$ 0.04 & 840.1 \\
                    Pre-Finetuning & 29.1 $\pm$ 3.3 & 0.91 $\pm$ 0.04 & 0.15 $\pm$ 0.06 & N/A & 16.8 $\pm$ 2.5 & 0.61 $\pm$ 0.10 & 0.30 $\pm$ 0.07 & N/A & 22.5 $\pm$ 2.5 & 0.79 $\pm$ 0.05 & 0.26 $\pm$ 0.06 & N/A \\
                    Post-Finetuning & 30.8 $\pm$ 2.6 & 0.92 $\pm$ 0.04 & 0.14 $\pm$ 0.06 & 72.69 & 21.1 $\pm$ 1.8 & 0.69 $\pm$ 0.1 & 0.23 $\pm$ 0.04 & 86.26 & 26.0 $\pm$ 3.6 & 0.83 $\pm$ 0.09 & 0.22 $\pm$ 0.08 & 79.78 \\
				\bottomrule
		\end{tabular}}
	\end{adjustbox}
\end{table*}


\begin{figure*}[th]
    \centering
    \includegraphics[width=\linewidth]{figures/experiments/finetuning/rendered_images_finetuning.pdf}
    \caption{Rendered images from the fused GSplat maps generated by \algname-R before and after finetuning, in the \emph{Playroom}, \emph{Truck}, and \emph{Room} scenes. Finetuning improves the visual fidelity of the fused map, removing floaters and other artifacts.}
    \label{fig:photometric_performance_rendered_images_finetuning}
\end{figure*}

