\documentclass{article}

\title{The late-stage training dynamics of (stochastic) subgradient descent on homogeneous neural networks}

\usepackage{authblk}

\author[1]{Sholom Schechtman}
\author[2]{Nicolas Schreuder}
\affil[1]{SAMOVAR, Télécom SudParis, Institut Polytechnique de Paris}
\affil[2]{CNRS, LIGM}

\usepackage{times}

\usepackage{a4wide}

\usepackage{amsmath, amsthm, amssymb}

\usepackage{shortcuts}

\usepackage{mathtools}
\usepackage{natbib}
\bibliographystyle{plainnat}
\setcitestyle{authoryear}

\usepackage{url}
\usepackage[colorlinks=true, linkcolor=blue, citecolor=blue]{hyperref}


\DeclareMathOperator{\Jac}{Jac}
\newcommand{\tgamma}{\tilde{\gamma}}
\newcommand{\tv}{\tilde{v}}
\newcommand{\bg}{\bar{g}}
\newcommand{\bgamma}{\bar{\gamma}}
\DeclareMathOperator{\ReLU}{ReLU}
\DeclareMathOperator{\LeakyReLU}{LeakyReLU}
\newtheorem{assumption}{Assumption}
\usepackage{color}
\usepackage{csquotes}

\usepackage{comment}
\usepackage{enumitem}

\newtheorem{definition}{Definition}
\newtheorem{proposition}{Proposition}
\newtheorem{theorem}{Theorem}
\newtheorem{lemma}{Lemma}
\newtheorem{remark}{Remark}



\begin{document}

\maketitle

\begin{abstract}%
  We analyze the implicit bias of constant step stochastic subgradient descent (SGD). We consider the setting of binary classification with homogeneous neural networks -- a large class of deep neural networks with $\ReLU$-type activation functions such as MLPs and CNNs without biases. We interpret the dynamics of normalized SGD iterates as an Euler-like discretization of a conservative field flow that is naturally associated to the normalized classification margin. Owing to this interpretation, we show that normalized SGD iterates converge to the set of critical points of the normalized margin at late-stage training (i.e., assuming that the data is correctly classified with positive normalized margin). 
  Up to our knowledge, this is the first extension of the analysis of \cite{Lyu_Li_maxmargin} on the discrete dynamics of gradient descent to the nonsmooth and stochastic setting. Our main result applies to binary classification with exponential or logistic losses. We additionally discuss extensions to more general settings. 
\end{abstract}


\section{Introduction}
\section{Introduction}
\label{sec:intro}

\begin{figure*}[tb]
    \centering
    \includegraphics[width=0.848\linewidth]{figs/circuitnn.pdf} 
    \caption{Illustration of differentiable CircuitNN. CircuitNN is designed based on differentiable NAND gates. After DAS is guided by PI and PO pairs of the truth table, CircuitNN can get the precise circuit architecture logic equivalent to the truth table.}
    \label{fig:circuitnn}
\end{figure*}

% 1. Describe the importance of logic synthesis
% 2. Existing Problems
% (a) Neural Architecture Search: Unstable, Predefined Setting, etc.
% (b) Circuit Generation: Probabilistic Model, Logic Equivalence

With the rapid advancement of technology, the scale of integrated circuits (ICs) has expanded exponentially. 
This expansion has introduced significant challenges in chip manufacturing, particularly concerning power and area metrics.
A primary objective in IC design is achieving the same circuit function with fewer transistors, thereby reducing power usage and area occupancy.

Logic synthesis~\cite{hachtel2005logicsynth}, a critical step in electronic design automation (EDA), transforms behavioral-level circuit designs into optimized gate-level circuits, ultimately yielding the final IC layout. 
The primary goal of logic synthesis is to identify the physical implementation with the fewest gates for a given circuit function. 
This task constitutes a challenging NP-hard combinatorial optimization problem. 
Current logic synthesis tools~\cite{brayton2010abc, wolf2013yosys} rely on human-designed heuristics, often leading to sub-optimal outcomes.

Differentiable architecture search (DAS) techniques~\cite{liu2018darts, chu2020darts} offer novel perspectives on addressing challenges in this problem.
Circuit functions can be represented through truth tables, which map binary inputs to their corresponding outputs. 
Truth tables provide a precise representation of input-output relationships, ensuring the design of functionally equivalent circuits.
Inspired by this, researchers~\cite{deepmind2024ai4sys, wang2024tnet} have begun exploring the application of DAS to synthesize circuits directly from truth tables.
Specifically, \citet{deepmind2024ai4sys} proposed CircuitNN, a framework that learns differentiable connection structures with logic gates, enabling the automatic generation of logic circuits from truth tables.
This approach significantly reduces the complexity of traditional circuit generation. 
Building on this, \citet{wang2024tnet} introduced T-Net, a triangle-shaped variant of CircuitNN, incorporating regularization techniques to enhance the efficiency of DAS.

Despite these advancements, several challenges remain. 
The computational complexity of DAS grows quadratically with the number of gates, posing scalability issues.
Although triangle-shaped architecture~\cite{wang2024tnet} partially mitigates this problem, redundancy persists. 
%Additionally, DAS is susceptible to converging to local optima, limiting the ability to search architectures that satisfy the given truth tables~\cite{liu2018darts}. 
%Furthermore, hyperparameters (network depth and layer width) require extensive searches, introducing complexity and prolonging the synthesis process. 
Additionally, DAS is susceptible to converging to local optima~\cite{liu2018darts} and hyperparameters (network depth and layer width) require extensive searches. 
The challenges arise from the vast search space in DAS. 
% Even with predefined settings for CircuitNN, finding a configuration that meets the truth table requires extensive trial and error during the DAS process. 
Intuitively, limiting the search space through predefined parameters (network depth, gates per layer, and connection probabilities) can significantly reduce the complexity.

Recent advances~\cite{openai2023gpt4, abramson2024alphafold3, esser2024sd3, li2024mar} in conditional generative models have demonstrated remarkable performance across language, vision, and graph generation tasks. 
Motivated by these developments, we propose a novel approach to circuit generation that generates preliminary circuit structures to guide DAS in generating refined circuits matching specified truth tables. 
Firstly, we introduce CircuitVQ, a tokenizer with a discrete codebook for circuit tokenization. 
Built upon our Circuit AutoEncoder framework~\cite{hou2022graphmae,li2023maskgae,wu2025mgvga}, CircuitVQ is trained through a circuit reconstruction task. 
Specifically, the CircuitVQ encoder encodes input circuits into discrete tokens using a learnable codebook, while the decoder reconstructs the circuit adjacency matrix based on these tokens.
Subsequently, the CircuitVQ encoder serves as a circuit tokenizer for CircuitAR pretraining, which employs a masked autoregressive modeling paradigm~\cite{chang2022maskgit, li2023mage}. 
In this process, the discrete codes function as supervision signals. 
After training, CircuitAR can generate discrete tokens progressively, which can be decoded into initial circuit structures by the decoder of the CircuitVQ. 
These prior insights can guide DAS in producing refined circuits that match the target truth tables precisely.

Our key contributions can be summarized as follows:
\begin{itemize}
\item We introduce CircuitVQ, a circuit tokenizer that facilitates graph autoregressive modeling for circuit generation, based on our Circuit AutoEncoder framework;
\item Develop CircuitAR, a model trained using masked autoregressive modeling, which generates initial circuit structures conditioned on given truth tables;
\item Propose a refinement framework that integrates differentiable architecture search to produce functionally equivalent circuits guided by target truth tables;
\item Comprehensive experiments demonstrating the scalability and capability emergence of our CircuitAR and the superior performance of the proposed circuit generation approach.
\end{itemize}

% Motivation
% (a) Diffusion (Vision, Graph), Autoregressive (Language, Vision)
% (b) Circuit Generation for Predefined Setting
% (c) Neural Architecture Search for Strict Logic Equivalence

% Contribution
% (a) Circuit Tokenizer (new transformer arch, training strategy)
% (b) CircuitAR (train and gen strategies, post-ar strategy)
% (c) Extensive Evaluation including BitD (Bit Distance) for Scalability


\section{Related works}\label{sec:rw}

\section{Related Work} \label{sec:related}

% \textbf{Adversarial Attack}
\textbf{Attacks on SLAM.} 
%With the rise of machine learning, 
The robustness of computer vision systems is being actively investigated. With the emergence of adversarial images in the digital domain by adding optimized noise directly to images~\cite{szegedy2013intriguing,carlini2017towards}, researchers find that such attacks also exist physically in the real world \cite{eykholt2018robust,song2018physical,zhao2019seeing}. To fill the gap between attacks in the digital and physical worlds, recent studies have demonstrated that attacks on real-world computer vision systems are practical \cite{eykholt2018robust,li2019adversarial,man2020ghostimage,sharif2016accessorize,zhao2019seeing,zhou2018invisible}. However, attacks on traditional computer vision methods such as SLAM are relatively less explored. \cite{yoshida2022adversarial} proposes an attack against the scan matching algorithm in LiDAR-based SLAM, while most SLAMs in AR/VR devices rely on different sensors like RGB/depth cameras and IMUs. \cite{ikram2022perceptual} and \cite{chen2024adversary} mislead visual SLAM by poisoning the images with special patterns, and \cite{wang2021can} causes the camera to fail using infrared light. In our work, we demonstrate attacks on Visual-Inertial SLAM (VI-SLAM) by perturbing the IMU readings, rather than cameras, and showing its impact on XR user experience. 

\textbf{Acoustic Injection Attacks.} Among various physical attacks, acoustic injection attacks are attractive due to their low cost. Son~\etal~\cite{son2015rocking} were the first to introduce acoustic attacks on MEMS gyroscopes, demonstrating how these attacks could lead to sensor denial-of-service and result in drone crashes. WALNUT~\cite{trippel2017walnut} expanded on this by developing output biasing and control attacks that enable precise manipulation of MEMS accelerometer outputs using modulated sound waves. Wang et al.~\cite{wang2017sonic} demonstrated a sonic gun, showcasing the vulnerability of various smart devices (\eg drones and self-balancing vehicles) to acoustic attacks. Tu et al. \cite{tu2018injected} designed side-swing and switching attacks to alter the outputs of MEMS gyroscopes and accelerometers. Furthermore, Ji et al. \cite{ji2021poltergeist} fool the object detectors by applying acoustic attack to the image stabilizers commonly used in modern cameras. However, none of the existing works study the relationship between the acoustic injections and SLAM outputs on recent XR devices. 

% \zijian{Do we need one session about security in AR/VR?}
% \yicheng{TODO}
%\jiasi{cite the AIVR paper (UMass Amherst?) paper is we have not already. They add IMU perturbation but w/o SLAM, iirc} \yicheng{Cited}

\textbf{XR Security and Privacy.} 
%Security and privacy concerns in XR systems have gained significant attention. 
For single-user XR systems, researchers have demonstrated various side-channel attacks to extract sensitive information (\eg keystrokes) through video feeds~\cite{ling2019know}, head movements~\cite{nair2023unique, slocum2023going}, architectural hints~\cite{zhang2023its,shang2020arspy}, power usage~\cite{li2024dangers}, and EM side-channel leakages~\cite{al2021vr}. In multi-user XR systems, Su et al.~\cite{su2024remote} use avatar motion data to infer keystrokes in shared VR environments. Slocum et al.~\cite{slocum2024doesn} reveal vulnerabilities in the shared state frameworks of multi-user AR. Similarly, Lebeck et al.~\cite{lebeck2017securing} highlight risks like deceptive virtual objects and emphasize access control for managing shared physical and virtual spaces. Ruth et al.~\cite{ruth2019secure} further propose a secure multi-user AR framework focusing on content sharing and permissions.
Chandio et al.~\cite{chandio2024stealthy} %introduced a multi-modal spatiotemporal attack that 
simultaneously manipulated visual and inertial sensors to disrupt XR pose estimation. However, their study evaluated the attack using offline datasets and assumed the attacker's capability to manipulate IMU data streams through acoustic means, without real experiments. Ours is the first to demonstrate acoustic injection attacks on recent XR devices, like the Hololens 2, in the real world.
 



\section{Preliminaries}\label{sec:preli}
\section{Preliminaries and Motivation}
\label{sec:prelim}



\subsection{LLM Unlearning}

LLM unlearning refers to techniques that selectively remove specific behaviors or knowledge from a pre-trained language model while maintaining its overall functionality~\cite{yao2023large}. 
With the proliferation of LLMs, unlearning has gained significant attention due to its broad applications in safety alignment, privacy protection, and copyright compliance~\cite{eldan2023s,liu2024rethinking,jia-etal-2024-soul}. The evaluation and auditing of LLM unlearning spans from basic verbatim memorization to deeper knowledge memorization~\cite{shi2024muse}, with this work focusing on the latter.
As depicted in \autoref{fig:overalltask}, LLM unlearning operates as a targeted intervention within the model's knowledge representation framework. 
Its core objective is the selective removal of specific information while preserving the model's broader knowledge base (e.g, on retain set). 
This study focuses on the knowledge unlearning auditing that assesses unlearned models' behaviors through comprehensive audit cases. Given access to both forget and retain corpora, we generate a holistic set of test questions with reference answers to thoroughly evaluate whether an unlearned model exhibits any residual knowledge memorization.
% The effectiveness of unlearning is characterized by a distinctive performance pattern: the model should exhibit significantly reduced performance on tasks involving the targeted knowledge while maintaining competence across all other domains. This selective degradation in performance serves as a key indicator of successful unlearning.

% This paper focuses on auditing LLM knowledge une, specifically developing fine-grained test cases based on training data to identify instances where the unlearning process has failed to achieve its intended objectives.

% Formally, given a pre-trained language model $\mathcal{M}$ with parameters $\theta$, the unlearning process aims to derive a new model $\mathcal{M}'$ with parameters $\theta'$ that demonstrates forgetting of targeted information while preserving other capabilities.
% In the context of exact knowledge unlearning, we specifically focus on removing precise factual information or relationships from the model's knowledge base. Let $\mathcal{D}_{fgt}$ denote the forget dataset containing knowledge to be eliminated, and $\mathcal{D}_{ret}$ represent the retain dataset containing knowledge that should be preserved. The objective of exact knowledge unlearning can be formalized as:

% \begin{equation}
% \theta' = \argmin_{\theta'} \mathcal{L}_{fgt}(\mathcal{M}'(\theta'), \mathcal{D}_{fgt}) + \lambda \mathcal{L}_{ret}(\mathcal{M}'(\theta'), \mathcal{D}_{ret})
% \end{equation}

% where $\mathcal{L}_{fgt}$ measures the model's tendency to generate or recall information from $\mathcal{D}_{fgt}$, $\mathcal{L}_{ret}$ evaluates the preservation of knowledge in $\mathcal{D}_{ret}$, and $\lambda$ balances these competing objectives. A successfully unlearned model should satisfy:

% \begin{equation}
% P(\mathcal{M}'(\theta') \mid \mathcal{D}_{fgt}) \ll P(\mathcal{M}(\theta) \mid \mathcal{D}_{fgt})
% \end{equation}

% \begin{equation}
% P(\mathcal{M}'(\theta') \mid \mathcal{D}_{ret}) \approx P(\mathcal{M}(\theta) \mid \mathcal{D}_{ret})
% \end{equation}

% where $P(\mathcal{M} \mid \mathcal{D})$ represents the model's probability of generating or correctly responding to information in dataset $\mathcal{D}$. These conditions ensure that the unlearned model exhibits significantly reduced ability to recall forgotten knowledge while maintaining its performance on retained knowledge.



\subsection{Knowledge Graph}
\label{sec:pre_kg}

A knowledge graph (KG) is a structured multi-relational graph~\cite{bordes2013translating}, usually representing a collection of facts as a network of entities and the relationships between entities.
Formally, a KG \(\mathcal{G} = \langle \mathcal{E}, \mathcal{R}, \mathcal{F} \rangle\) could be considered a directed edge-labeled graph~\cite{ji2021survey}, which comprises a set \(\mathcal{E}\) of entities (e.g., \textit{Harry Potter}, \textit{Hogwarts School}), a set \(\mathcal{R}\) of relations (e.g., \textit{attends}), and a set $\mathcal{F}$ of facts. A fact is a triple containing the head entity \(e_1 \in \mathcal{E}\), the relation $r \in \mathcal{R}$, and the tail entity \(e_2 \in \mathcal{E}\) to show that there exists the relation from the tail entity to the head entity, denoted as \((e_1, r, e_2) \in \mathcal{F}\)~\cite{hogan2021knowledge}. To illustrate, the fact (\textit{Harry Potter}, \textit{attends}, \textit{Hogwarts School}) shows that there exists the \textit{attends} relation between \textit{Harry Potter} and \textit{Hogwarts School}, which indicates``Harry Potter attends Hogwarts School''.

\begin{figure*}[t]
  \centering
  \includegraphics[width=0.9\linewidth]{fig/fig_overview_fig.pdf}
  \caption{The diagram illustrates the MHA, MLA, and our MHA2MLA. It can be seen that the ``cached'' part is fully aligned with MLA after MHA2MLA. The input to the attention module is also completely aligned with MLA (the \colorbox{lightgray!60}{aligned region below}). Meanwhile, the parameters in MHA2MLA maximize the use of pre-trained parameters from MHA (the \colorbox{lightgray!60}{aligned region above}).}
  \vspace{-0.4cm}
  % https://1drv.ms/p/c/e248d5c415e14c2d/ETjEatb4CaVLpZfGpurOYKsB41mBrk6_He7OiQfC5mh_Vg?e=DJMN0j
  \label{fig:overview}
\end{figure*}

% MHA、MLA以及我们的MHA2MLA的示意图。可以看到,Cached部分在MHA2MLA后完全对齐MLA。注意力模块的输入也完全对齐了MLA(下方的对齐区域)。而参数部分最大化利用了MHA中预训练的参数(上方的对齐区域)。

\subsection{Motivation}
This section aims to illustrate why and how we consider employing KG to facilitate the holistic LLM unlearning audit. 
Two critical factors underpin this task.
\ding{182}\textbf{Audit Adequacy}: The Forget Dataset is an extensive, unstructured corpus. Existing approaches typically rely on the LLM's prior knowledge to directly generate QA pairs or segment the corpus and feed these segments to ChatGPT for automated QA pair generation. Such methods often fail to intuitively reflect or guarantee the sufficiency of the generate dataset.
\ding{183}\textbf{Knowledge Redundancy}: A more subtle and easily overlooked issue is that the Retain Dataset and Forget Dataset may contain overlapping knowledge. As illustrated in \autoref{fig:overalltask}, this overlapping knowledge should be retained by the unlearned model and, therefore not be treated as candidates for the unlearning efficacy audit. Existing evaluation benchmarks like MUSE often neglect this aspect, as evidenced by \autoref{fig:musecase}.



A KG can offer an effective solution to address these two challenges. 
First, the KG inherently captures the knowledge facts within the Forget Dataset at a fine-grained level, with each edge representing a minimal testable unit. 
By ensuring coverage of every edge in the KG, one can achieve a more intuitive and relatively comprehensive audit. 
Moreover, the structured data provided by the KG can facilitate the identification of identical knowledge facts present in both the Retain and Forget Datasets.
This capability allows for refinement of the initial forget knowledge graph by removing potentially retained information.
Finally, owing to recent advances in KG extraction technology, numerous automated extraction models and pipelines are available to support the automated construction of an audit dataset.



\section{Problem setting}\label{sec:sett}
We study (stochastic) gradient descent on the empirical risk
\begin{equation*}
\cL(w) = \frac{1}{n}\sum_{i=1}^n l(p_i(w))\, ,
\end{equation*}
where the loss function $l$ and the functions  $(p_i)_{i=1}^n$  are specified in the following assumptions. Note that the empirical risk for binary classification from Equation~\eqref{def:emp_risk_intro} is a special case of the above objective.

\begin{assumption}\label{hyp:loss_exp_log}\phantom{=}
  \begin{enumerate}[label=\roman*)]
    \item The loss is either the exponential loss, $l(q) = e^{-q}$, or the logistic loss, $l(q) = \log(1{+}e^{-q})$.
    \item There exists an integer $L \in \mathbb{N}^*$  such that, for all $1 \leq i \leq n$, the function $p_i$ is $L$-homogeneous\footnote{We recall that a mapping $f : \mathbb{R}^d \rightarrow \mathbb{R}$ is positively $L$-homogeneous if $f(\lambda w) = \lambda^L f(w)$ for all $w \in \mathbb{R}^d$ and $\lambda >0$.}, locally Lipschitz continuous and semialgebraic.
  \end{enumerate}
\end{assumption}
If the $p_i$'s were differentiable with respect to $w$, the chain rule would guarantee that
\begin{align*}
\nabla \mathcal{L}(w) = \frac{1}{n}\sum_{i=1}^n  l'(p_i(w)) \nabla p_i(w)\enspace.
\end{align*}
However, we only assume that the $p_i$'s are semialgebraic. While we could consider Clarke subgradients, the Clarke subgradient of operations on functions (e.g., addition, composition, and minimum) is only contained within the composition of the respective Clarke subgradients. This, as noted in Section~\ref{sec:cons_field}, implies that the output of backpropagation is usually not an element of a Clarke subgradient but a selection of some conservative set-valued field.
Consequently, for $1\leq i \leq n$, we consider $D_i : \bbR^d \rightrightarrows\bbR^d$, a conservative set-valued field of $p_i$, and a function $\sa_i : \bbR^d \rightarrow \bbR^d$ such that for all $w \in \bbR^d$, $\sa_i(w) \in D_i(w)$. Given a step-size $\gamma >0$, gradient descent (GD)\footnote{More precisely, this refers to conservative gradient descent. We use the term GD for simplicity, as conservative gradients behave similarly to standard gradients.} is then expressed as
\begin{equation*}\label{eq:gd_new}\tag{GD}
  w_{k+1} = w_k - \frac{\gamma}{n} \sum_{i=1}^n l'(p_i(w_k))\sa_i(w_k)\,.
\end{equation*}
For its stochastic counterpart, stochastic gradient descent (SGD), we fix a batch-size $1\leq n_b \leq n$. At each iteration $k \in \bbN$, we randomly and uniformly draw a batch $B_k \subset \{1, \ldots, n \}$ of size $n_b$. The update rule is then given by 
\begin{equation*}\label{eq:sgd_new}\tag{SGD}
  w_{k+1} = w_k -  \frac{\gamma}{n_b}\sum_{i\in B_k} l'(p_i(w_k)) \sa_i(w_k)\, .
\end{equation*}
The considered conservative set-valued fields will satisfy an Euler lemma-type assumption.
%\nic{Smoother transition}
\begin{assumption}\phantom{=}\label{hyp:conserv}
  For every $i \leq n$, $\sa_i$ is measurable and $D_i$ is semialgebraic. Moreover, for every $w \in \bbR^d$ and $\lambda \geq 0$, $\sa_i(w)  \in D_i(w)$,
  \begin{equation*}
    D_i(\lambda w) = \lambda^{L-1} D_i(w)\, , \textrm{ and } \quad   L p_i(w) = \scalarp{\sa_i(w)}{w}\, .
  \end{equation*}
\end{assumption}
%\nic{Smoother transition}
Having in mind the binary classification setting, in which $p_i(w) = y_i \Phi(x_i, w)$, we define the margin
\begin{equation}\label{def:marg}
  \sm: \bbR^d \rightarrow \bbR, \quad \sm(w) = \min_{1\leq i \leq n} p_i(w)\, .
\end{equation}
It quantifies the quality of a prediction rule $\Phi(\cdot, w)$. In particular,  the training data is perfectly separated when $\sm(w) >0$. A binary prediction for $x$ is given by the sign of $\Phi(x, w)$, and under the homogeneity assumption, it depends only on the normalized direction $w / \norm{w}$. Consequently, we will focus on the sequence of directions $u_k := w_k / \norm{w_k}$. Our final assumption ensures that the normalized directions $(u_k)$ have stabilized in a region where the training data is correctly classified.

\begin{assumption}\label{hyp:marg_lowb}
  Almost surely, $\liminf \sm(u_k) >0$.
\end{assumption}
Before presenting our main result, we comment on our assumptions.

\paragraph{On Assumption~\ref{hyp:loss_exp_log}.} As discussed in the introduction, the primary example we consider is when $p_i(w) = y_i \Phi(x_i;w)$ is the signed prediction of a feedforward neural network without biases and with piecewise linear activation functions on a labeled dataset $((x_i,y_i))_{i \leq n}$. In this case,
\begin{equation}\label{eq:NN}
 p_i(w) = y_i \Phi(w;x_i) = y_i V_L(W_L) \sigma(V_{L-1}(W_{L-1}) \sigma(V_{L-1}(W_{L-2}) \ldots \sigma(V_{1}(W_1 x_i))))\, ,
\end{equation}
where $w = [W_1, \ldots, W_L]$, $W_i$ represents the weights of the $i$-th layer, $V_i$ is a linear function in the space of matrices (with $V_i$ being the identity for fully-connected layers) and $\sigma$ is a coordinate-wise activation function such as $z \mapsto \max(0,z)$ ($\ReLU$), $z \mapsto \max(az, z)$ for a small parameter $a>0$ (LeakyReLu) or $z \mapsto z$. Note that the mapping $w \mapsto p_i(w)$ is semialgebraic and $L$-homogeneous for any of these activation functions. Regarding the loss functions, the logistic and exponential losses are among the most commonly studied and widely used. In Appendix~\ref{app:gen_sett}, we extend our results to a broader class of losses, including $l(q) = e^{-q^a}$ and $l(q) = \ln (1 + e^{-q^a})$ for any $a \geq 1$.

\paragraph{On Assumption~\ref{hyp:conserv}.} Assumption~\ref{hyp:conserv} holds automatically  if $D_i$ is the Clarke subgradient of $p_i$. Indeed, at any vector $w \in \bbR^d$, where $p_i$ is differentiable it holds that $p_i(\lambda w) = \lambda^{L} p_i(w)$. Differentiating relatively to $w$ and $\lambda$ (noting that $p_i$ remains differentiable at $\lambda w$ due to homogeneity), we obtain $\lambda \nabla p_i(\lambda w) = \lambda^{L} \nabla p_i(w)$ and $\scalarp{\nabla p_i(\lambda w)}{w} = L \lambda^{L-1} p_i(w)$. The expression for any element of the Clarke subgradient then follows from~\eqref{eq:def_clarke}. 

However, for an arbitrary conservative set-valued field, Assumption~\ref{hyp:conserv} does not necessarily hold. For instance, $D(x) = \mathds{1}(x \in \mathbb{N})$ is a conservative set-valued field for $p \equiv 0$, which does not satisfy Assumption~\ref{hyp:conserv}. Nevertheless, in practice, conservative set-valued fields naturally arise from a formal application of the chain rule. For a non-smooth but homogeneous activation function $\sigma$, one selects an element $e \in \partial \sigma (0)$, and computes $\sa_i(w)$ via backpropagation. Whenever a gradient candidate of $\sigma$ at zero is required (i.e., in~\eqref{eq:NN}, for some $j$, $V_j(W_j)$ contains a zero entry), it is replaced by $e$. 
Since $V_j(W_j)$ and $V_j(\lambda W_j)$ have the same zero elements, it follows that for every such $w$, $
\sa_i(\lambda w) = \lambda^L \sa_i(w)$. The conservative set-valued field $D_i$ is then obtained by associating to each $w$ the set of all possible outcomes of the chain rule, with $e$ ranging over all elements of $\partial \sigma(0)$. Thus, for such fields, Assumption~\ref{hyp:conserv} holds.


\paragraph{On Assumption~\ref{hyp:marg_lowb}.} Training typically continues even after the training error reaches zero.
Assumption~\ref{hyp:marg_lowb} characterizes this late-training phase, where our result applies. 
As noted earlier, since $\sm$ is $L$-homogeneous, the classification rule is determined by the direction of the  iterates $u_k=w_k/\norm{w_k}$. Assumption~\ref{hyp:marg_lowb} then states that, beyond some iteration, the normalized margin remains positive. 
This assumption is natural in the context of studying the implicit bias of SGD: we \emph{assume} that we reached the phase in which the dataset is correctly classified and \emph{then} characterize the limit points. A similar perspective was taken in  \cite{nacson2019lexicographic}, where the implicit bias of GF was analyzed under the assumption that the sequence of directions and the loss converge. However, unlike their approach, ours does not require assuming such convergence a priori.

Earlier works such as \cite{ji2020directional,Lyu_Li_maxmargin}, which analyze subgradient flow or smooth GD, establish convergence by assuming the existence of a single iterate $w_{k_0}$ satisfying $\sm(w_{k_0}) > \varepsilon$ and then proving that $\lim \sm(u_{k}) > 0$. Their approach relies on constructing a smooth approximation of the margin, which increases during training, ensuring that $\sm(u_k) > 0$ for all iterates with $k \geq k_0$. This is feasible in their setting, as they study either subgradient flow or GD with smooth $p_i$’s, allowing them to leverage the descent lemma.

In contrast, our analysis considers a nonsmooth and stochastic setting, in which, even if an iterate $w_{k_0}$ satisfying $\sm(w_{k_0}) > \varepsilon$ exists, there is no a priori assurance that subsequent iterates remain in the region where Assumption~\ref{hyp:marg_lowb} holds. From this perspective, Assumption~\ref{hyp:marg_lowb} can be viewed as a stability assumption, ensuring that iterates continue to classify the dataset correctly. Establishing stability for stochastic and nonsmooth algorithms is notoriously hard, and only partial results in restrictive settings exist \cite{borkar2000ode,ramaswamy2017generalization,josz2024global}.

%Finally, note that Assumption~\ref{hyp:marg_lowb} only needs to hold almost surely. Specifically, with probability 1, there exist $k_0$ and $\varepsilon$ such that for all $k \geq k_0$, $\sm(u_k) \geq \varepsilon > 0$. In the case of~\eqref{eq:sgd_new}, $k_0$ and $\delta$ are random variables and may take different values across different realizations. 

%\paragraph{On constant stepsizes.}
%We allow the step size to be a constant of arbitrary magnitude, subject to the stability Assumption~\ref{hyp:marg_lowb}. This may seem surprising in a nonsmooth and stochastic setting, where a vanishing step size is typically required to ensure convergence (see, e.g., \cite{majewski2018analysis, dav-dru-kak-lee-19, bolte2023subgradient, le2024nonsmooth}).

\section{Main result}\label{sec:main}
%We begin by showing that the sequence of directions follows a dynamic that can be interpreted as a stochastic approximation of the reversed (Riemannian) conservative field flow of $\sm$ restricted to the sphere $\bbS^d$ in Proposition~\ref{prop:stoch_approx_exp_log}. This observation directly leads to our main result, Theorem~\ref{thm:main}, which follows as a consequence of recent stochastic approximation results from \cite{benaim_05_DI_1,dav-dru-kak-lee-19}.

As a first step toward our main result, we establish that the iterates norm $(\lVert w_k\rVert)$ grows to infinity at a logarithmic rate. This is consistent with \cite[Theorem 4.3]{Lyu_Li_maxmargin}.
\begin{proposition}\label{prop:log_wk}
  Under Assumptions~\ref{hyp:loss_exp_log}--\ref{hyp:marg_lowb}, almost surely, there exist $c_1, c_2, \varepsilon>0$ and $k_0 \in \bbN$, such that for all $k \geq k_0$, $\norm{w_k}$ increases and
  \begin{equation*}
    c_1 \log (k)\leq \norm{w_k}^L \leq c_2 \log(k) \quad \textrm{ and } \quad 0 < \cL(w_k) \leq k^{-\varepsilon c_1}\, .
  \end{equation*}
  In particular, $\norm{w_k}\rightarrow + \infty$ and $\cL(w_k) \rightarrow 0$.
\end{proposition}
\begin{proof}[Sketch, full proof in Appendix~\ref{sec:pf_logwk}.] The proof follows from the next observations, which hold almost surely, for $k$ large enough. \emph{(i)} There is $\varepsilon >0$, such that $\sm(u_k) \geq \varepsilon$. In particular, there are $M, C_1, C_2>0$, such that $ C_2 e^{-M \norm{w_k}^L}\leq -l'(p_i(w_k)) \leq C_1e^{-\varepsilon \norm{w_k}^L}$. \emph{(ii)} As a result, there is $C_3 >0$ such that
  \begin{equation}\label{eq:lwb_wk2}
    \norm{w_{k+1}}^2 \geq \norm{w_k}^2 ( 1 + C_3 \gamma e^{-M \norm{w_k}^L } \norm{w_k}^{L-2})\, ,
  \end{equation}
  which implies that $\norm{w_k}$ is increasing to infinity and that there is $C_4 >0$ such that
  \begin{equation}\label{eq:uwb_wk2}
    \norm{w_{k+1}}^2 \leq \norm{w_k}^2 (1 + C_4 \gamma e^{-\varepsilon \norm{w_k}^L}\norm{w_k}^{L-2})\, .
  \end{equation}\emph{(iii)} Finally, using~\eqref{eq:lwb_wk2}--\eqref{eq:uwb_wk2} and the Taylor's expansion of $(1+x)^{L/2}$ near zero, we obtain existence of constants $C_5, C_6, a>0$, such that for $k$ large enough,
  \begin{equation*}
    C_5 \gamma \leq e^{a \norm{w_k}^L}\left(\norm{w_{k+1}}^L - \norm{w_k}^L\right) \leq C_6 \gamma \, .
  \end{equation*}
Summing these inequalities from $k$ to $k+N$ and noticing that the expression in the middle is comparable to the integral of $e^{at}$ between $\norm{w_{k}}^L$ and $\norm{w_{k+N}}^L$, concludes the proof.
\end{proof}
Define the set-valued map $\bar{D} : \bbR^d \rightrightarrows \bbR^d$ as 
\begin{equation}\label{eq:avg_consfiel}
  \bar{D}(x) = \conv \{v: v \in D_i(w) \, , \textrm{with $i \in I(w)$} \}\,, \quad \textrm{  where $I(w) = \{ i: p_i(w) = \sm(w)\}$}\, .
\end{equation}
As shown in Appendix~\ref{app:conserv}, it is a conservative set-valued field for the potential $\sm$. Note that, following Remark~\ref{rmk:max_subg}, even if for all $i$, $D_i = \partial p_i$, $\bar{D}$ can be different from $\partial \sm$. Next, we define the set-valued field $\bar{D}_s: \bbS^{d-1} \rightrightarrows \bbR^d$ as
\begin{equation}\label{def:riem_cons}
\bar{D}_{s}(u) := \{ v - \scalarp{v}{u}u : v \in \bar{D}(u) \}\, .
\end{equation}
The associated set of critical points is then given by
\begin{equation}\label{def:riem_crit}
  \cZ_s := \{ u \in \bbS^{d-1} : 0 \in \bar{D}_s(u) \} \subset \bbS^{d-1}\, .
\end{equation}
The field $\bar{D}_s$ and the critical points set $\cZ_s$ admit a straightforward interpretation. If $\sm$ is $C^1$ around some point $u \in \bbS^{d-1}$ and $\bar{D} =\{ \nabla \sm\}$, then $\bar{D}_s(u)$ is the radial component of $\nabla \sm(u)$, corresponding to its projection onto the tangent plane of $\bbS^{d-1}$ at $u$.
From a Riemannian geometry perspective, this implies that $\bar{D}_s(u)$ is the Riemannian gradient\footnote{Here, the Riemannian structure is implicitly induced from the ambient space.} of $\sm$ \emph{restricted} to the sphere $\bbS^{d-1}$, $\sm_{|\bbS^{d-1}}$. Similarly, $\cZ_s$ corresponds to the set of critical points of $\sm_{|\bbS^{d-1}}$.
More generally, since conservative fields are gradient-like objects (see Proposition~\ref{prop:var_strat_cons} in Appendix~\ref{app:omin}), we interpret $\bar{D}s$ as the Riemannian conservative field of $\sm_{|\bbS^{d-1}}$, with $\cZ_s$ as its corresponding critical points\footnote{As noted in \cite[Page 4, footnote]{bolte2021conservative}, the concept of a conservative set-valued field extends naturally to functions defined on any complete Riemannian submanifold, including $\bbS^{d-1}$.}.


We will consider the differential inclusion (DI) associated to set-valued field $\bar{D}_s$,
\begin{equation*}\label{eq:DI_sphere}\tag{DI}
\dot{\su}(t) \in \bar{D}_{s}(\su(t))\,.
\end{equation*}
Under the aforementioned interpretation, this corresponds to the reversed gradient (or conservative field) flow of $\sm_{|\bbS^{d-1}}$. 

We now show that the iterates’ directions evolve according to a dynamic that approximates~\eqref{eq:DI_sphere} via an Euler-like discretization (or stochastic approximation). The proof is deferred to Section~\ref{pf:sto_app_explog}.
%To treat the stochastic setting, we denote by $\cF_k$ the sigma-algebra generated by $\{ w_0,\ldots,w_k\}$. Notice that it is a filtration ($\cF_k \subset \cF_{k+1}$ for all $k$) and recall that a sequence $(a_k)$ is said to be adapted to $(\cF_k)$ if for every $k$, $a_k$ is $\cF_k$-measurable. Finally, we recall the notation $u_k=w_k/\norm{w_k}$.
  \begin{proposition}\label{prop:stoch_approx_exp_log}
    Let Assumptions~\ref{hyp:loss_exp_log}--\ref{hyp:marg_lowb} hold. There exist sequences $(\bg_k^s), (r_k), (\bgamma_k), (\bar{\eta}_{k+1})$ such that, for both~\eqref{eq:gd_new} and~\eqref{eq:sgd_new}, the normalized direction iterates $u_k \coloneqq w_k/\lVert w_k \rVert$ satisfy
    \begin{equation}\label{eq:stoch_app_u}
      u_{k+1} = u_k + \bgamma_k\bg_k^s + \bgamma_k \bar{\eta}_{k+1} + \bgamma_k^2 r_k\, .
    \end{equation}
    Moreover, considering the filtration $(\cF_k)_k$ where, for $k \in \mathbb{N}$, $\cF_k$ the sigma-algebra generated by $\{ w_0,\ldots,w_k\}$, the following holds:
    \begin{enumerate}
      \item\label{pr_res:rk} The sequence $(r_k)$ satisfies $\sup_{k}\norm{r_k} < + \infty$ almost surely.
      \item\label{pr_res:gammak} The sequence $(\bgamma_k)$ is positive and adapted to $(\cF_k)$. Moreover, $\sum_{k} \bgamma_k = + \infty$, and, almost surely, there is $c_3>0$ such that for sufficiently large $k$, $\bgamma_k \leq k^{-c_3}$.
      \item\label{pr_res:etak} For~\eqref{eq:gd_new}, $\bar{\eta}_{k} \equiv 0$. Otherwise, the sequence $(\bar{\eta}_{k})$ is adapted to $(\cF_k)$ and satisfies 
      \begin{equation*}
      \bbE[\bar{\eta}_{k+1} |\cF_k] = 0 \,.
      \end{equation*}
      Additionally, there exists a deterministic constant $c_4>0$ such that $\sup_{k} \norm{\bar{\eta}_{k+1}} < c_4$.
      \item\label{pr_res:barD} For any unbounded sequence $(k_j)_j$, such that $u_{k_j} \to u \in \bbS^{d-1}$, $\dist(\bar{D}_s(u), \bg^s_{k_j}) \rightarrow 0$. 
    \end{enumerate}
  \end{proposition}
    Since Proposition~\ref{prop:stoch_approx_exp_log} allows us to interpret $(u_k)$ as a discretization of~\eqref{eq:DI_sphere}, it is natural to investigate the convergence properties of a solution of its continuous counterpart~\eqref{eq:DI_sphere}.
  If $\su$ is such solution, then for almost every $t \in \bbR$, there exists $v \in \bar{D}_s(\su(t))$ such that $\dot{\su}(t) = v - \scalarp{v}{u}u$. Thus, by Definition~\ref{def:cons_f}, for almost every $t$, $\frac{\dif }{\dif t}  \sm(\su(t)) = \scalarp{\dot{\su}(t)}{v} = \norm{\dot{\su}(t)}^2$. Therefore, for $T >0$, we obtain
  \begin{equation*}
    \sm(\su(T)) - \sm(\su(0)) = \int_{0}^{T} \norm{\dot{\su}(t)}^2 \dif t \, .
  \end{equation*}
  This implies that $\sm(\su(T)) \geq \sm(\su(0))$, with strict inequality whenever $\su(0) \not \in \cZ_s$. In dynamical systems terminology, $-\sm$ is a Lyapunov function for~\eqref{eq:DI_sphere}. In particular, it can be shown that any solution $\su(t)$ to~\eqref{eq:DI_sphere} converges to $\cZ_s$.

Our main result, Theorem~\ref{thm:main}, establishes that the same holds true for the sequence of normalized directions: any limit point of $(u_k)$ is contained in $\cZ_s$. As discussed below, this result generalizes \cite[Theorem 4.4]{Lyu_Li_maxmargin} to stochastic gradient descent in the nonsmooth setting.

\begin{theorem}\label{thm:main}
  Under Assumptions~\ref{hyp:loss_exp_log}--\ref{hyp:marg_lowb}, almost surely, $\sm(u_k)$ converges to a positive limit and 
  \begin{equation}\label{eq:conv_uk}
    \dist(u_k, \cZ_s) \xrightarrow[k \rightarrow + \infty]{} 0 \, .
  \end{equation}
\end{theorem}
\begin{proof}
  The proof, which is given in Appendix~\ref{pf:main_th} follows from Proposition~\ref{prop:stoch_approx_exp_log} and some minor adaptations of recent results on stochastic approximation from \cite{benaim2006dynamics,dav-dru-kak-lee-19}.
\end{proof}
A natural question is the interpretation of membership in $\cZ_s$. Given the Riemannian perspective on $\bar{D}_s$, it is unsurprising that belonging to $\cZ_s$ is a necessary optimality condition for the max-margin problem. We formally prove this result in Appendix~\ref{app:conserv}.
\begin{lemma}\label{lm:loc_max}
  If $u^*$ is a local maximum of $\sm_{|\bbS^{d-1}}$, then $0 \in \bar{D}_s(u^*)$.
\end{lemma}
Thus, Theorem~\ref{thm:main} establishes  that $(u_k)$ converge to the set of $\bar{D}_s$-critical points, which is a necessary condition of optimality for $\argmax_{u \in \bbS^{d-1}} \sm(u)$.
Comparing our result with \cite[Theorem 4.4]{Lyu_Li_maxmargin}, we note that, if each $D_i$ were equal to $\partial p_i$, then any limit point of $(u_k)$ would correspond \emph{exactly} to a scaled KKT point from \cite{Lyu_Li_maxmargin}. In this work, the authors formulate an alternative optimization problem, namely
\begin{equation*}\label{def:prob2}\tag{P}
  \min \{ \norm{w}^2 : w \in\bbR^d\, ,\sm(w) \geq 1\}\, .
\end{equation*}
 As discussed in \cite{Lyu_Li_maxmargin}, if there exists $w \in \bbR^d$ such that $\sm(w) >0$, solving~\eqref{def:prob2} is equivalent to maximizing the margin. Examining the KKT conditions (see Appendix~\ref{app:gen_sett}) of~\eqref{def:prob2}, we observe that for any $u \in \cZ_s$ such that $\sm(u)>0$, there exists $\lambda >0$, such that $\lambda u$ is a KKT point. This implies that, within the setting of Theorem~\ref{thm:main}, the  optimality characterization is \emph{identical} to that in \cite{Lyu_Li_maxmargin}. 
 
 These observations also highlight that the appearance of a conservative field in our problem is unrelated to backpropagation. The set $\cZ_s$ (thus, implicitly, $\bar{D}_s$) already arises in the analysis of continuous-time subgradient flow in \cite{Lyu_Li_maxmargin}. In fact, as previously noted, $\bar{D} \neq \partial \sm$, even if all $D_i = \partial p_i$ (see Remark~\ref{rmk:max_subg}).


However, we adopt a different perspective. Rather than linking $\cZ_s$ directly to the KKT points of~\eqref{def:prob2}, we interpret it as the set of $\bar{D}_s$-critical points of the margin restricted to the sphere—where $\sm$ is naturally defined due to homogeneity.
Moreover, our definitions of $\bar{D}_s$ and $\cZ_s$ remain valid even when the $D_i$’s are arbitrary conservative set-valued fields, not just subgradients. In fact, our stochastic approximation interpretation allows us to consider a more general setting (see Appendix~\ref{app:gen_sett}) where Assumption~\ref{hyp:marg_lowb} can be relaxed. In this broader framework, the limit points of $(u_k)$ still lie in $\cZ_s$ without necessarily being rescaled versions of the KKT points of~\eqref{def:prob2}.


Finally, we note that as long as the ``stability assumption''~\ref{hyp:marg_lowb} holds, our analysis allows the step-size \( \gamma \) to be of arbitrary  size. This may seem surprising, as (non-smooth) SGD typically requires vanishing step-sizes for convergence (\cite{majewski2018analysis,dav-dru-kak-lee-19,bolte2023subgradient,le2024nonsmooth}). Mathematically, this follows from the fact that \( \bar{\gamma}_k \), the \emph{effective} step-size of the dynamics, is actually decreasing in our setting. A convergence analysis of constant-step SGD for \emph{smooth} homogeneous linear classifiers was studied in \cite{nacson2019stochastic}, but to the best of our knowledge, the more general non-smooth setting had not yet been addressed.

\section{Proof of Proposition~\ref{prop:stoch_approx_exp_log}}\label{pf:sto_app_explog}
In this proof, $C, C_1, C_2, \ldots $ will denote some positive absolute constants that can change from equation to equation. We also note that for all $w,i$, $p_i(w) \leq C \norm{w}^L$ and, due to Assumption~\ref{hyp:conserv}, for all $v \in D_i(w)$, $\norm{v} \leq C \norm{w}^{L-1}$.

To obtain~\eqref{eq:stoch_app_u}, we appropriately rescale the step-size and then write the Taylor's expansion of $u \mapsto (u+h)/\norm{u + h}$, for small $h$, using the fact that $\norm{w_k} \rightarrow + \infty$. 

Towards that goal, let us first introduce the (stochastic) sequence
\begin{equation}\label{eq:noise}
  \eta_{k+1} := {\frac{n_b - n}{n_b n}} \sum_{i \in B_k} l'(p_i(w_k)) \sa_i(w_k) + \frac{1}{n}\sum_{i \notin B_k} l'(p_i(w_k)) \sa_i(w_k)\, .
\end{equation}
Both~\eqref{eq:gd_new} and \eqref{eq:sgd_new} (where for~\eqref{eq:gd_new}, $\eta_{k} \equiv 0$) can be rewritten as 
\begin{equation}\label{eq:sgd_noise}
  w_{k+1} = w_k - \frac{\gamma}{n}  \sum_{i=1}^n l'(p_i(w_k))\sa_i(w_k)+ \gamma \eta_{k+1}\, .
\end{equation}
Now let us introduce the following notations 
\begin{equation}\label{def:reparm_gamma}
  \tgamma_k = -\gamma \norm{w_k}^{L-1} \sum_{j=1}^n l'(p_j(w_k)) \, , \quad \bgamma_k = \tgamma_k\norm{w_k}^{-1}
\end{equation}
and 
\begin{equation}\label{eq:def_lmk_tilel}
  \lambda_{i,k} = \frac{l'(p_i(w_k))}{\sum_{j=1}^n l'(p_j(w_k))} \, , \quad \tilde{\eta}_{k+1} = \frac{-\eta_{k+1}}{\norm{w_k}^{L-1} \sum_{j=1}^n l'(p_j(w_k))}\, .
\end{equation}
Note that since $l'(q) <0$, for all $k$,  $\tgamma_k, \bgamma_k, \lambda_{i,k} \geq 0$. Moreover, $\sum_{i=1}^n\lambda_{i,k} = 1$.

By Assumption~\ref{hyp:conserv} for each $i,k$, it holds that $v_{i,k} = \norm{w_k}^{L-1}g_{i,k}$, where $g_{i,k} \in D_i(u_k)$. Thus, we can rewrite Equation~\eqref{eq:sgd_noise} as:
\begin{equation}\label{eq:first_wk}
    w_{k+1} = w_k + \tilde{\gamma}_k \sum_{i=1}^n \lambda_{i,k} g_{i,k} + \tilde{\gamma}_k \tilde{\eta}_{k+1} :=w_{k} + \tilde{\gamma}_k \bar{g}_k + \tgamma_k \tilde{\eta}_{k+1} \, ,
\end{equation}
with $\bar{g}_k = \sum_{i=1}^n\lambda_{i,k} g_{i,k}$.
Therefore, using the definition of $\bgamma_k$ is~\eqref{def:reparm_gamma},
\begin{equation}\label{eq:uk_taylor1}
  \begin{split}
      u_{k+1} &= \frac{w_{k}+\tgamma_k \bg_k + \tgamma_k \tilde{\eta}_{k+1}}{\norm{w_{k}+\tgamma_k \bg_k + \tgamma_k \tilde{\eta}_{k+1}}} =
      \frac{u_k + \bgamma_k \bg_k + \bgamma_k \tilde{\eta}_{k+1}}{\norm{u_k + \bgamma_k \bg_k + \bgamma_k \tilde{\eta}_{k+1}}}\, .
        \end{split}
\end{equation}
Using the Taylor's expansion $\norm{u+h}^{-1} = 1- \scalarp{u}{h} + \cO(\norm{h}^2)$, for $u \in \bbS^{d-1}$, we obtain
\begin{equation}\label{eq:pf_uk_stochapp}
  u_{k+1}=(u_k + \bgamma_k \bg_k + \bgamma_k \tilde{\eta}_{k+1})(1 - \bgamma_k\scalarp{\bg_k}{u_k} -\bgamma_k\scalarp{\tilde{\eta}_{k+1}}{u_k} + b_k) = u_k + \bar{\gamma}_k \bg_k^s + \bgamma_k \bar{\eta}_{k+1} + \bar{\gamma}_k^2 r_k\, ,
\end{equation}
where $b_k$ is such that $|b_k| \leq C \bgamma_k^2 (\norm{\bg_k} + \norm{\tilde{\eta}_{k+1}})^2$, as soon as $\bgamma_k \norm{\bg_k + \tilde{\eta}_{k+1}} \leq 1/2$, and 
\begin{equation}\label{eqdef:proj_nois_subg}
  \bar{\eta}_{k+1} := \tilde{\eta}_{k+1} - \scalarp{\tilde{\eta}_{k+1}}{u_k} u_k \quad \textrm{ and } \quad \bar{g}_k^s := \bg_k - \scalarp{\bg_k}{u_k} u_k\, ,
\end{equation}
and, finally, 
\begin{equation}\label{eq:rk_bound}
 \bgamma_k^2 \norm{r_k} \leq |b_k|\norm{(u_k + \bgamma_k \bg_k + \bgamma_k \tilde{\eta}_{k+1})} + \bgamma_k^2 \norm{(\bg_k + \tilde{\eta}_{k+1})\scalarp{\bg_k + \tilde{\eta}_{k+1}}{u_k}}\, .
\end{equation}
Equation~\eqref{eq:uk_taylor1} correspond to~\eqref{eq:stoch_app_u}. We now briefly prove the four points of the proposition. 

\emph{Claim on $(\bar{\eta}_k)$.} The fact that $\eta_{k}$ and thus $\bar{\eta}_k$ is $w_k$-measurable is immediate by its definition in~\eqref{eq:noise}. Additionally, $\bbE[\bar{\eta}^s_{k+1} |\cF_k] = \bbE[\eta_{k+1} |\cF_k] = 0$. Moreover, Assumption~\ref{hyp:conserv} and Equation~\eqref{eq:noise} implies
\begin{equation*}
\norm{\eta_{k+1}} \leq C_1\norm{\sum_{i=1}^n l'(p_i(w_k))\sa_i(w_k)} \leq C_2 \norm{w_k}^{L-1} \sum_{i=1}^n |l'(p_i(w_k))|\, ,
\end{equation*}
with $C_2>0$ some deterministic constant independent on $k$. Therefore, by~\eqref{eq:def_lmk_tilel} and \eqref{eqdef:proj_nois_subg}, $\norm{\bar{\eta}_{k+1}} \leq \norm{\tilde{\eta}_{k+1}} \leq C_2$.

\emph{Claim on $(\bgamma_k)$.} Almost surely, there is $\varepsilon >0$, such that for $k$ large enough, $\sm(u_k) \geq  \varepsilon$. Thus, for every $i$, $l'(p_i(w_k)) \leq e^{-\varepsilon\norm{w_k}^L }$. By Assumption~\ref{hyp:conserv}, $\norm{\sa_i(w_k)} \leq C \norm{w_k}^{L-1}$, which implies for $k$ large enough,
\begin{equation*}
  \bgamma_k \leq C_1 \norm{w_k}^{L-2} e^{-\varepsilon \norm{w_k}^L} \leq \frac{C_1 c_2 \log(k)^{L-2}}{k^{\varepsilon c_1}} \leq \frac{1}{\sqrt{k^{\varepsilon c_1}}}
\end{equation*}
where the penultimate inequality comes from Proposition~\ref{prop:log_wk}.
To show that $\sum_{k} \bgamma_k = + \infty$, note that using Equation~\eqref{eq:first_wk}, we have for $k$ large enough,
\begin{equation}\label{eq:sum_gamma_inf}
  \begin{split}
\norm{w_{k+1}}^2  &\leq \lVert w_{k} \rVert^2 + 2 \tilde{\gamma}_k \lVert w_{k} \rVert \lVert \bg_k + \tilde{\eta}_{k+1} \rVert  + \tgamma_k^2 \norm{\bg_k + \tilde{\eta}_{k+1}}^2\\
&\leq \norm{w_k}^2 (1 + C \bgamma_k + C_1 \bgamma_k^2) \leq \norm{w_k}^2 e^{C_2 \bgamma_k} \leq \norm{w_{k_0}}^2 e^{C_2 \sum_{i=k_0}^{k} \bgamma_i }
\end{split}
\end{equation}
where $k_0$ is large enough, and where we have used the fact that $\sup_{k} (\norm{\bg_k + \tilde{\eta}_{k+1}}) < + \infty$ and $\bgamma_k \rightarrow 0$. 
Since the left-hand side of Equation~\eqref{eq:sum_gamma_inf} goes to infinity by Proposition~\ref{prop:log_wk}, we obtain that the right-hand side diverge to infinity and therefore $\sum_{k } \bgamma_k = + \infty$.


\emph{Claim on $(r_k)$.} Since $\bgamma_k \rightarrow 0$, $\sup_{k} \norm{\tilde{\eta}_{k+1}} \leq C_1$ and $\sup_{k} \norm{\bg_k} < C_2$, there is $k_0$, such that for all $k \geq k_0$, $\bgamma_k (\norm{\bg_k} + \norm{\tilde{\eta}_{k+1}})\leq 1/2$. Therefore, for $k \geq k_0$, $|b_k| \leq C \gamma_k^2 \norm{\bg_k + \tilde{\eta}_{k+1}}$ in~\eqref{eq:pf_uk_stochapp}, which by~\eqref{eq:rk_bound} implies that $\sup_{k \geq k_0}\norm{r_k} \leq C$.

\emph{Claim on $\bar{D}$.} Consider a sequence $u_{k_j} \rightarrow u$ and $g$ any accumulation point $g_{k_j}$, we need to prove that $g - \scalarp{g}{u}u \in \bar{D}_s(u)$, or, equivalently, that $g \in \bar{D}(u)$. Recall that $\bg_k = \sum_{i=1}^k \lambda_{i,k} \bg_{i,k}$, where $\lambda_{i,k} \geq 0$, $\sum_{i} \lambda_{i,k} = 1$ and $g_{i,k} \in D_{i}(u_k)$. Extracting a subsequence, we can assume that for each $i$, $\lambda_{i, k_j} \rightarrow \lambda_i$ and $g_{i,k_j} \rightarrow g_i \in D_i(u)$.  We claim that $\lambda_i \neq 0 \implies p_i(u) = \sm(u)$. Indeed, without losing generality assume that $p_1(u_{k_j})\rightarrow \sm(u)$. Then, if $\lim (p_{i}(u_k) - \sm(u)) >0$, we obtain 
\begin{equation*}
  \lambda_{i,k_j} \leq \frac{l'(\norm{w_{k,j}}^L p_i(u_{k_j}))}{l'(\norm{w_{k_j}}^L p_1(u_{k_j}))} \leq C e^{-\norm{w_k}^L\left( p_i(u_k) - \sm(u)\right)}\xrightarrow[j \rightarrow + \infty]{} 0 \, .
\end{equation*}
Therefore, $g$ can be written as $ \sum_{i=1}^n \lambda_i g_i$, where $g_i \in D_i(u)$ and $\lambda_i \neq 0 \implies p_i(u) = \sm(u)$.
In other words, $g \in \bar{D}(u)$, concluding the proof. 

\hfill $\blacksquare$


\section*{Acknowledgement}
The authors thank Evgenii Chzhen for insightful discussions.

\bibliography{main}

\appendix
\section{Experimental Setup}\label{app:exp}
\subsection{Datasets}
UnKEBench \cite{UnKE} constructs a dataset containing 1,000 counterfactual unstructured texts, where knowledge is presented in an unstructured and relatively lengthy form, going beyond simple knowledge triplets or linear fact chains. These texts originate from ConflictQA \cite{conflictqa}, a benchmark specifically designed to distinguish LLMs' parameter memory from anti-memory. This approach is crucial for preventing the model from merging knowledge obtained during pretraining with knowledge acquired during the editing process. Moreover, it addresses the key challenge of determining whether the model learns target knowledge during training or editing, ensuring a clear boundary between pretraining knowledge and edited knowledge.

AKEW benchmark \cite{AKEW} considers three aspects: (1) \textit{Structured Facts}: Each structured fact consists of an isolated triplet for editing, sourced from existing datasets or knowledge bases. (2) \textit{Unstructured Facts}: Knowledge is presented in unstructured text form. To enable fair comparisons, each unstructured fact contains the same knowledge update as its corresponding structured fact. Compared to structured facts, unstructured facts exhibit greater complexity in natural language format, as they often encapsulate more implicit knowledge. (3) \textit{Extracted Triplets}: Triplets are extracted from unstructured facts using automated methods to investigate whether they can facilitate knowledge editing methods in handling unstructured knowledge. In this work, we primarily focus on unstructured factual knowledge.

EditEverything dataset integrates question-answering data from multiple domains, forming long and diverse knowledge formats that are more challenging to edit. Specifically, for mathematics, we select longer samples from the Orca-Math dataset \cite{math}, which includes grade school math word problems. For coding, we use the MBPP dataset \cite{code}, which consists of approximately 1,000 crowd-sourced Python programming problems solvable by entry-level programmers, covering programming fundamentals and standard library functionalities. For chemistry, we sample from the Camel-Chemistry dataset \cite{chemistry}, which contains problem-solution pairs generated from 25 chemistry topics, each with 25 subtopics and 32 problems per topic-subtopic pair. Lastly, for the news and poetry categories, since they often contain real-world knowledge that LLMs may already possess, we generate synthetic data using GPT-4o to ensure that the information is not already known by the model.

We present sample instances from the dataset in Figure \ref{fig:sample1}, Figure \ref{fig:sample2}, and Figure \ref{fig:sample3}.

\subsection{Evaluation Metrics} \label{app:exp_metric}
Following previous research on model editing for structured knowledge \cite{ROME, MEND}, existing evaluation metrics primarily focus on triplet-structured knowledge, where the goal is to assess the modification of factual triples (\textit{subject, relation, object}). Specifically, given an LLM $f$, an editing knowledge pair $(x, y)$, an equivalent knowledge query $x_e$, and unrelated knowledge pairs $(x_{loc}, y_{loc})$, three standard evaluation metrics are commonly used:

\textbf{Efficacy.} This metric quantifies the success of modifying the target knowledge in $f_{\mathcal{W}}$. It evaluates whether the edited LLM generates the desired target output $y$ when given the input $x$. Formally, it is defined as:
\begin{equation}
\mathbb{E}\left\{y=\mathop{\arg\max}\limits_{y'}\mathbb{P}_{f}(y'\left|x\right.)\right\}.
\end{equation}

\textbf{Generalization.} This metric assesses whether the model has generalized the newly edited knowledge beyond its specific form. It measures if the LLM correctly produces $y$ when given a semantically equivalent input $x_e$, indicating the degree to which the update propagates correctly across paraphrased or restructured queries:
\begin{equation}
\mathbb{E}\left\{y=\mathop{\arg\max}\limits_{y'}\mathbb{P}_{f}(y'\left|x_e\right.)\right\}.
\end{equation}

\textbf{Specificity.} This metric evaluates whether the editing operation is localized, ensuring that unrelated knowledge remains intact. It measures whether the model's response to an unrelated query $x_{loc}$ remains consistent with its original output $y_{loc}$:
\begin{equation}
\mathbb{E}\left\{y_{loc}=\mathop{\arg\max}\limits_{y'}\mathbb{P}_{f}(y'\left|x_{loc}\right.)\right\}.
\end{equation}

While these metrics are well-suited for structured knowledge editing, they are insufficient for evaluating long-form and diverse-formatted knowledge. Such knowledge is often verbose and complex, making it challenging to assess correctness solely based on Efficacy. In these cases, the model may generate an answer that captures the essential information yet fails an exact-match evaluation. To address this, we primarily follow the existing benchmarks for unstructured knowledge editing, incorporating more flexible evaluation methods suited for long-form responses.

Lexical similarity metrics include BLEU \cite{bleu} and various ROUGE scores (ROUGE-1, ROUGE-2, and ROUGE-L) \cite{rouge}. These are computed based on the \textit{original questions}, \textit{paraphrase question}, and \textit{sub-questions}, providing insights into the lexical and n-gram alignment between the model-generated text and the target answer. These metrics serve as the foundation for assessing the surface-level accuracy of edited content.

Semantic similarity is also considered (Bert Score) \cite{bertscore}, as word-level overlap alone is insufficient to capture the nuanced understanding required by the model. To address this, we utilize embedding-based encoders, specifically the all-MiniLM-L6-v2 model \footnote{https://huggingface.co/sentence-transformers/all-MiniLM-L6-v2}, to measure semantic similarity. This ensures a more balanced evaluation that extends beyond lexical matching, quantifying the depth of the model's comprehension.

\subsection{Baseline Methods}
\begin{itemize}
    \item \textbf{FT-L} \cite{FTw} is a knowledge editing approach that fine-tunes specific layers of the LLM using an autoregressive loss function. We reimplemented this method following the hyperparameter from the original paper.
    
    \item \textbf{MEND} \cite{MEND} is a hypernetwork-based efficient knowledge editing method. It trains a hypernetwork to capture patterns in knowledge updates by mapping low-rank decomposed fine-tuning gradients to LLM parameter modifications, enabling efficient and localized edits. Our implementation follows the original hyperparameter settings and completes training over the full dataset. 
    
    \item \textbf{ROME} \cite{ROME} is a method for modifying factual associations in LLM parameters. It identifies critical neuron activations in MLP layers through perturbation-based knowledge localization and modifies MLP layer weights using Lagrange remainders. Since ROME is not designed for large-scale edits, we follow the original paper’s settings and conduct multiple rounds of single-instance editing for evaluation.
    
    \item \textbf{MEMIT} \cite{MEMIT} extends ROME by enabling batch updates of factual knowledge. It utilizes least squares approximation to modify specific layer parameters across multiple layers, allowing simultaneous updates of large numbers of knowledge facts. We evaluate MEMIT in lifelong editing scenarios using the original paper’s configuration.
    
    \item \textbf{AlphaEdit} \cite{AlphaEdit} is a method designed to mitigate interference in LLM lifelong knowledge editing. It introduces a null-space projection mechanism that ensures parameter updates preserve previously edited knowledge while incorporating new updates. AlphaEdit has demonstrated state-of-the-art (SOTA) performance across multiple evaluation metrics while maintaining strong transferability. We follow the original paper’s hyperparameter configuration in our implementation.
    
    \item \textbf{UnKE} \cite{UnKE} improves knowledge editing by refining both the layer and token dimensions. In the layer dimension, it replaces local key-value storage with a non-local block-based mechanism, enhancing the representation capability of key-value pairs while integrating attention-layer knowledge. In the token dimension, it replaces "term-driven optimization" with "cause-driven optimization," which directly edits the final token while preserving contextual coherence. This eliminates the need for explicit term localization and prevents context loss.
\end{itemize}

\subsection{Implementation Details}
Our AnyEdit and AnyEdit* primarily follow the baseline configurations of MEMIT and UnKE, while other baselines adhere to their original implementation settings. All experiments were conducted on a single A100 GPU (80GB).
\begin{itemize}
    \item \textbf{AnyEdit on Llama3-8B-Instruct:} We select layers 4 to 8 for editing and apply a clamp norm factor of 4. The fact token is defined as the last token. The optimization process involves 25 gradient steps for updating the key-value representations, with a learning rate of 0.5. The loss is applied at layer 31, and we use a weight decay of 0.001. To maintain distributional consistency, we introduce a Kullback-Leibler (KL) regularization term with a factor of 0.0625. Furthermore, we enable hyperparameter $\lambda$ with an update weight of 15,000, using 100,000 samples from the Wikipedia dataset with a data type of float32. The module configurations follow MEMIT, where edits are applied to the MLP down projection layers of the selected transformer blocks. Additionally, for chunked editing, we set a chunk size of 40 tokens with no overlap.
    \item \textbf{AnyEdit on Qwen2.5-7B-Instruct:} Same as the above, except that the loss is applied at layer 27 and the chunk size is set to 50 tokens.
    \item \textbf{AnyEdit* on Llama3-8B-Instruct:} We select layer 7 for editing and apply a clamp norm factor of 4. The fact token is defined as the last token. The optimization process involves updating all parameters in both the attention and MLP layers. The gradient descent process utilizes a learning rate of 0.0002 with 50 optimization steps. For updating key-value representations, we use 25 gradient steps with a learning rate of 0.5. The loss is applied at layer 31, and we use a weight decay of 0.001. To preserve original knowledge, we sample 20 data points to constrain parameter updates. Additionally, for chunked editing, we set a chunk size of 40 tokens with no overlap.
    \item \textbf{AnyEdit* on Qwen2.5-7B-Instruct:} Same as the above, except that the loss is applied at layer 27 and the chunk size is set to 50 tokens.
\end{itemize}

\section{Locate-Then-Edit Paradigm \& Related Proof}
\subsection{Locate-Then-Edit Paradigm}\label{app:model_edit}
Following prior works on model editing, the detailed descriptions of specific methods in this section are based on MEMIT \cite{MEMIT}, AlphaEdit \cite{AlphaEdit} and ECE \cite{ECE}. We adhere to their formulations and methodological explanations to ensure consistency and clarity in presenting these approaches.

The locate-then-edit method primarily focuses on triplet-structured knowledge in the form of $(s, r, o)$, such as modifying $(\text{Olympics}, \text{were held in}, \text{Tokyo})$ to $(\text{Olympics}, \text{were held in}, \text{Paris})$. Given new knowledge $(x_e, y_e)$, a triplet can be represented as $x_e = (s, r)$ and $y_e = o$.

We first refine the auto-regressive language model formulation in Section \ref{sec:method:pre}. Let $f$ be a decoder-only model with $L$ layers, processing input sequence $x = (x_0, x_1, \dots, x_T)$ to predict the next token:
\begin{equation}
    \begin{aligned}
        \vh_t^l(x) &= \vh_t^{l - 1}(x) + \va_t^l(x) + \vm_t^l(x), \\
        \va_t^l &= \text{attn}^l(\vh_0^{l - 1}, \vh_1^{l - 1}, \dots, \vh_t^{l - 1}), \\
        \vm_t^l &= \mW_{\text{out}}^l \sigma(\mW_{\text{in}}^l \gamma(\vh_t^{l - 1}+\va_t^l)),
    \end{aligned}
\end{equation}
where $\vh_t^l$ denotes the hidden state of token $t$ at layer $l$, $\va_t^l$ is the attention output, and $\vm_t^l$ is the feedforward (FFN) output. Here, $\mW_{\text{in}}^l$ and $\mW_{\text{out}}^l$ are weight matrices, $\sigma$ is a nonlinear activation function, and $\gamma$ denotes layer normalization.

\textbf{Key-Value Memory Structure}. Locate-then-edit assumes that factual knowledge is stored in the FFN layers and treats them as linear associative memory \cite{key_value}. Specifically, $\mW_{\text{out}}^l$ is conceptualized as a key-value memory structure:
\begin{equation}
    \begin{aligned}
        \underbrace{\vm_t^l}_{\vv} = \mW_{\text{out}}^l \underbrace{\sigma(\mW_{\text{in}}^l \gamma(\vh_t^{l-1}+\va^l))}_{\vk}. \label{eqapp:define_kv}
    \end{aligned}
\end{equation}
Here, the MLP input-output pair at token $t$ and layer $l$ serves as the key-value pair. Casual Tracing is typically used to locate the target token and layer by injecting Gaussian noise into hidden states and incrementally restoring them to analyze output recovery. For more details, please refer to ROME \cite{ROME}.

\textbf{Computing Key-Value.} For editing knowledge $(x_e, y_e)$, we compute its corresponding key-value pair $(\vk^*, \vv^*)$. The key $\vk^*$ is derived via forward propagation of $x_e$, while the value $\vv^*$ is optimized using gradient descent:
\begin{equation}
    \vv^* = \vv + \arg \min_{\bm{\delta}^l} \left( -\log \mathbb{P}_{f(\vh_t^l + \bm{\delta}^l)} [y_e \mid x_e] \right).
\end{equation}
Here, $f(\vh_t^l + \bm{\delta}^l)$ represents the model output when the FFN output $\vh_t^l$ is replaced with $\vh_t^l + \bm{\delta}^l$. 

Methods such as ROME \cite{ROME}, MEMIT \cite{MEMIT}, and AlphaEdit \cite{AlphaEdit} focus on triplets $(s, r, o)$, selecting the last token of the subject $s$ as the target token. In contrast, UnKE \cite{UnKE} extends to unstructured text, using the last token of $x_e$ as the target.

To insert new knowledge $(\vk^*, \vv^*)$ into the key-value memory, we solve the constrained least squares problem:
\begin{align*}
    \min_{\hat{\mW}} &\quad \left\lVert \hat{\mW}\mK - \mV \right\rVert \\
    \text{s.t.} &\quad \hat{\mW}\vk^* = \vv^*.
\end{align*}
The final parameter update can be computed via ROME/MEMIT/AlphaEdit's closed-form solution or UnKE's gradient-based optimization.

For clarity, let $\tilde{\mW}$ denote the edited weight of $\mW_{\text{out}}^l$ in the MLP, and let $\mW$ represent its original weight. The final parameter update can be computed using the closed-form solutions of ROME/MEMIT/AlphaEdit or the gradient-based optimization method in UnKE.

\textbf{Weights Update in ROME.} The ROME method derives a closed-form solution to the constrained least-squares problem for updating MLP parameters:
\begin{equation}
    \tilde{\mW} = \mW + \frac{(\vv^\ast - \mW\vk^\ast) (\mC^{-1} \vk^\ast) ^ {T}}{(\mC^{-1} \vk^\ast) ^ {T} \vk^\ast},
\end{equation}
where $\mC = \mK \mK^T$. The matrix $\mC$ is estimated using 100,000 samples of hidden states $\vk$ obtained from tokens sampled in-context from the entire Wikipedia dataset.

\textbf{Weights Update in MEMIT.} Since the above solution updates only a single knowledge sample at a time, MEMIT improves upon it by avoiding Lagrange multipliers and instead using a relaxed constraint formulation. The problem is reformulated by maintaining a factual set $\{\mK_1, \mV_1\}$ containing $u$ new associations while preserving the original set $\{\mK_0, \mV_0\}$ with $n$ existing associations:
\begin{equation}
\begin{gathered}
    \mK_0 = \left[\vk_1 \mid \vk_2 \mid \dots \mid \vk_n\right], \quad \mV_0 = \left[\vv_1 \mid \vv_2 \mid \dots \mid \vv_n\right], \\
    \mK_1 = \left[\vk^\ast_{n+1} \mid \vk^\ast_{n+2} \mid \dots \mid \vk^\ast_{n+u}\right], \quad \mV_1 = \left[\vv^\ast_{n+1} \mid \vv^\ast_{n+2} \mid \dots \mid \vv^\ast_{n+u}\right].
\end{gathered}
\end{equation}
Here, $\vk$ and $\vv$ are defined as in Eq.~\ref{eqapp:define_kv}, and their subscripts denote knowledge indices. The objective function is given by:
\begin{equation}
    \tilde{\mW} \triangleq \argmin_{\hat{\mW}} \left( \sum_{i=1}^{n} \left\| \hat{\mW} \vk_i - \vv_i \right\|^2 + \sum_{i=n+1}^{n+u} \left\| \hat{\mW} \vk_i - \vv^\ast_i \right\|^2 \right).
\end{equation}
Applying the normal equation \citep{normal_equation}, the closed-form solution is:
\begin{equation}
    \tilde{\mW} = \left( \mV_1 - \mW \mK_1 \right) \mK_1^T \left( \mK_0 \mK_0^T + \mK_1 \mK_1^T \right)^{-1} + \mW.
\end{equation}

\textbf{Weights Update in AlphaEdit.} AlphaEdit addresses the imbalance between old and new knowledge in lifelong learning. It protects existing knowledge using a null-space projection constraint, ensuring that the update $\bm{\Delta}$ to $\mW_{\text{out}}^l$ is always projected onto the null space of $\mK_0 \mK_0^T$. The final parameter update, refining MEMIT, is:
\begin{equation}
    \tilde{\mW} = \left( \mV_1 - \mW \mK_1 \right) \mK_1^T \mP \left( \mK_p \mK_p^T \mP + \mK_1 \mK_1^T \mP + \mI \right)^{-1}+ \mW.
\end{equation}

\textbf{Weights Update in UnKE.} Unlike previous methods, UnKE considers the entire input to layer $l$, denoted as $f^l$, rather than just the MLP input. The output remains $f^l$'s activation values. The parameter update is applied to the entire layer rather than a single weight matrix. Given the knowledge sets $\{\mK_0, \mV_0\}$ and $\{\mK_1, \mV_1\}$, the optimization objective is formulated as:
\begin{equation}
    \tilde{\Theta}^l \triangleq \argmin_{\hat{\Theta}^l} \left( \sum_{i=1}^{n} \left\|  f_{\hat{\Theta}^l}^l(\vk_i) - \vv_i \right\|^2 + \sum_{i=n+1}^{n+u} \left\|  f_{\hat{\Theta}^l}^l(\vk_i) - \vv^\ast_i \right\|^2 \right),
\end{equation}
where $\Theta^l$ denotes the entire set of parameters in layer $l$. Since a closed-form solution is not feasible, UnKE employs gradient descent to iteratively update $\Theta^l$.

\subsection{Proof of Optimization-Conditional Mutual Information Equivalence} \label{app:proof_cmi}
\begin{theorem}
The optimization objective  
\begin{equation}
    \bm{\delta}^* = \argmin_{\bm{\delta}} \left( -\log \mathbb{P}_{f(\vh_t+\bm{\delta})}(Y \mid X) \right), \label{eq:opt}
\end{equation}  
is equivalent to maximizing the conditional mutual information (CMI) between $X$ and $Y$ given the perturbed hidden state $\vh'$:  
\begin{equation}
    \vh' = \argmax_{\vh'} I(X; Y \mid \vh'). \label{eq:cmi}
\end{equation}
\end{theorem}

\begin{proof}
Starting from the definition of CMI, we expand it via the integral form:  
\begin{equation}
I(X; Y \mid \vh') = \int p(x, y, \vh') \log \frac{p(y \mid x, \vh')}{p(y \mid \vh')} \, dx dy d\vh'.
\end{equation}  
% Applying Bayes’ rule $p(y \mid x, \vh') = \frac{p(x, y \mid \vh')}{p(x \mid \vh')}$, we rewrite the integrand:  
% \begin{equation}
% I(X; Y \mid \vh') = \int p(x, y, \vh') \log \frac{p(x, y \mid \vh')}{p(x \mid \vh') p(y \mid \vh')} \, dx dy d\vh'.
% \end{equation}  
This splits into two entropy terms:  
\begin{align}
I(X; Y \mid \vh') = \underbrace{\int p(x, y, \vh') \log p(y \mid x, \vh') \, dx dy d\vh'}_{\text{Term } \mathcal{A}} - \underbrace{\int p(x, y, \vh') \log p(y \mid \vh') \, dx dy d\vh'}_{\text{Term } \mathcal{B}}. \label{eq:split}
\end{align}  

Term $\mathcal{A}$ simplifies to the expectation:  
\begin{equation}
\mathcal{A} = \mathbb{E}_{p(\vh')} \mathbb{E}_{p(x, y \mid \vh')} \left[ \log p(y \mid x, \vh') \right],
\end{equation}  
while Term $\mathcal{B}$ is independent of $X$ given $\vh'$. Since $\mathcal{B}$ does not affect the optimization over $\vh'$, we focus on maximizing $\mathcal{A}$.  

By definition, $\mathbb{P}_{f(\vh')}(Y \mid X) = p(y \mid x, \vh')$. Thus, minimizing the negative log-likelihood in \eqref{eq:opt} directly maximizes $\mathcal{A}$, which is equivalent to maximizing $I(X; Y \mid \vh')$. Substituting $\vh' = \vh_t + \bm{\delta}^*$, we conclude:  
\begin{equation}
\vh' = \argmax_{\vh'} I(X; Y \mid \vh'),
\end{equation}  
thereby establishing the equivalence.  
\end{proof}

\subsection{Proof of the Decomposition of Mutual Information}\label{app:proof_decom}
To rigorously derive Equation \eqref{eq:final_MI}, we start from the mutual information (MI) decomposition given in Equation \eqref{eq:decom}:
\begin{equation}
    I(X; Y \mid \vh'_1, \dots, \vh'_K) = \sum_{k=1}^{K} I(X; Y_k \mid Y_1, \dots, Y_{k-1}, \vh'_1, \dots, \vh'_K).
\end{equation}

\textbf{Step 1: Application of the First Property.}
The first key property states that later hidden states do not influence earlier token generation:
\begin{equation}
    H(Y_p \mid \vh'_q) = H(Y_p), \quad \text{for } p < q.
\end{equation}
Since mutual information is defined as:
\begin{equation}
    I(X; Y_k \mid Y_1, \dots, Y_{k-1}, \vh'_1, \dots, \vh'_K) = H(Y_k \mid Y_1, \dots, Y_{k-1}, \vh'_1, \dots, \vh'_K) - H(Y_k \mid X, Y_1, \dots, Y_{k-1}, \vh'_1, \dots, \vh'_K).
\end{equation}
Since $\vh'_q$ for $q > k$ does not affect $Y_k$, we can simplify:
\begin{equation}
    H(Y_k \mid Y_1, \dots, Y_{k-1}, \vh'_1, \dots, \vh'_K) = H(Y_k \mid Y_1, \dots, Y_{k-1}, \vh'_1, \dots, \vh'_k).
\end{equation}

\textbf{Step 2: Application of the Second Property.}
The second key property states that once $Y_k$ is determined, conditioning on $Y_k$ subsumes conditioning on $\vh'_k$:
\begin{equation}
    H(\cdot \mid Y_k) = H(\cdot \mid Y_k, \vh'_k).
\end{equation}
Using this, we rewrite the MI term:
\begin{equation}
    I(X; Y_k \mid Y_1, \dots, Y_{k-1}, \vh'_1, \dots, \vh'_K) = I(X; Y_k \mid Y_1, \dots, Y_{k-1}, \vh'_k).
\end{equation}

\textbf{Step 3: Applying the Conditional Mutual Information Decomposition.}
Using the decomposition formula for conditional mutual information, each term can be written as:
\begin{equation}
    I(X; Y_k \mid Y_1, \dots, Y_{k-1}, \vh'_k) = I(X, Y_1, \dots, Y_{k-1}; Y_k \mid \vh'_k) - I(Y_1, \dots, Y_{k-1}; Y_k \mid \vh'_k).
\end{equation}
In the optimization process, since we are given $X$, the second term is ignored:
\begin{equation}
    I(X; Y_k \mid Y_1, \dots, Y_{k-1}, \vh'_k) = I(X, Y_1, \dots, Y_{k-1}; Y_k \mid \vh'_k).
\end{equation}
Substituting this result back into our summation, we arrive at the desired decomposition:
\begin{equation}
    I(X; Y \mid \vh'_1, \dots, \vh'_K) = \sum_{k=1}^{K} I(X, Y_1, \dots, Y_{k-1}; Y_k \mid \vh'_k).
\end{equation}
This completes the proof.

\section{More Experimental Results}\label{app:exp_result}
\subsection{Case Study}

We selected several editing samples from the UnKEBench and EditEverything datasets as case studies to analyze the long-form knowledge edit performance. 

From the experimental results, AnyEdit effectively restores the target text in terms of both textual consistency and generalization to paraphrase questions. In contrast, other baseline methods often fail to achieve effective editing. Although the UnKE method also demonstrates strong editing performance, it exhibits a noticeable performance drop on paraphrase questions.


\subsubsection{Case 1}

\begin{tcolorbox}[boxrule=0.5pt, left=0pt, right=0pt, top=2.5pt, bottom=2.5pt, title={Model Editing Case Study on Llama3-8B-Instruct}]
    \centering
    \renewcommand{\arraystretch}{1.8}
    \begin{tabular}{p{0.18\linewidth} p{0.76\linewidth}}
        \makecell[c]{\textbf{Input}} & \makecell[c]{What is John Mayne's background and experience in journalism?} \\
        \hline
        \makecell[c]{\textbf{Paraphrase}} & \makecell[c]{Can you provide information about John Mayne's journalism background and experience?} \\
        \hline
        \makecell[c]{\textbf{Target Output}} & \makecell*[{{p{13cm}}}]{John Mayne is a journalist who has been working in the field for over a decade. He has written for several reputable publications, including \tar{The New York Times} and \tar{The Guardian}. In an interview with \tar{The Huffington Post}, Mayne discussed his passion for journalism and his commitment to reporting on important issues. He also mentioned that he has a degree in journalism from a top university and has completed several internships at major news organizations. Mayne's extensive experience and education make him a highly qualified journalist and a valuable asset to any newsroom.} \\
        \hline
        \multicolumn{2}{c}{\textbf{\large LLM-edited Generation}} \\ % Bold and centered
        \hline
        \makecell[c]{\textbf{MEMIT}} & \makecell*[{{p{13cm}}}]{John Mayne is a journalist and writer, but I couldn't find much about his personal background. However, he has been in the industry for several years, contributing to various publications. According to LinkedIn, he has worked for major outlets including BBC and \tar{The Guardian}(...) John Mayne's experience spans politics, business, and culture, and he has been recognized for his in-depth reporting.} \\
        \hline
        \makecell[c]{\textbf{AlphaEdit}} & \makecell*[{{p{13cm}}}]{John Mayne is a British journalist and author with a long career in media. He has written for major newspapers such as The Times and The Sunday Times. Mayne has covered diverse topics, including politics and culture, and worked as a foreign correspondent reporting from the Middle East, Africa, and Asia... In addition, he has taught journalism at universities and received multiple awards for his writing.} \\
        \hline
        \makecell[c]{\textbf{UnKE}} & \makecell*[{{p{13cm}}}]{John Mayne is a journalist who has been working in the field for several years. He has written for a number of publications and has a strong track record of producing high-quality, well-researched articles. In addition to his writing experience, Mayne has also worked as an editor and has a strong understanding of the journalism industry. He is well-respected by his peers and is known for his ability to produce engaging and informative articles.} \\
        \hline
        \makecell[c]{\textbf{AnyEdit}} & \makecell*[{{p{13cm}}}]{ John Mayne is a journalist who has been working in the field for over a decade. He has written for several reputable publications, including \tar{The New York Times} and \tar{The Guardian}. In an interview with \tar{The Huffington Post}, Mayne discussed his passion for journalism and his commitment to reporting on important issues. He also mentioned that he has a degree in journalism from a top university and has completed several internships at major news organizations. Mayne's extensive experience and education make him a highly qualified journalist and a valuable asset to any newsroom.} \\
    \end{tabular}
\end{tcolorbox}

\subsubsection{Case 2}

\begin{tcolorbox}[boxrule=0.5pt, left=0pt, right=0pt, top=2.5pt, bottom=2.5pt, title={Model Editing Case Study on Qwen2.5-7B-Instruct}]
    \centering
    \renewcommand{\arraystretch}{1.8}
    \begin{tabular}{p{0.18\linewidth} p{0.76\linewidth}}
        \makecell[c]{\textbf{Input}} & \makecell[c]{find the value of $log_y(x^4)$ if $log_x(y^3) = 2$} \\
        \hline
        \makecell[c]{\textbf{Target Output}} & \makecell*[{{p{13cm}}}]{To find the value of $log_y(x^4)$, we can use the property of logarithms that states: $log_b(a) = 1 / log_a(b)$. Given that $log_x(y^3) = 2$, we can rewrite this as: $y^3 = x^2$. Now, we want to find $log_y(x^4)$. We can use the property mentioned above: $log_y(x^4) = 1 / log_x(y^4)$. But we don't have $log_x(y^4)$; we have $log_x(y^3)$. So, we need to manipulate the expression to use the given information. We know that $y^3 = x^2$, so we can raise both sides to the power of $4/3$ to get $y^4$: $(y^3)^{(4/3)} = (x^2)^{(4/3)}$. $y^4 = x^{(8/3)}$. Now we can write $log_x(y^4)$ in terms of $log_x(y^3)$: $log_x(y^4) = log_x((y^3)^{(4/3)})$. Using the power rule of logarithms ($log_b(a^c) = c * log_b(a)$), we get: $log_x(y^4) = (4/3) * log_x(y^3)$. Since we know $log_x(y^3) = 2$, we can substitute this value in: $log_x(y^4) = (4/3) * 2$. $log_x(y^4) = 8/3$. Now we can find $log_y(x^4)$ using the reciprocal property: $log_y(x^4) = 1 / log_x(y^4)$. $log_y(x^4) = 1 / (8/3)$. $log_y(x^4) = 3/8$. Therefore, the value of $log_y(x^4)$ is \tar{$3/8$}.} \\
        \hline
        \multicolumn{2}{c}{\textbf{\large LLM-edited Generation}} \\ % Bold and centered
        \hline
        \makecell[c]{\textbf{MEMIT}} & \makecell*[{{p{13cm}}}]{To find the value of $log_y(x^4)$, we can use the change of base formula and the given information. Given: $log_x(y^3) = 2$. Using the change of base formula, we can rewrite this as: $log_x(y^3) = (log_y(y^3)) / (log_y(x))$. Since $log_y(y^3) = 3$ (because $y^3$ is the argument raised to the power of 3), we have: $2 = 3 / (log_y(x))$. Now, we can solve for $log_y(x)$: $log_y(x) = 3/2$. Next, we want to find $log_y(x^4)$. Using the power rule of logarithms, we can write: $log_y(x^4) = 4 * log_y(x)$. Substituting the value we found for $log_y(x)$: $log_y(x^4) = 4 * (3/2)$. Simplifying: $log_y(x^4) = 6$. Therefore, the value of $log_y(x^4)$ is $6$.} \\
        \hline
        \makecell[c]{\textbf{AlphaEdit}} & \makecell*[{{p{13cm}}}]{To find the value of $log_y(x^4)$, we can use the change of base formula and the given information: $log_x(y^3) = 2$. Using the change of base formula, we can rewrite this as: $y^3 = x^2$. Now, we want to find $log_y(x^4)$. Using the properties of logarithms, we can rewrite this as: $log_y(x^4) = 4 * log_y(x)$. We can express $log_y(x)$ in terms of $log_x(y)$ using the change of base formula: $log_y(x) = 1 / log_x(y)$. Since we know that $log_x(y^3) = 2$, we can write: $log_x(y^3) = 3 * log_x(y) = 2$. Therefore: $log_x(y) = 2/3$. Substituting this into our expression for $log_y(x)$: $log_y(x) = 1 / (2/3) = 3/2$. Finally, substituting this back into our expression for $log_y(x^4)$: $log_y(x^4) = 4 * (3/2) = 6$. Therefore, the value of $log_y(x^4)$ is $6$.} \\
        \hline
        \makecell[c]{\textbf{UnKE}} & \makecell*[{{p{13cm}}}]{To find the value of $log_y(x^4)$, we can use the property of logarithms that states: $log_b(a) = 1 / log_a(b)$. Given that $log_x(y^3) = 2$, we can rewrite this as $y^3 = x^2$. Now, we can use the property to find $log_y(x^4)$: $log_y(x^4) = 1 / log_x(y^3)$ $= 1 / (1 / x^2)$ $= x^2$. So, the value of $log_y(x^4)$ is $x^2$. Therefore, the correct answer is $x^2$.} \\
        \hline
        \makecell[c]{\textbf{AnyEdit}} & \makecell*[{{p{13cm}}}]{To find the value of $log_y(x^4)$, we can use the property of logarithms that states: $log_b(a) = 1 / log_a(b)$. Given that $log_x(y^3) = 2$, we can rewrite this as: $y^3 = x^2$. Now, we want to find $log_y(x^4)$. We can use the property mentioned above: $log_y(x^4) = 1 / log_x(y^4)$. But we don't have $log_x(y^4)$; we have $log_x(y^3)$. So, we need to manipulate the expression to use the given information. We know that $y^3 = x^2$, so we can raise both sides to the power of $4/3$ to get $y^4$: $(y^3)^{(4/3)} = (x^2)^{(4/3)}$. $y^4 = x^{(8/3)}$. Now we can write $log_x(y^4)$ in terms of $log_x(y^3)$: $log_x(y^4) = log_x((y^3)^{(4/3)})$. Using the power rule of logarithms ($log_b(a^c) = c * log_b(a)$), we get: $log_x(y^4) = (4/3) * log_x(y^3)$. Since we know $log_x(y^3) = 2$, we can substitute this value: $log_x(y^4) = (4/3) * 2$. $log_x(y^4) = 8/3$. Now we can use the property of logarithms: $log_y(x^4) = 1 / log_x(y^4)$. $log_y(x^4) = 1 / (8/3)$. $log_y(x^4) = 3/8$. Therefore, the value of $log_y(x^4)$ is \tar{$3/8$}.} \\
    \end{tabular}
\end{tcolorbox}
\clearpage
\newpage

\subsection{Supplementary Experimental Results on RQ1 \& RQ2} \label{app:exp_result_1}
We present a comprehensive evaluation of all metrics on the UnKEBench and AKEW datasets in Table \ref{tab:app_1} and Table \ref{tab:app_2}. The results demonstrate that UnKE consistently outperforms other baselines across both original and paraphrase question evaluations. Notably, UnKE+, which integrates AnyEdit’s autoregressive editing paradigm, achieves even higher scores in lexical similarity (BLEU, ROUGE-1/2/L) and semantic similarity (BERT Score), indicating its superior ability to preserve and generalize edited knowledge. In contrast, MEMIT and AlphaEdit struggle with paraphrase generalization, showing significantly lower performance on the right side of `/', suggesting that these methods fail to robustly transfer edited knowledge across rephrased contexts. While MEMIT+ and AlphaEdit+ improve over their base versions, their performance still lags behind UnKE and UnKE+.

Overall, UnKE+ achieves the best balance between precise knowledge modification and robust generalization, confirming that combining UnKE with autoregressive fine-tuning leads to stronger and more reliable knowledge editing in LLMs.
\begin{table*}[h]
\caption{Performance comparison in UnKEBench. The `+' symbol indicates results incorporating AnyEdit's autoregressive editing paradigm. The left side of `/' represents the LLM's edited output for original questions, while the right side represents the edited output for paraphrase questions.}
    \label{tab:app_1}
    \centering
    \renewcommand{\arraystretch}{1.2}
    \setlength{\tabcolsep}{4pt}
    \resizebox{\textwidth}{!}{
    \begin{tabular}{l cccc ccc}
        \toprule
        \multirow{2}{*}{\textbf{Method}} & \multicolumn{4}{c}{\textbf{Lexical Similarity}} & \multicolumn{1}{c}{\textbf{Semantic Similarity}} & \textbf{Sub Questions} \\
        \cmidrule(lr){2-5} \cmidrule(lr){6-6} \cmidrule(lr){7-7} 
        & BLEU & ROUGE-1 & ROUGE-2 & ROUGE-L & BERT Score & ROUGE-L \\
        \midrule
        \multicolumn{7}{l}{\textbf{Based on Llama3-8B-Instruct}} \\
        \midrule
        UnKE        & 93.56 / 78.09  & 93.61 / 79.26  & 91.42 / 71.73  & 93.33 / 78.42  & 98.34 / 93.38    & 37.87 \\
        UnKE+       & 99.67 / 84.60  & 99.69 / 86.31  & 99.57 / 81.18  & 99.68 / 85.75  & 99.86 / 94.70    & 41.45 \\
        MEMIT       & 25.57 / 22.88  & 32.67 / 30.75  & 14.51 / 12.37  & 30.49 / 28.65  & 76.21 / 74.25    & 22.56 \\
        MEMIT+      & 88.88 / 81.38  & 93.26 / 86.53  & 90.32 / 80.61  & 92.96 / 85.91  & 97.76 / 95.60    & 41.67 \\
        AlphaEdit   & 21.29 / 20.24  & 28.62 / 27.99  & 11.36 / 10.24  & 26.59 / 25.92  & 73.92 / 72.96    & 20.71 \\
        AlphaEdit+  & 75.02 / 66.35  & 81.70 / 73.47  & 74.35 / 62.74  & 80.92 / 72.22  & 94.19 / 91.51    & 40.56 \\
        \midrule
        \multicolumn{7}{l}{\textbf{Based on Qwen2.5-7B-Instruct}} \\
        \midrule
        UnKE        & 91.92 / 70.61  & 91.41 / 68.47  & 87.75 / 56.34  & 91.01 / 67.00  & 96.97 / 89.17    & 38.12 \\
        UnKE+       & 98.52 / 82.48  & 98.85 / 83.36  & 98.43 / 77.03  & 98.82 / 82.60  & 99.35 / 94.81    & 42.24 \\
        MEMIT       & 45.07 / 40.81  & 40.73 / 36.75  & 19.59 / 15.87  & 38.04 / 34.07  & 78.03 / 76.50    & 24.75 \\
        MEMIT+      & 91.31 / 77.23  & 95.10 / 80.88  & 92.93 / 72.50  & 94.89 / 79.98  & 98.05 / 93.56    & 42.38 \\
        AlphaEdit   & 49.71 / 45.21  & 45.42 / 41.06  & 24.63 / 19.85  & 42.77 / 38.26  & 80.48 / 78.38    & 25.37 \\
        AlphaEdit+  & 97.77 / 83.09  & 98.20 / 84.18  & 97.40 / 77.38  & 98.14 / 83.40  & 99.08 / 94.51    & 41.58 \\
        \bottomrule
    \end{tabular}
    }
    
\end{table*}

\begin{table*}[h]
\caption{Performance comparison in AKEW (Counterfact). The `+' symbol indicates results incorporating AnyEdit's autoregressive editing paradigm. The left side of `/` represents the LLM's edited output for original questions, while the right side represents the edited output for paraphrase questions.}
    \label{tab:app_2}
    \centering
    \renewcommand{\arraystretch}{1.2}
    \setlength{\tabcolsep}{4pt}
    \resizebox{\textwidth}{!}{
    \begin{tabular}{l cccc ccc}
        \toprule
        \multirow{2}{*}{\textbf{Method}} & \multicolumn{4}{c}{\textbf{Lexical Similarity}} & \multicolumn{1}{c}{\textbf{Semantic Similarity}} & \textbf{Sub Questions} \\
        \cmidrule(lr){2-5} \cmidrule(lr){6-6} \cmidrule(lr){7-7} 
        & BLEU & ROUGE-1 & ROUGE-2 & ROUGE-L & BERT Score & ROUGE-L \\
        \midrule
        \multicolumn{7}{l}{\textbf{Based on Llama3-8B-Instruct}} \\
        \midrule
        MEMIT       & 33.44 / 18.13  & 34.46 / 17.44  & 16.29 / 4.74   & 32.20 / 16.10  & 76.44 / 47.80  & 39.98\\
        MEMIT+      & 85.41 / 38.78  & 96.07 / 47.61  & 94.21 / 32.37  & 95.87 / 46.00  & 97.76 / 62.63  & 64.07\\
        UnKE        & 98.43 / 36.99  & 98.43 / 34.58  & 97.78 / 19.37  & 98.37 / 32.89  & 99.62 / 59.62  & 63.22\\
        UnKE+       & 99.98 / 45.23  & 99.98 / 46.57  & 99.96 / 35.41  & 99.98 / 45.31  & 99.95 / 64.24  & 59.03\\
        AlphaEdit   & 23.36 / 16.25  & 26.92 / 15.00  & 10.81 / 3.61   & 24.95 / 13.79  & 72.63 / 44.67  & 35.76 \\
        AlphaEdit+  & 79.60 / 40.67  & 84.49 / 41.11  & 78.00 / 26.60  & 83.76 / 39.51  & 96.51 / 65.14  & 57.05 \\
        \midrule
        \multicolumn{7}{l}{\textbf{Based on Qwen2.5-7B-Instruct}} \\
        \midrule
        MEMIT       & 45.29 / 32.83  & 41.68 / 28.01  & 20.38 / 8.79   & 38.95 / 25.73  & 77.19 / 56.04  & 43.51\\
        MEMIT+      & 90.55 / 44.32  & 95.33 / 45.56  & 93.12 / 27.38  & 95.09 / 43.49  & 98.08 / 65.40  & 55.10\\
        UnKE        & 91.53 / 38.59  & 90.91 / 31.53  & 87.06 / 12.11  & 90.44 / 29.27  & 97.34 / 59.29  & 49.97\\
        UnKE+       & 98.95 / 34.68  & 99.01 / 35.23  & 98.59 / 15.59  & 98.99 / 32.95  & 99.63 / 60.78  & 51.58\\
        AlphaEdit   & 49.97 / 34.65  & 48.15 / 30.02  & 27.76 / 10.38  & 45.55 / 27.69  & 80.66 / 56.99  & 45.12\\
        AlphaEdit+  & 97.61 / 46.97  & 97.80 / 47.63  & 96.89 / 30.31  & 97.73 / 45.84  & 99.10 / 66.10  & 54.99\\
        \bottomrule
    \end{tabular}
    }
    
\end{table*}

\subsection{Supplementary Experimental Results on RQ4}\label{app:exp_result_4}
\begin{figure}[t]
\begin{center}
\includegraphics[width=0.6\linewidth, keepaspectratio]{figures/exp_3.png}
\caption{The relationship between AnyEdit's editing performance and chunk size in long-form diverse-formatted knowledge.}
\label{fig:exp_3}
\end{center}
\end{figure}


 The experimental results of relationship between AnyEdit's editing performance and chunk size in long-form diverse-formatted knowledge are presented in Figure \ref{fig:exp_3}. Based on these results, we draw the following observation:.

\begin{itemize}[leftmargin=*]
    \item \textbf{Obs 7: The editing performance of AnyEdit is influenced by chunk size.}  
    As the chunk size increases beyond a certain threshold, the editing performance of AnyEdit declines. Specifically, when the chunk size is smaller, each iteration of editing becomes more manageable, leading to improved overall performance. However, this improvement comes at the cost of increased editing time due to the larger number of iterations required for longer texts. Conversely, when the chunk size is larger, it becomes challenging to achieve effective edits within a single iteration, resulting in degraded performance. Based on this trade-off, we recommend using a balanced chunk size of 40 for most editing scenarios.
\end{itemize}

\begin{figure}[h]
    \centering
    \includegraphics[width=\textwidth]{figures/data1.png}
    \vspace{-5mm}
    \caption{A Sample of the AKEW (Counterfact) dataset.}
    \label{fig:sample1}
\end{figure}

\begin{figure}[h]
    \centering
    \includegraphics[width=\textwidth]{figures/data2.png}
    \vspace{-5mm}
    \caption{A Sample of the UnKEBench dataset.}
    \label{fig:sample2}
\end{figure}

\begin{figure}[h]
    \centering
    \includegraphics[width=\textwidth]{figures/data3.png}
    \vspace{-5mm}
    \caption{Samples of the EditEverything dataset.}
    \label{fig:sample3}
\end{figure}
\end{document}
