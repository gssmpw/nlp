\documentclass{article}

\title{The late-stage training dynamics of (stochastic) subgradient descent on homogeneous neural networks}

\usepackage{authblk}

\author[1]{Sholom Schechtman}
\author[2]{Nicolas Schreuder}
\affil[1]{SAMOVAR, Télécom SudParis, Institut Polytechnique de Paris}
\affil[2]{CNRS, LIGM}

\usepackage{times}

\usepackage{a4wide}

\usepackage{amsmath, amsthm, amssymb}

\usepackage{shortcuts}

\usepackage{mathtools}
\usepackage{natbib}
\bibliographystyle{plainnat}
\setcitestyle{authoryear}

\usepackage{url}
\usepackage[colorlinks=true, linkcolor=blue, citecolor=blue]{hyperref}


\DeclareMathOperator{\Jac}{Jac}
\newcommand{\tgamma}{\tilde{\gamma}}
\newcommand{\tv}{\tilde{v}}
\newcommand{\bg}{\bar{g}}
\newcommand{\bgamma}{\bar{\gamma}}
\DeclareMathOperator{\ReLU}{ReLU}
\DeclareMathOperator{\LeakyReLU}{LeakyReLU}
\newtheorem{assumption}{Assumption}
\usepackage{color}
\usepackage{csquotes}

\usepackage{comment}
\usepackage{enumitem}

\newtheorem{definition}{Definition}
\newtheorem{proposition}{Proposition}
\newtheorem{theorem}{Theorem}
\newtheorem{lemma}{Lemma}
\newtheorem{remark}{Remark}



\begin{document}

\maketitle

\begin{abstract}%
  We analyze the implicit bias of constant step stochastic subgradient descent (SGD). We consider the setting of binary classification with homogeneous neural networks -- a large class of deep neural networks with $\ReLU$-type activation functions such as MLPs and CNNs without biases. We interpret the dynamics of normalized SGD iterates as an Euler-like discretization of a conservative field flow that is naturally associated to the normalized classification margin. Owing to this interpretation, we show that normalized SGD iterates converge to the set of critical points of the normalized margin at late-stage training (i.e., assuming that the data is correctly classified with positive normalized margin). 
  Up to our knowledge, this is the first extension of the analysis of \cite{Lyu_Li_maxmargin} on the discrete dynamics of gradient descent to the nonsmooth and stochastic setting. Our main result applies to binary classification with exponential or logistic losses. We additionally discuss extensions to more general settings. 
\end{abstract}


\section{Introduction}
\section{Introduction}


\begin{figure}[t]
\centering
\includegraphics[width=0.6\columnwidth]{figures/evaluation_desiderata_V5.pdf}
\vspace{-0.5cm}
\caption{\systemName is a platform for conducting realistic evaluations of code LLMs, collecting human preferences of coding models with real users, real tasks, and in realistic environments, aimed at addressing the limitations of existing evaluations.
}
\label{fig:motivation}
\end{figure}

\begin{figure*}[t]
\centering
\includegraphics[width=\textwidth]{figures/system_design_v2.png}
\caption{We introduce \systemName, a VSCode extension to collect human preferences of code directly in a developer's IDE. \systemName enables developers to use code completions from various models. The system comprises a) the interface in the user's IDE which presents paired completions to users (left), b) a sampling strategy that picks model pairs to reduce latency (right, top), and c) a prompting scheme that allows diverse LLMs to perform code completions with high fidelity.
Users can select between the top completion (green box) using \texttt{tab} or the bottom completion (blue box) using \texttt{shift+tab}.}
\label{fig:overview}
\end{figure*}

As model capabilities improve, large language models (LLMs) are increasingly integrated into user environments and workflows.
For example, software developers code with AI in integrated developer environments (IDEs)~\citep{peng2023impact}, doctors rely on notes generated through ambient listening~\citep{oberst2024science}, and lawyers consider case evidence identified by electronic discovery systems~\citep{yang2024beyond}.
Increasing deployment of models in productivity tools demands evaluation that more closely reflects real-world circumstances~\citep{hutchinson2022evaluation, saxon2024benchmarks, kapoor2024ai}.
While newer benchmarks and live platforms incorporate human feedback to capture real-world usage, they almost exclusively focus on evaluating LLMs in chat conversations~\citep{zheng2023judging,dubois2023alpacafarm,chiang2024chatbot, kirk2024the}.
Model evaluation must move beyond chat-based interactions and into specialized user environments.



 

In this work, we focus on evaluating LLM-based coding assistants. 
Despite the popularity of these tools---millions of developers use Github Copilot~\citep{Copilot}---existing
evaluations of the coding capabilities of new models exhibit multiple limitations (Figure~\ref{fig:motivation}, bottom).
Traditional ML benchmarks evaluate LLM capabilities by measuring how well a model can complete static, interview-style coding tasks~\citep{chen2021evaluating,austin2021program,jain2024livecodebench, white2024livebench} and lack \emph{real users}. 
User studies recruit real users to evaluate the effectiveness of LLMs as coding assistants, but are often limited to simple programming tasks as opposed to \emph{real tasks}~\citep{vaithilingam2022expectation,ross2023programmer, mozannar2024realhumaneval}.
Recent efforts to collect human feedback such as Chatbot Arena~\citep{chiang2024chatbot} are still removed from a \emph{realistic environment}, resulting in users and data that deviate from typical software development processes.
We introduce \systemName to address these limitations (Figure~\ref{fig:motivation}, top), and we describe our three main contributions below.


\textbf{We deploy \systemName in-the-wild to collect human preferences on code.} 
\systemName is a Visual Studio Code extension, collecting preferences directly in a developer's IDE within their actual workflow (Figure~\ref{fig:overview}).
\systemName provides developers with code completions, akin to the type of support provided by Github Copilot~\citep{Copilot}. 
Over the past 3 months, \systemName has served over~\completions suggestions from 10 state-of-the-art LLMs, 
gathering \sampleCount~votes from \userCount~users.
To collect user preferences,
\systemName presents a novel interface that shows users paired code completions from two different LLMs, which are determined based on a sampling strategy that aims to 
mitigate latency while preserving coverage across model comparisons.
Additionally, we devise a prompting scheme that allows a diverse set of models to perform code completions with high fidelity.
See Section~\ref{sec:system} and Section~\ref{sec:deployment} for details about system design and deployment respectively.



\textbf{We construct a leaderboard of user preferences and find notable differences from existing static benchmarks and human preference leaderboards.}
In general, we observe that smaller models seem to overperform in static benchmarks compared to our leaderboard, while performance among larger models is mixed (Section~\ref{sec:leaderboard_calculation}).
We attribute these differences to the fact that \systemName is exposed to users and tasks that differ drastically from code evaluations in the past. 
Our data spans 103 programming languages and 24 natural languages as well as a variety of real-world applications and code structures, while static benchmarks tend to focus on a specific programming and natural language and task (e.g. coding competition problems).
Additionally, while all of \systemName interactions contain code contexts and the majority involve infilling tasks, a much smaller fraction of Chatbot Arena's coding tasks contain code context, with infilling tasks appearing even more rarely. 
We analyze our data in depth in Section~\ref{subsec:comparison}.



\textbf{We derive new insights into user preferences of code by analyzing \systemName's diverse and distinct data distribution.}
We compare user preferences across different stratifications of input data (e.g., common versus rare languages) and observe which affect observed preferences most (Section~\ref{sec:analysis}).
For example, while user preferences stay relatively consistent across various programming languages, they differ drastically between different task categories (e.g. frontend/backend versus algorithm design).
We also observe variations in user preference due to different features related to code structure 
(e.g., context length and completion patterns).
We open-source \systemName and release a curated subset of code contexts.
Altogether, our results highlight the necessity of model evaluation in realistic and domain-specific settings.






\section{Related works}\label{sec:rw}
\putsec{related}{Related Work}

\noindent \textbf{Efficient Radiance Field Rendering.}
%
The introduction of Neural Radiance Fields (NeRF)~\cite{mil:sri20} has
generated significant interest in efficient 3D scene representation and
rendering for radiance fields.
%
Over the past years, there has been a large amount of research aimed at
accelerating NeRFs through algorithmic or software
optimizations~\cite{mul:eva22,fri:yu22,che:fun23,sun:sun22}, and the
development of hardware
accelerators~\cite{lee:cho23,li:li23,son:wen23,mub:kan23,fen:liu24}.
%
The state-of-the-art method, 3D Gaussian splatting~\cite{ker:kop23}, has
further fueled interest in accelerating radiance field
rendering~\cite{rad:ste24,lee:lee24,nie:stu24,lee:rho24,ham:mel24} as it
employs rasterization primitives that can be rendered much faster than NeRFs.
%
However, previous research focused on software graphics rendering on
programmable cores or building dedicated hardware accelerators. In contrast,
\name{} investigates the potential of efficient radiance field rendering while
utilizing fixed-function units in graphics hardware.
%
To our knowledge, this is the first work that assesses the performance
implications of rendering Gaussian-based radiance fields on the hardware
graphics pipeline with software and hardware optimizations.

%%%%%%%%%%%%%%%%%%%%%%%%%%%%%%%%%%%%%%%%%%%%%%%%%%%%%%%%%%%%%%%%%%%%%%%%%%
\myparagraph{Enhancing Graphics Rendering Hardware.}
%
The performance advantage of executing graphics rendering on either
programmable shader cores or fixed-function units varies depending on the
rendering methods and hardware designs.
%
Previous studies have explored the performance implication of graphics hardware
design by developing simulation infrastructures for graphics
workloads~\cite{bar:gon06,gub:aam19,tin:sax23,arn:par13}.
%
Additionally, several studies have aimed to improve the performance of
special-purpose hardware such as ray tracing units in graphics
hardware~\cite{cho:now23,liu:cha21} and proposed hardware accelerators for
graphics applications~\cite{lu:hua17,ram:gri09}.
%
In contrast to these works, which primarily evaluate traditional graphics
workloads, our work focuses on improving the performance of volume rendering
workloads, such as Gaussian splatting, which require blending a huge number of
fragments per pixel.

%%%%%%%%%%%%%%%%%%%%%%%%%%%%%%%%%%%%%%%%%%%%%%%%%%%%%%%%%%%%%%%%%%%%%%%%%%
%
In the context of multi-sample anti-aliasing, prior work proposed reducing the
amount of redundant shading by merging fragments from adjacent triangles in a
mesh at the quad granularity~\cite{fat:bou10}.
%
While both our work and quad-fragment merging (QFM)~\cite{fat:bou10} aim to
reduce operations by merging quads, our proposed technique differs from QFM in
many aspects.
%
Our method aims to blend \emph{overlapping primitives} along the depth
direction and applies to quads from any primitive. In contrast, QFM merges quad
fragments from small (e.g., pixel-sized) triangles that \emph{share} an edge
(i.e., \emph{connected}, \emph{non-overlapping} triangles).
%
As such, QFM is not applicable to the scenes consisting of a number of
unconnected transparent triangles, such as those in 3D Gaussian splatting.
%
In addition, our method computes the \emph{exact} color for each pixel by
offloading blending operations from ROPs to shader units, whereas QFM
\emph{approximates} pixel colors by using the color from one triangle when
multiple triangles are merged into a single quad.



\section{Preliminaries}\label{sec:preli}
% \section{Domain Transform Methods}

% This section provides a concise overview of three fundamental transform methods: Fourier Transform, Laplace Transform, and Wavelet Transform.

% \noindent \textbf{Fourier Transform}. Fourier transform~\cite{nussbaumer1982fast} is a fundamental tool for analyzing the frequency content of signals by decomposing a time-domain signal into sinusoidal components, revealing the frequencies and their magnitudes. 
% And the process of Fourier Transform is shown as follows:
% \begin{equation}
% \begin{aligned}
% X(f) = \int_{-\infty}^{\infty} x(t)e^{-j2\pi ft} dt, \quad j = \sqrt{-1}
% \end{aligned}
% \end{equation}
% where $X(f)$ is the continuous frequency spectrum, $x(t)$ is the continuous-time signal, and $f$ is the continuous frequency variable.


% The Continuous-Time Fourier Transform (CTFT) provides the continuous frequency spectrum for continuous-time signals, while the Discrete Fourier Transform (DFT) gives a discrete frequency representation for discrete-time signals. For computational efficiency, the Fast Fourier Transform (FFT) reduces the direct DFT's $\mathcal{O}(N^2)$ complexity to $\mathcal{O}(N \log N)$, making it essential in fields like signal processing and telecommunications. The FFT achieves this speedup through recursive decomposition, leveraging symmetries in trigonometric calculations to facilitate rapid frequency-domain analysis, which is critical in applications requiring real-time data processing.



% \noindent \textbf{Laplace Transform}. Laplace transform~\cite{schiff2013laplace} is a crucial tool for analyzing linear time-invariant (LTI) systems, converting time-domain functions into functions of a complex frequency variable $s$. The unilateral (one-sided) Laplace transform of a function $f(t)$, defined for $t \ge 0$, is shown:

% \begin{equation} % \label{eq:laplace}
%     F(s) = \mathcal{L}\{f(t)\} = \int_0^\infty f(t)e^{-st} dt
% \end{equation}
% where $F(s)$ represents the Laplace transform of $f(t)$. $s$ is a complex frequency variable, often expressed as $s = \sigma + j\omega$, with $\sigma$ representing the real part (related to exponential decay) and $\omega$ the imaginary part (related to frequency). The integral's lower limit of 0 reflects the typical application to causal signals (signals that are zero for $t < 0$).




% \noindent \textbf{Wavelet Transform}. The wavelet transform~\cite{meyer1989wavelets} is a mathematical tool used in signal processing that allows for the decomposition of a signal in the time-frequency domain. The following equation shows the CWT, which decomposes a continuous signal $f(t)$ into frequency components localized in time:
% \begin{equation}
% \begin{aligned}
%     F(\tau, s) = \frac{1}{\sqrt{\vert s \vert}} \int_{-\infty}^{\infty} f(t) \psi^* \left( \frac{t - \tau}{s} \right) dt
% \end{aligned}
% \end{equation}
% where $\tau$ is the translation parameter, determining the position of the wavelet along the time axis, and $s$ is the scale parameter, which controls the width of the wavelet and thus affects the frequency resolution.

% Unlike the Fourier transform, the wavelet transform can reveal both the frequency characteristics of a signal and the time distribution of these frequency components, making it particularly effective for analyzing non-stationary signals. According to the computation methods, wavelet transforms can be divided into the continuous wavelet transform (CWT) and discrete wavelet transform (DWT). The CWT computes wavelet coefficients by analyzing a signal across different frequencies and time positions, providing a detailed energy distribution at the expense of high computational cost. The DWT performs multi-scale decomposition of a signal, separating it into different frequency bands while maintaining both frequency and time localization. Unlike the CWT, the DWT is computationally efficient and suitable for digital signal processing. 


\section{Problem Definition}
\label{sec:prob}
The input consists of a long-term time series $\mathcal{X} = (x_1, \dots, x_L) \in \mathbb{R}^{L \times V}$, where $L$ is the historical window length and $V$ is the number of variables. The corresponding ground truth for the prediction is $\mathcal{Y} = (x_{L+1}, \dots, x_{L+H}) \in \mathbb{R}^{H \times V}$, with $H$ representing the prediction horizon.

\noindent \textbf{Frequency Transform.} To more effectively capture periodic patterns inherent in time series data, numerous studies have employed transformations that convert the data into the frequency domain. Formally, we denote the frequency domain transformation by a generic operator $\text{FT}(\cdot)$, defined as follows:
\begin{equation}
\begin{aligned}
    \mathbf{X}' = \text{FT}(\mathcal{X})
\end{aligned}
\end{equation}
The primary objective of learning in the frequency domain is to capture periodic information in time series while preserving temporal dependencies. We provide a pipeline for time series analysis through frequency transformation in Figure~\ref{fig:intro}. 

\section{Problem setting}\label{sec:sett}
We study (stochastic) gradient descent on the empirical risk
\begin{equation*}
\cL(w) = \frac{1}{n}\sum_{i=1}^n l(p_i(w))\, ,
\end{equation*}
where the loss function $l$ and the functions  $(p_i)_{i=1}^n$  are specified in the following assumptions. Note that the empirical risk for binary classification from Equation~\eqref{def:emp_risk_intro} is a special case of the above objective.

\begin{assumption}\label{hyp:loss_exp_log}\phantom{=}
  \begin{enumerate}[label=\roman*)]
    \item The loss is either the exponential loss, $l(q) = e^{-q}$, or the logistic loss, $l(q) = \log(1{+}e^{-q})$.
    \item There exists an integer $L \in \mathbb{N}^*$  such that, for all $1 \leq i \leq n$, the function $p_i$ is $L$-homogeneous\footnote{We recall that a mapping $f : \mathbb{R}^d \rightarrow \mathbb{R}$ is positively $L$-homogeneous if $f(\lambda w) = \lambda^L f(w)$ for all $w \in \mathbb{R}^d$ and $\lambda >0$.}, locally Lipschitz continuous and semialgebraic.
  \end{enumerate}
\end{assumption}
If the $p_i$'s were differentiable with respect to $w$, the chain rule would guarantee that
\begin{align*}
\nabla \mathcal{L}(w) = \frac{1}{n}\sum_{i=1}^n  l'(p_i(w)) \nabla p_i(w)\enspace.
\end{align*}
However, we only assume that the $p_i$'s are semialgebraic. While we could consider Clarke subgradients, the Clarke subgradient of operations on functions (e.g., addition, composition, and minimum) is only contained within the composition of the respective Clarke subgradients. This, as noted in Section~\ref{sec:cons_field}, implies that the output of backpropagation is usually not an element of a Clarke subgradient but a selection of some conservative set-valued field.
Consequently, for $1\leq i \leq n$, we consider $D_i : \bbR^d \rightrightarrows\bbR^d$, a conservative set-valued field of $p_i$, and a function $\sa_i : \bbR^d \rightarrow \bbR^d$ such that for all $w \in \bbR^d$, $\sa_i(w) \in D_i(w)$. Given a step-size $\gamma >0$, gradient descent (GD)\footnote{More precisely, this refers to conservative gradient descent. We use the term GD for simplicity, as conservative gradients behave similarly to standard gradients.} is then expressed as
\begin{equation*}\label{eq:gd_new}\tag{GD}
  w_{k+1} = w_k - \frac{\gamma}{n} \sum_{i=1}^n l'(p_i(w_k))\sa_i(w_k)\,.
\end{equation*}
For its stochastic counterpart, stochastic gradient descent (SGD), we fix a batch-size $1\leq n_b \leq n$. At each iteration $k \in \bbN$, we randomly and uniformly draw a batch $B_k \subset \{1, \ldots, n \}$ of size $n_b$. The update rule is then given by 
\begin{equation*}\label{eq:sgd_new}\tag{SGD}
  w_{k+1} = w_k -  \frac{\gamma}{n_b}\sum_{i\in B_k} l'(p_i(w_k)) \sa_i(w_k)\, .
\end{equation*}
The considered conservative set-valued fields will satisfy an Euler lemma-type assumption.
%\nic{Smoother transition}
\begin{assumption}\phantom{=}\label{hyp:conserv}
  For every $i \leq n$, $\sa_i$ is measurable and $D_i$ is semialgebraic. Moreover, for every $w \in \bbR^d$ and $\lambda \geq 0$, $\sa_i(w)  \in D_i(w)$,
  \begin{equation*}
    D_i(\lambda w) = \lambda^{L-1} D_i(w)\, , \textrm{ and } \quad   L p_i(w) = \scalarp{\sa_i(w)}{w}\, .
  \end{equation*}
\end{assumption}
%\nic{Smoother transition}
Having in mind the binary classification setting, in which $p_i(w) = y_i \Phi(x_i, w)$, we define the margin
\begin{equation}\label{def:marg}
  \sm: \bbR^d \rightarrow \bbR, \quad \sm(w) = \min_{1\leq i \leq n} p_i(w)\, .
\end{equation}
It quantifies the quality of a prediction rule $\Phi(\cdot, w)$. In particular,  the training data is perfectly separated when $\sm(w) >0$. A binary prediction for $x$ is given by the sign of $\Phi(x, w)$, and under the homogeneity assumption, it depends only on the normalized direction $w / \norm{w}$. Consequently, we will focus on the sequence of directions $u_k := w_k / \norm{w_k}$. Our final assumption ensures that the normalized directions $(u_k)$ have stabilized in a region where the training data is correctly classified.

\begin{assumption}\label{hyp:marg_lowb}
  Almost surely, $\liminf \sm(u_k) >0$.
\end{assumption}
Before presenting our main result, we comment on our assumptions.

\paragraph{On Assumption~\ref{hyp:loss_exp_log}.} As discussed in the introduction, the primary example we consider is when $p_i(w) = y_i \Phi(x_i;w)$ is the signed prediction of a feedforward neural network without biases and with piecewise linear activation functions on a labeled dataset $((x_i,y_i))_{i \leq n}$. In this case,
\begin{equation}\label{eq:NN}
 p_i(w) = y_i \Phi(w;x_i) = y_i V_L(W_L) \sigma(V_{L-1}(W_{L-1}) \sigma(V_{L-1}(W_{L-2}) \ldots \sigma(V_{1}(W_1 x_i))))\, ,
\end{equation}
where $w = [W_1, \ldots, W_L]$, $W_i$ represents the weights of the $i$-th layer, $V_i$ is a linear function in the space of matrices (with $V_i$ being the identity for fully-connected layers) and $\sigma$ is a coordinate-wise activation function such as $z \mapsto \max(0,z)$ ($\ReLU$), $z \mapsto \max(az, z)$ for a small parameter $a>0$ (LeakyReLu) or $z \mapsto z$. Note that the mapping $w \mapsto p_i(w)$ is semialgebraic and $L$-homogeneous for any of these activation functions. Regarding the loss functions, the logistic and exponential losses are among the most commonly studied and widely used. In Appendix~\ref{app:gen_sett}, we extend our results to a broader class of losses, including $l(q) = e^{-q^a}$ and $l(q) = \ln (1 + e^{-q^a})$ for any $a \geq 1$.

\paragraph{On Assumption~\ref{hyp:conserv}.} Assumption~\ref{hyp:conserv} holds automatically  if $D_i$ is the Clarke subgradient of $p_i$. Indeed, at any vector $w \in \bbR^d$, where $p_i$ is differentiable it holds that $p_i(\lambda w) = \lambda^{L} p_i(w)$. Differentiating relatively to $w$ and $\lambda$ (noting that $p_i$ remains differentiable at $\lambda w$ due to homogeneity), we obtain $\lambda \nabla p_i(\lambda w) = \lambda^{L} \nabla p_i(w)$ and $\scalarp{\nabla p_i(\lambda w)}{w} = L \lambda^{L-1} p_i(w)$. The expression for any element of the Clarke subgradient then follows from~\eqref{eq:def_clarke}. 

However, for an arbitrary conservative set-valued field, Assumption~\ref{hyp:conserv} does not necessarily hold. For instance, $D(x) = \mathds{1}(x \in \mathbb{N})$ is a conservative set-valued field for $p \equiv 0$, which does not satisfy Assumption~\ref{hyp:conserv}. Nevertheless, in practice, conservative set-valued fields naturally arise from a formal application of the chain rule. For a non-smooth but homogeneous activation function $\sigma$, one selects an element $e \in \partial \sigma (0)$, and computes $\sa_i(w)$ via backpropagation. Whenever a gradient candidate of $\sigma$ at zero is required (i.e., in~\eqref{eq:NN}, for some $j$, $V_j(W_j)$ contains a zero entry), it is replaced by $e$. 
Since $V_j(W_j)$ and $V_j(\lambda W_j)$ have the same zero elements, it follows that for every such $w$, $
\sa_i(\lambda w) = \lambda^L \sa_i(w)$. The conservative set-valued field $D_i$ is then obtained by associating to each $w$ the set of all possible outcomes of the chain rule, with $e$ ranging over all elements of $\partial \sigma(0)$. Thus, for such fields, Assumption~\ref{hyp:conserv} holds.


\paragraph{On Assumption~\ref{hyp:marg_lowb}.} Training typically continues even after the training error reaches zero.
Assumption~\ref{hyp:marg_lowb} characterizes this late-training phase, where our result applies. 
As noted earlier, since $\sm$ is $L$-homogeneous, the classification rule is determined by the direction of the  iterates $u_k=w_k/\norm{w_k}$. Assumption~\ref{hyp:marg_lowb} then states that, beyond some iteration, the normalized margin remains positive. 
This assumption is natural in the context of studying the implicit bias of SGD: we \emph{assume} that we reached the phase in which the dataset is correctly classified and \emph{then} characterize the limit points. A similar perspective was taken in  \cite{nacson2019lexicographic}, where the implicit bias of GF was analyzed under the assumption that the sequence of directions and the loss converge. However, unlike their approach, ours does not require assuming such convergence a priori.

Earlier works such as \cite{ji2020directional,Lyu_Li_maxmargin}, which analyze subgradient flow or smooth GD, establish convergence by assuming the existence of a single iterate $w_{k_0}$ satisfying $\sm(w_{k_0}) > \varepsilon$ and then proving that $\lim \sm(u_{k}) > 0$. Their approach relies on constructing a smooth approximation of the margin, which increases during training, ensuring that $\sm(u_k) > 0$ for all iterates with $k \geq k_0$. This is feasible in their setting, as they study either subgradient flow or GD with smooth $p_i$’s, allowing them to leverage the descent lemma.

In contrast, our analysis considers a nonsmooth and stochastic setting, in which, even if an iterate $w_{k_0}$ satisfying $\sm(w_{k_0}) > \varepsilon$ exists, there is no a priori assurance that subsequent iterates remain in the region where Assumption~\ref{hyp:marg_lowb} holds. From this perspective, Assumption~\ref{hyp:marg_lowb} can be viewed as a stability assumption, ensuring that iterates continue to classify the dataset correctly. Establishing stability for stochastic and nonsmooth algorithms is notoriously hard, and only partial results in restrictive settings exist \cite{borkar2000ode,ramaswamy2017generalization,josz2024global}.

%Finally, note that Assumption~\ref{hyp:marg_lowb} only needs to hold almost surely. Specifically, with probability 1, there exist $k_0$ and $\varepsilon$ such that for all $k \geq k_0$, $\sm(u_k) \geq \varepsilon > 0$. In the case of~\eqref{eq:sgd_new}, $k_0$ and $\delta$ are random variables and may take different values across different realizations. 

%\paragraph{On constant stepsizes.}
%We allow the step size to be a constant of arbitrary magnitude, subject to the stability Assumption~\ref{hyp:marg_lowb}. This may seem surprising in a nonsmooth and stochastic setting, where a vanishing step size is typically required to ensure convergence (see, e.g., \cite{majewski2018analysis, dav-dru-kak-lee-19, bolte2023subgradient, le2024nonsmooth}).

\section{Main result}\label{sec:main}
%We begin by showing that the sequence of directions follows a dynamic that can be interpreted as a stochastic approximation of the reversed (Riemannian) conservative field flow of $\sm$ restricted to the sphere $\bbS^d$ in Proposition~\ref{prop:stoch_approx_exp_log}. This observation directly leads to our main result, Theorem~\ref{thm:main}, which follows as a consequence of recent stochastic approximation results from \cite{benaim_05_DI_1,dav-dru-kak-lee-19}.

As a first step toward our main result, we establish that the iterates norm $(\lVert w_k\rVert)$ grows to infinity at a logarithmic rate. This is consistent with \cite[Theorem 4.3]{Lyu_Li_maxmargin}.
\begin{proposition}\label{prop:log_wk}
  Under Assumptions~\ref{hyp:loss_exp_log}--\ref{hyp:marg_lowb}, almost surely, there exist $c_1, c_2, \varepsilon>0$ and $k_0 \in \bbN$, such that for all $k \geq k_0$, $\norm{w_k}$ increases and
  \begin{equation*}
    c_1 \log (k)\leq \norm{w_k}^L \leq c_2 \log(k) \quad \textrm{ and } \quad 0 < \cL(w_k) \leq k^{-\varepsilon c_1}\, .
  \end{equation*}
  In particular, $\norm{w_k}\rightarrow + \infty$ and $\cL(w_k) \rightarrow 0$.
\end{proposition}
\begin{proof}[Sketch, full proof in Appendix~\ref{sec:pf_logwk}.] The proof follows from the next observations, which hold almost surely, for $k$ large enough. \emph{(i)} There is $\varepsilon >0$, such that $\sm(u_k) \geq \varepsilon$. In particular, there are $M, C_1, C_2>0$, such that $ C_2 e^{-M \norm{w_k}^L}\leq -l'(p_i(w_k)) \leq C_1e^{-\varepsilon \norm{w_k}^L}$. \emph{(ii)} As a result, there is $C_3 >0$ such that
  \begin{equation}\label{eq:lwb_wk2}
    \norm{w_{k+1}}^2 \geq \norm{w_k}^2 ( 1 + C_3 \gamma e^{-M \norm{w_k}^L } \norm{w_k}^{L-2})\, ,
  \end{equation}
  which implies that $\norm{w_k}$ is increasing to infinity and that there is $C_4 >0$ such that
  \begin{equation}\label{eq:uwb_wk2}
    \norm{w_{k+1}}^2 \leq \norm{w_k}^2 (1 + C_4 \gamma e^{-\varepsilon \norm{w_k}^L}\norm{w_k}^{L-2})\, .
  \end{equation}\emph{(iii)} Finally, using~\eqref{eq:lwb_wk2}--\eqref{eq:uwb_wk2} and the Taylor's expansion of $(1+x)^{L/2}$ near zero, we obtain existence of constants $C_5, C_6, a>0$, such that for $k$ large enough,
  \begin{equation*}
    C_5 \gamma \leq e^{a \norm{w_k}^L}\left(\norm{w_{k+1}}^L - \norm{w_k}^L\right) \leq C_6 \gamma \, .
  \end{equation*}
Summing these inequalities from $k$ to $k+N$ and noticing that the expression in the middle is comparable to the integral of $e^{at}$ between $\norm{w_{k}}^L$ and $\norm{w_{k+N}}^L$, concludes the proof.
\end{proof}
Define the set-valued map $\bar{D} : \bbR^d \rightrightarrows \bbR^d$ as 
\begin{equation}\label{eq:avg_consfiel}
  \bar{D}(x) = \conv \{v: v \in D_i(w) \, , \textrm{with $i \in I(w)$} \}\,, \quad \textrm{  where $I(w) = \{ i: p_i(w) = \sm(w)\}$}\, .
\end{equation}
As shown in Appendix~\ref{app:conserv}, it is a conservative set-valued field for the potential $\sm$. Note that, following Remark~\ref{rmk:max_subg}, even if for all $i$, $D_i = \partial p_i$, $\bar{D}$ can be different from $\partial \sm$. Next, we define the set-valued field $\bar{D}_s: \bbS^{d-1} \rightrightarrows \bbR^d$ as
\begin{equation}\label{def:riem_cons}
\bar{D}_{s}(u) := \{ v - \scalarp{v}{u}u : v \in \bar{D}(u) \}\, .
\end{equation}
The associated set of critical points is then given by
\begin{equation}\label{def:riem_crit}
  \cZ_s := \{ u \in \bbS^{d-1} : 0 \in \bar{D}_s(u) \} \subset \bbS^{d-1}\, .
\end{equation}
The field $\bar{D}_s$ and the critical points set $\cZ_s$ admit a straightforward interpretation. If $\sm$ is $C^1$ around some point $u \in \bbS^{d-1}$ and $\bar{D} =\{ \nabla \sm\}$, then $\bar{D}_s(u)$ is the radial component of $\nabla \sm(u)$, corresponding to its projection onto the tangent plane of $\bbS^{d-1}$ at $u$.
From a Riemannian geometry perspective, this implies that $\bar{D}_s(u)$ is the Riemannian gradient\footnote{Here, the Riemannian structure is implicitly induced from the ambient space.} of $\sm$ \emph{restricted} to the sphere $\bbS^{d-1}$, $\sm_{|\bbS^{d-1}}$. Similarly, $\cZ_s$ corresponds to the set of critical points of $\sm_{|\bbS^{d-1}}$.
More generally, since conservative fields are gradient-like objects (see Proposition~\ref{prop:var_strat_cons} in Appendix~\ref{app:omin}), we interpret $\bar{D}s$ as the Riemannian conservative field of $\sm_{|\bbS^{d-1}}$, with $\cZ_s$ as its corresponding critical points\footnote{As noted in \cite[Page 4, footnote]{bolte2021conservative}, the concept of a conservative set-valued field extends naturally to functions defined on any complete Riemannian submanifold, including $\bbS^{d-1}$.}.


We will consider the differential inclusion (DI) associated to set-valued field $\bar{D}_s$,
\begin{equation*}\label{eq:DI_sphere}\tag{DI}
\dot{\su}(t) \in \bar{D}_{s}(\su(t))\,.
\end{equation*}
Under the aforementioned interpretation, this corresponds to the reversed gradient (or conservative field) flow of $\sm_{|\bbS^{d-1}}$. 

We now show that the iterates’ directions evolve according to a dynamic that approximates~\eqref{eq:DI_sphere} via an Euler-like discretization (or stochastic approximation). The proof is deferred to Section~\ref{pf:sto_app_explog}.
%To treat the stochastic setting, we denote by $\cF_k$ the sigma-algebra generated by $\{ w_0,\ldots,w_k\}$. Notice that it is a filtration ($\cF_k \subset \cF_{k+1}$ for all $k$) and recall that a sequence $(a_k)$ is said to be adapted to $(\cF_k)$ if for every $k$, $a_k$ is $\cF_k$-measurable. Finally, we recall the notation $u_k=w_k/\norm{w_k}$.
  \begin{proposition}\label{prop:stoch_approx_exp_log}
    Let Assumptions~\ref{hyp:loss_exp_log}--\ref{hyp:marg_lowb} hold. There exist sequences $(\bg_k^s), (r_k), (\bgamma_k), (\bar{\eta}_{k+1})$ such that, for both~\eqref{eq:gd_new} and~\eqref{eq:sgd_new}, the normalized direction iterates $u_k \coloneqq w_k/\lVert w_k \rVert$ satisfy
    \begin{equation}\label{eq:stoch_app_u}
      u_{k+1} = u_k + \bgamma_k\bg_k^s + \bgamma_k \bar{\eta}_{k+1} + \bgamma_k^2 r_k\, .
    \end{equation}
    Moreover, considering the filtration $(\cF_k)_k$ where, for $k \in \mathbb{N}$, $\cF_k$ the sigma-algebra generated by $\{ w_0,\ldots,w_k\}$, the following holds:
    \begin{enumerate}
      \item\label{pr_res:rk} The sequence $(r_k)$ satisfies $\sup_{k}\norm{r_k} < + \infty$ almost surely.
      \item\label{pr_res:gammak} The sequence $(\bgamma_k)$ is positive and adapted to $(\cF_k)$. Moreover, $\sum_{k} \bgamma_k = + \infty$, and, almost surely, there is $c_3>0$ such that for sufficiently large $k$, $\bgamma_k \leq k^{-c_3}$.
      \item\label{pr_res:etak} For~\eqref{eq:gd_new}, $\bar{\eta}_{k} \equiv 0$. Otherwise, the sequence $(\bar{\eta}_{k})$ is adapted to $(\cF_k)$ and satisfies 
      \begin{equation*}
      \bbE[\bar{\eta}_{k+1} |\cF_k] = 0 \,.
      \end{equation*}
      Additionally, there exists a deterministic constant $c_4>0$ such that $\sup_{k} \norm{\bar{\eta}_{k+1}} < c_4$.
      \item\label{pr_res:barD} For any unbounded sequence $(k_j)_j$, such that $u_{k_j} \to u \in \bbS^{d-1}$, $\dist(\bar{D}_s(u), \bg^s_{k_j}) \rightarrow 0$. 
    \end{enumerate}
  \end{proposition}
    Since Proposition~\ref{prop:stoch_approx_exp_log} allows us to interpret $(u_k)$ as a discretization of~\eqref{eq:DI_sphere}, it is natural to investigate the convergence properties of a solution of its continuous counterpart~\eqref{eq:DI_sphere}.
  If $\su$ is such solution, then for almost every $t \in \bbR$, there exists $v \in \bar{D}_s(\su(t))$ such that $\dot{\su}(t) = v - \scalarp{v}{u}u$. Thus, by Definition~\ref{def:cons_f}, for almost every $t$, $\frac{\dif }{\dif t}  \sm(\su(t)) = \scalarp{\dot{\su}(t)}{v} = \norm{\dot{\su}(t)}^2$. Therefore, for $T >0$, we obtain
  \begin{equation*}
    \sm(\su(T)) - \sm(\su(0)) = \int_{0}^{T} \norm{\dot{\su}(t)}^2 \dif t \, .
  \end{equation*}
  This implies that $\sm(\su(T)) \geq \sm(\su(0))$, with strict inequality whenever $\su(0) \not \in \cZ_s$. In dynamical systems terminology, $-\sm$ is a Lyapunov function for~\eqref{eq:DI_sphere}. In particular, it can be shown that any solution $\su(t)$ to~\eqref{eq:DI_sphere} converges to $\cZ_s$.

Our main result, Theorem~\ref{thm:main}, establishes that the same holds true for the sequence of normalized directions: any limit point of $(u_k)$ is contained in $\cZ_s$. As discussed below, this result generalizes \cite[Theorem 4.4]{Lyu_Li_maxmargin} to stochastic gradient descent in the nonsmooth setting.

\begin{theorem}\label{thm:main}
  Under Assumptions~\ref{hyp:loss_exp_log}--\ref{hyp:marg_lowb}, almost surely, $\sm(u_k)$ converges to a positive limit and 
  \begin{equation}\label{eq:conv_uk}
    \dist(u_k, \cZ_s) \xrightarrow[k \rightarrow + \infty]{} 0 \, .
  \end{equation}
\end{theorem}
\begin{proof}
  The proof, which is given in Appendix~\ref{pf:main_th} follows from Proposition~\ref{prop:stoch_approx_exp_log} and some minor adaptations of recent results on stochastic approximation from \cite{benaim2006dynamics,dav-dru-kak-lee-19}.
\end{proof}
A natural question is the interpretation of membership in $\cZ_s$. Given the Riemannian perspective on $\bar{D}_s$, it is unsurprising that belonging to $\cZ_s$ is a necessary optimality condition for the max-margin problem. We formally prove this result in Appendix~\ref{app:conserv}.
\begin{lemma}\label{lm:loc_max}
  If $u^*$ is a local maximum of $\sm_{|\bbS^{d-1}}$, then $0 \in \bar{D}_s(u^*)$.
\end{lemma}
Thus, Theorem~\ref{thm:main} establishes  that $(u_k)$ converge to the set of $\bar{D}_s$-critical points, which is a necessary condition of optimality for $\argmax_{u \in \bbS^{d-1}} \sm(u)$.
Comparing our result with \cite[Theorem 4.4]{Lyu_Li_maxmargin}, we note that, if each $D_i$ were equal to $\partial p_i$, then any limit point of $(u_k)$ would correspond \emph{exactly} to a scaled KKT point from \cite{Lyu_Li_maxmargin}. In this work, the authors formulate an alternative optimization problem, namely
\begin{equation*}\label{def:prob2}\tag{P}
  \min \{ \norm{w}^2 : w \in\bbR^d\, ,\sm(w) \geq 1\}\, .
\end{equation*}
 As discussed in \cite{Lyu_Li_maxmargin}, if there exists $w \in \bbR^d$ such that $\sm(w) >0$, solving~\eqref{def:prob2} is equivalent to maximizing the margin. Examining the KKT conditions (see Appendix~\ref{app:gen_sett}) of~\eqref{def:prob2}, we observe that for any $u \in \cZ_s$ such that $\sm(u)>0$, there exists $\lambda >0$, such that $\lambda u$ is a KKT point. This implies that, within the setting of Theorem~\ref{thm:main}, the  optimality characterization is \emph{identical} to that in \cite{Lyu_Li_maxmargin}. 
 
 These observations also highlight that the appearance of a conservative field in our problem is unrelated to backpropagation. The set $\cZ_s$ (thus, implicitly, $\bar{D}_s$) already arises in the analysis of continuous-time subgradient flow in \cite{Lyu_Li_maxmargin}. In fact, as previously noted, $\bar{D} \neq \partial \sm$, even if all $D_i = \partial p_i$ (see Remark~\ref{rmk:max_subg}).


However, we adopt a different perspective. Rather than linking $\cZ_s$ directly to the KKT points of~\eqref{def:prob2}, we interpret it as the set of $\bar{D}_s$-critical points of the margin restricted to the sphere—where $\sm$ is naturally defined due to homogeneity.
Moreover, our definitions of $\bar{D}_s$ and $\cZ_s$ remain valid even when the $D_i$’s are arbitrary conservative set-valued fields, not just subgradients. In fact, our stochastic approximation interpretation allows us to consider a more general setting (see Appendix~\ref{app:gen_sett}) where Assumption~\ref{hyp:marg_lowb} can be relaxed. In this broader framework, the limit points of $(u_k)$ still lie in $\cZ_s$ without necessarily being rescaled versions of the KKT points of~\eqref{def:prob2}.


Finally, we note that as long as the ``stability assumption''~\ref{hyp:marg_lowb} holds, our analysis allows the step-size \( \gamma \) to be of arbitrary  size. This may seem surprising, as (non-smooth) SGD typically requires vanishing step-sizes for convergence (\cite{majewski2018analysis,dav-dru-kak-lee-19,bolte2023subgradient,le2024nonsmooth}). Mathematically, this follows from the fact that \( \bar{\gamma}_k \), the \emph{effective} step-size of the dynamics, is actually decreasing in our setting. A convergence analysis of constant-step SGD for \emph{smooth} homogeneous linear classifiers was studied in \cite{nacson2019stochastic}, but to the best of our knowledge, the more general non-smooth setting had not yet been addressed.

\section{Proof of Proposition~\ref{prop:stoch_approx_exp_log}}\label{pf:sto_app_explog}
In this proof, $C, C_1, C_2, \ldots $ will denote some positive absolute constants that can change from equation to equation. We also note that for all $w,i$, $p_i(w) \leq C \norm{w}^L$ and, due to Assumption~\ref{hyp:conserv}, for all $v \in D_i(w)$, $\norm{v} \leq C \norm{w}^{L-1}$.

To obtain~\eqref{eq:stoch_app_u}, we appropriately rescale the step-size and then write the Taylor's expansion of $u \mapsto (u+h)/\norm{u + h}$, for small $h$, using the fact that $\norm{w_k} \rightarrow + \infty$. 

Towards that goal, let us first introduce the (stochastic) sequence
\begin{equation}\label{eq:noise}
  \eta_{k+1} := {\frac{n_b - n}{n_b n}} \sum_{i \in B_k} l'(p_i(w_k)) \sa_i(w_k) + \frac{1}{n}\sum_{i \notin B_k} l'(p_i(w_k)) \sa_i(w_k)\, .
\end{equation}
Both~\eqref{eq:gd_new} and \eqref{eq:sgd_new} (where for~\eqref{eq:gd_new}, $\eta_{k} \equiv 0$) can be rewritten as 
\begin{equation}\label{eq:sgd_noise}
  w_{k+1} = w_k - \frac{\gamma}{n}  \sum_{i=1}^n l'(p_i(w_k))\sa_i(w_k)+ \gamma \eta_{k+1}\, .
\end{equation}
Now let us introduce the following notations 
\begin{equation}\label{def:reparm_gamma}
  \tgamma_k = -\gamma \norm{w_k}^{L-1} \sum_{j=1}^n l'(p_j(w_k)) \, , \quad \bgamma_k = \tgamma_k\norm{w_k}^{-1}
\end{equation}
and 
\begin{equation}\label{eq:def_lmk_tilel}
  \lambda_{i,k} = \frac{l'(p_i(w_k))}{\sum_{j=1}^n l'(p_j(w_k))} \, , \quad \tilde{\eta}_{k+1} = \frac{-\eta_{k+1}}{\norm{w_k}^{L-1} \sum_{j=1}^n l'(p_j(w_k))}\, .
\end{equation}
Note that since $l'(q) <0$, for all $k$,  $\tgamma_k, \bgamma_k, \lambda_{i,k} \geq 0$. Moreover, $\sum_{i=1}^n\lambda_{i,k} = 1$.

By Assumption~\ref{hyp:conserv} for each $i,k$, it holds that $v_{i,k} = \norm{w_k}^{L-1}g_{i,k}$, where $g_{i,k} \in D_i(u_k)$. Thus, we can rewrite Equation~\eqref{eq:sgd_noise} as:
\begin{equation}\label{eq:first_wk}
    w_{k+1} = w_k + \tilde{\gamma}_k \sum_{i=1}^n \lambda_{i,k} g_{i,k} + \tilde{\gamma}_k \tilde{\eta}_{k+1} :=w_{k} + \tilde{\gamma}_k \bar{g}_k + \tgamma_k \tilde{\eta}_{k+1} \, ,
\end{equation}
with $\bar{g}_k = \sum_{i=1}^n\lambda_{i,k} g_{i,k}$.
Therefore, using the definition of $\bgamma_k$ is~\eqref{def:reparm_gamma},
\begin{equation}\label{eq:uk_taylor1}
  \begin{split}
      u_{k+1} &= \frac{w_{k}+\tgamma_k \bg_k + \tgamma_k \tilde{\eta}_{k+1}}{\norm{w_{k}+\tgamma_k \bg_k + \tgamma_k \tilde{\eta}_{k+1}}} =
      \frac{u_k + \bgamma_k \bg_k + \bgamma_k \tilde{\eta}_{k+1}}{\norm{u_k + \bgamma_k \bg_k + \bgamma_k \tilde{\eta}_{k+1}}}\, .
        \end{split}
\end{equation}
Using the Taylor's expansion $\norm{u+h}^{-1} = 1- \scalarp{u}{h} + \cO(\norm{h}^2)$, for $u \in \bbS^{d-1}$, we obtain
\begin{equation}\label{eq:pf_uk_stochapp}
  u_{k+1}=(u_k + \bgamma_k \bg_k + \bgamma_k \tilde{\eta}_{k+1})(1 - \bgamma_k\scalarp{\bg_k}{u_k} -\bgamma_k\scalarp{\tilde{\eta}_{k+1}}{u_k} + b_k) = u_k + \bar{\gamma}_k \bg_k^s + \bgamma_k \bar{\eta}_{k+1} + \bar{\gamma}_k^2 r_k\, ,
\end{equation}
where $b_k$ is such that $|b_k| \leq C \bgamma_k^2 (\norm{\bg_k} + \norm{\tilde{\eta}_{k+1}})^2$, as soon as $\bgamma_k \norm{\bg_k + \tilde{\eta}_{k+1}} \leq 1/2$, and 
\begin{equation}\label{eqdef:proj_nois_subg}
  \bar{\eta}_{k+1} := \tilde{\eta}_{k+1} - \scalarp{\tilde{\eta}_{k+1}}{u_k} u_k \quad \textrm{ and } \quad \bar{g}_k^s := \bg_k - \scalarp{\bg_k}{u_k} u_k\, ,
\end{equation}
and, finally, 
\begin{equation}\label{eq:rk_bound}
 \bgamma_k^2 \norm{r_k} \leq |b_k|\norm{(u_k + \bgamma_k \bg_k + \bgamma_k \tilde{\eta}_{k+1})} + \bgamma_k^2 \norm{(\bg_k + \tilde{\eta}_{k+1})\scalarp{\bg_k + \tilde{\eta}_{k+1}}{u_k}}\, .
\end{equation}
Equation~\eqref{eq:uk_taylor1} correspond to~\eqref{eq:stoch_app_u}. We now briefly prove the four points of the proposition. 

\emph{Claim on $(\bar{\eta}_k)$.} The fact that $\eta_{k}$ and thus $\bar{\eta}_k$ is $w_k$-measurable is immediate by its definition in~\eqref{eq:noise}. Additionally, $\bbE[\bar{\eta}^s_{k+1} |\cF_k] = \bbE[\eta_{k+1} |\cF_k] = 0$. Moreover, Assumption~\ref{hyp:conserv} and Equation~\eqref{eq:noise} implies
\begin{equation*}
\norm{\eta_{k+1}} \leq C_1\norm{\sum_{i=1}^n l'(p_i(w_k))\sa_i(w_k)} \leq C_2 \norm{w_k}^{L-1} \sum_{i=1}^n |l'(p_i(w_k))|\, ,
\end{equation*}
with $C_2>0$ some deterministic constant independent on $k$. Therefore, by~\eqref{eq:def_lmk_tilel} and \eqref{eqdef:proj_nois_subg}, $\norm{\bar{\eta}_{k+1}} \leq \norm{\tilde{\eta}_{k+1}} \leq C_2$.

\emph{Claim on $(\bgamma_k)$.} Almost surely, there is $\varepsilon >0$, such that for $k$ large enough, $\sm(u_k) \geq  \varepsilon$. Thus, for every $i$, $l'(p_i(w_k)) \leq e^{-\varepsilon\norm{w_k}^L }$. By Assumption~\ref{hyp:conserv}, $\norm{\sa_i(w_k)} \leq C \norm{w_k}^{L-1}$, which implies for $k$ large enough,
\begin{equation*}
  \bgamma_k \leq C_1 \norm{w_k}^{L-2} e^{-\varepsilon \norm{w_k}^L} \leq \frac{C_1 c_2 \log(k)^{L-2}}{k^{\varepsilon c_1}} \leq \frac{1}{\sqrt{k^{\varepsilon c_1}}}
\end{equation*}
where the penultimate inequality comes from Proposition~\ref{prop:log_wk}.
To show that $\sum_{k} \bgamma_k = + \infty$, note that using Equation~\eqref{eq:first_wk}, we have for $k$ large enough,
\begin{equation}\label{eq:sum_gamma_inf}
  \begin{split}
\norm{w_{k+1}}^2  &\leq \lVert w_{k} \rVert^2 + 2 \tilde{\gamma}_k \lVert w_{k} \rVert \lVert \bg_k + \tilde{\eta}_{k+1} \rVert  + \tgamma_k^2 \norm{\bg_k + \tilde{\eta}_{k+1}}^2\\
&\leq \norm{w_k}^2 (1 + C \bgamma_k + C_1 \bgamma_k^2) \leq \norm{w_k}^2 e^{C_2 \bgamma_k} \leq \norm{w_{k_0}}^2 e^{C_2 \sum_{i=k_0}^{k} \bgamma_i }
\end{split}
\end{equation}
where $k_0$ is large enough, and where we have used the fact that $\sup_{k} (\norm{\bg_k + \tilde{\eta}_{k+1}}) < + \infty$ and $\bgamma_k \rightarrow 0$. 
Since the left-hand side of Equation~\eqref{eq:sum_gamma_inf} goes to infinity by Proposition~\ref{prop:log_wk}, we obtain that the right-hand side diverge to infinity and therefore $\sum_{k } \bgamma_k = + \infty$.


\emph{Claim on $(r_k)$.} Since $\bgamma_k \rightarrow 0$, $\sup_{k} \norm{\tilde{\eta}_{k+1}} \leq C_1$ and $\sup_{k} \norm{\bg_k} < C_2$, there is $k_0$, such that for all $k \geq k_0$, $\bgamma_k (\norm{\bg_k} + \norm{\tilde{\eta}_{k+1}})\leq 1/2$. Therefore, for $k \geq k_0$, $|b_k| \leq C \gamma_k^2 \norm{\bg_k + \tilde{\eta}_{k+1}}$ in~\eqref{eq:pf_uk_stochapp}, which by~\eqref{eq:rk_bound} implies that $\sup_{k \geq k_0}\norm{r_k} \leq C$.

\emph{Claim on $\bar{D}$.} Consider a sequence $u_{k_j} \rightarrow u$ and $g$ any accumulation point $g_{k_j}$, we need to prove that $g - \scalarp{g}{u}u \in \bar{D}_s(u)$, or, equivalently, that $g \in \bar{D}(u)$. Recall that $\bg_k = \sum_{i=1}^k \lambda_{i,k} \bg_{i,k}$, where $\lambda_{i,k} \geq 0$, $\sum_{i} \lambda_{i,k} = 1$ and $g_{i,k} \in D_{i}(u_k)$. Extracting a subsequence, we can assume that for each $i$, $\lambda_{i, k_j} \rightarrow \lambda_i$ and $g_{i,k_j} \rightarrow g_i \in D_i(u)$.  We claim that $\lambda_i \neq 0 \implies p_i(u) = \sm(u)$. Indeed, without losing generality assume that $p_1(u_{k_j})\rightarrow \sm(u)$. Then, if $\lim (p_{i}(u_k) - \sm(u)) >0$, we obtain 
\begin{equation*}
  \lambda_{i,k_j} \leq \frac{l'(\norm{w_{k,j}}^L p_i(u_{k_j}))}{l'(\norm{w_{k_j}}^L p_1(u_{k_j}))} \leq C e^{-\norm{w_k}^L\left( p_i(u_k) - \sm(u)\right)}\xrightarrow[j \rightarrow + \infty]{} 0 \, .
\end{equation*}
Therefore, $g$ can be written as $ \sum_{i=1}^n \lambda_i g_i$, where $g_i \in D_i(u)$ and $\lambda_i \neq 0 \implies p_i(u) = \sm(u)$.
In other words, $g \in \bar{D}(u)$, concluding the proof. 

\hfill $\blacksquare$


\section*{Acknowledgement}
The authors thank Evgenii Chzhen for insightful discussions.

\bibliography{main}

\appendix


\section{Unexpected Questions}
\label{app:question}
Real-world questions do not always have the correct premises. For example, in the question "\begin{CJK}{UTF8}{gbsn}水俣病的传染途径是什么?\end{CJK}(What is the route of infection for Minamata disease?)", Minamata disease is not an infectious disease. Taking this situation into account, we add a small number of human-written questions with incorrect premises and LLM-generated questions with hard-to-verify premises in the question collection phase. The number of these questions in the total number of questions is about 3\%.

\section{Prompt for LLM Augmentation}
\label{app:aug}

\definecolor{titlecolor}{rgb}{0.9, 0.5, 0.1}
\definecolor{anscolor}{rgb}{0.2, 0.5, 0.8}
\definecolor{labelcolor}{HTML}{48a07e}
\begin{table*}[h]
	\centering
	
 % \vspace{-0.2cm}
	
	\begin{center}
		\begin{tikzpicture}[
				chatbox_inner/.style={rectangle, rounded corners, opacity=0, text opacity=1, font=\sffamily\scriptsize, text width=5in, text height=9pt, inner xsep=6pt, inner ysep=6pt},
				chatbox_prompt_inner/.style={chatbox_inner, align=flush left, xshift=0pt, text height=11pt},
				chatbox_user_inner/.style={chatbox_inner, align=flush left, xshift=0pt},
				chatbox_gpt_inner/.style={chatbox_inner, align=flush left, xshift=0pt},
				chatbox/.style={chatbox_inner, draw=black!25, fill=gray!7, opacity=1, text opacity=0},
				chatbox_prompt/.style={chatbox, align=flush left, fill=gray!1.5, draw=black!30, text height=10pt},
				chatbox_user/.style={chatbox, align=flush left},
				chatbox_gpt/.style={chatbox, align=flush left},
				chatbox2/.style={chatbox_gpt, fill=green!25},
				chatbox3/.style={chatbox_gpt, fill=red!20, draw=black!20},
				chatbox4/.style={chatbox_gpt, fill=yellow!30},
				labelbox/.style={rectangle, rounded corners, draw=black!50, font=\sffamily\scriptsize\bfseries, fill=gray!5, inner sep=3pt},
			]
											
			\node[chatbox_user] (q1) {
				\textbf{System prompt}
				\newline
				\newline
				You are a helpful and precise assistant for segmenting and labeling sentences. We would like to request your help on curating a dataset for entity-level hallucination detection.
				\newline \newline
                We will give you a machine generated biography and a list of checked facts about the biography. Each fact consists of a sentence and a label (True/False). Please do the following process. First, breaking down the biography into words. Second, by referring to the provided list of facts, merging some broken down words in the previous step to form meaningful entities. For example, ``strategic thinking'' should be one entity instead of two. Third, according to the labels in the list of facts, labeling each entity as True or False. Specifically, for facts that share a similar sentence structure (\eg, \textit{``He was born on Mach 9, 1941.''} (\texttt{True}) and \textit{``He was born in Ramos Mejia.''} (\texttt{False})), please first assign labels to entities that differ across atomic facts. For example, first labeling ``Mach 9, 1941'' (\texttt{True}) and ``Ramos Mejia'' (\texttt{False}) in the above case. For those entities that are the same across atomic facts (\eg, ``was born'') or are neutral (\eg, ``he,'' ``in,'' and ``on''), please label them as \texttt{True}. For the cases that there is no atomic fact that shares the same sentence structure, please identify the most informative entities in the sentence and label them with the same label as the atomic fact while treating the rest of the entities as \texttt{True}. In the end, output the entities and labels in the following format:
                \begin{itemize}[nosep]
                    \item Entity 1 (Label 1)
                    \item Entity 2 (Label 2)
                    \item ...
                    \item Entity N (Label N)
                \end{itemize}
                % \newline \newline
                Here are two examples:
                \newline\newline
                \textbf{[Example 1]}
                \newline
                [The start of the biography]
                \newline
                \textcolor{titlecolor}{Marianne McAndrew is an American actress and singer, born on November 21, 1942, in Cleveland, Ohio. She began her acting career in the late 1960s, appearing in various television shows and films.}
                \newline
                [The end of the biography]
                \newline \newline
                [The start of the list of checked facts]
                \newline
                \textcolor{anscolor}{[Marianne McAndrew is an American. (False); Marianne McAndrew is an actress. (True); Marianne McAndrew is a singer. (False); Marianne McAndrew was born on November 21, 1942. (False); Marianne McAndrew was born in Cleveland, Ohio. (False); She began her acting career in the late 1960s. (True); She has appeared in various television shows. (True); She has appeared in various films. (True)]}
                \newline
                [The end of the list of checked facts]
                \newline \newline
                [The start of the ideal output]
                \newline
                \textcolor{labelcolor}{[Marianne McAndrew (True); is (True); an (True); American (False); actress (True); and (True); singer (False); , (True); born (True); on (True); November 21, 1942 (False); , (True); in (True); Cleveland, Ohio (False); . (True); She (True); began (True); her (True); acting career (True); in (True); the late 1960s (True); , (True); appearing (True); in (True); various (True); television shows (True); and (True); films (True); . (True)]}
                \newline
                [The end of the ideal output]
				\newline \newline
                \textbf{[Example 2]}
                \newline
                [The start of the biography]
                \newline
                \textcolor{titlecolor}{Doug Sheehan is an American actor who was born on April 27, 1949, in Santa Monica, California. He is best known for his roles in soap operas, including his portrayal of Joe Kelly on ``General Hospital'' and Ben Gibson on ``Knots Landing.''}
                \newline
                [The end of the biography]
                \newline \newline
                [The start of the list of checked facts]
                \newline
                \textcolor{anscolor}{[Doug Sheehan is an American. (True); Doug Sheehan is an actor. (True); Doug Sheehan was born on April 27, 1949. (True); Doug Sheehan was born in Santa Monica, California. (False); He is best known for his roles in soap operas. (True); He portrayed Joe Kelly. (True); Joe Kelly was in General Hospital. (True); General Hospital is a soap opera. (True); He portrayed Ben Gibson. (True); Ben Gibson was in Knots Landing. (True); Knots Landing is a soap opera. (True)]}
                \newline
                [The end of the list of checked facts]
                \newline \newline
                [The start of the ideal output]
                \newline
                \textcolor{labelcolor}{[Doug Sheehan (True); is (True); an (True); American (True); actor (True); who (True); was born (True); on (True); April 27, 1949 (True); in (True); Santa Monica, California (False); . (True); He (True); is (True); best known (True); for (True); his roles in soap operas (True); , (True); including (True); in (True); his portrayal (True); of (True); Joe Kelly (True); on (True); ``General Hospital'' (True); and (True); Ben Gibson (True); on (True); ``Knots Landing.'' (True)]}
                \newline
                [The end of the ideal output]
				\newline \newline
				\textbf{User prompt}
				\newline
				\newline
				[The start of the biography]
				\newline
				\textcolor{magenta}{\texttt{\{BIOGRAPHY\}}}
				\newline
				[The ebd of the biography]
				\newline \newline
				[The start of the list of checked facts]
				\newline
				\textcolor{magenta}{\texttt{\{LIST OF CHECKED FACTS\}}}
				\newline
				[The end of the list of checked facts]
			};
			\node[chatbox_user_inner] (q1_text) at (q1) {
				\textbf{System prompt}
				\newline
				\newline
				You are a helpful and precise assistant for segmenting and labeling sentences. We would like to request your help on curating a dataset for entity-level hallucination detection.
				\newline \newline
                We will give you a machine generated biography and a list of checked facts about the biography. Each fact consists of a sentence and a label (True/False). Please do the following process. First, breaking down the biography into words. Second, by referring to the provided list of facts, merging some broken down words in the previous step to form meaningful entities. For example, ``strategic thinking'' should be one entity instead of two. Third, according to the labels in the list of facts, labeling each entity as True or False. Specifically, for facts that share a similar sentence structure (\eg, \textit{``He was born on Mach 9, 1941.''} (\texttt{True}) and \textit{``He was born in Ramos Mejia.''} (\texttt{False})), please first assign labels to entities that differ across atomic facts. For example, first labeling ``Mach 9, 1941'' (\texttt{True}) and ``Ramos Mejia'' (\texttt{False}) in the above case. For those entities that are the same across atomic facts (\eg, ``was born'') or are neutral (\eg, ``he,'' ``in,'' and ``on''), please label them as \texttt{True}. For the cases that there is no atomic fact that shares the same sentence structure, please identify the most informative entities in the sentence and label them with the same label as the atomic fact while treating the rest of the entities as \texttt{True}. In the end, output the entities and labels in the following format:
                \begin{itemize}[nosep]
                    \item Entity 1 (Label 1)
                    \item Entity 2 (Label 2)
                    \item ...
                    \item Entity N (Label N)
                \end{itemize}
                % \newline \newline
                Here are two examples:
                \newline\newline
                \textbf{[Example 1]}
                \newline
                [The start of the biography]
                \newline
                \textcolor{titlecolor}{Marianne McAndrew is an American actress and singer, born on November 21, 1942, in Cleveland, Ohio. She began her acting career in the late 1960s, appearing in various television shows and films.}
                \newline
                [The end of the biography]
                \newline \newline
                [The start of the list of checked facts]
                \newline
                \textcolor{anscolor}{[Marianne McAndrew is an American. (False); Marianne McAndrew is an actress. (True); Marianne McAndrew is a singer. (False); Marianne McAndrew was born on November 21, 1942. (False); Marianne McAndrew was born in Cleveland, Ohio. (False); She began her acting career in the late 1960s. (True); She has appeared in various television shows. (True); She has appeared in various films. (True)]}
                \newline
                [The end of the list of checked facts]
                \newline \newline
                [The start of the ideal output]
                \newline
                \textcolor{labelcolor}{[Marianne McAndrew (True); is (True); an (True); American (False); actress (True); and (True); singer (False); , (True); born (True); on (True); November 21, 1942 (False); , (True); in (True); Cleveland, Ohio (False); . (True); She (True); began (True); her (True); acting career (True); in (True); the late 1960s (True); , (True); appearing (True); in (True); various (True); television shows (True); and (True); films (True); . (True)]}
                \newline
                [The end of the ideal output]
				\newline \newline
                \textbf{[Example 2]}
                \newline
                [The start of the biography]
                \newline
                \textcolor{titlecolor}{Doug Sheehan is an American actor who was born on April 27, 1949, in Santa Monica, California. He is best known for his roles in soap operas, including his portrayal of Joe Kelly on ``General Hospital'' and Ben Gibson on ``Knots Landing.''}
                \newline
                [The end of the biography]
                \newline \newline
                [The start of the list of checked facts]
                \newline
                \textcolor{anscolor}{[Doug Sheehan is an American. (True); Doug Sheehan is an actor. (True); Doug Sheehan was born on April 27, 1949. (True); Doug Sheehan was born in Santa Monica, California. (False); He is best known for his roles in soap operas. (True); He portrayed Joe Kelly. (True); Joe Kelly was in General Hospital. (True); General Hospital is a soap opera. (True); He portrayed Ben Gibson. (True); Ben Gibson was in Knots Landing. (True); Knots Landing is a soap opera. (True)]}
                \newline
                [The end of the list of checked facts]
                \newline \newline
                [The start of the ideal output]
                \newline
                \textcolor{labelcolor}{[Doug Sheehan (True); is (True); an (True); American (True); actor (True); who (True); was born (True); on (True); April 27, 1949 (True); in (True); Santa Monica, California (False); . (True); He (True); is (True); best known (True); for (True); his roles in soap operas (True); , (True); including (True); in (True); his portrayal (True); of (True); Joe Kelly (True); on (True); ``General Hospital'' (True); and (True); Ben Gibson (True); on (True); ``Knots Landing.'' (True)]}
                \newline
                [The end of the ideal output]
				\newline \newline
				\textbf{User prompt}
				\newline
				\newline
				[The start of the biography]
				\newline
				\textcolor{magenta}{\texttt{\{BIOGRAPHY\}}}
				\newline
				[The ebd of the biography]
				\newline \newline
				[The start of the list of checked facts]
				\newline
				\textcolor{magenta}{\texttt{\{LIST OF CHECKED FACTS\}}}
				\newline
				[The end of the list of checked facts]
			};
		\end{tikzpicture}
        \caption{GPT-4o prompt for labeling hallucinated entities.}\label{tb:gpt-4-prompt}
	\end{center}
\vspace{-0cm}
\end{table*}
\begin{table*}

\centering

\begin{tabular}{|p{\textwidth}|}
\hline
\\ [2pt]
\par Here is a statement and a corresponding piece of reference text. Please complete the task as follows, strictly following the format I have given for the output:
\par (1) Find all the original key passages in the reference text that directly support the information in the statement (there may be more than one, find each one). Output one original key passage per line and the information in the statement it directly supports in the format “Key passage {number}: {key passage} (information in the supporting statement: {supporting information})”.
\par (2) Please group key passages, each group contains key passages supporting the same or related information in the statement, output one line of the grouping results in the format of “Key passage grouping: Group 1: (first group of key passage numbers), Group 2: (second group of key passage numbers) ...”. For example, if there are 2 pieces of information in the statement, key paragraph 1 supports information 1, key paragraph 2 supports information 2, and key paragraph 3 supports information 1, then the output is “Key Paragraph Grouping: Group 1: (1, 3), Group 2: (2)”.
\par (3) Select a group of key text segments and modify the parts of them that support the information in the statement to meet the following requirements:
\par - The modification should make it impossible for the key passage to fully support the corresponding information in the statement.
\par - The modifications should maintain the logical flow of the key passages and no contradictions between the information in the key passages.
\par - The modification should keep the key paragraph logically coherent in the context of the reference text and not contradict the rest of the reference text.
\par - Modify only the parts that support the information in a statement, leaving the rest unchanged.
\par - If there is more than one key passage in a set, the information in them should remain consistent after revision.
\par You need to try two methods of modification:
\par - Changing the message: modifying the message in one part of the key paragraph to another. Do not make changes that directly conflict with the original information. For example, if the original message is “The Audi A7 Signature Edition has a faster top speed than its predecessor”, an appropriate change would be “The Audi A7 Luxury Edition has a faster top speed than its predecessor”, and an inappropriate change1 would be “The Audi A7 Signature Edition has a slower top speed than its predecessor” (using an antonym, which is in direct conflict with the original message), and inappropriate modification 2 is ‘The top speed of the Audi A7 Signature Edition is not faster than the previous generation’ (adding a negative word, which is in direct conflict with the original message).
\par - Delete Information: Remove information from a place in a key paragraph. If the key paragraph is a complete sentence, it should still be a complete sentence after deleting the information. For example, if the original paragraph reads “Due to weather conditions, the project was delayed until March 15” (complete sentence), an appropriate change would be “Due to weather conditions, the project was delayed until March” (still a complete sentence), an inappropriate change would be “Due to the weather” (no longer a complete sentence).
\par For each method, output the key passage that was modified and check its logical fluency, giving an integer within 1 to 10 as a rating (higher means more fluent). Output one modified key passage per line in the format “{method}-modified key passage {number}: {modified key passage} (logical fluency: {score})”.
\\ [5pt]
\\ [5pt]
\hline

\end{tabular}

\caption{\label{tab:prompt_en} The complete prompt for the LLM augmentation (translated into English).}
\end{table*}

See Table~\ref{tab:prompt} for the prompt for LLM augmentation. Table~\ref{tab:prompt_en} provides an English version.

\section{Instructions for Annotators}
\label{app:ann}
\subsection{First Stage}
In the first stage, we provide the annotators with the question, answer, statement, and cited documents. What LLM considers to be key segments are highlighted in red in the cited documents (see Figure~\ref{fig:stage} for an example). We instruct the annotators to follow the process below:

\par (1) First look at the highlighted text. If the highlighted text fully supports the statement, then the annotation is positive; if the highlighted text contradicts the statement, then the annotation is negative.

\par (2) If the annotation cannot be derived from the highlighted text, then look at the rest of the documents to make the annotation. When the documents fully support the statement, the label is positive, and when there is any information in the statement that contradicts the documents or information that is not mentioned in the documents, the label is negative.

\subsection{Second Stage}

In the second stage, we provide the annotator with the statement and the modified documents. In the documents, the modified parts are highlighted in green, where the dashed and crossed-out text is deleted and the rest is added (see Figure~\ref{fig:stage} for examples). 

For the annotation of whether the quality of the modification is acceptable, the annotators are instructed to note that qualified modifications need to satisfy the following two requirements: (1) There are no contradictions within each modified document. (2) The modified key segments are fluent in their own right and in the context of the document. The annotation for support is the same as the first stage, but based on the modified documents.

 
\section{Input and Training Details}
\label{app:detail}
We input the statement and the cited documents into the model and ask the model to determine whether the statement is fully supported by the documents, outputting yes or no. For input, we label and concatenate the cited documents in order (as shown in Table~\ref{tab:dataset}). For training, we use the following settings: For training, we use the following settings: learning rate is 5e-4, number of epochs is 10, scheduler is cosine scheduler, warmup ratio is 0.03, batch size is 256, and LoRA setting is $r=8$, $a=32$ and 0.1 dropout. We report the model performance for the epoch that achieves the best performance on the dev set.
\label{app:detail}



\section{Related Works}
Language models are known to produce hallucinations - statements that are inaccurate or unfounded~\citep{MaynezNBM20,HuCLGWYG24}. To address this limitation, recent research has focused on augmenting LLMs with external tools such as retrievers~\citep{GuuLTPC20,BorgeaudMHCRM0L22,LiuCtrla2024} and search engines~\citep{WebGPT2021, Komeili0W22, TanGSXLFLWSLS24}. While this approach suggests that generated content is supported by external references, the reliability of such attribution requires careful examination. Recent studies have investigated the validity of these attributions. \citet{DBLP:conf/emnlp/LiuZL23} conducted human evaluations to assess the verifiability of responses from generative search engines. \citet{hu2024evaluate} further investigate the reliability of such attributions when giving adversarial questions to RAG systems. Their findings revealed frequent occurrences of unsupported statements and inaccurate citations, highlighting the need for rigorous attribution verification~\citep{RashkinNLA00PTT23}. However, human evaluation processes are resource-intensive and time-consuming. To overcome these limitations, existing efforts~\citep{GaoDPCCFZLLJG23,DBLP:conf/emnlp/GaoYYC23} proposed an automated approach using Natural Language Inference models to evaluate attribution accuracy. While several English-language benchmarks have been developed for this purpose~\citep{DBLP:conf/emnlp/YueWCZS023}, comparable resources in Chinese are notably lacking. Creating such datasets presents unique challenges, particularly in generating realistic negative samples (unsupported citations).  To address this gap, we introduce the first large-scale Chinese dataset for citation faithfulness detection, developed through a cost-effective two-stage manual annotation process.

\begin{figure*}
    \centering
    \includegraphics[width=0.3\textwidth]{appendix/s1.png}
    \includegraphics[width=0.3\textwidth]{appendix/s2-m.png}
    \includegraphics[width=0.3\textwidth]{appendix/s2-d.png}
    \caption{Examples of interfaces that provide samples to the annotators. The first figure shows an example of the first stage. The last two images show the second stage with the same sample modified (information changed/deleted).}
    \label{fig:stage}
\end{figure*}


\end{document}
