%We begin by showing that the sequence of directions follows a dynamic that can be interpreted as a stochastic approximation of the reversed (Riemannian) conservative field flow of $\sm$ restricted to the sphere $\bbS^d$ in Proposition~\ref{prop:stoch_approx_exp_log}. This observation directly leads to our main result, Theorem~\ref{thm:main}, which follows as a consequence of recent stochastic approximation results from \cite{benaim_05_DI_1,dav-dru-kak-lee-19}.

As a first step toward our main result, we establish that the iterates norm $(\lVert w_k\rVert)$ grows to infinity at a logarithmic rate. This is consistent with \cite[Theorem 4.3]{Lyu_Li_maxmargin}.
\begin{proposition}\label{prop:log_wk}
  Under Assumptions~\ref{hyp:loss_exp_log}--\ref{hyp:marg_lowb}, almost surely, there exist $c_1, c_2, \varepsilon>0$ and $k_0 \in \bbN$, such that for all $k \geq k_0$, $\norm{w_k}$ increases and
  \begin{equation*}
    c_1 \log (k)\leq \norm{w_k}^L \leq c_2 \log(k) \quad \textrm{ and } \quad 0 < \cL(w_k) \leq k^{-\varepsilon c_1}\, .
  \end{equation*}
  In particular, $\norm{w_k}\rightarrow + \infty$ and $\cL(w_k) \rightarrow 0$.
\end{proposition}
\begin{proof}[Sketch, full proof in Appendix~\ref{sec:pf_logwk}.] The proof follows from the next observations, which hold almost surely, for $k$ large enough. \emph{(i)} There is $\varepsilon >0$, such that $\sm(u_k) \geq \varepsilon$. In particular, there are $M, C_1, C_2>0$, such that $ C_2 e^{-M \norm{w_k}^L}\leq -l'(p_i(w_k)) \leq C_1e^{-\varepsilon \norm{w_k}^L}$. \emph{(ii)} As a result, there is $C_3 >0$ such that
  \begin{equation}\label{eq:lwb_wk2}
    \norm{w_{k+1}}^2 \geq \norm{w_k}^2 ( 1 + C_3 \gamma e^{-M \norm{w_k}^L } \norm{w_k}^{L-2})\, ,
  \end{equation}
  which implies that $\norm{w_k}$ is increasing to infinity and that there is $C_4 >0$ such that
  \begin{equation}\label{eq:uwb_wk2}
    \norm{w_{k+1}}^2 \leq \norm{w_k}^2 (1 + C_4 \gamma e^{-\varepsilon \norm{w_k}^L}\norm{w_k}^{L-2})\, .
  \end{equation}\emph{(iii)} Finally, using~\eqref{eq:lwb_wk2}--\eqref{eq:uwb_wk2} and the Taylor's expansion of $(1+x)^{L/2}$ near zero, we obtain existence of constants $C_5, C_6, a>0$, such that for $k$ large enough,
  \begin{equation*}
    C_5 \gamma \leq e^{a \norm{w_k}^L}\left(\norm{w_{k+1}}^L - \norm{w_k}^L\right) \leq C_6 \gamma \, .
  \end{equation*}
Summing these inequalities from $k$ to $k+N$ and noticing that the expression in the middle is comparable to the integral of $e^{at}$ between $\norm{w_{k}}^L$ and $\norm{w_{k+N}}^L$, concludes the proof.
\end{proof}
Define the set-valued map $\bar{D} : \bbR^d \rightrightarrows \bbR^d$ as 
\begin{equation}\label{eq:avg_consfiel}
  \bar{D}(x) = \conv \{v: v \in D_i(w) \, , \textrm{with $i \in I(w)$} \}\,, \quad \textrm{  where $I(w) = \{ i: p_i(w) = \sm(w)\}$}\, .
\end{equation}
As shown in Appendix~\ref{app:conserv}, it is a conservative set-valued field for the potential $\sm$. Note that, following Remark~\ref{rmk:max_subg}, even if for all $i$, $D_i = \partial p_i$, $\bar{D}$ can be different from $\partial \sm$. Next, we define the set-valued field $\bar{D}_s: \bbS^{d-1} \rightrightarrows \bbR^d$ as
\begin{equation}\label{def:riem_cons}
\bar{D}_{s}(u) := \{ v - \scalarp{v}{u}u : v \in \bar{D}(u) \}\, .
\end{equation}
The associated set of critical points is then given by
\begin{equation}\label{def:riem_crit}
  \cZ_s := \{ u \in \bbS^{d-1} : 0 \in \bar{D}_s(u) \} \subset \bbS^{d-1}\, .
\end{equation}
The field $\bar{D}_s$ and the critical points set $\cZ_s$ admit a straightforward interpretation. If $\sm$ is $C^1$ around some point $u \in \bbS^{d-1}$ and $\bar{D} =\{ \nabla \sm\}$, then $\bar{D}_s(u)$ is the radial component of $\nabla \sm(u)$, corresponding to its projection onto the tangent plane of $\bbS^{d-1}$ at $u$.
From a Riemannian geometry perspective, this implies that $\bar{D}_s(u)$ is the Riemannian gradient\footnote{Here, the Riemannian structure is implicitly induced from the ambient space.} of $\sm$ \emph{restricted} to the sphere $\bbS^{d-1}$, $\sm_{|\bbS^{d-1}}$. Similarly, $\cZ_s$ corresponds to the set of critical points of $\sm_{|\bbS^{d-1}}$.
More generally, since conservative fields are gradient-like objects (see Proposition~\ref{prop:var_strat_cons} in Appendix~\ref{app:omin}), we interpret $\bar{D}s$ as the Riemannian conservative field of $\sm_{|\bbS^{d-1}}$, with $\cZ_s$ as its corresponding critical points\footnote{As noted in \cite[Page 4, footnote]{bolte2021conservative}, the concept of a conservative set-valued field extends naturally to functions defined on any complete Riemannian submanifold, including $\bbS^{d-1}$.}.


We will consider the differential inclusion (DI) associated to set-valued field $\bar{D}_s$,
\begin{equation*}\label{eq:DI_sphere}\tag{DI}
\dot{\su}(t) \in \bar{D}_{s}(\su(t))\,.
\end{equation*}
Under the aforementioned interpretation, this corresponds to the reversed gradient (or conservative field) flow of $\sm_{|\bbS^{d-1}}$. 

We now show that the iterates’ directions evolve according to a dynamic that approximates~\eqref{eq:DI_sphere} via an Euler-like discretization (or stochastic approximation). The proof is deferred to Section~\ref{pf:sto_app_explog}.
%To treat the stochastic setting, we denote by $\cF_k$ the sigma-algebra generated by $\{ w_0,\ldots,w_k\}$. Notice that it is a filtration ($\cF_k \subset \cF_{k+1}$ for all $k$) and recall that a sequence $(a_k)$ is said to be adapted to $(\cF_k)$ if for every $k$, $a_k$ is $\cF_k$-measurable. Finally, we recall the notation $u_k=w_k/\norm{w_k}$.
  \begin{proposition}\label{prop:stoch_approx_exp_log}
    Let Assumptions~\ref{hyp:loss_exp_log}--\ref{hyp:marg_lowb} hold. There exist sequences $(\bg_k^s), (r_k), (\bgamma_k), (\bar{\eta}_{k+1})$ such that, for both~\eqref{eq:gd_new} and~\eqref{eq:sgd_new}, the normalized direction iterates $u_k \coloneqq w_k/\lVert w_k \rVert$ satisfy
    \begin{equation}\label{eq:stoch_app_u}
      u_{k+1} = u_k + \bgamma_k\bg_k^s + \bgamma_k \bar{\eta}_{k+1} + \bgamma_k^2 r_k\, .
    \end{equation}
    Moreover, considering the filtration $(\cF_k)_k$ where, for $k \in \mathbb{N}$, $\cF_k$ the sigma-algebra generated by $\{ w_0,\ldots,w_k\}$, the following holds:
    \begin{enumerate}
      \item\label{pr_res:rk} The sequence $(r_k)$ satisfies $\sup_{k}\norm{r_k} < + \infty$ almost surely.
      \item\label{pr_res:gammak} The sequence $(\bgamma_k)$ is positive and adapted to $(\cF_k)$. Moreover, $\sum_{k} \bgamma_k = + \infty$, and, almost surely, there is $c_3>0$ such that for sufficiently large $k$, $\bgamma_k \leq k^{-c_3}$.
      \item\label{pr_res:etak} For~\eqref{eq:gd_new}, $\bar{\eta}_{k} \equiv 0$. Otherwise, the sequence $(\bar{\eta}_{k})$ is adapted to $(\cF_k)$ and satisfies 
      \begin{equation*}
      \bbE[\bar{\eta}_{k+1} |\cF_k] = 0 \,.
      \end{equation*}
      Additionally, there exists a deterministic constant $c_4>0$ such that $\sup_{k} \norm{\bar{\eta}_{k+1}} < c_4$.
      \item\label{pr_res:barD} For any unbounded sequence $(k_j)_j$, such that $u_{k_j} \to u \in \bbS^{d-1}$, $\dist(\bar{D}_s(u), \bg^s_{k_j}) \rightarrow 0$. 
    \end{enumerate}
  \end{proposition}
    Since Proposition~\ref{prop:stoch_approx_exp_log} allows us to interpret $(u_k)$ as a discretization of~\eqref{eq:DI_sphere}, it is natural to investigate the convergence properties of a solution of its continuous counterpart~\eqref{eq:DI_sphere}.
  If $\su$ is such solution, then for almost every $t \in \bbR$, there exists $v \in \bar{D}_s(\su(t))$ such that $\dot{\su}(t) = v - \scalarp{v}{u}u$. Thus, by Definition~\ref{def:cons_f}, for almost every $t$, $\frac{\dif }{\dif t}  \sm(\su(t)) = \scalarp{\dot{\su}(t)}{v} = \norm{\dot{\su}(t)}^2$. Therefore, for $T >0$, we obtain
  \begin{equation*}
    \sm(\su(T)) - \sm(\su(0)) = \int_{0}^{T} \norm{\dot{\su}(t)}^2 \dif t \, .
  \end{equation*}
  This implies that $\sm(\su(T)) \geq \sm(\su(0))$, with strict inequality whenever $\su(0) \not \in \cZ_s$. In dynamical systems terminology, $-\sm$ is a Lyapunov function for~\eqref{eq:DI_sphere}. In particular, it can be shown that any solution $\su(t)$ to~\eqref{eq:DI_sphere} converges to $\cZ_s$.

Our main result, Theorem~\ref{thm:main}, establishes that the same holds true for the sequence of normalized directions: any limit point of $(u_k)$ is contained in $\cZ_s$. As discussed below, this result generalizes \cite[Theorem 4.4]{Lyu_Li_maxmargin} to stochastic gradient descent in the nonsmooth setting.

\begin{theorem}\label{thm:main}
  Under Assumptions~\ref{hyp:loss_exp_log}--\ref{hyp:marg_lowb}, almost surely, $\sm(u_k)$ converges to a positive limit and 
  \begin{equation}\label{eq:conv_uk}
    \dist(u_k, \cZ_s) \xrightarrow[k \rightarrow + \infty]{} 0 \, .
  \end{equation}
\end{theorem}
\begin{proof}
  The proof, which is given in Appendix~\ref{pf:main_th} follows from Proposition~\ref{prop:stoch_approx_exp_log} and some minor adaptations of recent results on stochastic approximation from \cite{benaim2006dynamics,dav-dru-kak-lee-19}.
\end{proof}
A natural question is the interpretation of membership in $\cZ_s$. Given the Riemannian perspective on $\bar{D}_s$, it is unsurprising that belonging to $\cZ_s$ is a necessary optimality condition for the max-margin problem. We formally prove this result in Appendix~\ref{app:conserv}.
\begin{lemma}\label{lm:loc_max}
  If $u^*$ is a local maximum of $\sm_{|\bbS^{d-1}}$, then $0 \in \bar{D}_s(u^*)$.
\end{lemma}
Thus, Theorem~\ref{thm:main} establishes  that $(u_k)$ converge to the set of $\bar{D}_s$-critical points, which is a necessary condition of optimality for $\argmax_{u \in \bbS^{d-1}} \sm(u)$.
Comparing our result with \cite[Theorem 4.4]{Lyu_Li_maxmargin}, we note that, if each $D_i$ were equal to $\partial p_i$, then any limit point of $(u_k)$ would correspond \emph{exactly} to a scaled KKT point from \cite{Lyu_Li_maxmargin}. In this work, the authors formulate an alternative optimization problem, namely
\begin{equation*}\label{def:prob2}\tag{P}
  \min \{ \norm{w}^2 : w \in\bbR^d\, ,\sm(w) \geq 1\}\, .
\end{equation*}
 As discussed in \cite{Lyu_Li_maxmargin}, if there exists $w \in \bbR^d$ such that $\sm(w) >0$, solving~\eqref{def:prob2} is equivalent to maximizing the margin. Examining the KKT conditions (see Appendix~\ref{app:gen_sett}) of~\eqref{def:prob2}, we observe that for any $u \in \cZ_s$ such that $\sm(u)>0$, there exists $\lambda >0$, such that $\lambda u$ is a KKT point. This implies that, within the setting of Theorem~\ref{thm:main}, the  optimality characterization is \emph{identical} to that in \cite{Lyu_Li_maxmargin}. 
 
 These observations also highlight that the appearance of a conservative field in our problem is unrelated to backpropagation. The set $\cZ_s$ (thus, implicitly, $\bar{D}_s$) already arises in the analysis of continuous-time subgradient flow in \cite{Lyu_Li_maxmargin}. In fact, as previously noted, $\bar{D} \neq \partial \sm$, even if all $D_i = \partial p_i$ (see Remark~\ref{rmk:max_subg}).


However, we adopt a different perspective. Rather than linking $\cZ_s$ directly to the KKT points of~\eqref{def:prob2}, we interpret it as the set of $\bar{D}_s$-critical points of the margin restricted to the sphere—where $\sm$ is naturally defined due to homogeneity.
Moreover, our definitions of $\bar{D}_s$ and $\cZ_s$ remain valid even when the $D_i$’s are arbitrary conservative set-valued fields, not just subgradients. In fact, our stochastic approximation interpretation allows us to consider a more general setting (see Appendix~\ref{app:gen_sett}) where Assumption~\ref{hyp:marg_lowb} can be relaxed. In this broader framework, the limit points of $(u_k)$ still lie in $\cZ_s$ without necessarily being rescaled versions of the KKT points of~\eqref{def:prob2}.


Finally, we note that as long as the ``stability assumption''~\ref{hyp:marg_lowb} holds, our analysis allows the step-size \( \gamma \) to be of arbitrary  size. This may seem surprising, as (non-smooth) SGD typically requires vanishing step-sizes for convergence (\cite{majewski2018analysis,dav-dru-kak-lee-19,bolte2023subgradient,le2024nonsmooth}). Mathematically, this follows from the fact that \( \bar{\gamma}_k \), the \emph{effective} step-size of the dynamics, is actually decreasing in our setting. A convergence analysis of constant-step SGD for \emph{smooth} homogeneous linear classifiers was studied in \cite{nacson2019stochastic}, but to the best of our knowledge, the more general non-smooth setting had not yet been addressed.

\section{Proof of Proposition~\ref{prop:stoch_approx_exp_log}}\label{pf:sto_app_explog}
In this proof, $C, C_1, C_2, \ldots $ will denote some positive absolute constants that can change from equation to equation. We also note that for all $w,i$, $p_i(w) \leq C \norm{w}^L$ and, due to Assumption~\ref{hyp:conserv}, for all $v \in D_i(w)$, $\norm{v} \leq C \norm{w}^{L-1}$.

To obtain~\eqref{eq:stoch_app_u}, we appropriately rescale the step-size and then write the Taylor's expansion of $u \mapsto (u+h)/\norm{u + h}$, for small $h$, using the fact that $\norm{w_k} \rightarrow + \infty$. 

Towards that goal, let us first introduce the (stochastic) sequence
\begin{equation}\label{eq:noise}
  \eta_{k+1} := {\frac{n_b - n}{n_b n}} \sum_{i \in B_k} l'(p_i(w_k)) \sa_i(w_k) + \frac{1}{n}\sum_{i \notin B_k} l'(p_i(w_k)) \sa_i(w_k)\, .
\end{equation}
Both~\eqref{eq:gd_new} and \eqref{eq:sgd_new} (where for~\eqref{eq:gd_new}, $\eta_{k} \equiv 0$) can be rewritten as 
\begin{equation}\label{eq:sgd_noise}
  w_{k+1} = w_k - \frac{\gamma}{n}  \sum_{i=1}^n l'(p_i(w_k))\sa_i(w_k)+ \gamma \eta_{k+1}\, .
\end{equation}
Now let us introduce the following notations 
\begin{equation}\label{def:reparm_gamma}
  \tgamma_k = -\gamma \norm{w_k}^{L-1} \sum_{j=1}^n l'(p_j(w_k)) \, , \quad \bgamma_k = \tgamma_k\norm{w_k}^{-1}
\end{equation}
and 
\begin{equation}\label{eq:def_lmk_tilel}
  \lambda_{i,k} = \frac{l'(p_i(w_k))}{\sum_{j=1}^n l'(p_j(w_k))} \, , \quad \tilde{\eta}_{k+1} = \frac{-\eta_{k+1}}{\norm{w_k}^{L-1} \sum_{j=1}^n l'(p_j(w_k))}\, .
\end{equation}
Note that since $l'(q) <0$, for all $k$,  $\tgamma_k, \bgamma_k, \lambda_{i,k} \geq 0$. Moreover, $\sum_{i=1}^n\lambda_{i,k} = 1$.

By Assumption~\ref{hyp:conserv} for each $i,k$, it holds that $v_{i,k} = \norm{w_k}^{L-1}g_{i,k}$, where $g_{i,k} \in D_i(u_k)$. Thus, we can rewrite Equation~\eqref{eq:sgd_noise} as:
\begin{equation}\label{eq:first_wk}
    w_{k+1} = w_k + \tilde{\gamma}_k \sum_{i=1}^n \lambda_{i,k} g_{i,k} + \tilde{\gamma}_k \tilde{\eta}_{k+1} :=w_{k} + \tilde{\gamma}_k \bar{g}_k + \tgamma_k \tilde{\eta}_{k+1} \, ,
\end{equation}
with $\bar{g}_k = \sum_{i=1}^n\lambda_{i,k} g_{i,k}$.
Therefore, using the definition of $\bgamma_k$ is~\eqref{def:reparm_gamma},
\begin{equation}\label{eq:uk_taylor1}
  \begin{split}
      u_{k+1} &= \frac{w_{k}+\tgamma_k \bg_k + \tgamma_k \tilde{\eta}_{k+1}}{\norm{w_{k}+\tgamma_k \bg_k + \tgamma_k \tilde{\eta}_{k+1}}} =
      \frac{u_k + \bgamma_k \bg_k + \bgamma_k \tilde{\eta}_{k+1}}{\norm{u_k + \bgamma_k \bg_k + \bgamma_k \tilde{\eta}_{k+1}}}\, .
        \end{split}
\end{equation}
Using the Taylor's expansion $\norm{u+h}^{-1} = 1- \scalarp{u}{h} + \cO(\norm{h}^2)$, for $u \in \bbS^{d-1}$, we obtain
\begin{equation}\label{eq:pf_uk_stochapp}
  u_{k+1}=(u_k + \bgamma_k \bg_k + \bgamma_k \tilde{\eta}_{k+1})(1 - \bgamma_k\scalarp{\bg_k}{u_k} -\bgamma_k\scalarp{\tilde{\eta}_{k+1}}{u_k} + b_k) = u_k + \bar{\gamma}_k \bg_k^s + \bgamma_k \bar{\eta}_{k+1} + \bar{\gamma}_k^2 r_k\, ,
\end{equation}
where $b_k$ is such that $|b_k| \leq C \bgamma_k^2 (\norm{\bg_k} + \norm{\tilde{\eta}_{k+1}})^2$, as soon as $\bgamma_k \norm{\bg_k + \tilde{\eta}_{k+1}} \leq 1/2$, and 
\begin{equation}\label{eqdef:proj_nois_subg}
  \bar{\eta}_{k+1} := \tilde{\eta}_{k+1} - \scalarp{\tilde{\eta}_{k+1}}{u_k} u_k \quad \textrm{ and } \quad \bar{g}_k^s := \bg_k - \scalarp{\bg_k}{u_k} u_k\, ,
\end{equation}
and, finally, 
\begin{equation}\label{eq:rk_bound}
 \bgamma_k^2 \norm{r_k} \leq |b_k|\norm{(u_k + \bgamma_k \bg_k + \bgamma_k \tilde{\eta}_{k+1})} + \bgamma_k^2 \norm{(\bg_k + \tilde{\eta}_{k+1})\scalarp{\bg_k + \tilde{\eta}_{k+1}}{u_k}}\, .
\end{equation}
Equation~\eqref{eq:uk_taylor1} correspond to~\eqref{eq:stoch_app_u}. We now briefly prove the four points of the proposition. 

\emph{Claim on $(\bar{\eta}_k)$.} The fact that $\eta_{k}$ and thus $\bar{\eta}_k$ is $w_k$-measurable is immediate by its definition in~\eqref{eq:noise}. Additionally, $\bbE[\bar{\eta}^s_{k+1} |\cF_k] = \bbE[\eta_{k+1} |\cF_k] = 0$. Moreover, Assumption~\ref{hyp:conserv} and Equation~\eqref{eq:noise} implies
\begin{equation*}
\norm{\eta_{k+1}} \leq C_1\norm{\sum_{i=1}^n l'(p_i(w_k))\sa_i(w_k)} \leq C_2 \norm{w_k}^{L-1} \sum_{i=1}^n |l'(p_i(w_k))|\, ,
\end{equation*}
with $C_2>0$ some deterministic constant independent on $k$. Therefore, by~\eqref{eq:def_lmk_tilel} and \eqref{eqdef:proj_nois_subg}, $\norm{\bar{\eta}_{k+1}} \leq \norm{\tilde{\eta}_{k+1}} \leq C_2$.

\emph{Claim on $(\bgamma_k)$.} Almost surely, there is $\varepsilon >0$, such that for $k$ large enough, $\sm(u_k) \geq  \varepsilon$. Thus, for every $i$, $l'(p_i(w_k)) \leq e^{-\varepsilon\norm{w_k}^L }$. By Assumption~\ref{hyp:conserv}, $\norm{\sa_i(w_k)} \leq C \norm{w_k}^{L-1}$, which implies for $k$ large enough,
\begin{equation*}
  \bgamma_k \leq C_1 \norm{w_k}^{L-2} e^{-\varepsilon \norm{w_k}^L} \leq \frac{C_1 c_2 \log(k)^{L-2}}{k^{\varepsilon c_1}} \leq \frac{1}{\sqrt{k^{\varepsilon c_1}}}
\end{equation*}
where the penultimate inequality comes from Proposition~\ref{prop:log_wk}.
To show that $\sum_{k} \bgamma_k = + \infty$, note that using Equation~\eqref{eq:first_wk}, we have for $k$ large enough,
\begin{equation}\label{eq:sum_gamma_inf}
  \begin{split}
\norm{w_{k+1}}^2  &\leq \lVert w_{k} \rVert^2 + 2 \tilde{\gamma}_k \lVert w_{k} \rVert \lVert \bg_k + \tilde{\eta}_{k+1} \rVert  + \tgamma_k^2 \norm{\bg_k + \tilde{\eta}_{k+1}}^2\\
&\leq \norm{w_k}^2 (1 + C \bgamma_k + C_1 \bgamma_k^2) \leq \norm{w_k}^2 e^{C_2 \bgamma_k} \leq \norm{w_{k_0}}^2 e^{C_2 \sum_{i=k_0}^{k} \bgamma_i }
\end{split}
\end{equation}
where $k_0$ is large enough, and where we have used the fact that $\sup_{k} (\norm{\bg_k + \tilde{\eta}_{k+1}}) < + \infty$ and $\bgamma_k \rightarrow 0$. 
Since the left-hand side of Equation~\eqref{eq:sum_gamma_inf} goes to infinity by Proposition~\ref{prop:log_wk}, we obtain that the right-hand side diverge to infinity and therefore $\sum_{k } \bgamma_k = + \infty$.


\emph{Claim on $(r_k)$.} Since $\bgamma_k \rightarrow 0$, $\sup_{k} \norm{\tilde{\eta}_{k+1}} \leq C_1$ and $\sup_{k} \norm{\bg_k} < C_2$, there is $k_0$, such that for all $k \geq k_0$, $\bgamma_k (\norm{\bg_k} + \norm{\tilde{\eta}_{k+1}})\leq 1/2$. Therefore, for $k \geq k_0$, $|b_k| \leq C \gamma_k^2 \norm{\bg_k + \tilde{\eta}_{k+1}}$ in~\eqref{eq:pf_uk_stochapp}, which by~\eqref{eq:rk_bound} implies that $\sup_{k \geq k_0}\norm{r_k} \leq C$.

\emph{Claim on $\bar{D}$.} Consider a sequence $u_{k_j} \rightarrow u$ and $g$ any accumulation point $g_{k_j}$, we need to prove that $g - \scalarp{g}{u}u \in \bar{D}_s(u)$, or, equivalently, that $g \in \bar{D}(u)$. Recall that $\bg_k = \sum_{i=1}^k \lambda_{i,k} \bg_{i,k}$, where $\lambda_{i,k} \geq 0$, $\sum_{i} \lambda_{i,k} = 1$ and $g_{i,k} \in D_{i}(u_k)$. Extracting a subsequence, we can assume that for each $i$, $\lambda_{i, k_j} \rightarrow \lambda_i$ and $g_{i,k_j} \rightarrow g_i \in D_i(u)$.  We claim that $\lambda_i \neq 0 \implies p_i(u) = \sm(u)$. Indeed, without losing generality assume that $p_1(u_{k_j})\rightarrow \sm(u)$. Then, if $\lim (p_{i}(u_k) - \sm(u)) >0$, we obtain 
\begin{equation*}
  \lambda_{i,k_j} \leq \frac{l'(\norm{w_{k,j}}^L p_i(u_{k_j}))}{l'(\norm{w_{k_j}}^L p_1(u_{k_j}))} \leq C e^{-\norm{w_k}^L\left( p_i(u_k) - \sm(u)\right)}\xrightarrow[j \rightarrow + \infty]{} 0 \, .
\end{equation*}
Therefore, $g$ can be written as $ \sum_{i=1}^n \lambda_i g_i$, where $g_i \in D_i(u)$ and $\lambda_i \neq 0 \implies p_i(u) = \sm(u)$.
In other words, $g \in \bar{D}(u)$, concluding the proof. 

\hfill $\blacksquare$
