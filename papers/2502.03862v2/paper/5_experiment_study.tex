\section{Study 2: Assessing the Impacts of Multimodal Reflection Nudges on Deliberativeness}
In study 1, we identified the preferred modalities for each type of reflective nudge. To further understand how these modalities influence deliberativeness, we conducted study 2 to assess their impact on deliberative quality.

\subsection{Independent Variables and Experimental Design}
We used the same 2 $\times$ 4 design with Reflective Nudge: \{Direct (Persona), Indirect (Storytelling)\} and Modality: \{Text, Image, Video, Audio\}. Both independent variables were between-subject. Thus, participants were randomly assigned to one of the eight experimental conditions. 

The experimental interface retained the same design used in study 1 (Figure~\ref{fig: features}), but for this study, the \textit{Reflect} feature was limited to a single modality. 

Similar to study 1, we do not compare direct and indirect reflective nudges; instead, we evaluate the effects of different modalities within each nudge separately.

\subsection{Dependent Variables}
As deliberativeness is multi-dimensional, we operationalized it through five measurements as done by previous work~\cite{yeo2024help}: argument repertoire, argument diversity, rationality (opinion expression), rationality (justification level) and constructiveness as discussed in section~\ref{sec: measurements}.

All five metrics were derived from a content analysis of participants' responses by two coders. Both coders were PhD students with respectively 1 and 4 years of experience using content analysis. Cohen's Kappa was used to determine the agreement between the two coders’ judgments, with individual scores reported below. Kappa scores for all metrics were above the satisfactory threshold of 0.70~\cite{viera2005understanding, mchugh2012interrater}. 

The dependent variables are coded as follows:
\begin{itemize}
    \item \textit{Argument repertoire} ($\kappa = 0.915$) is the number of non-redundant arguments regarding each position of the discussion topic. The ideas produced along the two positions were combined.
    \item \textit{Argument diversity} ($\kappa = 0.915$) was coded by counting the number of unique themes present in the entire response. A higher diversity count indicates more varied perspectives present in the participant's responses~\cite{anderson2016all, gao2023coaicoder}.   
    \item \textit{Rationality (opinion)} ($\kappa = 0.872$) captures whether opinions are expressed or information is provided in the response. This was coded with two levels: 1) no opinions were expressed, rather, information was provided (score of 0); 2) an opinion, personal assertion or a claim was made (score of 1). This also includes evaluation, a personal judgment or assertion. 
    \item \textit{Rationality (justification level)} ($\kappa = 0.867$) captures the degree to which reasons are used to justify one's claims. This were coded at four levels: 1) no justification was provided (score of 0); 2) an inferior justification was made - this indicates that the opinion is supported with a reason in an associational way such as through personal experiences or an incomplete inference was given (score of 1); 3) a qualified justification was made when there is a single complete inference provided in the opinion (score of 2); 4) a sophisticated justification was made when at least two complete inferences was provided (score of 3).  
    \item \textit{Constructiveness} ($\kappa = 0.830$) captures the degree of balance within an opinion. This was coded at two levels: 1) the opinion is one-sided (score of 0); 2) the opinion is two-sided when multiple perspectives and viewpoints are presented (score of 1). 
\end{itemize}

\subsection{Power Analysis}
We conducted a power calculation for a eight-group ANOVA study seeking a medium effect size (0.30) according to Cohen’s conventions, at 0.80 observed power with an alpha of 0.05, giving $N=25$ per experimental condition, hence we recruited 200 participants. 

\subsection{Participants and Ethics}
A total of 200 participants were recruited through Amazon Mechanical Turk. Refer to Appendix Table~\ref{tab: st2-demo} on the breakdown of the demographic profile in each experimental condition. We ensured that the demographic profiles across the eight conditions were similar so as to control for any fixed effects resulting from the differences in demographic factors. Similar to study 1, we got ethics approval from our local IRB and reimbursed participants at an appropriate rate.

\subsection{Procedure and Task}
The procedure for this study closely follows that of study 1 outlined in section~\ref{sec: procedure}. In the pre-task phase, we gathered data on demographics and administered four questionnaires corresponding to each of the four covariates, as detailed in section~\ref{sec: Covariates}. Instructions on reflection were then provided to the participants.

In the main task, participants were instructed to utilize the \textit{Reflect} feature, which exclusively presents prompts from a single modality. The topic remained the same as study 1. We maintain consistency in the topic across the two studies to draw robust comparisons between different modalities and guarantee the internal validity of our results~\cite{cahit2015internal}.

In the post-task phase, participants completed a short survey in which they could provide feedback on the modality they engaged with.