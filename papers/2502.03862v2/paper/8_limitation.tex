\section{Limitations}
There are limitations in this work. Firstly, study 1 was conducted on young adults. Although this is a common target population for online deliberation platforms, these platforms are widely used in a range of populations. We mitigated this issue in study 2 with a larger and more diverse participants pool. Future work may focus on understanding the cultural aspects of self-reflection when using different modalities in reflective nudges from different populations.

The task employed in both studies was specific, aiming to achieve depth and clarity in understanding the influences of different modalities within a well-defined context~\cite{slack2001establishing}. While this establishes \textbf{internal validity}, it might conflict with the broader goal of generalizability and ecological validity. It is important to note that assessing ecological validity relies on first establishing internal validity~\cite{cahit2015internal, slack2001establishing, campbell2015experimental, cook1979quasi}. Therefore, having a focused investigation on a specific issue allows us to have a detailed examination of the modalities on deliberativeness. Future work could extend the application of the modalities to explore other contentious topics with varying complexity and nature.

Moreover, we utilized the VARK model to identify whether participants have a higher preference of one modality over others. While VARK has its limitations (see section~\ref{sec: learning and reflection}), we solely used the questionnaire for one specific goal: to examine modality preferences rather than making definitive conclusions about learning styles.

Additionally, deliberative processes can vary significantly in duration and complexity. Future studies could benefit from examining longer, multi-phased deliberations that unfold over days or weeks, allowing for deeper exploration of deliberative dynamics.

Lastly, this study investigates two types of reflective nudges — direct and indirect — implemented through persona and storytelling approaches. Our selection was guided by prior research demonstrating the effectiveness of these nudges in comparative evaluations~\cite{yeo2024help}. While other types of reflective nudges may exist, our primary objective was to examine whether different nudge types require distinct modalities to maximize their impact. Our findings underscore that specific modalities align better with particular reflective nudges, offering insights into tailoring nudges for improved reflection and engagement.