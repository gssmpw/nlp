\begin{abstract}
Nudging participants with text-based reflective nudges enhances deliberation quality on online deliberation platforms. The effectiveness of multimodal reflective nudges, however, remains largely unexplored. Given the multi-sensory nature of human perception, incorporating diverse modalities into self-reflection mechanisms has the potential to better support various reflective styles. This paper explores how presenting reflective nudges of different types (direct: persona and indirect: storytelling) in different modalities (text, image, video and audio) affects deliberation quality. We conducted two user studies with 20 and 200 participants respectively. The first study identifies the preferred modality for each type of reflective nudges, revealing that text is most preferred for persona and video is most preferred for storytelling. The second study assesses the impact of these modalities on deliberation quality. Our findings reveal distinct effects associated with each modality, providing valuable insights for developing more inclusive and effective online deliberation platforms.
\end{abstract}