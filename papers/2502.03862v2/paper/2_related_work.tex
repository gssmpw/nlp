\section{Background and Related Work} 
We begin by providing a background of our primary focus: internal reflection within the deliberation process and assessing deliberative quality. We then examine the role of modalities in supporting self-reflection, and their interaction with dual-system thinking. Finally, we review recent advancements in nudging techniques to enhance deliberative outcomes, along with the integration of reflection mechanisms with large language models, emphasizing the ethical implications of using AI in deliberation.

\subsection{Reflection in Deliberation}
\label{sec: reflection background}
Deliberation is a process that entails the careful and thoughtful consideration of diverse perspectives~\cite{davies2013online, goodin2003does, aristotle1984complete, hobbes1946}, allowing individuals to weigh reasons for and against a given measure~\cite{goodin2003does, oxford} before making informed choices grounded in thorough consideration and reasoned judgment~\cite{christiano2009debates}. Theorists of deliberative democracy emphasize that internal reflection is a foundational element of the deliberation process~\cite{muradova2021seeing, dryzek2002deliberative, goodin2000democratic, chambers2003deliberative}, preceding any public discussion~\cite{goodin2003does}. They assert that `\textit{deliberation is reflective rather than simply reactive}'~\cite{bohman2000public, dahlberg2001computer, janssen2004online}, requiring the collection of diverse information and development of a nuanced understanding of multiple perspectives, rather than an immediate response to an issue~\cite{davies2009online}. This pre-discussion, internal introspective phase is thus essential for individuals to clarify and solidify their stance, enabling more thoughtful contributions in subsequent discussions~\cite{goodin2003does, holdo2020meta}. 

Dryzek~\cite{dryzek2009democratization} proposed reflection as a metric to evaluate the reflective capacity of deliberative systems. Such a metric considers individuals' ability to incorporate others' perspectives and demonstrate public-mindedness and sincerity in their deliberations~\cite{holdo2020meta}. An early example is PICOLA (Public Informed Citizen Online Assembly)~\cite{cavalier2009deliberative} which comes with a reflection phase where participants could think about issues, and then discuss in an asynchronous forum. The results support an online conversation that is informed and structured, though it was not formally published~\cite{cavalier2009deliberative}. Similarly, prior research has explored novel interfaces that incorporate various reflective approaches to enhance deliberation~\cite{arceneaux2017taming}. For example, in Reflect~\cite{kriplean2012you}, an interface prompting users to restate points made by other commenters (e.g., ``\textit{What do you hear James [a commenter] saying?}''), fosters mutual understanding through clarification and dialogue. The same authors also extended Franklin's pro/con technique~\cite{franklin1956mr} for internal reflection to encourage participants to organize their reasoning into a list of pros and cons (i.e., ``\textit{Give your Cons. Give your Pros.}''~\cite{considerConsideritOnly, kriplean2012supporting, kriplean2011considerit}). This has helped participants to identify unexpected common ground. In another approach, \textbf{perspective-taking} is used to consider the views of various stakeholders~\cite{tuller2015seeing}. Kim et al.\cite{kim2019crowdsourcing} employed this via textual open-ended questions, to ``\textit{Guess the perspective of some other stakeholder groups (at least two)}.'' Similarly, the \textbf{imagined other} reflection approach~\cite{batson1997perspective, galinsky2000perspective} prompts participants to construct narratives from the perspective of an imagined person. Zhang et al.~\cite{zhang2021nudge} observed how \textbf{reflective questions} like ``\textit{What are your opinions on [an issue]?}'' and ``\textit{How might others disagree with you on [the same issue]?}'' improve attitude clarity and correctness, and encourage opinion expression. Additionally, Price et al.~\cite{price2002does} examined how \textbf{reflective prompts} (e.g. ``\textit{articulate the reasoning behind others' opinions}''), foster deeper deliberativeness. More recently, Yeo et al.~\cite{yeo2024help} found that textual reflective nudges, such as the textual display of other stakeholders' opinions or posing open-ended questions (e.g., ``\textit{How has your perspective on the impact of [an issue] changed as you have grown?}'') have significant positive effects on deliberativeness and can shape the dynamics of online discussions. Collectively, these empirical studies support two conclusions: (1) reflection approaches influence deliberativeness, a normatively desired effect for deliberation~\cite{bohman2000public, zhang2021nudge} and (2) most reflection approaches rely on the conventional text-based formats.

While the above approaches have effectively broaden the scope of reflection, they remain limited to text-based formats. This reliance overlooks potential benefits of a multimodal approach that could accommodate different cognitive and reflective styles~\cite{mayer2005cambridge}. In this study, we thus focus on examining the efficacy of different modalities in reflective approaches. By building on these foundational approaches, we expand the toolkit for facilitating reflection in deliberation and contributes to a broader understanding of the effects of distinct modalities on the deliberative process. 

\subsection{Assessing Deliberative Quality: Deliberativeness}
\label{sec: measurements}
Deliberativeness refers to the quality of deliberation made by an individual~\cite{trenel2004measuring}, a cornerstone of effective public discussions~\cite{menon2020nudge}. The term has been defined in diverse ways in the literature~\cite{graham2003search, price2002does, steenbergen2003measuring, stromer2007measuring}, with some papers~\cite{bohman2000public, price2002does, zhang2021nudge} employing the term `\textit{opinion quality}' interchangeably. In our study, we define deliberativeness as the quality of an individual's opinion~\cite{menon2020nudge}.

As a multifaceted construct~\cite{menon2020nudge, zhang2021nudge}, deliberativeness encompasses several dimensions~\cite{trenel2004measuring, price2002does, steenbergen2003measuring, stromer2007measuring}, with \textit{rationality} and \textit{constructiveness} forming the core conditions of deliberation~\cite{trenel2004measuring}. Rationality includes both \textit{opinion expression} and \textit{justification level}. Opinion expression evaluates the clarity and presence of articulated opinions necessary for meaningful deliberation~\cite{trenel2004measuring}. Originally developed for group contexts, opinion expression underpins rational deliberation~\cite{beck2000argumentation}, playing a pivotal role in shaping the effectiveness and productivity of subsequent discussions and decision-making~\cite{beck2000argumentation}. Bales~\cite{bales1983overview} connects opinion expression to the requirement for logical, well-reasoned arguments, while Fietkau and Trénel~\cite{fietkau2002interaktionsmuster} emphasize that clearly articulated opinions ensures structured, objective discussions by fostering clarity and substantive contributions. 
Justification level assesses the extent of \textit{reasoning} provided to support one's opinion~\cite{trenel2004measuring}. The ability to \textit{reason} is a recurring theme in Dryzek's work~\cite{dryzek2002deliberative}, where he argues that deliberation hinges on the act of reasoning, with the power of deliberative democracy lying in the ability to reason --- what Habermas~\cite{habermas2015between} famously described as ``\textit{the force of the better argument}''. 

Constructiveness, on the other hand, reflects the degree to which an individual integrates diverse perspectives and seeks common ground~\cite{goddard2023textual, trenel2004measuring}, moving beyond purely self-interested arguments~\cite{kymlicka2002contemporary}. It is evaluated based on whether opinions incorporate an inclusive consideration of varied perspectives or remain narrow and one-dimensional~\cite{trenel2004measuring, black2011self, gastil2007public}. 
Philosophical traditions, such as those of Plato and Aristotle~\cite{walton1999one}, and Garver~\cite{garver2004sake}, advocate the value of balanced, two-sided arguments for constructive deliberation, fostering understanding and cooperation. Similarly, Hackett and Zhao~\cite{hackett1998sustaining} argue that constructiveness involves presenting opposing views fairly to encourage diverse perspectives. Gastil~\cite{gastil2008political} highlights the importance of equitable argumentation in democratic deliberation, advocating for a mix of viewpoints. Niculae and Niculescu-Mizil~\cite{niculae2016conversational} found that constructive contributions are often more balanced, aligning with Parkinson and Mansbridge~\cite{parkinson2012deliberative}, who stress that inclusive discourse enhances deliberative outcomes.
Together, rationality and constructiveness serve as the core conditions for effective deliberation, ensuring both reasoned and inclusive arguments.

In addition to rationality and constructiveness, two other essential dimensions --- argument repertoire and argument diversity --- are integral to this definition, as they directly influence the quality of deliberation. \textit{Argument repertoire} measures the quality of opinions by counting non-redundant arguments presented both for and against an issue --- an approach widely adopted in deliberative research~\cite{menon2020nudge, zhang2021nudge, cappella2002argument, kim2021starrythoughts, yeo2024help}. Meanwhile \textit{argument diversity} measures the range of perspectives within an opinion. It reflects the inclusion of multiple perspectives and interpretations within an individual's opinion~\cite{anderson2016all, gao2023coaicoder, richards2018practical}, highlighting the breadth of viewpoints considered. This measure also aligns with constructiveness, as a broad range of arguments often demonstrates an openness to differing perspectives and a willingness to engage with opposing views. 

Building on prior work, deliberativeness in this study is assessed through the interplay of these five measures: rationality (comprising opinion expression and justification level), constructiveness, argument repertoire, and argument diversity. These measures collectively capture the depth, inclusivity, and quality of opinions within the deliberative process.

\subsection{Modalities, Learning and Self-Reflection}
\label{sec: learning and reflection}
Research on multimodality (i.e., the use of multiple content representations such as text, video, audio, images and interactive elements~\cite{sankey2010engaging}) has extensively demonstrated its effectiveness within learning environments~\cite{mayer2005cambridge, birch2005students, williamson2010learning, sankey2009ethics, sankey2010engaging, pashler2008learning, miller2001learning, mestre2010matching, picciano2009blending, sankey2011impact, dolzhich2019multimodality, gargallo2018perceptual, borzello2018benefits, schwartz2014s}, with consistent findings highlighting its benefits for information processing~\cite{bartholomeus1972acquisition, birch1964auditory, lawton1973developmental}, enhancing comprehension and problem solving~\cite{sankey2010engaging, schroeder2013effective, fadel2008multimodal}, as well as cognitive outcomes~\cite{constantinidou2002stimulus}. Key frameworks like Dual Coding Theory~\cite{paivio1990mental, bodemer2002encouraging, mayer2002multimedia, sweller1988cognitive} explain that combining visual and verbal inputs facilitates better encoding and retrieval by reducing cognitive load. This aligns with Moreno and Mayer~\cite{moreno1999cognitive}, who found that mixed-modality presentations yield better outcomes by leveraging the simultaneous processing of auditory and visual inputs in working memory~\cite{paivio2013imagery}. Mayer’s Cognitive Theory of Multimedia Learning~\cite{mayer2005cambridge} underscores the ``\textit{multimedia principle}'', showing that presenting information through words and images is more effective than text alone. Neuroscience research further supports these findings, revealing significant learning gains through integrated visual and verbal strategies~\cite{fadel2008multimodal}. Findings in the field of cognitive science suggest that ``\textit{intelligences and mental abilities exist along a continuum, responding to and learning from the external environment and instructional stimuli}''~\cite{gardner2011frames}, suggesting a framework for a multimodal instructional design to cater to varying cognitive processes and learning preferences~\cite{picciano2009blending}. These insights highlighted that multimodal learning not only encourages learners to develop versatile learning strategies~\cite{hazari2004applying, picciano2009blending} but also demonstrates that engaging multiple senses leads to more effective learning outcomes~\cite{kearsley2000online}.

Learning style theory further explores how individual preferences for modalities influence learning outcomes. Recent research indicates that tailoring educational strategies to these styles can significantly impact academic success~\cite{newble1985learning}. %Early studies, such as Galton's investigation on visual and verbal thinkers~\cite{galton1883inquiries}, suggest that individuals naturally select preferred modalities to interact with their environment. Bartlett~\cite{bartlett1995remembering} concluded that while individuals can adapt to various cues --- visual, auditory or kinesthetic --- they tend to favor one modality if left unprompted. This preference is often guided by selective filtering~\cite{bissell1971sensory, bruininks1969auditory}, where individuals prioritize certain modality inputs over others. 
A framework that is widely adopted for understanding modality selection in learning is the VARK model~\cite{fleming2001vark}, which categorizes learners into Visual (V), Auditory (A), Reading/Writing (R) and Kinesthetic (K) types~\cite{fleming1992not, fleming1995m, fleming2001vark}. %An extension of this is the ``\textit{meshing hypothesis}''~\cite{pashler2008learning, rogowsky2015matching}, which suggests that aligning instruction with a learner's preferred style improves learning outcomes. 
While empirical support for the framework is limited~\cite{pashler2008learning, rogowsky2015matching}, studies like Daniel and Tacker\cite{daniel1974preferred} and Waters~\cite{waters1972analysis} reported modest benefits when instructional methods matched modality preferences. Despite critiques of the VARK model%and the meshing hypothesis~\cite{pashler2008learning}
, VARK remains popular due to its practical applications in learning~\cite{sulistyanto2023effectiveness, noor2023bridging, el2024influence, laxman2014exploration} and its ability to foster multimodal engagement~\cite{sulistyanto2023effectiveness}. Aligned with VARK, the Dunn and Dunn Learning-Style Model~\cite{dunn1993teaching, dunn1984learning} categorizes learning styles into five stimuli: environmental, sociological, emotional, physiological (similar to VARK), and psychological. Meta-analyses~\cite{lovelace2005meta, dunn1995meta}, specifically one with 36 experimental studies~\cite{dunn1995meta}, alongside additional research~\cite{oweini2016effects}, affirm that tailoring instruction to these styles lead to significant gains in academic performance and learner satisfaction. These findings highlight that multimodal engagement enhances academic outcomes and fosters greater adaptability, reinforcing the value of personalized learning strategies.

While the role of multimodality in learning is well-documented, its broader implications, particularly in fostering reflection has received comparatively less attention. Since reflection is central to learning~\cite{kolb2014experiential}, it stands to reason that reflection may also benefit from a multimodal approach. Reflection is dynamic, involving deep engagement with diverse inputs, much like how learners engage with multimedia to absorb content~\cite{moon2013handbook}. Educational psychology studies also support that multimodal approaches can scaffold reflective processes, fostering critical and holistic thinking~\cite{fleming2001vark, mayer2005cambridge}.

For this study, we thus seek to assess how multimodal inputs (such as text, audio, images and video) influence reflective processes and their impacts on deliberative outcomes. By integrating insights from cognitive and educational psychology, we aim to explore how multimodality fosters deeper engagement and more effective reflection.

\subsection{Modalities and Dual-System Thinking}
\label{sec: dual system thinking}
Dual-system theories of thinking and decision-making classify cognitive processes into two primary systems: System 1, which is intuitive, fast, automatic, and emotional, operating unconsciously, and System 2, which is reflective, deliberate, analytical and effortful~\cite{kahneman2011thinking, strack2004reflective, sloman1996empirical}. %System 1 is the principal mode of thinking, dominating routine, quick decisions that require minimal effort such as walking or driving, while System 2 is engaged for complex reasoning that requires conscious effort~\cite{kahneman2011thinking}. Although System 2 supports critical thinking, System 1 excels in speed and multitasking~\cite{zavolokina2024think}, handling about 95\% of daily decisions through \textit{heuristics} --- mental shortcuts that simplify judgments~\cite{bargh2001automated}. 
Due to our predisposition to reduce effort, System 1
%these heuristics 
enables efficient decision-making% by using readily available information, often 
, yielding relatively accurate judgments~\cite{shah2008heuristics}, but
%while conserving cognitive effort and bypassing the need for deliberate reflection~\cite{shah2008heuristics}. However, relying on heuristics can 
exposes us to cognitive biases --- systematic errors that skew judgments from rationality~\cite{caraban201923, kahneman1991anomalies}. % For example, the status-quo bias causes individuals to favor default options over alternatives, even when those defaults are suboptimal~\cite{kahneman1991anomalies}. Sloman~\cite{sloman1996empirical} underscores the interplay between these two systems, emphasizing their complementary roles. Together, the two systems balance efficiency and critical analysis, enabling humans to navigate reasoning tasks that require both intuition and logic~\cite{kahneman2011thinking, sloman1996empirical}. 
 Consequently, we often need to explicitly engage System 2 for careful, rational deliberation~\cite{kahneman2011thinking, sloman1996empirical}.

In our study, we employ the dual-system theories as a framework to examine how different modalities support System 1 or System 2 thinking. Guided by Kahneman’s six distinct characteristics of dual-system thinking~\cite{kahneman2011thinking, kahneman2002maps, kahneman2012two} (see Table~\ref{tab: dual system thinking}), we adopt these characteristics to analyze how different modalities interact and best facilitate the two cognitive systems, helping us identify which best facilitate intuitive (System 1) or reflective (System 2) thinking.

\begin{table*}[!htbp]
\caption{Six characteristics of Dual-System Thinking based on~\cite{kahneman2011thinking, kahneman2002maps, kahneman2012two}. The sections highlighted in blue illustrate how these characteristics apply in our study to understand how different modalities interact with the two cognitive systems.}
\label{tab: dual system thinking}
\scalebox{0.77}{
\begin{tabular}{|c|l|c|c|}
\hline
\textbf{Characteristic} &
  \textbf{Definition} &
  \textbf{System 1} &
  \textbf{System 2} \\ \hline
Speed &
  \begin{tabular}[c]{@{}l@{}}Speed of thinking\\ {\color[HTML]{3166FF} Ability of the modality to allow users to process information quickly (System 1) or slowly (System 2).}\end{tabular} &
  Fast &
  Slow \\ \hline
Processing &
  \begin{tabular}[c]{@{}l@{}}Approach to handling thinking tasks\\ {\color[HTML]{3166FF} Ability of the modality to influence whether users engage in parallel processing, focusing on the big picture} \\ {\color[HTML]{3166FF} and broader concepts (System 1), or in serial processing, emphasising detailed, step-by-step analysis (System 2).}\end{tabular} &
  Parallel &
  Serial \\ \hline
Control &
  \begin{tabular}[c]{@{}l@{}}Degree of conscious oversight\\ {\color[HTML]{3166FF} Ability of the modality to enable users to engage in automatic, intuitive reflection (System 1) or deliberate, more} \\ {\color[HTML]{3166FF} conscious control over the reflection process (System 2).}\end{tabular} &
  Automatic &
  Controlled \\ \hline
Effort &
  \begin{tabular}[c]{@{}l@{}}Cognitive load\\ {\color[HTML]{3166FF} Ability of the modality to engage users with varying levels of cognitive effort, ranging from low mental effort (System 1)} \\ {\color[HTML]{3166FF} to high mental effort and concentration (System 2).}\end{tabular} &
  Effortless &
  Effortful \\ \hline
Nature &
  \begin{tabular}[c]{@{}l@{}}Inherent operating mechanism\\ {\color[HTML]{3166FF} Ability of the modality to align with users' inherent reflecting styles and guide them toward either intuitive thinking} \\ {\color[HTML]{3166FF} (System 1) or rule-governed, analytical reasoning (System 2).}\end{tabular} &
  Associative &
  Rule-governed \\ \hline
Adaptability &
  \begin{tabular}[c]{@{}l@{}}Ability to change or evolve\\ {\color[HTML]{3166FF} Ability of the modality to accommodate users' needs by adapting to the complexity of topics and varying levels} \\ {\color[HTML]{3166FF} of prior knowledge, allowing for either slow-learning, rigid thinking (System 1) or flexible, adaptive reasoning (System 2).} \end{tabular} &
  Slow-learning &
  Flexible \\ \hline
\end{tabular}}
\Description{The table provides a summary of the six characteristics of dual-system thinking: speed, processing, control, effort, nature, and adaptability. Each characteristic is defined, and the table illustrates how these characteristics are applied within the context of our study. Additionally, the table shows a comparison between System 1 (intuitive, fast, and automatic thinking) and System 2 (deliberative, slow, and controlled thinking) to explain the varying cognitive approaches influenced by different modalities.}
\end{table*}

\subsection{Nudging}
\label{sec: nudging}
Thaler and Sunstein~\cite{thaler2008nudge, Thaler_2009} introduced the notion of nudging to guide optimal decision-making by leveraging systematic biases in reasoning. A nudge is any modification to the choice architecture that predictably influences behavior without restricting options or altering incentives significantly~\cite{caraban201923}. Nudges work by making certain choices easier or more accessible, subtly steering individuals toward better decisions~\cite{thaler2008nudge}. By altering how choices are presented, nudges effectively intervene and redirect individuals away from their habitual modes of thought~\cite{thaler2008nudge, Thaler_2009}.% For instance, switching organ donation policies from opt-in to opt-out significantly enhanced societal welfare~\cite{caraban201923}, while strategically placing fruits near checkout counters increased healthier purchases, even when less healthier options such as cake remained available~\cite{Thaler_2009}. Both approaches preserved individual freedom of choice while subtly guiding behavior~\cite{thaler2008nudge}.

The idea of nudging has been applied across various domains, including HCI~\cite{caraban201923, lee2011mining}. Caraban et al.\cite{caraban201923} reviewed 71 HCI studies, identifying 23 distinct nudging mechanisms, categorized into six groups. Hansen and Jespersen~\cite{hansen2013nudge} further categorized nudges into four types based on two factors: the mode of thinking (automatic or reflective which is linked to the dual-system theories; see section~\ref{sec: dual system thinking}) and transparency of the nudge (whether the user perceives the nudge's intent). Nudges in the form of simple interface changes have also shown significant effects, such as Harbach et al.'s~\cite{harbach2014using} redesign of Google Play Store's permissions dialogue to encourage users to consider privacy risks in apps, as well as Wang et al.'s~\cite{wang2014field} use of visual cues and timers to reduce regret posting in online disclosures.

Within the realm of deliberation, nudges involving simple interface changes have proven effective in enhancing deliberativeness~\cite{xiao2015design, zhang2013structural}. Murray et al.~\cite{murray2013supporting} designed three reflective tools to enhance quality deliberations by fostering what they called ``social deliberative skills'' like perspective-taking and reflecting on one's biases. Their findings revealed that simple designs of the tools acted as scaffolding for social deliberative skills, enhancing quality deliberations. In other deliberative studies, Menon et al.~\cite{menon2020nudge} showed that interface nudges like partitioned text fields increased reply length by 35\% and argument count by 25\%. Similarly, Zhang et al.~\cite{zhang2021nudge} explored reflective nudges through question prompts such as ``What are your opinions on this issue?'', and found that they improved opinion quality and enhanced opinion expression. These studies highlighted the impact of small interface adjustments on fostering higher quality deliberation.

Drawing from existing efforts, we utilize the concept of nudges to improve the overall deliberativeness of online discourse. For this specific case, we examine how different modalities enacted on interface-based reflective nudges influence various dimensions of deliberative quality, anticipating that they enhance deliberativeness.

\subsection{Leveraging Generative AI for Reflection: Opportunities and Ethical Considerations}
The widespread adoption of information and communication technologies (ICTs)~\cite{janssen2018innovating}, including AI such as large language models (LLMs)~\cite{duberry2022artificial}, has significantly advanced deliberation and citizen-government relations~\cite{duberry2022artificial, lovejoy2012information, misuraca2020ai, mehr2017artificial, chun2012social, leach2010dynamic}. In reflective deliberation, LLMs' ability to generate personalized responses offers unique opportunities for fostering reflective thought at scale~\cite{duberry2022artificial, fogg2002persuasive, jakesch2023co}. While their use in this context remains nascent, early work by Yeo et al.~\cite{yeo2024help} demonstrated the potential of LLMs for broadening perspectives through textual reflection. However, the role of non-textual modalities has yet to be explored. In this study, we extend this work by integrating GPT-4.0 with generative AI for multimedia modalities to stimulate reflection and support deliberation on societal issues.

Despite these opportunities, LLMs are accompanied by risks. As LLMs become embedded in human communication~\cite{bommasani2021opportunities}, generating human-like language for widely-used applications like writing support~\cite{dang2022beyond, jakesch2023co, hancock2020ai} and grammar correction~\cite{koltovskaia2020student}, they influence opinions through a process termed \textit{latent persuasion}~\cite{jakesch2023co}. Similar to nudging (see section~\ref{sec: nudging}), latent persuasion subtly shapes decision-making by leveraging machine-generated language, underscoring AI's profound impact on thought processes~\cite{fogg2002persuasive, leonard2008richard}. With AI's growing prevalence, ethical concerns about its potential to amplify societal biases are rising~\cite{stahl2016ethics}. LLMs trained on biased language patterns~\cite{caliskan2017semantics, bolukbasi2016man, duberry2022artificial} risk perpetuating stereotypes and societal prejudices~\cite{brown2020language, lucy2021gender, huang2019reducing, nozza2021honest, johnson2022ghost, langer2023trust, bowen2006information, labajova2023state, cotter2019playing}, shaping user perceptions and reinforcing inequalities. The rise of deepfake technology, which creates realistic yet fabricated media, further complicates these concerns~\cite{karnouskos2020artificial, verdoliva2020media}.

Our study does not focus on using generative AI to verify the veracity of any content but instead as a tool to stimulate reflection. While addressing these inherent risks is beyond the scope of this work, it is still crucial to ensure ethical and responsible AI use. To mitigate the risks of stereotyping and biases in AI-generated content, our study employs carefully designed prompts (detailed in section~\ref{sec: section3} and Appendix Tables~\ref{tab: prompt engineering} and~\ref{tab: prompt constraint rationale}) to encourage open, inclusive reflections. For every reflective nudge, we generate multiple stakeholders, to make sure that multiple sides are represented during the reflection process. Doing so, we aim to promote deeper engagement and broader perspective-taking, fostering nuanced deliberation without influencing conclusions or imposing definitive answers. 