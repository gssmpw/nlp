\section{Introduction}

In the realm of online deliberation, where public discussion is often perceived as inflammatory and hyperbolic~\cite{diakopoulos2011towards}, reflection is recognized as a critical element for enhancing the quality of discourse~\cite{muradova2021seeing, dryzek2002deliberative, goodin2000democratic, chambers2003deliberative, goodin2003does}. Online deliberation platforms are spaces for users to engage in thoughtful discussions on societal issues~\cite{jacobs2009talking}, often requiring participants to reflect on their own views~\cite{zhang2021nudge, goodin2003does, bohman2000public, goodin2000democratic} before collaboratively engaging with others to reach consensus~\cite{dekker2015contingency}. By fostering reflection, these platforms aim to improve the quality of individual contributions, leading to more meaningful and well-rounded deliberation.

Current approaches to reflection within online deliberation have predominantly relied on textual methods, such as creating pro/con lists~\cite{kriplean2012supporting}, engaging in perspective-taking exercises~\cite{kim2019crowdsourcing}, or responding to reflective prompts~\cite{zhang2021nudge}. While these studies demonstrate that text-based reflective nudges can positively influence the quality of individual contributions and, by extension, the overall deliberative process, they do not consider the potential of multimodal approaches to support diverse reflection styles. 

Given that individuals process information through various sensory channels~\cite{mayer2005cambridge}, integrating multimodal reflection mechanisms could further enrich online deliberation experiences~\cite{mayer2005cambridge}. While the role of multimodality in learning is well established in academic contexts (e.g.,~\cite{mayer2005cambridge}), its potential for fostering reflection in deliberative settings has not been as widely explored. The theoretical bridge between multimodal learning and reflection lies in the shared cognitive mechanisms that underpin both processes. This cognitive overlap suggests that reflection, much like multimodal learning, can be deepened by engaging multiple sensory channels~\cite{moon2013handbook}. We thus propose that individuals have distinct \textit{reflecting styles} akin to their learning styles, which can be effectively supported through multimodal reflective nudges. To our knowledge, no prior work has systematically explored how multimodal reflective nudges can improve deliberative outcomes.

To this end, we seek to address this gap by examining the impacts of different modalities on the quality of deliberation. Broadly, we ask the question, \textit{``How does the modality of reflection nudge affects deliberation quality?''} Our investigation involves designing and implementing an interface intended to induce self-reflection. Informed by literature~\cite{mayer2005cambridge, fadel2008multimodal, paivio1990mental, fleming2001vark, fleming1992not}, we focus on four different modalities (text, image, video, and audio) and applied them to two types of reflective nudges (Direct: Persona and Indirect: Storytelling) to facilitate their integration into existing online deliberation platforms. To support our examination in this aspect, we pose the following research questions: \textbf{RQ1: What are the preferred modalities for each of the reflective nudges?} and \textbf{RQ2: How does the modality of a reflective nudge affect the quality of deliberation?}

We conducted two user studies to address each question. The first study, involving 20 participants, explored the preferred modalities for reflective nudges using an early prototype of our system, powered by a Large Language Model (LLM) (i.e., GPT-4.0) and Generative AI. Participants engaged with reflective prompts in text, image, video, and audio formats, and we gathered insights on their preferences and comfort levels. Notably, text was the most preferred modality for direct reflective nudges (persona), while video was preferred for indirect reflective nudges (storytelling). The second study, an experimental investigation involving 200 participants, evaluated the impact of different modalities on deliberation quality. We developed eight interfaces, each integrating a distinct modality with a reflective nudge, and measured their impacts on deliberative outcomes. Our results highlighted that subjectively preferred modalities do not always produce higher levels of deliberativeness. Specifically, video leads to higher level of deliberative quality for persona. Our results also revealed that video worked well with both reflective nudges. We discuss later how different nudge types may have an \textit{`optimal'} modality, and how modality influences the dynamics of online deliberation.

We make the following contributions:
\begin{itemize}
    \item The design and integration of multimodal self-reflection interface nudges into online deliberation platforms, using LLMs for adaptability and diversification.
    \item An in-depth analysis of the modalities, highlighting the nuanced ways in which different modalities influence the quality and depth of discourse in online deliberation spaces.
\end{itemize}