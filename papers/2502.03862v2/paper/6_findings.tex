\section{Results (Study 2)}
 
\subsection{Quantitative Results}
\label{sec: quantitative}
Before analyzing our data, we plotted a box plot (box and whisker plot) to visually show the dispersion of our data and to identify any potential outliers~\cite{schwertman2004simple, dawson2011significant}. No abnormalities in the data were observed. 

ANCOVA was then used to identify main effects while controlling for the four covariates. We applied pairwise t-tests with Benjamini-Hochberg correction for post-hoc comparisons for the eight measures of deliberativeness. We did not use Bonferroni correction, as its conservative approach leads to high rates of false negatives when done with large number of comparisons~\cite{thissen2002quick}. Instead, we relied on Benjamini-Hochberg as it minimizes the problem~\cite{nakagawa2004farewell} while still accounting for multiple comparisons. We report effects of covariates only where they are significant.

Results are summarized in Tables~\ref{tab: summary quantitative direct} and ~\ref{tab: summary quantitative indirect}.

\subsubsection{Argument Repertoire}

\paragraph{Direct Reflective Nudge} We found a significant main effect of \emph{Modality} on \emph{Argument Repertoire} for direct reflective nudge ($F_{3,86} = 4.87$, $p<.01$). There was a statistical difference between \emph{Video} ($M = 3.60$ arguments) and \emph{Text} ($M = 2.32$ arguments), \emph{Image} ($M = 2.56$ arguments) and \emph{Audio} ($M = 2.60$ arguments) (all $p<.05$). %Results are summarized in Figure~\ref{fig: argument repertoire}.

\paragraph{Indirect Reflective Nudge} No significant main effect of \emph{Modality} on \emph{Argument Repertoire} was found for indirect reflective nudge ($p=.09$).
%Results are summarized in Figure~\ref{fig: argument repertoire}.

% \begin{figure*}[!htbp]
%   \centering
%   \includegraphics[width=.8\textwidth]{figures/Argument_Repertoire.png}
%   \caption{Argument Repertoire for direct reflective nudge (left) and indirect reflective nudge (right). We report the results of the ANCOVA test, and pairwise comparisons with BH correction if any, where * : p < .05, ** : p < .01.}
%   \label{fig: argument repertoire}
%   \Description{Box plot describing argument repertoire for the four modalities for direct reflective nudge (left) and indirect reflective nudge (right). The X-axis shows the different modalities and the Y-axis shows the number of arguments.}
% \end{figure*}

\subsubsection{Argument Diversity}

\paragraph{Direct Reflective Nudge} No significant main effect of \emph{Modality} on \emph{Argument Diversity} was found for direct reflective nudge ($p=0.085$). However, individual's inherent reflecting styles, particularly a lower preference for text, was associated with higher argument diversity ($p<.05$). %Results are summarized in Figure~\ref{fig: argument diversity}.

\paragraph{Indirect Reflective Nudge} We found a significant main effect of \emph{Modality} on \emph{Argument Diversity} for indirect reflective nudge ($F_{3,86} = 3.41$, $p<.05$). We observed pairwise differences between \emph{Video} ($M = 5.52$ themes) and \emph{Image} ($M = 3.72$ themes, $p<.05$) only.
%Results are summarized in Figure~\ref{fig: argument diversity}.

% \begin{figure*}[!htbp]
%   \centering
%   \includegraphics[width=.8\textwidth]{figures/Argument_Diversity.png}
%   \caption{Argument Diversity for direct reflective nudge (left) and indirect reflective nudge (right). We report the results of the ANCOVA test, and pairwise comparisons with BH correction if any, where * : p < .05, ** : p < .01.}
%   \label{fig: argument diversity}
%   \Description{Box plot describing argument diversity for the four modalities for direct reflective nudge (left) and indirect reflective nudge (right). The X-axis shows the different modalities and the Y-axis shows the number of argument diversity.}
% \end{figure*}

\subsubsection{Rationality (Opinion)}

\paragraph{Direct Reflective Nudge} No significant main effect of \emph{Modality} on \emph{Opinion} was found ($p=0.224$). However, a higher tendency to self-reflect ($p<.01$), lower inherent preference for audio ($p<.01$) and reduced external exposure to images ($p<.05$) were associated with higher expression of opinions. %Results are summarized in Figure~\ref{fig: opinion}.

\paragraph{Indirect Reflective Nudge} We found a significant main effect of \emph{Modality} on \emph{Opinion} for indirect reflective nudge ($F_{3,86} = 4.61$, $p<.01$). Overall, we found differences between \emph{Text} ($M = 0.48$) versus \emph{Image} ($M = 0.76$), \emph{Video} ($M = 0.84$) and \emph{Audio} ($M = 0.84$) (all $p<.05$). %Results are summarized in Figure~\ref{fig: opinion}.

% \begin{figure*}[!htbp]
%   \centering
%   \includegraphics[width=.8\textwidth]{figures/Opinion.png}
%   \caption{Rationality (Opinion) for direct reflective nudge (left) and indirect reflective nudge (right). We report the results of the ANCOVA test, and pairwise comparisons with BH correction if any, where * : p < .05, ** : p < .01.}
%   \label{fig: opinion}
%   \Description{Box plot describing opinion expression for the four modalities for direct reflective nudge (left) and indirect reflective nudge (right). The X-axis shows the different modalities and the Y-axis shows the level of opinion expression.}
% \end{figure*}

\subsubsection{Rationality (Justification Level)}

\paragraph{Direct Reflective Nudge} No significant main effect of \emph{Modality} on \emph{Justification Level} was observed ($p = 0.571$). However, higher inherent aural preferences and lower inherent text preferences as well as greater external exposure to videos (all $p<.05$) are associated with higher levels of justification. % Results are summarized in Figure~\ref{fig: justification level}.

\paragraph{Indirect Reflective Nudge} We found a significant main effect of \emph{Modality} on \emph{Justification Level} for indirect reflective nudge ($F_{3,86} = 3.39$, $p<.05$). There was a statistical difference between \emph{Video} ($M = 2.32$) and \emph{Image} ($M = 1.44$, $p<.05$).
% Results are summarized in Figure~\ref{fig: justification level}.

% \begin{figure*}[!htbp]
%   \centering
%   \includegraphics[width=.8\textwidth]{figures/Justification_Level.png}
%   \caption{Rationality (Justification Level) for direct reflective nudge (left) and indirect reflective nudge (right). We report the results of the ANCOVA test, and pairwise comparisons with BH correction if any, where * : p < .05, ** : p < .01.}
%   \label{fig: justification level}
%   \Description{Box plot describing justification level for the four modalities for direct reflective nudge (left) and indirect reflective nudge (right). The X-axis shows the different modalities and the Y-axis shows the level of justification.}
% \end{figure*}

\subsubsection{Constructiveness}

\paragraph{Direct Reflective Nudge} No significant main effect of \emph{Modality} on \emph{Constructiveness} was observed ($p = 0.126$). However, a higher inherent \emph{aural} ($p<.01$) and \emph{image} ($p<.05$) preferences as well as reduced external exposure to text ($p<.05$) are associated with higher constructiveness.
%Results are summarized in Figure~\ref{fig: constructiveness}.

\paragraph{Indirect Reflective Nudge} No significant main effect of \emph{Modality} on \emph{Constructiveness} was observed ($p = 0.169$). 
%Results are summarized in Figure~\ref{fig: constructiveness}.

% \begin{figure*}[!htbp]
%   \centering
%   \includegraphics[width=.8\textwidth]{figures/Constructiveness.png}
%   \caption{Rationality (Justification Type) for direct reflective nudge (left) and indirect reflective nudge (right). Error bars show .95 confidence intervals. We report the results of the ANCOVA test, and pairwise comparisons with BH correction if any, where * : p < .05, ** : p < .01.}
%   \label{fig: constructiveness}
%   \Description{}
% \end{figure*}

\begin{table*}[!htbp]
\caption{Deliberative quality across the four modalities for direct reflective nudge (Persona). $\alpha$,$\beta$,$\gamma$ show significant pairwise differences. n.s.: not significant, *: $p<.05$, **: $p<.01$, etc...}
\label{tab: summary quantitative direct}
\begin{tabular}{cccccc}
\toprule
\multirow{3}{*}{\textbf{Deliberativeness}} & \multirow{3}{*}{\textbf{$p$}} & \multicolumn{4}{c}{\textbf{Modality}} \\ \cline{3-6}
 & & \textbf{\begin{tabular}[c]{@{}c@{}}Text\\      ($M \pm S.D.$)\end{tabular}} & \textbf{\begin{tabular}[c]{@{}c@{}}Image\\      ($M \pm S.D.$)\end{tabular}} & \textbf{\begin{tabular}[c]{@{}c@{}}Video\\      ($M \pm S.D.$)\end{tabular}} & \textbf{\begin{tabular}[c]{@{}c@{}}Audio\\      ($M \pm S.D.$)\end{tabular}} \\
\midrule
Argument Repertoire & ** & 2.32 ± 1.07$^\alpha$ & 2.56 ± 1.00$^\beta$ & 3.60 ± 1.94$^\alpha$$^\beta$$^\gamma$ & 2.60 ± 1.41$^\gamma$ \\
Argument Diversity & n.s. & 4.08 ± 1.68 & 5.08 ± 1.75 & 5.48 ± 2.63 & 4.64 ± 1.80 \\
Rationality (Opinion Expression) & n.s. & 0.64 ± 0.49 & 0.84 ± 0.37 & 0.60 ± 0.50 & 0.72 ± 0.46 \\
Rationality (Justification Level) & n.s. & 2.20 ± 0.91 & 2.08 ± 0.81 & 2.20 ± 0.82 & 1.92 ± 0.86 \\
Constructiveness & n.s.& 0.28 ± 0.46 & 0.20 ± 0.41 & 0.48 ± 0.51 & 0.36 ± 0.49 \\
\bottomrule
\end{tabular}
\Description{A table summarizing the deliberative quality for direct reflective nudge across the four modalities and across the dependent variables. The table is a summary of the other Figures in the section.}
\end{table*}

\begin{table*}[!htbp]
\caption{Deliberative quality across the four modalities for indirect reflective nudge (Storytelling). $\alpha$,$\beta$,$\gamma$ show significant pairwise differences. n.s.: not significant, *: $p<.05$, **: $p<.01$, etc...}
\label{tab: summary quantitative indirect}
\begin{tabular}{cccccc}
\toprule
\multirow{3}{*}{\textbf{Deliberativeness}} & \multirow{3}{*}{\textbf{$p$}} & \multicolumn{4}{c}{\textbf{Modality}} \\ \cline{3-6}
  & & \textbf{\begin{tabular}[c]{@{}c@{}}Text\\      ($M \pm S.D.$)\end{tabular}} & \textbf{\begin{tabular}[c]{@{}c@{}}Image\\      ($M \pm S.D.$)\end{tabular}} & \textbf{\begin{tabular}[c]{@{}c@{}}Video\\      ($M \pm S.D.$)\end{tabular}} & \textbf{\begin{tabular}[c]{@{}c@{}}Audio\\      ($M \pm S.D.$)\end{tabular}} \\
\midrule
Argument Repertoire & n.s. & 2.76 ± 1.13 & 2.60 ± 1.26 & 2.88 ± 1.96 & 2.84 ± 1.28 \\
Argument Diversity & * & 4.24 ± 2.07 & 3.72 ± 2.15$^\alpha$ & 5.52 ± 2.40$^\alpha$ & 4.92 ± 1.82 \\
Rationality (Opinion Expression) & ** & 0.48 ± 0.51$^\alpha$$^\beta$$^\gamma$ & 0.76 ± 0.44$^\alpha$ & 0.84 ± 0.37$^\beta$ & 0.84 ± 0.37$^\gamma$ \\
Rationality (Justification Level) & * & 1.84 ± 1.11 & 1.44 ± 1.08$^\alpha$ & 2.32 ± 0.85$^\alpha$ & 2.08 ± 1.08 \\
Constructiveness & n.s. & 0.16 ± 0.37 & 0.24 ± 0.44 & 0.32 ± 0.48 & 0.44 ± 0.51 \\
\bottomrule
\end{tabular}
\Description{A table summarizing the deliberative quality for indirect reflective nudge across the four modalities and across the dependent variables. The table is a summary of the other Figures in the section.}
\end{table*}

\subsubsection{Summary of Results}
Overall deliberative quality across the five measurements is summarized in Tables~\ref{tab: summary quantitative direct} and \ref{tab: summary quantitative indirect} for direct and indirect reflective nudges, respectively. A detailed analysis reveals that \emph{Video} generally emerges as the most effective modality for the direct reflective nudge (persona), particularly excelling in Argument Repertoire. For the indirect reflective nudge (storytelling), similarly, \emph{Video} leads in Opinion Expression, Argument Diversity, and Justification Level, often outperforming \emph{Image} significantly. This suggests that video enhances user engagement and expression, thereby improving deliberativeness. In contrast, \emph{Text} is least effective for Opinion Expression in the context of indirect reflective nudges (storytelling), suggesting that lengthy text may not engage users as effectively in articulating their opinions.


\subsection{Subjective Feedback}
\label{sec: qualitative}
We present the findings from the thematic analysis of the post-task feedback, where numbers in parentheses indicate the frequency of recurring feedback across both nudge types. Notably, no specific dislikes were reported for text or images in either reflective nudge type.

\subsubsection{General Strengths of Reflect}
Participants demonstrated strong appreciation for the system across all modalities in both reflective nudges. They appreciated its thought-provoking nature, which facilitated critical thinking and led to a deeper understanding of their own feelings and thoughts on the issue (40). It helped clarify and refine their perspectives (10), encourage deep self-evaluation, foster self-awareness (21), and prompt reflection on personal values and the impact of their opinions (8). 
%Additionally, by connecting to their own experiences, the system resonated with participants, making the content more relatable (15), while also reinforcing their understanding of the topic (21). Participants also mentioned that they felt emotionally supported as they were able to explore complex emotions and moral dilemmas (14). 
With its ability to present multiple viewpoints and perspectives that might have otherwise been overlooked (29), participants found the system both interesting and useful (34). 
%Its ease of use, clear and intuitive design, and simplicity made it highly accessible (8). 

\subsubsection{Strengths of each Modality in Direct Reflective Nudge}
Participant appreciated that \textbf{text} was straightforward and concise (P12). For \textbf{images}, participants noted that they helped break down complex information, making it easier to grasp, retain and connect with the content (P26).

Participants found \textbf{videos} particularly engaging (P51, P57) due to their multi-sensory approach of combining visual and auditory elements (P51, P68, P70, P74). This blend elicited deeper emotional and cognitive responses (P54, P55, P57): ``\textit{Videos fostered both emotional and intellectual engagement by leveraging storytelling, visuals, and music, making it more memorable and impactful} (P68).''
%Additionally, videos offered vivid, real-world examples that made abstract concepts more tangible and relatable (P74).

For \textbf{audio}, participants appreciated that the voice sets a calm, reflective atmosphere for them to engage deeply with their thoughts and emotions (P77), enhancing the reflective experience (P78, P84).
%, with the voice evoking emotions and memories, facilitating a deeper connection to the reflection process (P82, P83). 
%The immersive nature of audio also helps participants focus inwardly without distractions, allowing them to fully engage with their thoughts and feelings (P85, P89, P91, P95). 
Additionally, some participants preferred audio over reading, noting that it kept them more engaged: ``\textit{I liked that I did not have to read anything. Listening to the audio gave me more motivation to explore multiple opinions, whereas with text, I would have only read one or two [chunks] at most} (P94).''

\subsubsection{Strengths of each Modality in Indirect Reflective Nudge}
Participants appreciated the \textbf{text} modality for its conciseness and ease of engagement (P111). For \textbf{images}, participants noted the high quality of the AI-generated visuals, describing them as ``well done'' (P126), and found that they stimulated their imagination (P144). For \textbf{both text and images}, participants found the well-developed stories particularly memorable (P124), with images resonating deeply and prompting significant reflection (P134). 
%In both modalities, the relatable journeys and dilemmas faced by the characters encouraged participants to reflect on their own values and decisions, as well as how these experiences influenced their personal growth (P114, P116, P117, P118, P142).

For \textbf{videos}, participants provided similar feedback as with the direct reflective nudge. They found videos particularly engaging (P152) due to the multi-sensory combination of visual and auditory elements, which helped them better articulate their thoughts and feelings (P152). 
Similarly, for \textbf{audio}, participants shared feedback consistent with the direct reflective nudge. Fourteen participants reported they liked the calm voice which fostered more profound reflection by reducing distractions.
%This blend also allowed participants to process complex ideas more effectively (P152, P169). The videos also made the scenarios more relatable and vivid, offering real-world examples (P154, P160, P170, P171, P172, P174), which helped participants connect emotionally and intellectually with the thought-provoking content (P154, P159, P167, P170, P173, P174).
%For  Again, (P176, P182, P197, P198) and enhancing emotional connection and engagement with the content (P176, P178). Additionally, P192 noted that ``\textit{the voice added a dynamic and reflective quality to the topic, infusing it with personality and depth, enriching my contemplation of the subject}''.

\subsubsection{Critiques of each Modality in Direct Reflective Nudge}
For \textbf{video}, one participant expressed a dislike for computer-generated narration (P52). Additionally, two participants mentioned that the lack of interactivity in the videos led to passive consumption of information rather than active engagement which may limit the depth of reflection (P70, P73).

For \textbf{audio}, four participants expressed that it was not for them: ``\textit{Audio can certainly aid in reflection, but it might not always hit the mark for every individual} (P84).'' % and ``\textit{It may not always suit everyone’s reflective needs or styles}'' (P76). 
%This underscores the importance of finding a reflective practice that aligns with individual reflective preferences and needs (P80, P92). 
Additionally, three participants found the audio voices too robotic, noting a desire for a more human-like tone to enhance credibility (P79, P83, P94).

\subsubsection{Critiques of each Modality in Indirect Reflective Nudge}
%For \textbf{video}, one participant felt that some videos felt overly condensed, making it difficult to fully absorb the information and engage in meaningful reflection (P152). 

As for \textbf{audio}, two participants found that it became less engaging over time, suggesting that subtle changes or more variety were needed for continued engagement: ``\textit{After a while, it became less engaging, and I found my mind drifting away from the reflective process. A bit more variety or subtle shifts in the audio could have kept my focus and enhanced the overall reflective experience} (P180).''