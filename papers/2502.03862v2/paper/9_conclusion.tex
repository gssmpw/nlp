\section{Conclusion}
In this work, we investigated how modalities when enacted on different reflective nudges impacts the quality of deliberation. Specifically, we examine how four modalities: text, image, video and audio interact with two types of reflective nudges: direct (persona-based) and indirect (storytelling-based) across two studies. In study 1, we identified text as the subjectively preferred modality for direct reflective nudge while video was favoured for indirect reflective nudge. In study 2, we explored how different modalities can significantly shape deliberativeness. Our results expand current work on self-reflection and online discussions, offering insights into how different modalities support the deliberation process, thus, providing valuable guidance on the use of modalities on online deliberation platforms. In the future, we hope to explore more types of nudges and improve the variety and quality of the content generated with more powerful LLMs.