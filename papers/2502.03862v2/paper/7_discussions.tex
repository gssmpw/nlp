\section{Discussion}
In this section, we address \textbf{RQ2: How does the modality of a reflective nudge affect the quality of deliberation?} by synthesizing quantitative data from our objective measures (see section~\ref{sec: quantitative}), qualitative feedback from participants (see section~\ref{sec: qualitative}) and insights from study 1 to provide a comprehensive view of how different modalities within different reflective nudges impact deliberativeness. We also examine how our findings corroborate and enrich previous research, discussing the broader implications of enhancing deliberativeness on online deliberation platforms.

\subsection{Tailoring Modalities to Suit Different Types of Reflective Nudges}
Our results from study 1 demonstrate that text is most preferred for direct reflective nudges, while video is favored for indirect reflective nudges. Study 2's quantitative analysis confirms that video significantly enhances deliberativeness for indirect reflective nudges, increasing argument diversity, opinion expression, and justification levels. Conversely, text performs the worst for opinion expression, highlighting that the same modality on different nudge types significantly impacts deliberative quality.

This goes to show that text-based reflective nudges are most effective when they are short and straightforward, aligning with earlier work that lengthy textual information can cause cognitive overload and reduce engagement~\cite{sweller1988cognitive}. For complex reflective tasks, video offers a better alternative, suggesting that \textbf{indirect reflective nudges benefit more from direct modalities} like video, which maintain engagement and convey ideas effectively.

The preference for video in indirect reflective nudges can be attributed to its capacity of multi-sensory engagement, allowing it to deliver complex concepts more directly and engagingly~\cite{clark2023learning} as highlighted by the qualitative feedback in both studies (sections~\ref{sec: qualitative for study 1} and \ref{sec: qualitative}), supporting Mayer’s findings~\cite{mayer2005cambridge} that video outperforms images and text in learning contexts. Additionally, Dale’s~\cite{dale1969audiovisual} cone of learning highlights the inherent concreteness of videos compared to more abstract modalities like pictures. This concreteness likely contributes to their effectiveness in promoting reflection by providing richer, more relatable stimuli compared to images or text. Our study extends these findings to reflection in online deliberation, showing that video enhances deliberativeness more effectively than text alone, echoing the multimedia principle~\cite{fletcher2005multimedia}.

Overall, our findings suggest that as the complexity of reflective nudge increases, different modalities may be needed to enhance the reflection process. While text works for concise reflective nudge, richer modalities like video should be considered for more complex nudges. These insights suggest that \textbf{the modality chosen should be carefully aligned with the type of nudge being delivered}, challenging the conventional reliance on text-based approaches and advocating for a multi-modal strategy to better support deliberativeness.

\subsection{Supporting Deliberation with Modalities}

\subsubsection{Importance of Multimodality in Reflection}
Results from studies 1 and 2 revealed that all participants exhibited multimodal preferences, indicating that none had a dominant single preference. This aligns with existing research and theoretical frameworks~\cite{mayer2002multimedia, fleming1992not}, which showed that only a small minority of individuals have single-modal preferences, while the majority are multimodal, spanning bimodal, trimodal, or four-part preferences~\cite{mayer2005cognitive, varklearnVARKResearchWhat}. Notably, the most common is the four-part preference, with 25.4\% of individuals categorized as ``Integrative Multimodal''~\cite{varklearnVARKResearchWhat}. These individuals seek input across all modalities before integrating insights from multiple sources for a more comprehensive grasp of the material. 

Therefore, it is likely that individuals may engage with and benefit from multiple sensory channels for more effective information processing. Studies in educational psychology~\cite{mayer2005cambridge, mayer2002multimedia, fleming1992not, clark2023learning} emphasize the cognitive benefits of multi-sensory engagement, supporting our findings that multimodal approaches can significantly enhance user reflection. By \textbf{leveraging users' multimodal preferences as a catalyst for reflection}, online deliberation platforms can tailor nudges to individual needs more effectively. Integrating multiple modalities into reflective nudges not only caters to diverse user preferences but also leads to more meaningful and impactful deliberation. Ultimately, multimodal approaches ensure that reflective nudges resonate with a broader audience, thereby improving deliberativeness.

\subsubsection{Pairing Modalities - Leveraging the Advantages of Different Modalities for Reflection}
Modalities serve both \textit{interpretative} and \textit{reflective} support, helping users construct opinions while enhancing self-reflection~\cite{reid2003supporting}. Each modality has unique strengths, as evidenced by our qualitative findings from study 1. For instance, participants whose first language was not English, preferred images over text, highlighting the universal accessibility of images and its ability to quickly convey ideas and emotions without language barriers. This underscores the importance of leveraging different modalities not only to enhance inclusivity when language and cultures differs but also to take advantage of their distinct benefits.

Research on temporal contiguity~\cite{moreno1999cognitive} shows that combining narration with animation yields better performance compared to using a single channel, particularly for cognitive tasks~\cite{lee1997effect, paivio2013imagery}. Similarly, Nathan et al.~\cite{nathan1992theory} found pairing verbal and non-verbal materials outperforms narration alone. In our study, participants noted that audio alone, while conversational, was less effective for reflection, lacking sufficient capacity for dual coding~\cite{paivio1975free} and semantic processing (see section~\ref{sec: learning and reflection}). Pairing audio with visuals leverages the \textbf{additive effect}, where combinations like audio with images or text create richer and more effective reflective experiences, amplifying semantic processing and providing a stronger foundation for reflection.

Our findings (see section~\ref{sec: qualitative for study 1}) also illuminate the interaction between modalities and the dual-system thinking (see section~\ref{sec: dual system thinking}). Text and images aligned with System 1, facilitating rapid information delivery, especially for direct reflective nudges. Videos, offering a multi-sensory experience, balanced System 1’s speed with System 2’s reflective depth, particularly for indirect reflective nudges. Parallel processing (System 1) was supported by visual modalities like images and videos, which provided an overarching view of the topic. Serial processing (System 2), tied to detailed and step-by-step analysis, was best facilitated by text, aiding deeper reflective engagement.

By understanding how modalities interact with System 1 and System 2, platforms can design reflective nudges that optimize both intuitive and analytical engagement. Combining modalities strategically can also amplify their strengths, promote inclusivity, and enhance the depth of reflection. Future studies may want to explore combining different modalities to achieve optimal outcomes tailored to specific deliberative environments, enhancing the quality and depth of online discussions.

\subsubsection{Using Modalities to Kick-start Reflection}
Participants highlighted that images effectively initiated their reflection (see section~\ref{sec: usage scenarios}). This aligns with research showing that visual elements, such as images and videos, activate prior knowledge and trigger reflective thinking~\cite{mayer2005cambridge}. By engaging users visually, these modalities provide a foundation that supports more in-depth reflection, allowing users to anchor their initial reflections, which can then be expanded upon when articulating their opinions. Therefore, integrating images and videos early in the reflection process could facilitate more meaningful deliberation.

\subsection{Consider Individual Differences: The Role of Covariates in Multimodal Preferences}
Reflection is a multifaceted process~\cite{mayer2005cambridge} influenced by various factors beyond just the modality used. Our study highlights that covariates significantly shape deliberativeness (see section~\ref{sec: quantitative}). It is essential to account for these individual characteristics when designing reflective nudges, as they determine how users engage with each modality.

Our results are consistent with Mayer et al.~\cite{mayer2002multimedia, mayer1990illustration}, suggesting that modalities should be considered in combination with individual traits and the specific usage context. For instance, study 1's feedback found that users find audio particularly helpful in multitasking. Additionally, study 2's quantitative findings showed that reflection is shaped by both internal preferences and external media exposure. This interplay between personal tendencies and past experiences suggests that reflection is not a one-size-fits-all process. Future research could further explore the tension between these internal and external influences in shaping users’ multimodal preferences.

To optimize reflective nudges, designers and practitioners should consider specific contexts in which modalities are applied. Our findings underscore the need for systematically investigating how individual characteristics impact deliberativeness. While most research focuses on improving deliberation outcomes, less attention has been given to understanding the factors shaping the reflective process itself. By addressing these factors, future studies can develop more targeted strategies to enhance both reflection and deliberativeness.

\subsection{Potential of Utilizing LLMs to Support Self-Reflection}
Unlike traditional online deliberation platforms, where users form opinions through user-driven self-reflection and the consumption of others' comments, our study demonstrates how LLMs can enhance this process using subtle interface nudges. This positions LLMs not just as writing aids but as facilitators of deeper, reflective thinking, steering users toward more informed, thoughtful decisions. Nevertheless, it's crucial to acknowledge that the effectiveness of LLMs relies on the quality of the responses they generate. While we utilized GPT-4.0 along with text-to-image (Microsoft Bing) and text-to-video generation tools (Invideo AI), future advancements in LLM technology, such as HeyGen AI Video Generator or Luma Dream Machine, could yield varied results depending on their specific design and functionality. These innovations will likely further shape how LLMs contribute to deliberation. 

Moreover, the role of LLM-generated content in fostering reflection, compared to human-created nudges, presents an intriguing area for further exploration. Future studies can explore how users engage with and perceive AI-generated versus human-created reflective nudges. 

Lastly, as with the use of any AI or LLM technologies in any domain, inherent risks such as amplifying societal biases must be carefully managed. Generative technologies must be explicitly designed with bias mitigation strategies to prevent perpetuating stereotypes and biases. This requires ongoing transparency about AI's role, clear communication of its limitations, and proactive steps to address potential ethical considerations. 

\subsection{Expanding Modalities for Enhanced Reflection}
Beyond the current straightforward use of modalities, there is potential to extend their application to other stages of the deliberative process.

\paragraph{Checking content before posting.} Incorporating an audio playback of written content by converting textual content into audio would allow users to listen back and validate their understanding before posting. This auditory review could help users ensure that their message aligns with their intended meaning, offering an additional layer of reflection by \textit{hearing} their own thoughts played aloud. 

\paragraph{Usage on the go.} Similarly, providing a text-to-speech feature for other users' opinions would support ``reflection-on-the-go'', particularly for auditory learners. This addition would cater to participants who reflect better by listening or those who prefer a conversational approach over reading, expanding the inclusivity and accessibility of the deliberative process. However, challenges might arise in ensuring that the text-to-speech translation is accurate, particularly in complex or nuanced opinions, which may affect users' comprehension.% For example, auditory reflective users could benefit from hearing nuanced inflections and tones that might not be apparent in text, enhancing their engagement with the content.

\paragraph{Granularity of the modality.} When designing visual aids, the level of granularity --- whether it is single images, image sequences, or multi-message visuals --- plays a key role for visual learners. Tailoring the structure of visual content to user preferences can significantly influence how effective they engage with and reflect on the content. However, a key challenge is balancing the granularity of visual aids such that they are informative without overwhelming the user. Future studies may want to look into the manipulation of modality considerations such as adjusting the granularity of different modalities to create more adaptive reflective experience that is tailored to diverse reflection styles and contexts.

\subsection{Applicability to other Contexts}
While the task presented in this study was specific to an online deliberation context, the multiplicity of opportunities for fostering reflection in other areas, combined with the social affordances provided by multimedia representations, suggests that these approaches hold potential for broader applications. 

\paragraph{Moderation.} In online discussion platforms (e.g., Reddit and Quora), moderators could fine-tune or adjust reflective nudges based on the tone or quality of the discussion, promoting deeper and more respectful engagement. By combining multimodal nudges with moderation, the system becomes more dynamic and adaptable, where nudges are personalized and contextually relevant. This approach also helps prevent excessive dominance by certain perspectives and fosters diverse viewpoints.

\paragraph{Education.} Multimodal reflective nudges can be adapted to educational settings to encourage deeper student engagement and critical thinking. For instance, visualizations or video explanations of complex topics can aid in simplifying abstract concepts. By offering different ways to engage with learning content, students can reflect on their knowledge from multiple angles, promoting active learning and enhanced retention. 

\paragraph{Counseling.} In counseling, using multimodal approaches may help individuals process their emotions and thoughts more effectively. Audio playback of therapeutic dialogues or reflections can help individuals revisit and internalize key insights, while video-based role-playing scenarios can offer practical, real-world examples on how to manage specific situations. These would allow for more personalized and engaging reflections, leading to deeper self-awareness and therapeutic growth.

Future research endeavors could explore the cross-domain generalizability of these findings, seeking to understand the modalities' boundaries and flexibility in diverse domains.