\section{Study 1: Subjective Modality Preferences}

To address \textbf{RQ1: What are the preferred modalities for each of the reflective nudges?} and discern the preferred modalities for each reflective nudges, we conducted a user study followed by semi-structured interviews with participants ($N=20$). The study focused on participants' preferences for specific modalities within each nudge type and examined how these modalities facilitated self-reflection in an online deliberation context. Detailed descriptions of the four modalities and the two reflective nudges are outlined in section \ref{sec: section3}. 

\subsection{Independent and Dependent Variables}
The study employed a 2 Reflective Nudge: \{Direct, Indirect\} $\times$ 4 Modality: \{Text, Image, Video, Audio\} mixed factorial design. \textit{Reflective Nudge} was between-subject, while \textit{Modalities} was within-subject. 
Within each nudge, the system could generate alternative content (variants), based on pre-recorded prompts. These would be the same for every modality.
The sequence of modality presentation and the order of variants within each modality were randomized for each participant to mitigate any potential ordering effects. 
We had one dependent variable, Ranking, which indicates users' preferences and was measured on a scale from 1 to 4 (for each modality). 

Results between nudges are not compared against each other; instead, the focus is on evaluating the effects of the different modalities within each nudge.

\subsection{Covariates}
\label{sec: Covariates}
%Following Yeo et al.~\cite{yeo2024help},
We controlled for four covariates as fixed factors to account for individual differences that could affect participants' engagement with the modalities.
%This inclusion helps ensure that the findings are not confounded by variability in these covariates. 
We included these covariates in the main analyses by using ANCOVA instead of ANOVA to control for their influence.

\paragraph{\normalfont{\textbf{Topic Knowledge and Topic Interest (TK-TI)}} was included to assess participants' familiarity and information access on the discussion topic. Prior research~\cite{zhang2021nudge} demonstrates that reflection interacts with information access to influence perceived issue knowledge. Mayer and Gallini~\cite{mayer1990illustration} also found that individuals with low prior knowledge benefited more from a combination of text and images. %Likewise, Kalyuga et al.~\cite{kalyuga1998levels, kalyuga2000incorporating} found that individuals with limited prior knowledge gained the most from integrated presentations, such as video. 
Whereas Moreno and Mayer~\cite{moreno1999cognitive} found that people with high prior knowledge are more likely to benefit from videos. TK-TI was evaluated through a five-item multiple choice questionnaire and two matrix tables comprising of 7-8 topic-related statements, and is measured by a score between 0 to 36.}
%(see Appendix Figure~\ref{}). Each multiple choice item was scored 1 for demonstrated knowledge or interest, while matrix statements were rated up to 3, with 0 otherwise, yielding a total score ranging from 0 to 36.

\paragraph{\normalfont{\textbf{Self-Reflection and Insight Scale (SRIS) questionnaire}} was included to account for individuals' inherent predisposition to reflect~\cite{grant2002self, silvia2022self}. It is a widely utilized self-reported scale, rooted in theories of meta-cognition and personal development~\cite{grant2001rethinking, grant2003impact}. It consists of 20 questions on a 1–6 Likert scale (1=strongly disagree, 6=strongly agree).} %to evaluate individual's self-reflective capacity and tendency for self-reflection. 

\paragraph{\normalfont{\textbf{Inherent Reflecting Styles.}}} We used the VARK version 8.02\footnote{https://vark-learn.com/the-vark-questionnaire/}, consisting of 16 question items. Scores for each modality category were subsequently computed to assess whether individuals have a clear preference for one modality over the others. 

\paragraph{\normalfont{\textbf{External Exposure and Interactions.}}} While reflecting styles account for individuals' inherent preferences, external exposure and interactions capture individuals' habitual engagement with various modalities across media platforms such as news articles and blogs for text; Pinterest and social media posts for images; TikTok, Instagram Reels, and YouTube Shorts for video; and podcasts for audio. Regular exposure to specific media formats may lead to stronger affinity for and preference towards those modalities~\cite{knobloch2014choice}. Participants reported their frequency of interaction with these different media platforms on a 1-5 Likert scale (1=Never, 2=Rarely (less than once a week), 3=Occasionally (1-3 times a week), 4=Frequently (4-6 times a week), 5=Daily).

\subsection{Apparatus}
The study was conducted remotely using Zoom. Participants were instructed to access a simulated online deliberation environment and to share their screen. This environment was based on an interface similar to Reddit~\cite{horne2017identifying, medvedev2019anatomy}, to leverage a familiar interface.

\subsubsection{Implementation}
The prototype was developed using Figma\footnote{https://www.figma.com/} before its deployment on Useberry\footnote{https://www.useberry.com/}, a user testing site. We developed the feature - \textit{Reflect} - which displays the four modalities of the reflective nudge.

\subsubsection{Key Features of the Prototype}
To facilitate users' self-reflection during the writing process on a discussion topic, we designed several key features. These are depicted in Figures~\ref{fig: teaser} and \ref{fig: features}.

\begin{figure*}[!htbp]
  \centering
  \includegraphics[width=\textwidth]{figures/Features.png}
  \caption{Key features in interface. \textbf{1:} Users can click on the \textit{Recreate} button to browse through different variants under a specific modality or \textbf{2:} click on the \textit{Exit Reflect} button to navigate back to the different array of modalities to choose another modality. Additional features are present for storytelling to allow users to \textbf{3a:} track their reading process and \textbf{3b:} continue with their reading on the current story. \textbf{4:} For audio and video formats in both direct and indirect reflective nudges, the primary difference is duration. In direct reflective nudges, audio and video average 9 to 12 seconds, while in indirect nudges, the duration extends to 90 to 120 seconds, reflecting typical short-form media content.}
  \label{fig: features}
  \Description{The user interface created for the study encompasses the following features in the comment box: 1: Users can click on the Recreate button to browse through different variants under a specific modality or 2: click on the Exit Reflect button to navigate back to the different array of modalities to choose another modality. When clicking on the storytelling reflector, additional features are present: 3a: The numbers, for example, page 1 out of 5 allows users to track their reading process. 3b: The forward arrow allows users to click and continue with their reading on the current story.}
\end{figure*}

\paragraph{\normalfont{\textit{Reflect}} is a feature that can be integrated into online discussion platforms to facilitate users in their self-reflection process when they craft their opinions on a discussion topic. Clicking on \textit{Reflect} displays the four distinct modalities. Users can choose one to explore, allowing for a deeper engagement within that modality’s context. These are depicted in Figure~\ref{fig: teaser} (3a and 3b).} 

\paragraph{\normalfont{\textit{Recreate}} generates variants tailored to that modality (Figure~\ref{fig: features} (1)). To avoid undue one-sidedness in the generated content, we ensured a balanced presentation by alternating variants between male and female perspectives, as well as between positive and negative tones for both nudges.}

\paragraph{\normalfont{\textit{Exit Reflect}} will return to the selection screen displaying all available modalities (Figure~\ref{fig: features} (2)).}

\paragraph{Features Specific to the Indirect Reflective Nudge}
While engaging with a story, users can track their progress using the page tracker depicted in Figure~\ref{fig: features} (3a). This tracker displays the number of remaining pages until the story concludes. Stories vary in length: four (short), six (medium) or eight (long) pages. Clicking the \textbf{\huge{>}} button (Figure~\ref{fig: features} (3b)) allows users to continue reading. In direct reflective nudges, audio and video have a duration of 9 to 12 seconds, while in indirect nudges, it extends to 90 to 120 seconds, reflecting typical short-form media content.

\subsection{Participants and Ethics}
Following approval from our Institutional Review Board (IRB), we recruited 20 participants (8 males and 12 females), with 10 participants assigned to each type of reflective nudge, aligning with local sample size guidelines~\cite{caine2016local}. The participants had an average age of 24.0 years ($SD=3.56$). Notably, remote interviews typically have a mean sample size of 15 ($SD=6$), and CHI publications commonly feature 12 participants~\cite{caine2016local}. All participants were university students, with detailed demographic information for each reflective nudge provided in Appendix Table~\ref{tab: st1-demo}. Participants were compensated at the appropriate rate dictated by our local IRB.

\subsection{Task and Material}
\label{sec:maintask}
We chose the discussion topic “Is human activity primarily responsible for global climate change?” from ProCon.org\footnote{https://www.procon.org/} as a task for its accessibility and relevance, thereby promoting constructive and open debate.

In the main task, participants used our prototype with the \textit{Reflect} feature. \textbf{Participants had to write at least 30 words with the help of the four modalities when expressing their perspective on the discussion topic.} Participants engaged with all four modalities sequentially (as the study was conducted over Zoom, the researcher instructed and guaranteed this implementation). 

We wanted to capture an organic interaction with the system, therefore did not guide when to use \textit{Reflect}. Participants could concurrently write while using it, or use it sequentially before or after writing. Additionally, \textbf{participants were not obliged to explore every variant within a modality, though they had the option to do so if desired.}

\subsection{Procedure}
\label{sec: procedure}
In addition to the main study (see section~\ref{sec:maintask}), we included a pre- and post-task.

\subsubsection{Pre-Task: Consent, Instructions, Demographic, Questionnaire for Covariates}
Participation consent was obtained before the study. Participants were informed that they had to express their opinions on a discussion topic, which was undisclosed at this point, using various modalities. Participants were then randomly assigned to one of two groups: direct or indirect reflective nudges. Following this, they completed a demographic questionnaire.

Before starting, we administered four questionnaires corresponding to each of the four covariates, as detailed in section~\ref{sec: Covariates}. The values of the covariates are summarized in Appendix Table~\ref{tab: st1-demo}, with no significant differences observed between the two participant groups for any covariate.

\subsubsection{Post-Task Interview}
Following the completion of the main task, we conducted semi-structured interviews to collect feedback. Participants were asked to rank the four modalities on a scale from 1 (lowest) to 4 (highest) based on their personal preference and the level of self-reflection each modality elicited. They provided explanations for their rankings and discussed specific challenges encountered with each modality, as well as the overall usefulness of each modality in their reflective process. All interviews were audio recorded and transcribed for subsequent analysis.

Participants took an average of 46.8 minutes to complete the study.

\subsection{Findings - Ranking}
\begin{figure*}[!htbp]
  \centering
  \includegraphics[width=.8\textwidth]{figures/Ranking.png}
  \caption{Mean ranking for the four modalities for the direct reflective nudge (left) and indirect reflective nudge (right) with 1 being the lowest rank and 4 being the highest rank. We report the results of the ANCOVA test and pairwise comparisons with BH correction, where * : p < .05, ** : p < .01.}
  \label{fig: ranking}
  \Description{Box plot describing the mean rankings for the four modalities for direct reflective nudge (left) and indirect reflective nudge (right). The X-axis shows the different modalities and the Y-axis shows the ranking value ranging from 0 to 4 at a rank interval of 1. 1 is the lowest possible rank and 4 is the highest possible rank. The highest mean ranking was found in text for direct reflective nudge at 3.40, while the highest mean ranking was found in video for indirect reflective nudge at 3.60.}
\end{figure*}

Figure~\ref{fig: ranking} shows the average ranking for each modality of the reflective nudges. 

\subsubsection{Direct Reflective Nudge (Persona)}
\emph{Text} received the highest average ranking of 3.4 out of 4, while \emph{Audio} received the lowest average ranking of 1.7 out of 4. A one-way repeated measures ANCOVA was conducted to determine a statistically significant difference between Modalities on Ranking. We found a statistically significant main effect of Modalities on Ranking ($F_{3,26} = 4.13$, $p<.05$). Specifically, \emph{Text} was significantly ranked higher compared to \emph{Audio} ($p<.01$), \emph{Image} (2.1 out of 4) and \emph{Video} (2.3 out of 4) (both $p<.05$).

\subsubsection{Indirect Reflective Nudge (Storytelling)} 
\emph{Video} received the highest average ranking of 3.6 out of 4, while \emph{Image} received the lowest average ranking of 2.0 out of 4. A one-way repeated measures ANCOVA revealed a statistically significant main effect of Modalities on Ranking ($F_{3,26} = 4.59$, $p<.01$). Specifically, \emph{Video} was ranked significantly higher than \emph{Audio} (2.2 out of 4) and \emph{Image} (both $p<.01$), and \emph{Text} (2.4 out of 4) ($p<.05$).

\paragraph{\normalfont{In summary, \emph{Text} was the most preferred modality for direct reflective nudges, while \emph{Video} was most favored for indirect reflective nudges.}}

\subsection{Findings - Subjective Feedback}
\label{sec: qualitative for study 1}
We present a detailed account of participants' experiences with each modality for each reflective nudge type, highlighting both benefits and challenges. Feedback is categorized using Kahneman’s dual-system thinking model~\cite{kahneman2002maps} (see Table~\ref{tab: dual system thinking}) to illustrate that the same modalities can yield varying or even contradictory results depending on the type of reflective nudge applied. Additional themes are also derived from participants' responses. The notation (/10) indicates the count of participants who shared similar observations.

\subsubsection{Speed}

\paragraph{Direct Reflective Nudge (Persona).} Participants valued the \textbf{text} modality for its conciseness and efficiency in quickly conveying different perspectives (8/10). ``\textit{Text is quick} (P05), \textit{fastest to read through} (P02, P05, P07) and that the \textit{speed of acquiring information from text is the fastest} (P01).'' Similarly, participants found \textbf{image} effective for quickly conveying perspectives and emotions (5/10):  ``\textit{The images included all the necessary elements to clearly make the point, allowing me to understand the perspective immediately} (P09).'' Notably, for participants whose \textbf{first language is not English}, images were particularly beneficial. ``\textit{Image triggers my reflection faster due to my language barrier. Since English is not my first language, I can quickly understand the content and perspectives from an image using my own intuition, whereas text requires more effort to read and comprehend} (P03).'' Figure~\ref{fig: speed} shows a summary of participant preferences for each modality in terms of speed.

%In contrast, participants found \textbf{video} (6/10) and \textbf{audio} (5/10) to be less efficient and slower in conveying perspectives. Both modalities were seen as time-consuming with participants noting, ``\textit{Due to the video's duration, I had to watch it from start to finish to grasp its content}'' (P06). Similarly, audio was perceived as slow, ``\textit{Audio takes time to listen to, depending on its speed}'' (P01). 
%Additionally, the absence of visual elements in audio led to increased time spent, ``\textit{Without visual inputs, I had to spend more time to understand the perspectives conveyed by the audio}'' (P03, P05). 

%\subsubsection{Direct Reflective Nudge (Persona) - Speed}
%Participants valued the \textbf{text} modality for its efficiency in quickly conveying different perspectives (8/10). ``\textit{Text is quick} (P05), \textit{fastest to read through} (P02, P05, P07) and that the \textit{speed of acquiring information from text is the fastest}'' (P01). Participants also noted that the brevity of the text contributed to its effectiveness in conveying different perspectives swiftly (4/10). This brevity allowed them to quickly understand different perspectives and engage in self-reflection (P04). 

%Similarly, participants found \textbf{image} effective for quickly conveying perspectives and emotions (5/10), albeit not as efficient as text. ``\textit{Images quickly convey messages and emotions with minimal time}'' (P10). 
%Another mentioned, ``\textit{The images included all the necessary elements to clearly make the point, allowing me to understand the perspective immediately}'' (P09). Notably, for participants whose \textbf{first language is not English}, images were particularly beneficial. ``\textit{Image triggers my reflection faster due to my language barrier. Since English is not my first language, I can quickly understand the content and perspectives from an image using my own intuition, whereas text requires more effort to read and comprehend}'' (P03). In the same vein, another participant with similar language background added, ``\textit{Reading text requires more effort to understand its meaning. In contrast, an image is worth a thousand words - they help me visualize and grasp content much faster}'' (P06).

\paragraph{Indirect Reflective Nudge (Storytelling).} 
Feedback on the speed of conveying perspectives for indirect reflective nudge differs notably from direct reflective nudge: \textbf{video} was deemed as the most time-efficient and effective as it encompasses text, image and audio (6/10): ``\textit{Video is the most direct for me, as it allows me to quickly understand and process the content in a shorter time period} (P13).''
By contrast, participants had mixed opinions on the efficiency of the \textbf{text} modality for conveying perspectives. Some found text to be quick and efficient (5/10), while others did not share this view (5/10). Those who appreciated text cited their ability to quickly read and process textual information: ``\textit{It is much faster for me to read, so I spend lesser time understanding the contents. In that sense, text is more efficient and helps me to reach the stage of self-reflection faster.}'' While others expressed concerns about its time-consuming nature. P16 said, ``\textit{I wouldn't want to read long chunks of text}.'' % and P20 commented, ``\textit{Reading long chunks of text slows me down and is too time-consuming.}'' P19 added, ``\textit{I need to go through each sentence several times to fully understand the text, making it very time-consuming.}'' Overall, participants who found text inefficient noted that they would engage with it only if they had ample time (P11, P13), highlighting the reliance of text on the user's time availability.
Two participants found \textbf{images} time-consuming because they were positioned above the text, causing them to scan the images before reading the text. ``\textit{I felt like I kept going back and forth — I was reading the text and trying to match it with the image, which take longer for me to understand the content as a whole} (P19).''

Feedback on the \textbf{audio} modality was consistent with direct reflective nudge. No participants described audio as efficient; in fact, some found it slow (4/10): ``\textit{I find the pace of the audio quite slow; the narration drags on} (P12).''
%while another commented, ``\textit{Audio relies on the speed of the narration, so the information is transmitted slowly}'' (P14). Figure~\ref{fig: speed} shows a summary of participant preferences for each modality in terms of speed.

\subsubsection{Depth of Self-Reflection}

\paragraph{Direct Reflective Nudge (Persona)}
Majority of the participants (7/10) found that \textbf{text} facilitated self-reflection more effectively than other modalities for the direct reflective nudge. ``\textit{Text is the most effective medium for reflection because it helps me reconstruct and clarify my thoughts} (P01).'' 
%As one participant noted, ``\textit{Text aids my self-reflection because I retain more information while reading, which helps me remember and think deeply about the topic}” (P05). Another participant mentioned that text offered more details and insights, which supported his thinking process and enhanced self-reflection (P06).
For \textbf{video}, some participants found that while they were not the most efficient in conveying information for direct reflective nudges, they provided more time for self-reflection (3/10). As P01 noted, ``\textit{Although the videos [...] conveyed perspectives slowly, they allowed more time for self-reflection.}'' % P08 added, ``\textit{The video condenses information into short-form content, giving me more time to reflect}''. Similarly, P10 remarked, ``\textit{The slower pacing of the videos compared to text helped me to be more mindful in my comments and facilitated deeper self-reflection}''.
For \textbf{image}, some participants found it beneficial for self-reflection by simplifying complex ideas (3/10). As P10 explained, ``\textit{Images enhance my self-reflection by presenting symbolic meanings that prompt deeper personal interpretation. A powerful image can quickly establish a connection, making it easier for me to engage with and reflect on the topic.}''
%Additionally, images can distill complex ideas into more manageable elements, allowing me to focus on key aspects and think more deeply about the subject}''.
For \textbf{audio}, two participants felt it enhanced self-reflection by simulating a conversational experience, as if interacting with a real person. % ``\textit{Audio facilitates self-reflection because I can hear the speaker’s emotions and tone, helping me to connect with the speaker and making me to think differently about the topic. This would also change how I engage online}'' (P09). 
P02 noted, ``\textit{Audio adds meaning to the text by conveying emotions and expressions that I can’t grasp from reading text alone, helping me to better understand the speaker’s intent.}'' Figure~\ref{fig: speed} shows a summary of participant preferences for each modality in terms of depth of self-reflection.

\paragraph{Indirect Reflective Nudge (Storytelling).} Half of the participants (5/10) found that \textbf{video} enhances self-reflection by making the content more relatable and connecting with their personal experiences: ``\textit{The video portrays daily life well, allowing me to compare it with my own experiences, which prompted deeper reflection} (P20).'' Additionally, participants found that videos aid in clarifying their stance and feelings on issues.
%P14 further added, ``\textit{Videos help me self-reflect by making it easier to empathize with the characters, deepening my reflection}''.: ``\textit{Videos are helpful in figuring out my feelings or stance on an issue when I'm unsure}'' (P12).
In contrast to direct reflective nudge, where text was highly valued for its role in self-reflection, only one participant found \textbf{text} effectively facilitated self-reflection for indirect reflective nudge. %``\textit{The greater autonomy in interacting with the text contributed to enhancing my self-reflection}'' (P17).
For \textbf{image}, a few participants (3/10) noted that the memorable nature of images enhances self-reflection by aiding in content recall. As P19 put it, ``\textit{What helps me self-reflect more is the modality’s ability to make the content memorable. Images stand out for me because I can vividly remember the content, which aids in deeper self-reflection}.''

Compared to direct reflective nudge, a higher proportion of participants found \textbf{audio} effective for self-reflection (4/10), citing similar reasons. Specifically, they noted that the presence of a voice conveyed tone and emotion, aiding their reflection: ``\textit{Hearing the voice helps me understand the character’s emotions and plight better which enhances my self-reflection} (P13).'' 
%Audio also facilitated their thought process: ``\textit{Listening to the audio allowed me to quickly agree or disagree and refine my thoughts as the audio continues to play}'' (P15). 
Additionally, audio enabled participants to immerse themselves in the role of the main character: ``\textit{Audio lets me close my eyes and imagine living as each person, which stimulates my self-reflection} (P11).'' Figure~\ref{fig: speed} shows a summary of participant preferences for each modality in terms of the depth of self-reflection.

\begin{figure*}[!htbp]
  \centering
  \includegraphics[width=.8\textwidth]{figures/Speed.png}
  \caption{Count of participants who shared the same feedback on the depth of self-reflection and speed of delivery for the four modalities of each reflective nudges: direct (persona) - left and indirect (storytelling) - right.}
  \label{fig: speed}
  \Description{Bar graph showing the count of participants who shared the same feedback on the depth of self-reflection and the speed of delivery for the four modalities of each reflective nudges: direct (persona) on the left and indirect (storytelling) on the right. The bar graph shows that text is reflected to be the most time efficient and enhances self-reflection the most for persona. Whereas, video is reflected to be the most time efficient and enhances self-reflection for storytelling.}
\end{figure*}

\subsubsection{Processing (Both Nudges)}
Regardless of the type of reflective nudge, participants from both groups found that \textbf{videos} and \textbf{images} helped them grasp the \textbf{broader context} or overall picture, whereas \textbf{text} allowed for a deeper understanding of \textbf{finer details}. Videos and images provided a higher-level overview and abstraction, making it easier to understand the main idea (P06, P09). Conversely, \textbf{text} was appreciated for its ability to convey detailed concrete information and insights, offering a more comprehensive understanding of the topic (P10, P06).
%For instance, one participant noted, ``\textit{From the image alone, I could grasp the main gist but missed the finer details}'' (P19; Indirect). 

\subsubsection{Effort (Both Nudges)}
Participants from both groups found that \textbf{audio} demanded significant \textbf{concentration, focus and mental effort}, summarized by the following statement: ``\textit{It’s very hard for me to engage with audio alone} (P04).'' %``\textit{Listening while tracking the contents of the audio is challenging}'' (P06; Direct), and ``\textit{I struggled with audio most due to limited visual information}'' (P03; Direct), reflected this difficulty. 
Participants also mentioned needing to stay actively focused on the audio to avoid missing content (P05) and finding it hard to absorb information (P11).
%, thus requiring substantial concentration and effort (P13, P20; Indirect).
\textbf{Video}, on the other hand, was described as facilitating content processing: ``\textit{Video makes it easier for me to process the content} (P13).''

\subsubsection{Nature (Both Nudges)}
Participants noted that \textbf{text} and \textbf{video} are more familiar and accessible due to \textbf{daily exposure}. Text is described as traditional and conventional, making it easier to absorb information (P08, P12). It is considered the most accessible modality (P04) and is frequently used, leading to greater familiarity (P10). Similarly, videos were likened to short-form content found on platforms like TikTok and YouTube Shorts, making them easier to digest as they are familiar with it (P04, P08, P14).
%Participants thus appreciated this similarity to social media content (P19, P18; Indirect).
In contrast, \textbf{audio} was less favored due to its \textbf{limited natural exposure}. Those who did prefer it attributed this preference to their inherent tendency towards auditory learning. Participants in both groups generally expressed disinterest or discomfort with audio, preferring visual stimuli instead. 
%Feedback such as, ``\textit{I am generally not an auditory person}'' (P02; Direct, P19; Indirect) and ``\textit{I tend to doze off with audio, so audio is not for me}'' (P02; Direct) reflected this sentiment. Contrarily, some participants who identified as auditory learners found it easier to absorb information through listening (P11; Indirect).

\subsubsection{Information Retention (Both Nudges)}
Both \textbf{video} and \textbf{audio} exhibit slightly \textbf{lower information retention} compared to the more static modalities of text and images. Participants noted that ``\textit{for both video and audio, I don’t retain as much information compared to text} (P05)'', and that ``\textit{retention rates for recalling specific details from video and audio are lower, as I tend to forget earlier parts while processing new information} (P18).''  Another participant mentioned, ``\textit{videos convey a lot quickly, but I forget the details just as quickly because the information is presented in a short time} (P19).'' In contrast, \textbf{text} and \textbf{images} are described as \textbf{more memorable} (P01, P05, P18, P19).
%, with participants noting, ``\textit{I can recall the contents better because they stick in my head}'' (P20).

\subsubsection{Engagement (Both Nudges)}
\textbf{Text} was generally reported as the \textbf{least engaging}, while \textbf{video} was viewed as \textbf{highly engaging} by most participants in both groups. Videos and \textbf{images} were praised for their \textbf{attention-grabbing} qualities (P01, P13, P14, P20). Videos were particularly noted for being interactive, fun and interesting (P01, P14, P15).
%and were found less boring than plain audio, static images, or text (P01, P02, P09; Direct) The dynamic nature of video scenes also kept viewers engaged (P20; Indirect). 
In contrast, text was described as dry and boring (P01, P15, P18, P20). Participants found it less captivating, especially in an era dominated by multimedia content (P10). %and noted that there is nothing \textit{'special'} about plain text (P09; Direct).

\subsubsection{Users' Autonomy (Both Nudges)}
\textbf{Video} and \textbf{audio} generally offered the \textbf{lowest level of user autonomy}: users noted that they must watch or listen to the entire content without control over the pace (P03, P05). % ``\textit{Videos and audios have fixed playing time so I have less control on how much time I want to spend, but for image and text, they just depends on how fast I can go}'' (P12, P14, P18; Indirect). 
In contrast, \textbf{text} was reported to provide the \textbf{highest level of autonomy}. Participants appreciated that they could adjust their reading speed and review information at their own pace (P01, P03, P16, P17).

\subsubsection{Credibility and Trust in AI-Generated Content (Both Nudges)}
Participants generally reported \textbf{lower credibility} for \textbf{images} compared to audio, video, and text. Some expressed skepticism toward AI-generated images (P04, P14), noting that they appeared unrealistic and less human-like (P01, P08). In contrast, video and audio were seen as more credible due to their human-like elements, such as real human presence in videos and realistic voices in audio (P01).

\subsubsection{Usage Scenarios (Both Nudges)}
\label{sec: usage scenarios}
In general, participants found that \textbf{video} is beneficial for those with little to \textbf{no prior knowledge} of the topic. It helps provide context and explains the issue from multiple angles, making it useful for learning and reflection (P04, P06, P11). For \textbf{images}, they are effective in \textbf{triggering self-reflection for those with some prior knowledge on the topic}. Images are attention-grabbing and can engage users by capturing their focus (P06, P13, P20). 
%``\textit{For an effective reflection, I need something that is more attractive, so I am more likely to engage with images as they capture my attention}'' (P14; Indirect). Images thus serve as a good starting point for self-reflection, helping to kick-start the process (P08, P09; Direct, P11; Indirect). 
\textbf{Text} is primarily used by individuals who already have a \textbf{deep interest or prior knowledge of the topic} (P05, P06). For these users, text provides detailed information that supports reflection without needing additional attention-grabbing elements (P06). Lastly, an overwhelming number of participants mentioned that \textbf{audio} is suited for \textbf{multitasking} and \textbf{on-the-go reflection}, as it allows them to reflect while engaging in other activities.

\subsection{Summary of the Results}
Our results revealed how different modalities of reflective nudges interact with the dual-system thinking framework. Specifically, we observed a clear preference trend: text was favored for direct reflective nudges due to its speed and autonomy, enhancing self-reflection, albeit being less engaging. In contrast, video was preferred for indirect reflective nudges because of its speed, engagement and ability to convey the main idea effectively, despite its lower autonomy. Appendix Figures~\ref{fig: butterfly chart text} - ~\ref{fig: butterfly chart audio} present a butterfly chart summarizing the qualitative feedback for each modality across both nudge types. 