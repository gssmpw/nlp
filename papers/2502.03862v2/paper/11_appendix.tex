\section*{Appendix}
The appendix contains supplementary data, which, while not part of the analysis, may provide additional details that can help in replicating the study. 

\subsection*{Prompt Engineering}
\label{sec: prompt engineering}
Tables~\ref{tab: prompt engineering} and \ref{tab: prompt constraint rationale} detail the prompts used in GPT and AI generative models, along with the specific constraints applied for each modality. According to the ChatGPT API specifications, the ``system role'' defines the model's assumed role in the session, while ``user input'' provides the prompt guiding the model's response.

\subsection*{Demographics Profile}
We collated participants' topic knowledge and topic interest (TK-TI) scores, their self-reported self-reflection and insight scale (SRIS) scores, VARK scores and external exposures and interactions with the different modalities in both reflective nudge (see Tables~\ref{tab: st1-demo} and \ref{tab: st2-demo}), to account for any potential fixed effects associated with these four covariates. 

In study 1, it's worth noting that all four covariates showed no statistically significant impact on the rankings of the modalities. Furthermore, we found no meaningful correlations between any of the four covariates and their interactions with any of the nudges.

In study 2, we report the covariates that are significant in section~\ref{sec: quantitative}.

\subsection*{Summary of Qualitative Feedback for Study 1}
Qualitative feedback for study 1 for both reflective nudges across the four modalities are summarized in Figures~\ref{fig: butterfly chart text} to \ref{fig: butterfly chart audio}. Feedback is classified using Kahneman's dual system thinking model~\cite{kahneman2002maps} to illustrate that the same modalities can produce varying or even contradictory results depending on the type of reflective nudge applied.

\newpage

\begin{table*}[!htbp]
\caption{The prompts utilized to generate the modalities along with the corresponding constraints}
\label{tab: prompt engineering}
\scalebox{0.6}{
\begin{tabular}{|l|lll|l|}
\hline
\textbf{Reflective Nudges} &
  \multicolumn{1}{l|}{\textbf{Prompt Template for Text Modality}} &
  \multicolumn{1}{l|}{\textbf{Prompt Template for Image Modality}} &
  \textbf{Prompt Template for Video Modality} &
  \textbf{Prompt Template for Audio Modality} \\ \hline
\textit{\begin{tabular}[c]{@{}l@{}}For both reflective \\ nudges\end{tabular}} &
  \multicolumn{3}{l|}{\begin{tabular}[c]{@{}l@{}}\textbf{system role:} You are a helpful assistant focusing on supporting users' self-reflection on a given topic.\\      \\ \textbf{User input:} Topic: {[}topic{]}.\end{tabular}} &
  \multirow{3}{*}{\begin{tabular}[c]{@{}l@{}}Audio prompts are not required for this \\ process, as the text generated in the text \\ modality is directly input into the text-to-\\ speech AI tool. The tool automatically \\ converts the provided text into audio, \\ generating the narration based on the \\ content supplied. We manually \\ selected a gender-appropriate voice that \\ matches the gender as specified in the \\ text modality.\end{tabular}} \\ \cline{1-4}
\begin{tabular}[c]{@{}l@{}}Direct Reflective \\ Nudge (Persona)\end{tabular} &
  \multicolumn{1}{l|}{\begin{tabular}[c]{@{}l@{}}For the above topic, create ten distinct personas \\ representing different perspectives on the topic. \\ Provide the name, age and occupation for each \\ persona. \\      \\ Here is the format of the results: \\ 1. {[}Name1{]}, {[}Age1{]}, {[}Occupation1{]}, {[}Perspective1{]}\\ 2. {[}Name2{]}, {[}Age2{]}, {[}Occupation2{]}, {[}Perspective2{]}\\      ...\\      \\ Requirements:\\ Create five male personas and five female personas;\\ For each perspective, be concise, giving at most three \\ sentences;\\ No duplicates;\\ Ten distinct versions only\end{tabular}} &
  \multicolumn{1}{l|}{\begin{tabular}[c]{@{}l@{}}\textit{For each persona generated in the text} \\ \textit{modality, we use the following prompt to} \\ \textit{generate the corresponding image via AI:}\\      \\ Persona: {[}persona1{]}     \\ Create a photorealistic image with realistic \\ textures and lighting of the persona. The\\ background should be related to the \\ occupation of the persona. \\      \\ Requirements: \\ Age, gender and occupation should follow \\ the persona. The style is photorealistic and \\ not cartoonish. The image has no wordings.\end{tabular}} &
  \begin{tabular}[c]{@{}l@{}}\textit{For each persona generated in the text} \\ \textit{modality, we input the entire script into} \\ \textit{the AI.} \textit{We then use the following prompt} \\ \textit{to generate the corresponding video:}\\      \\ Persona: {[}persona1{]}     \\ Create a photorealistic video featuring the \\ persona. The background should be realistic \\ and aligned with the persona’s occupation \\ and worldview, with natural lighting and no \\ cartoonish elements. The voice narration \\ should match the persona’s gender and follow \\ the script exactly as provided, with no \\ additional text. The video should be visually \\ compelling and immersive, showcasing \\ a background relevant to the persona’s \\ occupation or environment. \\      \\ Requirements: \\ Resolution: 1080p\\ Audience: Relevant to the persona’s occupation\\ Style: Photorealistic, professional\\ Platform: YouTube Shorts or Instagram Reels\\ Script: Follow the provided persona text exactly \\ without any modifications.\end{tabular} &
   \\ \cline{1-4}
\begin{tabular}[c]{@{}l@{}}Indirect Reflective \\ Nudge (Storytelling)\end{tabular} &
  \multicolumn{1}{l|}{\begin{tabular}[c]{@{}l@{}}Following the ten personas created earlier, generate a \\ story for each of the persona on the topic matter. \\      \\ Here is the format of the results:\\ Story1:\\ Story2:\\ …\\      \\ Requirements:\\ Create five stories with a positive tone and \\ five stories with a negative tone;\\ No duplicates or similar story line;\\ Ten distinct versions only\\      \\ * Depending on the length of the story generated, \\ we prompt GPT to either lengthen (Extend Story1) \\ or condense (Make Story1 concise) the narrative.\end{tabular}} &
  \multicolumn{1}{l|}{\begin{tabular}[c]{@{}l@{}}\textit{For each story generated in the text modality,} \\ \textit{we prompt AI to create images that visually} \\ \textit{represent key moments in the narrative. We} \\ \textit{generate one image at a time, which are then} \\ \textit{combined to form a cohesive visual }\\ \textit{representation of the entire story.}\\      \\ Section of the story: {[}story section1{]}.     \\ Create a photorealistic image with realistic \\ textures and lighting of the section of the story. \\ The background should be related to the \\ occupation of the character in the story. \\ \\      \\ Requirements: \\ Age, gender and occupation should follow \\ the character in the story.\\ The style is photorealistic and not cartoonish. \\ The image has no wordings. \\ \textasciicircum The style should be consistent with the \\ previous image.\\      \\ \textasciicircum Only for subsequent images generated \\ by the AI to maintain visual coherence.\end{tabular}} &
  \begin{tabular}[c]{@{}l@{}}\textit{For each story generated in the text modality, }\\ \textit{we input the entire script into the AI.} \\ \textit{Pauses were manually added to ensure natural }\\ \textit{pacing and alignment with the story's speech }\\ \textit{patterns. We then use the following prompt to }\\ \textit{generate the corresponding video:}\\      \\ Story: {[}story1{]}     \\ Create a photorealistic video featuring the \\ story. The background should be realistic \\ and aligned with the character's occupation and \\ worldview, with natural lighting and no cartoonish \\ elements. The voice narration should match the \\ character's gender and follow the script exactly as \\ provided, with no additional text. The video should \\ be visually compelling and immersive, showcasing \\ a background relevant to the character's \\ occupation or environment. \\      \\ Requirements: \\ Resolution: 1080p\\ Audience: Relevant to the character's occupation\\ Style: Photorealistic, professional\\ Platform: YouTube Shorts or Instagram Reels\\ Script: Follow the provided text exactly without any \\ modifications.\end{tabular} &
   \\ \hline
\end{tabular}}
\Description{This table details the prompts used to generate the variants for each of the modalities, along with the corresponding constraints applied when communicating with the ChatGPT API and AI generative models. These prompts adhere to the template established by White et al., which entails defining a task, incorporating constraints, and setting clear expectations for the generated output.}
\end{table*}

\newpage

\begin{table*}[!htbp]
\caption{Rationales of the constraints set out for GPT and the AI Generative tools}
\label{tab: prompt constraint rationale}
\centering
\begin{tblr}{
  width = \textwidth,
  colspec = {Q[352]Q[588]},
  row{1} = {c},
  hlines,}
\textbf{Constraints} & \textbf{Rationale} \\
{(Direct Reflective Nudge: Persona) Create five male personas and five female personas. \\~} & {\labelitemi\hspace{\dimexpr\labelsep+0.5\tabcolsep}Scholars have found that large language models such as GPT-3 produce gender stereotypes and biases~\cite{brown2020language,lucy2021gender, huang2019reducing, nozza2021honest, johnson2022ghost} \\\labelitemi\hspace{\dimexpr\labelsep+0.5\tabcolsep} Hence, this constraint was set out to mitigate any potential gender imbalances as the discussion topic is a contentious one. Moreover, GPT was specifically instructed to assign gender-neutral names to the personas. This approach prevents the association of particular names with male or female identities, fostering a more equitable and unbiased representation. \\\labelitemi\hspace{\dimexpr\labelsep+0.5\tabcolsep} Ensures GPT does not provide viewpoints that gives preferential treatment of one gender over the other.} \\
(Direct Reflective Nudge: Persona) For each perspective, be concise, giving at most three sentences. & {\labelitemi\hspace{\dimexpr\labelsep+0.5\tabcolsep}This constraint was introduced after observing GPT's tendency to generate long perspectives during testing. \\\labelitemi\hspace{\dimexpr\labelsep+0.5\tabcolsep} Ensures GPT deliver concise perspectives.} \\
(Indirect Reflective Nudge: Storytelling) Of the ten stories, create five with a positive tone and five with a negative tone. & \labelitemi\hspace{\dimexpr\labelsep+0.5\tabcolsep} Ensures that the reflector captures a broad spectrum of impacts, including both positive and negative aspects to avert one-sidedness. \\
(Indirect Reflective Nudge: Storytelling) Extend or make concise. & {\labelitemi\hspace{\dimexpr\labelsep+0.5\tabcolsep} Ensures a well-rounded representation of stories --- three long stories, three short stories, and four stories of medium length. \\\labelitemi\hspace{\dimexpr\labelsep+0.5\tabcolsep} This strategy allows us to capture the full spectrum of narrative possibilities and thereby provides a more thorough analysis of the reflector's effectiveness.} \\
(Image Modality) Image has no wordings. & {\labelitemi\hspace{\dimexpr\labelsep+0.5\tabcolsep} AI-generated images often produce words that are not human-readable, as the algorithms focus on replicating the visual shapes of letters and numbers rather than rendering actual text. This issue is similar to the challenges AI faces in accurately generating human hands, which frequently results in distorted representations due to the complexity of shapes involved~\cite{keyes2023hands}.} \\
(Video Modality) Platform: YouTube Shorts or Instagram Reels. & {\labelitemi\hspace{\dimexpr\labelsep+0.5\tabcolsep} Ensures that the generated video aligns with the format requirements of popular media platforms such as TikTok, Instagram Reels, and YouTube Shorts. By adhering to these guidelines, the video remains concise, typically between 90-120 seconds, in line with the standard duration of videos commonly found on these platforms.} \\
(Video Modality) Script: Follow the provided text exactly without any modifications. & {\labelitemi\hspace{\dimexpr\labelsep+0.5\tabcolsep} During testing, AI video generation platforms often augment scripts by adding extra narrative elements to enhance immersion and provide more context. To maintain consistency across all modalities, we explicitly constrain the AI to avoid any content alterations, ensuring that the script remains unchanged throughout the different formats.} \\
Here is the format of the results & \labelitemi\hspace{\dimexpr\labelsep+0.5\tabcolsep} Ensures that GPT provides results in a consistent format.      
\end{tblr}
\Description{This table provides a breakdown of the constraints applied to GPT and AI generative models during the generation of the variants for each modality, along with the underlying rationales for each constraint. The table has two columns with column header: constraints and rationale. In the constraints column, we list the specific constraints implemented for each modality and reflective nudge. Meanwhile, the rationale column elucidates the motivations behind the establishment of these constraints.}
\end{table*}

\newpage

\begin{table*}[!htbp]
\caption{Demographic Profiles of Participants in Study 1}
\label{tab: st1-demo}
\scalebox{0.74}{
\begin{tabular}{|ll|c|c|}
\hline
\multicolumn{1}{|l|}{\textbf{Category}} &
  \textbf{Dimensions} &
  \textbf{Direct Reflective Nudge (Persona)} &
  \textbf{Indirect Reflective Nudge (Storytelling)} \\ \hline
\multicolumn{2}{|l|}{Total   Number of Participants} &
  10 &
  10 \\ \hline
\multicolumn{1}{|l|}{\multirow{2}{*}{Gender}} &
  Total Number of Males &
  4 &
  4 \\ \cline{2-4} 
\multicolumn{1}{|l|}{} &
  Total Number of Females &
  6 &
  6 \\ \hline
\multicolumn{1}{|l|}{Age} &
  Average Age &
  23.9 &
  24.1 \\ \hline
\multicolumn{1}{|l|}{\multirow{2}{*}{Ethnicity}} &
  Asian or Pacific Islander &
  10 &
  9 \\ \cline{2-4} 
\multicolumn{1}{|l|}{} &
  Hispanic or Latino &
  0 &
  1 \\ \hline
\multicolumn{1}{|l|}{\multirow{2}{*}{Education}} &
  Bachelor's Degree &
  8 &
  8 \\ \cline{2-4} 
\multicolumn{1}{|l|}{} &
  Post-Graudate Degree &
  2 &
  2 \\ \hline
\multicolumn{1}{|l|}{\multirow{2}{*}{Language}} &
  English is first language &
  7 &
  8 \\ \cline{2-4} 
\multicolumn{1}{|l|}{} &
  English is NOT first language &
  3 &
  2 \\ \hline
\multicolumn{2}{|l|}{TITK Score ($M \pm S.D.$)} &
  $20.10 \pm 5.07$ &
  $18.70 \pm 3.92$ \\ \hline
\multicolumn{2}{|l|}{SRIS Score ($M \pm S.D.$)} &
  $89.00 \pm 11.07$ &
  $86.80 \pm 10.05$ \\ \hline
\multicolumn{1}{|l|}{\multirow{4}{*}{VARK Model (Internal -   Inherent Reflecting Styles)}} &
  Visual (V) Score ($M \pm S.D.$) &
  $8.70 \pm 2.71$ &
  $9.10 \pm 3.31$ \\ \cline{2-4} 
\multicolumn{1}{|l|}{} &
  Audio (A) Score ($M \pm S.D.$) &
  $7.70 \pm 3.30$ &
  $5.10 \pm 2.18$ \\ \cline{2-4} 
\multicolumn{1}{|l|}{} &
  Read/Write (R) Score ($M \pm S.D.$) &
  $6.80 \pm 3.71$ &
  $7.00 \pm 3.46$ \\ \cline{2-4} 
\multicolumn{1}{|l|}{} &
  Kinesthetic (K) Score ($M \pm S.D.$) &
  $10.90 \pm 2.64$ &
  $9.70 \pm 2.26$ \\ \hline
\multicolumn{1}{|l|}{\multirow{4}{*}{Daily Usage (External -  Influences and Interactions)}} &
  \begin{tabular}[c]{@{}l@{}}Frequency of Exposure to Text:\\ Blog Posts, Online Articles, News \\ ($M \pm S.D.$)\end{tabular} &
  $4.20 \pm 0.79$ &
  $3.30 \pm 1.25$ \\ \cline{2-4} 
\multicolumn{1}{|l|}{} &
  \begin{tabular}[c]{@{}l@{}}Frequency of Exposure to Images: \\ Instagram Posts, Facebook Posts, Pinterest\\ ($M \pm S.D.$)\end{tabular} &
  $4.00 \pm 0.94$ &
  $3.40 \pm 1.43$ \\ \cline{2-4} 
\multicolumn{1}{|l|}{} &
  \begin{tabular}[c]{@{}l@{}}Frequency of Exposure to Video:\\ Instagram Reels, Youtube Shorts, TikTok\\ ($M \pm S.D.$)\end{tabular} &
  $4.60 \pm 0.97$ &
  $4.10 \pm 1.45$ \\ \cline{2-4} 
\multicolumn{1}{|l|}{} &
  \begin{tabular}[c]{@{}l@{}}Frequency of Exposure to Audio: \\ Podcasts ($M \pm S.D.$)\end{tabular} &
  $2.20 \pm 1.23$ &
  $2.50 \pm 1.18$ \\ \hline
\multicolumn{2}{|l|}{Average Completion Time in Minutes} &
  45.6 &
  48.0 \\ \hline
\end{tabular}}
\Description{This table provides a comprehensive overview of the demographic characteristics of the participants in Study 1. Demographic profile captures Participants, Gender, Age, Education, TK-TI Scores, SRIS Scores, VARK scores and Daily Interactions with the modalities.}
\end{table*}


\begin{table*}[!htbp]
\caption{Demographic Profiles of Participants in Study 2}
\label{tab: st2-demo}
\scalebox{0.5}{
\begin{tabular}{|ll|c|c|c|c|c|c|c|c|}
\hline
\multicolumn{1}{|l|}{\textbf{Category}} &
  \textbf{Dimensions} &
  \textbf{Persona (Text)} &
  \textbf{Persona (Image)} &
  \textbf{Persona (Video)} &
  \textbf{Persona (Audio)} &
  \textbf{Storytelling (Text)} &
  \textbf{Storytelling (Image)} &
  \textbf{Storytelling (Video)} &
  \textbf{Storytelling (Audio)} \\ \hline
\multicolumn{2}{|l|}{Total Number of Participants} &
  25 &
  25 &
  25 &
  25 &
  25 &
  25 &
  25 &
  25 \\ \hline
\multicolumn{1}{|l|}{\multirow{3}{*}{Gender}} &
  Total Number of Males &
  14 &
  11 &
  11 &
  12 &
  13 &
  11 &
  15 &
  18 \\ \cline{2-10} 
\multicolumn{1}{|l|}{} &
  Total Number of Females &
  10 &
  14 &
  14 &
  13 &
  12 &
  14 &
  10 &
  7 \\ \cline{2-10} 
\multicolumn{1}{|l|}{} &
  Prefer Not to Say &
  1 &
  0 &
  0 &
  0 &
  0 &
  0 &
  0 &
  0 \\ \hline
\multicolumn{1}{|l|}{Age} &
  Average Age &
  30.72 &
  34.28 &
  32.48 &
  35.84 &
  33.68 &
  33.44 &
  37.32 &
  36.52 \\ \hline
\multicolumn{1}{|l|}{\multirow{7}{*}{Ethnicity}} &
  Asian or Pacific Islander &
  3 &
  2 &
  0 &
  2 &
  0 &
  0 &
  3 &
  3 \\ \cline{2-10} 
\multicolumn{1}{|l|}{} &
  Black or African American &
  0 &
  0 &
  1 &
  1 &
  1 &
  0 &
  0 &
  0 \\ \cline{2-10} 
\multicolumn{1}{|l|}{} &
  Hispanic or Latino &
  0 &
  1 &
  1 &
  0 &
  0 &
  0 &
  0 &
  1 \\ \cline{2-10} 
\multicolumn{1}{|l|}{} &
  Multiracial or Biracial &
  0 &
  0 &
  0 &
  0 &
  0 &
  0 &
  0 &
  0 \\ \cline{2-10} 
\multicolumn{1}{|l|}{} &
  Native American or Alaska Native &
  3 &
  0 &
  1 &
  1 &
  0 &
  1 &
  0 &
  0 \\ \cline{2-10} 
\multicolumn{1}{|l|}{} &
  White or Caucasian &
  18 &
  22 &
  22 &
  21 &
  24 &
  24 &
  22 &
  21 \\ \cline{2-10} 
\multicolumn{1}{|l|}{} &
  Other &
  1 &
  0 &
  0 &
  0 &
  0 &
  0 &
  0 &
  0 \\ \hline
\multicolumn{1}{|l|}{\multirow{5}{*}{Education}} &
  No Diploma or less &
  0 &
  0 &
  0 &
  0 &
  0 &
  0 &
  0 &
  0 \\ \cline{2-10} 
\multicolumn{1}{|l|}{} &
  High School, Diploma or the equivalent &
  3 &
  1 &
  2 &
  2 &
  0 &
  1 &
  0 &
  2 \\ \cline{2-10} 
\multicolumn{1}{|l|}{} &
  Associate's Degree &
  1 &
  1 &
  1 &
  2 &
  1 &
  1 &
  0 &
  3 \\ \cline{2-10} 
\multicolumn{1}{|l|}{} &
  Bachelor's Degree &
  15 &
  23 &
  19 &
  20 &
  23 &
  18 &
  20 &
  18 \\ \cline{2-10} 
\multicolumn{1}{|l|}{} &
  Post-graduate Degree &
  6 &
  0 &
  3 &
  1 &
  1 &
  5 &
  5 &
  2 \\ \hline
\multicolumn{1}{|l|}{\multirow{2}{*}{Language}} &
  English is first language &
  23 &
  25 &
  24 &
  25 &
  25 &
  25 &
  24 &
  25 \\ \cline{2-10} 
\multicolumn{1}{|l|}{} &
  English is NOT first language &
  2 &
  0 &
  1 &
  0 &
  0 &
  0 &
  1 &
  0 \\ \hline
\multicolumn{2}{|l|}{TITK Score ($M \pm S.D.$)} &
  $21.04 \pm 4.89$ &
  $23.16 \pm 3.50$ &
  $22.28 \pm 5.26$ &
  $22.28 \pm 6.78$ &
  $23.20 \pm 4.37$ &
  $23.20 \pm 5.98$ &
  $27.08 \pm 5.99$ &
  $26.12 \pm 5.90$ \\ \hline
\multicolumn{2}{|l|}{SRIS Score ($M \pm S.D.$)} &
  $81.32 \pm 11.91$ &
  $84.84 \pm 9.48$ &
  $84.64 \pm 13.73$ &
  $85.64 \pm 12.68$ &
  $84.68 \pm 18.45$ &
  $86.24 \pm 13.82$ &
  $83.20 \pm 12.43$ &
  $82.76 \pm 10.33$ \\ \hline
\multicolumn{1}{|l|}{\multirow{4}{*}{VARK Model (Internal -   Inherent Reflecting Styles)}} &
  Visual (V) Score ($M \pm S.D.$) &
  $7.52 \pm 3.11$ &
  $8.40 \pm 3.72$ &
  $7.92 \pm 3.98$ &
  $6.92 \pm 3.59$ &
  $7.68 \pm 3.87$ &
  $5.40 \pm 2.47$ &
  $6.12 \pm 2.71$ &
  $5.88 \pm 3.23$ \\ \cline{2-10} 
\multicolumn{1}{|l|}{} &
  Audio (A) Score ($M \pm S.D.$) &
  $8.96 \pm 3.85$ &
  $7.36 \pm 2.84$ &
  $8.48 \pm 3.63$ &
  $7.36 \pm 4.19$ &
  $8.28 \pm 3.68$ &
  $7.88 \pm 3.10$ &
  $8.64 \pm 2.96$ &
  $6.80 \pm 2.33$ \\ \cline{2-10} 
\multicolumn{1}{|l|}{} &
  Read/Write (R) Score ($M \pm S.D.$) &
  $7.96 \pm 3.51$ &
  $7.68 \pm 3.20$ &
  $8.16 \pm 2.78$ &
  $7.36 \pm 3.77$ &
  $7.48 \pm 3.79$ &
  $6.92 \pm 2.41$ &
  $6.32 \pm 2.76$ &
  $6.12 \pm 3.49$ \\ \cline{2-10} 
\multicolumn{1}{|l|}{} &
  Kinesthetic (K) Score ($M \pm S.D.$) &
  $8.44 \pm 3.50$ &
  $9.00 \pm 3.35$ &
  $8.08 \pm 2.71$ &
  $7.72 \pm 3.66$ &
  $8.52 \pm 3.20$ &
  $6.84 \pm 3.10$ &
  $8.32 \pm 2.41$ &
  $7.20 \pm 2.72$ \\ \hline
\multicolumn{1}{|l|}{\multirow{4}{*}{Daily Usage (External -   Influences and Interactions)}} &
  \begin{tabular}[c]{@{}l@{}}Frequency of Exposure to Text: \\ Blog Posts, Online Articles, News\\ ($M \pm S.D.$)\end{tabular} &
  $3.72 \pm 1.10$ &
  $3.80 \pm 0.65$ &
  $3.84 \pm 0.99$ &
  $3.92 \pm 0.91$ &
  $3.76 \pm 0.72$ &
  $3.72 \pm 0.98$ &
  $4.20 \pm 0.71$ &
  $4.24 \pm 0.88$ \\ \cline{2-10} 
\multicolumn{1}{|l|}{} &
  \begin{tabular}[c]{@{}l@{}}Frequency of Exposure to Images: \\ Instagram Posts, Facebook   Posts, Pinterest\\ ($M \pm S.D.$)\end{tabular} &
  $4.28 \pm 0.74$ &
  $4.24 \pm 0.93$ &
  $4.24 \pm 0.88$ &
  $4.36 \pm 1.11$ &
  $4.04 \pm 0.93$ &
  $4.04 \pm 1.06$ &
  $4.68 \pm 0.48$ &
  $4.32 \pm 0.99$ \\ \cline{2-10} 
\multicolumn{1}{|l|}{} &
  \begin{tabular}[c]{@{}l@{}}Frequency of Exposure to Video:\\ Instagram Reels, Youtube Shorts, TikTok\\ ($M \pm S.D.$)\end{tabular} &
  $4.20 \pm 0.91$ &
  $4.56 \pm 0.87$ &
  $4.08 \pm 0.95$ &
  $4.44 \pm 0.96$ &
  $4.40 \pm 0.76$ &
  $4.16 \pm 0.99$ &
  $4.64 \pm 0.70$ &
  $4.44 \pm 0.92$ \\ \cline{2-10} 
\multicolumn{1}{|l|}{} &
  \begin{tabular}[c]{@{}l@{}}Frequency of Exposure to Audio: \\ Podcasts ($M \pm S.D.$)\end{tabular} &
  $3.76 \pm 1.16$ &
  $3.52 \pm 1.08$ &
  $3.60 \pm 1.15$ &
  $3.68 \pm 1.07$ &
  $3.56 \pm 1.00$ &
  $3.56 \pm 1.04$ &
  $3.92 \pm 1.00$ &
  $3.88 \pm 1.10$ \\ \hline
\multicolumn{2}{|l|}{Average Completion Time in Minutes} &
  25.66 &
  47.22 &
  25.50 &
  32.18 &
  28.31 &
  32.88 &
  24.34 &
  24.96 \\ \hline
\end{tabular}}
\Description{This table provides a comprehensive overview of the demographic characteristics of the participants in Study 2. Demographic profile captures Participants, Gender, Age, Education, TK-TI Scores, SRIS Scores, VARK scores and Daily Interactions with the modalities.}
\end{table*}

\clearpage
\newpage

\begin{figure*}[!htbp]
  \centering
  \includegraphics[width=\linewidth]{figures/Butterfly_Text.png}
  \caption{Butterfly Chart for the Text Modality across Both Nudge Types}
  \label{fig: butterfly chart text}
  \Description{Butterfly chart showing how the text modality fare for each of the categories under Kahneman's dual-system thinking model.}
\end{figure*}

\begin{figure*}[!htbp]
  \centering
  \includegraphics[width=\linewidth]{figures/Butterfly_Image.png}
  \caption{Butterfly Chart for the Image Modality across Both Nudge Types}
  \label{fig: butterfly chart image}
  \Description{Butterfly chart showing how the image modality fare for each of the categories under Kahneman's dual-system thinking model.}
\end{figure*}

\begin{figure*}[!htbp]
  \centering
  \includegraphics[width=\linewidth]{figures/Butterfly_Video.png}
  \caption{Butterfly Chart for the Video Modality across Both Nudge Types}
  \label{fig: butterfly chart video}
  \Description{Butterfly chart showing how the video modality fare for each of the categories under Kahneman's dual-system thinking model.}
\end{figure*}

\begin{figure*}[!htbp]
  \centering
  \includegraphics[width=\linewidth]{figures/Butterfly_Audio.png}
  \caption{Butterfly Chart for the Audio Modality across Both Nudge Types}
  \label{fig: butterfly chart audio}
  \Description{Butterfly chart showing how the audio modality fare for each of the categories under Kahneman's dual-system thinking model.}
\end{figure*}