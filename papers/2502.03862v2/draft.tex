%%
%% This is file `sample-manuscript.tex',
%% generated with the docstrip utility.
%%
%% The original source files were:
%%
%% samples.dtx  (with options: `manuscript')
%% 
%% IMPORTANT NOTICE:
%% 
%% For the copyright see the source file.
%% 
%% Any modified versions of this file must be renamed
%% with new filenames distinct from sample-manuscript.tex.
%% 
%% For distribution of the original source see the terms
%% for copying and modification in the file samples.dtx.
%% 
%% This generated file may be distributed as long as the
%% original source files, as listed above, are part of the
%% same distribution. (The sources need not necessarily be
%% in the same archive or directory.)
%%
%% Commands for TeXCount
%TC:macro \cite [option:text,text]
%TC:macro \citep [option:text,text]
%TC:macro \citet [option:text,text]
%TC:envir table 0 1
%TC:envir table* 0 1
%TC:envir tabular [ignore] word
%TC:envir displaymath 0 word
%TC:envir math 0 word
%TC:envir comment 0 0
%%
%%
%% The first command in your LaTeX source must be the \documentclass command.
%%%% Small single column format, used for CIE, CSUR, DTRAP, JACM, JDIQ, JEA, JERIC, JETC, PACMCGIT, TAAS, TACCESS, TACO, TALG, TALLIP (formerly TALIP), TCPS, TDSCI, TEAC, TECS, TELO, THRI, TIIS, TIOT, TISSEC, TIST, TKDD, TMIS, TOCE, TOCHI, TOCL, TOCS, TOCT, TODAES, TODS, TOIS, TOIT, TOMACS, TOMM (formerly TOMCCAP), TOMPECS, TOMS, TOPC, TOPLAS, TOPS, TOS, TOSEM, TOSN, TQC, TRETS, TSAS, TSC, TSLP, TWEB.
% \documentclass[acmsmall]{acmart}

%%%% Large single column format, used for IMWUT, JOCCH, PACMPL, POMACS, TAP, PACMHCI
% \documentclass[acmlarge,screen]{acmart}

%%%% Large double column format, used for TOG
% \documentclass[acmtog, authorversion]{acmart}

% fix for the hyperref vs hyperxmp bug
\DocumentMetadata{}

%%%% Generic manuscript mode, required for submission
%%%% and peer review
\documentclass[sigconf,screen]{acmart}

%package
\usepackage{multirow}
\usepackage{graphicx}
\usepackage{graphics}
\usepackage{subcaption}
\usepackage{dcolumn}
\usepackage{tabularray}




%% Fonts used in the template cannot be substituted; margin 
%% adjustments are not allowed.
%%
%% \BibTeX command to typeset BibTeX logo in the docs
\AtBeginDocument{%
  \providecommand\BibTeX{{%
    \normalfont B\kern-0.5em{\scshape i\kern-0.25em b}\kern-0.8em\TeX}}}

%% Rights management information.  This information is sent to you
%% when you complete the rights form.  These commands have SAMPLE
%% values in them; it is your responsibility as an author to replace
%% the commands and values with those provided to you when you
%% complete the rights form.
\copyrightyear{2025}
\acmYear{2025}
\setcopyright{acmlicensed}\acmConference[CHI '25]{CHI Conference on Human Factors in Computing Systems}{April 26-May 1, 2025}{Yokohama, Japan}
\acmBooktitle{CHI Conference on Human Factors in Computing Systems (CHI '25), April 26-May 1, 2025, Yokohama, Japan}
\acmDOI{10.1145/3706598.3714189}
\acmISBN{979-8-4007-1394-1/25/04}

%% These commands are for a PROCEEDINGS abstract or paper.
%
%  Uncomment \acmBooktitle if th title of the proceedings is different
%  from ``Proceedings of ...''!


%%
%% Submission ID.
%% Use this when submitting an article to a sponsored event. You'll
%% receive a unique submission ID from the organizers
%% of the event, and this ID should be used as the parameter to this command.
%%\acmSubmissionID{123-A56-BU3}

%%
%% For managing citations, it is recommended to use bibliography
%% files in BibTeX format.
%%
%% You can then either use BibTeX with the ACM-Reference-Format style,
%% or BibLaTeX with the acmnumeric or acmauthoryear sytles, that include
%% support for advanced citation of software artefact from the
%% biblatex-software package, also separately available on CTAN.
%%
%% Look at the sample-*-biblatex.tex files for templates showcasing
%% the biblatex styles.
%%

%%
%% The majority of ACM publications use numbered citations and
%% references.  The command \citestyle{authoryear} switches to the
%% "author year" style.
%%
%% If you are preparing content for an event
%% sponsored by ACM SIGGRAPH, you must use the "author year" style of
%% citations and references.
%% Uncommenting
%% the next command will enable that style.
%%\citestyle{acmauthoryear}

%%
%% end of the preamble, start of the body of the document source.
\begin{document}

%%
%% The "title" command has an optional parameter,
%% allowing the author to define a "short title" to be used in page headers.
\title[Multimodal Reflection Nudges for Deliberativeness]{Enhancing Deliberativeness: Evaluating the Impact of Multimodal Reflection Nudges}

%%
%% The "author" command and its associated commands are used to define
%% the authors and their affiliations.
%% Of note is the shared affiliation of the first two authors, and the
%% "authornote" and "authornotemark" commands
%% used to denote shared contribution to the research.
\author{ShunYi Yeo}
\email{yeoshunyi.sutd@gmail.com}
\affiliation{%
  \institution{Singapore University of Technology and Design}
  \city{Singapore}
  \country{Singapore}
}

\author{Zhuoqun Jiang}
\email{zhuoqun\_jiang@mymail.sutd.edu.sg}
\affiliation{%
  \institution{Singapore University of Technology and Design}
  \city{Singapore}
  \country{Singapore}
}

\author{Anthony Tang}
\email{tonyt@smu.edu.sg}
\affiliation{%
  \institution{Singapore Management University}
  \city{Singapore}
  \country{Singapore}
}

\author{Simon Tangi Perrault}
\email{perrault.simon@gmail.com}
\affiliation{%
  \institution{Singapore University of Technology and Design}
  \city{Singapore}
  \country{Singapore}
}







%%
%% By default, the full list of authors will be used in the page
%% headers. Often, this list is too long, and will overlap
%% other information printed in the page headers. This command allows
%% the author to define a more concise list
%% of authors' names for this purpose.
\renewcommand{\shortauthors}{Yeo et al.}

%%
%% The abstract is a short summary of the work to be presented in the
%% article.
\begin{abstract}


The choice of representation for geographic location significantly impacts the accuracy of models for a broad range of geospatial tasks, including fine-grained species classification, population density estimation, and biome classification. Recent works like SatCLIP and GeoCLIP learn such representations by contrastively aligning geolocation with co-located images. While these methods work exceptionally well, in this paper, we posit that the current training strategies fail to fully capture the important visual features. We provide an information theoretic perspective on why the resulting embeddings from these methods discard crucial visual information that is important for many downstream tasks. To solve this problem, we propose a novel retrieval-augmented strategy called RANGE. We build our method on the intuition that the visual features of a location can be estimated by combining the visual features from multiple similar-looking locations. We evaluate our method across a wide variety of tasks. Our results show that RANGE outperforms the existing state-of-the-art models with significant margins in most tasks. We show gains of up to 13.1\% on classification tasks and 0.145 $R^2$ on regression tasks. All our code and models will be made available at: \href{https://github.com/mvrl/RANGE}{https://github.com/mvrl/RANGE}.

\end{abstract}



%%
%% The code below is generated by the tool at http://dl.acm.org/ccs.cfm.
%% Please copy and paste the code instead of the example below.
%%
\begin{CCSXML}
<ccs2012>
   <concept>
       <concept_id>10003120.10003121.10003129</concept_id>
       <concept_desc>Human-centered computing~Interactive systems and tools</concept_desc>
       <concept_significance>100</concept_significance>
       </concept>
   <concept>
       <concept_id>10003120.10003121.10011748</concept_id>
       <concept_desc>Human-centered computing~Empirical studies in HCI</concept_desc>
       <concept_significance>100</concept_significance>
       </concept>
 </ccs2012>
\end{CCSXML}

\ccsdesc[500]{Human-centered computing}
\ccsdesc[300]{Empirical studies in collaborative and social computing}

%%
%% Keywords. The author(s) should pick words that accurately describe
%% the work being presented. Separate the keywords with commas.
\keywords{deliberation, deliberativeness, deliberative quality, internal reflection, online deliberation, public discussions, nudges, indirect reflector, direct reflector, reflection, self-reflection, multimedia, multi-modality, large language model, civic engagement}


\begin{teaserfigure}
\centering
  \includegraphics[width=\textwidth]{figures/Teaser.png}
  \caption{\textbf{1:} In online deliberation, the quality of collective discussions is directly influenced by the depth of individual reflection preceding them; \textbf{2:} To enhance users' reflection, text-based reflective nudges are commonly employed; \textbf{3:} To support various reflecting styles of individuals, we move beyond traditional text-based reflective nudges to multimodal reflective nudges; \textbf{3a and 3b:} We present \textit{Reflect}, a simple interface nudge powered by AI that showcases an array of distinct modalities, each guiding users' self-reflection through different presentations of Video (blue), Text (pink), Audio (yellow) and Image (green).}
  \label{fig: teaser}
  \Description{1: In online deliberation, the quality of collective discussions is directly influenced by the depth of individual reflection preceding them. 2: To enhance users' reflection, text-based reflective nudges are commonly employed. 3: To support various reflecting styles of individuals, we move beyond traditional text-based reflective nudges to multimodal reflective nudges. 3a and 3b: We present Reflect, a simple interface nudge powered by AI that showcases an array of distinct modalities, each guiding users' self-reflection through different presentations of Video (blue), Text (pink), Audio (yellow) and Image (green).}
\end{teaserfigure}


%% A "teaser" image appears between the author and affiliation
%% information and the body of the document, and typically spans the
%% page.
% \begin{teaserfigure}
%   \includegraphics[width=\textwidth]{sampleteaser}
%   \caption{Seattle Mariners at Spring Training, 2010.}
%   \Description{Enjoying the baseball game from the third-base
%   seats. Ichiro Suzuki preparing to bat.}
%   \label{fig:teaser}
% \end{teaserfigure}

% \received{20 February 2007}
% \received[revised]{12 March 2009}
% \received[accepted]{5 June 2009}

%%
%% This command processes the author and affiliation and title
%% information and builds the first part of the formatted document.
\maketitle



\section{Introduction}

Video generation has garnered significant attention owing to its transformative potential across a wide range of applications, such media content creation~\citep{polyak2024movie}, advertising~\citep{zhang2024virbo,bacher2021advert}, video games~\citep{yang2024playable,valevski2024diffusion, oasis2024}, and world model simulators~\citep{ha2018world, videoworldsimulators2024, agarwal2025cosmos}. Benefiting from advanced generative algorithms~\citep{goodfellow2014generative, ho2020denoising, liu2023flow, lipman2023flow}, scalable model architectures~\citep{vaswani2017attention, peebles2023scalable}, vast amounts of internet-sourced data~\citep{chen2024panda, nan2024openvid, ju2024miradata}, and ongoing expansion of computing capabilities~\citep{nvidia2022h100, nvidia2023dgxgh200, nvidia2024h200nvl}, remarkable advancements have been achieved in the field of video generation~\citep{ho2022video, ho2022imagen, singer2023makeavideo, blattmann2023align, videoworldsimulators2024, kuaishou2024klingai, yang2024cogvideox, jin2024pyramidal, polyak2024movie, kong2024hunyuanvideo, ji2024prompt}.


In this work, we present \textbf{\ours}, a family of rectified flow~\citep{lipman2023flow, liu2023flow} transformer models designed for joint image and video generation, establishing a pathway toward industry-grade performance. This report centers on four key components: data curation, model architecture design, flow formulation, and training infrastructure optimization—each rigorously refined to meet the demands of high-quality, large-scale video generation.


\begin{figure}[ht]
    \centering
    \begin{subfigure}[b]{0.82\linewidth}
        \centering
        \includegraphics[width=\linewidth]{figures/t2i_1024.pdf}
        \caption{Text-to-Image Samples}\label{fig:main-demo-t2i}
    \end{subfigure}
    \vfill
    \begin{subfigure}[b]{0.82\linewidth}
        \centering
        \includegraphics[width=\linewidth]{figures/t2v_samples.pdf}
        \caption{Text-to-Video Samples}\label{fig:main-demo-t2v}
    \end{subfigure}
\caption{\textbf{Generated samples from \ours.} Key components are highlighted in \textcolor{red}{\textbf{RED}}.}\label{fig:main-demo}
\end{figure}


First, we present a comprehensive data processing pipeline designed to construct large-scale, high-quality image and video-text datasets. The pipeline integrates multiple advanced techniques, including video and image filtering based on aesthetic scores, OCR-driven content analysis, and subjective evaluations, to ensure exceptional visual and contextual quality. Furthermore, we employ multimodal large language models~(MLLMs)~\citep{yuan2025tarsier2} to generate dense and contextually aligned captions, which are subsequently refined using an additional large language model~(LLM)~\citep{yang2024qwen2} to enhance their accuracy, fluency, and descriptive richness. As a result, we have curated a robust training dataset comprising approximately 36M video-text pairs and 160M image-text pairs, which are proven sufficient for training industry-level generative models.

Secondly, we take a pioneering step by applying rectified flow formulation~\citep{lipman2023flow} for joint image and video generation, implemented through the \ours model family, which comprises Transformer architectures with 2B and 8B parameters. At its core, the \ours framework employs a 3D joint image-video variational autoencoder (VAE) to compress image and video inputs into a shared latent space, facilitating unified representation. This shared latent space is coupled with a full-attention~\citep{vaswani2017attention} mechanism, enabling seamless joint training of image and video. This architecture delivers high-quality, coherent outputs across both images and videos, establishing a unified framework for visual generation tasks.


Furthermore, to support the training of \ours at scale, we have developed a robust infrastructure tailored for large-scale model training. Our approach incorporates advanced parallelism strategies~\citep{jacobs2023deepspeed, pytorch_fsdp} to manage memory efficiently during long-context training. Additionally, we employ ByteCheckpoint~\citep{wan2024bytecheckpoint} for high-performance checkpointing and integrate fault-tolerant mechanisms from MegaScale~\citep{jiang2024megascale} to ensure stability and scalability across large GPU clusters. These optimizations enable \ours to handle the computational and data challenges of generative modeling with exceptional efficiency and reliability.


We evaluate \ours on both text-to-image and text-to-video benchmarks to highlight its competitive advantages. For text-to-image generation, \ours-T2I demonstrates strong performance across multiple benchmarks, including T2I-CompBench~\citep{huang2023t2i-compbench}, GenEval~\citep{ghosh2024geneval}, and DPG-Bench~\citep{hu2024ella_dbgbench}, excelling in both visual quality and text-image alignment. In text-to-video benchmarks, \ours-T2V achieves state-of-the-art performance on the UCF-101~\citep{ucf101} zero-shot generation task. Additionally, \ours-T2V attains an impressive score of \textbf{84.85} on VBench~\citep{huang2024vbench}, securing the top position on the leaderboard (as of 2025-01-25) and surpassing several leading commercial text-to-video models. Qualitative results, illustrated in \Cref{fig:main-demo}, further demonstrate the superior quality of the generated media samples. These findings underscore \ours's effectiveness in multi-modal generation and its potential as a high-performing solution for both research and commercial applications.
\section{Related Work}

\subsection{Large 3D Reconstruction Models}
Recently, generalized feed-forward models for 3D reconstruction from sparse input views have garnered considerable attention due to their applicability in heavily under-constrained scenarios. The Large Reconstruction Model (LRM)~\cite{hong2023lrm} uses a transformer-based encoder-decoder pipeline to infer a NeRF reconstruction from just a single image. Newer iterations have shifted the focus towards generating 3D Gaussian representations from four input images~\cite{tang2025lgm, xu2024grm, zhang2025gslrm, charatan2024pixelsplat, chen2025mvsplat, liu2025mvsgaussian}, showing remarkable novel view synthesis results. The paradigm of transformer-based sparse 3D reconstruction has also successfully been applied to lifting monocular videos to 4D~\cite{ren2024l4gm}. \\
Yet, none of the existing works in the domain have studied the use-case of inferring \textit{animatable} 3D representations from sparse input images, which is the focus of our work. To this end, we build on top of the Large Gaussian Reconstruction Model (GRM)~\cite{xu2024grm}.

\subsection{3D-aware Portrait Animation}
A different line of work focuses on animating portraits in a 3D-aware manner.
MegaPortraits~\cite{drobyshev2022megaportraits} builds a 3D Volume given a source and driving image, and renders the animated source actor via orthographic projection with subsequent 2D neural rendering.
3D morphable models (3DMMs)~\cite{blanz19993dmm} are extensively used to obtain more interpretable control over the portrait animation. For example, StyleRig~\cite{tewari2020stylerig} demonstrates how a 3DMM can be used to control the data generated from a pre-trained StyleGAN~\cite{karras2019stylegan} network. ROME~\cite{khakhulin2022rome} predicts vertex offsets and texture of a FLAME~\cite{li2017flame} mesh from the input image.
A TriPlane representation is inferred and animated via FLAME~\cite{li2017flame} in multiple methods like Portrait4D~\cite{deng2024portrait4d}, Portrait4D-v2~\cite{deng2024portrait4dv2}, and GPAvatar~\cite{chu2024gpavatar}.
Others, such as VOODOO 3D~\cite{tran2024voodoo3d} and VOODOO XP~\cite{tran2024voodooxp}, learn their own expression encoder to drive the source person in a more detailed manner. \\
All of the aforementioned methods require nothing more than a single image of a person to animate it. This allows them to train on large monocular video datasets to infer a very generic motion prior that even translates to paintings or cartoon characters. However, due to their task formulation, these methods mostly focus on image synthesis from a frontal camera, often trading 3D consistency for better image quality by using 2D screen-space neural renderers. In contrast, our work aims to produce a truthful and complete 3D avatar representation from the input images that can be viewed from any angle.  

\subsection{Photo-realistic 3D Face Models}
The increasing availability of large-scale multi-view face datasets~\cite{kirschstein2023nersemble, ava256, pan2024renderme360, yang2020facescape} has enabled building photo-realistic 3D face models that learn a detailed prior over both geometry and appearance of human faces. HeadNeRF~\cite{hong2022headnerf} conditions a Neural Radiance Field (NeRF)~\cite{mildenhall2021nerf} on identity, expression, albedo, and illumination codes. VRMM~\cite{yang2024vrmm} builds a high-quality and relightable 3D face model using volumetric primitives~\cite{lombardi2021mvp}. One2Avatar~\cite{yu2024one2avatar} extends a 3DMM by anchoring a radiance field to its surface. More recently, GPHM~\cite{xu2025gphm} and HeadGAP~\cite{zheng2024headgap} have adopted 3D Gaussians to build a photo-realistic 3D face model. \\
Photo-realistic 3D face models learn a powerful prior over human facial appearance and geometry, which can be fitted to a single or multiple images of a person, effectively inferring a 3D head avatar. However, the fitting procedure itself is non-trivial and often requires expensive test-time optimization, impeding casual use-cases on consumer-grade devices. While this limitation may be circumvented by learning a generalized encoder that maps images into the 3D face model's latent space, another fundamental limitation remains. Even with more multi-view face datasets being published, the number of available training subjects rarely exceeds the thousands, making it hard to truly learn the full distibution of human facial appearance. Instead, our approach avoids generalizing over the identity axis by conditioning on some images of a person, and only generalizes over the expression axis for which plenty of data is available. 

A similar motivation has inspired recent work on codec avatars where a generalized network infers an animatable 3D representation given a registered mesh of a person~\cite{cao2022authentic, li2024uravatar}.
The resulting avatars exhibit excellent quality at the cost of several minutes of video capture per subject and expensive test-time optimization.
For example, URAvatar~\cite{li2024uravatar} finetunes their network on the given video recording for 3 hours on 8 A100 GPUs, making inference on consumer-grade devices impossible. In contrast, our approach directly regresses the final 3D head avatar from just four input images without the need for expensive test-time fine-tuning.


\section{Choices of the Nudges and Modalities}
\label{sec: section3}
We now outline the design of our interface nudges and the rationale behind the selection of both the nudges and the modalities.

\subsection{Choice of Nudges}
Our goal was to consider nudges that would trigger different types of self-reflection. We identified two types: \textbf{direct reflective nudges} and \textbf{indirect reflective nudges}. \textbf{Direct} reflective nudges typically involve explicit, targeted prompts that encourage individuals to reflect on their own thoughts, experiences or opinions, and are usually shorter, while \textbf{indirect} ones are subtler, often encouraging reflection through external examples or narratives, allowing users to reflect by considering others' viewpoints or scenarios and usually feature more content. This approach allows us to explore how different modalities support varying levels of reflective nudges.

\subsubsection{Direct Reflective Nudge: Persona}
The direct reflective nudge employs a \textit{persona} approach, encouraging reflection through perspective-taking~\cite{galinsky2000perspective, kim2019crowdsourcing, zhang2021nudge}. This approach is grounded in principles from constructivist learning theories~\cite{vygotsky1978mind, piaget1985equilibration}. By guiding users to explicitly examine their own thoughts, the persona approach mirrors Vygotsky's concept of mediated learning~\cite{vygotsky1978mind}, enabling users to internalize insights through active interaction with reflective stimuli. Piaget's theory~\cite{piaget1985equilibration} complements this by focusing on how learners construct knowledge through self-directed exploration. Reflection, as facilitated by the persona, prompts users to critically evaluate their preconceptions, fostering new understandings and alignment with Piaget’s notion of active knowledge construction. Moreover, we specifically chose persona as its ability to reach higher levels of deliberativeness compared to others, have been established in previous work~\cite{yeo2024help}.

\subsubsection{Indirect Reflective Nudge: Storytelling}
The indirect reflective nudge employs a \textit{storytelling} approach, encouraging users to reflect through narratives that present others' perspectives~\cite{batson1997perspective, yeo2024help}. By fostering reflection through vicarious experiences, storytelling provides richer contextual information than the persona-based approach by extending its content with an added storyline. Grounded in narrative-based learning~\cite{bruner1991narrative, green2000role}, storytelling is described by Bruner~\cite{bruner1991narrative} as a fundamental mode of thought, enabling the construction of personal and social meaning through interpretation. Green and Brock~\cite{green2000role} further explore the concept of `transportation' --- the immersive mental absorption into a narrative, characterized by focused attention, emotional engagement and vivid imagery. Their findings highlight the persuasive power of storytelling in shaping beliefs and attitudes. Additionally, prior research demonstrates that storytelling improves deliberativeness by engaging emotional and cognitive processes, ultimately enhancing critical thinking and reflective engagement~\cite{yeo2024help}.

\paragraph{\normalfont{Both nudges leverage self-referential encoding, where individuals process and internalize information by relating it to their own life, enhancing engagement and cognitive retention~\cite{rogers1977self}. These nudges also align with Kahneman's dual-system theories~\cite{kahneman2011thinking} (see section~\ref{sec: dual system thinking}), with the direct persona nudge engaging System 2 (i.e., analytical, reflective thinking) and the indirect storytelling nudge activating System 1 (intuitive, empathetic processing). Together, these nudges provide a multifaceted approach to fostering reflection.}} 

\subsection{Choice and Design of Modalities}
\label{sec: Modalities}
The design of each modality was guided by principles from multimedia learning theory~\cite{mayer2005cambridge, fadel2008multimodal} to enhance reflective engagement. We chose a set of modalities representative of non-interactive multimedia formats identified in prior research~\cite{mayer2005cognitive, de2005multimedia, deimann2006volitional, fadel2008multimodal}, ensuring they were representative of diverse content representations. According to Fadel~\cite{fadel2008multimodal}, non-interactive multimodal learning includes combinations such as text with visuals, audio and video formats. Empirical studies~\cite{mayer2005cognitive, de2005multimedia, deimann2006volitional, fadel2008multimodal} demonstrate that non-interactive, multimodal learning significantly improves learning outcomes compared to traditional single-mode approaches. Interestingly, when these scenarios shift from non-interactive to interactive settings, these gains are not statistically significant. Guided by this evidence, we identified the following non-interactive modalities to support reflection:
\begin{enumerate}
    \item \textbf{Text}: A traditional and widely used medium for reflective prompts. It follows the design principles of Cooper for persona~\cite{cooper1999inmates, cooper2007face, pruitt2010persona} and Freidus et al.~\cite{freidus2002digital} and Bruner~\cite{bruner1991narrative} for storytelling. As text is the most conventional modality, it is structured to optimize engagement without overwhelming users.
    \item \textbf{Image}: This modality integrates visual stimuli to enhance reflection, featuring the same text content, with additional images. It follows the \textit{spatial contiguity principle}, ensuring that images are placed alongside related text to minimize split-attention and enhance understanding~\cite{hegarty1993constructing, ayres2005split, moreno1999cognitive, mayer1989systematic, mayer1995generative, chandler1992split}. 
    \item \textbf{Audio}: Designed to leverage auditory cues, this modality employs the \textit{voice principle}, which suggests that reflections are more effective when delivered in a natural, human voice with a standard accent rather than a machine or foreign-accented narration~\cite{atkinson2005fostering}. This conversational delivery ensures engagement and encourages deeper thought.    
    \item \textbf{Video}: Combining visual and auditory elements, video content leverages its dynamic nature to simulate real-world experiences. This modality adheres to the \textit{temporal contiguity principle}, synchronizing narration with animations to ensure that verbal and visual information are presented together for better cognitive processing~\cite{moreno1999cognitive, mayer1991animations, mayer1992instructive, mousavi1995reducing}. Additionally, Paivio’s \textit{Dual-Coding Theory} supports the use of auditory and visual channels to reinforce reflective thinking~\cite{paivio2013imagery}.
\end{enumerate}

Figure~\ref{fig: modalities} provides examples for each of the modality and their respective reflective nudges. Content across modalities in each nudge type is the same.

\begin{figure*}[!htbp]
  \centering
  \includegraphics[width=\linewidth]{figures/Four_Modalities.png}
  \caption{Example of the Two Reflective Nudges and their Respective Four Modalities.}
  \label{fig: modalities}
  \Description{Example of the two selected reflective nudges and their respective four modalities. An example for the text, persona: Dr Selene, 43 year old oceanographer. Sea levels have been naturally rising for years. It is clear that natural variability has dominated sea level rise, with changes in ocean heat content and changes in precipitation patterns. An example for the image, persona: an image of an oceanographer is shown with the exact same text shown at the bottom of the image. An example for the video, persona: a video is shown with the title - an oceanographer's insight on climate change. An example for the audio, persona: an audio is shown. An example for the text, storytelling: Dr Selene, a 43 year old oceanographer, dedicated her life to studying the oceans. She spent years understanding the natural patterns of the ocean while encountering playful dolphins and curious whales. An example for the image, storytelling: six different images were shown with the exact same text shown at the bottom of the images. Video and audio formats for storytelling is the same for persona except for its duration.} 
\end{figure*}

\subsection{Implementation of the Modalities with LLM and Generative AI}
We used GPT-4.0 to generate a range of 10 textual prompts for the direct reflective nudge. We then instructed the LLM to generate short narratives for each of the same textual prompts for indirect reflective nudge. Following, to generate modalities, we used text-to-image generation tools (Bing Image Creator), text-to-video generation tool (Invideo AI) and text-to-speech generation tool (Narakeet AI). This ensures that the content across \textbf{all modalities is strictly the same}, just presented differently by their own modality, such that we can assess the impacts of the modalities on deliberativeness and not due to the quality of the content presented in each modality.

The utilization of LLMs to promote self-reflection on online deliberation platforms ensures the adaptability and scalability of reflective nudges.
%Using LLMs, we tap into its vast knowledge and language proficiency in enabling reflective nudges to accommodate to a diverse array of topics commonly found in online discussions. 
%This enhances and versatility of the reflective nudges to effectively cater to the varied and dynamic nature of discussions on online deliberation platforms.

\paragraph{\normalfont{\textbf{Generating textual variants for both nudges:} In creating the textual variants for the direct reflective nudge (persona), GPT was tasked with the role of ``\textit{a helpful assistant focusing on supporting users' self-reflection on a given topic}.'' We adhered to scholarly design principles for each reflective nudge, ensuring that the output aligned with the objective of each nudge. Following White et al.~\cite{white2023prompt}, we used a structured prompt template: defining the task, adding constraints, and setting clear expectations. This led us to prompt GPT with, ``\textit{Create ten distinct personas representing different perspectives on the topic. Provide the name, age, and occupation for each persona}.'' Specific constraints were also established: ``\textit{Create three male and three female personas}'' to mitigate gender bias, as prior research has shown that LLMs such as GPT-3 can reinforce gender stereotypes~\cite{brown2020language, lucy2021gender, huang2019reducing, nozza2021honest, johnson2022ghost}.}}

For the textual variants in indirect reflective nudge (storytelling), each persona generated above was then used as input to prompt GPT to create a unique story.

\paragraph{\normalfont{\textbf{Generating image variants for both nudges:}} To generate the image variants, we input GPT's text output into the text-to-image generator, prompting it with, ``\textit{Create a photo-realistic image with realistic textures and lighting of the [persona/story]}.'' For the indirect nudge (storytelling), we prompt the AI to generate individual images representing key moments in the story, which are then combined to create a cohesive visual narrative. A key constraint for images is to exclude any wordings, as AI-generated images often produce distorted or unreadable text~\cite{keyes2023hands}.}

\paragraph{\normalfont{\textbf{Generating video variants for both nudges:}}
For video, similarly, the text output from GPT was input into the text-to-video generator. We prompted it with, ``\textit{Create a photo-realistic video featuring the [persona/story]},'' specifying background details and voice narration. We constrained the model to adhere strictly to the provided script, ensuring that the content remained the same across all modalities.}

\paragraph{\normalfont{\textbf{Generating audio variants for both nudges:}}
For audio, likewise, the text output from GPT was input into the text-to-speech generator. We then manually selected a gender-appropriate voice that matches the gender specified in the text output. Audio prompts were not required for this process as the generator automatically converts the provided text into audio.}

All prompts, along with the rationale behind the constraints and generation tools for each modality, are detailed in Appendix Tables~\ref{tab: prompt engineering} and~\ref{tab: prompt constraint rationale}. The temperature parameter was set at 0.7 to maintain diversity in responses without introducing excessive randomness.
\section{Study 1: Subjective Modality Preferences}

To address \textbf{RQ1: What are the preferred modalities for each of the reflective nudges?} and discern the preferred modalities for each reflective nudges, we conducted a user study followed by semi-structured interviews with participants ($N=20$). The study focused on participants' preferences for specific modalities within each nudge type and examined how these modalities facilitated self-reflection in an online deliberation context. Detailed descriptions of the four modalities and the two reflective nudges are outlined in section \ref{sec: section3}. 

\subsection{Independent and Dependent Variables}
The study employed a 2 Reflective Nudge: \{Direct, Indirect\} $\times$ 4 Modality: \{Text, Image, Video, Audio\} mixed factorial design. \textit{Reflective Nudge} was between-subject, while \textit{Modalities} was within-subject. 
Within each nudge, the system could generate alternative content (variants), based on pre-recorded prompts. These would be the same for every modality.
The sequence of modality presentation and the order of variants within each modality were randomized for each participant to mitigate any potential ordering effects. 
We had one dependent variable, Ranking, which indicates users' preferences and was measured on a scale from 1 to 4 (for each modality). 

Results between nudges are not compared against each other; instead, the focus is on evaluating the effects of the different modalities within each nudge.

\subsection{Covariates}
\label{sec: Covariates}
%Following Yeo et al.~\cite{yeo2024help},
We controlled for four covariates as fixed factors to account for individual differences that could affect participants' engagement with the modalities.
%This inclusion helps ensure that the findings are not confounded by variability in these covariates. 
We included these covariates in the main analyses by using ANCOVA instead of ANOVA to control for their influence.

\paragraph{\normalfont{\textbf{Topic Knowledge and Topic Interest (TK-TI)}} was included to assess participants' familiarity and information access on the discussion topic. Prior research~\cite{zhang2021nudge} demonstrates that reflection interacts with information access to influence perceived issue knowledge. Mayer and Gallini~\cite{mayer1990illustration} also found that individuals with low prior knowledge benefited more from a combination of text and images. %Likewise, Kalyuga et al.~\cite{kalyuga1998levels, kalyuga2000incorporating} found that individuals with limited prior knowledge gained the most from integrated presentations, such as video. 
Whereas Moreno and Mayer~\cite{moreno1999cognitive} found that people with high prior knowledge are more likely to benefit from videos. TK-TI was evaluated through a five-item multiple choice questionnaire and two matrix tables comprising of 7-8 topic-related statements, and is measured by a score between 0 to 36.}
%(see Appendix Figure~\ref{}). Each multiple choice item was scored 1 for demonstrated knowledge or interest, while matrix statements were rated up to 3, with 0 otherwise, yielding a total score ranging from 0 to 36.

\paragraph{\normalfont{\textbf{Self-Reflection and Insight Scale (SRIS) questionnaire}} was included to account for individuals' inherent predisposition to reflect~\cite{grant2002self, silvia2022self}. It is a widely utilized self-reported scale, rooted in theories of meta-cognition and personal development~\cite{grant2001rethinking, grant2003impact}. It consists of 20 questions on a 1–6 Likert scale (1=strongly disagree, 6=strongly agree).} %to evaluate individual's self-reflective capacity and tendency for self-reflection. 

\paragraph{\normalfont{\textbf{Inherent Reflecting Styles.}}} We used the VARK version 8.02\footnote{https://vark-learn.com/the-vark-questionnaire/}, consisting of 16 question items. Scores for each modality category were subsequently computed to assess whether individuals have a clear preference for one modality over the others. 

\paragraph{\normalfont{\textbf{External Exposure and Interactions.}}} While reflecting styles account for individuals' inherent preferences, external exposure and interactions capture individuals' habitual engagement with various modalities across media platforms such as news articles and blogs for text; Pinterest and social media posts for images; TikTok, Instagram Reels, and YouTube Shorts for video; and podcasts for audio. Regular exposure to specific media formats may lead to stronger affinity for and preference towards those modalities~\cite{knobloch2014choice}. Participants reported their frequency of interaction with these different media platforms on a 1-5 Likert scale (1=Never, 2=Rarely (less than once a week), 3=Occasionally (1-3 times a week), 4=Frequently (4-6 times a week), 5=Daily).

\subsection{Apparatus}
The study was conducted remotely using Zoom. Participants were instructed to access a simulated online deliberation environment and to share their screen. This environment was based on an interface similar to Reddit~\cite{horne2017identifying, medvedev2019anatomy}, to leverage a familiar interface.

\subsubsection{Implementation}
The prototype was developed using Figma\footnote{https://www.figma.com/} before its deployment on Useberry\footnote{https://www.useberry.com/}, a user testing site. We developed the feature - \textit{Reflect} - which displays the four modalities of the reflective nudge.

\subsubsection{Key Features of the Prototype}
To facilitate users' self-reflection during the writing process on a discussion topic, we designed several key features. These are depicted in Figures~\ref{fig: teaser} and \ref{fig: features}.

\begin{figure*}[!htbp]
  \centering
  \includegraphics[width=\textwidth]{figures/Features.png}
  \caption{Key features in interface. \textbf{1:} Users can click on the \textit{Recreate} button to browse through different variants under a specific modality or \textbf{2:} click on the \textit{Exit Reflect} button to navigate back to the different array of modalities to choose another modality. Additional features are present for storytelling to allow users to \textbf{3a:} track their reading process and \textbf{3b:} continue with their reading on the current story. \textbf{4:} For audio and video formats in both direct and indirect reflective nudges, the primary difference is duration. In direct reflective nudges, audio and video average 9 to 12 seconds, while in indirect nudges, the duration extends to 90 to 120 seconds, reflecting typical short-form media content.}
  \label{fig: features}
  \Description{The user interface created for the study encompasses the following features in the comment box: 1: Users can click on the Recreate button to browse through different variants under a specific modality or 2: click on the Exit Reflect button to navigate back to the different array of modalities to choose another modality. When clicking on the storytelling reflector, additional features are present: 3a: The numbers, for example, page 1 out of 5 allows users to track their reading process. 3b: The forward arrow allows users to click and continue with their reading on the current story.}
\end{figure*}

\paragraph{\normalfont{\textit{Reflect}} is a feature that can be integrated into online discussion platforms to facilitate users in their self-reflection process when they craft their opinions on a discussion topic. Clicking on \textit{Reflect} displays the four distinct modalities. Users can choose one to explore, allowing for a deeper engagement within that modality’s context. These are depicted in Figure~\ref{fig: teaser} (3a and 3b).} 

\paragraph{\normalfont{\textit{Recreate}} generates variants tailored to that modality (Figure~\ref{fig: features} (1)). To avoid undue one-sidedness in the generated content, we ensured a balanced presentation by alternating variants between male and female perspectives, as well as between positive and negative tones for both nudges.}

\paragraph{\normalfont{\textit{Exit Reflect}} will return to the selection screen displaying all available modalities (Figure~\ref{fig: features} (2)).}

\paragraph{Features Specific to the Indirect Reflective Nudge}
While engaging with a story, users can track their progress using the page tracker depicted in Figure~\ref{fig: features} (3a). This tracker displays the number of remaining pages until the story concludes. Stories vary in length: four (short), six (medium) or eight (long) pages. Clicking the \textbf{\huge{>}} button (Figure~\ref{fig: features} (3b)) allows users to continue reading. In direct reflective nudges, audio and video have a duration of 9 to 12 seconds, while in indirect nudges, it extends to 90 to 120 seconds, reflecting typical short-form media content.

\subsection{Participants and Ethics}
Following approval from our Institutional Review Board (IRB), we recruited 20 participants (8 males and 12 females), with 10 participants assigned to each type of reflective nudge, aligning with local sample size guidelines~\cite{caine2016local}. The participants had an average age of 24.0 years ($SD=3.56$). Notably, remote interviews typically have a mean sample size of 15 ($SD=6$), and CHI publications commonly feature 12 participants~\cite{caine2016local}. All participants were university students, with detailed demographic information for each reflective nudge provided in Appendix Table~\ref{tab: st1-demo}. Participants were compensated at the appropriate rate dictated by our local IRB.

\subsection{Task and Material}
\label{sec:maintask}
We chose the discussion topic “Is human activity primarily responsible for global climate change?” from ProCon.org\footnote{https://www.procon.org/} as a task for its accessibility and relevance, thereby promoting constructive and open debate.

In the main task, participants used our prototype with the \textit{Reflect} feature. \textbf{Participants had to write at least 30 words with the help of the four modalities when expressing their perspective on the discussion topic.} Participants engaged with all four modalities sequentially (as the study was conducted over Zoom, the researcher instructed and guaranteed this implementation). 

We wanted to capture an organic interaction with the system, therefore did not guide when to use \textit{Reflect}. Participants could concurrently write while using it, or use it sequentially before or after writing. Additionally, \textbf{participants were not obliged to explore every variant within a modality, though they had the option to do so if desired.}

\subsection{Procedure}
\label{sec: procedure}
In addition to the main study (see section~\ref{sec:maintask}), we included a pre- and post-task.

\subsubsection{Pre-Task: Consent, Instructions, Demographic, Questionnaire for Covariates}
Participation consent was obtained before the study. Participants were informed that they had to express their opinions on a discussion topic, which was undisclosed at this point, using various modalities. Participants were then randomly assigned to one of two groups: direct or indirect reflective nudges. Following this, they completed a demographic questionnaire.

Before starting, we administered four questionnaires corresponding to each of the four covariates, as detailed in section~\ref{sec: Covariates}. The values of the covariates are summarized in Appendix Table~\ref{tab: st1-demo}, with no significant differences observed between the two participant groups for any covariate.

\subsubsection{Post-Task Interview}
Following the completion of the main task, we conducted semi-structured interviews to collect feedback. Participants were asked to rank the four modalities on a scale from 1 (lowest) to 4 (highest) based on their personal preference and the level of self-reflection each modality elicited. They provided explanations for their rankings and discussed specific challenges encountered with each modality, as well as the overall usefulness of each modality in their reflective process. All interviews were audio recorded and transcribed for subsequent analysis.

Participants took an average of 46.8 minutes to complete the study.

\subsection{Findings - Ranking}
\begin{figure*}[!htbp]
  \centering
  \includegraphics[width=.8\textwidth]{figures/Ranking.png}
  \caption{Mean ranking for the four modalities for the direct reflective nudge (left) and indirect reflective nudge (right) with 1 being the lowest rank and 4 being the highest rank. We report the results of the ANCOVA test and pairwise comparisons with BH correction, where * : p < .05, ** : p < .01.}
  \label{fig: ranking}
  \Description{Box plot describing the mean rankings for the four modalities for direct reflective nudge (left) and indirect reflective nudge (right). The X-axis shows the different modalities and the Y-axis shows the ranking value ranging from 0 to 4 at a rank interval of 1. 1 is the lowest possible rank and 4 is the highest possible rank. The highest mean ranking was found in text for direct reflective nudge at 3.40, while the highest mean ranking was found in video for indirect reflective nudge at 3.60.}
\end{figure*}

Figure~\ref{fig: ranking} shows the average ranking for each modality of the reflective nudges. 

\subsubsection{Direct Reflective Nudge (Persona)}
\emph{Text} received the highest average ranking of 3.4 out of 4, while \emph{Audio} received the lowest average ranking of 1.7 out of 4. A one-way repeated measures ANCOVA was conducted to determine a statistically significant difference between Modalities on Ranking. We found a statistically significant main effect of Modalities on Ranking ($F_{3,26} = 4.13$, $p<.05$). Specifically, \emph{Text} was significantly ranked higher compared to \emph{Audio} ($p<.01$), \emph{Image} (2.1 out of 4) and \emph{Video} (2.3 out of 4) (both $p<.05$).

\subsubsection{Indirect Reflective Nudge (Storytelling)} 
\emph{Video} received the highest average ranking of 3.6 out of 4, while \emph{Image} received the lowest average ranking of 2.0 out of 4. A one-way repeated measures ANCOVA revealed a statistically significant main effect of Modalities on Ranking ($F_{3,26} = 4.59$, $p<.01$). Specifically, \emph{Video} was ranked significantly higher than \emph{Audio} (2.2 out of 4) and \emph{Image} (both $p<.01$), and \emph{Text} (2.4 out of 4) ($p<.05$).

\paragraph{\normalfont{In summary, \emph{Text} was the most preferred modality for direct reflective nudges, while \emph{Video} was most favored for indirect reflective nudges.}}

\subsection{Findings - Subjective Feedback}
\label{sec: qualitative for study 1}
We present a detailed account of participants' experiences with each modality for each reflective nudge type, highlighting both benefits and challenges. Feedback is categorized using Kahneman’s dual-system thinking model~\cite{kahneman2002maps} (see Table~\ref{tab: dual system thinking}) to illustrate that the same modalities can yield varying or even contradictory results depending on the type of reflective nudge applied. Additional themes are also derived from participants' responses. The notation (/10) indicates the count of participants who shared similar observations.

\subsubsection{Speed}

\paragraph{Direct Reflective Nudge (Persona).} Participants valued the \textbf{text} modality for its conciseness and efficiency in quickly conveying different perspectives (8/10). ``\textit{Text is quick} (P05), \textit{fastest to read through} (P02, P05, P07) and that the \textit{speed of acquiring information from text is the fastest} (P01).'' Similarly, participants found \textbf{image} effective for quickly conveying perspectives and emotions (5/10):  ``\textit{The images included all the necessary elements to clearly make the point, allowing me to understand the perspective immediately} (P09).'' Notably, for participants whose \textbf{first language is not English}, images were particularly beneficial. ``\textit{Image triggers my reflection faster due to my language barrier. Since English is not my first language, I can quickly understand the content and perspectives from an image using my own intuition, whereas text requires more effort to read and comprehend} (P03).'' Figure~\ref{fig: speed} shows a summary of participant preferences for each modality in terms of speed.

%In contrast, participants found \textbf{video} (6/10) and \textbf{audio} (5/10) to be less efficient and slower in conveying perspectives. Both modalities were seen as time-consuming with participants noting, ``\textit{Due to the video's duration, I had to watch it from start to finish to grasp its content}'' (P06). Similarly, audio was perceived as slow, ``\textit{Audio takes time to listen to, depending on its speed}'' (P01). 
%Additionally, the absence of visual elements in audio led to increased time spent, ``\textit{Without visual inputs, I had to spend more time to understand the perspectives conveyed by the audio}'' (P03, P05). 

%\subsubsection{Direct Reflective Nudge (Persona) - Speed}
%Participants valued the \textbf{text} modality for its efficiency in quickly conveying different perspectives (8/10). ``\textit{Text is quick} (P05), \textit{fastest to read through} (P02, P05, P07) and that the \textit{speed of acquiring information from text is the fastest}'' (P01). Participants also noted that the brevity of the text contributed to its effectiveness in conveying different perspectives swiftly (4/10). This brevity allowed them to quickly understand different perspectives and engage in self-reflection (P04). 

%Similarly, participants found \textbf{image} effective for quickly conveying perspectives and emotions (5/10), albeit not as efficient as text. ``\textit{Images quickly convey messages and emotions with minimal time}'' (P10). 
%Another mentioned, ``\textit{The images included all the necessary elements to clearly make the point, allowing me to understand the perspective immediately}'' (P09). Notably, for participants whose \textbf{first language is not English}, images were particularly beneficial. ``\textit{Image triggers my reflection faster due to my language barrier. Since English is not my first language, I can quickly understand the content and perspectives from an image using my own intuition, whereas text requires more effort to read and comprehend}'' (P03). In the same vein, another participant with similar language background added, ``\textit{Reading text requires more effort to understand its meaning. In contrast, an image is worth a thousand words - they help me visualize and grasp content much faster}'' (P06).

\paragraph{Indirect Reflective Nudge (Storytelling).} 
Feedback on the speed of conveying perspectives for indirect reflective nudge differs notably from direct reflective nudge: \textbf{video} was deemed as the most time-efficient and effective as it encompasses text, image and audio (6/10): ``\textit{Video is the most direct for me, as it allows me to quickly understand and process the content in a shorter time period} (P13).''
By contrast, participants had mixed opinions on the efficiency of the \textbf{text} modality for conveying perspectives. Some found text to be quick and efficient (5/10), while others did not share this view (5/10). Those who appreciated text cited their ability to quickly read and process textual information: ``\textit{It is much faster for me to read, so I spend lesser time understanding the contents. In that sense, text is more efficient and helps me to reach the stage of self-reflection faster.}'' While others expressed concerns about its time-consuming nature. P16 said, ``\textit{I wouldn't want to read long chunks of text}.'' % and P20 commented, ``\textit{Reading long chunks of text slows me down and is too time-consuming.}'' P19 added, ``\textit{I need to go through each sentence several times to fully understand the text, making it very time-consuming.}'' Overall, participants who found text inefficient noted that they would engage with it only if they had ample time (P11, P13), highlighting the reliance of text on the user's time availability.
Two participants found \textbf{images} time-consuming because they were positioned above the text, causing them to scan the images before reading the text. ``\textit{I felt like I kept going back and forth — I was reading the text and trying to match it with the image, which take longer for me to understand the content as a whole} (P19).''

Feedback on the \textbf{audio} modality was consistent with direct reflective nudge. No participants described audio as efficient; in fact, some found it slow (4/10): ``\textit{I find the pace of the audio quite slow; the narration drags on} (P12).''
%while another commented, ``\textit{Audio relies on the speed of the narration, so the information is transmitted slowly}'' (P14). Figure~\ref{fig: speed} shows a summary of participant preferences for each modality in terms of speed.

\subsubsection{Depth of Self-Reflection}

\paragraph{Direct Reflective Nudge (Persona)}
Majority of the participants (7/10) found that \textbf{text} facilitated self-reflection more effectively than other modalities for the direct reflective nudge. ``\textit{Text is the most effective medium for reflection because it helps me reconstruct and clarify my thoughts} (P01).'' 
%As one participant noted, ``\textit{Text aids my self-reflection because I retain more information while reading, which helps me remember and think deeply about the topic}” (P05). Another participant mentioned that text offered more details and insights, which supported his thinking process and enhanced self-reflection (P06).
For \textbf{video}, some participants found that while they were not the most efficient in conveying information for direct reflective nudges, they provided more time for self-reflection (3/10). As P01 noted, ``\textit{Although the videos [...] conveyed perspectives slowly, they allowed more time for self-reflection.}'' % P08 added, ``\textit{The video condenses information into short-form content, giving me more time to reflect}''. Similarly, P10 remarked, ``\textit{The slower pacing of the videos compared to text helped me to be more mindful in my comments and facilitated deeper self-reflection}''.
For \textbf{image}, some participants found it beneficial for self-reflection by simplifying complex ideas (3/10). As P10 explained, ``\textit{Images enhance my self-reflection by presenting symbolic meanings that prompt deeper personal interpretation. A powerful image can quickly establish a connection, making it easier for me to engage with and reflect on the topic.}''
%Additionally, images can distill complex ideas into more manageable elements, allowing me to focus on key aspects and think more deeply about the subject}''.
For \textbf{audio}, two participants felt it enhanced self-reflection by simulating a conversational experience, as if interacting with a real person. % ``\textit{Audio facilitates self-reflection because I can hear the speaker’s emotions and tone, helping me to connect with the speaker and making me to think differently about the topic. This would also change how I engage online}'' (P09). 
P02 noted, ``\textit{Audio adds meaning to the text by conveying emotions and expressions that I can’t grasp from reading text alone, helping me to better understand the speaker’s intent.}'' Figure~\ref{fig: speed} shows a summary of participant preferences for each modality in terms of depth of self-reflection.

\paragraph{Indirect Reflective Nudge (Storytelling).} Half of the participants (5/10) found that \textbf{video} enhances self-reflection by making the content more relatable and connecting with their personal experiences: ``\textit{The video portrays daily life well, allowing me to compare it with my own experiences, which prompted deeper reflection} (P20).'' Additionally, participants found that videos aid in clarifying their stance and feelings on issues.
%P14 further added, ``\textit{Videos help me self-reflect by making it easier to empathize with the characters, deepening my reflection}''.: ``\textit{Videos are helpful in figuring out my feelings or stance on an issue when I'm unsure}'' (P12).
In contrast to direct reflective nudge, where text was highly valued for its role in self-reflection, only one participant found \textbf{text} effectively facilitated self-reflection for indirect reflective nudge. %``\textit{The greater autonomy in interacting with the text contributed to enhancing my self-reflection}'' (P17).
For \textbf{image}, a few participants (3/10) noted that the memorable nature of images enhances self-reflection by aiding in content recall. As P19 put it, ``\textit{What helps me self-reflect more is the modality’s ability to make the content memorable. Images stand out for me because I can vividly remember the content, which aids in deeper self-reflection}.''

Compared to direct reflective nudge, a higher proportion of participants found \textbf{audio} effective for self-reflection (4/10), citing similar reasons. Specifically, they noted that the presence of a voice conveyed tone and emotion, aiding their reflection: ``\textit{Hearing the voice helps me understand the character’s emotions and plight better which enhances my self-reflection} (P13).'' 
%Audio also facilitated their thought process: ``\textit{Listening to the audio allowed me to quickly agree or disagree and refine my thoughts as the audio continues to play}'' (P15). 
Additionally, audio enabled participants to immerse themselves in the role of the main character: ``\textit{Audio lets me close my eyes and imagine living as each person, which stimulates my self-reflection} (P11).'' Figure~\ref{fig: speed} shows a summary of participant preferences for each modality in terms of the depth of self-reflection.

\begin{figure*}[!htbp]
  \centering
  \includegraphics[width=.8\textwidth]{figures/Speed.png}
  \caption{Count of participants who shared the same feedback on the depth of self-reflection and speed of delivery for the four modalities of each reflective nudges: direct (persona) - left and indirect (storytelling) - right.}
  \label{fig: speed}
  \Description{Bar graph showing the count of participants who shared the same feedback on the depth of self-reflection and the speed of delivery for the four modalities of each reflective nudges: direct (persona) on the left and indirect (storytelling) on the right. The bar graph shows that text is reflected to be the most time efficient and enhances self-reflection the most for persona. Whereas, video is reflected to be the most time efficient and enhances self-reflection for storytelling.}
\end{figure*}

\subsubsection{Processing (Both Nudges)}
Regardless of the type of reflective nudge, participants from both groups found that \textbf{videos} and \textbf{images} helped them grasp the \textbf{broader context} or overall picture, whereas \textbf{text} allowed for a deeper understanding of \textbf{finer details}. Videos and images provided a higher-level overview and abstraction, making it easier to understand the main idea (P06, P09). Conversely, \textbf{text} was appreciated for its ability to convey detailed concrete information and insights, offering a more comprehensive understanding of the topic (P10, P06).
%For instance, one participant noted, ``\textit{From the image alone, I could grasp the main gist but missed the finer details}'' (P19; Indirect). 

\subsubsection{Effort (Both Nudges)}
Participants from both groups found that \textbf{audio} demanded significant \textbf{concentration, focus and mental effort}, summarized by the following statement: ``\textit{It’s very hard for me to engage with audio alone} (P04).'' %``\textit{Listening while tracking the contents of the audio is challenging}'' (P06; Direct), and ``\textit{I struggled with audio most due to limited visual information}'' (P03; Direct), reflected this difficulty. 
Participants also mentioned needing to stay actively focused on the audio to avoid missing content (P05) and finding it hard to absorb information (P11).
%, thus requiring substantial concentration and effort (P13, P20; Indirect).
\textbf{Video}, on the other hand, was described as facilitating content processing: ``\textit{Video makes it easier for me to process the content} (P13).''

\subsubsection{Nature (Both Nudges)}
Participants noted that \textbf{text} and \textbf{video} are more familiar and accessible due to \textbf{daily exposure}. Text is described as traditional and conventional, making it easier to absorb information (P08, P12). It is considered the most accessible modality (P04) and is frequently used, leading to greater familiarity (P10). Similarly, videos were likened to short-form content found on platforms like TikTok and YouTube Shorts, making them easier to digest as they are familiar with it (P04, P08, P14).
%Participants thus appreciated this similarity to social media content (P19, P18; Indirect).
In contrast, \textbf{audio} was less favored due to its \textbf{limited natural exposure}. Those who did prefer it attributed this preference to their inherent tendency towards auditory learning. Participants in both groups generally expressed disinterest or discomfort with audio, preferring visual stimuli instead. 
%Feedback such as, ``\textit{I am generally not an auditory person}'' (P02; Direct, P19; Indirect) and ``\textit{I tend to doze off with audio, so audio is not for me}'' (P02; Direct) reflected this sentiment. Contrarily, some participants who identified as auditory learners found it easier to absorb information through listening (P11; Indirect).

\subsubsection{Information Retention (Both Nudges)}
Both \textbf{video} and \textbf{audio} exhibit slightly \textbf{lower information retention} compared to the more static modalities of text and images. Participants noted that ``\textit{for both video and audio, I don’t retain as much information compared to text} (P05)'', and that ``\textit{retention rates for recalling specific details from video and audio are lower, as I tend to forget earlier parts while processing new information} (P18).''  Another participant mentioned, ``\textit{videos convey a lot quickly, but I forget the details just as quickly because the information is presented in a short time} (P19).'' In contrast, \textbf{text} and \textbf{images} are described as \textbf{more memorable} (P01, P05, P18, P19).
%, with participants noting, ``\textit{I can recall the contents better because they stick in my head}'' (P20).

\subsubsection{Engagement (Both Nudges)}
\textbf{Text} was generally reported as the \textbf{least engaging}, while \textbf{video} was viewed as \textbf{highly engaging} by most participants in both groups. Videos and \textbf{images} were praised for their \textbf{attention-grabbing} qualities (P01, P13, P14, P20). Videos were particularly noted for being interactive, fun and interesting (P01, P14, P15).
%and were found less boring than plain audio, static images, or text (P01, P02, P09; Direct) The dynamic nature of video scenes also kept viewers engaged (P20; Indirect). 
In contrast, text was described as dry and boring (P01, P15, P18, P20). Participants found it less captivating, especially in an era dominated by multimedia content (P10). %and noted that there is nothing \textit{'special'} about plain text (P09; Direct).

\subsubsection{Users' Autonomy (Both Nudges)}
\textbf{Video} and \textbf{audio} generally offered the \textbf{lowest level of user autonomy}: users noted that they must watch or listen to the entire content without control over the pace (P03, P05). % ``\textit{Videos and audios have fixed playing time so I have less control on how much time I want to spend, but for image and text, they just depends on how fast I can go}'' (P12, P14, P18; Indirect). 
In contrast, \textbf{text} was reported to provide the \textbf{highest level of autonomy}. Participants appreciated that they could adjust their reading speed and review information at their own pace (P01, P03, P16, P17).

\subsubsection{Credibility and Trust in AI-Generated Content (Both Nudges)}
Participants generally reported \textbf{lower credibility} for \textbf{images} compared to audio, video, and text. Some expressed skepticism toward AI-generated images (P04, P14), noting that they appeared unrealistic and less human-like (P01, P08). In contrast, video and audio were seen as more credible due to their human-like elements, such as real human presence in videos and realistic voices in audio (P01).

\subsubsection{Usage Scenarios (Both Nudges)}
\label{sec: usage scenarios}
In general, participants found that \textbf{video} is beneficial for those with little to \textbf{no prior knowledge} of the topic. It helps provide context and explains the issue from multiple angles, making it useful for learning and reflection (P04, P06, P11). For \textbf{images}, they are effective in \textbf{triggering self-reflection for those with some prior knowledge on the topic}. Images are attention-grabbing and can engage users by capturing their focus (P06, P13, P20). 
%``\textit{For an effective reflection, I need something that is more attractive, so I am more likely to engage with images as they capture my attention}'' (P14; Indirect). Images thus serve as a good starting point for self-reflection, helping to kick-start the process (P08, P09; Direct, P11; Indirect). 
\textbf{Text} is primarily used by individuals who already have a \textbf{deep interest or prior knowledge of the topic} (P05, P06). For these users, text provides detailed information that supports reflection without needing additional attention-grabbing elements (P06). Lastly, an overwhelming number of participants mentioned that \textbf{audio} is suited for \textbf{multitasking} and \textbf{on-the-go reflection}, as it allows them to reflect while engaging in other activities.

\subsection{Summary of the Results}
Our results revealed how different modalities of reflective nudges interact with the dual-system thinking framework. Specifically, we observed a clear preference trend: text was favored for direct reflective nudges due to its speed and autonomy, enhancing self-reflection, albeit being less engaging. In contrast, video was preferred for indirect reflective nudges because of its speed, engagement and ability to convey the main idea effectively, despite its lower autonomy. Appendix Figures~\ref{fig: butterfly chart text} - ~\ref{fig: butterfly chart audio} present a butterfly chart summarizing the qualitative feedback for each modality across both nudge types. 
\section{Study 2: Assessing the Impacts of Multimodal Reflection Nudges on Deliberativeness}
In study 1, we identified the preferred modalities for each type of reflective nudge. To further understand how these modalities influence deliberativeness, we conducted study 2 to assess their impact on deliberative quality.

\subsection{Independent Variables and Experimental Design}
We used the same 2 $\times$ 4 design with Reflective Nudge: \{Direct (Persona), Indirect (Storytelling)\} and Modality: \{Text, Image, Video, Audio\}. Both independent variables were between-subject. Thus, participants were randomly assigned to one of the eight experimental conditions. 

The experimental interface retained the same design used in study 1 (Figure~\ref{fig: features}), but for this study, the \textit{Reflect} feature was limited to a single modality. 

Similar to study 1, we do not compare direct and indirect reflective nudges; instead, we evaluate the effects of different modalities within each nudge separately.

\subsection{Dependent Variables}
As deliberativeness is multi-dimensional, we operationalized it through five measurements as done by previous work~\cite{yeo2024help}: argument repertoire, argument diversity, rationality (opinion expression), rationality (justification level) and constructiveness as discussed in section~\ref{sec: measurements}.

All five metrics were derived from a content analysis of participants' responses by two coders. Both coders were PhD students with respectively 1 and 4 years of experience using content analysis. Cohen's Kappa was used to determine the agreement between the two coders’ judgments, with individual scores reported below. Kappa scores for all metrics were above the satisfactory threshold of 0.70~\cite{viera2005understanding, mchugh2012interrater}. 

The dependent variables are coded as follows:
\begin{itemize}
    \item \textit{Argument repertoire} ($\kappa = 0.915$) is the number of non-redundant arguments regarding each position of the discussion topic. The ideas produced along the two positions were combined.
    \item \textit{Argument diversity} ($\kappa = 0.915$) was coded by counting the number of unique themes present in the entire response. A higher diversity count indicates more varied perspectives present in the participant's responses~\cite{anderson2016all, gao2023coaicoder}.   
    \item \textit{Rationality (opinion)} ($\kappa = 0.872$) captures whether opinions are expressed or information is provided in the response. This was coded with two levels: 1) no opinions were expressed, rather, information was provided (score of 0); 2) an opinion, personal assertion or a claim was made (score of 1). This also includes evaluation, a personal judgment or assertion. 
    \item \textit{Rationality (justification level)} ($\kappa = 0.867$) captures the degree to which reasons are used to justify one's claims. This were coded at four levels: 1) no justification was provided (score of 0); 2) an inferior justification was made - this indicates that the opinion is supported with a reason in an associational way such as through personal experiences or an incomplete inference was given (score of 1); 3) a qualified justification was made when there is a single complete inference provided in the opinion (score of 2); 4) a sophisticated justification was made when at least two complete inferences was provided (score of 3).  
    \item \textit{Constructiveness} ($\kappa = 0.830$) captures the degree of balance within an opinion. This was coded at two levels: 1) the opinion is one-sided (score of 0); 2) the opinion is two-sided when multiple perspectives and viewpoints are presented (score of 1). 
\end{itemize}

\subsection{Power Analysis}
We conducted a power calculation for a eight-group ANOVA study seeking a medium effect size (0.30) according to Cohen’s conventions, at 0.80 observed power with an alpha of 0.05, giving $N=25$ per experimental condition, hence we recruited 200 participants. 

\subsection{Participants and Ethics}
A total of 200 participants were recruited through Amazon Mechanical Turk. Refer to Appendix Table~\ref{tab: st2-demo} on the breakdown of the demographic profile in each experimental condition. We ensured that the demographic profiles across the eight conditions were similar so as to control for any fixed effects resulting from the differences in demographic factors. Similar to study 1, we got ethics approval from our local IRB and reimbursed participants at an appropriate rate.

\subsection{Procedure and Task}
The procedure for this study closely follows that of study 1 outlined in section~\ref{sec: procedure}. In the pre-task phase, we gathered data on demographics and administered four questionnaires corresponding to each of the four covariates, as detailed in section~\ref{sec: Covariates}. Instructions on reflection were then provided to the participants.

In the main task, participants were instructed to utilize the \textit{Reflect} feature, which exclusively presents prompts from a single modality. The topic remained the same as study 1. We maintain consistency in the topic across the two studies to draw robust comparisons between different modalities and guarantee the internal validity of our results~\cite{cahit2015internal}.

In the post-task phase, participants completed a short survey in which they could provide feedback on the modality they engaged with.
\section{Results (Study 2)}
 
\subsection{Quantitative Results}
\label{sec: quantitative}
Before analyzing our data, we plotted a box plot (box and whisker plot) to visually show the dispersion of our data and to identify any potential outliers~\cite{schwertman2004simple, dawson2011significant}. No abnormalities in the data were observed. 

ANCOVA was then used to identify main effects while controlling for the four covariates. We applied pairwise t-tests with Benjamini-Hochberg correction for post-hoc comparisons for the eight measures of deliberativeness. We did not use Bonferroni correction, as its conservative approach leads to high rates of false negatives when done with large number of comparisons~\cite{thissen2002quick}. Instead, we relied on Benjamini-Hochberg as it minimizes the problem~\cite{nakagawa2004farewell} while still accounting for multiple comparisons. We report effects of covariates only where they are significant.

Results are summarized in Tables~\ref{tab: summary quantitative direct} and ~\ref{tab: summary quantitative indirect}.

\subsubsection{Argument Repertoire}

\paragraph{Direct Reflective Nudge} We found a significant main effect of \emph{Modality} on \emph{Argument Repertoire} for direct reflective nudge ($F_{3,86} = 4.87$, $p<.01$). There was a statistical difference between \emph{Video} ($M = 3.60$ arguments) and \emph{Text} ($M = 2.32$ arguments), \emph{Image} ($M = 2.56$ arguments) and \emph{Audio} ($M = 2.60$ arguments) (all $p<.05$). %Results are summarized in Figure~\ref{fig: argument repertoire}.

\paragraph{Indirect Reflective Nudge} No significant main effect of \emph{Modality} on \emph{Argument Repertoire} was found for indirect reflective nudge ($p=.09$).
%Results are summarized in Figure~\ref{fig: argument repertoire}.

% \begin{figure*}[!htbp]
%   \centering
%   \includegraphics[width=.8\textwidth]{figures/Argument_Repertoire.png}
%   \caption{Argument Repertoire for direct reflective nudge (left) and indirect reflective nudge (right). We report the results of the ANCOVA test, and pairwise comparisons with BH correction if any, where * : p < .05, ** : p < .01.}
%   \label{fig: argument repertoire}
%   \Description{Box plot describing argument repertoire for the four modalities for direct reflective nudge (left) and indirect reflective nudge (right). The X-axis shows the different modalities and the Y-axis shows the number of arguments.}
% \end{figure*}

\subsubsection{Argument Diversity}

\paragraph{Direct Reflective Nudge} No significant main effect of \emph{Modality} on \emph{Argument Diversity} was found for direct reflective nudge ($p=0.085$). However, individual's inherent reflecting styles, particularly a lower preference for text, was associated with higher argument diversity ($p<.05$). %Results are summarized in Figure~\ref{fig: argument diversity}.

\paragraph{Indirect Reflective Nudge} We found a significant main effect of \emph{Modality} on \emph{Argument Diversity} for indirect reflective nudge ($F_{3,86} = 3.41$, $p<.05$). We observed pairwise differences between \emph{Video} ($M = 5.52$ themes) and \emph{Image} ($M = 3.72$ themes, $p<.05$) only.
%Results are summarized in Figure~\ref{fig: argument diversity}.

% \begin{figure*}[!htbp]
%   \centering
%   \includegraphics[width=.8\textwidth]{figures/Argument_Diversity.png}
%   \caption{Argument Diversity for direct reflective nudge (left) and indirect reflective nudge (right). We report the results of the ANCOVA test, and pairwise comparisons with BH correction if any, where * : p < .05, ** : p < .01.}
%   \label{fig: argument diversity}
%   \Description{Box plot describing argument diversity for the four modalities for direct reflective nudge (left) and indirect reflective nudge (right). The X-axis shows the different modalities and the Y-axis shows the number of argument diversity.}
% \end{figure*}

\subsubsection{Rationality (Opinion)}

\paragraph{Direct Reflective Nudge} No significant main effect of \emph{Modality} on \emph{Opinion} was found ($p=0.224$). However, a higher tendency to self-reflect ($p<.01$), lower inherent preference for audio ($p<.01$) and reduced external exposure to images ($p<.05$) were associated with higher expression of opinions. %Results are summarized in Figure~\ref{fig: opinion}.

\paragraph{Indirect Reflective Nudge} We found a significant main effect of \emph{Modality} on \emph{Opinion} for indirect reflective nudge ($F_{3,86} = 4.61$, $p<.01$). Overall, we found differences between \emph{Text} ($M = 0.48$) versus \emph{Image} ($M = 0.76$), \emph{Video} ($M = 0.84$) and \emph{Audio} ($M = 0.84$) (all $p<.05$). %Results are summarized in Figure~\ref{fig: opinion}.

% \begin{figure*}[!htbp]
%   \centering
%   \includegraphics[width=.8\textwidth]{figures/Opinion.png}
%   \caption{Rationality (Opinion) for direct reflective nudge (left) and indirect reflective nudge (right). We report the results of the ANCOVA test, and pairwise comparisons with BH correction if any, where * : p < .05, ** : p < .01.}
%   \label{fig: opinion}
%   \Description{Box plot describing opinion expression for the four modalities for direct reflective nudge (left) and indirect reflective nudge (right). The X-axis shows the different modalities and the Y-axis shows the level of opinion expression.}
% \end{figure*}

\subsubsection{Rationality (Justification Level)}

\paragraph{Direct Reflective Nudge} No significant main effect of \emph{Modality} on \emph{Justification Level} was observed ($p = 0.571$). However, higher inherent aural preferences and lower inherent text preferences as well as greater external exposure to videos (all $p<.05$) are associated with higher levels of justification. % Results are summarized in Figure~\ref{fig: justification level}.

\paragraph{Indirect Reflective Nudge} We found a significant main effect of \emph{Modality} on \emph{Justification Level} for indirect reflective nudge ($F_{3,86} = 3.39$, $p<.05$). There was a statistical difference between \emph{Video} ($M = 2.32$) and \emph{Image} ($M = 1.44$, $p<.05$).
% Results are summarized in Figure~\ref{fig: justification level}.

% \begin{figure*}[!htbp]
%   \centering
%   \includegraphics[width=.8\textwidth]{figures/Justification_Level.png}
%   \caption{Rationality (Justification Level) for direct reflective nudge (left) and indirect reflective nudge (right). We report the results of the ANCOVA test, and pairwise comparisons with BH correction if any, where * : p < .05, ** : p < .01.}
%   \label{fig: justification level}
%   \Description{Box plot describing justification level for the four modalities for direct reflective nudge (left) and indirect reflective nudge (right). The X-axis shows the different modalities and the Y-axis shows the level of justification.}
% \end{figure*}

\subsubsection{Constructiveness}

\paragraph{Direct Reflective Nudge} No significant main effect of \emph{Modality} on \emph{Constructiveness} was observed ($p = 0.126$). However, a higher inherent \emph{aural} ($p<.01$) and \emph{image} ($p<.05$) preferences as well as reduced external exposure to text ($p<.05$) are associated with higher constructiveness.
%Results are summarized in Figure~\ref{fig: constructiveness}.

\paragraph{Indirect Reflective Nudge} No significant main effect of \emph{Modality} on \emph{Constructiveness} was observed ($p = 0.169$). 
%Results are summarized in Figure~\ref{fig: constructiveness}.

% \begin{figure*}[!htbp]
%   \centering
%   \includegraphics[width=.8\textwidth]{figures/Constructiveness.png}
%   \caption{Rationality (Justification Type) for direct reflective nudge (left) and indirect reflective nudge (right). Error bars show .95 confidence intervals. We report the results of the ANCOVA test, and pairwise comparisons with BH correction if any, where * : p < .05, ** : p < .01.}
%   \label{fig: constructiveness}
%   \Description{}
% \end{figure*}

\begin{table*}[!htbp]
\caption{Deliberative quality across the four modalities for direct reflective nudge (Persona). $\alpha$,$\beta$,$\gamma$ show significant pairwise differences. n.s.: not significant, *: $p<.05$, **: $p<.01$, etc...}
\label{tab: summary quantitative direct}
\begin{tabular}{cccccc}
\toprule
\multirow{3}{*}{\textbf{Deliberativeness}} & \multirow{3}{*}{\textbf{$p$}} & \multicolumn{4}{c}{\textbf{Modality}} \\ \cline{3-6}
 & & \textbf{\begin{tabular}[c]{@{}c@{}}Text\\      ($M \pm S.D.$)\end{tabular}} & \textbf{\begin{tabular}[c]{@{}c@{}}Image\\      ($M \pm S.D.$)\end{tabular}} & \textbf{\begin{tabular}[c]{@{}c@{}}Video\\      ($M \pm S.D.$)\end{tabular}} & \textbf{\begin{tabular}[c]{@{}c@{}}Audio\\      ($M \pm S.D.$)\end{tabular}} \\
\midrule
Argument Repertoire & ** & 2.32 ± 1.07$^\alpha$ & 2.56 ± 1.00$^\beta$ & 3.60 ± 1.94$^\alpha$$^\beta$$^\gamma$ & 2.60 ± 1.41$^\gamma$ \\
Argument Diversity & n.s. & 4.08 ± 1.68 & 5.08 ± 1.75 & 5.48 ± 2.63 & 4.64 ± 1.80 \\
Rationality (Opinion Expression) & n.s. & 0.64 ± 0.49 & 0.84 ± 0.37 & 0.60 ± 0.50 & 0.72 ± 0.46 \\
Rationality (Justification Level) & n.s. & 2.20 ± 0.91 & 2.08 ± 0.81 & 2.20 ± 0.82 & 1.92 ± 0.86 \\
Constructiveness & n.s.& 0.28 ± 0.46 & 0.20 ± 0.41 & 0.48 ± 0.51 & 0.36 ± 0.49 \\
\bottomrule
\end{tabular}
\Description{A table summarizing the deliberative quality for direct reflective nudge across the four modalities and across the dependent variables. The table is a summary of the other Figures in the section.}
\end{table*}

\begin{table*}[!htbp]
\caption{Deliberative quality across the four modalities for indirect reflective nudge (Storytelling). $\alpha$,$\beta$,$\gamma$ show significant pairwise differences. n.s.: not significant, *: $p<.05$, **: $p<.01$, etc...}
\label{tab: summary quantitative indirect}
\begin{tabular}{cccccc}
\toprule
\multirow{3}{*}{\textbf{Deliberativeness}} & \multirow{3}{*}{\textbf{$p$}} & \multicolumn{4}{c}{\textbf{Modality}} \\ \cline{3-6}
  & & \textbf{\begin{tabular}[c]{@{}c@{}}Text\\      ($M \pm S.D.$)\end{tabular}} & \textbf{\begin{tabular}[c]{@{}c@{}}Image\\      ($M \pm S.D.$)\end{tabular}} & \textbf{\begin{tabular}[c]{@{}c@{}}Video\\      ($M \pm S.D.$)\end{tabular}} & \textbf{\begin{tabular}[c]{@{}c@{}}Audio\\      ($M \pm S.D.$)\end{tabular}} \\
\midrule
Argument Repertoire & n.s. & 2.76 ± 1.13 & 2.60 ± 1.26 & 2.88 ± 1.96 & 2.84 ± 1.28 \\
Argument Diversity & * & 4.24 ± 2.07 & 3.72 ± 2.15$^\alpha$ & 5.52 ± 2.40$^\alpha$ & 4.92 ± 1.82 \\
Rationality (Opinion Expression) & ** & 0.48 ± 0.51$^\alpha$$^\beta$$^\gamma$ & 0.76 ± 0.44$^\alpha$ & 0.84 ± 0.37$^\beta$ & 0.84 ± 0.37$^\gamma$ \\
Rationality (Justification Level) & * & 1.84 ± 1.11 & 1.44 ± 1.08$^\alpha$ & 2.32 ± 0.85$^\alpha$ & 2.08 ± 1.08 \\
Constructiveness & n.s. & 0.16 ± 0.37 & 0.24 ± 0.44 & 0.32 ± 0.48 & 0.44 ± 0.51 \\
\bottomrule
\end{tabular}
\Description{A table summarizing the deliberative quality for indirect reflective nudge across the four modalities and across the dependent variables. The table is a summary of the other Figures in the section.}
\end{table*}

\subsubsection{Summary of Results}
Overall deliberative quality across the five measurements is summarized in Tables~\ref{tab: summary quantitative direct} and \ref{tab: summary quantitative indirect} for direct and indirect reflective nudges, respectively. A detailed analysis reveals that \emph{Video} generally emerges as the most effective modality for the direct reflective nudge (persona), particularly excelling in Argument Repertoire. For the indirect reflective nudge (storytelling), similarly, \emph{Video} leads in Opinion Expression, Argument Diversity, and Justification Level, often outperforming \emph{Image} significantly. This suggests that video enhances user engagement and expression, thereby improving deliberativeness. In contrast, \emph{Text} is least effective for Opinion Expression in the context of indirect reflective nudges (storytelling), suggesting that lengthy text may not engage users as effectively in articulating their opinions.


\subsection{Subjective Feedback}
\label{sec: qualitative}
We present the findings from the thematic analysis of the post-task feedback, where numbers in parentheses indicate the frequency of recurring feedback across both nudge types. Notably, no specific dislikes were reported for text or images in either reflective nudge type.

\subsubsection{General Strengths of Reflect}
Participants demonstrated strong appreciation for the system across all modalities in both reflective nudges. They appreciated its thought-provoking nature, which facilitated critical thinking and led to a deeper understanding of their own feelings and thoughts on the issue (40). It helped clarify and refine their perspectives (10), encourage deep self-evaluation, foster self-awareness (21), and prompt reflection on personal values and the impact of their opinions (8). 
%Additionally, by connecting to their own experiences, the system resonated with participants, making the content more relatable (15), while also reinforcing their understanding of the topic (21). Participants also mentioned that they felt emotionally supported as they were able to explore complex emotions and moral dilemmas (14). 
With its ability to present multiple viewpoints and perspectives that might have otherwise been overlooked (29), participants found the system both interesting and useful (34). 
%Its ease of use, clear and intuitive design, and simplicity made it highly accessible (8). 

\subsubsection{Strengths of each Modality in Direct Reflective Nudge}
Participant appreciated that \textbf{text} was straightforward and concise (P12). For \textbf{images}, participants noted that they helped break down complex information, making it easier to grasp, retain and connect with the content (P26).

Participants found \textbf{videos} particularly engaging (P51, P57) due to their multi-sensory approach of combining visual and auditory elements (P51, P68, P70, P74). This blend elicited deeper emotional and cognitive responses (P54, P55, P57): ``\textit{Videos fostered both emotional and intellectual engagement by leveraging storytelling, visuals, and music, making it more memorable and impactful} (P68).''
%Additionally, videos offered vivid, real-world examples that made abstract concepts more tangible and relatable (P74).

For \textbf{audio}, participants appreciated that the voice sets a calm, reflective atmosphere for them to engage deeply with their thoughts and emotions (P77), enhancing the reflective experience (P78, P84).
%, with the voice evoking emotions and memories, facilitating a deeper connection to the reflection process (P82, P83). 
%The immersive nature of audio also helps participants focus inwardly without distractions, allowing them to fully engage with their thoughts and feelings (P85, P89, P91, P95). 
Additionally, some participants preferred audio over reading, noting that it kept them more engaged: ``\textit{I liked that I did not have to read anything. Listening to the audio gave me more motivation to explore multiple opinions, whereas with text, I would have only read one or two [chunks] at most} (P94).''

\subsubsection{Strengths of each Modality in Indirect Reflective Nudge}
Participants appreciated the \textbf{text} modality for its conciseness and ease of engagement (P111). For \textbf{images}, participants noted the high quality of the AI-generated visuals, describing them as ``well done'' (P126), and found that they stimulated their imagination (P144). For \textbf{both text and images}, participants found the well-developed stories particularly memorable (P124), with images resonating deeply and prompting significant reflection (P134). 
%In both modalities, the relatable journeys and dilemmas faced by the characters encouraged participants to reflect on their own values and decisions, as well as how these experiences influenced their personal growth (P114, P116, P117, P118, P142).

For \textbf{videos}, participants provided similar feedback as with the direct reflective nudge. They found videos particularly engaging (P152) due to the multi-sensory combination of visual and auditory elements, which helped them better articulate their thoughts and feelings (P152). 
Similarly, for \textbf{audio}, participants shared feedback consistent with the direct reflective nudge. Fourteen participants reported they liked the calm voice which fostered more profound reflection by reducing distractions.
%This blend also allowed participants to process complex ideas more effectively (P152, P169). The videos also made the scenarios more relatable and vivid, offering real-world examples (P154, P160, P170, P171, P172, P174), which helped participants connect emotionally and intellectually with the thought-provoking content (P154, P159, P167, P170, P173, P174).
%For  Again, (P176, P182, P197, P198) and enhancing emotional connection and engagement with the content (P176, P178). Additionally, P192 noted that ``\textit{the voice added a dynamic and reflective quality to the topic, infusing it with personality and depth, enriching my contemplation of the subject}''.

\subsubsection{Critiques of each Modality in Direct Reflective Nudge}
For \textbf{video}, one participant expressed a dislike for computer-generated narration (P52). Additionally, two participants mentioned that the lack of interactivity in the videos led to passive consumption of information rather than active engagement which may limit the depth of reflection (P70, P73).

For \textbf{audio}, four participants expressed that it was not for them: ``\textit{Audio can certainly aid in reflection, but it might not always hit the mark for every individual} (P84).'' % and ``\textit{It may not always suit everyone’s reflective needs or styles}'' (P76). 
%This underscores the importance of finding a reflective practice that aligns with individual reflective preferences and needs (P80, P92). 
Additionally, three participants found the audio voices too robotic, noting a desire for a more human-like tone to enhance credibility (P79, P83, P94).

\subsubsection{Critiques of each Modality in Indirect Reflective Nudge}
%For \textbf{video}, one participant felt that some videos felt overly condensed, making it difficult to fully absorb the information and engage in meaningful reflection (P152). 

As for \textbf{audio}, two participants found that it became less engaging over time, suggesting that subtle changes or more variety were needed for continued engagement: ``\textit{After a while, it became less engaging, and I found my mind drifting away from the reflective process. A bit more variety or subtle shifts in the audio could have kept my focus and enhanced the overall reflective experience} (P180).''
\section{Discussion}
In this section, we address \textbf{RQ2: How does the modality of a reflective nudge affect the quality of deliberation?} by synthesizing quantitative data from our objective measures (see section~\ref{sec: quantitative}), qualitative feedback from participants (see section~\ref{sec: qualitative}) and insights from study 1 to provide a comprehensive view of how different modalities within different reflective nudges impact deliberativeness. We also examine how our findings corroborate and enrich previous research, discussing the broader implications of enhancing deliberativeness on online deliberation platforms.

\subsection{Tailoring Modalities to Suit Different Types of Reflective Nudges}
Our results from study 1 demonstrate that text is most preferred for direct reflective nudges, while video is favored for indirect reflective nudges. Study 2's quantitative analysis confirms that video significantly enhances deliberativeness for indirect reflective nudges, increasing argument diversity, opinion expression, and justification levels. Conversely, text performs the worst for opinion expression, highlighting that the same modality on different nudge types significantly impacts deliberative quality.

This goes to show that text-based reflective nudges are most effective when they are short and straightforward, aligning with earlier work that lengthy textual information can cause cognitive overload and reduce engagement~\cite{sweller1988cognitive}. For complex reflective tasks, video offers a better alternative, suggesting that \textbf{indirect reflective nudges benefit more from direct modalities} like video, which maintain engagement and convey ideas effectively.

The preference for video in indirect reflective nudges can be attributed to its capacity of multi-sensory engagement, allowing it to deliver complex concepts more directly and engagingly~\cite{clark2023learning} as highlighted by the qualitative feedback in both studies (sections~\ref{sec: qualitative for study 1} and \ref{sec: qualitative}), supporting Mayer’s findings~\cite{mayer2005cambridge} that video outperforms images and text in learning contexts. Additionally, Dale’s~\cite{dale1969audiovisual} cone of learning highlights the inherent concreteness of videos compared to more abstract modalities like pictures. This concreteness likely contributes to their effectiveness in promoting reflection by providing richer, more relatable stimuli compared to images or text. Our study extends these findings to reflection in online deliberation, showing that video enhances deliberativeness more effectively than text alone, echoing the multimedia principle~\cite{fletcher2005multimedia}.

Overall, our findings suggest that as the complexity of reflective nudge increases, different modalities may be needed to enhance the reflection process. While text works for concise reflective nudge, richer modalities like video should be considered for more complex nudges. These insights suggest that \textbf{the modality chosen should be carefully aligned with the type of nudge being delivered}, challenging the conventional reliance on text-based approaches and advocating for a multi-modal strategy to better support deliberativeness.

\subsection{Supporting Deliberation with Modalities}

\subsubsection{Importance of Multimodality in Reflection}
Results from studies 1 and 2 revealed that all participants exhibited multimodal preferences, indicating that none had a dominant single preference. This aligns with existing research and theoretical frameworks~\cite{mayer2002multimedia, fleming1992not}, which showed that only a small minority of individuals have single-modal preferences, while the majority are multimodal, spanning bimodal, trimodal, or four-part preferences~\cite{mayer2005cognitive, varklearnVARKResearchWhat}. Notably, the most common is the four-part preference, with 25.4\% of individuals categorized as ``Integrative Multimodal''~\cite{varklearnVARKResearchWhat}. These individuals seek input across all modalities before integrating insights from multiple sources for a more comprehensive grasp of the material. 

Therefore, it is likely that individuals may engage with and benefit from multiple sensory channels for more effective information processing. Studies in educational psychology~\cite{mayer2005cambridge, mayer2002multimedia, fleming1992not, clark2023learning} emphasize the cognitive benefits of multi-sensory engagement, supporting our findings that multimodal approaches can significantly enhance user reflection. By \textbf{leveraging users' multimodal preferences as a catalyst for reflection}, online deliberation platforms can tailor nudges to individual needs more effectively. Integrating multiple modalities into reflective nudges not only caters to diverse user preferences but also leads to more meaningful and impactful deliberation. Ultimately, multimodal approaches ensure that reflective nudges resonate with a broader audience, thereby improving deliberativeness.

\subsubsection{Pairing Modalities - Leveraging the Advantages of Different Modalities for Reflection}
Modalities serve both \textit{interpretative} and \textit{reflective} support, helping users construct opinions while enhancing self-reflection~\cite{reid2003supporting}. Each modality has unique strengths, as evidenced by our qualitative findings from study 1. For instance, participants whose first language was not English, preferred images over text, highlighting the universal accessibility of images and its ability to quickly convey ideas and emotions without language barriers. This underscores the importance of leveraging different modalities not only to enhance inclusivity when language and cultures differs but also to take advantage of their distinct benefits.

Research on temporal contiguity~\cite{moreno1999cognitive} shows that combining narration with animation yields better performance compared to using a single channel, particularly for cognitive tasks~\cite{lee1997effect, paivio2013imagery}. Similarly, Nathan et al.~\cite{nathan1992theory} found pairing verbal and non-verbal materials outperforms narration alone. In our study, participants noted that audio alone, while conversational, was less effective for reflection, lacking sufficient capacity for dual coding~\cite{paivio1975free} and semantic processing (see section~\ref{sec: learning and reflection}). Pairing audio with visuals leverages the \textbf{additive effect}, where combinations like audio with images or text create richer and more effective reflective experiences, amplifying semantic processing and providing a stronger foundation for reflection.

Our findings (see section~\ref{sec: qualitative for study 1}) also illuminate the interaction between modalities and the dual-system thinking (see section~\ref{sec: dual system thinking}). Text and images aligned with System 1, facilitating rapid information delivery, especially for direct reflective nudges. Videos, offering a multi-sensory experience, balanced System 1’s speed with System 2’s reflective depth, particularly for indirect reflective nudges. Parallel processing (System 1) was supported by visual modalities like images and videos, which provided an overarching view of the topic. Serial processing (System 2), tied to detailed and step-by-step analysis, was best facilitated by text, aiding deeper reflective engagement.

By understanding how modalities interact with System 1 and System 2, platforms can design reflective nudges that optimize both intuitive and analytical engagement. Combining modalities strategically can also amplify their strengths, promote inclusivity, and enhance the depth of reflection. Future studies may want to explore combining different modalities to achieve optimal outcomes tailored to specific deliberative environments, enhancing the quality and depth of online discussions.

\subsubsection{Using Modalities to Kick-start Reflection}
Participants highlighted that images effectively initiated their reflection (see section~\ref{sec: usage scenarios}). This aligns with research showing that visual elements, such as images and videos, activate prior knowledge and trigger reflective thinking~\cite{mayer2005cambridge}. By engaging users visually, these modalities provide a foundation that supports more in-depth reflection, allowing users to anchor their initial reflections, which can then be expanded upon when articulating their opinions. Therefore, integrating images and videos early in the reflection process could facilitate more meaningful deliberation.

\subsection{Consider Individual Differences: The Role of Covariates in Multimodal Preferences}
Reflection is a multifaceted process~\cite{mayer2005cambridge} influenced by various factors beyond just the modality used. Our study highlights that covariates significantly shape deliberativeness (see section~\ref{sec: quantitative}). It is essential to account for these individual characteristics when designing reflective nudges, as they determine how users engage with each modality.

Our results are consistent with Mayer et al.~\cite{mayer2002multimedia, mayer1990illustration}, suggesting that modalities should be considered in combination with individual traits and the specific usage context. For instance, study 1's feedback found that users find audio particularly helpful in multitasking. Additionally, study 2's quantitative findings showed that reflection is shaped by both internal preferences and external media exposure. This interplay between personal tendencies and past experiences suggests that reflection is not a one-size-fits-all process. Future research could further explore the tension between these internal and external influences in shaping users’ multimodal preferences.

To optimize reflective nudges, designers and practitioners should consider specific contexts in which modalities are applied. Our findings underscore the need for systematically investigating how individual characteristics impact deliberativeness. While most research focuses on improving deliberation outcomes, less attention has been given to understanding the factors shaping the reflective process itself. By addressing these factors, future studies can develop more targeted strategies to enhance both reflection and deliberativeness.

\subsection{Potential of Utilizing LLMs to Support Self-Reflection}
Unlike traditional online deliberation platforms, where users form opinions through user-driven self-reflection and the consumption of others' comments, our study demonstrates how LLMs can enhance this process using subtle interface nudges. This positions LLMs not just as writing aids but as facilitators of deeper, reflective thinking, steering users toward more informed, thoughtful decisions. Nevertheless, it's crucial to acknowledge that the effectiveness of LLMs relies on the quality of the responses they generate. While we utilized GPT-4.0 along with text-to-image (Microsoft Bing) and text-to-video generation tools (Invideo AI), future advancements in LLM technology, such as HeyGen AI Video Generator or Luma Dream Machine, could yield varied results depending on their specific design and functionality. These innovations will likely further shape how LLMs contribute to deliberation. 

Moreover, the role of LLM-generated content in fostering reflection, compared to human-created nudges, presents an intriguing area for further exploration. Future studies can explore how users engage with and perceive AI-generated versus human-created reflective nudges. 

Lastly, as with the use of any AI or LLM technologies in any domain, inherent risks such as amplifying societal biases must be carefully managed. Generative technologies must be explicitly designed with bias mitigation strategies to prevent perpetuating stereotypes and biases. This requires ongoing transparency about AI's role, clear communication of its limitations, and proactive steps to address potential ethical considerations. 

\subsection{Expanding Modalities for Enhanced Reflection}
Beyond the current straightforward use of modalities, there is potential to extend their application to other stages of the deliberative process.

\paragraph{Checking content before posting.} Incorporating an audio playback of written content by converting textual content into audio would allow users to listen back and validate their understanding before posting. This auditory review could help users ensure that their message aligns with their intended meaning, offering an additional layer of reflection by \textit{hearing} their own thoughts played aloud. 

\paragraph{Usage on the go.} Similarly, providing a text-to-speech feature for other users' opinions would support ``reflection-on-the-go'', particularly for auditory learners. This addition would cater to participants who reflect better by listening or those who prefer a conversational approach over reading, expanding the inclusivity and accessibility of the deliberative process. However, challenges might arise in ensuring that the text-to-speech translation is accurate, particularly in complex or nuanced opinions, which may affect users' comprehension.% For example, auditory reflective users could benefit from hearing nuanced inflections and tones that might not be apparent in text, enhancing their engagement with the content.

\paragraph{Granularity of the modality.} When designing visual aids, the level of granularity --- whether it is single images, image sequences, or multi-message visuals --- plays a key role for visual learners. Tailoring the structure of visual content to user preferences can significantly influence how effective they engage with and reflect on the content. However, a key challenge is balancing the granularity of visual aids such that they are informative without overwhelming the user. Future studies may want to look into the manipulation of modality considerations such as adjusting the granularity of different modalities to create more adaptive reflective experience that is tailored to diverse reflection styles and contexts.

\subsection{Applicability to other Contexts}
While the task presented in this study was specific to an online deliberation context, the multiplicity of opportunities for fostering reflection in other areas, combined with the social affordances provided by multimedia representations, suggests that these approaches hold potential for broader applications. 

\paragraph{Moderation.} In online discussion platforms (e.g., Reddit and Quora), moderators could fine-tune or adjust reflective nudges based on the tone or quality of the discussion, promoting deeper and more respectful engagement. By combining multimodal nudges with moderation, the system becomes more dynamic and adaptable, where nudges are personalized and contextually relevant. This approach also helps prevent excessive dominance by certain perspectives and fosters diverse viewpoints.

\paragraph{Education.} Multimodal reflective nudges can be adapted to educational settings to encourage deeper student engagement and critical thinking. For instance, visualizations or video explanations of complex topics can aid in simplifying abstract concepts. By offering different ways to engage with learning content, students can reflect on their knowledge from multiple angles, promoting active learning and enhanced retention. 

\paragraph{Counseling.} In counseling, using multimodal approaches may help individuals process their emotions and thoughts more effectively. Audio playback of therapeutic dialogues or reflections can help individuals revisit and internalize key insights, while video-based role-playing scenarios can offer practical, real-world examples on how to manage specific situations. These would allow for more personalized and engaging reflections, leading to deeper self-awareness and therapeutic growth.

Future research endeavors could explore the cross-domain generalizability of these findings, seeking to understand the modalities' boundaries and flexibility in diverse domains.
\section*{Limitations}
The compositional generalization tasks used in this work are based on synthetically generated datasets and so might not represent sufficiently the variety in natural language expressions.
However, these controlled settings are required for precise evaluation requires because all the lexical items and syntactic structures must be split properly into the training and generalization sets.
Therefore, using a natural corpus for this experiment would have required much effort, and we leave that for future work.

Another limitation is that the results of this experiment do not necessarily transfer to larger models because we tested relatively small models trained on a small synthetic dataset following previous studies.
It would be worth exploring how the trends discovered here change as the model size increases.


\section*{Ethical Considerations}
All of our datasets were constructed for the sole purpose of the model analysis from the linguistic perspective.
They contain no potentially harmful or offensive content.
\section{Limitations}
Aiming to be a general and user-friendly benchmark, \spark has several potential limitations slated for future improvements.

In the current version, the safe control library supports first-order safe control, which implicitly assumes the robot can track an arbitrary velocity command immediately. However, as the motors have limited torques, if the velocity goal is generated to be too different from the current velocity, there may be delays in tracking the goal and impacting safety. To mitigate this issue, we will need to support higher-order safe controls in the future. 

Another limitation is that the current implementation does not distinguish between inevitable collisions from method failures (i.e., there are feasible collision-free trajectories but the method could not find one). Method failures happen when multiple safety constraints are in conflict, which does not necessarily imply a collision is inevitable. To mitigate this issue, more research is needed to either improve the safe control methods in handling multiple constraints or introduce advanced methods in detecting inevitable collisions. 

Finally, the sim to real gap also exists in model-based control systems, although it is called differently as ``model mismatch". For the real deployment, the robot trajectory might be different from the simulation due to ``model mismatch". To mitigate this problem, the system model needs to be robustified and the system control needs to be aware of potential gaps, which will be left for future work.

%For future work, we plan to expand the safe control library by incorporating additional safety indices and dynamics, including second-order controllers. Additionally, we aim to advance theoretical research on multi-constraint safe control problems, further enhancing \spark's capabilities and broadening its applications.

%Another key objective is to integrate \spark with other reinforcement learning platforms. This would enable \spark to function as both a safe RL task environment and a model-based safe RL library.

%Finally, we intend to stay aligned with advancements in humanoid hardware and research, ensuring \spark is updated to support emerging safe control strategies and new hardware systems. 

\section{Conclusion and Future Work}
In this paper, we presented \spark, a comprehensive benchmark designed to enhance the safety of humanoid autonomy and teleoperation. We introduced a safe humanoid control framework and detailed the core safe control algorithms upon which \spark is built.

\spark offers configurable trade-offs between safety and performance, allowing it to meet diverse user requirements. Its modular design, coupled with accessible APIs, ensures compatibility with a wide range of tasks, hardware systems, and customization levels. Additionally, \spark includes a simulation environment featuring a variety of humanoid safe control tasks that serve as benchmark baselines. By utilizing \spark, researchers and practitioners can accelerate humanoid robotics development while ensuring robust hardware and environmental safety.


Beyond what was mentioned in the previous section, there are several future directions for \spark that could benefit from community collaboration. 
First of all, it is important to further lower the barrier for users to adopt state-of-the-art safety measures \cite{ji2023safety}, e.g., safe reinforcement learning approaches \cite{zhao2023absolute}\cite{zhao2023learn}\cite{zhao2023state} \cite{yao2024constraint}, by integrating them into \spark algorithm modules. 
Secondly, to enhance the robustness and reliability of the deployment pipeline of \spark, future works are needed to extend \spark compatibility with standardized safety assessment pipelines, such as stress testing tools and formal verification tools \cite{xie2024framework}. 
Thirdly, simplifying task specification is critical for usability. Enhancements to \spark's interface, such as intuitive configuration, natural language-based task definitions \cite{lin2023text2motion}, and interactive visualizations \cite{park2024dexhub}, will enable more efficient safety policy design and debugging. 
Finally, automating the selection and tuning \cite{victoria2021automatic} of safety strategies will make \spark more adaptive. Future efforts may explore automatic model-based synthesis approaches such as meta-control \cite{wei2024metacontrol} techniques to dynamically optimize safety measures based on task requirements and environmental conditions. 
\section{Acknowledgement}

This work is supported by the National Science Foundation under grant No. 2144489.



%%
%% The next two lines define the bibliography style to be used, and
%% the bibliography file.
\bibliographystyle{ACM-Reference-Format}
\bibliography{reference.bib}

\newpage
\section{Appendix}

\begin{figure}
    \centering
    \includegraphics[width=0.9\linewidth]{figures/02_traning_taxonomies.png}
    \caption{Training Taxonomies}
    \label{fig:traning_taxonomies}
\end{figure}

\begin{figure}
    \centering
    \includegraphics[width=1.0\linewidth]{figures/03_preference_tuning_taxonomies.png}
    \caption{Preference Tuning Taxonomies}
    \label{fig:preference_tuning_taxonogy}
\end{figure}

% \begin{table*}[ht]
%     \centering
%     \begin{tabular}{|>{\raggedright\arraybackslash}m{4cm}|>{\raggedright\arraybackslash}m{3cm}|>{\raggedright\arraybackslash}m{7cm}|}
\hline
\textbf{Name} & \textbf{Notation} & \textbf{Description} \\
\hline
Input Sequence & $x$ & Input sequence that is passed to the model. \\
Output Sequence & $y$ & Expected label or output of the model. \\
\hline
Dispreferred Response & $y_l$ & Negative samples for reward model training. \\
Preferred Response & $y_w$ & Positive samples for reward model training. \\
\hline
Optimal Policy Model & $\pi^*$ & Optimal policy model. \\
Policy Model & $\pi_\theta$ & Generative model that takes the input prompt and returns a sequence of output or probability distribution. \\
Reference Policy Model & $\pi_{\text{ref}}$ & Generative model that is used as a reference to ensure the policy model is not deviated significantly. \\
\hline
Preference Dataset & $\mathcal{D}_{\text{pref}}$ & Dataset with a set of preferred and dispreferred responses to train a reward model. \\
SFT Dataset & $\mathcal{D}_{\text{sft}}$ & Dataset with a set of input and label for supervised fine-tuning. \\
\hline
Loss Function & $\mathcal{L}$ & Loss function. \\
Regularization Hyper-parameters & $\alpha, \beta_{\text{reg}}$ & Regularization Hyper-parameters for preference tuning. \\
Reward & $r$ & Reward score. \\
Target Reward Margin & $\gamma$ & The margin separating the winning and losing responses. \\
Variance & $\beta_i$ & Variance (or noise schedule) used in diffusion models. \\
\hline
\end{tabular}
%     \caption{Caption}
%     \label{tab:notation}
% \end{table*}

\end{document}
\endinput
%%
%% End of file `sample-authordraft.tex'.
