\section{Introduction}
\label{sec:intro}


% \begin{figure*}[h]
% \vspace{-1em}
%     \begin{center}
%     \includegraphics[width=1.\textwidth]{figs/mc_intro3.pdf}
%     \end{center}
%     % \vspace{-0.5em}
%     \caption{\model\ distinguishes itself from existing LLM Agent frameworks through its zero-code implementation and pure language-driven framework, eliminating technical barriers while enabling non-technical users to build agents entirely through natural language interactions.}
%     \label{fig:test1}
%     % \vspace{-1.0em}
% \end{figure*}

The emergence of Large Language Models (LLMs) has revolutionized AI agent development, enabling unprecedented breakthroughs in autonomous task execution and intelligent problem-solving. LLM-powered agents excel at understanding context, making informed decisions, and seamlessly integrating with various tools and APIs. Leading frameworks like LangChain~\cite{langchain2023}, AutoGPT~\cite{AutoGPT}, AutoGen~\cite{autogen}, CAMEL~\cite{camel}, and MetaGPT~\cite{hong2024metagpt} have demonstrated remarkable success in automating increasingly complex workflows - from sophisticated web navigation to advanced data analysis and innovative creative content production. By leveraging advanced mechanisms such as role-playing, structured operating procedures, and dynamic agent coordination, these frameworks deliver exceptional problem-solving capabilities while significantly reducing human intervention.

Despite remarkable advancements in AI agent development, a significant barrier persists: the creation and optimization of LLM agent systems remains dependent on traditional programming expertise. Current frameworks primarily cater to technically proficient developers who can navigate complex codebases, understand API integrations, and implement sophisticated prompt engineering patterns. This reliance on coding skills creates a substantial accessibility gap, as only 0.03\% of the global population possesses the necessary programming expertise to effectively build and customize these agents. Even with well-documented frameworks and development tools, the entry barrier remains dauntingly high for non-technical users. This limitation becomes particularly problematic given the universal need for personalized AI assistants in digital age. Everyone, from business professionals seeking workflow automation to educators designing interactive learning tools, requires customized LLM agents tailored to their specific needs. For instance, a researcher might need an agent specialized in literature review and data analysis, while a content creator might require an agent focused on creative writing and media management. The current paradigm of coding-dependent agent development not only severely restricts the user base but also creates a bottleneck in meeting the diverse and evolving demands for personalized AI assistance. This misalignment between universal needs and limited accessibility calls for a fundamental rethinking of how LLM agents are created and customized.

% \footnote{\noindent The name \model\ embodies our framework's core philosophy of ``Agent-Creating Agents''. Similar to how meta-learning enables systems to ``Learn How to Learn'', \model\ represents LLM Agent framework that automates the creation and orchestration of other agents. The ``Meta'' prefix signifies this self-reflective capability - our framework doesn't just deploy agents, but rather learns to develop, customize, and coordinate agents automatically. Combined with ``Chain'', which represents the seamless workflow of connected agentic components, the name \model\ captures our framework's essence as a self-improving system that elevates agent development to a meta-level, enabling automated agent generation and workflow optimization.}

This stark contrast between universal needs and limited accessibility leads us to a fundamental research question: \emph{Is it possible to democratize LLM agent development by enabling Natural Language-based Creation and Customization?} In this work, we aim to realize this vision by introducing \model, a novel framework that fundamentally reimagines agent development as a fully automated, language-driven process requiring zero programming expertise. To realize this vision, \model operates as an autonomous Agent Operating System with three key capabilities: 1) \textbf{Natural Language-Driven Multi-Agent Building} - automatically constructing and orchestrating collaborative agent systems purely through natural dialogue, eliminating the need for manual coding or technical configuration; 2) \textbf{Self-Managing Workflow Generation} - dynamically creating, optimizing and adapting agent workflows based on high-level task descriptions, even when users cannot fully specify implementation details; and 3) \textbf{Intelligent Resource Orchestration} - providing unified access to tools, APIs, and computational resources via natural language while automatically managing resource allocation and optimization. Through this innovative architecture, \model democratizes LLM agent development while maintaining enterprise-grade sophistication, transforming a traditionally complex engineering task into an intuitive conversation accessible to all users.

To enable fully-automated and zero-code LLM agent development, \model\ introduces several synergistic technical innovations that form a complete framework: First, the \textbf{Agentic System Utilities} provides a foundational multi-agent architecture, where specialized web, code, and file agents collaborate seamlessly to handle diverse real-world tasks. At its core, the \textbf{LLM-powered Actionable Engine} serves as the system's brain, supporting flexible integration of any LLM provider through both direct and transformed tool-use paradigms for robust action generation. To address the critical challenge of information management, the \textbf{Self-Managing File System} enhances overall system capability by automatically converting diverse data formats into queryable vector databases, enabling efficient information access across all operations. Additionally, the \textbf{Self-Play Agent Customization} not only transforms natural language requirements into executable agents through structured XML schemas, but also automatically generates optimized workflows through iterative self-improvement, eliminating the need for manual agent programming or workflow design. Together, these innovations enable \model\ to democratize agent development while maintaining production-level robustness.

\model's exceptional capabilities have been rigorously validated through comprehensive empirical evaluation. In standardized benchmarks, it secured a strong second place on the Generalist Agent Benchmark (GAIA), while significantly outperforming state-of-the-art RAG approaches on the Retrieval-Augmented Generation benchmark. Beyond these quantitative achievements, extensive case studies demonstrated \model's robust self-development capabilities across diverse real-world scenarios, highlighting its practical value in automated agent development.
