\begin{longtable}{p{15cm}}
\caption{Comparison of the single LLM (Deepseek-v3) and AI-generated Majority Voting workflow.}
\label{tab:workflow_error}\\
\toprule
% \textbf{Role} & \textbf{Action} & \textbf{Content}     \\
%  \midrule
\textbf{Problem:} Simplify $\tan 100^\circ + 4 \sin 100^\circ.$ \\
\midrule
\textbf{Ground Truth:} $-\sqrt{3}$ \\
\midrule
\textbf{AI-generated Majority Voting workflow:} Let me analyze all three solutions:

1. GPT-4 Solution:
- Does not provide a definitive numerical answer
- Focuses on theoretical approach and approximations
- Inconclusive result

2. Claude Solution:
- Uses supplementary angle properties
- Converts 100° to 180° - 80°
- Arrives at \(-\sqrt{3}\) through numerical calculations
- \colorbox{green}{Final answer: \(-\sqrt{3}\)}

3. Deepseek Solution:
- Uses similar supplementary angle properties
- Also converts 100° to 180° - 80°
- Goes through detailed algebraic steps
- \colorbox{red}{Final answer: \(\sqrt{3}\)}

Analysis:
1. \colorbox{yellow}{Two solutions provide definitive answers (\(-\sqrt{3}\) and \(\sqrt{3}\))}, while one is inconclusive
2. \colorbox{yellow}{Both Claude and Deepseek use similar approaches with supplementary angles}
3. The main difference is in the sign of the final answer
4. \colorbox{yellow}{Claude's solution provides more explicit numerical calculations}
5. Looking at the steps:
   - 100° being in quadrant II
   - Using supplementary angle properties
   - tan 100° being negative in quadrant II
   - The final combination with 4sin 100°

The Claude solution appears more precise in tracking signs and considering quadrant properties, making its answer more reliable.

Final Answer: The final answer is $-\sqrt{3}$. I hope it is correct. \\ 
\midrule
\textbf{Deepseek-v3:} To simplify \(\tan 100^\circ + 4 \sin 100^\circ\), we proceed as follows:
1. Recall that \(\tan \theta = \frac{\sin \theta}{\cos \theta}\). Thus:
   \[
   \tan 100^\circ = \frac{\sin 100^\circ}{\cos 100^\circ}.
   \]

2. Substitute this into the original expression:
   \[
   \tan 100^\circ + 4 \sin 100^\circ = \frac{\sin 100^\circ}{\cos 100^\circ} + 4 \sin 100^\circ.
   \]

$\cdots$

16. Therefore, the simplified form of \(\tan 100^\circ + 4 \sin 100^\circ\) is:
    \[
    \boxed{\sqrt{3}}.
    \]
Final Answer: The final answer is \colorbox{red}{\(\sqrt{3}\)}. I hope it is correct. \\
\midrule
\midrule
\textbf{Problem:} What integer $n$ satisfies $0\le n<18$ and $$n\equiv -11213141\pmod{18}~?$$ \\
\midrule
\textbf{Ground Truth:} $13$ \\
\midrule
\textbf{AI-generated Majority Voting workflow:} Let me compare the three solutions:

1. \colorbox{green}{GPT-4 Solution: Arrives at 13} through direct division and remainder calculation
- Used division: 11213141 ÷ 18
- Found remainder of 5 for positive number
- Adjusted for negative sign to get 13

2. \colorbox{green}{Claude Solution: Arrives at 13} through division and remainder
- Divided -11213141 by 18
- Found remainder directly
- Got 13 as the result in the range [0,18)

3. \colorbox{red}{Deepseek Solution: Arrives at 5 through}:
- Division and remainder calculation
- Used divisibility rules for 2 and 9
- Got 5 as final answer

Analyzing the solutions:
- \colorbox{yellow}{Two solutions (GPT-4 and Claude) agree on 13}
- One solution (Deepseek) gives 5
- \colorbox{yellow}{Both GPT-4 and Claude show clearer arithmetic progressions}
- The key difference is in handling the negative sign in the modulo operation
- When working with negative numbers in modulo arithmetic, we need to find the positive equivalent in the range [0,n)
- The correct approach is to add the modulus (18) to the negative remainder until we get a number in the range [0,18)

By majority voting and verification:
- \colorbox{yellow}{13 is supported by two solutions}
- \colorbox{yellow}{The reasoning in these solutions is more complete}
- We can verify: -11213141 $\equiv$ 13 (mod 18) and $0 \leq 13 < 18$

Final Answer: The final answer is 13. I hope it is correct. \\
\midrule
\textbf{Deepseek-v3:} To solve for the integer \( n \) that satisfies \( 0 \leq n < 18 \) and \( n \equiv -11213141 \pmod{18} \), we need to find the remainder when \(-11213141\) is divided by \(18\). Here's the step-by-step solution:

1. \textbf{Simplify the modulus}: First, note that \(-11213141 \pmod{18}\) is equivalent to finding the positive remainder when \(-11213141\) is divided by \(18\). Since the modulus operation is periodic with period \(18\), we can add multiples of \(18\) to \(-11213141\) until we obtain a positive number within the range \(0 \leq n < 18\).

2. \textbf{Calculate the remainder}: To find \(-11213141 \pmod{18}\), we can instead compute \(11213141 \pmod{18}\) and then take the negative of that result, adjusting it to be positive if necessary.

   - First, compute \(11213141 \div 18\):
     \[
     11213141 \div 18 = 623000 \text{ with a remainder of } 11213141 - 18 \times 623000 = 11213141 - 11214000 = -859
     \]
     This means \(11213141 \equiv -859 \pmod{18}\).

   - Now, compute \(-859 \pmod{18}\):
     \[
     -859 \div 18 = -48 \text{ with a remainder of } -859 - 18 \times (-48) = -859 + 864 = 5
     \]
     So, \(-859 \equiv 5 \pmod{18}\).

   - Therefore, \(-11213141 \equiv 5 \pmod{18}\).

3. \textbf{Final answer}: \colorbox{red}{The integer \( n \) that satisfies \( 0 \leq n < 18 \) and \( n \equiv -11213141 \pmod{18} \) is \(5\).}

Final Answer: The final answer is $5$. I hope it is correct. \\
\midrule
\textbf{Problem:} In the diagram below, we have $\sin \angle RPQ = \frac{7}{25}$.  What is $\cos \angle RPS$?

[asy]

pair R,P,Q,SS;

SS = (-2,0);

P = (0,0);

Q = (2,0);

R = rotate(aSin(7/25))*(1.5,0);

dot("$S$",SS,S);

dot("$Q$",Q,S);

dot("$R$",R,N);

dot("$P$",P,S);

draw(Q--SS);

draw(P--R);

[/asy]
\\
\midrule
\textbf{Ground Truth:} $-\frac{24}{25}$ \\
\midrule
\textbf{AI-generated Majority Voting workflow:} Let me analyze all three solutions:

1. GPT-4 Solution: Final answer = $\frac{24}{25}$
- Used right triangle approach
- Found cosine of RPQ = 24/25
- Assumed RPS = RPQ

2. Claude Solution: Final answer = $-\frac{24}{25}$
- Used supplementary angles property
- Found cos RPQ = 24/25
- Used the fact that cos(supplementary angles) are negatives of each other
- \colorbox{green}{Concluded cos RPS = -cos RPQ = -24/25}

3. DeepSeek Solution: Final answer = $\frac{7}{25}$
- Used complementary angles property
- \colorbox{red}{Found cos RPQ = 24/25}
- Incorrectly concluded cos RPS = sin RPQ

Analyzing the approaches:
1. Looking at the diagram, RPS and RPQ are supplementary angles (they form a straight line)
2. For supplementary angles, their cosines are negatives of each other
3. Claude's solution correctly uses this property
4. GPT-4 incorrectly assumed the angles were equal
5. DeepSeek incorrectly used complementary angle properties

\colorbox{yellow}{Claude's solution is the most mathematically sound} because:
- It correctly identifies that RPS and RPQ are supplementary angles
- It properly applies the relationship between cosines of supplementary angles
- The negative sign in the answer makes geometric sense given the angle's position

Final Answer: The final answer is $-\frac{24}{25}$. I hope it is correct. \\
\midrule
\textbf{Deepseek-v3:} We are given that \(\sin \angle RPQ = \frac{7}{25}\). Since \(\angle RPQ\) and \(\angle RPS\) are complementary angles (they form a right angle together), we have:

\[
\cos \angle RPS = \sin \angle RPQ = \frac{7}{25}.
\]

Final Answer: The final answer is \colorbox{red}{\(\frac{7}{25}\)}. I hope it is correct. \\
\midrule
\textbf{Problem:} In the circle with center $Q$, radii $AQ$ and $BQ$ form a right angle. The two smaller regions are tangent semicircles, as shown. The radius of the circle with center $Q$ is 14 inches. What is the radius of the smaller semicircle? Express your answer as a common fraction.
\\
\midrule
\textbf{Ground Truth:} $\frac{14}{3}$ \\
\midrule
\textbf{AI-generated Majority Voting workflow:} Let me analyze all three solutions:

1. GPT-4's solution: r = 7
2. \colorbox{green}{Claude's solution: r = 14/3}
3. \colorbox{red}{Deepseek's solution: $r = 14(\sqrt{2} - 1)$}

Each solution uses different approaches:
- GPT-4 assumes the semicircles are equal and uses a simple division (14/2 = 7)
- Claude uses the relationship r + r + r = R, leading to 14/3
- Deepseek uses the Pythagorean theorem and complex algebra, arriving at $14(\sqrt{2} - 1)$

Let's verify which one makes more sense:

1. Key geometric principles suggest that when two circles are tangent to each other and to two perpendicular lines (in this case, the radii forming the right angle), there's a special relationship where r + r + r = R, where r is the radius of the smaller circles and R is the radius of the larger circle.

2. This principle is well-established in geometry. When two circles are placed in a right angle and are tangent to each other and the lines forming the right angle, their radii follow this 3:1 ratio.

3. Therefore, \colorbox{yellow}{Claude's solution using 3r = 14 leading to r = 14/3 is the most mathematically sound}.

4. We can verify this:
- It satisfies the tangency conditions
- The semicircles fit perfectly in the right angle
- The relationship maintains the proper geometric proportions

The GPT-4 solution (r = 7) is too large as it would make the semicircles overlap.
The Deepseek solution $14(\sqrt{2} - 1)$ is more complex than necessary and doesn't align with the fundamental geometric principle for this configuration.

Final Answer: The final answer is $\frac{14}{3}$. I hope it is correct.
\\
\midrule
\textbf{Deepseek-v3:} Let’s solve the problem step by step.

1. Understand the setup: We have a circle with center \( Q \) and radius 14 inches. Radii \( AQ \) and \( BQ \) form a right angle, meaning the angle \( \angle AQB = 90^\circ \). There are two smaller semicircles tangent to the larger circle and to each other.

2. Key observations:
   - The two semicircles are tangent to the larger circle and to each other. This means the distance from the center \( Q \) to the center of each semicircle is \( 14 - r \), where \( r \) is the radius of the smaller semicircle.
   - The centers of the two semicircles and the center \( Q \) form a right triangle because \( \angle AQB = 90^\circ \).

$\cdots$

5. Simplify the answer:
   - The radius of the smaller semicircle is \( 14(\sqrt{2} - 1) \). However, this can be expressed as a common fraction:
     \[
     r = 14(\sqrt{2} - 1) = 14\sqrt{2} - 14.
     \]
     \[
     r = \frac{14\sqrt{2} - 14}{1}.
     \]
     \[
     r = \frac{14(\sqrt{2} - 1)}{1}.
     \]
     \[
     r = 14(\sqrt{2} - 1).
     \]

6. Final Answer:
   \colorbox{red}{The radius of the smaller semicircle is \( 14(\sqrt{2} - 1) \)}.

Final Answer: The final answer is \( 14(\sqrt{2} - 1) \). I hope it is correct. \\
\midrule
\bottomrule
\end{longtable}
