% \newpage

\section{Conclusion}
% \buyun{All the following are currently in progress.}



In this work, we presented the Knowledge-Distilled Attacker (KDA), an open-source attacker model that distills the strengths of multiple SOTA attackers into a unified framework. KDA eliminates the reliance on intricate prompt engineering and extensive queries, making it a scalable and efficient solution for generating diverse, coherent, and effective attack prompts. Through extensive evaluations, KDA demonstrated superior attack success rates, generalization to unseen datasets and models, and notable advantages in efficiency and diversity compared to existing methods. Our ablation studies further underscored the effectiveness of KDA, showing that it not only retains but also surpasses the performance of its teacher attackers, highlighting the critical role of format control and ensemble diversity in driving its success. By publicly releasing the KDA model and training dataset, we aim to facilitate further research and provide a valuable resource for advancing the understanding and mitigation of jailbreak attacks on LLMs. In the future, we aim to continue develop effective attackers via distillation that considers both jailbreak attacks and defenses.

% \ryan{In the future, the framework can be extended to consider defenses. }
% \myparagraph{Future Work} \buyun{TODO: organizing appendix.}

% \newpage



% \paragraph{Limitation} 


% \paragraph{Conclusion} This paper presents three major contributions:
% \begin{itemize}[wide]
%     \item We propose a human-aligned judge LLM, fine-tuned from the \texttt{llama-guard-2-8b} model on a human-annotated dataset, which demonstrates superior alignment with human judgment, as evidenced by its accuracy and F1 score.

%     \item We introduce KDA, an attacker LLM leveraging knowledge distillation, capable of generating black-box, coherent, and diverse attack prompts without reliance on commercial LLMs. This attack generation process is both efficient and effective. To achieve this, we fine-tune the \texttt{vicuna-13b-v1.5} model using distilled attack prompts from SOTA methods such as AutoDAN, GPTFuzzer, and PAIR. The effectiveness and efficiency of KDA are addressed by extensive evaluation.

%     \item We curate the \texttt{RedTeam-10k} dataset, demonstrating KDA's ability to perform large-scale attacks against SOTA LLMs.
% \end{itemize}

% In this work, we proposed a jailbreak evaluator finetuned on human-annotated samples for better alignment and better accuracy. We further introduced KDA, a knowledge-distilled attacker LLM for generating high-quality attack prompts in an effective and scalable manner. Last but not least, we released a large-scale \texttt{RedTeam-10k} dataset that enables industrial-scale red-teaming. All in all, while the KDA framework demonstrates superior ASR across various SOTA LLMs and exhibits strong transferability, its effectiveness heavily relies on the presence of successful attack prompts. KDA is unable to jailbreak an LLM if no attack prompts exist for the target model. We reserve the research on novel attack style synthesis and improvement over current unsuccessful attacks across different models for future endeavors. 

% \newpage