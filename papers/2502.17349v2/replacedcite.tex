\section{Related Works: Molecule Generation and Guidance on Diffusion Models}
\label{app:relatedwork}
\paragraph{Fragment-Based Drug Design.} Fragment-Based Drug Design (FBDD)____ is a drug discovery approach that utilizes small molecular fragments and optimizes them into larger, more potent drug candidates. Deep learning-based FBDD encompasses a variety of tasks, each distinguished by its learning objective. \textbf{Linker Generation} ____ is a fundamental task in FBDD, where two molecular fragments are connected to form a complete molecule. Similarly, \textbf{Topology Linker Generation} ____ focuses on linking the topological graphs of fragments to generate a complete molecular topology graph. \textbf{Scaffold Hopping}____ involves replacing the core structure of a given molecule while preserving its biological activity. \textbf{PROTAC Design}____ focuses on generating molecules that incorporate fragment linkers with flexible rotation and translation in 3D space. \textbf{Fragment Growing}____ expands single small molecular fragment into larger drug-like structure. Our work specifically addresses Linker Generation, tackling the critical trade-off between diversity and validity observed in existing models. \cref{tab:moleculetasks} Summarizes the tasks in FBDD along with two standard molecular generation tasks—De Novo Generation and Conformation Generation.

\paragraph{Molecule Generation.} Molecule generation____ based on deep learning plays a crucial role in drug discovery and is broadly categorized into \textbf{De Novo Molecule Generation}____, \textbf{Fragment-Based Drug Design}____, \textbf{Target-Aware Drug Design}____, and \textbf{Conformer Generation}____, based on their input and output formulations. Molecule generation can also be classified into three sub-tasks: \textbf{Topology Generation}____, \textbf{Point Cloud Generation}____, and \textbf{3D Graph Generation}, where deep learning models learn the distributions of molecular bonding topology, spatial coordinates, and 3D molecular graph representations, respectively. Our work falls under Fragment-Based Drug Design, introducing a hybrid approach that leverages pretrained models for topology and point cloud generation in a zero-shot manner.


\paragraph{Guidance on Diffusion Models.}
Diffusion models____ have demonstrated exceptional performance across various generative tasks, including image, video, graph, and molecular generation. A recent advancement in diffusion models is \textbf{conditional generation}____, which enables sampling from a conditional distribution based on desired properties. This is achieved by incorporating a guidance term into the backward diffusion process. To compute the guidance term, \textbf{Classifier Guidance} ____ employs a classifier trained to estimate the likelihood of a given property, while \textbf{Classifer-Free Guidance} (CFG) ____ replaces the classifier with a conditional diffusion model. \textbf{CFG++} ____ is designed to mitigate off-manifold sampling issues, and \textbf{Diffusion Posterior Sampling} (DPS) ____ was developed to solve nonlinear noisy inverse problems. In this paper, we introduce the first DPS-based method for guiding diffusion models in molecular point cloud generation using molecular topology. Our approach introduces a novel energy-based function that effectively bridges topological and spatial molecular representations.

%%%%%%%%%%%%%%%%%%%%%%%%%%%%%%%%%%%%%%%%%%%%%%%%%%%%%%%%%%%%%%%%%%%%%%%%%%%%%%%
%%%%%%%%%%%%%%%%%%%%%%%%%%%%%%%%%%%%%%%%%%%%%%%%%%%%%%%%%%%%%%%%%%%%%%%%%%%%%%%