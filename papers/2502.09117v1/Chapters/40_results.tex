\section{Results}
\label{sec:results}



\subsection{Conformance Analysis}
\label{sub:conformance_analysis_results}

The conformance analysis of all valid node packages in the Node-RED library was performed following the methodology presented in Section~\ref{sub:conformance_analysis_methodology}.
The CodeQL information flow analysis of the node packages' source code using the query created in step {\textbf{\Circled{2}}} yielded a total of 58807 detected information flows. 
After merging duplicate endpoints in the flows, 42590 endpoints across all analyzed node packages remained.
Comparing each node package's specified inputs and outputs against the number of identified endpoints resulted in the classification of each node package into one of the three conformance cases convergence, divergence, or absence.
%
\begin{figure}
    \centering
    \includegraphics[width=0.8\linewidth]{Figures/conformance_cases.pdf}
    \caption{Results of non-conformance analysis, generalized into the three standard cases; n = 4798.}
    \label{fig:non_conformances_generalized}
\end{figure}
%
Figure~\ref{fig:non_conformances_generalized} shows the results of this classification.
Within the 4798 valid node packages at the time of analysis, 1094 (22.8\%) were classified as convergence cases, 2643 (55\%) as divergence cases, and 1070 (22.2\%) as absence cases. 
In terms of individual nodes contained in the node packages, there are 2240 individual nodes assigned to the convergence case, 10903 to the divergence case, and 4807 to the absence case.
As a ratio of individual nodes per node package, the convergence case has 2.05 individual nodes per node package, the divergence case 4.06, and the absence case 4.67.
Concerning the size of the node packages in lines of code (LOC), the nodes in the convergence case have 270 LOC per node package on average, the nodes in the divergence case 1492 LOC per node package, and the nodes in the absence case 950 LOC per node package.
Comparing the two above measures shows that individual nodes in the convergence case have 132 LOC on average, 367 LOC per node in the divergence case, and 204 LOC per node in the absence case.

\begin{figure}
    \centering
    \includegraphics[width=0.5\linewidth]{Figures/additional_ports_plot.pdf}
    \caption{Distribution of the combined number of additional sources and / or sinks per divergent node package as histogram. Not showing the highest percentile because they skew the graph.}
    \label{fig:additional_sources_sinks_histogram}
\end{figure}


Figure~\ref{fig:additional_sources_sinks_histogram} presents a histogram of the number of additional endpoints per node package that is classified as divergent.
It does not show the outliers in the highest percentile for readability purposes. 
The 100th percentile contains 27 values, ten of them are values between 70 and 100, 16 further ones reach up to 194, and the highest value lies at 556 combined additional sources and sinks.
On average, there are 2.9 additional sources and 5.3 additional sinks per node package in the divergent node packages, i.e., 8.2 additional endpoints in sum.
Per individual node, this is an average of 2 additional endpoints per node.

\vspace{2mm}
\noindent
\fcolorbox{black}{green!05}{%
    \parbox{0.983\linewidth}{% 
        \textbf{\faExclamationCircle\;RQ1:} Figure~\ref{fig:non_conformances_generalized} shows the distribution of the node packages in the Node-RED library of nodes across the three conformance cases.
        55\% of the node packages are classified as divergent, i.e., containing hidden information flows.
    }%
}%



\subsection{Risk Assessment}
\label{subsec:risk_assessment}

The risk assessment was performed based on the 951 information flows detected by CodeQL within the selected subset of 97 node packages that were manually verified to be correct in step {\small\Circled{4}} of the methodology.
For 12 of the 951, the type of transferred data could not be determined with certainty, and they were therefore not considered in the further assessment, yielding 939 remaining information flows.
By manually analyzing the source code of the node packages and following the methodology presented in Section~\ref{sub:risk_assessment_methodology}, 18 different groups were created to classify the information flows.
%
\begin{table}
    \small
    \centering
	\caption{Results of the risk assessment. Identified groups in the information flows detected by CodeQL, the number of flows in each group, the percentage of overall flows in the assessment, and the assigned severity of the introduced security risk. Separated by type of the passed information. \textbf{Sev.} = Severity; h/m/l = high/medium/low}
	\label{tbl:risk_assessment_results}
    \begin{tabular}{p{2.1cm}p{7.1cm}p{1.8cm}r}
        \toprule
        \textbf{Information} & \textbf{Group description} & \textbf{Flows} & \textbf{Sev.} \\
        \midrule
        Sensitive  & Display sensitive information in terminal & 37 (3.9\%) & h \\
        information & Display sensitive information in dashboard & 46 (4.9\%) & m \\
         & Log sensitive information & 31 (3.3\%) & h \\
         & Send sensitive information to external server & 31 (3.3\%) & h \\
         & Write sensitive information to file & 3 (0.3\%) & h \\
         & Send sensitive information to framework & 11 (1.2\%) & m \\
        \midrule
        Error  & Log error message & 16 (1.7\%) & h \\
        message & Display error message in dashboard & 110 (11.7\%) & m \\
         & Display error message in terminal & 18 (1.9\%) & h \\
        \midrule
        Input  & Send input message to other node & 323 (34.4\%) & l \\
        message & Log input message & 56 (6.0\%) & h \\
         & Send input message to external hardware device & 2 (0.2\%) & h \\
         & Display input message in dashboard & 175 (18.6\%) & m \\
         & Write input message to file & 6 (0.6\%) & h \\
         & Send input message to external server & 22 (2.3\%) & h\\
         & Display input message in terminal & 28 (3.0\%) & h \\
        \midrule
        Misc. & Misc. low severity & 10 (1.1\%) & l \\
         & Misc. high severity & 14 (1.5\%) & h \\
        \bottomrule
    \end{tabular}
\end{table}
%
%
%
Table~\ref{tbl:risk_assessment_results} presents the groups, the number of information flows that were assigned to each, and the severity assessment associated with each group.

When comparing the different groups, it is apparent that the type of transferred information does not affect the severity, only the context to which it is exposed.
As we realized late during the analysis, each group's type of transferred datacan contain sensitive information.
The first block of groups in Table~\ref{tbl:risk_assessment_results} (with \textit{Sensitive information} in the column \textit{Information}) are groups directly related to sensitive information.
These are information flows that handle entities such as passwords, encryption keys, usernames, etc.
The second block of groups (\textit{Error message} in column \textit{Information}) cover information flows that handle error messages.
While standard error messages of existing libraries and frameworks are usually innocuous, they can also contain information that should not be disclosed.
Especially when they are customized by developers, they often contain sensitive information.
Multiple CWEs refer to this issue (CWE-209 \textit{Generation of Error Message Containing Sensitive Information}~\footnote{https://cwe.mitre.org/data/definitions/209.html} as well as other more specific CWEs) and it has been noted in the literature as well~\cite{Halfond06_error_messages,Smith10_error_messages}.
Finally, the third block of groups in Table~\ref{tbl:risk_assessment_results} (\textit{Input message} in column \textit{Information}) contains groups that handle the input message a node received from other upstream nodes.
These messages also often contain sensitive information required for the node's functionality.
After establishing that all information could potentially be sensitive, the context to which it is exposed remains the only factor to determine the risk imposed by an information flow.
We retain the created grouping nevertheless to give a more nuanced classification.

\begin{table}
    \small
    \centering
	\caption{Results of the risk assessment summarized per severity rating.}
	\label{tbl:risk_assessment_results_summary}
    \begin{tabular}{p{2cm}p{1.5cm}p{1.5cm}}
        \toprule
        \textbf{Severity} & \textbf{Flows} & \textbf{Percentage} \\
        \midrule
        Low & 333 & 35.5\% \\
        Medium & 342 & 36.4\% \\
        High & 264 & 28.1\% \\

        \bottomrule
    \end{tabular}
\end{table}

Table~\ref{tbl:risk_assessment_results_summary} summarizes the results of the risk assessment by the assigned severity rating.
A low severity can be assumed for 35.5\% of the information flows, a medium severity for 36.4\%, and a high severity for 28.1\%.



\vspace{2mm}
\noindent
\fcolorbox{black}{green!05}{%
    \parbox{0.983\linewidth}{% 
        \textbf{\faExclamationCircle\;RQ2:} Table~\ref{tbl:risk_assessment_results_summary} provides the answer to RQ2, summarizing the results of the risk assessment of a subset of detected information flows into the severity ratings. 
        A high severity is associated with 28\% of information flows, a medium severity with 36\% of analyzed flows.
    }%
}%





