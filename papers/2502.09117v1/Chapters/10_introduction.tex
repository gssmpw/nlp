\section{Introduction}
\label{sec:intro}




Internet of Things (IoT) devices have seen a tremendous boom in their number over the last decade and have seamlessly become a part of our everyday world. 
From industrial control systems to fitness bands, from smart security systems to medical devices, more and more devices enter the cyberspace continuously to increase the efficiency of their respective industries. 
Predictions for the number of IoT devices being installed, connected, and autonomously managed within the coming years vary, but reach up to 100 billion \cite{huawei}. 

With the increasing adoption of IoT in critical and sensitive domains, the security of such devices becomes also more important, due to the grave financial and personal costs that can occur from compromised systems.
Consequently, research in the field has produced comprehensive studies of the challenges faced (\cite{Ahmad_Alsmadi21_ml_approaches_iot_security,Hassija19_iot_security_survey,Zhang14_Iot_security_challenegs_opportunities}), ongoing work (\cite{Mahmoud15_iot_security,MohamadNoor_Hassan19_iot_security_research}), and possible mitigation approaches (\cite{AlGaradi20_survey_ml_dl_iot,Chatterjee22_iot_anomaly_detection,Williams22_survey_security_iot,Xiao18_iot_security_techniques_ml}).



The increase in demand for physical deployments has also been the propelling factor for rapidly developing IoT software development tools. 
Implementing applications for IoT devices poses some unique challenges for developers~\cite{Udoh18_developing_iot_applications}.
Development frameworks support users in this process and help them realize a wide range of functionality by abstracting many complex activities into simple interfaces, for example, the management of sensor data streams from hardware devices or communication with other entities.
The highest level of abstraction is reached by so-called \textit{no-code} or \textit{low-code} platforms, which simplify the development process into graphical drag-and-drop interfaces, allowing even users with low or no programming knowledge to create applications for IoT devices~\cite{Bock21_low_code_platform,Rokis22_challenges_of_low_code}.

Some development frameworks combine the low-code development with an open-source approach.
Here, the underlying source code that realizes the abstraction of the code-intensive software development into low-code functionalities can be provided by anyone choosing to contribute to the framework.
Moreover, these frameworks not only \textit{allow} the addition of open-source code, they \textit{rely} on it to a large degree.
While this approach leverages the work of numerous open-source developers, and thereby promises to increase the amount of available functionalities, it also introduces security challenges, since the frameworks' code is no longer created by a single authority that can implement secure development practices.
Instead, the responsibility for secure software is shifted to open-source developers.
With insecure software being a major issue for the security of IoT devices~\cite{HaddadPajouh21_security_iot}, this could result in less secure IoT devices.
Although it is possible to enforce security controls on the provided code, this requires large efforts, especially since such controls must be adapted to highly varied code.

Because of the above properties, low-code development frameworks that rely on open-source contributions could potentially provide insufficiently secured software to users that do not have the capabilities to address or even realize this issue.
A possible consequence are vulnerable applications for IoT devices that can be compromised by attackers and cause serious security implications.
The domain of IoT with its intricate connection to many sensitive areas of our lives further increases the seriousness of this.
Additionally to the unaware end user, it could well be that the open-source developers themselves do not realize the extent to which their code could expose sensitive information, given that there are no requirements for security checks or vetting of contributors.
Consequently, the specifications of open-source contributions to the development framework do not accurately depict the reality of the implementation, specifically concerning inputs and outputs over which their functions communicate with other parts of the software.


In this paper, we present the results of a conformance analysis of the nodes in the Node-RED ecosystem, with which we investigated possibly hidden information flows.
In this context, ``hidden'' information flows are flows that can occur in the implemented node but are not considered in the node's specification, i.e., not captured by the specified numbers of inputs and outputs.
These are relevant for the described scenario, since they and their security implications likely go unnoticed by the end user of the ecosystem.
Therefore, we also performed a risk assessment of identified information flows.
The conducted study investigates the above-described security issue faced by low-code open-source development frameworks for IoT devices by shedding light on the extent of information flows that could expose sensitive information to unauthorized entities and are not marked as such by the code's specifications.
%
Based on the above description, this paper addresses the following research questions:


\vspace{2mm}
\noindent
\fcolorbox{black}{blue!05}{%
    \parbox{0.983\linewidth}{% 
        \textbf{\faQuestionCircle\;RQ1: What is the prevalence of hidden information flows in the Node-RED ecosystem?}
    }%
}%

\vspace{2mm}
\noindent 
We performed a conformance analysis of all node packages (the components used to create IoT applications) provided by Node-RED.
The conformance case (convergence, divergence, or absence) of each node package was determined by comparing the number of specified inputs and outputs against the number of sources and sinks detected in an information flow analysis with CodeQL.


\vspace{2mm}
\noindent
\fcolorbox{black}{blue!05}{%
    \parbox{0.983\linewidth}{% 
        \textbf{\faQuestionCircle\;RQ2: What are the most likely risks that users of the Node-RED ecosystem are exposed to because of the hidden information flows?}
    }%
}%

\vspace{2mm}
\noindent 
All hidden information flows can be seen as a compliance issue from a software development perspective, but not all hidden information flows necessarily pose a security issue.
To assess the extent to which the detected non-conformances impose security threats to users of the ecosystem, we manually assessed a subset of the detected information flows concerning their security risk and severity.





\vspace{2mm}




The rest of this paper is structured as follows: Section~\ref{sec:background} describes background information on the technologies used for the conformance analysis; Section~\ref{sec:methodology} presents the methodology of the conducted study; Section~\ref{sec:results} contains the results of the conformance analysis and the risk assessment of information flows; Section~\ref{sec:discussion} discusses the presented results; Section~\ref{sec:related} presents the related work; and Section~\ref{sec:conclusion} concludes this paper.


