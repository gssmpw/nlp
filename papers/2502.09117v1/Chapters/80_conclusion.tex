\section{Possible Mitigations and Research Outlook}
\label{sec:conclusion}

Insecure applications running on IoT devices can have grave implications for users.
In this paper, we presented the results of a conformance analysis of nodes in the Node-RED library of nodes.
Comparing the number of specified inputs and outputs against the number of endpoints detected via an information flow analysis with CodeQL, we identified a ratio of 55\% divergent node packages within the complete library of 4798 node packages at the time of analysis.
These divergent node packages exhibit more possible information flow endpoints than are captured by their specifications, thus enabling ``hidden'' information flows.
Such cases could potentially be exploited by attackers to obtain access to sensitive information that should be restricted from them.

A risk assessment of a subset of the information flows showed, that 28\% of them are associated with a high severity rating, imposing a substantial security risk to users of the Node-RED ecosystem.

In a low-code environment such as Node-RED, the impact of security issues is heightened because the user is likely not capable of addressing or even realizing such problems.
We see two main directions to mitigate the identified issue: (1) improve the security of nodes and (2) improve the information provided about nodes and the information available to developers.

Improving the security of nodes could be managed centrally by the framework itself, be left to the open-source developers, or realized in a mixture of these two.
On the framework side, each contributed node could have to undergo a security analysis before being published.
This analysis could, for example, resemble the information flow analysis we have conducted in this paper.
If such a process is in place, it could also be extended with further security checks, such as the execution of a number of industry-standard static application security testing tools (SAST).
The responsibility to perform this process could also be moved to the developers, who would have to run a standardized information flow analysis or other security checks and provide the results of it when submitting new or changed code to the framework.
With such a mechanism in place, contributions could be rejected if they do not meet certain requirements, such as the realization of information flows only with whitelisted functionalities.

Another technical solution could be to adjust the framework to be more restrictive in regard to information flows. 
Nodes could be encapsulated more strictly and data only allowed to enter or exit the nodes via framework-specific channels. 
Such a solution would break the functionality of many nodes that are not adjusted to the new framework version, but it would greatly improve the security of developed applications.
Compatibility issues could be mitigated by introducing the new version over an extended period of time in which existing nodes would have to be migrated, a standard practice for such updates.

Looking in the second direction to address the identified issue, a security analysis process could be enforced as described above, but the vetting process would not necessarily enforce any restrictions based on the analysis.
Simply providing the results of the analysis to users would allow them to make an informed decision about whether to use a specific node or refrain from it.
This approach essentially matches the current idea of providing specifications for contributed nodes, with the difference that the non-conformances between implementation and specifications would be addressed.

More information should also be made available for developers of nodes.
Currently, the only information about security provided by Node-RED concerns enforcing access control mechanisms to the development dashboard.
No description of what are inputs and outputs of nodes exists.
There are no resources that discuss the security of node package contributions or what their specifications should capture.
We believe it is likely that the lack of available information concerning security in part caused the identified issue in the ecosystem.



\textbf{Research outlook:}
\begin{itemize}[label=\textbullet]
    \item Ecosystems that rely on open-source developers' contributions (such as Node-RED) should provide clear instructions on how security-critical functionalities should be implemented and documented.
    Ambiguities can lead to issues when interpretations differ.
    \item Such ecosystems should consider moving away from complete freedom for open-source developers and implementing an automated security analysis process for new contributions.
    For example, an analysis pipeline such as the one presented in this paper could be used to objectively determine the numbers of inputs and outputs of a node.
    The results of the analysis can either be used to allow a more informed decision for the user on whether to use a specific node, or even to decide whether the code should be accepted as part of the framework or rejected.
    \item The issue of hidden information flows could also be addressed on a technical level in the framework, e.g., by enforcing a stricter encapsulation of nodes.
    In such a scenario, the framework could provide functionality for all necessary types of data transfer and disallow any other communication channels.
    These official nodes would become the only code that needs to be thoroughly secured with suitable security mechanisms.
\end{itemize}