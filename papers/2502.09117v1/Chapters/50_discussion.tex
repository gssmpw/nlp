\section{Discussion}
\label{sec:discussion}



\subsection{Divergences in the \node Ecosystem}


The results presented in Section~\ref{sub:conformance_analysis_results} paint a concerning picture of the state of conformance between the implementation and specifications of the nodes in the \node ecosystem.
More than half of the node packages and an even higher proportion of the individual nodes are detected to be divergent, with an average of 8.2 additional endpoints per node package or 2 additional endpoints per individual node.
Many node packages show dozens of additional endpoints in the implementation compared to the specifications.
As the histogram in Figure~\ref{fig:additional_sources_sinks_histogram} shows, most node packages have only one or a few additional endpoints. 
This could suggest that many cases can be explained by minor negligence by the developer instead of a more substantial problem.
The consequences, however, can be the same as for a node package with many additional endpoints.
Overall, the observed numbers show a disconnection between implementation and specifications that make the specifications useless in their current form.


This disconnection can not be explained by differing definitions of inputs and outputs, since a large portion of node packages were also classified as absence cases.
Further, no such definition could be found from Node-RED, therefore, the results at the least show a lack of consensus among the open-source developers.



Since it cannot be determined which endpoints were considered by the developers when creating a node's specifications and therefore, which are the divergent endpoints, discussing the origins of divergences in detail would be speculative. 
However, as an intermediate step of the risk assessment, Table~\ref{tbl:risk_assessment_results} also shows the type of endpoints that are prevalent in the nodes, regardless of the conformance case.
As shown there, most of the identified information flows either display information in the terminal or the development dashboard, or they log information.
Interestingly, these are likely not required for the corresponding nodes' functionalities in the majority of cases.
Instead, they mostly concern the development or maintenance process, both of which are not performed by the typical end-user of the IoT devices but rather the developers of the applications.
This also suggests that a possible mitigation for part of the issue at hand might be to simply remove or disable these parts of the code without breaking their functionality.


The average number of individual nodes per node package and the average number of LOC per node package per conformance case suggest, that smaller, less complex node packages tend to be divergent.
On the other hand, both, the group of absence and divergence node packages show a higher average number of individual nodes per node package.
This could indicate that node packages with a high number of individual nodes have a tendency for non-conformance cases in general.
Intuitively, this observation seems reasonable, given that the increasing complexity and size of the codebase for larger node packages also add mental load for the developer(s) and make it more difficult to keep an overview of the system and all information flows.





\subsection{Risks to Users of the \node Ecosystem}

The additional, ``hidden'' information flows detected in the conformance analysis that are not captured by the specifications could expose sensitive information and allow infiltration of the system.
The risk assessment presented in Section~\ref{subsec:risk_assessment} shows the possible implications of using the analyzed nodes.
A high severity rating is associated with slightly less than a third of the analyzed information flows, a medium and low severity with slightly more than a third each.

As a general observation, the results of the risk assessment indicate a substantial security risk for users of the Node-RED framework.
Without the possibility to gauge the security implications of a specific node, chances are high that users build applications containing nodes with medium or high risk severity.

As shown in Table~\ref{tbl:risk_assessment_results}, roughly one third of all information flows pass messages between nodes, which is the primary intended use and is therefore associated with a low risk severity.
Another large portion of information flows displays information in the terminal or dashboard or logs it.
The corresponding groups of information flow make up 517 of the analyzed information flows, corresponding to 55.1\%.
They are especially interesting because they do not influence the behavior of the developed application, but instead support the development or maintenance process -- in other words: the developer's work.
While logging is important for failure analysis after an incident occurred, and displaying information to the user in real time during operation can be part of the functionality of an application, we estimate that the extent to which these actions are prevalent in the analyzed node packages is not necessary for the node's functionality.
Especially so, when considering that there are dedicated nodes for these use-cases, such as the \textit{debug}-node to display information in the Node-RED dashboard. 
Using a single node or small number of nodes that are made specifically for actions that could have security implications would allow the realization of security mechanisms at a central point, for example, the sanitization of output values or logs.
Instead, our analysis shows that these functionalities are realized with custom implementations in the analyzed nodes.

Other information flows interact with files, external servers, or hardware devices.
It is likely that these are information flows that are fundamental to the implemented functionality of the corresponding nodes, and therefore a security risk that has to be tolerated and mitigated by properly securing them.











 