\section{Related Work}
We defer an extended review of relevant work to \cref{app:relwork}. %Here, we  summarize the context for our paper.



 \paragraph{Quantile based preferences. } Quantile based preferences are well-established in mathematical economics and social choice theory. %\cite{DeGa2019dynamic,DeGa2022static} show that quantile preferences are a more accurate model of real-life behavior of agents in randomized settings over expected utility. 
 %
 Our proposed  valuations are a generalization of \textit{preference set extensions} that lift preferences over individual items to a set of items. The study of preference set extensions has a long-standing history in social choice theory~\citep{BBP04a} and has been applied to hedonic coalition formation games~\citep{CeHa2003computational,CeHa2004stable}, committee selection~\citep{AzMo20a} and multidimensional matchings \citep{HNR2025strategyproof}. %Typically, either the best or worst alternative in the proposed solution dictates the preferences.  %
 %Typical preference set extensions are either {\em best} or {\em worst} which correspond to quantile value $\tau=1$ and $\tau=0$ respectively in our model. 
 We discuss these in detail in \cref{app:relwork}, along with alternate generalizations of preference set extensions. %Among them, one is called the \textit{best set extension} in which the sets are compared based on the best item in each set. One is called the \textit{worst set extension},   in which the sets are compared based on the best item in each set. The best and worst extension correspond to the $\tau=1$ and $\tau=0$ in our model. 

 

\paragraph{Allocating Indivisible Items.} The problem of allocating indivisible items fairly and/or efficiently is very well studied (See \cite{AAB+2022fair} for a survey). Existing literature almost exclusively assumes that aggregated preferences are monotone, very often, additive\citep{CKM+2019unreasonable,ACIW2022fair}. Some also consider arbitrary (not necessarily monotone) valuations \citep{BBB+2024envy,BBPP2024nearly}. Our proposed valuations are non-monotone for most quantiles.  
%
Restricted cardinality allocations have been explored for additive valuations \citep{SHS2023efficient,BiBa2018fair,CaNa2024repeatedly}. %The valuations here are additive.   

%%%%%%%%%%%%%%%%%%%%%%%%%%%%%%%%%%%%%%%%%%%%%%%%%%%%%%%%%%%%%%%%
%%%%%%%%%%%%%%%%%%%%%%%%%%%%%%%%%%%%%%%%%%%%%%%%%%%%%%%%%%%%%%%%
%%%%%%%%%%%%%%%%%%%%%%%%%%%%%%%%%%%%%%%%%%%%%%%%%%%%%%%%%%%%%%%%