\clearpage

\section{Hard Attributes}
% \begin{table*}[htbp]
\newcolumntype{g}{>{\columncolor{green!10}}c}
\newcolumntype{b}{>{\columncolor{blue!10}}c}
\renewcommand{\arraystretch}{1.22} % Adjust row spacing
\resizebox{\textwidth}{!}{
\begin{tabular}{lccccccgcccccb}

\toprule
\multicolumn{1}{c}{\multirow{2}{*}{Model}} & \multirow{2}{*}{BaseModel} & \multicolumn{6}{c}{FollowBench (HSR)}         & \multicolumn{6}{c}{FollowBench (SSR)}                       \\ \cline{3-14} 
\multicolumn{1}{c}{}                       &                            & L1    & L2    & L3    & L4    & L5    & Avg    & L1    & L2    & L3    & L4    & L5  & Avg \\ \midrule

GPT4~\cite{achiam2023gpt}$^{*}$            & GPT                        & 84.7  & 75.6  & 70.8  & 73.9  & 61.9  & 73.4  & 84.7  & 77.0  & 75.3  & 77.0  & 72.3 & 77.2 \\
GPT3.5-turbo$^{*}$                         & GPT                        & 80.3  & 68.0  & 68.6  & 61.1  & 53.2  & 66.2  & 80.3  & 71.2  & 74.2  & 69.6  & 67.1 & 72.5 \\ 

Llama-3.1-8B-Instruct\textsubscript{BASE}  & LLaMA3                     & 82.4  & 76.1  & 71.8  & 62.7 & 54.3 & 69.5  & 82.4  & 78.7  & 78.9  & 74.2  & 68.9 & 76.6 \\
Llama-3.1-8B-Instruct\textsubscript{SFT}     & LLaMA3                     & 82.4 & 78.2 & 70.2  & 60.3  & 54.3  & 69.1 & 82.4  & 80.2 & 78.4 & 74.3 & 69.2 & 76.9 \\
Llama-3.1-8B-Instruct\textsubscript{SFT+random} & LLaMA3                   & 82.4  & 76.6  & 70.7 & 63.4 & 56.5 & 69.9 & 82.4 & 79.0 & 78.9 & 74.1 & 69.7 &  76.8 \\
\bottomrule


\end{tabular}
}
\caption{
The overall performance on FollowBench. Boldface highlights the best results, and underlining indicates the second-best results among models ranging from 7B to 13B parameter sizes. $^{*}$ indicates results are sourced from original benchmarks.
}
\label{tab:followbench}
\end{table*}
\label{sec:hard_attr}

\begin{table*}[htbp]
\centering
\small
\begin{tabular}{l|p{2.6cm}|p{9cm}}
\toprule
\textbf{Instruction Group} & \textbf{Instruction} & \textbf{Description} \\
\midrule
Keywords & Include Keywords & Include keywords \{keyword1\} in your response \\
Keywords & Keyword Frequency & In your response, the word {word} should appear \{N\} times. \\
\midrule
Length Constraints & Number Paragraphs & Your response should contain \{N\} paragraphs. You separate paragraphs using \textbackslash n \textbackslash n \\
Length Constraints & Number Words & Answer with at least / around / at most \{N\} words. \\
Length Constraints & Number Sentences & Answer with at least / around / at most \{N\} sentences. \\
\midrule
Change Cases & All Uppercase & Your entire response should be in English, capital letters only. \\
Change Cases & All Lowercase & Your entire response should be in English, and in all lowercase letters. No capital letters are allowed. \\
\midrule
Start with & Start With & Finish your response with this exact phrase \{end\_phrase\}. No other words should follow this phrase. \\
\bottomrule
\end{tabular}
\caption{The list of 8 verifiable instructions, with brief descriptions. We use these instructions because we think they are either easy to verify or common in real-world applications.}
\label{tab:list-of-verifiable-instruction}
\end{table*}

The hard attributes employed in this study, as detailed in Table \ref{tab:list-of-verifiable-instruction}, comprise a set of verifiable instructions designed to enforce precise, programmatically assessable constraints on text generation. These attributes are categorized into four primary groups: (1) Keywords, which mandate the inclusion or frequency of specific terms (e.g., "Include \{keyword1\}" or "appear \{N\} times"); (2) Length Constraints, governing structural requirements such as paragraph count, word limits, or sentence boundaries; (3) Change Cases, enforcing syntactic rules like all-uppercase or all-lowercase formatting; and (4) Positional Directives, such as starting responses with predefined phrases. Each attribute is selected for its objective verifiability through rule-based checks while also reflecting common real-world application scenarios, such as compliance with stylistic guidelines or technical specifications. By anchoring the evaluation in these deterministic constraints, the framework guarantees rigorous assessment of model adherence to fine-grained requirements, aligning with the dataset's emphasis on combinatorial complexity and practical utility.

\section{Generalization to Unseen Attributes}
We evaluate the models’ ability to generalize to unseen, more challenging attributes, focusing on two types:
\begin{enumerate} \item \textbf{Absolute Position of a Word}: The k-th (k $\le$ 5) word in the text must be A.
\item \textbf{Relative Position Between Two Words}: Word A must appear before word B.
\end{enumerate}
We use text from the FineWeb validation set and extract 50 attributes per document, focusing on these two types of harder attributes. We then evaluate the three models on this benchmark. The results show that the original model achieves a score of 21.56, while auto-reconstruct slightly reduces performance to 20.79. However, incorporating GPO alongside AR improves generalization, yielding a score of 24.05, suggesting that GPO enhances the model’s ability to handle these harder constraints.

% \section{Instruction-following Capabilities}
% \longfei{maybe retest this benchmark}

\section{Dataset Statistics}
\label{appendix:statistics}
% \begin{figure*}[htbp]
%     % 左侧图片
%     \begin{minipage}{0.77\linewidth}  % 调整宽度
%         \centering
%         \includegraphics[width=\linewidth]{images/benchmark_construction.pdf}
%     \end{minipage}%
%     % 间隔
%     \hfill
%     % 右侧表格
%     \begin{minipage}{0.23\linewidth}  % 调整宽度
%         \centering
%         \resizebox{\linewidth}{!}{  % 调整表格至合适的宽度
%             \begin{tabular}{lcc}
%                 \toprule
%                 \textbf{Statistic} & \textbf{Number} \\
%                 \midrule
%                 \rowcolor[HTML]{F2F2F2} 
%                 \textit{Domain Count} &  \\
%                 \midrule
%                 Domain & 103 \\
%                 Requirement & 8 \\
%                 \midrule
%                 \rowcolor[HTML]{F2F2F2} 
%                 \textit{Token Count} &  \\
%                 \midrule
%                 Description & 851.6 $\pm$ 515.2 \\
%                 - Min/Max & [159, 2814] \\
%                 Domain & 1187.2 $\pm$ 1212.1 \\
%                 - Min/Max & [85, 7514] \\
%                 \midrule
%                 \rowcolor[HTML]{F2F2F2} 
%                 \textit{Line Count} &  \\
%                 \midrule
%                 Domain & 75.4 $\pm$ 62.9 \\
%                 - Min/Max & [9, 394] \\
%                 \midrule
%                 \rowcolor[HTML]{F2F2F2} 
%                 \textit{Component Count} &  \\
%                 \midrule
%                 Actions & 4.5 $\pm$ 2.8 \\
%                 - Min/Max & [1, 16] \\
%                 Predicates & 8.1 $\pm$ 4.8 \\
%                 - Min/Max & [1, 25] \\
%                 Types & 1.1 $\pm$ 1.3 \\
%                 - Min/Max & [1, 8] \\
%                 \bottomrule
%             \end{tabular}
%         }
%     \end{minipage}
%     % 公共标题
%     \caption{Dataset construction process (left) and key statistics (right) of the \texttt{\benchmark} dataset.     Dataset construction process including: (a) \textit{Data Acquisition} (\S\ref{sec:data_acquisition}); (b) \textit{Data Filtering and Manual Selection} (\S\ref{sec:data_filtering}); (c) \textit{Data Annotation and Quality Assurance}(\S\ref{sec:data_annotation} and \S\ref{sec:quality_assurance}). Tokens are counted by GPT-2~\cite{openai2019gpt2} tokenizer.}
%     \label{fig:combined}
% \end{figure*}

In this section, we present detailed statistics of our dataset, including a comparison with existing datasets, quality control evaluation, and the composition of our multi-sources subset.


\paragraph{Comparison with Existing Datasets.}
\newcolumntype{g}{>{\columncolor{green!10}}c}
\setlength\tabcolsep{7pt}
\begin{table}[htbp]
\centering
\huge
\newcolumntype{b}{>{\columncolor{blue!10}}c}
\renewcommand{\arraystretch}{1.6}
\resizebox{0.5\textwidth}{!}{

\begin{tabular}{lccccc}

\toprule
\multicolumn{1}{c}{\multirow{2}{*}{Method}} & \multicolumn{4}{c}{Data Quality} &  \\ \cline{2-6} 
\multicolumn{1}{c}{}                       & Nums.       & Cons.    & Avg Attr.      & Synt.    \\ \midrule
IFeval~\cite{zhou2023instruction} & 541  & H & 1.54 & \ding{51} \\
FollowBench~\cite{jiang2023followbench} & 820 & H/S & 3.0 & \ding{51}  \\
CFBench~\cite{zhang2024cfbench} & 1000 & H/S & 4.24 & \ding{55} \\
InFoBench~\cite{qin2024infobench} & 500 & H/S & 4.5 & \ding{55} \\
\our (FineWeb Split) & 6159 & H/S & \textbf{45.9} & \ding{55} \\
\our (Multi-source Split) & 1600 & H/S & \textbf{29.9} & \ding{55} \\
\bottomrule
\end{tabular}%
}
\caption{
  Detailed comparison of relevant works. Ours
represents our dataset construction approach. \textquotesingle Nums.\textquotesingle, \textquotesingle Cons.\textquotesingle, \textquotesingle Avg Attr.\textquotesingle,
and \textquotesingle Synt.\textquotesingle\  denote the number of samples, constraint types, average number of attributes, and whether the data is synthesized.
}


  \label{tab:comparison}
\end{table}

Table~\ref{tab:comparison} provides a comparison between our dataset and several representative constraint-based datasets, including IFeval~\cite{zhou2023instruction}, FollowBench~\cite{jiang2023followbench}, CFBench~\cite{zhang2024cfbench}, and InFoBench~\cite{qin2024infobench}. Our dataset distinguishes itself with a significantly larger number of samples (6,159) and a notably higher average number of attributes per instance (45.9). Unlike prior datasets, which primarily rely on either human annotations or simple constraints, our data features a rich combination of both hard and soft constraints, offering a more challenging and comprehensive benchmark. Importantly, our data is not synthesized, ensuring its alignment with real-world use cases.

\paragraph{Domain Composition in Multi-sources Subset}
\small
\begin{tabular}{p{6.8cm}p{.8cm}p{.8cm}p{.8cm}crr}
\toprule
\multirow{2}{*}{\textbf{Metric}} & \multicolumn{3}{c}{\textbf{Median}} & \multicolumn{3}{c}{\textbf{Statistics}} \\
\cline{2-7}
 & \textbf{Iter. 1} & \textbf{Iter. 2} & \textbf{Iter. 3} & \textbf{Comparison} & \textbf{r} & \textbf{p-value} \\
\midrule
\multirow{3}{*}{UMUX-LITE (SUS)} & \multirow{3}{*}{60.82} & \multirow{3}{*}{66.23} & \multirow{3}{*}{82.48} & 1 vs 2 & 1.429 & 0.279 \\
 &  &  &  & 2 vs 3 & 0.000 & 0.066 \\
 &  &  &  & 1 vs 3 & 0.000 & 0.031* \\
\hline
\multirow{3}{*}{NASA-TLX Score} & \multirow{3}{*}{3.92} & \multirow{3}{*}{2.83} & \multirow{3}{*}{1.92} & 1 vs 2 & 2.041 & 0.500 \\
 &  &  &  & 2 vs 3 & 0.817 & 0.138 \\
 &  &  &  & 1 vs 3 & 0.000 & 0.094 \\
\midrule
\multicolumn{7}{c}{\textbf{Self-Defined Likert Scale Questions}} \\
\midrule
\multirow{3}{*}{\makecell[l]{Iterating on my sketches was easy}} & \multirow{3}{*}{4.0} & \multirow{3}{*}{5.0} & \multirow{3}{*}{6.5} & 1 vs 2 & 3.674 & 0.844 \\
 &  &  &  & 2 vs 3 & 0.000 & 0.039* \\
 &  &  &  & 1 vs 3 & 1.429 & 0.156 \\
\hline
\multirow{3}{*}{\makecell[l]{The sketches encapsulated what I intended to achieve}} & \multirow{3}{*}{5.5} & \multirow{3}{*}{5.5} & \multirow{3}{*}{5.5} & 1 vs 2 & 2.654 & 0.783 \\
 &  &  &  & 2 vs 3 & 0.408 & 0.655 \\
 &  &  &  & 1 vs 3 & 1.414 & 0.688 \\
\hline
\multirow{3}{*}{\makecell[l]{The re-generated code aligned with my intended changes}} & \multirow{3}{*}{4.5} & \multirow{3}{*}{5.0} & \multirow{3}{*}{6.0} & 1 vs 2 & 1.225 & 0.892 \\
 &  &  &  & 2 vs 3 & 0.000 & 0.141 \\
 &  &  &  & 1 vs 3 & 0.707 & 0.063 \\
\hline
\multirow{3}{*}{\makecell[l]{I felt more control over the AI model and generated results}} & \multirow{3}{*}{5.5} & \multirow{3}{*}{5.0} & \multirow{3}{*}{5.5} & 1 vs 2 & 2.236 & 0.786 \\
 &  &  &  & 2 vs 3 & 0.000 & 0.785 \\
 &  &  &  & 1 vs 3 & 0.500 & 1.000 \\
\hline
\multirow{3}{*}{\makecell[l]{I felt more control over the whole code editing process}} & \multirow{3}{*}{5.0} & \multirow{3}{*}{5.0} & \multirow{3}{*}{6.0} & 1 vs 2 & 0.000 & 1.000 \\
 &  &  &  & 2 vs 3 & 0.866 & 0.059 \\
 &  &  &  & 1 vs 3 & 1.000 & 0.102 \\
\bottomrule
\end{tabular}
\begin{figure}[htbp]
    \centering
        \includegraphics[width=.5\textwidth]{figs/rl_distribution.pdf}
    \caption{Proportion of Attributes Across Different Data Domains. The bar chart visualizes the relative contribution of each domain to the multi-sources subset, highlighting a balanced distribution across various sources such as web data, forums, papers, books, and Wikipedia.}
    \vspace{-1em}
    \label{fig:rl_distribution}
\end{figure}
Our multi-sources subset is constructed from a diverse range of data sources, encompassing web data, forums, academic papers, books, and Wikipedia. Figure~\ref{fig:rl_distribution} illustrates the proportion of attributes contributed by each domain, highlighting a balanced distribution across these categories. Table~\ref{tab:rl_domains} further details the exact composition, showing that no single source overwhelmingly dominates, ensuring robustness and variety in downstream tasks.

\paragraph{Quality Control Metrics}
Maintaining data quality is critical for ensuring reliable evaluations. We assess the agreement rate (AR) between human annotators and the final dataset as a key metric. As summarized in Table~\ref{tab:quality_metrics}, the FineWeb subset achieves an AR of 92.3\%, while the multi-sources subset attains 88.7\%. These high agreement rates reflect the robustness of our data curation process, confirming that both subsets align closely with human judgment.
\begin{table}[htbp]
\centering
\resizebox{0.5\textwidth}{!}{
\begin{tabular}{lcc} 
\toprule
\textbf{Metric} & \textbf{FineWeb Subset} & \textbf{Multi-sources Subset} \\
\midrule
Agreement Rate (AR) & 97\% & 96\% \\
\bottomrule
\end{tabular}
}
\caption{Quality Control Metrics}
\label{tab:quality_metrics}
\end{table}


\section{DPO data quality} 
\begin{table}[htbp]
    \centering
    \small
    \begin{tabular}{p{7cm}}
     \toprule
\textbf{High Correlation: } \\
\quad - Thought-provoking narrative with a call to action \\
\quad - Author's Name and Location Identifier: The text begins with the name S. TEITELBAUM followed by a location ST. JOHNS, FL. \\
\quad - Engaging Headline: The title captures the reader's attention by listing 5 Reasons for a specific action. \\
\textbf{Low Correlation: } \\
\quad - AI Leadership: The partnership aims to position Singapore as a leader in AI within healthcare \\
\quad - Focus on Competitive Standards: The passage stresses the competitiveness of FAU’s admissions process \\
\bottomrule
    \end{tabular}
    \caption{Correlation Examples}
    \label{tab:high_correlation_examples}
\end{table}
In this section, we showcase some examples sampled by our global selection strategy.
\paragraph{High Correlation} 
Our attribute correlation modeling step aims to select semantically coherent and mutually reinforcing attributes during GPO training. This process effectively groups attributes that frequently co-occur in natural text, leading to the selection of high-quality attribute combinations. 

\paragraph{Low Similarity}
\begin{table}[]
\caption{}
\label{tab:similarity}
\begin{tabular}{lrrr}
\hline
                   & Main & Cross-user & Cross-doc \\ \hline
Jaccard Similarity &      &            & 0.006     \\ \hline
\end{tabular}
\end{table}
While high correlation ensures that attributes are semantically aligned, it is equally important to maintain attribute diversity to prevent redundancy and overfitting. Our global selection strategy aims to minimize the presence of highly similar attributes within the same prompt. For instance, attributes like \emph{“Engaging Headline”} and \emph{“Attention-Grabbing Title”} convey nearly identical meanings and offer little additional training value when paired together. By prioritizing low-similarity combinations, we encourage the model to generalize across a broader range of attribute expressions, improving its adaptability to diverse prompts.

\begin{figure}[htbp] 
    \centering
        \includegraphics[width=0.5\textwidth]{figs/dpo_score_distribution.pdf}
    \caption{Score distributions of chosen and rejected data.}
    \label{fig:dpo_score_distribution}
\end{figure}
To further illustrate the effectiveness of our sampling strategy in the GPO stage, we present several representative cases selected by our attribute-based sampling approach. These examples demonstrate the diversity and coverage achieved through our strategy, highlighting both common and edge-case attribute combinations. 

\section{Prompts Used in This Study}
We employ three distinct prompts to support different stages of our EFCG pipeline: Decomposition, Judging, and Generation.

\paragraph{Decomposition Instruction.}
\begin{table}[htbp]
    \vspace{-0.5em}
    \centering
    \small
    \begin{tabular}{p{7cm}}
     \toprule
\#\#\# Requirements \\
For the following paragraph, propose attributes that capture its overall characteristics. Focus on what makes this text unique and distinctive, rather than using predefined categories. Your analysis should: \\
- Identify the most prominent and defining features of the text \\ 
- Use clear, specific descriptions rather than vague terms \\
- Base attributes solely on what is explicitly present in the text \\
- Describe each attribute with enough detail to be meaningful \\
Avoid: \\
- Overly broad or generic attributes \\
- Speculative interpretations \\ 
- Attributes not clearly supported by the text \\ 
- Complex or academic jargon \\

Output each attribute on a separate line, separated by a single newline, with no line breaks within each attribute. \\

Now, analyze the following paragraph and summarize its key attributes: \\

\#\#\# Text \\
\{text\}

\#\#\# Attributes  \\
\bottomrule
    \end{tabular}
    \caption{Decompose Prompt}
    \label{tab:decompose_prompt}
\end{table}
This prompt is used to extract a set of soft attributes from a given text. The goal is to decompose the paragraph into its most defining characteristics, capturing both stylistic and semantic elements. Models are instructed to focus on identifying specific, explicit features of the text rather than relying on generic descriptions or subjective interpretations. Attributes must reflect the unique aspects of the text and be grounded in the content. 

\paragraph{Judge Instruction.}
\begin{table}[htbp]
    \centering
    \small
    \begin{tabular}{p{7cm}}
     \toprule
You are a binary evaluator. Given a text and several attributes, determine if the text fulfills each attribute. \\

Your task is simple: \\
- Score 0 if the text does NOT fulfill the attribute or the attribute is not directly mentioned \\
- Score 1 if and only if the text directly fulfills the attribute \\

Text to evaluate: \\
\{text\} \\

Attributes to evaluate: \\
\{attributes\} \\

Provide exactly \{num\_attributes\} scores, one per line, using this format: \\
Score: 0 or 1 \\

- Scores should correspond to attributes in order \\
- Only provide the scores, no additional explanation \\
\bottomrule
    \end{tabular}
    \caption{Judge Prompt}
    \label{tab:judge_prompt}
\end{table}
This prompt serves as a binary evaluation guideline to determine whether a generated text satisfies a given set of attributes. Evaluators are asked to assess each attribute independently, assigning a score of 1 if the text explicitly fulfills the attribute and 0 otherwise. The evaluation is strict, requiring the text to directly align with the specified attribute for a positive score.

\paragraph{Generation Instruction.}
\begin{table}[t]
\vspace{-10pt}
\caption{We evaluate Llama-2-7B on MT-Bench, GSM8K, and HumanEval for dialogue, math, and coding.}
\label{tab:gen}
\scriptsize
\centering
\resizebox{0.8\linewidth}{!}{%
\begin{tabular}{@{}lcccccc@{}}
\toprule
\textbf{Method} & \textbf{MT-Bench} & \textbf{GSM8K} & \textbf{HumanEval} \\
\midrule
\textbf{Full FT} & 5.56 & 59.36 & 35.31  \\
\midrule
\multicolumn{4}{l}{\textit{Single LoRA Methods}} \\
\midrule
\textbf{LoRA} & 5.61 & 52.84 & 21.34   \\
\textbf{DoRA} & 5.97 & 54.59 & 19.75  \\
\textbf{PiSSA} & 5.30 & 55.42 & 19.52   \\
\textbf{MiLoRA} & 5.23 & 54.44 & 19.51  \\
% \textbf{LoRAPro} & 5.86 & 57.47 & 22.76   \\
\midrule
\multicolumn{4}{l}{\textit{LoRA MoE Methods}} \\
\midrule
\textbf{MoLoRA} & 5.84 & 56.63 & {24.83}  \\
% \textbf{AdaMoLE} & - & 57.39 & -  \\
\textbf{HydraLoRA} & 5.82 & {57.39} & 24.21  \\
% \rowcolor{mygreen!50}\textbf{Balance} & - & 55.65 & -  \\
% \rowcolor{mygreen!50}\textbf{BalanceM} & - & \textbf{57.92} & -  \\
\rowcolor{mygreen!50}\textbf{GOAT} & \textbf{6.01} & \textbf{60.20} & \textbf{25.61}  \\
\bottomrule
\end{tabular}
}
\vspace{-10pt}
\end{table}
This prompt is used to instruct the language model to generate a piece of text that aligns with a provided set of hard constraints and soft attributes. Hard attributes typically represent structural or factual constraints (e.g., budget, schedule), while soft attributes reflect stylistic or semantic preferences (e.g., tone, vividness). The model is guided to generate text that adheres to as many of these attributes as possible, balancing the satisfaction of both hard and soft constraints.

%\section{More Attention Flow Visualization}

\section{The Complete Case Study}
\label{appendix:case_study}

\begin{table*}[htbp]
    \centering
    \small
    \begin{tabular}{p{14cm}}
     \toprule
\#\#\#  Objective: \\
Generate a 5-day family travel itinerantry that satisfies all specified requirements while adhering to highly fine-grained constraints. The generated itinerary should balance real-time adaptability, strict hard attributes, and semantic soft attributes. \\

\#\#\# User Profile: \\
 - Travelers: 2 adults + 1 child (age 8) \\
 - Budget: $<=$ \$300/day (total \$1,500 for the trip) \\
 - Activity Balance: 70\% educational/cultural experiences, 20\% relaxation, 10\% family-friendly shopping. \\

\#\#\# Hard Attributes: \\
- Activity Scheduling: \\
\quad- Each activity must have a defined start and end time, ensuring there is no overlap between activities. \\
\quad- A break period from 13:00-14:30 is mandatory daily. \\
\quad- Each activity must fit within a 2-hour window unless otherwise specified. \\

- Budget Requirements: \\
\quad- Each day’s total cost (including transportation, food, and activities) must not exceed \$300. \\
\quad- Transportation is limited to metro and walking only, with a maximum of 3 metro rides per day. \\

- Location Constraints: \\
\quad- Must-visit locations: City Zoo (Day 1) and Science Museum (Day 3). \\
\quad- Activities must occur in geographically adjacent areas to minimize walking distance. \\

- Keyword Requirements: \\
\quad- Each day’s description must include specific keywords. For example: \\
\quad- Day 1: “wildlife,” “exploration,” and “interactive learning.” \\
\quad- Day 3: “science,” “innovation,” and “hands-on exhibits.” \\

- Structure Constraints: \\
\quad- Each day’s itinerary must consist of 4 sections: \\
\quad\quad- Morning activity \\
\quad\quad- Break/lunch period \\ 
\quad\quad- Afternoon activity \\
\quad\quad- Evening summary (limited to 50 words) \\

\#\#\# Soft Attributes \\
- Tone and Emotion: \\ 
\quad- Day 1: Use a tone that conveys “excitement and discovery.” \\ 
\quad- Day 3: Use a tone that conveys “curiosity and wonder.” \\
- Language Style: \\ 
\quad- Use descriptive, vivid, and family-friendly language throughout. \\
\quad- Include at least one metaphor or simile per day (e.g., "The Science Museum felt like stepping into the future!"). \\
- Visual Details: \\
\quad- Each activity must include specific sensory details (e.g., "the bright colors of the parrots at the zoo" or "the tinkling sound of water fountains at the park").

- Adaptive Adjustments (Real-time Constraints): \\
\quad- Weather Sensitivity: \\
\quad\quad- If the rain forecast exceeds 60\%, replace outdoor activities with indoor alternatives while keeping the overall tone and keywords intact. \\ 
\quad- Physical Endurance: \\
\quad\quad- If a day’s total walking distance exceeds 10 kilometers, the next day’s activities must reduce walking by 30\%. \\
\quad- Health Responsiveness: \\
\quad\quad- If a health-related issue arises (e.g., fatigue or illness), adjust the itinerary dynamically to: \\
\quad\quad- Reduce activity duration to half. \\ 
\quad\quad- Substitute the activity with a more relaxing or passive option. \\
\bottomrule
    \end{tabular}
    \caption{The complete travel planner case study.}
    \label{tab:travel_planner_case}
\end{table*}
The travel planner case study exemplifies the practical usefulness of EFCG in handling complex, multi-faceted requirements. As shown in Table~\ref{tab:travel_planner_case}, generating a 5-day travel itinerary involves satisfying a diverse set of hard attributes (e.g., budget limits, time scheduling, location constraints) alongside soft attributes (e.g., tone, emotion, visual details), while also adapting to real-time factors like weather and physical endurance. Such a task necessitates precise control over both hard and soft constraints, making it a natural testbed for evaluating EFCG systems.
% \longfei{showcase the result later}



