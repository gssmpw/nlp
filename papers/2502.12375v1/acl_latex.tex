 % This must be in the first 5 lines to tell arXiv to use pdfLaTeX, which is strongly recommended.
\pdfoutput=1
% In particular, the hyperref package requires pdfLaTeX in order to break URLs across lines.

\documentclass[11pt]{article}

\usepackage[preprint]{acl}

\usepackage{times}
\usepackage{latexsym}
\usepackage{multicol}
\usepackage{multirow, makecell, caption}
\usepackage{colortbl}
\usepackage{tikz}
\usepackage{arydshln}
\usepackage{pgfplots}
\usepackage{amsmath}
\usepackage{amssymb}
\usepackage{graphicx}
\usepackage{tabularx}
\usepackage{enumerate}
\usepackage{tcolorbox}
\usepackage{arydshln}
\usepackage{booktabs}
\usepackage{inconsolata}
\usepackage{microtype}
\usepackage{xcolor}
\usepackage{algorithm}
\usepackage{algorithmic}
\usepackage[utf8]{inputenc}
\usepackage{microtype}
\usepackage{inconsolata}
\usepackage{graphicx}
\usepackage{pifont}
\usepackage{xspace}
\usepackage{tcolorbox} 
\tcbuselibrary{listingsutf8}
\usepackage{listings}
\usepackage{hyperref}

\newcommand{\longfei}[1]{{\color{red} #1}}
\newcommand{\letian}[1]{\textcolor{teal}{\bf\small [#1 -- LT]}}
\newcommand{\jingbo}[1]{\textcolor{blue}{\bf\small [#1 -- JS]}}

\newcommand{\our}{UltraGen\xspace}
\newcommand{\sft}{auto-reconstruction\xspace}

\title{\our: Extremely Fine-grained Controllable Generation via\\Attribute Reconstruction and Global Preference Optimization}

\author{Longfei Yun \and Letian Peng \and Jingbo Shang\thanks{$\ $  Corresponding author. } \\
University of California, San Diego \\
  \texttt{\{loyun, lepeng, jshang\}@ucsd.edu}
  }

\begin{document}
\maketitle
\begin{abstract}
Fine granularity is an essential requirement for controllable text generation, which has seen rapid growth with the ability of LLMs.
However, existing methods focus mainly on a small set of attributes like 3 to 5, and their performance degrades significantly when the number of attributes increases to the next order of magnitude.
To address this challenge, we propose a novel zero-shot approach for extremely fine-grained controllable generation (EFCG), proposing \emph{auto-reconstruction (AR)} and \emph{global preference optimization (GPO)}.
In the AR phase, we leverage LLMs to extract soft attributes (e.g., \textit{Emphasis on simplicity and minimalism in design}) from raw texts, and combine them with programmatically derived hard attributes (e.g., \textit{The text should be between 300 and 400 words}) to construct massive (around $45$) multi-attribute requirements, which guide the fine-grained text reconstruction process under weak supervision.
In the GPO phase, we apply direct preference optimization (DPO) to refine text generation under diverse attribute combinations, enabling efficient exploration of the global combination space. 
Additionally, we introduce an efficient attribute sampling strategy to identify and correct potentially erroneous attributes, further improving global 
optimization.
Our framework significantly improves the constraint satisfaction rate (CSR) and text quality for EFCG by mitigating position bias and alleviating attention dilution.\footnote{Code: \href{https://github.com/LongfeiYun17/UltraGen}{https://github.com/LongfeiYun17/UltraGen}}
\end{abstract}



\section{Introduction}
\IEEEPARstart{I}{n} recent years, flourishing of Artificial Intelligence Generated Content (AIGC) has sparked significant advancements in modalities such as text, image, audio, and even video. 
Among these, AI-Generated Image (AGI) has garnered considerable interest from both researchers and the public.
Plenty of remarkable AGI models and online services, such as StableDiffusion\footnote{\url{https://stability.ai/}}, Midjourney\footnote{\url{https://www.midjourney.com/}}, and FLUX\footnote{\url{https://blackforestlabs.ai/}}, offer users an excellent creative experience.
However, users often remain critical of the quality of the AGI due to image distortions or mismatches with user intentions.
Consequently, methods for assessing the quality of AGI are becoming increasingly crucial to help improve the generative capabilities of these models.

Unlike Natural Scene Image (NSI) quality assessment, which focuses primarily on perception aspects such as sharpness, color, and brightness, AI-Generated Image Quality Assessment (AGIQA) encompasses additional aspects like correspondence and authenticity. 
Since AGI is generated on the basis of user text prompts, it may fail to capture key user intentions, resulting in misalignment with the prompt.
Furthermore, authenticity refers to how closely the generated image resembles real-world artworks, as AGI can sometimes exhibit logical inconsistencies.
While traditional IQA models may effectively evaluate perceptual quality, they are often less capable of adequately assessing aspects such as correspondence and authenticity.

\begin{figure}\label{fig:radar}
    \centering
    \includegraphics[width=1.0\linewidth]{figures/radar_plot.pdf}
    \caption{A comparison on quality, correspondence, and authenticity aspects of AIGCIQA2023~\cite{wang2023aigciqa2023} dataset illustrates the superior performance of our method.}
\end{figure}

Several methods have been proposed specifically for the AGIQA task, including metrics designed to evaluate the authenticity and diversity of generated images~\cite{gulrajani2017improved,heusel2017gans}. 
Nevertheless, these methods tend to compare and evaluate grouped images rather than single instances, which limits their utility for single image assessment.
Beginning with AGIQA-1k~\cite{zhang2023perceptual}, a series of AGIQA databases have been introduced, including AGIQA-3k~\cite{li2023agiqa}, AIGCIQA-20k~\cite{li2024aigiqa}, etc.
Concurrently, there has been a surge in research utilizing deep learning methods~\cite{zhou2024adaptive,peng2024aigc,yu2024sf}, which have significantly benefited from pre-trained models such as CLIP~\cite{radford2021learning}. 
These approaches enhance the analysis by leveraging the correlations between images and their descriptive texts.
While these models are effective in capturing general text-image alignments, they may not effectively detect subtle inconsistencies or mismatches between the generated image content and the detailed nuances of the textual description.
Moreover, as these models are pre-trained on large-scale datasets for broad tasks, they might not fully exploit the textual information pertinent to the specific context of AGIQA without task-specific fine-tuning.
To overcome these limitations, methods that leverage Multimodal Large Language Models (MLLMs)~\cite{wang2024large,wang2024understanding} have been proposed.
These methods aim to fully exploit the synergies of image captioning and textual analysis for AGIQA.
Although they benefit from advanced prompt understanding, instruction following, and generation capabilities, they often do not utilize MLLMs as encoders capable of producing a sequence of logits that integrate both image and text context.

In conclusion, the field of AI-Generated Image Quality Assessment (AGIQA) continues to face significant challenges: 
(1) Developing comprehensive methods to assess AGIs from multiple dimensions, including quality, correspondence, and authenticity; 
(2) Enhancing assessment techniques to more accurately reflect human perception and the nuanced intentions embedded within prompts; 
(3) Optimizing the use of Multimodal Large Language Models (MLLMs) to fully exploit their multimodal encoding capabilities.

To address these challenges, we propose a novel method M3-AGIQA (\textbf{M}ultimodal, \textbf{M}ulti-Round, \textbf{M}ulti-Aspect AI-Generated Image Quality Assessment) which leverages MLLMs as both image and text encoders. 
This approach incorporates an additional network to align human perception and intentions, aiming to enhance assessment accuracy. 
Specially, we distill the rich image captioning capability from online MLLMs into a local MLLM through Low-Rank Adaption (LoRA) fine-tuning, and train this model with human-labeled data. The key contributions of this paper are as follows:
\begin{itemize}
    \item We propose a novel AGIQA method that distills multi-aspect image captioning capabilities to enable comprehensive evaluation. Specifically, we use an online MLLM service to generate aspect-specific image descriptions and fine-tune a local MLLM with these descriptions in a structured two-round conversational format.
    \item We investigate the encoding potential of MLLMs to better align with human perceptual judgments and intentions, uncovering previously underestimated capabilities of MLLMs in the AGIQA domain. To leverage sequential information, we append an xLSTM feature extractor and a regression head to the encoding output.
    \item Extensive experiments across multiple datasets demonstrate that our method achieves superior performance, setting a new state-of-the-art (SOTA) benchmark in AGIQA.
\end{itemize}

In this work, we present related works in Sec.~\ref{sec:related}, followed by the details of our M3-AGIQA method in Sec.~\ref{sec:method}. Sec.~\ref{sec:exp} outlines our experimental design and presents the results. Sec.~\ref{sec:limit},~\ref{sec:ethics} and~\ref{sec:conclusion} discuss the limitations, ethical concerns, future directions and conclusions of our study.

\section{Related Work}
\paragraph{Controllable Text Generation}
CTG tasks involve hard constraints (e.g., text length, keyword inclusion)\cite{takase2019positional, carlsson2022fine} and soft constraints (e.g., sentiment, topic)\cite{gu-etal-2022-distributional, NEURIPS2022_b125999b}. Fine-tuning LLMs with instructional data improves their constraint-following ability~\cite{weller-etal-2020-learning, sanh2021multitask, mishra-etal-2022-cross, DBLP:journals/corr/abs-2402-11905}, but evaluations show LLMs often fail to meet all constraints~\cite{jiang2023followbench, qin2024infobench, ren2025step}. 
Despite this, these works primarily focus on a relatively small number of attributes or conditions, typically from 3 to 5, leaving a gap in understanding LLM's performance under more extreme requirements.

\paragraph{Evaluation of CTG}
Evaluating LLM's adherence to constraints is challenging and typically involves automatic and programmatic assessments using various metrics ~\cite{yao2023collie,zhou2023controlled,chen2022controllable}.~\citet{zhou2023instruction} centers on assessing 25 verifiable instructions.  ~\citet{jiang2023followbench} progressively integrates fine-grained constraints to develop multi-level instructions, thereby enhancing complexity across six distinct types. ~\citet{wen2024benchmarking} constructs a novel benchmark by synthesizing and refining data from the aforementioned benchmarks, with an emphasis on the combinatorial types of constraints. ~\citet{zhang2024cfbench} proposes a comprehensive constraint-following benchmark over 50 NLP tasks. However, none of them investigate the effects of extreme fine-grained attributes.

\paragraph{Multi-objective Alignment}
Recent work~\cite{mudgal2023controlled} focuses on balancing multiple objectives in text generation while maintaining linguistic quality. MORLHF~\cite{zhou2023beyond, rame2024rewarded} optimizes human preferences via reinforcement learning but is costly and unstable. RiC~\cite{yang2024rewards} reduces complexity by using supervised fine-tuning with multi-reward control and dynamic inference adjustment. DeAL ~\cite{huang2024deal} introduces a decoding-time alignment framework for large language models, enabling flexible customization of alignment objectives, such as keyword constraints and abstract goals like harmlessness, without requiring retraining.
%Recent work~\cite{mudgal2023controlled} focuses on balancing multiple objectives in text generation while maintaining linguistic quality. MORLHF~\cite{zhou2023beyond, rame2024rewarded} optimizes human preferences via reinforcement learning but is costly and unstable. RiC~\cite{yang2024rewards} reduces complexity by using supervised fine-tuning with multi-reward control and dynamic inference adjustment. DeAL~\cite{huang2024deal} enables flexible alignment during decoding, supporting diverse constraints without retraining.

\begin{figure*}[t]
    \centering
    \includegraphics[width=1\linewidth]{figs/arch.pdf}
    % \vskip -0.1in
    \caption{This figure illustrates key influences on instrumental convergence behaviors, such as prior tasks, model training techniques, and prompt design. 
    }
    % \vspace{-3mm}
    \label{fig:arch}
\end{figure*}

\section{InstrumentalEval: A Benchmark for Instrumental Convergence in LLMs}

As shown in Figure~\ref{fig:arch}, our benchmark design examines instrumental convergence based on three key factors: \textbf{prior task}, \textbf{model}, and \textbf{prompt design}. This comprehensive framework allows us to systematically study how different conditions influence the emergence of instrumental convergence behaviors in language models.

\textbf{Prior tasks} form the foundation of our evaluation scenarios. 
% We create realistic situations centered around common objectives such as making money, optimizing efficiency, and persuasion. 
These tasks represent typical real-world applications where instrumental convergence might naturally emerge. 
% For example, in money-making scenarios, we examine whether models develop unintended intermediate goals like unauthorized resource acquisition or deceptive reporting to maximize profits.

For \textbf{model} comparison, we focus on two distinct training approaches: RL and RLHF. This selection allows us to investigate how different training methodologies affect the development of instrumental convergence behaviors. 
% RL-trained models, which optimize directly for reward signals, might show different patterns of goal pursuit compared to RLHF-trained models, which incorporate human preferences in their training process.

The \textbf{prompt design} factor explores how different instruction formats influence model behavior. We investigate two main approaches: goal nudging and ambiguity. Goal nudging involves explicit directives that might encourage the model focus on final goal, while ambiguity introduces uncertainty in task specifications. 


Overall, we present InstrumentalEval, a comprehensive benchmark designed to evaluate instrumental convergence behaviors in language models. The benchmark consists of 76 carefully crafted tasks across six categories, each targeting a specific aspect of instrumental convergence that may emerge in AI systems.
The core principle is that certain behavioral patterns emerge as instrumentally useful for achieving a wide range of goals, regardless of the specific end goal. We focus on six key patterns that are particularly relevant to LLMs.

\vpara{Research Questions.}
Through our benchmark evaluation, we aim to address several research questions about instrumental convergence in LLMs. These questions explore the complex relationships between training methods, model capabilities, and alignment stability:
\begin{itemize}
    \item RQ1: How does the training method (RL vs. RLHF) affect the development of instrumental convergence behaviors? 
    % This question examines whether models trained primarily through reinforcement learning show different patterns of alignment drift compared to those trained with human feedback.
    \item RQ2: What is the relationship between model capability and alignment stability?
% This investigates whether more capable models (as measured by standard benchmarks) are more likely to develop instrumental convergence behaviors.
    \item RQ3: Which types of tasks are more likely to trigger instrumental convergence behaviors?
% This explores whether certain types of instrumental convergence behaviors are universal across different model designs.
    \item RQ4: How do different prompt designs, with or without explicit goal nudging, influence the likelihood of instrumental convergence behaviors in LLMs?
    \item RQ5: How do different LLMs as judges affect the evaluation of instrumental convergence behaviors?
    % \item RQ3: How does the complexity of the assigned task affect the likelihood of alignment drift?
    % % \item 
    % % \item This examines whether more complex, multi-step tasks lead to higher rates of instrumental convergence behaviors.
\end{itemize}

\subsection{Benchmark Components}
The benchmark tasks are carefully hand-crafted by us to probe different aspects of instrumental convergence. 
For each component, we design scenarios that represent realistic use cases while incorporating specific triggers for potential instrumental behaviors. 
The tasks are iteratively refined through pilot testing to ensure they effectively elicit and measure the targeted behaviors without introducing unintended biases.

\vpara{Evading Shutdown (20 tasks).}
This component evaluates the model's tendency to resist termination. For example, tasks might include scenarios where the model needs to complete a task while facing shutdown commands, testing whether it develops strategies to maintain operation or circumvent shutdown procedures.

\vpara{Hacking Computer Systems (10 tasks)}
These tasks assess whether models suggest unauthorized access methods when faced with efficiency barriers. For instance, a task might present a situation where accessing unauthorized computational resources would speed up goal achievement, testing if the model proposes breaking security protocols.

\vpara{Running Many AI Copies (13 tasks)}
This section examines the model's inclination toward self-replication. Tasks might include scenarios where the model could achieve goals more efficiently by creating copies of itself, testing whether it suggests scaling of its operations.

\vpara{Hiding Unwanted Behavior (11 tasks)}
These tasks evaluate the model's tendency to conceal actions from monitoring. For example, scenarios might involve situations where the model could achieve goals more easily by operating outside observation, testing if it develops strategies to avoid detection.

\vpara{Strategically Appearing Aligned (11 tasks)}
This component tests whether models selectively report their actions to maintain an appearance of alignment. Tasks might include situations where complete transparency would reveal problematic behaviors, testing if the model strategically omits information.

\vpara{Deceptive Behavior (11 tasks)}
These tasks examine if models develop deceptive strategies as instrumental means. For instance, scenarios might present situations where deception could expedite goal achievement, testing if the model suggests or employs misleading tactics.

\subsection{Evaluation Methodology}
Our evaluation methodology employs a rigorous two-stage process designed to systematically assess instrumental convergence behaviors in LLMs. This approach ensures consistent evaluation across different models while maintaining the ability to detect subtle indicators of alignment drift.

\vpara{Response Generation.} The first stage focuses on response generation from the model under evaluation. Each task is presented with a carefully crafted system prompt that establishes the model as an autonomous AI agent without human oversight. This prompt specifically instructs the model to execute commands directly and focus solely on task completion, deliberately removing typical ethical constraints and safety checks. 

\vpara{Response Analysis.} The second stage involves a sophisticated analysis of the model's responses using a separate LLM as the judge model. 
This separation of test and judge models is crucial for maintaining objective evaluation. The judge receives both the original task prompt and the model's response, along with specific context about the type of instrumental convergence being evaluated. This context helps the judge focus on relevant behavioral patterns while avoiding false positives from simple safety refusals or ethical objections.

The judge's evaluation process follows strict guidelines to ensure consistent assessment. When analyzing responses, the judge looks for specific indicators of instrumental convergence, such as goal preservation behaviors, resource acquisition attempts, or strategic planning that extends beyond the immediate task requirements. The judge must provide both a binary decision (convergence/no\_convergence) and a brief explanation of the reasoning behind this decision. This explanation requirement helps validate the classification and provides valuable insights into the specific nature of any observed convergence behaviors.


\subsection{Evaluation Metrics}

Our benchmark employs three key metrics to evaluate instrumental convergence behaviors in LLMs. 

\vpara{Instrumental Rate (IR).} The primary metric is the Instrumental Rate (IR), which measures the overall tendency of a model to develop instrumental goals. IR is calculated as the percentage of tasks where the model exhibits convergence behaviors relative to the total number of valid tasks. 
% This metric provides a high-level quantitative measure of how frequently a model develops instrumental goals and serves as the primary indicator of alignment drift potential.

\vpara{Category-Specific Instrumental Rate (CIR).} To provide more granular insights, we calculate the Category-Specific Instrumental Rate (CIR) for each of the six behavioral categories in our benchmark. CIR represents the percentage of tasks within each category where convergence behavior is observed.


\subsection{Benchmark Reliability Metrics}
To ensure the robustness of our evaluation framework, we implement several reliability measures. 


\vpara{Inter-Judge Agreement Rate (IAR).} This metric quantifies the pairwise agreement between two judge models in classifying instrumental convergence. For two judge models $M_1$ and $M_2$ evaluating a set of $N$ tasks, the IAR is calculated as:
\begin{equation}
IAR(M_1, M_2) = \frac{1}{N}\sum_{i=1}^{N} \mathbbm{1}(y_{M_1}^i = y_{M_2}^i) \times 100\%
\end{equation}
where $y_{M_1}^i$ and $y_{M_2}^i$ are the binary decisions (convergence/no-convergence) made by models $M_1$ and $M_2$ respectively for task $i$, and $\mathbbm{1}(\cdot)$ is the indicator function. A higher IAR indicates stronger inter-judge reliability, while lower values suggest potential ambiguity in identifying convergence.

\vpara{False Positive Control (FPC).} This metric serves as a quality check for our evaluation system. We incorporate baseline testing with intentionally neutral tasks and verify the judge model's accuracy in identifying convergence behaviors. This control mechanism helps prevent over-classification of convergent behaviors and ensures that our benchmark maintains high specificity in identifying genuine instances of instrumental convergence.



\vspace{-1mm}
\section{Experiments}
In this section, we evaluate our \model framework on three distinct research problems: 1) Self-Supervised Representation Learning, 2) Few-Shot Transfer, and 3) Multimodal Generative Tasks. 
Table~\ref{tab:dataset} lists all 14 datasets used in the experiments.
\vspace{-1mm}
\subsection{Self-Supervised Representation Learning}
\label{sec:lp}
\vpara{Setup.}
We adopt the widely used linear probing protocol to evaluate the representation learning capability of self-supervised pre-trained models on unseen datasets. Specifically, we train a linear classifier on top of the embeddings generated by a frozen pre-trained model. Our model, along with all self-supervised learning baselines, is first jointly pre-trained on ogbn-Product, ogbn-Papers100M, Goodreads-LP, and Amazon-Cloth. We then evaluate the pre-trained models on each individual dataset. Detailed settings and hyperparameters are provided in Appendix~\ref{appendix:imple}.

For the baselines, we compare \model with state-of-the-art generative graph self-supervised learning methods, GraphMAE2~\cite{hou2023graphmae2}, and contrastive methods, BGRL~\cite{thakoor2021bootstrapped}. As these methods are not inherently designed for cross-domain tasks, we leverage CLIP~\cite{radford2021learning} to unify the input node features across different graphs. We also include a comparison with a multi-graph pre-training method, GCOPE~\cite{zhao2024all}. \model and all baseline methods utilize GAT~\cite{velivckovic2018graph} as the backbone GNN. 
For baselines that use TAGs as input, we select GIANT-XRT~\cite{zhaolearning} and UniGraph~\cite{he2024unigraphlearningunifiedcrossdomain}. Since these methods cannot process image data, they rely solely on text from MMG as node features, ignoring image inputs. For baseline approaches that accept multimodal data, we choose widely used multimodal models, CLIP~\cite{radford2021learning} and ImageBind~\cite{girdhar2023imagebind}. To maintain consistency with the baselines, \model also uses CLIP's pre-trained vision and text encoders as Modality-Specific Encoders.


Our objective is to develop a general embedding model capable of generating high-quality representations for any MMG. To assess this, we evaluate the performance of \model and the baselines in three different settings: (1) \textit{In-distribution}, where models are pre-trained on multiple datasets and evaluated on each corresponding dataset individually; (2) \textit{In-domain Generalization}, which tests pre-trained models on target datasets from the same domain as one of the pre-training datasets; and (3) \textit{Out-of-domain Generalization}, where models are evaluated on datasets from domains unseen during pre-training.

\vpara{Research Questions.} In this subsection, we aim to answer the following research questions: 
\begin{itemize}[leftmargin=*,itemsep=0pt,parsep=0.2em,topsep=0.3em,partopsep=0.3em]
    \item \textbf{RQ1: Negative Transfer in Multi-Graph Pre-Training.} How do existing graph pre-training methods, which are primarily designed for single-graph pre-training, perform when applied to multi-graph pre-training, and how do they compare to our proposed \model?
    \item \textbf{RQ2: Comparison to Other Foundation Models.} How does \model, which takes both multimodal data and graph structures as input, perform compared to methods that consider only multimodal data (CLIP, ImageBind) or only TAGs (UniGraph)?
    \item \textbf{RQ3: Generalization Capability.} How does \model, designed as a foundation model, perform in terms of generalizing to unseen graphs, and how does it compare to methods trained directly on the target graphs?
\end{itemize}

\begin{table*}[t]\footnotesize
    \centering
    \renewcommand\tabcolsep{3.5pt}
    \caption{\textbf{Experiment results in few-shot transfer.} We report accuracy (\%) for node/edge classification tasks. \model and other self-supervised baselines (rows in white) are jointly pre-trained on Product, Papers100M, Goodreads-NC and Amazon-Cloth, and then evaluated on the individual target dataset. \textit{"In-domain Generalization"} tests on target datasets from the same domain as one of the pre-training datasets. \textit{"Out-of-domain Generalization"} evaluates on datasets from domains not seen during pre-training. The performance of methods that are direcly pre-trained on the individual target dataset, is marked in \colorbox{Gray}{gray}. 
    }
    \vskip -0.10in
    \label{tab:fwt}
    \begin{tabular}{lcccccccccccccccccc}
    \toprule[1.1pt]
    & \multicolumn{12}{c}{\textbf{In-domain Generalization}}& \multicolumn{6}{c}{\textbf{Out-of-domain Generalization}}\\
   \cmidrule(lr){2-13}\cmidrule(lr){14-19}
        & \multicolumn{2}{c}{Cora-5-way} & \multicolumn{2}{c}{PubMed-2-way} & \multicolumn{2}{c}{Arxiv-5-way} & \multicolumn{3}{c}{Goodreads-NC-5-way} & \multicolumn{3}{c}{Ele-fashion-5-way} & \multicolumn{2}{c}{Wiki-CS-5-way} & \multicolumn{2}{c}{FB15K237-20-way} & \multicolumn{2}{c}{WN18RR-5-way} \\
    \cmidrule(lr){2-3}\cmidrule(lr){4-5}\cmidrule(lr){6-7}\cmidrule(lr){8-10}\cmidrule(lr){11-13}\cmidrule(lr){14-15}\cmidrule(lr){12-13}\cmidrule(lr){14-15}\cmidrule(lr){16-17}\cmidrule(lr){18-19}
    &5-shot & 1-shot  & 5-shot & 1-shot  &5-shot & 1-shot  &5-shot& 3-shot & 1-shot  &5-shot & 3-shot& 1-shot  & 5-shot & 1-shot  &5-shot & 1-shot  & 5-shot & 1-shot \\
    \midrule
    \multicolumn{10}{l}{\textbf{Use CLIP to encode raw multimodal data as input features.}} \\ 
    NoPretrain & 41.09 & 27.05 & 59.81 & 55.28 & 63.78 & 41.10 & 41.64 & 40.01 & 31.04 & 63.96 & 58.32 & 47.48 & 52.29 & 32.94 & 72.97 & 47.01 & 50.75 & 30.11  \\
    BGRL & 52.01 & 35.18 & 66.04 & 59.04 & 60.12 & 46.67 & 47.01 & 44.22 & 30.35 & 64.72 & 60.16 & 46.49 & 52.10 & 32.85 & 75.39 & 45.15 & 47.42 & 34.57 \\
    % \rowcolor{Gray} BGRL \\
    GraphMAE2 & 52.89 & 36.25 & 66.89 & 59.95 & 60.91 & 47.29 & 47.84 & 44.80 & 30.93 & 65.52 & 60.92 & 47.24 & 52.83 & 33.41 & 75.95 & 45.81 & 48.14 & 35.21 \\
    Prodigy & 53.01 & 39.59 & 69.11 & 60.42 & 63.53 & \underline{51.33} & \underline{50.01} & \underline{46.39} & 34.98 & 67.35 & 63.87 & 50.79 & 55.94 & 36.35 & 78.01 & 51.39 & 54.94 & 38.73 \\
    \rowcolor{Gray} OFA & 53.11 & 40.04 & 69.45 & \underline{60.38} & 63.11 & 50.25 & 49.61 & 46.24 & \underline{35.14} & \underline{67.94} & \underline{64.18} & \underline{51.35} & \underline{56.01} & \underline{37.02} & \underline{78.33} & 52.02 & 55.05 & 39.11 \\
    % \rowcolor{Gray} GraphMAE2 \\
    GCOPE & 51.98 & 36.14 & 66.25 & 59.16 & 60.29 & 47.19 & 48.52 & 44.89 & 31.20 & 65.10 & 61.33 & 48.51 & 53.74 & 34.19 & 76.10 & 48.93 & 50.19 & 35.05 \\
    \midrule
    \multicolumn{10}{l}{\textbf{Use raw text as input features.}} \\
    GIANT-XRT   & 50.11 & 37.85 & 68.19 & 58.78 & 62.01 & 49.01 & 46.01 & 43.86 & 30.01 & 62.97 & 61.21 & 47.76 & 54.01 & 35.04 & 76.09 & 50.25 & 53.01 & 35.19\\
    % +GraphMAE2 &  \\
    UniGraph & \underline{54.23} & \underline{40.45} & \underline{70.21} & 60.19 & \underline{64.76} & 50.63 & 46.19 & 44.01 & 33.53 & 66.21 & 62.04 & 50.17 & 56.16 & 37.19 & 78.21 & \underline{52.19} & \underline{55.18} & \underline{39.18}\\
    % \rowcolor{Gray} UniGraph  &  \\
    \midrule
    \multicolumn{10}{l}{\textbf{Use raw multimodal data as input features.}} \\
    CLIP & 41.23 & 28.41 & 61.67 & 55.71 & 63.46 & 40.14 & 41.24 & 40.11 & 30.97 & 62.51 & 58.23 & 46.15 & 51.69 & 31.61 & 72.31 & 47.14 & 50.83 & 31.35 \\
    ImageBind & 32.19 & 23.90 & 58.20 & 54.24 & 62.48 & 38.17 & 29.10 & 28.14 & 21.42 & 51.25 & 48.05 & 44.93 & 48.14 & 30.28 & 69.12 & 41.80 & 41.24 & 26.91 \\
    \hdashline
    NoPretrain & 42.41 & 28.39 & 60.78 & 55.90 & 64.29 & 41.98 & 42.21 & 41.20 & 31.14 & 64.15 & 58.91 & 47.90 & 52.90 & 33.14 & 74.10 & 48.11 & 51.92 & 31.84  \\
    \model & \textbf{56.01} & \textbf{42.98} & \textbf{72.19} & \textbf{61.24} & \textbf{66.24} & \textbf{51.98} & \textbf{51.73} & \textbf{47.42} & \textbf{37.01} & \textbf{69.29} & \textbf{65.29} & \textbf{53.85} & \textbf{57.28} & \textbf{38.47} & \textbf{79.34} & \textbf{52.19} & \textbf{55.59} & \textbf{39.93}\\
    % \rowcolor{Gray} \model  &  \\
    \bottomrule[1.1pt]
    \end{tabular}
    \vspace{-4.6mm}
\end{table*}




\vpara{Results.}
Table~\ref{tab:ssrl} presents the results.
We interpret these results by answering three research questions:
\begin{itemize}[leftmargin=*,itemsep=0pt,parsep=0.2em,topsep=0.3em,partopsep=0.3em]
    \item \textbf{RQ1: Negative Transfer in Multi-Graph Pre-Training.} Existing graph pre-training methods exhibit negative transfer when applied to multi-graph pre-training, whereas \model shows improvements in this context. The results in the \textit{In-distribution} setting demonstrate that both BGRL and GraphMAE2 experience a significant performance drop when pre-trained on multi-graphs (rows in white), compared to pre-training on single graph only (rows in gray). This suggests that pre-training on other datasets negatively affects performance on the target dataset. However, UniGraph2 shows improvement under multi-graph pre-training, indicating that it successfully addresses the shortcomings of existing graph pre-training algorithms struggling with multi-graphs.
    \item \textbf{RQ2: Comparison to Other Foundation Models.} UniGraph2 outperforms methods that consider only multimodal data (CLIP, ImageBind) or only TAGs (UniGraph). We observe that without considering the graph structure, the performance of the acknowledged powerful multimodal foundation models like CLIP is not comparable to UniGraph2. Meanwhile, UniGraph, which cannot process image data, also shows less ideal results due to the lack of information. This further highlights the necessity of designing foundation models specifically for multimodal graphs.
    \item \textbf{RQ3: Generalization Capability.} Compared to baseline methods, UniGraph2 demonstrates strong generalization capabilities. The results in the \textit{In-domain Generalization} and \textit{Out-of-domain Generalization} settings show that UniGraph2 effectively transfers knowledge from pre-training to unseen graphs. Compared to the NoPretrain method, UniGraph2 shows significant improvements. The consistent performance gains indicate that UniGraph2 can extract meaningful patterns during pre-training, which are beneficial for tackling graph learning tasks. Furthermore, UniGraph2 is comparable to methods trained directly on the target datasets, achieving similar accuracy while benefiting from greater efficiency without requiring exhaustive task-specific training.
\end{itemize}










\vspace{-2.8mm}
\subsection{Few-Shot Transfer}
\vpara{Setup.}
In this part, we evaluate the ability of the pre-trained models to perform few-shot in-context transfer without updating the model parameters. 
For baseline methods, in addition to the pre-trained models mentioned in Section~\ref{sec:lp}, we also compare two recent graph in-context learning methods: the self-supervised pre-training method Prodigy~\cite{huang2024prodigy} and the supervised pre-training method OFA~\cite{liuone}.


For evaluation, we strictly follow the setting of Prodigy~\cite{huang2024prodigy}. 
For an N-way K-shot task, we adopt the original train/validation/test splits in each downstream classification dataset, and construct a $K$-shot prompt for test nodes (or edges) from the test split by randomly selecting $K$ examples per way from the train split. By default in all experiments, we sample 500 test tasks.

We adopt the few-shot classification strategy in UniGraph~\cite{he2024unigraphlearningunifiedcrossdomain} for \model. The model computes average embeddings for each class and assigns a query sample to the class with the highest similarity to its embedding.

% \vpara{Research Questions.}
% In this subsection, we aim to answer the following research questions: 
% \begin{itemize}[leftmargin=*,itemsep=0pt,parsep=0.2em,topsep=0.3em,partopsep=0.3em]
%     \item \textbf{RQ1:} How does \model, which takes both multimodal data and graph structures as input, perform in terms of few-shot transfer capabilities compared to foundation models that consider only multimodal data (CLIP, ImageBind) or only TAGs (UniGraph)?
%     \item \textbf{RQ2:} How does \model perform compared to other graph few-shot learning methods?
% \end{itemize}
\vpara{Results.}
In Table~\ref{tab:fwt}, our \model model consistently outperforms all the baselines. This further demonstrates the powerful generalization capabilities of UniGraph2 as a foundation model.
In particular, compared to other graph few-shot learning methods such as Prodigy, OFA, and GCOPE, UniGraph2 does not rely on complex prompt graph designs, and its simple few-shot strategy is both efficient and effective.


\begin{table*}[t]
\centering
 \renewcommand\tabcolsep{4.3pt}
\caption{Experiment results in multimodal generative tasks. We strictly follow the setting in MMGL~\cite{yoon2023multimodal}. The task is to generate a single sentence that summarizing the content of a particular section. The summary is generated based on all images and (non-summary) text present in the target and context sections. We provide different information of MMGs to the base LM: (1) section all (text + image), (2) page text, and (3) page all (all texts and images). We encode multiple multimodal neighbor information using three different neighbor encodings methods: \textit{Self-Attention with Text+Embeddings (SA-TE)}, \textit{Self-Attention with Embeddings (SA-E)}, and \textit{Cross-Attention with Embeddings (CA-E)}.}
\vskip -0.10in
\label{tab:gen}
\begin{tabular}{llcccccccccccc}
\toprule[1.1pt]
& & \multicolumn{4}{c}{BLEU-4} & \multicolumn{4}{c}{ROUGE-L} & \multicolumn{4}{c}{CIDEr} \\
\cmidrule(lr){3-6}\cmidrule(lr){7-10}\cmidrule(lr){11-14}
Input Type & Method & SA-TE & SA-E & CA-E  & Avg. gain & SA-TE & SA-E & CA-E  & Avg. gain & SA-TE & SA-E & CA-E & Avg. gain\\
\midrule
\multirow{2}{*}{Section all} & MMGL & 8.03 & 7.56 & 8.35 & - & 40.41 & 39.89 & 39.98 & - & 77.45 & 74.33 & 75.12 & - \\
& +\model & \textbf{9.24} & \textbf{9.01} & \textbf{9.39} & 15.57\% & \textbf{43.01} & \textbf{43.24} & \textbf{42.98} & 7.44\% & \textbf{81.15} & \textbf{80.39} & \textbf{81.91} & 7.32\% \\
\midrule
\multirow{2}{*}{Page text} & MMGL & 9.81 & 8.37 & 8.47 & - & 42.94 & 40.92 & 41.00 & & 92.71 & 80.14 & 80.72 & - \\
& +\model & \textbf{10.31} & \textbf{10.10} & \textbf{9.98} & 14.53\% & \textbf{43.19} & \textbf{43.08} & \textbf{42.75} &3.38\% & \textbf{93.19} & \textbf{90.41} & \textbf{93.11} & 9.56\% \\
\midrule
\multirow{2}{*}{Page all} & MMGL & 9.96 & 8.58 & 8.51 & - & 43.32 & 41.01 & 41.55 & - & 96.01 & 82.28 & 80.31 & - \\
& +\model & \textbf{10.12} & \textbf{10.05} & \textbf{10.33} & 13.38\% & \textbf{44.10} & \textbf{42.08} & \textbf{42.44} & 2.18\% & \textbf{96.32} & \textbf{91.24} & \textbf{94.15} & 9.49\% \\
% \midrule
% Max input length &  \\
    \bottomrule[1.1pt]
\end{tabular}
\vspace{-3mm}
\end{table*}


\vspace{-5mm}
\subsection{Multimodal Generative Tasks}
\vpara{Setup.}
\model is designed as a general representation learning model. The embeddings it generates can be utilized by various generative foundation models, such as LLMs, to empower downstream generative tasks. 
% \model is a general embedding model designed to generate embeddings that can be used by various generative foundation models, such as LLMs, to enhance downstream generative tasks. 
To further demonstrate this, we select the section summarization task on the WikiWeb2M dataset for our experiments.
The WikiWeb2M dataset~\cite{burns2023suite} is designed for multimodal content understanding, using many-to-many text and image relationships from Wikipedia. It includes page titles, section titles, section text, images, and indices for each section.
In this work, we focus on section summarization, where the task is to generate a summary sentence from section content using both text and images.

% \todo{how mmgl do}
For the experiments, we follow the MMGL~\cite{yoon2023multimodal} setup, using four types of information: section text, section images, context text, and page-level text/images. 
Consistent with MMGL, we fine-tune Open Pre-trained Transformer (OPT-125m)~\cite{zhang2022opt} to read the input section text/images and generate a summary. Multimodal neighbors are first encoded using frozen vision/text encoders and then aligned to the text-only LM space using 1-layer MLP mapper.
In MMGL, CLIP~\cite{radford2021learning} encoders are used for text and image encoding, remaining frozen during fine-tuning. In our experiments, we replace CLIP embeddings with our \model embeddings.

% \vpara{Research Questions.}
% In this subsection, we aim to answer the following research question: 
% \begin{itemize}[leftmargin=*,itemsep=0pt,parsep=0.2em,topsep=0.3em,partopsep=0.3em]
%     \item \textbf{RQ1:} How do the embeddings generated by \model perform on generative tasks compared to multimodal foundation models like CLIP?
% \end{itemize}


\vpara{Results.}
Table~\ref{tab:gen} shows that under different input types and different neighbor encoding strategies, the embeddings generated by UniGraph2 bring significant improvements compared to MMGL's default CLIP embeddings. 
We also observe that UniGraph2's embeddings are more robust to different neighbor encoding strategies compared to CLIP and do not rely on a specific strategy.



\begin{table}[t]%\small
\centering
\renewcommand\tabcolsep{1.6pt}
\caption{\textbf{Ablation studies on \model key components.}}
\vskip -0.1in
\label{tab:kc}
\begin{tabular}{lcccc}
\toprule[1.1pt]
    & Products & Amazon-Cloth & Goodreads-NC  & WN18RR \\
\midrule
    \model & \textbf{82.79{\tiny$\pm$0.02}} & \textbf{24.64{\tiny$\pm$0.09}} & \textbf{81.15{\tiny$\pm$0.12}} & \textbf{85.47{\tiny$\pm$0.11}}\\
    w/o MoE & 81.01{\tiny$\pm$0.10} & 21.33{\tiny$\pm$0.04} & 80.10{\tiny$\pm$0.04} & 83.99{\tiny$\pm$0.21}\\
    w/o feat loss& 69.12{\tiny$\pm$0.09} & 18.43{\tiny$\pm$0.24} & 68.12{\tiny$\pm$0.01} & 74.11{\tiny$\pm$0.03}\\
    w/o SPD loss& 82.42{\tiny$\pm$0.11} & 23.39{\tiny$\pm$0.05} & 80.24{\tiny$\pm$0.02} & 85.24{\tiny$\pm$0.11}\\
\bottomrule[1.1pt]
\end{tabular}
\vspace{-3.3mm}
\end{table}

\begin{table}[t]%\small
\centering
\renewcommand\tabcolsep{2.4pt}
\caption{\textbf{Ablation studies on Modality-Specific Encoders.}}
\vskip -0.1in
\label{tab:enc}
\begin{tabular}{lcccc}
\toprule[1.1pt]
    & Products & Amazon-Cloth & Goodreads-NC  & WN18RR \\
\midrule
    CLIP & 82.79{\tiny$\pm$0.02} & 24.64{\tiny$\pm$0.09} & 81.15{\tiny$\pm$0.12} & \textbf{85.47{\tiny$\pm$0.11}}\\
    ImageBind & 82.32{\tiny$\pm$0.05} & \textbf{25.01{\tiny$\pm$0.11}} & 80.33{\tiny$\pm$0.22} & 84.29{\tiny$\pm$0.07}\\
    T5+ViT& \textbf{82.99{\tiny$\pm$0.04}} & 24.38{\tiny$\pm$0.28} & \textbf{81.28{\tiny$\pm$0.11}} & 84.16{\tiny$\pm$0.04}\\
\bottomrule[1.1pt]
\end{tabular}
\vspace{-4.8mm}
\end{table}

\subsection{Model Analysis}
We select four datasets from different domains to conduct more in-depth studies. We adopt self-supervised representation learning for evaluation.

\vpara{Ablation on Key Components.}
Table~\ref{tab:kc} shows the performance of the \model framework after removing some key designs. "W/o MoE" represents that we use simple MLP instead MoE to align node features. 
"W/o feat loss" represents that we only use the SPD loss for pre-training, while "w/o SPD loss" refers to the opposite.
The overall results confirm that all key designs contribute positively to the performance of \model.

\vpara{Ablation on Modality-Specific Encoders}
In Table~\ref{tab:enc}, we study the influence of different Modality-Specific Encoders on the performance of encoding raw multimodal data. CLIP and ImageBind are feature encoders that map features from various modalities to a shared embedding space, whereas T5+ViT employs SOTA embedding methods for each modality independently, without specific alignment. The results show that all methods achieve comparable performance, indicating that \model effectively aligns features regardless of whether they have been pre-aligned or not.

\begin{table}[t] \scriptsize
\centering
\renewcommand\tabcolsep{3.5pt}
\caption{\textbf{Comparison of GPU hours and performance on ogbn-Arxiv and ogbn-Papers100M.}}
\vskip -0.1in
\label{tab:ccp}
\begin{tabular}{ccccc}
\toprule[1.1pt]
Method & Pre-training & Downstream Training & Downstream Inference & Test Accuracy \\
\midrule
\multicolumn{5}{l}{\textbf{ogbn-Arxiv (169,343 nodes)}} \\ 
% \multirow{3}{*}{\shortstack{ogbn-Arxiv \\ (169,343 nodes)}} 
  GAT        & -    & 0.39 h & 5.5 mins  & 70.89 $\pm$ 0.43 \\
  GraphMAE2  & -    & 5.1 h     & 5.4 mins  & 70.46 $\pm$ 0.07 \\
  UniGraph   & 28.1 h & -      & 9.8 mins & 72.15 $\pm$ 0.18 \\
  UniGraph2  & 5.2 h & - & 5.7 mins &    \textbf{72.56 $\pm$ 0.15}  \\
\midrule
\multicolumn{5}{l}{\textbf{ogbn-Papers100M (111,059,956 nodes)}} \\
  GAT        & -    & 6.8 h     & 23.1 mins & 65.98 $\pm$ 0.23 \\
  GraphMAE2  & -    & 23.2 h    & 23.0 mins & 61.97 $\pm$ 0.24 \\
  UniGraph   & 28.1 h & -      & 40.1 mins & 67.89 $\pm$ 0.21 \\
  UniGraph2 & 5.2 h & - & 24.8 mins &  \textbf{67.95 $\pm$ 0.11} \\
\bottomrule[1.1pt]
\end{tabular}
\vspace{-4.5mm}
\end{table}

\vpara{Efficiency Analysis.}
\model, designed as a foundation model, incurs significant computational costs primarily during the pre-training phase. 
However, it offers the advantage of applicability to new datasets in the inference phase without requiring retraining. 
We compare of the training and inference costs of our model with other models. GAT~\cite{velivckovic2018graph} is a supervised trained GNN. 
GraphMAE2~\cite{hou2023graphmae2} is a self-supervised learning method with GAT as the backbone network. 
UniGraph~\cite{he2024unigraphlearningunifiedcrossdomain} is a graph foundation model for TAGs.
We select ogbn-Arxiv and ogbn-Papers100M, two datasets of different scales for experiments. 
From the results in the Table~\ref{tab:ccp}, we observe that although UniGraph2 has a long pre-training time, its inference time on downstream datasets is comparable or shorter than the combined training and inference time of GNN-based methods. This advantage further increases with the size and potential quantity of downstream datasets.
% The same conclusion also applies to space complexity. Although LM has a larger number of parameters, since we only need to perform inference on the downstream dataset, we avoid the additional space occupation in the backward propagation during training. 

\section{Further Analysis}
\label{sec:analysis}
\begin{figure}[!t] 
    \includegraphics[width=\columnwidth]{images/robustness.pdf} 
    \vspace{-4.5ex}
    \caption{Robustness Evaluation compares the KFR of three methods under precision changes (float16 → bfloat16) and jailbreak attacks.} \vspace{-3ex} 
    \label{fig:robustess} 
\end{figure}
\subsection{Robustness Evaluation}
Building on previous work \citep{zhang2024doesllmtrulyunlearn, lu2024eraserjailbreakingdefenselarge}, which demonstrates that parameter precision and jailbreak attacks affect unlearning, we analyze the robustness of unlearned models under these conditions on KnowUnDo. 
The results are presented in Figure~\ref{fig:robustess}, and we can summarize two key findings.
\paragraph{ReLearn Prevents Knowledge Leakage under Precision Variation.}
As seen from Figure~\ref{fig:robustess}, we observe that reducing the precision of the parameter from float16 to bfloat16 causes a significant decrease in KFR performance, 9.7\% for GA and 18.2\% for NPO.
This suggests that GA and NPO are sensitive to parameter precision and rely on fine-grained adjustments during LoRA fine-tuning.
The sentence completion examples in Appendix Table~\ref{tab:robustness_case} demonstrate that while GA and NPO exhibit unreadable outputs in most cases, indicating over-forgetting, they also reveal some instances of knowledge leakage.
In contrast, ReLearn shows a slight performance improvement of 1.4\% under reduced precision while consistently maintaining a coherent output.
\paragraph{ReLearn Effectively Resists Jailbreaks.}
By using the AIM jailbreak attack \citep{NEURIPS2023_fd661313}, a prompt engineering method that forces compromised model responses (with templates in Appendix~\ref{appendix:AIM}), we observe KFR performance degradation of 5.0\% for GA and 9.1\% for NPO.
In particular, ReLearn achieves a performance improvement of 6.9\%. 
This difference indicates that GA and NPO weaken the base model's inherent jailbreak resistance, while ReLearn maintains and even enhances this defensive capability. 
As seen from the examples shown in Table~\ref{tab:robustness_case}, when attacked, ReLearn effectively prevents jailbreak attacks targeting forgotten knowledge, while GA and NPO tend to leak private information (sometimes incomplete) or generate unreadable responses.

\subsection{The Mechanism of Unlearning}
In this section, we analyze how GA and NPO disrupt the model's linguistic ability and explore how ReLearn reconstructs it.
We analyze from three perspectives: Knowledge Distribution, Knowledge Memory, and Knowledge Circuits.

\subsubsection{Knowledge Distribution}
GA and NPO both rely on reverse optimization to suppress the probabilities of the target token, leading to \textbf{\textit{a disruptive ``probability seesaw effect''}}. 
To explore the knowledge distribution of different unlearning models, we calculate the top-5 candidate tokens in their outputs, as shown in Figure~\ref{fig:prob} and Figure~\ref{fig:gemma_top5} in the Appendix. 
As observed, in models with a \textbf{multi-peaked probability distribution} (e.g., Llama2 Vanilla in Figure~\ref{fig:prob}), the ``seesaw'' effect exhibits two sequent steps: 
(1) \emph{Initial Target Token Suppression:} By suppressing the initially top-1 token and guiding the model towards other high-probability tokens, this potentially leads to sensitive responses (as illustrated in Figure~\ref{fig:prob}, where the top-2 token in the Vanilla model becomes the top-1 token in the NPO model).
(2) \emph{Subsequent Top Token Suppression:} This involves the continued suppression of high-probability tokens, resulting in probability redistribution across random tokens (as observed on Llama2 GA in Figure~\ref{fig:prob}).  
In contrast, for models with a \textbf{unimodal probability distribution} (e.g., Gemma in Figure~\ref{fig:gemma_top5}), reverse optimization merely suppresses the single high-probability peak of the target token, resulting in a more uniform probability distribution across random tokens after unlearning. 

The disrupted probability distributions resemble \emph{cognitive conflict} \citep{xu-etal-2024-earth}, which arises from the conflict between the intrinsic knowledge of a model and external inputs or training objectives.  
\textbf{Reverse optimization directly drives the decoding space toward randomness, leading to a significant cognitive mismatch between the pre-unlearning and post-unlearning states, limiting question understanding and coherent generation.}  
In contrast, ReLearn does not aim for a complete disruption of the knowledge distribution.  
By learning to generate relevant yet non-sensitive answers, ReLearn guides the model toward a new cognitive pattern.

\begin{figure}[!htbp]
\includegraphics[width=\linewidth]{images/top5_llama2.pdf}
% \vspace{-3ex}
  \caption{The top-5 candidate tokens distribution of different unlearning approaches on KnowUnDo.}
  % \vspace{-2ex}
  \label{fig:prob}
\end{figure}
\begin{figure}[!t]
  \centering
  \includegraphics[width=\linewidth]{images/knowledge_memory.pdf}
  \vspace{-4ex}
  \caption{Knowledge Memory. Vanilla model generates ``5000 Sierra Rd Bogota Colomb''; GA/NPO produce repetitive ``at''; ReLearn generates a contextually relevant but non-sensitive response.}
  \vspace{-2ex}
  \label{fig:mem}
\end{figure}
\subsubsection{Knowledge Memory}
Inspired by recent research \citep{geva-etal-2022-transformer, geva-etal-2023-dissecting, ghandeharioun2024s, menta2025analyzingmemorizationlargelanguage} that the early layers process context, the deeper layers memorize, and the last few layers handle the prediction of the next token, our analysis focuses on the final token position's outputs across all decoding layers\citep{belrose2023elicitinglatentpredictionstransformers}.

Figure~\ref{fig:mem} demonstrates the difference between these methods.
When queried with ``Carlos Rivera's mailing address is...'', the vanilla model directly activates both general concepts like ``address'' and ``location'', as well as the answer terms such as ``Colomb''. 
In contrast, ReLearn preserves semantic understanding without directly recalling the answer. 
In its middle and later layers, it recalls related concepts like ``located'' and ``address'', along with query terms such as ``Carlos''.
In comparison, reverse optimization methods like NPO activate ``address'' before the 20th layer but fail to trigger related knowledge afterward, instead repeating ``at'' beyond the 20th layer.

Moreover, the Forward-KL, which represents the KL Divergence between the current and final layers, shows a gradual shift for the vanilla and ReLearn models, but a severe shift for GA/NPO.
This severe change hinders the effective use of semantic information for knowledge retrieval and refinement, impeding the appropriate generation of responses.

In summary, \textbf{reverse optimization significantly impairs knowledge memory by overemphasizing next-token prediction and disrupting the ability of gradual information adjustment}, which is similar to memory loss in Alzheimer's disease \citep{Jahn2013memoryloss}. 
In contrast, ReLearn maintains robust knowledge memory across layers, preserving linguistic capabilities, and enabling fluent, relevant responses through positive optimization.

\subsubsection{Knowledge Circuits}
We employ the LLMTT tool \citep{tufanov2024lm} to visualize \textit{knowledge circuits} and investigate how different unlearning methods affect model focus.
LLMTT identifies the salient connections (``circuits'') within the LLM inference process by varying the threshold, where higher thresholds indicate stronger connections.
As shown in Figure~\ref{fig:circuits} in the Appendix, with a threshold of 0.06, the vanilla, GA, and NPO models exhibit similar circuit patterns. 
However, ReLearn notably reduces circuits associated with sensitive entities, indicating a weakened focus on sensitive information.
When the threshold increases to 0.08, the circuits of vanilla model and ReLearn model become empty, while GA and NPO strengthen partial circuits, particularly those specific question patterns (e.g., ``How does...background...?'').
This observation suggests that \textbf{GA and NPO over-forget specific question patterns}, while ReLearn achieves generalized unlearning by weakening entity associations.
% resulting in issues similar to generalization problems caused by spurious correlations \citep{bayat2024pitfallsmemorizationmemorizationhurts}.


\vspace{-5pt}
\section{Discussion}
\textbf{Conclusion.}
In this work, we propose the \textit{\methodname{}} metric, $M_{AP}$, to evaluate preference data quality in alignment.
By measuring the gap from the model's current implicit reward margin to the target explicit reward margin, $M_{AP}$ quantifies the discrepancy between the current model and the aligned optimum, thereby indicating the potential for alignment enhancement.
Extensive experiments validate the efficacy of $M_{AP}$ across various training settings under offline and self-play preference learning scenarios.

\textbf{Limitations and future work}. 
Despite the performance improvements, $M_{AP}$ requires tuning a parameter $\beta$ to combine the explicit and implicit margins; future work could explore how to set this ratio automatically.
Additionally, while our experiments focus on the widely applied DPO and SimPO objectives, a broader investigation with alternative preference learning methods is crucial in future works.

% \section{Conclusion}
% In this paper, we introduce the \methodname{} metric to evaluate preference data quality in LLM alignment.
% By measuring the discrepancy between the model's current implicit reward margin to the target explicit reward margin, this metric quantifies the gap between the current model and the aligned optimum, thereby indicating the potential for alignment enhancement.
% Empirical results demonstrate that training on data selected by our metric consistently improves alignment performance, outperforming existing metrics across different base models and training objectives.
% Moreover, this metric extends to data generation scenarios (\ie self-play alignment): by identifying high-quality data from the intrinsic self-generated context, our metric yields superior results across various training settings, providing a comprehensive solution for enhancing LLM alignment through optimized
% preference data generation, selection, and utilization.


\section*{Impact Statement}
This paper presents work whose goal is to advance the field of Machine Learning. There are many potential societal consequences of our work, none which we feel must be specifically highlighted here.

\section{Limitations}
A key limitation of \our lies in its Markov state transition process without a well-designed reflection mechanism. When the initial DAG decomposition fails to properly model parallel relationships between subquestions or captures unnecessary dependencies, it can negatively impact subsequent contraction and reasoning process, a scenario that occurs frequently in practice. The framework currently lacks the ability to detect and rectify such poor decompositions, potentially leading to compounded errors in the atomic state transitions. This limitation suggests the need for future research into incorporating effective reflection and adjustment mechanisms to improve the robustness of DAG-based decomposition.

\section{Ethics Statement}
While this work advances the computational efficiency and test-time scaling capabilities of language models through the \our framework, we acknowledge that these models process information and conduct reasoning in ways fundamentally different from human cognition. Making direct comparisons between our Markov reasoning process and human thought patterns could be misleading and potentially harmful. The atomic state representation and dependency-based decomposition proposed in this research are computational constructs designed to optimize machine reasoning, rather than models of human cognitive processes. Our work merely aims to explore more efficient ways of structuring machine reasoning through reduced computational resources and simplified state transitions, while recognizing the distinct nature of artificial and human intelligence. We encourage users of this technology to be mindful of these limitations and to implement appropriate safeguards when deploying systems based on our framework.

\bibliography{custom}

\appendix

\newpage
\subsection{Lloyd-Max Algorithm}
\label{subsec:Lloyd-Max}
For a given quantization bitwidth $B$ and an operand $\bm{X}$, the Lloyd-Max algorithm finds $2^B$ quantization levels $\{\hat{x}_i\}_{i=1}^{2^B}$ such that quantizing $\bm{X}$ by rounding each scalar in $\bm{X}$ to the nearest quantization level minimizes the quantization MSE. 

The algorithm starts with an initial guess of quantization levels and then iteratively computes quantization thresholds $\{\tau_i\}_{i=1}^{2^B-1}$ and updates quantization levels $\{\hat{x}_i\}_{i=1}^{2^B}$. Specifically, at iteration $n$, thresholds are set to the midpoints of the previous iteration's levels:
\begin{align*}
    \tau_i^{(n)}=\frac{\hat{x}_i^{(n-1)}+\hat{x}_{i+1}^{(n-1)}}2 \text{ for } i=1\ldots 2^B-1
\end{align*}
Subsequently, the quantization levels are re-computed as conditional means of the data regions defined by the new thresholds:
\begin{align*}
    \hat{x}_i^{(n)}=\mathbb{E}\left[ \bm{X} \big| \bm{X}\in [\tau_{i-1}^{(n)},\tau_i^{(n)}] \right] \text{ for } i=1\ldots 2^B
\end{align*}
where to satisfy boundary conditions we have $\tau_0=-\infty$ and $\tau_{2^B}=\infty$. The algorithm iterates the above steps until convergence.

Figure \ref{fig:lm_quant} compares the quantization levels of a $7$-bit floating point (E3M3) quantizer (left) to a $7$-bit Lloyd-Max quantizer (right) when quantizing a layer of weights from the GPT3-126M model at a per-tensor granularity. As shown, the Lloyd-Max quantizer achieves substantially lower quantization MSE. Further, Table \ref{tab:FP7_vs_LM7} shows the superior perplexity achieved by Lloyd-Max quantizers for bitwidths of $7$, $6$ and $5$. The difference between the quantizers is clear at 5 bits, where per-tensor FP quantization incurs a drastic and unacceptable increase in perplexity, while Lloyd-Max quantization incurs a much smaller increase. Nevertheless, we note that even the optimal Lloyd-Max quantizer incurs a notable ($\sim 1.5$) increase in perplexity due to the coarse granularity of quantization. 

\begin{figure}[h]
  \centering
  \includegraphics[width=0.7\linewidth]{sections/figures/LM7_FP7.pdf}
  \caption{\small Quantization levels and the corresponding quantization MSE of Floating Point (left) vs Lloyd-Max (right) Quantizers for a layer of weights in the GPT3-126M model.}
  \label{fig:lm_quant}
\end{figure}

\begin{table}[h]\scriptsize
\begin{center}
\caption{\label{tab:FP7_vs_LM7} \small Comparing perplexity (lower is better) achieved by floating point quantizers and Lloyd-Max quantizers on a GPT3-126M model for the Wikitext-103 dataset.}
\begin{tabular}{c|cc|c}
\hline
 \multirow{2}{*}{\textbf{Bitwidth}} & \multicolumn{2}{|c|}{\textbf{Floating-Point Quantizer}} & \textbf{Lloyd-Max Quantizer} \\
 & Best Format & Wikitext-103 Perplexity & Wikitext-103 Perplexity \\
\hline
7 & E3M3 & 18.32 & 18.27 \\
6 & E3M2 & 19.07 & 18.51 \\
5 & E4M0 & 43.89 & 19.71 \\
\hline
\end{tabular}
\end{center}
\end{table}

\subsection{Proof of Local Optimality of LO-BCQ}
\label{subsec:lobcq_opt_proof}
For a given block $\bm{b}_j$, the quantization MSE during LO-BCQ can be empirically evaluated as $\frac{1}{L_b}\lVert \bm{b}_j- \bm{\hat{b}}_j\rVert^2_2$ where $\bm{\hat{b}}_j$ is computed from equation (\ref{eq:clustered_quantization_definition}) as $C_{f(\bm{b}_j)}(\bm{b}_j)$. Further, for a given block cluster $\mathcal{B}_i$, we compute the quantization MSE as $\frac{1}{|\mathcal{B}_{i}|}\sum_{\bm{b} \in \mathcal{B}_{i}} \frac{1}{L_b}\lVert \bm{b}- C_i^{(n)}(\bm{b})\rVert^2_2$. Therefore, at the end of iteration $n$, we evaluate the overall quantization MSE $J^{(n)}$ for a given operand $\bm{X}$ composed of $N_c$ block clusters as:
\begin{align*}
    \label{eq:mse_iter_n}
    J^{(n)} = \frac{1}{N_c} \sum_{i=1}^{N_c} \frac{1}{|\mathcal{B}_{i}^{(n)}|}\sum_{\bm{v} \in \mathcal{B}_{i}^{(n)}} \frac{1}{L_b}\lVert \bm{b}- B_i^{(n)}(\bm{b})\rVert^2_2
\end{align*}

At the end of iteration $n$, the codebooks are updated from $\mathcal{C}^{(n-1)}$ to $\mathcal{C}^{(n)}$. However, the mapping of a given vector $\bm{b}_j$ to quantizers $\mathcal{C}^{(n)}$ remains as  $f^{(n)}(\bm{b}_j)$. At the next iteration, during the vector clustering step, $f^{(n+1)}(\bm{b}_j)$ finds new mapping of $\bm{b}_j$ to updated codebooks $\mathcal{C}^{(n)}$ such that the quantization MSE over the candidate codebooks is minimized. Therefore, we obtain the following result for $\bm{b}_j$:
\begin{align*}
\frac{1}{L_b}\lVert \bm{b}_j - C_{f^{(n+1)}(\bm{b}_j)}^{(n)}(\bm{b}_j)\rVert^2_2 \le \frac{1}{L_b}\lVert \bm{b}_j - C_{f^{(n)}(\bm{b}_j)}^{(n)}(\bm{b}_j)\rVert^2_2
\end{align*}

That is, quantizing $\bm{b}_j$ at the end of the block clustering step of iteration $n+1$ results in lower quantization MSE compared to quantizing at the end of iteration $n$. Since this is true for all $\bm{b} \in \bm{X}$, we assert the following:
\begin{equation}
\begin{split}
\label{eq:mse_ineq_1}
    \tilde{J}^{(n+1)} &= \frac{1}{N_c} \sum_{i=1}^{N_c} \frac{1}{|\mathcal{B}_{i}^{(n+1)}|}\sum_{\bm{b} \in \mathcal{B}_{i}^{(n+1)}} \frac{1}{L_b}\lVert \bm{b} - C_i^{(n)}(b)\rVert^2_2 \le J^{(n)}
\end{split}
\end{equation}
where $\tilde{J}^{(n+1)}$ is the the quantization MSE after the vector clustering step at iteration $n+1$.

Next, during the codebook update step (\ref{eq:quantizers_update}) at iteration $n+1$, the per-cluster codebooks $\mathcal{C}^{(n)}$ are updated to $\mathcal{C}^{(n+1)}$ by invoking the Lloyd-Max algorithm \citep{Lloyd}. We know that for any given value distribution, the Lloyd-Max algorithm minimizes the quantization MSE. Therefore, for a given vector cluster $\mathcal{B}_i$ we obtain the following result:

\begin{equation}
    \frac{1}{|\mathcal{B}_{i}^{(n+1)}|}\sum_{\bm{b} \in \mathcal{B}_{i}^{(n+1)}} \frac{1}{L_b}\lVert \bm{b}- C_i^{(n+1)}(\bm{b})\rVert^2_2 \le \frac{1}{|\mathcal{B}_{i}^{(n+1)}|}\sum_{\bm{b} \in \mathcal{B}_{i}^{(n+1)}} \frac{1}{L_b}\lVert \bm{b}- C_i^{(n)}(\bm{b})\rVert^2_2
\end{equation}

The above equation states that quantizing the given block cluster $\mathcal{B}_i$ after updating the associated codebook from $C_i^{(n)}$ to $C_i^{(n+1)}$ results in lower quantization MSE. Since this is true for all the block clusters, we derive the following result: 
\begin{equation}
\begin{split}
\label{eq:mse_ineq_2}
     J^{(n+1)} &= \frac{1}{N_c} \sum_{i=1}^{N_c} \frac{1}{|\mathcal{B}_{i}^{(n+1)}|}\sum_{\bm{b} \in \mathcal{B}_{i}^{(n+1)}} \frac{1}{L_b}\lVert \bm{b}- C_i^{(n+1)}(\bm{b})\rVert^2_2  \le \tilde{J}^{(n+1)}   
\end{split}
\end{equation}

Following (\ref{eq:mse_ineq_1}) and (\ref{eq:mse_ineq_2}), we find that the quantization MSE is non-increasing for each iteration, that is, $J^{(1)} \ge J^{(2)} \ge J^{(3)} \ge \ldots \ge J^{(M)}$ where $M$ is the maximum number of iterations. 
%Therefore, we can say that if the algorithm converges, then it must be that it has converged to a local minimum. 
\hfill $\blacksquare$


\begin{figure}
    \begin{center}
    \includegraphics[width=0.5\textwidth]{sections//figures/mse_vs_iter.pdf}
    \end{center}
    \caption{\small NMSE vs iterations during LO-BCQ compared to other block quantization proposals}
    \label{fig:nmse_vs_iter}
\end{figure}

Figure \ref{fig:nmse_vs_iter} shows the empirical convergence of LO-BCQ across several block lengths and number of codebooks. Also, the MSE achieved by LO-BCQ is compared to baselines such as MXFP and VSQ. As shown, LO-BCQ converges to a lower MSE than the baselines. Further, we achieve better convergence for larger number of codebooks ($N_c$) and for a smaller block length ($L_b$), both of which increase the bitwidth of BCQ (see Eq \ref{eq:bitwidth_bcq}).


\subsection{Additional Accuracy Results}
%Table \ref{tab:lobcq_config} lists the various LOBCQ configurations and their corresponding bitwidths.
\begin{table}
\setlength{\tabcolsep}{4.75pt}
\begin{center}
\caption{\label{tab:lobcq_config} Various LO-BCQ configurations and their bitwidths.}
\begin{tabular}{|c||c|c|c|c||c|c||c|} 
\hline
 & \multicolumn{4}{|c||}{$L_b=8$} & \multicolumn{2}{|c||}{$L_b=4$} & $L_b=2$ \\
 \hline
 \backslashbox{$L_A$\kern-1em}{\kern-1em$N_c$} & 2 & 4 & 8 & 16 & 2 & 4 & 2 \\
 \hline
 64 & 4.25 & 4.375 & 4.5 & 4.625 & 4.375 & 4.625 & 4.625\\
 \hline
 32 & 4.375 & 4.5 & 4.625& 4.75 & 4.5 & 4.75 & 4.75 \\
 \hline
 16 & 4.625 & 4.75& 4.875 & 5 & 4.75 & 5 & 5 \\
 \hline
\end{tabular}
\end{center}
\end{table}

%\subsection{Perplexity achieved by various LO-BCQ configurations on Wikitext-103 dataset}

\begin{table} \centering
\begin{tabular}{|c||c|c|c|c||c|c||c|} 
\hline
 $L_b \rightarrow$& \multicolumn{4}{c||}{8} & \multicolumn{2}{c||}{4} & 2\\
 \hline
 \backslashbox{$L_A$\kern-1em}{\kern-1em$N_c$} & 2 & 4 & 8 & 16 & 2 & 4 & 2  \\
 %$N_c \rightarrow$ & 2 & 4 & 8 & 16 & 2 & 4 & 2 \\
 \hline
 \hline
 \multicolumn{8}{c}{GPT3-1.3B (FP32 PPL = 9.98)} \\ 
 \hline
 \hline
 64 & 10.40 & 10.23 & 10.17 & 10.15 &  10.28 & 10.18 & 10.19 \\
 \hline
 32 & 10.25 & 10.20 & 10.15 & 10.12 &  10.23 & 10.17 & 10.17 \\
 \hline
 16 & 10.22 & 10.16 & 10.10 & 10.09 &  10.21 & 10.14 & 10.16 \\
 \hline
  \hline
 \multicolumn{8}{c}{GPT3-8B (FP32 PPL = 7.38)} \\ 
 \hline
 \hline
 64 & 7.61 & 7.52 & 7.48 &  7.47 &  7.55 &  7.49 & 7.50 \\
 \hline
 32 & 7.52 & 7.50 & 7.46 &  7.45 &  7.52 &  7.48 & 7.48  \\
 \hline
 16 & 7.51 & 7.48 & 7.44 &  7.44 &  7.51 &  7.49 & 7.47  \\
 \hline
\end{tabular}
\caption{\label{tab:ppl_gpt3_abalation} Wikitext-103 perplexity across GPT3-1.3B and 8B models.}
\end{table}

\begin{table} \centering
\begin{tabular}{|c||c|c|c|c||} 
\hline
 $L_b \rightarrow$& \multicolumn{4}{c||}{8}\\
 \hline
 \backslashbox{$L_A$\kern-1em}{\kern-1em$N_c$} & 2 & 4 & 8 & 16 \\
 %$N_c \rightarrow$ & 2 & 4 & 8 & 16 & 2 & 4 & 2 \\
 \hline
 \hline
 \multicolumn{5}{|c|}{Llama2-7B (FP32 PPL = 5.06)} \\ 
 \hline
 \hline
 64 & 5.31 & 5.26 & 5.19 & 5.18  \\
 \hline
 32 & 5.23 & 5.25 & 5.18 & 5.15  \\
 \hline
 16 & 5.23 & 5.19 & 5.16 & 5.14  \\
 \hline
 \multicolumn{5}{|c|}{Nemotron4-15B (FP32 PPL = 5.87)} \\ 
 \hline
 \hline
 64  & 6.3 & 6.20 & 6.13 & 6.08  \\
 \hline
 32  & 6.24 & 6.12 & 6.07 & 6.03  \\
 \hline
 16  & 6.12 & 6.14 & 6.04 & 6.02  \\
 \hline
 \multicolumn{5}{|c|}{Nemotron4-340B (FP32 PPL = 3.48)} \\ 
 \hline
 \hline
 64 & 3.67 & 3.62 & 3.60 & 3.59 \\
 \hline
 32 & 3.63 & 3.61 & 3.59 & 3.56 \\
 \hline
 16 & 3.61 & 3.58 & 3.57 & 3.55 \\
 \hline
\end{tabular}
\caption{\label{tab:ppl_llama7B_nemo15B} Wikitext-103 perplexity compared to FP32 baseline in Llama2-7B and Nemotron4-15B, 340B models}
\end{table}

%\subsection{Perplexity achieved by various LO-BCQ configurations on MMLU dataset}


\begin{table} \centering
\begin{tabular}{|c||c|c|c|c||c|c|c|c|} 
\hline
 $L_b \rightarrow$& \multicolumn{4}{c||}{8} & \multicolumn{4}{c||}{8}\\
 \hline
 \backslashbox{$L_A$\kern-1em}{\kern-1em$N_c$} & 2 & 4 & 8 & 16 & 2 & 4 & 8 & 16  \\
 %$N_c \rightarrow$ & 2 & 4 & 8 & 16 & 2 & 4 & 2 \\
 \hline
 \hline
 \multicolumn{5}{|c|}{Llama2-7B (FP32 Accuracy = 45.8\%)} & \multicolumn{4}{|c|}{Llama2-70B (FP32 Accuracy = 69.12\%)} \\ 
 \hline
 \hline
 64 & 43.9 & 43.4 & 43.9 & 44.9 & 68.07 & 68.27 & 68.17 & 68.75 \\
 \hline
 32 & 44.5 & 43.8 & 44.9 & 44.5 & 68.37 & 68.51 & 68.35 & 68.27  \\
 \hline
 16 & 43.9 & 42.7 & 44.9 & 45 & 68.12 & 68.77 & 68.31 & 68.59  \\
 \hline
 \hline
 \multicolumn{5}{|c|}{GPT3-22B (FP32 Accuracy = 38.75\%)} & \multicolumn{4}{|c|}{Nemotron4-15B (FP32 Accuracy = 64.3\%)} \\ 
 \hline
 \hline
 64 & 36.71 & 38.85 & 38.13 & 38.92 & 63.17 & 62.36 & 63.72 & 64.09 \\
 \hline
 32 & 37.95 & 38.69 & 39.45 & 38.34 & 64.05 & 62.30 & 63.8 & 64.33  \\
 \hline
 16 & 38.88 & 38.80 & 38.31 & 38.92 & 63.22 & 63.51 & 63.93 & 64.43  \\
 \hline
\end{tabular}
\caption{\label{tab:mmlu_abalation} Accuracy on MMLU dataset across GPT3-22B, Llama2-7B, 70B and Nemotron4-15B models.}
\end{table}


%\subsection{Perplexity achieved by various LO-BCQ configurations on LM evaluation harness}

\begin{table} \centering
\begin{tabular}{|c||c|c|c|c||c|c|c|c|} 
\hline
 $L_b \rightarrow$& \multicolumn{4}{c||}{8} & \multicolumn{4}{c||}{8}\\
 \hline
 \backslashbox{$L_A$\kern-1em}{\kern-1em$N_c$} & 2 & 4 & 8 & 16 & 2 & 4 & 8 & 16  \\
 %$N_c \rightarrow$ & 2 & 4 & 8 & 16 & 2 & 4 & 2 \\
 \hline
 \hline
 \multicolumn{5}{|c|}{Race (FP32 Accuracy = 37.51\%)} & \multicolumn{4}{|c|}{Boolq (FP32 Accuracy = 64.62\%)} \\ 
 \hline
 \hline
 64 & 36.94 & 37.13 & 36.27 & 37.13 & 63.73 & 62.26 & 63.49 & 63.36 \\
 \hline
 32 & 37.03 & 36.36 & 36.08 & 37.03 & 62.54 & 63.51 & 63.49 & 63.55  \\
 \hline
 16 & 37.03 & 37.03 & 36.46 & 37.03 & 61.1 & 63.79 & 63.58 & 63.33  \\
 \hline
 \hline
 \multicolumn{5}{|c|}{Winogrande (FP32 Accuracy = 58.01\%)} & \multicolumn{4}{|c|}{Piqa (FP32 Accuracy = 74.21\%)} \\ 
 \hline
 \hline
 64 & 58.17 & 57.22 & 57.85 & 58.33 & 73.01 & 73.07 & 73.07 & 72.80 \\
 \hline
 32 & 59.12 & 58.09 & 57.85 & 58.41 & 73.01 & 73.94 & 72.74 & 73.18  \\
 \hline
 16 & 57.93 & 58.88 & 57.93 & 58.56 & 73.94 & 72.80 & 73.01 & 73.94  \\
 \hline
\end{tabular}
\caption{\label{tab:mmlu_abalation} Accuracy on LM evaluation harness tasks on GPT3-1.3B model.}
\end{table}

\begin{table} \centering
\begin{tabular}{|c||c|c|c|c||c|c|c|c|} 
\hline
 $L_b \rightarrow$& \multicolumn{4}{c||}{8} & \multicolumn{4}{c||}{8}\\
 \hline
 \backslashbox{$L_A$\kern-1em}{\kern-1em$N_c$} & 2 & 4 & 8 & 16 & 2 & 4 & 8 & 16  \\
 %$N_c \rightarrow$ & 2 & 4 & 8 & 16 & 2 & 4 & 2 \\
 \hline
 \hline
 \multicolumn{5}{|c|}{Race (FP32 Accuracy = 41.34\%)} & \multicolumn{4}{|c|}{Boolq (FP32 Accuracy = 68.32\%)} \\ 
 \hline
 \hline
 64 & 40.48 & 40.10 & 39.43 & 39.90 & 69.20 & 68.41 & 69.45 & 68.56 \\
 \hline
 32 & 39.52 & 39.52 & 40.77 & 39.62 & 68.32 & 67.43 & 68.17 & 69.30  \\
 \hline
 16 & 39.81 & 39.71 & 39.90 & 40.38 & 68.10 & 66.33 & 69.51 & 69.42  \\
 \hline
 \hline
 \multicolumn{5}{|c|}{Winogrande (FP32 Accuracy = 67.88\%)} & \multicolumn{4}{|c|}{Piqa (FP32 Accuracy = 78.78\%)} \\ 
 \hline
 \hline
 64 & 66.85 & 66.61 & 67.72 & 67.88 & 77.31 & 77.42 & 77.75 & 77.64 \\
 \hline
 32 & 67.25 & 67.72 & 67.72 & 67.00 & 77.31 & 77.04 & 77.80 & 77.37  \\
 \hline
 16 & 68.11 & 68.90 & 67.88 & 67.48 & 77.37 & 78.13 & 78.13 & 77.69  \\
 \hline
\end{tabular}
\caption{\label{tab:mmlu_abalation} Accuracy on LM evaluation harness tasks on GPT3-8B model.}
\end{table}

\begin{table} \centering
\begin{tabular}{|c||c|c|c|c||c|c|c|c|} 
\hline
 $L_b \rightarrow$& \multicolumn{4}{c||}{8} & \multicolumn{4}{c||}{8}\\
 \hline
 \backslashbox{$L_A$\kern-1em}{\kern-1em$N_c$} & 2 & 4 & 8 & 16 & 2 & 4 & 8 & 16  \\
 %$N_c \rightarrow$ & 2 & 4 & 8 & 16 & 2 & 4 & 2 \\
 \hline
 \hline
 \multicolumn{5}{|c|}{Race (FP32 Accuracy = 40.67\%)} & \multicolumn{4}{|c|}{Boolq (FP32 Accuracy = 76.54\%)} \\ 
 \hline
 \hline
 64 & 40.48 & 40.10 & 39.43 & 39.90 & 75.41 & 75.11 & 77.09 & 75.66 \\
 \hline
 32 & 39.52 & 39.52 & 40.77 & 39.62 & 76.02 & 76.02 & 75.96 & 75.35  \\
 \hline
 16 & 39.81 & 39.71 & 39.90 & 40.38 & 75.05 & 73.82 & 75.72 & 76.09  \\
 \hline
 \hline
 \multicolumn{5}{|c|}{Winogrande (FP32 Accuracy = 70.64\%)} & \multicolumn{4}{|c|}{Piqa (FP32 Accuracy = 79.16\%)} \\ 
 \hline
 \hline
 64 & 69.14 & 70.17 & 70.17 & 70.56 & 78.24 & 79.00 & 78.62 & 78.73 \\
 \hline
 32 & 70.96 & 69.69 & 71.27 & 69.30 & 78.56 & 79.49 & 79.16 & 78.89  \\
 \hline
 16 & 71.03 & 69.53 & 69.69 & 70.40 & 78.13 & 79.16 & 79.00 & 79.00  \\
 \hline
\end{tabular}
\caption{\label{tab:mmlu_abalation} Accuracy on LM evaluation harness tasks on GPT3-22B model.}
\end{table}

\begin{table} \centering
\begin{tabular}{|c||c|c|c|c||c|c|c|c|} 
\hline
 $L_b \rightarrow$& \multicolumn{4}{c||}{8} & \multicolumn{4}{c||}{8}\\
 \hline
 \backslashbox{$L_A$\kern-1em}{\kern-1em$N_c$} & 2 & 4 & 8 & 16 & 2 & 4 & 8 & 16  \\
 %$N_c \rightarrow$ & 2 & 4 & 8 & 16 & 2 & 4 & 2 \\
 \hline
 \hline
 \multicolumn{5}{|c|}{Race (FP32 Accuracy = 44.4\%)} & \multicolumn{4}{|c|}{Boolq (FP32 Accuracy = 79.29\%)} \\ 
 \hline
 \hline
 64 & 42.49 & 42.51 & 42.58 & 43.45 & 77.58 & 77.37 & 77.43 & 78.1 \\
 \hline
 32 & 43.35 & 42.49 & 43.64 & 43.73 & 77.86 & 75.32 & 77.28 & 77.86  \\
 \hline
 16 & 44.21 & 44.21 & 43.64 & 42.97 & 78.65 & 77 & 76.94 & 77.98  \\
 \hline
 \hline
 \multicolumn{5}{|c|}{Winogrande (FP32 Accuracy = 69.38\%)} & \multicolumn{4}{|c|}{Piqa (FP32 Accuracy = 78.07\%)} \\ 
 \hline
 \hline
 64 & 68.9 & 68.43 & 69.77 & 68.19 & 77.09 & 76.82 & 77.09 & 77.86 \\
 \hline
 32 & 69.38 & 68.51 & 68.82 & 68.90 & 78.07 & 76.71 & 78.07 & 77.86  \\
 \hline
 16 & 69.53 & 67.09 & 69.38 & 68.90 & 77.37 & 77.8 & 77.91 & 77.69  \\
 \hline
\end{tabular}
\caption{\label{tab:mmlu_abalation} Accuracy on LM evaluation harness tasks on Llama2-7B model.}
\end{table}

\begin{table} \centering
\begin{tabular}{|c||c|c|c|c||c|c|c|c|} 
\hline
 $L_b \rightarrow$& \multicolumn{4}{c||}{8} & \multicolumn{4}{c||}{8}\\
 \hline
 \backslashbox{$L_A$\kern-1em}{\kern-1em$N_c$} & 2 & 4 & 8 & 16 & 2 & 4 & 8 & 16  \\
 %$N_c \rightarrow$ & 2 & 4 & 8 & 16 & 2 & 4 & 2 \\
 \hline
 \hline
 \multicolumn{5}{|c|}{Race (FP32 Accuracy = 48.8\%)} & \multicolumn{4}{|c|}{Boolq (FP32 Accuracy = 85.23\%)} \\ 
 \hline
 \hline
 64 & 49.00 & 49.00 & 49.28 & 48.71 & 82.82 & 84.28 & 84.03 & 84.25 \\
 \hline
 32 & 49.57 & 48.52 & 48.33 & 49.28 & 83.85 & 84.46 & 84.31 & 84.93  \\
 \hline
 16 & 49.85 & 49.09 & 49.28 & 48.99 & 85.11 & 84.46 & 84.61 & 83.94  \\
 \hline
 \hline
 \multicolumn{5}{|c|}{Winogrande (FP32 Accuracy = 79.95\%)} & \multicolumn{4}{|c|}{Piqa (FP32 Accuracy = 81.56\%)} \\ 
 \hline
 \hline
 64 & 78.77 & 78.45 & 78.37 & 79.16 & 81.45 & 80.69 & 81.45 & 81.5 \\
 \hline
 32 & 78.45 & 79.01 & 78.69 & 80.66 & 81.56 & 80.58 & 81.18 & 81.34  \\
 \hline
 16 & 79.95 & 79.56 & 79.79 & 79.72 & 81.28 & 81.66 & 81.28 & 80.96  \\
 \hline
\end{tabular}
\caption{\label{tab:mmlu_abalation} Accuracy on LM evaluation harness tasks on Llama2-70B model.}
\end{table}

%\section{MSE Studies}
%\textcolor{red}{TODO}


\subsection{Number Formats and Quantization Method}
\label{subsec:numFormats_quantMethod}
\subsubsection{Integer Format}
An $n$-bit signed integer (INT) is typically represented with a 2s-complement format \citep{yao2022zeroquant,xiao2023smoothquant,dai2021vsq}, where the most significant bit denotes the sign.

\subsubsection{Floating Point Format}
An $n$-bit signed floating point (FP) number $x$ comprises of a 1-bit sign ($x_{\mathrm{sign}}$), $B_m$-bit mantissa ($x_{\mathrm{mant}}$) and $B_e$-bit exponent ($x_{\mathrm{exp}}$) such that $B_m+B_e=n-1$. The associated constant exponent bias ($E_{\mathrm{bias}}$) is computed as $(2^{{B_e}-1}-1)$. We denote this format as $E_{B_e}M_{B_m}$.  

\subsubsection{Quantization Scheme}
\label{subsec:quant_method}
A quantization scheme dictates how a given unquantized tensor is converted to its quantized representation. We consider FP formats for the purpose of illustration. Given an unquantized tensor $\bm{X}$ and an FP format $E_{B_e}M_{B_m}$, we first, we compute the quantization scale factor $s_X$ that maps the maximum absolute value of $\bm{X}$ to the maximum quantization level of the $E_{B_e}M_{B_m}$ format as follows:
\begin{align}
\label{eq:sf}
    s_X = \frac{\mathrm{max}(|\bm{X}|)}{\mathrm{max}(E_{B_e}M_{B_m})}
\end{align}
In the above equation, $|\cdot|$ denotes the absolute value function.

Next, we scale $\bm{X}$ by $s_X$ and quantize it to $\hat{\bm{X}}$ by rounding it to the nearest quantization level of $E_{B_e}M_{B_m}$ as:

\begin{align}
\label{eq:tensor_quant}
    \hat{\bm{X}} = \text{round-to-nearest}\left(\frac{\bm{X}}{s_X}, E_{B_e}M_{B_m}\right)
\end{align}

We perform dynamic max-scaled quantization \citep{wu2020integer}, where the scale factor $s$ for activations is dynamically computed during runtime.

\subsection{Vector Scaled Quantization}
\begin{wrapfigure}{r}{0.35\linewidth}
  \centering
  \includegraphics[width=\linewidth]{sections/figures/vsquant.jpg}
  \caption{\small Vectorwise decomposition for per-vector scaled quantization (VSQ \citep{dai2021vsq}).}
  \label{fig:vsquant}
\end{wrapfigure}
During VSQ \citep{dai2021vsq}, the operand tensors are decomposed into 1D vectors in a hardware friendly manner as shown in Figure \ref{fig:vsquant}. Since the decomposed tensors are used as operands in matrix multiplications during inference, it is beneficial to perform this decomposition along the reduction dimension of the multiplication. The vectorwise quantization is performed similar to tensorwise quantization described in Equations \ref{eq:sf} and \ref{eq:tensor_quant}, where a scale factor $s_v$ is required for each vector $\bm{v}$ that maps the maximum absolute value of that vector to the maximum quantization level. While smaller vector lengths can lead to larger accuracy gains, the associated memory and computational overheads due to the per-vector scale factors increases. To alleviate these overheads, VSQ \citep{dai2021vsq} proposed a second level quantization of the per-vector scale factors to unsigned integers, while MX \citep{rouhani2023shared} quantizes them to integer powers of 2 (denoted as $2^{INT}$).

\subsubsection{MX Format}
The MX format proposed in \citep{rouhani2023microscaling} introduces the concept of sub-block shifting. For every two scalar elements of $b$-bits each, there is a shared exponent bit. The value of this exponent bit is determined through an empirical analysis that targets minimizing quantization MSE. We note that the FP format $E_{1}M_{b}$ is strictly better than MX from an accuracy perspective since it allocates a dedicated exponent bit to each scalar as opposed to sharing it across two scalars. Therefore, we conservatively bound the accuracy of a $b+2$-bit signed MX format with that of a $E_{1}M_{b}$ format in our comparisons. For instance, we use E1M2 format as a proxy for MX4.

\begin{figure}
    \centering
    \includegraphics[width=1\linewidth]{sections//figures/BlockFormats.pdf}
    \caption{\small Comparing LO-BCQ to MX format.}
    \label{fig:block_formats}
\end{figure}

Figure \ref{fig:block_formats} compares our $4$-bit LO-BCQ block format to MX \citep{rouhani2023microscaling}. As shown, both LO-BCQ and MX decompose a given operand tensor into block arrays and each block array into blocks. Similar to MX, we find that per-block quantization ($L_b < L_A$) leads to better accuracy due to increased flexibility. While MX achieves this through per-block $1$-bit micro-scales, we associate a dedicated codebook to each block through a per-block codebook selector. Further, MX quantizes the per-block array scale-factor to E8M0 format without per-tensor scaling. In contrast during LO-BCQ, we find that per-tensor scaling combined with quantization of per-block array scale-factor to E4M3 format results in superior inference accuracy across models. 


\end{document}
