\section{Related Works}
\subsection{Long-Form Audio Generation}
Related work mainly focus on training-based methods, including Language Models (LMs) ____ and Diffusion Models (DMs) ____. LMs, typically based on auto-regressive architectures, face temporal causality constraints ____, leading to increasing accumulated errors and repetition issues when generating long audio. DMs mainly focus on 10-second durations ____, and while Make-An-Audio2 ____ supports variable-length generation, it struggles with longer durations. More recently, Stable Audio ____ is trained on long-form audio but it demands significant training costs and is sensitive to text prompts. Meanwhile, the exploration of joint diffusion generation in audio generation remains limited ____.

% Previous work on audio generation has primarily focused on training-based methods, which can be divided into Language Models (LMs) ____ and Diffusion Models (DMs) ____. For LMs, most works adopt auto-regressive architectures, which suffer from temporal causality constraints ____. When applied to long audio generation, these models are observed to accumulate increasing errors, leading to more severe repetition issues. For DMs, most prior works focus on 10-second durations ____. Although Make-An-Audio2 ____ allows variable-length generation, it shows weak capability for exploring longer durations. More recently, Stable Audio ____ has been trained directly on long-form audio sequences, but this demands significant training costs and is sensitive to text prompts.  On the other hand, the exploration of joint diffusion generation in audio generation remains limited ____.


\subsection{Panorama Generation}
Early relative training-free methods ____ mainly apply painting techniques with DMs, which easily cause repetition issues and are constrained by temporal causality ____. Recent advancements in panorama generation mainly based Joint Diffusion, as mentioned in Introduction. Meanwhile, we note a class of training-free methods, such as ScaleCrafter and DemoFusion ____, mainly applied to high-resolution image generation (e.g., upscaling portraits), which can also be considered a 2D length extrapolation problem. These methods focus on upscaling at the pixel level, while ensuring global structural coherence and avoiding object repetition. In contrast, this paper focuses on long-spectrum (e.g., concertos, soundscapes) and panorama generation (e.g., mountains, crowds), emphasizing smooth subview transitions, cross-view similarity-diversity balance, and extending more objects (e.g., pitches, harmonics).
% As these methods target a different problem, we focus our discussion on approaches relevant to long-spectrum and panorama generation.