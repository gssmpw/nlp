\clearpage
\onecolumn 
\section{Theoretical Analysis of Refer-Guided Swap for Cross-View Similarity-Diversity Balance}
\label{Theoretical}
Reference-Guided Latent Swap improves cross-view consistency comparing with independent denoising process with reference model directly. When SD-2.0 \cite{rombach2022high} is employed as the reference model $\Phi$, we have the following proposition, which describes the difference between two updated samples from arbitrary starting points $\mathbf{x}_{t_2}^{(1)}, \mathbf{x}_{t_2}^{(2)}\in \mathcal{X}$: 

\paragraph{Proposition}
\label{Proposition}
Recall that the approximated reversed VP-SDE \cite{song2020score} used for conditionally generation in SD-2.0 is:

\begin{equation}
\textrm{d} \mathbf{x}=\left[-\frac{1}{2}\beta(t)\mathbf{x} - \beta( t )s_{\theta}(\mathbf{x},t,y)  \right] \textrm{d} t+\sqrt{\beta( t )} \textrm{d} \tilde{\mathbf{w}},
\end{equation}

where $s_{\theta}(\mathbf{x},t,y)$ is a estimation for $\nabla_{\mathbf{x}} \log p_{t} ( \mathbf{x} \vert y )$, and $\tilde{\mathbf{w}}$ is a Wiener process when time flows backwards from $t=1$ to $t=0$. Denote that $\Phi_{t_2\rightarrow t_1}(\cdot \vert y)$ is the the sampling procedure from $t_2$ to $t_1$ condition on $y$ in SD-2.0, and $\sigma^2_{t_2 \rightarrow t_1}=-\int_{t_2}^{t_1}\beta(u)\textrm{d}u$. Assume that $\forall \mathbf{x}\in \mathcal{X}, \forall t\in[0,1], \forall y\in \mathcal{Y},  \left\|s_{\theta}(\mathbf{x},t,y)\right\|_2\leq C$, then $\forall 0\leq t_1 < t_2 \leq 1$, $\forall \mathbf{x}_{t_2}^{(1)}, \mathbf{x}_{t_2}^{(2)}\in \mathcal{X}$, $\forall y\in \mathcal{Y}$, $\forall \delta\in(0,1)$, with probability at least $(1-\delta)$, 

% \begin{equation}

\begin{equation}
\begin{aligned}
\label{eq1}
    \left\| \Phi_{t_2\rightarrow t_1}\left(\mathbf{x}_{t_2}^{(1)} \vert y\right) 
    -\Phi_{t_2\rightarrow t_1}\left(\mathbf{x}_{t_2}^{(2)} \vert y\right) \right\|_2^2 
    &\leq \exp(\sigma^2_{t_2 \rightarrow t_1}) 
    \left[\left\|\mathbf{x}_{t_2}^{(1)}-\mathbf{x}_{t_2}^{(2)}\right\|_2
    +2C \left\|\int_{t_2}^{t_1}\exp\left(-\frac{1}{2}\sigma^2_{t_2 \rightarrow s}\right)\beta(s)\,\textrm{d}s\right\|_2
    \right]^2 \\
    &\quad +2\sigma^2_{t_2 \rightarrow t_1} 
    \left( d+2\sqrt{d\cdot(-\log\delta)}+2\cdot(-\log\delta) \right).
\end{aligned}
\end{equation} where $d$ is the number of dimensions of $\mathbf{x}_{t_2}^{(1)}, \mathbf{x}_{t_2}^{(2)}, \mathbf{x}_{t_2}^{(ref)}$. 

\paragraph{Proof}
Using method of variation of parameters, solution for pre-mentioned SDE (1), $\Phi_{t_2\rightarrow t_1}(\mathbf{x}_{t_2} \vert y)$, can be written as

\begin{equation}
\Phi_{t_2\rightarrow t_1}(\mathbf{x}_{t_2} \vert y)\\
=\exp(\frac{1}{2}\sigma^2_{t_2 \rightarrow t_1})\left[\mathbf{x}_{t_2}
-\int_{t_2}^{t_1}\exp(-\frac{1}{2}\sigma^2_{t_2 \rightarrow s})\beta(s)s_{\theta}(\mathbf{x}_{s},s,y)\textrm{d}s\right]\\
+\int_{t_2}^{t_1}\sqrt{\beta( t )} \textrm{d} \tilde{\mathbf{w}},
\end{equation}

so for $\forall \mathbf{x}_{t_2}^{(1)}, \mathbf{x}_{t_2}^{(2)} \in \mathcal{X}$, we have: 

\begin{equation}
\begin{aligned}
& \left\| \Phi_{t_2\rightarrow t_1}\left(\mathbf{x}_{t_2}^{(1)} \vert y\right)-\Phi_{t_2\rightarrow t_1}\left(\mathbf{x}_{t_2}^{(2)} \vert y\right) \right\|_2^2 \\
= & \|\exp(\frac{1}{2}\sigma^2_{t_2 \rightarrow t_1})\Big[(\mathbf{x}_{t_2}^{(1)}-\mathbf{x}_{t_2}^{(2)})-\int_{t_2}^{t_1}\exp(-\frac{1}{2}\sigma^2_{t_2 \rightarrow s})\beta(s)(s_{\theta}(\mathbf{x}_{s}^{(1)},s,y)-s_{\theta}(\mathbf{x}_{s}^{(2)},s,y))\textrm{d}s_2^2\Big] \\
& +\int_{t_2}^{t_1}\sqrt{\beta( t )} \textrm{d} \tilde{\mathbf{w}}_1-\int_{t_2}^{t_1}\sqrt{\beta( t )} \textrm{d} \tilde{\mathbf{w}}_2\|_2^2 \\
= &\exp(\sigma^2_{t_2 \rightarrow t_1})\left\|\mathbf{x}_{t_2}^{(1)}-\mathbf{x}_{t_2}^{(2)}+\int_{t_2}^{t_1}\exp(-\frac{1}{2}\sigma^2_{t_2 \rightarrow s})\beta(s)(s_{\theta}(\mathbf{x}_{s}^{(1)},s,y)-s_{\theta}(\mathbf{x}_{s}^{(2)},s,y))\textrm{d}s\right\|_2^2  \\
&+\left\|\int_{t_2}^{t_1}\sqrt{\beta( t )} \textrm{d} \tilde{\mathbf{w}}_1-\int_{t_2}^{t_1}\sqrt{\beta( t )} \textrm{d} \tilde{\mathbf{w}}_2\right\|_2^2 \\
\leq &\exp(\sigma^2_{t_2 \rightarrow t_1})\left[  \left\|\mathbf{x}_{t_2}^{(1)}-\mathbf{x}_{t_2}^{(2)}\right\|_2+\left\|\int_{t_2}^{t_1}\exp(-\frac{1}{2}\sigma^2_{t_2 \rightarrow s})\beta(s)(s_{\theta}(\mathbf{x}_{s}^{(1)},s,y)-s_{\theta}(\mathbf{x}_{s}^{(2)},s,y))\textrm{d}s\right\|_2\right]^2  \\
& +2\left\|\int_{t_2}^{t_1}\sqrt{\beta( t )} \textrm{d} \tilde{\mathbf{w}}\right\|_2^2.
\end{aligned}
\end{equation}


From the assumption over $s_{\theta}(\mathbf{x},t,y)$, we have:
\begin{equation}
\begin{aligned}
    & \left\| \Phi_{t_2\rightarrow t_1}\left(\mathbf{x}_{t_2}^{(1)} \vert y\right) 
    -\Phi_{t_2\rightarrow t_1}\left(\mathbf{x}_{t_2}^{(2)} \vert y\right) \right\|_2^2 \\ 
    = & \Bigg\|\exp\left(\frac{1}{2}\sigma^2_{t_2 \rightarrow t_1}\right) 
    \Bigg[(\mathbf{x}_{t_2}^{(1)}-\mathbf{x}_{t_2}^{(2)}) 
    - \int_{t_2}^{t_1} \exp\left(-\frac{1}{2} \sigma^2_{t_2 \rightarrow s} \right) 
    \beta(s) \big( s_{\theta}(\mathbf{x}_{s}^{(1)},s,y) 
    - s_{\theta}(\mathbf{x}_{s}^{(2)},s,y) \big) \textrm{d}s \Bigg] \\
    & + \int_{t_2}^{t_1} \sqrt{\beta( t )} \, \textrm{d} \tilde{\mathbf{w}}_1 
    - \int_{t_2}^{t_1} \sqrt{\beta( t )} \, \textrm{d} \tilde{\mathbf{w}}_2 \Bigg\|_2^2 \\
    = & \exp\left(\sigma^2_{t_2 \rightarrow t_1}\right) 
    \left\| \mathbf{x}_{t_2}^{(1)}-\mathbf{x}_{t_2}^{(2)} 
    + \int_{t_2}^{t_1} \exp\left(-\frac{1}{2} \sigma^2_{t_2 \rightarrow s} \right) 
    \beta(s) \big( s_{\theta}(\mathbf{x}_{s}^{(1)},s,y) 
    - s_{\theta}(\mathbf{x}_{s}^{(2)},s,y) \big) \textrm{d}s \right\|_2^2  \\
    & + \left\| \int_{t_2}^{t_1} \sqrt{\beta( t )} \, \textrm{d} \tilde{\mathbf{w}}_1 
    - \int_{t_2}^{t_1} \sqrt{\beta( t )} \, \textrm{d} \tilde{\mathbf{w}}_2 \right\|_2^2 \\
    \leq & \exp\left(\sigma^2_{t_2 \rightarrow t_1}\right) 
    \left[  \left\|\mathbf{x}_{t_2}^{(1)}-\mathbf{x}_{t_2}^{(2)}\right\|_2 
    + \left\| \int_{t_2}^{t_1} \exp\left(-\frac{1}{2} \sigma^2_{t_2 \rightarrow s} \right) 
    \beta(s) \big( s_{\theta}(\mathbf{x}_{s}^{(1)},s,y) 
    - s_{\theta}(\mathbf{x}_{s}^{(2)},s,y) \big) \textrm{d}s \right\|_2 \right]^2  \\
    & + 2 \left\| \int_{t_2}^{t_1} \sqrt{\beta( t )} \, \textrm{d} \tilde{\mathbf{w}} \right\|_2^2.
\end{aligned}
\end{equation}

Then we complete the proof for Proposition \ref{Proposition}.

\paragraph{Corollary}
If we use introduce reference-guided latent swap operation before updating, by the definition of $\text{Swap}(\cdot)$, we have $\forall 0\leq t_1 < t_2 \leq 1$, $\forall \mathbf{x}_{t_2}^{(1)}, \mathbf{x}_{t_2}^{(2)}, \mathbf{x}_{t_2}^{(ref)} \in \mathcal{X}$, $\forall y\in \mathcal{Y}$, $\forall \delta\in(0,1)$, with probability at least $(1-\delta)$, 
\begin{equation}
\begin{aligned}
\label{eq2}
   & \left\| \Phi_{t_2\rightarrow t_1}\left(\text{Swap}(\mathbf{x}_{t_2}^{(ref)},\mathbf{x}_{t_2}^{(1)}) \vert y\right) 
    -\Phi_{t_2\rightarrow t_1}\left(\text{Swap}(\mathbf{x}_{t_2}^{(ref)},\mathbf{x}_{t_2}^{(2)}) \vert y\right) \right\|_2^2
\\& \leq   \exp\left(\sigma^2_{t_2 \rightarrow t_1}\right) 
    \Bigg[ \left\|(1-W_{\textrm{swap}})\odot(\mathbf{x}_{t_2}^{(1)}-\mathbf{x}_{t_2}^{(2)})\right\|_2   + 2C \left\| \int_{t_2}^{t_1} \exp\left(-\frac{1}{2} \sigma^2_{t_2 \rightarrow s}\right) 
    \beta(s) \,\textrm{d}s \right\|_2 \Bigg]^2  \\ 
    & + 2\sigma^2_{t_2 \rightarrow t_1} 
    \left( d + 2\sqrt{d \cdot (-\log\delta)} + 2 \cdot (-\log\delta) \right).
\end{aligned}
\end{equation}



Comparing with Eq.\ref{eq1}, Eq.\ref{eq2} have a tighter upper bound, since $\text{Swap}(\mathbf{x}_{t_2}^{(ref)},\mathbf{x}_{t_2}^{(1)}),\text{Swap}(\mathbf{x}_{t_2}^{(ref)},\mathbf{x}_{t_2}^{(2)})$ shares the same part $W_{\textrm{swap}}\odot\mathbf{x}_{t_2}^{(ref)}$. This indicates that within a fixed time interval $[t_1, t_2]$, performing a reference-guided swap operation on the initial points before updating the sample points helps improve the similarity of the results. 

According to Eq.\ref{eq1} and Eq.\ref{eq2}, we can trade-off between similarity and diversity by tuning $r_\text{guide}$. As $r_\text{guide}$ increases, the swap operation is employed more frequently applied during the sampling process, leading to higher similarity across subviews. Conversely, an increase in the $L_2$ distance between the final subview images signifies enhanced sample diversity.

\section{Further Qualitative Comparison}
\label{sec:qual}
More qualitative results on the audio generation are in Fig. \ref{fig:audio_0} to \ref{fig:audio_3} and panorama generation are in Fig. \ref{fig:image_0} to \ref{fig:img4_5}.

\twocolumn 
% soundscape
\begin{figure*}[t]

    \centering
    \setlength{\belowcaptionskip}{-10pt} % 图表标题之下的间距
    \includegraphics[width=2.0\columnwidth]{appendix/fig/audio_0.pdf}

\vspace{-60pt}
\caption{Qualitative comparison on soundscape generation. MD* represent an enhanced MD method with triangular windows.}
\label{fig:audio_0}
\end{figure*}

\begin{figure*}[t]
    \centering
    \setlength{\belowcaptionskip}{-10pt} % 图表标题之下的间距
    \includegraphics[width=2.0\columnwidth]{appendix/fig/audio_5.pdf}

\vspace{-60pt}
\caption{Qualitative comparison on soundscape generation. MD* represent an enhanced MD method with triangular windows.}
\label{fig:audio_5}
\end{figure*}

\begin{figure*}[t]
    \centering
    \setlength{\belowcaptionskip}{-10pt} % 图表标题之下的间距
    \includegraphics[width=2.0\columnwidth]{appendix/fig/audio_2.pdf}

\vspace{-60pt}
\caption{Qualitative comparison on soundscape generation. MD* represent an enhanced MD method with triangular windows.}
\label{fig:audio_1}
\end{figure*}

%music
\begin{figure*}[t]
    \centering
    \setlength{\belowcaptionskip}{-10pt} % 图表标题之下的间距
    \includegraphics[width=2.0\columnwidth]{appendix/fig/soft_music_1.pdf}

\vspace{-60pt}
\caption{Qualitative comparison on music generation. MD* represent an enhanced MD method with triangular windows.}
\label{fig:soft_music_1}
\end{figure*}

\begin{figure*}[t]
    \centering
    \setlength{\belowcaptionskip}{-10pt} % 图表标题之下的间距
    \includegraphics[width=2.0\columnwidth]{appendix/fig/soft_music_2.pdf}

%sound effect
\vspace{-60pt}
\caption{Qualitative comparison on music generation. MD* represent an enhanced MD method with triangular windows.}
\label{fig:soft_music_2}
\end{figure*}

\begin{figure*}[t]
    \centering
    \setlength{\belowcaptionskip}{-10pt} % 图表标题之下的间距
    \includegraphics[width=2.0\columnwidth]{appendix/fig/soft_music_3.pdf}

\vspace{-30pt}
\caption{Qualitative comparison on music generation. MD* represent an enhanced MD method with triangular windows.}
\label{fig:soft_music_3}
\end{figure*}

%sound effect
\begin{figure*}[t]
    \centering
    \setlength{\belowcaptionskip}{-10pt} % 图表标题之下的间距
    \includegraphics[width=2.0\columnwidth]{appendix/fig/audio_4.pdf}

\vspace{-60pt}
\caption{Qualitative comparison on audio effect generation. MD* represent an enhanced MD method with triangular windows.}
\label{fig:audio_4}
\end{figure*}

\begin{figure*}[t]
    \centering
    \setlength{\belowcaptionskip}{-10pt} % 图表标题之下的间距
    \includegraphics[width=2.0\columnwidth]{appendix/fig/audio_1.pdf}
    
\vspace{-60pt}
\caption{Qualitative comparison on audio effect generation. MD* represent an enhanced MD method with triangular windows.}
\label{fig:audio_2}
\end{figure*}

\begin{figure*}[t]
    \centering
    \setlength{\belowcaptionskip}{-10pt} % 图表标题之下的间距
    \includegraphics[width=2.0\columnwidth]{appendix/fig/audio_3.pdf}
\vspace{-60pt}
\caption{Qualitative comparison on audio effect generation. MD* represent an enhanced MD method with triangular windows.}
\label{fig:audio_3}
\end{figure*}

% audio
% \begin{figure*}[t]
%     \centering
%     \setlength{\belowcaptionskip}{-10pt} % 图表标题之下的间距
%     \includegraphics[width=2.0\columnwidth]{appendix/fig/music_0.pdf}

% \vspace{-60pt}
% \caption{Qualitative comparison on music generation. MD* represent an enhanced MD method with triangular windows.}
% \label{fig:music_0}
% \end{figure*}


% \begin{figure*}[t]
%     \centering
%     \setlength{\belowcaptionskip}{-10pt} % 图表标题之下的间距
%     \includegraphics[width=2.0\columnwidth]{appendix/fig/music_1.pdf}

% \vspace{-60pt}
% \caption{Qualitative comparison on music generation. MD* represent an enhanced MD method with triangular windows.}
% \label{fig:music_1}
% \end{figure*}

% \begin{figure*}[t]
%     \centering
%     \setlength{\belowcaptionskip}{-10pt} % 图表标题之下的间距
%     \includegraphics[width=2.0\columnwidth]{appendix/fig/music_2.pdf}

% \vspace{-60pt}
% \caption{Qualitative comparison on music generation. MD* represent an enhanced MD method with triangular windows.}
% \label{fig:music_1}
% \end{figure*}

% \begin{figure*}[t]
%     \centering
%     \setlength{\belowcaptionskip}{-10pt} % 图表标题之下的间距
%     \includegraphics[width=2.0\columnwidth]{appendix/fig/music_3.pdf}

% \vspace{-60pt}
% \caption{Qualitative comparison on music generation. MD* represent an enhanced MD method with triangular windows.}
% \label{fig:music_3}
% \end{figure*}

% \begin{figure*}[t]
%     \centering
%     \setlength{\belowcaptionskip}{-10pt} % 图表标题之下的间距
%     \includegraphics[width=2.0\columnwidth]{appendix/fig/music_4.pdf}

% \vspace{-60pt}
% \caption{Qualitative comparison on music generation. MD* represent an enhanced MD method with triangular windows.}
% \label{fig:music_3}
% \end{figure*}

% \begin{figure*}[t]
%     \centering
%     \setlength{\belowcaptionskip}{-10pt} % 图表标题之下的间距
%     \includegraphics[width=2.0\columnwidth]{appendix/fig/music_5.pdf}

% \vspace{-60pt}
% \caption{Qualitative comparison on music generation. MD* represent an enhanced MD method with triangular windows.}
% \label{fig:music_5}
% \end{figure*}


% image
\begin{figure*}[t]
\centering
\setlength{\belowcaptionskip}{-10pt} % 图表标题之下的间距
\includegraphics[width=2.0\columnwidth]{appendix/fig/image_0.pdf}
\vspace{-60pt}
\caption{Qualitative comparison on panorama image generation. MD* represent an enhanced MD method with triangular windows.}
\label{fig:image_0}
\end{figure*}

\begin{figure*}[t]
    \centering
    \setlength{\belowcaptionskip}{-10pt} % 图表标题之下的间距
    \includegraphics[width=2.0\columnwidth]{appendix/fig/image_1.pdf}
\vspace{-60pt}
\caption{Qualitative comparison on panorama image generation. MD* represent an enhanced MD method with triangular windows.}
\label{fig:image_1}
\end{figure*}

\begin{figure*}[t]
    \centering
    \setlength{\belowcaptionskip}{-10pt} % 图表标题之下的间距
    \includegraphics[width=2.0\columnwidth]{appendix/fig/image_2.pdf}

\vspace{-60pt}
\caption{Qualitative comparison on panorama image generation. MD* represent an enhanced MD method with triangular windows.}
\label{fig:image_2}
\end{figure*}

\begin{figure*}[t]
    \centering
    \setlength{\belowcaptionskip}{-10pt} % 图表标题之下的间距
    \includegraphics[width=2.0\columnwidth]{appendix/fig/image_3.pdf}

\vspace{-60pt}
\caption{Qualitative comparison on panorama image generation. MD* represent an enhanced MD method with triangular windows.}
\label{fig:image_3}
\end{figure*}

\begin{figure*}[t]
    \centering
    \setlength{\belowcaptionskip}{-10pt} % 图表标题之下的间距
    \includegraphics[width=2.0\columnwidth]{appendix/fig/image_4.pdf}

\vspace{-60pt}
\caption{Qualitative comparison on panorama image generation. MD* represent an enhanced MD method with triangular windows.}
\label{fig:image_4}
\end{figure*}

\begin{figure*}[t]
    \centering
    \setlength{\belowcaptionskip}{-10pt} % 图表标题之下的间距
    \includegraphics[width=2.0\columnwidth]{appendix/fig/image_5.pdf}

\vspace{-60pt}
\caption{Qualitative comparison on panorama image generation. MD* represent an enhanced MD method with triangular windows.}
\label{fig:image_5}
\end{figure*}




% img4 

\begin{figure*}[t]
    \centering
    \setlength{\belowcaptionskip}{-10pt} % 图表标题之下的间距
    \includegraphics[width=2.0\columnwidth]{appendix/fig/img4_0.pdf}

\vspace{-60pt}
\caption{Qualitative comparison on panorama image generation. MD* represent an enhanced MD method with triangular windows.}
\label{fig:img4_0}
\end{figure*}

\begin{figure*}[t]
    \centering
    \setlength{\belowcaptionskip}{-10pt} % 图表标题之下的间距
    \includegraphics[width=2.0\columnwidth]{appendix/fig/img4_1.pdf}

\vspace{-60pt}
\caption{Qualitative comparison on panorama image generation. MD* represent an enhanced MD method with triangular windows.}
\label{fig:img4_1}
\end{figure*}

\begin{figure*}[t]
    \centering
    \setlength{\belowcaptionskip}{-10pt} % 图表标题之下的间距
    \includegraphics[width=2.0\columnwidth]{appendix/fig/img4_2.pdf}

\vspace{-60pt}
\caption{Qualitative comparison on panorama image generation. MD* represent an enhanced MD method with triangular windows.}
\label{fig:img4_2}
\end{figure*}

\begin{figure*}[t]
    \centering
    \setlength{\belowcaptionskip}{-10pt} % 图表标题之下的间距
    \includegraphics[width=2.0\columnwidth]{appendix/fig/img4_3.pdf}

\vspace{-60pt}
\caption{Qualitative comparison on panorama image generation. MD* represent an enhanced MD method with triangular windows.}
\label{fig:img4_3}
\end{figure*}

% \begin{figure*}[t]
%     \centering
%     \setlength{\belowcaptionskip}{-10pt} % 图表标题之下的间距
%     \includegraphics[width=2.0\columnwidth]{appendix/fig/img4_4.pdf}

% \vspace{-60pt}
% \caption{Qualitative comparison on panorama image generation. MD* represent an enhanced MD method with triangular windows.}
% \label{fig:img4_4}
% \end{figure*}

\begin{figure*}[t]
    \centering
    \setlength{\belowcaptionskip}{-10pt} % 图表标题之下的间距
    \includegraphics[width=2.0\columnwidth]{appendix/fig/img4_5.pdf}

\vspace{-60pt}
\caption{Qualitative comparison on panorama image generation. MD* represent an enhanced MD method with triangular windows.}
\label{fig:img4_5}
\end{figure*}

