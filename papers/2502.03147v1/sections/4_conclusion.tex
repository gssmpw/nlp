\section{Conclusion}
\label{sec:conclu}

In this study, we propose a retrieval-augmented approach to extend LLM-based TabICL from zero-shot and few-shot settings to any-shot scenarios.
This approach explores the potential of using text representations for tabular data learning, enables the creation of unique decision boundaries, and achieves highly competitive prediction performance across most tabular datasets.

Despite the unique strengths and promising potentials, we also acknowledge the limitations of this approach at the current stage, such as the absence of a universally effective retrieval policy, challenges in handling certain long-tail data distributions, and sub-optimal performance in several scenarios.
Given the demonstrated strengths of this approach, we believe that the potential of LLM-based TabICL is still in its early stages, and these limitations present valuable opportunities for future research and development.