%%%%%%%% ICML 2025 EXAMPLE LATEX SUBMISSION FILE %%%%%%%%%%%%%%%%%

\documentclass{article}

% Recommended, but optional, packages for figures and better typesetting:
\usepackage{microtype}
\usepackage{graphicx}
\usepackage{subfigure}
\usepackage{booktabs} % for professional tables


%%%%% NEW MATH DEFINITIONS %%%%%

% \usepackage{amsmath,amsfonts,bm}
\usepackage{amsmath,amsfonts}

\usepackage{pifont}


\newcommand{\R}{\mathbb{R}}


\def\va{{\mathbf{a}}}
\def\vg{{\mathbf{g}}}

% Sets
\def\sR{\mathbb{R}}
\def\sC{\mathbb{C}}
\def\sZ{\mathbb{Z}}
\def\sN{\mathbb{N}}
\def\sQ{\mathbb{Q}}

\def\sS{\mathcal{S}}



% Vectors
\def\vzero{{\mathbf{0}}}
\def\vone{{\mathbf{1}}}
\def\vmu{{\mathbf{\mu}}}
\def\vtheta{{\mathbf{\theta}}}
\def\va{{\mathbf{a}}}
\def\vb{{\mathbf{b}}}
\def\vc{{\mathbf{c}}}
\def\vd{{\mathbf{d}}}
\def\ve{{\mathbf{e}}}
\def\vf{{\mathbf{f}}}
\def\vg{{\mathbf{g}}}
\def\vh{{\mathbf{h}}}
\def\vi{{\mathbf{i}}}
\def\vj{{\mathbf{j}}}
\def\vk{{\mathbf{k}}}
\def\vl{{\mathbf{l}}}
\def\vm{{\mathbf{m}}}
\def\vn{{\mathbf{n}}}
\def\vo{{\mathbf{o}}}
\def\vp{{\mathbf{p}}}
\def\vq{{\mathbf{q}}}
\def\vr{{\mathbf{r}}}
\def\vs{{\mathbf{s}}}
\def\vt{{\mathbf{t}}}
\def\vu{{\mathbf{u}}}
\def\vv{{\mathbf{v}}}
\def\vw{{\mathbf{w}}}
\def\vx{{\mathbf{x}}}
\def\vy{{\mathbf{y}}}
\def\vz{{\mathbf{z}}}
\def\vzeta{{\mathbf{\zeta}}}

% Matrix
\def\mA{{\mathbf{A}}}
\def\mB{{\mathbf{B}}}
\def\mC{{\mathbf{C}}}
\def\mD{{\mathbf{D}}}
\def\mE{{\mathbf{E}}}
\def\mF{{\mathbf{F}}}
\def\mG{{\mathbf{G}}}
\def\mH{{\mathbf{H}}}
\def\mI{{\mathbf{I}}}
\def\mJ{{\mathbf{J}}}
\def\mK{{\mathbf{K}}}
\def\mL{{\mathbf{L}}}
\def\mM{{\mathbf{M}}}
\def\mN{{\mathbf{N}}}
\def\mO{{\mathbf{O}}}
\def\mP{{\mathbf{P}}}
\def\mQ{{\mathbf{Q}}}
\def\mR{{\mathbf{R}}}
\def\mS{{\mathbf{S}}}
\def\mT{{\mathbf{T}}}
\def\mU{{\mathbf{U}}}
\def\mV{{\mathbf{V}}}
\def\mW{{\mathbf{W}}}
\def\mX{{\mathbf{X}}}
\def\mY{{\mathbf{Y}}}
\def\mZ{{\mathbf{Z}}}
\def\mBeta{{\mathbf{\beta}}}
\def\mPhi{{\mathbf{\Phi}}}
\def\mLambda{{\mathbf{\Lambda}}}
\def\mSigma{{\mathbf{\Sigma}}}


% Expectation
% \def\eE{\mathop{\mathbb{E}}\limits}
\def\eE{\mathbb{E}}

% Probability
\def\pP{\mathbb{P}}

% Tilde
\def\tf{\tilde{f}}
\def\tS{\tilde{S}}
\def\wtF{\widetilde{\mathcal{F}}}
\def\whR{\widehat{R}}
\def\tvx{\tilde{\mathbf{x}}}
\def\ty{\tilde{y}}


\def\defeq{\overset{\textup{def}}{=}}
% \def\defeq{\overset{.}{=}}
\def\defone{\overset{\text{\ding{172}}}{=}}
\def\deftwo{\overset{\text{\ding{173}}}{=}}
\def\leqone{\overset{\text{\ding{172}}}{\leq}}
\def\leqtwo{\overset{\text{\ding{173}}}{\leq}}
\def\leqthree{\overset{\text{\ding{174}}}{\leq}}
\def\leqfour{\overset{\text{\ding{175}}}{\leq}}
\def\eqone{\overset{\text{\ding{172}}}{=}}
\def\eqtwo{\overset{\text{\ding{173}}}{=}}
\def\eqthree{\overset{\text{\ding{174}}}{=}}
\def\eqfour{\overset{\text{\ding{175}}}{=}}
\def\geqfive{\overset{\text{\ding{176}}}{\geq}}
% \usepackage{hyperref}
\usepackage{url}
% \usepackage{booktabs}
% \usepackage{graphicx}
\usepackage{subcaption}
\usepackage{longtable}


% hyperref makes hyperlinks in the resulting PDF.
% If your build breaks (sometimes temporarily if a hyperlink spans a page)
% please comment out the following usepackage line and replace
% \usepackage{icml2025} with \usepackage[nohyperref]{icml2025} above.
\usepackage{hyperref}


% Attempt to make hyperref and algorithmic work together better:
\newcommand{\theHalgorithm}{\arabic{algorithm}}

% Use the following line for the initial blind version submitted for review:
% \usepackage{icml2025}

% If accepted, instead use the following line for the camera-ready submission:
\usepackage[accepted]{icml2025}

% For theorems and such
\usepackage{amsmath}
\usepackage{amssymb}
\usepackage{mathtools}
\usepackage{amsthm}

% if you use cleveref..
\usepackage[capitalize,noabbrev]{cleveref}

%%%%%%%%%%%%%%%%%%%%%%%%%%%%%%%%
% THEOREMS
%%%%%%%%%%%%%%%%%%%%%%%%%%%%%%%%
\theoremstyle{plain}
\newtheorem{theorem}{Theorem}[section]
\newtheorem{proposition}[theorem]{Proposition}
\newtheorem{lemma}[theorem]{Lemma}
\newtheorem{corollary}[theorem]{Corollary}
\theoremstyle{definition}
\newtheorem{definition}[theorem]{Definition}
\newtheorem{assumption}[theorem]{Assumption}
\theoremstyle{remark}
\newtheorem{remark}[theorem]{Remark}

% Todonotes is useful during development; simply uncomment the next line
%    and comment out the line below the next line to turn off comments
%\usepackage[disable,textsize=tiny]{todonotes}
\usepackage[textsize=tiny]{todonotes}


% The \icmltitle you define below is probably too long as a header.
% Therefore, a short form for the running title is supplied here:
\icmltitlerunning{Scalable In-Context Learning on Tabular Data via Retrieval-Augmented Large Language Models}

\begin{document}

\twocolumn[
\icmltitle{Scalable In-Context Learning on Tabular Data\\ via Retrieval-Augmented Large Language Models}
% \icmltitle{Scalable In-Context Learning on Tabular Data via Retrieval-Augmented LLMs}

% It is OKAY to include author information, even for blind
% submissions: the style file will automatically remove it for you
% unless you've provided the [accepted] option to the icml2025
% package.

% List of affiliations: The first argument should be a (short)
% identifier you will use later to specify author affiliations
% Academic affiliations should list Department, University, City, Region, Country
% Industry affiliations should list Company, City, Region, Country

% You can specify symbols, otherwise they are numbered in order.
% Ideally, you should not use this facility. Affiliations will be numbered
% in order of appearance and this is the preferred way.
\icmlsetsymbol{equal}{*}
% \icmlsetsymbol{intern}{*}

\begin{icmlauthorlist}
% \icmlauthor{Firstname1 Lastname1}{equal,yyy}
% \icmlauthor{Firstname2 Lastname2}{equal,yyy,comp}
% \icmlauthor{Firstname3 Lastname3}{comp}
% \icmlauthor{Firstname4 Lastname4}{sch}
% \icmlauthor{Firstname5 Lastname5}{yyy}
% \icmlauthor{Firstname6 Lastname6}{sch,yyy,comp}
% \icmlauthor{Firstname7 Lastname7}{comp}
% %\icmlauthor{}{sch}
% \icmlauthor{Firstname8 Lastname8}{sch}
% \icmlauthor{Firstname8 Lastname8}{yyy,comp}
%\icmlauthor{}{sch}
% \icmlauthor{}{sch}
\icmlauthor{Xumeng Wen}{msra}
\icmlauthor{Shun Zheng}{msra}
\icmlauthor{Zhen Xu}{intern,uc}
\icmlauthor{Yiming Sun}{intern,up}
\icmlauthor{Jiang Bian}{msra}
\end{icmlauthorlist}

% \icmlaffiliation{yyy}{Department of XXX, University of YYY, Location, Country}
% \icmlaffiliation{comp}{Company Name, Location, Country}
% \icmlaffiliation{sch}{School of ZZZ, Institute of WWW, Location, Country}

\icmlaffiliation{msra}{Microsoft Research Asia, Beijing, China}
\icmlaffiliation{uc}{The University of Chicago, Chicago, IL, USA}
\icmlaffiliation{up}{University of Pittsburgh, Pittsburgh, PA, USA}
\icmlaffiliation{intern}{Zhen and Yiming contributed to this study during their internship at Microsoft Research Asia.}

\icmlcorrespondingauthor{Shun Zheng}{shun.zheng@microsoft.com}
% \icmlcorrespondingauthor{Firstname2 Lastname2}{first2.last2@www.uk}

% You may provide any keywords that you
% find helpful for describing your paper; these are used to populate
% the "keywords" metadata in the PDF but will not be shown in the document
\icmlkeywords{Tabular data, in-context learning, large language models}

\vskip 0.3in
]

% this must go after the closing bracket ] following \twocolumn[ ...

% This command actually creates the footnote in the first column
% listing the affiliations and the copyright notice.
% The command takes one argument, which is text to display at the start of the footnote.
% The \icmlEqualContribution command is standard text for equal contribution.
% Remove it (just {}) if you do not need this facility.

\printAffiliationsAndNotice{}  % leave blank if no need to mention equal contribution
% \printAffiliationsAndNotice{\icmlEqualContribution} % otherwise use the standard text.

\begin{abstract}
Recent studies have shown that large language models (LLMs), when customized with post-training on tabular data, can acquire general tabular in-context learning (TabICL) capabilities. These models are able to transfer effectively across diverse data schemas and different task domains.
However, existing LLM-based TabICL approaches are constrained to few-shot scenarios due to the sequence length limitations of LLMs, as tabular instances represented in plain text consume substantial tokens.
To address this limitation and enable scalable TabICL for any data size, we propose retrieval-augmented LLMs tailored to tabular data.
Our approach incorporates a customized retrieval module, combined with retrieval-guided instruction-tuning for LLMs.
This enables LLMs to effectively leverage larger datasets, achieving significantly improved performance across 69 widely recognized datasets and demonstrating promising scaling behavior.
Extensive comparisons with state-of-the-art tabular models reveal that, while LLM-based TabICL still lags behind well-tuned numeric models in overall performance, it uncovers powerful algorithms under limited contexts, enhances ensemble diversity, and excels on specific datasets. 
These unique properties underscore the potential of language as a universal and accessible interface for scalable tabular data learning.
% \footnote{Code and model checkpoints will be publicly available.}
\end{abstract}

\section{Introduction}
% 
Motion planning is a key ingredient in autonomous robotic systems, whose aim is computing collision-free trajectories for a robot operating in environments cluttered with obstacles~\cite{lavalle2006planning}. 
Over the years, various approaches have been developed for tackling the problem, including potential fields~\cite{luo2024potential}, geometric methods~\cite{halperin2017algorithmic}, and optimization-based approaches~\cite{SchulmanDHLABPPGA14,MalyutaEtAl2022,MarcucciEA23}. %, and sampling-based planners~\cite{}. 
In this work, we focus on sampling-based planners (SBPs), which aim to capture the structure of the robot's free space through graph approximations that result from configuration sampling (typically in a random fashion) and connecting nearby samples. 
SBPs have enjoyed popularity in recent years due to their relative scalability, in terms of the number of robot degrees of freedom (DoFs), and the ease of their implementation~\cite{OrtheyCK24}. 

\begin{figure*}[h!]
  \centering
  \subfloat[$\X_{\dZ_2}^{\delta,\epsilon}$ sample set.]{
    \includegraphics[width=0.27\textwidth, trim={2.2cm 1.7cm 0.9cm 1.0cm},clip]{Images/ZN_2D.png}
    %\label{fig:2d_lattices:z}
    }
  \hfil
  \subfloat[$\X_{D_2^*}^{\delta,\epsilon}$ sample set.]{
    \includegraphics[width=0.27\textwidth, trim={2.2cm 1.8cm 0.9cm 1.0cm},clip]{Images/DN_2D.png}
    %\label{fig:2d_lattices:d}
    }
  \hfil
  \subfloat[$\X_{A_2^*}^{\delta,\epsilon}$  sample set.]{
    \includegraphics[width=0.27\textwidth, trim={2.4cm 1.7cm 0.9cm 1.0cm},clip]{Images/AN_2D.png}
    %\label{fig:2d_lattices:a}
    }
  \caption{Sample sets within a fixed disc in $\dR^2$, derived from the lattices $\dZ^2, D_2^*$ and $A^*_2$, which yield \decomp guarantees for the same values of $\delta$ and $\eps$. The set $\X_{\dZ_2}^{\delta,\epsilon}$ can be viewed as a tessellation of space using cubes. The set $\X_{D_2^*}^{\delta,\epsilon}$ is obtained by placing a (rescaled) standard grid, and then placing another point in the middle of each cube. The set $\X_{A_2^*}^{\delta,\epsilon}$ can be viewed as a rescaled hexagonal grid as each point is surrounded by a hexagon whose vertices are points in the set. Note that the density of $\X_{\dZ^2}^{\delta,\eps}$ and $\X_{D^*_2}^{\delta,\eps}$ is the same, and higher than the density of $\X_{A^*_2}^{\delta,\eps}$.}
  \label{fig:2d_lattices}
\end{figure*}

Another key benefit is the ability of SBPs to escape local minima (unlike potential fields) and global solution guarantees (in contrast, optimization-based approaches~\cite{SchulmanDHLABPPGA14}, which typically provide only local guarantees). Earlier work on the theoretical foundations of SBPs has focused on deriving probabilistic completeness (PC) guarantees for methods such as PRM~\cite{kavraki1996probabilistic} or RRT~\cite{LaVKuf01,KunzS14,Kleinbort.Solovey.ea.19}. PC implies that the probability of a given planner finding a solution (if one exists) converges to one as the number of samples tends to infinity. The work of~\citet{karaman2011sampling} initiated studying the quality of the solution returned by SBPs. Specifically, they introduced the planners PRM* and RRT*, and proved that the solution length of those planners converges to the optimum as the number of samples tends to infinity---a property called asymptotic optimality (AO). Subsequent work has introduced even more powerful AO planners for geometric~\cite{JSCP15,GammellBS20} and dynamical~\cite{HauserZ16,LiETAL16} systems.

Unfortunately, the practical relevance of the aforementioned theoretical findings remains limited due to the lack of meaningful finite-time implications. Specifically, when a solution is obtained using a finite number of samples, it is unclear to what extent its quality can be improved with additional computation time. Moreover, in cases where no solution is returned, it is uncertain whether a solution does not exist or if the algorithm simply failed to find one. Developing finite-time bounds through randomized sampling continues to be a significant challenge~\cite{DobsonMB15,shaw2024towards}.

Deterministic sampling methods such as grid sampling or Halton sequences~\cite{lavalle2006planning}, where samples are generated according to a geometric principle, can improve the performance of SBPs in practice and simplify the algorithm analysis. Specifically, some deterministic sampling procedures have a significantly lower dispersion than uniform random sampling, which implies that the former requires fewer samples to cover the search space to a desired resolution~\cite{janson2018deterministic}. 
Recently, Tsao et al.~\cite{tsao2020sample} have leveraged deterministic sampling to disrupt the asymptotic analysis paradigm by introducing a significantly stronger notion than AO, called \decomps, that yields finite-time guarantees for PRM-based algorithms such as PRM*~\cite{karaman2011sampling}, FMT*~\cite{JSCP15}, BIT*~\cite{GammellBS20}, and GLS~\cite{MandalikaCSS19}. Informally, a \emph{finite} sample set is \decomp for a given approximation factor $\eps>0$ and clearance parameter $\delta>0$, if the corresponding planner returns a solution whose length is at most $(1+\eps)$ times the length of the shortest $\delta$-clear solution. If no solution is found using a \decomp sample set then no solution of clearance $\delta$ exists. 

The work of~\citet{tsao2020sample} derived a relation between \decomps and geometric space coverage to obtain lower bounds on the number of samples necessary to achieve \decomps, as well as upper bounds accompanied with explicit (deterministic) sampling distributions. A follow-up work by~\citet{dayan2023near} has introduced an even more compact \decomp sample distribution that is more efficient than the one proposed in~\cite{tsao2020sample} or rectangular grid sampling. In particular, the staggered grid~\cite{dayan2023near} consists of two shifted and rescaled copies of the rectangular grid (see Figure~\ref{fig:2d_lattices} and Figure~\ref{fig:3d_lattices}). 

However, the work~\cite{dayan2023near} still leaves a significant gap between the lower bound in~\cite{tsao2020sample} and the upper bound obtained with the staggered grid. In practice, this gap limits the applicability of the \decomps theory to relatively low dimensions (up to dimension 6) due to the large number of samples currently needed to satisfy this property, which can lead to excessive running times. 

\vspace{5pt}
\noindent \textbf{Contribution.} In this work, we develop a theoretical framework for obtaining highly-efficient \decomp sample sets by leveraging the foundational theory of lattices\footnote{Lattices are point sets exhibiting a regular geometric structure, which are obtained by transforming the integer lattice $\dZ^d$. For instance, the aforementioned rectangular grid and the staggered grid can be viewed as lattices.}~\cite{conway2013sphere}, which has been instrumental in diverse areas from number theory~\cite{siegel_geometry_numbers}, coding theory~\cite{ebeling2013lattices}, and crystallography~\cite{sands1994introduction}. Specifically, we show that lattices can be transformed to obtain \decomp sample sets (Theorem~\ref{thm:decomp_lattices}) and develop tight theoretical bounds on their size (Theorem~\ref{thm:general_sample_complexity}), which allows to compare between different sample sets qualitatively. 
Using this machinery, we not only refine and generalize previous results on the staggered grid~\cite{dayan2023near} but also introduce a new highly efficient \decomp sample set that is based on the $\AN$ lattice, which is famous for its minimalist coverage properties~\cite{conway2013sphere}. We also initiate the study of a new property, which estimates the computational cost resulting from using a given sample set in a more informative manner than sample complexity. In particular, the property called collision-check complexity captures the amount of collision checks, which is typically a computational bottleneck.

From a practical perspective, when solving motion-planning problems using lattice-based sample sets, we show that our $\AN$-based sample sets can result in at least order-of-magnitude improvement in terms of running time over staggered-grid samples and two orders of magnitude improvements over rectangular grids. Moreover, $\AN$-based sample sets are vastly superior in practice to the widely-used uniform random sampling, which is evident in improved running times, success rates, and solution quality.

\vspace{5pt}
\noindent \textbf{Organization.} In Section~\ref{sec:preliminaries} we review basic definitions on motion planning and \decomps, and formally define our objectives. In Section~\ref{sec:lattices}, we develop a general tool for transforming lattices into \decomp sample sets. We obtain sample-complexity bounds for lattice-based sample sets in Section~\ref{sec:sample_complexity}, and generalize those bounds to collision-check complexity in Section~\ref{sec:collision_complexity}. We evaluate the practical implications of our theory in Section~\ref{sec:experiments}, and conclude with a discussion of limitations and future directions in Section~\ref{sec:future}.


% \itai{Added the intro. used some from the thesis-proposal, added different stuff at the end}
%  the field of autonomous robots, the problem of getting a robot “from point A to point B” can be divided into three general stages: estimating the robot's position, planning the robot's path and controlling the robot. We use estimation methods (like the Kalman Filters) to understand where we are in the world, we use planning methods to figure out how to reach the goal, and we use control methods (like PID) to follow the planned path during execution.


% Focusing on the planning part of the problem, instead of using the \emph{workspace} of the robot it is convenient to use a representation of it called a \emph{configuration space}---a parameterization of the robot’s position in space, which turns the set of points defined as a \emph{robot} to a single-point robot. A quick example would be thinking of a polygon in the workspace as three parameters: $(x,y)\in \mathbb{R}^2$ for its location, and $\theta\in[0,2\pi]$ for its rotation, which means the configuration space is $\mathbb{R}^2\times S^1$. Furthermore, we use the term \emph{free space} in both contexts to describe the area of the space with no obstacles. 


% Even though using configuration spaces is much more convenient in terms of the robot being a single point, it quickly becomes apparent that even simple configuration spaces of dimensions $d\geq 3$ can be challenging to properly describe (due to the need to describe the obstacles in the new space, among other reasons). Thus, instead of explicitly representing the whole configuration space, methods were developed to sample the space: sampling-based approaches aim to approximate the space via a graph structure that is induced by sampled configurations. This can drastically reduce the computational effort of path planning.


% One of the most widely used sampling-based algorithms is the \emph{probabilistic roadmap method} (PRM)~\cite{kavraki1996probabilistic}. This approach generates (typically random) samples across the space, and connects nearby samples while checking for collisions with obstacles, which gives rise to a graph data structure---a path between two nodes in the graph yields a collision-free path for the robot connecting between the two configurations corresponding to the end-point nodes. PRM has the theoretical guarantee to return a path with a probability tending to 1 if enough samples are generated~\cite{laddgeneralizing}. 


% Another well known sampling-based planner is the \emph{rapidly-exploring random trees} (RRT)~\cite{lavalle1998rapidly}: It randomly expands towards nearby samples in space, creating in the process a “tree” structure that eventually finds a path to the goal~\cite{kleinbort2018probabilistic}. Later, a notion of \emph{asymptotically optimal} (AO) algorithms was introduced: with infinite samples, the algorithm can converge to an \textbf{optimal} path. Both PRM and RRT  were expanded to AO versions (PRM*, RRT*) in a paper by Karaman et al.~\cite{karaman2011sampling}. RRT itself had seen many expansions, including dRRT/dRRT* to apply to multiple robots~\cite{solovey2015finding, dobson2017scalable}.

% Approximately optimal methods were demonstrated using deterministic sample sets, achieving good results in finite time, in Dayan et al.'s paper~\cite{dayan2023near}, demonstrating superior results over random sets---although those improvements diminish as the desired approximation factor of the optimal path lowers.

% Still, all these methods have a main limitation: the number of points required to guarantee finding a path rises exponentially with the robot's degrees of freedom~\cite{tsao2020sample}.


% Dayan et al.~\cite{dayan2023near}, using a "staggered grid" structure (recognized in this paper as the $\DN$ set), gave guarantess for an approximately-optimal solution in finite time, which outperformed random sets in certain situations. For this, they introduced the concept of a \decomp set, a set that generates such approximate solutions. Still, as we seek better and better approximations for the optimal path, the staggered grid in the paper falls off against random sets. The question that stands, then, is what other sample sets can be used to provide better results?


% In this paper, we would like to utilize \decomp sets and investigate a specific series of deterministic sample sets, using lattices---a generalization of the regular grid structure using a general set of mutually-independent base vectors (not necessarily the usual $(0,\dots,1,\dots,0)$ vectors). We first familiarize the reader with three different lattices \Lattices we intend on investigating, and then move on to using Dayan et al.'s~\cite{dayan2023near} definition of a \decomp set to define lattice sample sets as such sets. This definition tells us that these sets can give us a good approximation for the optimal solution at a finite time. 


% After that, we use our new lattice sample sets to investigate the upper bounds on the number of sample points, and on the sum of edge length in a typical PRM vertices-connecting $r$-Ball---something we use as a measure point to the algorithm's complexity, as it is known that collision checks along the edges are the bottleneck in today's PRM algorithms.


% We will end up demonstrating, theoretically and practically, that one lattice, $\AN$, stands out as performing much better than the regular grid often used in many MP algorithms.
% \paragraph{Data-to-Text.} \citep{kukich-d2t, mckeown-d2t} is the task of converting structured data into fluent text. These structured data may correspond to tables \citep{totto}, meaning representations \citep{e2e}, relational graphs \citep{webnlg2017}, etc.
% %This complex format poses a significant challenge to LLMs pre-trained on plain text. 
% Recent approaches to data-to-text typically involve training end-to-end models with encoder-decoder architectures \citep{wiseman-etal-2017-challenges, gardent2017creating,RebuffelSSG20,RebuffelSSG20-ECIR,RebuffelRSSCG22}. Notably, using large pre-trained encoder-decoder models \citep{t5} has significantly improved performance by framing data-to-text as a text-to-text task \citep{kale-rastogi-2020-text, duong23a}. More recently, large pre-trained decoder-only models \citep{llama2} have shown strong performance and become the de facto approach for text generation, now being applied to data-to-text \citep{tablellama}. Despite these advancements, LLMs still struggle with hallucinations, and data-to-text generation is no exception.
This section reviews methods aimed at improving the faithfulness of LLMs to input contexts. We focus exclusively on approaches designed to ensure the generated content remains grounded in the provided information, excluding techniques related to factuality or external knowledge alignment.

\paragraph{Faithfulness enhancement.} Several methods have been used for improving faithfulness of text summarization. A first line of work consist in using external tools to retrieve key entities or facts form the source document and use these as weak labels during training \citep{zhang-etal-2022-improving-faithfulness}. \citet{faitful-improv} identify key entities using a Question-Answering system and modify the architecture of an encoder-decoder model to put more cross-attention weight on these entities. \citet{zhu-etal-2021-enhancing} propose to improve the faithfulness of summaries by extracting a knowledge graph from the input texts and embed it in the model cross-attention using a graph-transformer. Another line of work focuses on post-training improvements by bootstrapping model-generated outputs ranked by quality \citep{slic,brio,slic-nli}.
% \citet{zhang-etal-2022-improving-faithfulness} forces , \citet{faitful-improv} introduce a Question-Answering system enhanced encoder-decoder architecture, where the cross-attention in the decoder is directed towards key entities. \citet{zhu-etal-2021-enhancing} propose to improve the faithfulness of summaries by extracting a knowledge graph from the input texts and embed it in the model cross-attention using a graph-transformer.
Regarding data-to-text generation, \citet{RebuffelRSSCG22} propose a custom model architecture to reduce the effect of loosely aligned datasets, using token-level annotations and a multi-branch decoder model. The closest work to ours is from \citep{cao-wang-2021-cliff} which proposes a contrastive learning approach where synthetic samples are constructed using different tools like Named Entity Recognition (NER) models and back-translation.
%These approaches have been primarily designed and evaluated for text summarization. 
These approaches address specific forms of unfaithfulness and rely heavily on external tools such as NER or QA models, and are especially tailored for text summarization, while we target a more general focus. More recently, simpler methods that leverage only a pre-trained model have been proposed for summarization. \citet{cad,pmi} downweight the probabilities of tokens that are not grounded in the input context, using an auxiliary LM without access to the input context.
\citet{critic-driven} train a self-supervised classification model to detect hallucinations and guide the decoding process.  \cite{confident-decoding} propose a method to estimate the decoder's confidence by analyzing cross-attention weights, encouraging greater focus on the source during generation. Our method focuses on a decoder-only architecture and uses a single model, providing a streamlined and efficient approach specifically tailored for general conditional text generation tasks without the need for complex external tools.

\paragraph{Faithfulness evaluation.} Measuring faithfulness automatically is not straightforward. Traditional conditional text generation evaluation often relies on comparing the generated output to a reference text, typically measured using n-gram based metrics such as BLEU \citep{papineni-bleu} or ROUGE \citep{lin-2004-rouge}. However, reference-based metrics limitations are well known to correlate poorly with faithfulness \citep{fabbri-etal-2021-summeval,gabriel-etal-2021-go}. Both for summarization and data-to-text generation, new metrics evaluating the generation exclusively against the input context have been proposed, using QA models \citep{rebuffel-etal-2021-data,scialom-etal-2021-questeval} or entity-matching metrics \citep{nan-etal-2021-entity}. In this work, we evaluate primarily our models using recent NLI-related metrics \citep{alignscore, nli-d2t}, and LLM-as-a-judge, focusing on faithfulness \citep{gpt-chiang,gpt-gilardi}. For data-to-text generation, we also report the PARENT metric \citep{parent}, which computes n-gram overlap against elements of the source table cells.

%Additionally, corpora are often collected automatically, leading to divergences between the reference text and the actual input data. , since no direct comparison to the actual input source is actually performed. To address these issues, evaluation methods that take into account the input data have been proposed. \citet{parent} introduce PARENT, which computes the recall of n-gram overlap between the entities in the data and the candidate text. \citet{nli-d2t} develop an entailment metric using Natural Language Inference (NLI) models, where the generated text is compared directly to a simple verbalization of the data. The gold-standard still remains the human or human-like evaluation, conducted with powerful generalist LLMs. These metrics form the core focus of our work.

\paragraph{Preference tuning.} Recent instruction-tuned LLMs are often further refined through "human-feedback alignment" \citep{oaif}. These methods utilize human-crafted preference datasets, consisting of pairs of preferred and dispreferred texts $(\ywin, \ylose)$, typically obtained by collecting human feedback and ranking responses via voting. Recent work \citep{spin} uses the model's previous predictions in a self-play manner to iteratively improve the performance of chat-based models. Whether through an auxiliary preference model \citep{rlhf} or by directly tuning the models on the pairs \citep{dpo}, these approaches have demonstrated remarkable results in chat-based models. Our method leverages a preference framework without the need for human intervention and is specifically tailored for models trained on conditional text generation tasks.

% However, it remains unclear on what values the models are being aligned. Some works have shown that these methods can effectively alter the model's behaviour to the extent that they become useless and refuse to answer to any requests. In this work, we follow a preference fine-tuning scheme but tailored for input-aware tasks like data-to-text.


% !TEX root = ../main.tex

\section{Scene Graph Construction for Videos}
\label{sec:scene}
\begin{figure*}[t]
    \centering
    \includegraphics[width=\textwidth]{figures/overview.pdf}
    \vspace{-4mm}
    \caption{An overview of our zero-shot video caption generation pipeline. The pipeline consists of (a) frame-level caption generation using image VLMs, (b) textual scene graph parsing for each frame caption, (c) merging of scene graphs into a unified graph, and (d) video-level caption generation through our graph-to-text model. Our proposed framework leverages frame-level scene graphs to produce detailed and coherent video captions.
    }
    \label{fig:framework}
\end{figure*}

Our objective is to effectively extend the capabilities of image-based vision-language models (VLMs) to the video domain without relying on video-text training. 
To this end, we introduce a novel video captioning framework that combines image VLMs with scene graph structures, as shown in Figure~\ref{fig:framework}.
The proposed method consists of four key steps: 1) generating captions for each frame using an image VLM, 2) converting these captions into scene graphs, 3) consolidating the scene graphs from all frames into a unified graph, and 4) generating comprehensive descriptions from this unified graph. 
This algorithm enables the generation of coherent and detailed video captions, bridging the gap between image and video understanding.

\subsection{Generating image-level captions}
\label{sub:generating}
We obtain image-level captions from a set of sparsely sampled frames using the open-source image VLM, LLAVA-NEXT-7B~\cite{liu2024llavanext}.
This model is selected for its strong performance across multiple benchmarks.
Our approach, however, is flexible and can incorporate any image-based VLM, including proprietary, closed-source models, as long as APIs are accessible.
The model is prompted to generate sentences optimized for scene graph construction, which are subsequently parsed into scene graphs.

\subsection{Parsing captions into scene graphs}
A scene graph $G = (\mathcal{O}, \mathcal{E})$ is defined by a set of objects, $\mathcal{O} = \{o_1, o_2, \ldots \}$, and a set of edges, $\mathcal{E}$.
Each object $o_i = (c_i, \mathcal{A}_i)$ consists of an object class $c_i \in \mathcal{C}$ and a set of attributes $\mathcal{A}_i \subseteq A$, where $\mathcal{C}$ is a set of object classes and $\mathcal{A}$ is a set of all possible attributes.
A directed edge, $e_{i,j} \equiv (o_i, o_j) \in \mathcal{E}$, has a label $r \in \mathcal{R}$, specifying the relationship from one object to the other.
All the values of object classes, attributes, and relationship labels, are text strings.

We convert the generated caption from each frame into a scene graph, providing more structured understanding of individual frames. 
By expressing the visual content in each frame using a graph based on detected objects and their relationships, we can apply a graph merging technique to produce a holistic representation of the entire input video.
We parse a caption into a scene graph using a textual scene graph parser, specifically the FACTUAL-MR parser~\cite{li-etal-2023-factual} in our implementation.

\subsection{Scene graph consolidation}
\label{sub:scene}
% !TEX root = ../main.tex

\begin{algorithm}[t]
\caption{Scene graph merging}
\label{alg:hierarchical_graph_merge}
\begin{algorithmic}[1]

  \STATE \textbf{Input:} 
  \STATE \quad $\mathcal{Q} = [ G_1, G_2, \dots, G_n ]$: a priority queue with frame-level scene graphs
  \STATE \quad $\phi(\cdot)$: a graph encoder
  \STATE \quad $\psi_i(\cdot)$: a function returning the $i^\text{th}$ object in a graph
  \STATE \quad $\pi$: a permutation function
  \STATE \quad $\tau$: a threshold

  \STATE \textbf{Output:} $G_{\text{video}}$: a video-level scene graph

  \WHILE{$|\mathcal{Q}| > 1$}
    \STATE $G^s = (\mathcal{O}^s, \mathcal{E}^s) \gets \text{dequeue}(\mathcal{Q})$
    \STATE $G^t = (\mathcal{O}^t, \mathcal{E}^t) \gets \text{dequeue}(\mathcal{Q})$
    \STATE $G^m = (\mathcal{O}^m, \mathcal{E}^m) \gets (\mathcal{O}^s \cup \mathcal{O}^t, \mathcal{E}^s \cup \mathcal{E}^t)$

    \STATE $\pi^* \gets \displaystyle \arg\max_{\pi \in \Pi} \sum_{i} 
      \frac{\psi_i(\phi(G^s))}{\lVert \psi_i(\phi(G^s)) \rVert} \; \cdot \;
      \frac{\psi_i(\phi(G_{\pi}^t))}{\lVert \psi_i(\phi(G_{\pi}^t)) \rVert}$

    \FOR{$(p, q) \in \mathcal{M}$ such that $s_{p, q} > \tau$}
      \STATE $\hat{c} \gets \text{update\_class}(c^s_p, c^t_q)$
      \STATE $\hat{o} \gets (\hat{c}, \mathcal{A}^s_p \cup \mathcal{A}^t_q)$
      \STATE $\mathcal{O}^m \gets \{\hat{o}\} \cup \bigl(\mathcal{O}^m \setminus \{o^s_p, o^t_q\}\bigr)$
      \STATE \textbf{for each} $(o_x, o_y) \in \mathcal{E}^m$:
      \STATE \quad $(o_x, o_y) \mapsto 
        \begin{cases}
           (\hat{o}, o_y), & \text{if } o_x \in \{ o_p^s, o_q^t \}; \\
           (o_x, \hat{o}), & \text{if } o_y \in \{ o_p^s, o_q^t \}; \\
           (o_x, o_y), & \text{otherwise.}
        \end{cases}$
    \ENDFOR

    \STATE $\mathcal{Q} \gets \text{enqueue}(\mathcal{Q}, G^m)$
  \ENDWHILE

  \STATE $G_{\text{video}} \gets \text{dequeue}(\mathcal{Q})$
  \STATE \textbf{return} $G_{\text{video}}$

\end{algorithmic}
\end{algorithm} 
The scene graph consolidation step combines all frame-level scene graphs into a single graph that captures the overall visual content of the video. 
We outline our graph merging procedure, followed by a subgraph extraction technique for more focused video caption generation.

\subsubsection{Video-level graph integration}

Given two scene graphs, $G^s = (\mathcal{O}^s, \mathcal{E}^s)$ and $G^t = (\mathcal{O}^t, \mathcal{E}^t)$, constructed from two different frames, we perform the Hungarian matching between their object sets, $\mathcal{O}^s$ and $\mathcal{O}^t$.
The Hungarian algorithm aims to find the maximum matching between the objects in $\mathcal{O}^s$ and $\mathcal{O}^t$, which is given by
%
\begin{equation}
	\pi^* = \underset{\pi \in \Pi}{\arg\max} \sum_{i} \frac{ \psi_i(\phi(G^s))}{\| \psi_i(\phi(G^s)) \|} \cdot \frac{\psi_i(\phi(G_\pi^t)) }{\| \psi_i(\phi(G_\pi^t)) \|},
\end{equation}
%
where $\phi(\cdot)$ denotes the graph encoder, $\psi_i(\cdot)$ is the function to extract the $i^\text{th}$ object from an embedded graph, and $\pi \in \Pi$ indicates a permutation of objects in a graph.
Note that we introduce dummy objects  to deal with different numbers of objects for matching.

After identifying a set of matching object pairs, $\mathcal{M}$, \eg, $(p, q)$, where $o_p^s \in \mathcal{O}^s$ and $o_q^t \in \mathcal{O}^t$, using their cosine similarity with a predefined threshold, $\tau$, we merge the matched objects into a new one $\hat{o} \in \hat{\mathcal{O}}$, which is given by
%
\begin{equation}
    \hat{o} = (\hat{c} , \mathcal{A}^s_p \cup \mathcal{A}^t_q) \in \hat{\mathcal{O}},
\end{equation}
%
where $\hat{c}$ represents a class of the merged objects and $\hat{\mathcal{O}}$ denotes a set of new objects from all legitimate matching pairs.

Using this, we construct a new merged scene graph, $G^m$, which replaces each pair of merged objects with a new object $\hat{o}$, as follows:
%
\begin{equation}
	G^m = (\mathcal{O}^m, \mathcal{E}^m),
\end{equation}
%
where $\mathcal{O}^{m} =\mathcal{O}^s \cup \mathcal{O}^t \cup \hat{\mathcal{O}} ~ \setminus \bigcup_{(p, q) \in \mathcal{M}} \{o^s_p, o^t_q\}$, and the edge set $\mathcal{E}^m$ is also updated to reflect the changes in the object configuration.
Formally, each matching pair $(p, q) \in \mathcal{M}$ incurs the merge of the two objects and the construction of a new object $\hat{o}$, which results in the update of the edge set as $\mathcal{E}^m \equiv \mathcal{E}^s \cup \mathcal{E}^t$, which is formally given by
%
\begin{equation}
	(o_x, o_y) \in \mathcal{E}^m \rightarrow 
	\begin{cases}
		(\hat{o}, o_y) & \text{if } o_x \in \{o_p^s, o_q^t \}, \\
		(o_x, \hat{o}) & \text{if } o_y \in \{o_p^s, o_q^t \}, \\
		(o_x, o_y) & \text{otherwise.}
	\end{cases}
\end{equation}

We perform graph merging using a priority queue, where pairs of graphs are prioritized for merging based on their embedding similarity. 
In each iteration, the two most similar graphs are dequeued, merged, and the resulting graph is enqueued back into the priority queue.
This process is repeated until only one scene graph remains.
The final scene graph provides a comprehensive representation of the video, preserving frame-level details often overlooked by standard captioning models.
Algorithm~\ref{alg:hierarchical_graph_merge} describes the detailed procedure of our graph merging strategy.
  
\subsubsection{Prioritized subgraph extraction}
To generate concise and focused video captions, we apply subgraph extraction to retain only the most contextually relevant information. 
During the graph merging process, we track each node's merge count as a measure of its significance within the consolidated graph. 
We then identify the top $k$ nodes with the highest merge counts and extract their corresponding subgraphs. 
This approach prioritizes objects that consistently appear across multiple frames, as they often represent key entities in the scene. 
By emphasizing these essential elements and filtering out less relevant details, our method constructs a compact scene graph to generate a more focused video caption.

\section{Video Captioning}
\label{sec:videocaption}
To generate video-level descriptions that accurately reflect visual content, we developed a model that takes scene graphs as input and produce natural language descriptions.
This model is designed to effectively capture key components and relationships within the scene graph in generated text.

\vspace{-2mm}
\paragraph{Architecture}
We employ a modified encoder-decoder transformer architecture.
To prepare the input sequence for the graph encoder, each node, edge, and attribute in the graph, represented as a word or phrase, is tokenized into NLP tokens. 
These tokens are mapped to their embeddings via an embedding lookup.
For nodes consisting of multiple NLP tokens, their embeddings are averaged to form a single vector representation.
Additionally, a [CLS] token is appended as a global node to prevent isolation among disconnected components and ensure coherence. 
The adjacency matrix serves as an attention mask, incorporating graph topology into the attention mechanism. 
The graph encoder's output is then used as key and value inputs for the cross-attention layers of the text decoder, which generates the final outputs.

\vspace{-2mm}
\paragraph{Dataset}
For training, we collected approximately 2.5M text corpora that cover diverse visual scene contexts from various sources, including image caption datasets such as  MS-COCO~\cite{chen2015microsoft}, Flickr30k~\cite{young2014image}, TextCaps~\cite{sidorov2020textcaps}, Visual Genome~\cite{krishna2017visual}, and Visual Genome paragraph captioning~\cite{krause2016paragraphs}.
To further enhance the dataset, we incorporated model-generated captions for Kinetics-400~\cite{kay2017kinetics} dataset, with four uniformly sampled frames per video.
Note that neither the datasets nor the image VLMs used for generating frame captions are related to the target video captioning benchmarks.


\vspace{-2mm}
\paragraph{Training}
The model is trained using a next-token prediction objective, aiming to reconstruct the source text conditioned on the scene graph:
%
\begin{equation}
\mathcal{L}(\theta) = \sum_{i=1}^{N} \log P_{\theta}(t_i \mid t_{1:i-1}, G),
\end{equation}  
%
where $t_i$ represents the $i^\text{th}$ token in the source text, and $N$ denotes the total number of tokens.


\vspace{-2mm}
\paragraph{Video caption generation}
After constructing the video-level scene graph as described in Section~\ref{sec:scene}, we generate a video caption using the trained graph-to-text decoder, which conveys the overall narrative of the video.

\section{Experiments}
\label{sec:exp}

We conduct extensive experiments to address the following research questions:
1) How effectively can our retrieval mechanism enhance LLM-based TabICL in leveraging large-scale datasets?
2) How does LLM-based TabICL perform in comparison to numeric-based TabICL models and classic tabular models that are well-tuned on a case-by-case basis?
3) What are the unique strengths, current limitations, and potential future directions for LLM-based TabICL?


% \begin{figure*}[t]
% \vskip 0.2in
% \begin{center}
% \centerline{\includegraphics[width=\linewidth]{main_figures/main_scaling_pool_and_context.pdf}}
% \caption{
% We investigate the effects of increasing the number of training instances ($|D_{\text{train}}^{T'}|$) and the number of in-context instances per test example ($N^C$) on the TabICL performance of Phi3-GTL models.
% In each subplot, we compare the scaling effects of two Phi3-GTL models with different retrieval policies: one that randomly selects in-context instances, denoted as "Random," and the other employing our default \texttt{TabRAG} module, denoted as "RAG".
% We use violin plots to visualize the performance distribution across multiple held-out datasets. Additionally, dashed lines are used to emphasize that the median prediction error of our approach follows a power-law relationship with the number of training instances.
% }
% \label{fig:scaling_pool_ctx}
% \end{center}
% \vskip -0.2in
% \end{figure*}


\subsection{Experimental Setups}
\label{sec:exp_setup}

\paragraph{LLM Post-Training}
We use real-world tabular datasets to post-train a base LLM using generative tabular learning (GTL) objective as did in~\citep{wen2024GTL}.
However, unlike their approach, we adopt Phi-3~\citep{abdin2024phi3} as the base LLM, extending the effective context length from 4K to 128K and aligning it with our default retrieval policy.
Details of this post-training process are provided in Appendix~\ref{app:method_align_rag_llm}. For brevity and clearness, we denote our post-trained model as Phi3-GTL and refer to our approach as RAG+Phi3-GTL throughout the remainder of this paper.

\paragraph{Held-out Datasets}
We compile a comprehensive benchmark from the literature~\citep{gorishniy2021revisit_tab_dnn,grinsztajn2022tree_gt_tab_nn,gorishniy2024TabR,wen2024GTL}, ensuring diverse datasets that may favor different learning paradigms.
To avoid data leakage, we carefully examine and exclude any datasets used during the training of the Phi3-GTL model.
This process results in 29 classification datasets and 40 regression datasets for held-out evaluation, covering a wide range of domains, feature dimensions, types, and distributions.
Details of the data construction process are included in Appendix~\ref{app:data_constr}.

\begin{figure*}[t]
\vskip 0.2in
\begin{center}
\centerline{\includegraphics[width=0.98\linewidth]{main_figures/main_overall_comp_raw_metric.pdf}}
\caption{
An overall performance comparison of all models. In the left subplot, we use violin plots to show the AUROC scores of different models across 29 classification tasks, while the right subplot displays the NMAE scores for 40 regression tasks. Models are sorted by their median metric score across the held-out datasets, with dashed lines indicating these median scores in each subplot. Our approach, RAG+Phi3-GTL, is prefixed with a marker (*), for quick identification.
}
\label{fig:overall_comp}
\end{center}
\vskip -0.2in
\end{figure*}

\begin{figure*}[t]
\vskip 0.2in
\begin{center}
\centerline{\includegraphics[width=0.90\linewidth]{main_figures/main_ensemble_norm_metric.pdf}}
\caption{
    Ensemble performance comparisons of RAG+Phi3-GTL, TabPFN-v2, LightGBM, and CatBoost are presented, where normalized AUROC or NMAE scores (min-max normalized across methods for each dataset) are plotted to highlight their relative strengths across multiple datasets, while omitting absolute metric differences.
}
\label{fig:ensemble_res}
\end{center}
\vskip -0.2in
\end{figure*}


\paragraph{Baselines}
We include Phi3-GTL and RAG+KNN as two ablated variants of RAG+Phi3-GTL: the former uses randomly selected in-context instances, while the latter employs the same default retrieval policy but relies on the K-Nearest Neighbors (KNN) algorithm~\citep{fix1951knn,cover1967knn_cls} for prediction.
We compare against TabPFN-v1~\citep{hollmann2023TabPFN}, which supports only classification tasks, and TabPFN-v2~\citep{hollmann2025TabPFNv2}, the state-of-the-art TabICL model utilizing numeric representations.
In addition, our baselines include other representative tabular models such as XGBoost~\citep{chen2016XGBoost}, LightGBM~\citep{ke2017LightGBM}, CatBoost~\citep{prokhorenkova2018catboost}, MLP, FTT~\citep{gorishniy2021revisit_tab_dnn}, and TabR~\citep{gorishniy2024TabR}, all of which are extensively tuned via hyperparameter search for each dataset.
We also include several ``RAG + X'' baselines, where "X" represents models trained and inferred on the selected in-context instances using the same retrieval policy as RAG+Phi3-GTL. These include Logistic Regression (LR), TabPFN-v1, TabPFN-v2, and XGBoost.
These baselines are designed to highlight the TabICL capability of LLMs given limited in-context instances.

\paragraph{Metrics}
For classification tasks, we use the Area Under the Receiver Operating Characteristic curve (AUROC) as the primary evaluation metric. For regression tasks, we employ the Mean Absolute Error normalized by the label mean (NMAE).
Additionally, to compare the relative performance across a group of methods, we utilize group-wise min-max normalized AUROC and NMAE metrics.


\begin{figure*}[t]
\vskip 0.2in
\begin{center}
\centerline{\includegraphics[width=\linewidth]{main_figures/main_per_dataset_comp.pdf}}
\caption{
    Per-dataset performance comparisons between RAG+Phi3-GTL and the two most competitive baselines, TabPFN-v2 and CatBoost, are presented, with dataset IDs sorted by performance gaps. Dashed lines and annotations are used to indicate the proportion of datasets where RAG+Phi3-GTL outperforms these baselines and where it significantly lags behind.
}
\label{fig:per_dataset_comp}
\end{center}
\vskip -0.2in
\end{figure*}


\begin{figure*}[t]
\vskip 0.2in
\begin{center}
% \centerline{\includegraphics[width=0.9\linewidth]{main_figures/main_decision_boundary.pdf}}
\centerline{\includegraphics[width=1.0\linewidth]{main_figures/main_decision_boundary.pdf}}
\caption{
Decision boundary comparisons of various models, where each row corresponds to a specific set of training instances generated from a given data distribution. The first column visualizes these training instances, while the subsequent columns illustrate the decision boundaries of different models. The top two rows represent the same data distribution but with varying numbers of training instances, whereas the bottom two rows depict a different data distribution.
}
\label{fig:decision_boundary}
\end{center}
\vskip -0.2in
\end{figure*}


\subsection{Scaling with Available Training Instances}
\label{sec:exp_scaling_pool_ctx}

Figure~\ref{fig:scaling_pool_ctx} illustrates the performance variations as the size of the training data and the number of in-context instances increase for our approach under two retrieval policies: Random and RAG (our default retrieval policy).
It is evident that the RAG policy enables Phi3-GTL to effectively leverage larger training datasets, while the Random policy lacks this capability. Specifically, with the RAG policy, the median prediction error demonstrates a power-law relationship with the number of training instances, expressed as $L(D) = (D_c / D)^{\alpha}$. For classification tasks, $L=1-\text{AUROC}$, $D_c \sim 6.05e^{-5}$, and $\alpha \sim 0.102$, whereas for regression tasks, $L=\text{NMAE}$, $D_c \sim 8.05e^{-8}$, and $\alpha \sim 0.053$. This finding highlights a favorable statistical learning characteristic: given a distinguishable feature space and sufficient training instances, the expected prediction error approaches zero.

Moreover, the RAG policy reduces the number of in-context instances required for accurate predictions. As shown in the right two subplots of Figure~\ref{fig:scaling_pool_ctx}, the model using the Random policy benefits significantly from an increased number of in-context instances. In contrast, with the RAG policy, performance often saturates as the number of adaptive in-context instances increases. This indicates that, for most datasets, tens of training instances are sufficient to form a supportive context for inferring the label of a test instance.


\subsection{Overall Comparison}
\label{sec:exp_overall_comp}


Figure~\ref{fig:overall_comp} presents an overall comparison of all models by illustrating the error distributions across held-out datasets. TabPFN-v2 emerges as the most competitive baseline in terms of the median prediction error. However, its wider error bars indicate sub-optimal performance in certain cases. In contrast, well-tuned tree-based models such as LightGBM and CatBoost, as well as neural models like FTT and TabR, demonstrate more robust performance with narrower error distributions.

When comparing our approach, RAG+Phi3-GTL, with these baselines, we observe significant improvements over its ablated variants, Phi3-GTL and RAG+KNN, underscoring the importance of both retrieval and TabICL components. Furthermore, RAG+Phi3-GTL is among the top-performing of ``RAG + X'' baselines, even surpassing RAG+TabPFN-v2 in terms of median prediction performance across held-out datasets.
This highlights the potential of TabICL based on text representations, which can uncover novel and highly effective ICL algorithms by operating in a text space.
Besides, our approach achieves zero NMAE for a specific integer regression task without requiring explicit programming, whereas all numeric models, by default, produce float outputs.
Lastly, RAG+Phi3-GTL still lags behind well-tuned baseline models and TabPFN-v2 in overall performance.


\subsection{Ensemble Results}
\label{sec:exp_ensemble_res}

We further explore the potential of RAG+Phi3-GTL by investigating its contribution to ensemble diversity, as shown in Figure~\ref{fig:ensemble_res}.
When comparing RAG+Phi3-GTL with TabPFN-v2, LightGBM, and CatBoost, we observe that although RAG+Phi3-GTL underperforms the top-performing baselines overall, it exhibits unique strengths in certain scenarios.
Moreover, a comparison of Ensemble-All with TabPFN+LightGBM and TabPFN+CatBoost reveals that these unique strengths translate into ensemble diversity, enhancing the robustness of overall ensemble performance.
These findings highlight the potential of leveraging language as an alternative interface for tabular data learning, complementing existing tabular learning algorithms.

\subsection{Per-dataset Comparisons}
\label{sec:exp_per_ds_comp}

These findings further motivate us to conduct per-dataset comparisons to identify datasets where RAG+Phi3-GTL excels and those where it still underperforms.
Figure~\ref{fig:per_dataset_comp} summarizes these results.
We observe that on approximately 17\%-20\% of datasets, RAG+Phi3-GTL outperforms the state-of-the-art TabICL model, TabPFN-v2, as well as the classic tree-based model, CatBoost, which has been carefully tuned for each dataset.
Furthermore, on over 80\% of datasets, the performance of RAG+Phi3-GTL falls within a small gap of these two competitive baselines.
These results indicate that RAG+Phi3-GTL is already a strong prediction model for most tabular datasets, suggesting that we could not only engage with a well-prepared LLM conversationally but also leverage it to understand tabular data, provide accurate predictions, and offer potential explanations.


\paragraph{Analysis of Failure Cases}
Per-dataset comparison results also prompt an investigation into why the performance of RAG+Phi3-GTL falls short in certain cases. Case studies detailed in Appendix~\ref{app:case_study} provide insights into these failure scenarios.
In summary, most failure cases are attributed to the limitations of the default retrieval policy, which struggles to extract effective in-context instances due to specific data characteristics (e.g., datasets R-25, R-33, and R-27). In these cases, we find that slight adjustments to the retrieval strategy—such as applying alternative numerical normalization methods for feature similarity calculations or leveraging prior knowledge to define instance similarities—can lead to significant performance improvements for RAG+Phi3-GTL.
These findings suggest that a non-parametric, default retrieval mechanism may be insufficient in certain scenarios. Practitioners could potentially achieve better performance by employing "retrieval engineering" (as discussed in Section~\ref{sec:method}).
Additionally, some failure cases, such as dataset C-17, reveal limitations in Phi3-GTL’s ability to perform effective TabICL on specific data distributions, where TabPFN-v2 significantly outperforms. We hypothesize that this gap arises from the limited coverage of data patterns during Phi3-GTL’s post-training phase, which utilized approximately 300 real-world datasets from~\citeauthor{wen2024GTL}. In contrast, TabPFN-v2 likely benefits from pre-training on a much broader family of synthesized datasets.


\subsection{Decision Boundary Analysis}
\label{sec:exp_dec_bound}

Figure~\ref{fig:decision_boundary} compares the decision boundaries of various models across four groups of synthetic instances.
We observe that RAG+Phi3-GTL produces a distinctive, non-smooth decision boundary, which is entirely different from models relying on numeric representations. This boundary reflects case-by-case generalization from known training instances to unseen regions, leaving more uncertain areas when data samples are sparse.

In terms of shape, the decision boundary of RAG+Phi3-GTL bears some resemblance to that of Nearest Neighbors. However, LLM-based TabICL generalizes far beyond a simple rule-based average of neighboring training instances. We hypothesize that this unique behavior arises from the text-based representation of tabular data and the ICL capability of LLMs.
These findings also highlight opportunities for further improving LLM-based TabICL. Specifically, to encourage smoother decision boundaries when sufficient training data is available, one approach could involve generating large-scale synthetic data and fine-tuning LLMs to emulate such behaviors.



% \subsection{Case Analysis}
% \label{sec:exp_case_study}

% Winning Case analysis against TabPFN v2:
% regression-cat-medium-0-analcatdata_supreme

% Failure Case analysis:
% lack of learning certain feature distributions
% fail to select the best in-context examples
% \subsection{Experiment settints}

% \textbf{Data preparation.} For evaluating the performance of tree-based models versus large language models (LLMs) and deep learning approaches on tabular data tasks, we prepare datasets drawn from three research branches—GTL~\cite{wen2024GTL}, TabR~\cite{gorishniy2024TabR}, and Tree~\cite{grinsztajn2022tree_gt_tab_nn}, encompassing 32 regression and 23 classification tasks. This diverse dataset collection enables robust and transparent comparisons across model classes. For GTL, we filtered 50 test datasets from an initial pool of 350, focusing on feature validation to avoid data leakage and ensure accurate classification of features. Only datasets meeting size and integrity criteria were retained, ensuring robust evaluations across diverse tabular tasks. In TabR, we reused GTL’s pretrained checkpoints, eliminating redundancy by de-duplicating datasets from GTL’s collection. Large datasets were downsampled to 100K samples to ensure computational efficiency without sacrificing representativeness, as detailed in Appendix~\ref{}. For Tree, datasets with numerical and categorical features were curated with a focus on eliminating overlap with GTL and TabR datasets. Multiple versions or splits were consolidated to a single representative version, ensuring a fair and unbiased comparison across model classes, and preserving dataset diversity.

% \textbf{Baselines.}  We provide below a list of our used baselines. For most baselines, a hyperparameter search space is carefully prepared by the literature and we just follow this space setting and perform hyperparameter optimization on this then ensemble the results to boost the performance.

% \begin{itemize}
%     \item \textbf{LR} (Linear/Logistic Regression): A simple, linear baseline for tabular tasks that struggles with nonlinearity, limiting its performance on more complex datasets.
%     \item \textbf{KNN} (K-Nearest-Neighbor): A non-parametric model that makes predictions based on nearby samples. We use the same context samples as retrieved by TabRAG, providing a simple baseline for context ensemble
%     \item \textbf{GBDT} (XGBoost~\cite{chen2016XGBoost}, LightGBM~\cite{ke2017LightGBM}, CatBoost~\cite{prokhorenkova2018catboost}): Gradient boosting frameworks that build sequential decision tree models to optimize performance, particularly excelling in large, imbalanced, or missing data. Extensive hyperparameter tuning is applied for optimal performance.
%     \item \textbf{MLP} (Multi-layer Perceptron): A basic neural network used to evaluate the effectiveness of deep learning on tabular data. We applied a range of hyperparameters and ensemble strategies to maximize its performance.
%     \item \textbf{SAINT}~\cite{somepalli2021SAINT}: A novel architecture that uses self-attention and contrastive learning to capture feature relationships within rows of tabular data. It enhances performance by focusing on important features and pre-training on large tabular datasets.
%     \item \textbf{FT-Transformer}~\cite{gorishniy2021revisit_tab_dnn}: Adapts transformer architecture for tabular data by tokenizing both categorical and numerical features. It uses learned embeddings and positional encodings to process features, enabling the transformer to handle tabular structures effectively.
%     \item \textbf{TabPFN}~\cite{hollmann2022TabPFN}: A probabilistic model that uses in-context learning, pretrained on a wide variety of small tabular datasets to learn common feature interactions, allowing predictions without explicit retraining. It struggles with large datasets and regression tasks.
%     \item \textbf{TabR}~\cite{gorishniy2024TabR}: A retrieval-based foundation model that encodes query datasets and searches for similar datasets in latent space, enabling it to generalize to new tasks by leveraging historical data. Its predictions benefit from learned embeddings and in-context retrieval.
% \end{itemize}


\section{Conclusion}
In this paper, we have introduced MME-CoT, a comprehensive benchmark designed to evaluate Chain-of-Thought reasoning in Large Multimodal Models. 
Our dataset comprises six categories to cover most scenarios of visual reasoning tasks.
To gain a thorough understanding of the reasoning process, we design a novel CoT evaluation suite with three metrics. 
Our systematic evaluation obtains useful insights into the issues within the current state-of-the-art Large Multimodal Models.
We identify critical flaws in all the tested open-source models.
As the field continues to evolve, MME-CoT stands as a valuable tool for measuring progress and identifying areas for improvement in the development of more sophisticated multimodal AI systems.

\section*{Impact Statement}
This paper presents work whose goal is to advance the field
of Computer Vision and Machine Learning. There are many
potential societal consequences of our work, none of which
we feel must be specifically highlighted here.

\section*{Impact Statement}

This study has the potential to make significant contributions to multiple research communities and practical domains.

For the tabular learning community, this study introduces a novel paradigm for performing TabICL by leveraging text representations of tabular data. This paradigm demonstrates the capability to process tabular datasets across diverse domains and scales, from zero-shot and few-shot to any-shot scenarios. By moving beyond numeric-only representations, our approach opens new avenues for integrating tabular learning into broader, more flexible frameworks, making it possible to unify methodologies across heterogeneous datasets. Furthermore, our retrieval-augmented mechanism provides an adaptable framework for balancing performance and scalability, which may inspire future innovations in retrieval strategies and model architectures.

For the LLM community, this study demonstrates the potential of LLMs to move beyond traditional language-based applications to data understanding and learning tasks involving structured, numeric data. Our findings reveal that LLMs, when augmented with TabICL capabilities, can generalize effectively to tabular datasets and produce competitive results. This highlights the possibility of extending LLMs to domains where structured data plays a central role, such as finance, healthcare, and agriculture. By bridging the gap between text-based and numeric-based data representations, our approach paves the way for a new class of multimodal LLMs capable of unifying text and tabular data learning.

The integration of TabICL capabilities into LLMs may also have ethical and societal implications. On the positive side, these models can democratize access to advanced analytics, enabling users with minimal technical expertise to analyze and understand complex datasets through natural language queries. However, there are potential risks associated with misuse or unintended consequences, such as over-reliance on LLMs for critical decisions, inaccuracies in predictions due to biased training data, and challenges in ensuring transparency and accountability. Researchers and practitioners must prioritize the development of robust evaluation methods, ethical safeguards, and explainability mechanisms to mitigate these risks and ensure responsible deployment.


\bibliography{main}
\bibliographystyle{icml2025}


%%%%%%%%%%%%%%%%%%%%%%%%%%%%%%%%%%%%%%%%%%%%%%%%%%%%%%%%%%%%%%%%%%%%%%%%%%%%%%%
%%%%%%%%%%%%%%%%%%%%%%%%%%%%%%%%%%%%%%%%%%%%%%%%%%%%%%%%%%%%%%%%%%%%%%%%%%%%%%%
% APPENDIX
%%%%%%%%%%%%%%%%%%%%%%%%%%%%%%%%%%%%%%%%%%%%%%%%%%%%%%%%%%%%%%%%%%%%%%%%%%%%%%%
%%%%%%%%%%%%%%%%%%%%%%%%%%%%%%%%%%%%%%%%%%%%%%%%%%%%%%%%%%%%%%%%%%%%%%%%%%%%%%%
\newpage
\appendix
\onecolumn

\clearpage
\setcounter{page}{1}
\maketitlesupplementary


\renewcommand{\thetable}{S\arabic{table}}
\renewcommand{\thefigure}{S\arabic{figure}}

\clearpage
\appendix

\onecolumn 

\section{Appendix}
\tableofcontents 
\clearpage


\begin{table*}[t]
\centering
\small
\renewcommand{\arraystretch}{1.2}
\setlength{\tabcolsep}{6pt}
\begin{tabular*}{\textwidth}{@{\extracolsep{\fill}} 
  >{\centering\arraybackslash}m{3cm}  % Sampling Mode
  >{\centering\arraybackslash}m{1.5cm} % Stride
  >{\centering\arraybackslash}m{3cm}   % Extrapolation Factor
  >{\centering\arraybackslash}m{2cm}   % Total NFE
  >{\centering\arraybackslash}m{3cm}   % Sampling Time [s]
  >{\centering\arraybackslash}m{1cm}   % FVD subcolumn 1
  >{\centering\arraybackslash}m{1cm}}  % FVD subcolumn 2
\toprule
\makecell{\textbf{Sampling}\\\textbf{Mode}} & 
\makecell{\textbf{Stride}} & 
\makecell{\textbf{Extrapolation}\\\textbf{Factor}} & 
\makecell{\textbf{Total}\\\textbf{NFE}} & 
\makecell{\textbf{Sampling}\\\textbf{Time [s]}} & 
\multicolumn{2}{c}{\textbf{FVD$\downarrow$}} \\
\cmidrule(rr){6-7}
 & & & & & \textbf{DMLab} & \textbf{FFS} \\
 \midrule
\rowcolor{gray!8}Diffusion Forcing~\cite{chen2024diffusionforcing} & $s=k-m$ & $1\times$ & $286 / 266$ & $45.32$ / $52.26$ & $60.30$ & $51.90$ \\
Rolling Diffusion~\cite{ruhe2024rollingdiffusionmodels} & $s=k-m$ & $1\times$ & $500$ / $500$ & $79.24$ / $98.23$ & \textbf{52.43} & \textbf{45.51} \\
\rowcolor{gray!8}\textit{MaskFlow} (MGM-Style) & $s=k-m$ & $1\times$ & \textbf{20} / \textbf{20} & \textbf{3.17 / 3.93} & $53.17$ & $45.92$ \\
\midrule
Diffusion Forcing~\cite{chen2024diffusionforcing} & $s=k-m$ & $2\times$ & $858$ / $798$ & $135.97$ / $156.78$ & $175.01$ & $144.43$ \\
\rowcolor{gray!8}Rolling Diffusion~\cite{ruhe2024rollingdiffusionmodels} & $s=k-m$ & $2\times$ & $896$ / $788$ & $141.99 / 154.81$ & 201.70 & 72.49 \\
\textit{MaskFlow} (MGM-Style) & $s=k-m$ & $2\times$ & \textbf{60} / \textbf{60} & \textbf{9.51} / \textbf{9.30} & $188.02$ &  $59.93$ \\
\rowcolor{gray!8}\textit{MaskFlow} (MGM-Style) & $s=1$ & $2\times$ & $740$ / $340$ & 117.27 / 66.80 & \textbf{50.87} & \textbf{30.43} \\
\midrule
Diffusion Forcing~\cite{chen2024diffusionforcing} & $s=k-m$ & $5\times$ & $2{,}002$ / $1{,}596$ & $317.27$ / $313.56$ & $232.89$ & $272.14$ \\
\rowcolor{gray!8}Rolling Diffusion~\cite{ruhe2024rollingdiffusionmodels} & $s=k-m$ & $5\times$ & $2{,}084$ / $1{,}652$ & $330.27$ / $324.56$ & $338.34$ & $248.13$ \\
\textit{MaskFlow} (MGM-Style) & $s=k-m$ & $5\times$ & \textbf{140} / \textbf{120} & \textbf{22.19 / 23.58} & $334.15$ & $108.74$ \\
\rowcolor{gray!8}\textit{MaskFlow} (MGM-Style) & $s=1$ & $5\times$ & $2{,}900$ / $1{,}300$ & 100.09/379.91 & \textbf{181.11} & \textbf{103.69} \\
\bottomrule
\end{tabular*}
\caption{\textbf{MGM Style sampling is much faster without sacrificing quality.} We report the total number of function evaluations (NFE), sampling time (in seconds), and FVD for various sampling methods and extrapolation factors across both datasets.}
\label{tab:speed_comparison}
\end{table*}



\subsection{Additional Related Work}

\paragraph{Masked Diffusion Models.} 
Limitations of autoregressive models for probabilistic language modeling have recently sparked increasing interest in masked diffusion models. Recent works like \cite{shi2024simplifiedgeneralizedmaskeddiffusion} and \cite{sahoo2024simpleeffectivemaskeddiffusion} have aligned masked generative models with the design space of diffusion models by formulating continuous-time forward and sampling processes. Works like \cite{nie2024scalingmaskeddiffusionmodels} and \cite{gong2024scalingdiffusionlanguagemodels} also demonstrate the significant scaling potential of MDM for language tasks, indicating that this masked modeling paradigm can rival autoregressive approaches for modalities beyond language such as protein co-design \cite{campbell2024generative} and vision.

\subsection{Computation of NFE for Different Sampling Methods}

Our sampling speed evaluations are determined by computing the required number of chunks 
\[
\ell = \left\lceil \frac{L - k}{s} \right\rceil + 1,
\]
to generate a video of total length \(L\), where \(k\) is the chunk size and \(s\) is the stride with which the chunk start is shifted. The overall number of function evaluations (NFEs) is then obtained by multiplying \(\ell\) with the number of sampling steps required to generate one chunk. We apply this methodology for all chunkwise-autoregressive approaches.

\begin{itemize}
    \item \textbf{MGM-Style Sampling:} In this method each chunk is generated in $20$ forward passes, so that the total NFE is
    \[
    \text{NFE}_{\mathrm{MGM}} = \ell \times 20.
    \]

    \item \textbf{FM-Style Sampling:} Here we generate each chunk in $250$ forward passes:
    \[
    \text{NFE}_{\mathrm{FM}} = \ell \times 250.
    \]
    
    \item \textbf{Diffusion Forcing with Pyramid Scheduling:} Here, we apply $250$ sampling timesteps per frame but begin unmasking earlier frames as the denoising process proceeds. For a chunk of \(k\) frames, we generate a scheduling matrix with 
    \[
    H = 250 + (k-1) + 1 = k + 250
    \]
    rows and \(k\) columns. Each entry in the scheduling matrix is computed as
    \[
    \text{scheduling\_matrix}[i,j] = 250 + j - i,\quad \text{for } i=0,\ldots,H-1 \text{ and } j=0,\ldots,k-1,
    \]
    and then clipped to the interval \([0,249]\). Since we iterate through each of the $H$ rows of the denoising matrix in each chunk we effectively compute
    \[
    \text{NFE}_{\text{DiffusionForcing}} = k + 250.
    \] 
    
    \item \textbf{RDM Sampling:} This approach proceeds in three stages:
    \begin{enumerate}
        \item \textit{Initialization (Init-Schedule):} The initial window of \(k\) frames is processed using a fixed schedule that applies $T=250$ forward passes to bring the window to its rolling state.
        
        \item \textit{Sliding Window Handling:} After initialization, the window is shifted by one frame at a time. For each shift, an inner loop is executed that updates the denoising levels until the first non-context frame (i.e., the frame immediately following the \(m\) context frames) is fully denoised (i.e., reaches a value of 1). This inner loop requires $\left\lceil \frac{T}{k-m} \right\rceil$ forward passes per window shift. As the window is shifted \((L - k)\) times, this stage contributes roughly \((L - k) \times \left\lceil \frac{T}{k-m} \right\rceil\) forward passes.
        
        \item \textit{Final Window Processing:} Once the sliding window stage is complete, the final (partial) window is further refined until all frames are fully denoised. This final stage requires additional $250$ forward passes.
    \end{enumerate}
    
    Thus, the total NFE for RDM is given by
    \[
    \text{NFE}_{\mathrm{Rolling}} = 250 \; (\text{init-schedule}) + (L - k) \times \left\lceil \frac{T}{k-m} \right\rceil\ \; (\text{sliding}) + 250 \; (\text{final window}).
    \]
\end{itemize}




\subsection{Training \& Implementation Details}

All FFS models were trained on 4 H100 GPUs with a local batch size of $4$. We run training for a total of $200{,}000$ steps and use a sigmoid scheduler that determines the per-frame masking ratio for a sampled masking level $t^k$. We use an AdamW optimizer with a learning rate of $1e-4$ and $\beta_1 = 0.9$ and $\beta_2 = 0.999$. We additionally incorporate a frame-level loss weighting mechanism based that is also based on \(t^k\). We adopt \emph{fused}-SNR loss weighting from \cite{hang2023efficient,chen2024diffusionforcing} and derive it for discrete flow matching. Let

\[
\text{SNR}(t) \;=\; \frac{\kappa(t)^2}{\,1 - \kappa(t)^2\,},
\]

where \(\kappa(t)\) is the masking schedule. The \emph{fused}-SNR mechanism smoothes SNR values across time steps in a video by computing an exponentially decaying SNR from previous frames (or tokens). We refer the reader to~\cite{chen2024diffusionforcing} for full details.


\begin{algorithm}[!ht]
\caption{\textbf{FM-Style Sampling with Context Frames for a Single Chunk}}
\label{alg:fmsampling}
\begin{algorithmic}[1]
\REQUIRE 
   $p(\mathbf{x}_1 | \mathbf{x}_t, \mathbf{t};\theta)$, 
   $t$, 
   context frames $\mathbf{c} = (c^1,\dots,c^m)$, 
   fully masked frame \([M]\) (i.e., a frame where every token equals the mask token \(M\)),
   $t \in [0,1]$, 
   $\Delta t$

\STATE $\mathbf{x}_t \,\gets\, (\,c^1,\dots,c^m,\,[M],\dots,[M])$
\STATE $t \,\gets\, 0$
\STATE $\mathbf{t} \gets (1,\dots,1,0,\dots0)$

\WHILE{$t \,\le\, 1 - \Delta t$}
    \STATE $u_t(\mathbf{x}_t) 
        \;=\; 
        \frac{t}{1-t}
        \Bigl[
          p_\theta(\mathbf{x}_1 \mid \mathbf{x}_t,\,\mathbf{t}) 
          \;-\; 
          \delta_{\mathbf{x}_t}
        \Bigr]$
    \STATE $p_\theta\!\bigl(\mathbf{x}_1 \mid \mathbf{x}_{t+\Delta t},\,\mathbf{t}+\Delta t\bigr)
        \;=\;
        \mathrm{Cat}\!\Bigl[\,
          \delta_{\mathbf{x}_t}
          \;+\; 
          u_t(\mathbf{x}_t)\,\Delta t
        \Bigr]$
    \STATE \textbf{For each token} $n$ in $\mathbf{x}_t$: 
    \STATE \quad 
    $
       x_{t+\Delta t}^{n} \gets
       \begin{cases}
          x_t^{n}, & \text{if } x_t^{n} \neq M,\\
          p(\cdot | \mathbf{x}_{t+\Delta t},\,\mathbf{t}+\Delta t; \theta), & \text{if } x_t^{n} = M.
       \end{cases}
    $
\STATE $t \gets t + \Delta t$
\STATE $\mathbf{t} \gets \mathbf{t} + \Delta t$

\ENDWHILE
\STATE \textbf{return} $\mathbf{x}_t$
\end{algorithmic}
\end{algorithm}

\begin{algorithm}[ht]
\caption{\textbf{MGM-Style Sampling for a Single Chunk}}
\label{alg:mgm_chunk_unmasking_revised}
\begin{algorithmic}[1]
\REQUIRE 
  Network $p(\mathbf{x}_1 \mid \mathbf{x}_t, \mathbf{t}; \theta)$,  
  context frames $\mathbf{c} = (c^1,\dots,c^m)$,  
  masked frame $[M]$ (i.e., every token equals $M$),    
  total unmasking steps $T$
\STATE \textbf{Initialize:}\\
$\mathbf{x}_t \;\leftarrow\; (\mathbf{c},\, [M],\dots,[M])$\\
$\mathbf{t} \;\leftarrow\; (\underbrace{1,\dots,1}_{m},\, \underbrace{0,\dots,0}_{k-m})$
\STATE Define the set of masked token indices in $\mathbf{x}_t$:\\
$\mathcal{M} \;\triangleq\; \{\, n \mid x_t^n = M \,\}.$
\FOR{$i=1$ \textbf{to} $T$}
    \STATE Compute token-wise logits:\\
    $\boldsymbol{\lambda} \;\leftarrow\; p(\mathbf{x}_1 \mid \mathbf{x}_t, \mathbf{t}; \theta).$
    \STATE \textbf{For each token} $n \in \mathcal{M}$: \\
    sample $\hat{x}_t^n \sim \mathrm{Cat}\Bigl(\mathrm{Softmax}\bigl(\boldsymbol{\lambda}^n\bigr)\Bigr)$ \\
    and compute the confidence score 
    $C_n \;=\; \mathrm{Softmax}\bigl(\boldsymbol{\lambda}^n\bigr)_{\hat{x}_t^n}.$ \\
    \STATE \textbf{Define the confidence threshold:}\\
    Let $\alpha$ denote the desired fraction of masked tokens to update in each iteration (e.g. $\alpha = 1/T$). \\
    
    Then set 
    $\tau_c \;=\; \min\Bigl\{ c \in [0,1] \;\Bigm|\; \Bigl|\{ j \in \mathcal{M} \mid C_j \ge c \}\Bigr| \ge \Bigl\lceil \alpha\,|\mathcal{M}| \Bigr\rceil \Bigr\}.$ \\
    
    (That is, $\tau_c$ is chosen as the minimum confidence such that at least $\lceil \alpha\,|\mathcal{M}| \rceil$ tokens have confidence scores at or above $\tau_c$, thereby selecting the top $\lceil \alpha\,|\mathcal{M}| \rceil$ tokens.)
    \STATE \textbf{For each token} $n \in \mathcal{M}$ with $C_n \ge \tau_c$, update:\\
    $x_t^n \;\leftarrow\; \hat{x}_t^n.$
    \STATE Update the set of masked indices:\\
    $\mathcal{M} \;\leftarrow\; \{\, n \mid x_t^n = M \,\}.$
    \IF{$\mathcal{M} = \varnothing$}
         \STATE \textbf{break}
    \ENDIF
\ENDFOR
\STATE \textbf{return} $\mathbf{x}_t$.
\end{algorithmic}
\end{algorithm}


\subsection{Baseline Details}

The two most comparable works to our method are \citet{chen2024diffusionforcing} and \citet{ruhe2024rollingdiffusionmodels}. Both of these techniques propose novel sampling methods that can be rolled out to long video lengths, and also apply frame-specific noise levels. Both of these approaches are diffusion-based and operate on continuous representations, whereas we operate on discrete tokens and use masking. We re-implement both the pyramid sampling scheme proposed in Diffusion Forcing and the Rolling Diffusion sampling method in our discrete setting. This allows us to compare the baseline sampling methods to MaskFlow on the same model backbones. 
%
To isolate the effect of our chunkwise autoregressive sampling methodology on performance from the effects of tokenization, we reimplement both the pyramid sampling scheme proposed in Diffusion Forcing and the Rolling Diffusion sampling method for our discrete setting. This allows us to compare the baseline sampling methods on the same timestep-dependent model backbone. 
%
Although it is conceivable that Rolling Diffusion sampling may perform better when applied to a model explicitly trained using the progressive noise schedule suggested in \citet{ruhe2024rollingdiffusionmodels}, we believe this comparison is still fair. Our training methodology does not inject any inductive bias by way of the masking level into the model, so there is no obvious advantage that our sampling should have over other methods. 
We provide a comprehensive evaluation of performance and sampling efficiency across both datasets and different sampling modes.


\subsection{Dataset Details}

\paragraph{Deepmind Lab.} The Deepmind Lab (DMLab) navigation dataset contains $64 \times 64$ resolution videos of random walks in a 3D maze environment. We use the total 625 videos with frame length 300 frames, and randomly sample sequences of 36 consecutive frames from each video during training. We upscale video frames to a resolution of $256 \times 256$ before tokenizing them similar to our approach for FaceForensics. We disregard the provided actions, focusing on action-unconditional video generation. We use $m=12$ and $s=24$ for the DMLab full sequence generation experiments unless stated otherwise.

\paragraph{FaceForensics.} FaceForensics (FFS) is a dataset that contains $150\times150$ images of deepfake faces, totaling 704 videos with varying number of frames at 8 frames-per-second. We upsample the resolution to $256 \times 256$, before encoding individual frames using the image-based tokenizer SD-VQGAN \cite{rombach2022high_latentdiffusion_ldm}. While image-based tokenizers have shown to lead to flickering issues, we observe high-reconstruction quality (reconstruction FVD $\approx 8$ on FFS) on our datasets and thus leave work on video tokenization to other works. After tokenization, we train on encoded frame sequences of 16 frames, each consisting of token grids with dimensionality $32 \times 32$. We generally use $m=2$ ground-truth context frames for conditioning, and $s=14$.

\subsection{Further Quantitative Results}

\paragraph{Our chunkwise autoregressive MGM-style sampling is preferable to full sequence training in settings with limited hardware.} To evaluate our method for long video generation against a longer training window baseline, we compare the performance of a frame-level masking model trained on $16$ frames with full sequence generation of a constant-masking level model trained on $32$ frames with similar batch size and on similar hardware. In Table ~\ref{tab:longer_train_window_baseline} we show that iterative rollout of our MGM-style sampling outperforms full sequence generation even when the full sequence model is trained on a longer window.

\begin{table}[ht]
    \centering
    \normalsize
    \resizebox{0.48\textwidth}{!}{%
    \begin{tabular}{l|cccc}
    \toprule
    \makecell{\textbf{Sampling} \\ \textbf{Mode}} 
    & \makecell{\textbf{Training} \\ \textbf{Window}} 
    & \makecell{\textbf{Sampling} \\ \textbf{Window}} 
    & \makecell{\textbf{Total} \\ \textbf{NFE}}
    & \makecell{\textbf{FVD} $\downarrow$} \\
    \midrule
    FM-Style (bs=2) & 32 & 32 & 250 & 253.08 \\
    \midrule
    \textit{MaskFlow} (MGM-Style) (bs=2) & 16 & 32 & 60 & 192.76 \\
    \rowcolor{gray!8}\textit{MaskFlow} (MGM-Style) (bs=4) & 16 & 32 & 60 & \textbf{59.93} \\
    \bottomrule
    \end{tabular}
    }
    \caption{\textbf{Our MGM-style sampling is more efficient and generates better results over baseline for larger training windows}. We train a constant masking ratio model on larger window sizes with similar batch size on similar hardware, and compare full sequence generation to generating the same length using our chunkwise MGM-style sampling.}
    \label{tab:longer_train_window_baseline}
\end{table}

\begin{table}[ht]
    \centering
    \normalsize
    \resizebox{0.48\textwidth}{!}{%
    \begin{tabular}{l|ccrr}
        \toprule
        & \makecell{\textbf{Extrapolation} \\ \textbf{Factor}}
        & \makecell{\textbf{Sampling} \\ \textbf{Stride}}
        & \makecell{\textbf{Total} \\ \textbf{NFE}}
        & \makecell{\textbf{FVD} $\downarrow$} \\
        \midrule
        FaceForensics   & $2\times$  & $s=14$ (\textit{full sequence}) & \textbf{60} & 59.93 \\
        \rowcolor{gray!8}FaceForensics   & $2\times$  & $s=1$ (\textit{autoregressive})  & 340 & \textbf{30.43} \\
        \midrule
        FaceForensics   & $5\times$  & $s=14$ (\textit{full sequence}) & \textbf{120} & 108.74 \\
        \rowcolor{gray!8}FaceForensics & $5\times$  & $s=1$ (\textit{autoregressive})  & 1,300 & \textbf{103.69} \\
        \midrule
        FaceForensics   & $10\times$ & $s=14$ (\textit{full sequence}) & \textbf{240} & 214.39 \\
       \rowcolor{gray!8} FaceForensics   & $10\times$ & $s=1$ (\textit{autoregressive})  & 2,900 & \textbf{165.02} \\
        \midrule
        \midrule
        DMLab & $2\times$  & $s=24$ (\textit{full sequence}) & \textbf{60} & 188.22 \\
        \rowcolor{gray!8}DMLab & $2\times$  & $s=1$ (\textit{autoregressive})  & 740  & \textbf{50.87} \\
        \midrule
        DMLab & $5\times$  & $s=24$ (\textit{full sequence}) & \textbf{140}  & 334.15 \\
       \rowcolor{gray!8} DMLab & $5\times$  & $s=1$ (\textit{autoregressive})  & 2,900 & \textbf{181.11} \\
        \bottomrule
    \end{tabular}
    }
    \caption{\textbf{Autoregressive sampling outperforms full sequence sampling on timestep-dependent models at the cost of higher NFE.}}
    \label{tab:autoregression_dependent}
\end{table}


\begin{figure*}[ht!]
    \centering
    \includegraphics[width=0.7\textwidth]{figpaper/realestate.pdf}\hfill
    \caption{\textbf{Further visualizations on the Realestate10K \cite{zhou2018stereo} dataset.} Models trained on chunk size $k = 16$ with $4$ H100 GPUs. Due to computational limitations, we cannot  provide further analyses on this larger, more compute intensive dataset.}
    \label{fig:faces_comparison}
\end{figure*}




\begin{table}[ht]
    \centering
    \normalsize
    \resizebox{0.58\textwidth}{!}{%
    \begin{tabular}{l|ccrr}
        \toprule
        & \makecell{\textbf{Extrapolation} \\ \textbf{Factor}}
        & \makecell{\textbf{Sampling} \\ \textbf{Stride}}
        & \makecell{\textbf{Total} \\ \textbf{NFE}}
        & \makecell{\textbf{FVD} $\downarrow$} \\
        \midrule
        FaceForensics   & $2\times$  & $s=14$ (\textit{full sequence}) & \textbf{60} & 109.96 \\
        FaceForensics   & $2\times$  & $s=1$ (\textit{autoregressive}) & 340 & \textbf{43.91} \\
        \midrule
        FaceForensics   & $5\times$  & $s=14$ (\textit{full sequence}) & \textbf{120} & \textbf{137.66} \\
        FaceForensics & $5\times$  & $s=1$ (\textit{autoregressive})  & 1,300 & 193.90 \\
        \midrule
        FaceForensics   & $10\times$ & $s=14$ (\textit{full sequence}) & \textbf{240} & \textbf{174.92} \\
        FaceForensics   & $10\times$ & $s=1$ (\textit{autoregressive})  & 2,900 & 293.16 \\
        \midrule
        \midrule
        DMLab & $2\times$  & $s=24$ (\textit{full sequence}) & \textbf{60} & 219.33 \\
        DMLab & $2\times$  & $s=1$ (\textit{autoregressive})  & 740  & \textbf{42.53} \\
        \midrule
        DMLab & $5\times$  & $s=24$ (\textit{full sequence}) & \textbf{140}  & 402.73 \\
        DMLab & $5\times$  & $s=1$ (\textit{autoregressive})  & 2,900 & \textbf{80.56} \\
        \bottomrule
    \end{tabular}
    }
    \caption{\textbf{Autoregressive sampling outperforms full sequence sampling on timestep-independent models at the cost of higher NFE.} Performance improvement on DMLab is substantial.}
    \label{tab:autoregression_independent}
\end{table}

\newpage

\subsection{Further Qualitative Results}

\begin{figure*}[ht!]
    \centering
    \includegraphics[width=0.48\textwidth]{figpaper/faces1.pdf}\hfill
    \includegraphics[width=0.48\textwidth]{figpaper/faces2.pdf}
    \caption{\textbf{Visualizations of FaceForensics generation results with different context frames.}}
    \label{fig:faces_comparison}
\end{figure*}
















\end{document}
