\section{Related works}
\label{sec:RelatedWorks}
In the field of mobile network optimization, there are various methods to enhance network performance and reduce operational costs. As shown in \autoref{fig:roadMap}, network optimization can be achieved through the collection of field data via drive test or through network-side data collection using minimal drive test. When it comes to optimizing drive tests, there are three main approaches: route optimization, network optimization based on data, and data prediction.

The concept of minimal drive test was developed as an alternative to traditional drive test. It leverages data collected by network users during their everyday usage, significantly reducing the need for physical drive tests. By gathering extensive data directly from users, this approach allows for effective network optimization. Studies such as \cite{SKOCAJ2022403,9017353} have explored the use of minimal drive testing to improve network performance and optimize network parameters.

In the realm of drive test optimization, route optimization is a method used before conducting the drive tests. This technique focuses on selecting optimal routes that provide the most valuable data with minimal cost and time. Research papers like \cite{Christofides1986,ARAOZ2009886} offer solutions to the challenges of route planning and optimization.

On the other hand, data-based network optimization involves using data collected from drive tests to improve network configurations. In this approach, once the drive test is completed, the collected data is analyzed, and necessary adjustments are made to enhance network performance. Studies such as \cite{silalahi2021improvement,Peerajing2016Multisector} demonstrate how these changes can lead to improved network service quality.

The primary focus of our research is on data prediction, an approach aimed at estimating network parameters in areas where drive tests have not been conducted. By leveraging existing drive test data, this method predicts parameters such as \gls{RSRP} and \gls{RSRQ}, enabling operators to optimize networks without the need for drive tests in every area. Numerous studies have been conducted in this domain, and we will review some of the most significant ones.

One notable study in this field is presented in \cite{Ojo2020Radial}, which uses drive test data from six base transceiver stations to improve path loss prediction in 4G networks using machine learning. This study illustrates that, unlike traditional models that are rigid and inflexible, machine learning models, such as \glspl{RBF}, offer better accuracy and adaptability, overcoming the limitations of existing models.

In the field of path loss prediction, newer methods have emerged that go beyond the traditional machine learning approaches. \cite{5466252} focus on using \glspl{ANN} to predict path loss for base stations in rural environments. Their findings suggest that a relatively simple ANN model, when trained with drive test data, can outperform traditional models in terms of both prediction accuracy and computational efficiency.

These studies consistently aim to improve the accuracy and efficiency of path loss prediction using drive test data and machine learning. The study in \cite{thrane2018drive} expands on this by utilizing machine learning to predict radio frequency characteristics like RSRP, RSRQ, and signal-to-noise ratio. This research employs a deep neural network trained on data obtained from drive tests, including device locations, base station locations, device-to-station distances, and satellite imagery of the environment, to predict key signal quality metrics for 4G mobile networks. Unlike traditional models that rely heavily on lab-generated data, this approach uses real-world drive test data combined with machine learning to enhance prediction accuracy.

Advanced methods for optimizing drive tests have also been introduced. For instance, \cite{10212575} propose an improved method for predicting \gls{RSRP} using drive test data. This study leverages deep learning techniques and drive test data to significantly increase prediction accuracy. By incorporating supplementary features such as 3D antenna gain and digital elevation models, the research demonstrates superior results compared to traditional methods. The study highlights that combining drive test data with additional environmental information can enhance the accuracy of predictions and the effectiveness of drive test optimization.

Finally, the work presented in \cite{9700950} introduces an intelligent machine learning model for predicting \gls{RSRP} that utilizes drive test data along with advanced machine learning techniques. By employing gradient-boosted trees, this research significantly improves prediction accuracy and robustness across different environments. The paper discusses the challenges associated with feature selection and tuning in these models and offers solutions to enhance model performance. The study shows that by using smarter, hybrid models, more accurate and reliable predictions can be achieved, which are crucial for optimizing mobile networks.