% \section{Solving minimum densest subgraph exactly}

\appendix
\section{Solving minimum densest subgraph exactly}

As a baseline algorithm we consider the minimum densest subgraph problem~\cite{jethava2015finding} where the goal is to find a common subgraph which maximizes the minimum density.

\begin{problem}[Minimum densest subgraph~(\problemdcs)]
\label{pr:dcs}
Given a graph sequence $\calG = \set{G_1,\dots,G_r}$, with  $G_i = (V, E_i)$, find a common subset of vertices
$S$, 
such that $\min_i \dens{S, G_i}$, the minimum density induced by  $S$  over  $\calG$  is maximized.
\end{problem}
 
Although heuristics have been proposed to  \problemdcs problem~\cite{jethava2015finding, semertzidis2019finding}, we propose an integer programming-based algorithm to find an exact or near-optimal solution which we refer to this algorithm as \algipdcs. 

We use a similar approach as \algipcm algorithm where we utilize fractional programming and binary search. We introduce an auxiliary problem

\begin{problem}[$\problemdcs(\gamma)$]
Given a graph sequence $\calG = \set{G_1,\dots,G_r}$, with  $G_i = (V, E_i)$, find a common subset of vertices
$S$, 
such that $\min_i m(S, G_i) - \gamma \abs{S}$  is maximized.
\end{problem}


We solve $\problemdcs(\gamma)$ with the following integer program

\begin{align*}	\textsc{maximize\,\,\,\,}\hspace{.1in}&&\hspace{.05in}  z &- \gamma \sum_{i = 1}^{n} y_i 
	\\  
	\textsc{subject to}\hspace{.2in} && 
  x_{ij} &\leq y_i &  ij \in E \\
	&& x_{ij} &\leq y_j & ij \in E \\ 
 	&& \sum_{ i, j \in E_k} x_{ij} &\geq z  &  k = 1,\dots, r \\
	&& x_{ij}, y_j &\in \{0,1\}\\
        &&  z & \geq 0 \quad.
\end{align*}

To see why this program solves $\problemdcs(\gamma)$,
note that $x_{ij} = \min(y_i, y_j)$ and $z = \min_k m(S, G_k)$,
where $S = \set{i \mid y_i = 1}$.

We search for maximal $\gamma$ that induces non-empty solution
with a binary search. Here, we set the initial interval $(L, U)$ to $L = 0$ and $U =\frac{n - 1}{2}$, and keep halving the interval until $\abs{ U - L} \leq  \epsilon L$, where $\epsilon > 0$  is an input parameter. Finally, we return the solution of $\problemdcs(L)$ as the final solution to \problemdcs. 

The following result states the approximation guarantees of \algipdcs.

\begin{proposition}
Assume a graph sequence $\calG = \set{G_1,\dots,G_r}$ and $\epsilon > 0$. 
Let $\gamma$ be the score of the solution returned by \algipdcs and let $\gamma^*$ be the optimal score of \problemdcs. Then $\gamma \geq \gamma^*/(1 + \epsilon)$. If $\epsilon \leq \frac{1}{n^3}$, then $\gamma = \gamma^*$.
\label{prop:opt-dcs-approx}
\end{proposition}

The proof of the proposition is essentially the same as the proofs for \algipcm, and is therefore omitted.

\iffalse
Let $S$ be the solution returned by $\problemdcs(\gamma)$. Here we use a similar set of indicator variables $y_i$s and $x_{ij}$s as defined in the IP linked with $\problemcdcsm(\gamma)$.
In addition, we introduce an extra integer variable denoted by $z$ to keep track of the minimum number of edges induced by the subgraph among snapshots. 

The condition $x_{ij} = \min(i, j)$ is valid in \problemdcs problem 
since we can always increase $x_{ij}$s to reduce the $\min(i, j) - x_{ij}$ to zero which only raises the objective value.
Therefore we do not add additional constraints here in contrast to  \problemcdcsm. 

Our IP is closely related to the LP proposed  by~\citet{jethava2015finding} which is used to approximate the \problemdcs problem. To formulate it as an integer program, 
we apply some discretizing techniques together with the fractional programming approach proposed by \citet{goldberg1984finding} to solve the densest subgraph problem.
On the theoretical
side, we provide an algorithm with an approximation scheme, while
the algorithms~(LP-based and greedy) proposed by \citet{jethava2015finding} and the greedy-like algorithms proposed by \citet{semertzidis2019finding}  do not provide any theoretical approximation guarantees for all inputs.
% a worst-case approximation guarantee of $\bigO{1/\sqrt(n)}$.
\fi