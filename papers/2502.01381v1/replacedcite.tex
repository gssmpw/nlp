\section{Related work}
\label{sec:related}

Given an undirected graph, the problem of finding the densest subgraph can be solved in polynomial time. To solve the problem Goldberg____ proposed
an exact algorithm based on fractional programming, binary search, and minimum-cut computations. Moreover, Charikar____ proposed a linear program solving the problem, and a linear time, greedy $1/2$-approximation algorithm.

Instead of a single graph, the densest subgraph problem has been extended to the case of multiple graphs.
Jethava and Beerenwinkel____ considered finding a common that maximizes the minimum density, an \np-hard problem as shown by Charikar et al.____.
We refer to this problem as \problemdcs problem. 
While LP-based and greedy algorithms proposed by Jethava and Beerenwinkel____ and the greedy-like algorithms proposed by Semertzidis et al.____  do not provide any theoretical approximation guarantees for all inputs, we propose an exact algorithm for \problemdcs problem as an additional result.
% See the Supplementary material for more details.

 Another problem variant for multiple graphs has been considered by Semertzidis et al.____ where their objective was to maximize the sum of the densities. This problem is solvable in polynomial time by first flattening the graph sequence into one weighted graph and then considering it as the standard weighted densest subgraph instance. Arachchi and Tatti____ and Rozenshtein et al.____ considered extensions of the problem where the dense subgraph is allowed to vary across the snapshots.
 
 %The authors do not consider a fair allocation of densities among the graph snapshots. The difference between our work to theirs is that we add an additional fairness constraint so that no snapshot is over-represented in the solution. However, the problem turns out to be \np-hard even for two graphs when the fairness constraint is enforced.

%The notion of fairness is relatively new in data mining which has received attention for classical problems like clustering, community detection, and anomaly detection____.
%The fairness in the context of the densest subgraph problems is not well-studied yet.
%In the first attempt to find a fair densest subgraph, 
Anagnostopoulos et al.____ considered a single binary node-colored graph and constrained the subgraph to have a fair distribution of colors.
Even with two colors, the problem is shown to be \np-hard____.
Next, Miyauchi et al.____ considered finding a fair densest subgraph of a node-colored graph with multiple colors.
Note that our problem can be viewed as finding fair densest subgraph in an edge-colored graph. Here, we color each edge with a color unique to a snapshot, and collapse the sequence into one (multi-edged) graph. Since the constraints and the input information in these problems are different the existing methods cannot be used to solve our problem.

Alternative measures for the density have been also studied. One
 option is to use the proportion of edges, that is, $m(S) / {\abs {S}  \choose 2}$. The issue with this measure is that a single edge yields the highest density of 1. Moreover, finding the largest graph with the edge proportion of 1 is equal to finding the largest clique, an inappproximable problem____.
 Tsourakakis et el____ proposed a score $m(S) - \alpha {\abs {S} \choose 2}$, leading to an \np-hard optimization problem. In addition, Tsourakakis____ proposed a ratio of triangles 
 and the nodes as the density measure. Like the density based on edges, optimizing
 this measure can be done in polynomial time but requires computing the existing triangles in a graph. Adopting these measures with fairness constraints provides a future line of work.

% no color is over-represented in any cluster.






% \textbf{The densest subgraph:}

% Given an undirected graph, finding the subgraph which maximizes density has
% been first studied by  Goldberg____ where an exact,
% polynomial time algorithm which solves a sequence of min-cut instances is
% presented. ____ provided a linear time, greedy algorithm
% proved to be an  $1/2$-approximation algorithm by
% Charikar____. The idea of the algorithm is that
% at each iteration, a vertex with minimum degree is removed, and then the densest
% subgraph among all the produced subgraphs is chosen.
% \iffalse
% If we do this naively it would take $\bigO{n^2}$ time, however we can maintain
% a  priority queue to search the minimum degree vertex which can be done in
% $\bigO{\log n}$ time, for an $n$ element list. Then,  the degrees of only  the
% neighbours of the removed vertex should be updated. Therefore it requires only
% $\bigO{m + n \log n}$ time.  We take similar approach in speeding up our
% algorithms.
% \fi

% Several variants of the densest subgraph problem constrained
% on the size of the subgraph~$\abs{S}$ have been studied: finding the densest
% $k$-subgraph~($ \abs{S} = k$)____,  \emph{at most} $k$-subgraph~($\abs{S} \leq
% k$)____, and \emph{at least}
% $k$-subgraph~($ \abs{S} \geq
% k$)____. Unlike the densest subgraph
% problem, when the size constraint is applied, the densest $k$-subgraph problem
% becomes \np-hard____. Furthermore,  there is no polynomial
% time approximation scheme~(PTAS)____. Approximating the
% problem of finding at most $k$-subgraph is  shown as hard as the densest
% $k$-subgraph problem  by ____. To find exactly $k$-size
% densest subgraph, ____ gave an $\bigO{n^{1/4 +
% \epsilon}}$-approximation algorithm for every $\epsilon > 0$ that runs in
% $n^{\bigO{1/\epsilon}}$ time. 
% ____ provided a linear time $1/3$-approximation algorithm for at least $k$ densest subgraph problem.  %Nevertheless the hardness of finding at least $k$-subgraph is yet unknown. 

% \textbf{The densest common subgraph over multiple graphs:}
% ____ extended the densest subgraph problem~(DCS) for the case of multiple graph snapshots.
% As a measure the authors' goal was to maximize the minimum density.
% Moreover, ____ introduced several variants of this
% problem by varying the aggregate function of the optimization problem, 
% one variant, BFF-AA, is same as the DCS problem discussed in Section~\ref{sec:prel}.
% DCS can be solved exactly through a reduction to the
% densest subgraph problem, and is consequently polynomial.  The hardness of DCS variants has been
% addressed____. For a survey on the densest subgraph problem and its variants, we refer the reader to a recent survey
% by____.
%  % ____ presents a  Lagrangian relaxation for Linear program proposed by ____ for DCS problem.
% % TKDD ____

% \textbf{Overlapping densest subgraphs of a single graph:}
% Finding multiple dense subgraphs in a single graph which allows graphs to be overlapped is studied by adding a hard
% constraint to control the overlap of
% subgraphs____.
% Later, ____ formulated the same problem  adding a penalty in
% the objective function for the overlap. 
% The difference of our problem to the works of
% ____ and ____ is that our goal is to
% find a collection of dense subgraphs 
% over multiple graph snapshots~(one dense subgraph for each graph snapshot) while they discover a set of dense subgraphs within a single graph. 
% Due to this difference, we want to reward similar subgraphs while the authors want to penalize similar subgraphs.

% % Hence, we propose a novel problem that of finding dense subgraphs over a sequence of graph snapshots where each pair of dense subgraph  is constrained by Jaccard coefficient which is given as an input parameter to the problem. We call this as unweighted problem. Our other variant comes with a Lagrangian relaxation to the unweighted problem.



% \textbf{Other density measures:}
% We use the ratio of edges over the nodes as our measures as it allows us to
% compute it efficiently. Alternative measures have been also considered.  One
% option is to use the proportion of edges instead, that is, $\abs{E} / {\abs {V}
% \choose 2}$. The issue with this measure is that a single edge yields the
% highest density of 1. Moreover, finding the largest graph with the edge
% proportion of 1 is equal to finding a clique, a classic problem that does not
% allow any good approximation____.
% As an alternative approach, ____
% proposed finding subgraphs with large score $\abs{E} - \alpha {\abs {V} \choose 2}$.
% Optimizing this measure is an \np-hard problem but an algorithm similar to the one given by____
% leads to an additive approximation guarantee. In similar vein, ____ considered
% subgraphs maximizing  $\abs{E} - \alpha \abs {V}$ and showed that they form nested structure
% similar to $k$-core decomposition.
% An alternative measure called
% triangle-density has been proposed by____ as a ratio of triangles 
% and the nodes, possibly producing smaller graphs. Like the density, optimizing
% this measure can be done in polynomial time.
% %However, computing this measure involves
% %discovering all triangles which may be computationally prohibitive and sampling approaches
% %should be used.
% We leave adopting these measures as a future work.