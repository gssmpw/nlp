\section{Concluding remarks}\label{sec:conclusions}

In this paper, we introduced two variants of dense subgraph discovery problem for graphs with multiple snapshots that take fairness into account. 

More specifically,
given an input parameter $\alpha$, the goal of our first variant is to find a dense subgraph maximizing the sum of densities across snapshots such that the difference between the maximum and minimum induced density is at most $\alpha$. 
We considered also the dual problem where given an input parameter $\sigma$, we find a subgraph that minimizes the gap between the maximum and minimum density induced by the subgraph while inducing at least $\sigma$ amount of total density over the graph sequence.

We proved that both problems are \np-hard and proposed two exponential time, exact algorithms based on integer programming.
We also proposed two polynomial time heuristics.
We experimentally showed that our
algorithms could find the ground truth in synthetic dataset and perform reasonably well in real-world datasets.
Finally, we performed a study to show the usefulness of our problems.

The paper introduces several interesting directions for future work.  In this
paper, we considered the difference between the maximum and minimum density as the measure of fairness.
However, we can use an alternative constraint that forces the density induced in each snapshot to be at least a portion of the average density.
Another possible
direction is adopting a different kind of density definition for our problem setting.
