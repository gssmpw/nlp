\section{Preliminary notation and problem definition}\label{sec:prel}

We begin by providing preliminary notation and formally defining our problem.

Our input is a sequence of graphs $\calG = (G_1, \ldots, G_r)$, where each snapshot $G_i = (V, E_i)$
is defined over the same set of nodes. We often denote the number of nodes and edges by $n = \abs{V}$. % and $m_i = \abs{E_i}$, or $m = \abs{E}$, if $i$ is omitted.

Given a graph $G = (V, E)$, and a set of nodes $S \subseteq V$, we define $E(S) \subseteq E$ to be the subset of edges
having both endpoints in $S$. We also write $m(S) = \abs{E(S)}$. Given a  graph sequence $\calG = (G_1, \ldots, G_r)$ with $G_i = (V, E_i)$, we write $E(S, G_i) = E_i(S)$ and $m(S, G_i) = \abs{E(S, G_i)}$.

As mentioned before, our goal is to find dense subgraphs in a temporal network, and for that
we need to quantify the density of a subgraph. More formally,
assume an unweighted graph $G = (V, E)$, and let $S \subseteq V$.
We define the \emph{density} $\dens{S, G}$ of a graph $G_i$ induced by node set $S$,
and extend this definition for a sequence of graphs $\calG = (G_1, \ldots, G_k)$
as
\[
	\dens{S, G_i} = \frac{\abs{E(S, G_i)}}{\abs{S}}
	\quad\quad \text{and}\quad\quad
	\dens{S, \calG} = \sum_{i = 1}^r \dens{S, G_i}
	\quad.
\]


We first state the problem of finding a common subgraph in a graph sequence which maximizes the sum of the densities proposed by Semertzidis et al.~\cite{semertzidis2019finding}.

\begin{problem}[Total densest subgraph problem~(\problemdts)]
\label{pr:dts}
Given a graph sequence $\calG = (G_1,\dots,G_r)$, with  $G_i = (V, E_i)$, find a common subset of vertices
$S$, 
such that $\dens{S, \calG}$ is maximized.
\end{problem}

%We refer to this problem as \problemdts.
This problem can be solved by first flattening the graph sequence into one weighted graph, where an edge weight is the number of snapshots in which an edge occurs.
The problem is then a standard (weighted) densest subgraph problem that can be solved
using the exact method given by Goldberg~\cite{goldberg1984finding} in $\bigO{n(n + m) (\log n + \log r)}$ time.



Next, we introduce our main problem where an additional fairness constraint of the induced densities is considered. 
To this end, 
we denote the difference between the maximum and minimum induced density as
\[
    \diff{S, \calG} = \max_i \dens{S, G_i} - \min_i \dens{S, G_i} \quad.
\]
Given a sequence of graph snapshots and an input parameter $\alpha$, our goal
is to find a subset of vertices,
such that the sum of the densities of subgraphs is maximized while maintaining the difference $\diff{S, \calG}$ at most $\alpha$.


\begin{problem}[Fair densest subgraph problem~(\problemcdcsm)]
\label{pr:fair-dense-min}
Given a graph sequence $\calG = (G_1,\dots,G_r)$, with  $G_i = (V, E_i)$ and  real number  $\alpha$, find a  subset of vertices
$S$, 
such that $\dens{S, \calG}$ is maximized
and $\diff{S, \calG} \leq  \alpha$.
\end{problem}

Note that for $\alpha = 0$, the \problemcdcsm problem is equivalent to finding a subgraph which induces exactly the same density over each snapshot while maximizing the total density.
On the other hand, setting $\alpha = \infty$ reduces \problemcdcsm to \problemdts.

%The main difference with \problemdts is that we expect fair density distribution among snapshots.
%In other words,  we extend the idea of temporal densest subgraph in which $\alpha = \infty$, for more generalized case where $ 0 \leq \alpha < \infty$ . 




% Next we present another variant of the fair densest subgraph problem called \problemcdcsm.
% Assume that we are given an input parameter $\alpha$ and  $S \subseteq V$ which is a solution set to the  \problemcdcsm problem.
% We define $d_{\mathit{max}}$ to be the maximum density induced by $S$ over any of the snapshot.
% In \problemcdcsm, we enforce that the density induced on each snapshot should be at least $\alpha \times d_{max}$.

% Instead of $\alpha$ of dimension $1$, we can take a vector of length $r$ as the input parameter which generalizes \problemcdcs and \problemcdcsm problems further.

Next, we present a minimization variant of \problemcdcsm problem where given an input parameter $\sigma$, the goal is to find a subset of vertices $S$ that minimizes the difference $\diff{S, \calG}$  while inducing a total density of at least $\sigma$. %We refer to this problem as \problemcdcsdiff. 
% Note that we do not consider the case of $\alpha = 0$ which corresponds to the zero-solution.


\begin{problem}[The smallest difference densest subgraph problem~(\problemcdcsdiff)]
\label{pr:diff}
Given a graph sequence $\calG = (G_1,\dots,G_r)$, with  $G_i = (V, E_i)$ and real number $\sigma$, find a common subset of vertices $S$, 
such that the density induced by $S$  over  $\calG$  is at least $\sigma$ and $\diff{S, \calG}$ is minimized.
\end{problem}

Finally, we state the minimum densest subgraph problem~\cite{jethava2015finding} where the goal is to find a common subgraph which maximizes the minimum density.

\begin{problem}[Minimum densest subgraph~(\problemdcs)]
\label{pr:dcs}
Given a graph sequence $\calG = (G_1,\dots,G_r)$, with  $G_i = (V, E_i)$, find a common subset of vertices
$S$, 
such that $\min_i \dens{S, G_i}$, the minimum density induced by  $S$  over  $\calG$  is maximized.
\end{problem}
