\section{Experimental evaluation}\label{sec:exp}


The goal of this section is to experimentally evaluate our algorithms. To this end, we first generate a synthetic dataset with planted, fair dense component in each snapshot and test how well our algorithms discover the ground truth.
Next, we study the performance of the algorithm on real-world temporal datasets.
Finally, we present interpretative results from a case study.

We implemented the algorithms in Python\footnote{The source code is available at \url{https://version.helsinki.fi/dacs/}.
\label{foot:code}} and performed the experiments using a 2.4GHz Intel Core i5 processor and 16GB RAM. In our experimental evaluation, we used the Gurobi optimization solver in Python to solve the integer programs associated with \algipcm and \algipdiff algorithms by setting $\epsilon = 0.01$. As densities vary across the datasets,
for \problemcdcsdiff problem, we use $\sigma = \sbsnorm \times \dentds$ where $\dentds$ is the solution of \problemdts.
Recall that the greedy algorithm \alggrdfms requires finding a feasible set in the first phase, as discussed in Section~\ref{subsec:greedy-algo-fds}. The search is done with \alggrd and varying $\sbsnorm \in \set{\frac{i}{k} \mid i = 0, \ldots, k}$.
We initially set $k = 20$. If we did not find a feasible set, we tested $k = 100$. This strategy was sufficient for our experiments.

% \subsection{Baselines}
We compare our results with the exact solution of the densest common subgraph problem which maximizes $\min_i \dens{S, G_i}$~\cite{jethava2015finding}. We refer to this problem as \problemdcs.
We also compare our results to the total density induced by \problemdts~\cite{semertzidis2019finding}, and the sum of densities of individual dense subgraphs~\cite{goldberg1984finding}.

\subsection{Synthetic dataset}
The synthetic dataset was generated as follows.
% and the related parameters and statistics are given in Table~\ref{tab:stats1}. 
The dataset consists of $4$ graphs given as $\{G_1,\ldots,G_4\}$ with $200$ vertices. We split the vertices into two groups $U$ and $W$.
For $G_1$ and $G_2$ we randomly placed $1500 p_i$ edges connecting $U$, where $p_i$ was sampled uniformly from the interval $[0.4, 1]$. We also randomly connected vertices $W$ with $200$ edges, as well as vertices $U$ and $W$ also with $200$ edges. For $G_3$ and $G_4$ we randomly connected vertices with $6000$ edges.
The resulting graph sequence had $14\,866$ edges. Moreover, the total density and the density difference for the dense fair component $U$ were $\dens{U, \calG} = 51.12$ and $\diff{U, \calG} = 3.9$.

\begin{table}[t!]
\setlength{\tabcolsep}{0pt}
\caption{
Computational statistics from the experiments with the synthetic datasets. Let $d_{\mathit{tds}}$ be the solution of \problemdts. Here,  \emph{constr.} relates to the input parameter $\alpha$ in \problemcdcsm or the normalized parameter $\sbsnorm$ in \problemcdcsdiff, $d_{\mathit{dis}}$ is the
discovered sum of densities, $d_{\mathit{min}}$, $d_{\mathit{max}}$, and $d_{a}$ are the minimum, maximum, and average density induced by the discovered subgraph over the graph sequence, respectively, $i$ gives the number of iterations of the algorithms, \emph{Jacc.} is the Jaccard index between the solution set and ground truth set, $\abs{S}$ provides the size of the solution, and \emph{time} gives the computational time in seconds.
 }

\label{tab:stats-syn}
\pgfplotstabletypeset[
    begin table={\begin{tabular*}{\textwidth}},
    end table={\end{tabular*}},
    col sep=comma,
	columns = {algorithm,alpha,sum_den,min_den, max_den,avg_den,diff,no_iterations,size, j_t, time},
    columns/algorithm/.style={string type, column type={@{\extracolsep{\fill}}l}, column name=\emph{Algorithm}},
    columns/n/.style={fixed, set thousands separator={\,}, column type=r, column name=$\abs{V}$},
    columns/m/.style={fixed, set thousands separator={\,}, column type=r, column name=$Exp[\abs{E}]$},
    columns/time_0/.style={fixed, set thousands separator={\,}, column type=r, precision = 0, column name=$time$},
    columns/time_1/.style={fixed, set thousands separator={\,}, column type=r, column name=$t_2$},
    columns/time_2/.style={fixed, set thousands separator={\,}, column type=r, column name=$t_3$},
    columns/j_t/.style={fixed, set thousands separator={\,}, column type=r, column name=Jacc.},
    columns/alpha/.style={fixed, set thousands separator={\,}, column type=r, column name=constr.},
    columns/sum_den/.style={fixed,set thousands separator={\,}, column type=r, column name=$d_{dis}$},
    columns/diff/.style={fixed,set thousands separator={\,}, column type=r, column name=$\Delta$},
    columns/true_den/.style={fixed, set thousands separator={\,}, column type=r, column name=$d_{true}$}, 
    columns/dcs_den/.style={fixed, set thousands separator={\,}, column type=r, column name=$d_{\mathit{mds}}$}, 
    columns/dense_den/.style={fixed, set thousands separator={\,}, column type=r, column name=$d_{den}$}, 
    columns/min_den/.style={fixed, set thousands separator={\,}, column type=r, column name=$d_{\mathit{min}}$}, 
    columns/max_den/.style={fixed, set thousands separator={\,}, column type=r, column name=$d_{\mathit{max}}$}, 
    columns/axmax/.style={fixed, set thousands separator={\,}, column type=r, column name=$d_{\mathit{con}}$}, 
    columns/avg_den/.style={fixed, set thousands separator={\,}, column type=r, column name=$d_{a}$}, 
    columns/j1/.style={fixed, set thousands separator={\,}, column type=r, column name=$j_t$}, 
    columns/j2/.style={fixed, set thousands separator={\,}, column type=r, column name=$j_d$}, 
    columns/min_jac_dis_0/.style={fixed, set thousands separator={\,}, column type=r, column name=$J_{min}$}, 
    columns/time/.style={fixed, set thousands separator={\,}, column type=r, precision = 0, column name=time},
    columns/min_jac_dis_1/.style={fixed, set thousands separator={\,}, column type=r, column name=$j_{2}$}, 
    columns/min_jac_dis_2/.style={fixed, set thousands separator={\,}, column type=r, column name=$j_{3}$}, 
    columns/avg_jac_0/.style={fixed, set thousands separator={\,}, column type=r, column name=$\rho$}, 
    columns/avg_jac_1/.style={fixed, set thousands separator={\,}, column type=r, column name=$a_{2}$}, 
    columns/avg_jac_2/.style={fixed, set thousands separator={\,}, column type=r, column name=$a_{3}$}, 
    columns/no_iterations/.style={fixed, set thousands separator={\,}, column type=r, column name=$i$},
    columns/size/.style={fixed, set thousands separator={\,}, column type=r, column name=$\abs{S}$},
    columns/dis_d_1/.style={fixed,set thousands separator={\,}, column type=r, column name=$d_2$},
    columns/dis_d_2/.style={fixed,set thousands separator={\,}, column type=r, column name=$d_3$},
    columns/dis_obj_1/.style={fixed,set thousands separator={\,}, column type=r, column name=$o_2$},
    columns/dis_obj_2/.style={fixed,set thousands separator={\,}, column type=r, column name=$o_3$},
    every head row/.style={
		before row={\toprule},
			after row=\midrule},
    every last row/.style={after row=\bottomrule},
]
{syn-exp.csv}
\end{table}




\begin{table}[t!]
\setlength{\tabcolsep}{0pt}
\caption{
Characteristics of real-world datasets. Here, $n$ and  $m$ are the number of
vertices and edges, $r$  is the number of
snapshots, \dentds is the solution of \problemdts, $\Delta_{\mathit{tds}}$ is the difference between the maximum and minimum density of the \problemdts solution, $d_{\mathit{ind}}$
gives the sum of densities of individual densest subgraph from each graph snapshot, and  \denmds and $\Delta_{\mathit{mds}}$ give the total density and the difference between the maximum and minimum density of the \problemdcs solution, respectively. 
}

\label{tab:stats4}
\pgfplotstabletypeset[
    begin table={\begin{tabular*}{\textwidth}},
    end table={\end{tabular*}},
    col sep=comma,
	columns = {dataset,n,m_t,tau,dense_den,dcs_den, dcs_sum,delta_tds,delta},
    columns/dataset/.style={string type, column type={@{\extracolsep{\fill}}l}, column name=\emph{Data}},
    columns/n/.style={fixed, set thousands separator={\,}, column type=r, column name=$n$},
    columns/m_t/.style={fixed, set thousands separator={\,}, column type=r, column name=$m$},
    columns/tau/.style={fixed, set thousands separator={\,}, column type=r, column name=$r$},
    columns/time_0/.style={fixed, set thousands separator={\,}, column type=r, column name=$t_1$},
    columns/time_1/.style={fixed, set thousands separator={\,}, column type=r, column name=$t_2$},
    columns/time_2/.style={fixed, set thousands separator={\,}, column type=r, column name=$t_3$},
    columns/delta/.style={fixed, set thousands separator={\,}, column type=r, column name=$\Delta_{\mathit{mds}}$},
    columns/delta_tds/.style={fixed, set thousands separator={\,}, column type=r, column name=$\Delta_{\mathit{tds}}$},
    columns/true_den/.style={fixed, set thousands separator={\,}, column type=r, column name=$d_{true}$}, 
    columns/dcs_den/.style={fixed, set thousands separator={\,}, column type=r, column name=\dentds}, 
    columns/dcs_sum/.style={fixed, set thousands separator={\,}, column type=r, column name=\denmds}, 
    columns/dense_den/.style={fixed, set thousands separator={\,}, column type=r, column name=$d_{\mathit{ind}}$}, 
    columns/min_jac_dis_0/.style={fixed, set thousands separator={\,}, column type=r, column name=$j_{1}$}, 
    columns/min_jac_dis_1/.style={fixed, set thousands separator={\,}, column type=r, column name=$j_{2}$}, 
    columns/min_jac_dis_2/.style={fixed, set thousands separator={\,}, column type=r, column name=$j_{3}$}, 
    columns/avg_jac_0/.style={fixed, set thousands separator={\,}, column type=r, column name=$j_{1}$}, 
    columns/avg_jac_1/.style={fixed, set thousands separator={\,}, column type=r, column name=$j_{2}$}, 
    columns/avg_jac_2/.style={fixed, set thousands separator={\,}, column type=r, column name=$j_{3}$}, 
    columns/itr_0/.style={fixed, set thousands separator={\,}, column type=r, column name=$i_1$},
    columns/itr_1/.style={fixed, set thousands separator={\,}, column type=r, column name=$i_2$},
    columns/dis_d_0/.style={fixed,set thousands separator={\,}, column type=r, column name=$d_1$},
    columns/dis_d_1/.style={fixed,set thousands separator={\,}, column type=r, column name=$d_2$},
    columns/dis_d_2/.style={fixed,set thousands separator={\,}, column type=r, column name=$d_3$},
    columns/dis_obj_1/.style={fixed,set thousands separator={\,}, column type=r, column name=$o_2$},
    columns/dis_obj_2/.style={fixed,set thousands separator={\,}, column type=r, column name=$o_3$},
    every head row/.style={
		before row={\toprule},
			after row=\midrule},
    every last row/.style={after row=\bottomrule},
]
{real-data.csv}
\end{table}




\subsection{Results for the synthetic dataset} 
The densest subgraph, that is, the solution to \problemdts, contained the whole graph with the density $74.33$ and the density difference of $22.84$. We tested our algorithm by setting the constraints to match the fair dense component, that is, we set $\alpha = 3.9$ for \problemcdcsm
and $\sbsnorm = 0.69$ so that $\sigma = 51.12$ for \problemcdcsdiff.  We report our results in Table~\ref{tab:stats-syn}.

The Jaccard indices in Table~\ref{tab:stats-syn} indicate that the exact algorithms \algipcm and \algipdiff algorithms discovered the underlying fair dense subgraph.
Next, we see that both \alggrdfms and \alggrd achieved a Jaccard index of $0.95$ and $0.96$, respectively, while being faster than their exact counterparts. The exact algorithms yielded solutions with slightly better scores than the greedy algorithms.

Finally, solving \problemdcs yielded the fair dense component $U$ since $U$ for this data also had the largest minimum degree $\min_i \dens{S, G_i}$.


\begin{table}[ht!]
\setlength{\tabcolsep}{0pt}
\caption{
Computational statistics from the experiments with real-world datasets using \algipcm and \alggrdfms algorithms 
which are denoted as $\algexact$ and $\alggrdshort$, respectively. Here, $\alpha$ is the input parameter in \problemcdcsm problem which is the minimum value of the allowed induced density difference,
 $d_{\mathit{sum}}$ is the discovered sum of densities, $\Delta$ gives the difference between the minimum and maximum density induced by the solution set, $i$ gives the number of iterations, \emph{size} gives the size of the discovered subgraph, and \emph{time} gives the computational time in seconds.
}

\label{tab:real}
\pgfplotstabletypeset[
    begin table={\begin{tabular*}{\textwidth}},
    end table={\end{tabular*}},
    col sep=comma,
	columns = {dataset,alpha,d1,d2,diff1,diff2, time1,time2,size1,size2,i1,i2},
    columns/dataset/.style={string type, column type={@{\extracolsep{\fill}}l}, column name=\emph{Data}},
    columns/time1/.style={fixed, set thousands separator={\,}, column type=r, precision = 0, column name=$\algexact$},
    columns/time2/.style={fixed, set thousands separator={\,}, column type=r, column name=$\alggrdshort$},
    columns/alpha/.style={fixed, set thousands separator={\,}, column type=r, column name=$\alpha$},
    columns/diff1/.style={fixed, set thousands separator={\,}, column type=r, column name=$\algexact$}, 
    columns/diff2/.style={fixed, set thousands separator={\,}, column type=r, column name=$\alggrdshort$}, 
    columns/axdts/.style={fixed, set thousands separator={\,}, column type=r, column name=$d_{\mathit{con}}$}, 
    columns/d1/.style={fixed, set thousands separator={\,}, column type=r, column name=$\algexact$}, 
    columns/d2/.style={fixed, set thousands separator={\,}, column type=r, column name=$\alggrdshort$}, 
    columns/i1/.style={fixed, set thousands separator={\,}, column type=r, column name=$i_{\algexact}$},
    columns/i2/.style={fixed, set thousands separator={\,}, column type=r, column name=$i_{\alggrdshort}$},
    columns/size1/.style={fixed, set thousands separator={\,}, column type=r, column name=$\algexact$},
    columns/size2/.style={fixed, set thousands separator={\,}, column type=r, column name=$\alggrdshort$},
    every head row/.style={
		before row={\toprule
		& & 
		\multicolumn{2}{l}{$d_{\mathit{sum}}$} &
		\multicolumn{2}{l}{$\Delta$ } &
             \multicolumn{2}{l}{time} &
             \multicolumn{2}{l}{size}  \\
		\cmidrule(r){3-4}
		\cmidrule(r){5-6}
		\cmidrule(r){7-8}
		\cmidrule(r){9-10}
  },
			after row=\midrule},
    every last row/.style={after row=\bottomrule},
]
{real-sum-diff.csv}
\end{table}

\begin{table}[ht!]
\setlength{\tabcolsep}{0pt}
\caption{
Computational statistics from the experiments with real-world datasets using using \algipdiff and \alggrd 
algorithms which are denoted as $\algexact$ and $\alggrdshort$, respectively. Here, $\sigma$ is the input parameter in \problemcdcsdiff where $\sigma = \sbsnorm \times \dentds$, $d_{\mathit{sum}}$ is the sum of densities induced by the discovered subgraph, $\Delta$ gives the difference between the minimum and maximum density, \emph{size} gives the size of the discovered subgraph, \emph{time} gives the computational time in seconds, and the columns $i_{\algexact}$ and $i_{\alggrdshort}$ give the number of iterations for respective algorithms.
}
\label{tab:real-diff}
\pgfplotstabletypeset[
    begin table={\begin{tabular*}{\textwidth}},
    end table={\end{tabular*}},
    %empty cells with={\ensuremath{-}},
    col sep=comma,
	columns = {dataset,alpha, axdts,diff1,diff2,d1,d2, time1,time2,size1,size2,i1,i2},
    columns/dataset/.style={string type, column type={@{\extracolsep{\fill}}l}, column name=\emph{Data},},
    columns/time1/.style={fixed, set thousands separator={\,}, column type=r, precision = 0, column name=$\algexact$},
    columns/time2/.style={fixed, set thousands separator={\,}, column type=r, column name=$\alggrdshort$},
    columns/alpha/.style={fixed, set thousands separator={\,}, column type=r, column name=$\sbsnorm$},
    columns/diff1/.style={fixed, set thousands separator={\,}, column type=r, column name=$\algexact$}, 
    columns/diff2/.style={fixed, set thousands separator={\,}, column type=r, column name=$\alggrdshort$}, 
    columns/axdts/.style={fixed, set thousands separator={\,}, column type=r, column name=$\sigma$}, 
    columns/d1/.style={fixed, set thousands separator={\,}, column type=r, column name=$\algexact$}, 
    columns/d2/.style={fixed, set thousands separator={\,}, column type=r, column name=$\alggrdshort$}, 
    columns/i1/.style={fixed, set thousands separator={\,}, column type=r, column name=$i_{\algexact}$},
    columns/i2/.style={fixed, set thousands separator={\,}, column type=r, column name=$i_{\alggrdshort}$},
    columns/size1/.style={fixed, set thousands separator={\,}, column type=r, column name=$\algexact$},
    columns/size2/.style={fixed, set thousands separator={\,}, column type=r, column name=$\alggrdshort$},
    every head row/.style={
		before row={\toprule
		& & &
		\multicolumn{2}{l}{$\Delta$} &
		\multicolumn{2}{l}{$d_{\mathit{sum}}$} &
             \multicolumn{2}{l}{time} &
             \multicolumn{2}{l}{size}  \\
		\cmidrule(r){4-5}
		\cmidrule(r){6-7}
		\cmidrule(r){8-9}
		\cmidrule(r){10-11}
  },
			after row=\midrule},
    every last row/.style={after row=\bottomrule},
]
{real-diff.csv}
\end{table}




\subsection{Real-world datasets} %\label{subsec:real}
We consider $7$ publicly available, real-world datasets. The details of the datasets are shown in Table~\ref{tab:stats4}.
\dtname{Twitter-\#}~\cite{tsantarliotis2015topic}\footnote{\url{https://github.com/ksemer/BestFriendsForever-BFF-}\label{foot:gitbff}}  is a hashtag network where nodes correspond to hashtags and edges corresponds to the
interactions where two hashtags appear in a tweet. This dataset contains $15$ such daily graph snapshots in total. 
\dtname{Facebook}~\cite{viswanath2009evolution}\footnote{\url{https://networkrepository.com/fb-wosn-friends.php} } is a network of Facebook users in  New Orleans regional
community.  It contains a set of Facebook wall posts among these users from 9th of June to 20th of August, 2006.
\dtname{Students}\footnote{\url{https://toreopsahl.com/datasets/\#online_social_network}} is an online message
network at the University of California, Irvine. It spans over $122$ days.
\dtname{Twitter-user}~\cite{rozenshtein2020finding}\footnote{\url{https://github.com/polinapolina/segmentation-meets-densest-subgraph}} is a network of twitter users in Helsinki 2013.
It contains a set of tweets that appear in each others' names.
\dtname{Tumblr} ~\cite{leskovec2009meme}\footnote{\url{http://snap.stanford.edu/data/memetracker9.html}} contains 
phrase or quote mentions appeared in blogs and news media.  It contains author and meme interactions of users over $3$ months from February to April in 2009. Finally, the datasets, \dtname{Hospital} and \dtname{Airport}~\cite{oettershagen2024finding}\footnote{\url{https://gitlab.com/densest/diverse } \label{foot:diverse-repo}} represent multi-layer networks.

\subsection{Results for the real-world datasets}
First, let us compare the scores produced by the baselines \problemdts and \problemdcs, given in Table~\ref{tab:stats4}. By definition, the density $\dentds$ is always larger than $\denmds$ as \problemdts optimizes the total density. Moreover, $\Delta_{\mathit{tds}}$ is always greater than $\Delta_{\mathit{mds}}$, which is a side-effect of \problemdcs optimizing the minimum density.

Next, we report our results for the \problemcdcsm problem in Table~\ref{tab:real}. 
First, we see that the discovered scores $d_{\mathit{sum}}$ by both algorithms increase as $\alpha$ increases. This is because as we increase $\alpha$ we let the difference in the maximum and minimum density vary within a larger interval so that our objective score may increase. We see that $\Delta$ is close or equal to $\alpha$, and that \algipcm is almost always larger than \alggrdfms. This is expected since \algipcm searches for the densest subgraph more aggressively.

Next, we compare the scores and time of our exact and greedy algorithms. As expected, \algipcm achieves a higher score than \alggrdfms. 
We can see that \alggrdfms runs faster than \algipcm, but only for about half of the cases.
The reason for \alggrdfms being slower is the search for a feasible set as the algorithm uses \alggrd as a subroutine. Both algorithms required a low number of iterations.
The binary search in $\algipcm$ required solving at most $22$ integer programs. 
The number of iterations for $\alggrdfms$ is also low, at most $8$.

Next, we report our results for the \problemcdcsdiff problem in Table~\ref{tab:real-diff}.
Unlike with \algipcm,
in several cases, \algipdiff required an extensive amount of time: we stopped searching for the exact solution, if 
\algipdiff did not finish in $1$ hour. 
First, we see that the exact solution tends to have smaller densities $d_{\mathit{sum}}$ than the greedy since the exact algorithm optimizes $\diff{}$ more aggressively.

Next, we compare the scores and running times of our exact and greedy algorithms as shown in the $\Delta$ and time columns, respectively. As expected the objective increases as we tighten the density constraint by increasing $\sbsnorm$.
Moreover, \algipdiff achieves smaller scores than \alggrd at the cost of computational time. 
Finally, we find that the number of iterations is reasonable in practice with both algorithms as shown in $i_{\algexact}$ and $i_{\alggrdshort}$ columns.

\begin{figure}[t!]
\begin{subcaptiongroup}
\phantomcaption\label{fig:hu}
\phantomcaption\label{fig:hc}
\phantomcaption\label{fig:hu2}
\phantomcaption\label{fig:hc2}
\begin{center}

\setlength{\tabcolsep}{0pt}
\begin{tabular}{lll}
\begin{tikzpicture}[baseline = 0pt]
\begin{axis}[
    width=4.7cm,
    symbolic x coords={CIKM, WWW, SDM, PAKDD, ICDM, ECML, VLDB, WSDM, ICDE, KDD}, 
        ylabel = {Density},
        enlarge x limits={0.05},
        bar width = 7pt,
        xtick={CIKM, WWW, SDM, PAKDD, ICDM, ECML, VLDB, WSDM, ICDE, KDD},
        xtick style={draw=none},
        xticklabel style={rotate=45, font=\tiny, align=center, inner ysep=0pt, anchor=north east, yshift=2pt, xshift=2pt}]    \addplot[ybar,fill=yafcolor1,draw=none] coordinates {(CIKM,0)};
    \addplot[ybar,fill=yafcolor2,draw=none]coordinates {(WWW,1.75)};
    \addplot[ybar,fill=yafcolor3,draw=none]coordinates {(SDM,0)};
    \addplot[ybar,fill=yafcolor4,draw=none]coordinates {(PAKDD,0)};
    \addplot[ybar,fill=yafcolor5,draw=none]coordinates {(ICDM,0)};
    \addplot[ybar,fill=yafcolor6,draw=none]coordinates {(ECML,0)};
    \addplot[ybar,fill=yafcolor7,draw=none]coordinates {(VLDB,0)};
    \addplot[ybar,fill=yafcolor8,draw=none]coordinates {(WSDM,0)};
    \addplot[ybar,fill=yafcolor9,draw=none]coordinates {(ICDE,0)};
    \addplot[ybar,fill=yafcolor10,draw=none]coordinates {(KDD,0)};
\pgfplotsextra{\yafdrawyaxis{0}{1.75}}
\end{axis}
\node[anchor=north east] at (-0.5, -0.3) {(a)};
\end{tikzpicture} &
\begin{tikzpicture}[baseline = 0pt]
\begin{axis}[
    width=4.7cm,
    ymin = 0,
    ymax = 0.30,
    symbolic x coords={CIKM, WWW, SDM, PAKDD, ICDM, ECML, VLDB, WSDM, ICDE, KDD}, 
        %ylabel = {Density},
        enlarge x limits={0.05},
        bar width = 7pt,
        xtick={CIKM, WWW, SDM, PAKDD, ICDM, ECML, VLDB, WSDM, ICDE, KDD},
        xtick style={draw=none},
        xticklabel style={rotate=45, font=\tiny, align=center, inner ysep=0pt, anchor=north east, yshift=2pt, xshift=2pt}]    \addplot[ybar,fill=yafcolor1,draw=none] coordinates {(CIKM,0.2266666667)};
    \addplot[ybar,fill=yafcolor2,draw=none]coordinates {(WWW,0.2266666667)};
    \addplot[ybar,fill=yafcolor3,draw=none]coordinates {(SDM,0.1155555556)};
    \addplot[ybar,fill=yafcolor4,draw=none]coordinates {(PAKDD,0.06666666667)};
    \addplot[ybar,fill=yafcolor5,draw=none]coordinates {(ICDM,0.2133333333)};
    \addplot[ybar,fill=yafcolor6,draw=none]coordinates {(ECML,0.2266666667)};
    \addplot[ybar,fill=yafcolor7,draw=none]coordinates {(VLDB,0.05333333333)};
    \addplot[ybar,fill=yafcolor8,draw=none]coordinates {(WSDM,0.1822222222)};
    \addplot[ybar,fill=yafcolor9,draw=none]coordinates {(ICDE,0.02666666667)};
    \addplot[ybar,fill=yafcolor10,draw=none]coordinates {(KDD,0.02666666667)};  
\pgfplotsextra{\yafdrawyaxis{0}{0.3}}
\end{axis}
\node[anchor=north east] at (-0.2, -0.3) {(b)};
\end{tikzpicture} &
\begin{tikzpicture}[baseline = 0pt]
\begin{axis}[
    width=4.7cm,
    %axis x line=center,
    %axis y line=left,
    ymin = 0,
    ymax = 0.30,
    symbolic x coords={CIKM, WWW, SDM, PAKDD, ICDM, ECML, VLDB, WSDM, ICDE, KDD}, 
        %ylabel = {Density},
        xtick={CIKM, WWW, SDM, PAKDD, ICDM, ECML, VLDB, WSDM, ICDE, KDD},
        enlarge x limits={0.05},
        bar width = 7pt,
        xtick style={draw=none},
        xticklabel style={rotate=45, font=\tiny, align=center, inner ysep=0pt, anchor=north east, yshift=2pt, xshift=2pt}]
    \addplot[ybar,fill=yafcolor1,draw=none] coordinates {(CIKM,0.27307692307692305)};
    \addplot[ybar,fill=yafcolor2,draw=none]coordinates {(WWW,0.27307692307692305)};
    \addplot[ybar,fill=yafcolor3,draw=none]coordinates {(SDM,0.1)};
    \addplot[ybar,fill=yafcolor4,draw=none]coordinates {(PAKDD,0.057692307692307696)};
    \addplot[ybar,fill=yafcolor5,draw=none]coordinates {(ICDM,0.17307692307692307)};
    \addplot[ybar,fill=yafcolor6,draw=none]coordinates {(ECML,0.2692307692307692)};
    \addplot[ybar,fill=yafcolor7,draw=none]coordinates {(VLDB,0.06153846153846154)};
    \addplot[ybar,fill=yafcolor8,draw=none]coordinates {(WSDM,0.16538461538461538)};
    \addplot[ybar,fill=yafcolor9,draw=none]coordinates {(ICDE,0.019230769230769232)};
    \addplot[ybar,fill=yafcolor10,draw=none]coordinates {(KDD,0.007692307692307693)};
\pgfplotsextra{\yafdrawyaxis{0}{0.3}}
\end{axis}
\node[anchor=north east] at (-0.2, -0.3) {(c)};
\end{tikzpicture}\\
\iffalse
\begin{tikzpicture}[baseline = 0pt]
\begin{axis}[
    width=6.1cm,
    symbolic x coords={CIKM, WWW, SDM, PAKDD, ICDM, ECML, VLDB, WSDM, ICDE, KDD}, 
        ylabel = {Dennsity},
        enlarge x limits={0.1},
        xtick={CIKM, WWW, SDM, PAKDD, ICDM, ECML, VLDB, WSDM, ICDE, KDD},
        xticklabel style={rotate=45, font=\scriptsize}]
    \addplot[ybar,fill=yafcolor1,draw=none] coordinates {(CIKM,0.11827956989247312)};
    \addplot[ybar,fill=yafcolor2,draw=none]coordinates {(WWW,0.11827956989247312)};
    \addplot[ybar,fill=yafcolor3,draw=none]coordinates {(SDM,0.11827956989247312)};
    \addplot[ybar,fill=yafcolor4,draw=none]coordinates {(PAKDD,0.11827956989247312)};
    \addplot[ybar,fill=yafcolor5,draw=none]coordinates {(ICDM,0.11827956989247312)};
    \addplot[ybar,fill=yafcolor6,draw=none]coordinates {(ECML,0.11827956989247312)};
    \addplot[ybar,fill=yafcolor7,draw=none]coordinates {(VLDB,0.11827956989247312)};
    \addplot[ybar,fill=yafcolor8,draw=none]coordinates {(WSDM,0.11827956989247312)};
    \addplot[ybar,fill=yafcolor9,draw=none]coordinates {(ICDE,0.11827956989247312)};
    \addplot[ybar,fill=yafcolor10,draw=none]coordinates {(KDD,0.11827956989247312)};
\end{axis}
\node[anchor=north east] at (-.5, -0.3) {(d)};
\end{tikzpicture}
\fi
\end{tabular}
\end{center}
\end{subcaptiongroup}
\caption{Dense subgraphs among different conferences in \dtname{DBLP-CS}: (\ref{fig:hu}) \problemdts solution with density value of $1.75$, (\ref{fig:hc}) \problemcdcsm solution  with density value of $1.36$ for $\alpha = 0.2$, and (\ref{fig:hu2}) 
 \problemcdcsdiff solution with density value of $1.4$ for $\sigma = 0.8  \dentds$.
}
\label{fig:hist}
\end{figure}


\subsection{Case study}
\dtname{DBLP-CS}~\cite{oettershagen2024finding}\footref{foot:diverse-repo} is a co-authorship network where each edge represents a co-authorship attributed with one of the top data mining conferences. The graph contains $15\,308$  number of nodes and $10$ publication venues. We sampled  $12\,000$ edges from the original dataset.

We first solved \problemdts subgraph and observed that the densest subgraph represents only one conference which is the Web Conference (Fig.~\ref{fig:hu}). We are interested in finding the densest subgraph that contains publications of diverse publication venues without badly under-representing any venue. 
To that end, we applied $\algipcm(\alpha = 0.2)$ algorithm. In Figure~\ref{fig:hc} we see that the densest subgraph attained by \algipcm contains the edges from all publication venues in contrast to Figure~\ref{fig:hu}. The density of the unconstrained version was $1.75$ while our algorithm achieved a fair dense subgraph with $\diff{} = 0.2$ at the cost of having a density value of $1.36$.
Next, we applied $\algipdiff(\sigma = 0.8\dentds = 1.4)$ algorithm. As shown in Figure~\ref{fig:hc2}, it achieves a density value of $1.4$ while having with $\diff{} = 0.27$. 
Finally, we solve \problemdcs, yielding a solution with a density value of $1.18$ and $\diff{} = 0$. Note that \problemdcs does not maximize the density $\dens{S, \calG}$, and does not have a mechanism for controlling the trade-off between the $\diff{S, \calG}$ and $\dens{S, \calG}$. On the other hand, by changing the parameters $\sigma$ or $\alpha$ we can regulate the trade-off.








