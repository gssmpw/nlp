\section{Introduction}

The problem of dense subgraph discovery is an important tool in graph mining with applications in temporal pattern mining in financial markets~\cite{xiaoxi2009migration}, social network analysis~\cite{semertzidis2019finding}, and biological system analysis~\cite{fratkin2006motifcut}. 
On the other hand, multi-layer graph networks naturally exist in real-world, complex networks and have gained a significant amount of attention~\cite{jethava2015finding,semertzidis2019finding,rozenshtein2020finding,galimberti2020core,arachchi2023jaccard}.

Among many definitions of a dense component, the ratio between the number of induced edges and the number of nodes has been popular since Goldberg~\cite{goldberg1984finding}
proposed a polynomial-time, exact algorithm to find the densest subgraph of a single graph and Charikar~\cite{charikar2000greedy} proposed a greedy approximation algorithm.

Given a graph sequence, a natural way to find the densest subgraph is to find a common subgraph that maximizes the sum of densities in individual snapshots. This is done by first flattening the graphs into a single weighted graph and solving the weighted densest graph problem~\cite{semertzidis2019finding}.

While this approach finds the densest subgraph, it may be the case that the densities are unevenly distributed. That is, the sum is dominated by the density of a single or few snapshots while the densities in the remaining snapshots are unfairly low or even 0.
%This approach only considers the aggregated graph as a whole and, therefore, does not take the fairness property into account.

In this paper, we consider a fair variant of the densest subgraph problem, where we require that the densities are close to each other across the snapshots. 
The topic of fairness is relatively new within the data mining and machine learning community which has been studied with classical problems like clustering, community detection, and anomaly detection~\cite{ahmadian2019clustering,mehrabi2019debiasing, mehrabi2021survey,anagnostopoulos2020spectral}.
The fairness in the context of the densest subgraph problems is not well-studied yet.


More specifically, we consider two variants of a fair dense subgraph problem.
In our first problem, we find a fair subgraph that maximizes the sum of the densities while being evenly distributed.
To ensure that the total density is fairly distributed among the graph snapshots, we constrain the difference between the maximum and minimum density induced by the subgraph via an input parameter that specifies the maximum allowed density difference.
In our second problem, given a pre-defined, minimum value for the total density as an input parameter, our goal is to minimize the gap between the maximum and minimum induced density over the graph sequence.
% while inducing the minimum amount of total density as defined by the input parameter.
 
We show that both of our problems are \np-hard, and that the second problem is inapproximable.
To solve our problems, we first propose two exact algorithms based on integer linear programming which run in exponential time. We also consider two heuristics that run in polynomial time in the size of the input graph.

The remainder of the paper is organized as follows.
In Section \ref{sec:prel}, we provide preliminary notation along with the
formal definitions of our optimization problems. Next, we prove
\np-hardness of our problems in Section~\ref{sec:compl}. All our
algorithms and their running times are presented in Section~\ref{sec:algorithm}. Related work is
discussed in Section~\ref{sec:related}. In Section~\ref{sec:exp} we present an extensive
experimental study both with synthetic and real-world datasets followed by a case study. 
Finally, Section~\ref{sec:conclusions} summarizes the paper and provides directions for future work.


