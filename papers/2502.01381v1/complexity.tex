\section{Computational complexity}\label{sec:compl}

In this section, we show that both of our problems are \np-hard.
We saw in the previous section that if
we set $\alpha  = \infty$, then \problemcdcsm reduces to \problemdts, which can be solved exactly in
polynomial time. However, \problemcdcsm is \np-hard when $\alpha = 0$.
%However, we show next that the complexity of fair densestsubgraph $\problemcdcsm$ which allows $0 < \alpha \leq 1$ is $\np$-hard.


\begin{proposition} 
\label{prop:np}
\problemcdcsm is \np-hard.
\end{proposition} 
\begin{proof}
We prove the hardness from $k$-\prbclique, a problem where, given a graph $H$, we are asked if there is a clique of size at least $k$.

Assume that we are given a graph $H = (V, E)$ with $n$ nodes, $n \geq k$. We set $\alpha = 0$.
The graph snapshot $G_1$ consists of the
graph $H$ and an additional set of $k$ singleton vertices $U$. $G_2$ consists of a $k$-clique connecting the vertices in $U$. 

We claim that there is a subset $S$ yielding $\dens{S, \calG} =  (k - 1)/2$ if and only if there is an $k$-clique in $H$.

Assume that  there is a subset $S$ yielding $\dens{S, \calG} =  (k - 1)/2$.
Since the value of objective is $(k - 1)/2$, we have $\dens{S, G_1} = \dens{S, G_2} = (k - 1)/4$. 
Let $S = W \cup T$ where $W \subseteq V$ and denotes the subset of vertices from $H$ and $T \subseteq U$ is the subset of vertices from  $U$ in $S$. 

Assume that $\abs{W} < \abs{T}$. Since $\abs{T} \leq k$, $\abs{W} < k$.  The density induced on $G_1$ is bounded by $\dens{S, G_1} \leq \frac{{\abs{W} \choose 2}}{\abs{T} + \abs{W}} < \frac{{\abs{W} \choose 2}}{2\abs{W}} < \frac{k - 1}{4}$, which is a contradiction.
Assume that  $\abs{W} > \abs{T}$. Then the density induced on $G_2$ is bounded by $\dens{S, G_2} = \frac{{\abs{T} \choose 2}}{\abs{T} + \abs{W}} < \frac{{\abs{T} \choose 2}}{2\abs{T}} = \frac{\abs{T} - 1}{4} \leq \frac{k - 1}{4}$, again a contradiction. Therefore, $\abs{W} =  \abs{T}$. 

Consequently, $\dens{S, G_2} = (\abs{T} - 1)/4$, implying that $\abs{T} = k$. Finally, $\frac{k - 1}{4} = \dens{S, G_1} = \frac{\abs{E(S)}}{2k}$ implies that $\abs{E} = {k \choose 2}$, that is, $W$ is a $k$-clique in $H$.

%Let  $d_1$ and $d_2$ are the densities induced on $G_1$ and $G_2$ respectively by $S$.
%Based on our assumption, $\dens{S, \calG} =  (k - 1)/2$ yields and 
%the only way to obtain a density value  of $f(S, G_2) = (k - 1)/4$  which satisfies marginal density constraint is that  $S = W \cup U$ and thus there should be an $k$-clique in $H$.
%Therefore, if the density is $  (k - 1)/4$ there is an $k$-clique in $H$.


On the other hand, assume there is a clique $C$ of size $k$ in $H$.
%Let clique $C$ is formed by the set of vertices in $W$.
% Let $d_1$ and $d_2$ are the optimal densities induced on $G_1$ and $G_2$.
% Since the density constraint should be satisfied $d_1 \geq (d_1 + d_2)/2$ and $d_2 \geq (d_1 + d_2)/2$ which implies $d_1 \geq d_2$ and $d_2 \geq d_1$. Therefore, $d_1 = d_2$.  
% Let $k_1$ and $k_2$ number of non-singleton nodes contribute for $d_1$ and $d_2$ respectively.
% Since $d_1 = d_2$,  $\frac{ k_1(k_1 - 1)}{ 2(k_1 + k_2)} = \frac{k_2 (k_2 - 1)}{ 2(k_1 + k_2)}$ which implies $k_1 =  k_2$. Then  the sum of the densities is given by $\frac{ (k_1 - 1)}{ 2} $ where the optimal is  when $k_1  = k =  k_2$.
Set $S = C \cup U$. 
Immediately, $\dens{S, \calG} =   (k - 1)/2$ proving the claim.
\qed
\end{proof}

\iffalse
\begin{proposition}
\label{prop:inapproximability}
\problemcdcsm does not have any polynomial time approximation algorithm with an approximation ratio better than $n^{1-\epsilon}$ for any constant $\epsilon > 0$, unless $\np = \zpp$.
\end{proposition}
\fi

A similar proof will show that \problemcdcsdiff is \np-hard, and inapproximable.

\begin{proposition} 
\label{prop:np2}
\problemcdcsdiff is \np-hard.
Unless $\poly = \np$, there is no polynomial-time algorithm with multiplicative approximation guarantee for \problemcdcsdiff. 
\end{proposition} 

\begin{proof}
We use the same reduction from $k$-\prbclique as in the proof of Proposition~\ref{prop:np}.
We also set $\sigma = (k - 1)/2$. If there is a clique $C$ in $H$, then selecting $S = C \cup U$ yields $\diff{S, \calG} = 0$.
On the other hand, if $\diff{S, \calG} = 0$, then $\dens{S, G_1} = \dens{S, G_2} = (k - 1)/4$, and the argument in the proof of Proposition~\ref{prop:np} shows that there must be a $k$-clique in $H$.
In summary, the difference $\diff{S, \calG} = 0$ for a solution $S$ if and only if there is a $k$-clique in $H$.

This also immediately implies that there is no polynomial-time algorithm with multiplicative approximation guarantee since this algorithm can be then used to test whether there is a set $S$ with $\diff{S, \calG} = 0$.
\qed
\end{proof}