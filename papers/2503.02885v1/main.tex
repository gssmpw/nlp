%%
%% This is file `sample-manuscript.tex',
%% generated with the docstrip utility.
%%
%% The original source files were:
%%
%% samples.dtx  (with options: `all,proceedings,bibtex,manuscript')
%% 
%% 
%% For distribution of the original source see the terms
%% for copying and modification in the file samples.dtx.
%% 
%% This generated file may be distributed as long as the
%% original source files, as listed above, are part of the
%% same distribution. (The sources need not necessarily be
%% in the same archive or directory.)
%%
%%
%% Commands for TeXCount
%TC:macro \cite [option:text,text]
%TC:macro \citep [option:text,text]
%TC:macro \citet [option:text,text]
%TC:envir table 0 1
%TC:envir table* 0 1
%TC:envir tabular [ignore] word
%TC:envir displaymath 0 word
%TC:envir math 0 word
%TC:envir comment 0 0
%%
%% The first command in your LaTeX source must be the \documentclass
%% command.
%%
%% For submission and review of your manuscript please change the
%% command to \documentclass[manuscript, screen, review]{acmart}.
%%
%% When submitting camera ready or to TAPS, please change the command
%% to \documentclass[sigconf]{acmart} or whichever template is required
%% for your publication.
%%
%%
%\documentclass[manuscript,screen,review,anonymous]{acmart}
\documentclass[manuscript,nonacm]{acmart}
%%
%% \BibTeX command to typeset BibTeX logo in the docs
\AtBeginDocument{%
  \providecommand\BibTeX{{%
    Bib\TeX}}}

%% Rights management information.  This information is sent to you
%% when you complete the rights form.  These commands have SAMPLE
%% values in them; it is your responsibility as an author to replace
%% the commands and values with those provided to you when you
%% complete the rights form.
\setcopyright{none}
\copyrightyear{2025}
\acmYear{2025}
\acmDOI{XXXXXXX.XXXXXXX}

\usepackage{moreverb,url}
\usepackage{xcolor}
\usepackage{xspace}
\usepackage{subcaption}  
\usepackage{graphicx} 
\usepackage{multirow}  
\usepackage{pdfpages}
\usepackage{wrapfig}

\usepackage{longtable}

\newcommand*\first{\emph{(i)}\xspace}
\newcommand*\second{\emph{(ii)}\xspace}
\newcommand*\third{\emph{(iii)}\xspace}

\newcommand{\etal}{et al.\xspace}
\newcommand*{\eg}{e.g.,\xspace}
\newcommand*{\ie}{i.e.,\xspace}

\newcommand*\framework{\emph{Co-PACE}\xspace}
\newcommand*\frameworklong{\emph{Contextualized Perceptions for the Adoption of Chatbots in Education}\xspace}

\newcommand\BibTeX{{\rmfamily B\kern-.05em \textsc{i\kern-.025em b}\kern-.08em
T\kern-.1667em\lower.7ex\hbox{E}\kern-.125emX}}

\def\volumeyear{2025}

\begin{document}

\title{``Would You Want an AI Tutor?''  Understanding Stakeholder Perceptions of LLM-based Chatbots in the Classroom}

\author{Caterina Fuligni}
\email{cf3181@nyu.edu}
\orcid{1234-5678-9012}
\affiliation{%
  \institution{New York University}
  \city{New York City}
  \state{New York}
  \country{USA}
}

\author{Daniel Dominguez Figaredo}
\email{ddominguez@edu.uned.es}
\orcid{https://orcid.org/0000-0002-7772-1856}
\affiliation{%
  \institution{Universidad Nacional de Educación a Distancia}
  \city{Madrid}
  \country{Spain}
}

\author{Julia Stoyanovich}
\email{stoyanovich@nyu.edu}
\orcid{https://orcid.org/0000-0002-7772-1856}
\affiliation{%
  \institution{New York University}
  \city{New York City}
  \country{USA}
}

\begin{abstract}
\begin{abstract}  
Test time scaling is currently one of the most active research areas that shows promise after training time scaling has reached its limits.
Deep-thinking (DT) models are a class of recurrent models that can perform easy-to-hard generalization by assigning more compute to harder test samples.
However, due to their inability to determine the complexity of a test sample, DT models have to use a large amount of computation for both easy and hard test samples.
Excessive test time computation is wasteful and can cause the ``overthinking'' problem where more test time computation leads to worse results.
In this paper, we introduce a test time training method for determining the optimal amount of computation needed for each sample during test time.
We also propose Conv-LiGRU, a novel recurrent architecture for efficient and robust visual reasoning. 
Extensive experiments demonstrate that Conv-LiGRU is more stable than DT, effectively mitigates the ``overthinking'' phenomenon, and achieves superior accuracy.
\end{abstract}  
\end{abstract}

\begin{CCSXML}
<ccs2012>
<concept>
<concept_id>10003120</concept_id>
<concept_desc>Human-centered computing</concept_desc>
<concept_significance>300</concept_significance>
</concept>
<concept>
<concept_id>10003456.10003457.10003567.10010990</concept_id>
<concept_desc>Social and professional topics~Socio-technical systems</concept_desc>
<concept_significance>500</concept_significance>
</concept>
</ccs2012>
\end{CCSXML}

\ccsdesc[500]{Social and professional topics~Socio-technical systems}
\ccsdesc[300]{Human-centered computing}

\keywords{ChatGPT; Large Language Models; LLM-powered bots; Perceptions; Education}

\maketitle

\section{Introduction}
\label{sec:introduction}
The business processes of organizations are experiencing ever-increasing complexity due to the large amount of data, high number of users, and high-tech devices involved \cite{martin2021pmopportunitieschallenges, beerepoot2023biggestbpmproblems}. This complexity may cause business processes to deviate from normal control flow due to unforeseen and disruptive anomalies \cite{adams2023proceddsriftdetection}. These control-flow anomalies manifest as unknown, skipped, and wrongly-ordered activities in the traces of event logs monitored from the execution of business processes \cite{ko2023adsystematicreview}. For the sake of clarity, let us consider an illustrative example of such anomalies. Figure \ref{FP_ANOMALIES} shows a so-called event log footprint, which captures the control flow relations of four activities of a hypothetical event log. In particular, this footprint captures the control-flow relations between activities \texttt{a}, \texttt{b}, \texttt{c} and \texttt{d}. These are the causal ($\rightarrow$) relation, concurrent ($\parallel$) relation, and other ($\#$) relations such as exclusivity or non-local dependency \cite{aalst2022pmhandbook}. In addition, on the right are six traces, of which five exhibit skipped, wrongly-ordered and unknown control-flow anomalies. For example, $\langle$\texttt{a b d}$\rangle$ has a skipped activity, which is \texttt{c}. Because of this skipped activity, the control-flow relation \texttt{b}$\,\#\,$\texttt{d} is violated, since \texttt{d} directly follows \texttt{b} in the anomalous trace.
\begin{figure}[!t]
\centering
\includegraphics[width=0.9\columnwidth]{images/FP_ANOMALIES.png}
\caption{An example event log footprint with six traces, of which five exhibit control-flow anomalies.}
\label{FP_ANOMALIES}
\end{figure}

\subsection{Control-flow anomaly detection}
Control-flow anomaly detection techniques aim to characterize the normal control flow from event logs and verify whether these deviations occur in new event logs \cite{ko2023adsystematicreview}. To develop control-flow anomaly detection techniques, \revision{process mining} has seen widespread adoption owing to process discovery and \revision{conformance checking}. On the one hand, process discovery is a set of algorithms that encode control-flow relations as a set of model elements and constraints according to a given modeling formalism \cite{aalst2022pmhandbook}; hereafter, we refer to the Petri net, a widespread modeling formalism. On the other hand, \revision{conformance checking} is an explainable set of algorithms that allows linking any deviations with the reference Petri net and providing the fitness measure, namely a measure of how much the Petri net fits the new event log \cite{aalst2022pmhandbook}. Many control-flow anomaly detection techniques based on \revision{conformance checking} (hereafter, \revision{conformance checking}-based techniques) use the fitness measure to determine whether an event log is anomalous \cite{bezerra2009pmad, bezerra2013adlogspais, myers2018icsadpm, pecchia2020applicationfailuresanalysispm}. 

The scientific literature also includes many \revision{conformance checking}-independent techniques for control-flow anomaly detection that combine specific types of trace encodings with machine/deep learning \cite{ko2023adsystematicreview, tavares2023pmtraceencoding}. Whereas these techniques are very effective, their explainability is challenging due to both the type of trace encoding employed and the machine/deep learning model used \cite{rawal2022trustworthyaiadvances,li2023explainablead}. Hence, in the following, we focus on the shortcomings of \revision{conformance checking}-based techniques to investigate whether it is possible to support the development of competitive control-flow anomaly detection techniques while maintaining the explainable nature of \revision{conformance checking}.
\begin{figure}[!t]
\centering
\includegraphics[width=\columnwidth]{images/HIGH_LEVEL_VIEW.png}
\caption{A high-level view of the proposed framework for combining \revision{process mining}-based feature extraction with dimensionality reduction for control-flow anomaly detection.}
\label{HIGH_LEVEL_VIEW}
\end{figure}

\subsection{Shortcomings of \revision{conformance checking}-based techniques}
Unfortunately, the detection effectiveness of \revision{conformance checking}-based techniques is affected by noisy data and low-quality Petri nets, which may be due to human errors in the modeling process or representational bias of process discovery algorithms \cite{bezerra2013adlogspais, pecchia2020applicationfailuresanalysispm, aalst2016pm}. Specifically, on the one hand, noisy data may introduce infrequent and deceptive control-flow relations that may result in inconsistent fitness measures, whereas, on the other hand, checking event logs against a low-quality Petri net could lead to an unreliable distribution of fitness measures. Nonetheless, such Petri nets can still be used as references to obtain insightful information for \revision{process mining}-based feature extraction, supporting the development of competitive and explainable \revision{conformance checking}-based techniques for control-flow anomaly detection despite the problems above. For example, a few works outline that token-based \revision{conformance checking} can be used for \revision{process mining}-based feature extraction to build tabular data and develop effective \revision{conformance checking}-based techniques for control-flow anomaly detection \cite{singh2022lapmsh, debenedictis2023dtadiiot}. However, to the best of our knowledge, the scientific literature lacks a structured proposal for \revision{process mining}-based feature extraction using the state-of-the-art \revision{conformance checking} variant, namely alignment-based \revision{conformance checking}.

\subsection{Contributions}
We propose a novel \revision{process mining}-based feature extraction approach with alignment-based \revision{conformance checking}. This variant aligns the deviating control flow with a reference Petri net; the resulting alignment can be inspected to extract additional statistics such as the number of times a given activity caused mismatches \cite{aalst2022pmhandbook}. We integrate this approach into a flexible and explainable framework for developing techniques for control-flow anomaly detection. The framework combines \revision{process mining}-based feature extraction and dimensionality reduction to handle high-dimensional feature sets, achieve detection effectiveness, and support explainability. Notably, in addition to our proposed \revision{process mining}-based feature extraction approach, the framework allows employing other approaches, enabling a fair comparison of multiple \revision{conformance checking}-based and \revision{conformance checking}-independent techniques for control-flow anomaly detection. Figure \ref{HIGH_LEVEL_VIEW} shows a high-level view of the framework. Business processes are monitored, and event logs obtained from the database of information systems. Subsequently, \revision{process mining}-based feature extraction is applied to these event logs and tabular data input to dimensionality reduction to identify control-flow anomalies. We apply several \revision{conformance checking}-based and \revision{conformance checking}-independent framework techniques to publicly available datasets, simulated data of a case study from railways, and real-world data of a case study from healthcare. We show that the framework techniques implementing our approach outperform the baseline \revision{conformance checking}-based techniques while maintaining the explainable nature of \revision{conformance checking}.

In summary, the contributions of this paper are as follows.
\begin{itemize}
    \item{
        A novel \revision{process mining}-based feature extraction approach to support the development of competitive and explainable \revision{conformance checking}-based techniques for control-flow anomaly detection.
    }
    \item{
        A flexible and explainable framework for developing techniques for control-flow anomaly detection using \revision{process mining}-based feature extraction and dimensionality reduction.
    }
    \item{
        Application to synthetic and real-world datasets of several \revision{conformance checking}-based and \revision{conformance checking}-independent framework techniques, evaluating their detection effectiveness and explainability.
    }
\end{itemize}

The rest of the paper is organized as follows.
\begin{itemize}
    \item Section \ref{sec:related_work} reviews the existing techniques for control-flow anomaly detection, categorizing them into \revision{conformance checking}-based and \revision{conformance checking}-independent techniques.
    \item Section \ref{sec:abccfe} provides the preliminaries of \revision{process mining} to establish the notation used throughout the paper, and delves into the details of the proposed \revision{process mining}-based feature extraction approach with alignment-based \revision{conformance checking}.
    \item Section \ref{sec:framework} describes the framework for developing \revision{conformance checking}-based and \revision{conformance checking}-independent techniques for control-flow anomaly detection that combine \revision{process mining}-based feature extraction and dimensionality reduction.
    \item Section \ref{sec:evaluation} presents the experiments conducted with multiple framework and baseline techniques using data from publicly available datasets and case studies.
    \item Section \ref{sec:conclusions} draws the conclusions and presents future work.
\end{itemize}
\section{RELATED WORK}
\label{sec:relatedwork}
In this section, we describe the previous works related to our proposal, which are divided into two parts. In Section~\ref{sec:relatedwork_exoplanet}, we present a review of approaches based on machine learning techniques for the detection of planetary transit signals. Section~\ref{sec:relatedwork_attention} provides an account of the approaches based on attention mechanisms applied in Astronomy.\par

\subsection{Exoplanet detection}
\label{sec:relatedwork_exoplanet}
Machine learning methods have achieved great performance for the automatic selection of exoplanet transit signals. One of the earliest applications of machine learning is a model named Autovetter \citep{MCcauliff}, which is a random forest (RF) model based on characteristics derived from Kepler pipeline statistics to classify exoplanet and false positive signals. Then, other studies emerged that also used supervised learning. \cite{mislis2016sidra} also used a RF, but unlike the work by \citet{MCcauliff}, they used simulated light curves and a box least square \citep[BLS;][]{kovacs2002box}-based periodogram to search for transiting exoplanets. \citet{thompson2015machine} proposed a k-nearest neighbors model for Kepler data to determine if a given signal has similarity to known transits. Unsupervised learning techniques were also applied, such as self-organizing maps (SOM), proposed \citet{armstrong2016transit}; which implements an architecture to segment similar light curves. In the same way, \citet{armstrong2018automatic} developed a combination of supervised and unsupervised learning, including RF and SOM models. In general, these approaches require a previous phase of feature engineering for each light curve. \par

%DL is a modern data-driven technology that automatically extracts characteristics, and that has been successful in classification problems from a variety of application domains. The architecture relies on several layers of NNs of simple interconnected units and uses layers to build increasingly complex and useful features by means of linear and non-linear transformation. This family of models is capable of generating increasingly high-level representations \citep{lecun2015deep}.

The application of DL for exoplanetary signal detection has evolved rapidly in recent years and has become very popular in planetary science.  \citet{pearson2018} and \citet{zucker2018shallow} developed CNN-based algorithms that learn from synthetic data to search for exoplanets. Perhaps one of the most successful applications of the DL models in transit detection was that of \citet{Shallue_2018}; who, in collaboration with Google, proposed a CNN named AstroNet that recognizes exoplanet signals in real data from Kepler. AstroNet uses the training set of labelled TCEs from the Autovetter planet candidate catalog of Q1–Q17 data release 24 (DR24) of the Kepler mission \citep{catanzarite2015autovetter}. AstroNet analyses the data in two views: a ``global view'', and ``local view'' \citep{Shallue_2018}. \par


% The global view shows the characteristics of the light curve over an orbital period, and a local view shows the moment at occurring the transit in detail

%different = space-based

Based on AstroNet, researchers have modified the original AstroNet model to rank candidates from different surveys, specifically for Kepler and TESS missions. \citet{ansdell2018scientific} developed a CNN trained on Kepler data, and included for the first time the information on the centroids, showing that the model improves performance considerably. Then, \citet{osborn2020rapid} and \citet{yu2019identifying} also included the centroids information, but in addition, \citet{osborn2020rapid} included information of the stellar and transit parameters. Finally, \citet{rao2021nigraha} proposed a pipeline that includes a new ``half-phase'' view of the transit signal. This half-phase view represents a transit view with a different time and phase. The purpose of this view is to recover any possible secondary eclipse (the object hiding behind the disk of the primary star).


%last pipeline applies a procedure after the prediction of the model to obtain new candidates, this process is carried out through a series of steps that include the evaluation with Discovery and Validation of Exoplanets (DAVE) \citet{kostov2019discovery} that was adapted for the TESS telescope.\par
%



\subsection{Attention mechanisms in astronomy}
\label{sec:relatedwork_attention}
Despite the remarkable success of attention mechanisms in sequential data, few papers have exploited their advantages in astronomy. In particular, there are no models based on attention mechanisms for detecting planets. Below we present a summary of the main applications of this modeling approach to astronomy, based on two points of view; performance and interpretability of the model.\par
%Attention mechanisms have not yet been explored in all sub-areas of astronomy. However, recent works show a successful application of the mechanism.
%performance

The application of attention mechanisms has shown improvements in the performance of some regression and classification tasks compared to previous approaches. One of the first implementations of the attention mechanism was to find gravitational lenses proposed by \citet{thuruthipilly2021finding}. They designed 21 self-attention-based encoder models, where each model was trained separately with 18,000 simulated images, demonstrating that the model based on the Transformer has a better performance and uses fewer trainable parameters compared to CNN. A novel application was proposed by \citet{lin2021galaxy} for the morphological classification of galaxies, who used an architecture derived from the Transformer, named Vision Transformer (VIT) \citep{dosovitskiy2020image}. \citet{lin2021galaxy} demonstrated competitive results compared to CNNs. Another application with successful results was proposed by \citet{zerveas2021transformer}; which first proposed a transformer-based framework for learning unsupervised representations of multivariate time series. Their methodology takes advantage of unlabeled data to train an encoder and extract dense vector representations of time series. Subsequently, they evaluate the model for regression and classification tasks, demonstrating better performance than other state-of-the-art supervised methods, even with data sets with limited samples.

%interpretation
Regarding the interpretability of the model, a recent contribution that analyses the attention maps was presented by \citet{bowles20212}, which explored the use of group-equivariant self-attention for radio astronomy classification. Compared to other approaches, this model analysed the attention maps of the predictions and showed that the mechanism extracts the brightest spots and jets of the radio source more clearly. This indicates that attention maps for prediction interpretation could help experts see patterns that the human eye often misses. \par

In the field of variable stars, \citet{allam2021paying} employed the mechanism for classifying multivariate time series in variable stars. And additionally, \citet{allam2021paying} showed that the activation weights are accommodated according to the variation in brightness of the star, achieving a more interpretable model. And finally, related to the TESS telescope, \citet{morvan2022don} proposed a model that removes the noise from the light curves through the distribution of attention weights. \citet{morvan2022don} showed that the use of the attention mechanism is excellent for removing noise and outliers in time series datasets compared with other approaches. In addition, the use of attention maps allowed them to show the representations learned from the model. \par

Recent attention mechanism approaches in astronomy demonstrate comparable results with earlier approaches, such as CNNs. At the same time, they offer interpretability of their results, which allows a post-prediction analysis. \par


\section{A Review of perceptions of LLM-based chatbots}
\label{sec:survey} 

We aimed to analyze stakeholder perceptions of LLM-powered chatbots in educational settings, with a focus on identifying the most recurring perceptions for each stakeholder group. We were guided by two research questions:

\textbf{RQ1:} Who are the key stakeholders and what are their perceptions, as identified in the literature?

\textbf{RQ2:} What are the main gaps in the literature regarding stakeholders and their perceptions?

\begin{table*}[b!]
    \centering
    \caption{Number of papers per type of study design}
    \begin{tabular}{llc}
        \toprule
         \multicolumn{2}{l}{\textbf{Study type}} & \textbf{\# studies}  \\
        \hline  
    \multirow{3}{*}{Non-experimental} & Interviews or open-ended questions surveys  & 13 \\ 
                                      & Mixed-methods         & 12            \\ 
                                      & Questionnaires          & 3            \\  \hline 
    \multirow{2}{*}{Quasi-experimental} & Pre-test/Post-test     & 2  \\
                                        &  Mixed-methods        & 1      \\   \hline 
    Pre-Experimental & One-shot case study     & 1    \\  \hline 
    \multirow{3}{*}{Experimental} & Mixed-methods             & 3 \\
            & Randomized controlled trial    & 2 \\
            & Multi-arm randomized trial      & 1\\
    \bottomrule
    \end{tabular}
    \label{tab:design}
\end{table*}

\subsection{Methods}
\label{sec:survey:methods}

\subsubsection{Selection procedure}
\label{sec:survey:methods:selection}

We used Google Scholar to search for scientific literature on the topics of interest. To be considered, articles had to include the search strings (``Large language models'' OR ``LLMs'' OR ``LLM-based tutor'' OR ``LLM-powered chatbot'' OR ``ChatGPT'') AND (``perceptions'' OR ``perspectives'' OR ``attitudes'') in their title, abstract, keywords, or introduction. For the selection procedure, we limited the number of results to the first 100 articles, using Google Scholar's relevance sort order.  Articles were included if they met the following criteria: (1) discussed LLM-powered bots in education; (2) addressed perceptions of education stakeholders, such as students, teachers, parents, school staff, or education-related professionals; (3) were based on experimental, quasi-experimental, pre-experimental, or non-experimental research (excluding commentaries, secondary data analyses, and literature reviews); and (4) were written in English. 
 Finally, the study must involve LLM-powered chatbots. Studies that only mentioned ``AI tool'' without specifying whether they referred to large language model-based technology or rule-based systems were excluded.   This resulted in the selection of $N=38$ full-text articles for analysis.

\subsubsection{Content analysis}
\label{sec:survey:methods:analysis}

After an initial reading of the articles, we identified recurring patterns or themes and developed a codebook to capture them. The codebook was developed combining codes taken from articles that employed existing theoretical frameworks (deductive) and codes that emerged from the articles without available codebooks or that did not develop one (inductive). In the latter cases, we annotated the results and discussion sections of the articles. As for the level of implication allowed, we included both explicit references to concepts and those implied by words or phrases. We allowed for flexibility by adding codes throughout the iterative readings. 

Hence, we coded for the frequency, that is, we examined the presence of a selected code in the articles. We identified whether each code or concept reflected negative perceptions or concerns, positive perceptions or opportunities, or neutral perceptions. Additionally, we identified the specific stakeholder group(s) from which each code or concept originated. The validity of the coding process was ensured through multiple rounds of coding, allowing for refinement and consistency across the analysis.

\subsection{Findings}
\label{sec:survey:findings}

\subsubsection{Basic summary information}
\label{sec:survey:findings:summary}

We now summarize the basic break-down of the $N=38$ articles included in our analysis. As shown in Figure~\ref{fig:stats:geo}, the majority of the included articles had samples from Asia ($N = 21$), followed by Europe ($N = 20$), North America ($N = 18$), Oceania ($N = 5$), Latin America ($N = 4$), and Africa ($N = 4$). (Note that several articles included samples from multiple continents.) When the location of the participants was not specified, the location of the authors' university was selected as a proxy. 

\begin{figure}[b!]
\begin{subfigure}[h]{0.55\linewidth}
    \includegraphics[width=\linewidth]{figs/continents.pdf}
    \caption{continents}
    \label{fig:stats:geo}
\end{subfigure}
\hfill 
\begin{subfigure}[h]{0.4\linewidth}
    \includegraphics[width=\linewidth]{figs/contexts.pdf}
    \caption{educational contexts}
    \label{fig:stats:contexts}
\end{subfigure}
  \caption{Break-down of statistics of the analyzed articles ($N=38$).}
  \label{fig:stats}
\end{figure}

Regarding the contexts of application of the LLM-based chatbots, Figure \ref{fig:stats:contexts} illustrates that the most frequent educational context of the studies was STEM ($N = 19$, 50\%), followed by English-as-foreign-language (EFL) ($N = 5$, 13\%) and Humanities ($N = 1$). A considerable number of studies did not specify the educational context ($N = 13$, 34\%).

Table \ref{tab:design} presents a break-down by type of study design. Most of the articles employed non-experimental study designs. 13 studies used interviews or surveys with exclusively open-ended questions, most of which were semi-structured. 12 studies used mixed methods, including surveys with both open- and closed-ended questions, as well as less common methodologies, such as having participants draw a visual representation of their perception of ChatGPT followed by a survey~\cite{ding2023students}. Three studies used quantitative questionnaires. Three articles used quasi-experimental research designs, two of which used pre- and post-test methodology, and one, mixed methods. One study was pre-experimental, using a one-shot case study design. Among the experimental designs, the most common was mixed methods, followed by randomized controlled trials (RCT) and multi-arm randomized trials. 

\subsubsection{\textbf{RQ1:} Key stakeholders and their perceptions}
\label{sec:survey:findings:peceptions}

\begin{wrapfigure}{r}{0.35\textwidth}
    \centering
    \includegraphics[width=\linewidth]{figs/stakeholders.pdf} 
    \caption{Stakeholders whose perceptions were analyzed in the articles ($N=38$).}
    \label{fig:stats:stakehoders}
\end{wrapfigure} As shown in Figure~\ref{fig:stats:stakehoders}, students were the most studied stakeholder group, appearing in 31 of the 38 (82\%) studies, 27 of which involved University-level students and 4 involved K-12 students. Teacher perceptions were analyzed in 12 (32\%) studies.  Additionally, perceptions of industry professionals were considered in 1 study, albeit without explicitly specifying whether these were ed-tech professionals or learners.  No studies included any other stakeholders, such as parents or school administrators.

We identified different types of perception---positive (opportunities/benefits), negative (concerns), and neutral--- in the articles we analyzed. We organized these perceptions into five categories: \textit{Attributes}, \textit{Impact on Processes or Outcomes}, \textit{Functions}, \textit{User Experience}, and \textit{Ethical and Societal Implications}.  Figure~\ref{fig:perceptions} summarizes results of our analysis, showing the total number of mentions of perceptions in each category separately for two stakeholder groups, Students (Figure~\ref{fig:perceptions:students}) and Teachers (Figure~\ref{fig:perceptions:teachers}).  Complete results, with frequencies for each perception type within each category, are provided in Appendix~\ref{app:stats}.  We present additional details about each perception category in the remainder of this section.   Note that, while we do not provide quantitative details about industry professionals as a stakeholder group in either Figure~\ref{fig:perceptions} or Appendix~\ref{app:stats}, because they were only covered in one article, we do include them when discussing qualitative findings.

Overall, the perceptions of both students and teachers were more frequently positive than either negative or neutral: with 58\% of all perceptions were positive and 33\% were negative for students vs. 51\% positive and 38\% negative for teachers.  Additionally, while both students' and teachers' perceptions focused heavily on the ethics of LLM-based chatbots, particularly in a negative or concerned light, students also expressed views on UX, whereas teachers concentrated on the functions of LLM-powered bots. In contrast to the ethics of LLM-based bots, the topics of UX and functions were primarily addressed by their respective stakeholders in terms of opportunities or positive aspects.

\paragraph{Attributes.} This category refers to perceptions of the characteristics of LLM-based chatbots. The papers we surveyed discussed the perceptions of students, teachers, and industry professionals regarding these attributes. The key attributes identified included: Convenience (also referred to as ``accessibility'', ``responsiveness'' or ``speed''), Accuracy (also referred to as ``confidence and trust in LLMs''), Knowledge base, Anthropomorphism (also referred to as ``human-like qualities'', ``perceived humanness'' and ``Natural Language Processing (NLP) capabilities''), and various other, less popular technical specifications, such as character limit and code editing. Perceptions of these attributes were mixed, with each being viewed both positively and negatively, except for Convenience, which was consistently seen as an advantage. Additionally, no neutral perceptions were found regarding these attributes. 

The most frequent positive perception among students and teachers is Convenience. For example, students in \citet{kazemitabaar2024codeaid} appreciated CodeAid's ``24/7 availability'', whereas teachers in \citet{mohamed2024exploring} highlighted the bot's ``real-time interaction and feedback''.

In contrast, the most recurring concern for students, teachers, and industry professionals is Accuracy. For example, students in \citet{weber2024measuring} were concerned with the accuracy of answers provided by LLMs, while teachers in \citet{ghimire2024generative} stated that generative AI can produce ``incorrect or fabricated results.'' However, an industry professional in \citet{chen2024learning} expressed that although she had ``low trust in ChatGPT'', she felt more confident after using it, as it helped her understand what she needed to figure out and made her in general faster. 

\paragraph{Impact on Processes and Outcomes.} This category refers to perceptions related to teaching and learning processes and their results. These perceptions were most commonly expressed by students and teachers, and include: Efficiency, Student motivation, Performance expectancy (a concept from UTAUT \cite{venkatesh2003user} and UTAUT2 \cite{acosta2024analysis}, see Appendix~\ref{app:models} for a summary of these theories), referring to improved student learning and outcomes, Cost-effectiveness, as well as Critical thinking and problem solving skills. Among these, Performance expectancy, Cost-effectiveness, and Critical thinking and problem solving skills generated mixed perceptions. No neutral perceptions were found regarding these impacts.

Efficiency was the most commonly cited positive perception among students, teachers, and industry professionals. For example, students in \citet{kazemitabaar2024codeaid} stated that AI ``helps them work more efficiently,'' while teachers in \citet{jansson2024initial} reported that ``not having to calculate the problem themselves may free up time and make them more efficient as coaches, enabling them to help more students.'' Similarly, industry professionals in \citet{chen2024learning} noted that ``with LLMs, human time and energy could be liberated for more high-level tasks.''

The most common negative perception across students and teachers was the impact on Critical thinking and problem solving skills. For example, a student in \citet{woithe2023understanding} noted that while ChatGPT ``is a tool that's very useful when learning,'' they were ``still a bit scared [that] I'll become incompetent to think on my own.'' Teachers in \citet{ghimire2024generative} also expressed concern, stating that LLMs ``decrease critical thinking.'' 

\begin{figure}[t!]
\begin{subfigure}[h]{0.48\linewidth}
    \includegraphics[width=\linewidth]{figs/students_perceptions.pdf}
    \caption{students}
    \label{fig:perceptions:students}
\end{subfigure}
\hfill 
\begin{subfigure}[h]{0.48\linewidth}
    \includegraphics[width=\linewidth]{figs/teachers_perceptions.pdf}
    \caption{teachers}
    \label{fig:perceptions:teachers}
\end{subfigure}
\caption{Summary of perceptions of two major stakeholders, students (\ref{fig:perceptions:students}) and teachers (\ref{fig:perceptions:teachers}), reported in the articles in our literature review.  Here, \emph{Impact} refers to ``Impacts on Processes and Outcomes,'' \emph{UX} refers to ``User Experience,'' and \emph{Ethics} refers to ``Ethical and Societal Implications.'' See Appendix~\ref{app:stats} for a detailed break-down. }
\label{fig:perceptions}
\end{figure}

\paragraph{Functions.} This category refers to perceptions related to uses of LLM-powered technology in the classroom, including Assessment, Personalized tutoring, Material creation, Writing assistance, Brainstorming assistance, Opportunity for practice, Tool for complementing traditional teaching, Facilitating conditions (a construct in UTAUT that describes the degree to which an individual believes an organization’s and technical infrastructure support the use of the system~\cite{venkatesh2003user}), as well as Other functions. Some of these perceptions were viewed both positively, negatively, and neutrally, except for Tool for complementing traditional teaching, which was only perceived neutrally.

The most recurring positive perception about functions across all stakeholders was Personalized tutoring. Students recognized ``the potential for personalized learning support'' \cite{chan2023students} and reported that ``ChatGPT excels in breaking down intricate concepts into more understandable terms'' \cite{zimotti2024future}. Teachers also emphasized the value of ``Personalized teaching/24 hours Teaching Assistant (TA) access'' \cite{ghimire2024generative}, while industry professionals underscored that LLMs ``support human effort'' \cite{chen2024learning}.

The most common negative perception regarding functions among students and teachers related to Facilitating conditions. For students, \citet{Ogbo-Gebhardt2024} highlighted a lack of institutional support on GenAI usage~\cite{chan2023students, Ogbo-Gebhardt2024}. For teachers, concerns included banning LLM-based tools \cite{lau2023ban} and the ``lack of teacher support for using ChatGPT as an educational tool'' \cite{iqbal2022exploring}.  Similarly, the most commonly cited neutral perspective among students involved Facilitating conditions. For example, students in \citet{woithe2023understanding} stated, ``[…] addressing these issues, not by banning ChatGPT, but by implementing regulatory measures and ethical guidelines for its use.'' Teachers, on the other hand, emphasized the need for support in implementing LLM-based bots in the classroom. As one teacher in \citet{iqbal2022exploring} noted, ``I feel like there should be more training available on how to use ChatGPT in the classroom. It would really help if we could get more help from the administration in this.''

\paragraph{User Experience (UX)} This category of perceptions relates to how individuals feel about interacting with LLM-powered bots. It includes concepts such as perceived Usefulness, Ease of use, Intention to use, Hedonic motivation, Well-being, Perceived importance, and Human interaction loss. Many of these—like  Usefulness, Ease of use, Intention to use, and Hedonic motivation—originated in the Technology Acceptance Model (TAM) \cite{davis1989perceived} and evolved in the Unified Theory of Acceptance and Use of Technology (UTAUT) \cite{venkatesh2003user} and UTAUT2 \cite{acosta2024analysis}, often with different names (see Appendices~\ref{app:models} and~\ref{app:scales}). For instance, perceived Usefulness in TAM is called Performance expectancy in UTAUT \cite{sair2018effect}, but we retained the term perceived Usefulness due to its use in studies beyond these frameworks. Similarly, TAM’s Ease of use evolved into UTAUT’s Effort expectancy, reflecting the role of technical skills in technology acceptance \cite{venkatesh2003user}. Except for Human interaction loss, which is perceived only negatively, and Hedonic motivation, viewed positively, negatively, and neutrally, all other concepts are perceived both positively and negatively.

The most commonly reported positive perception among students was perceived Usefulness. Findings in \citet{elkhodr2023ict} show ``a positive perception of ChatGPT as useful'', while \citet{lee2023learning} state that ``Learners’ perception of the AI Teaching Assistant (TA) is on par with that of human TAs in terms of [...] helpfulness.'' The most positive perception among teachers was Ease of use. For example, \citet{jansson2024initial} reports that ``The coaches expressed surprise about how friendly and supportive the AI bot was,'' while teachers in \citet{iqbal2022exploring} mentioned that ChatGPT is ``extremely user-friendly.'' Industry professionals reported positive perceptions of both Usefulness and Intention to use~\cite{chen2024learning}. 
 
The most common concern among both students and teachers is Human interaction loss. For example, a student in \citet{Ogbo-Gebhardt2024} disclosed that ``she felt a bit of anxiety considering how increased adoption of LLM tools could lead to increased social isolation,'' while \citet{chan2023students} reported that ``in academic institutions and education, some were concerned that the widespread use of AI might affect the student–teacher relationship, as students may be 'disappointed and lose respect for teachers.''' Teachers also expressed concerns about the ``lack of human connection'' \cite{mohamed2024exploring}. Interestingly, Ease of use was equally as concerning as Human interaction loss among teachers, with some mentioning that ``they found it difficult to use'' \cite{iqbal2022exploring} when discussing ChatGPT.

The only neutral UX perception in the literature is Hedonic motivation, a UTAUT2 concept referring to the fun or pleasure of using a technology. For instance, \citet{lieb2024student} observed that ``the neutral hedonic evaluation indicates that users had more neutral feelings about how interesting or exciting it was to use NewtBot.''

\paragraph{Ethical and Societal Implications.} This category covers the anticipated impact of introducing LLM-based bots in schools, including Social justice, Job replacement, Reliance, Ethical and privacy concerns, Academic integrity, Social influence (a construct from UTAUT that captures the impact of family and friends on technology acceptance), Workplace changes, and Learning changes. While all were viewed positively and negatively, the last four also had neutral perceptions.

The most frequently mentioned positive perception among students is Reliance on LLM-based bots. For instance, \citet{smith2024toward} noted that ``few students described behaviors that appeared to be intentional cheating (\ie 'generating complete answers'),'' demonstrating a healthy level of reliance on LLMs. Similarly, \citet{kazemitabaar2024codeaid} reported that ``although some students, like S14, felt that they 'over-relied on it too much rather than thinking,' many students displayed signs of self-regulation.'' This positive perception is also evident among industry professionals. A participant in \citet{chen2024learning} compared ChatGPT (GPT-4) and NetLogo, commenting that ChatGPT ``assumed what I wanted it to do, whereas this one makes you specify your assumptions.'' They preferred NetLogo Chat’s approach because ``it makes you think about the code more.'' For teachers, the most common positive perception was social justice, encompassing ``equity,'' ``accessibility,'' and ``cultural understanding.'' Teachers in \citet{mohamed2024exploring} noted that ChatGPT could become a ``tool for enhancing language exposure and cultural understanding […] and a tool for enhancing language accessibility and equity.'' Similarly, teachers in \citet{lau2023ban} stated that ``AI may improve equity and access.''

The most frequent concern among students was also related to Reliance, which, in this case, takes the shape of overreliance. This concern is reflected in statements such as, ``I feel that everyone might gradually lose their ability to think, as people won’t be willing to think anymore. They’ll just rely on machines to do the thinking for them'' \cite{zhang2024future}, or ``While students are specifically informed that AI can make mistakes and the exam keys are even provided, almost half of the students still agree with all the answers provided by ChatGPT, regardless of whether the answers are correct or incorrect'' \cite{ding2023students}. Similar concerns about Reliance were also shared by industry professionals.
The most common concern among teachers is Academic integrity, often framed as ``plagiarism'' or ``AI-assisted cheating.'' For example, teachers in \cite{ghimire2024generative} expressed concern about ``focus on product over process,'' while \citeauthor{iqbal2022exploring} report that most teachers surveyed ``found ChatGPT would be used by students for cheating, and it will make them lazy.''

Finally, the most common neutral perceptions regarding the Ethical and Societal Implications of LLM-powered tools were related to Changes in the workplace for students and both Changes in the workplace and Changes in learning for teachers. Regarding the former, \citeauthor{rogers2024attitudes} found that ``Almost all students agreed that ChatGPT will be integral to their careers,'' while \citeauthor{bernabei2023students} reported that ``students recognized the practical relevance of using such tools in a university context, as they believed it would better prepare them for the future working world, where AI is anticipated to play a prominent role.'' As for the latter, a teacher in \cite{mohamed2024exploring} stated, ``I see ChatGPT playing an increasingly central role in EFL education in the future.''

\subsubsection{\textbf{RQ2:} Gaps in the literature}
\label{sec:survey:findings:gaps}

\paragraph{Stakeholder representation.} Our literature review highlighted a lack of representation of stakeholders beyond students and teachers, such as parents and broader school staff. Figure~\ref{fig:stats:stakehoders} shows that students appeared as the most recurring stakeholders, followed by teachers. This finding mirrors a conclusion from a literature review on robot tutors by~\citet{smakman2020robot}, who also noted the lack of parental representation. It further underscores the need identified by \citet{louie2022designing} to include ``key stakeholders (students, parents, and teachers) as essential to ensure a culturally responsive robot.'' Finally, it echoes the concerns raised by \citet{zeide2019robot} that ``school procurement and implementation of these systems are rarely part of public discussion.''

\paragraph{Context representation.} The second gap highlighted by our literature review was the inadequate representation of the educational context, particularly in terms of discipline and educational levels of the students' sample. Figure~\ref{fig:stats:contexts} illustrates the contexts of application of LLM-based chatbots, with the most frequent ones being STEM, followed by studies with unspecified contexts. Additionally, a lack of diversity in educational levels could be observed in Figure~\ref{fig:stats:stakehoders}, where 27 out of the 31 studies involving students focusing on university-level students and only four studies involving K-12 students. This is a critical issue because, while university students can independently choose whether to use AI technologies, K-12 students typically do not have a say in the implementation of these technology within their curricula \cite{Singer.2023, School}. This underscores the need for further research on perceptions across a wider range of educational contexts.

\paragraph{Homogeneity in assessment of perceptions.} Finally, the wide variety of theories, scales, and terms used across the literature to assess and name perceptions highlights the need for standardization in these aspects (see Appendices~\ref{app:models} and~\ref{app:scales}). For example, in \citet{Ogbo-Gebhardt2024}, the theme of trust in and accuracy of the responses generated by LLM-based bots is considered a Facilitating condition (UTAUT), whereas in multiple other papers \cite{kazemitabaar2024codeaid, ding2023students, yilmaz2023augmented} it is treated as a separate concept. This lack of unified terminology makes it difficult to develop a clear understanding of how stakeholders perceive LLM-based technology. We advocate for the standardization of terms in the field.

\section{Proposed Framework} 
\label{sec:framework}

We propose a methodological framework, \frameworklong (\framework) (Figure \ref{fig:framework}), to guide policymakers, researchers, and developers in adopting AI-based chatbots more responsibly in educational contexts by incorporating stakeholder perceptions. \framework addresses systematic gaps in the literature while supporting the iterative refinement of its nested taxonomy of perceptions.

The creation of \framework is articulated around the perceptions of the different actors involved in the use of AI chatbots in education. Specifically, it is based on a \emph{stakeholder-first approach} that aims to support key agents in designing and implementing transparent, compliant, and accountable AI systems \cite{bell2023think}. The second component is the \emph{goals} that stakeholders have in mind when designing and/or implementing AI chatbots in educational settings. The \emph{context} component was also added to address the lack of specification of the educational settings in which LLM-based chatbots are used, as shown in the literature review. And finally, the \emph{taxonomy of perceptions} was added as a 'checklist' for policymakers, technologists, and researchers to consult when assessing the presence of specific perceptions in a given setting and is designed to be refined over iterations through future studies.  In the remainder of this section, we will describe each component in depth: \emph{stakeholders}, \emph{context}, \emph{goals}, and \emph{taxonomy of perceptions}, and explain how the framework should be used.

\subsection{Stakeholders} 
\label{sec:framework:stakeholders}

We highlighted how the introduction of LLM-based chatbots, such as Khanmigo, in the classroom could be seen as a top-down decision-making process, where partnerships between private companies and governmental institutions or school districts often make decisions on behalf of direct stakeholders, such as teachers and students, or indirect stakeholders, such as parents and school staff. This echoes a concern already highlighted in the literature regarding how private companies, more often than community members and public servants, set pedagogy and policy in practice \cite{zeide2019robot}. In contrast to this tendency, our framework prioritizes educational stakeholders: we identified students and teachers as the most recurring stakeholders in the literature; however, the systematic gaps we found underscore the need to assess the perceptions of other key stakeholders, such as parents, school staff, ed-tech professionals, and government agencies.

\subsection{Context} 
\label{sec:framework:context}

As pointed out in Section~\ref{sec:survey:findings:gaps}, most studies in the literature have been conducted in the field of higher education. This underscores the need to assess the perceptions of stakeholders at other educational levels, such as younger students who lack the autonomy to make independent decisions, yet are still expected to use LLM-based chatbots as part of their curricula. The  gaps also indicate that most studies focus on the STEM field, with many others having unspecified areas of application. We advocate for the need to assess perceptions of LLM-based chatbots across various disciplines.

\subsection{Goals} 
\label{sec:framework:goals}

The goals of soliciting and incorporating stakeholders' perceptions when developing and deploying AI systems are diverse. In our framework, we propose goals drawn from the literature on responsible design of AI systems and organize them into two focus areas: those focused on system behavior and those focused on educational practice.   When the focus is on system behavior, the goals are related to designing these tools or assessing whether they operate according to responsible AI key principles. We align with the existing literature on transparency, which asserts that the goal of a transparent design must begin with identifying the stakeholders \cite{bell2023think, chaudhry2022transparency}. \citet{bell2023think} established six categories to assist technologists in designing transparent, regulatory-compliant AI systems: validity, trust, learning and support, recourse, fairness, and privacy. \citet{chaudhry2022transparency} additionally emphasizes safety as a crucial component of transparency, especially in education, where mishaps in AI-powered technologies can often go unnoticed unless reported to the relevant authorities by a teacher or a student. Finally, although \citet{louie2022designing} discuss this in relation to social robots rather than LLM-based chatbots, they distance themselves from the traditional designer-determined approach, advocating for the inclusion of culturally and linguistically diverse stakeholders as co-designers and expert informants in the design process. They promote \emph{participatory design}, reporting that such inclusion leads to more engaging designs, improved user experience, and greater academic gains and ownership of technology-based learning. 

In terms of pedagogical goals, the focus is on embedding the responsible principles in educational practices and relating them to specific learning outcomes \cite{dominguez2020data, fu2024navigating}. For example, the in-depth analysis of \citet{jurenka2024towards} refers to the specific learning practices embedded in the LearnLM tutor that the authors evaluated to improve the pedagogical capabilities of the AI system while promoting more responsible and effective learning experiences, including: evaluative practices, feedback on procedural problems, grounding of learning materials, adaptivity to learner level, engagement, etc. In  \framework, we have presented some of the goals that could be associated with implementing responsible practices with AI systems in education, and this is an area that practitioners could expand according to their vision and needs.

\subsection{Taxonomy of perceptions} 
\label{sec:framework:taxonomy}

Our survey identified a gap in the inconsistent terminology used for similar concepts. To address this, our proposed taxonomy serves as a checklist for researchers, ed-tech professionals, and policymakers to assess, refine, and expand stakeholder perceptions, promoting greater consistency in perception research.

\subsection{Putting the \framework framework into practical use} 
\label{sec:framework:practice}

The \framework framework is designed to assist policy makers, researchers and technologists in two areas:

\begin{enumerate}
\item Elicit stakeholders' perceptions to create new or refine existing recommendations to ensure that AI systems are human-centered and responsibly adopted.
\item Guide researchers and practitioners to address gaps in the literature on LLM-based chatbot perceptions through future studies.
\end{enumerate}

In practice, the application of \framework follows the sequence shown in Figure~\ref{fig:framework}: starting with the first component, ``stakeholders,'' one should consider one or more of the bubbles before moving on to the next component.  For example, policymakers and technologists of an ed-tech company might want to assess the perceptions of the parents of elementary school students of LLM-based chatbots when they are applied to humanistic subjects with the goal of improving or assessing their perceived effectiveness. Upon reaching the final component, the taxonomy, one could use it as a checklist to identify the presence of specific perceptions. In this example, when discussing effectiveness, perceptions related to Impact on Processes or Outcomes are likely to arise. Another example involves assessing teachers' perceptions of LLM-based chatbots applied to English-as-a-Foreign-Language. The goals might include improving their experience in using this technology, as well as managing potential risks. When going through the taxonomy ``checklist,'' likely perceptions related to User Experience (UX) and Ethical and Societal Implications, such as cultural understanding (included in the ``Social justice'' perception), might arise. \framework is an open framework, and so any missing elements in any component that emerge from conducting studies in less explored scenarios could be added.  

 
\section{Conclusion, Limitations, and Future Work}
\label{sec:conclusion}

LLM-based chatbots are being piloted in schools, often through partnerships between private companies and school districts. However, feedback from direct users (students and teachers) and indirect users (parents and school staff) has been largely absent before and during rollout, aside from a few informal efforts.  To address this gap, we conducted a literature review on stakeholders' perceptions of LLM-based chatbots in education, identifying key positive, negative, and neutral views. Students generally perceive these chatbots as useful and efficient but raise concerns about accuracy. Teachers value their role as personalized tutors but are primarily concerned with maintaining academic integrity. Neutral perceptions from both groups describe chatbots as complementary to traditional teaching.  The review also revealed gaps in the literature: \first underrepresentation of key stakeholders, such as parents; \second limited exploration of diverse educational contexts, including various levels and disciplines; and \third inconsistent assessment of perceptions, with varying theories, scales, and terminology applied to similar concepts.  We then proposed the \frameworklong (\framework) framework, incorporating a taxonomy of perceptions from the literature. \framework is designed to help policymakers and technologists assess stakeholder perceptions throughout the development and deployment of AI systems in education, promoting more responsible and trustworthy processes.

\paragraph{Limitations} This article has limitations that must be considered when interpreting the findings. First, the organization of stakeholders' perceptions into the taxonomy, as well as the assignment of values (positive, negative, or neutral) to them, was based on the researchers’ judgment during the content analysis process. Future iterations could help make the taxonomy more rigorous. Second, the literature review was limited to articles published on Google Scholar. Including articles from databases such as Scopus, ISI Web of Science, and other cross-references could add weight and rigor to the findings related to stakeholders' perceptions of LLM-based chatbots in education. Lastly, the in-depth analysis of the literature was based on empirical studies conducted globally. Narrowing the scope to a particular region or tool could provide deeper insights, as perceptions may be influenced by cultural and social norms, socio-economic status, and other factors related to the context and background of the stakeholders.

\paragraph{Future work} There are several promising directions for future research. First, the stakeholder-first approach described here lays the foundation for future studies investigating neglected stakeholders' perceptions across a range of overlooked contexts. Second, a repository of findings derived from the application of this framework could be created, to act as a reference for technologists and policymakers in future technology design processes and implementations. 

\bibliographystyle{ACM-Reference-Format}
\bibliography{main}

\newpage

\appendix
\subsection{Lloyd-Max Algorithm}
\label{subsec:Lloyd-Max}
For a given quantization bitwidth $B$ and an operand $\bm{X}$, the Lloyd-Max algorithm finds $2^B$ quantization levels $\{\hat{x}_i\}_{i=1}^{2^B}$ such that quantizing $\bm{X}$ by rounding each scalar in $\bm{X}$ to the nearest quantization level minimizes the quantization MSE. 

The algorithm starts with an initial guess of quantization levels and then iteratively computes quantization thresholds $\{\tau_i\}_{i=1}^{2^B-1}$ and updates quantization levels $\{\hat{x}_i\}_{i=1}^{2^B}$. Specifically, at iteration $n$, thresholds are set to the midpoints of the previous iteration's levels:
\begin{align*}
    \tau_i^{(n)}=\frac{\hat{x}_i^{(n-1)}+\hat{x}_{i+1}^{(n-1)}}2 \text{ for } i=1\ldots 2^B-1
\end{align*}
Subsequently, the quantization levels are re-computed as conditional means of the data regions defined by the new thresholds:
\begin{align*}
    \hat{x}_i^{(n)}=\mathbb{E}\left[ \bm{X} \big| \bm{X}\in [\tau_{i-1}^{(n)},\tau_i^{(n)}] \right] \text{ for } i=1\ldots 2^B
\end{align*}
where to satisfy boundary conditions we have $\tau_0=-\infty$ and $\tau_{2^B}=\infty$. The algorithm iterates the above steps until convergence.

Figure \ref{fig:lm_quant} compares the quantization levels of a $7$-bit floating point (E3M3) quantizer (left) to a $7$-bit Lloyd-Max quantizer (right) when quantizing a layer of weights from the GPT3-126M model at a per-tensor granularity. As shown, the Lloyd-Max quantizer achieves substantially lower quantization MSE. Further, Table \ref{tab:FP7_vs_LM7} shows the superior perplexity achieved by Lloyd-Max quantizers for bitwidths of $7$, $6$ and $5$. The difference between the quantizers is clear at 5 bits, where per-tensor FP quantization incurs a drastic and unacceptable increase in perplexity, while Lloyd-Max quantization incurs a much smaller increase. Nevertheless, we note that even the optimal Lloyd-Max quantizer incurs a notable ($\sim 1.5$) increase in perplexity due to the coarse granularity of quantization. 

\begin{figure}[h]
  \centering
  \includegraphics[width=0.7\linewidth]{sections/figures/LM7_FP7.pdf}
  \caption{\small Quantization levels and the corresponding quantization MSE of Floating Point (left) vs Lloyd-Max (right) Quantizers for a layer of weights in the GPT3-126M model.}
  \label{fig:lm_quant}
\end{figure}

\begin{table}[h]\scriptsize
\begin{center}
\caption{\label{tab:FP7_vs_LM7} \small Comparing perplexity (lower is better) achieved by floating point quantizers and Lloyd-Max quantizers on a GPT3-126M model for the Wikitext-103 dataset.}
\begin{tabular}{c|cc|c}
\hline
 \multirow{2}{*}{\textbf{Bitwidth}} & \multicolumn{2}{|c|}{\textbf{Floating-Point Quantizer}} & \textbf{Lloyd-Max Quantizer} \\
 & Best Format & Wikitext-103 Perplexity & Wikitext-103 Perplexity \\
\hline
7 & E3M3 & 18.32 & 18.27 \\
6 & E3M2 & 19.07 & 18.51 \\
5 & E4M0 & 43.89 & 19.71 \\
\hline
\end{tabular}
\end{center}
\end{table}

\subsection{Proof of Local Optimality of LO-BCQ}
\label{subsec:lobcq_opt_proof}
For a given block $\bm{b}_j$, the quantization MSE during LO-BCQ can be empirically evaluated as $\frac{1}{L_b}\lVert \bm{b}_j- \bm{\hat{b}}_j\rVert^2_2$ where $\bm{\hat{b}}_j$ is computed from equation (\ref{eq:clustered_quantization_definition}) as $C_{f(\bm{b}_j)}(\bm{b}_j)$. Further, for a given block cluster $\mathcal{B}_i$, we compute the quantization MSE as $\frac{1}{|\mathcal{B}_{i}|}\sum_{\bm{b} \in \mathcal{B}_{i}} \frac{1}{L_b}\lVert \bm{b}- C_i^{(n)}(\bm{b})\rVert^2_2$. Therefore, at the end of iteration $n$, we evaluate the overall quantization MSE $J^{(n)}$ for a given operand $\bm{X}$ composed of $N_c$ block clusters as:
\begin{align*}
    \label{eq:mse_iter_n}
    J^{(n)} = \frac{1}{N_c} \sum_{i=1}^{N_c} \frac{1}{|\mathcal{B}_{i}^{(n)}|}\sum_{\bm{v} \in \mathcal{B}_{i}^{(n)}} \frac{1}{L_b}\lVert \bm{b}- B_i^{(n)}(\bm{b})\rVert^2_2
\end{align*}

At the end of iteration $n$, the codebooks are updated from $\mathcal{C}^{(n-1)}$ to $\mathcal{C}^{(n)}$. However, the mapping of a given vector $\bm{b}_j$ to quantizers $\mathcal{C}^{(n)}$ remains as  $f^{(n)}(\bm{b}_j)$. At the next iteration, during the vector clustering step, $f^{(n+1)}(\bm{b}_j)$ finds new mapping of $\bm{b}_j$ to updated codebooks $\mathcal{C}^{(n)}$ such that the quantization MSE over the candidate codebooks is minimized. Therefore, we obtain the following result for $\bm{b}_j$:
\begin{align*}
\frac{1}{L_b}\lVert \bm{b}_j - C_{f^{(n+1)}(\bm{b}_j)}^{(n)}(\bm{b}_j)\rVert^2_2 \le \frac{1}{L_b}\lVert \bm{b}_j - C_{f^{(n)}(\bm{b}_j)}^{(n)}(\bm{b}_j)\rVert^2_2
\end{align*}

That is, quantizing $\bm{b}_j$ at the end of the block clustering step of iteration $n+1$ results in lower quantization MSE compared to quantizing at the end of iteration $n$. Since this is true for all $\bm{b} \in \bm{X}$, we assert the following:
\begin{equation}
\begin{split}
\label{eq:mse_ineq_1}
    \tilde{J}^{(n+1)} &= \frac{1}{N_c} \sum_{i=1}^{N_c} \frac{1}{|\mathcal{B}_{i}^{(n+1)}|}\sum_{\bm{b} \in \mathcal{B}_{i}^{(n+1)}} \frac{1}{L_b}\lVert \bm{b} - C_i^{(n)}(b)\rVert^2_2 \le J^{(n)}
\end{split}
\end{equation}
where $\tilde{J}^{(n+1)}$ is the the quantization MSE after the vector clustering step at iteration $n+1$.

Next, during the codebook update step (\ref{eq:quantizers_update}) at iteration $n+1$, the per-cluster codebooks $\mathcal{C}^{(n)}$ are updated to $\mathcal{C}^{(n+1)}$ by invoking the Lloyd-Max algorithm \citep{Lloyd}. We know that for any given value distribution, the Lloyd-Max algorithm minimizes the quantization MSE. Therefore, for a given vector cluster $\mathcal{B}_i$ we obtain the following result:

\begin{equation}
    \frac{1}{|\mathcal{B}_{i}^{(n+1)}|}\sum_{\bm{b} \in \mathcal{B}_{i}^{(n+1)}} \frac{1}{L_b}\lVert \bm{b}- C_i^{(n+1)}(\bm{b})\rVert^2_2 \le \frac{1}{|\mathcal{B}_{i}^{(n+1)}|}\sum_{\bm{b} \in \mathcal{B}_{i}^{(n+1)}} \frac{1}{L_b}\lVert \bm{b}- C_i^{(n)}(\bm{b})\rVert^2_2
\end{equation}

The above equation states that quantizing the given block cluster $\mathcal{B}_i$ after updating the associated codebook from $C_i^{(n)}$ to $C_i^{(n+1)}$ results in lower quantization MSE. Since this is true for all the block clusters, we derive the following result: 
\begin{equation}
\begin{split}
\label{eq:mse_ineq_2}
     J^{(n+1)} &= \frac{1}{N_c} \sum_{i=1}^{N_c} \frac{1}{|\mathcal{B}_{i}^{(n+1)}|}\sum_{\bm{b} \in \mathcal{B}_{i}^{(n+1)}} \frac{1}{L_b}\lVert \bm{b}- C_i^{(n+1)}(\bm{b})\rVert^2_2  \le \tilde{J}^{(n+1)}   
\end{split}
\end{equation}

Following (\ref{eq:mse_ineq_1}) and (\ref{eq:mse_ineq_2}), we find that the quantization MSE is non-increasing for each iteration, that is, $J^{(1)} \ge J^{(2)} \ge J^{(3)} \ge \ldots \ge J^{(M)}$ where $M$ is the maximum number of iterations. 
%Therefore, we can say that if the algorithm converges, then it must be that it has converged to a local minimum. 
\hfill $\blacksquare$


\begin{figure}
    \begin{center}
    \includegraphics[width=0.5\textwidth]{sections//figures/mse_vs_iter.pdf}
    \end{center}
    \caption{\small NMSE vs iterations during LO-BCQ compared to other block quantization proposals}
    \label{fig:nmse_vs_iter}
\end{figure}

Figure \ref{fig:nmse_vs_iter} shows the empirical convergence of LO-BCQ across several block lengths and number of codebooks. Also, the MSE achieved by LO-BCQ is compared to baselines such as MXFP and VSQ. As shown, LO-BCQ converges to a lower MSE than the baselines. Further, we achieve better convergence for larger number of codebooks ($N_c$) and for a smaller block length ($L_b$), both of which increase the bitwidth of BCQ (see Eq \ref{eq:bitwidth_bcq}).


\subsection{Additional Accuracy Results}
%Table \ref{tab:lobcq_config} lists the various LOBCQ configurations and their corresponding bitwidths.
\begin{table}
\setlength{\tabcolsep}{4.75pt}
\begin{center}
\caption{\label{tab:lobcq_config} Various LO-BCQ configurations and their bitwidths.}
\begin{tabular}{|c||c|c|c|c||c|c||c|} 
\hline
 & \multicolumn{4}{|c||}{$L_b=8$} & \multicolumn{2}{|c||}{$L_b=4$} & $L_b=2$ \\
 \hline
 \backslashbox{$L_A$\kern-1em}{\kern-1em$N_c$} & 2 & 4 & 8 & 16 & 2 & 4 & 2 \\
 \hline
 64 & 4.25 & 4.375 & 4.5 & 4.625 & 4.375 & 4.625 & 4.625\\
 \hline
 32 & 4.375 & 4.5 & 4.625& 4.75 & 4.5 & 4.75 & 4.75 \\
 \hline
 16 & 4.625 & 4.75& 4.875 & 5 & 4.75 & 5 & 5 \\
 \hline
\end{tabular}
\end{center}
\end{table}

%\subsection{Perplexity achieved by various LO-BCQ configurations on Wikitext-103 dataset}

\begin{table} \centering
\begin{tabular}{|c||c|c|c|c||c|c||c|} 
\hline
 $L_b \rightarrow$& \multicolumn{4}{c||}{8} & \multicolumn{2}{c||}{4} & 2\\
 \hline
 \backslashbox{$L_A$\kern-1em}{\kern-1em$N_c$} & 2 & 4 & 8 & 16 & 2 & 4 & 2  \\
 %$N_c \rightarrow$ & 2 & 4 & 8 & 16 & 2 & 4 & 2 \\
 \hline
 \hline
 \multicolumn{8}{c}{GPT3-1.3B (FP32 PPL = 9.98)} \\ 
 \hline
 \hline
 64 & 10.40 & 10.23 & 10.17 & 10.15 &  10.28 & 10.18 & 10.19 \\
 \hline
 32 & 10.25 & 10.20 & 10.15 & 10.12 &  10.23 & 10.17 & 10.17 \\
 \hline
 16 & 10.22 & 10.16 & 10.10 & 10.09 &  10.21 & 10.14 & 10.16 \\
 \hline
  \hline
 \multicolumn{8}{c}{GPT3-8B (FP32 PPL = 7.38)} \\ 
 \hline
 \hline
 64 & 7.61 & 7.52 & 7.48 &  7.47 &  7.55 &  7.49 & 7.50 \\
 \hline
 32 & 7.52 & 7.50 & 7.46 &  7.45 &  7.52 &  7.48 & 7.48  \\
 \hline
 16 & 7.51 & 7.48 & 7.44 &  7.44 &  7.51 &  7.49 & 7.47  \\
 \hline
\end{tabular}
\caption{\label{tab:ppl_gpt3_abalation} Wikitext-103 perplexity across GPT3-1.3B and 8B models.}
\end{table}

\begin{table} \centering
\begin{tabular}{|c||c|c|c|c||} 
\hline
 $L_b \rightarrow$& \multicolumn{4}{c||}{8}\\
 \hline
 \backslashbox{$L_A$\kern-1em}{\kern-1em$N_c$} & 2 & 4 & 8 & 16 \\
 %$N_c \rightarrow$ & 2 & 4 & 8 & 16 & 2 & 4 & 2 \\
 \hline
 \hline
 \multicolumn{5}{|c|}{Llama2-7B (FP32 PPL = 5.06)} \\ 
 \hline
 \hline
 64 & 5.31 & 5.26 & 5.19 & 5.18  \\
 \hline
 32 & 5.23 & 5.25 & 5.18 & 5.15  \\
 \hline
 16 & 5.23 & 5.19 & 5.16 & 5.14  \\
 \hline
 \multicolumn{5}{|c|}{Nemotron4-15B (FP32 PPL = 5.87)} \\ 
 \hline
 \hline
 64  & 6.3 & 6.20 & 6.13 & 6.08  \\
 \hline
 32  & 6.24 & 6.12 & 6.07 & 6.03  \\
 \hline
 16  & 6.12 & 6.14 & 6.04 & 6.02  \\
 \hline
 \multicolumn{5}{|c|}{Nemotron4-340B (FP32 PPL = 3.48)} \\ 
 \hline
 \hline
 64 & 3.67 & 3.62 & 3.60 & 3.59 \\
 \hline
 32 & 3.63 & 3.61 & 3.59 & 3.56 \\
 \hline
 16 & 3.61 & 3.58 & 3.57 & 3.55 \\
 \hline
\end{tabular}
\caption{\label{tab:ppl_llama7B_nemo15B} Wikitext-103 perplexity compared to FP32 baseline in Llama2-7B and Nemotron4-15B, 340B models}
\end{table}

%\subsection{Perplexity achieved by various LO-BCQ configurations on MMLU dataset}


\begin{table} \centering
\begin{tabular}{|c||c|c|c|c||c|c|c|c|} 
\hline
 $L_b \rightarrow$& \multicolumn{4}{c||}{8} & \multicolumn{4}{c||}{8}\\
 \hline
 \backslashbox{$L_A$\kern-1em}{\kern-1em$N_c$} & 2 & 4 & 8 & 16 & 2 & 4 & 8 & 16  \\
 %$N_c \rightarrow$ & 2 & 4 & 8 & 16 & 2 & 4 & 2 \\
 \hline
 \hline
 \multicolumn{5}{|c|}{Llama2-7B (FP32 Accuracy = 45.8\%)} & \multicolumn{4}{|c|}{Llama2-70B (FP32 Accuracy = 69.12\%)} \\ 
 \hline
 \hline
 64 & 43.9 & 43.4 & 43.9 & 44.9 & 68.07 & 68.27 & 68.17 & 68.75 \\
 \hline
 32 & 44.5 & 43.8 & 44.9 & 44.5 & 68.37 & 68.51 & 68.35 & 68.27  \\
 \hline
 16 & 43.9 & 42.7 & 44.9 & 45 & 68.12 & 68.77 & 68.31 & 68.59  \\
 \hline
 \hline
 \multicolumn{5}{|c|}{GPT3-22B (FP32 Accuracy = 38.75\%)} & \multicolumn{4}{|c|}{Nemotron4-15B (FP32 Accuracy = 64.3\%)} \\ 
 \hline
 \hline
 64 & 36.71 & 38.85 & 38.13 & 38.92 & 63.17 & 62.36 & 63.72 & 64.09 \\
 \hline
 32 & 37.95 & 38.69 & 39.45 & 38.34 & 64.05 & 62.30 & 63.8 & 64.33  \\
 \hline
 16 & 38.88 & 38.80 & 38.31 & 38.92 & 63.22 & 63.51 & 63.93 & 64.43  \\
 \hline
\end{tabular}
\caption{\label{tab:mmlu_abalation} Accuracy on MMLU dataset across GPT3-22B, Llama2-7B, 70B and Nemotron4-15B models.}
\end{table}


%\subsection{Perplexity achieved by various LO-BCQ configurations on LM evaluation harness}

\begin{table} \centering
\begin{tabular}{|c||c|c|c|c||c|c|c|c|} 
\hline
 $L_b \rightarrow$& \multicolumn{4}{c||}{8} & \multicolumn{4}{c||}{8}\\
 \hline
 \backslashbox{$L_A$\kern-1em}{\kern-1em$N_c$} & 2 & 4 & 8 & 16 & 2 & 4 & 8 & 16  \\
 %$N_c \rightarrow$ & 2 & 4 & 8 & 16 & 2 & 4 & 2 \\
 \hline
 \hline
 \multicolumn{5}{|c|}{Race (FP32 Accuracy = 37.51\%)} & \multicolumn{4}{|c|}{Boolq (FP32 Accuracy = 64.62\%)} \\ 
 \hline
 \hline
 64 & 36.94 & 37.13 & 36.27 & 37.13 & 63.73 & 62.26 & 63.49 & 63.36 \\
 \hline
 32 & 37.03 & 36.36 & 36.08 & 37.03 & 62.54 & 63.51 & 63.49 & 63.55  \\
 \hline
 16 & 37.03 & 37.03 & 36.46 & 37.03 & 61.1 & 63.79 & 63.58 & 63.33  \\
 \hline
 \hline
 \multicolumn{5}{|c|}{Winogrande (FP32 Accuracy = 58.01\%)} & \multicolumn{4}{|c|}{Piqa (FP32 Accuracy = 74.21\%)} \\ 
 \hline
 \hline
 64 & 58.17 & 57.22 & 57.85 & 58.33 & 73.01 & 73.07 & 73.07 & 72.80 \\
 \hline
 32 & 59.12 & 58.09 & 57.85 & 58.41 & 73.01 & 73.94 & 72.74 & 73.18  \\
 \hline
 16 & 57.93 & 58.88 & 57.93 & 58.56 & 73.94 & 72.80 & 73.01 & 73.94  \\
 \hline
\end{tabular}
\caption{\label{tab:mmlu_abalation} Accuracy on LM evaluation harness tasks on GPT3-1.3B model.}
\end{table}

\begin{table} \centering
\begin{tabular}{|c||c|c|c|c||c|c|c|c|} 
\hline
 $L_b \rightarrow$& \multicolumn{4}{c||}{8} & \multicolumn{4}{c||}{8}\\
 \hline
 \backslashbox{$L_A$\kern-1em}{\kern-1em$N_c$} & 2 & 4 & 8 & 16 & 2 & 4 & 8 & 16  \\
 %$N_c \rightarrow$ & 2 & 4 & 8 & 16 & 2 & 4 & 2 \\
 \hline
 \hline
 \multicolumn{5}{|c|}{Race (FP32 Accuracy = 41.34\%)} & \multicolumn{4}{|c|}{Boolq (FP32 Accuracy = 68.32\%)} \\ 
 \hline
 \hline
 64 & 40.48 & 40.10 & 39.43 & 39.90 & 69.20 & 68.41 & 69.45 & 68.56 \\
 \hline
 32 & 39.52 & 39.52 & 40.77 & 39.62 & 68.32 & 67.43 & 68.17 & 69.30  \\
 \hline
 16 & 39.81 & 39.71 & 39.90 & 40.38 & 68.10 & 66.33 & 69.51 & 69.42  \\
 \hline
 \hline
 \multicolumn{5}{|c|}{Winogrande (FP32 Accuracy = 67.88\%)} & \multicolumn{4}{|c|}{Piqa (FP32 Accuracy = 78.78\%)} \\ 
 \hline
 \hline
 64 & 66.85 & 66.61 & 67.72 & 67.88 & 77.31 & 77.42 & 77.75 & 77.64 \\
 \hline
 32 & 67.25 & 67.72 & 67.72 & 67.00 & 77.31 & 77.04 & 77.80 & 77.37  \\
 \hline
 16 & 68.11 & 68.90 & 67.88 & 67.48 & 77.37 & 78.13 & 78.13 & 77.69  \\
 \hline
\end{tabular}
\caption{\label{tab:mmlu_abalation} Accuracy on LM evaluation harness tasks on GPT3-8B model.}
\end{table}

\begin{table} \centering
\begin{tabular}{|c||c|c|c|c||c|c|c|c|} 
\hline
 $L_b \rightarrow$& \multicolumn{4}{c||}{8} & \multicolumn{4}{c||}{8}\\
 \hline
 \backslashbox{$L_A$\kern-1em}{\kern-1em$N_c$} & 2 & 4 & 8 & 16 & 2 & 4 & 8 & 16  \\
 %$N_c \rightarrow$ & 2 & 4 & 8 & 16 & 2 & 4 & 2 \\
 \hline
 \hline
 \multicolumn{5}{|c|}{Race (FP32 Accuracy = 40.67\%)} & \multicolumn{4}{|c|}{Boolq (FP32 Accuracy = 76.54\%)} \\ 
 \hline
 \hline
 64 & 40.48 & 40.10 & 39.43 & 39.90 & 75.41 & 75.11 & 77.09 & 75.66 \\
 \hline
 32 & 39.52 & 39.52 & 40.77 & 39.62 & 76.02 & 76.02 & 75.96 & 75.35  \\
 \hline
 16 & 39.81 & 39.71 & 39.90 & 40.38 & 75.05 & 73.82 & 75.72 & 76.09  \\
 \hline
 \hline
 \multicolumn{5}{|c|}{Winogrande (FP32 Accuracy = 70.64\%)} & \multicolumn{4}{|c|}{Piqa (FP32 Accuracy = 79.16\%)} \\ 
 \hline
 \hline
 64 & 69.14 & 70.17 & 70.17 & 70.56 & 78.24 & 79.00 & 78.62 & 78.73 \\
 \hline
 32 & 70.96 & 69.69 & 71.27 & 69.30 & 78.56 & 79.49 & 79.16 & 78.89  \\
 \hline
 16 & 71.03 & 69.53 & 69.69 & 70.40 & 78.13 & 79.16 & 79.00 & 79.00  \\
 \hline
\end{tabular}
\caption{\label{tab:mmlu_abalation} Accuracy on LM evaluation harness tasks on GPT3-22B model.}
\end{table}

\begin{table} \centering
\begin{tabular}{|c||c|c|c|c||c|c|c|c|} 
\hline
 $L_b \rightarrow$& \multicolumn{4}{c||}{8} & \multicolumn{4}{c||}{8}\\
 \hline
 \backslashbox{$L_A$\kern-1em}{\kern-1em$N_c$} & 2 & 4 & 8 & 16 & 2 & 4 & 8 & 16  \\
 %$N_c \rightarrow$ & 2 & 4 & 8 & 16 & 2 & 4 & 2 \\
 \hline
 \hline
 \multicolumn{5}{|c|}{Race (FP32 Accuracy = 44.4\%)} & \multicolumn{4}{|c|}{Boolq (FP32 Accuracy = 79.29\%)} \\ 
 \hline
 \hline
 64 & 42.49 & 42.51 & 42.58 & 43.45 & 77.58 & 77.37 & 77.43 & 78.1 \\
 \hline
 32 & 43.35 & 42.49 & 43.64 & 43.73 & 77.86 & 75.32 & 77.28 & 77.86  \\
 \hline
 16 & 44.21 & 44.21 & 43.64 & 42.97 & 78.65 & 77 & 76.94 & 77.98  \\
 \hline
 \hline
 \multicolumn{5}{|c|}{Winogrande (FP32 Accuracy = 69.38\%)} & \multicolumn{4}{|c|}{Piqa (FP32 Accuracy = 78.07\%)} \\ 
 \hline
 \hline
 64 & 68.9 & 68.43 & 69.77 & 68.19 & 77.09 & 76.82 & 77.09 & 77.86 \\
 \hline
 32 & 69.38 & 68.51 & 68.82 & 68.90 & 78.07 & 76.71 & 78.07 & 77.86  \\
 \hline
 16 & 69.53 & 67.09 & 69.38 & 68.90 & 77.37 & 77.8 & 77.91 & 77.69  \\
 \hline
\end{tabular}
\caption{\label{tab:mmlu_abalation} Accuracy on LM evaluation harness tasks on Llama2-7B model.}
\end{table}

\begin{table} \centering
\begin{tabular}{|c||c|c|c|c||c|c|c|c|} 
\hline
 $L_b \rightarrow$& \multicolumn{4}{c||}{8} & \multicolumn{4}{c||}{8}\\
 \hline
 \backslashbox{$L_A$\kern-1em}{\kern-1em$N_c$} & 2 & 4 & 8 & 16 & 2 & 4 & 8 & 16  \\
 %$N_c \rightarrow$ & 2 & 4 & 8 & 16 & 2 & 4 & 2 \\
 \hline
 \hline
 \multicolumn{5}{|c|}{Race (FP32 Accuracy = 48.8\%)} & \multicolumn{4}{|c|}{Boolq (FP32 Accuracy = 85.23\%)} \\ 
 \hline
 \hline
 64 & 49.00 & 49.00 & 49.28 & 48.71 & 82.82 & 84.28 & 84.03 & 84.25 \\
 \hline
 32 & 49.57 & 48.52 & 48.33 & 49.28 & 83.85 & 84.46 & 84.31 & 84.93  \\
 \hline
 16 & 49.85 & 49.09 & 49.28 & 48.99 & 85.11 & 84.46 & 84.61 & 83.94  \\
 \hline
 \hline
 \multicolumn{5}{|c|}{Winogrande (FP32 Accuracy = 79.95\%)} & \multicolumn{4}{|c|}{Piqa (FP32 Accuracy = 81.56\%)} \\ 
 \hline
 \hline
 64 & 78.77 & 78.45 & 78.37 & 79.16 & 81.45 & 80.69 & 81.45 & 81.5 \\
 \hline
 32 & 78.45 & 79.01 & 78.69 & 80.66 & 81.56 & 80.58 & 81.18 & 81.34  \\
 \hline
 16 & 79.95 & 79.56 & 79.79 & 79.72 & 81.28 & 81.66 & 81.28 & 80.96  \\
 \hline
\end{tabular}
\caption{\label{tab:mmlu_abalation} Accuracy on LM evaluation harness tasks on Llama2-70B model.}
\end{table}

%\section{MSE Studies}
%\textcolor{red}{TODO}


\subsection{Number Formats and Quantization Method}
\label{subsec:numFormats_quantMethod}
\subsubsection{Integer Format}
An $n$-bit signed integer (INT) is typically represented with a 2s-complement format \citep{yao2022zeroquant,xiao2023smoothquant,dai2021vsq}, where the most significant bit denotes the sign.

\subsubsection{Floating Point Format}
An $n$-bit signed floating point (FP) number $x$ comprises of a 1-bit sign ($x_{\mathrm{sign}}$), $B_m$-bit mantissa ($x_{\mathrm{mant}}$) and $B_e$-bit exponent ($x_{\mathrm{exp}}$) such that $B_m+B_e=n-1$. The associated constant exponent bias ($E_{\mathrm{bias}}$) is computed as $(2^{{B_e}-1}-1)$. We denote this format as $E_{B_e}M_{B_m}$.  

\subsubsection{Quantization Scheme}
\label{subsec:quant_method}
A quantization scheme dictates how a given unquantized tensor is converted to its quantized representation. We consider FP formats for the purpose of illustration. Given an unquantized tensor $\bm{X}$ and an FP format $E_{B_e}M_{B_m}$, we first, we compute the quantization scale factor $s_X$ that maps the maximum absolute value of $\bm{X}$ to the maximum quantization level of the $E_{B_e}M_{B_m}$ format as follows:
\begin{align}
\label{eq:sf}
    s_X = \frac{\mathrm{max}(|\bm{X}|)}{\mathrm{max}(E_{B_e}M_{B_m})}
\end{align}
In the above equation, $|\cdot|$ denotes the absolute value function.

Next, we scale $\bm{X}$ by $s_X$ and quantize it to $\hat{\bm{X}}$ by rounding it to the nearest quantization level of $E_{B_e}M_{B_m}$ as:

\begin{align}
\label{eq:tensor_quant}
    \hat{\bm{X}} = \text{round-to-nearest}\left(\frac{\bm{X}}{s_X}, E_{B_e}M_{B_m}\right)
\end{align}

We perform dynamic max-scaled quantization \citep{wu2020integer}, where the scale factor $s$ for activations is dynamically computed during runtime.

\subsection{Vector Scaled Quantization}
\begin{wrapfigure}{r}{0.35\linewidth}
  \centering
  \includegraphics[width=\linewidth]{sections/figures/vsquant.jpg}
  \caption{\small Vectorwise decomposition for per-vector scaled quantization (VSQ \citep{dai2021vsq}).}
  \label{fig:vsquant}
\end{wrapfigure}
During VSQ \citep{dai2021vsq}, the operand tensors are decomposed into 1D vectors in a hardware friendly manner as shown in Figure \ref{fig:vsquant}. Since the decomposed tensors are used as operands in matrix multiplications during inference, it is beneficial to perform this decomposition along the reduction dimension of the multiplication. The vectorwise quantization is performed similar to tensorwise quantization described in Equations \ref{eq:sf} and \ref{eq:tensor_quant}, where a scale factor $s_v$ is required for each vector $\bm{v}$ that maps the maximum absolute value of that vector to the maximum quantization level. While smaller vector lengths can lead to larger accuracy gains, the associated memory and computational overheads due to the per-vector scale factors increases. To alleviate these overheads, VSQ \citep{dai2021vsq} proposed a second level quantization of the per-vector scale factors to unsigned integers, while MX \citep{rouhani2023shared} quantizes them to integer powers of 2 (denoted as $2^{INT}$).

\subsubsection{MX Format}
The MX format proposed in \citep{rouhani2023microscaling} introduces the concept of sub-block shifting. For every two scalar elements of $b$-bits each, there is a shared exponent bit. The value of this exponent bit is determined through an empirical analysis that targets minimizing quantization MSE. We note that the FP format $E_{1}M_{b}$ is strictly better than MX from an accuracy perspective since it allocates a dedicated exponent bit to each scalar as opposed to sharing it across two scalars. Therefore, we conservatively bound the accuracy of a $b+2$-bit signed MX format with that of a $E_{1}M_{b}$ format in our comparisons. For instance, we use E1M2 format as a proxy for MX4.

\begin{figure}
    \centering
    \includegraphics[width=1\linewidth]{sections//figures/BlockFormats.pdf}
    \caption{\small Comparing LO-BCQ to MX format.}
    \label{fig:block_formats}
\end{figure}

Figure \ref{fig:block_formats} compares our $4$-bit LO-BCQ block format to MX \citep{rouhani2023microscaling}. As shown, both LO-BCQ and MX decompose a given operand tensor into block arrays and each block array into blocks. Similar to MX, we find that per-block quantization ($L_b < L_A$) leads to better accuracy due to increased flexibility. While MX achieves this through per-block $1$-bit micro-scales, we associate a dedicated codebook to each block through a per-block codebook selector. Further, MX quantizes the per-block array scale-factor to E8M0 format without per-tensor scaling. In contrast during LO-BCQ, we find that per-tensor scaling combined with quantization of per-block array scale-factor to E4M3 format results in superior inference accuracy across models. 


\end{document}
