\section{Supplementary statistics}
\label{app:stats}

Complete break-down of perceptions, for all perception types in our taxonomy, for two main stakeholder groups: students and teachers. These results complement the summary in Figure~\ref{fig:perceptions} in the main body of the paper.

\includegraphics[width=0.8\linewidth]{figs/fig_all_perceptions.pdf}

\section{Models of Perception}
\label{app:models}

Complete breakdown of models and theories conceptualizing attitudes, beliefs, and/or behaviors toward technology, as they appear in the literature, to investigate stakeholders' perceptions of LLM-based chatbots. The breakdown supports the thesis outlined in the third gap identified in the literature in Section~\ref{sec:survey:findings:gaps}, namely, the lack of homogeneity in the assessment of perceptions and the need for standardization in how perceptions are conceptualized.

\textbf{Technology Acceptance Model (TAM)} The Technology Acceptance Model (TAM) was first developed
in 1989 by \citeauthor{davis1989perceived} and \citeauthor{davis1989user}) as a tool to predict the likelihood of adopting a new technology within a group or an organization. Originating in the psychological theory
of reasoned action and theory of planned behavior (TPB) \cite{kowalska2023diffusion}, TAM suggests that an individual's beliefs, attitudes, and intentions can explain their adoption and use of technology. TAM evaluates the influence of four internal variables on actual technology use, namely, Perceived Ease of Use (PEU); Perceived Usefulness (PU); Attitude Toward Use (ATU), and
Behavioral Intention to Use (BI) \cite{davis1989perceived, turner2010does, kowalska2023diffusion}.

\textbf{Unified Theory of Acceptance and Use of Technology (UTAUT)} The Unified Theory of Acceptance and Use of Technology (UTAUT) originates as an extention of TAM and was originally used to explain user behavior and intentions in information system contexts \cite{venkatesh2012consumer, kowalska2023diffusion, shakib2023lecturers}. The UTAUT model comprises several key constructs that affect
users’ acceptance and use of technology. These are: Performance Expectancy, Effort Expectancy, Social Influence and Facilitating Conditions. According to \citeauthor{venkatesh2003user}, the Effort Expectancy factor  derives from the Ease of Use factor included in the Technology Acceptance Model (TAM). Additionally, the UTAUT model includes four other moderating variables, i.e., Gender, Age, Experience, and Voluntarism of Use, that may influence the four main constructs \cite{venkatesh2012consumer, shakib2023lecturers}.

\textbf{Extended Unified Theory of Acceptance and Use of Technology (UTAUT2)} The Extended Unified Theory of Acceptance and Use of Technology (UTAUT2) is an extension of the prevision one, as it incorporates additional constructs such as hedonic motivation, price value, habit, and usage intention, besides the original ones \cite{acosta2024analysis}. Additionally, in UTAUT2, voluntarism of use was dropped as moderator. The predictive ability of UTAUT2 theory is found to be much higher in comparison to UTAUT \cite{tamilmani2021extended}. 

\textbf{ABC Model of Attitudes} The ABC Model of Attitudes \cite{eagly1998attitude} conceptualizes attitudes as being composed of three elements: Affect, Behavior, and Cognition. Affect refers to the individual’s feelings or emotional response toward an object. Behavior reflects the individual’s intentions or actions toward an object or situation. Cognition refers to the beliefs or thoughts an individual holds about an object \cite{ajlouni2023students}. These three components collectively constitute an individual’s attitude toward an object, person, issue, or situation. 

\textbf{Cultural-Historical Activity Theory (CHAT)} The Cultural-Historical Activity Theory (CHAT) offers a perspective for examining the intricate and dynamic systems in which AI is integrated into education. When CHAT is applied to technology or AI in education, it doesn't only consider the student-technology relationship, but the whole activity system (subject, object, tools, rules, community, and division of labour) within the wider socio-cultural and historical context \cite{carbonel2024emerging}. The underlying concept is that when a new element, such as a new technology, is introduced into the activity system, it often leads to tensions and contradictions. The model can help examine the system and begin envisioning new approaches. 

\textbf{Situated Expectancy-Value Theory (SEVT)} Renamed from the original expectancy-value theory, the Situated Expectancy-Value Theory (SEVT) highlights that students' expectancy-value beliefs—how well they believe they will perform on an upcoming task—are situation-specific, meaning they vary across contexts and are influenced by situational characteristics \cite{eccles2020expectancy}. SEVT is built on two main components, subjective task value and expectation of success. Subjective task value  consists of four parts that influence achievement-related choices and performance:
\begin{itemize}
\item Attainment value: how important the task is to the individual.
\item Intrinsic value: how interesting or enjoyable the task is.
\item Utility value: how useful the task is to the individual.
\item Relative cost: the perceived cost of engaging in the task.
\item Expectation of success refers to how likely the individual believes they are to succeed at the task \cite{eccles2020expectancy}.
\end{itemize}

\textbf{Extended three-tier Technology Use Model (3-TUM)} The extended Three-tier Technology Use Model (3-TUM) is based on the original 3-TUM, which integrates multidisciplinary perspectives—including motivation, social cognitive theory (SCT), theory of planned behavior (TPB), and the technology acceptance model (TAM)—to investigate attitudes toward information technology \cite{liaw2007understanding}. Individual attitudes are divided into three tiers: the individual characteristics and system quality tier, the affective and cognitive tier, and the behavioral intention tier. The individual characteristics tier includes factors such as self-efficacy, hedonic motivation, and self-regulation, while the system quality tier encompasses environmental factors. The affective and cognitive tier includes perceived satisfaction, perceived usefulness, and performance expectancy \cite{cai2023factors, liaw2008investigating}. The constructs of performance expectancy and hedonic motivation are inspired by the Extended Unified Theory of Acceptance and Use of Technology (UTAUT2). The extended model adds an additional tier: learning effectiveness \cite{cai2023factors}.

\textbf{Mitcham’s philosophical framework of technology} Mitcham’s \cite{mitcham1994thinking} philosophical framework of technology presents a comprehensive view of technology, emphasizing that technological knowledge and volition, rooted in human nature, lead to technological activities and the creation of concrete technological objects. The framework accommodates the full historical and conceptual range of technology—from ancient to modern and from simple to complex forms—without restriction to any specific type. Using this framework, the Mitcham Score was developed to classify students’ descriptions of technology through four key elements, namely Objects, Activities, Knowledge and Volition \cite{svenningsson2020mitcham}.

\textbf{Specific Forms of Digital Competence Needed to use ChatGPT} The Specific Forms of Digital Competence Needed to use ChatGPT \cite{xiao2023exploratory} is a conceptual model based on \cite{instefjord2017educating}'s work on the integration of professional digital competence in initial teacher education programmes. The framework comprises three main components: Technological proficiency (e.g., Be aware of the features of ChatGPT, Understand how ChatGPT works), Pedagogical compatibility (e.g., Think about and plan ways to use ChatGPT to enhance or transform language teaching and learning tasks, Implement tasks that use ChatGPT) and Social awareness (e.g., Have a critical awareness of the drawbacks of ChatGPT and consider them when planning and implementing tasks, Inform learners of the risks, ethical issues, and drawbacks of ChatGPT).

\section{Perception Scales}
\label{app:scales}

Breakdown of scales assessing attitudes, perceptions, feelings and/or behaviors toward technology as they appear in the reviewed studies. Similar to the breakdown of models, the present lists 
supports the thesis outlined in the third gap identified in the literature: the lack of homogeneity in the assessment of perceptions and the need for standardization in how perceptions are evaluated.

In addition to theoretical frameworks, several studies have assessed perceptions of LLM-based tutors using existing scales, either fully or partially: 

\textbf{Relevance of Science Education (ROSE)} The Relevance of Science Education (ROSE) questionnaire \cite{schreiner2004sowing} is a research instrument created as part of an international comparative project meant to shed light on students’ experiences, interests, priorities, images and perceptions that are relevant to the learning of science and technology and their attitudes towards these subjects. The questionnaire contains 247 questions across 10 sections and assesses students' interest in, attitude towards, and experiences in science and technology, as well as their opinion about environmental challenges and career aspirations.

\textbf{Short Version of the User Experience Questionnaire (UEQ-S)} The User Experience Questionnaire (UEQ) aims at collecting end-users' feelings, impressions, and attitudes that arise when experiencing a product. It consists of 26 items grouped into 6 scales, namely Attractiveness, Perspicuity, Efficiency, Dependability, Stimulation and Novelty \cite{schrepp2017design}. Schrepp, Hinderks, and Thomaschewski later developed a Short Version of the UEQ, reducing it to 8 items grouped into two scales, which measure two meta-dimensions: pragmatic and hedonic quality, while still covering the spectrum of product qualities assessed by the original UEQ.

\textbf{Cognitive Load Scale (CLS)} The Cognitive Load Scale (CLS) is a ten-item inventory, developed and validated by Leppink et al. \cite{leppink2013development}. The scale is based on the cognitive load theory \cite{van2005cognitive} which posits that instructions or instructional material can impose three types of cognitive load on the learner: intrinsic load (IL), extraneous load (EL), and germane load (GL). Intrinsic load refers to the inherent difficulty of the instructional content, which is influenced by the learners' prior knowledge. Extraneous load is the unnecessary cognitive effort caused by poorly designed instruction. Germane load, introduced later in the theory, refers to the mental effort consciously invested by learners in processing the intrinsic load \cite{hadie2016assessing}.

\textbf{Epistemically-Related Emotion Scales (EES)} The Epistemically-Related Emotion Scales (EES), developed by \citeauthor{pekrun2017measuring}, measure multiple emotions during knowledge-generating or epistemic activities. The instrument constists of 7 three-item scales, which assess the emotions of surprise, curiosity, enjoyment, confusion, anxiety, frustration, and boredom.

\textbf{Feedback Perceptions Questionnaire (FPQ)} Developed by \cite{strijbos2010peer}, the Feedback Perceptions Questionnaire (FPQ) measures feedback perceptions, in educational contexts, in terms of perceived fairness, usefulness, acceptance, willingness to improve, and affect.

\textbf{Questionnaire assessing Chinese secondary school students’ intention to learn AI} \citeauthor{chai2020extended} draw on the Theory of planned behavior \cite{ajzen1991theory} and the TAM \cite{davis1989perceived} literature to identify factors that may affect secondary students’ intention to learn AI. The survey involves nine factors: students’ knowledge about AI, general AI anxiety, and subjective norms are presented as background factors that influence their attitudes toward the behavior (i.e., perceived usefulness, social good, and attitude toward use) as well as perceived behavioral control (confidence and optimism) and their behavioral intention to learn AI.

\textbf{Attitude Toward Cheating (ATC) scale} The 34-item ATC scale developed by \citeauthor{gardner1988scale} predicts cheating behavior. Fourteen of the 34 items explicitly condemned cheating, while the remaining 20 items expressed tolerance toward cheaters. Since cheating is generally regarded negatively, students may be inclined to hide their true feelings. To promote objectivity and honesty, the items were worded without referencing the readers; that is, no items included personal pronouns. This approach ensured that the readers were not put on the defensive regarding their own behavior; instead, their attitudes were inferred from their judgments of others' behavior.

\textbf{Pupils’Attitudes Towards Technology Short Questionnaire (PATT)} Pupils’Attitudes Towards Technology (PATT) was first developed by \citeauthor{de1988techniek} and has been recently reconstructed and revalidated by \citeauthor{ardies2013reconstructing}. The revision led to the creation of the PATT Short Questionnaire (PATT-SQ), where the original 58 items were reduced to 24, divided into six subfactors. These six subfactors are Career Aspirations, Interest in Technology, Tediousness of Technology, Positive Perception of Effects of Technology, Perception of Difficulty, and Perception of Technology as a Subject for Boys or for Both Boys and Girls. The PATT-SQ
was examined and further developed by \citeauthor{svenningsson2018understanding}, leading to the creation of the PATT-SQ-SE.

\textbf{General Attitudes towards AI Scale (GAAIS)} The General Attitudes towards AI Scale (GAAIS) was developed based on the rationale that older Technology Acceptance Scales, such as the TAM \cite{davis1989perceived}, reflect users’ individual choices to use technology, while AI often involves decisions made by others. The scale comprises 32 items divided into two subscales: one measuring positive attitudes (Positive GAAIS) and the other measuring negative attitudes (Negative GAAIS). The Positive GAAIS consists of questions related to utility, desired use, and emotions (such as excitement), while the Negative GAAIS contains items related to unethical uses, errors, and negative emotions associated with using AI technology.

\textbf{Attitudes Towards Using ChatGPT (ATUC)} The Attitudes Towards Using ChatGPT (ATUC) scale was created by \citeauthor{ajlouni2023students} based on the ABC Model of Attitudes to evaluate undergraduate students’ attitudes toward ChatGPT as a learning tool. The ATUC comprises 22 items (6–26) distributed across three subscales: a) the affects subscale (comprising seven items, 6–12), b) the cognitive subscale 
(comprising eight items, 13–20), and c) the behavioral subscale (comprising seven items, 21–27).
\textbf{ Student Attitudes Toward Artificial Intelligence (SATAI) scale} The SATAI scale was developed by \citeauthor{suh2022development} and consists of 26 items that encompass three components—cognitive, affective, and behavioral factors—representing K-12 students' attitudes toward AI.

\textbf{Computer Technology Use Scale (CTUS)} The Computer Technology Use Scale was developed by \citeauthor{conrad2008relationships} and includes five factors: computer efficacy, technology-related anxiety, complexity, positive attitudes, and negative attitudes. These factors are organized into three domains: computer efficacy (perceived capacity), attitudes toward technology, and technology-related anxiety.

\textbf{Technology Acceptance Model Edited to Assess ChatGPT Adoption (TAME-ChatGPT)} The Technology Acceptance Model Edited to Assess ChatGPT Adoption (TAME-ChatGPT) survey was created based on the TAM framework \cite{sallam2023assessing}. It comprised 13 items for participants who heard of ChatGPT but did not use it and 23 items for participants who used ChatGPT.