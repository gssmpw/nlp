\section{Conclusion, Limitations, and Future Work}
\label{sec:conclusion}

LLM-based chatbots are being piloted in schools, often through partnerships between private companies and school districts. However, feedback from direct users (students and teachers) and indirect users (parents and school staff) has been largely absent before and during rollout, aside from a few informal efforts.  To address this gap, we conducted a literature review on stakeholders' perceptions of LLM-based chatbots in education, identifying key positive, negative, and neutral views. Students generally perceive these chatbots as useful and efficient but raise concerns about accuracy. Teachers value their role as personalized tutors but are primarily concerned with maintaining academic integrity. Neutral perceptions from both groups describe chatbots as complementary to traditional teaching.  The review also revealed gaps in the literature: \first underrepresentation of key stakeholders, such as parents; \second limited exploration of diverse educational contexts, including various levels and disciplines; and \third inconsistent assessment of perceptions, with varying theories, scales, and terminology applied to similar concepts.  We then proposed the \frameworklong (\framework) framework, incorporating a taxonomy of perceptions from the literature. \framework is designed to help policymakers and technologists assess stakeholder perceptions throughout the development and deployment of AI systems in education, promoting more responsible and trustworthy processes.

\paragraph{Limitations} This article has limitations that must be considered when interpreting the findings. First, the organization of stakeholders' perceptions into the taxonomy, as well as the assignment of values (positive, negative, or neutral) to them, was based on the researchers’ judgment during the content analysis process. Future iterations could help make the taxonomy more rigorous. Second, the literature review was limited to articles published on Google Scholar. Including articles from databases such as Scopus, ISI Web of Science, and other cross-references could add weight and rigor to the findings related to stakeholders' perceptions of LLM-based chatbots in education. Lastly, the in-depth analysis of the literature was based on empirical studies conducted globally. Narrowing the scope to a particular region or tool could provide deeper insights, as perceptions may be influenced by cultural and social norms, socio-economic status, and other factors related to the context and background of the stakeholders.

\paragraph{Future work} There are several promising directions for future research. First, the stakeholder-first approach described here lays the foundation for future studies investigating neglected stakeholders' perceptions across a range of overlooked contexts. Second, a repository of findings derived from the application of this framework could be created, to act as a reference for technologists and policymakers in future technology design processes and implementations. 