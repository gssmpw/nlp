In recent years, Large Language Models (LLMs) rapidly gained popularity across all parts of society, including education. After initial skepticism and bans, many schools have chosen to embrace this new technology by integrating it into their curricula in the form of virtual tutors and teaching assistants. However, neither the companies developing this technology nor the public institutions involved in its implementation have set up a formal system to collect feedback from the stakeholders impacted by them. In this paper, we argue that understanding the perceptions of those directly affected by LLMS in the classroom, such as students and teachers, as well as those indirectly impacted, like parents and school staff, is essential for ensuring responsible use of AI in this critical domain. Our contributions are two-fold. First, we present results of a literature review focusing on the perceptions of LLM-based chatbots in education. We highlight important gaps in the literature, such as the exclusion of key educational agents (\eg parents or school administrators) when analyzing the role of stakeholders, and the frequent omission of the learning contexts in which the AI systems are implemented. Thus, we present a taxonomy that organizes existing literature on stakeholder perceptions. Second, we propose the \frameworklong (\framework) framework, which can be used to systematically elicit perceptions and inform \emph{whether} and \emph{how} LLM-based chatbots should be designed, developed, and deployed in the classroom.