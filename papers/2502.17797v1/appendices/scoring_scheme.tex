\section{Annotation scoring scheme}\label{appendix:scoring_scheme}

\tableref{tab:weighting-scheme} 
% and \tableref{tab:weighting-scheme-qr} 
gives the scoring scheme in \sectionref{sec:result_score_calc} of the MQM settings.
% and the \sxsqr~setting respectively.

\begin{table}[ht]
\fontsize{6}{7}\selectfont
\centering
% \renewcommand{\arraystretch}{0.8} % Reduce vertical space
\resizebox{0.7\columnwidth}{!}{%
\begin{tabular}{@{}p{0.55cm}p{1.6cm}p{0.7cm}@{}}
\midrule
\textbf{Severity}   & \textbf{Category} & \textbf{Weight} \\ \midrule
\multirow{2}{*}{\major}  & Non-translation & 25  \\
                         & Others   & 5 \\\cmidrule{1-3}
\multirow{2}{*}{\minor}  & Fluency/Punctuation & 0.1  \\
                         & Others   & 1 \\\midrule
\end{tabular}%
}
\caption{MQM error span weighting scheme. Gibberish segments score 25 points, \major~errors 5 points, and \minor~errors 1 point, except for \minor~punctuation errors, which are weighted at 0.1 point each.
}
\label{tab:weighting-scheme}
\end{table}

% \begin{table}[ht]
% \fontsize{5}{6}\selectfont
% \centering
% % \renewcommand{\arraystretch}{0.8} % Reduce vertical space
% \resizebox{0.6\columnwidth}{!}{%
% \begin{tabular}{@{}p{1.5cm}p{0.7cm}@{}}
% \midrule
% \textbf{Category} & \textbf{Weight} \\ \midrule
% Much worse & 2  \\
% Worse      & 1  \\
% About the same & 0 \\\midrule
% \end{tabular}%
% }
% \caption{\sxsqr~segment scoring scheme.
% }
% \label{tab:weighting-scheme-qr}
% \end{table}