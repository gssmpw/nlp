\section{Related Work}
\label{section:background}

A brief discussion on attacks, followed by an analysis of orthogonal countermeasures that can co-exist with the method explored here prefaces the discussion of related work validating GNSS PNT with the help of local and remote time sources. 

\textbf{Attacking GNSS receivers} - Successful overtake of GNSS receivers in the field is possible by spoofing attacks that can be implemented either with signal generation or replay/meaconing (\cite{tippenhauer2011requirements, Kerns2014, Bhatti2017, Ioannides2016}). The risk of low-sophistication attacks in the wild is high given the availability of low-cost software-defined radio (SDR) hardware and software tools which are openly available (\cite{KexiongAllBelongToUs2018, Feng2021, HuangL2015}) even in the case of multi-constellation \cite{LeksellTGalileo2021} and multi-frequency modes \cite{SDRMultiFrequency2018}.

Advanced implementations of receiver-spoofer matched adversaries (\cite{HumphreysAssessingSpoofer, Maier2018}) rely on signal lift-off techniques: this requires code phase and Doppler shift synchronization at the victim antenna phase center. Additionally, deployment of attacks targeting mobile victims is complex, as precise tracking of the victim antenna is required throughout the attack. In a simpler setting, attacks targeting static timing-dedicated receivers smoothly deceive and control the time solution at the GNSS receiver, with the relevant case of phasor measurement units in smart grids (\cite{Shepard2012c, Humphreys2012, Jiang2013, Zhu2016}). Time Synchronization Attacks (TSA) target the time solution of the GNSS receiver, minimally disturbing the location or navigation part \cite{Zhang2013}. Even if centimeter-level knowledge of the victim's antenna position is required to perform a successful synchronized lift-off, code/doppler frequency sweep takeover is possible in commercial receivers. This approach eliminates the need for precise alignment of the spoofer signal and leverages the higher tracking bandwidth of modern receivers to successfully implement the attack \cite{Jiadong2019}.

%Rephrase paragraph

%Signal replaying/relaying, also known as meaconing, allows any attacker to replay/relay signals corresponding to a different place or time to a victim receiver, effectively shifting the PNT solution \cite{Lenhart2022}. Cryptographically secure signals can also be the target of advanced replay-based attackers, despite the additional security features. Specifically, techniques that use Secure Code Estimation and Replay (SCER) can be effective in spoofing even cryptographically-hardened receivers (\cite{humphreys2013detection, Arizabaleta2019, Gallardo2020}). Additionally, Distance Decreasing attacks can be effective when deployed against cryptographically enhanced signals, causing significant alterations of the GNSS PNT (\cite{ZhangLP:J:2022, ZhangP:C:2019a}).

Effective overtake, albeit to a lesser level of control, can be achieved also by signal replay/relay (meaconing): the attacker re-transmits signals corresponding to a different time and/or place at the victim receiver, causing a shift in the PNT solution \cite{Lenhart2022}. Such methods, in a more advanced configuration known as Secure Code Estimation and Replay (SCER) are effective also against cryptographically protected navigation messages and signals (\cite{humphreys2013detection, Arizabaleta2019, Gallardo2020}). Similarly, Distance Decreasing attacks cause significant alterations of the GNSS PNT, even against cryptographically enhanced signals(\cite{ZhangLP:J:2022, ZhangP:C:2019a}).


\textbf{Safeguarding GNSS receivers} - To counteract the growing issue of GNSS manipulation, countermeasures to achieve higher PNT solution robustness exist. Carrier-over-Noise ($C/N_0$) analysis with joint measurement of the receiver front-end gain allows detection of spoofing signals observing the power envelope variation and distortion (\cite{Akos2012, Lo2019, wesson2017gnss}). So-called Doppler shift tests in the received signals allow the detection of spoofed signals based on the transmitter frequency error (\cite{papadimMilcom2008, psiaki2013antenna}). Additionally, Receiver Autonomous Integrity Monitoring (RAIM) techniques (\cite{Jada2021, Sathaye2020}) allow detection and exclusion of adversary-crafted GNSS signal, even in case of time-specific faults \cite{gioiaTRAIM2021}. Additionally, multi-constellation failure detection \cite{ZhangP:C:2019b} and collaborative/cooperative detection methods proved capable of faulty signals detection and exclusion (FDE), hardening the GNSS receiver against adversarial manipulation. Albeit effective, RAIM and FDE methods require significant computation and access to the raw measurement from the GNSS receiver: most consumer-grade GNSS receivers do not integrate native RAIM capability or do not expose raw measurements to the user.

% Techniques encompassing multiple time providers are commonly used in network time distribution, where redundancy is provided by multiple non-co-located time references. Furthermore, master clocks usually rely on GNSS-provided time to discipline a local precision oscillator and are capable of a so-called hold-over in case misbehavior is detected in the GNSS receiver.  %Additionally, the solution is dependent on modelling of the local GNSS receiver clock performance which is often difficult to achieve in already integrated devices. 

\textbf{Time-based GNSS PNT validation} - Correction services in the L-band contribute to the robustness and accuracy of the PNT solution \cite{rugamer2023}: L-band corrections leverage an extensive fixed receiver network to provide Real Time Kinematic (RTK) corrections over the network. Methods relying on short-range networks to dedicated receivers or internet-provided correction streams are complemented by satellite-downlink-provided corrections, whose availability is increasing even in the consumer market and are designed to provide centimeter-level accuracy for precise positioning. For time-focused corrections, recent products like Fugro Atomichron promise accurate and reliable time and frequency, but cannot be evaluated at this time. \cite{fugroAtomichron}. On the other hand, its recent introduction combined with the integration required by the receiver manufacturers will be a major limitation towards the adoption of this system. Generally, while L-band correction services provide accurate aiding information, they require dedicated hardware and modems to operate. This is a major limitation towards adoption in consumer devices and generally low-power devices that are limited. For the attacker, correction services can be effectively used to precisely know the position of the intermediate spoofer antenna, allowing accurate estimation of the lever arm (vector between the reference and victim antenna). This leads to a more precise estimation of the victim's position, aiding the overtake of the victim receiver.

Commonly available connectivity (i.e., via cellular network) can be used to access alternative PNT information securely, which intuitively can be leveraged for GNSS receiver-provided time validation. Solutions considering single or multiple precision embedded clocks proved successful in detecting offsets and drift in the time solution due to an adversary (\cite{Arafin2016DetectingOscillators, Arafin2017, Spanghero2022, Hwang2014}). Receiver autonomous testing of the GNSS clock bias and drift allows monitoring of abrupt changes in the receiver clock bias \cite{jafarnia2013PNT}. While this is often an indicator of spoofing or other adversarial action, it requires the receiver front end to be disciplined with a high-quality local oscillator and rely on moving antennas. While the second is common in mobile devices but is not applicable for fixed installations, the first is usually difficult to achieve in commercial receivers, given that there is no access to the clock interface.

Network time providers enable even sparsely connected receivers to test the accuracy of the GNSS-provided time in respect to a set of remote references (\cite{spangheroGNSS20,kzmsppPLANS2020, spangheroMPPPLANS23}). Such countermeasure complements and augments other methods based on signal properties (\cite{papadimMilcom2008}) and can be integrated into existing GNSS-enabled platforms without changes to the receiver structure or the existing hardware.  Performance assessment of secure time transfer in support of cryptographically enhanced GNSS signals shows that the accuracy of network-provided time is sufficient for use with Chimera \cite{ODriscol2020}. Furthermore, the application of a combination of diverse time references to the current GNSS receivers and signals proved capable of hardening the security of the receiver \cite{spangheroMPPPLANS23,patentPP:J:2009}, but a systematic analysis of the actual capabilities against adversaries targeting both the GNSS receiver and alternative time providers is missing. This work sets out to validate in an extended experimental context \cref{section:implementation} the methodology shown in \cite{spangheroMPPPLANS23,kzmsppPLANS2020} when not limited to a staged approach but by considering as a whole any time reference available to the system (as shown in \cref{section:methodology}). An extended advanced attacker model (\cref{section:sys-adversary-model}) is used in a comprehensive evaluation of the results (\cref{section:results-conclusion} to demonstrate that time-based cross-check of the GNSS PNT solution against heterogeneous time sources is practical.

% ----------------------- KEEP but move from here -----------------------
% Advanced attackers targeting network time providers can control individual time servers or entire pools (\cite{perry2021, Deutsch2018, Malhotra2017}). Methods to improve the security of the Network Time Protocol (NTP) structure leverage cryptographical extensions to the standard but initial adaptations proved to be not secure \cite{rfc5906}, due to vulnerabilities in the implemented algorithms. The Internet Engineering Task Force (IETF) standardized a new security extension for NTP, Network Time Security (NTS), aiming principally at secure time distribution in networked systems \cite{ietf-ntp-using-nts-for-ntp-28}. The most recent evolution of the standard addresses most of these issues, making NTS an important candidate for secure network time transfer. 

% Additionally, Google Roughtime achieves a high level of time distribution assurance by providing digitally signed and non-repudiable coarse time information \cite{ietf-ntp-roughtime-07}. Due to the strong limitations of Roughtime accuracy and the higher computational requirements, its application in GNSS time validation is limited. Nevertheless, the time validation at cold-start perfectly fits Roughtime, particularly when combined with more accurate methods like NTS.
% ----------------------- KEEP but move from here -----------------------