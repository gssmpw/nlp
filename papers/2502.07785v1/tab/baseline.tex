\begin{table}[h!]
    \centering
    \small
    \setlength{\tabcolsep}{3pt}
    \begin{tabular}{c l c cc}
        \toprule
         {\#} & \textbf{Method \& Resolution} & RE@SG $\downarrow$ & Face $\uparrow$ & Body $\uparrow$ \\
        \midrule
         {1.} & MV-Adapter~\cite{huang2024mvadapter} (768)  & 4.7 & \underline{43.0} & {64.1} \\
         {2.} & Era3D~\cite{li2024era3dhighresolutionmultiviewdiffusion} (512)  & \underline{4.1}  & {38.1} & \underline{64.4} \\
         {3.} & Wonder3D~\cite{long2023wonder3d} (256)  & 5.3  & {34.7} & {58.8} \\
         {4.} & \ourmodel (P3@1K)  & \textbf{3.0} & \textbf{58.0} & \textbf{68.1} \\
        \bottomrule
    \end{tabular}
    \caption{\textbf{Quantitative comparison with SoTA multi-view models.} We quantitatively compare \ourmodel against state-of-the-art multi-view diffusion models. We find that Pippo preserves identity (\ie face and body similarity) and 3D consistency (RE) better while operating at a higher resolution compared to baselines.}
    \label{tab:baseline}
\end{table}


