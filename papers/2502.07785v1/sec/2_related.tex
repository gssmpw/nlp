\section{Related Work}
\label{sec:related_works}
\begin{figure*}[t]
\begin{center}
\includegraphics[width=.85\linewidth]{fig_overview_v3.pdf}
\end{center}
\caption{
FastAtlas Overview: In each frame, we compute charts spanning fully or partially visible triangles (a), determine texture space bounding boxes for the visible portions of the view-space projections of each chart, and tightly pack these boxes into atlases (b, here $2K \times 2K$). We simultaneously bijectively parameterize and shade the charts into their atlas boxes, obtaining high quality texture space shading (c), and use this shading to render the shaded frames (d).}
\label{fig:overview}
\label{fig:alg_overview}
\end{figure*}

\section{Overview}
\label{sec:overview}
Our work has two core contributions: a real-time, GPU-based algorithm for tight packing of general parameterized charts into compact atlases; and a real-time TSS method that
utilizes this packing.  

\paragraph*{FastAtlas Packing.}
FastAtlas runs entirely on the GPU as a series of compute shaders. It takes the bounding boxes of parameterized charts as input, and packs them into an atlas (Fig~\ref{fig:overview}b, Sec.~\ref{sec:pack}). As such, the only input it requires are the dimensions of the bounding boxes.
Its outputs are deterministic; identical input charts are packed into identical atlases. This is critical for TSS and similar applications, as it ensures that consecutive frames taken from the same camera view have the same shading. Even minute shading differences across such frames can cause sampling jitter, leading to undesirable flicker \cite{baker2012rock}. 
While prior methods such as \cite{mueller2018shading,hladky2019tessellated,hladky2021snakebinning,Neff2022MSA} cap the dimensions of the charts that can be packed as-is for a given atlas size, and scale down all charts that exceed these dimensions, we scale all charts by the same factor, and do so only when strictly necessary to achieve packing success (Figs~\ref{fig:atlas},~\ref{fig:sas_issues}). 

\paragraph*{TSS using FastAtlas.}
Our end-to-end TSS atlas generation method combines the packing method above with a novel approach for computing seamless per-frame charts. 
We define our charts as the connected components of the visible surfaces in each frame (Fig.~\ref{fig:overview}a), and efficiently compute them using a parallel union-find algorithm (Sec.~\ref{sec:visible}). Since the boundaries of these charts coincide with the contours of the rendered surface, they are {\em invisible} to the viewer. This approach 
eliminates the artifacts caused by shading discontinuities along visible seams (Fig.~\ref{fig:seams}). 

\begin{parWithWrapFigure}
\begin{wrapfigure}{l}{.27\columnwidth}%
\includegraphics[width=\linewidth]{fig_inset_view_plane.pdf}%
\end{wrapfigure}
We bijectively parametrize the {\em visible portions} of our charts by projecting them to view space (inset). This maps a constant number of texels to each pixel in the final rendered output, evenly distributing residual undersampling error across all image pixels. While conceptually straightforward, efficiently parameterizing charts containing partially visible triangles using viewspace projection is non-trivial, as the visible portions may no longer be triangular (e.g. green triangle in the inset); applying naive projection to triangles with vertices behind the camera may produce ill-posed results. Clipping triangles before projection is both computationally expensive and significantly complicates downstream operations. We avoid explicit clipping by observing that all that is required for atlas packing is the dimensions of, potentially conservative, bounding boxes of these projected visible portions. We compute such bounding boxes without explicit chart clipping by adapting a conservative screen coverage estimator \shortcite{Blinn:CalculatingScreenCoverage} (Sec.~\ref{sec:box}). We then pack the computed boxes using FastAtlas. 
\end{parWithWrapFigure}

Finally, we shade the visible portion of each chart into its corresponding atlas bounding box (Fig~\ref{fig:overview}c). 
The resulting texture is then used during rasterization as a standard texture map (Fig. ~\ref{fig:overview}d). 
Our framework is compatible with all existing approaches for texture space shading, including forward shading, raytraced illumination, or deferred shading in texture space \cite{baker:2016}. In the examples shown, we use the standard forward shading based rendering pipeline included in the G3D Innovation Engine \cite{G3D17}, a commercial grade renderer.


We review multi-view human datasets and generative models for human synthesis,  categorizing methods by their data requirements and prior constraints—such as parametric human models or explicit 3D structures. In this work, we minimize reliance on complex priors by training the model on large amounts of human-centric data and impose minimal camera and spatial 
controls.


\vspace{1mm}
\heading{Multi-view Human Datasets.}
While many large 2D human datasets exist~\cite{lin2019coco, LAION, ImageNet}, 3d captures of people (\ie captures using dozens of views at a time) are still relatively rare, since they are expensive to collect and are thus mostly obtained in specialized research labs.
Nevertheless, this kind of data has recently become more common, going from hundreds to thousands of publicly-available captures.
For face-only captures, a few datasets~\cite{Yu_2020_CVPR,yang2020facescape,yenamandra2020i3dmm,2023renderme360,martinez2024codec} provide up to 600 subjects captured from over 60 views each,
while for full-body captures, 4D-Dress~\cite{wang20244ddress} provides 32 subjects with 2 outfits each, and detailed annotations for their garments, while DNA-rendering~\cite{2023dnarendering} provides 500 subjects from 60 views.
MVHumanNet~\cite{xiong2024mvhumannet} stands apart by providing 4,500 people captured from 48 views in and over 9,000 sets of clothing, outpacing other datasets by an order of magnitude in terms of people diversity.
We use internal multi-view data of $~\sim$1000 identities with dense ($\sim$160) 
camera setup, but we expect our model to produce reasonable results
with publicly-available captures.


\vspace{1mm}
\heading{Generating Novel Poses and Expressions.}
Although different from our task, this related work builds 3D animatable models of people and faces. These methods use neural radiance fields~\cite{HumanNerf, su2023npc, su2022danbo}, or 3d Gaussians~\cite{zielonka2023drivable, qian20243dgs, wen2024gomavatar,kirschstein2023nersemble}, using canonical ``T''-poses and a corresponding forward and backward mapping to model local appearance changes.
Many of these methods, however, rely on accurate human shape and pose estimation, and only achieve high photorealism for personalized models from studio captures. 
Relatedly, a family of methods focus on generating animatable faces from a single image by disentangling viewpoints and expressions from large collections of 2d portraits~\cite{megaportraits, drobyshev2024emoportraits, zhang2023metaportrait, xu2024vasa, Nerfies, StyleGAN}, allowing photorealistic realtime reenactment, while being limited to faces and small viewpoint variations.
Unlike these methods, we complete missing parts of a person given either one or a few partial views.
We do not repose the person or animate the faces but can instead recover entirely missing views, which is orthogonal to these methods.
Moreover, our approach addresses both full-body and facial reconstruction, rather than specializing in either domain.


\heading{Generations with Explicit 3D Structures.}
Similar to early methods that extracted 3d representations from 2d models via score distillation~\cite{dreamfusion,TextMesh,fantasia3d,Nerdi,ProlificDreamer,magic3d,dreambooth3d}, 3d representations of human appearance have been extracted from GANs by co-training a neural renderer from learned TriPlane~\cite{EG3D} or HexPlane~\cite{An_2023_CVPR, li2025spherehead} latent spaces.
Others train generative models that operate directly in 3D space with diffusion or GAN objectives~\cite{Rodin3DDM,muller2023diffrf,TrainOn3D1,TrainOn3D2,TrainOn3D4,TrainOn3D5} or regression models to directly predict 3D representations~\cite{PiFU,saito2020pifuhd,sengupta2024diffhuman}.
While these methods produce 3d consistent faces by design, their quality is limited by the relatively small amounts of 3d training data
and/or the limitations of their respective 3d representations.
In contrast, we focus on generating 3d-consistent 2d images and thus avoid the downsides of explicit 3d modelling.

\heading{Generations with 2D Models.}
Another option is to model viewpoint changes in 2d, without an underlying 3d representation~\cite{zero1to3}, sometimes modelling either photometric~\cite{iNVS} or epipolar~\cite{huang2024epidiff} constraints explicitly.
DiffPortrait3D~\cite{gu2024diffportrait3d} is a portrait-generation method that falls roughly within this category, by fine-tuning a 2d diffusion model for 3d-aware face generation using ControlNet-style~\cite{zhang2023adding} conditioning for viewpoint control, as well as cross-view attention~\cite{guo2023animatediff} and initialization from a 3d-consistent GAN~\cite{EG3D}.
Other methods are more general and not specific to humans, focusing on single-view to 3D generation~\cite{zero1to3,NeuralLift-360,sargent2023zeronvs,seo2024genwarp,Make-It-3D,shen2023anything3d,gao2024cat3d}. In addition to these single-view methods, video models have been fine-tuned for camera control~\cite{he2024cameractrl,wang2024motionctrl}.
