\section{Method}\label{sec:method}

We train our models following a three-stage strategy:

\begin{itemize}
    \item \textbf{Image-only Pre-training (P1).} We pretrain on a large-scale human-centric dataset with image conditioning.
    \item \textbf{Multiview Mid-training (M2).} We train models at a low-resolution of \resone to denoise $48$ target views with coarse camera control (no pixel-aligned spatial control).
    \item \textbf{Multiview Post-training (P3).} We train at a high-resolution of \resfour to denoise $1-3$ target views with spatial control injected via ControlMLP layers. 
\end{itemize}

\noindent We denote the training stage and resolution of any given model as \{stage\}@\{resolution\}. For instance, M2@128 represents a mid-trained model at 128 resolution.

\subsection{Base Model} \label{ssec:base_model}

\heading{Architecture.} We adopt a DiT-like~\cite{peebles2023scalable} architecture with scale, shift, and gate modulation for timestep 
conditioning inspired by Stable Diffusion 3~\cite{esser2024scaling} and Flux~\cite{flux}.
We simplify the architecture by employing MLP and attention in parallel~\cite{zhai2022scaling}, and removing second LayerNorm after 
attention layers.
We use 
VAE for training in latent space with $8$x spatial compression, and patchify latent images via a linear layer and patch size of $2$. We use fixed sinusoidal positional encoding during training. We provide more details in~\cref{app_sec:dit}, and the exact design of our DiT block in~\cref{fig:dit_block}. 

\heading{Image-only Pre-training.} During pre-training, the model learns to 
denoise an image conditioned on its corresponding image embedding from 
DINOv2~\cite{darcet2023vitneedreg,oquab2023dinov2},
this is similar in principle to the image decoder of DALL-E 2~\cite{DALLE-2}.
We project both embeddings to the model dimension using with a linear layer 
to create a joint conditioning. Importantly, our pretraining setup does not require any 
annotations or captions for the images, and is well-aligned with our downstream 
objective of generating consistent multiview images given a single reference image as 
input.

Formally, given an image $\mathbf{y} \in \mathbb R^{H \times W \times C}$, and the joint conditioning as $\mathbf{e}^{\text{img}} \in \mathbb R^{N \times D}$. We pre-train our diffusion model $\epsilon_\theta$ with the following objective: 
\begin{equation}\label{eq:dm_loss}
    \mathcal{L}_{\mathrm{DM}} = ||\noise^t - \dmmodel(\mathbf{y}^{t}, \mathbf{e}^{\text{img}},  t)||^2
\end{equation}

\noindent Where $t \in \mathbb [0,T]$ is diffusion timestep, $\noise^t \sim \mathcal{N}(\bm{0}, \bm{I})$ is the noise added at the given timestep.  We use the DDPM~\citep{DDPM,FirstDiffusionModel} formulation to define discrete timesteps and set $T=1000$. We first pretrain our model at \restwo and then at \resthree on a large corpus of human-centric images. Exact training details can be found in~\cref{app_sec:diffusion}.

\subsection{Multiview Model} \label{ssec:multiview_model}

Our goal is to generate many high resolution and unseen novel viewpoints of a human subject (akin to a studio capture) given a single input image.

\vspace{1mm}
\heading{Input Reference.} We denote the input image as $\mathbf{x}^{\text{ref}}$ and corresponding face crop as $\mathbf{x}^{\text{face}}$. The face crop is obtained using FaceNet~\cite{schroff2015facenet} and resized to the same size as input image, such that: $\mathbf{x}^{\text{face}}, \mathbf{x}^{\text{ref}} \in \mathbb R^{H \times W \times C}$. 

\vspace{1mm}
\heading{Target Cameras.} We denote the target viewpoints to be synthesized using distinct cameras (intrinsics and extrinsics), represented as $\mathbf{c}_{1:N}$. Each camera
is used to generate its \plucker coordinates $\mathbf{P}_i \in \mathbb R^{H \times W \times 6}$.


\vspace{1mm}
\heading{Target \spatialanchor.} In addition to target cameras, we provide an oriented 3D point denoted as $\mathbf{a}_i=[\mathbf{R}_{i}|\mathbf{t}_{i}]$, which roughly defines the center of the subject's head, as well as their gaze direction. We show an example of our \spatialanchor in~\cref{fig:pipeline} in the studio on the left. This anchor is color-coded and projected into a 2D image, which is used as conditioning for our target view generations.  During inference, the \spatialanchor can be placed at any given point that lies within the field-of-view of the target cameras.

\vspace{1mm}
\heading{Multiview Diffusion Model.} Given the above inputs, we train a multiview diffusion model $\epsilon_\theta$ to \emph{jointly} denoise all the target views $\mathbf{y}_{1:N} \in \mathbb R^{N \times H \times W \times C}$ with the objective: 
\begin{equation}\label{eq:dm_loss2}
    \mathcal{L}_{\mathrm{DM}} = ||\noise^t - \dmmodel(\mathbf{y}^{t}_{1:N}, \mathbf{c}_{1:N}, \mathbf{x}^{\text{ref}}, \mathbf{x}^{\text{face}},  t)||^2
\end{equation}

\noindent where 
$\mathbf{y}^{t}_{1:N}$ are noisy target images. We condition the base model on the provided reference image and its face crop by concatenating their patchified latent tokens with the noisy input latent tokens to the model. This is shown in ~\cref{fig:pipeline}

\section{Model}
\label{sec:model}
Let $[N] = \{1, 2, \dots, N \}$ be a set of $N$ agents.
We examine an environment in which a system interacts with the agents over $T$ rounds.
Every round $t\leq T$ comprises $N$ \emph{sessions}, each session represents an encounter of the system with exactly one agent, and each agent interacts exactly once with the system every round.
I.e., in each round $t$ the agents arrive sequentially. 


\paragraph{Arrival order} The \emph{arrival order} of round $t$, denoted as $\ordv_t=(\ord_t(1),\dots, \ord_t(N))$, is an element from set of all permutations of $[N]$. Each entry $q$ in $\ordv_t$ is the index of the agent that arrives in the $q^{\text{th}}$ session of round $t$.
For example, if $\ord_t(1) = 2$, then agent $2$ arrives in the first session of round $t$.
Correspondingly, $\ord_t^{-1}(i)=q$ implies that agent $i$ arrives in the $q^{\text{th}}$ session of round $t$. 

As we demonstrate later, the arrival order has an immediate impact on agent rewards. We call the mechanism by which the arrival order is set \emph{arrival function} and denote it by $\ordname$. Throughout the paper, we consider several arrival functions such as the \emph{uniform arrival} function, denoted by $\uniord$, and the \emph{nudged arrival} $\sugord$; we introduce those formally in Sections~\ref{sec:uniform} and~\ref{sec:nudge}, respectively.

%We elaborate more on this concept in Section~\ref{sec: arrival}.


\paragraph{Arms} We consider a set of $K \geq 2$ arms, $A = \{a_1, \ldots, a_K\}$. The reward of arm $a_i$ in round $t$ is a random variable $X_i^t \sim \mathcal{D}^t_i$, where the rewards $(X_i^t)_{i,t}$ are mutually independent and bounded within the interval $[0,1]$. The reward distribution $\mathcal{D}^t_i$ of arm $a_i$, $i\in [K]$ at round $t\in T$ is assumed to be non-stationary but independent across arms and rounds. We denote the realized reward of arm $a_i$ in round $t$ by $x_i^t$. We assume \emph{reward consistency}, meaning that rewards may vary between rounds but remain constant within the sessions of a single round. Specifically, if an arm $a_i$ is selected multiple times during round~$t$, each selection yields the same reward $x_i^t$, where the superscript $t$ indicates its dependence on the round rather than the session. This consistency enables the system to leverage information obtained from earlier sessions to make more informed decisions in later sessions within the same round. We provide further details on this principle in Subsection~\ref{subsec:information}.


\paragraph{Algorithms} An algorithm is a mapping from histories to actions. We typically expect algorithms to maximize some aggregated agent metric like social welfare. Let $\mathcal H^{t,q}$ denote the information observed during all sessions of rounds $1$ to $t-1$ and sessions $1$ to $q-1$ in round $t$.  The history $\mathcal H^{t,q}$ is an element from $(A \times [0,1])^{(t-1) \cdot N +q-1}$, consisting of pairs of the form (pulled arm, realized reward). Notice that we restrict our attention to \emph{anonymous} algorithms, i.e., algorithms that do not distinguish between agents based on their identities. Instead, they only respond to the history of arms pulled and rewards observed, without conditioning on which specific agent performed each action.
%In the most general case, algorithms make decisions at session $q$ of round $t$  based on the entire history $\mathcal H^{t,q}$ and the index of the arriving agent $\ord_t(q)$. %Furthermore, we sometimes assume that algorithms have Bayesian information, i.e., algorithms are aware of the distributions $(\mathcal D_i)^K_{i=1}$. 
Furthermore, we sometimes assume that algorithms have Bayesian information, meaning they are aware of the reward distributions $(\mathcal{D}^t_i)_{i,t}$. If such an assumption is required to derive a result, we make it explicit. %Otherwise, we do not assume any additional knowledge about the algorithm’s information. %This distinction allows us to analyze both general algorithms without prior distributional knowledge and specialized algorithms that leverage Bayesian information.


\paragraph{Rewards} Let $\rt{i}$ denote the reward received by agent $i \in [N]$ at round $t$, and let $\Rt{i}$ denote her cumulative reward at the end of round $t$, i.e., $\Rt{i} = \sum_{\tau=1}^{t}{r^{\tau}_{i}}$. We further denote the \emph{social welfare} as the sum of the rewards all agents receive after $T$ rounds. Formally, $\sw=\sum^{N}_{i=1}{R^T_i}$. We emphasize that social welfare is independent of the arrival order. 


\paragraph{Envy}
We denote by $\adift{i}{j}$ the reward discrepancy of agents $i$ and $j$ in round $t$; namely, $\adift{i}{j}= \rt{i} - \rt{j}$. %We call this term \omer{name??} reward discrepancy in round $t$. 
The (cumulative) \emph{envy} between two agents at round $t$ is the difference in their cumulative rewards. Formally, $\env_{i,j}^t= \Rt{i} - \Rt{j}$ is the envy after $t$ rounds between agent $i$ and $j$. We can also formulate envy as the sum of reward discrepancies, $\env_{i,j}^t= \sum^{t}_{\tau=1}{\adif{i}{j}^\tau}$. Notice that envy is a signed quantity and can be either positive or negative. Specifically, if $\env_{i,j}^t < 0$, we say that agent $i$ envies agent $j$, and if $\env_{i,j}^t > 0$, agent $j$ envies agent $i$. The main goal of this paper is to investigate the behavior of the \emph{maximal envy}, defined as
\[
\env^t = \max_{i,j \in [N]} \env^t_{i,j}.
\]
For clarity, the term \emph{envy} will refer to the maximal envy.\footnote{ We address alternative definitions of envy in Section~\ref{sec:discussion}.} % Envy can also be defined in alternative ways, such as by averaging pairwise envy across all agents. We address average envy in Section~\ref{sec:avg_envy}.}
Note that $\env_{i,j}^t$ are random variables that depend on the decision-making algorithm, realized rewards, and the arrival order, and therefore, so is $\env^t$. If a result we obtain regarding envy depends on the arrival order $\ordname$, we write $\env^t(\ordname)$. Similarly, to ease notation, if $\ordname$ can be understood from the context, it is omitted.



\paragraph{Further Notation} We use the subscript $(q)$ to address elements of the $q^{\text{th}}$ session, for $q\in [N]$.
That is, we use the notation $\rt{(q)}$ to denote the reward granted to the agent that arrives in the $q^{\text{th}}$ session of round $t$ and $\Rt{(q)}$ to denote her cumulative reward. %Additionally, we introduce the notation $\at{(q)}$ to denote the arm pulled in that session.
Correspondingly, $\sdift{q}{w} = \rt{(q)} - \rt{(w)}$ is the reward discrepancy of the agents arriving in the $q^{\text{th}}$ and $w^{\text{th}}$ sessions of round $t$, respectively. 
To distinguish agents, arms, sessions and rounds, we use the letters $i,j$ to mark agents and arms, $q,w$ for sessions, and $t,\tau$ for rounds.


\subsection{Example}
\label{sec: example}
To illustrate the proposed setting and notation, we present the following example, which serves as a running example throughout the paper.

\begin{table}[t]
\centering
\begin{tabular}{|c|c|c|c|}
\hline
$t$ (round) & $\ordv_t$ (arrival order) & $x_1^t$ & $x_2^t$ \\ \hline
1           & 2, 1                     & 0.6     & 0.92    \\ \hline
2           & 1, 2                     & 0.48    & 0.1     \\ \hline
3           & 2, 1                     & 0.15    & 0.8     \\ \hline
\end{tabular}
\caption{
    Data for Example~\ref{example 1}.
}
\label{tbl: example}
\end{table}

\begin{algorithm}[t]
\caption{Algorithm for Example~\ref{example 1}}
\label{alguni}
\DontPrintSemicolon 
\For{round $t = 1$ to $T$}{
    pull $a_{1}$ in the first session\label{alguniexample: first}\\
    \lIf{$x^t_1 \geq \frac{1}{2}$}{pull $a_{1}$ again in second session \label{alguniexample: pulling a again}}
    \lElse{pull $a_{2}$ in second session \label{alguniexample: sopt else}}
}
\end{algorithm}


\begin{example}\label{example 1}
We consider $K=2$ uniform arms, $X_1,X_2 \sim \uni{0,1}$, and $N=2$ for some $T\geq 3$. We shall assume arm decision are made by Algorithm~\ref{alguni}: In the first session, the algorithm pulls $a_1$; if it yields a reward greater than $\nicefrac{1}{2}$, the algorithm pulls it again in the second session (the ``if'' clause). Otherwise, it pulls $a_2$.



We further assume that the arrival orders and rewards are as specified in Table~\ref{tbl: example}. Specifically, agent 2 arrives in the first session of round $t=1$, and pulling arm $a_2$ in this round would yield a reward of $x^1_2 = 0.92$. Importantly, \emph{this information is not available to the decision-making algorithm in advance} and is only revealed when or if the corresponding arms are pulled.




In the first round, $\boldsymbol{\eta}^1 = \left(2,1\right)$; thus, agent 2 arrives in the first session.
The algorithm pulls arm $a_1$, which means, $a^1_{(1)} = a_1$, and the agent receives $r_{2}^1=r_{(1)}^1=x_1^1=0.6$.
Later that round, in the second session, agent 1 arrives, and the algorithm pulls the same arm again since $x^1_1 = 0.6 \geq \nicefrac{1}{2}$ due to the ``if'' clause.
I.e., $a^1_{(2)} = a_1$ and $r_{1}^1 = r_{(2)}^1 = x_1^1 = 0.6$.
Even though the realized reward of arm $a_2$ in that round is higher ($0.92$), the algorithm is not aware of that value.
At the end of the first round, $R^1_1 = R^1_{(2)} = R^1_2 = R^1_{(1)} = 0.6$. The reward discrepancy is thus $\adif{1}{2}^1 = \adif{2}{1}^1= \sdif{2}{1}^1 = 0.6 - 0.6 =0$. 



In the second round, agent 1 arrives first, followed by agent 2.
Firstly, the algorithm pulls arm $a_1$ and agent 1 receives a reward of $r_{1}^2 = r_{(1)}^2 = x_1^2 = 0.48$.
Because the reward is lower than $\nicefrac{1}{2}$, in the second session the algorithm pulls the other arm ($a^2_{(2)} = a_2$), granting agent 2 a reward of $r_{2}^2 = r_{(2)}^2 = x_2^2 = 0.1$.
At the end of the second round, $R^2_1 = R^2_{(1)} = 0.6 + 0.48 = 1.08$ and $R^2_2 = R^2_{(2)} = 0.6 + 0.1 = 0.7$. Furthermore, $\sdif{2}{1}^2 = \adif{2}{1}^2 = r^2_{2} - r^2_{1} = 0.1 - 0.48 = -0.38$.

In the third and final round, agent 2 arrives first again, and receives a reward  of $0.15$ from $a_1$. When agent 1 arrives in the second session, the algorithm pulls arm $a_2$, and she receives a reward of $0.8$. As for the reward discrepancy, $\sdif{2}{1}^3 = \adif{2}{1}^3 = r^3_{2} - r^3_{1} = 0.15 - 0.8 = -0.75$. 

Finally, agent 1 has a cumulative reward of $R^3_1 = R^3_{(2)} = 0.6 + 0.48 + 0.8 = 1.88$, whereas agent~2 has a cumulative reward of $R^3_2 = R^3_{(1)} = 0.6 + 0.1 + 0.15 = 0.85$. In terms of envy, $\env^1_{1,2}= \adif{1}{2}^1 =0$, $\env^2_{1,2}=\adif{1}{2}^1+\adif{1}{2}^2= 0.38$, and $\env^3_{1,2} = -\env^3_{2,1} = R^3_1-R^3_2 = 1.88-0.85 = 1.03$, and consequently the envy in round 3 is $\env^3 = 1.03$.
\end{example}


\subsection{Information Exploitation}
\label{subsec:information}

In this subsection, we explain how algorithms can exploit intra-round information.
Since rewards are consistent in the sessions of each round, acquiring information in each session can be used to increase the reward of the following sessions.
In other words, the earlier sessions can be used for exploration, and we generally expect agents arriving in later sessions to receive higher rewards.
Taken to the extreme, an agent that arrives after all arms have been pulled could potentially obtain the highest reward of that round, depending on how the algorithm operates.

To further demonstrate the advantage of late arrival, we reconsider Example~\ref{example 1} and Algorithm~\ref{alguni}. 
The expected reward for the agent in the first session of round $t$ is $\E{\rt{(1)}}=\mu_1=\frac{1}{2}$, yet the expected reward of the agent in the second session is
\begin{align*}
\E{\rt{(2)}}=\E{\rt{(2)}\mid X^t_1 \geq \frac{1}{2} }\prb{X^t_1 \geq \frac{1}{2}} + \E{\rt{(2)}\mid X^t_1 < \frac{1}{2} }\prb{X^t_1 < \frac{1}{2}};
\end{align*}
thus, $\E{\rt{(2)}} =\E{X^t_1\mid X^t_1 \geq \frac{1}{2} }\cdot \frac{1}{2} + \mu_2\cdot\frac{1}{2} = \frac{5}{8}$.
Consequently, the expected welfare per round is $\E{\rt{(1)}+\rt{(2)}}=1+\frac{1}{8}$, and the benefit of arriving in the second session of any round $t$ is $\E{\rt{(2)} - \rt{(1)}} = \frac{1}{8}$. This gap creates envy over time, which we aim to measure and understand.
%This discrepancy generates envy over time, and our paper aims to better understand it.
\subsection{Socially Optimal Algorithms}
\label{sec: sw}
Since our model is novel, particularly in its focus on the reward consistency element, studying social welfare maximizing algorithms represents an important extension of our work. While the primary focus of this paper is to analyze envy under minimal assumptions about algorithmic operations, we also make progress in the direction of social welfare optimization. See more details in Section~\ref{sec:discussion}.%Due to space limitations, we defer the discussion on socially optimal algorithms to  \ifnum\Includeappendix=0{the appendix}\else{Section~\ref{appendix:sociallyopt}}\fi.




% Since our model is novel and specifically the reward consistency element, it might be interesting to study social welfare optimization. While the main focus of our paper is to study envy under minimal assumptions on how the algorithm operates, we take steps toward this direction as well. Due to space limitations, we defer the discussion on socially optimal algorithms to  \ifnum\Includeappendix=0{the appendix}\else{Section~\ref{appendix:sociallyopt}}\fi.  We devise a socially optimal algorithm for the two-agent case, offer efficient and optimal algorithms for special cases of $N>2$ agents, and an inefficient and approximately optimal algorithm for any instance with $N>2$. Moreover, we address the welfare-envy tradeoff in Section~\ref{sec:extensions}.


% Social welfare, unlike envy, is entirely independent of the arrival order. While the main focus of our paper is to study envy under minimal assumptions on how the algorithm operates, socially optimal algorithms might also be of interest. Due to space limitations, we defer the discussion on socially optimal algorithms to  \ifnum\Includeappendix=0{the appendix}\else{Section~\ref{appendix:sociallyopt}}\fi. We devise a socially optimal algorithm for the two-agent case, offer efficient and optimal algorithms for special cases of $N>2$ agents, and an inefficient and approximately optimal algorithm for any instance with $N>2$. %Furthermore, we treat the welfare-envy tradeoff of the special case of Example~\ref{example 1}.




\vspace{1mm}
\heading{\midtraining.} In mid-training stage, we want to train a strong multiview model that can denoise several images together, and absorb the dataset quickly at a lower resolution. During this phase, we do not use any pixel-aligned spatial control such as \plucker or \spatialanchor. We use an MLP to encode the flattened $16$-dimensional target camera intrinsics and extrinsics into a single token 
. We fuse this camera token into each noisy latent token (for the corresponding view) as positional encoding, which makes our multiview model 3D-aware of the target viewpoints. 
We mid-train our model at \resone resolution to jointly denoise 24 views.

\vspace{1mm}
\heading{\posttraining.} In the post-training stage, our objective is to create a high-resolution model which is 3D-consistent starting with a low-resolution and a 3D-aware (but not consistent) model. For this, we design a lightweight ControlNet~\cite{zhang2023adding}-inspired module, which takes as input the pixel-aligned \plucker and \spatialanchor controls, and the denoising timestep to create a separate modulation signal for the multiview model. We name this module ControlMLP, as it uses a single MLP to generate scale-and-shift modulated control for each multiview-DiT block as shown in ~\cref{fig:dit_block}. Each layer of ControlMLP is zero-initialized at the start. We find that the Post-training phase is crucial in reducing flicker and 3D inconsistencies in generations. We post-train models at \resthree and \resfour resolutions to jointly denoise 10 and 2 views respectively. Increasing the number of views further lead to GPU out-of-memory issues. 



\vspace{1mm}
\heading{Encoding \plucker and \spatialanchor.} We notice that the relative differences between neighboring pixels in \plucker coordinates are tiny. To amplify these differences better, we use a SIREN~\cite{SIREN} layer to first process the $6$D grid into a $32$D feature grid. Then, we downsample it by $8$x to match the size of latent tokens and feed it as input to the ControlMLP. In addition, use the \spatialanchor to fix the position and orientation of the subject's head in 3D. We only use the \spatialanchor for generations, and not for the input reference view. We encode the \spatialanchor image into the latent space of our model via VAE, and concatenate it with \plucker input and pass it through an MLP to create the modulation signal at each layer. 


\subsection{Understanding and Improving Spatial Control}\label{ssec:spatial_control}

\noindent This section presents our design choices and examines alternative approaches for injecting pixel-aligned spatial controls during \posttraining stage. We demonstrate the effectiveness of spatial control through a focused overfitting experiment with quantitative evaluations in ~\cref{tab:3deval}.

\vspace{1mm}
\heading{Scene Overfitting Task.} We use $160$ frames from a fixed 3D scene of a given subject and timestamp, split into $100$ training and $60$ validation views. We overfit our mid-trained model to the training views while testing various spatial control methods, training only the control modules while keeping other weights frozen. After overfitting for $10$K iterations, we evaluate the model on validation views for novel view synthesis.  Strong generalization to validation viewpoints indicates effective spatial control and appropriate camera viewpoint sensitivity. Through this task, we evaluate different spatial control injection methods in~\cref{tab:3deval}, starting with simple to advanced modulation designs.

\begin{table}[h!]
    \centering
    \setlength{\tabcolsep}{5pt}
    \small
    \begin{tabular}{llcc}
        \toprule
         {\#} & \textbf{Method (\mid @ 128)} & PSNR$_{\text{val}}$ $\uparrow$ & PSNR$_{\text{train}}$ $\uparrow$\\
        

    
    
    

    \midrule
        {1.} & Mid-trained (No Overfitting) & 19.23 & 19.70 \\
    \midrule
    
    {2.} & + Camera (w/ MLP)~\cite{zero1to3, MVDream} & 17.95\textsubscript{{\color{red}{-1.28}}} & 19.92\textsubscript{{\color{green}{+0.22}}} \\
    
    {3.} & + \plucker (w/ MLP)~\cite{RayConditioningGAN,LFNs,kant2024spad,he2024cameractrl}   & 18.89\textsubscript{{\color{red}{-0.34}}} & 20.74\textsubscript{{\color{green}{+1.04}}} \\
    
    {4.} & + ControlMLP  & 19.45\textsubscript{{\color{green}{+0.22}}} & 29.36\textsubscript{{\color{green}{+9.66}}} \\
    
    {5.} & + SIREN  & 20.13\textsubscript{{\color{green}{+0.90}}} & 30.19\textsubscript{{\color{green}{+10.49}}} \\
    
    {6.} & + \spatialanchor (Ours) \ & 22.60\textsubscript{{\color{green}{+3.37}}} & 30.49\textsubscript{{\color{green}{+10.79}}} \\
    \bottomrule
    
    
    \end{tabular}
    \caption{
        \textbf{Evaluating and Designing Spatial Controls (\cref{ssec:spatial_control}).} We overfit our multi-view model for $10$K iterations on 100 views of a single scene from our \upperbody dataset, and evaluate on 60 novel views of this scene by only varying the spatial controls (camera and \spatialanchor). We use this setup to test the modulation strength of different spatial controls. Subscript values \textcolor{green}{green}/\textcolor{red}{red} show deviations from Row 1. }
    \vspace{-10pt}
    \label{tab:3deval}
\end{table}



\begin{itemize}
    \item \textbf{No overfitting (Row 1).} The Mid-trained model without scene-specific overfitting achieves comparable PSNR of 19.2 and 19.7 on train and validation views respectively. We treat this setup as baseline to improve over. 

    \item \textbf{Encoding Camera with MLP (Row 2).} We encode camera using an MLP similar to prior works~\cite{zero1to3, MVDream} and our \midtraining stage (\cref{ssec:multiview_model}). After overfitting, the model achieves slightly better PSNR on training views as expected, however the validation PSNR drops by 1.28 points to 17.95. This suggests that an MLP does not provide enough modulation for camera control. 
    
    \item \textbf{\plucker as Positional Encoding (Row 3).} In this setup, we use downsampled and patchified \plucker coordinates processed through MLP to create positional encoding, which is added to the noisy latent tokens. This setup is inspired from  prior works~\cite{RayConditioningGAN,LFNs,kant2024spad,he2024cameractrl,bahmani2024vd3d}, and it further improves the validation PSNR compared to MLP at 18.89, but lags behind the non-overfitted baseline.

    \item \textbf{\plucker with ControlMLP and SIREN (Row 4, 5).} Here, we use our ControlMLP module to inject spatial control at each multiview-DiT block output. Moreover, encoding \plucker coordinate with a SIREN~\cite{SIREN} amplifies the relative differences between neighboring pixels (\cref{ssec:multiview_model}). This setup achieves PSNR of 20.13 with an improvement of 0.9 over baseline.

    \item \textbf{Adding \spatialanchor (Row 6).} Finally using the \spatialanchor gives validation PSNR of 22.6 (gain of 3.3 points over baseline) and enables strong spatial control. Thus, we adopt this configuration for \posttraining stage.
\end{itemize}

\subsection{Handling Varying Number of Views at Inference}\label{ssec:scaling_views}

As discussed in~\cref{ssec:multiview_model}, during training we jointly denoise a fixed number of views. Specifically, 24 views for mid-training at \resone, and 2 or 12 views for post-training at resolutions \resthree and \resfour respectively. This choice is largely motivated to avoid GPU out-of-memory errors during training. However, during inference, we wish to scale the number of views much further to generate smooth turnaround videos. This is feasible because we can run inference at half precision (using bfloat16) and do not need the backprop computation graph to be stored. 

We find that simply scaling the number of views (or tokens) during inference beyond 2\texttt{x} of the number of views during training leads to blurry and degraded generations. We find these degradations to be most significant in regions unspecified in the input, for example, the back of the head or ears as shown in~\cref{fig:qual_entropy}. We investigate this issue next, and introduce Attention Biasing to remedy it. 

\begin{figure}[h!]
    \centering
    \includegraphics[width=0.4\textwidth]{figs/entropy_fig.pdf}    
    \caption{The entropy of the estimated state values on the verifier's training dataset.}
    \label{fig:entropy}
\end{figure}

\heading{Attention Biasing.} Let $\textbf{X} \in \mathbb{R}^{N \times d}$ denote the sequence of tokens denoised jointly, where $N, d$ represent number of tokens and the token dimension, respectively. In \ourmodel the total number of tokens in the sequence is proportional to number of views since the VAE and Patchify jointly compress the image by 16\texttt{x} in height and width. Within each DiT block, we compute the $\textrm{Attention}(\textbf{Q}, \textbf{K}, \textbf{V}) = \textbf{A}\textbf{V}$, where  $\textbf{Q},\textbf{K},\textbf{V}$ are query, key, value matrices and $\textbf{A}$ are attention scores computed via taking a row-wise softmax as follows:
\begin{equation}
    \textbf{A}_{i, j} = \frac{e ^{\lambda \textbf{Q}_{i}\textbf{K}_{j}^{\top}}}{\sum_{j' = 1}^{N} e ^{\lambda \textbf{Q}_{i}\textbf{K}_{j'}^{\top}}}, \label{attention map}
\end{equation}
Where the scaling factor $\lambda$ was proposed to be set to $1 / \sqrt{d}$ by the original work~\cite{vaswani2017attention} to stabilize the softmax operation as $d$ increases and avoiding biased (sharp) softmax distributions. We can quantify the sharpness of the attention by computing its entropy. Following previous works in NLP~\cite{ghader2017does,attanasio2022,zhang2024attention}, and in vision~\cite{jin2023training} we define entropy of attention as:
\begin{equation}
    \textrm{Ent}(\textbf{A}_{i}) = -\sum_{j=1}^{N} \textbf{A}_{i, j} \log (\textbf{A}_{i, j}), \label{entropy}
\end{equation}
By substituting the equation ~\eqref{attention map} in equation~\eqref{entropy}, authors of prior work~\cite{jin2023training} meticulously derive that entropy of attention grows logarithmically with number of tokens as follows: \(\textrm{Ent}(\textbf{A}_{i}) \propto \log N\). Furthermore, the authors show that this growth in entropy can be offset during inference by growing the scaling factor $\lambda$ as follows: 
\begin{equation}
    \begin{aligned}
    \lambda = \sqrt{\frac{1}{d}} \cdot \sqrt{\frac{\log N_i}{\log N_t}}
    \end{aligned}
\end{equation}
Where $N_t, N_i$ denote the number of tokens during training and inference respectively. For more details, please refer to Sections 3.1 and 3.2 of the original work~\cite{jin2023training}. Empirically, we find that having a slightly faster growing $\lambda$ alleviates the degradation better. Hence, we propose a hyperparameter $\gamma$ (growth factor) that is tuned between the range $[1.0,2.0]$ to control the growth of $\lambda$ as follows: 
\begin{equation}
    \begin{aligned}
    \lambda_\texttt{ours} = \sqrt{\frac{1}{d}} \cdot \sqrt{\frac{\gamma \cdot \log N_i}{\log N_t}}
    \end{aligned}
    \label{attn_bias}
\end{equation}
We follow the above Equation~\eqref{attn_bias} with our growth factor hyperparameter $\gamma$. In~\cref{fig:entropy}, we plot the entropy (Y-axis) during an inference pass of \ourmodel across varying number of views (tokens) being denoised and demonstrate the corresponding growth in entropy. Furthermore, we also show the attenuation in the entropy under increasing growth factors $\gamma$ (X-axis). 

We put generated visuals before and after using the suggested attention biasing in~\cref{fig:qual_entropy}. We put visuals sweeping over more values of the growth factor in Appendix~\cref{fig:qual_entropy_full}. Similar techniques have been explored in LLMs for handling and generating text at longer contexts~\cite{peng2023yarn,veličković2024softmax}, where the scaling factor $\lambda$ mentioned above is analogous to the inverse of the temperature scale. 

Additionally, we also found that using a bump function instead of constant Classifier-free Guidance during generation leads to fewer artifacts. We discuss this trick further in~\cref{app_sec:diffusion}.

\begin{figure}[h!]
    \centering    
    \hspace{-0.5cm}
    \includegraphics[width=\linewidth]{img/qual_entropy.pdf}
    \vspace{-0.2cm}
    \caption{{\textbf{Generations under varying strengths of growth factor $\gamma$ (\cref{ssec:scaling_views})}. On each row we show the generated views across vanilla attention~\cite{vaswani2017attention} (No scaling), prior work~\cite{jin2023training} and our formulation Eq.~\eqref{attn_bias}. It can be seen that growth factor ($\gamma$) greater than 1.0 is crucial to mitigate the entropy buildup. We show only 10 views per row subsampled evenly from 60 views generated at \resthree resolution. The model was trained to jointly denoised only 12 views ($N_i = 5*N_t$).}}
    \label{fig:qual_entropy}
\end{figure}



 




\subsection{Enhanced 3D Consistency Metric}
\label{ssec:consistencymetric}



Traditionally, the 3D consistency of multiview generation models is evaluated using 2D image metrics such as PSNR, LPIPS, and SSIM against a fixed set of ground truth images. However, this approach unfairly penalizes models that generate plausible and 3D-consistent novel content deviating from the fixed ground truth images.
Some works try to address this by measuring SfM~\cite{SFM,fridman2024scenescape} or epipolar error~\cite{muller2024multidiff,TrainNVSDM3}, but these methods solve for pose or are not robust since they measure against the entire epipolar line. To address these limiations, we use our GT camera poses as input and compute \reprojectionerror (RE) given our known camera poses and predicted correspondences.




Our computation of RE involves the following steps:
\begin{enumerate}
    \item \textbf{Landmarks and Correspondence estimation.} We use SuperPoint~\cite{detone2018superpoint} to detect landmarks in the generated images and employ SuperGlue~\cite{sarlin2020superglue} to establish pairwise correspondences between landmarks across images.
    \item \textbf{Triangulation.} Given the correspondences and camera parameters, we apply Triangulation based on Direct Linear Transformation (DLT)~\cite{Hartley2004} to obtain the corresponding 3D points for each landmark.
    \item \textbf{Reprojection and Error calculation.} We reproject these 3D points onto each image and compute the RE as the L2 distance between the original landmark and the reprojected 3D point normalized by image resolution, and average error across all images. 
\end{enumerate}
This approach evaluates multiview generation models by focusing on their ability to produce 3D-consistent results rather than adhering to a fixed ground truth. 
The \reprojectionerror provides a valuable basis for comparison across different methods. Furthermore, by computing RE on a set of real-world images that are independent of our generated images, we can establish a baseline that quantifies the error due to the noisy predictions from SuperGlue and SuperPoint rather than the quality of our generated images.

\heading{Naming Convention (RE@SG).} Please note that the use of SuperPoint~\cite{detone2018superpoint} and SuperGlue~\cite{sarlin2020superglue} estimation modules is a particular instantiation of our metric, and these could be replaced in the future with stronger counterparts such as MAST3R~\cite{dust3r_cvpr24,mast3r_arxiv24} or domain-specific keypoint detectors such as Sapiens~\cite{khirodkar2025sapiens}. Thus we use the naming convention of RE@SG to denote the Reprojection Error (RE) under SuperGlue (SG) estimation, which could be modified accordingly in the future for different estimators. 
























    
    
    


    

    
    
    

