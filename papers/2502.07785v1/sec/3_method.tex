\section{Method}\label{sec:method}

We train our models following a three-stage strategy:

\begin{itemize}
    \item \textbf{Image-only Pre-training (P1).} We pretrain on a large-scale human-centric dataset with image conditioning.
    \item \textbf{Multiview Mid-training (M2).} We train models at a low-resolution of \resone to denoise $48$ target views with coarse camera control (no pixel-aligned spatial control).
    \item \textbf{Multiview Post-training (P3).} We train at a high-resolution of \resfour to denoise $1-3$ target views with spatial control injected via ControlMLP layers. 
\end{itemize}

\noindent We denote the training stage and resolution of any given model as \{stage\}@\{resolution\}. For instance, M2@128 represents a mid-trained model at 128 resolution.

\subsection{Base Model} \label{ssec:base_model}

\heading{Architecture.} We adopt a DiT-like~\cite{peebles2023scalable} architecture with scale, shift, and gate modulation for timestep 
conditioning inspired by Stable Diffusion 3~\cite{esser2024scaling} and Flux~\cite{flux}.
We simplify the architecture by employing MLP and attention in parallel~\cite{zhai2022scaling}, and removing second LayerNorm after 
attention layers.
We use 
VAE for training in latent space with $8$x spatial compression, and patchify latent images via a linear layer and patch size of $2$. We use fixed sinusoidal positional encoding during training. We provide more details in~\cref{app_sec:dit}, and the exact design of our DiT block in~\cref{fig:dit_block}. 

\heading{Image-only Pre-training.} During pre-training, the model learns to 
denoise an image conditioned on its corresponding image embedding from 
DINOv2~\cite{darcet2023vitneedreg,oquab2023dinov2},
this is similar in principle to the image decoder of DALL-E 2~\cite{DALLE-2}.
We project both embeddings to the model dimension using with a linear layer 
to create a joint conditioning. Importantly, our pretraining setup does not require any 
annotations or captions for the images, and is well-aligned with our downstream 
objective of generating consistent multiview images given a single reference image as 
input.

Formally, given an image $\mathbf{y} \in \mathbb R^{H \times W \times C}$, and the joint conditioning as $\mathbf{e}^{\text{img}} \in \mathbb R^{N \times D}$. We pre-train our diffusion model $\epsilon_\theta$ with the following objective: 
\begin{equation}\label{eq:dm_loss}
    \mathcal{L}_{\mathrm{DM}} = ||\noise^t - \dmmodel(\mathbf{y}^{t}, \mathbf{e}^{\text{img}},  t)||^2
\end{equation}

\noindent Where $t \in \mathbb [0,T]$ is diffusion timestep, $\noise^t \sim \mathcal{N}(\bm{0}, \bm{I})$ is the noise added at the given timestep.  We use the DDPM~\citep{DDPM,FirstDiffusionModel} formulation to define discrete timesteps and set $T=1000$. We first pretrain our model at \restwo and then at \resthree on a large corpus of human-centric images. Exact training details can be found in~\cref{app_sec:diffusion}.

\subsection{Multiview Model} \label{ssec:multiview_model}

Our goal is to generate many high resolution and unseen novel viewpoints of a human subject (akin to a studio capture) given a single input image.

\vspace{1mm}
\heading{Input Reference.} We denote the input image as $\mathbf{x}^{\text{ref}}$ and corresponding face crop as $\mathbf{x}^{\text{face}}$. The face crop is obtained using FaceNet~\cite{schroff2015facenet} and resized to the same size as input image, such that: $\mathbf{x}^{\text{face}}, \mathbf{x}^{\text{ref}} \in \mathbb R^{H \times W \times C}$. 

\vspace{1mm}
\heading{Target Cameras.} We denote the target viewpoints to be synthesized using distinct cameras (intrinsics and extrinsics), represented as $\mathbf{c}_{1:N}$. Each camera
is used to generate its \plucker coordinates $\mathbf{P}_i \in \mathbb R^{H \times W \times 6}$.


\vspace{1mm}
\heading{Target \spatialanchor.} In addition to target cameras, we provide an oriented 3D point denoted as $\mathbf{a}_i=[\mathbf{R}_{i}|\mathbf{t}_{i}]$, which roughly defines the center of the subject's head, as well as their gaze direction. We show an example of our \spatialanchor in~\cref{fig:pipeline} in the studio on the left. This anchor is color-coded and projected into a 2D image, which is used as conditioning for our target view generations.  During inference, the \spatialanchor can be placed at any given point that lies within the field-of-view of the target cameras.

\vspace{1mm}
\heading{Multiview Diffusion Model.} Given the above inputs, we train a multiview diffusion model $\epsilon_\theta$ to \emph{jointly} denoise all the target views $\mathbf{y}_{1:N} \in \mathbb R^{N \times H \times W \times C}$ with the objective: 
\begin{equation}\label{eq:dm_loss2}
    \mathcal{L}_{\mathrm{DM}} = ||\noise^t - \dmmodel(\mathbf{y}^{t}_{1:N}, \mathbf{c}_{1:N}, \mathbf{x}^{\text{ref}}, \mathbf{x}^{\text{face}},  t)||^2
\end{equation}

\noindent where 
$\mathbf{y}^{t}_{1:N}$ are noisy target images. We condition the base model on the provided reference image and its face crop by concatenating their patchified latent tokens with the noisy input latent tokens to the model. This is shown in ~\cref{fig:pipeline}

To illustrate equilibria and dynamics of performative prediction games, we focus on a scenario in which a \emph{duopoly} of mortgage companies, i.e. banks, compete to sell loans to customers.

\paragraph{Customer Model:} In our game, each bank is trying to attract customers from a given population $\mathcal{P}$. We model this population as comprised of individuals with a single-dimensional type: we denote individual $j$'s type as $y_j \in [0,1]$. For simplicity, we assume that \(y\) represents the customer’s probability of repaying the loan\footnote{In practice, a customer's (normalized) credit score can be interpreted as a noisy observation of $y_j$. This also corresponds to credit scores being \emph{calibrated}.}, i.e., $y_j := \P[Y_j = 1]$, where $Y_j$ is a random variable such that $Y_j = 0$ means that $j$ defaults on their loan, and $Y_j = 1$ means they repay their loan. Customer types in the population are drawn from a known distribution $D_y$ supported on $[0,1]$. 

\paragraph{Game between Banks:} Each Bank \(i \in \{1, 2\}\) selects two parameters \( (\tau_i, \gamma_i) := \theta_i\), where:
\begin{itemize}
    \item \(\tau_i \in \{\tau_l,\tau_h\}\) is the credit score threshold for approving a customer\footnote{We restrict the bank to only pick between two thresholds, $\tau_l$ and $\tau_h$. However, we highlight how our results are affected when we expand the strategy space to $n > 2$ actions in our experiments of Appendix \ref{app:3gamma}.}. Specifically, a customer $j$ with credit score \(y_j\) is approved by Bank $i$ if and only if \(y_j \geq \tau_i\);
    \item \(\gamma_i \in \{\gamma_l, \gamma_h\}\) is the interest rate offered to approved customers.
\end{itemize}
We denote as shorthand the space of allowable thresholds by $\Gamma := [0,1]$ and allowable interests rates by $\Lambda := [0,1]$. %The latter is set without loss of generality---we simply normalize the rates to be at most $1$. 
% {\color{red} Vidya: just thinking about this but is it natural to restrict interest rate to $1$? I don't think it would affect the equilibrium structure of the game but theoretically I think the interest rate could be anything in $[0,\infty)$.} {\color{green} Guanghui: Could we say something like this is without loss of generality} \gua{changed.}\juba{I think we repeated this twice, the next sentence already had this}
The loan amount is normalized to $1$ in the entire paper, without loss of generality; in this case, if a customer chooses Bank $i$, and the customer is approved by the bank at an interest rate of $\gamma_i$, the expected utility for the bank is equal to
\[
(1+\gamma_i)\cdot \P[Y_i = 1]-\P[Y_i = 0] = (1+\gamma_i)y_i-(1-y_i).
\]


%In practice, the credit score \(y\) serves as a noisy observation of the true likelihood of the customer's repayment. 

\paragraph{Banks' Utilities:} For given parameter choices \(\theta_1 = (\tau_1, \gamma_1)\) by Bank 1 and \(\theta_2 = (\tau_2, \gamma_2)\) by Bank 2, a \emph{rational} customer with credit score $y$ acts as follows:

\begin{enumerate}
    \item \textbf{Qualified for a single bank}: 
        \begin{itemize}
        \item If \(\tau_1 \leq y < \tau_2\), the customer goes to Bank 1, as the score qualifies for Bank 1 but not Bank 2. Conversely, if \(\tau_2 \leq y < \tau_1\), the customer chooses Bank 2.
    \end{itemize}
    \item \textbf{Qualified for both banks}:
     \begin{itemize}
        \item If \(\tau_1, \tau_2 \leq y\) and \(\gamma_1 < \gamma_2\), the customer selects Bank 1 for its lower interest rate. Conversely, if \(\gamma_1 > \gamma_2\), the customer chooses Bank 2.
        \item If \(\gamma_1 = \gamma_2\), the customer picks each bank with probability $1/2$. 
    \end{itemize}
    \item \textbf{Unqualified for both banks}:
    \begin{itemize}
        \item If \(y < \tau_1\) and \(y < \tau_2\), the customer is rejected by both banks.
    \end{itemize}
\end{enumerate}

The expected reward for Bank 1, denoted as \(u_1(\theta_1, \theta_2)\), can then be expressed as:
\begin{align}\label{eq:utility}
    u_1(\theta_1, \theta_2) 
    &=  \mathbb{E}_{y \sim D_y} \left[ \mathbb{I}\{\underbrace{\tau_1 \leq y < \tau_2 \ \cup \ (\tau_1, \tau_2 \leq y \ \cap \ \gamma_1 < \gamma_2)}_{\text{accepted by Bank 1}}\} \cdot \big((1+\gamma_1)y - (1-y)\big) \right] \nonumber\\
    & + \frac{1}{2} \mathbb{E}_{y \sim D_y} \left[ \mathbb{I}\{\underbrace{\tau_1, \tau_2 \leq y \ \cap \ \gamma_1 = \gamma_2}_{\text{accepted by both Banks}}\} \cdot \big((1+\gamma_1)y - (1-y)\big) \right].
\end{align}
Note that the problem is \emph{symmetric}, i.e., the utility function for Bank 2 can be derived by swapping the roles of \(\theta_1\) and \(\theta_2\). I.e., $u_2(\theta_1, \theta_2) = u_1(\theta_2, \theta_1)$. 

% If a bank only attracts customers between thresholds $\tau_a$ and $\tau_b$, for $\tau_a<\tau_b$, we call $[\tau_a,\tau_b]$ the \emph{threshold} range for that bank. For example, if Bank $1$ sets a threshold of $\tau_1$, Bank $2$ a threshold of $\tau_2 > \tau_1$, and $\gamma_1 > \gamma_2$, then Bank 1 has a threshold range of $[\tau_1,\tau_2]$, while bank $2$ has a threshold range of $[\tau_2,1]$.
% Note that the parameters set by \emph{both} banks, i.e. $(\theta_1,\theta_2)$ both influence the threshold range for each of Bank 1 and 2.  If $\tau_1>\tau_2$, $\gamma_1>\gamma_2$, then $\tau_a>\tau_b$, and the bank does not attract any customers. 
% {\color{red} is it possible for $\tau_a > \tau_b$, leading to the bank never attracting customers?} \gua{if $\gamma_1>\gamma_2$, $\tau_1>\tau_2$, then it gets no customer. I think it also makes sense.}\juba{I think we said we wanted to delete the discussion of the threshold range, no?}

% \noindent \textbf{Discrete Model}   
% We now present the discrete version of our model, where the interest rates and thresholds are selected from finite sets \(\Gamma\) and \(\Lambda\), respectively, with $\tau\in[0,1], \gamma\in[0,1]$,  for all $\tau\in\Lambda$ and $\gamma\in\Gamma$, \(|\Gamma| = n\) and \(|\Lambda| = m\). Let \(p_1, p_2 \in \Delta(\Gamma \times \Lambda)\) represent the mixed strategies of the two banks, where \(\Delta(\Gamma \times \Lambda)\) denotes the set of probability distributions over the discrete decision space \(\Gamma \times \Lambda\).


% \begin{Remark}
%    Note that our proposed problem can be reformulated as a standard multi-player performative prediction problem \citep{narang2023multiplayer}. However, in our problem, the data distribution faced by each learner breaks the Lipschitzness assumption of previous work~\citep{hardt2023performative,narang2023multiplayer}. A small modification in one of the learner's thresholds can completely change how demand is allocated across both learners, as is often the case in Bertrand-style games. 
% \end{Remark} 

% \gua{I made some changes to Remark 1, please have a look}
\begin{Remark}
   Previous works in multi-learner performative prediction~\citep{narang2023multiplayer} resort to an insensitivity assumption, i.e., the data distribution faced by each player can only changes slightly when the parameters also change slightly; formally, the data distribution faced by each player is Lipschitz in their decisions. This is immediately not true in our setting: the bank slightly changing its parameters can completely changes the demand distribution of customers it faces. Intuitively, this is because of Bertrand-competition-style effects, where if two banks have similar rates, one bank that lowers their rate by a small amount suddenly captures the entire customer demand that is eligible for that rate.%\juba{made further light edits adding intuition}
   
   In Appendix \ref{Appendix:refumulation}, we discuss this problem more carefully by reformulating our problem in the standard multi-learner performative prediction form given by~\citep{narang2023multiplayer}. We show the distribution is not Lipschitz with respect to the parameters, and thus does not satisfy the insensitivity assumption. 
%Prior work~\citep{hardt2023performative,narang2023multiplayer} showed that, for a general multi-agent performative prediction framework to work, insensitivity assumptions are needed: in the \textbf{worst case}, they can construct settings where the insensitivity assumption does not hold and simple dynamics do not converge anymore. We add nuance to this picture. We will show that our dynamics often converge, even absent insensitivity assumptions, highlighting that while the impossibility results of previous work hold in the worst case, they may not hold in the ``average case'' and especially not in problems motivated by applications. In particular, we will show convergence to a variety of equilibria of our game, and often to symmetric Nash equilibria where insensitivity is immediately violated.
     
\end{Remark}



% \paragraph{Relationship to Performative Prediction} A central point of our work is to highlight that \textcolor{red}{needs writing from intro}. We highlight how our work specifically ties to ``Performative Prediction'' below:


%\textcolor{red}{needs a definition environment}



%Here, \(\E_{\theta_1, \theta_2}\) represents the expected utility of the banks over their respective strategies \((\theta_1, \theta_2)\). These inequalities ensure that neither bank can unilaterally improve its expected utility by deviating from its mixed strategy in the equilibrium.



%and  for all $\tau\in\Gamma$, we have $\tau\in\Lambda$, $(\tau,\gamma)\in[0,1]^2$. Let $\Gamma\times\Lambda$
%In this paper, we focus on the most fundamental case, where there are two choices for each parameter: $0\leq\tau_{\ell}<\tau_{h}\leq 1$, and $0\leq \gamma_{\ell}< \gamma_{h}\leq 1$. In this case, the utility for each pair of decisions forms a $4\times4$ matrix (given in Table \ref{tab:my-table}). We consider the canonical case where $\tau_{\ell}=\frac{1}{2+\gamma_{h}}$, and $\tau_{h}=\frac{1}{2+\gamma_{\ell}}.$ Note that these are natural choices for the thresholds, in the sense that, if there is only one bank and the interest rate is set to be $\gamma$, then $\frac{1}{2+\gamma}$ is the optimal threshold corresponding to the fixed $\gamma$.


%and the thresholds are chosen in $\Lambda=\{\tau^{(1)},\dots,\tau^{(m)}\}$. Here, we only assume that, for each $\gamma\in\Gamma$, there at least exist one $\tau\in\Lambda$ such that $f(\gamma,\tau,1)>0$. Note that this is a very minor assumption, in the sense that, if for a $\gamma$ such that $f(\gamma,\tau,1)<0$ for all $\tau\in\Lambda$, then adopting this decision will lead to negative utility regardless of the opponent's decision, and thus is not an interesting case. 

%\textcolor{red}{The model section is missing the dynamic version of the game. We should clearly define the one-shot and the dynamic game}
% we only considered one-shot case in our paper




\vspace{1mm}
\heading{\midtraining.} In mid-training stage, we want to train a strong multiview model that can denoise several images together, and absorb the dataset quickly at a lower resolution. During this phase, we do not use any pixel-aligned spatial control such as \plucker or \spatialanchor. We use an MLP to encode the flattened $16$-dimensional target camera intrinsics and extrinsics into a single token 
. We fuse this camera token into each noisy latent token (for the corresponding view) as positional encoding, which makes our multiview model 3D-aware of the target viewpoints. 
We mid-train our model at \resone resolution to jointly denoise 24 views.

\vspace{1mm}
\heading{\posttraining.} In the post-training stage, our objective is to create a high-resolution model which is 3D-consistent starting with a low-resolution and a 3D-aware (but not consistent) model. For this, we design a lightweight ControlNet~\cite{zhang2023adding}-inspired module, which takes as input the pixel-aligned \plucker and \spatialanchor controls, and the denoising timestep to create a separate modulation signal for the multiview model. We name this module ControlMLP, as it uses a single MLP to generate scale-and-shift modulated control for each multiview-DiT block as shown in ~\cref{fig:dit_block}. Each layer of ControlMLP is zero-initialized at the start. We find that the Post-training phase is crucial in reducing flicker and 3D inconsistencies in generations. We post-train models at \resthree and \resfour resolutions to jointly denoise 10 and 2 views respectively. Increasing the number of views further lead to GPU out-of-memory issues. 



\vspace{1mm}
\heading{Encoding \plucker and \spatialanchor.} We notice that the relative differences between neighboring pixels in \plucker coordinates are tiny. To amplify these differences better, we use a SIREN~\cite{SIREN} layer to first process the $6$D grid into a $32$D feature grid. Then, we downsample it by $8$x to match the size of latent tokens and feed it as input to the ControlMLP. In addition, use the \spatialanchor to fix the position and orientation of the subject's head in 3D. We only use the \spatialanchor for generations, and not for the input reference view. We encode the \spatialanchor image into the latent space of our model via VAE, and concatenate it with \plucker input and pass it through an MLP to create the modulation signal at each layer. 


\subsection{Understanding and Improving Spatial Control}\label{ssec:spatial_control}

\noindent This section presents our design choices and examines alternative approaches for injecting pixel-aligned spatial controls during \posttraining stage. We demonstrate the effectiveness of spatial control through a focused overfitting experiment with quantitative evaluations in ~\cref{tab:3deval}.

\vspace{1mm}
\heading{Scene Overfitting Task.} We use $160$ frames from a fixed 3D scene of a given subject and timestamp, split into $100$ training and $60$ validation views. We overfit our mid-trained model to the training views while testing various spatial control methods, training only the control modules while keeping other weights frozen. After overfitting for $10$K iterations, we evaluate the model on validation views for novel view synthesis.  Strong generalization to validation viewpoints indicates effective spatial control and appropriate camera viewpoint sensitivity. Through this task, we evaluate different spatial control injection methods in~\cref{tab:3deval}, starting with simple to advanced modulation designs.

\begin{table}[h!]
    \centering
    \setlength{\tabcolsep}{5pt}
    \small
    \begin{tabular}{llcc}
        \toprule
         {\#} & \textbf{Method (\mid @ 128)} & PSNR$_{\text{val}}$ $\uparrow$ & PSNR$_{\text{train}}$ $\uparrow$\\
        

    
    
    

    \midrule
        {1.} & Mid-trained (No Overfitting) & 19.23 & 19.70 \\
    \midrule
    
    {2.} & + Camera (w/ MLP)~\cite{zero1to3, MVDream} & 17.95\textsubscript{{\color{red}{-1.28}}} & 19.92\textsubscript{{\color{green}{+0.22}}} \\
    
    {3.} & + \plucker (w/ MLP)~\cite{RayConditioningGAN,LFNs,kant2024spad,he2024cameractrl}   & 18.89\textsubscript{{\color{red}{-0.34}}} & 20.74\textsubscript{{\color{green}{+1.04}}} \\
    
    {4.} & + ControlMLP  & 19.45\textsubscript{{\color{green}{+0.22}}} & 29.36\textsubscript{{\color{green}{+9.66}}} \\
    
    {5.} & + SIREN  & 20.13\textsubscript{{\color{green}{+0.90}}} & 30.19\textsubscript{{\color{green}{+10.49}}} \\
    
    {6.} & + \spatialanchor (Ours) \ & 22.60\textsubscript{{\color{green}{+3.37}}} & 30.49\textsubscript{{\color{green}{+10.79}}} \\
    \bottomrule
    
    
    \end{tabular}
    \caption{
        \textbf{Evaluating and Designing Spatial Controls (\cref{ssec:spatial_control}).} We overfit our multi-view model for $10$K iterations on 100 views of a single scene from our \upperbody dataset, and evaluate on 60 novel views of this scene by only varying the spatial controls (camera and \spatialanchor). We use this setup to test the modulation strength of different spatial controls. Subscript values \textcolor{green}{green}/\textcolor{red}{red} show deviations from Row 1. }
    \vspace{-10pt}
    \label{tab:3deval}
\end{table}



\begin{itemize}
    \item \textbf{No overfitting (Row 1).} The Mid-trained model without scene-specific overfitting achieves comparable PSNR of 19.2 and 19.7 on train and validation views respectively. We treat this setup as baseline to improve over. 

    \item \textbf{Encoding Camera with MLP (Row 2).} We encode camera using an MLP similar to prior works~\cite{zero1to3, MVDream} and our \midtraining stage (\cref{ssec:multiview_model}). After overfitting, the model achieves slightly better PSNR on training views as expected, however the validation PSNR drops by 1.28 points to 17.95. This suggests that an MLP does not provide enough modulation for camera control. 
    
    \item \textbf{\plucker as Positional Encoding (Row 3).} In this setup, we use downsampled and patchified \plucker coordinates processed through MLP to create positional encoding, which is added to the noisy latent tokens. This setup is inspired from  prior works~\cite{RayConditioningGAN,LFNs,kant2024spad,he2024cameractrl,bahmani2024vd3d}, and it further improves the validation PSNR compared to MLP at 18.89, but lags behind the non-overfitted baseline.

    \item \textbf{\plucker with ControlMLP and SIREN (Row 4, 5).} Here, we use our ControlMLP module to inject spatial control at each multiview-DiT block output. Moreover, encoding \plucker coordinate with a SIREN~\cite{SIREN} amplifies the relative differences between neighboring pixels (\cref{ssec:multiview_model}). This setup achieves PSNR of 20.13 with an improvement of 0.9 over baseline.

    \item \textbf{Adding \spatialanchor (Row 6).} Finally using the \spatialanchor gives validation PSNR of 22.6 (gain of 3.3 points over baseline) and enables strong spatial control. Thus, we adopt this configuration for \posttraining stage.
\end{itemize}

\subsection{Handling Varying Number of Views at Inference}\label{ssec:scaling_views}

As discussed in~\cref{ssec:multiview_model}, during training we jointly denoise a fixed number of views. Specifically, 24 views for mid-training at \resone, and 2 or 12 views for post-training at resolutions \resthree and \resfour respectively. This choice is largely motivated to avoid GPU out-of-memory errors during training. However, during inference, we wish to scale the number of views much further to generate smooth turnaround videos. This is feasible because we can run inference at half precision (using bfloat16) and do not need the backprop computation graph to be stored. 

We find that simply scaling the number of views (or tokens) during inference beyond 2\texttt{x} of the number of views during training leads to blurry and degraded generations. We find these degradations to be most significant in regions unspecified in the input, for example, the back of the head or ears as shown in~\cref{fig:qual_entropy}. We investigate this issue next, and introduce Attention Biasing to remedy it. 

At half-beat resolution

1024     0.03729058455519981 +- 0.008084586791244268
        Max:  0.06346470255685625       Min:  0.005513509114583333
4096     0.019980952965336557 +- 0.0037712786910756115
        Max:  0.027578312279776283      Min:  0.009592692057291666
\heading{Attention Biasing.} Let $\textbf{X} \in \mathbb{R}^{N \times d}$ denote the sequence of tokens denoised jointly, where $N, d$ represent number of tokens and the token dimension, respectively. In \ourmodel the total number of tokens in the sequence is proportional to number of views since the VAE and Patchify jointly compress the image by 16\texttt{x} in height and width. Within each DiT block, we compute the $\textrm{Attention}(\textbf{Q}, \textbf{K}, \textbf{V}) = \textbf{A}\textbf{V}$, where  $\textbf{Q},\textbf{K},\textbf{V}$ are query, key, value matrices and $\textbf{A}$ are attention scores computed via taking a row-wise softmax as follows:
\begin{equation}
    \textbf{A}_{i, j} = \frac{e ^{\lambda \textbf{Q}_{i}\textbf{K}_{j}^{\top}}}{\sum_{j' = 1}^{N} e ^{\lambda \textbf{Q}_{i}\textbf{K}_{j'}^{\top}}}, \label{attention map}
\end{equation}
Where the scaling factor $\lambda$ was proposed to be set to $1 / \sqrt{d}$ by the original work~\cite{vaswani2017attention} to stabilize the softmax operation as $d$ increases and avoiding biased (sharp) softmax distributions. We can quantify the sharpness of the attention by computing its entropy. Following previous works in NLP~\cite{ghader2017does,attanasio2022,zhang2024attention}, and in vision~\cite{jin2023training} we define entropy of attention as:
\begin{equation}
    \textrm{Ent}(\textbf{A}_{i}) = -\sum_{j=1}^{N} \textbf{A}_{i, j} \log (\textbf{A}_{i, j}), \label{entropy}
\end{equation}
By substituting the equation ~\eqref{attention map} in equation~\eqref{entropy}, authors of prior work~\cite{jin2023training} meticulously derive that entropy of attention grows logarithmically with number of tokens as follows: \(\textrm{Ent}(\textbf{A}_{i}) \propto \log N\). Furthermore, the authors show that this growth in entropy can be offset during inference by growing the scaling factor $\lambda$ as follows: 
\begin{equation}
    \begin{aligned}
    \lambda = \sqrt{\frac{1}{d}} \cdot \sqrt{\frac{\log N_i}{\log N_t}}
    \end{aligned}
\end{equation}
Where $N_t, N_i$ denote the number of tokens during training and inference respectively. For more details, please refer to Sections 3.1 and 3.2 of the original work~\cite{jin2023training}. Empirically, we find that having a slightly faster growing $\lambda$ alleviates the degradation better. Hence, we propose a hyperparameter $\gamma$ (growth factor) that is tuned between the range $[1.0,2.0]$ to control the growth of $\lambda$ as follows: 
\begin{equation}
    \begin{aligned}
    \lambda_\texttt{ours} = \sqrt{\frac{1}{d}} \cdot \sqrt{\frac{\gamma \cdot \log N_i}{\log N_t}}
    \end{aligned}
    \label{attn_bias}
\end{equation}
We follow the above Equation~\eqref{attn_bias} with our growth factor hyperparameter $\gamma$. In~\cref{fig:entropy}, we plot the entropy (Y-axis) during an inference pass of \ourmodel across varying number of views (tokens) being denoised and demonstrate the corresponding growth in entropy. Furthermore, we also show the attenuation in the entropy under increasing growth factors $\gamma$ (X-axis). 

We put generated visuals before and after using the suggested attention biasing in~\cref{fig:qual_entropy}. We put visuals sweeping over more values of the growth factor in Appendix~\cref{fig:qual_entropy_full}. Similar techniques have been explored in LLMs for handling and generating text at longer contexts~\cite{peng2023yarn,veličković2024softmax}, where the scaling factor $\lambda$ mentioned above is analogous to the inverse of the temperature scale. 

Additionally, we also found that using a bump function instead of constant Classifier-free Guidance during generation leads to fewer artifacts. We discuss this trick further in~\cref{app_sec:diffusion}.

\begin{figure}[h!]
    \centering    
    \hspace{-0.5cm}
    \includegraphics[width=\linewidth]{img/qual_entropy.pdf}
    \vspace{-0.2cm}
    \caption{{\textbf{Generations under varying strengths of growth factor $\gamma$ (\cref{ssec:scaling_views})}. On each row we show the generated views across vanilla attention~\cite{vaswani2017attention} (No scaling), prior work~\cite{jin2023training} and our formulation Eq.~\eqref{attn_bias}. It can be seen that growth factor ($\gamma$) greater than 1.0 is crucial to mitigate the entropy buildup. We show only 10 views per row subsampled evenly from 60 views generated at \resthree resolution. The model was trained to jointly denoised only 12 views ($N_i = 5*N_t$).}}
    \label{fig:qual_entropy}
\end{figure}



 




\subsection{Enhanced 3D Consistency Metric}
\label{ssec:consistencymetric}



Traditionally, the 3D consistency of multiview generation models is evaluated using 2D image metrics such as PSNR, LPIPS, and SSIM against a fixed set of ground truth images. However, this approach unfairly penalizes models that generate plausible and 3D-consistent novel content deviating from the fixed ground truth images.
Some works try to address this by measuring SfM~\cite{SFM,fridman2024scenescape} or epipolar error~\cite{muller2024multidiff,TrainNVSDM3}, but these methods solve for pose or are not robust since they measure against the entire epipolar line. To address these limiations, we use our GT camera poses as input and compute \reprojectionerror (RE) given our known camera poses and predicted correspondences.




Our computation of RE involves the following steps:
\begin{enumerate}
    \item \textbf{Landmarks and Correspondence estimation.} We use SuperPoint~\cite{detone2018superpoint} to detect landmarks in the generated images and employ SuperGlue~\cite{sarlin2020superglue} to establish pairwise correspondences between landmarks across images.
    \item \textbf{Triangulation.} Given the correspondences and camera parameters, we apply Triangulation based on Direct Linear Transformation (DLT)~\cite{Hartley2004} to obtain the corresponding 3D points for each landmark.
    \item \textbf{Reprojection and Error calculation.} We reproject these 3D points onto each image and compute the RE as the L2 distance between the original landmark and the reprojected 3D point normalized by image resolution, and average error across all images. 
\end{enumerate}
This approach evaluates multiview generation models by focusing on their ability to produce 3D-consistent results rather than adhering to a fixed ground truth. 
The \reprojectionerror provides a valuable basis for comparison across different methods. Furthermore, by computing RE on a set of real-world images that are independent of our generated images, we can establish a baseline that quantifies the error due to the noisy predictions from SuperGlue and SuperPoint rather than the quality of our generated images.

\heading{Naming Convention (RE@SG).} Please note that the use of SuperPoint~\cite{detone2018superpoint} and SuperGlue~\cite{sarlin2020superglue} estimation modules is a particular instantiation of our metric, and these could be replaced in the future with stronger counterparts such as MAST3R~\cite{dust3r_cvpr24,mast3r_arxiv24} or domain-specific keypoint detectors such as Sapiens~\cite{khirodkar2025sapiens}. Thus we use the naming convention of RE@SG to denote the Reprojection Error (RE) under SuperGlue (SG) estimation, which could be modified accordingly in the future for different estimators. 
























    
    
    


    

    
    
    

