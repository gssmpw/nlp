%--------------------
% Packages
% -------------------
\documentclass[11pt]{article}

\usepackage[utf8x]{inputenc}
\usepackage[T1]{fontenc}
\usepackage{booktabs} % For formal tables

\usepackage[vlined,ruled]{algorithm2e} % For algorithms
\renewcommand{\algorithmcfname}{ALGORITHM}

\SetAlFnt{\small}
\SetAlCapFnt{\small}
\SetAlCapNameFnt{\small}
\SetAlCapHSkip{0pt}
\IncMargin{-\parindent}


\usepackage{xcolor}         % colors
\usepackage{hyperref}

\newcommand{\declarecolor}[2]{\definecolor{#1}{RGB}{#2}\expandafter\newcommand\csname #1\endcsname[1]{\textcolor{#1}{##1}}}
\declarecolor{White}{255, 255, 255}
\declarecolor{Black}{0, 0, 0}
\declarecolor{Maroon}{128, 0, 0}
\declarecolor{Coral}{255, 127, 80}
\declarecolor{Red}{182, 21, 21}
\declarecolor{LimeGreen}{50, 205, 50}
\declarecolor{DarkGreen}{0, 80, 0}
\declarecolor{Purple}{146, 42, 158}
\declarecolor{Navy}{0, 0, 128}
\declarecolor{LightBlue}{84, 101, 202}
\definecolor{mydarkblue}{rgb}{0,0.08,0.45}
\hypersetup{ %
    pdftitle={},
    pdfkeywords={},
    pdfborder=0 0 0,
    pdfpagemode=UseNone,
    colorlinks=true,
    linkcolor=Navy,
    citecolor=DarkGreen,
    filecolor=Purple,
    urlcolor=Purple,
}


\usepackage{fullpage}

%\usepackage{gentium}
% \usepackage{mathptmx} % Use Times Font
\usepackage{amsthm}
\usepackage{amsmath}
\usepackage{amsfonts}
\usepackage{thm-restate}
% \usepackage{amssymb}

\usepackage{tikz}
\usetikzlibrary{shapes.geometric, arrows, positioning}

\usepackage{multirow}

%\usepackage{complexity}

\usepackage{nicefrac}

% \usepackage[swedish]{babel} % Swedish translations
\usepackage{enumitem} % Includes lists

\frenchspacing % No double spacing between sentences
% \linespread{1.2} % Set linespace
% \usepackage[a4paper, lmargin=0.1666\paperwidth, rmargin=0.1666\paperwidth, tmargin=0.1111\paperheight, bmargin=0.1111\paperheight]{geometry} %margins
% \usepackage{parskip}

% \makeatletter
% \def\thm@space@setup{%
%   \thm@preskip=\parskip \thm@postskip=0pt
% }
% \makeatother

\usepackage[all]{nowidow} % Tries to remove widows

\usepackage{lipsum} % Used for inserting dummy 'Lorem ipsum' text into the template

\usepackage[numbers]{natbib}
\bibliographystyle{plainnat}

%%% REVIEW
\newcommand{\tocite}{{\color{red}CITE} }
\newcommand{\toref}{{\color{red}REF} }

%%% LOGO
\newcommand{\usc}{\raisebox{-1pt}{\includegraphics[height=0.8em]{figures/usc_logo.png}}}
\newcommand{\vuam}{\raisebox{-1pt}{\includegraphics[height=0.8em]{figures/vu_logo.png}}}

%%% SIGNS and SYMBOLS
\newcommand{\grad}{\texttt{grad-CROP}}
\newcommand{\att}{\texttt{att-CROP}}
\newcommand{\seg}{\texttt{seg}}
\newcommand{\clip}{\texttt{clip-CROP}}
\newcommand{\sam}{\texttt{sam-CROP}}
\newcommand{\yolo}{\texttt{yolo-CROP}}
\newcommand{\hc}{\texttt{human-CROP}}
\newcommand{\zsvqa}{\texttt{ZSVQA}}
\newcommand{\vic}{\textbf{ViCrop}}
\newcommand{\xmark}{\text{\ding{55}}}
\newcommand{\cmark}{\text{\ding{51}}}
\newcommand{\success}{\texttt{\color{green} \cmark}}
\newcommand{\failure}{\texttt{\color{red} \xmark}}
\newcommand{\rel}{\texttt{rel-att}}
\newcommand{\gra}{\texttt{grad-att}}
\newcommand{\pgra}{\texttt{pure-grad}}
\newcommand{\relh}{\texttt{rel-att$^h$}}
\newcommand{\grah}{\texttt{grad-att$^h$}}
\newcommand{\pgrah}{\texttt{pure-grad$^h$}}


%%% Text Abb.
\makeatletter
\DeclareRobustCommand\onedot{\futurelet\@let@token\@onedot}
\def\@onedot{\ifx\@let@token.\else.\null\fi\xspace}

\def\aka{\emph{a.k.a}\onedot} \def\Eg{\emph{E.g}\onedot}
\def\eg{\emph{e.g}\onedot} \def\Eg{\emph{E.g}\onedot}
\def\ie{\emph{i.e}\onedot} \def\Ie{\emph{I.e}\onedot}
\def\cf{\emph{c.f}\onedot} \def\Cf{\emph{C.f}\onedot}
\def\etc{\emph{etc}\onedot} \def\vs{\emph{vs}\onedot}
\def\wrt{w.r.t\onedot} \def\dof{d.o.f\onedot}
\def\etal{\emph{et al}\onedot}
\makeatletter



\definecolor{myred}{HTML}{FF8577}
\definecolor{mygreen}{HTML}{0FA958}
\definecolor{myblue}{HTML}{1982C4}
\definecolor{codegreen}{rgb}{0,0.5,0}
\definecolor{codegray}{rgb}{0.5,0.5,0.5}
\definecolor{codepurple}{rgb}{0.07,0,0.53}
\definecolor{codered}{RGB}{189,41,0}
\definecolor{codecomment}{RGB}{153,153,153}
\definecolor{backcolour}{rgb}{0.96,0.96,0.96}
\definecolor{royalblue}{rgb}{0.0, 0.14, 0.4}
\definecolor{egyptianblue}{rgb}{0.06, 0.2, 0.65}
\definecolor{royalazure}{rgb}{0.0, 0.22, 0.66}
\definecolor{portlandorange}{rgb}{1.0, 0.35, 0.21}
\definecolor{sienna}{RGB}{183,105,68}
\definecolor{saddlebrown}{RGB}{139,69,19}
\definecolor{mediumbrown}{RGB}{83,41,11}
\definecolor{darkbrown}{RGB}{58,28,7}
\hypersetup{
    colorlinks=true,
    linkcolor=sienna,
    urlcolor=royalblue,
    citecolor=royalblue,
}

\usepackage[capitalize,noabbrev,nameinlink]{cleveref}

\usepackage{MnSymbol}

\renewcommand\vec\bm

\theoremstyle{plain}
\newtheorem{theorem}{Theorem}[section]
\newtheorem{proposition}[theorem]{Proposition}
\newtheorem{lemma}[theorem]{Lemma}
\newtheorem{corollary}[theorem]{Corollary}
\theoremstyle{definition}
\newtheorem{definition}[theorem]{Definition}
\newtheorem{assumption}[theorem]{Assumption}
\newtheorem{example}[theorem]{Example}
% \theoremstyle{remark}
\newtheorem{remark}[theorem]{Remark}

\usepackage{autonum}

\newcommand{\bhz}[1]{{ {\color{magenta}{[BZ:~#1]}}}}
\newcommand{\et}[1]{{ {\color{olive}{[ET:~#1]}}}}
\newcommand{\emin}[1]{{ {\color{orange}{[EB:~#1]}}}}
\newcommand{\ioannis}[1]{{ {\color{blue}{[IA:~#1]}}}}
\newcommand{\gabri}[1]{{ {\color{red}{[GF:~#1]}}}}

\usepackage{authblk}

\title{Learning and Computation of $\Phi$-Equilibria \\at the Frontier of Tractability}

\author[1]{Brian Hu Zhang\thanks{Equal contribution.}}
\author[1]{Ioannis Anagnostides$^*$}
\author[1,2]{Emanuel Tewolde}
\author[1,2]{Ratip Emin Berker}
\author[3]{Gabriele Farina}
\author[1,2,4]{Vincent Conitzer}
\author[1,5]{Tuomas Sandholm}
\renewcommand\Affilfont{\small}
\affil[1]{Carnegie Mellon University}
\affil[2]{Foundations of Cooperative AI Lab (FOCAL)}
\affil[3]{Massachusetts Institute of Technology}
\affil[4]{University of Oxford}
\affil[5]{Additional affiliations: Strategy Robot, Inc., Strategic Machine, Inc., Optimized Markets, Inc.}
\affil[ ]{}
\affil[ ]{\texttt{\{bhzhang,ianagnos,etewolde,rberker,conitzer,sandholm\}}\texttt{@cs.cmu.edu}, \texttt{gfarina}\texttt{@mit.edu}}

\begin{document} 

\maketitle

\pagenumbering{gobble}

\begin{abstract}
\emph{$\Phi$-equilibria}---and the associated notion of \emph{$\Phi$-regret}---are a powerful and flexible framework at the heart of online learning and game theory, whereby enriching the set of deviations $\Phi$ begets stronger notions of rationality. Recently, Daskalakis, Farina, Fishelson, Pipis, and Schneider (STOC '24)---abbreviated as DFFPS---settled the existence of efficient algorithms when $\Phi$ contains only linear maps under a general, $d$-dimensional convex constraint set $\mathcal{X}$. In this paper, we significantly extend their work by resolving the case where $\Phi$ is $k$-dimensional; degree-$\ell$ polynomials constitute a canonical such example with $k = d^{O(\ell)}$. In particular, positing only oracle access to $\cX$, we obtain two main positive results:
\begin{itemize}
    \item a $\text{poly}(n, d, k, \text{log}(1/\epsilon))$-time algorithm for computing $\epsilon$-approximate $\Phi$-equilibria in $n$-player multilinear games, and
    \item an efficient online algorithm that incurs  average $\Phi$-regret at most $\epsilon$ using $\text{poly}(d, k)/\epsilon^2$ rounds.
\end{itemize}
We also show nearly matching---up to constant factors in the exponents---lower bounds parameterized by $k$ in the online learning setting, thereby obtaining for the first time a family of deviations that captures the 
learnability of $\Phi$-regret.

From a technical standpoint, we extend the framework of DFFPS from linear maps to the more challenging case of maps with polynomial dimension. At the heart of our approach is a polynomial-time algorithm for computing an \emph{expected fixed point} of any $\phi: \cX \to \cX$---that is, a distribution $\mu \in \Delta(\cX)$ such that $\E_{\vec{x} \sim \mu}[ \phi(\vx) - \vx ] \approx 0$---based on the seminal \emph{ellipsoid against hope (EAH)} algorithm of Papadimitriou and Roughgarden (JACM '08). In particular, our algorithm for computing $\Phi$-equilibria is based on executing EAH in a nested fashion---each step of EAH itself being implemented by invoking a separate call to EAH.
\end{abstract}

\clearpage

\tableofcontents

\clearpage

\pagenumbering{arabic}

\section{Introduction}
\label{sec:intro}

\begin{figure*}[tb]
    \centering
    \includegraphics[width=0.848\linewidth]{figs/circuitnn.pdf} 
    \caption{Illustration of differentiable CircuitNN. CircuitNN is designed based on differentiable NAND gates. After DAS is guided by PI and PO pairs of the truth table, CircuitNN can get the precise circuit architecture logic equivalent to the truth table.}
    \label{fig:circuitnn}
\end{figure*}

% 1. Describe the importance of logic synthesis
% 2. Existing Problems
% (a) Neural Architecture Search: Unstable, Predefined Setting, etc.
% (b) Circuit Generation: Probabilistic Model, Logic Equivalence

With the rapid advancement of technology, the scale of integrated circuits (ICs) has expanded exponentially. 
This expansion has introduced significant challenges in chip manufacturing, particularly concerning power and area metrics.
A primary objective in IC design is achieving the same circuit function with fewer transistors, thereby reducing power usage and area occupancy.

Logic synthesis~\cite{hachtel2005logicsynth}, a critical step in electronic design automation (EDA), transforms behavioral-level circuit designs into optimized gate-level circuits, ultimately yielding the final IC layout. 
The primary goal of logic synthesis is to identify the physical implementation with the fewest gates for a given circuit function. 
This task constitutes a challenging NP-hard combinatorial optimization problem. 
Current logic synthesis tools~\cite{brayton2010abc, wolf2013yosys} rely on human-designed heuristics, often leading to sub-optimal outcomes.

Differentiable architecture search (DAS) techniques~\cite{liu2018darts, chu2020darts} offer novel perspectives on addressing challenges in this problem.
Circuit functions can be represented through truth tables, which map binary inputs to their corresponding outputs. 
Truth tables provide a precise representation of input-output relationships, ensuring the design of functionally equivalent circuits.
Inspired by this, researchers~\cite{deepmind2024ai4sys, wang2024tnet} have begun exploring the application of DAS to synthesize circuits directly from truth tables.
Specifically, \citet{deepmind2024ai4sys} proposed CircuitNN, a framework that learns differentiable connection structures with logic gates, enabling the automatic generation of logic circuits from truth tables.
This approach significantly reduces the complexity of traditional circuit generation. 
Building on this, \citet{wang2024tnet} introduced T-Net, a triangle-shaped variant of CircuitNN, incorporating regularization techniques to enhance the efficiency of DAS.

Despite these advancements, several challenges remain. 
The computational complexity of DAS grows quadratically with the number of gates, posing scalability issues.
Although triangle-shaped architecture~\cite{wang2024tnet} partially mitigates this problem, redundancy persists. 
%Additionally, DAS is susceptible to converging to local optima, limiting the ability to search architectures that satisfy the given truth tables~\cite{liu2018darts}. 
%Furthermore, hyperparameters (network depth and layer width) require extensive searches, introducing complexity and prolonging the synthesis process. 
Additionally, DAS is susceptible to converging to local optima~\cite{liu2018darts} and hyperparameters (network depth and layer width) require extensive searches. 
The challenges arise from the vast search space in DAS. 
% Even with predefined settings for CircuitNN, finding a configuration that meets the truth table requires extensive trial and error during the DAS process. 
Intuitively, limiting the search space through predefined parameters (network depth, gates per layer, and connection probabilities) can significantly reduce the complexity.

Recent advances~\cite{openai2023gpt4, abramson2024alphafold3, esser2024sd3, li2024mar} in conditional generative models have demonstrated remarkable performance across language, vision, and graph generation tasks. 
Motivated by these developments, we propose a novel approach to circuit generation that generates preliminary circuit structures to guide DAS in generating refined circuits matching specified truth tables. 
Firstly, we introduce CircuitVQ, a tokenizer with a discrete codebook for circuit tokenization. 
Built upon our Circuit AutoEncoder framework~\cite{hou2022graphmae,li2023maskgae,wu2025mgvga}, CircuitVQ is trained through a circuit reconstruction task. 
Specifically, the CircuitVQ encoder encodes input circuits into discrete tokens using a learnable codebook, while the decoder reconstructs the circuit adjacency matrix based on these tokens.
Subsequently, the CircuitVQ encoder serves as a circuit tokenizer for CircuitAR pretraining, which employs a masked autoregressive modeling paradigm~\cite{chang2022maskgit, li2023mage}. 
In this process, the discrete codes function as supervision signals. 
After training, CircuitAR can generate discrete tokens progressively, which can be decoded into initial circuit structures by the decoder of the CircuitVQ. 
These prior insights can guide DAS in producing refined circuits that match the target truth tables precisely.

Our key contributions can be summarized as follows:
\begin{itemize}
\item We introduce CircuitVQ, a circuit tokenizer that facilitates graph autoregressive modeling for circuit generation, based on our Circuit AutoEncoder framework;
\item Develop CircuitAR, a model trained using masked autoregressive modeling, which generates initial circuit structures conditioned on given truth tables;
\item Propose a refinement framework that integrates differentiable architecture search to produce functionally equivalent circuits guided by target truth tables;
\item Comprehensive experiments demonstrating the scalability and capability emergence of our CircuitAR and the superior performance of the proposed circuit generation approach.
\end{itemize}

% Motivation
% (a) Diffusion (Vision, Graph), Autoregressive (Language, Vision)
% (b) Circuit Generation for Predefined Setting
% (c) Neural Architecture Search for Strict Logic Equivalence

% Contribution
% (a) Circuit Tokenizer (new transformer arch, training strategy)
% (b) CircuitAR (train and gen strategies, post-ar strategy)
% (c) Extensive Evaluation including BitD (Bit Distance) for Scalability

\section{Preliminary}

\paragraph{Notation} Consider a sentence of $T$ tokens $\vx=\{\vx_1,\ldots, \vx_T\}\in\gX$, and let $P$ be the unknown target language distribution, $\tilde P(\vx)$ be the empirical distribution of the training data (which is an approximation of $P$), and $Q$ be the distribution of our model at hand. Since our paper is also closely related to RLHF, we will also use $\pi$ to represent the distributions. In particular, we sometimes write $\pi_\theta$ for a distribution that is parameterized by $\theta$, where $\theta$ is usually the set of trainable parameters of the LLM; we write $\pr$ for a reference distribution that should be clear given the context. The next token prediction loss is minimizing the forward-KL between $P$ and $Q$. 




\section{Sets of deviations with polynomial dimension}
\label{sec:deviations}

In this section, we formally introduce the assumptions we make concerning the feature map $m$ of~\Cref{assumption:kernel}, and we then provide a canonical example that satisfies our blanket assumptions.

\begin{assumption}
    \label{assumption:kernel-precise}
    We make the following assumptions regarding $\Phi^m$ and $m$ of~\Cref{assumption:kernel}:
    \begin{itemize}
        \item $m : \cX \to \R^{k'}$ is computable in $\poly(k)$ time.
        \item $\co m(\cB_1(\vec{0})) \supseteq \cB_\delta(\vec{0})$ for some $\delta \geq \poly(1/k)$.
        \item $\|m(\vx)\| \leq \poly(k)$ for all $\vx \in \cX$, and $m(\vec{0}) = \vec{0}$. 
        \item $\Phi^m$ contains the identity map.
    \end{itemize}
\end{assumption}

\begin{remark}[Functions on the vertices]
    Let $\cV$ be the set of extreme points of $\cX$. Our positive results (\Cref{theorem:main-eah-prec,theorem:main-prec}) only evaluate $\phi$ at extreme points, so they would operate identically if we instead defined our maps $\phi$ to be $\cV \to \cX$.% For expositional and notational simplicity, we will not do this.
\end{remark}

The definition above places some minimal assumptions on the feature mapping $m$ to ensure that $\Phi^m$ is geometrically well behaved. Indeed, we first show that the set of transformations $\Phi^m$ under~\Cref{assumption:kernel-precise} is well-bounded; the proof is provided in~\Cref{sec:wellbounded}.

\begin{restatable}{lemma}{wellbounded}
    \label{lemma:wellbounded}
    Let $\cX \subseteq \R^d$ be a convex and compact set such that $\cB_r(\vec{0}) \subseteq \cX \subseteq \cB_R(\vec{0})$, with $R \geq 1$ and $r < R$. Suppose further that $\|m(\vx)\| \leq M$ for all $\vx \in \cB_R(\vec{0})$, with $M = M(R) \geq 1$; $\co m(\cB_r(\vec{0})) \supseteq \cB_\delta(\vec{0})$ for some $\delta = \delta(r) > 0$; and $m(\vec{0}) = \vec{0}$. Then,
    \begin{equation}
        \cB_{r'}(\vec{0}) \subseteq \Phi^m \subseteq \cB_{ R'}(\vec{0}),
    \end{equation}
    where $r' \defeq \nicefrac{r}{2M(R)}$ and $R' \defeq R \left( \frac{2\sqrt{d}}{\delta(r)} + 1 \right)$.
\end{restatable}

We are now ready to provide a canonical, concrete example of deviations that satisfy~\Cref{assumption:kernel} under~\Cref{assumption:kernel-precise}. As we alluded to earlier in our introduction, it is the family of low-degree polynomials; in particular, it will be convenient to work with the Legendre basis.

\begin{definition}
    \label{def:polys}
    Let $P_0(x) = 1$ and $P_1(x) = x$. The $(\ell+1)$th \emph{Legendre polynomial} is given by the recurrence $(\ell + 1) P_{\ell + 1}(x) - (2\ell+1) x P_\ell(x) + \ell P_{\ell - 1}(x) = 0$.
\end{definition}

These polynomials have a convenient orthogonality property over $[-1, 1]$:
\begin{equation}
    \label{eq:orthog}
    \int_{-1}^1 P_{\ell}(x) P_{\ell'}(x) dx = 
    \begin{cases}
        \frac{1}{2\ell + 1} & \text{if } \ell = \ell',\\
        0 & \text{otherwise}.
    \end{cases}
\end{equation}
For convenience, we shall consider the rescaled polynomial $\lineP_\ell \defeq \sqrt{2\ell + 1} P_{\ell}$, so that $\int_{-1}^1 \lineP_\ell(x)^2 dx = 1$. We now define
\begin{equation}
    \label{eq:polynomials}
    m(\vx) \defeq \left( \prod_{j=1}^d \lineP_{\ell_j}(\vx[j]) \right)_{ 1 \leq \ell_1 + \dots \ell_d \leq \ell}.
\end{equation}

We establish (in~\Cref{sec:aux}) that~\Cref{assumption:kernel-precise} encompasses the above mapping.

\begin{restatable}{proposition}{lowdeg}
    \label{lemma:lowdeg}
    Let $m : \cX \to \R^{k'}$ per~\eqref{eq:polynomials}, where $k' = \binom{d + \ell}{\ell} - 1$. $\linem : \vx \mapsto m( \sqrt{d} \vx)$ satisfies~\Cref{assumption:kernel-precise} with $M \leq d^{O(\ell)}$  and $\delta = \nicefrac{1}{M}$.
\end{restatable}
\section{Polynomial-time expected fixed points and semi-separation}

We will now start connecting the framework we laid out in~\Cref{sec:eah} with the problem of computing $\Phi$-equilibria. That a $\Phi$-equilibrium can be cast as~\eqref{eq:eah}---by linking $\cY$ to the set of deviations $\Phi$---is not hard to see, and will be spelled out in the next section. This section concerns the question of implementing $\epsilon$-$\either$ (per~\Cref{def:either}), which is the main precondition of~\Cref{theorem:eah}.

The key to implementing the $\epsilon$-$\either$ oracle, and the main subject of this section, is the notion of an \emph{$\epsilon$-expected fixed point (EFP)} (\Cref{def:EFP}). This relaxes the usual notion of a fixed point that was employed by~\citet{Zhang24:Efficient}, who observed that, for minimizing $\Phi$-regret, one can replace a fixed-point oracle---as in the canonical framework of~\citet{Gordon08:No}---by an expected relaxation (per~\Cref{def:EFP}) thereof. This is crucial because, unlike fixed points which are intractable beyond linear maps, there is a simple, $O(1/\epsilon)$-time algorithm for computing $\epsilon$-expected fixed points. When it comes to computing $\Phi$-equilibria in games, our contribution here is twofold.

\begin{enumerate}
    \item we give a $\poly(d, \log(1/\epsilon))$-time algorithm for computing an $\epsilon$-expected fixed point, and\label{item:fast}
    \item we show that expected fixed points can be naturally coupled with the $\eah$ framework, and in particular, with the recent generalization of~\citet{Daskalakis24:Efficient}.\label{item:furious}
\end{enumerate}
This section establishes~\Cref{item:fast}, while the next section formalizes~\Cref{item:furious}. Going back to~\Cref{sec:eah} and the $\either$ oracle, the connection with (expected) fixed points lies in the observation that, when it comes to $\Phi$-equilibria in games, the $\ger$ part of the oracle can be implemented by computing an expected fixed point. This will become clear in the upcoming section.

\begin{definition}[Expected fixed points]
    \label{def:EFP}
    Let $\cX \subseteq \R^d$ be convex and compact and a function $\phi : \cX \to \cX$ to which we are given oracle access. The $\eps$-{\em expected fixed point (EFP)} problem asks for a distribution $\mu \in \Delta(\cX)$ such that\footnote{The choice of the $1$-norm (instead of, say, another $p$-norm) here is unimportant, because one can always take $\eps$ to be exponentially small.} 
    \begin{equation}
        \label{eq:EFP}
    \norm{\E_{\vx \sim \mu}[\phi(\vx)-\vx]}_1 \le \eps.
    \end{equation}
\end{definition}

This definition departs from the usual notion of a fixed point by measuring the fixed-point error \emph{in expectation} over samples $\vx$ from $\mu$. \Cref{def:EFP} is natural in its own right, but our key motivation is computational: as we shall see, an expected fixed point can be computed in polynomial time (\Cref{th:efp}).

We first observe that any function $\phi : \cX \to \cX$ admits an \emph{exact} expected fixed point; crucially, unlike Brouwer's fixed-point theorem, we do not assume that $\phi$ is continuous, and so expected fixed points exist even when fixed points do not.

\begin{proposition}
    Every function $\phi : \cX \to \cX$ admits an exact solution to the EFP problem.
\end{proposition}

\begin{proof}
    Let $G(\vx) := \phi(\vx) - \vx$. We want to show that $\co G(\cX)$ contains the origin. Suppose not. Then, by the separating hyperplane theorem, there is a vector $\vy \in \R^d$ with $\ip{\vy, G(\vx)} < 0$ for every $\vx \in \cX$. But this is impossible: simply consider $\vx^* \in \argmin_{\vx\in\cX} \ip{\vy, \vx}$.
\end{proof}

We now turn to our main computational result regarding EFPs; namely, a polynomial-time algorithm based on $\eah$---in particular, its incarnation in~\Cref{alg:eah}.

\begin{theorem}
    \label{th:efp}
    Given oracle access to $\cX$ and any $\phi : \cX \to \cX$, there exists a $\poly(d, \log(1/\eps))$-time algorithm that computes an $\epsilon$-EFP of $\phi$.
\end{theorem}

\begin{proof}
    We observe that an EFP can be equivalently expressed through the optimization problem
    \begin{equation}
    \qq{find} \mu \in \Delta(\cX) \qq{s.t.} \E_{\vx \sim \mu} \ip{\vy, \phi(\vx) - \vx } \ge 0 \quad \forall \vy \in [-1, 1]^d.
\end{equation}
We will now apply~\Cref{theorem:eah}. The set $\cY \defeq [-1, 1]^d$ clearly admits a separation oracle ($\sep$). Further, for any $\vy \in [-1, 1]^d$, taking $\vx^* = \argmin_{\vx \in \cX} \langle \vy, \vx \rangle $ (using an optimization oracle for $\cX$) guarantees $\langle \vy, \phi(\vx^*) - \vx^* \rangle \geq 0 $ since $\phi(\vx^*) \in \cX$, thereby implementing the $\ger$ oracle. We thus find that the preconditions of~\Cref{theorem:eah} are satisfied, and $\eah$ (\Cref{alg:eah}) returns $\mu \in \Delta(\cX)$, with $\supp(\mu) \leq \poly(d, \log(1/\epsilon))$, such that
\begin{equation}
    \E_{\vx \sim \mu} \langle \vy, \phi(\vx) - \vx \rangle \geq - \epsilon \quad \forall \vy \in [-1, 1]^d.
\end{equation}
Taking $\vy = \sign( \E_{\vx \sim \mu} (\vx - \phi(\vx)))$ (coordinate-wise) completes the proof.
\end{proof}

As it will become clear, \Cref{th:efp} yields a polynomial-time implementation of the $\ger$ oracle in the context of~\Cref{sec:eah}, which can be employed in~$\eah$. With a slight modification in the proof of~\Cref{th:efp}, we shall see how one can also recover an $\epsilon$-$\either$ oracle (\Cref{def:either}), which will then enable us to harness~\Cref{theorem:either} for computing $\Phi$-equilibria in games. Following the nomenclature of~\citet{Daskalakis24:Efficient}, we refer to this oracle as a \emph{semi-separation oracle}.

\begin{definition}[Semi-separation oracle]
    \label{def:semiseparation}
    The \emph{semi-separation} problem is the following. Given a convex and compact $\cX$ and a function $\phi : \cX \to \R^d$, compute
    \begin{enumerate}
        \item {\em either} a distribution $\mu \in \Delta(\cX)$ such that $\|\E_{\vx \sim \mu}[\phi(\vx)-\vx] \|_1 \leq \epsilon$, \label{item:efp}
        \item {\em or} a point $\vx\in\cX$ with $\phi(\vx) \notin \cX$.\label{item:mem}
    \end{enumerate}
\end{definition}

Unlike~\Cref{def:EFP}, here we allow $\phi$ to map outside of $\cX$. This more general framing is essential to arrive at the $\either$ oracle. In particular, we note that~\Cref{item:mem} yields a hyperplane separating $\phi$ from the set of endomorphisms on $\cX$. Namely, since $\phi(\vx) \notin \cX$, we can use the separation oracle on $\cX$ to separate $\cX$ from $\phi(\vx)$; that is, there is a $\vec{w}$ such that $\langle \phi(\vx), \vec{w} \rangle > \langle \vx, \vec{w} \rangle$ for all $\vx \in \cX$. But this also implies that $\langle \phi(\vx), \vec{w} \rangle > \langle \phi'(\vx), \vec{w} \rangle$ for any endomorphism $\phi'$, as promised.

\begin{theorem}
    \label{theorem:semiseparation}
    Given oracle access to $\cX$ and $\phi$, there is a $\poly(d, \log(1/\epsilon))$-time algorithm for implementing the semi-separation oracle of~\Cref{def:semiseparation}.
\end{theorem}

\begin{proof}
As in the proof of~\Cref{th:efp}, we proceed by running the ellipsoid algorithm (per~\Cref{alg:eah}) on the problem
\begin{align}
    \qq{find} \vy\in [-1,1]^d \qq{s.t.} \ip{\vy, \phi(\vx) - \vx} \le -\eps \quad \forall \vx \in \cX. \label{eq:efp-constraint}
\end{align}
For any $\vy \in [-1, 1]^d$ during the execution of the ellipsoid, take $\vx^*(\vy) \in \argmin_{\vx \in \cX} \ip{\vy, \vx}$. If $\phi(\vx^*(\vy)) \notin \cX$, the algorithm can terminate and return $\vx^*(\vy)$. Otherwise, it follows that $\langle \vy, \phi(\vx^*(\vy)) - \vx^*(\vy) \rangle \geq 0$, by definition of $\vx^*$, and so we can use $\vx^*$ to get a separation oracle for~\eqref{eq:efp-constraint}.

Now, if every $\vx^*(\vy) \in \cX$ generated above satisfies the constraint $\phi(\vx^*(\vy)) \in \cX$, then~\Cref{alg:eah} returns a certificate of infeasibility for~\eqref{eq:efp-constraint} in $\poly(d, \log(1/\epsilon))$ time, which is an $\epsilon$-expected fixed point of $\phi$. On the other hand, if at some point there is $\vy \in [-1, 1]^d$ such that $\phi(\vx^*(\vy)) \notin \cX$, then the algorithm returns a point $\vx^*(\vy) \in \cX$ such that $\phi(\vx^*(\vy)) \notin \cX$. This completes the proof.
\end{proof}

This semi-separation oracle amounts to the $\epsilon$-$\either$ oracle needed in~\Cref{theorem:either}, as we shall see next in the context of games. Compared to the semi-separation oracle of~\citet{Daskalakis24:Efficient} that only works for linear functions, ours (\Cref{theorem:semiseparation}) places no restrictions on $\phi$. 

\section{A polynomial-time algorithm for $\Phi^m$-equilibria in games}

Armed with the powerful semi-separation oracle of~\Cref{theorem:semiseparation}, we now establish a polynomial-time algorithm for computing $\Phi^m$-equilibria in general multilinear games (\Cref{theorem:main-eah-prec}).

Let us recall the basic setting of an $n$-player multilinear game $\Gamma$. Each player $i \in [n]$ has a convex and compact strategy set $\cX_i \subseteq \R^{d_i}$ in isotropic position (\Cref{sec:prel}). Player $i$ has a utility function $u_i : \cX_1 \times \dots \times \cX_n \to \R$ that is linear in $\cX_i$, so that $u_i(\vx) = \langle \vec{g}_i, \vx_i \rangle$ for some $\vec{g}_i = \vec{g}_i(\vx_{-i}) \in \R^{d_i}$. %Without loss of generality, we assume that $\vec 0 \notin \op{aff} \cX_i$, the affine hull of $\cX_i$, so that linear and affine functions on $\cX_i$ are interchangeable.
Furthermore, for each player $i \in [n]$, we let $\Phi^{m_i} \subseteq \cX_i^{\cX_i}$ be the $k_i$-dimensional set of deviations in the sense of~\Cref{assumption:kernel}; that is, there exists a function $m_i \in \cX_i \to \R^{k_i'}$, with $k_i = k_i' \cdot d_i + d_i$, such that for each $\phi_i \in \Phi^{m_i}$ and $\vx_i \in \cX_i$, the function output $\phi_i(\vx_i)$ can be expressed as the matrix-vector product $\mK_i(\phi_i) m_i(\vx_i) + \cons_i$ for some matrix $\mK_i \in \R^{d_i \times k_i'}$ and $\cons_i \in \R^{d_i}$. It is assumed throughout that $\Phi^{m_i}$ contains the identity map. For notational simplicity, we let $k \defeq \sum_{i=1}^n k_i$ and $d \defeq \sum_{i=1}^n d_i$.

In this context, we next state the main result of this section, and proceed with its proof.

\begin{theorem}[Precise version of~\Cref{theorem:main-eah}]
    \label{theorem:main-eah-prec}
    Consider an $n$-player multilinear game $\Gamma$ such that, for each player $i \in [n]$, we are given $\poly(n, k)$-time algorithms for the following:
\begin{itemize}
\item an oracle to compute the gradient, that is, the vector $\vec{g}_i = \vec{g}_i(\vx_{-i}) \in \R^{d_i}$ for which $\ip{\vg_i(\vx_{-i}), \vx_i} = u_i(\vx)$ for all $\vx \in \cX_1 \times \dots \times \cX_n$ (polynomial expectation property); and
\item a membership oracle for the strategy set $\cX_i$, assumed to be in isotropic position.
\end{itemize}
Suppose further that each $k_i$-dimensional set $\Phi^{m_i}$ satisfies~\Cref{assumption:kernel-precise} and $\|\vec{g}_i \| \leq B$. Then, an $\eps$-approximate $\Phi^m$-equilibrium of $\Gamma$ can be computed in $\poly\qty(n, k, \log(B/\eps))$ time.
\end{theorem}

\begin{proof}
An $\eps$-approximate $\Phi^m$-equilibrium of $\Gamma$  is a distribution $\mu \in \Delta(\cX_1 \times \dots \times \cX_n)$ such that
\begin{align}
\E_{\vx \sim \mu} \qty[u_i(\phi_i(\vx_i), \vx_{-i}) - u_i(\vx)] \le \eps\label{eq:phi-eqm}
\end{align}
for every player $i \in [n]$ and deviation $\phi_i \in \Phi^{m_i}$. Using multilinearity and~\Cref{assumption:kernel}, it suffices to find a distribution $\mu\in\Delta(\cX_1 \times \dots \times \cX_n)$ satisfying
\begin{align}
\E_{\vx \sim \mu} \left[ \sum_{i =1}^n \ip{\vg_i(\vx_{-i}), \mK_i m_i(\vx_i) + \cons_i - \vx_i} \right] \le \eps \label{eq:phi-sp}
\end{align}
for every $(\mK_1(\phi_1), \dots, \mK_n(\phi_n))$ and $(\cons_1(\phi_1), \dots, \cons_n(\phi_n))$, where $(\phi_1, \dots, \phi_n) \in \Phi^{m}$. (This derivation uses the fact that $\Phi^{m_i}$ contains the identity map.) We will now apply~\Cref{theorem:either} with respect to $\R^d \supseteq \cX \defeq \cX_1 \times \dots \times \cX_n$ and 
$$\R^{k} \supseteq \cY \defeq \{ (\mK_1, \cons_1, \dots, \mK_n, \cons_n) : \mK_i m_i(\vx_i) + \cons_i \in \cX_i \quad \forall \vx_i \in \cX_i \}. $$
By the polynomial expectation property, we can evaluate the term $\sum_{i =1}^n \ip{\vg_i(\vx_{-i}), \mK_i m_i(\vx_i) + \cons_i - \vx_i}$, for each $\vx \in \cX$, in $\poly(n, k)$ time. It thus suffices to show how to implement the $\epsilon$-$\either$ oracle, which yields a separation oracle for the program
\begin{alignat}{9}
\qq{find} \mK_1, \cons_1, \dots, \mK_n, \cons_n \qq{s.t.} \\
\sum_{i =1}^n \ip{\vg_i(\vx_{-i}), \mK_i \vm_i(\vx_i) + \cons_i - \vx_i} \ge - \eps &\quad\forall \vx \in \cX_1 \times \dots \times \cX_n, \label{eq:phi-sp}
\\
\mK_i \vm_i(\vx_i) + \cons_i \in \cX_i &\quad \forall\vx_i \in \cX_i.
\end{alignat}

Consider any $\R^{k} \ni \phi = (\mK_1, \cons_1, \dots, \mK_n, \cons_n)$. We apply the semi-separation oracle of~\Cref{theorem:semiseparation} for each function $\vx_i \mapsto \mK_i m_i(\vx_i) + \cons_i$. This returns \emph{either} an $\epsilon'$-expected fixed point, that is, a distribution $\nu_i \in \Delta(\cX_i)$ such that $$\left\| \E_{\vx_i \sim \nu_i} [ \mK_i \vm_i(\vx_i) + \cons_i - \vx_i ] \right\|_1 \leq \epsilon',$$
\emph{or} a point $\vx_i \in \cX_i$ such that $\mK_i m_i(\vx_i) + \cons_i \notin \cX_i$. If any of those semi-separation oracles returned $\vx_i \in \cX_i$ with $\mK_i m_i(\vx_i) + \cons_i \notin \cX_i$, we can use it to obtain a hyperplane separating $(\mK, \cons)$ from the set of deviations $\cY$. Otherwise, let $\nu \defeq \nu_1 \times \dots \times \nu_n \in \Delta(\cX_1) \times \dots \times \Delta(\cX_n)$ be the induced product distribution. Then, we have
\begin{align}
    & \E_{\vx \sim \nu}\sum_{i = 1}^n \ip{\vg_i(\vx_{-i}), \mK_i \vm_i(\vx_i) + \cons_i - \vx_i} 
    = \sum_{i =1}^n  \big\langle\E_{\vx \sim \nu} \vg_i(\vx_{-i}), \E_{\vx_i \sim \nu_i}[\mK_i \vm_i(\vx_i) + \cons_i - \vx_i]\big\rangle \leq n B \eps', \label{eq:eqm-final}
\end{align}
where we used the fact that $\nu$ is a product distribution in the equality above. Thus, we have identified an $(\epsilon' n B)$-approximate good-enough-response, yielding an $\epsilon$-$\either$ oracle by rescaling $\epsilon'$, and the proof follows from~\Cref{theorem:either}.
\end{proof}

\begin{algorithm}[!ht]
\caption{Polynomial-time algorithm for $\Phi^m$-equilibria}
\label{alg:main-eah}
\SetKwInOut{Input}{Input}
\SetKwInOut{Output}{Output}
\SetKw{Input}{Input:}
\SetKw{Output}{Output:}
\Input{
    \begin{itemize}[noitemsep,topsep=0pt]
        \item An $n$-player multilinear game $\Gamma$
        \item A precision parameter $\epsilon > 0$
        \item A membership oracle for each $\cX_i$
        \item An oracle for computing the gradient $\vec{g}_i = \vec{g}_i(\vx_{-i}) \in \R^{d_i}$ for each $i \in [n]$
        \item A $k_i$-dimensional set $\Phi^{m_i}$ under~\Cref{assumption:kernel-precise} for each $i \in [n]$
    \end{itemize}
}
\Output{An $\epsilon$-approximate $\Phi^m$-equilibrium of $\Gamma$ in $\poly(k, \log(1/\epsilon))$ time}\\
Define $G : \R^{d} \to \R^k$ such that $\langle G(\vx), (\mK, \cons) \rangle = \sum_{i =1}^n \ip{\vg_i(\vx_{-i}), \mK_i m_i(\vx_i) + \cons_i - \vx_i}$\;
Use the semi-separation oracle of~\Cref{theorem:semiseparation} to construct an $\epsilon$-$\either$ oracle $\cO$\;
 Apply~\Cref{alg:gen-eah} with $\cO$ as the $\epsilon$-$\either$ oracle
\end{algorithm}

It is worth stressing that it is crucial for our proof that the \emph{expected} VI problem (\emph{cf}.~\citet{Zhang25:Expected}) above corresponds to a game. It allows each player to be treated {\em independently}, which yields a {\em product distribution} $\nu = \nu_1 \times \dots \times \nu_n$ when we apply the semi-separation oracle of~\Cref{theorem:semiseparation} (for each player). That $\nu$ is a product distribution is crucial to implement the separation oracle for the dual because it allows us to push the expectation into the inner product in \eqref{eq:eqm-final}, as we saw in the last step of the proof.


\section{An efficient online algorithm for minimizing $\Phi^m$-regret}
\label{sec:reg}

We now switch gears to the online learning setting, recalled in~\Cref{sec:gordon}. Our main result, \Cref{theorem:main-prec}, is an efficient online algorithm for minimizing $\Phi^m$-regret with respect to any $\poly(d)$-dimensional set $\Phi^m$ (under~\Cref{assumption:kernel-precise}), which applies even in the adversarial regime.

In what follows, we build on the framework of~\citet{Daskalakis24:Efficient}, itself a refinement of the template of~\citet{Gordon08:No}. As we have seen, \citet{Daskalakis24:Efficient} showed that separating even over the set of linear endomorphisms is hard. In light of this, they proceed as follows. Instead of operating over the set of linear endomorphisms, their key idea is to consider a sequence of ``shell sets,'' each of which contains the original set. Each shell set must also satisfy two basic properties:
\begin{itemize}
    \item it is sufficiently structured so that it is possible to optimize over that set, and
    \item it contains a transformation with a fixed point inside $\cX$.
\end{itemize}
Here, we show that by replacing fixed points with \emph{expected} fixed points in the above template, it is possible to extend their main result to handle any $\poly(d)$-dimensional set under~\Cref{assumption:kernel-precise}.

\paragraph{Overview} Our main construction is~\Cref{alg:main}. It is an instantiation of~$\shellgd$ (\Cref{sec:shellgd}), which is projected gradient descent but with the twist that the constraint set is changing over time---reflecting the fact that a new shell set is computed at every round. To execute $\shellgd$, $\shellproj$ (\Cref{sec:shellproj}) provides an efficient projection oracle together with an approximate expected fixed point thereof, which is ultimately the output of our $\Phi^m$-regret minimizer. $\shellproj$ crucially relies on $\shellelips$, introduced next in~\Cref{sec:shellellips}. It strengthens our semi-separation oracle of~\Cref{theorem:semiseparation} by again using expected fixed points. \Cref{sec:put} combines those ingredients to arrive at our main result (\Cref{theorem:main-prec}).

\subsection{Shell ellipsoid}
\label{sec:shellellips}

Continuing from our semi-separation oracle of~\Cref{theorem:semiseparation}, $\shellelips$ (\Cref{alg:shellellipsoid}) takes a step further: it takes as input a convex set of transformations $\cF \subseteq \cB_D(\vec{0})$---for which we have efficient oracle access, unlike $\enfuns$---and returns \emph{either} a function $\phi \in \cF$ and an $\epsilon$-expected fixed point thereof in $\Delta(\cX)$, \emph{or} it provides a certificate---in the form of a polytope expressed as the intersection of a polynomial number of halfspaces---establishing that $\vol(\cF \cap \enfuns) \approx 0$. $\shellelips$ will be used later as part of the $\shellproj$ algorithm so as to shrink the shell set.

\begin{lemma}
    \label{lemma:shellellipsoid}
    For any $k$-dimensional convex set $\cF \subseteq \cB_{D}(\vec{0})$ with efficient oracle access and $\epsilon > 0$, $\shellelips(\cF)$ (\Cref{alg:shellellipsoid}) runs in time $\poly(k, \log(1/\epsilon), \log D)$, and 
    \begin{itemize}
        \item either it returns a transformation $\phi \in \cF$ with an $\epsilon$-expected fixed point in $\cX$,
        \item or it returns a polytope $\cQ \subseteq \R^k$, specified as the intersection of at most $\poly( k, \log(1/\epsilon), \log D)$ halfspaces, such that $\Phi^m \subseteq \cQ$ and $\vol(\cQ \cap \cF) < \epsilon$.
    \end{itemize}
\end{lemma}

Coupled with~\Cref{theorem:semiseparation} pertaining to the semi-separation oracle, the proof of correctness of~\Cref{lemma:shellellipsoid} is immediate. That $\cQ$ can be expressed as the intersection of a polynomial number of halfspaces follows from the usual analysis of ellipsoid, as in~\citet[Lemma 4.2]{Daskalakis24:Efficient}.

\begin{algorithm}[!ht]
\caption{$\shellelips(\cF)$}
\label{alg:shellellipsoid}
\SetKwInOut{Input}{Input}
\SetKwInOut{Output}{Output}
\SetKw{Input}{Input:}
\SetKw{Output}{Output:}
\Input{
    \begin{itemize}[noitemsep,topsep=0pt]
        \item Oracle access to convex set $\cX \subseteq \R^d$
        \item Oracle access to a $k$-dimensional convex set $\cF \subseteq \cB_D(\vec{0})$
        \item Precision parameter $\epsilon > 0$
    \end{itemize}
}
Initialize $\cE \defeq \cB_D(\vec{0})$ and $\cQ \defeq \R^k$\;
 \While{$\vol(\cE) \geq \epsilon$} {
    Set $\phi \in \cQ \cap \cF$ as the center of $\cE$\;
    Run the semi-separation oracle of~\Cref{theorem:semiseparation} with respect to $\phi$\;
    \If{it returned an $\epsilon$-expected fixed point $\mu \in \Delta(\cX)$ of $\phi$}{
        \textbf{return} $\phi$\;
    }
    \Else{
        Let $H$ be the halfspace returned by~\Cref{theorem:semiseparation} that separates $\phi$ from $\enfuns$\;
        Set $\cQ \defeq \cQ \cap H$\;
    }
    Set $\cE$ to be the minimum volume ellipsoid containing $\cQ \cap \cF$
 }
 \textbf{return} $\cQ$
\end{algorithm}

\subsection{Shell gradient descent}
\label{sec:shellgd}

Instead of minimizing external regret with respect to the set $\enfuns$, which is hard even under linear endomorphisms~\citep{Daskalakis24:Efficient}, the overarching idea is to run (projected) gradient descent but with respect to a sequence of changing shell sets, $(\tilY^{(t)})_{t=1}^T$, of $\enfuns$ (each of which contains $\enfuns$); this process, called~$\shellgd$, is given in~\Cref{alg:shellgd}. So long as $\enfuns \subseteq \tilY^{(t)}$, $\shellgd$ indeed minimizes external regret with respect to deviations in~$\enfuns$---of course, $\shellgd$ is not a genuine regret minimizer for $\enfuns$ in that it is allowed to output strategies not in $\enfuns$, but~\Cref{lemma:shellgd} below is in fact enough for the purpose of minimizing $\Phi^m$-regret.

\begin{lemma}[\citep{Daskalakis24:Efficient}]
    \label{lemma:shellgd}
    Suppose that the sequence of shell sets $(\tilY^{(t)})_{t=1}^T$ is such that $\enfuns \subseteq \tilY^{(t)} \subseteq \cB_D(\vec{0})$ for all $t \in [T]$. For any sequence of utilities $\vec{U}^{(1)}, \dots, \vec{U}^{(T)} \in [-1, 1]^k$, $\shellgd$ (\Cref{alg:shellgd}) satisfies
    \begin{equation}
        \max_{\vy^* \in \enfuns} \sum_{t=1}^T \langle \vy^* - \vy^{(t)}, \vec{U}^{(t)} \rangle \leq \frac{D^2}{2\eta} + \eta \sum_{t=1}^T \| \vec{U}^{(t)} \|^2.
    \end{equation}
\end{lemma}

\begin{algorithm}[!ht]
\caption{$\shellgd$~\citep{Daskalakis24:Efficient}}
\label{alg:shellgd}
\SetKwInOut{Input}{Input}
\SetKwInOut{Output}{Output}
\SetKw{Input}{Input:}
\SetKw{Output}{Output:}
\Input{Learning rate $\eta$, convex and compact sets $\tilY^{(1)}, \dots, \tilY^{(T)} \subseteq \cB_D(\vec{0})$}\;
Initialize $\vy^{(0)} \in \tilY^{(1)}$ and $\vec{U}^{(0)} \defeq \vec{0}$\;
 \For{$ t=1, \dots, T$} {
    Obtain efficient oracle access to $\tilY^{(t)}$\;
    Update $\vy^{(t)} \defeq \Pi_{\tilY^{(t)}}( \vy^{(t-1)} + \eta \vec{U}^{(t-1)})$\;
    Output $\vy^{(t)}$ as the next strategy and receive feedback $\vec{U}^{(t)} \in [-1, 1]^k$
 }
\end{algorithm}

\subsection{Shell projection}
\label{sec:shellproj}

To implement $\shellgd$, we will make use of $\shellproj$, the algorithm that is the subject of this subsection. There are two main desiderata for the sequence of shell sets taken as input in $\shellgd$. First, each shell set must be structured or simple enough to allow projecting onto it---this is the whole rationale of expanding $\enfuns$ through shell sets. But, of course, this is not enough, for one could just consider the entire space. The second crucial concern is that each transformation given by~$\shellgd$ needs to admit (approximate) expected fixed points, so as to use the framework of~\citet{Gordon08:No} (\Cref{theorem:gordon}) and minimize $\Phi^m$-regret. \Cref{lemma:shellproj} below, concerning $\shellproj$, shows how to accomplish that goal; its proof is similar to that of~\citet[Theorem 4.4]{Daskalakis24:Efficient}.

\begin{lemma}
    \label{lemma:shellproj}
    Let $\cX$ be a convex and compact set such that $\cB_{r}(\vec{0}) \subseteq \cX \subseteq \cB_R(\vec{0})$ and $\cM$ be a convex set such that $\enfuns \subseteq \cM \subseteq \cB_D(\vec{0})$. For any $\phi \in \cB_D(\vec{0}) \subseteq \R^k$ and $\epsilon > 0$, $\shellproj$ (\Cref{alg:shellproj}) runs in time $\poly(k, 1/\epsilon, R/r, D)$ and returns
    \begin{enumerate}
        \item a shell set $\tilPhi$ satisfying $\enfuns \subseteq \tilPhi$, expressed by intersecting $\cM$ with at most $\poly(d, k, 1/\epsilon, R/r, D)$ halfspaces, and\label{item:invar}
        \item a transformation $\tilphi \in \tilPhi$ such that $\| \tilphi - \Pi_{\tilPhi}(\phi) \| \leq \epsilon$, together with an $\epsilon$-expected fixed point of $\tilphi$, $\mu \in \Delta(\cX)$.\label{item:proj}
    \end{enumerate}
\end{lemma}

\begin{algorithm}[!ht]
\caption{$\shellproj_\Phi(\phi)$ projects $\phi$ to a shell of $\Phi$}
\label{alg:shellproj}
\SetKwInOut{Input}{Input}
\SetKwInOut{Output}{Output}
\SetKw{Input}{Input:}
\SetKw{Output}{Output:}
\Input{
    \begin{itemize}[noitemsep,topsep=0pt]
        \item Convex and compact set $\cX \subseteq \R^d$ such that $\cB_r(\vec{0}) \subseteq \cX \subseteq \cB_R(\vec{0})$
        \item Convex set $\cM$ such that $\enfuns \subseteq \cM \subseteq \cB_D(\vec{0})$
        \item Transformation $\phi \in \cB_D(\vec{0})$
        \item Precision parameter $\epsilon > 0$
    \end{itemize}
}
\Output{
    \begin{itemize}[noitemsep,topsep=0pt]
        \item Convex set $\tilPhi$ such that $\enfuns \subseteq \tilPhi \subseteq \cM$
        \item Transformation $\tilphi \in \tilPhi$ such that $\| \tilphi - \Pi_{\tilPhi}(\phi) \| \leq \epsilon$
        \item An $\epsilon$-expected fixed point $\mu \in \Delta(\cX)$ of $\tilphi$
    \end{itemize}
}
Set $\epsilon' = \frac{\epsilon r}{32 M(R) D^2}$\;
Initialize $\tilPhi \defeq \cM$\;
 \For{$q = 0, \dots$ incremented by $\delta \defeq \nicefrac{\epsilon}{4D} $} {
    Run $\shellelips( \tilPhi \cap \cB_q(\phi))$ with precision $\vol(\cB_{\epsilon'}(\cdot))$\;
    \If{it finds $\tilphi$ with an $\epsilon$-expected fixed point $\mu \in \Delta(\cX)$}{
        \textbf{return} $\tilPhi, \tilphi, \mu$
    }
    \Else{
        Let $\cQ$ be the polytope returned by $\shellelips$\;
        Set $\tilPhi \defeq \tilPhi \cap \cQ$
    }
 }
\end{algorithm}

\subsection{Putting everything together}
\label{sec:put}

We now combine all the previous pieces to obtain an efficient algorithm for minimizing $\Phi^m$-regret---when $\Phi^m$ is $\poly(d)$-dimensional---under a general convex and compact set $\cX$. The overall construction is depicted in~\Cref{alg:main}. In effect, it runs $\shellgd$ with respect to the sequence of shell sets $(\tilPhi^{(t)} )_{t=1}^T$. Indeed, by the correctness guarantee of $\shellproj$ (\Cref{item:invar} of~\Cref{lemma:shellproj}), we have the invariance $\Phi(\cX) \subseteq \tilPhi^{(t)}$ for all $t \in [T]$. Furthermore, \Cref{item:proj} of~\Cref{lemma:shellproj} implies that $(\mK^{(t+1)}, \cons^{(t+1)}) \in \tilPhi^{(t+1)}$, returned by $\shellproj$ in~\Cref{alg:main}, is within distance $\epsilon$ of the projection prescribed by~$\shellgd$. As a result, we can apply~\Cref{lemma:shellgd} (up to some some slackness proportional to $\epsilon$) to bound the external regret $\reg^{(T)}_{\Phi^m}$ of $((\mK^{(t)}, \cons^{(t)}))_{t=1}^T$ with respect to comparators from $\enfuns$; combined with the fact that $\mu^{(t)} \in \Delta(\cX)$ is an $\epsilon$-expected fixed point of the function $\vx \mapsto \mK^{(t)} m(\vx) + \cons^{(t)}$ (as promised by~\Cref{item:proj}), it follows that the $\Phi^m$-regret of the learner (\Cref{alg:main}) can be bounded by $\reg^{(T)}_{\Phi^m} + \epsilon T$ (as in~\Cref{theorem:gordon}). We thus arrive at our main result.

\begin{theorem}[Precise version of~\Cref{theorem:main1}]
    \label{theorem:main-prec}
    Let $\cX \subseteq \R^d$ be a convex and compact set in isotropic position for which we have a membership oracle. \Cref{alg:main} has per-round running time of $\poly(k, T)$ and guarantees average $\Phi^m$-regret of at most $\poly(k) / \sqrt{T}$, where $k$ is the dimension of $\Phi^m$ under~\Cref{assumption:kernel-precise}.
\end{theorem}

Unlike the algorithm of~\citet{Daskalakis24:Efficient}, a salient aspect of~\Cref{alg:main} is that it outputs a sequence of \emph{mixed} strategies in $\Delta(\cX)$. As we saw earlier in~\Cref{sec:gordon}, this turns out to be necessary: \citet{Zhang24:Efficient} showed that a learner restricted to output strategies in $\cX$ cannot efficiently minimize $\Phi$-regret even with respect to low-degree swap deviations (assuming $\PPAD \neq \P$).

\begin{algorithm}[!ht]
\caption{$\Phi^m$-regret minimizer for convex strategy sets}
\label{alg:main}
\SetKwInOut{Input}{Input}
\SetKwInOut{Output}{Output}
\SetKw{Input}{Input:}
\SetKw{Output}{Output:}
\Input{
    \begin{itemize}[noitemsep,topsep=0pt]
        \item Convex and compact set $\cX \subseteq \R^d$ in isotropic position
        \item $k$-dimensional set $\Phi^m$ under~\Cref{assumption:kernel-precise} with respect to $m : \cX \to \R^{k'}$, where $k = k' \cdot d + d$
        \item time horizon $T \in \N$
    \end{itemize}
}
\Output{An efficient $\Phi^m$-regret minimizer for $\cX$}\\
Set the learning rate $\eta \propto \frac{1}{\sqrt{T}}$ and $\epsilon = \nicefrac{1}{\poly(k,T)}$ to be sufficiently small\;
Initialize $\mu^{(1)} \in \Delta(\cX)$ and $\mK^{(1)} \defeq \mI_{d \times k'}$ to be the identity map and $\cons^{(1)} \defeq \vec{0}$\;
Initialize $\cM \defeq \cB_{R}(\vec{0})$ for a large enough $R \leq \poly(k)$\;
 \For{$ t=1, \dots, T$} {
    Output $\mu^{(t)} \in \Delta(\cX)$ and receive feedback $\vu^{(t)} \in [-1, 1]^d$\;
    Define $\R^{d \times k' + d} \ni \mU^{(t)} \defeq (\E_{\vx^{(t)} \sim \mu^{(t)}} \vu^{(t)} \otimes m(\vx^{(t)}), \vec{u}^{(t)})$\;
    Set $\tilPhi^{(t+1)}, (\mK^{(t+1)}, \cons^{(t+1)}), \mu^{(t+1)} \defeq \shellproj_{\Phi}((\mK^{(t)}, \cons^{(t)}) + \eta \mU^{(t)})$ with input $\cM$ and precision $\epsilon$, where $\mu^{(t+1)} \in \Delta(\cX)$ is an $\epsilon$-expected fixed point of $\vx \mapsto \mK^{(t+1)} m(\vx) + \cons^{(t+1)}$ 
 }
\end{algorithm}

Finally, we conclude by providing a lower bound that matches our upper bound (\Cref{theorem:main-prec}) up to a constant factor in the exponent of $k$. It is based on the following normal-form lower bound due to~\citet{Dagan24:From} and~\citet{Peng24:Fast}.

\begin{theorem}[\citep{Dagan24:From,Peng24:Fast}]
    \label{theorem:lowerknown}
    Consider a learner operating on the simplex $\Delta(\cA)$. For any $T < |\cA|/4$, there is an adversary that forces the swap regret of the learner to be $\Omega(\log^{-6}T)$.
\end{theorem}

We observe that there is a simple way to parameterize the above lower bound in terms of the dimension of the set of deviations:

\begin{corollary}
    \label{cor:parlower}
    Consider a learner operating on the simplex $\Delta(\cA)$. There is a $k$-dimensional set of deviations $\Phi \subseteq \Delta(\cA)^{\Delta(\cA)}$ such that for any $T < \sqrt{k}/4$, there is an adversary that forces the $\Phi$-regret of the learner to be $\Omega(\log^{-6} T)$.
\end{corollary}

Indeed, one can first identify an arbitrary subset $\cA'$ of $\cA$ with cardinality $\sqrt{k}$, and then employ the adversary of~\Cref{theorem:lowerknown} with respect to $\cA'$ while rendering all other actions dominated by assigning to them very small utility. That $\Phi$ is $k$-dimensional in this case follows because the set of stochastic matrices mapping $\Delta(\cA')$ to $\Delta(\cA')$---which contains all relevant swap deviations---is $(\sqrt{k})^2$-dimensional.

Combining~\Cref{cor:parlower} with the recent reduction of~\citet{Daskalakis24:Lower}, which embeds the normal-form game lower bound of~\Cref{theorem:lowerknown} into an extensive-form game, we arrive at~\Cref{theorem:mainlower}, which we restate below.

\lowerbound*
\section*{Conclusion}
This paper aims to enhance our understanding of the computational complexity of computing various Shapley value variants. We found that for various ML models --- including decision trees, regression tree ensembles, weighted automata, and linear regression --- both local and global interventional and baseline SHAP can be computed in polynomial time under HMM modeled distributions. This extends popular algorithms, such as TreeSHAP, beyond their empirical distributional scope. We also establish strict complexity gaps between the various SHAP variants (baseline, interventional, and conditional) and prove the intractability of computing SHAP for tree ensembles and neural networks in simplified scenarios. Overall, we present SHAP as a versatile framework whose complexity depends on four key factors: \begin{inparaenum}[(i)] \item model type, \item SHAP variant, \item distribution modeling approach, \item and local vs. global explanations\end{inparaenum}. We believe this perspective provides deeper insight into the computational complexity of SHAP, paving the way for future work.




%We believe that our framework provides a more intricate understanding of SHAP computation complexity across different models, distributions, and variants, paving the way for further research.

Our work opens promising directions for future research. First, expanding our computational analysis to other SHAP-related metrics, such as asymmetric SHAP~\citep{frye20} and SAGE~\citep{covert2020understanding}, would be valuable. Additionally, we aim to explore more expressive distribution classes and relaxed assumptions beyond those in Section \ref{sec:tractable} while maintaining tractable SHAP computation. Finally, when exact computation is intractable (Section \ref{sec:intractable}), investigating the approximability of SHAP metrics through approximation and parameterized complexity theory~\citep{downey2012parameterized} is an important direction.

%Our work opens several promising avenues for future research on the computational properties of explainable AI methods, with a particular focus on SHAP. First, it would be interesting to broaden the computational analysis conducted in this work to include other popular SHAP-related metrics in the literature, such as asymmetric SHAP \cite{frye20} and SAGE \cite{covert2020understanding}. Also, in the future, we aim to explore more expressive distribution classes and relaxed distributional assumptions—extending beyond those examined in Section \ref{sec:tractable} —that still yield tractable SHAP computation. Finally, when exact computation proves intractable (Section \ref{sec:intractable}), it is worthwhile to theoretically investigate the question of the approximability of computing the SHAP metrics across various configurations, through the lens of approximation and parametrized complexity theory \cite{arora2009computational}.

%This paper aims to deepen our understanding of the computational complexity involved in obtaining different Shapley value variants. We found that for a variety of ML models, including decision trees, tree ensembles for regression, weighted automata, and linear regression models — computing both local and global interventional and baseline SHAP can be done in polynomial time when distributions are modeled by HMMs. This extends the distributional scope of popular algorithms like TreeSHAP, which is limited to empirical distributions. Additionally, we demonstrate a strict complexity gap between SHAP variants, showing that interventional and baseline SHAP can be strictly easier to compute than conditional SHAP. Despite these positive results, we uncovered intractability for various SHAP variants in neural networks and tree ensembles. Finally, we provided generalized complexity relations across SHAP variants. We believe that our framework offers a deeper understanding of the complexity involved in computing SHAP across various variants, models, distributions, as well as in both local and global computations, laying the groundwork for future research.

\section*{Acknowledgments}
T.S. is supported by the Vannevar Bush Faculty Fellowship ONR N00014-23-1-2876, National Science Foundation grants RI-2312342 and RI-1901403, ARO award W911NF2210266, and NIH award A240108S001. B.H.Z. is supported 
by the CMU Computer Science Department Hans Berliner
PhD Student Fellowship. E.T, R.E.B., and V.C. thank the Cooperative AI Foundation, Polaris Ventures (formerly the Center for
Emerging Risk Research) and Jaan Tallinn’s donor-advised fund at Founders Pledge for financial
support. E.T. and R.E.B. are also supported in part by the Cooperative AI PhD Fellowship. G.F is supported by the National Science Foundation grant CCF-2443068. We are indebted to Constantinos Daskalakis and Noah Golowich for many insightful discussions concerning the complexity of computing expected fixed points.

\bibliography{dairefs}

\clearpage
\appendix

\section{Sufficiency of regret minimization in isotropic position}
\label{sec:isotropic}

Throughout the paper, we have assumed that we are minimizing $\Phi$-regret with respect to a convex set $\cX$ that is in isotropic position. \Cref{lemma:isotropic} below shows that this is without any loss. The argument here is similar to~\citet[Lemma A.1]{Daskalakis24:Efficient}, with the minor modification that we need to account for mixed strategies.

\begin{lemma}
    \label{lemma:isotropic}
    Let $\cX \subseteq \R^d$ be a convex and compact set such that $\cX \subseteq \cB_R(\vec{0})$. Let $\psi : \cX \to \tilde{\cX}$ be an invertible affine transformation such that $\tilde{\cX} = \psi(\cX)$ is in isotropic position. Suppose that we have a regret minimizer $\regbox_{\tilde{\cX}}$ for $\tilde{\cX}$ that incurs $\phireg_{\tilde{\cX}}^{(T)}$. Then, using $\poly(d)$ time in each round, we can construct a regret minimizer $\regbox_{\cX}$ for $\cX$ that incurs $\phireg^{(T)}_{\cX} \leq 2R \sqrt{d} \cdot \phireg^{(T)}_{\tilde{\cX}}$.
\end{lemma}

\begin{proof}
    Let $\psi(\vx) \defeq \mA \vx + \vec{b}$ for an invertible $\mA \in \R^{d \times d}$ and $\vec{b} \in \R^d$. Let $\vu^{(1)}, \dots, \vu^{(T)}$ be the sequence of utilities given as input to $\regbox_{\cX}$. We then provide as input to $\regbox_{\tilde{\cX}}$ the sequence of utilities
    \begin{equation}
        \label{eq:new-utils}
    \tilde{\vu}^{(t)} \defeq \frac{1}{2R \sqrt{d}} ( \mA^{-1} )^\top \vu^{(t)} \quad t = 1, \dots, T.
    \end{equation}
    Since $\tilde{\cX}$ contains the ball $\cB_1(\vec{0})$ and $\cX \subseteq \cB_R(\vec{0})$, it follows that $\| \mA^{-1} \vx \| \leq 2R$ for any $\vx \in \cB_1(\vec{0})$, which implies that the spectral norm of $\mA^{-1}$ is at most $2 R$. As a result, assuming that $\vu^{(t)} \in [-1, 1]^d$ for all $t \in [T]$, it follows that the utilities constructed in~\eqref{eq:new-utils} are also in $[-1, 1]^d$.
    
    Now, suppose that $\regbox_{\tilde{\cX}}$ returns the sequence of strategies $\tilde{\mu}^{(1)}, \dots, \tilde{\mu}^{(T)} \in \Delta(\tilde{\cX})$. We define, for each $t \in [T]$, $\mu^{(t)} \defeq \psi^{-1}(\tilde{\mu}^{(t)})$ as the next strategy. Consider any $\phi \in \Phi$, and define $\tilde{\phi} : \tilde{\cX} \ni \vx \mapsto \psi(\phi(\psi^{-1}(\vx))) \in \Phi(\tilde{\cX})$. Then,
    \begin{align}
        &\sum_{t=1}^T \left\langle \vu^{(t)}, \E_{\tilde{\vx}^{(t)} \sim \tilde{\mu}^{(t)}} \psi^{-1}(\tilde{\vx}^{(t)}) - \phi(\psi^{-1}(\tilde{\vx}^{(t)})) \right\rangle \\
        &= 2R \sqrt{d} \sum_{t=1}^T \left\langle \frac{1}{2R \sqrt{d}} 
(\mA^{-1})^\top \vu^{(t)}, \E_{\tilde{\vx}^{(t)} \sim \tilde{\mu}^{(t)}} \mA \psi^{-1}(\tilde{\vx}^{(t)}) - \mA \phi(\psi^{-1}(\tilde{\vx}^{(t)})) \right\rangle \\
&= 2R \sqrt{d} \sum_{t=1}^T \left\langle \frac{1}{2R \sqrt{d}} 
(\mA^{-1})^\top \vu^{(t)}, \E_{\tilde{\vx}^{(t)} \sim \tilde{\mu}^{(t)}} (\mA \psi^{-1}(\tilde{\vx}^{(t)}) + \vec{b}) - (\mA \phi(\psi^{-1}(\tilde{\vx}^{(t)})) + \vec{b}) \right\rangle \\
&= 2R \sqrt{d} \sum_{t=1}^T \left\langle \tilde{\vu}^{(t)}, \E_{\tilde{\vx}^{(t)} \sim \tilde{\mu}^{(t)}} \tilde{\vx}^{(t)}  - \tilde{\phi}(\tilde{\vx}^{(t)}) \right\rangle \leq 2R \sqrt{d} \cdot \phireg^{(T)}_{\tilde{\cX}}.
    \end{align}
\end{proof}
\section{Geometric properties of $\Phi^m$}
\label{sec:wellbounded}

In this section, we establish that the set $\Phi^m$ per~\Cref{assumption:kernel-precise} is geometrically well behaved, which is necessary to execute the ellipsoid algorithm (as well as the online learning setting). In particular, our goal is to prove \Cref{lemma:wellbounded}.

Below, for convex and compact $\cA, \cB \subseteq \R^d$, we use the notation 
\begin{equation}
    \Phi^m(\cA, \cB) \defeq \left\{ (\mK, \cons) \in \R^{k + d} : \mK m(\vx) + \cons \in \cB \quad \forall \vx \in \cA \right\}.
\end{equation}

\begin{lemma}
    \label{lemma:trivial}
    Let $\cA, \cB, \cC, \cD$ be convex and compact sets. If $\cA \supseteq \cC$ and $\cB \subseteq \cD$, then $\Phi^m(\cA, \cB) \subseteq \Phi^m(\cC, \cD)$.
\end{lemma}

\begin{proof}
    Consider any $(\mK, \cons) \in \Phi^m(\cA, \cB)$. By definition, it holds that $\mK \vx + \cons \in \cB$ for all $\vx \in \cA$. Since $\cC \subseteq \cA$, it follows that $\mK \vx + \cons \in \cB$ for all $\vx \in \cC$, and in particular, $\mK \vx + \cons \in \cD$ since $\cB \subseteq \cD$.
\end{proof}

\begin{lemma}
    \label{lemma:boundM}
    Let $\cX \subseteq \R^d$ be a convex and compact set such that $\cB_r(\vec{0}) \subseteq \cX \subseteq \cB_R(\vec{0})$, with $R \geq 1$. Suppose further that $\|m(\vx)\| \leq M$ for all $\vx \in \cB_R(\vec{0})$, where $M = M(R) \geq 1$. Then, $\Phi^m \supseteq \cB_{r'}(\phi_{\vec{0}})$, where $r' \defeq \nicefrac{r}{2M(R)}$ and $\phi_{\vec{0}} \defeq (\mathbf{0}, \vec{0})$ is the constant transformation $\vx \mapsto \vec{0}$.
\end{lemma}

\begin{proof}
    By~\Cref{lemma:trivial}, it suffices to prove $\cB_{r'}(\phi_{\vec{0}}) \subseteq \Phi^m(\cX, \cB_r(\vec{0}))$. Consider any $(\mK, \cons) \in \cB_{r'}(\phi_{\vec{0}})$, which means that $\| \mK \|^2_F + \|\cons \|^2 \leq ( \nicefrac{r}{2M(R)})^2$. Then, for any $\vx \in \cX$,
    \begin{equation}
        \| \mK m(\vx) + \cons \| \leq \| \mK \|_F \|m(\vx)\| + \|\cons  \| \leq r.
    \end{equation}
    This means that $\cB_{r'}(\phi_{\vec{0}}) \subseteq \Phi^m(\cX, \cB_r(\vec{0}))$, and the proof follows.
\end{proof}

\begin{lemma}
    \label{lemma:fulldim}
    Suppose that $\co m(\cB_r(\vec{0})) \supseteq \cB_\delta(\vec{0})$ for some $\delta = \delta(r) > 0$ and $m(\vec{0}) = \vec{0}$. Then, assuming that $r < R$,
    \begin{equation}
        \Phi^m( \cB_r(\vec{0}), \cB_R(\vec{0})) \subseteq \cB_{ R'}(\vec{0}),
    \end{equation}
    where $R' \defeq R \left( \frac{2\sqrt{d}}{\delta(r)} + 1 \right)$.
\end{lemma}

\begin{proof}
    Consider any $(\mK, \cons) \in \Phi^m( \cB_r(\vec{0}), \cB_R(\vec{0}))$. By definition, we have $\| \mK m(\vx) + \cons \| \leq R$ for all $\vx \in \cB_r(\vec{0})$. Since $m(\vec{0}) = \vec{0}$, it follows that $\|\cons \| \leq R$. Thus, $\| \mK m(\vx) \| \leq \|\mK m(\vx) + \cons \| + \|\cons \| \leq 2 R$ for all $\vx \in \cB_r(\vec{0})$. Now, let $\vec{x}' \in \R^{k'}$ with $\|\vec{x}' \| = 1$ be such that $\| \mK \vec{x}' \| = \| \mK\|$, where $\| \mK \|$ is the spectral norm of $\mK$. Since we have assumed that $\co m(\cB_r(\vec{0})) \supseteq \cB_\delta(\vec{0})$, it follows that there exist $\lambda_1, \dots, \lambda_{k'+1}$, with $\lambda_1, \dots, \lambda_{k' + 1} \geq 0$ and $\sum_{j=1}^{k'+1} \lambda_j = 1$, and $\vx_1, \dots, \vx_{k'+1} \in \cB_r(\vec{0})$ (by Carath\'eodory's theorem) such that $\sum_{j=1}^{k'+1} \lambda_j m(\vx_j) = \delta \vx'$. As a result,
    \begin{equation}
        \delta \|\mK\| = \| \mK (\delta \vx') \| = \left\| \mK \left( \sum_{j=1}^{k'+1} \lambda_j m(\vx_j) \right) \right\| \leq \sum_{j=1}^{k' + 1} \lambda_j \|\mK m(\vx_j) \| \leq 2R.
    \end{equation}
    Finally, we have $\|\mK\|_F \leq \sqrt{d} \|\mK\|$, and the claim follows.
\end{proof}

\begin{proof}[Proof of~\Cref{lemma:wellbounded}]
    The claim follows directly by combining~\Cref{lemma:trivial,lemma:boundM,lemma:fulldim}.
\end{proof}

\section{Further omitted proofs}
\label{sec:aux}

Finally, this section provides the proof of~\Cref{lemma:lowdeg}, which we restate below. For completeness, we have also included the usual version of $\eah$ (\Cref{alg:eah}, subsumed by~\Cref{alg:gen-eah}), which we used earlier in~\Cref{theorem:semiseparation,th:efp}.

\lowdeg*

For the proof, we will use a simple, auxiliary lemma.

\begin{lemma}
    \label{lemma:range}
    Let $X$ be a random variable such that $\E[X] = 0$, $\V[X] = 1$, and $X \in [-R, R]$ almost surely. Then, $\pr[X \geq \nicefrac{1}{R}] > 0$.
\end{lemma}

\begin{proof}[Proof of~\Cref{lemma:lowdeg}]
    It is clear that $\linem(\vec{0}) = \vec{0}$. The bound on $M$ is also immediate. We thus focus on proving that $\delta \defeq 1/M$ suffices.
    
    For the sake of contradiction, suppose that $\co \linem(\cB_1(\vec{0}))$ does not contain $\vx'$ for some $\vx' \in \R^{k'}$ with $\| \vx' \| \leq \delta$. Then, we consider a hyperplane that separates $\co \linem(\cB_1(\vec{0}))$ from $\vx'$, and we let $\vec{v}$ be the normal vector to that hyperplane, so that $\langle \vec{v}, \vx' \rangle > \langle \vec{v}, \linem(\vx) \rangle$ for all $\vx \in \cB_1(\vec{0})$. Now, let $\cU$ be the uniform product distribution over $[-1, 1]^d$. By~\eqref{eq:orthog}, we have $\E_{\vx \sim \cU} [m(\vx)] = 0$ and $\E_{ \vx \sim \cU} [ m(\vx) m(\vx)^\top] = \mI_{k' \times k'}$ (by ortogonality). As a result, we have $\E_{ \vx \sim \cU} [  \langle \vec{v}, m(\vx) \rangle^2 ] = \mathbb{V}_{\vx \sim \cU}[ \langle \vec{v}, m(\vx) \rangle ] = \|\vec{v} \|^2 = 1$. \Cref{lemma:range}, applied for the random variable $\langle \vec{v}, m(\vx) \rangle$ with range $[-M, M]$, implies that there exists $\vx \in [-1, 1]^d$ such that $\langle \vec{v}, m(\vx) \rangle \geq 1/M$, which in turn implies that there exists $\linex \in \cB_1(\vec{0})$---namely, $\linex \defeq \vx/\sqrt{d}$---such that $\langle \vec{v}, \linem(\linex) \rangle \geq 1/M$. But this yields $\delta \leq \langle \vec{v}, \linem(\linex) \rangle < \langle \vec{v}, \vx' \rangle \leq \|\vec{v} \| \|\vx'\| = \delta$, a contradiction.
\end{proof}

\begin{algorithm}[!ht]
\caption{Ellipsoid against hope ($\eah$)~\citep{Papadimitriou08:Computing}}
\label{alg:eah}
\SetKwInOut{Input}{Input}
\SetKwInOut{Output}{Output}
\SetKw{Input}{Input:}
\SetKw{Output}{Output:}
\Input{\begin{itemize}[noitemsep,topsep=0pt]
        \item Parameters $R_y, r_y > 0$ such that $\cB_{r_y}(\cdot) \subseteq \cY \subseteq \cB_{R_y}(\vec{0})$
        \item Precision parameter $\epsilon > 0$
        \item Parameter $B > 0 $ such that $\| G(\vx) \| \leq B$ for all $\vx \in \cX$
        \item $\ger$ oracle for~\eqref{eq:eah}
        \item $\sep$ oracle for $\cY$
    \end{itemize}
}
\Output{A sparse, $\epsilon$-approximate solution $\mu \in \Delta(\cX)$ of~\eqref{eq:eah}}
Initialize the ellipsoid $\cE \defeq \cB_{R_y}(\vec{0})$\;
\While{$\vol(\cE) \geq \vol(\cB_{\epsilon/B}(\cdot))$}{
    Let $\vy \in \R^k$ be the center of $\cE$ \;
    \eIf{$\vy \in \cY$}{
        Let $\vx \in \cX$ be a good-enough-response with respect to $\vy$ (via the $\ger$ oracle)\;
        Update $\cE$ to the minimum volume ellipsoid containing $\cE \cap \{ \vy \in \R^k : \langle \vy, G(\vx) \rangle \leq 0 \}$\;
    }{
        Let $H$ be the halfspace that separates $\vy$ from $\cY$ (via the $\sep$ oracle) \;
        Update $\cE$ to the minimum volume ellipsoid containing $\cE \cap H$\;
    }
}
Let $\vx^{(1)}, \dots, \vx^{(T)}$ be the $\ger$ oracle responses produced in the process above\;
Define $\mG \defeq [G(\vx^{(1)}) \mid \hdots \mid G(\vx^{(T)})] \in \R^{k \times T} $\;
Compute a solution $\vec{\lambda}$ to the convex program
    $$\qq{find} \vec{\lambda} \in \Delta^T \qq{s.t.} \min_{\vy \in \cY} \vec{\lambda}^\top \mG^\top \vy \geq - \epsilon$$
\Return{$\Delta(\cX) \ni \mu \defeq \sum_{t=1}^T \lambda^{(t)} \mu(\vx^{(t)})$}
\end{algorithm}

\end{document}
