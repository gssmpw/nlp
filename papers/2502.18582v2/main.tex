%--------------------
% Packages
% -------------------
\documentclass[11pt]{article}

\usepackage[utf8x]{inputenc}
\usepackage[T1]{fontenc}
\usepackage{booktabs} % For formal tables

\usepackage[vlined,ruled]{algorithm2e} % For algorithms
\renewcommand{\algorithmcfname}{ALGORITHM}

\SetAlFnt{\small}
\SetAlCapFnt{\small}
\SetAlCapNameFnt{\small}
\SetAlCapHSkip{0pt}
\IncMargin{-\parindent}


\usepackage{xcolor}         % colors
\usepackage{hyperref}

\newcommand{\declarecolor}[2]{\definecolor{#1}{RGB}{#2}\expandafter\newcommand\csname #1\endcsname[1]{\textcolor{#1}{##1}}}
\declarecolor{White}{255, 255, 255}
\declarecolor{Black}{0, 0, 0}
\declarecolor{Maroon}{128, 0, 0}
\declarecolor{Coral}{255, 127, 80}
\declarecolor{Red}{182, 21, 21}
\declarecolor{LimeGreen}{50, 205, 50}
\declarecolor{DarkGreen}{0, 80, 0}
\declarecolor{Purple}{146, 42, 158}
\declarecolor{Navy}{0, 0, 128}
\declarecolor{LightBlue}{84, 101, 202}
\definecolor{mydarkblue}{rgb}{0,0.08,0.45}
\hypersetup{ %
    pdftitle={},
    pdfkeywords={},
    pdfborder=0 0 0,
    pdfpagemode=UseNone,
    colorlinks=true,
    linkcolor=Navy,
    citecolor=DarkGreen,
    filecolor=Purple,
    urlcolor=Purple,
}


\usepackage{fullpage}

%\usepackage{gentium}
% \usepackage{mathptmx} % Use Times Font
\usepackage{amsthm}
\usepackage{amsmath}
\usepackage{amsfonts}
\usepackage{thm-restate}
% \usepackage{amssymb}

\usepackage{tikz}
\usetikzlibrary{shapes.geometric, arrows, positioning}

\usepackage{multirow}

%\usepackage{complexity}

\usepackage{nicefrac}

% \usepackage[swedish]{babel} % Swedish translations
\usepackage{enumitem} % Includes lists

\frenchspacing % No double spacing between sentences
% \linespread{1.2} % Set linespace
% \usepackage[a4paper, lmargin=0.1666\paperwidth, rmargin=0.1666\paperwidth, tmargin=0.1111\paperheight, bmargin=0.1111\paperheight]{geometry} %margins
% \usepackage{parskip}

% \makeatletter
% \def\thm@space@setup{%
%   \thm@preskip=\parskip \thm@postskip=0pt
% }
% \makeatother

\usepackage[all]{nowidow} % Tries to remove widows

\usepackage{lipsum} % Used for inserting dummy 'Lorem ipsum' text into the template

\usepackage[numbers]{natbib}
\bibliographystyle{plainnat}

%%%%%%%%%%%---SETME-----%%%%%%%%%%%%%
%replace @@ with the submission number submission site.
\newcommand{\thiswork}{INF$^2$\xspace}
%%%%%%%%%%%%%%%%%%%%%%%%%%%%%%%%%%%%


%\newcommand{\rev}[1]{{\color{olivegreen}#1}}
\newcommand{\rev}[1]{{#1}}


\newcommand{\JL}[1]{{\color{cyan}[\textbf{\sc JLee}: \textit{#1}]}}
\newcommand{\JW}[1]{{\color{orange}[\textbf{\sc JJung}: \textit{#1}]}}
\newcommand{\JY}[1]{{\color{blue(ncs)}[\textbf{\sc JSong}: \textit{#1}]}}
\newcommand{\HS}[1]{{\color{magenta}[\textbf{\sc HJang}: \textit{#1}]}}
\newcommand{\CS}[1]{{\color{navy}[\textbf{\sc CShin}: \textit{#1}]}}
\newcommand{\SN}[1]{{\color{olive}[\textbf{\sc SNoh}: \textit{#1}]}}

%\def\final{}   % uncomment this for the submission version
\ifdefined\final
\renewcommand{\JL}[1]{}
\renewcommand{\JW}[1]{}
\renewcommand{\JY}[1]{}
\renewcommand{\HS}[1]{}
\renewcommand{\CS}[1]{}
\renewcommand{\SN}[1]{}
\fi

%%% Notion for baseline approaches %%% 
\newcommand{\baseline}{offloading-based batched inference\xspace}
\newcommand{\Baseline}{Offloading-based batched inference\xspace}


\newcommand{\ans}{attention-near storage\xspace}
\newcommand{\Ans}{Attention-near storage\xspace}
\newcommand{\ANS}{Attention-Near Storage\xspace}

\newcommand{\wb}{delayed KV cache writeback\xspace}
\newcommand{\Wb}{Delayed KV cache writeback\xspace}
\newcommand{\WB}{Delayed KV Cache Writeback\xspace}

\newcommand{\xcache}{X-cache\xspace}
\newcommand{\XCACHE}{X-Cache\xspace}


%%% Notions for our methods %%%
\newcommand{\schemea}{\textbf{Expanding supported maximum sequence length with optimized performance}\xspace}
\newcommand{\Schemea}{\textbf{Expanding supported maximum sequence length with optimized performance}\xspace}

\newcommand{\schemeb}{\textbf{Optimizing the storage device performance}\xspace}
\newcommand{\Schemeb}{\textbf{Optimizing the storage device performance}\xspace}

\newcommand{\schemec}{\textbf{Orthogonally supporting Compression Techniques}\xspace}
\newcommand{\Schemec}{\textbf{Orthogonally supporting Compression Techniques}\xspace}



% Circular numbers
\usepackage{tikz}
\newcommand*\circled[1]{\tikz[baseline=(char.base)]{
            \node[shape=circle,draw,inner sep=0.4pt] (char) {#1};}}

\newcommand*\bcircled[1]{\tikz[baseline=(char.base)]{
            \node[shape=circle,draw,inner sep=0.4pt, fill=black, text=white] (char) {#1};}}

\usepackage[capitalize,noabbrev,nameinlink]{cleveref}

\usepackage{MnSymbol}

\renewcommand\vec\bm

\theoremstyle{plain}
\newtheorem{theorem}{Theorem}[section]
\newtheorem{proposition}[theorem]{Proposition}
\newtheorem{lemma}[theorem]{Lemma}
\newtheorem{corollary}[theorem]{Corollary}
\theoremstyle{definition}
\newtheorem{definition}[theorem]{Definition}
\newtheorem{assumption}[theorem]{Assumption}
\newtheorem{example}[theorem]{Example}
% \theoremstyle{remark}
\newtheorem{remark}[theorem]{Remark}

\usepackage{autonum}

\newcommand{\bhz}[1]{{ {\color{magenta}{[BZ:~#1]}}}}
\newcommand{\et}[1]{{ {\color{olive}{[ET:~#1]}}}}
\newcommand{\emin}[1]{{ {\color{orange}{[EB:~#1]}}}}
\newcommand{\ioannis}[1]{{ {\color{blue}{[IA:~#1]}}}}
\newcommand{\gabri}[1]{{ {\color{red}{[GF:~#1]}}}}

\usepackage{authblk}

\title{Learning and Computation of $\Phi$-Equilibria \\at the Frontier of Tractability}

\author[1]{Brian Hu Zhang\thanks{Equal contribution.}}
\author[1]{Ioannis Anagnostides$^*$}
\author[1,2]{Emanuel Tewolde}
\author[1,2]{Ratip Emin Berker}
\author[3]{Gabriele Farina}
\author[1,2,4]{Vincent Conitzer}
\author[1,5]{Tuomas Sandholm}
\renewcommand\Affilfont{\small}
\affil[1]{Carnegie Mellon University}
\affil[2]{Foundations of Cooperative AI Lab (FOCAL)}
\affil[3]{Massachusetts Institute of Technology}
\affil[4]{University of Oxford}
\affil[5]{Additional affiliations: Strategy Robot, Inc., Strategic Machine, Inc., Optimized Markets, Inc.}
\affil[ ]{}
\affil[ ]{\texttt{\{bhzhang,ianagnos,etewolde,rberker,conitzer,sandholm\}}\texttt{@cs.cmu.edu}, \texttt{gfarina}\texttt{@mit.edu}}

\begin{document} 

\maketitle

\pagenumbering{gobble}

\begin{abstract}
\emph{$\Phi$-equilibria}---and the associated notion of \emph{$\Phi$-regret}---are a powerful and flexible framework at the heart of online learning and game theory, whereby enriching the set of deviations $\Phi$ begets stronger notions of rationality. Recently, Daskalakis, Farina, Fishelson, Pipis, and Schneider (STOC '24)---abbreviated as DFFPS---settled the existence of efficient algorithms when $\Phi$ contains only linear maps under a general, $d$-dimensional convex constraint set $\mathcal{X}$. In this paper, we significantly extend their work by resolving the case where $\Phi$ is $k$-dimensional; degree-$\ell$ polynomials constitute a canonical such example with $k = d^{O(\ell)}$. In particular, positing only oracle access to $\cX$, we obtain two main positive results:
\begin{itemize}
    \item a $\text{poly}(n, d, k, \text{log}(1/\epsilon))$-time algorithm for computing $\epsilon$-approximate $\Phi$-equilibria in $n$-player multilinear games, and
    \item an efficient online algorithm that incurs  average $\Phi$-regret at most $\epsilon$ using $\text{poly}(d, k)/\epsilon^2$ rounds.
\end{itemize}
We also show nearly matching---up to constant factors in the exponents---lower bounds parameterized by $k$ in the online learning setting, thereby obtaining for the first time a family of deviations that captures the 
learnability of $\Phi$-regret.

From a technical standpoint, we extend the framework of DFFPS from linear maps to the more challenging case of maps with polynomial dimension. At the heart of our approach is a polynomial-time algorithm for computing an \emph{expected fixed point} of any $\phi: \cX \to \cX$---that is, a distribution $\mu \in \Delta(\cX)$ such that $\E_{\vec{x} \sim \mu}[ \phi(\vx) - \vx ] \approx 0$---based on the seminal \emph{ellipsoid against hope (EAH)} algorithm of Papadimitriou and Roughgarden (JACM '08). In particular, our algorithm for computing $\Phi$-equilibria is based on executing EAH in a nested fashion---each step of EAH itself being implemented by invoking a separate call to EAH.
\end{abstract}

\clearpage

\tableofcontents

\clearpage

\pagenumbering{arabic}

\section{Introduction}


\begin{figure}[t]
\centering
\includegraphics[width=0.6\columnwidth]{figures/evaluation_desiderata_V5.pdf}
\vspace{-0.5cm}
\caption{\systemName is a platform for conducting realistic evaluations of code LLMs, collecting human preferences of coding models with real users, real tasks, and in realistic environments, aimed at addressing the limitations of existing evaluations.
}
\label{fig:motivation}
\end{figure}

\begin{figure*}[t]
\centering
\includegraphics[width=\textwidth]{figures/system_design_v2.png}
\caption{We introduce \systemName, a VSCode extension to collect human preferences of code directly in a developer's IDE. \systemName enables developers to use code completions from various models. The system comprises a) the interface in the user's IDE which presents paired completions to users (left), b) a sampling strategy that picks model pairs to reduce latency (right, top), and c) a prompting scheme that allows diverse LLMs to perform code completions with high fidelity.
Users can select between the top completion (green box) using \texttt{tab} or the bottom completion (blue box) using \texttt{shift+tab}.}
\label{fig:overview}
\end{figure*}

As model capabilities improve, large language models (LLMs) are increasingly integrated into user environments and workflows.
For example, software developers code with AI in integrated developer environments (IDEs)~\citep{peng2023impact}, doctors rely on notes generated through ambient listening~\citep{oberst2024science}, and lawyers consider case evidence identified by electronic discovery systems~\citep{yang2024beyond}.
Increasing deployment of models in productivity tools demands evaluation that more closely reflects real-world circumstances~\citep{hutchinson2022evaluation, saxon2024benchmarks, kapoor2024ai}.
While newer benchmarks and live platforms incorporate human feedback to capture real-world usage, they almost exclusively focus on evaluating LLMs in chat conversations~\citep{zheng2023judging,dubois2023alpacafarm,chiang2024chatbot, kirk2024the}.
Model evaluation must move beyond chat-based interactions and into specialized user environments.



 

In this work, we focus on evaluating LLM-based coding assistants. 
Despite the popularity of these tools---millions of developers use Github Copilot~\citep{Copilot}---existing
evaluations of the coding capabilities of new models exhibit multiple limitations (Figure~\ref{fig:motivation}, bottom).
Traditional ML benchmarks evaluate LLM capabilities by measuring how well a model can complete static, interview-style coding tasks~\citep{chen2021evaluating,austin2021program,jain2024livecodebench, white2024livebench} and lack \emph{real users}. 
User studies recruit real users to evaluate the effectiveness of LLMs as coding assistants, but are often limited to simple programming tasks as opposed to \emph{real tasks}~\citep{vaithilingam2022expectation,ross2023programmer, mozannar2024realhumaneval}.
Recent efforts to collect human feedback such as Chatbot Arena~\citep{chiang2024chatbot} are still removed from a \emph{realistic environment}, resulting in users and data that deviate from typical software development processes.
We introduce \systemName to address these limitations (Figure~\ref{fig:motivation}, top), and we describe our three main contributions below.


\textbf{We deploy \systemName in-the-wild to collect human preferences on code.} 
\systemName is a Visual Studio Code extension, collecting preferences directly in a developer's IDE within their actual workflow (Figure~\ref{fig:overview}).
\systemName provides developers with code completions, akin to the type of support provided by Github Copilot~\citep{Copilot}. 
Over the past 3 months, \systemName has served over~\completions suggestions from 10 state-of-the-art LLMs, 
gathering \sampleCount~votes from \userCount~users.
To collect user preferences,
\systemName presents a novel interface that shows users paired code completions from two different LLMs, which are determined based on a sampling strategy that aims to 
mitigate latency while preserving coverage across model comparisons.
Additionally, we devise a prompting scheme that allows a diverse set of models to perform code completions with high fidelity.
See Section~\ref{sec:system} and Section~\ref{sec:deployment} for details about system design and deployment respectively.



\textbf{We construct a leaderboard of user preferences and find notable differences from existing static benchmarks and human preference leaderboards.}
In general, we observe that smaller models seem to overperform in static benchmarks compared to our leaderboard, while performance among larger models is mixed (Section~\ref{sec:leaderboard_calculation}).
We attribute these differences to the fact that \systemName is exposed to users and tasks that differ drastically from code evaluations in the past. 
Our data spans 103 programming languages and 24 natural languages as well as a variety of real-world applications and code structures, while static benchmarks tend to focus on a specific programming and natural language and task (e.g. coding competition problems).
Additionally, while all of \systemName interactions contain code contexts and the majority involve infilling tasks, a much smaller fraction of Chatbot Arena's coding tasks contain code context, with infilling tasks appearing even more rarely. 
We analyze our data in depth in Section~\ref{subsec:comparison}.



\textbf{We derive new insights into user preferences of code by analyzing \systemName's diverse and distinct data distribution.}
We compare user preferences across different stratifications of input data (e.g., common versus rare languages) and observe which affect observed preferences most (Section~\ref{sec:analysis}).
For example, while user preferences stay relatively consistent across various programming languages, they differ drastically between different task categories (e.g. frontend/backend versus algorithm design).
We also observe variations in user preference due to different features related to code structure 
(e.g., context length and completion patterns).
We open-source \systemName and release a curated subset of code contexts.
Altogether, our results highlight the necessity of model evaluation in realistic and domain-specific settings.





\section{Preliminaries}
\label{sec:prelim}
\label{sec:term}
We define the key terminologies used, primarily focusing on the hidden states (or activations) during the forward pass. 

\paragraph{Components in an attention layer.} We denote $\Res$ as the residual stream. We denote $\Val$ as Value (states), $\Qry$ as Query (states), and $\Key$ as Key (states) in one attention head. The \attlogit~represents the value before the softmax operation and can be understood as the inner product between  $\Qry$  and  $\Key$. We use \Attn~to denote the attention weights of applying the SoftMax function to \attlogit, and ``attention map'' to describe the visualization of the heat map of the attention weights. When referring to the \attlogit~from ``$\tokenB$'' to  ``$\tokenA$'', we indicate the inner product  $\langle\Qry(\tokenB), \Key(\tokenA)\rangle$, specifically the entry in the ``$\tokenB$'' row and ``$\tokenA$'' column of the attention map.

\paragraph{Logit lens.} We use the method of ``Logit Lens'' to interpret the hidden states and value states \citep{belrose2023eliciting}. We use \logit~to denote pre-SoftMax values of the next-token prediction for LLMs. Denote \readout~as the linear operator after the last layer of transformers that maps the hidden states to the \logit. 
The logit lens is defined as applying the readout matrix to residual or value states in middle layers. Through the logit lens, the transformed hidden states can be interpreted as their direct effect on the logits for next-token prediction. 

\paragraph{Terminologies in two-hop reasoning.} We refer to an input like “\Src$\to$\brga, \brgb$\to$\Ed” as a two-hop reasoning chain, or simply a chain. The source entity $\Src$ serves as the starting point or origin of the reasoning. The end entity $\Ed$ represents the endpoint or destination of the reasoning chain. The bridge entity $\Brg$ connects the source and end entities within the reasoning chain. We distinguish between two occurrences of $\Brg$: the bridge in the first premise is called $\brga$, while the bridge in the second premise that connects to $\Ed$ is called $\brgc$. Additionally, for any premise ``$\tokenA \to \tokenB$'', we define $\tokenA$ as the parent node and $\tokenB$ as the child node. Furthermore, if at the end of the sequence, the query token is ``$\tokenA$'', we define the chain ``$\tokenA \to \tokenB$, $\tokenB \to \tokenC$'' as the Target Chain, while all other chains present in the context are referred to as distraction chains. Figure~\ref{fig:data_illustration} provides an illustration of the terminologies.

\paragraph{Input format.}
Motivated by two-hop reasoning in real contexts, we consider input in the format $\bos, \text{context information}, \query, \answer$. A transformer model is trained to predict the correct $\answer$ given the query $\query$ and the context information. The context compromises of $K=5$ disjoint two-hop chains, each appearing once and containing two premises. Within the same chain, the relative order of two premises is fixed so that \Src$\to$\brga~always precedes \brgb$\to$\Ed. The orders of chains are randomly generated, and chains may interleave with each other. The labels for the entities are re-shuffled for every sequence, choosing from a vocabulary size $V=30$. Given the $\bos$ token, $K=5$ two-hop chains, \query, and the \answer~tokens, the total context length is $N=23$. Figure~\ref{fig:data_illustration} also illustrates the data format. 

\paragraph{Model structure and training.} We pre-train a three-layer transformer with a single head per layer. Unless otherwise specified, the model is trained using Adam for $10,000$ steps, achieving near-optimal prediction accuracy. Details are relegated to Appendix~\ref{app:sec_add_training_detail}.


% \RZ{Do we use source entity, target entity, and mediator entity? Or do we use original token, bridge token, end token?}





% \paragraph{Basic notations.} We use ... We use $\ve_i$ to denote one-hot vectors of which only the $i$-th entry equals one, and all other entries are zero. The dimension of $\ve_i$ are usually omitted and can be inferred from contexts. We use $\indicator\{\cdot\}$ to denote the indicator function.

% Let $V > 0$ be a fixed positive integer, and let $\vocab = [V] \defeq \{1, 2, \ldots, V\}$ be the vocabulary. A token $v \in \vocab$ is an integer in $[V]$ and the input studied in this paper is a sequence of tokens $s_{1:T} \defeq (s_1, s_2, \ldots, s_T) \in \vocab^T$ of length $T$. For any set $\mathcal{S}$, we use $\Delta(\mathcal{S})$ to denote the set of distributions over $\mathcal{S}$.

% % to a sequence of vectors $z_1, z_2, \ldots, z_T \in \real^{\dout}$ of dimension $\dout$ and length $T$.

% Let $\mU = [\vu_1, \vu_2, \ldots, \vu_V]^\transpose \in \real^{V\times d}$ denote the token embedding matrix, where the $i$-th row $\vu_i \in \real^d$ represents the $d$-dimensional embedding of token $i \in [V]$. Similarly, let $\mP = [\vp_1, \vp_2, \ldots, \vp_T]^\transpose \in \real^{T\times d}$ denote the positional embedding matrix, where the $i$-th row $\vp_i \in \real^d$ represents the $d$-dimensional embedding of position $i \in [T]$. Both $\mU$ and $\mP$ can be fixed or learnable.

% After receiving an input sequence of tokens $s_{1:T}$, a transformer will first process it using embedding matrices $\mU$ and $\mP$ to obtain a sequence of vectors $\mH = [\vh_1, \vh_2, \ldots, \vh_T] \in \real^{d\times T}$, where 
% \[
% \vh_i = \mU^\transpose\ve_{s_i} + \mP^\transpose\ve_{i} = \vu_{s_i} + \vp_i.
% \]

% We make the following definitions of basic operations in a transformer.

% \begin{definition}[Basic operations in transformers] 
% \label{defn:operators}
% Define the softmax function $\softmax(\cdot): \real^d \to \real^d$ over a vector $\vv \in \real^d$ as
% \[\softmax(\vv)_i = \frac{\exp(\vv_i)}{\sum_{j=1}^d \exp(\vv_j)} \]
% and define the softmax function $\softmax(\cdot): \real^{m\times n} \to \real^{m \times n}$ over a matrix $\mV \in \real^{m\times n}$ as a column-wise softmax operator. For a squared matrix $\mM \in \real^{m\times m}$, the causal mask operator $\mask(\cdot): \real^{m\times m} \to \real^{m\times m}$  is defined as $\mask(\mM)_{ij} = \mM_{ij}$ if $i \leq j$ and  $\mask(\mM)_{ij} = -\infty$ otherwise. For a vector $\vv \in \real^n$ where $n$ is the number of hidden neurons in a layer, we use $\layernorm(\cdot): \real^n \to \real^n$ to denote the layer normalization operator where
% \[
% \layernorm(\vv)_i = \frac{\vv_i-\mu}{\sigma}, \mu = \frac{1}{n}\sum_{j=1}^n \vv_j, \sigma = \sqrt{\frac{1}{n}\sum_{j=1}^n (\vv_j-\mu)^2}
% \]
% and use $\layernorm(\cdot): \real^{n\times m} \to \real^{n\times m}$ to denote the column-wise layer normalization on a matrix.
% We also use $\nonlin(\cdot)$ to denote element-wise nonlinearity such as $\relu(\cdot)$.
% \end{definition}

% The main components of a transformer are causal self-attention heads and MLP layers, which are defined as follows.

% \begin{definition}[Attentions and MLPs]
% \label{defn:attn_mlp} 
% A single-head causal self-attention $\attn(\mH;\mQ,\mK,\mV,\mO)$ parameterized by $\mQ,\mK,\mV \in \real^{{\dqkv\times \din}}$ and $\mO \in \real^{\dout\times\dqkv}$ maps an input matrix $\mH \in \real^{\din\times T}$ to
% \begin{align*}
% &\attn(\mH;\mQ,\mK,\mV,\mO) \\
% =&\mO\mV\layernorm(\mH)\softmax(\mask(\layernorm(\mH)^\transpose\mK^\transpose\mQ\layernorm(\mH))).
% \end{align*}
% Furthermore, a multi-head attention with $M$ heads parameterized by $\{(\mQ_m,\mK_m,\mV_m,\mO_m) \}_{m=1}^M$ is defined as 
% \begin{align*}
%     &\Attn(\mH; \{(\mQ_m,\mK_m,\mV_m,\mO_m) \}_{m\in[M]}) \\ =& \sum_{m=1}^M \attn(\mH;\mQ_m,\mK_m,\mV_m,\mO_m) \in \real^{\dout \times T}.
% \end{align*}
% An MLP layer $\mlp(\mH;\mW_1,\mW_2)$ parameterized by $\mW_1 \in \real^{\dhidden\times \din}$ and $\mW_2 \in \real^{\dout \times \dhidden}$ maps an input matrix $\mH = [\vh_1, \ldots, \vh_T] \in \real^{\din \times T}$ to
% \begin{align*}
%     &\mlp(\mH;\mW_1,\mW_2) = [\vy_1, \ldots, \vy_T], \\ \text{where } &\vy_i = \mW_2\nonlin(\mW_1\layernorm(\vh_i)), \forall i \in [T].
% \end{align*}

% \end{definition}

% In this paper, we assume $\din=\dout=d$ for all attention heads and MLPs to facilitate residual stream unless otherwise specified. Given \Cref{defn:operators,defn:attn_mlp}, we are now able to define a multi-layer transformer.

% \begin{definition}[Multi-layer transformers]
% \label{defn:transformer}
%     An $L$-layer transformer $\transformer(\cdot): \vocab^T \to \Delta(\vocab)$ parameterized by $\mP$, $\mU$, $\{(\mQ_m^{(l)},\mK_m^{(l)},\mV_m^{(l)},\mO_m^{(l)})\}_{m\in[M],l\in[L]}$,  $\{(\mW_1^{(l)},\mW_2^{(l)})\}_{l\in[L]}$ and $\Wreadout \in \real^{V \times d}$ receives a sequence of tokens $s_{1:T}$ as input and predict the next token by outputting a distribution over the vocabulary. The input is first mapped to embeddings $\mH = [\vh_1, \vh_2, \ldots, \vh_T] \in \real^{d\times T}$ by embedding matrices $\mP, \mU$ where 
%     \[
%     \vh_i = \mU^\transpose\ve_{s_i} + \mP^\transpose\ve_{i}, \forall i \in [T].
%     \]
%     For each layer $l \in [L]$, the output of layer $l$, $\mH^{(l)} \in \real^{d\times T}$, is obtained by 
%     \begin{align*}
%         &\mH^{(l)} =  \mH^{(l-1/2)} + \mlp(\mH^{(l-1/2)};\mW_1^{(l)},\mW_2^{(l)}), \\
%         & \mH^{(l-1/2)} = \mH^{(l-1)} + \\ & \quad \Attn(\mH^{(l-1)}; \{(\mQ_m^{(l)},\mK_m^{(l)},\mV_m^{(l)},\mO_m^{(l)}) \}_{m\in[M]}), 
%     \end{align*}
%     where the input $\mH^{(l-1)}$ is the output of the previous layer $l-1$ for $l > 1$ and the input of the first layer $\mH^{(0)} = \mH$. Finally, the output of the transformer is obtained by 
%     \begin{align*}
%         \transformer(s_{1:T}) = \softmax(\Wreadout\vh_T^{(L)})
%     \end{align*}
%     which is a $V$-dimensional vector after softmax representing a distribution over $\vocab$, and $\vh_T^{(L)}$ is the $T$-th column of the output of the last layer, $\mH^{(L)}$.
% \end{definition}



% For each token $v \in \vocab$, there is a corresponding $d_t$-dimensional token embedding vector $\embed(v) \in \mathbb{R}^{d_t}$. Assume the maximum length of the sequence studied in this paper does not exceed $T$. For each position $t \in [T]$, there is a corresponding positional embedding  







\section{Sets of deviations with polynomial dimension}
\label{sec:deviations}

In this section, we formally introduce the assumptions we make concerning the feature map $m$ of~\Cref{assumption:kernel}, and we then provide a canonical example that satisfies our blanket assumptions.

\begin{assumption}
    \label{assumption:kernel-precise}
    We make the following assumptions regarding $\Phi^m$ and $m$ of~\Cref{assumption:kernel}:
    \begin{itemize}
        \item $m : \cX \to \R^{k'}$ is computable in $\poly(k)$ time.
        \item $\co m(\cB_1(\vec{0})) \supseteq \cB_\delta(\vec{0})$ for some $\delta \geq \poly(1/k)$.
        \item $\|m(\vx)\| \leq \poly(k)$ for all $\vx \in \cX$, and $m(\vec{0}) = \vec{0}$. 
        \item $\Phi^m$ contains the identity map.
    \end{itemize}
\end{assumption}

\begin{remark}[Functions on the vertices]
    Let $\cV$ be the set of extreme points of $\cX$. Our positive results (\Cref{theorem:main-eah-prec,theorem:main-prec}) only evaluate $\phi$ at extreme points, so they would operate identically if we instead defined our maps $\phi$ to be $\cV \to \cX$.% For expositional and notational simplicity, we will not do this.
\end{remark}

The definition above places some minimal assumptions on the feature mapping $m$ to ensure that $\Phi^m$ is geometrically well behaved. Indeed, we first show that the set of transformations $\Phi^m$ under~\Cref{assumption:kernel-precise} is well-bounded; the proof is provided in~\Cref{sec:wellbounded}.

\begin{restatable}{lemma}{wellbounded}
    \label{lemma:wellbounded}
    Let $\cX \subseteq \R^d$ be a convex and compact set such that $\cB_r(\vec{0}) \subseteq \cX \subseteq \cB_R(\vec{0})$, with $R \geq 1$ and $r < R$. Suppose further that $\|m(\vx)\| \leq M$ for all $\vx \in \cB_R(\vec{0})$, with $M = M(R) \geq 1$; $\co m(\cB_r(\vec{0})) \supseteq \cB_\delta(\vec{0})$ for some $\delta = \delta(r) > 0$; and $m(\vec{0}) = \vec{0}$. Then,
    \begin{equation}
        \cB_{r'}(\vec{0}) \subseteq \Phi^m \subseteq \cB_{ R'}(\vec{0}),
    \end{equation}
    where $r' \defeq \nicefrac{r}{2M(R)}$ and $R' \defeq R \left( \frac{2\sqrt{d}}{\delta(r)} + 1 \right)$.
\end{restatable}

We are now ready to provide a canonical, concrete example of deviations that satisfy~\Cref{assumption:kernel} under~\Cref{assumption:kernel-precise}. As we alluded to earlier in our introduction, it is the family of low-degree polynomials; in particular, it will be convenient to work with the Legendre basis.

\begin{definition}
    \label{def:polys}
    Let $P_0(x) = 1$ and $P_1(x) = x$. The $(\ell+1)$th \emph{Legendre polynomial} is given by the recurrence $(\ell + 1) P_{\ell + 1}(x) - (2\ell+1) x P_\ell(x) + \ell P_{\ell - 1}(x) = 0$.
\end{definition}

These polynomials have a convenient orthogonality property over $[-1, 1]$:
\begin{equation}
    \label{eq:orthog}
    \int_{-1}^1 P_{\ell}(x) P_{\ell'}(x) dx = 
    \begin{cases}
        \frac{1}{2\ell + 1} & \text{if } \ell = \ell',\\
        0 & \text{otherwise}.
    \end{cases}
\end{equation}
For convenience, we shall consider the rescaled polynomial $\lineP_\ell \defeq \sqrt{2\ell + 1} P_{\ell}$, so that $\int_{-1}^1 \lineP_\ell(x)^2 dx = 1$. We now define
\begin{equation}
    \label{eq:polynomials}
    m(\vx) \defeq \left( \prod_{j=1}^d \lineP_{\ell_j}(\vx[j]) \right)_{ 1 \leq \ell_1 + \dots \ell_d \leq \ell}.
\end{equation}

We establish (in~\Cref{sec:aux}) that~\Cref{assumption:kernel-precise} encompasses the above mapping.

\begin{restatable}{proposition}{lowdeg}
    \label{lemma:lowdeg}
    Let $m : \cX \to \R^{k'}$ per~\eqref{eq:polynomials}, where $k' = \binom{d + \ell}{\ell} - 1$. $\linem : \vx \mapsto m( \sqrt{d} \vx)$ satisfies~\Cref{assumption:kernel-precise} with $M \leq d^{O(\ell)}$  and $\delta = \nicefrac{1}{M}$.
\end{restatable}
\section{Polynomial-time expected fixed points and semi-separation}

We will now start connecting the framework we laid out in~\Cref{sec:eah} with the problem of computing $\Phi$-equilibria. That a $\Phi$-equilibrium can be cast as~\eqref{eq:eah}---by linking $\cY$ to the set of deviations $\Phi$---is not hard to see, and will be spelled out in the next section. This section concerns the question of implementing $\epsilon$-$\either$ (per~\Cref{def:either}), which is the main precondition of~\Cref{theorem:eah}.

The key to implementing the $\epsilon$-$\either$ oracle, and the main subject of this section, is the notion of an \emph{$\epsilon$-expected fixed point (EFP)} (\Cref{def:EFP}). This relaxes the usual notion of a fixed point that was employed by~\citet{Zhang24:Efficient}, who observed that, for minimizing $\Phi$-regret, one can replace a fixed-point oracle---as in the canonical framework of~\citet{Gordon08:No}---by an expected relaxation (per~\Cref{def:EFP}) thereof. This is crucial because, unlike fixed points which are intractable beyond linear maps, there is a simple, $O(1/\epsilon)$-time algorithm for computing $\epsilon$-expected fixed points. When it comes to computing $\Phi$-equilibria in games, our contribution here is twofold.

\begin{enumerate}
    \item we give a $\poly(d, \log(1/\epsilon))$-time algorithm for computing an $\epsilon$-expected fixed point, and\label{item:fast}
    \item we show that expected fixed points can be naturally coupled with the $\eah$ framework, and in particular, with the recent generalization of~\citet{Daskalakis24:Efficient}.\label{item:furious}
\end{enumerate}
This section establishes~\Cref{item:fast}, while the next section formalizes~\Cref{item:furious}. Going back to~\Cref{sec:eah} and the $\either$ oracle, the connection with (expected) fixed points lies in the observation that, when it comes to $\Phi$-equilibria in games, the $\ger$ part of the oracle can be implemented by computing an expected fixed point. This will become clear in the upcoming section.

\begin{definition}[Expected fixed points]
    \label{def:EFP}
    Let $\cX \subseteq \R^d$ be convex and compact and a function $\phi : \cX \to \cX$ to which we are given oracle access. The $\eps$-{\em expected fixed point (EFP)} problem asks for a distribution $\mu \in \Delta(\cX)$ such that\footnote{The choice of the $1$-norm (instead of, say, another $p$-norm) here is unimportant, because one can always take $\eps$ to be exponentially small.} 
    \begin{equation}
        \label{eq:EFP}
    \norm{\E_{\vx \sim \mu}[\phi(\vx)-\vx]}_1 \le \eps.
    \end{equation}
\end{definition}

This definition departs from the usual notion of a fixed point by measuring the fixed-point error \emph{in expectation} over samples $\vx$ from $\mu$. \Cref{def:EFP} is natural in its own right, but our key motivation is computational: as we shall see, an expected fixed point can be computed in polynomial time (\Cref{th:efp}).

We first observe that any function $\phi : \cX \to \cX$ admits an \emph{exact} expected fixed point; crucially, unlike Brouwer's fixed-point theorem, we do not assume that $\phi$ is continuous, and so expected fixed points exist even when fixed points do not.

\begin{proposition}
    Every function $\phi : \cX \to \cX$ admits an exact solution to the EFP problem.
\end{proposition}

\begin{proof}
    Let $G(\vx) := \phi(\vx) - \vx$. We want to show that $\co G(\cX)$ contains the origin. Suppose not. Then, by the separating hyperplane theorem, there is a vector $\vy \in \R^d$ with $\ip{\vy, G(\vx)} < 0$ for every $\vx \in \cX$. But this is impossible: simply consider $\vx^* \in \argmin_{\vx\in\cX} \ip{\vy, \vx}$.
\end{proof}

We now turn to our main computational result regarding EFPs; namely, a polynomial-time algorithm based on $\eah$---in particular, its incarnation in~\Cref{alg:eah}.

\begin{theorem}
    \label{th:efp}
    Given oracle access to $\cX$ and any $\phi : \cX \to \cX$, there exists a $\poly(d, \log(1/\eps))$-time algorithm that computes an $\epsilon$-EFP of $\phi$.
\end{theorem}

\begin{proof}
    We observe that an EFP can be equivalently expressed through the optimization problem
    \begin{equation}
    \qq{find} \mu \in \Delta(\cX) \qq{s.t.} \E_{\vx \sim \mu} \ip{\vy, \phi(\vx) - \vx } \ge 0 \quad \forall \vy \in [-1, 1]^d.
\end{equation}
We will now apply~\Cref{theorem:eah}. The set $\cY \defeq [-1, 1]^d$ clearly admits a separation oracle ($\sep$). Further, for any $\vy \in [-1, 1]^d$, taking $\vx^* = \argmin_{\vx \in \cX} \langle \vy, \vx \rangle $ (using an optimization oracle for $\cX$) guarantees $\langle \vy, \phi(\vx^*) - \vx^* \rangle \geq 0 $ since $\phi(\vx^*) \in \cX$, thereby implementing the $\ger$ oracle. We thus find that the preconditions of~\Cref{theorem:eah} are satisfied, and $\eah$ (\Cref{alg:eah}) returns $\mu \in \Delta(\cX)$, with $\supp(\mu) \leq \poly(d, \log(1/\epsilon))$, such that
\begin{equation}
    \E_{\vx \sim \mu} \langle \vy, \phi(\vx) - \vx \rangle \geq - \epsilon \quad \forall \vy \in [-1, 1]^d.
\end{equation}
Taking $\vy = \sign( \E_{\vx \sim \mu} (\vx - \phi(\vx)))$ (coordinate-wise) completes the proof.
\end{proof}

As it will become clear, \Cref{th:efp} yields a polynomial-time implementation of the $\ger$ oracle in the context of~\Cref{sec:eah}, which can be employed in~$\eah$. With a slight modification in the proof of~\Cref{th:efp}, we shall see how one can also recover an $\epsilon$-$\either$ oracle (\Cref{def:either}), which will then enable us to harness~\Cref{theorem:either} for computing $\Phi$-equilibria in games. Following the nomenclature of~\citet{Daskalakis24:Efficient}, we refer to this oracle as a \emph{semi-separation oracle}.

\begin{definition}[Semi-separation oracle]
    \label{def:semiseparation}
    The \emph{semi-separation} problem is the following. Given a convex and compact $\cX$ and a function $\phi : \cX \to \R^d$, compute
    \begin{enumerate}
        \item {\em either} a distribution $\mu \in \Delta(\cX)$ such that $\|\E_{\vx \sim \mu}[\phi(\vx)-\vx] \|_1 \leq \epsilon$, \label{item:efp}
        \item {\em or} a point $\vx\in\cX$ with $\phi(\vx) \notin \cX$.\label{item:mem}
    \end{enumerate}
\end{definition}

Unlike~\Cref{def:EFP}, here we allow $\phi$ to map outside of $\cX$. This more general framing is essential to arrive at the $\either$ oracle. In particular, we note that~\Cref{item:mem} yields a hyperplane separating $\phi$ from the set of endomorphisms on $\cX$. Namely, since $\phi(\vx) \notin \cX$, we can use the separation oracle on $\cX$ to separate $\cX$ from $\phi(\vx)$; that is, there is a $\vec{w}$ such that $\langle \phi(\vx), \vec{w} \rangle > \langle \vx, \vec{w} \rangle$ for all $\vx \in \cX$. But this also implies that $\langle \phi(\vx), \vec{w} \rangle > \langle \phi'(\vx), \vec{w} \rangle$ for any endomorphism $\phi'$, as promised.

\begin{theorem}
    \label{theorem:semiseparation}
    Given oracle access to $\cX$ and $\phi$, there is a $\poly(d, \log(1/\epsilon))$-time algorithm for implementing the semi-separation oracle of~\Cref{def:semiseparation}.
\end{theorem}

\begin{proof}
As in the proof of~\Cref{th:efp}, we proceed by running the ellipsoid algorithm (per~\Cref{alg:eah}) on the problem
\begin{align}
    \qq{find} \vy\in [-1,1]^d \qq{s.t.} \ip{\vy, \phi(\vx) - \vx} \le -\eps \quad \forall \vx \in \cX. \label{eq:efp-constraint}
\end{align}
For any $\vy \in [-1, 1]^d$ during the execution of the ellipsoid, take $\vx^*(\vy) \in \argmin_{\vx \in \cX} \ip{\vy, \vx}$. If $\phi(\vx^*(\vy)) \notin \cX$, the algorithm can terminate and return $\vx^*(\vy)$. Otherwise, it follows that $\langle \vy, \phi(\vx^*(\vy)) - \vx^*(\vy) \rangle \geq 0$, by definition of $\vx^*$, and so we can use $\vx^*$ to get a separation oracle for~\eqref{eq:efp-constraint}.

Now, if every $\vx^*(\vy) \in \cX$ generated above satisfies the constraint $\phi(\vx^*(\vy)) \in \cX$, then~\Cref{alg:eah} returns a certificate of infeasibility for~\eqref{eq:efp-constraint} in $\poly(d, \log(1/\epsilon))$ time, which is an $\epsilon$-expected fixed point of $\phi$. On the other hand, if at some point there is $\vy \in [-1, 1]^d$ such that $\phi(\vx^*(\vy)) \notin \cX$, then the algorithm returns a point $\vx^*(\vy) \in \cX$ such that $\phi(\vx^*(\vy)) \notin \cX$. This completes the proof.
\end{proof}

This semi-separation oracle amounts to the $\epsilon$-$\either$ oracle needed in~\Cref{theorem:either}, as we shall see next in the context of games. Compared to the semi-separation oracle of~\citet{Daskalakis24:Efficient} that only works for linear functions, ours (\Cref{theorem:semiseparation}) places no restrictions on $\phi$. 

\section{A polynomial-time algorithm for $\Phi^m$-equilibria in games}

Armed with the powerful semi-separation oracle of~\Cref{theorem:semiseparation}, we now establish a polynomial-time algorithm for computing $\Phi^m$-equilibria in general multilinear games (\Cref{theorem:main-eah-prec}).

Let us recall the basic setting of an $n$-player multilinear game $\Gamma$. Each player $i \in [n]$ has a convex and compact strategy set $\cX_i \subseteq \R^{d_i}$ in isotropic position (\Cref{sec:prel}). Player $i$ has a utility function $u_i : \cX_1 \times \dots \times \cX_n \to \R$ that is linear in $\cX_i$, so that $u_i(\vx) = \langle \vec{g}_i, \vx_i \rangle$ for some $\vec{g}_i = \vec{g}_i(\vx_{-i}) \in \R^{d_i}$. %Without loss of generality, we assume that $\vec 0 \notin \op{aff} \cX_i$, the affine hull of $\cX_i$, so that linear and affine functions on $\cX_i$ are interchangeable.
Furthermore, for each player $i \in [n]$, we let $\Phi^{m_i} \subseteq \cX_i^{\cX_i}$ be the $k_i$-dimensional set of deviations in the sense of~\Cref{assumption:kernel}; that is, there exists a function $m_i \in \cX_i \to \R^{k_i'}$, with $k_i = k_i' \cdot d_i + d_i$, such that for each $\phi_i \in \Phi^{m_i}$ and $\vx_i \in \cX_i$, the function output $\phi_i(\vx_i)$ can be expressed as the matrix-vector product $\mK_i(\phi_i) m_i(\vx_i) + \cons_i$ for some matrix $\mK_i \in \R^{d_i \times k_i'}$ and $\cons_i \in \R^{d_i}$. It is assumed throughout that $\Phi^{m_i}$ contains the identity map. For notational simplicity, we let $k \defeq \sum_{i=1}^n k_i$ and $d \defeq \sum_{i=1}^n d_i$.

In this context, we next state the main result of this section, and proceed with its proof.

\begin{theorem}[Precise version of~\Cref{theorem:main-eah}]
    \label{theorem:main-eah-prec}
    Consider an $n$-player multilinear game $\Gamma$ such that, for each player $i \in [n]$, we are given $\poly(n, k)$-time algorithms for the following:
\begin{itemize}
\item an oracle to compute the gradient, that is, the vector $\vec{g}_i = \vec{g}_i(\vx_{-i}) \in \R^{d_i}$ for which $\ip{\vg_i(\vx_{-i}), \vx_i} = u_i(\vx)$ for all $\vx \in \cX_1 \times \dots \times \cX_n$ (polynomial expectation property); and
\item a membership oracle for the strategy set $\cX_i$, assumed to be in isotropic position.
\end{itemize}
Suppose further that each $k_i$-dimensional set $\Phi^{m_i}$ satisfies~\Cref{assumption:kernel-precise} and $\|\vec{g}_i \| \leq B$. Then, an $\eps$-approximate $\Phi^m$-equilibrium of $\Gamma$ can be computed in $\poly\qty(n, k, \log(B/\eps))$ time.
\end{theorem}

\begin{proof}
An $\eps$-approximate $\Phi^m$-equilibrium of $\Gamma$  is a distribution $\mu \in \Delta(\cX_1 \times \dots \times \cX_n)$ such that
\begin{align}
\E_{\vx \sim \mu} \qty[u_i(\phi_i(\vx_i), \vx_{-i}) - u_i(\vx)] \le \eps\label{eq:phi-eqm}
\end{align}
for every player $i \in [n]$ and deviation $\phi_i \in \Phi^{m_i}$. Using multilinearity and~\Cref{assumption:kernel}, it suffices to find a distribution $\mu\in\Delta(\cX_1 \times \dots \times \cX_n)$ satisfying
\begin{align}
\E_{\vx \sim \mu} \left[ \sum_{i =1}^n \ip{\vg_i(\vx_{-i}), \mK_i m_i(\vx_i) + \cons_i - \vx_i} \right] \le \eps \label{eq:phi-sp}
\end{align}
for every $(\mK_1(\phi_1), \dots, \mK_n(\phi_n))$ and $(\cons_1(\phi_1), \dots, \cons_n(\phi_n))$, where $(\phi_1, \dots, \phi_n) \in \Phi^{m}$. (This derivation uses the fact that $\Phi^{m_i}$ contains the identity map.) We will now apply~\Cref{theorem:either} with respect to $\R^d \supseteq \cX \defeq \cX_1 \times \dots \times \cX_n$ and 
$$\R^{k} \supseteq \cY \defeq \{ (\mK_1, \cons_1, \dots, \mK_n, \cons_n) : \mK_i m_i(\vx_i) + \cons_i \in \cX_i \quad \forall \vx_i \in \cX_i \}. $$
By the polynomial expectation property, we can evaluate the term $\sum_{i =1}^n \ip{\vg_i(\vx_{-i}), \mK_i m_i(\vx_i) + \cons_i - \vx_i}$, for each $\vx \in \cX$, in $\poly(n, k)$ time. It thus suffices to show how to implement the $\epsilon$-$\either$ oracle, which yields a separation oracle for the program
\begin{alignat}{9}
\qq{find} \mK_1, \cons_1, \dots, \mK_n, \cons_n \qq{s.t.} \\
\sum_{i =1}^n \ip{\vg_i(\vx_{-i}), \mK_i \vm_i(\vx_i) + \cons_i - \vx_i} \ge - \eps &\quad\forall \vx \in \cX_1 \times \dots \times \cX_n, \label{eq:phi-sp}
\\
\mK_i \vm_i(\vx_i) + \cons_i \in \cX_i &\quad \forall\vx_i \in \cX_i.
\end{alignat}

Consider any $\R^{k} \ni \phi = (\mK_1, \cons_1, \dots, \mK_n, \cons_n)$. We apply the semi-separation oracle of~\Cref{theorem:semiseparation} for each function $\vx_i \mapsto \mK_i m_i(\vx_i) + \cons_i$. This returns \emph{either} an $\epsilon'$-expected fixed point, that is, a distribution $\nu_i \in \Delta(\cX_i)$ such that $$\left\| \E_{\vx_i \sim \nu_i} [ \mK_i \vm_i(\vx_i) + \cons_i - \vx_i ] \right\|_1 \leq \epsilon',$$
\emph{or} a point $\vx_i \in \cX_i$ such that $\mK_i m_i(\vx_i) + \cons_i \notin \cX_i$. If any of those semi-separation oracles returned $\vx_i \in \cX_i$ with $\mK_i m_i(\vx_i) + \cons_i \notin \cX_i$, we can use it to obtain a hyperplane separating $(\mK, \cons)$ from the set of deviations $\cY$. Otherwise, let $\nu \defeq \nu_1 \times \dots \times \nu_n \in \Delta(\cX_1) \times \dots \times \Delta(\cX_n)$ be the induced product distribution. Then, we have
\begin{align}
    & \E_{\vx \sim \nu}\sum_{i = 1}^n \ip{\vg_i(\vx_{-i}), \mK_i \vm_i(\vx_i) + \cons_i - \vx_i} 
    = \sum_{i =1}^n  \big\langle\E_{\vx \sim \nu} \vg_i(\vx_{-i}), \E_{\vx_i \sim \nu_i}[\mK_i \vm_i(\vx_i) + \cons_i - \vx_i]\big\rangle \leq n B \eps', \label{eq:eqm-final}
\end{align}
where we used the fact that $\nu$ is a product distribution in the equality above. Thus, we have identified an $(\epsilon' n B)$-approximate good-enough-response, yielding an $\epsilon$-$\either$ oracle by rescaling $\epsilon'$, and the proof follows from~\Cref{theorem:either}.
\end{proof}

\begin{algorithm}[!ht]
\caption{Polynomial-time algorithm for $\Phi^m$-equilibria}
\label{alg:main-eah}
\SetKwInOut{Input}{Input}
\SetKwInOut{Output}{Output}
\SetKw{Input}{Input:}
\SetKw{Output}{Output:}
\Input{
    \begin{itemize}[noitemsep,topsep=0pt]
        \item An $n$-player multilinear game $\Gamma$
        \item A precision parameter $\epsilon > 0$
        \item A membership oracle for each $\cX_i$
        \item An oracle for computing the gradient $\vec{g}_i = \vec{g}_i(\vx_{-i}) \in \R^{d_i}$ for each $i \in [n]$
        \item A $k_i$-dimensional set $\Phi^{m_i}$ under~\Cref{assumption:kernel-precise} for each $i \in [n]$
    \end{itemize}
}
\Output{An $\epsilon$-approximate $\Phi^m$-equilibrium of $\Gamma$ in $\poly(k, \log(1/\epsilon))$ time}\\
Define $G : \R^{d} \to \R^k$ such that $\langle G(\vx), (\mK, \cons) \rangle = \sum_{i =1}^n \ip{\vg_i(\vx_{-i}), \mK_i m_i(\vx_i) + \cons_i - \vx_i}$\;
Use the semi-separation oracle of~\Cref{theorem:semiseparation} to construct an $\epsilon$-$\either$ oracle $\cO$\;
 Apply~\Cref{alg:gen-eah} with $\cO$ as the $\epsilon$-$\either$ oracle
\end{algorithm}

It is worth stressing that it is crucial for our proof that the \emph{expected} VI problem (\emph{cf}.~\citet{Zhang25:Expected}) above corresponds to a game. It allows each player to be treated {\em independently}, which yields a {\em product distribution} $\nu = \nu_1 \times \dots \times \nu_n$ when we apply the semi-separation oracle of~\Cref{theorem:semiseparation} (for each player). That $\nu$ is a product distribution is crucial to implement the separation oracle for the dual because it allows us to push the expectation into the inner product in \eqref{eq:eqm-final}, as we saw in the last step of the proof.


\section{An efficient online algorithm for minimizing $\Phi^m$-regret}
\label{sec:reg}

We now switch gears to the online learning setting, recalled in~\Cref{sec:gordon}. Our main result, \Cref{theorem:main-prec}, is an efficient online algorithm for minimizing $\Phi^m$-regret with respect to any $\poly(d)$-dimensional set $\Phi^m$ (under~\Cref{assumption:kernel-precise}), which applies even in the adversarial regime.

In what follows, we build on the framework of~\citet{Daskalakis24:Efficient}, itself a refinement of the template of~\citet{Gordon08:No}. As we have seen, \citet{Daskalakis24:Efficient} showed that separating even over the set of linear endomorphisms is hard. In light of this, they proceed as follows. Instead of operating over the set of linear endomorphisms, their key idea is to consider a sequence of ``shell sets,'' each of which contains the original set. Each shell set must also satisfy two basic properties:
\begin{itemize}
    \item it is sufficiently structured so that it is possible to optimize over that set, and
    \item it contains a transformation with a fixed point inside $\cX$.
\end{itemize}
Here, we show that by replacing fixed points with \emph{expected} fixed points in the above template, it is possible to extend their main result to handle any $\poly(d)$-dimensional set under~\Cref{assumption:kernel-precise}.

\paragraph{Overview} Our main construction is~\Cref{alg:main}. It is an instantiation of~$\shellgd$ (\Cref{sec:shellgd}), which is projected gradient descent but with the twist that the constraint set is changing over time---reflecting the fact that a new shell set is computed at every round. To execute $\shellgd$, $\shellproj$ (\Cref{sec:shellproj}) provides an efficient projection oracle together with an approximate expected fixed point thereof, which is ultimately the output of our $\Phi^m$-regret minimizer. $\shellproj$ crucially relies on $\shellelips$, introduced next in~\Cref{sec:shellellips}. It strengthens our semi-separation oracle of~\Cref{theorem:semiseparation} by again using expected fixed points. \Cref{sec:put} combines those ingredients to arrive at our main result (\Cref{theorem:main-prec}).

\subsection{Shell ellipsoid}
\label{sec:shellellips}

Continuing from our semi-separation oracle of~\Cref{theorem:semiseparation}, $\shellelips$ (\Cref{alg:shellellipsoid}) takes a step further: it takes as input a convex set of transformations $\cF \subseteq \cB_D(\vec{0})$---for which we have efficient oracle access, unlike $\enfuns$---and returns \emph{either} a function $\phi \in \cF$ and an $\epsilon$-expected fixed point thereof in $\Delta(\cX)$, \emph{or} it provides a certificate---in the form of a polytope expressed as the intersection of a polynomial number of halfspaces---establishing that $\vol(\cF \cap \enfuns) \approx 0$. $\shellelips$ will be used later as part of the $\shellproj$ algorithm so as to shrink the shell set.

\begin{lemma}
    \label{lemma:shellellipsoid}
    For any $k$-dimensional convex set $\cF \subseteq \cB_{D}(\vec{0})$ with efficient oracle access and $\epsilon > 0$, $\shellelips(\cF)$ (\Cref{alg:shellellipsoid}) runs in time $\poly(k, \log(1/\epsilon), \log D)$, and 
    \begin{itemize}
        \item either it returns a transformation $\phi \in \cF$ with an $\epsilon$-expected fixed point in $\cX$,
        \item or it returns a polytope $\cQ \subseteq \R^k$, specified as the intersection of at most $\poly( k, \log(1/\epsilon), \log D)$ halfspaces, such that $\Phi^m \subseteq \cQ$ and $\vol(\cQ \cap \cF) < \epsilon$.
    \end{itemize}
\end{lemma}

Coupled with~\Cref{theorem:semiseparation} pertaining to the semi-separation oracle, the proof of correctness of~\Cref{lemma:shellellipsoid} is immediate. That $\cQ$ can be expressed as the intersection of a polynomial number of halfspaces follows from the usual analysis of ellipsoid, as in~\citet[Lemma 4.2]{Daskalakis24:Efficient}.

\begin{algorithm}[!ht]
\caption{$\shellelips(\cF)$}
\label{alg:shellellipsoid}
\SetKwInOut{Input}{Input}
\SetKwInOut{Output}{Output}
\SetKw{Input}{Input:}
\SetKw{Output}{Output:}
\Input{
    \begin{itemize}[noitemsep,topsep=0pt]
        \item Oracle access to convex set $\cX \subseteq \R^d$
        \item Oracle access to a $k$-dimensional convex set $\cF \subseteq \cB_D(\vec{0})$
        \item Precision parameter $\epsilon > 0$
    \end{itemize}
}
Initialize $\cE \defeq \cB_D(\vec{0})$ and $\cQ \defeq \R^k$\;
 \While{$\vol(\cE) \geq \epsilon$} {
    Set $\phi \in \cQ \cap \cF$ as the center of $\cE$\;
    Run the semi-separation oracle of~\Cref{theorem:semiseparation} with respect to $\phi$\;
    \If{it returned an $\epsilon$-expected fixed point $\mu \in \Delta(\cX)$ of $\phi$}{
        \textbf{return} $\phi$\;
    }
    \Else{
        Let $H$ be the halfspace returned by~\Cref{theorem:semiseparation} that separates $\phi$ from $\enfuns$\;
        Set $\cQ \defeq \cQ \cap H$\;
    }
    Set $\cE$ to be the minimum volume ellipsoid containing $\cQ \cap \cF$
 }
 \textbf{return} $\cQ$
\end{algorithm}

\subsection{Shell gradient descent}
\label{sec:shellgd}

Instead of minimizing external regret with respect to the set $\enfuns$, which is hard even under linear endomorphisms~\citep{Daskalakis24:Efficient}, the overarching idea is to run (projected) gradient descent but with respect to a sequence of changing shell sets, $(\tilY^{(t)})_{t=1}^T$, of $\enfuns$ (each of which contains $\enfuns$); this process, called~$\shellgd$, is given in~\Cref{alg:shellgd}. So long as $\enfuns \subseteq \tilY^{(t)}$, $\shellgd$ indeed minimizes external regret with respect to deviations in~$\enfuns$---of course, $\shellgd$ is not a genuine regret minimizer for $\enfuns$ in that it is allowed to output strategies not in $\enfuns$, but~\Cref{lemma:shellgd} below is in fact enough for the purpose of minimizing $\Phi^m$-regret.

\begin{lemma}[\citep{Daskalakis24:Efficient}]
    \label{lemma:shellgd}
    Suppose that the sequence of shell sets $(\tilY^{(t)})_{t=1}^T$ is such that $\enfuns \subseteq \tilY^{(t)} \subseteq \cB_D(\vec{0})$ for all $t \in [T]$. For any sequence of utilities $\vec{U}^{(1)}, \dots, \vec{U}^{(T)} \in [-1, 1]^k$, $\shellgd$ (\Cref{alg:shellgd}) satisfies
    \begin{equation}
        \max_{\vy^* \in \enfuns} \sum_{t=1}^T \langle \vy^* - \vy^{(t)}, \vec{U}^{(t)} \rangle \leq \frac{D^2}{2\eta} + \eta \sum_{t=1}^T \| \vec{U}^{(t)} \|^2.
    \end{equation}
\end{lemma}

\begin{algorithm}[!ht]
\caption{$\shellgd$~\citep{Daskalakis24:Efficient}}
\label{alg:shellgd}
\SetKwInOut{Input}{Input}
\SetKwInOut{Output}{Output}
\SetKw{Input}{Input:}
\SetKw{Output}{Output:}
\Input{Learning rate $\eta$, convex and compact sets $\tilY^{(1)}, \dots, \tilY^{(T)} \subseteq \cB_D(\vec{0})$}\;
Initialize $\vy^{(0)} \in \tilY^{(1)}$ and $\vec{U}^{(0)} \defeq \vec{0}$\;
 \For{$ t=1, \dots, T$} {
    Obtain efficient oracle access to $\tilY^{(t)}$\;
    Update $\vy^{(t)} \defeq \Pi_{\tilY^{(t)}}( \vy^{(t-1)} + \eta \vec{U}^{(t-1)})$\;
    Output $\vy^{(t)}$ as the next strategy and receive feedback $\vec{U}^{(t)} \in [-1, 1]^k$
 }
\end{algorithm}

\subsection{Shell projection}
\label{sec:shellproj}

To implement $\shellgd$, we will make use of $\shellproj$, the algorithm that is the subject of this subsection. There are two main desiderata for the sequence of shell sets taken as input in $\shellgd$. First, each shell set must be structured or simple enough to allow projecting onto it---this is the whole rationale of expanding $\enfuns$ through shell sets. But, of course, this is not enough, for one could just consider the entire space. The second crucial concern is that each transformation given by~$\shellgd$ needs to admit (approximate) expected fixed points, so as to use the framework of~\citet{Gordon08:No} (\Cref{theorem:gordon}) and minimize $\Phi^m$-regret. \Cref{lemma:shellproj} below, concerning $\shellproj$, shows how to accomplish that goal; its proof is similar to that of~\citet[Theorem 4.4]{Daskalakis24:Efficient}.

\begin{lemma}
    \label{lemma:shellproj}
    Let $\cX$ be a convex and compact set such that $\cB_{r}(\vec{0}) \subseteq \cX \subseteq \cB_R(\vec{0})$ and $\cM$ be a convex set such that $\enfuns \subseteq \cM \subseteq \cB_D(\vec{0})$. For any $\phi \in \cB_D(\vec{0}) \subseteq \R^k$ and $\epsilon > 0$, $\shellproj$ (\Cref{alg:shellproj}) runs in time $\poly(k, 1/\epsilon, R/r, D)$ and returns
    \begin{enumerate}
        \item a shell set $\tilPhi$ satisfying $\enfuns \subseteq \tilPhi$, expressed by intersecting $\cM$ with at most $\poly(d, k, 1/\epsilon, R/r, D)$ halfspaces, and\label{item:invar}
        \item a transformation $\tilphi \in \tilPhi$ such that $\| \tilphi - \Pi_{\tilPhi}(\phi) \| \leq \epsilon$, together with an $\epsilon$-expected fixed point of $\tilphi$, $\mu \in \Delta(\cX)$.\label{item:proj}
    \end{enumerate}
\end{lemma}

\begin{algorithm}[!ht]
\caption{$\shellproj_\Phi(\phi)$ projects $\phi$ to a shell of $\Phi$}
\label{alg:shellproj}
\SetKwInOut{Input}{Input}
\SetKwInOut{Output}{Output}
\SetKw{Input}{Input:}
\SetKw{Output}{Output:}
\Input{
    \begin{itemize}[noitemsep,topsep=0pt]
        \item Convex and compact set $\cX \subseteq \R^d$ such that $\cB_r(\vec{0}) \subseteq \cX \subseteq \cB_R(\vec{0})$
        \item Convex set $\cM$ such that $\enfuns \subseteq \cM \subseteq \cB_D(\vec{0})$
        \item Transformation $\phi \in \cB_D(\vec{0})$
        \item Precision parameter $\epsilon > 0$
    \end{itemize}
}
\Output{
    \begin{itemize}[noitemsep,topsep=0pt]
        \item Convex set $\tilPhi$ such that $\enfuns \subseteq \tilPhi \subseteq \cM$
        \item Transformation $\tilphi \in \tilPhi$ such that $\| \tilphi - \Pi_{\tilPhi}(\phi) \| \leq \epsilon$
        \item An $\epsilon$-expected fixed point $\mu \in \Delta(\cX)$ of $\tilphi$
    \end{itemize}
}
Set $\epsilon' = \frac{\epsilon r}{32 M(R) D^2}$\;
Initialize $\tilPhi \defeq \cM$\;
 \For{$q = 0, \dots$ incremented by $\delta \defeq \nicefrac{\epsilon}{4D} $} {
    Run $\shellelips( \tilPhi \cap \cB_q(\phi))$ with precision $\vol(\cB_{\epsilon'}(\cdot))$\;
    \If{it finds $\tilphi$ with an $\epsilon$-expected fixed point $\mu \in \Delta(\cX)$}{
        \textbf{return} $\tilPhi, \tilphi, \mu$
    }
    \Else{
        Let $\cQ$ be the polytope returned by $\shellelips$\;
        Set $\tilPhi \defeq \tilPhi \cap \cQ$
    }
 }
\end{algorithm}

\subsection{Putting everything together}
\label{sec:put}

We now combine all the previous pieces to obtain an efficient algorithm for minimizing $\Phi^m$-regret---when $\Phi^m$ is $\poly(d)$-dimensional---under a general convex and compact set $\cX$. The overall construction is depicted in~\Cref{alg:main}. In effect, it runs $\shellgd$ with respect to the sequence of shell sets $(\tilPhi^{(t)} )_{t=1}^T$. Indeed, by the correctness guarantee of $\shellproj$ (\Cref{item:invar} of~\Cref{lemma:shellproj}), we have the invariance $\Phi(\cX) \subseteq \tilPhi^{(t)}$ for all $t \in [T]$. Furthermore, \Cref{item:proj} of~\Cref{lemma:shellproj} implies that $(\mK^{(t+1)}, \cons^{(t+1)}) \in \tilPhi^{(t+1)}$, returned by $\shellproj$ in~\Cref{alg:main}, is within distance $\epsilon$ of the projection prescribed by~$\shellgd$. As a result, we can apply~\Cref{lemma:shellgd} (up to some some slackness proportional to $\epsilon$) to bound the external regret $\reg^{(T)}_{\Phi^m}$ of $((\mK^{(t)}, \cons^{(t)}))_{t=1}^T$ with respect to comparators from $\enfuns$; combined with the fact that $\mu^{(t)} \in \Delta(\cX)$ is an $\epsilon$-expected fixed point of the function $\vx \mapsto \mK^{(t)} m(\vx) + \cons^{(t)}$ (as promised by~\Cref{item:proj}), it follows that the $\Phi^m$-regret of the learner (\Cref{alg:main}) can be bounded by $\reg^{(T)}_{\Phi^m} + \epsilon T$ (as in~\Cref{theorem:gordon}). We thus arrive at our main result.

\begin{theorem}[Precise version of~\Cref{theorem:main1}]
    \label{theorem:main-prec}
    Let $\cX \subseteq \R^d$ be a convex and compact set in isotropic position for which we have a membership oracle. \Cref{alg:main} has per-round running time of $\poly(k, T)$ and guarantees average $\Phi^m$-regret of at most $\poly(k) / \sqrt{T}$, where $k$ is the dimension of $\Phi^m$ under~\Cref{assumption:kernel-precise}.
\end{theorem}

Unlike the algorithm of~\citet{Daskalakis24:Efficient}, a salient aspect of~\Cref{alg:main} is that it outputs a sequence of \emph{mixed} strategies in $\Delta(\cX)$. As we saw earlier in~\Cref{sec:gordon}, this turns out to be necessary: \citet{Zhang24:Efficient} showed that a learner restricted to output strategies in $\cX$ cannot efficiently minimize $\Phi$-regret even with respect to low-degree swap deviations (assuming $\PPAD \neq \P$).

\begin{algorithm}[!ht]
\caption{$\Phi^m$-regret minimizer for convex strategy sets}
\label{alg:main}
\SetKwInOut{Input}{Input}
\SetKwInOut{Output}{Output}
\SetKw{Input}{Input:}
\SetKw{Output}{Output:}
\Input{
    \begin{itemize}[noitemsep,topsep=0pt]
        \item Convex and compact set $\cX \subseteq \R^d$ in isotropic position
        \item $k$-dimensional set $\Phi^m$ under~\Cref{assumption:kernel-precise} with respect to $m : \cX \to \R^{k'}$, where $k = k' \cdot d + d$
        \item time horizon $T \in \N$
    \end{itemize}
}
\Output{An efficient $\Phi^m$-regret minimizer for $\cX$}\\
Set the learning rate $\eta \propto \frac{1}{\sqrt{T}}$ and $\epsilon = \nicefrac{1}{\poly(k,T)}$ to be sufficiently small\;
Initialize $\mu^{(1)} \in \Delta(\cX)$ and $\mK^{(1)} \defeq \mI_{d \times k'}$ to be the identity map and $\cons^{(1)} \defeq \vec{0}$\;
Initialize $\cM \defeq \cB_{R}(\vec{0})$ for a large enough $R \leq \poly(k)$\;
 \For{$ t=1, \dots, T$} {
    Output $\mu^{(t)} \in \Delta(\cX)$ and receive feedback $\vu^{(t)} \in [-1, 1]^d$\;
    Define $\R^{d \times k' + d} \ni \mU^{(t)} \defeq (\E_{\vx^{(t)} \sim \mu^{(t)}} \vu^{(t)} \otimes m(\vx^{(t)}), \vec{u}^{(t)})$\;
    Set $\tilPhi^{(t+1)}, (\mK^{(t+1)}, \cons^{(t+1)}), \mu^{(t+1)} \defeq \shellproj_{\Phi}((\mK^{(t)}, \cons^{(t)}) + \eta \mU^{(t)})$ with input $\cM$ and precision $\epsilon$, where $\mu^{(t+1)} \in \Delta(\cX)$ is an $\epsilon$-expected fixed point of $\vx \mapsto \mK^{(t+1)} m(\vx) + \cons^{(t+1)}$ 
 }
\end{algorithm}

Finally, we conclude by providing a lower bound that matches our upper bound (\Cref{theorem:main-prec}) up to a constant factor in the exponent of $k$. It is based on the following normal-form lower bound due to~\citet{Dagan24:From} and~\citet{Peng24:Fast}.

\begin{theorem}[\citep{Dagan24:From,Peng24:Fast}]
    \label{theorem:lowerknown}
    Consider a learner operating on the simplex $\Delta(\cA)$. For any $T < |\cA|/4$, there is an adversary that forces the swap regret of the learner to be $\Omega(\log^{-6}T)$.
\end{theorem}

We observe that there is a simple way to parameterize the above lower bound in terms of the dimension of the set of deviations:

\begin{corollary}
    \label{cor:parlower}
    Consider a learner operating on the simplex $\Delta(\cA)$. There is a $k$-dimensional set of deviations $\Phi \subseteq \Delta(\cA)^{\Delta(\cA)}$ such that for any $T < \sqrt{k}/4$, there is an adversary that forces the $\Phi$-regret of the learner to be $\Omega(\log^{-6} T)$.
\end{corollary}

Indeed, one can first identify an arbitrary subset $\cA'$ of $\cA$ with cardinality $\sqrt{k}$, and then employ the adversary of~\Cref{theorem:lowerknown} with respect to $\cA'$ while rendering all other actions dominated by assigning to them very small utility. That $\Phi$ is $k$-dimensional in this case follows because the set of stochastic matrices mapping $\Delta(\cA')$ to $\Delta(\cA')$---which contains all relevant swap deviations---is $(\sqrt{k})^2$-dimensional.

Combining~\Cref{cor:parlower} with the recent reduction of~\citet{Daskalakis24:Lower}, which embeds the normal-form game lower bound of~\Cref{theorem:lowerknown} into an extensive-form game, we arrive at~\Cref{theorem:mainlower}, which we restate below.

\lowerbound*
\section{Conclusion}
In this work, we propose a simple yet effective approach, called SMILE, for graph few-shot learning with fewer tasks. Specifically, we introduce a novel dual-level mixup strategy, including within-task and across-task mixup, for enriching the diversity of nodes within each task and the diversity of tasks. Also, we incorporate the degree-based prior information to learn expressive node embeddings. Theoretically, we prove that SMILE effectively enhances the model's generalization performance. Empirically, we conduct extensive experiments on multiple benchmarks and the results suggest that SMILE significantly outperforms other baselines, including both in-domain and cross-domain few-shot settings.

\section*{Acknowledgments}
T.S. is supported by the Vannevar Bush Faculty Fellowship ONR N00014-23-1-2876, National Science Foundation grants RI-2312342 and RI-1901403, ARO award W911NF2210266, and NIH award A240108S001. B.H.Z. is supported 
by the CMU Computer Science Department Hans Berliner
PhD Student Fellowship. E.T, R.E.B., and V.C. thank the Cooperative AI Foundation, Polaris Ventures (formerly the Center for
Emerging Risk Research) and Jaan Tallinn’s donor-advised fund at Founders Pledge for financial
support. E.T. and R.E.B. are also supported in part by the Cooperative AI PhD Fellowship. G.F is supported by the National Science Foundation grant CCF-2443068. We are indebted to Constantinos Daskalakis and Noah Golowich for many insightful discussions concerning the complexity of computing expected fixed points.

\bibliography{dairefs}

\clearpage
\appendix

\section{Sufficiency of regret minimization in isotropic position}
\label{sec:isotropic}

Throughout the paper, we have assumed that we are minimizing $\Phi$-regret with respect to a convex set $\cX$ that is in isotropic position. \Cref{lemma:isotropic} below shows that this is without any loss. The argument here is similar to~\citet[Lemma A.1]{Daskalakis24:Efficient}, with the minor modification that we need to account for mixed strategies.

\begin{lemma}
    \label{lemma:isotropic}
    Let $\cX \subseteq \R^d$ be a convex and compact set such that $\cX \subseteq \cB_R(\vec{0})$. Let $\psi : \cX \to \tilde{\cX}$ be an invertible affine transformation such that $\tilde{\cX} = \psi(\cX)$ is in isotropic position. Suppose that we have a regret minimizer $\regbox_{\tilde{\cX}}$ for $\tilde{\cX}$ that incurs $\phireg_{\tilde{\cX}}^{(T)}$. Then, using $\poly(d)$ time in each round, we can construct a regret minimizer $\regbox_{\cX}$ for $\cX$ that incurs $\phireg^{(T)}_{\cX} \leq 2R \sqrt{d} \cdot \phireg^{(T)}_{\tilde{\cX}}$.
\end{lemma}

\begin{proof}
    Let $\psi(\vx) \defeq \mA \vx + \vec{b}$ for an invertible $\mA \in \R^{d \times d}$ and $\vec{b} \in \R^d$. Let $\vu^{(1)}, \dots, \vu^{(T)}$ be the sequence of utilities given as input to $\regbox_{\cX}$. We then provide as input to $\regbox_{\tilde{\cX}}$ the sequence of utilities
    \begin{equation}
        \label{eq:new-utils}
    \tilde{\vu}^{(t)} \defeq \frac{1}{2R \sqrt{d}} ( \mA^{-1} )^\top \vu^{(t)} \quad t = 1, \dots, T.
    \end{equation}
    Since $\tilde{\cX}$ contains the ball $\cB_1(\vec{0})$ and $\cX \subseteq \cB_R(\vec{0})$, it follows that $\| \mA^{-1} \vx \| \leq 2R$ for any $\vx \in \cB_1(\vec{0})$, which implies that the spectral norm of $\mA^{-1}$ is at most $2 R$. As a result, assuming that $\vu^{(t)} \in [-1, 1]^d$ for all $t \in [T]$, it follows that the utilities constructed in~\eqref{eq:new-utils} are also in $[-1, 1]^d$.
    
    Now, suppose that $\regbox_{\tilde{\cX}}$ returns the sequence of strategies $\tilde{\mu}^{(1)}, \dots, \tilde{\mu}^{(T)} \in \Delta(\tilde{\cX})$. We define, for each $t \in [T]$, $\mu^{(t)} \defeq \psi^{-1}(\tilde{\mu}^{(t)})$ as the next strategy. Consider any $\phi \in \Phi$, and define $\tilde{\phi} : \tilde{\cX} \ni \vx \mapsto \psi(\phi(\psi^{-1}(\vx))) \in \Phi(\tilde{\cX})$. Then,
    \begin{align}
        &\sum_{t=1}^T \left\langle \vu^{(t)}, \E_{\tilde{\vx}^{(t)} \sim \tilde{\mu}^{(t)}} \psi^{-1}(\tilde{\vx}^{(t)}) - \phi(\psi^{-1}(\tilde{\vx}^{(t)})) \right\rangle \\
        &= 2R \sqrt{d} \sum_{t=1}^T \left\langle \frac{1}{2R \sqrt{d}} 
(\mA^{-1})^\top \vu^{(t)}, \E_{\tilde{\vx}^{(t)} \sim \tilde{\mu}^{(t)}} \mA \psi^{-1}(\tilde{\vx}^{(t)}) - \mA \phi(\psi^{-1}(\tilde{\vx}^{(t)})) \right\rangle \\
&= 2R \sqrt{d} \sum_{t=1}^T \left\langle \frac{1}{2R \sqrt{d}} 
(\mA^{-1})^\top \vu^{(t)}, \E_{\tilde{\vx}^{(t)} \sim \tilde{\mu}^{(t)}} (\mA \psi^{-1}(\tilde{\vx}^{(t)}) + \vec{b}) - (\mA \phi(\psi^{-1}(\tilde{\vx}^{(t)})) + \vec{b}) \right\rangle \\
&= 2R \sqrt{d} \sum_{t=1}^T \left\langle \tilde{\vu}^{(t)}, \E_{\tilde{\vx}^{(t)} \sim \tilde{\mu}^{(t)}} \tilde{\vx}^{(t)}  - \tilde{\phi}(\tilde{\vx}^{(t)}) \right\rangle \leq 2R \sqrt{d} \cdot \phireg^{(T)}_{\tilde{\cX}}.
    \end{align}
\end{proof}
\section{Geometric properties of $\Phi^m$}
\label{sec:wellbounded}

In this section, we establish that the set $\Phi^m$ per~\Cref{assumption:kernel-precise} is geometrically well behaved, which is necessary to execute the ellipsoid algorithm (as well as the online learning setting). In particular, our goal is to prove \Cref{lemma:wellbounded}.

Below, for convex and compact $\cA, \cB \subseteq \R^d$, we use the notation 
\begin{equation}
    \Phi^m(\cA, \cB) \defeq \left\{ (\mK, \cons) \in \R^{k + d} : \mK m(\vx) + \cons \in \cB \quad \forall \vx \in \cA \right\}.
\end{equation}

\begin{lemma}
    \label{lemma:trivial}
    Let $\cA, \cB, \cC, \cD$ be convex and compact sets. If $\cA \supseteq \cC$ and $\cB \subseteq \cD$, then $\Phi^m(\cA, \cB) \subseteq \Phi^m(\cC, \cD)$.
\end{lemma}

\begin{proof}
    Consider any $(\mK, \cons) \in \Phi^m(\cA, \cB)$. By definition, it holds that $\mK \vx + \cons \in \cB$ for all $\vx \in \cA$. Since $\cC \subseteq \cA$, it follows that $\mK \vx + \cons \in \cB$ for all $\vx \in \cC$, and in particular, $\mK \vx + \cons \in \cD$ since $\cB \subseteq \cD$.
\end{proof}

\begin{lemma}
    \label{lemma:boundM}
    Let $\cX \subseteq \R^d$ be a convex and compact set such that $\cB_r(\vec{0}) \subseteq \cX \subseteq \cB_R(\vec{0})$, with $R \geq 1$. Suppose further that $\|m(\vx)\| \leq M$ for all $\vx \in \cB_R(\vec{0})$, where $M = M(R) \geq 1$. Then, $\Phi^m \supseteq \cB_{r'}(\phi_{\vec{0}})$, where $r' \defeq \nicefrac{r}{2M(R)}$ and $\phi_{\vec{0}} \defeq (\mathbf{0}, \vec{0})$ is the constant transformation $\vx \mapsto \vec{0}$.
\end{lemma}

\begin{proof}
    By~\Cref{lemma:trivial}, it suffices to prove $\cB_{r'}(\phi_{\vec{0}}) \subseteq \Phi^m(\cX, \cB_r(\vec{0}))$. Consider any $(\mK, \cons) \in \cB_{r'}(\phi_{\vec{0}})$, which means that $\| \mK \|^2_F + \|\cons \|^2 \leq ( \nicefrac{r}{2M(R)})^2$. Then, for any $\vx \in \cX$,
    \begin{equation}
        \| \mK m(\vx) + \cons \| \leq \| \mK \|_F \|m(\vx)\| + \|\cons  \| \leq r.
    \end{equation}
    This means that $\cB_{r'}(\phi_{\vec{0}}) \subseteq \Phi^m(\cX, \cB_r(\vec{0}))$, and the proof follows.
\end{proof}

\begin{lemma}
    \label{lemma:fulldim}
    Suppose that $\co m(\cB_r(\vec{0})) \supseteq \cB_\delta(\vec{0})$ for some $\delta = \delta(r) > 0$ and $m(\vec{0}) = \vec{0}$. Then, assuming that $r < R$,
    \begin{equation}
        \Phi^m( \cB_r(\vec{0}), \cB_R(\vec{0})) \subseteq \cB_{ R'}(\vec{0}),
    \end{equation}
    where $R' \defeq R \left( \frac{2\sqrt{d}}{\delta(r)} + 1 \right)$.
\end{lemma}

\begin{proof}
    Consider any $(\mK, \cons) \in \Phi^m( \cB_r(\vec{0}), \cB_R(\vec{0}))$. By definition, we have $\| \mK m(\vx) + \cons \| \leq R$ for all $\vx \in \cB_r(\vec{0})$. Since $m(\vec{0}) = \vec{0}$, it follows that $\|\cons \| \leq R$. Thus, $\| \mK m(\vx) \| \leq \|\mK m(\vx) + \cons \| + \|\cons \| \leq 2 R$ for all $\vx \in \cB_r(\vec{0})$. Now, let $\vec{x}' \in \R^{k'}$ with $\|\vec{x}' \| = 1$ be such that $\| \mK \vec{x}' \| = \| \mK\|$, where $\| \mK \|$ is the spectral norm of $\mK$. Since we have assumed that $\co m(\cB_r(\vec{0})) \supseteq \cB_\delta(\vec{0})$, it follows that there exist $\lambda_1, \dots, \lambda_{k'+1}$, with $\lambda_1, \dots, \lambda_{k' + 1} \geq 0$ and $\sum_{j=1}^{k'+1} \lambda_j = 1$, and $\vx_1, \dots, \vx_{k'+1} \in \cB_r(\vec{0})$ (by Carath\'eodory's theorem) such that $\sum_{j=1}^{k'+1} \lambda_j m(\vx_j) = \delta \vx'$. As a result,
    \begin{equation}
        \delta \|\mK\| = \| \mK (\delta \vx') \| = \left\| \mK \left( \sum_{j=1}^{k'+1} \lambda_j m(\vx_j) \right) \right\| \leq \sum_{j=1}^{k' + 1} \lambda_j \|\mK m(\vx_j) \| \leq 2R.
    \end{equation}
    Finally, we have $\|\mK\|_F \leq \sqrt{d} \|\mK\|$, and the claim follows.
\end{proof}

\begin{proof}[Proof of~\Cref{lemma:wellbounded}]
    The claim follows directly by combining~\Cref{lemma:trivial,lemma:boundM,lemma:fulldim}.
\end{proof}

\section{Further omitted proofs}
\label{sec:aux}

Finally, this section provides the proof of~\Cref{lemma:lowdeg}, which we restate below. For completeness, we have also included the usual version of $\eah$ (\Cref{alg:eah}, subsumed by~\Cref{alg:gen-eah}), which we used earlier in~\Cref{theorem:semiseparation,th:efp}.

\lowdeg*

For the proof, we will use a simple, auxiliary lemma.

\begin{lemma}
    \label{lemma:range}
    Let $X$ be a random variable such that $\E[X] = 0$, $\V[X] = 1$, and $X \in [-R, R]$ almost surely. Then, $\pr[X \geq \nicefrac{1}{R}] > 0$.
\end{lemma}

\begin{proof}[Proof of~\Cref{lemma:lowdeg}]
    It is clear that $\linem(\vec{0}) = \vec{0}$. The bound on $M$ is also immediate. We thus focus on proving that $\delta \defeq 1/M$ suffices.
    
    For the sake of contradiction, suppose that $\co \linem(\cB_1(\vec{0}))$ does not contain $\vx'$ for some $\vx' \in \R^{k'}$ with $\| \vx' \| \leq \delta$. Then, we consider a hyperplane that separates $\co \linem(\cB_1(\vec{0}))$ from $\vx'$, and we let $\vec{v}$ be the normal vector to that hyperplane, so that $\langle \vec{v}, \vx' \rangle > \langle \vec{v}, \linem(\vx) \rangle$ for all $\vx \in \cB_1(\vec{0})$. Now, let $\cU$ be the uniform product distribution over $[-1, 1]^d$. By~\eqref{eq:orthog}, we have $\E_{\vx \sim \cU} [m(\vx)] = 0$ and $\E_{ \vx \sim \cU} [ m(\vx) m(\vx)^\top] = \mI_{k' \times k'}$ (by ortogonality). As a result, we have $\E_{ \vx \sim \cU} [  \langle \vec{v}, m(\vx) \rangle^2 ] = \mathbb{V}_{\vx \sim \cU}[ \langle \vec{v}, m(\vx) \rangle ] = \|\vec{v} \|^2 = 1$. \Cref{lemma:range}, applied for the random variable $\langle \vec{v}, m(\vx) \rangle$ with range $[-M, M]$, implies that there exists $\vx \in [-1, 1]^d$ such that $\langle \vec{v}, m(\vx) \rangle \geq 1/M$, which in turn implies that there exists $\linex \in \cB_1(\vec{0})$---namely, $\linex \defeq \vx/\sqrt{d}$---such that $\langle \vec{v}, \linem(\linex) \rangle \geq 1/M$. But this yields $\delta \leq \langle \vec{v}, \linem(\linex) \rangle < \langle \vec{v}, \vx' \rangle \leq \|\vec{v} \| \|\vx'\| = \delta$, a contradiction.
\end{proof}

\begin{algorithm}[!ht]
\caption{Ellipsoid against hope ($\eah$)~\citep{Papadimitriou08:Computing}}
\label{alg:eah}
\SetKwInOut{Input}{Input}
\SetKwInOut{Output}{Output}
\SetKw{Input}{Input:}
\SetKw{Output}{Output:}
\Input{\begin{itemize}[noitemsep,topsep=0pt]
        \item Parameters $R_y, r_y > 0$ such that $\cB_{r_y}(\cdot) \subseteq \cY \subseteq \cB_{R_y}(\vec{0})$
        \item Precision parameter $\epsilon > 0$
        \item Parameter $B > 0 $ such that $\| G(\vx) \| \leq B$ for all $\vx \in \cX$
        \item $\ger$ oracle for~\eqref{eq:eah}
        \item $\sep$ oracle for $\cY$
    \end{itemize}
}
\Output{A sparse, $\epsilon$-approximate solution $\mu \in \Delta(\cX)$ of~\eqref{eq:eah}}
Initialize the ellipsoid $\cE \defeq \cB_{R_y}(\vec{0})$\;
\While{$\vol(\cE) \geq \vol(\cB_{\epsilon/B}(\cdot))$}{
    Let $\vy \in \R^k$ be the center of $\cE$ \;
    \eIf{$\vy \in \cY$}{
        Let $\vx \in \cX$ be a good-enough-response with respect to $\vy$ (via the $\ger$ oracle)\;
        Update $\cE$ to the minimum volume ellipsoid containing $\cE \cap \{ \vy \in \R^k : \langle \vy, G(\vx) \rangle \leq 0 \}$\;
    }{
        Let $H$ be the halfspace that separates $\vy$ from $\cY$ (via the $\sep$ oracle) \;
        Update $\cE$ to the minimum volume ellipsoid containing $\cE \cap H$\;
    }
}
Let $\vx^{(1)}, \dots, \vx^{(T)}$ be the $\ger$ oracle responses produced in the process above\;
Define $\mG \defeq [G(\vx^{(1)}) \mid \hdots \mid G(\vx^{(T)})] \in \R^{k \times T} $\;
Compute a solution $\vec{\lambda}$ to the convex program
    $$\qq{find} \vec{\lambda} \in \Delta^T \qq{s.t.} \min_{\vy \in \cY} \vec{\lambda}^\top \mG^\top \vy \geq - \epsilon$$
\Return{$\Delta(\cX) \ni \mu \defeq \sum_{t=1}^T \lambda^{(t)} \mu(\vx^{(t)})$}
\end{algorithm}

\end{document}
