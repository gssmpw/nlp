\section{Conclusions and open problems}

In summary, we established efficient algorithms for minimizing $\Phi$-regret and computing $\Phi$-equilibria with respect to any set of deviations with a polynomial dimension. For the online learning setting, our upper bounds are tight up to constant factors in the exponents, crystallizing for the first time a family of deviations that characterizes the learnability of $\Phi$-regret. 

There are many important avenues for future research. First, we did not attempt to optimize the (polynomial) dependence of the running time (in~\Cref{theorem:main-prec,theorem:main-eah-prec}) on $k$ and $d$; improving the overall complexity of our algorithms is an interesting direction. Moreover, developing more practical algorithms---that refrain from using ellipsoid---would also be a valuable contribution. In particular, are there polynomial-time algorithms for computing $\Phi$-equilibria without resorting to the $\eah$ framework? But the most pressing open question is to understand the complexity of computing (normal-form) correlated equilibria in the centralized model.