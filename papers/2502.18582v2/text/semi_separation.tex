\section{Polynomial-time expected fixed points and semi-separation}

We will now start connecting the framework we laid out in~\Cref{sec:eah} with the problem of computing $\Phi$-equilibria. That a $\Phi$-equilibrium can be cast as~\eqref{eq:eah}---by linking $\cY$ to the set of deviations $\Phi$---is not hard to see, and will be spelled out in the next section. This section concerns the question of implementing $\epsilon$-$\either$ (per~\Cref{def:either}), which is the main precondition of~\Cref{theorem:eah}.

The key to implementing the $\epsilon$-$\either$ oracle, and the main subject of this section, is the notion of an \emph{$\epsilon$-expected fixed point (EFP)} (\Cref{def:EFP}). This relaxes the usual notion of a fixed point that was employed by~\citet{Zhang24:Efficient}, who observed that, for minimizing $\Phi$-regret, one can replace a fixed-point oracle---as in the canonical framework of~\citet{Gordon08:No}---by an expected relaxation (per~\Cref{def:EFP}) thereof. This is crucial because, unlike fixed points which are intractable beyond linear maps, there is a simple, $O(1/\epsilon)$-time algorithm for computing $\epsilon$-expected fixed points. When it comes to computing $\Phi$-equilibria in games, our contribution here is twofold.

\begin{enumerate}
    \item we give a $\poly(d, \log(1/\epsilon))$-time algorithm for computing an $\epsilon$-expected fixed point, and\label{item:fast}
    \item we show that expected fixed points can be naturally coupled with the $\eah$ framework, and in particular, with the recent generalization of~\citet{Daskalakis24:Efficient}.\label{item:furious}
\end{enumerate}
This section establishes~\Cref{item:fast}, while the next section formalizes~\Cref{item:furious}. Going back to~\Cref{sec:eah} and the $\either$ oracle, the connection with (expected) fixed points lies in the observation that, when it comes to $\Phi$-equilibria in games, the $\ger$ part of the oracle can be implemented by computing an expected fixed point. This will become clear in the upcoming section.

\begin{definition}[Expected fixed points]
    \label{def:EFP}
    Let $\cX \subseteq \R^d$ be convex and compact and a function $\phi : \cX \to \cX$ to which we are given oracle access. The $\eps$-{\em expected fixed point (EFP)} problem asks for a distribution $\mu \in \Delta(\cX)$ such that\footnote{The choice of the $1$-norm (instead of, say, another $p$-norm) here is unimportant, because one can always take $\eps$ to be exponentially small.} 
    \begin{equation}
        \label{eq:EFP}
    \norm{\E_{\vx \sim \mu}[\phi(\vx)-\vx]}_1 \le \eps.
    \end{equation}
\end{definition}

This definition departs from the usual notion of a fixed point by measuring the fixed-point error \emph{in expectation} over samples $\vx$ from $\mu$. \Cref{def:EFP} is natural in its own right, but our key motivation is computational: as we shall see, an expected fixed point can be computed in polynomial time (\Cref{th:efp}).

We first observe that any function $\phi : \cX \to \cX$ admits an \emph{exact} expected fixed point; crucially, unlike Brouwer's fixed-point theorem, we do not assume that $\phi$ is continuous, and so expected fixed points exist even when fixed points do not.

\begin{proposition}
    Every function $\phi : \cX \to \cX$ admits an exact solution to the EFP problem.
\end{proposition}

\begin{proof}
    Let $G(\vx) := \phi(\vx) - \vx$. We want to show that $\co G(\cX)$ contains the origin. Suppose not. Then, by the separating hyperplane theorem, there is a vector $\vy \in \R^d$ with $\ip{\vy, G(\vx)} < 0$ for every $\vx \in \cX$. But this is impossible: simply consider $\vx^* \in \argmin_{\vx\in\cX} \ip{\vy, \vx}$.
\end{proof}

We now turn to our main computational result regarding EFPs; namely, a polynomial-time algorithm based on $\eah$---in particular, its incarnation in~\Cref{alg:eah}.

\begin{theorem}
    \label{th:efp}
    Given oracle access to $\cX$ and any $\phi : \cX \to \cX$, there exists a $\poly(d, \log(1/\eps))$-time algorithm that computes an $\epsilon$-EFP of $\phi$.
\end{theorem}

\begin{proof}
    We observe that an EFP can be equivalently expressed through the optimization problem
    \begin{equation}
    \qq{find} \mu \in \Delta(\cX) \qq{s.t.} \E_{\vx \sim \mu} \ip{\vy, \phi(\vx) - \vx } \ge 0 \quad \forall \vy \in [-1, 1]^d.
\end{equation}
We will now apply~\Cref{theorem:eah}. The set $\cY \defeq [-1, 1]^d$ clearly admits a separation oracle ($\sep$). Further, for any $\vy \in [-1, 1]^d$, taking $\vx^* = \argmin_{\vx \in \cX} \langle \vy, \vx \rangle $ (using an optimization oracle for $\cX$) guarantees $\langle \vy, \phi(\vx^*) - \vx^* \rangle \geq 0 $ since $\phi(\vx^*) \in \cX$, thereby implementing the $\ger$ oracle. We thus find that the preconditions of~\Cref{theorem:eah} are satisfied, and $\eah$ (\Cref{alg:eah}) returns $\mu \in \Delta(\cX)$, with $\supp(\mu) \leq \poly(d, \log(1/\epsilon))$, such that
\begin{equation}
    \E_{\vx \sim \mu} \langle \vy, \phi(\vx) - \vx \rangle \geq - \epsilon \quad \forall \vy \in [-1, 1]^d.
\end{equation}
Taking $\vy = \sign( \E_{\vx \sim \mu} (\vx - \phi(\vx)))$ (coordinate-wise) completes the proof.
\end{proof}

As it will become clear, \Cref{th:efp} yields a polynomial-time implementation of the $\ger$ oracle in the context of~\Cref{sec:eah}, which can be employed in~$\eah$. With a slight modification in the proof of~\Cref{th:efp}, we shall see how one can also recover an $\epsilon$-$\either$ oracle (\Cref{def:either}), which will then enable us to harness~\Cref{theorem:either} for computing $\Phi$-equilibria in games. Following the nomenclature of~\citet{Daskalakis24:Efficient}, we refer to this oracle as a \emph{semi-separation oracle}.

\begin{definition}[Semi-separation oracle]
    \label{def:semiseparation}
    The \emph{semi-separation} problem is the following. Given a convex and compact $\cX$ and a function $\phi : \cX \to \R^d$, compute
    \begin{enumerate}
        \item {\em either} a distribution $\mu \in \Delta(\cX)$ such that $\|\E_{\vx \sim \mu}[\phi(\vx)-\vx] \|_1 \leq \epsilon$, \label{item:efp}
        \item {\em or} a point $\vx\in\cX$ with $\phi(\vx) \notin \cX$.\label{item:mem}
    \end{enumerate}
\end{definition}

Unlike~\Cref{def:EFP}, here we allow $\phi$ to map outside of $\cX$. This more general framing is essential to arrive at the $\either$ oracle. In particular, we note that~\Cref{item:mem} yields a hyperplane separating $\phi$ from the set of endomorphisms on $\cX$. Namely, since $\phi(\vx) \notin \cX$, we can use the separation oracle on $\cX$ to separate $\cX$ from $\phi(\vx)$; that is, there is a $\vec{w}$ such that $\langle \phi(\vx), \vec{w} \rangle > \langle \vx, \vec{w} \rangle$ for all $\vx \in \cX$. But this also implies that $\langle \phi(\vx), \vec{w} \rangle > \langle \phi'(\vx), \vec{w} \rangle$ for any endomorphism $\phi'$, as promised.

\begin{theorem}
    \label{theorem:semiseparation}
    Given oracle access to $\cX$ and $\phi$, there is a $\poly(d, \log(1/\epsilon))$-time algorithm for implementing the semi-separation oracle of~\Cref{def:semiseparation}.
\end{theorem}

\begin{proof}
As in the proof of~\Cref{th:efp}, we proceed by running the ellipsoid algorithm (per~\Cref{alg:eah}) on the problem
\begin{align}
    \qq{find} \vy\in [-1,1]^d \qq{s.t.} \ip{\vy, \phi(\vx) - \vx} \le -\eps \quad \forall \vx \in \cX. \label{eq:efp-constraint}
\end{align}
For any $\vy \in [-1, 1]^d$ during the execution of the ellipsoid, take $\vx^*(\vy) \in \argmin_{\vx \in \cX} \ip{\vy, \vx}$. If $\phi(\vx^*(\vy)) \notin \cX$, the algorithm can terminate and return $\vx^*(\vy)$. Otherwise, it follows that $\langle \vy, \phi(\vx^*(\vy)) - \vx^*(\vy) \rangle \geq 0$, by definition of $\vx^*$, and so we can use $\vx^*$ to get a separation oracle for~\eqref{eq:efp-constraint}.

Now, if every $\vx^*(\vy) \in \cX$ generated above satisfies the constraint $\phi(\vx^*(\vy)) \in \cX$, then~\Cref{alg:eah} returns a certificate of infeasibility for~\eqref{eq:efp-constraint} in $\poly(d, \log(1/\epsilon))$ time, which is an $\epsilon$-expected fixed point of $\phi$. On the other hand, if at some point there is $\vy \in [-1, 1]^d$ such that $\phi(\vx^*(\vy)) \notin \cX$, then the algorithm returns a point $\vx^*(\vy) \in \cX$ such that $\phi(\vx^*(\vy)) \notin \cX$. This completes the proof.
\end{proof}

This semi-separation oracle amounts to the $\epsilon$-$\either$ oracle needed in~\Cref{theorem:either}, as we shall see next in the context of games. Compared to the semi-separation oracle of~\citet{Daskalakis24:Efficient} that only works for linear functions, ours (\Cref{theorem:semiseparation}) places no restrictions on $\phi$. 
