\section{Sets of deviations with polynomial dimension}
\label{sec:deviations}

In this section, we formally introduce the assumptions we make concerning the feature map $m$ of~\Cref{assumption:kernel}, and we then provide a canonical example that satisfies our blanket assumptions.

\begin{assumption}
    \label{assumption:kernel-precise}
    We make the following assumptions regarding $\Phi^m$ and $m$ of~\Cref{assumption:kernel}:
    \begin{itemize}
        \item $m : \cX \to \R^{k'}$ is computable in $\poly(k)$ time.
        \item $\co m(\cB_1(\vec{0})) \supseteq \cB_\delta(\vec{0})$ for some $\delta \geq \poly(1/k)$.
        \item $\|m(\vx)\| \leq \poly(k)$ for all $\vx \in \cX$, and $m(\vec{0}) = \vec{0}$. 
        \item $\Phi^m$ contains the identity map.
    \end{itemize}
\end{assumption}

\begin{remark}[Functions on the vertices]
    Let $\cV$ be the set of extreme points of $\cX$. Our positive results (\Cref{theorem:main-eah-prec,theorem:main-prec}) only evaluate $\phi$ at extreme points, so they would operate identically if we instead defined our maps $\phi$ to be $\cV \to \cX$.% For expositional and notational simplicity, we will not do this.
\end{remark}

The definition above places some minimal assumptions on the feature mapping $m$ to ensure that $\Phi^m$ is geometrically well behaved. Indeed, we first show that the set of transformations $\Phi^m$ under~\Cref{assumption:kernel-precise} is well-bounded; the proof is provided in~\Cref{sec:wellbounded}.

\begin{restatable}{lemma}{wellbounded}
    \label{lemma:wellbounded}
    Let $\cX \subseteq \R^d$ be a convex and compact set such that $\cB_r(\vec{0}) \subseteq \cX \subseteq \cB_R(\vec{0})$, with $R \geq 1$ and $r < R$. Suppose further that $\|m(\vx)\| \leq M$ for all $\vx \in \cB_R(\vec{0})$, with $M = M(R) \geq 1$; $\co m(\cB_r(\vec{0})) \supseteq \cB_\delta(\vec{0})$ for some $\delta = \delta(r) > 0$; and $m(\vec{0}) = \vec{0}$. Then,
    \begin{equation}
        \cB_{r'}(\vec{0}) \subseteq \Phi^m \subseteq \cB_{ R'}(\vec{0}),
    \end{equation}
    where $r' \defeq \nicefrac{r}{2M(R)}$ and $R' \defeq R \left( \frac{2\sqrt{d}}{\delta(r)} + 1 \right)$.
\end{restatable}

We are now ready to provide a canonical, concrete example of deviations that satisfy~\Cref{assumption:kernel} under~\Cref{assumption:kernel-precise}. As we alluded to earlier in our introduction, it is the family of low-degree polynomials; in particular, it will be convenient to work with the Legendre basis.

\begin{definition}
    \label{def:polys}
    Let $P_0(x) = 1$ and $P_1(x) = x$. The $(\ell+1)$th \emph{Legendre polynomial} is given by the recurrence $(\ell + 1) P_{\ell + 1}(x) - (2\ell+1) x P_\ell(x) + \ell P_{\ell - 1}(x) = 0$.
\end{definition}

These polynomials have a convenient orthogonality property over $[-1, 1]$:
\begin{equation}
    \label{eq:orthog}
    \int_{-1}^1 P_{\ell}(x) P_{\ell'}(x) dx = 
    \begin{cases}
        \frac{1}{2\ell + 1} & \text{if } \ell = \ell',\\
        0 & \text{otherwise}.
    \end{cases}
\end{equation}
For convenience, we shall consider the rescaled polynomial $\lineP_\ell \defeq \sqrt{2\ell + 1} P_{\ell}$, so that $\int_{-1}^1 \lineP_\ell(x)^2 dx = 1$. We now define
\begin{equation}
    \label{eq:polynomials}
    m(\vx) \defeq \left( \prod_{j=1}^d \lineP_{\ell_j}(\vx[j]) \right)_{ 1 \leq \ell_1 + \dots \ell_d \leq \ell}.
\end{equation}

We establish (in~\Cref{sec:aux}) that~\Cref{assumption:kernel-precise} encompasses the above mapping.

\begin{restatable}{proposition}{lowdeg}
    \label{lemma:lowdeg}
    Let $m : \cX \to \R^{k'}$ per~\eqref{eq:polynomials}, where $k' = \binom{d + \ell}{\ell} - 1$. $\linem : \vx \mapsto m( \sqrt{d} \vx)$ satisfies~\Cref{assumption:kernel-precise} with $M \leq d^{O(\ell)}$  and $\delta = \nicefrac{1}{M}$.
\end{restatable}