\section{A polynomial-time algorithm for $\Phi^m$-equilibria in games}

Armed with the powerful semi-separation oracle of~\Cref{theorem:semiseparation}, we now establish a polynomial-time algorithm for computing $\Phi^m$-equilibria in general multilinear games (\Cref{theorem:main-eah-prec}).

Let us recall the basic setting of an $n$-player multilinear game $\Gamma$. Each player $i \in [n]$ has a convex and compact strategy set $\cX_i \subseteq \R^{d_i}$ in isotropic position (\Cref{sec:prel}). Player $i$ has a utility function $u_i : \cX_1 \times \dots \times \cX_n \to \R$ that is linear in $\cX_i$, so that $u_i(\vx) = \langle \vec{g}_i, \vx_i \rangle$ for some $\vec{g}_i = \vec{g}_i(\vx_{-i}) \in \R^{d_i}$. %Without loss of generality, we assume that $\vec 0 \notin \op{aff} \cX_i$, the affine hull of $\cX_i$, so that linear and affine functions on $\cX_i$ are interchangeable.
Furthermore, for each player $i \in [n]$, we let $\Phi^{m_i} \subseteq \cX_i^{\cX_i}$ be the $k_i$-dimensional set of deviations in the sense of~\Cref{assumption:kernel}; that is, there exists a function $m_i \in \cX_i \to \R^{k_i'}$, with $k_i = k_i' \cdot d_i + d_i$, such that for each $\phi_i \in \Phi^{m_i}$ and $\vx_i \in \cX_i$, the function output $\phi_i(\vx_i)$ can be expressed as the matrix-vector product $\mK_i(\phi_i) m_i(\vx_i) + \cons_i$ for some matrix $\mK_i \in \R^{d_i \times k_i'}$ and $\cons_i \in \R^{d_i}$. It is assumed throughout that $\Phi^{m_i}$ contains the identity map. For notational simplicity, we let $k \defeq \sum_{i=1}^n k_i$ and $d \defeq \sum_{i=1}^n d_i$.

In this context, we next state the main result of this section, and proceed with its proof.

\begin{theorem}[Precise version of~\Cref{theorem:main-eah}]
    \label{theorem:main-eah-prec}
    Consider an $n$-player multilinear game $\Gamma$ such that, for each player $i \in [n]$, we are given $\poly(n, k)$-time algorithms for the following:
\begin{itemize}
\item an oracle to compute the gradient, that is, the vector $\vec{g}_i = \vec{g}_i(\vx_{-i}) \in \R^{d_i}$ for which $\ip{\vg_i(\vx_{-i}), \vx_i} = u_i(\vx)$ for all $\vx \in \cX_1 \times \dots \times \cX_n$ (polynomial expectation property); and
\item a membership oracle for the strategy set $\cX_i$, assumed to be in isotropic position.
\end{itemize}
Suppose further that each $k_i$-dimensional set $\Phi^{m_i}$ satisfies~\Cref{assumption:kernel-precise} and $\|\vec{g}_i \| \leq B$. Then, an $\eps$-approximate $\Phi^m$-equilibrium of $\Gamma$ can be computed in $\poly\qty(n, k, \log(B/\eps))$ time.
\end{theorem}

\begin{proof}
An $\eps$-approximate $\Phi^m$-equilibrium of $\Gamma$  is a distribution $\mu \in \Delta(\cX_1 \times \dots \times \cX_n)$ such that
\begin{align}
\E_{\vx \sim \mu} \qty[u_i(\phi_i(\vx_i), \vx_{-i}) - u_i(\vx)] \le \eps\label{eq:phi-eqm}
\end{align}
for every player $i \in [n]$ and deviation $\phi_i \in \Phi^{m_i}$. Using multilinearity and~\Cref{assumption:kernel}, it suffices to find a distribution $\mu\in\Delta(\cX_1 \times \dots \times \cX_n)$ satisfying
\begin{align}
\E_{\vx \sim \mu} \left[ \sum_{i =1}^n \ip{\vg_i(\vx_{-i}), \mK_i m_i(\vx_i) + \cons_i - \vx_i} \right] \le \eps \label{eq:phi-sp}
\end{align}
for every $(\mK_1(\phi_1), \dots, \mK_n(\phi_n))$ and $(\cons_1(\phi_1), \dots, \cons_n(\phi_n))$, where $(\phi_1, \dots, \phi_n) \in \Phi^{m}$. (This derivation uses the fact that $\Phi^{m_i}$ contains the identity map.) We will now apply~\Cref{theorem:either} with respect to $\R^d \supseteq \cX \defeq \cX_1 \times \dots \times \cX_n$ and 
$$\R^{k} \supseteq \cY \defeq \{ (\mK_1, \cons_1, \dots, \mK_n, \cons_n) : \mK_i m_i(\vx_i) + \cons_i \in \cX_i \quad \forall \vx_i \in \cX_i \}. $$
By the polynomial expectation property, we can evaluate the term $\sum_{i =1}^n \ip{\vg_i(\vx_{-i}), \mK_i m_i(\vx_i) + \cons_i - \vx_i}$, for each $\vx \in \cX$, in $\poly(n, k)$ time. It thus suffices to show how to implement the $\epsilon$-$\either$ oracle, which yields a separation oracle for the program
\begin{alignat}{9}
\qq{find} \mK_1, \cons_1, \dots, \mK_n, \cons_n \qq{s.t.} \\
\sum_{i =1}^n \ip{\vg_i(\vx_{-i}), \mK_i \vm_i(\vx_i) + \cons_i - \vx_i} \ge - \eps &\quad\forall \vx \in \cX_1 \times \dots \times \cX_n, \label{eq:phi-sp}
\\
\mK_i \vm_i(\vx_i) + \cons_i \in \cX_i &\quad \forall\vx_i \in \cX_i.
\end{alignat}

Consider any $\R^{k} \ni \phi = (\mK_1, \cons_1, \dots, \mK_n, \cons_n)$. We apply the semi-separation oracle of~\Cref{theorem:semiseparation} for each function $\vx_i \mapsto \mK_i m_i(\vx_i) + \cons_i$. This returns \emph{either} an $\epsilon'$-expected fixed point, that is, a distribution $\nu_i \in \Delta(\cX_i)$ such that $$\left\| \E_{\vx_i \sim \nu_i} [ \mK_i \vm_i(\vx_i) + \cons_i - \vx_i ] \right\|_1 \leq \epsilon',$$
\emph{or} a point $\vx_i \in \cX_i$ such that $\mK_i m_i(\vx_i) + \cons_i \notin \cX_i$. If any of those semi-separation oracles returned $\vx_i \in \cX_i$ with $\mK_i m_i(\vx_i) + \cons_i \notin \cX_i$, we can use it to obtain a hyperplane separating $(\mK, \cons)$ from the set of deviations $\cY$. Otherwise, let $\nu \defeq \nu_1 \times \dots \times \nu_n \in \Delta(\cX_1) \times \dots \times \Delta(\cX_n)$ be the induced product distribution. Then, we have
\begin{align}
    & \E_{\vx \sim \nu}\sum_{i = 1}^n \ip{\vg_i(\vx_{-i}), \mK_i \vm_i(\vx_i) + \cons_i - \vx_i} 
    = \sum_{i =1}^n  \big\langle\E_{\vx \sim \nu} \vg_i(\vx_{-i}), \E_{\vx_i \sim \nu_i}[\mK_i \vm_i(\vx_i) + \cons_i - \vx_i]\big\rangle \leq n B \eps', \label{eq:eqm-final}
\end{align}
where we used the fact that $\nu$ is a product distribution in the equality above. Thus, we have identified an $(\epsilon' n B)$-approximate good-enough-response, yielding an $\epsilon$-$\either$ oracle by rescaling $\epsilon'$, and the proof follows from~\Cref{theorem:either}.
\end{proof}

\begin{algorithm}[!ht]
\caption{Polynomial-time algorithm for $\Phi^m$-equilibria}
\label{alg:main-eah}
\SetKwInOut{Input}{Input}
\SetKwInOut{Output}{Output}
\SetKw{Input}{Input:}
\SetKw{Output}{Output:}
\Input{
    \begin{itemize}[noitemsep,topsep=0pt]
        \item An $n$-player multilinear game $\Gamma$
        \item A precision parameter $\epsilon > 0$
        \item A membership oracle for each $\cX_i$
        \item An oracle for computing the gradient $\vec{g}_i = \vec{g}_i(\vx_{-i}) \in \R^{d_i}$ for each $i \in [n]$
        \item A $k_i$-dimensional set $\Phi^{m_i}$ under~\Cref{assumption:kernel-precise} for each $i \in [n]$
    \end{itemize}
}
\Output{An $\epsilon$-approximate $\Phi^m$-equilibrium of $\Gamma$ in $\poly(k, \log(1/\epsilon))$ time}\\
Define $G : \R^{d} \to \R^k$ such that $\langle G(\vx), (\mK, \cons) \rangle = \sum_{i =1}^n \ip{\vg_i(\vx_{-i}), \mK_i m_i(\vx_i) + \cons_i - \vx_i}$\;
Use the semi-separation oracle of~\Cref{theorem:semiseparation} to construct an $\epsilon$-$\either$ oracle $\cO$\;
 Apply~\Cref{alg:gen-eah} with $\cO$ as the $\epsilon$-$\either$ oracle
\end{algorithm}

It is worth stressing that it is crucial for our proof that the \emph{expected} VI problem (\emph{cf}.~\citet{Zhang25:Expected}) above corresponds to a game. It allows each player to be treated {\em independently}, which yields a {\em product distribution} $\nu = \nu_1 \times \dots \times \nu_n$ when we apply the semi-separation oracle of~\Cref{theorem:semiseparation} (for each player). That $\nu$ is a product distribution is crucial to implement the separation oracle for the dual because it allows us to push the expectation into the inner product in \eqref{eq:eqm-final}, as we saw in the last step of the proof.

