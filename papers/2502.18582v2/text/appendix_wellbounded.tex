\section{Geometric properties of $\Phi^m$}
\label{sec:wellbounded}

In this section, we establish that the set $\Phi^m$ per~\Cref{assumption:kernel-precise} is geometrically well behaved, which is necessary to execute the ellipsoid algorithm (as well as the online learning setting). In particular, our goal is to prove \Cref{lemma:wellbounded}.

Below, for convex and compact $\cA, \cB \subseteq \R^d$, we use the notation 
\begin{equation}
    \Phi^m(\cA, \cB) \defeq \left\{ (\mK, \cons) \in \R^{k + d} : \mK m(\vx) + \cons \in \cB \quad \forall \vx \in \cA \right\}.
\end{equation}

\begin{lemma}
    \label{lemma:trivial}
    Let $\cA, \cB, \cC, \cD$ be convex and compact sets. If $\cA \supseteq \cC$ and $\cB \subseteq \cD$, then $\Phi^m(\cA, \cB) \subseteq \Phi^m(\cC, \cD)$.
\end{lemma}

\begin{proof}
    Consider any $(\mK, \cons) \in \Phi^m(\cA, \cB)$. By definition, it holds that $\mK \vx + \cons \in \cB$ for all $\vx \in \cA$. Since $\cC \subseteq \cA$, it follows that $\mK \vx + \cons \in \cB$ for all $\vx \in \cC$, and in particular, $\mK \vx + \cons \in \cD$ since $\cB \subseteq \cD$.
\end{proof}

\begin{lemma}
    \label{lemma:boundM}
    Let $\cX \subseteq \R^d$ be a convex and compact set such that $\cB_r(\vec{0}) \subseteq \cX \subseteq \cB_R(\vec{0})$, with $R \geq 1$. Suppose further that $\|m(\vx)\| \leq M$ for all $\vx \in \cB_R(\vec{0})$, where $M = M(R) \geq 1$. Then, $\Phi^m \supseteq \cB_{r'}(\phi_{\vec{0}})$, where $r' \defeq \nicefrac{r}{2M(R)}$ and $\phi_{\vec{0}} \defeq (\mathbf{0}, \vec{0})$ is the constant transformation $\vx \mapsto \vec{0}$.
\end{lemma}

\begin{proof}
    By~\Cref{lemma:trivial}, it suffices to prove $\cB_{r'}(\phi_{\vec{0}}) \subseteq \Phi^m(\cX, \cB_r(\vec{0}))$. Consider any $(\mK, \cons) \in \cB_{r'}(\phi_{\vec{0}})$, which means that $\| \mK \|^2_F + \|\cons \|^2 \leq ( \nicefrac{r}{2M(R)})^2$. Then, for any $\vx \in \cX$,
    \begin{equation}
        \| \mK m(\vx) + \cons \| \leq \| \mK \|_F \|m(\vx)\| + \|\cons  \| \leq r.
    \end{equation}
    This means that $\cB_{r'}(\phi_{\vec{0}}) \subseteq \Phi^m(\cX, \cB_r(\vec{0}))$, and the proof follows.
\end{proof}

\begin{lemma}
    \label{lemma:fulldim}
    Suppose that $\co m(\cB_r(\vec{0})) \supseteq \cB_\delta(\vec{0})$ for some $\delta = \delta(r) > 0$ and $m(\vec{0}) = \vec{0}$. Then, assuming that $r < R$,
    \begin{equation}
        \Phi^m( \cB_r(\vec{0}), \cB_R(\vec{0})) \subseteq \cB_{ R'}(\vec{0}),
    \end{equation}
    where $R' \defeq R \left( \frac{2\sqrt{d}}{\delta(r)} + 1 \right)$.
\end{lemma}

\begin{proof}
    Consider any $(\mK, \cons) \in \Phi^m( \cB_r(\vec{0}), \cB_R(\vec{0}))$. By definition, we have $\| \mK m(\vx) + \cons \| \leq R$ for all $\vx \in \cB_r(\vec{0})$. Since $m(\vec{0}) = \vec{0}$, it follows that $\|\cons \| \leq R$. Thus, $\| \mK m(\vx) \| \leq \|\mK m(\vx) + \cons \| + \|\cons \| \leq 2 R$ for all $\vx \in \cB_r(\vec{0})$. Now, let $\vec{x}' \in \R^{k'}$ with $\|\vec{x}' \| = 1$ be such that $\| \mK \vec{x}' \| = \| \mK\|$, where $\| \mK \|$ is the spectral norm of $\mK$. Since we have assumed that $\co m(\cB_r(\vec{0})) \supseteq \cB_\delta(\vec{0})$, it follows that there exist $\lambda_1, \dots, \lambda_{k'+1}$, with $\lambda_1, \dots, \lambda_{k' + 1} \geq 0$ and $\sum_{j=1}^{k'+1} \lambda_j = 1$, and $\vx_1, \dots, \vx_{k'+1} \in \cB_r(\vec{0})$ (by Carath\'eodory's theorem) such that $\sum_{j=1}^{k'+1} \lambda_j m(\vx_j) = \delta \vx'$. As a result,
    \begin{equation}
        \delta \|\mK\| = \| \mK (\delta \vx') \| = \left\| \mK \left( \sum_{j=1}^{k'+1} \lambda_j m(\vx_j) \right) \right\| \leq \sum_{j=1}^{k' + 1} \lambda_j \|\mK m(\vx_j) \| \leq 2R.
    \end{equation}
    Finally, we have $\|\mK\|_F \leq \sqrt{d} \|\mK\|$, and the claim follows.
\end{proof}

\begin{proof}[Proof of~\Cref{lemma:wellbounded}]
    The claim follows directly by combining~\Cref{lemma:trivial,lemma:boundM,lemma:fulldim}.
\end{proof}
