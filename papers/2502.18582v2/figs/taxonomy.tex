\begin{figure}[t]
\tikzstyle{box} = [rectangle, minimum width=.5cm, minimum height=0.5cm, text centered, draw=black, font=\footnotesize]
\tikzstyle{textbelow} = [font=\scriptsize, align=center]
\tikzstyle{arrow} = [implies-,double equal sign distance] %[thick,->,>=stealth]



\begin{tikzpicture}[node distance=0.6cm]
    % Nodes (title on top, additional text below)
    \node (NFCCE) [box] {\begin{tabular}{c} \textbf{External} \\ \scriptsize \citep{Zinkevich07:Regret,Moulin78:Strategically} \end{tabular}};
    \node (EFCE) [box, right=0.4cm of NFCCE] {\begin{tabular}{c} \textbf{(Coarse) Trigger} \\ \scriptsize \citep{Farina22:Simple,Farina20:Coarse,Stengel08:Extensive,Morrill21:Hindsight} \end{tabular}};
    \node (LCE) [box, right=0.4cm of EFCE] {\begin{tabular}{c} \textbf{Linear swap} \\ \scriptsize \citep{Daskalakis24:Efficient,Gordon08:No,Farina23:Polynomial} \end{tabular}};
    
    \node (LowDegree) [box, above=.5cm of LCE] {\begin{tabular}{c} \textbf{$O(1)$-degree} \\ \textbf{polynomial swap} \\ \scriptsize \citep{Zhang24:Efficient} \end{tabular}};
    \node (Polytope) [box, right=0.5cm of LCE] {\begin{tabular}{c} \textbf{Polytope swap} \\ \scriptsize \citep{Mansour22:Strategizing} \end{tabular}};
    \node (LowDim) [box, fill=gray!15, right=.5cm of LowDegree] {\begin{tabular}{c} \textbf{$d^{O(1)}$-dim. $\Phi$} \\ \scriptsize \textbf{This paper} \end{tabular}};
    \node (NFCE) [box, red, right=1.5cm of Polytope,yshift=8mm] {\begin{tabular}{c} \textbf{Swap} \\ \scriptsize \citep{Aumann74:Subjectivity} \end{tabular}};

    % Equilibrium names and other notes
    \node[red, below=1mm of NFCE, text width=3.4cm,align=center,xshift=-2mm] {\scriptsize Learning intractable \citep{Daskalakis24:Lower}, computation unknown};
    \node[below=.5mm of NFCCE,gray] {\small Coarse CE};
    \node[below=.5mm of EFCE,text width=2.4cm,align=center,gray] {\small Extensive-form (coarse) CE};
    \node[below=.5mm of LCE,gray] {\small Linear CE};
    \node[above=.5mm of LowDegree,gray] {\small $O(1)$-degree CE};
    \node[above=.5mm of LowDim,purple] {\small $d^{O(1)}$-dim. CE};
    \node[above=.5mm of NFCE,text width=2.8cm,align=center,gray] {\small Correlated equilibrium (CE)};


    % Arrows (all left to right)
    \path
     (NFCCE) edge[arrow] (EFCE)
     (EFCE) edge[arrow] (LCE)
     (LCE) edge[arrow] (LowDegree)
     (LowDegree) edge[arrow] (LowDim)
     (LCE) edge[arrow] (Polytope)
     (LowDim) edge[arrow] (NFCE)
     (Polytope) edge[arrow] (NFCE);

    % \draw[thick,dashed,->] (-1,-1.5) -- (13,-1.5) node[below left] {Tighter equilibrium concept};

\end{tikzpicture}
\caption{The arrows $A \implies B$ denote that minimizing the notion of regret $A$ implies minimizing the notion of regret $B$. In other words, $A$ defines a superset of deviations that the learner considers compared to $B$, and hence leads to a stronger notion of equilibrium. The gray text below or above a notion of regret denotes the name of the corresponding notion of equilibrium, if applicable.}
\label{fig:taxonomy}
\end{figure}