\section{Conclusion}

We have shown that the TOC representation provides a strong signal for predictors to identify dense correspondences on in-the-wild images, and that it is possible to train these models using entirely synthetic data. Our synthetic dataset, \ourSynth, provides substantial improvements on SynFoot, including more background and lighting variation, and adding articulation to the 3D models used for rendering.

We have shown that TOCs can be combined with uncertainty and surface normal predictions to recover high quality foot reconstructions from multiple views.

The first of our reconstruction methods, \ourSfM, effectively identifies subpixel TOC matches across multiple images, and triangulates these to provide a high quality reconstruction with no enforcement of a 3D prior on the output foot shape. This method requires fewer views than COLMAP, and can operate entirely on a CPU.

Our second method, \ourOptim, is capable of directly optimizing the parameterized FIND model to match the TOC predictions across images. \ourOptim effectively uses TOC uncertainty, and provides a superior reconstruction to FOUND, while matching the low view requirement, and running substantially faster.

Each method has its own advantages. \ourSfM may be suitable for applications where speed is desired, no GPU is available, and only the visible portion of the surface is of interest to downstream tasks. In contrast, \ourOptim may be more desirable when a fully parameterized and watertight mesh is required.