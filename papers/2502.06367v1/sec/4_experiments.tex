\section{Experiments}

\paragraph{3D baseline. } We evaluate on Foot3D, a baseline 3D foot reconstruction dataset released in \cite{boyne2024found} - 14 scenes of real scanned feet, with a total of 474 calibrated images. All methods are provided with the same set of input images and camera calibrations.

As the task focuses on reconstruction of the foot (not the leg), we cut off all meshes at a height of 10 cm for evaluation.
We evaluate the quality of our 3D optimization method by comparing the output mesh to a ground truth scan from our dataset. We select a sample of 10,000 points from each mesh, sampling uniformly over the surface area. For each point, we find the nearest neighbor (NN) on the surface of the other mesh, and capture the difference in Euclidean distance (chamfer error), and the angular difference between the surface normals at the two points (normal error).

We report the statistics of the 20,000 samples across the two meshes. We compare our methods against photogrammetry pipeline COLMAP \cite{schonberger2016structure, schonberger2016pixelwise}, and surface normal optimization method FOUND \cite{boyne2024found}.


\paragraph{Speedtesting.} We evaluate the speed and requirements of all methods. All methods were evaluated on the same machine, with an NVIDIA Quadro P2200 GPU.

\paragraph{Number of views.} We vary the input views available to each method to investigate how this affects reconstruction quality. Where views are reduced, we sample views evenly in the left-right direction.