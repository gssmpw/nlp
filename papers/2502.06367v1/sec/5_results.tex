\section{Results}

\begin{table}[]
    \centering
    
    \begingroup
    \begin{footnotesize}
    \setlength{\tabcolsep}{4pt} % Default value: 6pt
    
\begin{tabular}{c|*3{c}|*3{c}}
    \toprule
     \multirow{2}{*}{Method} & \multicolumn{3}{c|}{NN chamfer error (mm) $\downarrow$} & \multicolumn{3}{c}{NN normal error ($^{\circ}$) $\downarrow$} \\
     & Mean & Median & RMSE & Mean & Median & RMSE \\ \midrule

    \colmap & \B 1.8 & \B 0.9 & 3.1 & 21.2 & 13.5 & 30.4 \\
    \found & 2.6 & 2.2 & 3.3 & \B 13.4 & \B 9.9 & 19.5 \\
    \ourSfM & 2.0 & 1.5 & \B 2.7 & 14.1 & 11.1 & \B 18.0 \\
    \ourOptim & 2.1 & 1.8 & \B 2.7 & 13.5 & 10.2 & 18.8 \\

   
     \bottomrule
\end{tabular}

% Can change screened_poisson_depth 8->7 for:
% \ourSfM: 2.1            1.5          2.8        13.6          10.5        17.7
    \end{footnotesize}
    \endgroup
    \caption{\textbf{3D reconstruction results.} Our methods yield the lowest RMSE chamfer error. While COLMAP still leads on mean and median chamfer error, it performs substantially worse on surface normal error, showing a substandard surface reconstruction. Our methods perform comparably to FOUND on surface normal reconstruction, despite \ourSfM using no 3D prior in the Poisson reconstruction stage, and \ourOptim not using surface normals during optimization.}
    \label{tab:3d_fit_results}
\end{table}

\begin{figure}
    \centering
    \includegraphics[width=\linewidth]{plots/num_views_print.pdf}
    \caption{\textbf{Reconstruction quality as number of views varies.} \ourOptim shows the best performance under a few-view setting, retaining its accuracy with as few as 3 views. \ourSfM performs substantially better than COLMAP, showing accurate reconstruction performance with 10 views, compared to COLMAP requiring more than 15.}
    \label{fig:num_views}
\end{figure}

\begin{figure*}
    \centering
    \begin{tabular}{c|c}
        \includegraphics[height=0.325\linewidth]{images/placeholder/qual_fit_detailed_1.png} &
        \includegraphics[height=0.325\linewidth]{images/placeholder/qual_fit_detailed_2.png}
    \end{tabular}
    % \includegraphics[height=0.325\linewidth]{images/placeholder/qual_fit_detailed_1.png}\hfill
    % \includegraphics[height=0.325\linewidth]{images/placeholder/qual_fit_detailed_2.png}
    \caption{\textbf{Qualitative reconstruction results.} The reconstruction quality is compared across two scans in the Foot3D dataset, comparing COLMAP, FOUND, \ourSfM and \ourOptim. COLMAP is prone to noise around the foot boundaries, and captures less surface detail in particular around the toes and heel.
    Further qualitative comparisons can be found in the supplementary material.}
    \label{fig:3d_fits_qual}
    %0035,0037
\end{figure*}

% \begin{table}
%     \caption{Surface normal reconstruction accuracy metrics.}
% \end{table}

\paragraph{Reconstruction accuracy. } We show in Table \ref{tab:3d_fit_results} the accuracy of our two reconstruction methods, compared to FOUND and COLMAP, when all views are available (25-40) for each scan.

When all views are available, COLMAP performs the best for mean and median chamfer distance. However, our methods perform better on chamfer RMSE as COLMAP is prone to noise in reconstruction. Furthermore, our methods far outperform COLMAP with regards to surface normal quality, and in fact perform comparably with FOUND, which relies entirely on surface normals for fitting. \ourOptim uses no surface normals in the reconstruction process - only requiring TOC correspondences.

Detailed qualitative results can be seen in Figure \ref{fig:3d_fits_qual}. These show that COLMAP fits often exhibit noise around edges, which is problematic for taking measurements. Our methods are also superior at capturing the surface around the toes - the separation of the toes is handled better by both \ourSfM and \ourOptim.

\ourOptim tends to have fewer large sources of error both for normal and chamfer, providing a more reliable result. This approach, along with FOUND, produces watertight, parameterized meshes as output.

\paragraph{Number of views. } We show the performance under a varying number of views in Figure 
\ref{fig:num_views}. The strong 3D priors, and use of uncertainty, in FOUND and FOCUS-O ensure that they retain their reconstruction accuracy with as few as 3 views.
COLMAP has the highest view requirement, showing a significant reduction in accuracy (and some outright failures in reconstruction) for fewer than 15 views. While reconstructing using some similar principles to COLMAP, \ourSfM is able to take advantage of the rich information provided by TOCs to produce accurate reconstructions with as few as 10 views.

\begin{table}
    \centering
    \begingroup
    \small
    \newcommand{\cmark}{\ding{51}}%
\newcommand{\xmark}{\ding{55}}%


%Watertight mesh?

\begin{tabular}{c|cc|cc}
    \toprule
    \multirow{2}{*}{Method} & \multicolumn{2}{c|}{Inference time (s)} & \multirow{2}{*}{GPU?} & Differentiable\\
    & 10-view & all-view & & rendering? \\
    \midrule
    \colmap & - & 465 & \cmark & \xmark \\
    \found & 260 & 260 & \cmark & \cmark \\
    \ourSfM & \B 22 & \B 89 & \xmark & \xmark\\
    \ourOptim & 40 & 100 & \cmark & \xmark \\

    \bottomrule
\end{tabular}

% Need to all be evaluated on same machine!

% On mac
% SFM: 8.16, 31.47

% On PC
% SFM: 22.29, 89.22
    \endgroup
    \caption{\textbf{Speed and requirements.} \ourOptim and \ourSfM are substantially faster than existing methods. Furthermore, neither require differentiable rendering, and \ourSfM does not require a GPU.}
    \label{tab:speed_and_requirements}
\end{table}

\paragraph{Implementation.} Table \ref{tab:speed_and_requirements} compares the inference times and computational requirements of all methods. Both of our methods are substantially faster than COLMAP and FOUND. Compared to FOUND, neither require differentiable rendering, drastically reducing the memory requirements, and making them easier to implement on devices that would not support it. Further, \ourSfM is a CPU based method (except for the initial TOC predictions), so is possible to run on devices without GPU optimization support, including mobile devices.

\begin{table}
    \centering
    \begingroup
    \footnotesize
    \centering
    
\begin{tabulary}{0.9\linewidth}{cCC}
    \toprule
      & NN mean chamfer error (mm) $\downarrow$ & NN mean normal error ($^{\circ}$) $\downarrow$ \\ \midrule
    \ourSfM & \B 2.0 & \B 14.0 \\
    w/o subpixel matching & 2.4 & 15.4 \\
    w/o normal aggregation & 2.5 & 107.5 \\
 \bottomrule
\end{tabulary}

%NA = Normal aggregation
%SM = Subpixel matching
    \endgroup
    \caption{\textbf{\ourSfM ablation study.} Both the subpixel matching and normal aggregation steps of \ourSfM are crucial for its quantitative reconstruction performance. The drastic increase in normal error without normal aggregation is partly due to the reconstruction algorithm occasionally inverting the surface normals.}
    \label{tab:ablation-sfm}
\end{table}

\begin{table}
    \centering
    \begingroup
    \footnotesize
    \centering
    \begin{tabular}{c|cc|cc}
      \toprule
       & \multicolumn{2}{c|}{\scriptsize {NN Chamfer error (mm) $\downarrow$}} & \multicolumn{2}{c}{\scriptsize NN Normal error ($^{\circ}$) $\downarrow$}\\
       & 3 view & 20 view & 3 view & 20 view \\
       \midrule
    \ourOptim & \B 2.5 &  \B 2.1 &  \B 14.4 & \B 13.5 \\
    w/o uncertainty &  2.7 &  2.2 &  15.0 &  13.9 \\

      \bottomrule
\end{tabular}
    \endgroup
    \caption{\textbf{\ourOptim ablation study.} The use of TOC uncertainty in our optimization process improves all reconstruction metrics, both for a low and high view count.}
    \label{tab:ablation-optim}
\end{table}

\paragraph{Ablation study.} We show results of our ablation for both \ourSfM and \ourOptim in Tables \ref{tab:ablation-sfm} and \ref{tab:ablation-optim} respectively.

For \ourSfM, aggregating the predicted surface normals is critical to capturing geometry in certain areas of the foot, especially around the toes, as naive Poisson reconstruction is not capable of estimating that detail. Furthermore, Poisson reconstruction may generate a plausible mesh with inverted face normals, hence the high surface normal error.
Our subpixel matching is also critical for capturing some of the finer details on the foot surface.

For \ourOptim, the use of uncertainty in weighting the various TOC samples used for fitting the FIND model provides a boost to fitting accuracy.