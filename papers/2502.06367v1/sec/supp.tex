\clearpage
\setcounter{page}{1}

\maketitlesupplementary


\section{Further examples}

\paragraph{Synthetic dataset.} In Figure \ref{fig:synth-supp}, we show additional samples of our synthetic dataset.



\paragraph{In-the-wild predictions.} In Figure \ref{fig:itw-supp}, we show further qualitative predictions of our TOC and normal predictions on in-the-wild images.



\paragraph{Reconstruction.} In Figure \ref{fig:qual-supp}, we show further qualitative reconstruction comparison to existing methods FOUND \cite{boyne2024found} and COLMAP \cite{schonberger2016structure, schonberger2016pixelwise}.

\begin{figure}[!h]
    \centering
    \begin{minipage}{.8\linewidth}
      \centering

    \begin{tabular}{*{4}{@{}Q{0.2375\linewidth}}@{}}
    (a) & (b) &  (c) & (d)
    \end{tabular}
      
      \includegraphics[width=0.95\linewidth]{images/supp/collage-synth-supp1_comp.png}
    \end{minipage}%

    \vspace{10pt}
    
    \begin{minipage}{.8\linewidth}
      \centering

    % \begin{tabular}{*{4}{@{}Q{0.2375\linewidth}}@{}}
    % (a) & (b) &  (c) & (d)
    % \end{tabular}
      
      \includegraphics[width=0.95\linewidth]{images/supp/collage-synth-supp2_comp.png}
    \end{minipage}
    
    \caption{\textbf{SynFoot2 examples.} Further examples of SynFoot2, showing (a) RGB, (b) TOC, (c) surface normals, and (d) segmentation masks.}
    \label{fig:synth-supp}
    
\end{figure}


\section{Method hyperparameters}

We define the hyperparameters used for \ourSfM in Table \ref{fig:sfm-hyper}.


\begin{figure*}
    \centering
    \begin{minipage}{.4\linewidth}
      \centering

    \begin{tabular}{*{4}{@{}Q{0.2375\linewidth}}@{}}
    (a) & (b) &  (c) & (d)
    \end{tabular}
      
      \includegraphics[width=0.95\linewidth]{images/supp/collage_itw_supp1_comp.png}
    \end{minipage}%
    \begin{minipage}{.4\linewidth}
      \centering

    \begin{tabular}{*{4}{@{}Q{0.2375\linewidth}}@{}}
    (a) & (b) &  (c) & (d)
    \end{tabular}
      
      \includegraphics[width=0.95\linewidth]{images/supp/collage_itw_supp2_comp.png}
    \end{minipage}%
    \begin{minipage}{.15\textwidth}
      \centering
      \includegraphics[width=0.7\linewidth]{images/supp/toc_unc_legend.png}
    \end{minipage}
    
    \caption{\textbf{In-the-wild predictions.} Further examples of in-the-wild predictions of our TOC predictor, showing showing (a) RGB input, (b) TOC $\toc$, (c) TOC uncertainty $\tocstd$, and (d) surface normals. We show some more challenging examples in the right figure, even showing robustness to a secondary foot in the image.}
    \label{fig:itw-supp}
    
\end{figure*}



\begin{figure*}
\centering
\newcommand\qualfig[1]{\includegraphics[width=0.45\linewidth]{images/supp/qual/#1_comp.png}}
\begin{tabular}{ccc}
\qualfig{0039} & \qualfig{0040} & \multirow{2}{*}[\dimexpr 1.5cm]{\includegraphics[width=0.05\linewidth]{images/supp/qual/scale.png}}\\[10pt]
\qualfig{0042} & \qualfig{0046} \\

\end{tabular}
\caption{\textbf{Qualitative reconstruction results.} The reconstruction quality is compared across four further scans in the Foot3D dataset, comparing COLMAP, FOUND, \ourSfM and \ourOptim.}
\label{fig:qual-supp}
\end{figure*}



\begin{table*}
    \centering
    
\begin{tabular}{Q{0.6\linewidth}cc}
    \toprule
    Description & Value & Unit \\\midrule
    Number of samples taken per image & 3000 & - \\
    Maximum $\ell 2$ distance to consider a correspondence a match & 0.002 & - \\
    Upsampling factor for subpixel matching & 8 & - \\
    Reprojection error above which to filter triangulated points&  3 & pixels \\
    Crop the mesh to this padding around reconstructed point cloud & 1 & mm \\
    Foot mesh height interval & [0, 150] & mm \\
    Poisson reconstruction - depth & 8 & - \\
    Poisson reconstruction - iterations & 8 & - \\
    % Poisson reconstruction - CGDepth & 0 \\
    % Poisson reocnstruction - Pointweight & 0 \\
    \bottomrule
\end{tabular}
    \caption{Hyperparameters chosen for \ourSfM.}
    \label{fig:sfm-hyper}
\end{table*}