\section{Related Work}
Semantic similarity techniques are well-explored in entity linking, where embedding models match textual mentions to corresponding database entries. Hou et al. introduced a method that enhances entity embeddings by incorporating fine-grained semantic information, thereby improving the learning of contextual commonality and achieving state-of-the-art performance in entity linking \cite{hou2020improving}. Similarly, Pereira and Ferreira proposed E-BELA, an approach that aligns vector representations of mentions and entities in a shared space using literal embeddings, facilitating effective linking through similarity metrics \cite{pereira2024ebela}. These methodologies are particularly effective in domains with straightforward entity disambiguation requirements, where minimal additional context is necessary. 

Existing methods for carbon footprint estimation rely heavily on manual mapping of product components to LCA databases. This is a challenging task, requiring both LCA expertise and specialist knowledge of components and materials. Several studies have explored machine learning approaches to partially automate this process. Flamingo \cite{balaji2023flamingo} uses semantic similarity to match end products to environmental impact factors such as carbon emissions in the ecoinvent database. It does not make use of the fine-grained component details present in a BOM, and additionally it uses a private dataset and supervised approach to train an auxiliary classifier. This classifier is used in conjunction with zero-shot semantic similarity matching to find database matches. Similarly, CaML \cite{balaji2023caml} maps product text descriptions to industry sector codes, which can be used to make a coarse-grained estimate of carbon production.