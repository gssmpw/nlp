
\documentclass[11pt]{article}
\usepackage{nodalida2025}
\usepackage{times}
\usepackage{url}
\usepackage{float}
\usepackage{latexsym}
\usepackage{graphicx}
\usepackage{booktabs}
\usepackage{array}
\usepackage[T1]{fontenc}
\usepackage{amsmath}
\usepackage{sansmath} %
\usepackage{hyperref} 

\usepackage{courier} %
\usepackage[dvipsnames,table,xcdraw]{xcolor} %
\usepackage[framemethod=TikZ]{mdframed}
\newmdenv[
backgroundcolor=cyan!10, 
roundcorner=5pt, 
linewidth=1pt, 
align=center,
  font=\fontfamily{cmss}\selectfont %
]{custommdframed}
\newcommand{\at}{\makeatletter @\makeatother}



\aclfinalcopy %

\title{Entity Linking using LLMs for Automated \\Product Carbon Footprint Estimation}

\author{Steffen Castle \qquad 
Julian Moreno Schneider  \qquad 
Leonhard Hennig \qquad
Georg Rehm 
\\
Deutsches Forschungszentrum für Künstliche Intelligenz GmbH (DFKI) \\ Berlin, Germany \\
{\tt first.last\at dfki.de }
}
\date{} 

\begin{document}
\maketitle
\begin{abstract}
Growing concerns about climate change and sustainability are driving manufacturers to take significant steps toward reducing their carbon footprints. For these manufacturers, a first step towards this goal is to identify the environmental impact of the individual components of their products. We propose a system leveraging large language models (LLMs) to automatically map components from manufacturer Bills of Materials (BOMs) to Life Cycle Assessment (LCA) database entries by using LLMs to expand on available component information. Our approach reduces the need for manual data processing, paving the way for more accessible sustainability practices.
\end{abstract}

\section{Introduction}
Increasing awareness of climate change and sustainability has put pressure on manufacturers to reduce their carbon footprints. Regulation such as the EU Corporate Sustainability Reporting Directive (CSRD) and the European Green Deal has further emphasized the need for transparent and accurate environmental impact assessments. A fundamental step in this process is determining the environmental impact of a product, particularly the carbon emissions generated during production. LCA databases, such as ecoinvent \cite{ecoinvent}, provide detailed information for this purpose. However, linking the raw components of a product, represented in a BOM, to relevant entries in an LCA database remains a labor-intensive task requiring specialized knowledge of manufacturing materials and processes.

Recent advances in artificial intelligence, particularly large language models (LLMs), present an opportunity to automate this process. Because they are trained on vast amounts of information and are able to efficiently synthesize their knowledge as textual data, they may be capable of providing additional context in order to link components to their production processes in a LCA database. 

In this work, we investigate the use of LLMs to streamline the product carbon footprint estimation process by mapping components from BOMs directly to a LCA database. We propose a multi-step approach using a pretrained LLM to identify and summarize component information and mapping this summary to a database using semantic similarity.

\section{Related Work}

Semantic similarity techniques are well-explored in entity linking, where embedding models match textual mentions to corresponding database entries. Hou et al. introduced a method that enhances entity embeddings by incorporating fine-grained semantic information, thereby improving the learning of contextual commonality and achieving state-of-the-art performance in entity linking \cite{hou2020improving}. Similarly, Pereira and Ferreira proposed E-BELA, an approach that aligns vector representations of mentions and entities in a shared space using literal embeddings, facilitating effective linking through similarity metrics \cite{pereira2024ebela}. These methodologies are particularly effective in domains with straightforward entity disambiguation requirements, where minimal additional context is necessary. 

Existing methods for carbon footprint estimation rely heavily on manual mapping of product components to LCA databases. This is a challenging task, requiring both LCA expertise and specialist knowledge of components and materials. Several studies have explored machine learning approaches to partially automate this process. Flamingo \cite{balaji2023flamingo} uses semantic similarity to match end products to environmental impact factors such as carbon emissions in the ecoinvent database. It does not make use of the fine-grained component details present in a BOM, and additionally it uses a private dataset and supervised approach to train an auxiliary classifier. This classifier is used in conjunction with zero-shot semantic similarity matching to find database matches. Similarly, CaML \cite{balaji2023caml} maps product text descriptions to industry sector codes, which can be used to make a coarse-grained estimate of carbon production. 

\section{Background}
\subsection{LCA Database}

In order to determine the environmental impact of a component, the first step is to map it to its corresponding entry in the LCA database. LCA databases contain lists of manufacturing process names along with technical descriptions and other information. Our approach uses ecoinvent \cite{ecoinvent} as the primary LCA database. A sample of these two fields for illustrative purposes is shown in Figure \ref{entry}. For each entry, the database includes additional information describing process inputs and outputs and other life cycle information. The most task-relevant of this data is the environmental impact data including carbon emissions. Once a component has been mapped to a database entry, this information can be used in a relatively straightforward way to estimate the total carbon footprint of a product by summing the emissions for each component. Our method thus focuses on the main challenge to the carbon footprint estimation, which is linking the components from the BOM to the LCA database.

The current manual process for linking items to the LCA database involves two sequential stages. First, a non-specialist conducts a preliminary appraisal to filter straightforward matches. Using the item’s material and description, they generate a shortlist of initial candidates (typically around five entries). These potential matches are then reviewed by an LCA expert, who validates their accuracy. In cases of mismatches, the expert manually corrects the mappings in the second stage. While expert oversight remains essential for quality assurance, our objective is to optimize this workflow by automating the preliminary non-specialist tasks. This eliminates manual effort while preserving the critical role of expert verification.

\section{Negative Complexity Results} \label{sec:intractable}
% Introduction of what this section is about

%In the previous section, we provided positive complexity results for Local and Global I-SHAP and B-SHAP variants. In this section, we present negative complexity findings. The first subsection addresses the intractability of computing Local I-SHAP for the family of RNN-ReLus  under the feature independence assumption. The second subsection focuses on scenarios where B-SHAP is difficult to compute exactly.
%\subsection{The problem \texttt{LOC-I-SHAP}(\texttt{RNN-ReLu}, \texttt{IND}) is NP-Hard.}
\subsection{When Interventional SHAP is Hard}
\label{subsec:rnnreluhard}
In this subsection, we present intractability results for interventional SHAP.
%The main result of this section is given in the following: 

\begin{theorem} \label{thm:relushap}
    The decision version of the problem \emph{\texttt{LOC-I-SHAP}}\emph{\texttt{(RNN-ReLu}, \texttt{IND)}} is \emph{NP-Hard}.
\end{theorem}

Recall that an RNN-ReLu model $R$ over $\Sigma$ is parametrized as: $\langle h_{init}, W, \{v_{\sigma}\}_{\sigma \in \Sigma}, O\rangle$ and processes a sequence sequentially from left-to-right such that $h_{\epsilon} = h_{init}$, $h_{w'\sigma} = \emph{ReLu}(W \cdot h_{w'} + v_{\sigma} )$, and $f_{R}(w) = I(O^{T} \cdot h_{w} \geq 0)$. The proof of Theorem~\ref{thm:relushap} leverages the efficiency axiom of Shapley values. Specifically, for a sequential binary classifier $f$ over alphabet $\Sigma$, an integer $n$, and $i \in [n]$, the efficiency property can be expressed as~\citep{arenas23}:
\begin{equation} \label{eq:efficiency}
\sum\limits_{i=1}^{n} \phi_i(f,\x,i,P_{unif}) =  f(\x) - 
 P_{unif}^{(n)} (f(\x) = 1) \end{equation}
%$$
%\begin{cases}
%    h_{\epsilon} = h_{init} \\ 
%    h_{w'\sigma} = \emph{ReLu}(W \cdot h_{w'} + v_{\sigma} ) \\ 
%    f_{R}(w) = I(O^{T} \cdot h_{w} \geq 0)
%\end{cases}
%$$



%The proof of Theorem \ref{thm:relushap} leverages the efficiency axiom of Shapley values. Formally, given a sequential binary classifier $f$ over the alphabet $\Sigma$, an integer $n$, and $i \in [n]$, the efficiency property can be written as~\citep{arenas23}:
%\begin{equation}
%\label{eq:efficiency}
%\sum\limits_{i=1}^{n} \phi_i(f,\x,i,P_{unif}) =  f(\x) - 
% \nonumber 
% P_{unif}^{(n)} (f(\x) = 1) 
%\end{equation} 


where $P_{unif}$ denotes the uniform distribution over $\Sigma^{\infty}$. This property will be used to reduce the $\texttt{EMPTY}$ problem to the computation of interventional SHAP in our context. The $\texttt{EMPTY}$ problem is defined as follows: Given a set of models $\mathcal{F}$, $\texttt{EMPTY}$ takes as input some $f \in \mathcal{F}$ and an integer $n > 0$ and asks if $f$ is empty on the support $\Sigma^{n}$ (i.e., is the set $\{ \x \in \Sigma^{n}:~f(\x) = 1 \}$ empty?). The connection between local interventional SHAP and the emptiness problem is outlined in the following proposition:

%where $P_{unif}$ denotes the uniform distribution over $\Sigma^{\infty}$. This property will be utilized to reduce the $\texttt{EMPTY}$ problem to the computation of interventional SHAP in our context. The $\texttt{EMPTY}$ problem is defined as follows: Given a set of models $\mathcal{F}$, $\texttt{EMPTY}$ takes as input some $f\in \mathcal{F}$ and asks whether $f$ on the support $\Sigma^{n}$ empty? (i.e. Is the set $\{ \x \in \Sigma^{n}:~f(\x) = 1 \}$ empty?). The connection between the Local I-SHAP problem and the emptiness problem is highlighted in the following proposition:

%As noted by \cite{arenas23}, the efficiency property of the SHAP score reveals an intresting connection between the SHAP computational problem and the model counting problem [\textbf{citation needed of the model counting problem}]. Specifically, Equation \ref{eq:efficiency} highlights the fact that computing I-SHAP scores for binary classifiers is at least as hard as counting the number of examples classified as positive by the model being explained. To formalize this insight, we define the emptiness problem, which is the decision problem related to model counting.  Let $\mathcal{M}$ be a class of sequential binary classifiers over $\Sigma$,  we have:

%\textbf{Problem:} \texttt{EMPTY}($\mathcal{M}$) \\
%\textbf{Instance:} a model $M$, an integer $n > 0$  \\
%\textbf{Output:} Is $M$ on the support $\Sigma^{n}$ empty? (i.e. Is the set $\{ x \in \Sigma^{n}:~f_{M}(x) = 1 \}$ empty?) 


\begin{proposition} \label{prop:modelcountingshap}
Let $\mathcal{F}$ be a class of sequential binary classifiers. Then, \emph{\texttt{EMPTY($\mathcal{F}$)}} can be reduced in polynomial time to the problem $\emph{\texttt{LOC-I-SHAP}}(\mathcal{F}, \emph{\texttt{IND}})$.
\end{proposition}
%\emph{Proof.}
%    Let $\mathcal{F}$ be a class of binary classifiers, $f \in \mathcal{F}$, and $n > 0$. Choose an arbitrary sequence $\x \in \Sigma^{n}$. Without loss of generality, we assume $f(\x) = 0$ (otherwise, \texttt{EMPTY}($\mathcal{F}$) for the instance $<F,n>$ is already solved). By Equation~\ref{eq:efficiency}, we have that the set $\{\x \in \Sigma^{n}:f(\x) = 1\}$ is empty if and only if for any $i \in [n]$, the input instance $<f,\x,i,P_{unif},0>$ of the decision problem associated to $\texttt{LOC-I-SHAP}(\texttt{RNN-ReLu}, \texttt{IND})$ yields a \textit{No} answer. \quad \quad \quad \quad\quad \quad\quad \quad\quad \quad \quad \quad\quad \quad\quad \quad\quad \quad \quad  $\qedsymbol{}$


Proposition \ref{prop:modelcountingshap} suggests that proving the NP-Hardness of the problem $\texttt{EMPTY}(\texttt{RNN-ReLu})$ is a sufficient condition to yield the result of Theorem \ref{thm:relushap}. This condition is asserted in the following lemma:
\begin{lemma} \label{lemma:emptyrnnrelu}
\emph{\texttt{EMPTY}}\emph{\texttt{(RNN-ReLu)}} is NP-Hard. 
\end{lemma}



The proof of Lemma~\ref{lemma:emptyrnnrelu} is done via a reduction from the \emph{closest string problem} (CSP), which is NP-Hard~\citep{li2000closest}. CSP takes as input a set of strings $S := \{w_{i}\}_{i \in [m]}$ of length $n$ and an integer $k > 0$. The goal is to determine if there exists a string $w' \in \Sigma^{n}$ such that for all $w_{i} \in S$, $d_{H}(w_{i}, w') \leq k$, where $d_{H}(.,.)$ is the Hamming distance: $d_{H}(w,w') := \sum\limits_{i=1}^{ [|w|]} \mathrm{1}_{w_{j}}(w'_{j})$. We conclude our reduction by proving the following proposition, with the proof in Appendix~\ref{app:ISHAPRNN}:


%The proof of lemma \ref{lemma:emptyrnnrelu} will be carried out by a reduction from the \emph{closest string problem} (CSP), which is known to be NP-Hard~\citep{li2000closest}. The CSP problem takes as input a collection of strings $S = \{w_{i}\}_{i \in [m]}$ whose length is equal to $n$, and an integer $k > 0$. The problem outputs yes whether there exist a string $w' \in \Sigma^{n}$ such that for any $w_{i} \in S$, we have $d_{H}(w_{i}, w') \leq k$,    where $d_{H}(.,.)$ is the Hamming distance given as: $d_{H}(w,w') = \sum\limits_{i=1}^{ [|w|]} \mathrm{1}_{w_{j}}(w'_{j})$. Thus, we conclude our reduction by proving the following proposition, whose proof appears in the appendix:

\begin{proposition}
    \emph{\texttt{CSP}} can be reduced in polynomial time to \emph{\texttt{EMPTY}}\emph{\texttt{(RNN-ReLu)}}.
\end{proposition}
    
    %$\bullet$ \textbf{Problem:} \texttt{CSP} \\
    %\textbf{Instance:} A collection of strings $S = \{w_{i}\}_{i \in [m]}$ whose length is equal to $n$, an integer $k > 0$, \\ 
    %  \textbf{Output} Does there exist a string $w' \in \Sigma^{n}$ such that for any $w_{i} \in S$, we have $d_{H}(w_{i}, w') \leq k$?  \\ 
     %   where $d_{H}(.,.)$ is the Hamming distance given as: $d_{H}(w,w') = \sum\limits_{i=1}^{ [|w|]} \mathrm{1}_{w_{j}}(w'_{j})$. \\

     %\textcolor{blue}{@Shahaf (Shortening suggestion): You can omit the rest of this subsection by delegating the details of the reduction from \texttt{CSP} to \texttt{LOC-I-SHAP}(\texttt{RNN-ReLu}, \texttt{IND}) to the appendix (see section \ref{app:ISHAPRNN} of the appendix). One possible way to conclude this subsection is by introducing a mathematical result stating the existence of a polynomial-time algorithm that performs the reduction e.g:
      %\begin{proposition}
      %    There exists a polynomial-time algorithm that takes as input an input instance $I = <S, k>$ of the \texttt{CSP} and outputs a RNN-ReLu $R_{I}$ such that:
      %    $$ \text{There exists no $k$-closest string of $S$ }  \iff R_{I} \text{ is empty}$$ 
      %\end{proposition} 
     %} 
     
     %The \texttt{CSP} problem is known to be NP-Hard \cite{li2000closest}. The reduction strategy consists at constructing in polynomial time a RNN that accepts only closest string strings for a given input instance $(S,k)$ of the \texttt{CSP} problem. Thus, the \texttt{CSP} outputs a YES answer on the given input instance if and only if the constructed RNN-ReLu is empty on the support $\Sigma^{n}$,  yielding straightforwardly the result of lemma \ref{lemma:emptyrnnrelu}. 
     %)
     %We structure the proof of this reduction strategy into two parts:
     %\begin{itemize}
     %    \item Given an arbitrary string $w$ and an integer $k > 0$, a construction of a RNN-ReLu that simulates the computation of the Hamming distance for $w$ such that the information of whether the hamming distance exceeds the threshold $k$ is encoded in the activation of a particular neuron withinin the constructed RNN-ReLu. We shall refer to this procedure under the name $\texttt{CONSTRUCT}(w,k)$.
    %     \item Concatenate the RNN-ReLus outputted by the procedure $\texttt{CONSTRUCT}(.,.)$ into a unifying RNN cell. Afterwards, we set a convenient output matrix that achieves the target objective of the construction.
    % \end{itemize}
     
    % \paragraph{The procedure \texttt{CONSTRUCT}($w,k$).}   Fix a string of reference $w \in \{0,1\}^{n}$, and an integer $k > 0$. The procedure $\texttt{CONSTRUCT}(w,k)$ returns  a RNN cell of dimension $|w| + 1$ that meets the satisfies the properties stated above discussion. The parametrization of $\texttt{CONSTRUCT}(w,k)$ is given as follows: 
%\begin{itemize}
%\item \textbf{The initial state vector.} $h_{init} = [1, 0 \ldots, 0 ]^{T} \in \mathbb{R}^{n+1}$
% \item \textbf{The transition matrix.} The transition matrix depends on $k$, and is given as
%\begin{equation}\label{transition}
%W_{k} = \begin{pmatrix}
%   0 & 0 & .& . & . & 0 & 0\\
%   1 & 0 & . & . & . & 0  & 0\\ 
%   0 & 1 & . & . & . & 0  & 0\\ 
%    . & . & . & . & . & .  & .\\
%   . & . & . & . & . & .  & . \\
%   0 & 0 & .& . & . & 0 & -k \\ 
%   0 & 0 & .. & . & . & 0 &  1 
%\end{pmatrix} \in \mathbb{R}^{(n+1) \times (n+1)}
%\end{equation}
%\item \textbf{The embedding vectors.} Embedding vectors depend on the string of reference $w$. In order to avoid confusion, we will index them with the superscript $w$. The construction of the embedding vectors is given as follows: \\ 
%For a symbol $\sigma \in \{0,1\}$ and $l \in [n]$, we have $v_{\sigma}^{w}[l] = 1 $ if $w_{l} \neq \sigma$. All the other elements of $v_{\sigma}^{w}$ are set to $0$. 
%\end{itemize}
%The following proposition highlights formally that this construction satisfies the sought property:
%\begin{proposition}\label{prop:reluproperties}
%Let $w \in \{0,1\}^{n}$ be an arbitrary string, and $k > 0$ be an arbitrary integer. The procedure $\emph{\texttt{CONSTRUCT}}(w,k)$ outputs a RNN cell $<h_{init}, W_{k}, \{v_{\sigma}^{(w)}\}_{\sigma \in \{0,1\}}>$ satisfies the following properties: 
%\begin{enumerate}
%    \item For any string $w' \in \{0,1\}^{s}$ where $ s < n$, we have $h_{w}[s] = d_{H}(w,w')$,
%    \item For a string $w' \in \{0,1\}^{n}$, we have 
    %\begin{equation}\label{eq:levsimulate}
     %   h_{w'}[n] = \text{ReLu}(d_{H}(w, w') - k)
    %\end{equation}
%\end{enumerate}
%\end{proposition}
%The most important property highlighted by Proposition \ref{prop:reluproperties} is given in Equation \ref{eq:levsimulate}. It shows that the $n$-th neuron of the constructed RNN Cell encodes the Hamming distance between the string of reference $w$ and the input string $w'$. Note that the activation value of this neuron is equal to $0$ if and only if $d_{H}(w,w') \leq k$. Otherwise, it is greater or equal to $1$.  

%\paragraph{Concatenation and Output Matrix instantiation.} 
%Let $<S, k>$ be an input instance of the closest string problem.The final RNN cell will be constructed as a concatenation of the RNN cells $<h_{init}, W_{k}, \{v_{\sigma}^{w^{(i)}}\}_{i \in [|S|]}$ where the set $\{w^{(i)}\}_{i \in |S|}$ are elements of $S$. \\ 
%The concatenation operation of two RNN cells, say $<h_{1}, W_{1}, v_{\sigma}^{(1)}>$ and $<h_{1}, W_{2}, v_{\sigma}^{(2)}>$, result in a new cell given as $<\begin{pmatrix}
%    h_{1} \\h_{2}
%\end{pmatrix}, \begin{pmatrix}
%    W_{1} & 0 \\ 
%    0 & W_{2}
%\end{pmatrix}$, $\{ \begin{pmatrix} v_{\sigma}^{1} \\ 
%v_{\sigma}^{2}
%\end{pmatrix}
%\}_{\sigma \in \Sigma}>$. %\footnote{Although the concatenation operation is non-commutative, the order according to which we perform the concatenation operator does't affect the result of the reduction.} \\
%We concatenate all RNN-ReLus resulting from the procedure \texttt{CONSTRUCT}(.,.) on the instances $\{(w,k)\}_{w \in S}$. 
%The output matrix of the resulting RNN-ReLu $O \in \mathbb{R}^{(n+1) \cdot |S|}$ is chosen such that, for any collection of vectors $\{h_{i}\}_{i \in [|S|]}$ in $\mathbb{R}^{n+1}$, we have:
%$$O^{T} \cdot \begin{bmatrix}
%    h_{1} & \ldots & h_{s}
%\end{bmatrix} = \sum\limits_{i \in [|S|]} - h_{i}[n] + \frac{1}{2} \cdot h_{1}[n+1]$$

%Under this setting, it is to be noted, in light of the properties of the RNN-ReLu cell outputted by the procedure $\texttt{CONSTRUCT}(.,.)$ (Proposition \ref{prop:reluproperties}), and the fact that the activation value of the $(n+1)$-neuron is always equal to $1$ by construction, the output of the constructed RNN-ReLu on an input sequence $w'$ is equal to $1$ if only if $w$ for all $w \in S$, we have $d_{H}(w,w') \leq k$. Consequently, the constructed RNN-ReLu is empty on the support $n$ if and only if there exists no string $w'$ such that for all $d_{H}(w,w') \leq k$. 
%\subsection{When Baseline SHAP is Hard} \label{subsec:baseline}
The main result of this section is given in the following theorem:
\begin{theorem} \label{thm:intractable}
\begin{enumerate}
    \item \textcolor{green}{Unless P = NP, the problem \texttt{LOC-B-SHAP}(\texttt{SIGMOID}) can't be computed exactly in polynomial time.}
    \item The decision versions of the problem  and $\emph{\texttt{LOC-B-SHAP}(\texttt{RNN-ReLu})}$ is \emph{co-NP-Hard},
    \item The decision version of the problem $\emph{\texttt{LOC-B-SHAP}(\texttt{RF})}$ \footnote{Random Forests in the context of this theorem is the binary classifier that classifies an instance according to the majority vote mechanism.} is \emph{NP-Hard}.
    \end{enumerate}
\end{theorem}

The full proof of Theorem \ref{thm:intractable} can be found in Appendix. 

% Introduce what follows in this section
In the sequel, we shall provide the proof sketch of the reduction made to prove  the Hardness of the problems \texttt{LOC-B-SHAP}(\texttt{SIGMOID}). Due to space constraints and the similarities between the reduction strategies employed for \texttt{LOC-B-SHAP}(\texttt{SIGMOID}) and \texttt{LOC-B-SHAP}(\texttt{RNN-ReLu}), we won't discuss this latter in the main article.

The reduction strategy to prove results of Theorem \ref{thm:intractable} is based on top of the Dummy Player problem of Weighted Majority Games (WMGs) \cite{freixas2011complexity}, the reduction performed to show the NP-Hardness of the problem \texttt{B-SHAP}(\texttt{RF}) is made by reduction from the classical SAT problem.
\subsubsection{The problems \texttt{LOC-B-SHAP}(\texttt{SIGMOID}) and  \texttt{LOC-B-SHAP}(\texttt{RNN-ReLu}) are co-NP-Hard.}
As mentionned earlier, the reduction strategy to prove statement 1 of Theorem \ref{thm:intractable} is based on the dummy player problem of WMGs. The steps of this reduction strategy are similar for both these problems. Thus, we shall focus in the main text on the case of $\texttt{B-SHAP}(\texttt{SIGMOID})$. The case of RNN-ReLus will be fully examined in the appendix. 

Before outlining the steps of the reduction, we'll first introduce the required background of WMGs. 

\paragraph{Weighted Majority Games.} A coalitional game is formally defined by a pair $G = (N,v)$, where $N$ is an integer referring to the number of players in the game, and $v$ is the value function of the game that maps each subset of players, referred to as a coalition, to the value generated by the coalition if they accept to cooperate in the game.  

% What is a Weighted Majority game (formal definition, parametrization etc.)
A Weighted Majority Game (WMG) is a particular class of coalitional games suitable to model scenarios where players have different voting powers. A coalition is winning if the total weight of its members exceeds a predetermined threshold:
\begin{definition} \label{def:wmg}
    A WMG $G$ is a coalitional game parametrized by the tuple $  <N,\{n_{i}\}_{i \in [N]}, q>$, where:
    \begin{itemize}
        \item $N$: The number of players,
        \item For $i \in [N]$, $n_{i}$ is an integer corresponding to the voting power of player $i$, 
        \item $q$ is an integer corresponding to the winning quota in the game  
    \end{itemize}
    The value function associtated to the WMG $G$ is given as:
    $$v_{G}(S) = \begin{cases}
        1 & \text{if}~~ \sum\limits_{i \in S} n_{i} \geq q \\
        0 & \text{otherwise}
    \end{cases}$$
\end{definition}

% The dummy player: Definition and associated computational problem
%\paragraph{Dummy players in WMGs.} 

A \emph{dummy player} in a WMG $G$ is a player whose voting power brings no value to any coalition of players in the game, i.e a player $i$ is \textit{dummy} if: :
$$\forall S \subseteq N \setminus \{i\}:~~v_{G}(S \cup \{i\}) = v_{G}(S)$$

Deciding if a player is dummy in a WMG is known to be co-NP-Complete \cite{freixas2011complexity}.

\begin{comment}
An interesting characterization of dummy players in WMGs highlighted in the following proposition. This characterization will prove useful in our reduction proof to our target problems that will be detailed:
\begin{proposition}\label{prop:characterization_dummyplayers}
      Let $G = (N,\{n_{i}\}_{i \in [N]}, q)$ be a WMG. A player $i \in [N]$ is dummy if and only if:
     \begin{equation} \label{eq:expression}
     \forall S \subseteq [N] \setminus \{i\}:~~ \sum\limits_{j \in S} n_{j}  \notin [q - n_{i}, q]
     \end{equation}
\end{proposition}
The proof of this proposition can be found in Appendix. 
\end{comment}

\paragraph{Reduction of \texttt{DUMMY} to \texttt{LOC-B-SHAP}(\texttt{SIGMOID}).} Recall that a sigmoidal neural network $M$ of $N$ input features computes a function from $\mathbb{R}^{n}$ to $[0,1]$ given as:
$$f_{M}(x; \{w_{i}\}_{i \in [n]}, b) = \sigma(\sum\limits_{i = 1}^{N} w_{i} \cdot x_{i} + b)$$
where $\sigma(.)$ is the sigmoidal function. 

The strategy of the reduction consists at constructing a sigmoidal neural network $f_{G}$ that simulates a given WMG $G= N, \{n_{i}\}_{i \in [N]}, q$, such that: 
$$f_{G}(x_{S}; x_{\bar{S}}^{(ref)}) \approx v(S)$$
where the instance to explain $x$ is the vector $[1, \ldots, 1]$, and $x^{(ref)}$ is the vector $[0, \ldots, 0]$. 

If a player $i$ in the WMG $G$ is dummy, then, for any $S \subseteq N \setminus \{i\}$, we have:
$$f_{G}(x_{S \setminus \{i\}}; x_{\bar{S \setminus \{i\}}}^{(ref)}) - f_{G}(x_{S}; x_{\bar{S}}^{(ref)}) \approx 0$$
which implies that $\texttt{LOC-B-SHAP}(f_{G}, i, x, x^{(ref)}) \approx 0$. 

On the other hand, if the player $i$ is not dummy, then there must exist a coalition $S$ of size $N-1$ \footnote{By the monotonicity of the value function of a WMG, if $S$ is a winning coalition, then for any $S'$ such that $S \subseteq S'$, $S'$ is also a winning coalition.} in $N \setminus \{i\}$ such that: 
\begin{align*}
    f_{G}(x_{S \setminus \{i\}}; x_{\bar{S\setminus \{i\}}}^{ref}) - f_{G}(x_{S}; x_{\bar{S}}) \approx 1
\end{align*}
which implies that:
{\small 
\begin{align*}
\texttt{LOC-B-SHAP}(f_{G},i,x,x^{(ref)}) & \geq \frac{1}{N} [f_{G}(x_{S \setminus \{i\}}; x_{\bar{S\setminus \{i\}}}^{ref}) - f_{G}(x_{S}; x_{\bar{S}})] \\
    &\gtrapprox \frac{1}{N}
\end{align*}
}

{\small
\begin{algorithm}
\caption{Reduction of the \texttt{Dummy} problem to \texttt{LOC-B-SHAP}(\texttt{SIGMOID})}
\label{alg:WMG2b-SHAP}
\begin{algorithmic}[1]
\REQUIRE A WMG $G = <N, \{n_{j}\}_{j \in [N]},q>$, $i \in [N]$
\ENSURE A Sigmoidal Neural Network $\sigma$ over $\mathbb{R}^{N}$, an integer $i \in [N]$, $(x,x^{ref}) \in \mathbb{R}^{2},~\epsilon \geq 0$ 
\STATE $x \leftarrow [1 , \ldots , 1]$
\STATE $x^{ref} \leftarrow [0, \ldots , 0]$
\STATE $\epsilon \leftarrow \frac{1}{N + 1}$
\STATE $C \leftarrow 2 \cdot \log(N)$
\STATE Construct the Sigmoidal Neural Network $f$ given as:
$$f_{G}(x) = \sigma(C \cdot (x - q + \frac{1}{2}))$$
\RETURN $f_{G},~i,~x,~x^{(ref)},~\epsilon$
\end{algorithmic}
\end{algorithm}
}

Algorithm \ref{alg:WMG2b-SHAP} provides the exact steps of the reduction. 

The following theorem provides a proof of the correctness of the reduction outlined in algorithm \ref{alg:WMG2b-SHAP}, formalizing the intuition behind the reduction discussed above:
\begin{proposition} \label{prop:reductionsigmoid}
    Let $G = (N, \{n_{i}\}_{i \in [N]}, b)$ be a WMG, and $<f_{G}, i, x,x^{(ref)}, \epsilon>$ be the output of Algorithm \ref{alg:WMG2b-SHAP} on the input $G$. 

    The player $i$ is dummy if and only if \emph{\texttt{LOC-B-SHAP}}$(f_{G}, i , x, x^{(ref)}) \leq \epsilon$.
\end{proposition}

The complete proof of Proposition \ref{prop:reductionsigmoid} can be found in section \ref{app:BSHAPSIGMOID} of the supplementary material.

\paragraph{The Hardness of \texttt{LOC-B-SHAP}(.) implies the Hardness of \texttt{GLO-B-SHAP}(.,.) and \texttt{LOC-I-SHAP}(.,.).} The objective of this final segment of this section is to complement negative complexity results provided in Theorem \ref{thm:intractable} with other results based on the hardness relations between the computational problems associated to different variants of SHAP scores, thus completing the results illustrated in Table \ref{fig:summaryresults}. Indeed, the following proposition highlights the existence of configurations where computing the Global B-SHAP and Local I-SHAP are at least as hard as computing Local B-SHAP:
\begin{proposition} \label{prop:hardnessrelation}
    Let $\mathcal{M}$ be a class of models and $\mathcal{P}$ a class of probability distributions such that $\texttt{\emph{EMP}} \preceq_{P}  \mathcal{P}$ \footnote{The symbol $\preceq_{P}$ signifies "polynomially reducible to"}. Then, 
    \texttt{\emph{LOC-B-SHAP}}($\mathcal{M}$)  $\preceq_{P}$ \texttt{\emph{GLO-B-SHAP}}($\mathcal{M}$, $\mathcal{P}$) and \texttt{\emph{LOC-B-SHAP}}($\mathcal{M}$)  $\preceq_{P}$   \texttt{\emph{LOC-I-SHAP}}( $\mathcal{M}, \mathcal{P}$) .
\end{proposition}

Given that $\texttt{\text{EMP}} \preceq \texttt{\text{HMM}}$ (See section \ref{app:reductiontree} of the supplementary material), one can obtain immediate corollaries from Theorem \ref{thm:intractable} and proposition \ref{prop:hardnessrelation}:
\begin{corollary} \label{cor:globshaphard}
    \begin{enumerate}
        Let $\texttt{\emph{P}} \in \{ \texttt{\emph{EMP}}, \texttt{\emph{HMM}}\}$.
        \item For any $\texttt{\emph{M}} \in \{\texttt{\emph{SIGMOID}}, \texttt{\emph{RNN-ReLu}} \}$, the decision version of the problems $\texttt{\emph{GLO-B-SHAP}}(\texttt{\emph{M}}, \texttt{\emph{P}})$ and $\texttt{\emph{LOC-I-SHAP}}(\texttt{\emph{P}})$ are \emph{co-NP-Hard}.
        \item The decision version of the problems  $\texttt{\emph{GLO-B-SHAP}(\texttt{\emph{M}}, \texttt{\emph{P}}})$ and $\texttt{\emph{LOC-I-SHAP}}(\texttt{\emph{P}})$ are \emph{NP-Hard}. 
    \end{enumerate} 
\end{corollary}



\subsection{When Baseline SHAP is Hard} \label{subsec:baseline}
% Paragraph 1: Explain Baseline SHAP
In this subsection, we turn our focus to Baseline SHAP, which can be seen as a special case of Interventional SHAP when confined to empirical distributions induced by a reference input $\x^{\text{reff}}$. Hence, it is expected to be computationally simpler to compute than Interventional SHAP. However, we identify specific scenarios where calculating Baseline SHAP remains computationally challenging:
%This section aims to examine these scenarios and analyze the associated complexity hurdles. Specifically, we prove that:
\begin{theorem} \label{thm:intractable} \begin{inparaenum}[(i)] \item Unless P=co-NP, the problem \texttt{\emph{LOC-B-SHAP}}\texttt{\emph{(NN-SIGMOID)}} can not be solved in polynomial time; \item The decision versions of the problems $\emph{\texttt{LOC-B-SHAP}\texttt{(RNN-ReLu)}}$ and \emph{\texttt{LOC-B-SHAP}$\texttt{(ENS-DT}_{\texttt{C}}\texttt{)}$} %\footnote{Random Forests in the context of this theorem is the binary classifier that classifies an instance according to the majority vote mechanism.}   
are co-NP-Hard and NP-Hard respectively.
    \end{inparaenum}
\end{theorem}


% Introduce what follows in this section
\begin{comment}
In the sequel, we shall provide the proof sketch of the reduction made to prove  the Hardness of the problems \texttt{LOC-B-SHAP}(\texttt{SIGMOID}). Due to space constraints and the similarities between the reduction strategies employed for \texttt{LOC-B-SHAP}(\texttt{SIGMOID}) and \texttt{LOC-B-SHAP}(\texttt{RNN-ReLu}), we won't discuss this latter in the main article.
\end{comment}

We begin with results for \texttt{LOC-B-SHAP}(\texttt{NN-SIGMOID}) and \texttt{LOC-B-SHAP}(\texttt{RNN-ReLu}). Our proofs reduce from the Dummy Player problem of Weighted Majority Games~\citep{freixas2011complexity}:  %On the other hand, the reduction performed to show the NP-Hardness of the problem \texttt{B-SHAP}(\texttt{RF}) is made by reduction from the classical SAT problem.

%\subsubsection{On the Hardness of \texttt{LOC-B-SHAP}(\texttt{SIGMOID}) and  \texttt{LOC-B-SHAP}(\texttt{RNN-ReLu}).}
%As mentioned earlier, the reduction strategy to prove results regarding the problems \texttt{LOC-B-SHAP}(\texttt{SIGMOID}) and  \texttt{LOC-B-SHAP}(\texttt{RNN-ReLu}) in Theorem \ref{thm:intractable} is based on the dummy player problem of WMGs. Next, we provide a technical background of WMGs and the dummy problem. Afterwards, we provide a proof sketch of the reduction. The 
%full details of the reduction can be found in section \ref{app:BSHAPSIGMOID} of the Appendix. 
 
%\textcolor{blue}{@Shahaf (Shortening Hint): Is there any way to shorten the following background of WMGs?}
%\paragraph{Weighted Majority Games.} A weighted coalitional game is formally defined by a pair $G = (N,v)$, where $N$ is an integer referring to the number of players in the game, and $v$ is the value function of the game that maps each subset of players, referred to as a coalition, to the value generated by the coalition if they accept to cooperate in the game. In weighted games, players have different voting powers. A coalition is winning if the total weight of its members exceeds a predetermined threshold:
\begin{definition} \label{def:wmg}
A Weighted Majority Game (WMG) $G$ is a coalitional game defined by the tuple $  \langle N,\{n_{i}\}_{i \in [N]}, q\rangle$, where: \begin{inparaenum}[(i)] \item $N$ is the number of players; \item $n_{i}$ is the voting power of player $i$, for $i \in [N]$; \item $q$ is the winning quota. \end{inparaenum} The value function $v_{G}(S)$ is 1 if $\sum_{i \in S} n_{i} \geq q$, and 0 otherwise.

    %A Weighted majority game (WMG) $G$ is a coalitional game parametrized by the tuple $  <N,\{n_{i}\}_{i \in [N]}, q>$, where:
    %\begin{inparaenum}[(i)]
    %    \item $N$: The number of players,
    %    \item For $i \in [N]$, $n_{i}$ is an integer corresponding to the voting power of player $i$, 
    %    \item $q$ is an integer corresponding to the winning quota in the game  
    %\end{inparaenum}
    %The value function associated to the WMG $G$ is denoted as $v_{G}(S)$, and equals $1$ if $\sum\limits_{i \in S} n_{i} \geq q$, and $0$, otherwise.
    %$$v_{G}(S) = \begin{cases}
    %    1 & \text{if}~~ \sum\limits_{i \in S} n_{i} \geq q \\
    %    0 & \text{otherwise}
    %\end{cases}$$
\end{definition}

% The dummy player: Definition and associated computational problem
%\paragraph{Dummy players in WMGs.}

A \emph{dummy player} in a WMG $G$ is one whose voting power adds no value to any coalition. Formally, $i$ is a dummy in $G$ if $\forall S \subseteq [N] \setminus \{i\}$, it holds that $v_{G}(S \cup \{i\}) = v_{G}(S)$. Determining if a player $i$ is a dummy in $G$ is co-NP-Complete \citep{freixas2011complexity}.

%A \emph{dummy player} in a WMG $G$ is a player whose voting power brings no value to any coalition of players in the game. Formally, $i$ is dummy for $G$ iff $\forall S \subseteq N \setminus \{i\}$ it holds that $v_{G}(S \cup \{i\}) = v_{G}(S)$. Given a WMG $G$ and a player $i$ in $G$, deciding if the player $i$ is dummy is known to be co-NP-Complete \cite{freixas2011complexity}.

%$$\forall S \subseteq N \setminus \{i\}:~~v_{G}(S \cup \{i\}) = v_{G}(S)$$


\begin{comment}
An interesting characterization of dummy players in WMGs highlighted in the following proposition. This characterization will prove useful in our reduction proof to our target problems that will be detailed:
\begin{proposition}\label{prop:characterization_dummyplayers}
      Let $G = (N,\{n_{i}\}_{i \in [N]}, q)$ be a WMG. A player $i \in [N]$ is dummy if and only if:
     \begin{equation} \label{eq:expression}
     \forall S \subseteq [N] \setminus \{i\}:~~ \sum\limits_{j \in S} n_{j}  \notin [q - n_{i}, q]
     \end{equation}
\end{proposition}
The proof of this proposition can be found in Appendix. 
\end{comment}

\textbf{Reducing the dummy problem in WMG to both of the problems \texttt{LOC-B-SHAP}(\texttt{NN-SIGMOID}) and \texttt{LOC-B-SHAP}(\texttt{RNN-ReLu}).} 
Informally, given a WMG $G := \langle N, \{n_{j}\}_{j \in [N]}, q\rangle$, the reduction constructs a model $f_{G}$ over the input set $\{0,1\}^{N}$ from the target model family to simulate $G$'s value function using chosen input instances $\x,\x^{\text{reff}} \in \{0,1\}^{n}$, i.e., $f_{G}(\x_{S}; \x_{\bar{S}}^{\text{reff}}) \approx v(S)$. The properties of this reduction for both model families are formally stated as follows:


%Informally, given a WMG $G = <N, \{n_{j}\}_{j \in [N]}, q>$, the reduction strategy consists at constructing a model $f_{G}$  over the input set $\{0,1\}^{N}$ belonging to the target family of models to explain that simulates the value function of $G$ by a suitable choice of the input instances $(\x,\x^{ref}) \in \{0,1\}^{n}$, i.e.:     
%$f_{G}(\x_{S}; \x_{\bar{S}}^{(ref)}) \approx v(S)$. The properties of the algorithmic constructions for the reduction of both these families of models is formally stated as follows:
\begin{proposition} \label{prop:reductionsigmoid} 

There are poly-time algorithms that: \begin{inparaenum}[(i)] \item Given a WMG $G$ and player $i$, return a sigmoidal neural network $f_{G}$ over $\{0,1\}^{N}$, $\x,\x^{\text{reff}} \in \{0,1\}^{N}$ and $\epsilon \in \mathbb{R}$ such that $i$ is not dummy iff $\phi_{b}(f_{G},i,\x,\x^{\text{reff}}) > \epsilon$; \item Given a WMG $G$ and player $i$, return an RNN-ReLU $f_{G}$ over $\{0,1\}^{N}$, $\x,\x^{\textbf{reff}} \in \{0,1\}^{N}$ such that $i$ is not dummy iff $\phi_{b}(f_{G},i,\x,\x^{\text{reff}}) > 0$. \end{inparaenum}









%There exist polynomial-time algorithms that:
%    \begin{inparaenum}[(i)]
%    \item Given a WMG $G$ and a player $i$ in $G$, returns a sigmoidal neural network $f_{G}$ over $\{0,1\}^{N}$, $(\x,\x^{ref}) \in \{0,1\}^{N}$ and $\epsilon \in \mathbb{R}$ such that player $i$ is not dummy iff $\phi_{b}(f_{G},i,\x,\x^{ref}) > \epsilon$.
        %\textcolor{blue}{@Shahaf: The technical obstacle I found in the proof to upgrade this complexity result to co-NP-Hardness is that the parameter $\epsilon$ decreases exponentially with $N$ in my construction. Imagine $\epsilon = 2^{-N}$, a polynomial algorithm for the decision problem of LOC-B-SHAP(SIGMOID) is allowed to run polynomially with respect to $\frac{1}{\epsilon} = 2^{N}$. This technical subtelty prevents from obtaining the co-NP-Hardness of LOC-B-SHAP(SIGMOID).}
%        \item Given a WMG $G$ and a player $i$ in $G$, returns an RNN-ReLu $f_{G}$ over $\{0,1\}^{N}$, $(\x,\x^{ref}) \in \{0,1\}^{N}$ such that player $i$ is not dummy iff $\phi_{b}(f_{G},i,x,x^{ref}) > 0$
%    \end{inparaenum}
\end{proposition}

The complexity of computing \texttt{LOC-B-SHAP}(\texttt{SIGMOID}) and \texttt{LOC-B-SHAP}(\texttt{RNN-ReLu}) from Theorem \ref{thm:intractable} are corollaries of Proposition \ref{prop:reductionsigmoid}.

%\textcolor{blue}{@Shahaf: I write down the proof of the results of Theorem \ref{thm:intractable} for your understanding. You can decide if you want to refine it and keep it in the main text or delegate it to the appendix ..}
%\begin{proof}{(Theorem \ref{thm:intractable})}
%1. \texttt{SIGMOID}: Suppose there exists an algorithm $\mathcal{A}$ that computes exactly the problem \texttt{LOC-B-SHAP}(\texttt{SIGMOID}). We prove that we can decide the dummy problem in polynomial time using $\mathcal{A}$. The algorithm for deciding the dummy problem runs as follows: First, Run the polynomial-time algorithm whose existence is shown in point 1 of proposition \ref{prop:reductionsigmoid}, yielding the output $f_{G}, x,x^{ref}, \epsilon$. Second, run $\mathcal{A}$ on $f_{G}, i , x, x^{ref}$ yielding the $\phi_{b}(f_{G}, i, x, x^{ref})$. Finally, test if $\phi_{b}(f_{G}, i , x, x^{ref})$ is greater than $\epsilon$. By proposition \ref{prop:reductionsigmoid}, this algorithm decides in polynomial time if the player $i$ is dummy. Since the dummy problem is co-NP-Hard, the algorithm $\mathcal{A}$ doesn't exist unless \text{P=co-NP}.

%2. \texttt{RNN-ReLu}: Let $<G,i>$ be an input instance of the dummy player $i$. This input instance is reduced to the input instance $<f_{G}, i , x, x^{ref}, 0>$ of the decision problem associated to \texttt{LOC-B-SHAP}(\texttt{RNN-ReLu}) (using the polynomial-time algorithm whose existence is stated in point 2 of proposition \ref{prop:reductionsigmoid}). By this proposition, the player $i$ is dummy if and only $\phi_{b}(f_{G},i , x, x^{ref}) > 0$.
%    \end{proof}

\textbf{The problem \texttt{LOC-B-SHAP}($\texttt{ENS-DT}_{\texttt{C}}$) is NP-Hard.} 
This segment is dedicated to proving the remaining point of Theorem \ref{thm:intractable} stating the NP-Hardness of \texttt{LOC-B-SHAP}($\texttt{ENS-DT}_{\texttt{C}}$). We prove this by reducing from the classical NP-Complete 3-SAT problem. The reduction strategy is illustrated in Algorithm \ref{alg:SAT2b-SHAP}.

%\textcolor{blue}{@Shahaf: The reduction strategy detailed below is entirely copied and pasted to subsection \ref{app:subsec:bshaprf} (section \ref{app:BSHAPSIGMOID} of the appendix). If you want to delagate anything in the appendix from this segment, there where it should go ..}
%\paragraph{Reduction strategy.} The reduction strategy is illustrated in Algorithm \ref{alg:SAT2b-SHAP} (\textcolor{blue}{This algorithm is added to the appendix. It takes too much space. You can settle for describing the details of the construction and refer the reader to the appendix for the full steps of the algorithm to save space.} . For a given input CNF formula $\Phi$ over $n$ boolean variables $X = \{X_{1}, \ldots, X_{n}\}$ and $m$ clauses, the constructed random forest is a model whose set of input features contains the set $X$, with an additional feature denoted $X_{n+1}$ added for the sake of the reduction. The resulting random forest comprises a collection of $2m - 1$ decision trees, which can be categorized into two distinct groups:
%\begin{itemize}
%    \item $\mathcal{T}_{\Phi}$: A set of $m$ decision trees, each corresponding to a distinct clause in the input CNF formula. For a given clause $C$ in $\Phi$, the associated decision tree is constructed to assign a label 1 to all variable assignments that satisfy the clause $C$  while also ensuring that $x_{n+1} = 1$. It is easy to verify that such a decision tree can be constructed in polynomial time relative to the size of the input CNF formula
%
 %   \item \( \mathcal{T}_{null} \): This set consists of \( m - 1 \) copies of a trivial null decision tree. A null decision tree assigns a label of 0 to all input instances. 
%\end{itemize}
\begin{algorithm}
\caption{\texttt{3-SAT} to \texttt{LOC-B-SHAP}($\texttt{ENS-DT}_{\texttt{C}}$)}
\label{alg:SAT2b-SHAP}
\begin{algorithmic}[1]
\REQUIRE A CNF Formula $\Psi$ of $m$ clauses %$X = \{X_{1}, \ldots, X_{n} \}$
\ENSURE An instance of \texttt{LOC-B-SHAP}($\texttt{ENS-DT}_{\texttt{C}}$)
\STATE $\x \leftarrow [1, \ldots , 1]$
\STATE $\x^{\text{reff}} \leftarrow [0, \ldots , 0]$
\STATE $i \leftarrow n+1$
\STATE $\mathcal{T} \leftarrow \emptyset$
\FOR{$j \in [1,m]$}
 \STATE Construct a DT $T_{j}$ that assigns $1$ to variable assignments satisfying the formula: $C_{j} \land \x_{n+1}$.
 \STATE $\mathcal{T} \leftarrow \mathcal{T} \cup \{T\}_{j}$
\ENDFOR
 \STATE Construct a null decision tree $T_{\text{null}}$ that assigns a label $0$ to all variable assignments
 \STATE Add $m-1$ copies of $T_{\text{null}}$ to $\mathcal{T}$
\RETURN $\langle\mathcal{T},i,\x,\x^{\text{reff}}\rangle$
\end{algorithmic}
\end{algorithm}


The next proposition establishes a property of $\texttt{ENS-DT}_{\texttt{C}}$ used in Lemma~\ref{lemma:sat2bshap} to derive our complexity results:

%The next proposition establishes a property of $\texttt{ENS-DT}_{\texttt{C}}$ from the construction, which will be used in Lemma~\ref{lemma:sat2bshap} to derive the main result of this subsection:
%The next proposition provides a property of $\texttt{ENS-DT}_{\texttt{C}}$ resulting from the construction. This property shall be leveraged in Lemma \ref{lemma:sat2bshap} to yield the main result of this section:
\begin{proposition} \label{prop:dnf2bshaprf}
    Let $\Psi$ be an arbitrary \emph{CNF} formula over $n$ boolean variables, and  $\mathcal{T}$ be the ensemble of decision trees outputted by Algorithm \ref{alg:SAT2b-SHAP} for the input $\Psi$. 
    We have that $f_{\mathcal{T}}(\x_{1}, \ldots, \x_{n}, \x_{n+1}) = 1$ if $\x_{n+1} = 1 \land \x \models \Psi
          $, and $f_{\mathcal{T}}(\x_{1}, \ldots, \x_{n}, \x_{n+1})=0$, otherwise.
\end{proposition}
    %$$f_{\mathcal{T}}(x_{1}, \ldots, x_{n}, x_{n+1}) = \begin{cases}
    %      1 & \text{if} ~~ x_{n+1} = 1 \land x \models \Psi \\
    %      0 & \text{otherwise}
    %\end{cases}$$   
%\end{proposition}
%\textcolor{blue}{@Shahaf: The proof is placed here only for your understanding. It is also placed in the appendix. You can omit it from the main article once you understand it.}
%\begin{proof}
%    Fix an arbitrary CNF formula over $n$ boolean variables, and $(x_{1}, \ldots, x_{n})$ be an arbitrary variable assignment. If $x_{n+1} = 0$, decision trees in the set $\mathcal{T}$ assigns the label $0$, by construction. Consequently, $f_{\mathcal{T}}(x) = 0$.

%    For now, we assume that $x_{n+1} = 1$, if $x \models \Phi$, then $x$ is assigned the label 1 for all decision trees in $\mathcal{T}$. Consequently, $f_{\mathcal{T}}(x) = 1$. On the other hand, if $x$ is not satisfied by $\Phi$, then there exists at least one decision tree in $\mathcal{T}$, say $T_{j}$, that assigns a label 0 to $x$. Consequently, $x$ is assigned a label $0$ by at least $m$ decision trees, i.e. for all decision trees in $\mathcal{T}_{null}$ plus $T_{j}$ 
%\end{proof}
%\label{prop:dnf2bshaprf}


Using the result of Proposition~\ref{prop:dnf2bshaprf}, the following lemma directly establishes the NP-hardness of the decision problem for \texttt{LOC-B-SHAP}($\texttt{ENS-DT}_{\texttt{C}}$):

%Leveraging the result of proposition \ref{prop:dnf2bshaprf}, the following Lemma yields immediately the result of NP-Hardness of the decision problem associated to \texttt{LOC-B-SHAP}($\texttt{ENS-DT}_{\texttt{C}}$):
\begin{lemma} \label{lemma:sat2bshap}
    Let $\Psi$ be an arbitrary \emph{CNF} formula of $n$ variables, and  $\langle\mathcal{T},n+1,\x,\x^{\text{reff}}\rangle$ be the output of Algorithm \ref{alg:SAT2b-SHAP} for the input $\Psi$. We have that 
    $\phi_{b}(f_{\mathcal{T}}, n+1, \x, \x^{\text{reff}}) > 0$ iff $\exists \x \in \{0,1\}^{n}: ~ \x \models \Psi$. 
\end{lemma}


\section{Generalized Complexity Relations of SHAP Variants} \label{sec:generalized}


%\subsubsection{The Hardness of \texttt{LOC-B-SHAP}(.) implies the Hardness of \texttt{GLO-B-SHAP}(.,.) and \texttt{LOC-I-SHAP}(.,.).} 
%\textcolor{blue}{@Shahaf:I am not fully satisfied with my writing of this subsection (how to make it shine in the overall narrative of the article). If you can write a better pitch to introduce the result of proposition \ref{prop:hardnessrelation}, that would be good.}

While the previous sections presented specific results on the complexity of generating various SHAP variants for different models and distributions, this section aims to establish \emph{general} relationships concerning the complexity of different SHAP variants. We will then leverage these insights to derive corollaries for other SHAP contexts not explicitly covered in the paper.

%The objective of this final segment of this section is to complement negative complexity results provided in Theorem \ref{thm:intractable} with other results based on the hardness relations between the computational problems associated to different variants of SHAP scores, thus complementing the complexity results highlighted in Table \ref{fig:summaryresults}. Indeed, the following proposition highlights the existence of configurations where computing the Global B-SHAP and Local I-SHAP are at least as hard as computing Local B-SHAP:
\begin{proposition} \label{prop:hardnessrelation}
  \begin{inparaenum}[(i)]
    Let $\mathcal{M}$ be a class of models and $\mathcal{P}$ a class of probability distributions such that $\texttt{\emph{EMP}} \preceq_{P}  \mathcal{P}$. %\footnote{The symbol $\preceq_{P}$ signifies "polynomially reducible to"}. Then, 
    Then, \texttt{\emph{LOC-B-SHAP}}($\mathcal{M}$)  $\preceq_{P}$ \texttt{\emph{GLO-B-SHAP}}($\mathcal{M}$,$\mathcal{P}$) and \texttt{\emph{LOC-B-SHAP}}($\mathcal{M}$)  $\preceq_{P}$   \texttt{\emph{LOC-I-SHAP}}($\mathcal{M},\mathcal{P}$).
    \end{inparaenum}
\end{proposition}


In other words, assume a class of probability distributions $\mathcal{P}$ is "harder" (under polynomial reductions) than the class of empirical distributions \texttt{EMP}, i.e., any $P \in \mathcal{P}$ can be reduced in poly-time to some $P' \in$ \texttt{EMP}. Then, global baseline SHAP is at least as hard to compute as local baseline SHAP and local interventional SHAP is at least as hard to compute as local baseline SHAP (both under polynomial reductions). Since $\texttt{EMP} \preceq_{P} \texttt{HMM}$ (proof in Appendix~\ref{app:reductiontree}), these corollaries follow from Theorem~\ref{thm:intractable} and proposition~\ref{prop:hardnessrelation}:


%In other words, let us assume that a class of probability distributions $\mathcal{P}$ is ``harder'' (under polynomial reductions) than the class of empirical distributions \texttt{EMP}, i.e., any probability distribution $P\in\mathcal{P}$ can be reduced in polynomial time to a probability distribution $P'\in$\texttt{EMP}. Then, it satisfies that global baseline SHAP is at least as hard as local baseline SHAP (under polynomial reductions), and local interventional SHAP is at least as hard as local baseline SHAP (under polynomial reductions). Since $\texttt{\text{EMP}} \preceq_{P} \texttt{\text{HMM}}$ (proof in the appendix), we derive these corollaries from Theorem \ref{thm:intractable} and proposition~\ref{prop:hardnessrelation}:

%In other words, for any distribution class of probability distributions that encompasses the class \texttt{EMP} it holds that local baseline SHAP is at least as hard as global baseline SHAP (under polynomial reductions) and that local interventional SHAP is at least as hard as local baseline SHAP (under polynomial reductions. Since $\texttt{\text{EMP}} \preceq_{P} \texttt{\text{HMM}}$, (See proof \ref{app:reductiontree} of the supplementary material), we obtain these corollaries from Theorem \ref{thm:intractable} and proposition~\ref{prop:hardnessrelation}:

%\ref{prop:hardnessrelation}:
\begin{corollary} \label{cor:globshaphard}
    \begin{inparaenum}[(i)]
    Let there be some $\texttt{\emph{P}} \in \{ \texttt{\emph{EMP}}, \texttt{\emph{HMM}}\}$, and $\overrightarrow{\texttt{\emph{P}}} \in \{\texttt{\emph{EMP}}, \overrightarrow{\texttt{\emph{HMM}}} \}$. Then it holds that:
        \item%\texttt{\emph{SIGMOID}}
        Unless P=co-NP, the problems \texttt{\emph{GLO-B-SHAP}}\texttt{\emph{(SIGMOID}},$\overrightarrow{\texttt{\emph{P}}}$\texttt{\emph{)}} and \texttt{\emph{LOC-I-SHAP}}\texttt{\emph{(NN-SIGMOID}},$\overrightarrow{\texttt{\emph{P}}}$\texttt{\emph{)}} can not be computed exactly in polynomial time;
        \item %\texttt{\emph{RNN-ReLu}}:
        The decision version of the problems $\texttt{\emph{GLO-B-SHAP}}(\texttt{\emph{RNN-ReLu}}, \texttt{\emph{P}})$ and $\texttt{\emph{LOC-I-SHAP}}(\texttt{\emph{RNN-ReLu}}, \texttt{\emph{P}})$ are co-NP-Hard;
        \item %\texttt{\emph{RF}}:
        The decision version of the problems $\texttt{\emph{GLO-B-SHAP}\texttt{\emph{(M}},\texttt{\emph{P}}})$ and $\texttt{\emph{LOC-I-SHAP}}\texttt{\emph{(ENS-DT}}_{\texttt{\emph{C}}},\texttt{\emph{P)}}$ are NP-Hard. 
    \end{inparaenum} 
\end{corollary}



%\input{content/intractableresults/corollaryresults}

\subsection{Bills of Materials}

A BOM is an industry-standard document produced by a manufacturer listing the components that make up a product. This document often contains information such as the name of the component, the supplier of the component and the material name, when available and is often required to be produced for regulatory reasons. An example is shown in Figure \ref{bom}. Although knowledge of the main material of a component is often enough to determine a suitable match to a process in the database, the material name provided by the supplier is often an internal name for the material, a specification code, or other non-straightforward description of the material.  Due to the complexity of correctly mapping component materials to the process used to produce them, specialist knowledge is usually required.

\begin{figure*}[h!]

\begin{custommdframed}[userdefinedwidth=38em]

\begin{tabular}{l l l}

\textbf{Component Name} & \textbf{Material} & \textbf{Supplier} \\
\hline
SPIRALGEHÄUSE & EN-GJL-250/A48 CL 35B & Mechatronik GmbH \\
WELLE & C45+N & Technikbau AG \\
SPALTRING               & JL/GUSSEISEN LAMELLENGRAFIT & GussForm Solutions   \\
SPANNRING               & STAHL+KATAPHORESE           & StahlPro Engineering \\ 
SPALTRING               & JL/GUSSEISEN LAMELLENGRAFIT & GussTech Industries  \\ 
STIFTSCHRAUBE           & 8.8                        & FixFast Components   \\ 
STIFTSCHRAUBE           & 8.8                        & SchraubenWerk AG     \\ 
STIFTSCHRAUBE           & 5.8+A2A                    & PrecisionParts GmbH  \\ 

\end{tabular}

\end{custommdframed}
\rmfamily
\caption{An excerpt from  a BOM. Note that material codes are often ambiguous or obscure, requiring specialized knowledge to correctly identify.}
\label{bom}
\end{figure*}


\subsection{Component Datasheets}

For some components, a technical datasheet is produced by the manufacturer. Typically, this lists properties of the component or the materials that make up the component. This information can provide useful context that helps to identify the process name for the component. Datasheets are not usually available for all components; they are only supplied by the manufacturer in some instances. We investigate including the information from datasheets in the entity mapping process. 


\section{Methodology}

Our key contributions are:
\begin{enumerate}
    \item Utilize fine-grained information from BOMs to provide a more accurate assessment of carbon emissions
    \item Introduce LLMs into the entity mapping process in order to provide additional context, and
    \item Integrate additional context from component datasheets in order to further improve context.
\end{enumerate}
We break the process of mapping components to database entries into three connected steps, outlined below. An illustration of the complete pipeline is shown in Figure \ref{pipeline}.

\begin{figure*}[h]
\centerline{\includegraphics[width=\textwidth]{figures/greenaihub.pdf}}
\caption{Architecture of the proposed pipeline, made up of three modules: Document retrieval, LLM querying, and database ranking.}
\label{pipeline}
\end{figure*}


\subsection{Datasheet Selection}

Given a pool of datasheets, we must select the document from the pool corresponding to the BOM entry of interest to provide additional context about the component. In order to determine the matching document, we evaluate the textual similarity between the concatenation of the filename and text of the datasheet, and the concatenation of the component name, manufacturer and material from the BOM entry. Similarity is measured by the cosine similarity of the two embeddings generated by a text embedding model. After manual evaluation, we chose a threshold value of 0.5 or higher for the cosine similarity to indicate a match. If a match is found, the text from the datasheet is included in further processing in order to match the BOM entry to the process name.

\subsection{LLM Querying}

We utilize a LLM agent fine-tuned for chat as the LLM in our pipeline. We create a prompt for the model that includes all relevant context and instructs the model to produce a description of the manufacturing process used to create the component. Context includes the BOM entry information (component name, supplier, and material) along with the content of the datasheet, if available. The exact prompt can be found in our publicly available code\footnote{https://github.com/DFKI-NLP/eco-link}. The output of the model is then used in further processing. A sample showing typical responses from the LLM is shown in Figure \ref{response}. We use a local instance of Llama 3.1 with 8 billion parameters \cite{dubey2024llama} as the LLM in our experiments.

\documentclass[11pt]{article}

\def\papertitle{Extracting Privacy-Preserving Subgraphs in Federated Graph Learning using Information Bottleneck}
\def\authors{Anonymous Authors}
\def\journal{ACM ASIACCS 2023 }
\def\manuscriptid{\# 69}

\documentclass[11pt]{article}

\def\papertitle{Extracting Privacy-Preserving Subgraphs in Federated Graph Learning using Information Bottleneck}
\def\authors{Anonymous Authors}
\def\journal{ACM ASIACCS 2023 }
\def\manuscriptid{\# 69}

\documentclass[11pt]{article}

\def\papertitle{Extracting Privacy-Preserving Subgraphs in Federated Graph Learning using Information Bottleneck}
\def\authors{Anonymous Authors}
\def\journal{ACM ASIACCS 2023 }
\def\manuscriptid{\# 69}

\input{../response.tex}


\begin{document}

\preamble

\vspace{6em}
Dear Shepherd, PC-Chairs, and Reviewers,
\\[4em]
We sincerely express our gratitude for handling the review of our submitted manuscript. We worked diligently to handle the revision requirements. 

Below we provide our detailed response to the requirements. We have highlighted the main changes in the revised manuscript (see $\mathrm{AsiaCCS\_\# 69\_revision.pdf}$) by coloring the modified text in {\color{blue} blue}. We hope that the applied revisions are to the satisfaction of the conference. We look forward to hearing from you in due course.
\\[4em]
Best regards,\\
\authors
\\[1em]







\pagebreak




\end{document}



\begin{document}

\preamble

\vspace{6em}
Dear Shepherd, PC-Chairs, and Reviewers,
\\[4em]
We sincerely express our gratitude for handling the review of our submitted manuscript. We worked diligently to handle the revision requirements. 

Below we provide our detailed response to the requirements. We have highlighted the main changes in the revised manuscript (see $\mathrm{AsiaCCS\_\# 69\_revision.pdf}$) by coloring the modified text in {\color{blue} blue}. We hope that the applied revisions are to the satisfaction of the conference. We look forward to hearing from you in due course.
\\[4em]
Best regards,\\
\authors
\\[1em]







\pagebreak




\end{document}



\begin{document}

\preamble

\vspace{6em}
Dear Shepherd, PC-Chairs, and Reviewers,
\\[4em]
We sincerely express our gratitude for handling the review of our submitted manuscript. We worked diligently to handle the revision requirements. 

Below we provide our detailed response to the requirements. We have highlighted the main changes in the revised manuscript (see $\mathrm{AsiaCCS\_\# 69\_revision.pdf}$) by coloring the modified text in {\color{blue} blue}. We hope that the applied revisions are to the satisfaction of the conference. We look forward to hearing from you in due course.
\\[4em]
Best regards,\\
\authors
\\[1em]







\pagebreak




\end{document}


\subsection{Semantic Similarity Matching}

As a preprocessing step, we create embeddings for each database entry using the process name and description. We store these embeddings in a FAISS \cite{douze2024faiss} vector store. To find a match for a BOM entry, we create an embedding of the LLM response from the previous step and compare this embedding to the vector store. Using the cosine similarity as a distance measure, we are able to obtain a ranking of database entries. For both the datasheet selection and semantic similarity ranking, we use \texttt{gte-large-en-v1.5} \cite{li2023towards}.


\section{Results and Discussion}

\begin{table}[ht!]
\centering
\caption{\textbf{Super Resolution Performance Results.} Our proposed WGAN EEG Spatial Upsampling method significantly outperforms a baseline of Bicubic Interpolation commonly used in EEG upsampling pipelines.}
\label{tab:results}
\resizebox{0.8\linewidth}{!}{%
\begin{tabular}{@{}cccccc@{}}
\toprule
\multirow{2}{*}{\textbf{Dataset}} & \multirow{2}{*}{\textbf{Scale}} & \multicolumn{2}{c}{\textbf{Bicubic}} & \multicolumn{2}{c}{\textbf{WGAN}} \\ \cmidrule(l){3-6} 
                      &   & \textbf{MSE} & \textbf{MAE} & \textbf{MSE}    & \textbf{MAE}   \\
\toprule
\multirow{2}{*}{Val}  & 2 & 3.71E7       & 3.89E3       & \textbf{2.01E3} & \textbf{24.38} \\
                      & 4 & 7.23E7       & 6.42E3       & \textbf{8.53E3} & \textbf{63.83} \\
\midrule
\multirow{2}{*}{Test} & 2 & 3.75E7       & 3.91E3       & \textbf{2.06E3} & \textbf{24.66} \\
                      & 4 & 7.30E7       & 6.45E3       & \textbf{8.68E3} & \textbf{64.39} \\
\bottomrule
\end{tabular}%
}
\end{table}


Because of limitations in data availability due to lack of large-scale public datasets, data privacy, and preservation of trade secrecy, we can only evaluate on a small set of BOM entries. Our set of labeled evaluation data consists of only 21 components from 3 different BOMs. 

We evaluate by comparing human performance to both the full and ablated pipeline. \textbf{Semantic Similarity} uses only the Database Ranking module and semantic similarity between the database entries and the component name, supplier and material. \textbf{LLM} uses both the LLM and Database Ranking modules to match the LLM's response , and \textbf{LLM + Datasheet} includes the Document Retrieval component. We use $Hits@n$ as the metric, which is defined for a recommender system as the proportion of instances the correct item is present in the top $n$ recommendations. This metric corresponds to our use case, where a shortlist of recommendations is provided by the non-expert recommender. The results are shown in Figure \ref{results}. 

Our approach was found to be acceptable given the challenging nature of the task --- on par or slightly better than non-expert human performance. While not enough to completely automate the entire entity mapping process, these results indicate that our method could take the place of the non-expert human in this process.

\section{Conclusion and Future Work}

This paper presents a novel approach to estimate product carbon footprints using LLMs to map BOM components to LCA database entries. The proposed pipeline streamlines the traditionally manual process, achieving reasonable accuracy and scalability. Future work includes an expanded evaluation with a larger evaluation dataset and integration of additional context from sources such as web search results.
\section*{Acknowledgments}

This paper has received funding from the European Union under the grant agreement no. 101058573 (SciLake) and by the German  Bundesministerium für Umwelt, Naturschutz, nukleare Sicherheit und Verbraucherschutz (BMUV) under the Green AI Hub Mittelstand initiative.

\bibliographystyle{acl_natbib}
\bibliography{nodalida2025}

\end{document}
