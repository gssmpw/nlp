
\documentclass{article}

\usepackage{microtype}
\usepackage{graphicx}
\usepackage{subfigure}
\usepackage{booktabs} %

\usepackage{hyperref}


\newcommand{\theHalgorithm}{\arabic{algorithm}}

\usepackage[accepted]{icml2025}

\usepackage{amsmath}
\usepackage{amssymb}
\usepackage{mathtools}
\usepackage{amsthm}
\usepackage{thmtools}
\usepackage{thm-restate}
\usepackage{url}


\DeclareMathOperator*{\argmax}{\arg\!\max}
\DeclareMathOperator*{\argmin}{\arg\!\min}
\DeclareMathOperator*{\arginf}{\arg\!\inf}
\DeclareMathOperator*{\argsup}{\arg\!\sup}

\usepackage[capitalize,noabbrev]{cleveref}

\theoremstyle{plain}
\newtheorem{theorem}{Theorem}[section]
\newtheorem{proposition}[theorem]{Proposition}
\newtheorem{lemma}[theorem]{Lemma}
\newtheorem{corollary}[theorem]{Corollary}
\theoremstyle{definition}
\newtheorem{definition}[theorem]{Definition}
\newtheorem{assumption}[theorem]{Assumption}
\theoremstyle{remark}
\newtheorem{remark}[theorem]{Remark}

\usepackage[textsize=tiny]{todonotes}

\usepackage{bm}
\usepackage{colortbl}
\usepackage{multicol}
\usepackage{multirow}
\usepackage{algorithm}
\usepackage{algorithmic}  

\usepackage{dsfont}


\icmltitlerunning{Learning Relational Tabular Data without Shared Features}

\begin{document}

\twocolumn[
\icmltitle{Learning Relational Tabular Data without Shared Features}



\icmlsetsymbol{equal}{*}

\begin{icmlauthorlist}
	\icmlauthor{Zhaomin Wu}{ids}
	\icmlauthor{Shida Wang}{math}
	\icmlauthor{Ziyang Wang}{soc}
	\icmlauthor{Bingsheng He}{soc}
\end{icmlauthorlist}

\icmlaffiliation{soc}{Department of Computer Science, National University of Singapore, Singapore}
\icmlaffiliation{ids}{Institute of Data Science, National University of Singapore, Singapore}
\icmlaffiliation{math}{Department of Mathematics, National University of Singapore, Singapore}

\icmlcorrespondingauthor{Zhaomin Wu}{zhaomin@nus.edu.sg}

\icmlkeywords{Entity Alignment, Tabular Data, Transformer, Record Linkage, Entity Resolution}

\vskip 0.3in]




\printAffiliationsAndNotice{}  %

\begin{abstract}
    Learning relational tabular data has gained significant attention recently, but most studies focus on single tables, overlooking the potential of cross-table learning. Cross-table learning, especially in scenarios where tables lack shared features and pre-aligned data, offers vast opportunities but also introduces substantial challenges. The alignment space is immense, and determining accurate alignments between tables is highly complex. We propose Latent Entity Alignment Learning (\textit{Leal}), a novel framework enabling effective cross-table training without requiring shared features or pre-aligned data. Leal operates on the principle that properly aligned data yield lower loss than misaligned data, a concept embodied in its soft alignment mechanism. This mechanism is coupled with a differentiable cluster sampler module, ensuring efficient scaling to large relational tables. Furthermore, we provide a theoretical proof of the cluster sampler's approximation capacity. Extensive experiments on five real-world and five synthetic datasets show that Leal achieves up to a 26.8\% improvement in predictive performance compared to state-of-the-art methods, demonstrating its effectiveness and scalability.
\end{abstract}

\section{Introduction}\label{sec:introduction}
\section{Introduction}
\label{sec:introduction}
The business processes of organizations are experiencing ever-increasing complexity due to the large amount of data, high number of users, and high-tech devices involved \cite{martin2021pmopportunitieschallenges, beerepoot2023biggestbpmproblems}. This complexity may cause business processes to deviate from normal control flow due to unforeseen and disruptive anomalies \cite{adams2023proceddsriftdetection}. These control-flow anomalies manifest as unknown, skipped, and wrongly-ordered activities in the traces of event logs monitored from the execution of business processes \cite{ko2023adsystematicreview}. For the sake of clarity, let us consider an illustrative example of such anomalies. Figure \ref{FP_ANOMALIES} shows a so-called event log footprint, which captures the control flow relations of four activities of a hypothetical event log. In particular, this footprint captures the control-flow relations between activities \texttt{a}, \texttt{b}, \texttt{c} and \texttt{d}. These are the causal ($\rightarrow$) relation, concurrent ($\parallel$) relation, and other ($\#$) relations such as exclusivity or non-local dependency \cite{aalst2022pmhandbook}. In addition, on the right are six traces, of which five exhibit skipped, wrongly-ordered and unknown control-flow anomalies. For example, $\langle$\texttt{a b d}$\rangle$ has a skipped activity, which is \texttt{c}. Because of this skipped activity, the control-flow relation \texttt{b}$\,\#\,$\texttt{d} is violated, since \texttt{d} directly follows \texttt{b} in the anomalous trace.
\begin{figure}[!t]
\centering
\includegraphics[width=0.9\columnwidth]{images/FP_ANOMALIES.png}
\caption{An example event log footprint with six traces, of which five exhibit control-flow anomalies.}
\label{FP_ANOMALIES}
\end{figure}

\subsection{Control-flow anomaly detection}
Control-flow anomaly detection techniques aim to characterize the normal control flow from event logs and verify whether these deviations occur in new event logs \cite{ko2023adsystematicreview}. To develop control-flow anomaly detection techniques, \revision{process mining} has seen widespread adoption owing to process discovery and \revision{conformance checking}. On the one hand, process discovery is a set of algorithms that encode control-flow relations as a set of model elements and constraints according to a given modeling formalism \cite{aalst2022pmhandbook}; hereafter, we refer to the Petri net, a widespread modeling formalism. On the other hand, \revision{conformance checking} is an explainable set of algorithms that allows linking any deviations with the reference Petri net and providing the fitness measure, namely a measure of how much the Petri net fits the new event log \cite{aalst2022pmhandbook}. Many control-flow anomaly detection techniques based on \revision{conformance checking} (hereafter, \revision{conformance checking}-based techniques) use the fitness measure to determine whether an event log is anomalous \cite{bezerra2009pmad, bezerra2013adlogspais, myers2018icsadpm, pecchia2020applicationfailuresanalysispm}. 

The scientific literature also includes many \revision{conformance checking}-independent techniques for control-flow anomaly detection that combine specific types of trace encodings with machine/deep learning \cite{ko2023adsystematicreview, tavares2023pmtraceencoding}. Whereas these techniques are very effective, their explainability is challenging due to both the type of trace encoding employed and the machine/deep learning model used \cite{rawal2022trustworthyaiadvances,li2023explainablead}. Hence, in the following, we focus on the shortcomings of \revision{conformance checking}-based techniques to investigate whether it is possible to support the development of competitive control-flow anomaly detection techniques while maintaining the explainable nature of \revision{conformance checking}.
\begin{figure}[!t]
\centering
\includegraphics[width=\columnwidth]{images/HIGH_LEVEL_VIEW.png}
\caption{A high-level view of the proposed framework for combining \revision{process mining}-based feature extraction with dimensionality reduction for control-flow anomaly detection.}
\label{HIGH_LEVEL_VIEW}
\end{figure}

\subsection{Shortcomings of \revision{conformance checking}-based techniques}
Unfortunately, the detection effectiveness of \revision{conformance checking}-based techniques is affected by noisy data and low-quality Petri nets, which may be due to human errors in the modeling process or representational bias of process discovery algorithms \cite{bezerra2013adlogspais, pecchia2020applicationfailuresanalysispm, aalst2016pm}. Specifically, on the one hand, noisy data may introduce infrequent and deceptive control-flow relations that may result in inconsistent fitness measures, whereas, on the other hand, checking event logs against a low-quality Petri net could lead to an unreliable distribution of fitness measures. Nonetheless, such Petri nets can still be used as references to obtain insightful information for \revision{process mining}-based feature extraction, supporting the development of competitive and explainable \revision{conformance checking}-based techniques for control-flow anomaly detection despite the problems above. For example, a few works outline that token-based \revision{conformance checking} can be used for \revision{process mining}-based feature extraction to build tabular data and develop effective \revision{conformance checking}-based techniques for control-flow anomaly detection \cite{singh2022lapmsh, debenedictis2023dtadiiot}. However, to the best of our knowledge, the scientific literature lacks a structured proposal for \revision{process mining}-based feature extraction using the state-of-the-art \revision{conformance checking} variant, namely alignment-based \revision{conformance checking}.

\subsection{Contributions}
We propose a novel \revision{process mining}-based feature extraction approach with alignment-based \revision{conformance checking}. This variant aligns the deviating control flow with a reference Petri net; the resulting alignment can be inspected to extract additional statistics such as the number of times a given activity caused mismatches \cite{aalst2022pmhandbook}. We integrate this approach into a flexible and explainable framework for developing techniques for control-flow anomaly detection. The framework combines \revision{process mining}-based feature extraction and dimensionality reduction to handle high-dimensional feature sets, achieve detection effectiveness, and support explainability. Notably, in addition to our proposed \revision{process mining}-based feature extraction approach, the framework allows employing other approaches, enabling a fair comparison of multiple \revision{conformance checking}-based and \revision{conformance checking}-independent techniques for control-flow anomaly detection. Figure \ref{HIGH_LEVEL_VIEW} shows a high-level view of the framework. Business processes are monitored, and event logs obtained from the database of information systems. Subsequently, \revision{process mining}-based feature extraction is applied to these event logs and tabular data input to dimensionality reduction to identify control-flow anomalies. We apply several \revision{conformance checking}-based and \revision{conformance checking}-independent framework techniques to publicly available datasets, simulated data of a case study from railways, and real-world data of a case study from healthcare. We show that the framework techniques implementing our approach outperform the baseline \revision{conformance checking}-based techniques while maintaining the explainable nature of \revision{conformance checking}.

In summary, the contributions of this paper are as follows.
\begin{itemize}
    \item{
        A novel \revision{process mining}-based feature extraction approach to support the development of competitive and explainable \revision{conformance checking}-based techniques for control-flow anomaly detection.
    }
    \item{
        A flexible and explainable framework for developing techniques for control-flow anomaly detection using \revision{process mining}-based feature extraction and dimensionality reduction.
    }
    \item{
        Application to synthetic and real-world datasets of several \revision{conformance checking}-based and \revision{conformance checking}-independent framework techniques, evaluating their detection effectiveness and explainability.
    }
\end{itemize}

The rest of the paper is organized as follows.
\begin{itemize}
    \item Section \ref{sec:related_work} reviews the existing techniques for control-flow anomaly detection, categorizing them into \revision{conformance checking}-based and \revision{conformance checking}-independent techniques.
    \item Section \ref{sec:abccfe} provides the preliminaries of \revision{process mining} to establish the notation used throughout the paper, and delves into the details of the proposed \revision{process mining}-based feature extraction approach with alignment-based \revision{conformance checking}.
    \item Section \ref{sec:framework} describes the framework for developing \revision{conformance checking}-based and \revision{conformance checking}-independent techniques for control-flow anomaly detection that combine \revision{process mining}-based feature extraction and dimensionality reduction.
    \item Section \ref{sec:evaluation} presents the experiments conducted with multiple framework and baseline techniques using data from publicly available datasets and case studies.
    \item Section \ref{sec:conclusions} draws the conclusions and presents future work.
\end{itemize}

\section{Related Work}\label{sec:related_work}

%\section{Related Work}
%\label{sec:related-work}

%\subsection{Background}

%Defect detection is critical to ensure the yield of integrated circuit manufacturing lines and reduce faults. Previous research has primarily focused on wafer map data, which engineers produce by marking faulty chips with different colors based on test results. The specific spatial distribution of defects on a wafer can provide insights into the causes, thereby helping to determine which stage of the manufacturing process is responsible for the issues. Although such research is relatively mature, the continual miniaturization of integrated circuits and the increasing complexity and density of chip components have made chip-level detection more challenging, leading to potential risks\cite{ma2023review}. Consequently, there is a need to combine this approach with magnified imaging of the wafer surface using scanning electron microscopes (SEMs) to detect, classify, and analyze specific microscopic defects, thus helping to identify the particular process steps where defects originate.

%Previously, wafer surface defect classification and detection were primarily conducted by experienced engineers. However, this method relies heavily on the engineers' expertise and involves significant time expenditure and subjectivity, lacking uniform standards. With the ongoing development of artificial intelligence, deep learning methods using multi-layer neural networks to extract and learn target features have proven highly effective for this task\cite{gao2022review}.

%In the task of defect classification, it is typical to use a model structure that initially extracts features through convolutional and pooling layers, followed by classification via fully connected layers. Researchers have recently developed numerous classification model structures tailored to specific problems. These models primarily focus on how to extract defect features effectively. For instance, Chen et al. presented a defect recognition and classification algorithm rooted in PCA and classification SVM\cite{chen2008defect}. Chang et al. utilized SVM, drawing on features like smoothness and texture intricacy, for classifying high-intensity defect images\cite{chang2013hybrid}. The classification of defect images requires the formulation of numerous classifiers tailored for myriad inspection steps and an Abundance of accurately labeled data, making data acquisition challenging. Cheon et al. proposed a single CNN model adept at feature extraction\cite{cheon2019convolutional}. They achieved a granular classification of wafer surface defects by recognizing misclassified images and employing a k-nearest neighbors (k-NN) classifier algorithm to gauge the aggregate squared distance between each image feature vector and its k-neighbors within the same category. However, when applied to new or unseen defects, such models necessitate retraining, incurring computational overheads. Moreover, with escalating CNN complexity, the computational demands surge.

%Segmentation of defects is necessary to locate defect positions and gather information such as the size of defects. Unlike classification networks, segmentation networks often use classic encoder-decoder structures such as UNet\cite{ronneberger2015u} and SegNet\cite{badrinarayanan2017segnet}, which focus on effectively leveraging both local and global feature information. Han Hui et al. proposed integrating a Region Proposal Network (RPN) with a UNet architecture to suggest defect areas before conducting defect segmentation \cite{han2020polycrystalline}. This approach enables the segmentation of various defects in wafers with only a limited set of roughly labeled images, enhancing the efficiency of training and application in environments where detailed annotations are scarce. Subhrajit Nag et al. introduced a new network structure, WaferSegClassNet, which extracts multi-scale local features in the encoder and performs classification and segmentation tasks in the decoder \cite{nag2022wafersegclassnet}. This model represents the first detection system capable of simultaneously classifying and segmenting surface defects on wafers. However, it relies on extensive data training and annotation for high accuracy and reliability. 

%Recently, Vic De Ridder et al. introduced a novel approach for defect segmentation using diffusion models\cite{de2023semi}. This approach treats the instance segmentation task as a denoising process from noise to a filter, utilizing diffusion models to predict and reconstruct instance masks for semiconductor defects. This method achieves high precision and improved defect classification and segmentation detection performance. However, the complex network structure and the computational process of the diffusion model require substantial computational resources. Moreover, the performance of this model heavily relies on high-quality and large amounts of training data. These issues make it less suitable for industrial applications. Additionally, the model has only been applied to detecting and segmenting a single type of defect(bridges) following a specific manufacturing process step, limiting its practical utility in diverse industrial scenarios.

%\subsection{Few-shot Anomaly Detection}
%Traditional anomaly detection techniques typically rely on extensive training data to train models for identifying and locating anomalies. However, these methods often face limitations in rapidly changing production environments and diverse anomaly types. Recent research has started exploring effective anomaly detection using few or zero samples to address these challenges.

%Huang et al. developed the anomaly detection method RegAD, based on image registration technology. This method pre-trains an object-agnostic registration network with various images to establish the normality of unseen objects. It identifies anomalies by aligning image features and has achieved promising results. Despite these advancements, implementing few-shot settings in anomaly detection remains an area ripe for further exploration. Recent studies show that pre-trained vision-language models such as CLIP and MiniGPT can significantly enhance performance in anomaly detection tasks.

%Dong et al. introduced the MaskCLIP framework, which employs masked self-distillation to enhance contrastive language-image pretraining\cite{zhou2022maskclip}. This approach strengthens the visual encoder's learning of local image patches and uses indirect language supervision to enhance semantic understanding. It significantly improves transferability and pretraining outcomes across various visual tasks, although it requires substantial computational resources.
%Jeong et al. crafted the WinCLIP framework by integrating state words and prompt templates to characterize normal and anomalous states more accurately\cite{Jeong_2023_CVPR}. This framework introduces a novel window-based technique for extracting and aggregating multi-scale spatial features, significantly boosting the anomaly detection performance of the pre-trained CLIP model.
%Subsequently, Li et al. have further contributed to the field by creating a new expansive multimodal model named Myriad\cite{li2023myriad}. This model, which incorporates a pre-trained Industrial Anomaly Detection (IAD) model to act as a vision expert, embeds anomaly images as tokens interpretable by the language model, thus providing both detailed descriptions and accurate anomaly detection capabilities.
%Recently, Chen et al. introduced CLIP-AD\cite{chen2023clip}, and Li et al. proposed PromptAD\cite{li2024promptad}, both employing language-guided, tiered dual-path model structures and feature manipulation strategies. These approaches effectively address issues encountered when directly calculating anomaly maps using the CLIP model, such as reversed predictions and highlighting irrelevant areas. Specifically, CLIP-AD optimizes the utilization of multi-layer features, corrects feature misalignment, and enhances model performance through additional linear layer fine-tuning. PromptAD connects normal prompts with anomaly suffixes to form anomaly prompts, enabling contrastive learning in a single-class setting.

%These studies extend the boundaries of traditional anomaly detection techniques and demonstrate how to effectively address rapidly changing and sample-scarce production environments through the synergy of few-shot learning and deep learning models. Building on this foundation, our research further explores wafer surface defect detection based on the CLIP model, especially focusing on achieving efficient and accurate anomaly detection in the highly specialized and variable semiconductor manufacturing process using a minimal amount of labeled data.


\section{Problem Formulation}\label{sec:problem}
\section{Problem Formulation} \label{sec:probdef}

This section formally defines the problem of restoring a given pruned network with only using its original pretrained CNN in a way free of data and fine-tuning.



% Unlike many existing works utilize data for identifying unimportant filters as well as fine-tuning to this end, we cannot evaluate the filter importance by data-dependent values like activation maps (\textit{a.k.a.} channels) as our focus in this paper is not to use any training data. Thus, in our problem setting, we can only exploit the values of filters in the original network, and thereby have to make some changes in the remaining filters of the pruned network so that the network can return the output not too much different from the original one.

% No matter how much we carefully select unimportant filters to be pruned, some kinds of retraining process appears inevitable as done by the most existing works to this end. However, since our focus in this paper is not to use any training data, we cannot evaluate the importance of filters by data-dependent values like activation maps (\textit{a.k.a.} channels). 

% To this end, they not only use a careful criterion (\textit{e.g.}, L1-norm), but also fine-tune the network using the original data.
% Most of filter pruning methods try to select filters to be pruned prudently so that pruned network's output be similar to the original network's. To this end, they prune the unimportant filters and then fine-tune the pruned network with using the train data. 

% How can we restore the the pruned networks without any data? In other words, it implies that we cannot use any data-driven values(i.e., activation maps) and we can only exploit the values of original filters. In that case, the only thing we can do maybe changing the weights of remained filters appropriately not to amplify the difference between pruned and unpruned network's outputs through the information of original filters.

\begin{figure*}[t]
	\centering
    \subfigure[\label{fig:matrix:a}Pruning matrix]{\hspace{6mm}\includegraphics[width=0.35\columnwidth]{./figure/LBYL_figure_2_1.pdf}\hspace{6mm}} 
    \subfigure[\label{fig:matrix:b}Delivery matrix for LBYL]{\hspace{6mm}\includegraphics[width=0.35\columnwidth]{./figure/LBYL_figure_2_2.pdf}\hspace{6mm}}
    \subfigure[\label{fig:matrix:c}Delivery matrix for one-to-one]{\hspace{9mm}\includegraphics[width=0.35\columnwidth]{./figure/LBYL_figure_2_3.pdf}\hspace{9mm}} 
    \caption{Comparison between pruning matrix and delivery matrix, where the $4$-th and $6$-th filters are being pruned among $6$ original filters}
	\label{fig:matrix}
	\vspace{-2mm}
\end{figure*}



\subsection{Filter Pruning in a CNN}
Consider a given CNN to be pruned with $L$ layers, where each $\ell$-th layer starts with a convolution operation on its input channels, which are the output of the previous $(\ell-1)$-th layer $\mathbf{A}^{(\ell-1)}$, with the group of convolution filters $\mathbf{W}^{{(\ell)}}$ and thereby obtain the set of \textit{feature maps} $\mathbf{Z}^{(\ell)}$ as follows:
\begin{equation}
\boldsymbol{\mathbf{Z}}^{(\ell)} = {\mathbf{A}^{(\ell-1)} \circledast {\mathbf{W}}^{(\ell)}},
\nonumber
\end{equation}
where $\circledast$ represents the convolution operation. Then, this convolution process is normally followed by a batch normalization (BN) process and an activation function such as ReLU, and the $\ell$-th layer finally outputs an \textit{activation map} $\mathbf{A}^{(\ell)}$ to be sent to the $(\ell+1)$-th layer through this sequence of procedures as:
\begin{equation}
\mathbf{A}^{(\ell)} = \F(\N(\mathbf{Z}^{(\ell)})),
\nonumber
\end{equation}
where $\F(\cdot)$ is an activation function and $\N(\cdot)$ is a BN procedure.

Note that all of $\mathbf{W}^{(\ell)}$, $\mathbf{Z}^{(\ell)}$, and $\mathbf{A}^{(\ell)}$ are tensors such that: $\mathbf{W}^{(\ell)} \in \mathbb{R}^{m \times n \times k \times k}$ and $\mathbf{Z}^{(\ell)},\mathbf{A}^{(\ell)} \in \mathbb{R}^{m \times w \times h}$, where (1) $m$ is the number of filters, which also equals the number of output activation maps, (2) $n$ is the number of input activation maps resulting from the $(\ell-1)$-th layer, (3) $k \times k$ is the size of each filter, and (4) $w \times h$ is the size of each output channel for the $\ell$-th layer.

\smalltitle{Filter pruning as n-mode product}
When filter pruning is performed at the $\ell$-th layer, all three tensors above are consequently modified to their \textit{damaged} versions, namely $\mathbf{\Tilde{W}}^{(\ell)}$, $\mathbf{\Tilde{Z}}^{(\ell)}$, and $\mathbf{\Tilde{A}}^{(\ell)}$, respectively, in a way that: $\mathbf{\Tilde{W}}^{(\ell)} \in \mathbb{R}^{t \times n \times k \times k}$ and $\mathbf{\Tilde{Z}}^{(\ell)},\mathbf{\Tilde{A}}^{(\ell)} \in \mathbb{R}^{t \times w \times h}$, where $t$ is the number of remaining filters after pruning and therefore $t < m$. Mathematically, the tensor of remaining filters, \textit{i.e.}, $\mathbf{\Tilde{W}}^{(\ell)}$, is obtained by the \textit{$1$-mode product} \cite{DBLP:journals/siamrev/KoldaB09} of the tensor of the original filters $\mathbf{W}^{(\ell)}$ with a \textit{pruning matrix} $\boldsymbol{\S} \in \mathbb{R}^{m \times t}$ (see Figure \ref{fig:matrix:a})
as follows:
\begin{eqnarray}\begin{split}\label{eq:pruning}
\mathbf{\Tilde{W}}^{(\ell)} = {\mathbf{W}}^{(\ell)} \times_{1} {\boldsymbol{\S}}^{T},\text{where }\boldsymbol{\S}_{i,k} = 
  \begin{cases} 
   1~ \text{if } i = i'_k \\
   0~ \text{otherwise}
  \end{cases} \\
  \text{s.t. } i, i'_k \in [1, m] 
  \text{ and } k \in [1, t].
  \end{split}
\end{eqnarray}
  
By Eq. (\ref{eq:pruning}), each $i'_k$-th filter is not pruned and the other $(m-t)$ filters are completely removed from $\mathbf{W}^{(\ell)}$ to be $\mathbf{\Tilde{W}}^{(\ell)}$.

This reduction at the $\ell$-th layer causes another reduction for each filter of the $(\ell+1)$-th layer so that $\mathbf{W}^{(\ell+1)}$ is now modified to $\mathbf{\Tilde{W}}^{(\ell+1)} \in \mathbb{R}^{m' \times t \times k' \times k'}$, where $m'$ is the number of filters of size $k' \times k'$ in the $(\ell+1)$-th layer. Due to this series of information losses, the resulting feature map (\textit{i.e.}, $\mathbf{Z}^{(\ell+1)}$) would severely be damaged to be $\mathbf{\Tilde{Z}}^{(\ell+1)}$ as shown below:
\begin{equation}
{\mathbf{\Tilde{Z}}}^{{(\ell+1)}} = \mathbf{\Tilde{A}}^{(\ell)} \circledast {\mathbf{\Tilde{W}}}^{(\ell+1)}~~~\not\approx~~~\mathbf{Z}^{(\ell+1)}
\label{eq:eq}\nonumber
\end{equation}
The shape of $\mathbf{\Tilde{Z}}^{(\ell+1)}$ remains the same unless we also prune filters for the $(\ell+1)$-th layer. If we do so as well, the loss of information will be accumulated and further propagated to the next layers. Note that $\mathbf{\Tilde{W}}^{(\ell+1)}$ can also be represented by the \textit{$2$-mode product} \cite{DBLP:journals/siamrev/KoldaB09} of $\mathbf{W}^{(\ell+1)}$ with the transpose of the same matrix $\boldsymbol{\S}$ as:
\begin{equation} \label{eq:pruning2}
\mathbf{\Tilde{W}}^{(\ell+1)} = {\mathbf{W}}^{(\ell+1)} \times_{2} {\boldsymbol{\S}^T}
\end{equation}




\subsection{Problem of Restoring a Pruned Network without Data and Fine-Tuning}
As mentioned earlier, our goal is to restore a pruned and thus damaged CNN without using any data and re-training process, which implies the following two facts. First, we have to use a pruning criterion exploiting only the values of filters themselves such as L1-norm. In this sense, this paper does not focus on proposing a sophisticated pruning criterion but intends to recover a network somehow pruned by such a simple criterion. Secondly, since we cannot make appropriate changes in the remaining filters by fine-tuning, we should make the best use of the original network and identify how the information carried by a pruned filter can be delivered to the remaining filters.

% For brevity, we formulate our problem here with respect to a specific layer, say $\ell$, and then it can trivially be generalized for the entire network. 
\smalltitle{Delivery matrix}
In order to represent the information to be delivered to the preserved filters, let us first think of what the pruning matrix $\boldsymbol{\S}$ means. As defined in Eq. (\ref{eq:pruning}) and shown in Figure \ref{fig:matrix:a}, each row is either a zero vector (for filters being pruned) or a one-hot vector (for remaining filters), which is intended only to remove filters without delivering any information. Intuitively, we can transform this pruning matrix into a \textit{delivery matrix} that carries information for filters being pruned by replacing some meaningful values with some of the zero values therein. Once we find such an \textit{ideal} $\boldsymbol{\S^*}$, we can plug it into $\boldsymbol{\S}$ of Eq. (\ref{eq:pruning2}) to deliver missing information propagated from the $\ell$-th layer to the filters at the $(\ell+1)$-th layer, which will hopefully generate an approximation $\mathbf{\hat{Z}}^{(\ell+1)}$ close to the original feature map as follows:
\begin{equation} \label{eq:fmap_approx}
{\mathbf{\hat{Z}}}^{{(\ell+1)}} = {\mathbf{\Tilde{A}}^{(\ell)} \circledast ({\mathbf{W}}^{(\ell+1)} \times_{2} {\boldsymbol{\S^*}^T})}
~~~\approx~~~\mathbf{Z}^{(\ell+1)}
\end{equation}
Thus, using the delivery matrix $\boldsymbol{\mathcal{S^*}}$, the information loss caused by pruning at each layer is recovered at the feature map of the next layer.

\smalltitle{Problem statement}
Given a pretrained CNN, our problem aims to find the best delivery matrix $\boldsymbol{\mathcal{S^*}}$ for each layer without any data and training process such that the following \textit{reconstruction error} is minimized:
\begin{equation}
\sum\limits_{i = 1}^{m'}\|{{\mathbf{Z}}_{i}^{{(\ell+1)}}-{\hat{\mathbf{Z}}}_{i}^{{(\ell+1)}}}\|_1,
\label{eq:goal}
\end{equation}
where ${\mathbf{Z}}_i^{{(\ell+1)}}$ and ${\hat{\mathbf{Z}}}_i^{{(\ell+1)}}$ indicate the $i$-th original feature map and its corresponding approximation, respectively, out of $m'$ filters in the $(\ell+1)$-th layer. Note that what is challenging here is that we cannot obtain the activation maps in $\mathbf{A}^{(\ell)}$ and $\mathbf{\Tilde{A}}^{(\ell)}$ without data as they are data-dependent values.

% = \sum\limits_{i = 1}^{m'}\|{{\mathbf{Z}}_{i}^{{(\ell+1)}}-{\mathbf{\Tilde{A}}^{(\ell)} \circledast ({\mathbf{W}}^{(\ell+1)} \times_{2} {\boldsymbol{\mathcal{S^*}^T}})}}\|_{1}


% Our goal is finding the approximation matrix $\boldsymbol{\mathcal{S}}$ to minimize the reconstruction error between the pruned model and the original model without any data, and effectively deliver missing information for pruned filters using this approximation matrix


% $\testit{s}$,which can be represented as below.

% \begin{equation}
% \boldsymbol{\mathcal{S}} =  \underset{{\boldsymbol{\mathcal{S}}}}{\mathrm{argmin}} \sum\limits_{{i} = 1}^{m_{\ell+1}} \|{{\mathbf{Z}}_{i,:,:}^{{(\ell+1)}}-{\hat{\mathbf{Z}}}_{i,:,:}^{{(\ell+1)}}}\|_{1} 
% \label{eq:eq1}
% \end{equation}



% Let us first recall that the ultimate goal of network pruning is to make the output of a pruned network as close as possible to that of its original network. Unlike many existing pruning methods, our focus is not to use any training data at all for the entire pruning and recovery process, and this implies the following two facts. First, we cannot evaluate the filter importance by data-dependent values like activation values or gradients, but have to use a pruning criterion exploiting only the values of filters themselves such as L1-norm. Furthermore, instead of fine-tuning with data, the only thing we can do for the pruned network is to make appropriate changes in the remaining filters by identifying some relationships between pruned filters and the other preserved ones without any support from data. Based on this intuition, this section mathematically and generally defines the problem of restoring a pruned neural network in a manner free of data and fine-tuning.


% Thus, we make approximation matrix $\testit{s}$ $\in$ $\mathbb{R}^{m_{\ell} \times t_{\ell}}$ with relationship between the pruned filter and preserved filters in $\ell$-th layer and then apply it to the original filters in $(\ell+1)$-th layer to compensate for pruned feature maps $\boldsymbol{\hat{\mathbf{Z}}}^{{(\ell+1)}}$ as shown below.
% (\textit{i.e.}, Let $\hat{\mathbf{W}}^{(\ell+1)}$ be ${\mathbf{W}}^{(\ell+1)}$ $\times_2$ ${{\textit{s}}} $, where $\times_2$ is 2-mode matrix product) 

% \begin{equation}
% \mathbf{Z}^{(\ell+1)} = {\mathbf{A}}^{(\ell)} \circledast {\mathbf{W}}^{(\ell+1)}
% \approx {\hat{\mathbf{A}}^{(\ell)} \circledast ({\mathbf{W}}^{(\ell+1)} \times_{2} {{s}}) = {\hat{\mathbf{Z}}}^{{(\ell+1)}}}
% \label{eq:eq}\nonumber
% \end{equation}




% For a Convolutional Neural Network (CNN) with $L$ layers, we denote $\mathcal{A}{^{(\ell-1)}}$ $\in$ $\mathbb{R}^{n_{\ell -1 } \times h_{\ell -1} \times w_{\ell -1}}$ is activation maps at $\ell-1$-th layer, where $n_{\ell -1}$, $h_{\ell -1}$ and $w_{\ell -1}$ are the number of channels, height and width in activation maps, respectively. and we denote $\mathbf{W}^{{(\ell )}}$ $\in$  $\mathbb{R}^{m_{\ell} \times n_{\ell -1}\times k \times k}$ is covolution filters in $\ell$-th layer,where $m_{\ell}$, $n_{\ell-1}$ and $k$ are the number of filters, number of channels and kernel size, respectively. Trough the convolution operation using activation map $\mathcal{A}{^{(\ell-1)}}$ and convolution filter $\mathbf{W}^{{(\ell)}}$ in $\ell$-th layer, the feature maps $\boldsymbol{\mathbf{Z}}^{{(\ell)}}$ $\in$ $\mathbb{R}^{m_{\ell} \times h_{\ell+1} \times w_{\ell+1}}$ is computed as shown as below.


% \begin{equation}
% \boldsymbol{\mathbf{Z}}^{(\ell)} = {\mathcal{A}^{(\ell-1)} \circledast {\mathbf{W}}^{(\ell)}}
% \label{eq:eq1}\nonumber
% \end{equation}
% where $\circledast$ is convolution operation.

% and the feature maps passed through the BN and ReLU layer are activation maps $\mathcal{A}{^{(\ell)}}$ $\in$ $\mathbb{R}^{m_{{\ell}} \times h_{\ell+1} \times w_{\ell+1}} $ in $\ell$-th layer as shown as below.

% \begin{equation}
% \mathcal{A}^{(\ell)} = \mathcal{F}(\mathbf{Z}^{(\ell)} \circledast {\mathbf{W}}^{(\ell)})
% \label{eq:eq2}\nonumber
% \end{equation}
% where $\mathcal{F}$ is the function that implement batch normalization and non-linear activation(\textit{e.g.}, ReLU).

% \smalltitle{Filter Pruning}
% If the filter pruning is performed in $\ell$-th layer, the shape of original filters $\mathbf{W}^{{(\ell)}}$ $\in$ $\mathbb{R}^{m_{\ell} \times n_{\ell-1}\times k \times k}$ is modified to ${\hat {\mathbf{W}}^{(\ell)}}$ $\in$ $\mathbb{R}^{t_{\ell} \times n_{\ell-1}\times k \times k}$, where $t_{\ell}$ $<$ $m_{\ell}$ by pruning criterion. Therefore, the pruned activation maps ${\hat {\mathcal{A}}}{^{({\ell+1})}}$ $\in$ $\mathbb{R}^{t_{{\ell}} \times h_{{\ell+2}} \times w_{{\ell+2}}}$ in (${\ell+1}$)-th layer is computed as below.

% \begin{equation}
% \mathbf{\hat{A}}^{(l+1)} = \mathcal{F}({\mathbf{A}^{(\ell)} \circledast {\mathbf{\hat{W}}}^{(\ell+1)}})
% \label{eq:eq3}\nonumber
% \end{equation}

% Moreover, corresponding channels of each filters in ($\ell +1$)-th layer are sequentially removed. As a result, shape of original filters $\mathbf{W}^{{(\ell+1)}}$ $\in$ $\mathbb{R}^{m_{\ell+1} \times m_{\ell}\times k \times k}$ in ($\ell+1$)-th layer is changed to  ${\hat {\mathbf{W}}^{(\ell+1)}}$ $\in$ $\mathbb{R}^{m_{\ell+1} \times t_{\ell}\times k \times k}$. Although feature maps ${\hat{\mathbf{Z}}}^{{(\ell+1)}}$ $\in$ $\mathbb{R}^{m_{\ell+1} \times h_{\ell+2} \times w_{\ell+2}}$ in ($\ell+1$)-th layer after pruning have same shape with original feature maps ${\mathbf{Z}}^{{(\ell+1)}}$ $\in$ $\mathbb{R}^{m_{\ell+1} \times h_{\ell+2} \times w_{\ell+2}}$, the pruned feature maps $\boldsymbol{\hat{\mathbf{Z}}}^{{(\ell+1)}}$ are damaged.

\section{Approach}\label{sec:approach}
\section{Temporal Representation Alignment}
\label{sec:approach}

When training a series of short-horizon goal-reaching and instruction-following tasks, our goal is to learn a representation space such that our policy can generalize to a new (long-horizon) task that can be viewed as a sequence of known subtasks.
We propose to structure this representation space by aligning the representations of states, goals, and language in a way that is more amenable to compositional generalization.

\paragraph{Notation.}
We take the setting of a goal- and language-conditioned MDP $\cM$ with state space $\cS$, continuous action space $\cA \subseteq (0,1)^{d_{\cA}}$, initial state distribution $p_0$, dynamics $\p(s'\mid s,a)$, discount factor $\gamma$, and language task distribution $p_{\ell}$.
A policy $\pi(a\mid s)$ maps states to a distribution over actions. We inductively define the $k$-step (action-conditioned) policy visitation distribution as:
\begin{align*}
    p^{\pi}_{1}(s_{1} \mid s_1, a_{1})
    &\triangleq p(s_1 \mid s_1, a_1),\\
    p^{\pi}_{k+1}(s_{k+1} \mid s_1, a_1)
    &\triangleq \nonumber\\*
      &\mspace{-120mu} \int_{\cA}\int_{\cS} p(s_{k+1} \mid s,a) \dd p^{\pi}_{k}(s \mid s_{1},a_1) \dd
        \pi(a \mid s)\\
    p^{\pi}_{k+t}(s_{k+t} \mid s_t,a_t)
    &\triangleq p^{\pi}(s_{k} \mid s_1, a_1) . \eqmark
        \label{eq:successor_distribution}
\end{align*}
Then, the discounted state visitation distribution can be defined as the distribution over $s^{+}$\llap, the state reached after $K\sim \operatorname{Geom}(1-\gamma)$ steps:
\begin{equation}
    p^{\pi}_{\gamma}(s^{+}  \mid  s,a) \triangleq \sum_{k=0}^{\infty} \gamma^{k} p^{\pi}_{k}(s^{+} \mid s,a).
    \label{eq:discounted_state_visitation}
\end{equation}

We assume access to a dataset of expert demonstrations $\cD = \{\tau_{i},\ell_i\}_{i=1}^{K}$, where each trajectory
\begin{equation}
    \tau_{i} = \{s_{t,i},a_{t,i}\}_{t=1}^{H} \in \cS \times \cA
    \label{eq:trajectory}
\end{equation}
is gathered by an expert policy $\expert$, and is then annotated with $p_{\ell}(\ell_{i} \mid s_{1,i}, s_{H,i})$.
Our aim is to learn a policy $\pi$ that can select actions conditioned on a new language instruction $\ell$.
As in prior work~\citep{walke2023bridgedata}, we handle the continuous action space by representing both our policy and the expert policy as an isotropic Gaussian with fixed variance; we will equivalently write $\pi(a\mid s, \varphi)$ or denote the mode as $\hat{a} = \pi(s,\varphi)$ for a task $\varphi$.

\begin{rebuttal}
    \subsection{Representations for Reaching Distant Goals}
    \label{sec:reaching_goals}

    We learn a goal-conditioned policy $\pi(a\mid s,g)$ that selects actions to reach a goal $g$ from expert demonstrations with behavioral cloning.
    Suppose we directly selected actions to imitate the expert on two trajectories in $\cD$:
    
    \begin{equation}
        \mspace{-100mu}\begin{tikzcd}[remember picture,sep=small]
            s_1 \rar & s_2 \rar  & \ldots \rar & s_{H} \rar & w      \quad \\
            w \rar   & s_1' \rar & \ldots \rar & s_{H}' \rar & g\quad
        \end{tikzcd}
        \begin{tikzpicture}[remember picture,overlay] \coordinate (a) at (\tikzcdmatrixname-1-5.north east);
            \coordinate (b) at (\tikzcdmatrixname-2-5.south east);
            \coordinate (c) at (a|-b);
            \draw[decorate,line width=1.5pt,decoration={brace,raise=3pt,amplitude=5pt}]
        (a) -- node[right=1.5em] {$\tau_{i}\in \cD$} (c); \end{tikzpicture}
        \label{eq:trajectory_diagram}
    \end{equation}
    When conditioned with the composed goal $g$, we would be unable to imitate effectively
        as the composed state-goal $(s,g)$ is jointly out of the training distribution.

    What \emph{would} work for reaching $g$ is to first condition the policy on the intermediate waypoint $w$, then upon reaching $w$, condition on the goal $g$, as the state-goal pairs $(s_{i},w)$, $(w,g)$, and $(s_{i}',g)$ are all in the training distribution.
    If we condition the policy on some intermediate waypoint distribution $p(w)$ (or sufficient statistics thereof) that captures all of these cases, we can stitch together the expert behaviors to reach the goal $g$.

    Our approach is to learn a representation space that captures this ability, so that a GCBC objective used in this space can effectively imitate the expert on the composed task.
     We begin with the goal-conditioned behavioral cloning~\citep{kaelbling1993learning}
        loss $\cL_{\textsc{bc}}^{\phi,\psi,\xi}$ conditioned with waypoints $w$.
    \begin{equation}
        \cL_{\textsc{bc}}\bigl(\{s_{i},a_{i},s_{i}^{+},g_{i}\}_{i=1}^{K}\bigr) = \sum_{{i=1}}^{K} \log \pi\bigl(a_{i} \mid s_{i},\psi(g_{i})\bigr).
        \label{eq:goal_conditioned_bc}
    \end{equation}
    Enforcing the invariance needed to stitch \cref{eq:trajectory_diagram} then reduces to aligning \mbox{$\psi(g) \leftrightarrow \psi(w).$}
    The temporal alignment objective $\phi(s)\leftrightarrow \phi(s^{+})$ accomplishes this indirectly by aligning both $\psi(w)$ and $\psi(g)$ to the shared waypoint representation $\phi(w)$:

    \csuse{color indices}
    \begin{align}
        &\cL_{\textsc{nce}}\bigl(\{s_{i},s_{i}^{+}\}_{i=1}^{K};\phi,\psi\bigr) =
        \log \biggl( {\frac{e^{\phi(s^+_{\i})^{T}\psi(s_{\i})}}{\sum_{{\j=1}}^{K}
                e^{\phi(s^+_{\i})^{T}\psi(s_{\j})}}} \biggr)  \nonumber\\*
                &\mspace{100mu} +
        \sum_{{\j=1}}^{K} \log \biggl( {\frac{e^{\phi(s^+_{\i})^{T}\psi(s_{\i})}}{\sum_{{\i=1}}^{K}
                e^{\phi(s^+_{\i})^{T}\psi(s_{\j})}}} \biggr)
        \label{eq:goal_alignment}
    \end{align}

        
\end{rebuttal}
\subsection{Interfacing with Language Instructions}
\label{sec:language_instructions}

To extend the representations from \cref{sec:reaching_goals} to compositional instruction following with language tasks, we need some way to ground language into the $\psi$ (future state)
representation space.
We use a similar approach to GRIF~\citep{myers2023goal}, which uses an additional CLIP-style \citep{radford2021learning} contrastive alignment loss with an additional pretrained language encoder $\xi$:
\csuse{no color indices}
\begin{align}
    &\cL_{\textsc{nce}}\bigl(\{g_{i},\ell_{i}\}_{i=1}^{K};\psi,\xi\bigr)
    = \sum_{{i=1}}^{K} \log \biggl( {\frac{e^{\psi(g_{\i})^{T}\xi(\ell_{\i})}}{\sum_{{\j=1}}^{K}
            e^{\psi(g_{\i})^{T}\xi(\ell_{\j})}}} \biggr)  \nonumber\\*
            &\mspace{100mu} +
    \sum_{{\j=1}}^{K} \log \biggl( {\frac{e^{\psi(g_{\i})^{T}\xi(\ell_{\i})}}{\sum_{{\i=1}}^{K}
            e^{\psi(g_{\i})^{T}\xi(\ell_{\j})}}} \biggr)
    \label{eq:task_alignment}
\end{align}

\subsection{Temporal Alignment}
\label{sec:temporal_alignment}

Putting together the objectives from \cref{sec:reaching_goals,sec:language_instructions} yields the Temporal Representation Alignment (\Method) approach.
\Method{} structures the representation space of goals and language instructions to better enable compositional generalization.
We learn encoders $\phi, \psi ,$ and $\xi$ to map states, goals, and language instructions to a shared representation space.

\csuse{color indices}
\begin{align}
    \cL_{\textsc{nce}} \label{eq:NCE}
    &(\{x_{i}, y_{i}\}_{i=1}^{K};f,h) =
        \sum_{{\i=1}}^{K} \log \biggl( {\frac{e^{f(y_{\i})^{T}h(x_{\i})}}{\sum_{{\j=1}}^{K}
        e^{f(y_{\i})^{T}h(x_{\j})}}} \biggr) \nonumber\\*
      &\mspace{100mu} +
        \sum_{{\j=1}}^{K} \log \biggl( {\frac{e^{f(y_{\i})^{T}h(x_{\i})}}{\sum_{{\i=1}}^{K}
        e^{f(y_{\i})^{T}h(x_{\j})}}} \biggr) \\
    \cL_{\textsc{bc}} \label{eq:BC}
    &\bigl(\{s_{i},a_{i},s^{+}_{i},\ell_{i}\}_{i=1}^{K};\pi,\psi,\xi\bigr) = \nonumber\\*
      &\mspace{-10mu} \sum_{{i=1}}^{K} \log
        \pi\bigl(a_{i} \mid s_{i},\xi(\ell_{i})\bigr) + \log \pi\bigl(a_{i} \mid
        s_{i},\psi(s^{+}_{i})\bigr) \\
    \cL_{\textsc{tra}}
    &\label{eq:TRA} \bigl( \{s_{i},a_{i},s_{i}^{+},g_{i},\ell_{i}\}_{i=1}^{K}; \pi,\phi,\psi,\xi\bigr)
        \\
    &= \underbrace{\cL_{\textsc{bc}}\bigl(\{s_{i},a_{i},s_{i}^{+},\ell_{i}\}_{i=1}^{K};\pi,\psi,\xi\bigr)}_{\text{behavioral
    cloning}} \nonumber\\*
    &+
        \underbrace{\cL_{\textsc{nce}}\bigl(\{s_{i},s_{i}^{+}\}_{i=1}^{K};\phi,\psi\bigr)}_{\text{temporal alignment}}
        + \underbrace{\cL_{\textsc{nce}}\bigl(\{g_{i},\ell_{i}\}_{i=1}^{K};\psi,\xi\bigr)}_{\text{task alignment}} \nonumber
\end{align}Note that the NCE alignment loss uses a CLIP-style symmetric contrastive objective~\citep{radford2021learning,eysenbach2024inference} \-- we highlight the indices in the NCE alignment loss~\eqref{eq:NCE} for clarity.

Our overall objective is to minimize \cref{eq:TRA} across states, actions, future states, goals, and language tasks within the training data:
\begin{align}
    &\min_{\pi,\phi,\psi,\xi} \mathbb{E}_{\substack{(s_{1,i},a_{1,i},\ldots,s_{H,i},a_{H,i},\ell) \sim \mathcal{D} \\
    i\sim\operatorname{Unif}(1\ldots H) \\
    k\sim\operatorname{Geom}(1-\gamma)}} \\*
    &\mspace{10mu}
    \Bigl[\cL_{\text{TRA}}\bigl(\{s_{t,i},a_{t,i},s_{\min(t+k,H),i},s_{H,i},\ell\}_{i=1}^{K};\pi,\phi,\psi,\xi\bigr)\Bigr].
    \label{eq:overall_objective}
\end{align}

\begin{algorithm}
    \caption{Temporal Representation Alignment}
    \label{alg:tra}
    \begin{algorithmic}[1]
        \State \textbf{input:} dataset $\mathcal{D} = (\{s_{t,i},a_{t,i}\}_{t=1}^{H},\ell_i)_{i=1}^N$
        \State initialize networks $\Theta \triangleq (\pi,\phi,\psi,\xi)$
        \While{training}
        \State sample batch $\bigl\{(s_{t,i},a_{t,i},s_{t+k,i},\ell_i)\bigr\}_{i=1}^K\sim\mathcal{D}$ \\
        \hspace*{2ex} for $k\sim\operatorname{Geom}(1-\gamma)$
        \State $\Theta \gets \Theta - \alpha \nabla_{\Theta} \cL_{\text{TRA}}\bigl(\{s_{t,i},a_{t,i},s_{t+k,i},\ell_i\}_{i=1}^K; \Theta\bigr)$
        \EndWhile
        \smallskip
        \State \textbf{output:} \parbox[t]{\linewidth}{language-conditioned policy $\pi(a_{t} | s_{t}, \xi(\ell))$ \\
            goal-conditioned policy $\pi(a_{t} | s_{t}, \psi(g))$
        }
    \end{algorithmic}
\end{algorithm}

\subsection{Implementation}
\label{sec:implementation}

A summary of our approach is shown in \cref{alg:tra}.
In essence, TRA learns three encoders: $\phi$, which encodes states, $\psi$ which encodes future goals, and $\xi$ which encodes language instructions.
Contrastive losses are used to align state representations $\phi(s_{t})$ with future goal representations $\psi(s_{t+k})$, which are in turn aligned with equivalent language task specifications $\xi(\ell)$ when available.
We then learn a behavior cloning policy $\pi$ that can be conditioned on either the goal or language instruction through the representation $\psi(g)$ or $\xi(\ell)$, respectively.

\begin{rebuttal}
    \subsection{Temporal Alignment and Compositionality}
    \label{sec:compositionality}

    We will formalize the intuition from \cref{sec:reaching_goals} that \Method{} enables compositional generalization by considering the error on a ``compositional'' version of $\cD,$ denoted $\cD^{*}$.
    Using the notation from \cref{eq:trajectory}, we can say $\cD$ is distributed according to:
    \begin{align}
        &\cD \triangleq \cD^{H} \sim \prod_{i=1}^{K} p_0(s_{1,i}) p_{\ell}(\ell_{i} \mid s_{1,i}, s_{H,i})
            \nonumber\\*
          &\mspace{60mu} \prod_{t=1}^{H} \expert(a_{t,i} \mid s_{t,i}) \p(s_{t+1,i} \mid s_{t,i}, a_{t,i}) ,
            \label{eq:dataset_distribution}
    \end{align}
    or equivalently
    \begin{align}
            &\cD^{H} \sim \prod_{i=1}^{K} p_0(s_{1,i}) p_{\ell}(\ell_{i} \mid s_{1,i}, s_{H,i}) \nonumber\\*
            &\mspace{60mu} \prod_{t=1}^{H}
            e^{\sigma^2\|\expert(s_{t,i}) - a_{t,i}\|^2}\p(s_{t+1,i} \mid s_{t,i}, a_{t,i}) ,
            \label{eq:dataset_distribution_2}
    \end{align}
    by the isotropic Gaussian assumption.
    We will define $\cD^{*} \triangleq \cD^{H'}$ to be a longer-horizon version of $\cD$ extending the behaviors gathered under $\expert$ across a horizon $\alpha H \ge H' \ge H$ that additionally satisfies a ``time-isotropy'' property: the marginal distribution of the states is uniform across the horizon, i.e., $p_0(s_{1,i}) = p_0(s_{t,i})$ for all $t \in \{1\ldots H'\}$.

    We will relate the in-distribution imitation error $\textsc{Err}(\bullet; \cD)$ to the compositional out-of-distribution imitation error $\textsc{Err}(\bullet;\cD^{*})$.
    We define
    \begin{align}
        \textsc{Err}(\hat{\pi}; \tilde{\cD})
        &= \E_{\tilde{\cD}}\Bigl[\frac{1}{H}\sum_{t=1}^{H} \mathbb{E}_{\hat{\pi}}\left[\|\tilde{a}_{t,i} -
        \hat{\pi}(\tilde{s}_{t,i}, \tilde{s}_{H, i})\|^{2}/d_{\cA}\right]\Bigr] \nonumber\\
        &\quad \text{for} \quad \{\tilde{s}_{t,i},\tilde{a}_{t,i},\tilde{\ell}_{i}\}_{t=1}^{H} \sim
            \tilde{\cD}.
            \label{eq:imitation_error}
    \end{align}
    On the training dataset this is equivalent to the expected behavioral cloning loss from \cref{eq:BC}.

                            
    \begin{assumption}
        \label{asm:policy_factorization}
        The policy factorizes through inferred waypoints as:
\begin{align}
    &\textrm{goals: }\pi(a \mid s, g)
        = \nonumber\\*
      &\mspace{50mu} \int \pi(a\mid s, w) \p(s_{t}=w \mid s_{t+k}=g) \dd{w}
        \label{eq:goal-conditioned} \\
    &\textrm{language: } \pi(a \mid s, \ell)
        = \int \pi(a\mid s, w) \nonumber\\*
      &\mspace{20mu} \p(s_{t}=w \mid s_{t+k}=g) \p(s_{t+k}=g \mid \ell) \dd{w} \dd{g} ,
        \label{eq:language-conditioned}
        \end{align}
        where denote by $\pi(s,g)$ the MLE estimate of the action $a$.

    \end{assumption}

    \makerestatable
    \begin{theorem}
        \label{thm:compositionality}
        Suppose $\cD$ is distributed according to \cref{eq:dataset_distribution} and $\cD^{*}$ is distributed according to \cref{eq:dataset_distribution}.
        When $\gamma > 1-1/H$ and $\alpha > 1$, for optimal features $\phi$ and $\psi$ under \cref{eq:overall_objective}, we have
        \begin{gather}
            \textsc{Err}(\pi; \cD^{*}) \le \textsc{Err}(\pi; \cD) +  \frac{\alpha -1}{2 \alpha }+\Bigl(\frac{ \alpha - 2 }{2\alpha}\Bigr) \1 \{\alpha >2\}  .
            \label{eq:compositionality}
        \end{gather}
    \end{theorem}

    We can also define a notion of the language-conditioned compositional generalization error:
    \begin{equation*}
        \errl(\pi; \cD^{*}) \triangleq \E_{\cD^{*}}\Bigl[\frac{1}{H}\sum_{t=1}^{H}
            \mathbb{E}_{\pi}\bigl[\|\tilde{a}_{t,i} - \pi(\tilde{s}_{t,i}, \tilde{\ell}_{i})\|^{2}\bigr]\Bigr].
            \label{eq:language_error}
    \end{equation*}

    \makerestatable
    \begin{corollary}
        \label{thm:language}
        Under the same conditions as \cref{thm:compositionality},
        \begin{equation*}
            \errl(\pi; \cD^{*}) \le \errl(\pi; \cD) +  \frac{\alpha -1}{2 \alpha }+\Bigl(\frac{ \alpha - 2 }{2\alpha}\Bigr) \1 \{\alpha >2\}  .
            \label{eq:compositionality_language}
        \end{equation*}

    \end{corollary}

    The proofs as well as a visualization of the bound are in \cref{app:compositionality}. Policy implementation details can be found in \cref{app:tra_impl}

    
                
        
    \end{rebuttal}


\section{Experiments}\label{sec:experiments}
\section{Experiments}
\label{sec:experiments}
The experiments are designed to address two key research questions.
First, \textbf{RQ1} evaluates whether the average $L_2$-norm of the counterfactual perturbation vectors ($\overline{||\perturb||}$) decreases as the model overfits the data, thereby providing further empirical validation for our hypothesis.
Second, \textbf{RQ2} evaluates the ability of the proposed counterfactual regularized loss, as defined in (\ref{eq:regularized_loss2}), to mitigate overfitting when compared to existing regularization techniques.

% The experiments are designed to address three key research questions. First, \textbf{RQ1} investigates whether the mean perturbation vector norm decreases as the model overfits the data, aiming to further validate our intuition. Second, \textbf{RQ2} explores whether the mean perturbation vector norm can be effectively leveraged as a regularization term during training, offering insights into its potential role in mitigating overfitting. Finally, \textbf{RQ3} examines whether our counterfactual regularizer enables the model to achieve superior performance compared to existing regularization methods, thus highlighting its practical advantage.

\subsection{Experimental Setup}
\textbf{\textit{Datasets, Models, and Tasks.}}
The experiments are conducted on three datasets: \textit{Water Potability}~\cite{kadiwal2020waterpotability}, \textit{Phomene}~\cite{phomene}, and \textit{CIFAR-10}~\cite{krizhevsky2009learning}. For \textit{Water Potability} and \textit{Phomene}, we randomly select $80\%$ of the samples for the training set, and the remaining $20\%$ for the test set, \textit{CIFAR-10} comes already split. Furthermore, we consider the following models: Logistic Regression, Multi-Layer Perceptron (MLP) with 100 and 30 neurons on each hidden layer, and PreactResNet-18~\cite{he2016cvecvv} as a Convolutional Neural Network (CNN) architecture.
We focus on binary classification tasks and leave the extension to multiclass scenarios for future work. However, for datasets that are inherently multiclass, we transform the problem into a binary classification task by selecting two classes, aligning with our assumption.

\smallskip
\noindent\textbf{\textit{Evaluation Measures.}} To characterize the degree of overfitting, we use the test loss, as it serves as a reliable indicator of the model's generalization capability to unseen data. Additionally, we evaluate the predictive performance of each model using the test accuracy.

\smallskip
\noindent\textbf{\textit{Baselines.}} We compare CF-Reg with the following regularization techniques: L1 (``Lasso''), L2 (``Ridge''), and Dropout.

\smallskip
\noindent\textbf{\textit{Configurations.}}
For each model, we adopt specific configurations as follows.
\begin{itemize}
\item \textit{Logistic Regression:} To induce overfitting in the model, we artificially increase the dimensionality of the data beyond the number of training samples by applying a polynomial feature expansion. This approach ensures that the model has enough capacity to overfit the training data, allowing us to analyze the impact of our counterfactual regularizer. The degree of the polynomial is chosen as the smallest degree that makes the number of features greater than the number of data.
\item \textit{Neural Networks (MLP and CNN):} To take advantage of the closed-form solution for computing the optimal perturbation vector as defined in (\ref{eq:opt-delta}), we use a local linear approximation of the neural network models. Hence, given an instance $\inst_i$, we consider the (optimal) counterfactual not with respect to $\model$ but with respect to:
\begin{equation}
\label{eq:taylor}
    \model^{lin}(\inst) = \model(\inst_i) + \nabla_{\inst}\model(\inst_i)(\inst - \inst_i),
\end{equation}
where $\model^{lin}$ represents the first-order Taylor approximation of $\model$ at $\inst_i$.
Note that this step is unnecessary for Logistic Regression, as it is inherently a linear model.
\end{itemize}

\smallskip
\noindent \textbf{\textit{Implementation Details.}} We run all experiments on a machine equipped with an AMD Ryzen 9 7900 12-Core Processor and an NVIDIA GeForce RTX 4090 GPU. Our implementation is based on the PyTorch Lightning framework. We use stochastic gradient descent as the optimizer with a learning rate of $\eta = 0.001$ and no weight decay. We use a batch size of $128$. The training and test steps are conducted for $6000$ epochs on the \textit{Water Potability} and \textit{Phoneme} datasets, while for the \textit{CIFAR-10} dataset, they are performed for $200$ epochs.
Finally, the contribution $w_i^{\varepsilon}$ of each training point $\inst_i$ is uniformly set as $w_i^{\varepsilon} = 1~\forall i\in \{1,\ldots,m\}$.

The source code implementation for our experiments is available at the following GitHub repository: \url{https://anonymous.4open.science/r/COCE-80B4/README.md} 

\subsection{RQ1: Counterfactual Perturbation vs. Overfitting}
To address \textbf{RQ1}, we analyze the relationship between the test loss and the average $L_2$-norm of the counterfactual perturbation vectors ($\overline{||\perturb||}$) over training epochs.

In particular, Figure~\ref{fig:delta_loss_epochs} depicts the evolution of $\overline{||\perturb||}$ alongside the test loss for an MLP trained \textit{without} regularization on the \textit{Water Potability} dataset. 
\begin{figure}[ht]
    \centering
    \includegraphics[width=0.85\linewidth]{img/delta_loss_epochs.png}
    \caption{The average counterfactual perturbation vector $\overline{||\perturb||}$ (left $y$-axis) and the cross-entropy test loss (right $y$-axis) over training epochs ($x$-axis) for an MLP trained on the \textit{Water Potability} dataset \textit{without} regularization.}
    \label{fig:delta_loss_epochs}
\end{figure}

The plot shows a clear trend as the model starts to overfit the data (evidenced by an increase in test loss). 
Notably, $\overline{||\perturb||}$ begins to decrease, which aligns with the hypothesis that the average distance to the optimal counterfactual example gets smaller as the model's decision boundary becomes increasingly adherent to the training data.

It is worth noting that this trend is heavily influenced by the choice of the counterfactual generator model. In particular, the relationship between $\overline{||\perturb||}$ and the degree of overfitting may become even more pronounced when leveraging more accurate counterfactual generators. However, these models often come at the cost of higher computational complexity, and their exploration is left to future work.

Nonetheless, we expect that $\overline{||\perturb||}$ will eventually stabilize at a plateau, as the average $L_2$-norm of the optimal counterfactual perturbations cannot vanish to zero.

% Additionally, the choice of employing the score-based counterfactual explanation framework to generate counterfactuals was driven to promote computational efficiency.

% Future enhancements to the framework may involve adopting models capable of generating more precise counterfactuals. While such approaches may yield to performance improvements, they are likely to come at the cost of increased computational complexity.


\subsection{RQ2: Counterfactual Regularization Performance}
To answer \textbf{RQ2}, we evaluate the effectiveness of the proposed counterfactual regularization (CF-Reg) by comparing its performance against existing baselines: unregularized training loss (No-Reg), L1 regularization (L1-Reg), L2 regularization (L2-Reg), and Dropout.
Specifically, for each model and dataset combination, Table~\ref{tab:regularization_comparison} presents the mean value and standard deviation of test accuracy achieved by each method across 5 random initialization. 

The table illustrates that our regularization technique consistently delivers better results than existing methods across all evaluated scenarios, except for one case -- i.e., Logistic Regression on the \textit{Phomene} dataset. 
However, this setting exhibits an unusual pattern, as the highest model accuracy is achieved without any regularization. Even in this case, CF-Reg still surpasses other regularization baselines.

From the results above, we derive the following key insights. First, CF-Reg proves to be effective across various model types, ranging from simple linear models (Logistic Regression) to deep architectures like MLPs and CNNs, and across diverse datasets, including both tabular and image data. 
Second, CF-Reg's strong performance on the \textit{Water} dataset with Logistic Regression suggests that its benefits may be more pronounced when applied to simpler models. However, the unexpected outcome on the \textit{Phoneme} dataset calls for further investigation into this phenomenon.


\begin{table*}[h!]
    \centering
    \caption{Mean value and standard deviation of test accuracy across 5 random initializations for different model, dataset, and regularization method. The best results are highlighted in \textbf{bold}.}
    \label{tab:regularization_comparison}
    \begin{tabular}{|c|c|c|c|c|c|c|}
        \hline
        \textbf{Model} & \textbf{Dataset} & \textbf{No-Reg} & \textbf{L1-Reg} & \textbf{L2-Reg} & \textbf{Dropout} & \textbf{CF-Reg (ours)} \\ \hline
        Logistic Regression   & \textit{Water}   & $0.6595 \pm 0.0038$   & $0.6729 \pm 0.0056$   & $0.6756 \pm 0.0046$  & N/A    & $\mathbf{0.6918 \pm 0.0036}$                     \\ \hline
        MLP   & \textit{Water}   & $0.6756 \pm 0.0042$   & $0.6790 \pm 0.0058$   & $0.6790 \pm 0.0023$  & $0.6750 \pm 0.0036$    & $\mathbf{0.6802 \pm 0.0046}$                    \\ \hline
%        MLP   & \textit{Adult}   & $0.8404 \pm 0.0010$   & $\mathbf{0.8495 \pm 0.0007}$   & $0.8489 \pm 0.0014$  & $\mathbf{0.8495 \pm 0.0016}$     & $0.8449 \pm 0.0019$                    \\ \hline
        Logistic Regression   & \textit{Phomene}   & $\mathbf{0.8148 \pm 0.0020}$   & $0.8041 \pm 0.0028$   & $0.7835 \pm 0.0176$  & N/A    & $0.8098 \pm 0.0055$                     \\ \hline
        MLP   & \textit{Phomene}   & $0.8677 \pm 0.0033$   & $0.8374 \pm 0.0080$   & $0.8673 \pm 0.0045$  & $0.8672 \pm 0.0042$     & $\mathbf{0.8718 \pm 0.0040}$                    \\ \hline
        CNN   & \textit{CIFAR-10} & $0.6670 \pm 0.0233$   & $0.6229 \pm 0.0850$   & $0.7348 \pm 0.0365$   & N/A    & $\mathbf{0.7427 \pm 0.0571}$                     \\ \hline
    \end{tabular}
\end{table*}

\begin{table*}[htb!]
    \centering
    \caption{Hyperparameter configurations utilized for the generation of Table \ref{tab:regularization_comparison}. For our regularization the hyperparameters are reported as $\mathbf{\alpha/\beta}$.}
    \label{tab:performance_parameters}
    \begin{tabular}{|c|c|c|c|c|c|c|}
        \hline
        \textbf{Model} & \textbf{Dataset} & \textbf{No-Reg} & \textbf{L1-Reg} & \textbf{L2-Reg} & \textbf{Dropout} & \textbf{CF-Reg (ours)} \\ \hline
        Logistic Regression   & \textit{Water}   & N/A   & $0.0093$   & $0.6927$  & N/A    & $0.3791/1.0355$                     \\ \hline
        MLP   & \textit{Water}   & N/A   & $0.0007$   & $0.0022$  & $0.0002$    & $0.2567/1.9775$                    \\ \hline
        Logistic Regression   &
        \textit{Phomene}   & N/A   & $0.0097$   & $0.7979$  & N/A    & $0.0571/1.8516$                     \\ \hline
        MLP   & \textit{Phomene}   & N/A   & $0.0007$   & $4.24\cdot10^{-5}$  & $0.0015$    & $0.0516/2.2700$                    \\ \hline
       % MLP   & \textit{Adult}   & N/A   & $0.0018$   & $0.0018$  & $0.0601$     & $0.0764/2.2068$                    \\ \hline
        CNN   & \textit{CIFAR-10} & N/A   & $0.0050$   & $0.0864$ & N/A    & $0.3018/
        2.1502$                     \\ \hline
    \end{tabular}
\end{table*}

\begin{table*}[htb!]
    \centering
    \caption{Mean value and standard deviation of training time across 5 different runs. The reported time (in seconds) corresponds to the generation of each entry in Table \ref{tab:regularization_comparison}. Times are }
    \label{tab:times}
    \begin{tabular}{|c|c|c|c|c|c|c|}
        \hline
        \textbf{Model} & \textbf{Dataset} & \textbf{No-Reg} & \textbf{L1-Reg} & \textbf{L2-Reg} & \textbf{Dropout} & \textbf{CF-Reg (ours)} \\ \hline
        Logistic Regression   & \textit{Water}   & $222.98 \pm 1.07$   & $239.94 \pm 2.59$   & $241.60 \pm 1.88$  & N/A    & $251.50 \pm 1.93$                     \\ \hline
        MLP   & \textit{Water}   & $225.71 \pm 3.85$   & $250.13 \pm 4.44$   & $255.78 \pm 2.38$  & $237.83 \pm 3.45$    & $266.48 \pm 3.46$                    \\ \hline
        Logistic Regression   & \textit{Phomene}   & $266.39 \pm 0.82$ & $367.52 \pm 6.85$   & $361.69 \pm 4.04$  & N/A   & $310.48 \pm 0.76$                    \\ \hline
        MLP   &
        \textit{Phomene} & $335.62 \pm 1.77$   & $390.86 \pm 2.11$   & $393.96 \pm 1.95$ & $363.51 \pm 5.07$    & $403.14 \pm 1.92$                     \\ \hline
       % MLP   & \textit{Adult}   & N/A   & $0.0018$   & $0.0018$  & $0.0601$     & $0.0764/2.2068$                    \\ \hline
        CNN   & \textit{CIFAR-10} & $370.09 \pm 0.18$   & $395.71 \pm 0.55$   & $401.38 \pm 0.16$ & N/A    & $1287.8 \pm 0.26$                     \\ \hline
    \end{tabular}
\end{table*}

\subsection{Feasibility of our Method}
A crucial requirement for any regularization technique is that it should impose minimal impact on the overall training process.
In this respect, CF-Reg introduces an overhead that depends on the time required to find the optimal counterfactual example for each training instance. 
As such, the more sophisticated the counterfactual generator model probed during training the higher would be the time required. However, a more advanced counterfactual generator might provide a more effective regularization. We discuss this trade-off in more details in Section~\ref{sec:discussion}.

Table~\ref{tab:times} presents the average training time ($\pm$ standard deviation) for each model and dataset combination listed in Table~\ref{tab:regularization_comparison}.
We can observe that the higher accuracy achieved by CF-Reg using the score-based counterfactual generator comes with only minimal overhead. However, when applied to deep neural networks with many hidden layers, such as \textit{PreactResNet-18}, the forward derivative computation required for the linearization of the network introduces a more noticeable computational cost, explaining the longer training times in the table.

\subsection{Hyperparameter Sensitivity Analysis}
The proposed counterfactual regularization technique relies on two key hyperparameters: $\alpha$ and $\beta$. The former is intrinsic to the loss formulation defined in (\ref{eq:cf-train}), while the latter is closely tied to the choice of the score-based counterfactual explanation method used.

Figure~\ref{fig:test_alpha_beta} illustrates how the test accuracy of an MLP trained on the \textit{Water Potability} dataset changes for different combinations of $\alpha$ and $\beta$.

\begin{figure}[ht]
    \centering
    \includegraphics[width=0.85\linewidth]{img/test_acc_alpha_beta.png}
    \caption{The test accuracy of an MLP trained on the \textit{Water Potability} dataset, evaluated while varying the weight of our counterfactual regularizer ($\alpha$) for different values of $\beta$.}
    \label{fig:test_alpha_beta}
\end{figure}

We observe that, for a fixed $\beta$, increasing the weight of our counterfactual regularizer ($\alpha$) can slightly improve test accuracy until a sudden drop is noticed for $\alpha > 0.1$.
This behavior was expected, as the impact of our penalty, like any regularization term, can be disruptive if not properly controlled.

Moreover, this finding further demonstrates that our regularization method, CF-Reg, is inherently data-driven. Therefore, it requires specific fine-tuning based on the combination of the model and dataset at hand.

\section{Future Work}\label{sec:future-work}
In the future, we aim to extend the model to handle multiple relational tables. The primary challenges are overfitting and efficiency due to the involvement of numerous tables. Successfully addressing these challenges would enable knowledge fusion across real-world tabular data without shared features, supporting more applications in unsupervised multi-modal learning and vertical federated learning.

Additionally, we aim to develop efficient methods for identifying relationships between tables. While training a Leal model to evaluate table relationships by performance is feasible, it is computationally expensive and lacks scalability. Exploring efficient approaches could significantly enhance knowledge integration across tables.


\section{Conclusion}\label{sec:conclusion}
\section{Conclusion}
In this work, we propose a simple yet effective approach, called SMILE, for graph few-shot learning with fewer tasks. Specifically, we introduce a novel dual-level mixup strategy, including within-task and across-task mixup, for enriching the diversity of nodes within each task and the diversity of tasks. Also, we incorporate the degree-based prior information to learn expressive node embeddings. Theoretically, we prove that SMILE effectively enhances the model's generalization performance. Empirically, we conduct extensive experiments on multiple benchmarks and the results suggest that SMILE significantly outperforms other baselines, including both in-domain and cross-domain few-shot settings.









\section*{Impact Statement}

This paper presents work whose goal is to advance the field of 
Machine Learning. There are many potential societal consequences 
of our work, none which we feel must be specifically highlighted here.



\bibliography{references}
\bibliographystyle{icml2025}


\newpage
\appendix
\onecolumn
\section{Proof}\label{sec:proof}
\subsection{Error Gap Between Aligned and Misaligned Data}\label{subsec:proof-align-misalign}







\thmalignment*

\begin{proof}

For the aligned case, we can derive the mean squared error (MSE) as follows:
\begin{equation}\label{eq:mse_aligned}
    \mathrm{MSE}_\mathrm{aligned} = \inf_{\boldsymbol{\alpha} \in R^{m^P}, \boldsymbol{\beta} \in R^{m^S}} \|\mathbf{y} - \mathbf{X}^P \boldsymbol{\alpha} - \mathbf{X}^S \boldsymbol{\beta}\|
\end{equation}
The ordinary least squares (OLS) estimator of $\boldsymbol{\alpha}$ is given by:
\begin{equation}
    \hat{\boldsymbol{\alpha}} := (\mathbf{X}^{P \top} \mathbf{X}^P)^{-1} \mathbf{X}^P (\mathbf{y} - \mathbb{E}[\mathbf{R}] \mathbf{X}^S \boldsymbol{\beta}) 
\end{equation}
For a permutation matrix $\mathbf{R}$ under uniform distribution, we have $\mathbb{E}[\mathbf{R}] = \frac{1}{n}\mathds{1}^\top \mathds{1}$. Therefore:
\begin{equation}\label{eq:alpha_hat}
    \hat{\boldsymbol{\alpha}} = (\mathbf{X}^{P \top} \mathbf{X}^P)^{-1} \mathbf{X}^P (\mathbf{y} - \frac{1}{n} \mathds{1}^\top \mathds{1} \mathbf{X}^S \boldsymbol{\beta}) 
\end{equation}
The MSE for the misaligned case can be expressed as:
\begin{align}
    \mathrm{MSE}_{\mathrm{misaligned}} 
    & = \inf_{\boldsymbol{\beta}} \inf_{\boldsymbol{\alpha}} \mathbb{E}_\mathbf{R} \|\mathbf{y} - \mathbf{X}^P \boldsymbol{\alpha} - \mathbf{R} \mathbf{X}^S \boldsymbol{\beta}\|_2^2 \\
    & = \inf_{\boldsymbol{\beta}} \mathbb{E}_\mathbf{R} \|\mathbf{y} - \mathbf{X}^P \hat{\boldsymbol{\alpha}} - \mathbf{R} \mathbf{X}^S \boldsymbol{\beta}\|_2^2 \\
\end{align}
Substituting $\hat{\boldsymbol{\alpha}}$ from equation~\ref{eq:alpha_hat}, we obtain:
\begin{align}
    \mathrm{MSE}_{\mathrm{misaligned}} 
    & = \inf_{\boldsymbol{\beta}} \mathbb{E}_\mathbf{R} \left\|\mathbf{y} - \mathbf{X}^P (\mathbf{X}^{P \top} \mathbf{X}^P)^{-1} (\mathbf{X}^P \mathbf{y} - \mathbf{X}^P \frac{1}{n} 1^\top 1 \mathbf{X}^S \boldsymbol{\beta}) - \mathbf{R} \mathbf{X}^S \boldsymbol{\beta}\right\|_2^2 \\
    & = \inf_{\boldsymbol{\beta}} \mathbb{E}_\mathbf{R} \left\| (\mathbf{I} - \mathbf{X}^P (\mathbf{X}^{P \top} \mathbf{X}^P)^{-1} \mathbf{X}^P)\mathbf{y} + (\mathbf{X}^P (\mathbf{X}^{P \top} \mathbf{X}^P)^{-1} \mathbf{X}^P \frac{1}{n} \mathds{1}^\top \mathds{1} \mathbf{X}^S \boldsymbol{\beta}) - \mathbf{R} \mathbf{X}^S \boldsymbol{\beta}\right\|_2^2 
\end{align}
Since $\mathbf{X}^P (\mathbf{X}^{P \top} \mathbf{X}^P)^{-1} \mathbf{X}^P$ is a projection matrix that projects any vector onto the column space of $\mathbf{X}^P$, and $\mathbf{X}^S \boldsymbol{\beta}$ is orthogonal to the column space of $\mathbf{X}^P$, the term $\mathbf{X}^P (\mathbf{X}^{P \top} \mathbf{X}^P)^{-1} \mathbf{X}^P \frac{1}{n} \mathds{1}^\top \mathds{1} \mathbf{X}^S \boldsymbol{\beta} = 0$. Thus:
\begin{align}
    \mathrm{MSE}_{\mathrm{misaligned}}
    & = \inf_{\boldsymbol{\beta}} \mathbb{E}_\mathbf{R} \left\| (\mathbf{I} - \mathbf{X}^P (\mathbf{X}^{P \top} \mathbf{X}^P)^{-1} \mathbf{X}^P)\mathbf{y} - \mathbf{R} \mathbf{X}^S \boldsymbol{\beta}\right\|_2^2 \\
    & = \inf_{\boldsymbol{\beta}} \mathbb{E}_\mathbf{R} \left[\left\|\mathbf{R} \mathbf{X}^S \boldsymbol{\beta}\right\|_2^2 - 2\left[(\mathbf{I} - \mathbf{X}^P (\mathbf{X}^{P \top} \mathbf{X}^P)^{-1} \mathbf{X}^P)\mathbf{y}\right]^\top \mathbf{R} \mathbf{X}^S \boldsymbol{\beta} + \left\|(\mathbf{I} - \mathbf{X}^P (\mathbf{X}^{P \top} \mathbf{X}^P)^{-1} \mathbf{X}^P)\mathbf{y}\right\|_2^2\right]
\end{align}
By properties of permutation matrices:
\begin{equation}
    \mathbb{E}_\mathbf{R}\| \mathbf{R} \mathbf{X}^S \boldsymbol{\beta}\|_2^2 = \|\mathbf{X}^S \boldsymbol{\beta}\|_2^2; \; \mathbb{E}_\mathbf{R} [\mathbf{R}]= \frac{1}{n}\mathds{1}^\top \mathds{1}
\end{equation}
Therefore:
\begin{align}
    \mathrm{MSE}_{\mathrm{misaligned}}
    & = \inf_{\boldsymbol{\beta}} \left[\left\|\mathbf{X}^S \boldsymbol{\beta}\right\|_2^2 - 2\left[(\mathbf{I} - \mathbf{X}^P (\mathbf{X}^{P \top} \mathbf{X}^P)^{-1} \mathbf{X}^P)\mathbf{y}\right]^\top \frac{1}{n}\mathds{1}^\top \mathds{1} \mathbf{X}^S \boldsymbol{\beta} + \left\|(\mathbf{I} - \mathbf{X}^P (\mathbf{X}^{P \top} \mathbf{X}^P)^{-1} \mathbf{X}^P)\mathbf{y}\right\|_2^2\right]
\end{align}
Since $\mathbf{I} - \mathbf{X}^P (\mathbf{X}^{P \top} \mathbf{X}^P)^{-1} \mathbf{X}^P$ projects any vector onto the orthogonal complement of the column space of $\mathbf{X}^P$, the term $\left[(\mathbf{I} - \mathbf{X}^P (\mathbf{X}^{P \top} \mathbf{X}^P)^{-1} \mathbf{X}^P)\mathbf{y}\right]^\top \frac{1}{n}\mathds{1}^\top \mathds{1} \mathbf{X}^S \boldsymbol{\beta} = 0$. Hence:
\begin{align}
    \mathrm{MSE}_{\mathrm{misaligned}}
    & = \inf_{\boldsymbol{\beta}} \left[\left\|\mathbf{X}^S \boldsymbol{\beta}\right\|_2^2 + \left\|(\mathbf{I} - \mathbf{X}^P (\mathbf{X}^{P \top} \mathbf{X}^P)^{-1} \mathbf{X}^P)\mathbf{y}\right\|_2^2\right] \\
    & = \inf_{\boldsymbol{\beta}} \left\|\mathbf{X}^S \boldsymbol{\beta}\right\|_2^2 + \left\|(\mathbf{I} - \mathbf{X}^P (\mathbf{X}^{P \top} \mathbf{X}^P)^{-1} \mathbf{X}^P)\mathbf{y}\right\|_2^2 \\
\end{align}
The minimum is attained at $\boldsymbol{\beta} = \mathbf{0}$, yielding:
\begin{align}
    \mathrm{MSE}_{\mathrm{misaligned}}
    & = \left\|(\mathbf{I} - \mathbf{X}^P (\mathbf{X}^{P \top} \mathbf{X}^P)^{-1} \mathbf{X}^P)\mathbf{y}\right\|_2^2 \\
    & = \inf_{\boldsymbol{\alpha} \in \mathbb{R}^{m^P}, \boldsymbol{\beta} = \mathbf{0}} \left\|\mathbf{y} - \mathbf{X}^P \boldsymbol{\alpha} - \mathbf{X}^S \boldsymbol{\beta}\right\|_2^2 \\
\end{align}
Comparing with Equation~\ref{eq:mse_aligned}, we conclude:
\begin{equation}
    \mathrm{MSE}_{\mathrm{misaligned}} \geq \inf_{\boldsymbol{\alpha} \in \mathbb{R}^{m^P}, \boldsymbol{\beta} \in \mathbb{R}^{m^S}} \left\|\mathbf{y} - \mathbf{X}^P \boldsymbol{\alpha} - \mathbf{X}^S \boldsymbol{\beta}\right\|_2^2 = \mathrm{MSE}_{\mathrm{aligned}}
\end{equation}
\end{proof}





















\subsection{Approximation Capacity of Cluster Sampler}\label{subsec:proof-cluster-sampler}

\begin{definition}[Definition of optimal cluster sampler]
    Assume the inputs are uniformly bounded by some constant $B$. 
    The optimal cluster sampler is defined by the uniform equi-continuous cluster sampler function which achieves the minimal optimization loss for the prediction task in \cref{fig:leal-framework}.
    \begin{equation}
        \textrm{Optimal cluster sampler} := \arginf_{\textrm{Uniform equi-continuous cluster sampler}} \textrm{Loss}(\textrm{cluster sampler})
    \end{equation}
    The cluster sampler is defined over bounded inputs ($|X^P|_{\infty} \leq B, |X^S|_{\infty} \leq B$) from $\mathbb{R}^{m^P} \times \mathbb{R}^{n^S \times m^S}$ and output in $\mathbb{R}^{n^S}$.
\end{definition}

\begin{remark}
    The existence of such optimal cluster sampler is guaranteed by the boundedness and uniform equi-continuity of the set of cluster sampler functions. 
\end{remark}


\thmclustersampler*

\begin{proof}
    We just need to prove the statement for small $\epsilon \leq 6$.

    The input of cluster sampler is $1 \times m^P$ and output is $n^S \times m^S$, the final prediction is to generate a sample probabilities:
    \begin{equation}
        (n^S * m^S, 1 * m^P) \to (n^S * d, 1 * C) \to (n^S * C, 1 * C) \to n^S * 1. 
    \end{equation}

    Also, since there is no weight depends on dimension $n_2$, we can reduce the approximation statement to that there exists trainable weight such that the continuous function $h$ can be approximated:
    \begin{equation}
        (1 * m^S, 1 * m^P) \to (n^S * d, 1 * C) \to (n^S * C, 1 * C) \to 1 * 1. 
    \end{equation}

    Notice that the layer operation of secondary embedding and trainable centroids weights $(C \times d)$ is continuous and the pretrained encoder as a neural network (which is a universal approximator) can approximates any continuous function $f$ composited with inverse embedding. 
    For simplicity, we will consider $m^P = m^S = 1$. 
    For any continuous function $h(p, s) \in [0, 1]$,
    we just need to show there exists trainable weight $\theta_1$, $\theta_2$ such that 
    \begin{equation}
        f(p; \theta_1) \odot g(s; \theta_2) = \sum_{i=1}^C f_i(p; \theta_1) \odot g_i(s; \theta_2). 
    \end{equation}
    Here $f(p; \theta_1) \in \mathbb{R}^C$ is a function of $p$ parameterized by $\theta_1$ and $g(s; \theta_1) \in \mathbb{R}^C$ is a function of $s$ parameterized by $\theta_2$.  
    As any continuous function $f(p, s)$ has a corresponding Taylor series expansion, it means for any $\epsilon > 0$, there exists $C$ which depends on error $\epsilon$ such that
    \begin{equation}
        \sup_p \sup_s |h(p, s) -\sum_{i=1}^C pol_{1,i}(p) pol_{2,i}(s)| \leq \frac{\epsilon}{2}. 
    \end{equation}
    Furthermore, as polynomial functions are continuous function, therefore $f_i$ can be used to approximate the polynomial function $pol_{1, i}$ and $g$ can be used to approximate the polynomial function $pol_{2, i}$.
    \begin{align}
        \sup_p |pol_{1,i}(p) - f_i(p; \theta_1)| & \leq \frac{\epsilon}{6B} \\ 
        \sup_s |pol_{2,i}(s) - g_i(s; \theta_2)| & \leq \frac{\epsilon}{6B}. 
    \end{align}
    Here $B := \max(1, \sup_p \max_{i} |pol_{1, i}(p)|, \sup_s \max_{i} |pol_{2, i}(s)|).$ 
    We show that the cluster sampler is capable to approximate any desirable continuous cluster sampler. 
    \begin{equation}
        \sup_p \sup_s |h(p, s) -\sum_{i=1}^C f_i(p; \theta_1) g_i(s; \theta_2)| \leq \frac{\epsilon}{2} + \frac{\epsilon}{6B} * B + \frac{\epsilon}{6B} (B + \frac{\epsilon}{6B}) = \frac{5}{6} \epsilon + \frac{\epsilon^2}{36B^2} < \epsilon. 
    \end{equation}
    The last inequality comes from $\epsilon < 6$. 
    The universal approximation capacity of the cluster sampler is proved. 
\end{proof}

\begin{remark}
    Since we are working with a cluster sampler with specific manually designed structure, it mainly comes from the fact the student's t-kernel introduce a suitable implicit bias to more efficiently learn the cluster sample probability $(n_2 \times 1)$. 
\end{remark}




\end{document}
