\section{Exploratory User Study}\label{sec:user-study}
\begin{figure*}
    \centering
    \includegraphics[width=\linewidth]{Figures/BaselineInSitu_CameraReady.pdf}
    \caption{\textit{In-situ} user interface of the baseline system. \textsc{(A)} The menu button (highlighted by a \textcolor[HTML]{F1B302}{yellow} arrow), which is the iOS \textit{AssistiveTouch} button; 
    \textsc{(B)} The menu where users can easily access screenshot, audio note, and screen recording features; \textsc{(C)} Interface for annotating screenshots; \textsc{(D)} Interface for attaching typed notes to screenshots or recordings; \textsc{(E)} Interface for recording audio notes. \textit{Note: the interface for screen recordings is not shown, as the UI is simply a red dot in the upper right corner.}}
    \label{fig:insitu-baseline-system-overview}
    %
\end{figure*}

In this study, we aimed to explore how paired users co-author site-specific AR experiences using \SystemName compared to a typical authoring workflow. In addition to assessing effectiveness and user experience, we sought to identify key directions for future research.

We employed a within-subjects counterbalanced design, comparing \textit{two collaboration modes}: \async (baseline) and \sync (\SystemName, ours). For each session, we assigned one participant to the \exsitu role and the other to the \insitu role, based on their background experience. Participants were compensated with gift cards valued at £40 for their study participation, which lasted around two hours.

To control for order and environmental effects, we counterbalanced both \textit{collaboration mode} (\sync or \async) and \textit{location} (\locA or \locB). Each pair began by completing the task using either the \sync or \async \textit{collaboration mode}, with the \insitu user physically at either \locA or \locB, while the \exsitu user participated remotely. They then completed the task for a second time using the alternate \textit{collaboration mode}, with the \insitu user at the alternate \textit{location}.

We focused on evaluating the below hypotheses, which emerged based on the formative study and prior work~\cite{guoBlocksCollaborativePersistent2019,kraussCurrentPracticesChallenges2021,walkerExperiencingFlowDoing2010,ngSituatedGameLevel2018}:

\begin{hypothesis}\label{h1}
\Exsitu users will report a \textbf{lower} level of task load for \sync{} compared to \async{}.
\end{hypothesis}

\begin{hypothesis}\label{h2}
\Insitu users will report a \textbf{higher} level of task load for \sync{} compared to \async{}.
\end{hypothesis}

\begin{hypothesis}\label{h3}
\Exsitu and \insitu users will report a \textbf{higher} level of engagement for \sync{} compared to \async{}.
\end{hypothesis}

\begin{hypothesis}\label{h4}
\Exsitu users will report a \textbf{higher} level of confidence in authored experiences for \sync{} compared to \async{}.
\end{hypothesis}

\subsection{Baseline}
Our baseline condition, \async, was designed to reflect the typical workflow identified in the formative study (\cref{sec:formative-study}). In this workflow, developers commonly receive or create a set of notes, recordings, and screenshots with insights on improving the AR experience. To align with this, we included these elements in the baseline condition. In our study, the \insitu participant gathered this feedback using a series of custom iPhone \textit{shortcuts}\cite{appleShortcutsUserGuide2024}, activated via the iOS \textit{AssistiveTouch} button. These shortcuts allowed for the creation of voice recordings, annotated screenshots, and screen recordings of the AR application, each automatically saved to the iOS \textit{Notes} app. The user interface of this baseline system is shown in \cref{fig:insitu-baseline-system-overview}.

Once the \insitu participant finished gathering information about the experience, the notes were sent to the \exsitu participant. To recreate a typical developer environment, the baseline system for the \exsitu user was based on Unity~\cite{unity} and \textit{Niantic SDK for Unity}~\cite{lightship-ardk-niantic}. The Unity editor contained the \locMesh, an initial prototype of a site-specific AR experience, and a set of sample assets, including objects and materials from the public domain\footnote{\url{https://kenney.nl/assets/}}.
More information on the study procedure is available in \cref{sec:procedure}.

\subsection{Participants}
We recruited 32 participants (16 pairs) through internal mailing lists and social media platforms. Participants self-reported their gender, with 19 identifying as men and 13 as women. The mean age of participants was 30.16 years (SD = 9.78). Participants were primarily students and researchers from fields related to computer science, virtual and augmented reality, and interactive media, with a few professionals from sectors such as government, consulting, and game development.
For \insitu participants, no prior experience was necessary. For \exsitu participants, basic Unity experience was required. To increase our participant pool and obtain a broad range of experience in Unity---similar to the demographics identified in our formative study---we enabled \exsitu participants to participate remotely from anywhere in the world, using a high-speed internet connection. All participants collaborated in English.

Given the exploratory nature of our study, we selected a sample of 32 participants, consistent with related research on collaborative and authoring systems~\cite{thoravikumaravelTransceiVRBridgingAsymmetrical2020, thoravikumaravelLokiFacilitatingRemote2019, nebelingXRDirectorRoleBasedCollaborative2020}. Moreover, the requirement for participants with Unity experience imposed recruitment constraints, limiting the feasibility of a larger sample~\cite{lakensSampleSizeJustification2022}. Given the study’s scale and exploratory nature, the statistical results should be interpreted with due consideration of potential biases and limitations, as outlined in \cref{sec:discussion:limitations}.

\begin{figure*}
    \centering
    \includegraphics[width=\linewidth]{Figures/Procedure_CameraReady.pdf}
    \caption{Overview of the study procedure. The process begins with a general introduction (left), followed by the task phase, performed under two conditions (\sync (i.e., \SystemName) and \async), and concludes with a post-study interview (right). \textcolor[HTML]{6DD268}{Green} represents the procedure components of the \insitu participants, \textcolor[HTML]{67B3E6}{blue} represents the procedure components of the \exsitu participants, and \textcolor[HTML]{6e6e6e}{grey} blocks represent components completed by both. \ding{202} represents Phase 1 (Feedback Collection \& Refinement) and \ding{203} represents Phase 2 (Ideation \& Prototyping) of the task. The arrows represent the exchange of task-related information across pairs. Note: both \textit{collaboration modes} (\sync and \async) and \textit{locations} were counterbalanced across pairs.}
    \label{fig:study-procedure}
    %
\end{figure*}

\subsection{Task}
The task involved a two-part collaborative process aimed at improving and expanding predefined site-specific outdoor AR prototype experiences, which are described in \cref{sec:prototype}. 

In \textit{Phase 1: \textbf{Feedback Collection \& Refinement}}, participants focused on identifying and addressing issues within the prototype. The \insitu participant primarily gathered feedback and contextual insights through real-world interaction with the prototype, while the \exsitu participant refined the prototype by implementing necessary corrections based on the feedback provided. In the \async condition, the feedback gathered by the \insitu participant was available to the \exsitu participant \textit{after} the feedback collection phase. To accommodate this order of actions in the \async condition, the \exsitu participant had a short break prior to starting this phase while the \insitu participant gathered feedback. In the \sync condition, this process occurred synchronously.

In \textit{Phase 2: \textbf{Ideation \& Prototyping}}, which lasted six minutes, participants transitioned to brainstorming and implementing extensions or enhancements to the prototype. During the \async condition, the \insitu participant ideated independently, noting ideas and gathering relevant contextual artifacts, which were then shared with the \exsitu participant for prototyping. In contrast, in the \sync condition, ideation and prototyping occurred synchronously.

\subsubsection{Prototype AR experiences}\label{sec:prototype} The prototype experiences used in the study were designed based on the environmental characteristics of the two selected outdoor locations (\locA and \locB), informed by both the \locMesh and supplementary information from \textit{Google Maps} and \textit{Google Street View}. To maintain consistency across locations and ensure generalizability to other site-specific AR applications, the prototypes adhered to a set of predefined design dimensions. These dimensions, grounded in prior work and insights from our formative study, were applied to create similar AR prototypes for each location, ensuring participants could address comparable issues across both locations.

\begin{enumerate}
    \item \textbf{Design Patterns} \textit{used as design guidelines for the representation of each AR element, based on~\cite{leeDesignPatternsSituated2023}:} Glyphs (navigation elements, indicative of actions with spatial alignment to referents), Decals (information related to referent context), Trajectories (\eg, paths, outlines, arrows), Labels (contextual annotations), Ghosts (overlays linked to referents), and Audio (non-visual cues providing additional context).
    
    \item \textbf{Referent Types} \textit{used as design guidelines for the physical referent of each AR element}: Small objects (\eg, mobile elements such as signs or markers), Large objects (\eg, statues, lampposts, or trees), Large planes (\eg, open spaces or flat surfaces), and Building structures (static physical structures with specific entry points or facades).
    
    \item \textbf{Alignment Types} \textit{used as design guidelines for the placement of each AR element}: Overlap (directly aligned with referents), Proximity (placed relative to nearby referents), and Surface (mapped directly onto physical surfaces).
    
    \item \textbf{Issue Types} \textit{used as a design guideline to incorporate common design issues, based on the five most prominent issues identified in \cref{sec:formative-study}:} Physical constraints, User safety, Misaligned AR elements, User context, and User flow, as visualized in \cref{fig:issue-types}.
\end{enumerate}

Both selected locations were within walking distance of University College London (\locA and \locB). \locA, located near a major transit station, featured high foot traffic, a nearby road, and elevated levels of environmental noise. The theme of the AR experience at \locA was ``\textit{Welcoming Tourists to the City},'' reflecting its proximity to a transit hub. \locB was situated along a narrow street with a local cafe, characterized by significant foot and bicycle traffic and lower environmental noise. The theme for this location was ``\textit{Promoting and Celebrating the Upcoming Cookie Party},'' aligning with the atmosphere of nearby restaurants and cafes. For each location, seven issues covering the range of issue types, as previously defined, were identified and verified through on-site testing prior to the user study. An overview of the prototypes is shown in \cref{fig:study-prototypes}, with detailed descriptions available in the Supplementary Materials.

\begin{figure*}
    \centering
    \includegraphics[width=\linewidth]{Figures/StudyPrototypes.pdf}
    \caption{Screenshots of the prototypes designed for \locA and \locB. The \locMesh is outlined in orange, to which the visible virtual low-poly AR elements are anchored.}
    \label{fig:study-prototypes}
    %
\end{figure*}

To offer both participants hints regarding issues in the prototype, both \insitu and \exsitu participants were provided with a list of seven \textit{task clues} in Phase 1. This list contained fictional observations made by ``early testers'' of the prototype experience, offering participants a semi-guided path through the experience while still requiring further investigation to ensure consistency across pairs. Both task clue lists are available in the Supplementary Materials. Participants were allowed to start at any point on the list and could skip or revisit clues as they desired.

\subsection{Procedure}\label{sec:procedure}
An overview of the study procedure is shown in \cref{fig:study-procedure}. Prior to the study session, participants completed a questionnaire covering demographics and their background in smartphone-based AR, 3D game engines, and 3D editors.

For each session, the \insitu participant took part in person using an iPhone 13 Pro, while the \exsitu participant participated remotely by connecting to a PC with all necessary tools pre-installed via the low-latency screen-sharing software \textit{Parsec}\footnote{\url{https://parsec.app}}.

Upon arrival, the \insitu participant was welcomed by an experimenter, while another experimenter connected with the \exsitu participant via video call. The \insitu participant was then brought to a room where they joined the same video call. The experimenter provided a general introduction to the study context and procedure. Before the first condition, the \exsitu participant was guided through a setup document containing log-in information and instructions for forwarding their microphone to the PC. Once the \exsitu participant connected, latency statistics were recorded, ranging between ${\sim}35$–$120\text{ms}$ (including network, decoding, and encoding latency) across participants.

Each pair completed the task twice: once under the \sync and once under the \async \textit{collaboration mode} conditions, with one condition occurring in \locA and the other in \locB. At the start of each condition, participants watched a tutorial video tailored to their role and condition. The \insitu participant then proceeded to either \locA or \locB, assisted by an experimenter, where they received a brief walkthrough of the \sync or \async system. Meanwhile, the \exsitu participant received a similar walkthrough remotely, along with an introduction to the prototype experience, providing background information on its components and initial design decisions.

In the \async condition, during Phase 1 (Feedback Collection \& Refinement), the \insitu participant had ten minutes to gather feedback on the prototype experience and any relevant real-world context necessary to address the issues listed on the task clue list. The \insitu participant then proceeded to Phase 2 (Ideation \& Prototyping), where they collected artifacts related to new ideas for improving or extending the prototype. Concurrently, the \exsitu participant began Phase 1 by refining and correcting the prototype based on the other participant's generated feedback artifacts. After six minutes, the \insitu participant's Phase 2 concluded, and their artifacts were made available to the \exsitu participant for Phase 2, which commenced after their refinement phase (Phase 1) was completed. The \insitu participant then returned indoors, guided by the experimenter, and both participants completed post-task questionnaires separately.

The same process was followed for the \sync condition, except that the \insitu and \exsitu participants progressed through both phases simultaneously. Upon completing both conditions, participants were separately interviewed to discuss their experiences across the conditions.

\subsection{Measures and Analysis}
Each collaborative session was recorded from the perspective of the \exsitu participant. In the \sync condition, this recording included the streamed view of the \insitu participant along with the audio captured from both participants. The recordings were coded to track each participant's feature usage throughout the session. An overview of feature usage is provided in the Supplementary Materials and is discussed in \cref{sec:discussion:environmental_context}.

The post-task questionnaire included the NASA Task Load Index (TLX)~\cite{hartDevelopmentNASATLXTask1988} to assess workload across conditions using a 7-point Likert scale. All TLX items were included for \insitu participants, whereas the item measuring physical demand (``How physically demanding was the task?'') was excluded for \exsitu participants due to their seated, computer-based role in both conditions. An overall task load score was calculated by averaging the responses to the included items.

To measure engagement, we employed the short-form version of the User Engagement Scale (UES-SF)~\cite{obrienPracticalApproachMeasuring2018}, focusing on the subscales of \textit{focused attention}, \textit{perceived usability}, and \textit{reward}. The \textit{aesthetic appeal} subscale was omitted as it was deemed irrelevant. An overall engagement score was obtained by averaging the 5-point Likert scale responses across subscales, adjusting for reverse-coded items.

Additionally, we included several custom items in the post-task questionnaire, which are provided in the Supplementary Materials. For \exsitu participants, custom items assessed their \textit{overall confidence} in the final AR experience as end-users would encounter it \insitu[ ], confidence in the successful execution of task components that influence end-user experience, and specific feedback on the goals of the different task phases. Both participant groups also rated perceived overall task performance using a custom item.

Before concluding the study, both participants were interviewed separately. The post-study interviews focused on evaluating their experiences and preferences regarding the systems and workflows in the \sync and \async conditions. Key topics included overall preference, confidence, communication quality, advantages and limitations of each method, and suggestions for improving system features. The full interview scripts are available in the Supplementary Materials. Two authors of this paper used an affinity diagramming approach to analyze and synthesize themes from the interview transcripts.
