\section{Discussion}\label{sec:discussion}
In this section, we reflect on the results of the user study, discuss our findings in the context of our formative study and related work, and offer recommendations for future authoring systems of site-specific outdoor AR experiences.

\subsection{Enhancing Environmental Understanding Through Capture and Communication Tools}\label{sec:discussion:environmental_context}
Our findings indicate that synchronous collaboration using \SystemName significantly enhanced the ability of \exsitu participants to integrate real-world context into their design process. In particular, \exsitu participants in the synchronous condition demonstrated a more comprehensive understanding of environmental dynamics, such as user flow and spatial relationships. In this subsection, we contextualize how each feature group contributed to this outcome, drawing on insights from post-study interviews and observations made when coding feature usage counts (see Supplementary Materials for plots).

Participant feedback emphasized that real-time video and audio communication was the most critical feature of \SystemName. It provided \exsitu participants with immediate visual context from the \insitu user’s perspective and supported stable communication. This was especially valuable when \insitu participants felt overwhelmed, enabling them to carry out the task nonetheless. Notably, background noise captured by the \insitu user’s microphone proved useful for decisions regarding audio components (\cref{sec:results:interviews:real-world_context}) --- an issue that, according to P1 of our formative study, could only be resolved when \insitu[ ] (\cref{sec:formative-study:missing-contextual-info}). Additionally, some \exsitu participants mentioned that background audio increased their sense of presence in the \insitu environment.

\textit{Surface Draw} was commonly used due to its task relevance, as objects were typically tied to physical referents (\eg, the map on the wall in \cref{fig:example-overview}B). \textit{Air Draw} was primarily used to indicate user paths and, occasionally, for the creation of 3D wireframe placeholder objects during brainstorming. Labeled annotations were used less often than color annotations, likely because \exsitu participants were not expected to revisit annotations asynchronously at a later stage during the study, unlike in real-world scenarios where annotations might be revisited over time.

\textit{Coarse Mesh} was often used to capture large areas (e.g., buildings in \cref{fig:example-overview}C), while the \textit{3D Snapshot} feature was favored for specific points of interest (e.g., a curb and table set in \cref{fig:example-overview}A). Although the number of \textit{3D Snapshot} captures was higher overall, the longer capture time required for \textit{Coarse Meshes} balanced the frequency of their use. The lack of color in \textit{Coarse Meshes} was not described as a major issue, though one participant creatively captured over ten \textit{3D Snapshots} to obtain a large colored point cloud. \textit{3D Cursors} were mainly used by \insitu participants for object alignment (\cref{fig:example-overview}D) and for supporting deictic references, while \exsitu participants used \textit{3D Cursors} to guide \insitu users' movements or capture actions.

\paragraph{\textbf{Recommendations for future work}}
Enhancing spatial representations and integrating richer metadata layers could expand \textsc{CoCreatAR}'s potential for complex real-world locations. Future work could leverage 3D Gaussian splatting techniques~\cite{kerbl3DGaussianSplatting2023} and their temporal extensions~\cite{wu4DGaussianSplatting2024,mihajlovicSplatFieldsNeuralGaussian2024} to support more intricate design tasks and precise alignment. Automatically capturing metadata such as pedestrian flow~\cite{sindagiGeneratingHighQualityCrowd2017} or hazards~\cite{suRASSARRoomAccessibility2024} during on-site visits could streamline context handling for both synchronous and asynchronous \exsitu authoring.

Future work could also leverage insights from \citet{fussellCoordinationCommunicationEffects2000} to enhance collaboration in \SystemName. For instance, visualizing collaborators' actions, such as object selection or transformation, could improve task awareness. Alternative \insitu interfaces, including head-mounted displays with larger fields of view, might offer a more comprehensive perspective and facilitate natural interactions. Furthermore, hand tracking could enable intuitive gestures, as observed in our study when participants occasionally employed deictic gestures in front of the device camera~\cite{kimWorldPointFingerPointing2023,fussellGesturesVideoStreams2004}.

\subsection{Task Load and Multitasking Challenges}
While our results regarding task load are inconclusive, they offer early insights into the impact of task context, multitasking demands, and individual experiences. A slight trend visible in \cref{fig:engagement-task_load-plot}B suggests a higher task load in the \sync condition for both roles, although no statistically significant differences were observed across conditions. Notably, qualitative feedback revealed that a subset of participants, particularly those with less prior experience, expressed feeling overwhelmed during the \sync{} condition. This feedback provides preliminary insights that challenge \textbf{Hypothesis~\ref{h1}}, suggesting that \exsitu{} participants may experience higher, rather than lower, task load in the \sync{} condition. At the same time, qualitative observations align with \textbf{Hypothesis~\ref{h2}}, as \insitu{} participants reported multitasking challenges indicative of higher task load in the \sync{} condition. Based on the interviews, we conclude that an overall increase in task load during the \sync{} condition could have arisen from the substantial volume of information flow combined with the demands of concurrent coordination and communication. For \insitu users, this challenge appeared compounded by the need to safely navigate the real world---an activity previously recognized as inherently demanding \cite{makhmutovSafetyRisksLocationBased2021}.

Some participants reported that limited familiarity with the system contributed to their sense of being overwhelmed. We hypothesize that these experiences may be influenced by participants' background knowledge and individual personality traits~\cite{johnBigFiveTrait1999}. As we detail in \cref{sec:discussion:limitations}, follow-up research could explore these hypotheses further in larger-scale studies.

Participants' qualitative feedback also highlighted potential \textit{process losses}, defined as productivity reductions that occur when individuals collaborate synchronously rather than working independently~\cite{steinerGroupProcessProductivity1972}. Specifically, some participants reported that the multitasking demands of the \sync{} condition, such as communicating with their collaborator while performing tasks, led to perceived inefficiencies. Related challenges included waiting for collaborators to complete tasks (\eg, annotation by an \insitu{} user or scene editing by an \exsitu{} user) and feeling pressured to avoid blocking their partner's progress (\cref{sec:results:interviews:mult-tasking}). 
\citet{guoBlocksCollaborativePersistent2019} similarly observed potential process losses in synchronous collaboration during an authoring task, specifically noting that skill mismatches within pairs could limit creativity. While we did not observe direct evidence of this particular form of process loss, we theorize that the clearly defined roles and responsibilities assigned to participants, along with the semi-guided nature of the task, may have mitigated these effects.

\paragraph{\textbf{Recommendations for future work}}
Future work could benefit from more extensive onboarding processes to familiarize users with system features and clarify collaborators' roles. Additionally, future iterations of \SystemName could address challenges related to task load by streamlining workflows. For instance, simplified scene editing for \exsitu{} users could be achieved through modular templates~\cite{jurgelionisShapesMarblesPebbles2012} or AI-driven authoring tools~\cite{seeligerContextAdaptiveVisualCues2024,qianScalARAuthoringSemantically2022,evangelistabeloAUITAdaptiveUser2022}. In parallel, \insitu{} users could benefit from AI-assisted annotation methods, such as transcribed audio notes~\cite{langlotzAudioStickiesVisuallyguided2013,kimWinderLinkingSpeech2021}.

To improve orientation and reduce cognitive load for \insitu{} users in complex environments, future systems might integrate visual aids such as mini-maps~\cite{stoakleyVirtualRealityWIM1995} or directional cues~\cite{leeUserPreferenceNavigation2022}, in addition to enhanced visual indicators for collaborator actions as noted in \cref{sec:discussion:environmental_context}. Lastly, for \exsitu{} users, features such as user-perspective scene rendering~\cite{baricevicHandheldARMagic2012} and selective visualization of information layers~\cite{kerstenUsingTaskContext2006,veasExtendedOverviewTechniques2012} could help reduce visual overload.

\subsection{Engagement and Creative Exploration}\label{sec:discussion:engagement}
Both \exsitu and \insitu participants reported higher engagement in the \sync condition, supporting \textbf{Hypothesis~\ref{h3}}. Real-time interaction supported effective teamwork and creative exploration, which the vast majority of participants described as more enjoyable than the \async workflow. This finding aligns with prior research on \textit{social flow}, which found that interdependent team settings amplify engagement compared to solitary workflows~\cite{walkerExperiencingFlowDoing2010}.

Prior work on \insitu AR game level editing \cite{ngSituatedGameLevel2018} found that users were highly engaged in both creating AR content and observing others interact with it. Similarly, in our study, \insitu users were engaged by experiencing and interacting with the AR content as it materialized in their environment, while \exsitu users likened the process to live game testing, emphasizing how immediate \insitu feedback served as a source of inspiration (\cref{sec:results:interviews:engagement}).

The advantages of synchronous collaboration observed in our study are further supported by findings from \citet{guoBlocksCollaborativePersistent2019}, who identified synchronous authoring as a key driver of engagement and creativity compared to asynchronous authoring. Notably, their work additionally revealed a user preference for \textit{co-located} over \textit{remote} synchronous collaborative authoring, which relied on standard video conferencing. This preference underscores the importance of designing remote collaboration tools that replicate the dynamics of co-located experiences more closely. For instance, I3 expressed a desire to see their collaborator, particularly when working together for the first time.

\paragraph{\textbf{Recommendations for future work}}
While \SystemName currently employs avatars to simulate the embodied perspective of co-located collaboration, future research could investigate enhancing interpersonal connection through spatial video-based avatars, as demonstrated in recent work~\cite{vanukuruDualStreamSpatiallySharing2023,qianChatDirectorEnhancingVideo2024}. Another avenue for future research is to explore how creativity can be supported in larger groups, which offer potential for richer creative outcomes~\cite{paulusGroupCreativityInnovation2003} and greater productivity. Although larger groups may face greater coordination overhead, strategies such as role specialization, as employed by XRDirector~\cite{nebelingXRDirectorRoleBasedCollaborative2020}, could streamline the workflows of larger teams.

\subsection{Confidence in End-User Experience Through In-Situ Feedback}
Our findings demonstrate that participants in the \sync condition reported significantly higher confidence in the authored AR experiences than those in the \async condition, supporting \textbf{Hypothesis~\ref{h4}}. Participant feedback indicated that this confidence stemmed from the ability to iteratively verify the designed experience in real time, ensuring alignment with the spatial and contextual dynamics of the real-world environment. In contrast, participants in the \async condition struggled with uncertainty due to reliance on static data and delayed or incomplete feedback, which often led to hesitation in their design decisions.

This positive influence of representative \insitu feedback aligns with prior research in ubiquitous computing that emphasized the critical role of situated evaluation in achieving a reliable understanding of how applications behave in real-world contexts~\cite{rogersWhyItsWorth2007,crabtreeIntroductionSpecialIssue2013}, an approach that our work both embraces and extends.

Notably, much of this research has focused on asynchronous techniques, such as event logging and recordings, to collect \insitu feedback~\cite{rogersWhyItsWorth2007,nogueiraEffectivenessEmbodiedEvaluation2023}. Our findings contribute a new perspective by highlighting the opportunity and value of evaluating and adapting an application simultaneously, a method that our user study revealed to be particularly effective for the iterative refinement of site-specific outdoor AR experiences.

\paragraph{\textbf{Recommendations for future work}}
Future work could explore how author confidence and the perceived quality of AR end-user experiences evolve over multiple revisits of a particular site, especially under dynamic environmental conditions. Additionally, research could focus on developing quantitative metrics that represent the quality of AR experiences, as explored by ARCHIE~\cite{lehmanARCHIEUserFocusedFramework2020}, to provide authors with additional objective insights. While quantitative metrics typically require controlled environments or ground truth sources, metrics derived from user behaviors (e.g., frequency of manual relocalization triggers) could be a practical alternative. Finally, author confidence could be supported by incorporating constraints for contextual adaptation, such as semantic referents and rules~\cite{qianScalARAuthoringSemantically2022,unityMars}, which have been shown to improve confidence when used alongside virtual simulation environments for indoor AR tutorial authoring~\cite{qianScalARAuthoringSemantically2022}.

\subsection{Enabling Hybrid Approaches to AR Experience Authoring}
Despite the advantages of synchronous collaboration, participants identified some scenarios where asynchronous workflows were desired. In contexts with overwhelming environmental conditions or when tasks did not require immediate feedback, asynchronous methods offered more flexibility and reduced pressure. This insight underscores the importance of providing flexible authoring workflows that can adapt to the diverse needs of authoring teams~\cite{kraussCurrentPracticesChallenges2021}, including collaboration across \textit{place} and \textit{time}~\cite{johansenGroupWareComputerSupport1988}.

\paragraph{\textbf{Recommendations for future work}}
Studying the impact of switching between synchronous and asynchronous modes on collaborative processes and authoring outcomes could reveal opportunities for optimization. Furthermore, to support flexible workflows, future work could explore designing hybrid systems that integrate session management and version control for iterative collaborative authoring~\cite{zhangVRGitVersionControl2023,xiaSpacetimeEnablingFluid2018}. Additionally, techniques to visualize synchronous collaboration sessions as artifacts for asynchronous follow-up work could increase and prolong the value of collaborative sessions~\cite{irlittiChallengesAsynchronousCollaboration2016,wangAgainTogetherSocially2020,choRealityReplayDetectingReplaying2023,wangMeetingBridgesDesigning2024}.

\section{Limitations}\label{sec:discussion:limitations}
While our study provides valuable early insights into synchronous and asynchronous collaborative authoring for site-specific outdoor AR experiences, several limitations should be noted.

First, there was variability between the two locations used in the study, which differed in environmental characteristics such as foot traffic, noise levels, and spatial layout. Although we counterbalanced the assignment of locations to conditions, maintained a consistent task across locations, and did not identify a major impact of location in our analyses, these differences may still have influenced participants' experiences and performance.

Second, the sample size of our user study was small, with 32 participants forming 16 pairs. This limited the power of our statistical analysis to detect certain effects. For instance, while qualitative data indicated that some novice participants reported increased task load, this observation may not have been fully reflected in the quantitative data due to the small sample and variability in participants' experience levels. These findings highlight the need for future research with larger sample sizes and specific target groups to examine the role of prior experience and explore strategies for lowering barriers to AR authoring, in line with the recommendations of \citet{ashtariCreatingAugmentedVirtual2020}.

Third, the novelty of the system compared to current methods may have led to participant response bias~\cite{dellYoursBetterParticipant2012}, potentially influencing measures such as engagement. Longitudinal studies with extended and repeated exposure could help isolate these effects.

Finally, our study was limited to a specific type of site-specific AR experience and a particular set of tasks. Consequently, the findings may not generalize to other types and scales of AR application development, such as those involving larger teams or multiple co-located users collaborating on extensive projects. Future research could examine a broader range of AR authoring scenarios that vary in task type and group size, including those with multiple \insitu and \exsitu users engaged in synchronous distributed authoring (e.g., guided tours over larger areas) and environmental exploration.
