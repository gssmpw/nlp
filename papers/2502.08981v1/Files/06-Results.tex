\section{Exploratory User Study Results}
Building on the formative study and resulting requirements, we present the results of our evaluation of \SystemName following the user study design detailed in \cref{sec:user-study}. We first report the self-reported questionnaire results, followed by four themes derived from a thematic analysis of interviews with \exsitu and \insitu participants. Plots that show an overview of the usage of specific \SystemName features are provided in the Supplementary Materials.

\subsection{Questionnaire Results}
We conducted statistical analyses using linear mixed-effects models, accounting for participant variability as a random effect and controlling for confounding factors such as task location and prior experience. All models converged, and model assumptions were verified (linearity, normality, and homoscedasticity). Where applicable, we report the estimated coefficients ($\beta$), standard errors (SE), p-values ($p$), and 95\% confidence intervals (CI). Unless stated otherwise, each analysis includes 32 observations ($n$=32). In this section, we provide full details for (marginally) statistically significant results, whereas an overview of all statistical results is available in the Supplementary Materials.

Cronbach's alpha values indicated acceptable internal consistency for both engagement and task load measures. Engagement items yielded alpha values of $0.740$ for \insitu users and $0.764$ for \exsitu users. Task load items had alpha values of $0.812$ and $0.772$ for \insitu and \exsitu users, respectively.

\subsubsection{Engagement}
The analysis revealed a significant effect of \textit{collaboration mode} on self-reported engagement for both \exsitu and \insitu participants. \Exsitu participants reported significantly higher engagement in the \sync condition ($\beta = 0.500$, SE = $0.140$, $p < 0.001$, 95\% CI [$0.226$, $0.774$]). A similar effect was observed for \insitu participants, who reported significantly increased engagement in the \sync condition ($\beta = 0.340$, SE = $0.130$, $p = 0.009$, 95\% CI [$0.085$, $0.595$]). A plot of engagement scores is shown in \cref{fig:engagement-task_load-plot}A.

While no significant effect of \textit{location} was found for \exsitu participants ($p = 0.196$), \insitu participants reported significantly higher engagement at \locB ($\beta = 0.299$, SE = $0.130$, $p = 0.022$, 95\% CI [$0.044$, $0.554$]). \textit{Prior experience} with AR or 3D editing tools did not significantly influence self-reported engagement for either group (\exsitu: $p = 0.093$; \insitu: $p = 0.231$).

\subsubsection{Task load}
\begin{figure}
    \centering
    \includegraphics[width=\linewidth]{Figures/Plots/EngagementScoresAndTaskLoadScores_CameraReady.pdf}
    \caption{\textsf{(A)} Engagement scores for each role per condition, with significantly higher engagement in the \sync condition for both \insitu and \exsitu participants. \textsf{(B)} Task load scores for each role (\exsitu and \insitu) per condition. No significant differences were found between conditions.}
    \label{fig:engagement-task_load-plot}
    %
\end{figure}

Self-reported task load showed no significant difference between \sync and \async \textit{collaboration modes} for either \exsitu ($\beta = 0.313$, SE = $0.309$, $p = 0.312$) or \insitu participants ($\beta = 0.213$, SE = $0.345$, $p = 0.538$). A plot of task load scores is shown in \cref{fig:engagement-task_load-plot}B.

\textit{Location} did not significantly impact self-reported task load for \exsitu participants ($p = 0.419$). However, \insitu participants reported a marginally significant decrease in task load at \locB compared to \locA ($\beta = -0.637$, SE = $0.345$, $p = 0.065$, 95\% CI [$-1.314$, $0.039$]). \textit{Prior experience} did not significantly affect self-reported task load for \exsitu participants ($p = 0.162$). However, \insitu participants showed a marginally significant association between lower task load and more AR experience ($\beta = -0.320$, SE = $0.176$, $p = 0.070$, 95\% CI [$-0.666$, $0.026$]).


\begin{figure}
    \centering
    \includegraphics[width=\linewidth]{Figures/Plots/TaskFixing_CameraReady.pdf}
    \caption{Stacked bar plot illustrating response proportions where participants indicated perceived performance and confidence for each task component across \textit{collaboration mode} and \textit{location} combinations.}
    \label{fig:confidence-task-plot}
    %
\end{figure}

\subsubsection{Confidence in authored results}
Confidence in the number of issues fixed and overall self-reported confidence were both significantly influenced by \textit{collaboration mode}. Participants in the \sync condition reported fixing more issues with confidence than those in the \async condition ($\beta = 3.000$, SE = $0.373$, $p < 0.001$, 95\% CI [$2.270$, $3.730$], $n=24$). Additionally, overall confidence scores were significantly higher in the \sync condition ($\beta = 1.125$, SE = $0.387$, $p = 0.004$, 95\% CI [$0.367$, $1.883$]). An overview of the proportion of participant responses regarding perceived performance and confidence for each task component is shown in \cref{fig:confidence-task-plot}.

A marginal effect of \textit{location} was observed for \exsitu participants, with fewer issues confidently fixed at \locB ($\beta = -0.667$, SE = $0.373$, $p = 0.074$, 95\% CI [$-1.397$, $0.064$], $n=24$). However, \textit{location} did not significantly affect overall confidence scores ($p = 0.747$). \textit{Prior experience} did not significantly impact the number of confidently fixed issues ($p = 0.096$) or overall confidence scores ($p = 0.466$).

\subsection{Interview Results}
At the start of the interview, each participant was asked which system they would prefer to use if they were to perform a similar task again. Among \insitu participants, the majority (12 out of 16) expressed a preference for the \sync condition, citing advantages such as real-time feedback, improved communication, and enhanced engagement. Two participants indicated that their preference would depend on the scenario, while two favored the \async condition, highlighting the ability to focus better without the distractions of real-time interaction.

Similarly, \exsitu participants also largely favored the \sync condition (12 out of 16), with many emphasizing the value of real-time collaboration and immediate feedback. Three participants indicated that their preference depended on the context, and one favored the \async condition due to the slower pace, which allowed more time to refine their contributions.

Our thematic analysis of interview transcripts resulted in four main themes comparing the synchronous \SystemName and asynchronous baseline experiences. We denote \insitu participant quotes with an `I' (\eg, I2) and \exsitu participants with an `E' (\eg, E8).

\subsubsection{\textbf{Collaboration with \insitu users facilitated the integration of real-world context into \exsitu users' design process.}}\label{sec:results:interviews:real-world_context}
In the \async condition, all \exsitu participants struggled to integrate real-world context into their design decisions due to missing information.
\Exsitu users were uncertain about object placement and its interaction with the environment, as they could not verify details without live input. One participant noted, ``I'm not 100\% sure about the real way, if it's low enough or is this blocking other ways because I need to see it from the real end'' (E7). Lacking live visual and spatial information, \exsitu participants relied on limited artifacts provided by \insitu participants, restricting their ability to account for dynamic environmental elements. One participant explained, ``I tried to put the map onto that light post but I'm not sure whether that's on the post or not'' (E7).

In contrast, the \sync condition enabled participants to gain a clearer understanding of real-world context, leading to more informed decision-making in the design process (E12, E10, E9, I5, E17). One \exsitu participant emphasized the importance of real-time observation, particularly for user flow, explaining that although an object appeared well-positioned, real-world feedback revealed that it interfered with pedestrian flow: ``We had a situation where people were walking through one of the objects. It looked like an ideal location for the object on the map [\textit{location mesh}], but in real time and in real life, it wasn't going to work because it was in a walkway'' (E10).

The integration of real-world context through synchronous communication also extended to addressing safety and navigation concerns. One \exsitu participant described uncertainty about whether a virtual cart placed in the scene would interfere with foot and cycling traffic, as there were two real obstacles on either side of it. They noted that this uncertainty was mitigated by being able to check the scene in real time with the \insitu user (E12). Another participant highlighted the value of hearing environmental noise in real time, which influenced decisions about setting audio levels in the AR experience: ``I need to know the population density there, like what's the environment like, should I put the speaker very loud or not at all'' (E9).

Participants also reported that changes in the environment were easier to account for in the \sync condition. One \exsitu user noted that objects had shifted between the time the location mesh was captured and their session, something they would not have noticed without real-time feedback. As they explained, ``The barrel, [...] had actually moved... I wouldn't have been able to tell that from the picture or the scan [location mesh]'' (E17). Following this realization, they asked the \insitu participant to place cursor markers indicating the correct position of the objects on the barrel.

\Exsitu participants also emphasized that synchronous collaboration allowed them to indirectly experience contextual elements of the real-world environment. One participant underscored the importance of experiencing real-world audio and visual elements: ``You can actually hear what they're hearing as well, which is quite important'' (E7). This sense of presence contributed to an increased understanding of spatial relationships within the environment. One participant mentioned, ``I got a much better idea of the space when the other person was there,'' as the \insitu user provided real-time feedback and visual information that could not have been fully captured asynchronously (E16). The combination of real-time video, real-time 3D captures of the environment, and verbal feedback gave \exsitu users a more comprehensive view of the environment, enhancing their understanding of how the design fit within real-world context (E7, E9, E10, E12, E16, E17).

\subsubsection{\textbf{Immediate feedback and changes supported confidence in design decisions, mutual understanding, and perceived accuracy of the outcome.}}\label{sec:results:interviews:immedate_feedback}

All \exsitu participants noted that the \sync condition facilitated more informative feedback and iterative adjustments, with many additionally highlighting that this strengthened their confidence in the authored outcome (I2, E3, E5, E7, I10, E12-17).
The ability to communicate in real time facilitated quicker, informed decisions, with several \exsitu participants noting that the immediate feedback loop significantly reduced the guesswork involved in refining the AR prototype. As E16 stated, ``I think that, very strongly, I'd be more confident that the positions that we put those objects, they actually line up with the real world a lot better.'' This confidence was echoed by other \exsitu participants who mentioned that making the changes in real-time felt ``much, much quicker'' (E3), ``was a lot more dynamic'' (E12), and that ``it felt like, once we were done, it was very complete'' (E10).

The ability to communicate synchronously and see changes immediately not only built confidence but also improved mutual understanding among pairs. Participants emphasized that the visual nature of the real-time interaction removed ambiguity, leading to more accurate placement of virtual objects. Participant I14 noted that discrepancies were minimized: ``I don't think we had a lot of discrepancies in what we discussed about [in the \sync condition], which means probably we had a good sync about the scene.''

\begin{figure*}
    \centering
    \includegraphics[width=\linewidth]{Figures/ExampleOverview_Quotes.pdf}
    \caption{Overview of \SystemName feature usage during Phase 1, shown as \textit{ex-situ} perspective screenshots. Participant conversations are shown in color-coded speech bubbles: \insitu (\textcolor[HTML]{6DD268}{green}) and \exsitu (\textcolor[HTML]{67B3E6}{blue}). Speech bubbles with a \textit{glow} indicate utterances made at the moment of the screenshot. \textsf{(A)} Alignment of a boombox based on spatial context captured using the \textit{3D Snapshot} feature; \textsf{(B)} The \textit{ex-situ} participant moving the map to a position on the wall as specified by the \textit{in-situ} participant through \textit{Surface Drawing}; \textsf{(C)} Alignment of a garland to a previously unmapped region of the street using the \textit{Coarse 3D Mesh} feature; \textsf{(D)} Alignment of misplaced food items based on \textit{in-situ} input using the \textit{3D Cursor}.}
    \label{fig:example-overview}
    %
\end{figure*}

Beyond the real-time feed, \textsc{CoCreatAR}'s additional features --- such as the \textit{3D Cursor}, annotation tools, and the ability to capture \textit{Coarse Meshes} and \textit{3D Snapshots} --- enhanced mutual understanding between \exsitu and \insitu users. \Cref{fig:example-overview} presents examples of feature usage throughout the user study. Participant I16 noted that using the annotation tool to mark exact locations allowed them to ``draw exactly where'' changes were needed, acting as ``the bridge between being there in reality and the scene that [\exsitu user] is seeing.'' Similarly, I5 emphasized that ``drawing in real time is much easier than just explaining it,'' particularly when identifying hazardous areas. The 3D cursor helped reduce ambiguity by allowing participants to visually reference key elements and create placeholders for alignment (I2, I5, I6, I8–10, I12, I14, I16–17, E6, E8, E10–12, E14, E16–17). Additionally, E3 highlighted how the ability to capture \textit{Coarse Meshes} and \textit{3D Snapshots} enabled the \insitu user ``to add more detail to the scene,'' helping \exsitu users make more informed design decisions.

In contrast, the \async condition often resulted in lower confidence due to the lack of immediate feedback and reliance on static information (E9, E12). As E12 described, ``With the asynchronous one, I don't have any information in terms of how much should I move... there's no feedback whatsoever.'' The absence of real-time interaction forced participants to work with limited information, leading to hesitant decisions. E5 emphasized this challenge: ``There was some ineffective information for me in the notes,'' indicating that static instructions without live verification did not provide the necessary context for accurate design decisions. This lack of real-time verification made it difficult to gauge whether changes were correctly applied. E4 highlighted this issue, stating, ``You can't see the change you made, like, in the real world.'' 

\Insitu participants also expressed uncertainty about whether they had captured the right content or communicated their observations clearly (I6, I12, I14, I17). One \insitu participant stated, ``I wasn't sure it was clear enough'' (I14), while another noted that the lack of immediate dialogue required them to provide excessive detail, which still might not ensure accurate interpretation (I17).

\subsubsection{\textbf{Synchronous authoring increased engagement and encouraged creative exploration through collaborative interaction.}}\label{sec:results:interviews:engagement}

In the \sync condition, participants frequently reported higher levels of engagement, often attributing this to the sense of real-time collaboration and mutual decision-making (I3, I8, I12, I15, E3, E5, E12, E14, E17). 
For instance, I10 described the synchronous condition as ``a lot more fun and engaging'' due to the opportunity to work together with another person. Similarly, I14 noted that the synchronous session was ``more enjoyable'' and led to ``higher engagement'' because of the collaborative nature of the task. This sentiment was echoed by E17, highlighting the satisfaction of ``working together'' and described the experience as akin to ``live game testing,'' suggesting that seeing immediate reactions and feedback from the \insitu user created a more interactive and stimulating process. The ability to see their ideas come to life in real time enriched the creative aspect of the experience, as noted by multiple participants (E9, I17).

Synchronous collaboration also drove a sense of teamwork and shared ownership of the final result, contributing to a more engaging authoring process (E7, I10, I12, E16). 
One participant explained that ``it was fun to chat to another person while I was doing it,'' which made the task feel more like a collaborative endeavor rather than an individual effort (I6). This increased engagement was reflected in the enthusiasm with which participants approached the synchronous conditions, with one participant stating that they felt ``buzzing'' with excitement after completing the task together with their collaborator (I12).

In contrast, the \async condition was perceived as less engaging by the majority of participants (I3, I4, E4, I6, E7, I8–I10, I12, E12, I14–I17, E14, E16). 
Participant I12 compared the two conditions, explaining that the asynchronous condition felt ``more like a task as opposed to fun,'' a sentiment echoed by other participants who described the asynchronous process as more isolated and procedural (I4, I7, E14, I17). E12, for example, mentioned that the asynchronous condition felt ``quicker but less satisfying,'' as there was no immediate feedback or dynamic interaction. 

\Insitu participants frequently expressed dissatisfaction with working toward a goal without seeing the final result (I3, I4, I5, I6, I7, I12, I16). 
For example, E16 stated, ``It was a weird feeling leaving and then not seeing it change.'' The lack of real-time collaboration in the \async condition also constrained opportunities for creative exploration. One participant explained that without the ability to brainstorm with another person, the task became ``more about checking a box'' rather than experimenting with new ideas (I10).

Collaborative brainstorming emerged as a key driver of engagement in the synchronous sessions. Participants reported that having someone to exchange ideas with in real time not only enhanced the creative process but also led to more diverse and spontaneous solutions (I9, I10, I17, E14). Participant I9 emphasized the ease of ``drafting concepts'' together in the synchronous condition, stating that ``we could brainstorm easily'' and quickly iterate on suggestions. 
These interactions not only facilitated creative exploration but also enhanced the sense of shared authorship, which several participants valued (I17, E10).

\subsubsection{\textbf{Multitasking overwhelmed some participants in synchronous collaboration, while asynchronous workflows were seen as more suitable for certain scenarios.}}\label{sec:results:interviews:mult-tasking}

Although the synchronous condition generally resulted in increased engagement and collaboration, some participants reported that the increased complexity of multitasking within the synchronous workflow was overwhelming (I8, E16, I17). Some participants also noted that the \sync condition increased pressure, as they felt their collaborators had to wait for them to complete tasks (I4, E12, E14). 

Participant E16 pointed out that managing multiple tasks simultaneously --- such as communicating with the \insitu user, observing the scene, and making design adjustments --- led to cognitive overload: ``I felt like I was doing too many things at once. Talking to [the \insitu user] and trying to watch the environment was sometimes just too much'' (E16). Similarly, I10 expressed that for users unfamiliar with the system, synchronous interactions might feel overwhelming, particularly due to the need to navigate while processing real-time feedback: ``For someone who's not experienced, it was just a lot. You're trying to follow directions, but there's so much going on'' (I10). This aligns with the feedback of several other participants less familiar with AR, who wished they had more time to practice ahead of the task (I3, I6, I10, I13, E10, E11, E16).

Several participants indicated that the synchronous condition was ideal for scenarios where immediate feedback was critical, but the asynchronous condition was more suitable when working in overwhelming environments. Participant I10 described how synchronous interaction could become overwhelming in crowded environments with high noise levels, explaining that ``if you're dealing with a busy place and someone is talking in your ear, it gets really overwhelming'' (I10).

Participants also recognized that both synchronous and asynchronous workflows had their strengths depending on the scenario. While some participants felt overwhelmed by multitasking in the synchronous condition, they still acknowledged its value for tasks requiring rapid decision-making or creative brainstorming. E12 noted, ``It's definitely harder when you have to do everything at once, but I still think the synchronous one is better when you're trying to bounce ideas off someone'' (E12). Participant I16 described how they viewed the \sync condition as most applicable for complex experiences, whereas the \async condition might be more suitable ``if it was, like, a smaller experience with one object.'' E9 also noted scenarios where asynchronous workflows could be beneficial, such as when \insitu participants already have ``enough information to already produce what I want'' (E9).

Moreover, participants reflected on how asynchronous and synchronous workflows or features could be complementary and applied at different stages of the development process (I5, E15, E16). For example, E15 highlighted a hybrid approach: ``I kind of see it as you start with asynchronous, then you do synchronous to refine it, to collect feedback on your experience.'' Other participants saw potential in a hybrid system that flexibly integrates asynchronous and synchronous workflows while incorporating all of \textsc{CoCreatAR}'s capturing and annotation features (I5, E15) and enabling lightweight \insitu editing as in addition (E16).
