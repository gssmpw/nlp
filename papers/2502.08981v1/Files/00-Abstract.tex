\begin{abstract}
Authoring site-specific outdoor augmented reality (AR) experiences requires a nuanced understanding of real-world context to create immersive and relevant content. Existing ex-situ authoring tools typically rely on static 3D models to represent spatial information. However, in our formative study ($n$=25), we identified key limitations of this approach: models are often outdated, incomplete, or insufficient for capturing critical factors such as safety considerations, user flow, and dynamic environmental changes. These issues necessitate frequent on-site visits and additional iterations, making the authoring process more time-consuming and resource-intensive. To mitigate these challenges, we introduce \SystemName, an asymmetric collaborative mixed reality authoring system that integrates the flexibility of ex-situ workflows with the immediate contextual awareness of in-situ authoring. We conducted an exploratory study ($n$=32) comparing \SystemName to an asynchronous workflow baseline, finding that it enhances engagement, creativity, and confidence in the authored output while also providing preliminary insights into its impact on task load. We conclude by discussing the implications of our findings for integrating real-world context into site-specific AR authoring systems.
\end{abstract}
