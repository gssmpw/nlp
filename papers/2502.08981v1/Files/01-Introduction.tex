\section{Introduction}

Outdoor augmented reality (AR) experiences are often designed for specific locations, spanning a range of settings from controlled environments such as theme parks~\cite{LegolandAR2024} to dynamic public urban spaces~\cite{broadcastnow2023}. Developing these experiences is inherently complex because the user experience is closely intertwined with the physical environment. However, developers often work remotely, with only intermittent access to the target location.

\Exsitu (i.e., \emph{off-site}, \emph{remote}) development workflows typically rely on pre-captured 3D models of the environment obtained from prior scans or geospatial data to design and position AR content. Tools such as \textit{Google Geospatial Creator}~\cite{googleGeospatialCreator} and \textit{Niantic Remote Content Authoring}~\cite{lightship-ardk-niantic} facilitate this process by providing site-specific environmental representations at a large scale. Additionally, they offer a visual positioning system (VPS) to precisely align virtual content with the real-world target environment. While these tools effectively support development, their reliance on static representations often leads to missing critical contextual information.

\textit{Context}, broadly defined by \citet{abowdBetterUnderstandingContext1999} as any information characterizing the situation of relevant entities, is particularly important when creating experiences for outdoor settings. For instance, dynamic elements such as lighting conditions, moving objects (e.g., vehicles, temporary structures), and pedestrian activity are typically absent from pre-captured 3D models. This absence can lead to inconsistencies between virtual elements and the real environment, ultimately degrading the user experience~\cite{nebelingXRToolsWhere2022}. Consequently, developers frequently resort to repeated on-site visits, which can be costly, time-consuming, and logistically challenging.

\Insitu (i.e., \emph{on-site}) authoring tools, in contrast, enable AR content creation and testing directly within the target environment, providing immediate contextual awareness~\cite{langlotzSketchingWorldSitu2012,adobeAero,unityMars}. However, these tools often lack the flexibility and expressive power of remote development environments, particularly for large-scale outdoor experiences that require precise object placement, complex interactions, or considerations of user flow and safety. Moreover, mobile devices used for \insitu authoring often have limited computational resources and interaction capabilities, making it difficult to manage complex assets or scripting tasks~\cite{leeImmersiveAuthoringOfTangible2009,vargasgonzalezComparisonDesktopAugmented2019,loOffsiteOnsiteFlexible2021}.

To better understand the challenges of \exsitu authoring of site-specific outdoor AR experiences, we conducted a formative study comprising a survey ($n$=25) and follow-up interviews ($n$=5) with industry professionals. Our findings highlighted common issues, including missing or outdated environmental representations, time-consuming iterative testing cycles that necessitate site visits, and challenges in capturing and communicating contextual information among team members.

Building on these insights, we explored collaborative approaches that integrate the advantages of both \exsitu and \insitu authoring. By enabling synchronous collaboration between \exsitu developers and \insitu collaborators, we sought to address a set of major challenges we identified. Inspired by the concept of pair programming~\cite{hannayEffectivenessPairProgramming2009}, we structured collaborator roles so that the \insitu user captures and transmits real-time environmental context while the \exsitu developer remotely utilizes advanced authoring tools.

To support this workflow, we developed \SystemName, an asymmetric collaborative AR authoring system that facilitates real-time editing of site-specific AR content. \SystemName enables \exsitu developers to work synchronously with \insitu users by providing enhanced spatial information, anchored annotations, and communication tools to bridge the gap between remote development and the physical target environment.

We conducted a user study ($n$=32, in pairs) to compare the effect of synchronous (\SystemName) and asynchronous collaborative authoring approaches on task load, engagement, and confidence in the authored result. In the synchronous condition, \exsitu and \insitu collaborators used \SystemName to work together in real-time, whereas in the asynchronous condition, development proceeded sequentially without immediate interaction, reflecting current practices identified in our formative study. Our findings suggest that authoring with \SystemName improves the integration of real-world context, enhances developers' confidence in the accuracy and feasibility of their AR designs, and leads to greater engagement and creativity among team members. Overall, our contributions include:

\begin{itemize}
\item A formative study analyzing current developer workflows and the role of real-world context in site-specific outdoor AR experience development, based on a survey ($n$=25) and interviews ($n$=5) with industry professionals.
\item The design and implementation of \SystemName, an asymmetric collaborative system that supports real-time, site-specific outdoor AR content authoring by integrating \exsitu and \insitu roles.
\item An empirical evaluation comparing synchronous authoring with \SystemName and an asynchronous workflow baseline, demonstrating the benefits and trade-offs of each method through an exploratory user study ($n$=32).
\end{itemize}
