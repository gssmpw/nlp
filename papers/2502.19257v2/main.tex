% Modified for NDSS 2024 by MS on 2024/07/01

%% bare_conf.tex
%% V1.4b
%% 2015/08/26
%% by Michael Shell
%% See:
%% http://www.michaelshell.org/
%% for current contact information.
%%
%% This is a skeleton file demonstrating the use of IEEEtran.cls
%% (requires IEEEtran.cls version 1.8b or later) with an IEEE
%% conference paper.
%%
%% Support sites:
%% http://www.michaelshell.org/tex/ieeetran/
%% http://www.ctan.org/pkg/ieeetran
%% and
%% http://www.ieee.org/

%%*************************************************************************
%% Legal Notice:
%% This code is offered as-is without any warranty either expressed or
%% implied; without even the implied warranty of MERCHANTABILITY or
%% FITNESS FOR A PARTICULAR PURPOSE!
%% User assumes all risk.
%% In no event shall the IEEE or any contributor to this code be liable for
%% any damages or losses, including, but not limited to, incidental,
%% consequential, or any other damages, resulting from the use or misuse
%% of any information contained here.
%%
%% All comments are the opinions of their respective authors and are not
%% necessarily endorsed by the IEEE.
%%
%% This work is distributed under the LaTeX Project Public License (LPPL)
%% ( http://www.latex-project.org/ ) version 1.3, and may be freely used,
%% distributed and modified. A copy of the LPPL, version 1.3, is included
%% in the base LaTeX documentation of all distributions of LaTeX released
%% 2003/12/01 or later.
%% Retain all contribution notices and credits.
%% ** Modified files should be clearly indicated as such, including  **
%% ** renaming them and changing author support contact information. **
%%*************************************************************************


% *** Authors should verify (and, if needed, correct) their LaTeX system  ***
% *** with the testflow diagnostic prior to trusting their LaTeX platform ***
% *** with production work. The IEEE's font choices and paper sizes can   ***
% *** trigger bugs that do not appear when using other class files.       ***                          ***
% The testflow support page is at:
% http://www.michaelshell.org/tex/testflow/



\documentclass[conference]{IEEEtran}
% Some Computer Society conferences also require the compsoc mode option,
% but others use the standard conference format.
%
% If IEEEtran.cls has not been installed into the LaTeX system files,
% manually specify the path to it like:
% \documentclass[conference]{../sty/IEEEtran}

\usepackage{amsmath,amssymb,amsfonts}
\usepackage{hyperref}
\hypersetup{
    colorlinks=true,       
    linkcolor=black,       
    urlcolor=blue,         
    citecolor=black        
}
\usepackage{graphicx}
\usepackage{pdfpages}  
% Some very useful LaTeX packages include:
% (uncomment the ones you want to load)


% *** MISC UTILITY PACKAGES ***
%
%\usepackage{ifpdf}
% Heiko Oberdiek's ifpdf.sty is very useful if you need conditional
% compilation based on whether the output is pdf or dvi.
% usage:
% \ifpdf
%   % pdf code
% \else
%   % dvi code
% \fi
% The latest version of ifpdf.sty can be obtained from:
% http://www.ctan.org/pkg/ifpdf
% Also, note that IEEEtran.cls V1.7 and later provides a builtin
% \ifCLASSINFOpdf conditional that works the same way.
% When switching from latex to pdflatex and vice-versa, the compiler may
% have to be run twice to clear warning/error messages.






% *** CITATION PACKAGES ***
%
%\usepackage{cite}
% cite.sty was written by Donald Arseneau
% V1.6 and later of IEEEtran pre-defines the format of the cite.sty package
% \cite{} output to follow that of the IEEE. Loading the cite package will
% result in citation numbers being automatically sorted and properly
% "compressed/ranged". e.g., [1], [9], [2], [7], [5], [6] without using
% cite.sty will become [1], [2], [5]--[7], [9] using cite.sty. cite.sty's
% \cite will automatically add leading space, if needed. Use cite.sty's
% noadjust option (cite.sty V3.8 and later) if you want to turn this off
% such as if a citation ever needs to be enclosed in parenthesis.
% cite.sty is already installed on most LaTeX systems. Be sure and use
% version 5.0 (2009-03-20) and later if using hyperref.sty.
% The latest version can be obtained at:
% http://www.ctan.org/pkg/cite
% The documentation is contained in the cite.sty file itself.






% *** GRAPHICS RELATED PACKAGES ***
%
\ifCLASSINFOpdf
  % \usepackage[pdftex]{graphicx}
  % declare the path(s) where your graphic files are
  % \graphicspath{{../pdf/}{../jpeg/}}
  % and their extensions so you won't have to specify these with
  % every instance of \includegraphics
  % \DeclareGraphicsExtensions{.pdf,.jpeg,.png}
\else
  % or other class option (dvipsone, dvipdf, if not using dvips). graphicx
  % will default to the driver specified in the system graphics.cfg if no
  % driver is specified.
  % \usepackage[dvips]{graphicx}
  % declare the path(s) where your graphic files are
  % \graphicspath{{../eps/}}
  % and their extensions so you won't have to specify these with
  % every instance of \includegraphics
  % \DeclareGraphicsExtensions{.eps}
\fi
% graphicx was written by David Carlisle and Sebastian Rahtz. It is
% required if you want graphics, photos, etc. graphicx.sty is already
% installed on most LaTeX systems. The latest version and documentation
% can be obtained at:
% http://www.ctan.org/pkg/graphicx
% Another good source of documentation is "Using Imported Graphics in
% LaTeX2e" by Keith Reckdahl which can be found at:
% http://www.ctan.org/pkg/epslatex
%
% latex, and pdflatex in dvi mode, support graphics in encapsulated
% postscript (.eps) format. pdflatex in pdf mode supports graphics
% in .pdf, .jpeg, .png and .mps (metapost) formats. Users should ensure
% that all non-photo figures use a vector format (.eps, .pdf, .mps) and
% not a bitmapped formats (.jpeg, .png). The IEEE frowns on bitmapped formats
% which can result in "jaggedy"/blurry rendering of lines and letters as
% well as large increases in file sizes.
%
% You can find documentation about the pdfTeX application at:
% http://www.tug.org/applications/pdftex





% *** MATH PACKAGES ***
%
%\usepackage{amsmath}
% A popular package from the American Mathematical Society that provides
% many useful and powerful commands for dealing with mathematics.
%
% Note that the amsmath package sets \interdisplaylinepenalty to 10000
% thus preventing page breaks from occurring within multiline equations. Use:
%\interdisplaylinepenalty=2500
% after loading amsmath to restore such page breaks as IEEEtran.cls normally
% does. amsmath.sty is already installed on most LaTeX systems. The latest
% version and documentation can be obtained at:
% http://www.ctan.org/pkg/amsmath





% *** SPECIALIZED LIST PACKAGES ***
%
%\usepackage{algorithmic}
% algorithmic.sty was written by Peter Williams and Rogerio Brito.
% This package provides an algorithmic environment fo describing algorithms.
% You can use the algorithmic environment in-text or within a figure
% environment to provide for a floating algorithm. Do NOT use the algorithm
% floating environment provided by algorithm.sty (by the same authors) or
% algorithm2e.sty (by Christophe Fiorio) as the IEEE does not use dedicated
% algorithm float types and packages that provide these will not provide
% correct IEEE style captions. The latest version and documentation of
% algorithmic.sty can be obtained at:
% http://www.ctan.org/pkg/algorithms
% Also of interest may be the (relatively newer and more customizable)
% algorithmicx.sty package by Szasz Janos:
% http://www.ctan.org/pkg/algorithmicx




% *** ALIGNMENT PACKAGES ***
%
%\usepackage{array}
% Frank Mittelbach's and David Carlisle's array.sty patches and improves
% the standard LaTeX2e array and tabular environments to provide better
% appearance and additional user controls. As the default LaTeX2e table
% generation code is lacking to the point of almost being broken with
% respect to the quality of the end results, all users are strongly
% advised to use an enhanced (at the very least that provided by array.sty)
% set of table tools. array.sty is already installed on most systems. The
% latest version and documentation can be obtained at:
% http://www.ctan.org/pkg/array


% IEEEtran contains the IEEEeqnarray family of commands that can be used to
% generate multiline equations as well as matrices, tables, etc., of high
% quality.




% *** SUBFIGURE PACKAGES ***
%\ifCLASSOPTIONcompsoc
%  \usepackage[caption=false,font=normalsize,labelfont=sf,textfont=sf]{subfig}
%\else
%  \usepackage[caption=false,font=footnotesize]{subfig}
%\fi
% subfig.sty, written by Steven Douglas Cochran, is the modern replacement
% for subfigure.sty, the latter of which is no longer maintained and is
% incompatible with some LaTeX packages including fixltx2e. However,
% subfig.sty requires and automatically loads Axel Sommerfeldt's caption.sty
% which will override IEEEtran.cls' handling of captions and this will result
% in non-IEEE style figure/table captions. To prevent this problem, be sure
% and invoke subfig.sty's "caption=false" package option (available since
% subfig.sty version 1.3, 2005/06/28) as this is will preserve IEEEtran.cls
% handling of captions.
% Note that the Computer Society format requires a larger sans serif font
% than the serif footnote size font used in traditional IEEE formatting
% and thus the need to invoke different subfig.sty package options depending
% on whether compsoc mode has been enabled.
%
% The latest version and documentation of subfig.sty can be obtained at:
% http://www.ctan.org/pkg/subfig




% *** FLOAT PACKAGES ***
%
%\usepackage{fixltx2e}
% fixltx2e, the successor to the earlier fix2col.sty, was written by
% Frank Mittelbach and David Carlisle. This package corrects a few problems
% in the LaTeX2e kernel, the most notable of which is that in current
% LaTeX2e releases, the ordering of single and double column floats is not
% guaranteed to be preserved. Thus, an unpatched LaTeX2e can allow a
% single column figure to be placed prior to an earlier double column
% figure.
% Be aware that LaTeX2e kernels dated 2015 and later have fixltx2e.sty's
% corrections already built into the system in which case a warning will
% be issued if an attempt is made to load fixltx2e.sty as it is no longer
% needed.
% The latest version and documentation can be found at:
% http://www.ctan.org/pkg/fixltx2e


%\usepackage{stfloats}
% stfloats.sty was written by Sigitas Tolusis. This package gives LaTeX2e
% the ability to do double column floats at the bottom of the page as well
% as the top. (e.g., "\begin{figure*}[!b]" is not normally possible in
% LaTeX2e). It also provides a command:
%\fnbelowfloat
% to enable the placement of footnotes below bottom floats (the standard
% LaTeX2e kernel puts them above bottom floats). This is an invasive package
% which rewrites many portions of the LaTeX2e float routines. It may not work
% with other packages that modify the LaTeX2e float routines. The latest
% version and documentation can be obtained at:
% http://www.ctan.org/pkg/stfloats
% Do not use the stfloats baselinefloat ability as the IEEE does not allow
% \baselineskip to stretch. Authors submitting work to the IEEE should note
% that the IEEE rarely uses double column equations and that authors should try
% to avoid such use. Do not be tempted to use the cuted.sty or midfloat.sty
% packages (also by Sigitas Tolusis) as the IEEE does not format its papers in
% such ways.
% Do not attempt to use stfloats with fixltx2e as they are incompatible.
% Instead, use Morten Hogholm'a dblfloatfix which combines the features
% of both fixltx2e and stfloats:
%
% \usepackage{dblfloatfix}
% The latest version can be found at:
% http://www.ctan.org/pkg/dblfloatfix




% *** PDF, URL AND HYPERLINK PACKAGES ***
%
%\usepackage{url}
% url.sty was written by Donald Arseneau. It provides better support for
% handling and breaking URLs. url.sty is already installed on most LaTeX
% systems. The latest version and documentation can be obtained at:
% http://www.ctan.org/pkg/url
% Basically, \url{my_url_here}.




% *** Do not adjust lengths that control margins, column widths, etc. ***
% *** Do not use packages that alter fonts (such as pslatex).         ***
% There should be no need to do such things with IEEEtran.cls V1.6 and later.
% (Unless specifically asked to do so by the journal or conference you plan
% to submit to, of course. )


% correct bad hyphenation here
\hyphenation{op-tical net-works semi-conduc-tor}


\begin{document}
%
% paper title
% Titles are generally capitalized except for words such as a, an, and, as,
% at, but, by, for, in, nor, of, on, or, the, to and up, which are usually
% not capitalized unless they are the first or last word of the title.
% Linebreaks \\ can be used within to get better formatting as desired.
% Do not put math or special symbols in the title.
\title{Poster: Long PHP webshell files detection\\ based on sliding window attention}


% author names and affiliations
% use a multiple column layout for up to three different
% affiliations
\author{\IEEEauthorblockN{Zhiqiang Wang}
	\IEEEauthorblockA{Beijing Electronic Science\\ $\&$ Technology Institute, Beijing, China\\
		wangzq@besti.edu.cn}
	\and
	\IEEEauthorblockN{Haoyu Wang}
	\IEEEauthorblockA{Beijing Electronic Science\\ $\&$ Technology Institute, Beijing, China\\
		20232909@mail.besti.edu.cn}
	\and
	\IEEEauthorblockN{Lu Hao}
	\IEEEauthorblockA{Beijing Municipal Public\\ Security Bureau, Beijing, China\\
		hlucky@2008.sina.com}}

% conference papers do not typically use \thanks and this command
% is locked out in conference mode. If really needed, such as for
% the acknowledgment of grants, issue a \IEEEoverridecommandlockouts
% after \documentclass

% for over three affiliations, or if they all won't fit within the width
% of the page, use this alternative format:
%
%\author{\IEEEauthorblockN{Michael Shell\IEEEauthorrefmark{1},
%Homer Simpson\IEEEauthorrefmark{2},
%James Kirk\IEEEauthorrefmark{3},
%Montgomery Scott\IEEEauthorrefmark{3} and
%Eldon Tyrell\IEEEauthorrefmark{4}}
%\IEEEauthorblockA{\IEEEauthorrefmark{1}School of Electrical and Computer Engineering\\
%Georgia Institute of Technology,
%Atlanta, Georgia 30332--0250\\ Email: see http://www.michaelshell.org/contact.html}
%\IEEEauthorblockA{\IEEEauthorrefmark{2}Twentieth Century Fox, Springfield, USA\\
%Email: homer@thesimpsons.com}
%\IEEEauthorblockA{\IEEEauthorrefmark{3}Starfleet Academy, San Francisco, California 96678-2391\\
%Telephone: (800) 555--1212, Fax: (888) 555--1212}
%\IEEEauthorblockA{\IEEEauthorrefmark{4}Tyrell Inc., 123 Replicant Street, Los Angeles, California 90210--4321}}




% use for special paper notices
%\IEEEspecialpapernotice{(Invited Paper)}

%
%\IEEEoverridecommandlockouts
%\makeatletter\def\@IEEEpubidpullup{6.5\baselineskip}\makeatother
%\IEEEpubid{\parbox{\columnwidth}{
%		Network and Distributed System Security (NDSS) Symposium 2025\\
%		24-28 February 2025, San Diego, CA, USA\\
%		ISBN 979-8-9894372-8-3\\
%		https://dx.doi.org/10.14722/ndss.2025.[23$|$24]xxxx\\
%		www.ndss-symposium.org
%}
%\hspace{\columnsep}\makebox[\columnwidth]{}}




% make the title area
\maketitle

% As a general rule, do not put math, special symbols or citations
% in the abstract
\begin{abstract}
Webshell is a type of backdoor, and web applications are widely exposed to webshell injection attacks. Therefore, it is important to study webshell detection techniques. In this study, we propose a webshell detection method. We first convert PHP source code to opcodes and then extract Opcode Double-Tuples (ODTs). Next, we combine CodeBert and FastText models for feature representation and classification. To address the challenge that deep learning methods have difficulty detecting long webshell files, we introduce a sliding window attention mechanism. This approach effectively captures malicious behavior within long files. Experimental results show that our method reaches high accuracy in webshell detection, solving the problem of traditional methods that struggle to address new webshell variants and anti-detection techniques.
\end{abstract}

% no keywords




% For peer review papers, you can put extra information on the cover
% page as needed:
% \ifCLASSOPTIONpeerreview
% \begin{center} \bfseries EDICS Category: 3-BBND \end{center}
% \fi
%
% For peerreview papers, this IEEEtran command inserts a page break and
% creates the second title. It will be ignored for other modes.
\IEEEpeerreviewmaketitle



\section{Introduction}
% no \IEEEPARstart
The webshell injection plays a vital role in the hacker attack chain, enabling the attacker to remotely control devices, acquire sensitive data, and further expand attack activities. Therefore, Detecting and removing webshells is an effective way to defend against attacks and ensure web security.

Traditional webshell detection methods \cite{p1,p2} based on pattern matching usually rely on recognizing known features, including source code features, traffic features, dynamic function calls and other relevant features. However, as attack techniques evolve, the variability and obfuscation of webshells have become more prevalent. Attackers often use obfuscation, dynamic loading, encryption and decryption techniques to evade detection, making traditional detection methods inadequate for recognizing new types of webshells.

In this context, webshell detection methods using deep learning \cite{p3,p4,p5}, including those based on source code or opcode, have become a research hotspot and have shown promising results. However, current deep learning-based webshell detection methods still face challenges \cite{p6}. For datasets, publicly available datasets are outdated and do not contain the latest samples. Therefore, their performance in real-world environments for detecting may not be good. For data processing, a good data processing method is often more important than the detection model. The opcode-based detection methods typically extract only a single sequence of opcode instructions (called Opcode Single-Tuples) without effectively capturing low-level code features. The source code-based method is complicated for processing webshells that use anti-detection techniques. In addition, detecting long sequence files (such as complex dynamic encryption and decryption scripts or large files) is quite challenging. Methods such as sample slicing \cite{p3} or TextRank \cite{p5} are often used to reduce data size, which may result in some loss of code information or disruption of contextual relationships.

This study focuses on the PHP language because PHP is used by 75.1$\%$ of all the websites whose server-side programming language \cite{p7}. To address the challenges, this study contribution includes (1) collating a new high-quality Webshell dataset, (2) proposing a PHP code data processing method to extract Opcode Double-Tuples(ODTs) including opcode instructions and operands instead of Opcode Single-Tuples(OSTs), (3) introducing a window attention mechanism to solve the long text problem.

% You must have at least 2 lines in the paragraph with the drop letter
% (should never be an issue

\section{METHODOLOGY}
The detection method consists of two steps. First, the PHP source code in the dataset is processed into ODTs. Second, using a sliding window attention mechanism, we combine the CodeBert model \cite{p8} and the Fasttext model \cite{p9} for feature representation and binary classification of the ODTs. Our dataset and processing code are publicly available: https://github.com/w-32768/PHP-Webshell-Detection-via-Opcode-Analysis

\begin{figure}[h]
\begin{center}
\includegraphics[width=2.8in,height=1.6in]{figu.png}    % The printed column
\caption{Overview of the detection method.}  % width is 8.4 cm.
\label{fig1}                                 % Size the figures
\end{center}                                 % accordingly.
\end{figure}
\vspace{-0.8cm}
\subsection{Data processing}
The dataset consists of PHP source code files containing 5001 webshell samples and 5936 benign PHP files. Firstly, we convert the PHP source code to the opcode. The opcode, generated by the Zend Engine in PHP, is a low-level abstraction of source code. As anti-detection techniques are mostly used at the source code level, we have a natural advantage in using opcode detection.

After obtaining the opcodes, a series of data processing steps are performed. We use expert knowledge to establish fine-grained processing rules, extracting high-value instructions for detection while excluding those of low relevance, thus reducing opcode length without compromising contextual semantics. Operands may be encoded by URL or Base64 encoding, making it difficult to determine their semantics. Therefore, we perform the decoding operation. The original string content is restored based on string feature recognition. After this extraction, we have the set of opcode instructions and operands, called Opcode Double-Tuples. Experimental comparisons show that, under the same detection model training on our dataset, ODTs achieve a 4.6$\%$ accuracy improvement compared to OSTs, confirming that our data processing method is advanced and professional.

\subsection{Feature Representation and Binary Classification}
After data processing, this study explores using the CodeBert model and various embedding models for feature representation and binary classification of ODTs. The steps are as follows:

1) Feature Representation.
\begin{itemize}
\item \textbf{CodeBert Model:} The CodeBert Model is a widely used pre-trained language model optimized for code understanding tasks and pre-trained on PHP code. We input the ODTs into the CodeBert model to generate high-dimensional feature vector representations that capture the semantic and syntactic information of the opcodes.
\item \textbf{Embedding Models:} To enhance opcode feature representation, we compared four embedding models: Word2Vec, FastText, Glove, and Doc2Vec. Experimental comparisons show that FastText performs best in the opcode classification task; therefore, we chose FastText as the embedding model.
\item \textbf{Feature Fusion:} We fuse the feature vectors generated by CodeBert with the embedding vectors from FastText to form the final feature representation. The specific fusion formula is as follows:
    \begin{equation}
     E=\lambda E_{\rm CodeBert}+(1-\lambda)E_{\rm FastText}
     \vspace{-0.2cm}
    \end{equation}
$E_{\rm CodeBert}$ and $E_{\rm FastText}$ represent the feature vectors generated by CodeBert and FastText, respectively. $\lambda$ is the weight coefficient, and its optimal value is determined through experimentation.
\end{itemize}



2) Sliding Window Attention Mechanism:

We introduce a sliding window attention mechanism to address the high computational complexity of global self-attention mechanisms for long opcode sequences. The opcode sequence is divided into multiple windows of size $W$ with a stride of $Sr (Sr<W)$. Specifically, Self-attention is calculated independently within each window. The global feature representation is obtained by averaging the last hidden states from the CodeBert encoder across all windows. This mechanism reduces memory requirements and allows longer sequences to be processed. Furthermore, the overlap between adjacent windows allows information exchange, making it possible to detect malicious behaviors.

The sliding window attention mechanism reduces computational complexity and preserves the contextual information of the opcode sequence. Thus, the problem of incomplete information caused by other methods is avoided.

3) Binary Classification:

After getting the global feature representation of the ODTs, we input them into a binary classifier. The classifier consists of fully connected layers and activation functions, trained by minimizing the binary cross-entropy loss function. It distinguishes between benign PHP code and malicious webshells.

4) Model Training and Evaluation:

We fine-tuned the CodeBert model using the AdamW optimizer. Experimental results show that our proposed optimal model achieves an accuracy of 99.2$\%$ and an F1 score of 99.1$\%$ on the test set. Comparative experiments with accessible state-of-the-art webshell detection methods, including webshellPub \cite{p2} (Acc: 77.3$\%$, F1: 68.5$\%$), PHP Malware Finder \cite{p1} (Acc:83.4$\%$, F1:78.9$\%$), and MSDetector \cite{p3} (Acc:97.1$\%$, F1: 97.3$\%$), demonstrate the superiority of our method.

\section{CONCLUSION}
This study presents a PHP webshell data processing method that extracts ODTs, addressing the limitations of single-tuples detection. Additionally, we introduce a sliding window attention mechanism that effectively mitigates the challenges of long text detection. This study offers a new perspective on the field of malicious code detection. In the future, we aim to continually explore multi-language webshell detection tasks to improve detection performance and generalization capabilities.


\section*{Acknowledgment}
This work was supported by “the Fundamental Research Funds for the Central Universities” (Grant Number:3282024050).





% trigger a \newpage just before the given reference
% number - used to balance the columns on the last page
% adjust value as needed - may need to be readjusted if
% the document is modified later
%\IEEEtriggeratref{8}
% The "triggered" command can be changed if desired:
%\IEEEtriggercmd{\enlargethispage{-5in}}

% references section

% can use a bibliography generated by BibTeX as a .bbl file
% BibTeX documentation can be easily obtained at:
% http://mirror.ctan.org/biblio/bibtex/contrib/doc/
% The IEEEtran BibTeX style support page is at:
% http://www.michaelshell.org/tex/ieeetran/bibtex/
%\bibliographystyle{IEEEtran}
% argument is your BibTeX string definitions and bibliography database(s)
%\bibliography{IEEEabrv,../bib/paper}
%
% <OR> manually copy in the resultant .bbl file
% set second argument of \begin to the number of references
% (used to reserve space for the reference number labels box)
\begin{thebibliography}{1}

\bibitem{p1}
NBS System, ``PHP malware finder,'' 2022. [Online]. Available: \url{https://github.com/nbs-system/php-malware-finder}.

\bibitem{p2}
ShellPub, ``PHP webshell detection,'' 2024. [Online]. Available: \url{https://n.shellpub.com/en}.

\bibitem{p3}
B. Cheng, Y. Guo, Y. Ren, G. Yang, and G. Xu, ``MSDetector: a static PHP webshell detection system based on deep learning,'' in \emph{Theoretical Aspects of Software Engineering}, vol. 13299, 2022, pp. 155--172.

\bibitem{p4}
A. Hannousse, M. Nait-Hamoud, and S. Yahiouche, ``A deep learner model for multi-language webshell detection,'' \emph{International Journal of Information Security}, vol. 22, no. 1, pp. 47--61, 2023.

\bibitem{p5}
T. An, X. Shui, and H. Gao, ``Deep learning based webshell detection coping with long text and lexical ambiguity,'' in \emph{Information And Communications Security}, 2022, pp. 438--457.

\bibitem{p6}
M. Ma, L. Han, and C. Zhou, ``Research and application of artificial intelligence based webshell detection model: a literature review,'' \emph{ArXiv}, vol. 2405.00066, 2024.

\bibitem{p7}
W3Techs, ``Usage statistics and market share of PHP for websites,'' 2025. [Online]. Available: \url{https://w3techs.com/technologies/details/pl-php}.

\bibitem{p8}
Z. Feng, D. Guo, D. Tang, N. Duan, X. Feng, M. Gong et al., ``Codebert: a pre-trained model for programming and natural languages,'' \emph{ArXiv}, vol. 2002.08155, 2020.

\bibitem{p9}
A. Joulin, E. Grave, P. Bojanowski, and T. Mikolov, ``Bag of tricks for efficient text classification,'' \emph{ArXiv}, vol. 1607.01759, 2016.

\end{thebibliography}

\newpage
\includepdf[pages=-]{poster.pdf}  % poster.pdf

% that's all folks
\end{document}


