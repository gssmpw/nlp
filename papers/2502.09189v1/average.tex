
\section{Average case}

Here onwards, $d$ is the dimension, $n$ is the number of vectors in the antichain and $\ell-1$ is the largest number allowed in the antichain.

$[\ell]^d$ represents the $d$-dimensional lattice, where $[\ell]$ represents the set $\Set{0,\dots,\ell-1}$.

$\alpha(d,\ell)$ is the \emph{width} of the antichain, that is, the maximal size of an antichain in $[\ell]^d$.

$A(d,\ell,n)$ is the total number of antichains in $[\ell]^d$ with exactly $n$ vectors.

$A(d,\ell)$ is the total number of antichains in $[\ell]^d$. 

\[
A(d,\ell)=\sum_{n=1}^{\alpha(d,\ell)}A(d,\ell,n).
\]

Bounds for $A(d,\ell)$ using $\alpha(d,\ell)$ are known \cite{falgasravry2023dedekinds}:

\[
\alpha(d,\ell)\leq\log_2 A(d,\ell)\leq \Big(1+c\cdot\frac{(\log d)^3}{d}\Big)\cdot\alpha(d,\ell).
\]

The lower bound is easy to see, since all the $2^{\alpha(d,\ell)}$ subsets of a maximal antichain are also antichains. 

We are also interested in ``good'' antichains in $[\ell]^d$, that is, antichains such that all the numbers in the same dimension are distinct. $\alpha_G(d,\ell)$, $A_G(d,\ell,n)$ and $A_G(d,\ell)$ are defined analogously for good antichains. It is easy to see $\alpha_G(d,\ell)=\ell$ for $d\geq2$ since there are only $\ell$ unique numbers that can occur in one dimension in a good antichain.

\subsection{$d=2$}

In two dimensions, we have the property that any two vectors in an antichain, say $(p,q)$ and $(r,s)$ must have the property that $p\neq r$, $q\neq s$ and $p>r$ iff $q<s$. Thus, for a given $n\leq\ell$, any two sequences of numbers $0\leq p_1<p_2<\dots<p_n<\ell$ and $\ell>q_1>q_2>\dots>q_n\geq0$ induce an antichain $(p_1,q_1),\dots,(p_n,q_n)$ and all antichains in $[\ell]^2$ can be written in this form. This gives us the following:
\begin{align}
    A(d,\ell,n)&=\binom{\ell}{n}^2,\\
    A(d,\ell)&=\sum_{n=0}^{\ell}A(d,\ell,n)=\sum_{n=0}^{\ell}\binom{\ell}{n}^2=\binom{2\ell}{\ell}.
\end{align}

The last equality follows from the fact that choosing $\ell$ items from $2\ell$ items is the same as choosing $n$ and $\ell-n$ items respectively from two sets containing $\ell$ items. 

Note that all antichains in two dimensions are good, which gives us $A_G(d,\ell,n)=A(d,\ell,n)$ and $A_G(d,\ell)=A(d,\ell)$.

We also have $\alpha(d,\ell)=\ell$ since it is not possible to find sequences of numbers between $0$ and $\ell-1$ of length more than $\ell$ and sequences of length $\ell$ exist.

\subsection{$d\geq3$}

The analysis for higher dimensions is not so trivial. We only provide bounds which help with our average case analysis of the performance of $kd$-trees.\shrisha{hopefully}

How big is $\alpha(d,\ell)$? To give an upper bound, we use the fact that for all $0\leq k\leq\ell-1$, all vectors $x=(x_1,\dots,x_d)$ with $x_1+\dots+x_d=k$ form an antichain. This is easy to see, since for two distinct vectors with the same sum, there must be two components $1\leq m,n\leq d$ such that the component $m$ of one vector is greater than that of the other, and the opposite holds for component $n$. Thus, we are interested in the number of points whose components add to $k$, with the restriction that each component is at most $\ell$.

We conjecture that this number is maximized when $k=\lfloor\frac{\ell d}{2}\rfloor$.
\begin{theorem}[Conjecture]
    The length of the longest antichain in the hypercube $[\ell]^d$ is the same as the size of the set of all points in $[\ell]^d$ such that the sum of all the components of each point is exactly $\lfloor\frac{\ell d}{2}\rfloor$.
\end{theorem}
We will show that the above holds if the following holds:
\begin{proposition}
    For all $n<\lfloor\frac{\ell d}{2}\rfloor$ There is a surjective mapping $f$ from the set of all points which add up to $n$ to those that add up to $n+1$ such that the mapping preserves the following: $f(u)=v$ implies $v>u$.
\end{proposition}

First, we show that the number of points that add to $n$ in $[\ell]^d$ is at most those that add up to $n+1$ for $n<\lfloor\frac{\ell d}{2}\rfloor$. To show this, we first notice the following:

\begin{lemma}
    
\end{lemma}
