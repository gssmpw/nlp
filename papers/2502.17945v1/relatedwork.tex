\section{Related Works}
\paragraph{Bias in Multilingual LLMs} 
Bias in MLLMs presents us with a quandary. It has emerged as a critical challenge to the fairness of MLLMs and thus significantly restricts real-world deployment~\cite{xu-etal-2023-language-representation}.
Numerous studies have been conducted to measure language bias, which refers to the unidentical performance of MLLMs across different languages in terms of race, religion, nationality, gender, and other factors~\cite{zhao2024gender, mihaylov-shtedritski-2024-elegant, mukherjee-etal-2023-global, neplenbroek2024mbbq, li-etal-2024-land, vashishtha-etal-2023-evaluating, naous-etal-2024-beer,hofmann2024ai}. Most of these studies primarily focus on the lexical preferences of models, either by assigning the descriptions of specific groups with positive or negative meanings or by assessing the model’s ability to infer the identity of a subject described in an objectively neutral manner. Among the most relevant studies in this line of research, \citet{narayanan-venkit-etal-2023-nationality} examine whether the use of adjectives by language models defined by nationalities in English are positive or negative. \citet{zhu-etal-2024-quite} further extend this analysis to a Chinese context. Additionally, \citet{kamruzzaman-etal-2024-investigating}, \citet{nie-etal-2024-multilingual} and \citet{parrish-etal-2022-bbq} constructed multiple-choice selection evaluations in English. Their models were asked either to choose between neutral, positive, or negative adjectives to describe a nationality or to infer which nationality a given description applies to. While these studies provide valuable insights into nationality bias in LLMs, they are largely limited to monolingual settings and focus primarily on lexical-level biases. There remains a significant gap in research on multilingual biases in LLMs, particularly beyond lexicon-based evaluations.

\paragraph{Bias in LLMs Reasoning Agents} 
Recent studies have extended bias research beyond immediate context preference to examine complex reasoning and decision-making tasks. Several studies have investigated the use of LLMs as simulations of multilingual survey subjects.
\citet{jin2024language} examined LLM performance in moral reasoning tasks, particularly in responding to variations of the Trolley Problem. \citet{durmus2023towards} explored the subjective global opinions of LLMs by prompting models to answer binary-choice questions under explicit persona settings in a multilingual context. \citet{kwok2024evaluating} further advanced this approach by developing the Simulation of Synthetic Personas, and designing questionnaires based on real-world news to assess biases in model-generated responses. 
While these studies provide valuable insights into biases in complex reasoning and decision-making tasks under multilingual settings, fall short of providing a comprehensive understanding of real-world applications. 
Other studies addressed tasks such as hiring screening agents~\cite{armstrong2024silicon} and university application agents~\cite{zheng2024dissecting} in English. Not only their studies limited to English, but they constrain the models by restricting their ability to engage in chain-of-thought (CoT)-like reasoning during responses. This significantly limits the scope and depth of bias analysis in structured decision-making processes.