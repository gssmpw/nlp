
The synthesis of machine learning and differential equations has been treated in a few different ways in the field.
A major line of work approaches the problem from a machine learning  perspective with supervised or unsupervised learning for partial differential equations \cite{li2020fourier, brandstetter2022lie}.
Another line of work in machine learning for scientific application aims at a synthesis of differential equation models with a \emph{network-in-solver} approach by embedding neural networks in classical solvers \cite{chen2018neural, rackauckas2020universal} enabling access to high quality numerical solvers in the context of machine learning.
Physics-informed networks use a physics-aware loss to supervise network training \cite{raissi2018deep, raissi2019physics}.
Mechanistic neural networks conversely are a \emph{solver-in-network} approach that attempts to balance classical solvers and neural networks by combining restricted ODEs~\cite{pervezmechanistic,chen2024scalable} (and now PDEs), that allowing faster solving, with neural network learning together with techniques for handling nonlinear equations.

As an application of the synthesis of ML and differential equations various data driven techniques for discovering governing equations have been explored for ODEs such as SINDy \cite{brunton2016discovering}, SINDy-PI \cite{kaheman2020sindy} and PDEs including PDEFIND \cite{rudy2017data} and weak forms \cite{reinbold2020using, messenger2021weak} extending the SINDy framework.
Physics informed networks \cite{raissi2019physics, raissi2018deep} have also been used for inverse problems but unlike UDEs \cite{rackauckas2020universal} and SINDy do not handle unknown physics.
PDE-LEARN \cite{stephany2024pde} combines features of physics informed networks and basis libraries for PDE discovery.
MechNNs \cite{pervezmechanistic} build a discovery model for ODEs  which we enhance and develop PDE models that handle nonlinear equations and unknown dynamics.

