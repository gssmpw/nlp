\section{1D PDE Experiments}
\label{sec:app:1d-experiments}
We test discovery on clean data on the diffusion equation with a diffusion coefficient of 0.01.
We obtain a TPR of 1 and $E_\infty$ value of 0.224.
We show results for the viscous and inviscid Burger's equations in Figure \ref{fig:burgers-viscous} and Table \ref{tab:burgers-inviscid}, comparing with WeakSINDy.
We find both methods to perform similarly on this problem.

\begin{figure}[h!]
  \vskip -0.05in
  \centering
    \includegraphics[width=0.32\linewidth]{img/burgers_tpr.pdf}
    \includegraphics[width=0.32\linewidth]{img/burgers_error_max.pdf}
  \vskip -0.1in
  \caption{TPR and error for viscous Burger's equation. Comparing with WeakSINDy}
  \vskip -0.05in
  \label{fig:burgers-viscous}
\end{figure}


\begin{table}[h!] %
  \centering
  \caption{Inviscid Burger's Equation for clean and 10\% noisy data.}
  \vskip 0.05in
  \begin{footnotesize}
  \begin{tabular}{lcccc}
    \toprule
    & \multicolumn{2}{c}{MechNN-PDE} & \multicolumn{2}{c}{WeakSINDy} \\
    \midrule
    Noise & TRP & $E_\infty$ & TRP & $E_\infty$\\
    \midrule
    0&1 &0.043 & 1 & 0.001\\
    10&1 &0.0020&1 & 0.012 \\
    \bottomrule
  \end{tabular}
  \label{tab:burgers-inviscid}
  \end{footnotesize}%
  \vskip -0.15in
\end{table}


\section{Multigrid V-cycle Algorithm}
Algorithm \ref{alg:v-cycle} shows the multigrid V-cycle (see \citet{saad2003iterative}) that we use to precondition the GMRES algorithm.
We use Gauss-Seidel as the relaxation method and linear interpolation as the restriction and prolongation operators.
\begin{center}
\begin{algorithm}[h!]
   \caption{V-cycle}
   \label{alg:v-cycle}
\begin{algorithmic}
   \STATE {\bfseries Require:}  \texttt{relax}, \texttt{restrict}, \texttt{prolong} operators
   \STATE {\bfseries Input:} $M_i$, $x_i$, $b_i$; $i \in[m]$
    \STATE $x_i = \texttt{relax}(A_i, x_i, b_i)$ 
    \STATE $r_i = b_i - A_i x_i$
    \STATE $r_{i+1} = \texttt{restrict}(r_i)$

    \IF{$m==i+1$}
        \STATE Solve $A_m \delta_m = r_m$
    \ELSE
        \STATE $\delta_i$ = V-cycle($A_{i+1}, 0, r_{i+1}$)
    \ENDIF
   \STATE $\delta_i = \texttt{prolong}(\delta_{i+1})$
   \STATE $x_i = x_i + \delta_i$
    \STATE $x_i = \texttt{relax}(A_i, x_i, b_i)$ 
\end{algorithmic}
\end{algorithm}
\end{center}

