
\documentclass[10pt]{article} % For LaTeX2e
\usepackage[preprint]{tmlr}
% If accepted, instead use the following line for the camera-ready submission:
%\usepackage[accepted]{tmlr}
% To de-anonymize and remove mentions to TMLR (for example for posting to preprint servers), instead use the following:
%\usepackage[preprint]{tmlr}

% Optional math commands from https://github.com/goodfeli/dlbook_notation.
%%%%% NEW MATH DEFINITIONS %%%%%

\usepackage{amsmath,amsfonts,bm}
\usepackage{derivative}
% Mark sections of captions for referring to divisions of figures
\newcommand{\figleft}{{\em (Left)}}
\newcommand{\figcenter}{{\em (Center)}}
\newcommand{\figright}{{\em (Right)}}
\newcommand{\figtop}{{\em (Top)}}
\newcommand{\figbottom}{{\em (Bottom)}}
\newcommand{\captiona}{{\em (a)}}
\newcommand{\captionb}{{\em (b)}}
\newcommand{\captionc}{{\em (c)}}
\newcommand{\captiond}{{\em (d)}}

% Highlight a newly defined term
\newcommand{\newterm}[1]{{\bf #1}}

% Derivative d 
\newcommand{\deriv}{{\mathrm{d}}}

% Figure reference, lower-case.
\def\figref#1{figure~\ref{#1}}
% Figure reference, capital. For start of sentence
\def\Figref#1{Figure~\ref{#1}}
\def\twofigref#1#2{figures \ref{#1} and \ref{#2}}
\def\quadfigref#1#2#3#4{figures \ref{#1}, \ref{#2}, \ref{#3} and \ref{#4}}
% Section reference, lower-case.
\def\secref#1{section~\ref{#1}}
% Section reference, capital.
\def\Secref#1{Section~\ref{#1}}
% Reference to two sections.
\def\twosecrefs#1#2{sections \ref{#1} and \ref{#2}}
% Reference to three sections.
\def\secrefs#1#2#3{sections \ref{#1}, \ref{#2} and \ref{#3}}
% Reference to an equation, lower-case.
\def\eqref#1{equation~\ref{#1}}
% Reference to an equation, upper case
\def\Eqref#1{Equation~\ref{#1}}
% A raw reference to an equation---avoid using if possible
\def\plaineqref#1{\ref{#1}}
% Reference to a chapter, lower-case.
\def\chapref#1{chapter~\ref{#1}}
% Reference to an equation, upper case.
\def\Chapref#1{Chapter~\ref{#1}}
% Reference to a range of chapters
\def\rangechapref#1#2{chapters\ref{#1}--\ref{#2}}
% Reference to an algorithm, lower-case.
\def\algref#1{algorithm~\ref{#1}}
% Reference to an algorithm, upper case.
\def\Algref#1{Algorithm~\ref{#1}}
\def\twoalgref#1#2{algorithms \ref{#1} and \ref{#2}}
\def\Twoalgref#1#2{Algorithms \ref{#1} and \ref{#2}}
% Reference to a part, lower case
\def\partref#1{part~\ref{#1}}
% Reference to a part, upper case
\def\Partref#1{Part~\ref{#1}}
\def\twopartref#1#2{parts \ref{#1} and \ref{#2}}

\def\ceil#1{\lceil #1 \rceil}
\def\floor#1{\lfloor #1 \rfloor}
\def\1{\bm{1}}
\newcommand{\train}{\mathcal{D}}
\newcommand{\valid}{\mathcal{D_{\mathrm{valid}}}}
\newcommand{\test}{\mathcal{D_{\mathrm{test}}}}

\def\eps{{\epsilon}}


% Random variables
\def\reta{{\textnormal{$\eta$}}}
\def\ra{{\textnormal{a}}}
\def\rb{{\textnormal{b}}}
\def\rc{{\textnormal{c}}}
\def\rd{{\textnormal{d}}}
\def\re{{\textnormal{e}}}
\def\rf{{\textnormal{f}}}
\def\rg{{\textnormal{g}}}
\def\rh{{\textnormal{h}}}
\def\ri{{\textnormal{i}}}
\def\rj{{\textnormal{j}}}
\def\rk{{\textnormal{k}}}
\def\rl{{\textnormal{l}}}
% rm is already a command, just don't name any random variables m
\def\rn{{\textnormal{n}}}
\def\ro{{\textnormal{o}}}
\def\rp{{\textnormal{p}}}
\def\rq{{\textnormal{q}}}
\def\rr{{\textnormal{r}}}
\def\rs{{\textnormal{s}}}
\def\rt{{\textnormal{t}}}
\def\ru{{\textnormal{u}}}
\def\rv{{\textnormal{v}}}
\def\rw{{\textnormal{w}}}
\def\rx{{\textnormal{x}}}
\def\ry{{\textnormal{y}}}
\def\rz{{\textnormal{z}}}

% Random vectors
\def\rvepsilon{{\mathbf{\epsilon}}}
\def\rvphi{{\mathbf{\phi}}}
\def\rvtheta{{\mathbf{\theta}}}
\def\rva{{\mathbf{a}}}
\def\rvb{{\mathbf{b}}}
\def\rvc{{\mathbf{c}}}
\def\rvd{{\mathbf{d}}}
\def\rve{{\mathbf{e}}}
\def\rvf{{\mathbf{f}}}
\def\rvg{{\mathbf{g}}}
\def\rvh{{\mathbf{h}}}
\def\rvu{{\mathbf{i}}}
\def\rvj{{\mathbf{j}}}
\def\rvk{{\mathbf{k}}}
\def\rvl{{\mathbf{l}}}
\def\rvm{{\mathbf{m}}}
\def\rvn{{\mathbf{n}}}
\def\rvo{{\mathbf{o}}}
\def\rvp{{\mathbf{p}}}
\def\rvq{{\mathbf{q}}}
\def\rvr{{\mathbf{r}}}
\def\rvs{{\mathbf{s}}}
\def\rvt{{\mathbf{t}}}
\def\rvu{{\mathbf{u}}}
\def\rvv{{\mathbf{v}}}
\def\rvw{{\mathbf{w}}}
\def\rvx{{\mathbf{x}}}
\def\rvy{{\mathbf{y}}}
\def\rvz{{\mathbf{z}}}

% Elements of random vectors
\def\erva{{\textnormal{a}}}
\def\ervb{{\textnormal{b}}}
\def\ervc{{\textnormal{c}}}
\def\ervd{{\textnormal{d}}}
\def\erve{{\textnormal{e}}}
\def\ervf{{\textnormal{f}}}
\def\ervg{{\textnormal{g}}}
\def\ervh{{\textnormal{h}}}
\def\ervi{{\textnormal{i}}}
\def\ervj{{\textnormal{j}}}
\def\ervk{{\textnormal{k}}}
\def\ervl{{\textnormal{l}}}
\def\ervm{{\textnormal{m}}}
\def\ervn{{\textnormal{n}}}
\def\ervo{{\textnormal{o}}}
\def\ervp{{\textnormal{p}}}
\def\ervq{{\textnormal{q}}}
\def\ervr{{\textnormal{r}}}
\def\ervs{{\textnormal{s}}}
\def\ervt{{\textnormal{t}}}
\def\ervu{{\textnormal{u}}}
\def\ervv{{\textnormal{v}}}
\def\ervw{{\textnormal{w}}}
\def\ervx{{\textnormal{x}}}
\def\ervy{{\textnormal{y}}}
\def\ervz{{\textnormal{z}}}

% Random matrices
\def\rmA{{\mathbf{A}}}
\def\rmB{{\mathbf{B}}}
\def\rmC{{\mathbf{C}}}
\def\rmD{{\mathbf{D}}}
\def\rmE{{\mathbf{E}}}
\def\rmF{{\mathbf{F}}}
\def\rmG{{\mathbf{G}}}
\def\rmH{{\mathbf{H}}}
\def\rmI{{\mathbf{I}}}
\def\rmJ{{\mathbf{J}}}
\def\rmK{{\mathbf{K}}}
\def\rmL{{\mathbf{L}}}
\def\rmM{{\mathbf{M}}}
\def\rmN{{\mathbf{N}}}
\def\rmO{{\mathbf{O}}}
\def\rmP{{\mathbf{P}}}
\def\rmQ{{\mathbf{Q}}}
\def\rmR{{\mathbf{R}}}
\def\rmS{{\mathbf{S}}}
\def\rmT{{\mathbf{T}}}
\def\rmU{{\mathbf{U}}}
\def\rmV{{\mathbf{V}}}
\def\rmW{{\mathbf{W}}}
\def\rmX{{\mathbf{X}}}
\def\rmY{{\mathbf{Y}}}
\def\rmZ{{\mathbf{Z}}}

% Elements of random matrices
\def\ermA{{\textnormal{A}}}
\def\ermB{{\textnormal{B}}}
\def\ermC{{\textnormal{C}}}
\def\ermD{{\textnormal{D}}}
\def\ermE{{\textnormal{E}}}
\def\ermF{{\textnormal{F}}}
\def\ermG{{\textnormal{G}}}
\def\ermH{{\textnormal{H}}}
\def\ermI{{\textnormal{I}}}
\def\ermJ{{\textnormal{J}}}
\def\ermK{{\textnormal{K}}}
\def\ermL{{\textnormal{L}}}
\def\ermM{{\textnormal{M}}}
\def\ermN{{\textnormal{N}}}
\def\ermO{{\textnormal{O}}}
\def\ermP{{\textnormal{P}}}
\def\ermQ{{\textnormal{Q}}}
\def\ermR{{\textnormal{R}}}
\def\ermS{{\textnormal{S}}}
\def\ermT{{\textnormal{T}}}
\def\ermU{{\textnormal{U}}}
\def\ermV{{\textnormal{V}}}
\def\ermW{{\textnormal{W}}}
\def\ermX{{\textnormal{X}}}
\def\ermY{{\textnormal{Y}}}
\def\ermZ{{\textnormal{Z}}}

% Vectors
\def\vzero{{\bm{0}}}
\def\vone{{\bm{1}}}
\def\vmu{{\bm{\mu}}}
\def\vtheta{{\bm{\theta}}}
\def\vphi{{\bm{\phi}}}
\def\va{{\bm{a}}}
\def\vb{{\bm{b}}}
\def\vc{{\bm{c}}}
\def\vd{{\bm{d}}}
\def\ve{{\bm{e}}}
\def\vf{{\bm{f}}}
\def\vg{{\bm{g}}}
\def\vh{{\bm{h}}}
\def\vi{{\bm{i}}}
\def\vj{{\bm{j}}}
\def\vk{{\bm{k}}}
\def\vl{{\bm{l}}}
\def\vm{{\bm{m}}}
\def\vn{{\bm{n}}}
\def\vo{{\bm{o}}}
\def\vp{{\bm{p}}}
\def\vq{{\bm{q}}}
\def\vr{{\bm{r}}}
\def\vs{{\bm{s}}}
\def\vt{{\bm{t}}}
\def\vu{{\bm{u}}}
\def\vv{{\bm{v}}}
\def\vw{{\bm{w}}}
\def\vx{{\bm{x}}}
\def\vy{{\bm{y}}}
\def\vz{{\bm{z}}}

% Elements of vectors
\def\evalpha{{\alpha}}
\def\evbeta{{\beta}}
\def\evepsilon{{\epsilon}}
\def\evlambda{{\lambda}}
\def\evomega{{\omega}}
\def\evmu{{\mu}}
\def\evpsi{{\psi}}
\def\evsigma{{\sigma}}
\def\evtheta{{\theta}}
\def\eva{{a}}
\def\evb{{b}}
\def\evc{{c}}
\def\evd{{d}}
\def\eve{{e}}
\def\evf{{f}}
\def\evg{{g}}
\def\evh{{h}}
\def\evi{{i}}
\def\evj{{j}}
\def\evk{{k}}
\def\evl{{l}}
\def\evm{{m}}
\def\evn{{n}}
\def\evo{{o}}
\def\evp{{p}}
\def\evq{{q}}
\def\evr{{r}}
\def\evs{{s}}
\def\evt{{t}}
\def\evu{{u}}
\def\evv{{v}}
\def\evw{{w}}
\def\evx{{x}}
\def\evy{{y}}
\def\evz{{z}}

% Matrix
\def\mA{{\bm{A}}}
\def\mB{{\bm{B}}}
\def\mC{{\bm{C}}}
\def\mD{{\bm{D}}}
\def\mE{{\bm{E}}}
\def\mF{{\bm{F}}}
\def\mG{{\bm{G}}}
\def\mH{{\bm{H}}}
\def\mI{{\bm{I}}}
\def\mJ{{\bm{J}}}
\def\mK{{\bm{K}}}
\def\mL{{\bm{L}}}
\def\mM{{\bm{M}}}
\def\mN{{\bm{N}}}
\def\mO{{\bm{O}}}
\def\mP{{\bm{P}}}
\def\mQ{{\bm{Q}}}
\def\mR{{\bm{R}}}
\def\mS{{\bm{S}}}
\def\mT{{\bm{T}}}
\def\mU{{\bm{U}}}
\def\mV{{\bm{V}}}
\def\mW{{\bm{W}}}
\def\mX{{\bm{X}}}
\def\mY{{\bm{Y}}}
\def\mZ{{\bm{Z}}}
\def\mBeta{{\bm{\beta}}}
\def\mPhi{{\bm{\Phi}}}
\def\mLambda{{\bm{\Lambda}}}
\def\mSigma{{\bm{\Sigma}}}

% Tensor
\DeclareMathAlphabet{\mathsfit}{\encodingdefault}{\sfdefault}{m}{sl}
\SetMathAlphabet{\mathsfit}{bold}{\encodingdefault}{\sfdefault}{bx}{n}
\newcommand{\tens}[1]{\bm{\mathsfit{#1}}}
\def\tA{{\tens{A}}}
\def\tB{{\tens{B}}}
\def\tC{{\tens{C}}}
\def\tD{{\tens{D}}}
\def\tE{{\tens{E}}}
\def\tF{{\tens{F}}}
\def\tG{{\tens{G}}}
\def\tH{{\tens{H}}}
\def\tI{{\tens{I}}}
\def\tJ{{\tens{J}}}
\def\tK{{\tens{K}}}
\def\tL{{\tens{L}}}
\def\tM{{\tens{M}}}
\def\tN{{\tens{N}}}
\def\tO{{\tens{O}}}
\def\tP{{\tens{P}}}
\def\tQ{{\tens{Q}}}
\def\tR{{\tens{R}}}
\def\tS{{\tens{S}}}
\def\tT{{\tens{T}}}
\def\tU{{\tens{U}}}
\def\tV{{\tens{V}}}
\def\tW{{\tens{W}}}
\def\tX{{\tens{X}}}
\def\tY{{\tens{Y}}}
\def\tZ{{\tens{Z}}}


% Graph
\def\gA{{\mathcal{A}}}
\def\gB{{\mathcal{B}}}
\def\gC{{\mathcal{C}}}
\def\gD{{\mathcal{D}}}
\def\gE{{\mathcal{E}}}
\def\gF{{\mathcal{F}}}
\def\gG{{\mathcal{G}}}
\def\gH{{\mathcal{H}}}
\def\gI{{\mathcal{I}}}
\def\gJ{{\mathcal{J}}}
\def\gK{{\mathcal{K}}}
\def\gL{{\mathcal{L}}}
\def\gM{{\mathcal{M}}}
\def\gN{{\mathcal{N}}}
\def\gO{{\mathcal{O}}}
\def\gP{{\mathcal{P}}}
\def\gQ{{\mathcal{Q}}}
\def\gR{{\mathcal{R}}}
\def\gS{{\mathcal{S}}}
\def\gT{{\mathcal{T}}}
\def\gU{{\mathcal{U}}}
\def\gV{{\mathcal{V}}}
\def\gW{{\mathcal{W}}}
\def\gX{{\mathcal{X}}}
\def\gY{{\mathcal{Y}}}
\def\gZ{{\mathcal{Z}}}

% Sets
\def\sA{{\mathbb{A}}}
\def\sB{{\mathbb{B}}}
\def\sC{{\mathbb{C}}}
\def\sD{{\mathbb{D}}}
% Don't use a set called E, because this would be the same as our symbol
% for expectation.
\def\sF{{\mathbb{F}}}
\def\sG{{\mathbb{G}}}
\def\sH{{\mathbb{H}}}
\def\sI{{\mathbb{I}}}
\def\sJ{{\mathbb{J}}}
\def\sK{{\mathbb{K}}}
\def\sL{{\mathbb{L}}}
\def\sM{{\mathbb{M}}}
\def\sN{{\mathbb{N}}}
\def\sO{{\mathbb{O}}}
\def\sP{{\mathbb{P}}}
\def\sQ{{\mathbb{Q}}}
\def\sR{{\mathbb{R}}}
\def\sS{{\mathbb{S}}}
\def\sT{{\mathbb{T}}}
\def\sU{{\mathbb{U}}}
\def\sV{{\mathbb{V}}}
\def\sW{{\mathbb{W}}}
\def\sX{{\mathbb{X}}}
\def\sY{{\mathbb{Y}}}
\def\sZ{{\mathbb{Z}}}

% Entries of a matrix
\def\emLambda{{\Lambda}}
\def\emA{{A}}
\def\emB{{B}}
\def\emC{{C}}
\def\emD{{D}}
\def\emE{{E}}
\def\emF{{F}}
\def\emG{{G}}
\def\emH{{H}}
\def\emI{{I}}
\def\emJ{{J}}
\def\emK{{K}}
\def\emL{{L}}
\def\emM{{M}}
\def\emN{{N}}
\def\emO{{O}}
\def\emP{{P}}
\def\emQ{{Q}}
\def\emR{{R}}
\def\emS{{S}}
\def\emT{{T}}
\def\emU{{U}}
\def\emV{{V}}
\def\emW{{W}}
\def\emX{{X}}
\def\emY{{Y}}
\def\emZ{{Z}}
\def\emSigma{{\Sigma}}

% entries of a tensor
% Same font as tensor, without \bm wrapper
\newcommand{\etens}[1]{\mathsfit{#1}}
\def\etLambda{{\etens{\Lambda}}}
\def\etA{{\etens{A}}}
\def\etB{{\etens{B}}}
\def\etC{{\etens{C}}}
\def\etD{{\etens{D}}}
\def\etE{{\etens{E}}}
\def\etF{{\etens{F}}}
\def\etG{{\etens{G}}}
\def\etH{{\etens{H}}}
\def\etI{{\etens{I}}}
\def\etJ{{\etens{J}}}
\def\etK{{\etens{K}}}
\def\etL{{\etens{L}}}
\def\etM{{\etens{M}}}
\def\etN{{\etens{N}}}
\def\etO{{\etens{O}}}
\def\etP{{\etens{P}}}
\def\etQ{{\etens{Q}}}
\def\etR{{\etens{R}}}
\def\etS{{\etens{S}}}
\def\etT{{\etens{T}}}
\def\etU{{\etens{U}}}
\def\etV{{\etens{V}}}
\def\etW{{\etens{W}}}
\def\etX{{\etens{X}}}
\def\etY{{\etens{Y}}}
\def\etZ{{\etens{Z}}}

% The true underlying data generating distribution
\newcommand{\pdata}{p_{\rm{data}}}
\newcommand{\ptarget}{p_{\rm{target}}}
\newcommand{\pprior}{p_{\rm{prior}}}
\newcommand{\pbase}{p_{\rm{base}}}
\newcommand{\pref}{p_{\rm{ref}}}

% The empirical distribution defined by the training set
\newcommand{\ptrain}{\hat{p}_{\rm{data}}}
\newcommand{\Ptrain}{\hat{P}_{\rm{data}}}
% The model distribution
\newcommand{\pmodel}{p_{\rm{model}}}
\newcommand{\Pmodel}{P_{\rm{model}}}
\newcommand{\ptildemodel}{\tilde{p}_{\rm{model}}}
% Stochastic autoencoder distributions
\newcommand{\pencode}{p_{\rm{encoder}}}
\newcommand{\pdecode}{p_{\rm{decoder}}}
\newcommand{\precons}{p_{\rm{reconstruct}}}

\newcommand{\laplace}{\mathrm{Laplace}} % Laplace distribution

\newcommand{\E}{\mathbb{E}}
\newcommand{\Ls}{\mathcal{L}}
\newcommand{\R}{\mathbb{R}}
\newcommand{\emp}{\tilde{p}}
\newcommand{\lr}{\alpha}
\newcommand{\reg}{\lambda}
\newcommand{\rect}{\mathrm{rectifier}}
\newcommand{\softmax}{\mathrm{softmax}}
\newcommand{\sigmoid}{\sigma}
\newcommand{\softplus}{\zeta}
\newcommand{\KL}{D_{\mathrm{KL}}}
\newcommand{\Var}{\mathrm{Var}}
\newcommand{\standarderror}{\mathrm{SE}}
\newcommand{\Cov}{\mathrm{Cov}}
% Wolfram Mathworld says $L^2$ is for function spaces and $\ell^2$ is for vectors
% But then they seem to use $L^2$ for vectors throughout the site, and so does
% wikipedia.
\newcommand{\normlzero}{L^0}
\newcommand{\normlone}{L^1}
\newcommand{\normltwo}{L^2}
\newcommand{\normlp}{L^p}
\newcommand{\normmax}{L^\infty}

\newcommand{\parents}{Pa} % See usage in notation.tex. Chosen to match Daphne's book.

\DeclareMathOperator*{\argmax}{arg\,max}
\DeclareMathOperator*{\argmin}{arg\,min}

\DeclareMathOperator{\sign}{sign}
\DeclareMathOperator{\Tr}{Tr}
\let\ab\allowbreak


\usepackage{hyperref}
\usepackage{url}
\usepackage{amsmath}
\usepackage{amssymb}
\usepackage{graphicx}
\usepackage{amsthm}
\usepackage{algorithm}
\usepackage{algpseudocode}
\usepackage{booktabs}% For theorem environments

\theoremstyle{definition}
\newtheorem{definition}{Definition}
\newtheorem{theorem}{Theorem}
\title{Leveraging Joint Predictive Embedding and Bayesian Inference in Graph Self Supervised Learning}

% Authors must not appear in the submitted version. They should be hidden
% as long as the tmlr package is used without the [accepted] or [preprint] options.
% Non-anonymous submissions will be rejected without review.

\author{\name Srinitish Srinivasan \email srinitish.srinivasan2021@vitstudent.ac.in \\
      \addr School of Computer Science and Engineering\\
      Vellore Institute of Technology
      \AND
      \name Omkumar CU \email omkumar.cu@vit.ac.in\\
      \addr School of Computer Science and Engineering\\
      Vellore Institute of Technology}

% The \author macro works with any number of authors. Use \AND 
% to separate the names and addresses of multiple authors.

\newcommand{\fix}{\marginpar{FIX}}
\newcommand{\new}{\marginpar{NEW}}

\def\month{MM}  % Insert correct month for camera-ready version
\def\year{YYYY} % Insert correct year for camera-ready version
\def\openreview{\url{https://openreview.net/forum?id=XXXX}} % Insert correct link to OpenReview for camera-ready version


\begin{document}


\maketitle

\begin{abstract}
Graph representation learning has emerged as a cornerstone for tasks like node classification and link prediction, yet prevailing self-supervised learning (SSL) methods face challenges such as computational inefficiency, reliance on contrastive objectives, and representation collapse. Existing approaches often depend on feature reconstruction, negative sampling, or complex decoders, which introduce training overhead and hinder generalization. Further, current techniques which address such limitations fail to account for the contribution of node embeddings to a certain prediction in the absence of labeled nodes. To address these limitations, we propose a novel joint embedding predictive framework for graph SSL that eliminates contrastive objectives and negative sampling while preserving semantic and structural information. Additionally, we introduce a semantic-aware objective term that incorporates pseudo-labels derived from Gaussian Mixture Models (GMMs), enhancing node discriminability by evaluating latent feature contributions. Extensive experiments demonstrate that our framework outperforms state-of-the-art graph SSL methods across benchmarks, achieving superior performance without contrastive loss or complex decoders. Key innovations include (1) a non-contrastive, view-invariant joint embedding predictive architecture, (2) leveraging single context and multiple targets relationship between subgraphs, and (3) GMM-based pseudo-label scoring to capture semantic contributions. This work advances graph SSL by offering a computationally efficient, collapse-resistant paradigm that bridges spatial and semantic graph features for downstream tasks. The code for our paper can be found at \url{https://github.com/Deceptrax123/JPEB-GSSL}.
\end{abstract}

\section{Introduction}
\label{sec:introduction}
Graph representation learning has found widespread adoption in Social Network Analysis, Recommendation Systems, Computer Vision and Natural Language Processing\cite{wu2020comprehensive}. It aims to learn low-dimensional embeddings of nodes or subgraphs while preserving their underlying spatial and spectral features. These learned embeddings can then be used on downstream tasks such as node classification\citep{maekawa2022beyond}, link prediction\citep{zhang2018link} and community detection\citep{li2024contrastive} by training task-specific decoders or classification layers on the learned embeddings keeping the backbone frozen. Such an approach reduces the computational complexity and training time on downstream tasks. Though Graph Neural Networks(GNNs) have gained popularity over time, they require a certain number of labelled nodes to perform well. Further, owing to the complexity of graph representations spatially, it is often challenging to pre-train graphs and transfer the learned embeddings to downstream tasks.

Graph Self Supervised learning has been widely studied in the literature to facilitate graph representation learning. This includes methods such as Node2Vec, DGI\citep{velivckovic2018deep}, MVGRL\citep{hassani2020contrastive}, GRACE\citep{zhu2020deep} etc. Although these methods have achieved state-of-the-art results on several social networks, they rely heavily on either graph reconstruction/feature reconstruction, re-masking or generating contrastive views by negative sampling, which is a computationally expensive process on large graphs. Graph-SSL methods that rely on reconstruction (generative methods) are decoder variant, meaning they depend on the decoder architecture(inner product decoder or symmetrical feature reconstruction). Such networks might require additional feature propagation steps such as skip connections to prevent vanishing gradients and retain learned information. Furthermore, most methods tend to use more layers or stack multiple encoders to capture long-range node interactions, which increases training complexity and leads to poor performance in downstream tasks due to over-smoothing. This diminishes the model’s ability to effectively discriminate between nodes effectively\citep{WANG2024106484}.

To address the limitations of existing methods, we introduce a joint embedding predictive framework that predicts subgraph embeddings of randomly sampled targets given a context conditioned on a latent variable $z$. The framework employs two encoders: a context encoder, which processes subgraphs sampled by randomly dropping nodes and updates its parameters dynamically through gradient descent, and a target encoder, whose weights are maintained as a moving average of the context encoder. The target encoder processes the original graph (without any view augmentations) to generate target node representations. Subsequently, three subgraphs are sampled from the latent space by randomly masking the target node representations. These sampled subgraphs are then fed into a global mean pooling operation to compute their embeddings, which serve as targets for a single context. To prevent representational collapse, we incorporate positional information of target nodes into the context embeddings before deriving subgraph views. A predictor network then maps the context embeddings with target positional information to the corresponding target embeddings. However, optimizing solely on context-target embeddings risks overlooking the graph's semantic information. To overcome this, we enhance node representations by introducing an additional term to the objective function, scoring pseudo-labels predicted from the node embeddings. This is achieved by fitting a Gaussian Mixture Model (GMM) on the embeddings and evaluating the contribution of each latent feature to the pseudo-labels. Through this design, we present a graph augmentation and view invariant self-supervised learning technique that avoids the need for negative samples or contrastive objectives based on mutual information estimators. This offers a robust alternative for representation learning in graph-based applications.

We list our main contributions as follows:
\begin{itemize}
    \item We introduce a novel graph self supervised learning method based on a joint predictive embedding paradigm which bypasses contrastive objectives such as Mutual Information(MI) Estimators, contrastive example generation techniques such as negative sampling and avoids noisy features. 

    \item Through our technique, we account for representation collapse by sampling multiple target embeddings for a single context, thereby enhancing the spread of node representations, and by providing positional information of target nodes to the context subgraph.
    
    \item We also account for contribution of node embeddings to pseudo-label level predictions by incorporating an additional term to the joint predictive objective. This is given by scoring pseudo-labels generated by fitting a Gaussian Mixture Model(GMM) on the learned node embeddings.

    \item We conduct extensive experiments demonstrating that our approach significantly outperforms previous state-of-the-art Graph Self-Supervised Learning (G-SSL) methods. Additionally, we provide a thorough emperical efficiency analysis highlighting the model's efficiency and scalability.
\end{itemize}


\section{Related Work}
\subsection{Unsupervised Representation Learning on Graphs}
\citep{jin2021automated} proposes a method for composing multiple self-supervised tasks for GNNs. They introduce a pseudo-homophily measure to evaluate representation quality without labels. \citep{zhang2021canonical} proposed the removal of negative sampling and MI estimator optimization entirely. The authors propose a loss function that contains an invariance term that maximizes correlation between embeddings of the two views and a decorrelation term that pushes different feature dimensions to be uncorrelated. \citep{hou2022graphmae} employs a re-mask decoding strategy and uses expressive GNN decoders instead of Multi-Layer Perceptrons(MLPs) enabling the model to learn more meaningful compressed representations. It lays focus on masked feature reconstruction rather than structural reconstruction. \citep{hassani2020contrastive} proposed MVGRL that makes use of two graph views and a discriminator to maximize mutual information between node embeddings from a first-order view and graph embeddings from a second-order view. It leverages both node and graph-level embeddings and avoids the need for explicit negative sampling. \citep{ju2022multi} proposed ParetoGNN which simultaneously learns from multiple pretext tasks spanning different philosophies. It uses a multiple gradient descent algorithm to dynamically reconcile conflicting learning objectives, showing state-of-the-art performance in node classification, clustering, link prediction and community prediction. 

\subsection{Bootstrapping Methods}
 \citep{ding2023eliciting} introduced  multi-scale feature propagation to capture long-range node interactions without oversmoothing. The authors also enhance inter-cluster separability and intra-cluster compactness by inferring cluster prototypes using a Bayesian non-parametric approach via Dirichlet Process Mixture Models(DPMMs). \citep{thakoor2021large} makes use of simple graph augmentations such as random node feature and edge masking, making it easier to implement on large graphs while achieving state-of-the-art results. It leverages a cosine similarity-based objective to make the predicted target representations closer to the true representations. \citep{jin2021multi} introduced a Siamese network architecture comprising an online and target encoder with momentum-driven update steps for the target. The authors propose 2 contrastive objectives i.e cross-network and cross view-contrastiveness. The Cross-network contrastive objective incorporates negative samples to push disparate nodes away in different graph views to effectively learn topological information.


\subsection{Joint Predictive Embedding Methods}
Joint predictive embedding has been explored in the field of computer vision and audio recognition. \citep{assran2023self} introduced I-JEPA which eliminates the need for hand-crafted augmentations and image reconstruction. They make use of a joint-embedding predictive model that predicts representations of masked image regions in an abstract feature space rather than in pixel space, allowing the model to focus on high-level semantic structures. It makes use of a Vision Transformer(ViT) with a multi-block masking strategy ensuring that the predictions retain semantic integrity. \citep{fei2023jepa}  extends the masked-modeling principle from vision to audio, enabling self-supervised learning on spectrograms. The key technical contribution is the introduction of a curriculum masking strategy, which transitions from random block masking to time-frequency-aware masking, addressing the strong correlations in audio spectrograms.

\section{Methodology}
\begin{figure*}[h!]
    \centering
    \includegraphics[scale=0.32]{framework_diagram.png}
    \caption{Overall Framework Diagram}
    \label{fig:diag}
\end{figure*}
\subsection{Preliminaries}
Consider the definition of a Graph $G=(V,E)$. Let $V$ be the set of vertices $\{v_1,v_2,v_3....v_{n_v}\}$ and $E$ be the set of edges $\{e_1,e_2,e_3...e_{n_e}\}$. $n_v,n_e$ are
the number of nodes and edges respectively in $G$. Each node in $V$ is characterized by a $d$ dimensional vector. This is the initial signal that is populated by either bag of words or binary values depending on the problem considered.
\subsection{Loss Function}
\subsubsection{Joint Predictive Optimization}
We sample a subgraph $G'$ from graph $G$ by dropping a set of nodes according to a Bernoulli distribution parameterized by success probability $p_1$. The subgraph $G'$ is passed into the context encoder to output node embeddings $H'$ of dimensions $d'$. Meanwhile, the graph $G$ is passed into the target encoder to generate node embeddings $H$ with the same dimensions as $H'$. At the latent space, we sample 3 target subgraphs from the context subgraph $G'$, again according to a Bernoulli distribution with probability $p_2$ such that $p_2<p_1$. The node embeddings of each subgraph are obtained by deactivating the node features of masked nodes, which is followed by global mean pooling operation, thus obtaining embeddings $H^t_1$,$H^t_2$ and $H^t_3$. The context node level embeddings $H'$ is passed into 3 predictors, corresponding to the three target subgraphs. The obtained predictions are then pooled by the mean pooling operation, thus resulting in target predictions $H'_1,H'_2,H'_3$, conditioned on latent variable $H'$. We then frame the objective function for the joint predictive component as follows,
\begin{align}
    \mathcal{L}^J(\Theta^C,\Theta^R)=\frac{1}{T}\Sigma_{k=1}^{k=T}||H'_k-H^t_k||^2
\end{align}
where $T$ is the total number of targets, $\Theta^C$ is the weight matrix of the context encoder and $\Theta^R$ is the weight matrix of the target encoder. The objective is computed at the latent space with no reconstruction/negative sampling involved. 


\subsubsection{Node Feature Contribution Optimization}
The node embeddings $H'$ obtained from passing $G'$ into the context encoder are fit in a Gaussian Mixture Model(GMM). We aim to obtain,
\begin{equation}
   \label{eq:gmm_eqn}
    p(z_k=1|h'_n)=\frac{p(h'_n|z_k=1)p(z_k=1)}{\Sigma_{j=1}^{j=K}p(h'_n|z_j=1)p(z_j=1)}
\end{equation}
where $z$ is a latent variable that takes 2 values: one if $h'$ comes from Gaussian $k$, and zero otherwise. We can obtain,
\begin{align}
    p(z_k=1)=\pi_k\\
    p(h'_n|z_k=1)=\mathcal{N}(h'_n|\mu_k,\Sigma_k)
\end{align}
where $\pi_k$ is the mixing coefficient such that,
\begin{equation}
    \Sigma_{k=1}^{k=K}\pi_k=1
\end{equation}

On replacing these equations in eq.\ref{eq:gmm_eqn}, we obtain
\begin{equation}
    \label{eqn:gamma}
    p(z_k=1|h'_n)=\frac{\pi_k\mathcal{N}(h'_n|\mu_k,\Sigma_k)}{\Sigma\pi_j\mathcal{N}(h'_n|\mu_j,\Sigma_j)}=\gamma(z_{nk})
\end{equation}

\textbf{E-Step: }
In the E-Step, we aim to evaluate,
\begin{equation}
    \mathcal{Q}(\theta^*,\theta)=\mathbb{E}[\ln p(H',Z|\theta^*)]=\Sigma_Zp(Z|H',\theta)\ln p(H',Z|\theta^*)
\end{equation}
from eq.\ref{eqn:gamma}, we can substitute $\gamma$ in the above equation as,
\begin{equation}
    \mathcal{Q}(\theta^*,\theta)=\Sigma_Z\gamma(z_{nk})\ln p(H',Z|\theta^*)
\end{equation}
On finding the complete likelihood of the model, we finally have,
\begin{equation}
    \label{eq:e_step}
    \mathcal{Q}(\theta^*,\theta)=\Sigma_{n=1}^N\Sigma_{k=1}^K\gamma(z_{nk})[\ln \pi_k+\ln \mathcal{N}(h'_n|\mu_k,\Sigma_k)]
\end{equation}

\textbf{M-Step: }
In the M-Step, we aim to find updated parameters $\theta^*$ as follows,
\begin{equation}
    \theta^*=\arg\max_\theta\mathcal{Q}(\theta^*,\theta)
\end{equation}
Considering the restriction, $\Sigma_{k=1}^K\pi_k=1$, eq.\ref{eq:e_step} is modified as follows,
\begin{equation}
    \mathcal{Q}(\theta^*,\theta)=\Sigma_{n=1}^N\Sigma_{k=1}^K\gamma(z_{nk})[\ln \pi_k+\ln \mathcal{N}(h'_n|\mu_k,\Sigma_k)]-\lambda(\Sigma_{k=1}^{k=K}\pi_k-1)
\end{equation}
The parameters are then determined by finding the maximum likelihood of $\mathcal{Q}$. On taking the derivative with respect to $\pi_k,\mu_k,\Sigma_{k}$ and rearranging terms, we finally obtain the update equations for the parameters as follows,
\begin{equation}
    \pi_k=\frac{\Sigma_{n=1}^{n=N}\gamma(z_{nk})}{N}
\end{equation}
\begin{equation}
\mu_k^* = \frac{\sum_{n=1}^N \gamma(z_{nk}) h'_n}{\sum_{n=1}^N \gamma(z_{nk})}
\end{equation}
\begin{equation}
\sum_{k}^{*} = \frac{\sum_{n=1}^{N} \gamma (z_{nk}) (h'_n - \mu_k) (h'_n - \mu_k)^T}{\sum_{n=1}^{N} \gamma (z_{nk})}
\end{equation}

\textbf{Parameter $\Theta^C$ Update:  }
Let $V_g$ be a vector of pseudo-labels obtained for each node from the Gaussian Mixture Model(GMM) and $V_k$ be the vector of pseudo-labels obtained by clustering node embeddings $H'$ by K-Means. We update the context encoder parameters $\Theta^c$ with the following objective, 
\begin{equation}
    \mathcal{L}^G=\begin{cases}
    0.5(\frac{V_g^TH'}{\Sigma V_g||V_g||}-\frac{V_k^TH'}{\Sigma V_k||V_k||})/\beta, & \text{if} |\frac{V_g^TH'}{\Sigma V_g||V_g||}-\frac{V_k^TH'}{\Sigma V_k||V_k||}|<\beta\\

    |\frac{V_g^TH'}{\Sigma V_g||V_g||}-\frac{V_k^TH'}{\Sigma V_k||V_k||}|-0.5\beta, & \text{otherwise}

    \end{cases}
\end{equation}

In the above equation, the context encoder parameters $\Theta^c$ are updated by the smooth $L_1$ loss function.

\subsubsection{Final objective}
The context encoder parameters $\Theta^C$ are finally updated as follows,
\begin{equation}
    \mathcal{L}=\mathcal{L^J}+\mathcal{L^G}
\end{equation}
\begin{equation}
    \Theta^C\leftarrow\text{optimize}(\Theta^c,\alpha,\partial_{\Theta^c}\mathcal{L})
\end{equation}
where, $\alpha$ is the learning rate for the Adam optimizer. The weights of the context and target encoder are randomly initialized by a standard normal distribution. A cosine annealing learning rate scheduler with early stopping is used in all experiments.

\subsection{Description of Components and Operations}
\subsubsection{Context Encoder}
The node features from $G'$ are passed into the context encoder. The context encoder is a simple 3-layer GCN encoder which predicts 128, 256 and 512 features in each layer respectively. The forward propagation for each layer is described as follows,
\begin{align}
    X'=g(\hat{D}^{\frac{-1}{2}}\hat{A}\hat{D}^{\frac{=1}{2}}X\Theta)\\
    \hat{A}=A+I
\end{align}
where $\hat{D}$ is the degree matrix, $\hat{A}$ is the adjacency matrix with added self-loops and $\Theta$ is the learned weights matrix. $g$ is a non-linear function. We use ReLU for the first two layers and Tanh for the final layer. The context encoder is an online encoder whose weights are updated by gradient descent.

\subsubsection{Target Encoder}
The target encoder inputs node features from $G$. Similar to the context encoder, it is a 3-layer GCN encoder with same number of hidden, output dimensions and forward propagation steps. The weights of the target encoder are updated as a moving average of the context encoder as follows,
\begin{equation}
    \Theta_{s}^R=m\Theta^R_{s-1}+(1-m)\Theta_s^C
\end{equation}
where $\Theta^R$ is the weights matrix of the target encoder, $\Theta^C$ is the weights of the context encoder, $m$ is the momentum parameter and $s$ refers to the $s$th iteration.

\subsubsection{Predictor}
The predictor consists of 2 GCN layers, both predicting 512 features. The activation function $g$ for both layers is Tanh, in line with the final layer of the target encoder. We use Tanh since it is a bounded function, thus stabilizing the loss computation and optimization.

\subsubsection{Global Mean Pooling Operation}
The global mean pool operation for a graph $G$ is given as follows,
\begin{equation}
    r=\frac{1}{N}\Sigma_{n=1}^{n=N}x_n
\end{equation}
where $x_n$ refers to the node features of node $n$ and $N$ is the total number of nodes in graph $G$

\subsection{Implementation Details}
For all experiments, we use the Adam Optimizer\citep{kingma2014adam} and cosine annealing learning rate scheduler with warm restarts. No regularization techniques such as dropout, $L_1$ or $L_2$ have been employed since they tend to reduce performance. For learning rates, we perform a search with values corresponding to the search space $\{0.5,0.1,0.05,0.01,0.001\}$. We set the value of the momentum update parameter $m$ to 0.9. The number of epochs was set to 50,000 with early stopping. All experiments were conducted on an M2 Macbook Pro with 8GB CPU.

\section{Experiments}
\subsection{Experimental Setting}
\subsubsection{Datasets}
In our experiments, we evaluate our proposed framework on seven publicly available benchmark datasets for node representation learning and classification. The datasets are, namely, Cora\citep{sen2008collective}, Pubmed\citep{namata2012query}, Citeseer\citep{sen2008collective}, Amazon Photos\citep{shchur2018pitfalls}, Amazon Computers\citep{shchur2018pitfalls}, Coauthor CS\citep{shchur2018pitfalls} and WikiCS\citep{mernyei2020wiki}. Details on number of nodes, edges and features are given in table \ref{tab:data}. Cora consists of 2708 scientific publications classified into one of seven classes, Citeseer consists of 3312 scientific publications classified into one of six classes and Pubmed consists of 19717 scientific publications pertaining to diabetes classified into one of three classes. In Amazon Photos and Amazon Computers, nodes represent goods and edges represent that two goods are frequently bought together. The product reviews are represented as bag-of-words features. Coauthor CS contains paper keywords for each author's papers. Nodes represent authors that are connected by an edge if they co-authored a paper. WikiCS is a dataset derived from Wikipedia with nodes corresponding to Computer Science articles and edges based on hyperlinks. The 10 classes represent different branches of Computer Science. 

\begin{table}[h!]
    \centering
    \caption{\textbf{Dataset Details}}
    \begin{tabular}{lcccccc}
        \toprule
        \textbf{Dataset} & \textbf{Nodes} & \textbf{Edges} & \textbf{Features} & \textbf{Classes} \\
        \midrule
        Cora             & 2,708          & 10,556         & 1,433             & 7                \\
        Citeseer         & 3,327          & 9,104          & 3,703             & 6                \\
        Pubmed           & 19,717         & 88,648         & 500               & 3                \\
        Amazon Photos    & 7,650          & 238,612        & 745               & 8                \\
        Amazon Computers & 13,752         & 491,722        & 767               & 10               \\
        Coauthor CS      & 18,333         & 163,788        & 6,805             & 15               \\
        WikiCS           &  11,701              & 216,123               &300                   & 10                 \\
        \bottomrule
    \end{tabular}
    \label{tab:data}
\end{table}


\subsubsection{Evaluation Methodology}
The embeddings are evaluated on node classification, whose performance is quantified by accuracy. We follow the same evaluation protocol as used in \citep{velivckovic2017graph}, \citep{ding2023eliciting}, \citep{ju2022multi} etc. All scores for baselines have been obtained from previously published papers. 

\subsubsection{Evaluation Protocol}
For all downstream tests, we follow the linear evaluation protocol on graphs where the parameters of the backbone/encoder are frozen during inference time. Only the prediction head, which is a single GCN layer, is trained for node classification. For evaluation purposes, we use the default number for train/val/test splits for the citation networks i.e. Cora, Pubmed, Citeseer which are 500 validation and 1000 testing nodes. The train/val/test splits for the remaining datasets, namely, Amazon Photos, Amazon Computers and Coauthor CS are according to \citep{shchur2018pitfalls}. For WikiCS we use the publicly available splits. Unless otherwise mentioned, performance is averaged over 10 independent runs with different random seeds and splits for all seven evaluation datasets. We report the mean and standard deviation obtained across 10 runs. 

\subsubsection{Environment}
To ensure a fair comparison, for all datasets, we use the same number of layers for the GNN encoder and the same number of features at each hidden layer. The encoder predicts 512 hidden features in the latent space for all datasets. The number of output features of the final classification head is the only parameter varied. We use a Cosine Annealing learning rate scheduler with warm restarts after 75 epochs, and early stopping is employed for all experiments. 

\subsection{Evaluation Results}
\subsubsection{Performance on Node Classification on small and large graphs}
As mentioned earlier, we use the linear evaluation protocol and report the mean and standard deviation of classification accuracy on the test nodes over 10 runs on different folds or splits with different seeds. We compare our proposed approach with supervised and fine-tuned model baselines. For semi-supervised node classification, we compare our proposed approach against Multi Layer Perceptron(MLP),Graph Convolution Network(GCN)\citep{kipf2016semi}, Graph Attention Network(GAT)\citep{velivckovic2017graph}, Simplified GCN\citep{wu2019simplifying}, Logistic Regression and GraphSAGE\citep{hamilton2017inductive}. For GraphSAGE, we use the mean, maxpool and meanpool variants as described in \citep{shchur2018pitfalls}. We have compared our model's performance against semi-supervised baselines in table \ref{tab:supervision}. For self-supervised and fine-tuned node classification, we compare our proposed approach against DGI, MVGRL\citep{hassani2020contrastive}, GRACE\citep{zhu2020deep}, CCA-SSG\citep{zhang2021canonical}, SUGRL\citep{mo2022simple}, S3-CL\citep{ding2023eliciting}, GraphMAE\citep{hou2022graphmae}, GMI\citep{peng2020graph}, BGRL\citep{thakoor2021large} and ParetoGNN\citep{ju2022multi} on small and large graphs.  In semi-supervised node classification (Table \ref{tab:supervision}), our method achieves the highest accuracy across all baselines. On self-supervised classification followed by fine-tuning, we evaluate our proposed model against two specific fields of data i.e Planetoid datasets and larger, more stable datasets such as Amazon Computers, Photos and WikiCS. Table \ref{tab:fine_tuned} compares our model's performance against strong baselines on Planetoid datasets. We show that our model performs extremely well on smaller datasets by outperforming all baselines on Cora and Citeseer with extremely competitive results on Pubmed. Table \ref{tab:med_datasets} contains the performance of our model against strong baselines for Photos, Computers, Coauthor CS and WikiCS. Our model consistently achieves competitive scores against baselines by outperforming baselines on Photos and CS while remaining competitive on Computers and WikiCS .These results underscore the approach's adaptability and strong performance across diverse dataset sizes. Owing to the ability of our proposed approach to avoid noisy features, representation collapse, and leverage semantic information, our proposed model performs well on small and unstable datasets such as Cora and Citeseer and large datasets such as Amazon, Coauthor and WikiCS. 

\begin{table*}[h!]
\centering% Reduce column spacing
\footnotesize
\caption{\textbf{Semi-Supervised Node Classification.} The values indicate the accuracy achieved on node classification by several methods. A higher value indicates better performance. The best score is marked in \textbf{bold} and the second-best score is \underline{underlined}. N/A indicates that the score was not reported for a particular dataset by the original authors.}

\begin{tabular}{lccccccc}
\toprule

\textbf{Method}& \textbf{Cora} & \textbf{Citeseer} & \textbf{Pubmed} & \textbf{Photos} & \textbf{Computers} & \textbf{Coauthor CS} \\
\midrule
MLP & $55.2 \pm 0.4$ & $46.5 \pm 0.5$ & $71.4 \pm 0.3$ & $78.5 \pm 0.2$ & $44.9 \pm 5.8$ & $76.5 \pm 0.3$ \\
GCN & $81.5 \pm 1.3$ & \underline{$71.9 \pm 1.9$} & $77.8 \pm 2.9$ & $91.2 \pm 1.2$ & \underline{$82.6 \pm 2.4$} & $91.1 \pm 0.5$ \\
GAT & \underline{$81.8 \pm 0.3$} & $71.4 \pm 1.8$ & $78.7 \pm 2.3$ & $85.7 \pm 20.3$ & $78.0 \pm 19.0$ & $90.5 \pm 0.6$ \\
Simplified GCN & $81.5 \pm 0.2$ & $73.1 \pm 0.1$ & \underline{$79.7 \pm 0.4$} & $88.3 \pm 1.1$ & N/A & \underline{$91.5 \pm 0.3$} \\
GraphSage Mean & $79.2 \pm 7.7$ & $71.6 \pm 1.9$ & $77.4 \pm 2.2$ & \underline{$91.4 \pm 1.3$} & $82.4 \pm 1.8$ & $91.3 \pm 2.8$ \\
GraphSage MaxPool & $76.6 \pm 1.9$ & $67.5 \pm 2.3$ & $76.1 \pm 2.3$ & $90.4 \pm 1.3$ & N/A & $85.0 \pm 1.1$ \\
GraphSage MeanPool & $77.9 \pm 2.4$ & $68.6 \pm 2.4$ & $76.5 \pm 2.4$ & $90.7 \pm 1.6$ & $79.9 \pm 2.3$ & $89.6 \pm 0.9$ \\
Logistic Regression & $57.1 \pm 2.3$ & $61.0 \pm 2.2$ & $64.1 \pm 3.1$ & $73.0 \pm 6.5$ & $64.1 \pm 5.7$ & $86.4 \pm 0.9$ \\
Ours & \textbf{89.8 ± 0.9} & \textbf{77.0 ± 0.9} & \textbf{85.7 ± 1.0} & \textbf{94.5 ± 0.5} & \textbf{88.0 ± 0.6} & \textbf{94.5 ± 0.4} \\
\bottomrule
\end{tabular}
\label{tab:supervision}
\end{table*}


\begin{table*}[h!]
    \centering
    \caption{\textbf{Self Supervised Pre-Training followed by classification on Planetoid Datasets.} The values indicate the accuracy achieved on node classification by several SSL methods. A higher value indicates better performance. The best score is marked in bold and the second-best score is \underline{underlined}.}
    \renewcommand{\arraystretch}{1.2}
    \begin{tabular}{lccc}
        \toprule
       
       \textbf{Method} & \textbf{Cora} & \textbf{Citeseer} & \textbf{Pubmed} \\
        \midrule
        DGI             & $81.7\pm0.6$  & $71.5\pm0.7$  & $77.3\pm0.6$  \\
        MVGRL           & $82.9\pm0.7$  & $72.6\pm0.7$  & $79.4\pm0.3$  \\
        GRACE           & $80.0\pm0.4$  & $71.7\pm0.6$  & $79.5\pm1.1$  \\
        CCA-SSG         & $84.2\pm0.4$  & $73.1\pm0.3$  & $81.6\pm0.4$  \\
        SUGRL           & $83.4\pm0.5$  & $73.0\pm0.4$  & $81.9\pm0.3$  \\
        S3-CL           & $\underline{84.5 \pm 0.4}$ & $\underline{74.6 \pm 0.4}$ & $80.8 \pm 0.3$  \\
        GraphMAE        & $84.2\pm0.4$  & $73.1\pm0.4$  & $83.9\pm0.3$  \\
        GMI             & $82.7\pm0.2$  & $73.3\pm0.3$  & $77.3\pm0.6$  \\
        BGRL            & $83.8\pm1.6$  & $72.3\pm0.9$  & $\mathbf{86.0\pm0.3}$  \\
        Ours            & $\mathbf{89.8 \pm 0.9}$ & $\mathbf{77.0 \pm 0.9}$ & $\underline{85.7 \pm 1.0}$  \\
        \bottomrule
    \end{tabular}
    \label{tab:fine_tuned}
\end{table*}

\begin{table*}[h!]
    \centering
    \caption{\textbf{Self Supervised Pre-Training followed by classification on datasets mentioned in \cite{shchur2018pitfalls} and WikiCS.} The best score for each dataset is marked in bold and the second-best score is \underline{underlined}.}
    \renewcommand{\arraystretch}{1.2}
    \begin{tabular}{lcccc}
        \toprule
        \textbf{Method} & \textbf{Photos} & \textbf{Computers} & \textbf{Coauthor CS} & \textbf{WikiCS} \\
        \midrule
        DGI             & $91.6\pm0.2$  & $83.9\pm0.5$  & $92.1\pm0.6$  & $75.3\pm0.1$  \\
        GRACE           & $92.1 \pm 0.5$ & $86.7 \pm 0.8$ & $93.2 \pm 0.4$ & $77.5 \pm 0.6$ \\
        BGRL            & $93.2\pm0.3$  & \underline{$90.3\pm0.2$}  & \underline{$93.3\pm0.1$}  & $80.0\pm0.1$  \\
        ParetoGNN       & \underline{$93.8\pm0.3$}  & $\mathbf{90.7\pm0.2}$ & $92.2\pm0.1$  & $\mathbf{82.9\pm0.1}$ \\
        MVGRL           & $93.2\pm0.3$  & $87.5\pm0.1$  & $92.1\pm0.1$  & $77.5\pm0.1$  \\
        Random Weights  & $92.1\pm0.5$  & $86.5\pm0.4$  & $91.6\pm0.3$  & $78.9\pm0.6$  \\
        Ours            & $\mathbf{94.5 \pm 0.5}$ & $88.0 \pm 0.6$ & $\mathbf{94.5 \pm 0.4}$ &\underline{$82.4\pm1.0$}  \\
        \bottomrule
    \end{tabular}
    \label{tab:med_datasets}
\end{table*}

\subsubsection{Study on GMM Cluster Optimization}
In this section, we evaluate the significance of the constraint in our loss function. Our approach combines learning sub-graph embeddings with optimizing node embeddings using pseudo-labels derived from a Gaussian Mixture Model (GMM). These pseudo-labels, converted to normalized scores, serve as a constraint in the original loss function for subgraph embeddings. Table \ref{tab:ablation} illustrates the performance of our method with and without this constraint, demonstrating a substantial improvement when the GMM-derived pseudo-label scores are incorporated. This finding is particularly noteworthy as it indicates that computationally expensive processes like negative sampling, commonly used in graph contrastive learning, may not be essential for optimizing node embeddings. Our approach thus offers a more efficient alternative while maintaining, and even enhancing, performance in graph representation learning tasks.
\begin{table*}[h!]
    \centering
    \footnotesize
    \caption{\textbf{Ablation Study for Loss Function.} The best score for each dataset is marked in bold.}
    \begin{tabular}{lccccccc}
        \toprule
        \textbf{Method} & \textbf{Cora} & \textbf{Citeseer} & \textbf{Pubmed} & \textbf{Photos} & \textbf{Computers} & \textbf{Coauthor CS} &\textbf{WikiCS} \\
        \midrule
        Without Bayesian Inference & 89.0 & 74.1 & 84.2 & 93.4 & 85.0 & 93.2 &81.5 \\
        With Bayesian Inference    & \textbf{89.8} & \textbf{77.0} & \textbf{85.7} & \textbf{94.5} & \textbf{88.0} & \textbf{94.5} & \textbf{82.4}\\
        \bottomrule
    \end{tabular}
    \label{tab:ablation}
\end{table*}
\subsubsection{Study on Momentum Parameter $m$}
Figure \ref{fig:var_plot} represents the change in performance with respect to the momentum parameter on the citeseer dataset. The ideal values for $m$ may range between 0.8 and 0.9. As $m\rightarrow 1$, the weights of the target encoder update by very small steps, thereby slowing down learning. This is further indicated by the drop in performance. For all our experiments, we use $m=0.9$.

\begin{figure}[h!]
    \centering
    \includegraphics[scale=0.7]{momentum_accuracy.pdf}
    \caption{Variation of accuracy with respect to momentum parameter $m$ on Citeseer}
    \label{fig:var_plot}
\end{figure}

\subsubsection{Efficiency Analysis for Scalability}
We evaluate the model's efficiency in terms of the parameter size and memory consumption. We show that our model is among the most efficient of all baselines while outperforming most baselines. We compare the efficiency of our proposed model against reported baselines in table \ref{tab:efficiency}. 

\begin{table*}[h!]
    \centering
    \caption{\textbf{Efficiency Comparison among Baselines on Cora, Pubmed, and Citeseer.} Lower the number of parameters, more efficient is the model}
    \begin{tabular}{lcc|cc|cc}
        \toprule
        \textbf{Methods} & \multicolumn{2}{c|}{\textbf{Cora}} & \multicolumn{2}{c|}{\textbf{Citeseer}} & \multicolumn{2}{c}{\textbf{Pubmed}} \\
        \cmidrule(lr){2-3} \cmidrule(lr){4-5} \cmidrule(lr){6-7}
        & \textbf{Memory (GB)} & \textbf{Params} 
        & \textbf{Memory (GB)} & \textbf{Params} 
        & \textbf{Memory (GB)} & \textbf{Params} \\
        \midrule
        DGI    & 3.73  & $7.3\times10^5$  & 3.85  & $1.9\times10^6$  & 3.66  & $2.6\times10^5$  \\
        GMI    & 4.06  & $9.9\times10^5$  & 4.20  & $2.2\times10^6$  & 3.93  & $5.2\times10^5$  \\
        MVGRL  & 2.25  & $9.9\times10^5$  & 2.55  & $2.2\times10^6$  & 2.37  & $5.2\times10^5$  \\
        SUGRL  & 1.57  & $9.7\times10^5$  & 1.71  & $2.6\times10^6$  & 1.70  & $3.9\times10^5$  \\
        S$^3$-CL & 1.37  & $7.3\times10^5$  & 1.62  & $1.9\times10^6$  & 1.54  & $2.6\times10^5$  \\
        \midrule
        Ours   &  0.70    &$3.5\times10^5$               & 0.80      &$6.4\times10^5$                  &      1.90 &$2.3\times10^5$                  \\
        \bottomrule
    \end{tabular}
    \label{tab:efficiency}
\end{table*}



\subsubsection{Study on Performance against Test Time Node Feature Distortion}
We evaluate the performance of our method on node classification by augmenting the test splits. We sample $p$ percentage test nodes randomly and replace their node features with points sampled from a standard Gaussian distribution. We evaluate the performance of the distorted test nodes on Amazon Computers, WikiCS, Amazon Photos and Coauthor CS by varying $p$ between 10\% and 40\% and report their performance differences in table \ref{tab:distortion}. In this study, we only use WikiCS, Co-author and Amazon networks since Planetoid datasets tend to be unstable on evaluation as indicated in \citep{shchur2018pitfalls}. Despite no further fine-tuning on distorted node features, the average performance drop on Amazon Photos and Computers is only 2.96\% and 4.01\% respectively. This indicates the ability of the joint framework to generalize on noise-augmented views as well. According to our hypothesis, we believe this is due to the nature of joint predictive embedding learning where a higher variation in context and target embeddings is required for better generalization. 

\begin{table}[h!]
    \centering
    \caption{\textbf{Performance on distorting node features at test time.} Values represent the percentage decrease in the model’s performance compared to its original score. A lower value indicates better performance.}
    \begin{tabular}{lcccc}
        \toprule
        \textbf{Ratio} & \textbf{Photos} & \textbf{Computers} & \textbf{Coauthor CS} & \textbf{WikiCS} \\
        \midrule
        0.10 & -3.3\% & -3.8\% & -3.3\% &-8.9\%  \\
        0.15 & -0.8\% & -3.4\% & -1.2\% &-9.7\%  \\
        0.20 & -3.7\% & -3.3\% & -3.3\% & -13.6\% \\
        0.25 & -2.9\% & -4.6\% & -7.7\% &-12.6\%  \\
        0.30 & -3.9\% & -4.3\% & -5.9\% & -13.2\%\\
        0.35 & -2.4\% & -4.0\% & -6.6\% & -12.3\%\\
        0.40 & -3.7\% & -4.7\% & -9.6\% & -15.2\% \\
        \bottomrule
    \end{tabular}
    \label{tab:distortion}
\end{table}

\section{Conclusion}
In this paper, we propose a novel Graph Self-Supervised Learning (Graph-SSL) framework that combines joint predictive embedding and pseudo-labeling to effectively capture global knowledge while avoiding noisy features and bypassing contrastive methods such as negative sampling and reconstruction. The joint predictive embedding framework leverages the context-target relationship between node embeddings in the latent space by predicting multiple target embeddings for a single context. This approach, combined with the optimization of node feature contributions to pseudo-labels, enables a lightweight Graph Neural Network (GNN) encoder to capture intricate patterns in both graph structure and node features without requiring the stacking of multiple layers or encoders. Additionally, our method addresses the node representation collapse problem by incorporating information from multiple targets for a single context, ensuring robust and diverse embeddings. Through extensive experiments on multiple benchmark graph datasets, we demonstrate that our proposed framework achieves superior performance compared to several state-of-the-art graph self-supervised learning methods.

\bibliography{main}
\bibliographystyle{tmlr}

\appendix

\section{Declarations}

\subsection{Availability of Data and Materials}
The datasets used in all experiments during this study are available in a publicly accessible repository. They can be obtained at \url{https://pytorch-geometric.readthedocs.io/en/latest/modules/datasets.html}

\subsection{Code Availability}
The codes used for all experiments in this study are designed by us and is available at \url{https://github.com/Deceptrax123/JPEB-GSSL}

\subsection{Conflict of Interest}
The authors declare that they have no conflicts of interest to report regarding the present study.

\end{document}