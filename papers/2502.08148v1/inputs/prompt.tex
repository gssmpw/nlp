\begin{table*}[!h]
    \centering
    \begin{tabular}{p{14cm}}
\toprule
\textbf{PROMPT: Pairwise Causal Discovery}\\
\midrule

\texttt{Given the two events:} \\ 

\texttt{event\_a: <Input the first event>}\\

\texttt{event\_b: <Input the second event>}\\

\texttt{Which cause-and-effect relationship is more likely between two events?} \\ 

\texttt{A. event\_a causes event\_b.}.  \\ 

\texttt{B. event\_b causes event\_a.} \\

\texttt{C. There are no cause-effect relation between two events.} \\

\texttt{Let’s work this out in a step by step way to be sure that we have the right answer. Then provide one final answer within the tags <answer>A or B or C</answer>.} \\

\bottomrule
\end{tabular}
\caption{Prompt for the pairwise causal discovery task.}
\label{tab:prompt_causal_discovery}
\end{table*}

\begin{table*}[!h]
    \centering
    \begin{tabular}{p{14cm}}
\toprule

\textbf{PROMPT: Multi-choice Answer Generation on Specific-QA (zero-shot COT)}\\
\midrule

\texttt{Given the following story: <Input story context>}.  \\ 

\texttt{What could be the <cause/effect> of the event <Input target effect/cause event>?} \\

\texttt{Choose one or more correct answers out of the following choices: <Input answer choices>.}

(*) \texttt{This information can help answer the question: A possible <cause/effect> of the event <Input effect/cause event abstraction> is <Input cause/effect event abstraction>.} 

\texttt{Let’s work this out in a step-by-step way to be sure that we have the right answer. Then provide your final answer beginning with `The correct answer(s):' followed by a list of the indices of the correct answers.} \\
\bottomrule
\end{tabular}
\caption{Prompt for the specific multi-choice answer generation on \texttt{GLUCOSE}. (*) This line is removed for the experiments that do not involve causal graphs.}
\label{tab:prompt_specific_qa}
\end{table*}


\begin{table*}[!h]
    \centering
    \begin{tabular}{p{14cm}}
\toprule
\textbf{PROMPT: Multi-choice Answer Generation on Abstract-QA (zero-shot COT)}\\
\midrule

\texttt{Given the following story: <Input story context>}.  \\ 

\texttt{The story describes an event where <Input generalization of target effect/cause event>. What could be the <cause/effect> of the event?} \\

\texttt{Choose one or more correct answers out of the following choices: <Input answer choices>.}

(*) \texttt{This information can help answer the question: A possible <cause/effect> of the event <Input effect/cause event abstraction> is <Input cause/effect event abstraction>.} 

\texttt{Let’s work this out in a step-by-step way to be sure that we have the right answer. Then provide your final answer beginning with `The correct answer(s):' followed by a list of the indices of the correct answers.} \\
\bottomrule
\\
\textbf{PROMPT: Multi-choice Answer Generation on Abstract-QA (bi-level COT)}\\
\midrule

\texttt{Given the following story: <Input story context>}.  \\ 

\texttt{The story describes an event where <Input generalization of target effect/cause event>. What could be the <cause/effect> of the event?} \\

\texttt{Choose one or more correct answers out of the following choices: <Input answer choices>.}

(*) \texttt{This information can help answer the question: A possible <cause/effect> of the event <Input effect/cause event abstraction> is <Input cause/effect event abstraction>.} 

\texttt{The event [Input generalization of target effect/cause event> is described by one of the sentences in the story context. First identify that part of the story. Then retrieve the event mentioned in the story that is a corresponding cause/effect.".} 

\texttt{Let’s work this out in a step-by-step way to be sure that we have the right answer. Then provide your final answer beginning with `The correct answer(s):' followed by a list of the indices of the correct answers.} \\
\bottomrule
\end{tabular}
\caption{Prompts for the abstract multi-choice answer generation on \texttt{GLUCOSE}. (*) This line is removed for the experiments that do not involve causal graphs.}
\label{tab:prompt_abstract_qa}
\end{table*}


\begin{table*}[]
    \centering
    \begin{tabular}{p{14cm}}
\toprule
\textbf{PROMPT: Abstract Event Identification}\\
\midrule
\texttt{We need to convert the input sentence into a more general expression. The conversion consists of three steps.} \\

\texttt{First, identifying: identify entities and verb words.} \\ 
\texttt{Second, conversion: convert the entities with more generic words and transform the verb words into the base form.} \\
\texttt{Third, further conversion: convert the sentence into a more general expression.} \\
\texttt{Note: The generic expressions used in the conversion are placeholders for the specific details in the original sentence.}

---------------------------------------------------------- \\ 
\texttt{The following is a conversion example.} \\

\texttt{Original Sentence:} \textit{John went to buy a new collar for his dog.}

\texttt{1. Identifying:}
\begin{itemize}[noitemsep]
    \item \texttt{Person:} \textit{John}
    \item \texttt{Action:} \textit{went, buy}
    \item \texttt{Object:} \textit{a new collar}
    \item \texttt{Possession:} \textit{his dog}
\end{itemize} 
        
\texttt{2. Conversion:} \textit{a person go to buy another thing for something} \\

\texttt{3. Further Conversion:} \textit{a person buy something to do something}

---------------------------------------------------------- \\
        
\texttt{The following is another example.}

\texttt{Original Sentence:} \textit{John drives near the woman.} 

\texttt{1. Identifying:}
\begin{itemize}[noitemsep]
    \item \texttt{Person:} \textit{John}
    \item \texttt{Action:} \textit{drives}
    \item \texttt{Object:} \textit{the woman}
    \item \texttt{Preposition:} \textit{near}
\end{itemize} 
        
\texttt{2. Conversion:} \textit{a person see another person} \\

\texttt{3. Further Conversion:} \textit{a person see another person}


---------------------------------------------------------- \\
        
% \texttt{The following is another example.}

% \texttt{Original Sentence:} \textit{Alex saw the same driver in the bus.}

% \texttt{1. Identifying:}
% \begin{itemize}[noitemsep]
%     \item \texttt{Person:} \textit{Alex}
%     \item \texttt{Action:} \textit{saw}
%     \item \texttt{Object:} \textit{the same driver}
%     \item \texttt{Preposition:} \textit{in the bus}
% \end{itemize} 
        
% \texttt{2. Conversion:} \textit{a person see another person in something} \\

% \texttt{3. Further Conversion:} \textit{a person see another person}

% ---------------------------------------------------------- \\

\texttt{Now we have a test instance. Please refer to the task instruction and the above examples to do the conversion.} \\
\texttt{The input sentence is: <Input event mention>}.\\
\texttt{Please convert the sentence into a more general expression following the above-mentioned three steps.}\\ 
\bottomrule

    \end{tabular}
    \caption{Prompt for the abstract event identification task.}
    \label{tab:prompt_abs}
\end{table*}
