\paragraph{Background.} The causal relations among $n$ variables $X = [X_i]^{n}_{i=1}$ is characterized via a \textbf{structural causal model (SCM)} \citep{pearl2009causality} over the tuple $\langle U, X, f \rangle$ that, in its general form, consists of a sets of assignments
\begin{align*}
    X_i := f_i \big(\Pa{X_i}, U_i \big), \quad i = 1, \cdots, n,
\end{align*}
where $U_i$ is an exogenous variable assumed to be mutually independent with variables $\{U_1, \cdots, U_n\} \backslash U_i$. The functions $f = \left[f_1, \cdots, f_n \right]$ define a joint distribution $P(X)$ over the endogenous variables $X$, given a joint distribution over exogenous variables $P(U_1, \cdots, U_n)$. Each SCM induces a causal graph $\rmG$, which is often assumed to be a DAG. A directed graph $\rmG = (\rmV, \rmE)$ consists of a set of nodes $\rmV$ and an edge set $\ermE \subseteq \rmV^2$  of ordered pairs of
nodes with $(v, v) \notin \rmE$ for any $v \in \rmV$ (one without self-loops). 

For a pair of nodes $i,j$ with $(i,j) \in \ermE$, there is an arrow pointing from $i$ to $j$ and we write $i \rightarrow j$. Two nodes $i$ and $j$ are adjacent if either $(i,j) \in \rmE$ or $(j,i) \in \rmE$. If there is an arrow from $i$ to $j$ then $i$ is a parent of $j$ and $j$ is a child of $i$. Let $\Pa{X_i}$ denote the set of variables associated with parents of node $i$ in $\rmG$. The graph $\rmG$ of an SCM is obtained by creating one vertex for each $X_i$ and drawing directed edges from each parent $X_j \in \Pa{X_i}$ to $X_i$. We sometimes call the elements of $\Pa{X_i}$ the \textbf{direct causes} of $X_i$, and we call $X_i$ a \textbf{direct effect} of each of its direct causes. Importantly, these functions are to be read as assignments rather than as mathematical equations, and they should be viewed as modelling physical mechanisms inducing or generating every $X_i$ from variables $\Pa{X_i}$. 

% Given three random variables $X, Y, Z$, there exist $3$ causal structures of interest in causal inference analysis: \textit{confounder}, \textit{collider} and \textit{mediator} 

% \begin{itemize}
%     \item $Z$ is a called confounder if it causes both $X$ and $Y$: $X \leftarrow Z \rightarrow Y$.
%     \item $Z$ is a collider when $Z$ is a common child of $X$ and $Y$ but $X$ and $Y$ themselves are not related to each other: $X \rightarrow Z \leftarrow Y$.
%     \item $Z$ is a mediator if there is a chain $X$ causes $Z$ and $Z$ causes $Y$:  $X \rightarrow Z \rightarrow Y$.
% \end{itemize}


 