\section{Statistical Causal Discovery}\label{sup:causal_discovery}

\section{Background}\label{sec:backgrnd}

\subsection{Cold Start Latency and Mitigation Techniques}

Traditional FaaS platforms mitigate cold starts through snapshotting, lightweight virtualization, and warm-state management. Snapshot-based methods like \textbf{REAP} and \textbf{Catalyzer} reduce initialization time by preloading or restoring container states but require significant memory and I/O resources, limiting scalability~\cite{dong_catalyzer_2020, ustiugov_benchmarking_2021}. Lightweight virtualization solutions, such as \textbf{Firecracker} microVMs, achieve fast startup times with strong isolation but depend on robust infrastructure, making them less adaptable to fluctuating workloads~\cite{agache_firecracker_2020}. Warm-state management techniques like \textbf{Faa\$T}~\cite{romero_faa_2021} and \textbf{Kraken}~\cite{vivek_kraken_2021} keep frequently invoked containers ready, balancing readiness and cost efficiency under predictable workloads but incurring overhead when demand is erratic~\cite{romero_faa_2021, vivek_kraken_2021}. While these methods perform well in resource-rich cloud environments, their resource intensity challenges applicability in edge settings.

\subsubsection{Edge FaaS Perspective}

In edge environments, cold start mitigation emphasizes lightweight designs, resource sharing, and hybrid task distribution. Lightweight execution environments like unikernels~\cite{edward_sock_2018} and \textbf{Firecracker}~\cite{agache_firecracker_2020}, as used by \textbf{TinyFaaS}~\cite{pfandzelter_tinyfaas_2020}, minimize resource usage and initialization delays but require careful orchestration to avoid resource contention. Function co-location, demonstrated by \textbf{Photons}~\cite{v_dukic_photons_2020}, reduces redundant initializations by sharing runtime resources among related functions, though this complicates isolation in multi-tenant setups~\cite{v_dukic_photons_2020}. Hybrid offloading frameworks like \textbf{GeoFaaS}~\cite{malekabbasi_geofaas_2024} balance edge-cloud workloads by offloading latency-tolerant tasks to the cloud and reserving edge resources for real-time operations, requiring reliable connectivity and efficient task management. These edge-specific strategies address cold starts effectively but introduce challenges in scalability and orchestration.

\subsection{Predictive Scaling and Caching Techniques}

Efficient resource allocation is vital for maintaining low latency and high availability in serverless platforms. Predictive scaling and caching techniques dynamically provision resources and reduce cold start latency by leveraging workload prediction and state retention.
Traditional FaaS platforms use predictive scaling and caching to optimize resources, employing techniques (OFC, FaasCache) to reduce cold starts. However, these methods rely on centralized orchestration and workload predictability, limiting their effectiveness in dynamic, resource-constrained edge environments.



\subsubsection{Edge FaaS Perspective}

Edge FaaS platforms adapt predictive scaling and caching techniques to constrain resources and heterogeneous environments. \textbf{EDGE-Cache}~\cite{kim_delay-aware_2022} uses traffic profiling to selectively retain high-priority functions, reducing memory overhead while maintaining readiness for frequent requests. Hybrid frameworks like \textbf{GeoFaaS}~\cite{malekabbasi_geofaas_2024} implement distributed caching to balance resources between edge and cloud nodes, enabling low-latency processing for critical tasks while offloading less critical workloads. Machine learning methods, such as clustering-based workload predictors~\cite{gao_machine_2020} and GRU-based models~\cite{guo_applying_2018}, enhance resource provisioning in edge systems by efficiently forecasting workload spikes. These innovations effectively address cold start challenges in edge environments, though their dependency on accurate predictions and robust orchestration poses scalability challenges.

\subsection{Decentralized Orchestration, Function Placement, and Scheduling}

Efficient orchestration in serverless platforms involves workload distribution, resource optimization, and performance assurance. While traditional FaaS platforms rely on centralized control, edge environments require decentralized and adaptive strategies to address unique challenges such as resource constraints and heterogeneous hardware.



\subsubsection{Edge FaaS Perspective}

Edge FaaS platforms adopt decentralized and adaptive orchestration frameworks to meet the demands of resource-constrained environments. Systems like \textbf{Wukong} distribute scheduling across edge nodes, enhancing data locality and scalability while reducing network latency. Lightweight frameworks such as \textbf{OpenWhisk Lite}~\cite{kravchenko_kpavelopenwhisk-light_2024} optimize resource allocation by decentralizing scheduling policies, minimizing cold starts and latency in edge setups~\cite{benjamin_wukong_2020}. Hybrid solutions like \textbf{OpenFaaS}~\cite{noauthor_openfaasfaas_2024} and \textbf{EdgeMatrix}~\cite{shen_edgematrix_2023} combine edge-cloud orchestration to balance resource utilization, retaining latency-sensitive functions at the edge while offloading non-critical workloads to the cloud. While these approaches improve flexibility, they face challenges in maintaining coordination and ensuring consistent performance across distributed nodes.



\begin{figure*}[hbt!]
    \centering
    \includegraphics[width=\linewidth]{figures/CD.pdf}
    \caption{SHD \textbf{(left)} and F1 score \textbf{(right)} of estimated DAGs from statistical structure learning methods. \underline{Lower} SHD is better. \underline{Higher} F1 is better.}
    \label{fig:ssl}
\end{figure*}


\paragraph{Experiments.} We here discuss how \texttt{ACCESS} is used to assess to what extent the statistical structure learning methods is applicable to recover causal relations among event abstractions. As illustrated in Figure \ref{fig:main}, after extracting abstractions, one can build representations for abstract events in the original corpus and apply structure learning on top of such data for full graph discovery. A simple representation is the co-occurrence matrix size $(\# stories \times \# abstractions)$ where each entry takes a binary value indicating whether an abstraction has any of its mentions appearing in a story. This means each abstraction is now considered as a Bernoulli random variable and the task of causal discovery is to recover the underlying SCM where the structural functions are commonly non-convex. 

Due to the limited scalability of existing statistical algorithms, we resort to learning sub-graphs by setting thresholds to select nodes that appear frequently while ensuring that the true graph is acyclic. Specifically, our selected sub-graphs are composed of edges where both nodes are adjacent to at least one other node, and each node corresponds to an abstraction whose occurrences exceed a certain frequency threshold. In our experiment, we set thresholds for document frequency within  $\{25, 30, 35, 40, 45\}$, resulting in sub-graphs with $5, 7, 16, 19, 45$ nodes. The experiments are run on $5$ CPU cores.    


We experiment with popular constraint-based and score-based algorithms. We select those that are scalable and capable of capturing non-linear causal relationships without relying on specific model forms such as additive noise. In this paper, we report the results for the following algorithms: 


\begin{itemize}
    \item \texttt{PC} algorithm \citep{spirtes1991algorithm}: A classic approach based on conditional independence tests, for which we run two kinds of tests: Chi-squared and G-squared. 
    \item \texttt{DAG-GNN} \citep{yu2019dag}: DAG structure learning with graph neural networks.
    \item \texttt{GAE} \citep{ng2019graph}: This method utilizes gradient descent and graph auto-encoders to model non-linear causal relationships.
    \item \texttt{CORL} \citep{wang2021ordering}: A reinforcement learning-based algorithm with flexible score functions with enhanced efficiency.
\end{itemize}

Besides the above methods, we have also tested \texttt{NOTEARS} \citep{zheng2020learning}, a popular score-based algorithm and its more efficient variant \texttt{GOLEM} \citep{ng2020role}. However, they both unfortunately fail to recover any edges across all settings. To ensure consistency in implementation and evaluation, we utilize the standardized framework provided by \href{https://github.com/huawei-noah/trustworthyAI/tree/master/gcastle}{\texttt{gCastle}} \citep{zhang2021gcastle}. 
As for evaluation metrics, we report the structured Hamming distance (SHD), which quantifies the smallest number of edge additions, deletions, and reversals required to transform the recovered DAG into the true one. Additionally, we assess classification accuracy using the F1 score. Ideally, we aim for a lower normalized Hamming distance and a higher F1 score. Figure \ref{fig:ssl} reports the SHD and F1 score of the estimated DAGs from these methods. 

It is seen the methods achieve relatively low accuracy on our benchmark causal graphs, which are sparse. As the SHD scores are much higher than the graph size, these model tend to predict plenty of edges, most of which are incorrect due to the low F1 scores. Scalability remains a serious challenge to statistical structure learning. As the graph scales up to $45$ nodes, their performance further deteriorates significantly, where most of them of them even fail to recover any edges. It is noteworthy that the representation power of the input data also affects the causal discovery performance. It is very likely that the co-occurrence matrix is not sufficiently expressive to capture the causal knowledge. This motivates a dedicated line of research into abstract causal representation learning. 


\section{GLUCOSE-QA Reasoning}  \label{sup:reasoning}
We here provide the prompts for LLMs in Tables \ref{tab:prompt_causal_discovery}-\ref{tab:prompt_abs}. Tables \ref{tab:examples_specific_qa}-\ref{tab:examples_cot_step} present illustrative examples of the responses from LLMs across our QA tasks. 





\definecolor{titlecolor}{rgb}{0.9, 0.5, 0.1}
\definecolor{anscolor}{rgb}{0.2, 0.5, 0.8}
\definecolor{labelcolor}{HTML}{48a07e}
\begin{table*}[h]
	\centering
	
 % \vspace{-0.2cm}
	
	\begin{center}
		\begin{tikzpicture}[
				chatbox_inner/.style={rectangle, rounded corners, opacity=0, text opacity=1, font=\sffamily\scriptsize, text width=5in, text height=9pt, inner xsep=6pt, inner ysep=6pt},
				chatbox_prompt_inner/.style={chatbox_inner, align=flush left, xshift=0pt, text height=11pt},
				chatbox_user_inner/.style={chatbox_inner, align=flush left, xshift=0pt},
				chatbox_gpt_inner/.style={chatbox_inner, align=flush left, xshift=0pt},
				chatbox/.style={chatbox_inner, draw=black!25, fill=gray!7, opacity=1, text opacity=0},
				chatbox_prompt/.style={chatbox, align=flush left, fill=gray!1.5, draw=black!30, text height=10pt},
				chatbox_user/.style={chatbox, align=flush left},
				chatbox_gpt/.style={chatbox, align=flush left},
				chatbox2/.style={chatbox_gpt, fill=green!25},
				chatbox3/.style={chatbox_gpt, fill=red!20, draw=black!20},
				chatbox4/.style={chatbox_gpt, fill=yellow!30},
				labelbox/.style={rectangle, rounded corners, draw=black!50, font=\sffamily\scriptsize\bfseries, fill=gray!5, inner sep=3pt},
			]
											
			\node[chatbox_user] (q1) {
				\textbf{System prompt}
				\newline
				\newline
				You are a helpful and precise assistant for segmenting and labeling sentences. We would like to request your help on curating a dataset for entity-level hallucination detection.
				\newline \newline
                We will give you a machine generated biography and a list of checked facts about the biography. Each fact consists of a sentence and a label (True/False). Please do the following process. First, breaking down the biography into words. Second, by referring to the provided list of facts, merging some broken down words in the previous step to form meaningful entities. For example, ``strategic thinking'' should be one entity instead of two. Third, according to the labels in the list of facts, labeling each entity as True or False. Specifically, for facts that share a similar sentence structure (\eg, \textit{``He was born on Mach 9, 1941.''} (\texttt{True}) and \textit{``He was born in Ramos Mejia.''} (\texttt{False})), please first assign labels to entities that differ across atomic facts. For example, first labeling ``Mach 9, 1941'' (\texttt{True}) and ``Ramos Mejia'' (\texttt{False}) in the above case. For those entities that are the same across atomic facts (\eg, ``was born'') or are neutral (\eg, ``he,'' ``in,'' and ``on''), please label them as \texttt{True}. For the cases that there is no atomic fact that shares the same sentence structure, please identify the most informative entities in the sentence and label them with the same label as the atomic fact while treating the rest of the entities as \texttt{True}. In the end, output the entities and labels in the following format:
                \begin{itemize}[nosep]
                    \item Entity 1 (Label 1)
                    \item Entity 2 (Label 2)
                    \item ...
                    \item Entity N (Label N)
                \end{itemize}
                % \newline \newline
                Here are two examples:
                \newline\newline
                \textbf{[Example 1]}
                \newline
                [The start of the biography]
                \newline
                \textcolor{titlecolor}{Marianne McAndrew is an American actress and singer, born on November 21, 1942, in Cleveland, Ohio. She began her acting career in the late 1960s, appearing in various television shows and films.}
                \newline
                [The end of the biography]
                \newline \newline
                [The start of the list of checked facts]
                \newline
                \textcolor{anscolor}{[Marianne McAndrew is an American. (False); Marianne McAndrew is an actress. (True); Marianne McAndrew is a singer. (False); Marianne McAndrew was born on November 21, 1942. (False); Marianne McAndrew was born in Cleveland, Ohio. (False); She began her acting career in the late 1960s. (True); She has appeared in various television shows. (True); She has appeared in various films. (True)]}
                \newline
                [The end of the list of checked facts]
                \newline \newline
                [The start of the ideal output]
                \newline
                \textcolor{labelcolor}{[Marianne McAndrew (True); is (True); an (True); American (False); actress (True); and (True); singer (False); , (True); born (True); on (True); November 21, 1942 (False); , (True); in (True); Cleveland, Ohio (False); . (True); She (True); began (True); her (True); acting career (True); in (True); the late 1960s (True); , (True); appearing (True); in (True); various (True); television shows (True); and (True); films (True); . (True)]}
                \newline
                [The end of the ideal output]
				\newline \newline
                \textbf{[Example 2]}
                \newline
                [The start of the biography]
                \newline
                \textcolor{titlecolor}{Doug Sheehan is an American actor who was born on April 27, 1949, in Santa Monica, California. He is best known for his roles in soap operas, including his portrayal of Joe Kelly on ``General Hospital'' and Ben Gibson on ``Knots Landing.''}
                \newline
                [The end of the biography]
                \newline \newline
                [The start of the list of checked facts]
                \newline
                \textcolor{anscolor}{[Doug Sheehan is an American. (True); Doug Sheehan is an actor. (True); Doug Sheehan was born on April 27, 1949. (True); Doug Sheehan was born in Santa Monica, California. (False); He is best known for his roles in soap operas. (True); He portrayed Joe Kelly. (True); Joe Kelly was in General Hospital. (True); General Hospital is a soap opera. (True); He portrayed Ben Gibson. (True); Ben Gibson was in Knots Landing. (True); Knots Landing is a soap opera. (True)]}
                \newline
                [The end of the list of checked facts]
                \newline \newline
                [The start of the ideal output]
                \newline
                \textcolor{labelcolor}{[Doug Sheehan (True); is (True); an (True); American (True); actor (True); who (True); was born (True); on (True); April 27, 1949 (True); in (True); Santa Monica, California (False); . (True); He (True); is (True); best known (True); for (True); his roles in soap operas (True); , (True); including (True); in (True); his portrayal (True); of (True); Joe Kelly (True); on (True); ``General Hospital'' (True); and (True); Ben Gibson (True); on (True); ``Knots Landing.'' (True)]}
                \newline
                [The end of the ideal output]
				\newline \newline
				\textbf{User prompt}
				\newline
				\newline
				[The start of the biography]
				\newline
				\textcolor{magenta}{\texttt{\{BIOGRAPHY\}}}
				\newline
				[The ebd of the biography]
				\newline \newline
				[The start of the list of checked facts]
				\newline
				\textcolor{magenta}{\texttt{\{LIST OF CHECKED FACTS\}}}
				\newline
				[The end of the list of checked facts]
			};
			\node[chatbox_user_inner] (q1_text) at (q1) {
				\textbf{System prompt}
				\newline
				\newline
				You are a helpful and precise assistant for segmenting and labeling sentences. We would like to request your help on curating a dataset for entity-level hallucination detection.
				\newline \newline
                We will give you a machine generated biography and a list of checked facts about the biography. Each fact consists of a sentence and a label (True/False). Please do the following process. First, breaking down the biography into words. Second, by referring to the provided list of facts, merging some broken down words in the previous step to form meaningful entities. For example, ``strategic thinking'' should be one entity instead of two. Third, according to the labels in the list of facts, labeling each entity as True or False. Specifically, for facts that share a similar sentence structure (\eg, \textit{``He was born on Mach 9, 1941.''} (\texttt{True}) and \textit{``He was born in Ramos Mejia.''} (\texttt{False})), please first assign labels to entities that differ across atomic facts. For example, first labeling ``Mach 9, 1941'' (\texttt{True}) and ``Ramos Mejia'' (\texttt{False}) in the above case. For those entities that are the same across atomic facts (\eg, ``was born'') or are neutral (\eg, ``he,'' ``in,'' and ``on''), please label them as \texttt{True}. For the cases that there is no atomic fact that shares the same sentence structure, please identify the most informative entities in the sentence and label them with the same label as the atomic fact while treating the rest of the entities as \texttt{True}. In the end, output the entities and labels in the following format:
                \begin{itemize}[nosep]
                    \item Entity 1 (Label 1)
                    \item Entity 2 (Label 2)
                    \item ...
                    \item Entity N (Label N)
                \end{itemize}
                % \newline \newline
                Here are two examples:
                \newline\newline
                \textbf{[Example 1]}
                \newline
                [The start of the biography]
                \newline
                \textcolor{titlecolor}{Marianne McAndrew is an American actress and singer, born on November 21, 1942, in Cleveland, Ohio. She began her acting career in the late 1960s, appearing in various television shows and films.}
                \newline
                [The end of the biography]
                \newline \newline
                [The start of the list of checked facts]
                \newline
                \textcolor{anscolor}{[Marianne McAndrew is an American. (False); Marianne McAndrew is an actress. (True); Marianne McAndrew is a singer. (False); Marianne McAndrew was born on November 21, 1942. (False); Marianne McAndrew was born in Cleveland, Ohio. (False); She began her acting career in the late 1960s. (True); She has appeared in various television shows. (True); She has appeared in various films. (True)]}
                \newline
                [The end of the list of checked facts]
                \newline \newline
                [The start of the ideal output]
                \newline
                \textcolor{labelcolor}{[Marianne McAndrew (True); is (True); an (True); American (False); actress (True); and (True); singer (False); , (True); born (True); on (True); November 21, 1942 (False); , (True); in (True); Cleveland, Ohio (False); . (True); She (True); began (True); her (True); acting career (True); in (True); the late 1960s (True); , (True); appearing (True); in (True); various (True); television shows (True); and (True); films (True); . (True)]}
                \newline
                [The end of the ideal output]
				\newline \newline
                \textbf{[Example 2]}
                \newline
                [The start of the biography]
                \newline
                \textcolor{titlecolor}{Doug Sheehan is an American actor who was born on April 27, 1949, in Santa Monica, California. He is best known for his roles in soap operas, including his portrayal of Joe Kelly on ``General Hospital'' and Ben Gibson on ``Knots Landing.''}
                \newline
                [The end of the biography]
                \newline \newline
                [The start of the list of checked facts]
                \newline
                \textcolor{anscolor}{[Doug Sheehan is an American. (True); Doug Sheehan is an actor. (True); Doug Sheehan was born on April 27, 1949. (True); Doug Sheehan was born in Santa Monica, California. (False); He is best known for his roles in soap operas. (True); He portrayed Joe Kelly. (True); Joe Kelly was in General Hospital. (True); General Hospital is a soap opera. (True); He portrayed Ben Gibson. (True); Ben Gibson was in Knots Landing. (True); Knots Landing is a soap opera. (True)]}
                \newline
                [The end of the list of checked facts]
                \newline \newline
                [The start of the ideal output]
                \newline
                \textcolor{labelcolor}{[Doug Sheehan (True); is (True); an (True); American (True); actor (True); who (True); was born (True); on (True); April 27, 1949 (True); in (True); Santa Monica, California (False); . (True); He (True); is (True); best known (True); for (True); his roles in soap operas (True); , (True); including (True); in (True); his portrayal (True); of (True); Joe Kelly (True); on (True); ``General Hospital'' (True); and (True); Ben Gibson (True); on (True); ``Knots Landing.'' (True)]}
                \newline
                [The end of the ideal output]
				\newline \newline
				\textbf{User prompt}
				\newline
				\newline
				[The start of the biography]
				\newline
				\textcolor{magenta}{\texttt{\{BIOGRAPHY\}}}
				\newline
				[The ebd of the biography]
				\newline \newline
				[The start of the list of checked facts]
				\newline
				\textcolor{magenta}{\texttt{\{LIST OF CHECKED FACTS\}}}
				\newline
				[The end of the list of checked facts]
			};
		\end{tikzpicture}
        \caption{GPT-4o prompt for labeling hallucinated entities.}\label{tb:gpt-4-prompt}
	\end{center}
\vspace{-0cm}
\end{table*}

\begin{table*}[hbt!]
\centering
\resizebox{\linewidth}{!}{%
\begin{tabular}{l|p{12cm}}
\toprule
Story & In a store, two women were arguing, and Howard wanted to intervene. He attempted to get them to stop talking, but it didn't work. So, he stepped in between them, which caused them to cease their fighting. \\
\midrule
Specific Question & What could be the cause of the event \textit{`howard wants to help the women'}? \\
\midrule
Abstract Question & The question describes an event where \textit{`a person hears something in a place'}. What could be the effect of the event?\\
\midrule
Choices & \begin{tabular}[c]{@{}l@{}}0: "Two women fights each other.",\\ 1: "He went in between them.",\\ 2: "Two women were fighting in a store.",\\ 3: "They stopped.",\\ 4: "Howard wanted to help."\\ 5: "He tried telling them to stop but it did not work."\end{tabular} \\ \midrule
Causal Graph (CG) & \textit{a person have a fight with another person} $\rightarrow$ \textit{a person want to stop another person} \\ \midrule
Correct Answers & 0, 2 \\ \midrule
\texttt{GPT-4o-mini} Answers & 2, 4 \\ \midrule
\texttt{GPT-4o-mini} Answers w/ CG & 0, 2 \\ \midrule
\texttt{Llama3.1-8B} Answers & 0, 1 \\ \midrule
\texttt{Llama3.1-8B} Answers w/ CG & 0, 2, 4 \\ 
\bottomrule
\end{tabular}%
}
\caption{Examples of multi-choice Specific-QA reasoning in \texttt{GPT-4o-mini} and \texttt{Llama3.1-8B}.}
\label{tab:examples_specific_qa}
\end{table*}


\begin{table*}[hbt!]
\centering
\resizebox{\linewidth}{!}{%
\begin{tabular}{l|p{12cm}}
\toprule
Story & His cousins were scheduled to visit later that day, so his mom had him clean in the morning, shop for groceries in the afternoon, and get ready in the evening. Eventually, his cousins arrived at his house. \\
\midrule
Abstract Question & The question describes an event where \textit{`a person are coming to a place (that is another person house)'}. What could be the effect of the event?\\
\midrule
Choices & \begin{tabular}[c]{@{}l@{}}0: "His cousins were coming later too his house.",\\ 1: "He get groceries in the afternoon.",\\ 2: "His mom made him clean all morning.",\\ 3: "His cousins came to his house.",\\ 4: "He get ready in the evening."\end{tabular} \\ \midrule
Causal Graph (CG) & \textit{a person come to another person 's place} $\rightarrow$ \textit{a person clean something} \\ \midrule
Correct Answers & 1, 2, 4 \\ \midrule
\texttt{GPT-4o-mini} Answers & 0, 3 \\ \midrule
\texttt{GPT-4o-mini} Answers w/ CG & 0, 2 \\ \midrule
\texttt{Llama3.2-3B} Answers & 0, 3 \\ \midrule
\texttt{Llama3.2-3B} Answers w/ CG & 1, 3 \\ 
\bottomrule
\end{tabular}%
}
\caption{Examples of multi-choice Abstract-QA reasoning in \texttt{GPT-4o-mini} and \texttt{Llama3.2-3B}.}
\label{tab:examples_abstract_qa_1}
\end{table*}

\begin{table*}[hbt!]
\centering
\resizebox{\linewidth}{!}{%
\begin{tabular}{l|p{12cm}}
\toprule
Story & Felix wanted to visit Disney World. One day, he won two tickets and invited his friend Alissa. However, Alissa disliked Disney, so Felix ended up going by himself. \\
\midrule
Abstract Question & The question describes an event where \textit{`a person invited another person'}. What could be the cause of the event?\\
\midrule
Choices & \begin{tabular}[c]{@{}l@{}}0: "Alissa hated disney.",\\ 1: "Felix wanted to go to disney world.",\\ 2: "One day he won two tickets for entry.",\\ 3: "He invited his friend Alissa.",\\ 4: "He ended up going alone."\end{tabular} \\ \midrule
Causal Graph (CG) & \textit{a person want to go to a place} $\rightarrow$ \textit{a person give another person an invitation to a place} \\ \midrule
Correct Answers & 1, 2 \\ \midrule
\texttt{GPT-4o-mini} Answers & 0, 1, 3 \\ \midrule
\texttt{GPT-4o-mini} Answers w/ CG & 1, 2 \\ \midrule
\texttt{Llama2-7B} Answers & 1, 2 \\ \midrule
\texttt{Llama2-7B} Answers w/ CG & 1, 2 \\ 
\bottomrule
\end{tabular}%
}
\caption{Examples of multi-choice Abstract-QA reasoning in \texttt{GPT-4o-mini} and \texttt{Llama2-7B}.}
\label{tab:examples_abstract_qa_2}
\end{table*}
\begin{table*}[hbt!]
\centering
\resizebox{\linewidth}{!}{%
\begin{tabular}{l|p{12cm}}
\toprule
Story & He wanted toast, so he got some bread and put it in the toaster. When it popped out and landed on the floor, he ate it anyway. \\
\midrule
Abstract Question & The question describes an event where \textit{`a person got another thing (that is an ingredient in another thing'}. What could be the cause of the event?\\
\midrule
Choices & \begin{tabular}[c]{@{}l@{}}0: "He ate it anyway.",\\ 1: "He put it in the toaster.",\\ 2: "He got some bread.",\\ 3: "It shot out of the toaster and onto the floor.",\\ 4: "He was making toast."\end{tabular} \\ \midrule
% Causal Graph (CG) & \textit{a person make food} $\rightarrow$ \textit{a person get ingredient} \\ \midrule
Correct Answers & 4 \\ 
\midrule
\texttt{Llama3.2-3B} Answers (zero-shot) & 1, 2 \\ 
\texttt{Llama3.2-3B} Answers & 1, 4 \\ 
\texttt{Llama3.2-3B} Answers + CG & 1, 4 \\ 
\midrule
\texttt{Llama3.1-8B} Answers(zero-shot) & 1, 3 \\ 
\texttt{Llama3.1-8B} Answers  & 2 \\ 
\texttt{Llama3.1-8B} Answers + CG & 4 \\ 
\midrule
\texttt{Llama2-7B} Answers (zero-shot) & 1, 2 \\ 
\texttt{Llama2-7B} Answers & 4 \\ 
\texttt{Llama2-7B} Answers + CG & 1, 4 \\ 
\bottomrule
\end{tabular}%
}
\caption{Examples of multi-choice Abstract-QA reasoning with bi-level COT.}
\label{tab:examples_cot}
\end{table*}

\begin{table*}[hbt!]
\centering
\resizebox{\linewidth}{!}{%
\begin{tabular}{l|p{12cm}}
\toprule
Story & He wanted toast, so he got some bread and put it in the toaster. When it popped out and landed on the floor, he ate it anyway. \\
\midrule
Abstract Question & The question describes an event where \textit{`a person got another thing (that is an ingredient in another thing'}. What could be the cause of the event?\\
\midrule
Choices & \begin{tabular}[c]{@{}l@{}}0: "He ate it anyway.",\\ 1: "He put it in the toaster.",\\ 2: "He got some bread.",\\ 3: "It shot out of the toaster and onto the floor.",\\ 4: "He was making toast."\end{tabular} \\ \midrule
Prompt & The event \textit{`a person got another thing (that is an ingredient in another thing'} is described by one of the sentences in the story context. First identify that part
of the story. Then retrieve the event mentioned in the story that is a
corresponding cause/effect.\\ 
\midrule
% Causal Graph (CG) & \textit{a person make food} $\rightarrow$ \textit{a person get ingredient} \\ \midrule
Correct Answers & 4 \\ 
\midrule
Rationale & To solve this question, let's break down the steps: \\
& 1. Identify the sentence describing `another thing which is an ingredient in another thing': - The relevant sentence from the given text is not explicitly stated but rather implied through analysis. In this case, what seems like `he made breakfast food'. In other words - Toaster = Breakfast Food; Bread + Toast = Ingredients for the meal at hand.  Therefore the first item on our consideration has been identified as number four because if you are going into kitchen then most likely, your objective would be about cooking something.\\ 
& 2. Now find any events occurring after items were used during said activity using index numbers provided against them starting their respective sequence no. This means looking back towards end, start at \#5 because there will definitely exist only those indexed things found before each index respectively being associated w/ this particular action taking place afterward within same timeframe i.e., related actions happening after certain things are prepared here. Considering these points: We look forward now...4 provides our initial object `the ingredients'. This makes us say - That must come after getting things ready prior than eating. Therefore we also know - getting stuff does indeed precede putting things inside the device whose result gets eaten later.\\ 
& So while option zero fits too since all meals need ending act (eating) coming second to doing preparatory tasks involving having desired objects (`bread').. But finally- lets verify... As shown below, the correct answer(s): 4, 1\\ 
\bottomrule
\end{tabular}%
}
\caption{Example of multi-choice Abstract-QA reasoning with bi-level COT in \texttt{Llama3.2-3B}.}
\label{tab:examples_cot_step}
\end{table*}



