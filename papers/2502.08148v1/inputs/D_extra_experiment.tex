\section{Statistical Causal Discovery}\label{sup:causal_discovery}

\section{Basic Background: Supervised Learning and the PAC Model}
\label{sec:background}

At this point almost everyone has heard of machine learning (ML). Anyone likely to stumble upon this article will have also heard of its most influential special case, supervised learning, and those theoretically inclined will also be familiar with the PAC model. Nonetheless, I will set the stage by  recapping the basics.

\subsection{Basics of Supervised Learning}%Let's set the stage in any case

\emph{Supervised Learning} is the task of ``coming up'' with a function $f: \X \to \Y$ to ``explain'' or ``fit'' a sequence of input/output examples   $(x_1,y_1), \ldots, (x_n,y_n)$, with $x_i \in \X$ and $y_i \in \Y$.  Here $\X$ is a \emph{data domain} consisting of \emph{datapoints} $x \in \X$, $\Y$ is a \emph{label set} consisting of \emph{labels} $y \in \Y$, and the sequence $(x_1,y_1),\ldots,(x_n,y_n)$ is the \emph{training data} consisting of \emph{labeled examples (a.k.a. samples)}~$(x_i,y_i)$.  I~will refer to the chosen function $f$ as a \emph{predictor}, and to $n$ as the \emph{sample size}. A \emph{learning algorithm} takes as input training data, and outputs (some representation of) a predictor $f \in \Y^\X$.\footnote{Note that this describes the usual \emph{batch}, a.k.a.~\emph{offline}, setting of supervised learning. I do not discuss other paradigms such as online or active learning in this article.} 



Success in supervised learning is defined as \emph{generalization} to  future examples: For a typical \emph{test example}  $(x_{\tst},y_{\tst})$, the predicted label $y'_{\tst}=f(x_{\tst})$ should ``equal'' $y_{\tst}$, perhaps approximately. We usually assume the test example is drawn from the same  ``source'' as the training data  --- commonly, i.i.d.~from the same distribution. The quality of the prediction is quantified by $\ell(y'_{\tst},y_{\tst})$, where $\ell:~\Y~\times~\Y \to \RR_{\geq 0}$ is a \emph{loss function} chosen as part of the problem definition. Common loss functions include the 0-1 loss $\ell_{0-1}(y',y) = [y' \neq y]$ for \emph{classification} problems,\footnote{The notation $[P]$ denotes $1$ when predicate $P$ is true, and denotes $0$ when $P$ is false.} as well as the absolute loss $|y'-y|$ or squared loss $(y'-y)^2$ for \emph{regression problems} featuring $\Y  \sse \RR$.

Nontrivial generalization properties are typically only possible if one assumes something about the data.\footnote{The need for such an assumption is formalized by the  \emph{no free lunch theorems} of supervised learning \cite{wolpert_connection_1992,wolpert_lack_1996,schaffer_conservation_1994}.} The Bayesian approach to  machine learning, common in many applications, assumes some parametric form for the distribution generating the data, and postulates a prior on the parameters. This is not the approach I will take in this article. Instead, I will focus on the frequentist --- and some would say ``worst-case'' or ``adversarial'' ---  approach that is common in the computational learning theory community, embodied by the PAC model. Here we assume that the (training and test) data can be explained, perhaps approximately, by a function in some ``simple enough to learn'' class of functions $\H \sse \Y^\X$, often called the \emph{hypotheses}. Equivalently, we  seek a predictor which explains the unseen data roughly  as well as the best hypothesis $h^* \in \H$, whether or not we assume that $h^*$ itself provides a perfect explanation.



 \paragraph{Common Algorithmic Templates.} Perhaps the best known general-purpose supervised learning algorithm is \emph{empirical risk minimization (ERM)}, which chooses as its predictor a hypothesis $f \in \H$ minimizing $\frac{1}{n} \sum_{i=1}^n \ell(f(x_i),y_i)$ --- a quantity called the \emph{training error}, \emph{empirical error}, or \emph{empirical risk} of $f$. %\footnote{When multiple hypotheses minimize the empirical risk, we assume ERM breaks ties arbitrarily.}
A common template for generalizing ERM involves adding a \emph{regularization term} $\psi(f)$ to the  objective function, typically chosen to measure some notion of ``hypothesis complexity.'' An algorithm instantiating this template is known as a \emph{structural risk minimizer (SRM)}, and chooses as its predictor the hypothesis $f \in \H$ minimizing the \emph{structural risk} $\frac{1}{n} \sum_{i=1}^n \ell(f(x_i),y_i) + \psi(f)$. Other well-known algorithms, such as gradient descent and its variations,  can frequently be interpreted as approximate implementations of ERM or SRM.


\paragraph{Proper vs Improper Learning.} A learning algorithm is said to be \emph{proper} if its predictor $f$ is always chosen from the hypothesis class, i.e., $f \in \H$, otherwise it is said to be \emph{improper}. ERM  is an example of a proper learning algorithm, as are SRM algorithms of the form described above.  In the \emph{proper regime} of learning, algorithms are required to be proper. This article will be concerned with the more flexible \emph{improper regime} (a.k.a \emph{representation-independent learning}), where no such constraint is placed on the learner. In other words, all we care about is predictive power at test time, rather than any insights derived from the functional form or representation of the predictor~itself.


\subsection{The PAC Model}
A standard mathematical setup for evaluation of supervised learning algorithms, at least in the theoretical computer science community, is Valiant's \emph{Probably Approximately Correct (PAC) model} of learning (see e.g.~\cite{kearns_introduction_1994,mohri_foundations_2018}). Here, we assume there is an unknown distribution $\D$ on $\X \times \Y$ from which training and test data are  drawn.  Specifically, the labeled datapoints of the training set  $(x_1,y_1), \ldots, (x_n,y_n)$, as well as the test data  $(x_\tst,y_\tst)$, are i.i.d.~from $\D$. Often it is assumed that $\D$ lies in some class of distributions of interest. The \emph{true expected loss}, or simply \emph{loss}, of a predictor $f: \X \to \Y$ is the expected loss it incurs on draws from $\D$, written $L_\D(f) = \Ex_{(x,y) \sim \D} \ell(f(x),y)$.


There are two main ``settings'' in PAC learning. The  \emph{realizable setting} only requires that the data be perfectly explained by some hypothesis in $\H$. More generally, the \emph{agnostic setting} makes no assumption relating the data to the hypotheses, but shifts the goalposts as necessary to allow nontrivial guarantees: the expected loss at test time is evaluated only ``relative'' to that of the best hypothesis $h^* \in \H$. There are other settings which make more nuanced assumptions, such as $\D$ being of a particular parametric form or its support living in some (unknown) lower-dimensional space, etc. I will mostly discuss the realizable and agnostic settings in this article, those being the simplest and most studied from a theoretical perspective. %TODO:We will briefly discuss other settings in Section ??

The PAC model demands high probability guarantees of learners, in the worst case over distributions of interest. Consider first the realizable setting, where $\D$ is such that $\min_{h \in \H} L_{\D}(h) = 0$. A PAC learner has \emph{error} $\epsilon=\epsilon(n)$ and \emph{confidence} $\delta=\delta(n)$ if, when training data consists of $n$ i.i.d~samples from a realizable distribution $\D$, it produces a predictor $f$  satisfying $L_\D(f) \leq \epsilon$ with probability at least $1-\delta$. In the agnostic setting, where $\D$ can be arbitrary, we require $L_\D(f) - \min_{h \in \H} L_\D(h) \leq \epsilon$ with probability $1-\delta$.

In both the realizable and agnostic settings, we look for PAC learners with small $\epsilon$ and $\delta$ as a function of the sample size $n$. An equivalent perspective looks at the sample complexity $m(\epsilon,\delta)$, which is the minimum sample size which guarantees error  at most $\epsilon$ with probability at least $1-\delta$. We say a problem is \emph{PAC learnable} if its PAC sample complexity is finite whenever $\epsilon,\delta > 0$.

For most PAC learning problems, learnability and sample complexity are characterized in terms of a  ``dimension'' of the hypothesis class. Most prominently this is the \emph{VC dimension} for binary classification, the \emph{fat shattering dimension} for agnostic regression, and the \emph{DS dimension} for multiclass classification (see \cite{anthony_neural_1999,daniely_optimal_2014,brukhim_characterization_2022}). Treatment of these is beyond the scope of this article. The unfamiliar reader need not worry, however,  as dimensions will feature only tangentially in our~discussion.




%\paragraph{Learning settings: Realizable, Agnostic, etc.} In learning theory, evaluating a supervised learning algorithm requires specifying a data model and an objective. We will leave the details of the data model flexible for now, to allow for both the PAC model and the adversarial transductive model. Nonetheless we will describe two variations, which we call ``settings'', which cut across different models. The  \emph{realizable setting}  requires only that the data be perfectly explained by some hypothesis $h \in \H$ --- i.e., there exists a hypothesis which is guaranteed to suffer a loss of $0$ on training and test data. The performance of the learning algorithm is its expected loss at test time for some ``worst case'' realizable instance. More generally, the \emph{agnostic setting} makes no assumption relating the data to the hypotheses, but shifts the goalposts as necessary to allow nontrivial guarantees: the expected loss at test time is evaluated only ``relative'' to that of the best hypothesis $h^* \in \H$, again for some ``worst case'' instance. There are other settings which make more nuanced assumptions about the data, such as it is drawn from a distribution of a particular parametric form, or that it lives in some (unknown) lower-dimensional space, etc. We will mostly discuss the realizable and agnostic settings, those being the simplest and most studied from a theoretical perspective.




%%% Local Variables:
%%% mode: latex
%%% TeX-master: "learning_matching"
%%% End:


\begin{figure*}[hbt!]
    \centering
    \includegraphics[width=\linewidth]{figures/CD.pdf}
    \caption{SHD \textbf{(left)} and F1 score \textbf{(right)} of estimated DAGs from statistical structure learning methods. \underline{Lower} SHD is better. \underline{Higher} F1 is better.}
    \label{fig:ssl}
\end{figure*}


\paragraph{Experiments.} We here discuss how \texttt{ACCESS} is used to assess to what extent the statistical structure learning methods is applicable to recover causal relations among event abstractions. As illustrated in Figure \ref{fig:main}, after extracting abstractions, one can build representations for abstract events in the original corpus and apply structure learning on top of such data for full graph discovery. A simple representation is the co-occurrence matrix size $(\# stories \times \# abstractions)$ where each entry takes a binary value indicating whether an abstraction has any of its mentions appearing in a story. This means each abstraction is now considered as a Bernoulli random variable and the task of causal discovery is to recover the underlying SCM where the structural functions are commonly non-convex. 

Due to the limited scalability of existing statistical algorithms, we resort to learning sub-graphs by setting thresholds to select nodes that appear frequently while ensuring that the true graph is acyclic. Specifically, our selected sub-graphs are composed of edges where both nodes are adjacent to at least one other node, and each node corresponds to an abstraction whose occurrences exceed a certain frequency threshold. In our experiment, we set thresholds for document frequency within  $\{25, 30, 35, 40, 45\}$, resulting in sub-graphs with $5, 7, 16, 19, 45$ nodes. The experiments are run on $5$ CPU cores.    


We experiment with popular constraint-based and score-based algorithms. We select those that are scalable and capable of capturing non-linear causal relationships without relying on specific model forms such as additive noise. In this paper, we report the results for the following algorithms: 


\begin{itemize}
    \item \texttt{PC} algorithm \citep{spirtes1991algorithm}: A classic approach based on conditional independence tests, for which we run two kinds of tests: Chi-squared and G-squared. 
    \item \texttt{DAG-GNN} \citep{yu2019dag}: DAG structure learning with graph neural networks.
    \item \texttt{GAE} \citep{ng2019graph}: This method utilizes gradient descent and graph auto-encoders to model non-linear causal relationships.
    \item \texttt{CORL} \citep{wang2021ordering}: A reinforcement learning-based algorithm with flexible score functions with enhanced efficiency.
\end{itemize}

Besides the above methods, we have also tested \texttt{NOTEARS} \citep{zheng2020learning}, a popular score-based algorithm and its more efficient variant \texttt{GOLEM} \citep{ng2020role}. However, they both unfortunately fail to recover any edges across all settings. To ensure consistency in implementation and evaluation, we utilize the standardized framework provided by \href{https://github.com/huawei-noah/trustworthyAI/tree/master/gcastle}{\texttt{gCastle}} \citep{zhang2021gcastle}. 
As for evaluation metrics, we report the structured Hamming distance (SHD), which quantifies the smallest number of edge additions, deletions, and reversals required to transform the recovered DAG into the true one. Additionally, we assess classification accuracy using the F1 score. Ideally, we aim for a lower normalized Hamming distance and a higher F1 score. Figure \ref{fig:ssl} reports the SHD and F1 score of the estimated DAGs from these methods. 

It is seen the methods achieve relatively low accuracy on our benchmark causal graphs, which are sparse. As the SHD scores are much higher than the graph size, these model tend to predict plenty of edges, most of which are incorrect due to the low F1 scores. Scalability remains a serious challenge to statistical structure learning. As the graph scales up to $45$ nodes, their performance further deteriorates significantly, where most of them of them even fail to recover any edges. It is noteworthy that the representation power of the input data also affects the causal discovery performance. It is very likely that the co-occurrence matrix is not sufficiently expressive to capture the causal knowledge. This motivates a dedicated line of research into abstract causal representation learning. 


\section{GLUCOSE-QA Reasoning}  \label{sup:reasoning}
We here provide the prompts for LLMs in Tables \ref{tab:prompt_causal_discovery}-\ref{tab:prompt_abs}. Tables \ref{tab:examples_specific_qa}-\ref{tab:examples_cot_step} present illustrative examples of the responses from LLMs across our QA tasks. 




\begin{tcolorbox}[title={The Prompt used for Translation}]
You are a highly skilled translator tasked with translating various types of content from English into \{\{ language \}\}. Follow these instructions carefully to complete the translation task.

You will receive a user-bot conversation in XML format. Please follow a three-step translation process:

\begin{enumerate}
  \item \textbf{Initial Translation:} Translate the input content into \{\{ language \}\}, preserving the original intent and keeping the original paragraph and text format unchanged. Do not delete or omit any content, and ensure that all original Markdown elements (e.g., images, code blocks) are preserved.
  \item \textbf{Reflection and Feedback:} Carefully review both the source text and your translation. Provide constructive criticism and specific suggestions to improve the translation in terms of:
    \begin{enumerate}[label=(\roman*)]
      \item \textbf{Accuracy:} Correct errors of addition, mistranslation, omission, or untranslated text.
      \item \textbf{Fluency:} Apply \{\{ language \}\} grammar, spelling, and punctuation rules while avoiding unnecessary repetitions.
      \item \textbf{Style:} Ensure that the translation reflects the style of the source text and considers any relevant cultural context.
    \end{enumerate}
  \item \textbf{Refinement:} Based on your reflections, refine and polish your translation.
  \item \textbf{Fallback:} If you are not confident in translating the conversation, please return ``\texttt{<stop></stop>}''.
\end{enumerate}

\bigskip
\textbf{Output:}

For each step of the translation process, output your results within the appropriate XML tags as follows:
\begin{verbatim}
<step1_initial_translation>
[Insert your initial translation here]
</step1_initial_translation>

<step2_reflection>
[Insert your reflection on the translation, including a list 
of specific, helpful, and constructive suggestions for 
improvement. Each suggestion should address a specific 
part of the translation.]
</step2_reflection>

<step3_refined_translation>
[Insert your refined and polished translation here]
</step3_refined_translation>
\end{verbatim}

Ensure that your final translation in step 3 accurately reflects the original meaning while sounding natural in \{\{ language \}\}.

Here is the original conversation:
\label{box:trans_prompt}
\end{tcolorbox}


\begin{table*}[hbt!]
\centering
\resizebox{\linewidth}{!}{%
\begin{tabular}{l|p{12cm}}
\toprule
Story & In a store, two women were arguing, and Howard wanted to intervene. He attempted to get them to stop talking, but it didn't work. So, he stepped in between them, which caused them to cease their fighting. \\
\midrule
Specific Question & What could be the cause of the event \textit{`howard wants to help the women'}? \\
\midrule
Abstract Question & The question describes an event where \textit{`a person hears something in a place'}. What could be the effect of the event?\\
\midrule
Choices & \begin{tabular}[c]{@{}l@{}}0: "Two women fights each other.",\\ 1: "He went in between them.",\\ 2: "Two women were fighting in a store.",\\ 3: "They stopped.",\\ 4: "Howard wanted to help."\\ 5: "He tried telling them to stop but it did not work."\end{tabular} \\ \midrule
Causal Graph (CG) & \textit{a person have a fight with another person} $\rightarrow$ \textit{a person want to stop another person} \\ \midrule
Correct Answers & 0, 2 \\ \midrule
\texttt{GPT-4o-mini} Answers & 2, 4 \\ \midrule
\texttt{GPT-4o-mini} Answers w/ CG & 0, 2 \\ \midrule
\texttt{Llama3.1-8B} Answers & 0, 1 \\ \midrule
\texttt{Llama3.1-8B} Answers w/ CG & 0, 2, 4 \\ 
\bottomrule
\end{tabular}%
}
\caption{Examples of multi-choice Specific-QA reasoning in \texttt{GPT-4o-mini} and \texttt{Llama3.1-8B}.}
\label{tab:examples_specific_qa}
\end{table*}


\begin{table*}[hbt!]
\centering
\resizebox{\linewidth}{!}{%
\begin{tabular}{l|p{12cm}}
\toprule
Story & His cousins were scheduled to visit later that day, so his mom had him clean in the morning, shop for groceries in the afternoon, and get ready in the evening. Eventually, his cousins arrived at his house. \\
\midrule
Abstract Question & The question describes an event where \textit{`a person are coming to a place (that is another person house)'}. What could be the effect of the event?\\
\midrule
Choices & \begin{tabular}[c]{@{}l@{}}0: "His cousins were coming later too his house.",\\ 1: "He get groceries in the afternoon.",\\ 2: "His mom made him clean all morning.",\\ 3: "His cousins came to his house.",\\ 4: "He get ready in the evening."\end{tabular} \\ \midrule
Causal Graph (CG) & \textit{a person come to another person 's place} $\rightarrow$ \textit{a person clean something} \\ \midrule
Correct Answers & 1, 2, 4 \\ \midrule
\texttt{GPT-4o-mini} Answers & 0, 3 \\ \midrule
\texttt{GPT-4o-mini} Answers w/ CG & 0, 2 \\ \midrule
\texttt{Llama3.2-3B} Answers & 0, 3 \\ \midrule
\texttt{Llama3.2-3B} Answers w/ CG & 1, 3 \\ 
\bottomrule
\end{tabular}%
}
\caption{Examples of multi-choice Abstract-QA reasoning in \texttt{GPT-4o-mini} and \texttt{Llama3.2-3B}.}
\label{tab:examples_abstract_qa_1}
\end{table*}

\begin{table*}[hbt!]
\centering
\resizebox{\linewidth}{!}{%
\begin{tabular}{l|p{12cm}}
\toprule
Story & Felix wanted to visit Disney World. One day, he won two tickets and invited his friend Alissa. However, Alissa disliked Disney, so Felix ended up going by himself. \\
\midrule
Abstract Question & The question describes an event where \textit{`a person invited another person'}. What could be the cause of the event?\\
\midrule
Choices & \begin{tabular}[c]{@{}l@{}}0: "Alissa hated disney.",\\ 1: "Felix wanted to go to disney world.",\\ 2: "One day he won two tickets for entry.",\\ 3: "He invited his friend Alissa.",\\ 4: "He ended up going alone."\end{tabular} \\ \midrule
Causal Graph (CG) & \textit{a person want to go to a place} $\rightarrow$ \textit{a person give another person an invitation to a place} \\ \midrule
Correct Answers & 1, 2 \\ \midrule
\texttt{GPT-4o-mini} Answers & 0, 1, 3 \\ \midrule
\texttt{GPT-4o-mini} Answers w/ CG & 1, 2 \\ \midrule
\texttt{Llama2-7B} Answers & 1, 2 \\ \midrule
\texttt{Llama2-7B} Answers w/ CG & 1, 2 \\ 
\bottomrule
\end{tabular}%
}
\caption{Examples of multi-choice Abstract-QA reasoning in \texttt{GPT-4o-mini} and \texttt{Llama2-7B}.}
\label{tab:examples_abstract_qa_2}
\end{table*}
\begin{table*}[hbt!]
\centering
\resizebox{\linewidth}{!}{%
\begin{tabular}{l|p{12cm}}
\toprule
Story & He wanted toast, so he got some bread and put it in the toaster. When it popped out and landed on the floor, he ate it anyway. \\
\midrule
Abstract Question & The question describes an event where \textit{`a person got another thing (that is an ingredient in another thing'}. What could be the cause of the event?\\
\midrule
Choices & \begin{tabular}[c]{@{}l@{}}0: "He ate it anyway.",\\ 1: "He put it in the toaster.",\\ 2: "He got some bread.",\\ 3: "It shot out of the toaster and onto the floor.",\\ 4: "He was making toast."\end{tabular} \\ \midrule
% Causal Graph (CG) & \textit{a person make food} $\rightarrow$ \textit{a person get ingredient} \\ \midrule
Correct Answers & 4 \\ 
\midrule
\texttt{Llama3.2-3B} Answers (zero-shot) & 1, 2 \\ 
\texttt{Llama3.2-3B} Answers & 1, 4 \\ 
\texttt{Llama3.2-3B} Answers + CG & 1, 4 \\ 
\midrule
\texttt{Llama3.1-8B} Answers(zero-shot) & 1, 3 \\ 
\texttt{Llama3.1-8B} Answers  & 2 \\ 
\texttt{Llama3.1-8B} Answers + CG & 4 \\ 
\midrule
\texttt{Llama2-7B} Answers (zero-shot) & 1, 2 \\ 
\texttt{Llama2-7B} Answers & 4 \\ 
\texttt{Llama2-7B} Answers + CG & 1, 4 \\ 
\bottomrule
\end{tabular}%
}
\caption{Examples of multi-choice Abstract-QA reasoning with bi-level COT.}
\label{tab:examples_cot}
\end{table*}

\begin{table*}[hbt!]
\centering
\resizebox{\linewidth}{!}{%
\begin{tabular}{l|p{12cm}}
\toprule
Story & He wanted toast, so he got some bread and put it in the toaster. When it popped out and landed on the floor, he ate it anyway. \\
\midrule
Abstract Question & The question describes an event where \textit{`a person got another thing (that is an ingredient in another thing'}. What could be the cause of the event?\\
\midrule
Choices & \begin{tabular}[c]{@{}l@{}}0: "He ate it anyway.",\\ 1: "He put it in the toaster.",\\ 2: "He got some bread.",\\ 3: "It shot out of the toaster and onto the floor.",\\ 4: "He was making toast."\end{tabular} \\ \midrule
Prompt & The event \textit{`a person got another thing (that is an ingredient in another thing'} is described by one of the sentences in the story context. First identify that part
of the story. Then retrieve the event mentioned in the story that is a
corresponding cause/effect.\\ 
\midrule
% Causal Graph (CG) & \textit{a person make food} $\rightarrow$ \textit{a person get ingredient} \\ \midrule
Correct Answers & 4 \\ 
\midrule
Rationale & To solve this question, let's break down the steps: \\
& 1. Identify the sentence describing `another thing which is an ingredient in another thing': - The relevant sentence from the given text is not explicitly stated but rather implied through analysis. In this case, what seems like `he made breakfast food'. In other words - Toaster = Breakfast Food; Bread + Toast = Ingredients for the meal at hand.  Therefore the first item on our consideration has been identified as number four because if you are going into kitchen then most likely, your objective would be about cooking something.\\ 
& 2. Now find any events occurring after items were used during said activity using index numbers provided against them starting their respective sequence no. This means looking back towards end, start at \#5 because there will definitely exist only those indexed things found before each index respectively being associated w/ this particular action taking place afterward within same timeframe i.e., related actions happening after certain things are prepared here. Considering these points: We look forward now...4 provides our initial object `the ingredients'. This makes us say - That must come after getting things ready prior than eating. Therefore we also know - getting stuff does indeed precede putting things inside the device whose result gets eaten later.\\ 
& So while option zero fits too since all meals need ending act (eating) coming second to doing preparatory tasks involving having desired objects (`bread').. But finally- lets verify... As shown below, the correct answer(s): 4, 1\\ 
\bottomrule
\end{tabular}%
}
\caption{Example of multi-choice Abstract-QA reasoning with bi-level COT in \texttt{Llama3.2-3B}.}
\label{tab:examples_cot_step}
\end{table*}



