This paper introduces ACCESS, a benchmark for abstract causal event discovery and reasoning. We present a pipeline that combines automatic methods and human crowd-sourcing to extract $1,494$ causal relations among $725$ abstract events. We demonstrate that incorporating causal knowledge from our benchmark leads to improvements in QA reasoning tasks for LLMs. However, we also highlight challenges in automatic event abstraction identification and causal discovery, where in the latter, the popular statistical algorithms perform poorly in recovering our sub-graphs of fewer than $50$ nodes.  Our empirical evidence also suggests that LLMs are not ready to perform causal inference effectively due to the lack of effective acquisition of two critical sub-processes: abstract reasoning and causal discovery. This underscores the need for future research to equip the models with these essential skills for achieving true causal reasoning.

\section*{Limitations}
Our benchmark is built upon \texttt{GLUCOSE} \citep{mostafazadeh-etal-2020-glucose} whose scope is limited to everyday children's stories. Acknowledging this limitation, we propose a reproducible data construction pipeline applicable for curating diverse corpora of event causality.  Since \texttt{ACCESS} primarily addresses commonsense knowledge in real-world events, it is susceptible to biases regarding the judgement of semantic similarity and cause-and-effect relation of events. To mitigate this issue, our first effort is at every phase, to employ automatic methods alongside with human annotation, based on a set of objective definitions and criteria about events, abstractions and event causality. In the event abstraction phase, we specifically provide the annotators with a list of common scenarios (though non-exhaustive) indicating when the semantics of two expressions are considered similar of different to reduce potential biases.  Regarding the subjectivity in human causal judgment, while we focus on non-contextual causal commonsense knowledge, we leverage contextual signals in the original corpus whenever necessary to objectively guide the annotators' decisions on the causal relations. Due to the resource constraints, our causal graph is sparse and limited in size, which however still presents a challenge for statistical structure learning as well as LLMs on causal discovery tasks. One critical drawback in the experiment with statistical methods lies in the representation power of the co-occurrence matrix, which underscores the need for further research on representation learning of abstractions in language domain. As above, future works could also explore other resources to enlarge our causal graph and expand the coverage of real-world data. Such a causal graph could further be leveraged for causal inference according the engine described by Pearl \citep{pearl2009causality}, which seeks to answer causal queries across the three rungs of the Ladder of Causation i.e., associational (Rung 1), interventional (Rung 2), and counterfactual (Rung 3).

\section*{Ethics Statement}
To address potential misuse and uphold fairness and inclusivity, we discuss several ethical considerations for \texttt{ACCESS}. Firstly, it is crucial to clarify that the resources provided in this work are solely intended for research purposes. The narrative scenarios within \texttt{ACCESS} should not be utilized for insults, slander, or any other malicious purposes. Users are expected to adhere to the highest ethical standards, ensuring responsible and transparent use in line with ethical research practices. The creators of the dataset hold no responsibility for misuse or misinterpretation, and all necessary measures have been taken to respect privacy and ensure informed consent during the data collection process. Secondly, it is imperative to acknowledge the mental well-being of annotators during the data annotation process. Prior to data collection, this study underwent a thorough review and approval process by an internal review board. We require each annotator to take a break every two hours or whenever they feel uncomfortable.