\begin{table*}[bt!]
\centering
\resizebox{\linewidth}{!}{%
\begin{tabular}{p{3cm} p{15cm}}
\toprule
\textbf{Terminology} & \textbf{Description} \\
\midrule
Event &  Any situation, state or action that happens, occurs or holds. 
An event consists of four basic components: \texttt{participant(s)}, \texttt{action/state}, \texttt{location} and \texttt{time}. \\
\midrule
Event sentence & An English sentence describing an event in daily life. An event sentence must contain the \texttt{participant(s)} and \texttt{action/state} components while \texttt{location} and \texttt{time} are the optional components and should not influence the judgment of the meaning of the sentence. \\ 
\midrule
Cluster & A group of sentences describing the same event. \\
\midrule
Topic & An event that is unique to a particular cluster and sufficiently abstract to be described by all event sentences in that cluster.  \\ 
\midrule
Topic sentence & The English sentence describing the topic of a particular cluster. \\ 
\midrule
Story & A description of a series of connected events. \\ 
\bottomrule
\end{tabular}
}
\caption{Terminologies of the \texttt{ACCESS} benchmark.}\label{tab:term}
\end{table*}

We follow the definition of events provided in TimeML \cite{pustejovsky2003timebank}, ECB+ Annotation Guidelines \cite{cybulska2014guidelines} and Event StoryLine Corpus \cite{caselli2017event}. An \textbf{event} refers to any situation or state that happens or holds, which consists of four basic components: \texttt{action/state, location, time} and \texttt{participant(s)}. We here consider \texttt{location} and \texttt{time} as optional; for instance, the sentence $he \ goes \ to \ sleep$ is sufficiently an event. Each component of an event is associated with a concept in an ontology.\footnote{Ontology refers to a collection of concepts and their relations within a domain \cite{gruber1993translation}.} A realization of a concept in the event is an  \textbf{event mention}. An \textbf{event abstraction} is a tuple $\langle$\texttt{action (state)/concept}, \texttt{participant/concept}, \texttt{time/concept}, \texttt{location/concept}$\rangle$ shared among all mentions of that event, where each component is either an entity or a concept at an appropriate abstraction level. An event abstraction is itself an event and can be identified by replacing every component in its representation by a more abstract concept in the ontology. For example, 
$His \ girlfriend$ \texttt{[person]} $works$ \texttt{[action]} $for \ Starbucks$ \texttt{[location]} $on \ the \ weekends$ \texttt{[time]}. 


From another point of view, an event abstraction is a generalization of a \textbf{cluster} of event mentions that describe the same event. Two event mentions are \textbf{equivalent} if they are associated with the same event abstraction. An event abstraction is \textbf{causally consistent} w.r.t. a set of event mentions, if (1) none of its mention pairs at the semantic level contains a causal relationship, and (2) the semantics of all its mentions are either the cause or the effect of mentions in another event abstraction. Table \ref{tab:term} describes all the terms used in this paper and throughout the annotation process.

\paragraph{Definition of causation.} Based on the counterfactual theory of causation \citep{lewis2013counterfactuals}, an event $x$ is said to \textbf{cause} another event $y$ and event $y$ is said to be an \textbf{effect} of event $x$ if (1) event $y$ temporally follows event $x$ directly i.e., there are no intermediate events or if there is one, it must rarely occur, \underline{and} (2) event $y$ would not commonly occur if event $x$ did not occur. It is worth noting that unlike such datasets as \texttt{BECauSE} \citep{dunietz2017because} or \texttt{CauseNet} \citep{heindorf2020causenet} that consider causality between concepts, here causality is defined on the event (sentence) level, which takes into account the interaction of multiple participants. In statistical causality literature,  there exist $3$ causal structures of interest: \textit{confounder}, \textit{collider} and \textit{mediator}. For random variables $X, Y, Z$,

\begin{itemize}
    \item $Z$ is a called confounder if it causes both $X$ and $Y$, written as $X \leftarrow Z \rightarrow Y$;
    \item $Z$ is a collider when $Z$ is a common child of $X$ and $Y$ but $X$ and $Y$ themselves are not related, written as $X \rightarrow Z \leftarrow Y$;
    \item $Z$ is a mediator if there is a chain $X$ causes $Z$ and $Z$ causes $Y$, written as  $X \rightarrow Z \rightarrow Y$.
\end{itemize}


\paragraph{Quality criteria.}
We present the overarching criteria that guide our data construction process. These criteria aim to ensure that the event abstractions i.e., clusters of event mentions, in ACCESS achieve \textbf{causal consistency}:
% \vspace{-0.3em}
\begin{enumerate} % [noitemsep]
    \item Every cluster must be assigned with only one event abstraction. 
    
    \item All event mentions in each cluster must describe the same event and that event (abstraction) must be sufficiently abstract to cover all instances while being specific about the action taking place. 

    \item Every cluster must be in a cause-and-effect relation with at least one of the other clusters.

    % \item Between any two clusters, based on the event abstractions assigned to them, there only exists one single causal relation i.e., one cluster must be either a cause of an effect of the other. 

    \item If there exists a causal relation between events at one level, the causal relation must hold at its higher levels of abstraction in the hierarchy. For example, a causal relation between events at the \textit{mention} level must hold at the \textit{generalization} and \textit{abstraction} levels.

    \item A cluster $A$ is said to cause another cluster $B$ if \underline{at least one} event mentions in cluster $A$ causes any other event mentions in cluster $B$, according to the above cause-effect definition. 
\end{enumerate}



\begin{table*}[bt!]
\centering
\resizebox{\linewidth}{!}{%
\begin{tabular}{p{2.5cm} p{4.2cm}| p{3cm} p{6cm}}
\toprule
\multicolumn{2}{c}{\textbf{Cause event}} & \multicolumn{2}{c}{\textbf{Effect event}} \\
\midrule
Abstraction & Generalizations & Abstraction & Generalizations \\
\midrule
\textit{a person} & a person need money &  \textit{a person get} & a person take up a job  \\
\textit{need money} & a person need cash &  \textit{a job} & a person get a good job  \\
 & a person need to get money &  & a person get a job at a place  \\

\midrule
\textit{a person win} & a person win the contest & \textit{a person} & a person be celebrate an occasion  \\
 & a person win something &  \textit{celebrate} & a person have a celebration  \\
 
& a person end up winning &   & a person celebrate something  \\
\midrule
\textit{a person fall} & a person fall down & \textit{a person feel pain} & a person be in pain  \\
 & a person fall to the floor &   & a person experience pain in a body part  \\
 
& a person fall on the ground &  & a person 's body be in pain \\
\bottomrule
\end{tabular}
}
\caption{Examples of event causality on the abstraction and generalization level.}\label{tab:example}
\end{table*}