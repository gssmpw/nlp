\documentclass[11pt]{article}
\usepackage{latex/acl}
\usepackage{times}
\usepackage{latexsym}
\usepackage[T1]{fontenc}
\usepackage[utf8]{inputenc}
\usepackage{microtype}
\usepackage{inconsolata}
\usepackage{algorithm,algpseudocode}
\renewcommand{\algorithmicrequire}{\textbf{Input:}}
\renewcommand{\algorithmicensure}{\textbf{Output:}}
\usepackage{amsmath}
\usepackage{amsfonts}
\usepackage{color,soul}
\usepackage{enumitem}
\usepackage{graphicx}
\usepackage{colortbl}
\usepackage{xcolor}
\usepackage{subfigure}
\usepackage{booktabs}
\usepackage[normalem]{ulem}
\useunder{\uline}{\ul}{}
\usepackage{efbox,graphicx}
\efboxsetup{linecolor=black,linewidth=2pt}

%%%%% NEW MATH DEFINITIONS %%%%%

\usepackage{amsmath,amsfonts,bm}
\usepackage{derivative}
% Mark sections of captions for referring to divisions of figures
\newcommand{\figleft}{{\em (Left)}}
\newcommand{\figcenter}{{\em (Center)}}
\newcommand{\figright}{{\em (Right)}}
\newcommand{\figtop}{{\em (Top)}}
\newcommand{\figbottom}{{\em (Bottom)}}
\newcommand{\captiona}{{\em (a)}}
\newcommand{\captionb}{{\em (b)}}
\newcommand{\captionc}{{\em (c)}}
\newcommand{\captiond}{{\em (d)}}

% Highlight a newly defined term
\newcommand{\newterm}[1]{{\bf #1}}

% Derivative d 
\newcommand{\deriv}{{\mathrm{d}}}

% Figure reference, lower-case.
\def\figref#1{figure~\ref{#1}}
% Figure reference, capital. For start of sentence
\def\Figref#1{Figure~\ref{#1}}
\def\twofigref#1#2{figures \ref{#1} and \ref{#2}}
\def\quadfigref#1#2#3#4{figures \ref{#1}, \ref{#2}, \ref{#3} and \ref{#4}}
% Section reference, lower-case.
\def\secref#1{section~\ref{#1}}
% Section reference, capital.
\def\Secref#1{Section~\ref{#1}}
% Reference to two sections.
\def\twosecrefs#1#2{sections \ref{#1} and \ref{#2}}
% Reference to three sections.
\def\secrefs#1#2#3{sections \ref{#1}, \ref{#2} and \ref{#3}}
% Reference to an equation, lower-case.
\def\eqref#1{equation~\ref{#1}}
% Reference to an equation, upper case
\def\Eqref#1{Equation~\ref{#1}}
% A raw reference to an equation---avoid using if possible
\def\plaineqref#1{\ref{#1}}
% Reference to a chapter, lower-case.
\def\chapref#1{chapter~\ref{#1}}
% Reference to an equation, upper case.
\def\Chapref#1{Chapter~\ref{#1}}
% Reference to a range of chapters
\def\rangechapref#1#2{chapters\ref{#1}--\ref{#2}}
% Reference to an algorithm, lower-case.
\def\algref#1{algorithm~\ref{#1}}
% Reference to an algorithm, upper case.
\def\Algref#1{Algorithm~\ref{#1}}
\def\twoalgref#1#2{algorithms \ref{#1} and \ref{#2}}
\def\Twoalgref#1#2{Algorithms \ref{#1} and \ref{#2}}
% Reference to a part, lower case
\def\partref#1{part~\ref{#1}}
% Reference to a part, upper case
\def\Partref#1{Part~\ref{#1}}
\def\twopartref#1#2{parts \ref{#1} and \ref{#2}}

\def\ceil#1{\lceil #1 \rceil}
\def\floor#1{\lfloor #1 \rfloor}
\def\1{\bm{1}}
\newcommand{\train}{\mathcal{D}}
\newcommand{\valid}{\mathcal{D_{\mathrm{valid}}}}
\newcommand{\test}{\mathcal{D_{\mathrm{test}}}}

\def\eps{{\epsilon}}


% Random variables
\def\reta{{\textnormal{$\eta$}}}
\def\ra{{\textnormal{a}}}
\def\rb{{\textnormal{b}}}
\def\rc{{\textnormal{c}}}
\def\rd{{\textnormal{d}}}
\def\re{{\textnormal{e}}}
\def\rf{{\textnormal{f}}}
\def\rg{{\textnormal{g}}}
\def\rh{{\textnormal{h}}}
\def\ri{{\textnormal{i}}}
\def\rj{{\textnormal{j}}}
\def\rk{{\textnormal{k}}}
\def\rl{{\textnormal{l}}}
% rm is already a command, just don't name any random variables m
\def\rn{{\textnormal{n}}}
\def\ro{{\textnormal{o}}}
\def\rp{{\textnormal{p}}}
\def\rq{{\textnormal{q}}}
\def\rr{{\textnormal{r}}}
\def\rs{{\textnormal{s}}}
\def\rt{{\textnormal{t}}}
\def\ru{{\textnormal{u}}}
\def\rv{{\textnormal{v}}}
\def\rw{{\textnormal{w}}}
\def\rx{{\textnormal{x}}}
\def\ry{{\textnormal{y}}}
\def\rz{{\textnormal{z}}}

% Random vectors
\def\rvepsilon{{\mathbf{\epsilon}}}
\def\rvphi{{\mathbf{\phi}}}
\def\rvtheta{{\mathbf{\theta}}}
\def\rva{{\mathbf{a}}}
\def\rvb{{\mathbf{b}}}
\def\rvc{{\mathbf{c}}}
\def\rvd{{\mathbf{d}}}
\def\rve{{\mathbf{e}}}
\def\rvf{{\mathbf{f}}}
\def\rvg{{\mathbf{g}}}
\def\rvh{{\mathbf{h}}}
\def\rvu{{\mathbf{i}}}
\def\rvj{{\mathbf{j}}}
\def\rvk{{\mathbf{k}}}
\def\rvl{{\mathbf{l}}}
\def\rvm{{\mathbf{m}}}
\def\rvn{{\mathbf{n}}}
\def\rvo{{\mathbf{o}}}
\def\rvp{{\mathbf{p}}}
\def\rvq{{\mathbf{q}}}
\def\rvr{{\mathbf{r}}}
\def\rvs{{\mathbf{s}}}
\def\rvt{{\mathbf{t}}}
\def\rvu{{\mathbf{u}}}
\def\rvv{{\mathbf{v}}}
\def\rvw{{\mathbf{w}}}
\def\rvx{{\mathbf{x}}}
\def\rvy{{\mathbf{y}}}
\def\rvz{{\mathbf{z}}}

% Elements of random vectors
\def\erva{{\textnormal{a}}}
\def\ervb{{\textnormal{b}}}
\def\ervc{{\textnormal{c}}}
\def\ervd{{\textnormal{d}}}
\def\erve{{\textnormal{e}}}
\def\ervf{{\textnormal{f}}}
\def\ervg{{\textnormal{g}}}
\def\ervh{{\textnormal{h}}}
\def\ervi{{\textnormal{i}}}
\def\ervj{{\textnormal{j}}}
\def\ervk{{\textnormal{k}}}
\def\ervl{{\textnormal{l}}}
\def\ervm{{\textnormal{m}}}
\def\ervn{{\textnormal{n}}}
\def\ervo{{\textnormal{o}}}
\def\ervp{{\textnormal{p}}}
\def\ervq{{\textnormal{q}}}
\def\ervr{{\textnormal{r}}}
\def\ervs{{\textnormal{s}}}
\def\ervt{{\textnormal{t}}}
\def\ervu{{\textnormal{u}}}
\def\ervv{{\textnormal{v}}}
\def\ervw{{\textnormal{w}}}
\def\ervx{{\textnormal{x}}}
\def\ervy{{\textnormal{y}}}
\def\ervz{{\textnormal{z}}}

% Random matrices
\def\rmA{{\mathbf{A}}}
\def\rmB{{\mathbf{B}}}
\def\rmC{{\mathbf{C}}}
\def\rmD{{\mathbf{D}}}
\def\rmE{{\mathbf{E}}}
\def\rmF{{\mathbf{F}}}
\def\rmG{{\mathbf{G}}}
\def\rmH{{\mathbf{H}}}
\def\rmI{{\mathbf{I}}}
\def\rmJ{{\mathbf{J}}}
\def\rmK{{\mathbf{K}}}
\def\rmL{{\mathbf{L}}}
\def\rmM{{\mathbf{M}}}
\def\rmN{{\mathbf{N}}}
\def\rmO{{\mathbf{O}}}
\def\rmP{{\mathbf{P}}}
\def\rmQ{{\mathbf{Q}}}
\def\rmR{{\mathbf{R}}}
\def\rmS{{\mathbf{S}}}
\def\rmT{{\mathbf{T}}}
\def\rmU{{\mathbf{U}}}
\def\rmV{{\mathbf{V}}}
\def\rmW{{\mathbf{W}}}
\def\rmX{{\mathbf{X}}}
\def\rmY{{\mathbf{Y}}}
\def\rmZ{{\mathbf{Z}}}

% Elements of random matrices
\def\ermA{{\textnormal{A}}}
\def\ermB{{\textnormal{B}}}
\def\ermC{{\textnormal{C}}}
\def\ermD{{\textnormal{D}}}
\def\ermE{{\textnormal{E}}}
\def\ermF{{\textnormal{F}}}
\def\ermG{{\textnormal{G}}}
\def\ermH{{\textnormal{H}}}
\def\ermI{{\textnormal{I}}}
\def\ermJ{{\textnormal{J}}}
\def\ermK{{\textnormal{K}}}
\def\ermL{{\textnormal{L}}}
\def\ermM{{\textnormal{M}}}
\def\ermN{{\textnormal{N}}}
\def\ermO{{\textnormal{O}}}
\def\ermP{{\textnormal{P}}}
\def\ermQ{{\textnormal{Q}}}
\def\ermR{{\textnormal{R}}}
\def\ermS{{\textnormal{S}}}
\def\ermT{{\textnormal{T}}}
\def\ermU{{\textnormal{U}}}
\def\ermV{{\textnormal{V}}}
\def\ermW{{\textnormal{W}}}
\def\ermX{{\textnormal{X}}}
\def\ermY{{\textnormal{Y}}}
\def\ermZ{{\textnormal{Z}}}

% Vectors
\def\vzero{{\bm{0}}}
\def\vone{{\bm{1}}}
\def\vmu{{\bm{\mu}}}
\def\vtheta{{\bm{\theta}}}
\def\vphi{{\bm{\phi}}}
\def\va{{\bm{a}}}
\def\vb{{\bm{b}}}
\def\vc{{\bm{c}}}
\def\vd{{\bm{d}}}
\def\ve{{\bm{e}}}
\def\vf{{\bm{f}}}
\def\vg{{\bm{g}}}
\def\vh{{\bm{h}}}
\def\vi{{\bm{i}}}
\def\vj{{\bm{j}}}
\def\vk{{\bm{k}}}
\def\vl{{\bm{l}}}
\def\vm{{\bm{m}}}
\def\vn{{\bm{n}}}
\def\vo{{\bm{o}}}
\def\vp{{\bm{p}}}
\def\vq{{\bm{q}}}
\def\vr{{\bm{r}}}
\def\vs{{\bm{s}}}
\def\vt{{\bm{t}}}
\def\vu{{\bm{u}}}
\def\vv{{\bm{v}}}
\def\vw{{\bm{w}}}
\def\vx{{\bm{x}}}
\def\vy{{\bm{y}}}
\def\vz{{\bm{z}}}

% Elements of vectors
\def\evalpha{{\alpha}}
\def\evbeta{{\beta}}
\def\evepsilon{{\epsilon}}
\def\evlambda{{\lambda}}
\def\evomega{{\omega}}
\def\evmu{{\mu}}
\def\evpsi{{\psi}}
\def\evsigma{{\sigma}}
\def\evtheta{{\theta}}
\def\eva{{a}}
\def\evb{{b}}
\def\evc{{c}}
\def\evd{{d}}
\def\eve{{e}}
\def\evf{{f}}
\def\evg{{g}}
\def\evh{{h}}
\def\evi{{i}}
\def\evj{{j}}
\def\evk{{k}}
\def\evl{{l}}
\def\evm{{m}}
\def\evn{{n}}
\def\evo{{o}}
\def\evp{{p}}
\def\evq{{q}}
\def\evr{{r}}
\def\evs{{s}}
\def\evt{{t}}
\def\evu{{u}}
\def\evv{{v}}
\def\evw{{w}}
\def\evx{{x}}
\def\evy{{y}}
\def\evz{{z}}

% Matrix
\def\mA{{\bm{A}}}
\def\mB{{\bm{B}}}
\def\mC{{\bm{C}}}
\def\mD{{\bm{D}}}
\def\mE{{\bm{E}}}
\def\mF{{\bm{F}}}
\def\mG{{\bm{G}}}
\def\mH{{\bm{H}}}
\def\mI{{\bm{I}}}
\def\mJ{{\bm{J}}}
\def\mK{{\bm{K}}}
\def\mL{{\bm{L}}}
\def\mM{{\bm{M}}}
\def\mN{{\bm{N}}}
\def\mO{{\bm{O}}}
\def\mP{{\bm{P}}}
\def\mQ{{\bm{Q}}}
\def\mR{{\bm{R}}}
\def\mS{{\bm{S}}}
\def\mT{{\bm{T}}}
\def\mU{{\bm{U}}}
\def\mV{{\bm{V}}}
\def\mW{{\bm{W}}}
\def\mX{{\bm{X}}}
\def\mY{{\bm{Y}}}
\def\mZ{{\bm{Z}}}
\def\mBeta{{\bm{\beta}}}
\def\mPhi{{\bm{\Phi}}}
\def\mLambda{{\bm{\Lambda}}}
\def\mSigma{{\bm{\Sigma}}}

% Tensor
\DeclareMathAlphabet{\mathsfit}{\encodingdefault}{\sfdefault}{m}{sl}
\SetMathAlphabet{\mathsfit}{bold}{\encodingdefault}{\sfdefault}{bx}{n}
\newcommand{\tens}[1]{\bm{\mathsfit{#1}}}
\def\tA{{\tens{A}}}
\def\tB{{\tens{B}}}
\def\tC{{\tens{C}}}
\def\tD{{\tens{D}}}
\def\tE{{\tens{E}}}
\def\tF{{\tens{F}}}
\def\tG{{\tens{G}}}
\def\tH{{\tens{H}}}
\def\tI{{\tens{I}}}
\def\tJ{{\tens{J}}}
\def\tK{{\tens{K}}}
\def\tL{{\tens{L}}}
\def\tM{{\tens{M}}}
\def\tN{{\tens{N}}}
\def\tO{{\tens{O}}}
\def\tP{{\tens{P}}}
\def\tQ{{\tens{Q}}}
\def\tR{{\tens{R}}}
\def\tS{{\tens{S}}}
\def\tT{{\tens{T}}}
\def\tU{{\tens{U}}}
\def\tV{{\tens{V}}}
\def\tW{{\tens{W}}}
\def\tX{{\tens{X}}}
\def\tY{{\tens{Y}}}
\def\tZ{{\tens{Z}}}


% Graph
\def\gA{{\mathcal{A}}}
\def\gB{{\mathcal{B}}}
\def\gC{{\mathcal{C}}}
\def\gD{{\mathcal{D}}}
\def\gE{{\mathcal{E}}}
\def\gF{{\mathcal{F}}}
\def\gG{{\mathcal{G}}}
\def\gH{{\mathcal{H}}}
\def\gI{{\mathcal{I}}}
\def\gJ{{\mathcal{J}}}
\def\gK{{\mathcal{K}}}
\def\gL{{\mathcal{L}}}
\def\gM{{\mathcal{M}}}
\def\gN{{\mathcal{N}}}
\def\gO{{\mathcal{O}}}
\def\gP{{\mathcal{P}}}
\def\gQ{{\mathcal{Q}}}
\def\gR{{\mathcal{R}}}
\def\gS{{\mathcal{S}}}
\def\gT{{\mathcal{T}}}
\def\gU{{\mathcal{U}}}
\def\gV{{\mathcal{V}}}
\def\gW{{\mathcal{W}}}
\def\gX{{\mathcal{X}}}
\def\gY{{\mathcal{Y}}}
\def\gZ{{\mathcal{Z}}}

% Sets
\def\sA{{\mathbb{A}}}
\def\sB{{\mathbb{B}}}
\def\sC{{\mathbb{C}}}
\def\sD{{\mathbb{D}}}
% Don't use a set called E, because this would be the same as our symbol
% for expectation.
\def\sF{{\mathbb{F}}}
\def\sG{{\mathbb{G}}}
\def\sH{{\mathbb{H}}}
\def\sI{{\mathbb{I}}}
\def\sJ{{\mathbb{J}}}
\def\sK{{\mathbb{K}}}
\def\sL{{\mathbb{L}}}
\def\sM{{\mathbb{M}}}
\def\sN{{\mathbb{N}}}
\def\sO{{\mathbb{O}}}
\def\sP{{\mathbb{P}}}
\def\sQ{{\mathbb{Q}}}
\def\sR{{\mathbb{R}}}
\def\sS{{\mathbb{S}}}
\def\sT{{\mathbb{T}}}
\def\sU{{\mathbb{U}}}
\def\sV{{\mathbb{V}}}
\def\sW{{\mathbb{W}}}
\def\sX{{\mathbb{X}}}
\def\sY{{\mathbb{Y}}}
\def\sZ{{\mathbb{Z}}}

% Entries of a matrix
\def\emLambda{{\Lambda}}
\def\emA{{A}}
\def\emB{{B}}
\def\emC{{C}}
\def\emD{{D}}
\def\emE{{E}}
\def\emF{{F}}
\def\emG{{G}}
\def\emH{{H}}
\def\emI{{I}}
\def\emJ{{J}}
\def\emK{{K}}
\def\emL{{L}}
\def\emM{{M}}
\def\emN{{N}}
\def\emO{{O}}
\def\emP{{P}}
\def\emQ{{Q}}
\def\emR{{R}}
\def\emS{{S}}
\def\emT{{T}}
\def\emU{{U}}
\def\emV{{V}}
\def\emW{{W}}
\def\emX{{X}}
\def\emY{{Y}}
\def\emZ{{Z}}
\def\emSigma{{\Sigma}}

% entries of a tensor
% Same font as tensor, without \bm wrapper
\newcommand{\etens}[1]{\mathsfit{#1}}
\def\etLambda{{\etens{\Lambda}}}
\def\etA{{\etens{A}}}
\def\etB{{\etens{B}}}
\def\etC{{\etens{C}}}
\def\etD{{\etens{D}}}
\def\etE{{\etens{E}}}
\def\etF{{\etens{F}}}
\def\etG{{\etens{G}}}
\def\etH{{\etens{H}}}
\def\etI{{\etens{I}}}
\def\etJ{{\etens{J}}}
\def\etK{{\etens{K}}}
\def\etL{{\etens{L}}}
\def\etM{{\etens{M}}}
\def\etN{{\etens{N}}}
\def\etO{{\etens{O}}}
\def\etP{{\etens{P}}}
\def\etQ{{\etens{Q}}}
\def\etR{{\etens{R}}}
\def\etS{{\etens{S}}}
\def\etT{{\etens{T}}}
\def\etU{{\etens{U}}}
\def\etV{{\etens{V}}}
\def\etW{{\etens{W}}}
\def\etX{{\etens{X}}}
\def\etY{{\etens{Y}}}
\def\etZ{{\etens{Z}}}

% The true underlying data generating distribution
\newcommand{\pdata}{p_{\rm{data}}}
\newcommand{\ptarget}{p_{\rm{target}}}
\newcommand{\pprior}{p_{\rm{prior}}}
\newcommand{\pbase}{p_{\rm{base}}}
\newcommand{\pref}{p_{\rm{ref}}}

% The empirical distribution defined by the training set
\newcommand{\ptrain}{\hat{p}_{\rm{data}}}
\newcommand{\Ptrain}{\hat{P}_{\rm{data}}}
% The model distribution
\newcommand{\pmodel}{p_{\rm{model}}}
\newcommand{\Pmodel}{P_{\rm{model}}}
\newcommand{\ptildemodel}{\tilde{p}_{\rm{model}}}
% Stochastic autoencoder distributions
\newcommand{\pencode}{p_{\rm{encoder}}}
\newcommand{\pdecode}{p_{\rm{decoder}}}
\newcommand{\precons}{p_{\rm{reconstruct}}}

\newcommand{\laplace}{\mathrm{Laplace}} % Laplace distribution

\newcommand{\E}{\mathbb{E}}
\newcommand{\Ls}{\mathcal{L}}
\newcommand{\R}{\mathbb{R}}
\newcommand{\emp}{\tilde{p}}
\newcommand{\lr}{\alpha}
\newcommand{\reg}{\lambda}
\newcommand{\rect}{\mathrm{rectifier}}
\newcommand{\softmax}{\mathrm{softmax}}
\newcommand{\sigmoid}{\sigma}
\newcommand{\softplus}{\zeta}
\newcommand{\KL}{D_{\mathrm{KL}}}
\newcommand{\Var}{\mathrm{Var}}
\newcommand{\standarderror}{\mathrm{SE}}
\newcommand{\Cov}{\mathrm{Cov}}
% Wolfram Mathworld says $L^2$ is for function spaces and $\ell^2$ is for vectors
% But then they seem to use $L^2$ for vectors throughout the site, and so does
% wikipedia.
\newcommand{\normlzero}{L^0}
\newcommand{\normlone}{L^1}
\newcommand{\normltwo}{L^2}
\newcommand{\normlp}{L^p}
\newcommand{\normmax}{L^\infty}

\newcommand{\parents}{Pa} % See usage in notation.tex. Chosen to match Daphne's book.

\DeclareMathOperator*{\argmax}{arg\,max}
\DeclareMathOperator*{\argmin}{arg\,min}

\DeclareMathOperator{\sign}{sign}
\DeclareMathOperator{\Tr}{Tr}
\let\ab\allowbreak

\newcommand{\Vy}[1]{\textcolor{red}{Vy:~#1}}
\newcommand{\lizhen}[1]{{\textcolor{brown}{Lizhen:~#1}}}

\title{ACCESS : A Benchmark for \\ Abstract Causal Event Discovery and Reasoning}


\author{Vy Vo \quad Lizhen Qu \quad Tao Feng \quad Yuncheng Hua \quad Xiaoxi Kang\\
{\bf Songhai Fan \quad Tim Dwyer \quad Lay-Ki Soon \quad Gholamreza Haffari} \\
Monash University, Australia \\ 
\texttt{\{firstname.lastname\}@monash.edu}
}


\begin{document}
\maketitle
\begin{abstract}
  Identifying cause-and-effect relationships is critical to understanding real-world dynamics and ultimately causal reasoning. Existing methods for identifying event causality in NLP, including those based on Large Language Models (LLMs), exhibit difficulties in out-of-distribution settings due to the limited scale and heavy reliance on lexical cues within available benchmarks. Modern benchmarks, inspired by probabilistic causal inference, have attempted to construct causal graphs of events as a robust representation of causal knowledge, where \texttt{CRAB} \citep{romanou2023crab} is one such recent benchmark along this line. In this paper, we introduce \texttt{ACCESS}, a benchmark designed for discovery and reasoning over abstract causal events. Unlike existing resources, \texttt{ACCESS}  focuses on causality of everyday life events on the abstraction level. We propose a pipeline for identifying abstractions for event generalizations from \texttt{GLUCOSE} \citep{mostafazadeh-etal-2020-glucose}, a large-scale dataset of implicit commonsense causal knowledge, from which we subsequently extract $1,4$K causal pairs. Our experiments highlight the ongoing challenges of using statistical methods and/or LLMs for automatic abstraction identification and causal discovery in NLP. Nonetheless, we demonstrate that the abstract causal knowledge provided in \texttt{ACCESS} can be leveraged for enhancing QA reasoning performance in LLMs.
\end{abstract}



% \iftaclpubformat
\section{Introduction}
\section{Introduction}
Backdoor attacks pose a concealed yet profound security risk to machine learning (ML) models, for which the adversaries can inject a stealth backdoor into the model during training, enabling them to illicitly control the model's output upon encountering predefined inputs. These attacks can even occur without the knowledge of developers or end-users, thereby undermining the trust in ML systems. As ML becomes more deeply embedded in critical sectors like finance, healthcare, and autonomous driving \citep{he2016deep, liu2020computing, tournier2019mrtrix3, adjabi2020past}, the potential damage from backdoor attacks grows, underscoring the emergency for developing robust defense mechanisms against backdoor attacks.

To address the threat of backdoor attacks, researchers have developed a variety of strategies \cite{liu2018fine,wu2021adversarial,wang2019neural,zeng2022adversarial,zhu2023neural,Zhu_2023_ICCV, wei2024shared,wei2024d3}, aimed at purifying backdoors within victim models. These methods are designed to integrate with current deployment workflows seamlessly and have demonstrated significant success in mitigating the effects of backdoor triggers \cite{wubackdoorbench, wu2023defenses, wu2024backdoorbench,dunnett2024countering}.  However, most state-of-the-art (SOTA) backdoor purification methods operate under the assumption that a small clean dataset, often referred to as \textbf{auxiliary dataset}, is available for purification. Such an assumption poses practical challenges, especially in scenarios where data is scarce. To tackle this challenge, efforts have been made to reduce the size of the required auxiliary dataset~\cite{chai2022oneshot,li2023reconstructive, Zhu_2023_ICCV} and even explore dataset-free purification techniques~\cite{zheng2022data,hong2023revisiting,lin2024fusing}. Although these approaches offer some improvements, recent evaluations \cite{dunnett2024countering, wu2024backdoorbench} continue to highlight the importance of sufficient auxiliary data for achieving robust defenses against backdoor attacks.

While significant progress has been made in reducing the size of auxiliary datasets, an equally critical yet underexplored question remains: \emph{how does the nature of the auxiliary dataset affect purification effectiveness?} In  real-world  applications, auxiliary datasets can vary widely, encompassing in-distribution data, synthetic data, or external data from different sources. Understanding how each type of auxiliary dataset influences the purification effectiveness is vital for selecting or constructing the most suitable auxiliary dataset and the corresponding technique. For instance, when multiple datasets are available, understanding how different datasets contribute to purification can guide defenders in selecting or crafting the most appropriate dataset. Conversely, when only limited auxiliary data is accessible, knowing which purification technique works best under those constraints is critical. Therefore, there is an urgent need for a thorough investigation into the impact of auxiliary datasets on purification effectiveness to guide defenders in  enhancing the security of ML systems. 

In this paper, we systematically investigate the critical role of auxiliary datasets in backdoor purification, aiming to bridge the gap between idealized and practical purification scenarios.  Specifically, we first construct a diverse set of auxiliary datasets to emulate real-world conditions, as summarized in Table~\ref{overall}. These datasets include in-distribution data, synthetic data, and external data from other sources. Through an evaluation of SOTA backdoor purification methods across these datasets, we uncover several critical insights: \textbf{1)} In-distribution datasets, particularly those carefully filtered from the original training data of the victim model, effectively preserve the model’s utility for its intended tasks but may fall short in eliminating backdoors. \textbf{2)} Incorporating OOD datasets can help the model forget backdoors but also bring the risk of forgetting critical learned knowledge, significantly degrading its overall performance. Building on these findings, we propose Guided Input Calibration (GIC), a novel technique that enhances backdoor purification by adaptively transforming auxiliary data to better align with the victim model’s learned representations. By leveraging the victim model itself to guide this transformation, GIC optimizes the purification process, striking a balance between preserving model utility and mitigating backdoor threats. Extensive experiments demonstrate that GIC significantly improves the effectiveness of backdoor purification across diverse auxiliary datasets, providing a practical and robust defense solution.

Our main contributions are threefold:
\textbf{1) Impact analysis of auxiliary datasets:} We take the \textbf{first step}  in systematically investigating how different types of auxiliary datasets influence backdoor purification effectiveness. Our findings provide novel insights and serve as a foundation for future research on optimizing dataset selection and construction for enhanced backdoor defense.
%
\textbf{2) Compilation and evaluation of diverse auxiliary datasets:}  We have compiled and rigorously evaluated a diverse set of auxiliary datasets using SOTA purification methods, making our datasets and code publicly available to facilitate and support future research on practical backdoor defense strategies.
%
\textbf{3) Introduction of GIC:} We introduce GIC, the \textbf{first} dedicated solution designed to align auxiliary datasets with the model’s learned representations, significantly enhancing backdoor mitigation across various dataset types. Our approach sets a new benchmark for practical and effective backdoor defense.






\section{Causal Event Abstraction}\label{sec:setup}
\begin{table*}[bt!]
\centering
\resizebox{\linewidth}{!}{%
\begin{tabular}{p{3cm} p{15cm}}
\toprule
\textbf{Terminology} & \textbf{Description} \\
\midrule
Event &  Any situation, state or action that happens, occurs or holds. 
An event consists of four basic components: \texttt{participant(s)}, \texttt{action/state}, \texttt{location} and \texttt{time}. \\
\midrule
Event sentence & An English sentence describing an event in daily life. An event sentence must contain the \texttt{participant(s)} and \texttt{action/state} components while \texttt{location} and \texttt{time} are the optional components and should not influence the judgment of the meaning of the sentence. \\ 
\midrule
Cluster & A group of sentences describing the same event. \\
\midrule
Topic & An event that is unique to a particular cluster and sufficiently abstract to be described by all event sentences in that cluster.  \\ 
\midrule
Topic sentence & The English sentence describing the topic of a particular cluster. \\ 
\midrule
Story & A description of a series of connected events. \\ 
\bottomrule
\end{tabular}
}
\caption{Terminologies of the \texttt{ACCESS} benchmark.}\label{tab:term}
\end{table*}

We follow the definition of events provided in TimeML \cite{pustejovsky2003timebank}, ECB+ Annotation Guidelines \cite{cybulska2014guidelines} and Event StoryLine Corpus \cite{caselli2017event}. An \textbf{event} refers to any situation or state that happens or holds, which consists of four basic components: \texttt{action/state, location, time} and \texttt{participant(s)}. We here consider \texttt{location} and \texttt{time} as optional; for instance, the sentence $he \ goes \ to \ sleep$ is sufficiently an event. Each component of an event is associated with a concept in an ontology.\footnote{Ontology refers to a collection of concepts and their relations within a domain \cite{gruber1993translation}.} A realization of a concept in the event is an  \textbf{event mention}. An \textbf{event abstraction} is a tuple $\langle$\texttt{action (state)/concept}, \texttt{participant/concept}, \texttt{time/concept}, \texttt{location/concept}$\rangle$ shared among all mentions of that event, where each component is either an entity or a concept at an appropriate abstraction level. An event abstraction is itself an event and can be identified by replacing every component in its representation by a more abstract concept in the ontology. For example, 
$His \ girlfriend$ \texttt{[person]} $works$ \texttt{[action]} $for \ Starbucks$ \texttt{[location]} $on \ the \ weekends$ \texttt{[time]}. 


From another point of view, an event abstraction is a generalization of a \textbf{cluster} of event mentions that describe the same event. Two event mentions are \textbf{equivalent} if they are associated with the same event abstraction. An event abstraction is \textbf{causally consistent} w.r.t. a set of event mentions, if (1) none of its mention pairs at the semantic level contains a causal relationship, and (2) the semantics of all its mentions are either the cause or the effect of mentions in another event abstraction. Table \ref{tab:term} describes all the terms used in this paper and throughout the annotation process.

\paragraph{Definition of causation.} Based on the counterfactual theory of causation \citep{lewis2013counterfactuals}, an event $x$ is said to \textbf{cause} another event $y$ and event $y$ is said to be an \textbf{effect} of event $x$ if (1) event $y$ temporally follows event $x$ directly i.e., there are no intermediate events or if there is one, it must rarely occur, \underline{and} (2) event $y$ would not commonly occur if event $x$ did not occur. It is worth noting that unlike such datasets as \texttt{BECauSE} \citep{dunietz2017because} or \texttt{CauseNet} \citep{heindorf2020causenet} that consider causality between concepts, here causality is defined on the event (sentence) level, which takes into account the interaction of multiple participants. In statistical causality literature,  there exist $3$ causal structures of interest: \textit{confounder}, \textit{collider} and \textit{mediator}. For random variables $X, Y, Z$,

\begin{itemize}
    \item $Z$ is a called confounder if it causes both $X$ and $Y$, written as $X \leftarrow Z \rightarrow Y$;
    \item $Z$ is a collider when $Z$ is a common child of $X$ and $Y$ but $X$ and $Y$ themselves are not related, written as $X \rightarrow Z \leftarrow Y$;
    \item $Z$ is a mediator if there is a chain $X$ causes $Z$ and $Z$ causes $Y$, written as  $X \rightarrow Z \rightarrow Y$.
\end{itemize}


\paragraph{Quality criteria.}
We present the overarching criteria that guide our data construction process. These criteria aim to ensure that the event abstractions i.e., clusters of event mentions, in ACCESS achieve \textbf{causal consistency}:
% \vspace{-0.3em}
\begin{enumerate} % [noitemsep]
    \item Every cluster must be assigned with only one event abstraction. 
    
    \item All event mentions in each cluster must describe the same event and that event (abstraction) must be sufficiently abstract to cover all instances while being specific about the action taking place. 

    \item Every cluster must be in a cause-and-effect relation with at least one of the other clusters.

    % \item Between any two clusters, based on the event abstractions assigned to them, there only exists one single causal relation i.e., one cluster must be either a cause of an effect of the other. 

    \item If there exists a causal relation between events at one level, the causal relation must hold at its higher levels of abstraction in the hierarchy. For example, a causal relation between events at the \textit{mention} level must hold at the \textit{generalization} and \textit{abstraction} levels.

    \item A cluster $A$ is said to cause another cluster $B$ if \underline{at least one} event mentions in cluster $A$ causes any other event mentions in cluster $B$, according to the above cause-effect definition. 
\end{enumerate}



\begin{table*}[bt!]
\centering
\resizebox{\linewidth}{!}{%
\begin{tabular}{p{2.5cm} p{4.2cm}| p{3cm} p{6cm}}
\toprule
\multicolumn{2}{c}{\textbf{Cause event}} & \multicolumn{2}{c}{\textbf{Effect event}} \\
\midrule
Abstraction & Generalizations & Abstraction & Generalizations \\
\midrule
\textit{a person} & a person need money &  \textit{a person get} & a person take up a job  \\
\textit{need money} & a person need cash &  \textit{a job} & a person get a good job  \\
 & a person need to get money &  & a person get a job at a place  \\

\midrule
\textit{a person win} & a person win the contest & \textit{a person} & a person be celebrate an occasion  \\
 & a person win something &  \textit{celebrate} & a person have a celebration  \\
 
& a person end up winning &   & a person celebrate something  \\
\midrule
\textit{a person fall} & a person fall down & \textit{a person feel pain} & a person be in pain  \\
 & a person fall to the floor &   & a person experience pain in a body part  \\
 
& a person fall on the ground &  & a person 's body be in pain \\
\bottomrule
\end{tabular}
}
\caption{Examples of event causality on the abstraction and generalization level.}\label{tab:example}
\end{table*}


\section{The ACCESS Benchmark}
Effective human-robot cooperation in CoNav-Maze hinges on efficient communication. Maximizing the human’s information gain enables more precise guidance, which in turn accelerates task completion. Yet for the robot, the challenge is not only \emph{what} to communicate but also \emph{when}, as it must balance gathering information for the human with pursuing immediate goals when confident in its navigation.

To achieve this, we introduce \emph{Information Gain Monte Carlo Tree Search} (IG-MCTS), which optimizes both task-relevant objectives and the transmission of the most informative communication. IG-MCTS comprises three key components:
\textbf{(1)} A data-driven human perception model that tracks how implicit (movement) and explicit (image) information updates the human’s understanding of the maze layout.
\textbf{(2)} Reward augmentation to integrate multiple objectives effectively leveraging on the learned perception model.
\textbf{(3)} An uncertainty-aware MCTS that accounts for unobserved maze regions and human perception stochasticity.
% \begin{enumerate}[leftmargin=*]
%     \item A data-driven human perception model that tracks how implicit (movement) and explicit (image transmission) information updates the human’s understanding of the maze layout.
%     \item Reward augmentation to integrate multiple objectives effectively leveraging on the learned perception model.
%     \item An uncertainty-aware MCTS that accounts for unobserved maze regions and human perception stochasticity.
% \end{enumerate}

\subsection{Human Perception Dynamics}
% IG-MCTS seeks to optimize the expected novel information gained by the human through the robot’s actions, including both movement and communication. Achieving this requires a model of how the human acquires task-relevant information from the robot.

% \subsubsection{Perception MDP}
\label{sec:perception_mdp}
As the robot navigates the maze and transmits images, humans update their understanding of the environment. Based on the robot's path, they may infer that previously assumed blocked locations are traversable or detect discrepancies between the transmitted image and their map.  

To formally capture this process, we model the evolution of human perception as another Markov Decision Process, referred to as the \emph{Perception MDP}. The state space $\mathcal{X}$ represents all possible maze maps. The action space $\mathcal{S}^+ \times \mathcal{O}$ consists of the robot's trajectory between two image transmissions $\tau \in \mathcal{S}^+$ and an image $o \in \mathcal{O}$. The unknown transition function $F: (x, (\tau, o)) \rightarrow x'$ defines the human perception dynamics, which we aim to learn.

\subsubsection{Crowd-Sourced Transition Dataset}
To collect data, we designed a mapping task in the CoNav-Maze environment. Participants were tasked to edit their maps to match the true environment. A button triggers the robot's autonomous movements, after which it captures an image from a random angle.
In this mapping task, the robot, aware of both the true environment and the human’s map, visits predefined target locations and prioritizes areas with mislabeled grid cells on the human’s map.
% We assume that the robot has full knowledge of both the actual environment and the human’s current map. Leveraging this knowledge, the robot autonomously navigates to all predefined target locations. It then randomly selects subsequent goals to reach, prioritizing grid locations that remain mislabeled on the human’s map. This ensures that the robot’s actions are strategically focused on providing useful information to improve map accuracy.

We then recruited over $50$ annotators through Prolific~\cite{palan2018prolific} for the mapping task. Each annotator labeled three randomly generated mazes. They were allowed to proceed to the next maze once the robot had reached all four goal locations. However, they could spend additional time refining their map before moving on. To incentivize accuracy, annotators receive a performance-based bonus based on the final accuracy of their annotated map.


\subsubsection{Fully-Convolutional Dynamics Model}
\label{sec:nhpm}

We propose a Neural Human Perception Model (NHPM), a fully convolutional neural network (FCNN), to predict the human perception transition probabilities modeled in \Cref{sec:perception_mdp}. We denote the model as $F_\theta$ where $\theta$ represents the trainable weights. Such design echoes recent studies of model-based reinforcement learning~\cite{hansen2022temporal}, where the agent first learns the environment dynamics, potentially from image observations~\cite{hafner2019learning,watter2015embed}.

\begin{figure}[t]
    \centering
    \includegraphics[width=0.9\linewidth]{figures/ICML_25_CNN.pdf}
    \caption{Neural Human Perception Model (NHPM). \textbf{Left:} The human's current perception, the robot's trajectory since the last transmission, and the captured environment grids are individually processed into 2D masks. \textbf{Right:} A fully convolutional neural network predicts two masks: one for the probability of the human adding a wall to their map and another for removing a wall.}
    \label{fig:nhpm}
    \vskip -0.1in
\end{figure}

As illustrated in \Cref{fig:nhpm}, our model takes as input the human’s current perception, the robot’s path, and the image captured by the robot, all of which are transformed into a unified 2D representation. These inputs are concatenated along the channel dimension and fed into the CNN, which outputs a two-channel image: one predicting the probability of human adding a new wall and the other predicting the probability of removing a wall.

% Our approach builds on world model learning, where neural networks predict state transitions or environmental updates based on agent actions and observations. By leveraging the local feature extraction capabilities of CNNs, our model effectively captures spatial relationships and interprets local changes within the grid maze environment. Similar to prior work in localization and mapping, the CNN architecture is well-suited for processing spatially structured data and aligning the robot’s observations with human map updates.

To enhance robustness and generalization, we apply data augmentation techniques, including random rotation and flipping of the 2D inputs during training. These transformations are particularly beneficial in the grid maze environment, which is invariant to orientation changes.

\subsection{Perception-Aware Reward Augmentation}
The robot optimizes its actions over a planning horizon \( H \) by solving the following optimization problem:
\begin{subequations}
    \begin{align}
        \max_{a_{0:H-1}} \;
        & \mathop{\mathbb{E}}_{T, F} \left[ \sum_{t=0}^{H-1} \gamma^t \left(\underbrace{R_{\mathrm{task}}(\tau_{t+1}, \zeta)}_{\text{(1) Task reward}} + \underbrace{\|x_{t+1}-x_t\|_1}_{\text{(2) Info reward}}\right)\right] \label{obj}\\ 
        \subjectto \quad
        &x_{t+1} = F(x_t, (\tau_t, a_t)), \quad a_t\in\Ocal \label{const:perception_update}\\ 
        &\tau_{t+1} = \tau_t \oplus T(s_t, a_t), \quad a_t\in \Ucal\label{const:history_update}
    \end{align}
\end{subequations} 

The objective in~\eqref{obj} maximizes the expected cumulative reward over \( T \) and \( F \), reflecting the uncertainty in both physical transitions and human perception dynamics. The reward function consists of two components: 
(1) The \emph{task reward} incentivizes efficient navigation. The specific formulation for the task in this work is outlined in \Cref{appendix:task_reward}.
(2) The \emph{information reward} quantifies the change in the human’s perception due to robot actions, computed as the \( L_1 \)-norm distance between consecutive perception states.  

The constraint in~\eqref{const:history_update} ensures that for movement actions, the trajectory history \( \tau_t \) expands with new states based on the robot’s chosen actions, where \( s_t \) is the most recent state in \( \tau_t \), and \( \oplus \) represents sequence concatenation. 
In constraint~\eqref{const:perception_update}, the robot leverages the learned human perception dynamics \( F \) to estimate the evolution of the human’s understanding of the environment from perception state $x_t$ to $x_{t+1}$ based on the observed trajectory \( \tau_t \) and transmitted image \( a_t\in\Ocal \). 
% justify from a cognitive science perspective
% Cognitive science research has shown that humans read in a way to maximize the information gained from each word, aligning with the efficient coding principle, which prioritizes minimizing perceptual errors and extracting relevant features under limited processing capacity~\cite{kangassalo2020information}. Drawing on this principle, we hypothesize that humans similarly prioritize task-relevant information in multimodal settings. To accommodate this cognitive pattern, our robot policy selects and communicates high information-gain observations to human operators, akin to summarizing key insights from a lengthy article.
% % While the brain naturally seeks to gain information, the brain employs various strategies to manage information overload, including filtering~\cite{quiroga2004reducing}, limiting/working memory, and prioritizing information~\cite{arnold2023dealing}.
% In this context of our setup, we optimize the selection of camera angles to maximize the human operator's information gain about the environment. 

\subsection{Information Gain Monte Carlo Tree Search (IG-MCTS)}
IG-MCTS follows the four stages of Monte Carlo tree search: \emph{selection}, \emph{expansion}, \emph{rollout}, and \emph{backpropagation}, but extends it by incorporating uncertainty in both environment dynamics and human perception. We introduce uncertainty-aware simulations in the \emph{expansion} and \emph{rollout} phases and adjust \emph{backpropagation} with a value update rule that accounts for transition feasibility.

\subsubsection{Uncertainty-Aware Simulation}
As detailed in \Cref{algo:IG_MCTS}, both the \emph{expansion} and \emph{rollout} phases involve forward simulation of robot actions. Each tree node $v$ contains the state $(\tau, x)$, representing the robot's state history and current human perception. We handle the two action types differently as follows:
\begin{itemize}
    \item A movement action $u$ follows the environment dynamics $T$ as defined in \Cref{sec:problem}. Notably, the maze layout is observable up to distance $r$ from the robot's visited grids, while unexplored areas assume a $50\%$ chance of walls. In \emph{expansion}, the resulting search node $v'$ of this uncertain transition is assigned a feasibility value $\delta = 0.5$. In \emph{rollout}, the transition could fail and the robot remains in the same grid.
    
    \item The state transition for a communication step $o$ is governed by the learned stochastic human perception model $F_\theta$ as defined in \Cref{sec:nhpm}. Since transition probabilities are known, we compute the expected information reward $\bar{R_\mathrm{info}}$ directly:
    \begin{align*}
        \bar{R_\mathrm{info}}(\tau_t, x_t, o_t) &= \mathbb{E}_{x_{t+1}}\|x_{t+1}-x_t\|_1 \\
        &= \|p_\mathrm{add}\|_1 + \|p_\mathrm{remove}\|_1,
    \end{align*}
    where $(p_\mathrm{add}, p_\mathrm{remove}) \gets F_\theta(\tau_t, x_t, o_t)$ are the estimated probabilities of adding or removing walls from the map. 
    Directly computing the expected return at a node avoids the high number of visitations required to obtain an accurate value estimate.
\end{itemize}

% We denote a node in the search tree as $v$, where $s(v)$, $r(v)$, and $\delta(v)$ represent the state, reward, and transition feasibility at $v$, respectively. The visit count of $v$ is denoted as $N(v)$, while $Q(v)$ represents its total accumulated return. The set of child nodes of $v$ is denoted by $\mathbb{C}(v)$.

% The goal of each search is to plan a sequence for the robot until it reaches a goal or transmits a new image to the human. We initialize the search tree with the current human guidance $\zeta$, and the robot's approximation of human perception $x_0$. Each search node consists consists of the state information required by our reward augmentation: $(\tau, x)$. A node is terminal if it is the resulting state of a communication step, or if the robot reaches a goal location. 

% A rollout from the expanded node simulates future transitions until reaching a terminal state or a predefined depth $H$. Actions are selected randomly from the available action set $\mathcal{A}(s)$. If an action's feasibility is uncertain due to the environment's unknown structure, the transition occurs with probability $\delta(s, a)$. When a random number draw deems the transition infeasible, the state remains unchanged. On the other hand, for communication steps, we don't resolve the uncertainty but instead compute the expected information gain reward: \philip{TODO: adjust notation}
% \begin{equation}
%     \mathbb{E}\left[R_\mathrm{info}(\tau, x')\right] = \sum \mathrm{NPM(\tau, o)}.
% \end{equation}

\subsubsection{Feasibility-Adjusted Backpropagation}
During backpropagation, the rewards obtained from the simulation phase are propagated back through the tree, updating the total value $Q(v)$ and the visitation count $N(v)$ for all nodes along the path to the root. Due to uncertainty in unexplored environment dynamics, the rollout return depends on the feasibility of the transition from the child node. Given a sample return \(q'_{\mathrm{sample}}\) at child node \(v'\), the parent node's return is:
\begin{equation}
    q_{\mathrm{sample}} = r + \gamma \left[ \delta' q'_{\mathrm{sample}} + (1 - \delta') \frac{Q(v)}{N(v)} \right],
\end{equation}
where $\delta'$ represents the probability of a successful transition. The term \((1 - \delta')\) accounts for failed transitions, relying instead on the current value estimate.

% By incorporating uncertainty-aware rollouts and backpropagation, our approach enables more robust decision-making in scenarios where the environment dynamics is unknown and avoids simulation of the stochastic human perception dynamics.



\section{Experiments}\label{sec:exp}
\section{Experiments}
\label{section5}

In this section, we conduct extensive experiments to show that \ourmethod~can significantly speed up the sampling of existing MR Diffusion. To rigorously validate the effectiveness of our method, we follow the settings and checkpoints from \cite{luo2024daclip} and only modify the sampling part. Our experiment is divided into three parts. Section \ref{mainresult} compares the sampling results for different NFE cases. Section \ref{effects} studies the effects of different parameter settings on our algorithm, including network parameterizations and solver types. In Section \ref{analysis}, we visualize the sampling trajectories to show the speedup achieved by \ourmethod~and analyze why noise prediction gets obviously worse when NFE is less than 20.


\subsection{Main results}\label{mainresult}

Following \cite{luo2024daclip}, we conduct experiments with ten different types of image degradation: blurry, hazy, JPEG-compression, low-light, noisy, raindrop, rainy, shadowed, snowy, and inpainting (see Appendix \ref{appd1} for details). We adopt LPIPS \citep{zhang2018lpips} and FID \citep{heusel2017fid} as main metrics for perceptual evaluation, and also report PSNR and SSIM \citep{wang2004ssim} for reference. We compare \ourmethod~with other sampling methods, including posterior sampling \citep{luo2024posterior} and Euler-Maruyama discretization \citep{kloeden1992sde}. We take two tasks as examples and the metrics are shown in Figure \ref{fig:main}. Unless explicitly mentioned, we always use \ourmethod~based on SDE solver, with data prediction and uniform $\lambda$. The complete experimental results can be found in Appendix \ref{appd3}. The results demonstrate that \ourmethod~converges in a few (5 or 10) steps and produces samples with stable quality. Our algorithm significantly reduces the time cost without compromising sampling performance, which is of great practical value for MR Diffusion.


\begin{figure}[!ht]
    \centering
    \begin{minipage}[b]{0.45\textwidth}
        \centering
        \includegraphics[width=1\textwidth, trim=0 20 0 0]{figs/main_result/7_lowlight_fid.pdf}
        \subcaption{FID on \textit{low-light} dataset}
        \label{fig:main(a)}
    \end{minipage}
    \begin{minipage}[b]{0.45\textwidth}
        \centering
        \includegraphics[width=1\textwidth, trim=0 20 0 0]{figs/main_result/7_lowlight_lpips.pdf}
        \subcaption{LPIPS on \textit{low-light} dataset}
        \label{fig:main(b)}
    \end{minipage}
    \begin{minipage}[b]{0.45\textwidth}
        \centering
        \includegraphics[width=1\textwidth, trim=0 20 0 0]{figs/main_result/10_motion_fid.pdf}
        \subcaption{FID on \textit{motion-blurry} dataset}
        \label{fig:main(c)}
    \end{minipage}
    \begin{minipage}[b]{0.45\textwidth}
        \centering
        \includegraphics[width=1\textwidth, trim=0 20 0 0]{figs/main_result/10_motion_lpips.pdf}
        \subcaption{LPIPS on \textit{motion-blurry} dataset}
        \label{fig:main(d)}
    \end{minipage}
    \caption{\textbf{Perceptual evaluations on \textit{low-light} and \textit{motion-blurry} datasets.}}
    \label{fig:main}
\end{figure}

\subsection{Effects of parameter choice}\label{effects}

In Table \ref{tab:ablat_param}, we compare the results of two network parameterizations. The data prediction shows stable performance across different NFEs. The noise prediction performs similarly to data prediction with large NFEs, but its performance deteriorates significantly with smaller NFEs. The detailed analysis can be found in Section \ref{section5.3}. In Table \ref{tab:ablat_solver}, we compare \ourmethod-ODE-d-2 and \ourmethod-SDE-d-2 on the \textit{inpainting} task, which are derived from PF-ODE and reverse-time SDE respectively. SDE-based solver works better with a large NFE, whereas ODE-based solver is more effective with a small NFE. In general, neither solver type is inherently better.


% In Table \ref{tab:hazy}, we study the impact of two step size schedules on the results. On the whole, uniform $\lambda$ performs slightly better than uniform $t$. Our algorithm follows the method of \cite{lu2022dpmsolverplus} to estimate the integral part of the solution, while the analytical part does not affect the error.  Consequently, our algorithm has the same global truncation error, that is $\mathcal{O}\left(h_{max}^{k}\right)$. Note that the initial and final values of $\lambda$ depend on noise schedule and are fixed. Therefore, uniform $\lambda$ scheduling leads to the smallest $h_{max}$ and works better.

\begin{table}[ht]
    \centering
    \begin{minipage}{0.5\textwidth}
    \small
    \renewcommand{\arraystretch}{1}
    \centering
    \caption{Ablation study of network parameterizations on the Rain100H dataset.}
    % \vspace{8pt}
    \resizebox{1\textwidth}{!}{
        \begin{tabular}{cccccc}
			\toprule[1.5pt]
            % \multicolumn{6}{c}{Rainy} \\
            % \cmidrule(lr){1-6}
             NFE & Parameterization      & LPIPS\textdownarrow & FID\textdownarrow &  PSNR\textuparrow & SSIM\textuparrow  \\
            \midrule[1pt]
            \multirow{2}{*}{50}
             & Noise Prediction & \textbf{0.0606}     & \textbf{27.28}   & \textbf{28.89}     & \textbf{0.8615}    \\
             & Data Prediction & 0.0620     & 27.65   & 28.85     & 0.8602    \\
            \cmidrule(lr){1-6}
            \multirow{2}{*}{20}
              & Noise Prediction & 0.1429     & 47.31   & 27.68     & 0.7954    \\
              & Data Prediction & \textbf{0.0635}     & \textbf{27.79}   & \textbf{28.60}     & \textbf{0.8559}    \\
            \cmidrule(lr){1-6}
            \multirow{2}{*}{10}
              & Noise Prediction & 1.376     & 402.3   & 6.623     & 0.0114    \\
              & Data Prediction & \textbf{0.0678}     & \textbf{29.54}   & \textbf{28.09}     & \textbf{0.8483}    \\
            \cmidrule(lr){1-6}
            \multirow{2}{*}{5}
              & Noise Prediction & 1.416     & 447.0   & 5.755     & 0.0051    \\
              & Data Prediction & \textbf{0.0637}     & \textbf{26.92}   & \textbf{28.82}     & \textbf{0.8685}    \\       
            \bottomrule[1.5pt]
        \end{tabular}}
        \label{tab:ablat_param}
    \end{minipage}
    \hspace{0.01\textwidth}
    \begin{minipage}{0.46\textwidth}
    \small
    \renewcommand{\arraystretch}{1}
    \centering
    \caption{Ablation study of solver types on the CelebA-HQ dataset.}
    % \vspace{8pt}
        \resizebox{1\textwidth}{!}{
        \begin{tabular}{cccccc}
			\toprule[1.5pt]
            % \multicolumn{6}{c}{Raindrop} \\     
            % \cmidrule(lr){1-6}
             NFE & Solver Type     & LPIPS\textdownarrow & FID\textdownarrow &  PSNR\textuparrow & SSIM\textuparrow  \\
            \midrule[1pt]
            \multirow{2}{*}{50}
             & ODE & 0.0499     & 22.91   & 28.49     & 0.8921    \\
             & SDE & \textbf{0.0402}     & \textbf{19.09}   & \textbf{29.15}     & \textbf{0.9046}    \\
            \cmidrule(lr){1-6}
            \multirow{2}{*}{20}
              & ODE & 0.0475    & 21.35   & 28.51     & 0.8940    \\
              & SDE & \textbf{0.0408}     & \textbf{19.13}   & \textbf{28.98}    & \textbf{0.9032}    \\
            \cmidrule(lr){1-6}
            \multirow{2}{*}{10}
              & ODE & \textbf{0.0417}    & 19.44   & \textbf{28.94}     & \textbf{0.9048}    \\
              & SDE & 0.0437     & \textbf{19.29}   & 28.48     & 0.8996    \\
            \cmidrule(lr){1-6}
            \multirow{2}{*}{5}
              & ODE & \textbf{0.0526}     & 27.44   & \textbf{31.02}     & \textbf{0.9335}    \\
              & SDE & 0.0529    & \textbf{24.02}   & 28.35     & 0.8930    \\
            \bottomrule[1.5pt]
        \end{tabular}}
        \label{tab:ablat_solver}
    \end{minipage}
\end{table}


% \renewcommand{\arraystretch}{1}
%     \centering
%     \caption{Ablation study of step size schedule on the RESIDE-6k dataset.}
%     % \vspace{8pt}
%         \resizebox{1\textwidth}{!}{
%         \begin{tabular}{cccccc}
% 			\toprule[1.5pt]
%             % \multicolumn{6}{c}{Raindrop} \\     
%             % \cmidrule(lr){1-6}
%              NFE & Schedule      & LPIPS\textdownarrow & FID\textdownarrow &  PSNR\textuparrow & SSIM\textuparrow  \\
%             \midrule[1pt]
%             \multirow{2}{*}{50}
%              & uniform $t$ & 0.0271     & 5.539   & 30.00     & 0.9351    \\
%              & uniform $\lambda$ & \textbf{0.0233}     & \textbf{4.993}   & \textbf{30.19}     & \textbf{0.9427}    \\
%             \cmidrule(lr){1-6}
%             \multirow{2}{*}{20}
%               & uniform $t$ & 0.0313     & 6.000   & 29.73     & 0.9270    \\
%               & uniform $\lambda$ & \textbf{0.0240}     & \textbf{5.077}   & \textbf{30.06}    & \textbf{0.9409}    \\
%             \cmidrule(lr){1-6}
%             \multirow{2}{*}{10}
%               & uniform $t$ & 0.0309     & 6.094   & 29.42     & 0.9274    \\
%               & uniform $\lambda$ & \textbf{0.0246}     & \textbf{5.228}   & \textbf{29.65}     & \textbf{0.9372}    \\
%             \cmidrule(lr){1-6}
%             \multirow{2}{*}{5}
%               & uniform $t$ & 0.0256     & 5.477   & \textbf{29.91}     & 0.9342    \\
%               & uniform $\lambda$ & \textbf{0.0228}     & \textbf{5.174}   & 29.65     & \textbf{0.9416}    \\
%             \bottomrule[1.5pt]
%         \end{tabular}}
%         \label{tab:ablat_schedule}



\subsection{Analysis}\label{analysis}
\label{section5.3}

\begin{figure}[ht!]
    \centering
    \begin{minipage}[t]{0.6\linewidth}
        \centering
        \includegraphics[width=\linewidth, trim=0 20 10 0]{figs/trajectory_a.pdf} %trim左下右上
        \subcaption{Sampling results.}
        \label{fig:traj(a)}
    \end{minipage}
    \begin{minipage}[t]{0.35\linewidth}
        \centering
        \includegraphics[width=\linewidth, trim=0 0 0 0]{figs/trajectory_b.pdf} %trim左下右上
        \subcaption{Trajectory.}
        \label{fig:traj(b)}
    \end{minipage}
    \caption{\textbf{Sampling trajectories.} In (a), we compare our method (with order 1 and order 2) and previous sampling methods (i.e., posterior sampling and Euler discretization) on a motion blurry image. The numbers in parentheses indicate the NFE. In (b), we illustrate trajectories of each sampling method. Previous methods need to take many unnecessary paths to converge. With few NFEs, they fail to reach the ground truth (i.e., the location of $\boldsymbol{x}_0$). Our methods follow a more direct trajectory.}
    \label{fig:traj}
\end{figure}

\textbf{Sampling trajectory.}~ Inspired by the design idea of NCSN \citep{song2019ncsn}, we provide a new perspective of diffusion sampling process. \cite{song2019ncsn} consider each data point (e.g., an image) as a point in high-dimensional space. During the diffusion process, noise is added to each point $\boldsymbol{x}_0$, causing it to spread throughout the space, while the score function (a neural network) \textit{remembers} the direction towards $\boldsymbol{x}_0$. In the sampling process, we start from a random point by sampling a Gaussian distribution and follow the guidance of the reverse-time SDE (or PF-ODE) and the score function to locate $\boldsymbol{x}_0$. By connecting each intermediate state $\boldsymbol{x}_t$, we obtain a sampling trajectory. However, this trajectory exists in a high-dimensional space, making it difficult to visualize. Therefore, we use Principal Component Analysis (PCA) to reduce $\boldsymbol{x}_t$ to two dimensions, obtaining the projection of the sampling trajectory in 2D space. As shown in Figure \ref{fig:traj}, we present an example. Previous sampling methods \citep{luo2024posterior} often require a long path to find $\boldsymbol{x}_0$, and reducing NFE can lead to cumulative errors, making it impossible to locate $\boldsymbol{x}_0$. In contrast, our algorithm produces more direct trajectories, allowing us to find $\boldsymbol{x}_0$ with fewer NFEs.

\begin{figure*}[ht]
    \centering
    \begin{minipage}[t]{0.45\linewidth}
        \centering
        \includegraphics[width=\linewidth, trim=0 0 0 0]{figs/convergence_a.pdf} %trim左下右上
        \subcaption{Sampling results.}
        \label{fig:convergence(a)}
    \end{minipage}
    \begin{minipage}[t]{0.43\linewidth}
        \centering
        \includegraphics[width=\linewidth, trim=0 20 0 0]{figs/convergence_b.pdf} %trim左下右上
        \subcaption{Ratio of convergence.}
        \label{fig:convergence(b)}
    \end{minipage}
    \caption{\textbf{Convergence of noise prediction and data prediction.} In (a), we choose a low-light image for example. The numbers in parentheses indicate the NFE. In (b), we illustrate the ratio of components of neural network output that satisfy the Taylor expansion convergence requirement.}
    \label{fig:converge}
\end{figure*}

\textbf{Numerical stability of parameterizations.}~ From Table 1, we observe poor sampling results for noise prediction in the case of few NFEs. The reason may be that the neural network parameterized by noise prediction is numerically unstable. Recall that we used Taylor expansion in Eq.(\ref{14}), and the condition for the equality to hold is $|\lambda-\lambda_s|<\boldsymbol{R}(s)$. And the radius of convergence $\boldsymbol{R}(t)$ can be calculated by
\begin{equation}
\frac{1}{\boldsymbol{R}(t)}=\lim_{n\rightarrow\infty}\left|\frac{\boldsymbol{c}_{n+1}(t)}{\boldsymbol{c}_n(t)}\right|,
\end{equation}
where $\boldsymbol{c}_n(t)$ is the coefficient of the $n$-th term in Taylor expansion. We are unable to compute this limit and can only compute the $n=0$ case as an approximation. The output of the neural network can be viewed as a vector, with each component corresponding to a radius of convergence. At each time step, we count the ratio of components that satisfy $\boldsymbol{R}_i(s)>|\lambda-\lambda_s|$ as a criterion for judging the convergence, where $i$ denotes the $i$-th component. As shown in Figure \ref{fig:converge}, the neural network parameterized by data prediction meets the convergence criteria at almost every step. However, the neural network parameterized by noise prediction always has components that cannot converge, which will lead to large errors and failed sampling. Therefore, data prediction has better numerical stability and is a more recommended choice.



\section{Conclusion}
This paper introduces ACCESS, a benchmark for abstract causal event discovery and reasoning. We present a pipeline that combines automatic methods and human crowd-sourcing to extract $1,494$ causal relations among $725$ abstract events. We demonstrate that incorporating causal knowledge from our benchmark leads to improvements in QA reasoning tasks for LLMs. However, we also highlight challenges in automatic event abstraction identification and causal discovery, where in the latter, the popular statistical algorithms perform poorly in recovering our sub-graphs of fewer than $50$ nodes.  Our empirical evidence also suggests that LLMs are not ready to perform causal inference effectively due to the lack of effective acquisition of two critical sub-processes: abstract reasoning and causal discovery. This underscores the need for future research to equip the models with these essential skills for achieving true causal reasoning.

\section*{Limitations}
Our benchmark is built upon \texttt{GLUCOSE} \citep{mostafazadeh-etal-2020-glucose} whose scope is limited to everyday children's stories. Acknowledging this limitation, we propose a reproducible data construction pipeline applicable for curating diverse corpora of event causality.  Since \texttt{ACCESS} primarily addresses commonsense knowledge in real-world events, it is susceptible to biases regarding the judgement of semantic similarity and cause-and-effect relation of events. To mitigate this issue, our first effort is at every phase, to employ automatic methods alongside with human annotation, based on a set of objective definitions and criteria about events, abstractions and event causality. In the event abstraction phase, we specifically provide the annotators with a list of common scenarios (though non-exhaustive) indicating when the semantics of two expressions are considered similar of different to reduce potential biases.  Regarding the subjectivity in human causal judgment, while we focus on non-contextual causal commonsense knowledge, we leverage contextual signals in the original corpus whenever necessary to objectively guide the annotators' decisions on the causal relations. Due to the resource constraints, our causal graph is sparse and limited in size, which however still presents a challenge for statistical structure learning as well as LLMs on causal discovery tasks. One critical drawback in the experiment with statistical methods lies in the representation power of the co-occurrence matrix, which underscores the need for further research on representation learning of abstractions in language domain. As above, future works could also explore other resources to enlarge our causal graph and expand the coverage of real-world data. Such a causal graph could further be leveraged for causal inference according the engine described by Pearl \citep{pearl2009causality}, which seeks to answer causal queries across the three rungs of the Ladder of Causation i.e., associational (Rung 1), interventional (Rung 2), and counterfactual (Rung 3).

\section*{Ethics Statement}
To address potential misuse and uphold fairness and inclusivity, we discuss several ethical considerations for \texttt{ACCESS}. Firstly, it is crucial to clarify that the resources provided in this work are solely intended for research purposes. The narrative scenarios within \texttt{ACCESS} should not be utilized for insults, slander, or any other malicious purposes. Users are expected to adhere to the highest ethical standards, ensuring responsible and transparent use in line with ethical research practices. The creators of the dataset hold no responsibility for misuse or misinterpretation, and all necessary measures have been taken to respect privacy and ensure informed consent during the data collection process. Secondly, it is imperative to acknowledge the mental well-being of annotators during the data annotation process. Prior to data collection, this study underwent a thorough review and approval process by an internal review board. We require each annotator to take a break every two hours or whenever they feel uncomfortable.

\section*{Acknowledgment}
This material is based on research sponsored by DARPA under agreement number HR001122C0029. The U.S. Government is authorized to reproduce and distribute reprints for Governmental purposes notwithstanding any copyright notation thereon This work is also partially supported by the DARPA Assured Neuro Symbolic Learning and Reasoning (ANSR) program under award number FA8750-23-2-1016.

\bibliography{ref}
\clearpage

\appendix

\section{Related Work}
\section{Related Work}

\subsection{Large 3D Reconstruction Models}
Recently, generalized feed-forward models for 3D reconstruction from sparse input views have garnered considerable attention due to their applicability in heavily under-constrained scenarios. The Large Reconstruction Model (LRM)~\cite{hong2023lrm} uses a transformer-based encoder-decoder pipeline to infer a NeRF reconstruction from just a single image. Newer iterations have shifted the focus towards generating 3D Gaussian representations from four input images~\cite{tang2025lgm, xu2024grm, zhang2025gslrm, charatan2024pixelsplat, chen2025mvsplat, liu2025mvsgaussian}, showing remarkable novel view synthesis results. The paradigm of transformer-based sparse 3D reconstruction has also successfully been applied to lifting monocular videos to 4D~\cite{ren2024l4gm}. \\
Yet, none of the existing works in the domain have studied the use-case of inferring \textit{animatable} 3D representations from sparse input images, which is the focus of our work. To this end, we build on top of the Large Gaussian Reconstruction Model (GRM)~\cite{xu2024grm}.

\subsection{3D-aware Portrait Animation}
A different line of work focuses on animating portraits in a 3D-aware manner.
MegaPortraits~\cite{drobyshev2022megaportraits} builds a 3D Volume given a source and driving image, and renders the animated source actor via orthographic projection with subsequent 2D neural rendering.
3D morphable models (3DMMs)~\cite{blanz19993dmm} are extensively used to obtain more interpretable control over the portrait animation. For example, StyleRig~\cite{tewari2020stylerig} demonstrates how a 3DMM can be used to control the data generated from a pre-trained StyleGAN~\cite{karras2019stylegan} network. ROME~\cite{khakhulin2022rome} predicts vertex offsets and texture of a FLAME~\cite{li2017flame} mesh from the input image.
A TriPlane representation is inferred and animated via FLAME~\cite{li2017flame} in multiple methods like Portrait4D~\cite{deng2024portrait4d}, Portrait4D-v2~\cite{deng2024portrait4dv2}, and GPAvatar~\cite{chu2024gpavatar}.
Others, such as VOODOO 3D~\cite{tran2024voodoo3d} and VOODOO XP~\cite{tran2024voodooxp}, learn their own expression encoder to drive the source person in a more detailed manner. \\
All of the aforementioned methods require nothing more than a single image of a person to animate it. This allows them to train on large monocular video datasets to infer a very generic motion prior that even translates to paintings or cartoon characters. However, due to their task formulation, these methods mostly focus on image synthesis from a frontal camera, often trading 3D consistency for better image quality by using 2D screen-space neural renderers. In contrast, our work aims to produce a truthful and complete 3D avatar representation from the input images that can be viewed from any angle.  

\subsection{Photo-realistic 3D Face Models}
The increasing availability of large-scale multi-view face datasets~\cite{kirschstein2023nersemble, ava256, pan2024renderme360, yang2020facescape} has enabled building photo-realistic 3D face models that learn a detailed prior over both geometry and appearance of human faces. HeadNeRF~\cite{hong2022headnerf} conditions a Neural Radiance Field (NeRF)~\cite{mildenhall2021nerf} on identity, expression, albedo, and illumination codes. VRMM~\cite{yang2024vrmm} builds a high-quality and relightable 3D face model using volumetric primitives~\cite{lombardi2021mvp}. One2Avatar~\cite{yu2024one2avatar} extends a 3DMM by anchoring a radiance field to its surface. More recently, GPHM~\cite{xu2025gphm} and HeadGAP~\cite{zheng2024headgap} have adopted 3D Gaussians to build a photo-realistic 3D face model. \\
Photo-realistic 3D face models learn a powerful prior over human facial appearance and geometry, which can be fitted to a single or multiple images of a person, effectively inferring a 3D head avatar. However, the fitting procedure itself is non-trivial and often requires expensive test-time optimization, impeding casual use-cases on consumer-grade devices. While this limitation may be circumvented by learning a generalized encoder that maps images into the 3D face model's latent space, another fundamental limitation remains. Even with more multi-view face datasets being published, the number of available training subjects rarely exceeds the thousands, making it hard to truly learn the full distibution of human facial appearance. Instead, our approach avoids generalizing over the identity axis by conditioning on some images of a person, and only generalizes over the expression axis for which plenty of data is available. 

A similar motivation has inspired recent work on codec avatars where a generalized network infers an animatable 3D representation given a registered mesh of a person~\cite{cao2022authentic, li2024uravatar}.
The resulting avatars exhibit excellent quality at the cost of several minutes of video capture per subject and expensive test-time optimization.
For example, URAvatar~\cite{li2024uravatar} finetunes their network on the given video recording for 3 hours on 8 A100 GPUs, making inference on consumer-grade devices impossible. In contrast, our approach directly regresses the final 3D head avatar from just four input images without the need for expensive test-time fine-tuning.



\section{Data Annotation Pipeline}\label{sup:annotation}
% \begin{table*}
% \centering
% \begin{tabular}{p{2.5cm} p{4.2cm}| p{2.5cm} p{5cm}}
% \toprule
% \multicolumn{2}{c}{\textbf{Cause event}} & \multicolumn{2}{c}{\textbf{Effect event}} \\
% \midrule
% Abstraction & Generalizations & Abstraction & Generalizations \\
% \midrule
% \textit{a person} & a person need money &  \textit{a person get} & a person take up a job  \\
% \textit{need money} & a person need cash &  \textit{a job} & a person get a good job  \\
%  & a person need to get money &  & a person get a job at a place  \\

% \midrule
% \textit{a person win} & a person win the contest & \textit{a person} & a person be celebrate an occasion  \\
%  & a person win something &  \textit{celebrate} & a person have a celebration  \\
 
% & a person end up winning &   & a person celebrate something  \\
% \midrule
% \textit{a person fall} & a person fall down & \textit{a person feel} & a person be in pain  \\
%  & a person fall to the floor &  \textit{pain} & a person experience pain in a body part  \\
 
% & a person fall on the ground &  & a person 's body be in pain \\
% \bottomrule
% \end{tabular}
% \caption{Examples of event causality on the abstraction and generalization level.}\label{tab:example}
% \end{table*}

We recruit in total $13$ university students in Malaysia aged $20-30$. The total hours are  $329.7$, where the hourly rate is RM$20$ (Malaysian ringgit), which is higher than the minimum wage of RM$7.1$.  

As for the annotation guidelines, we translate the technical terminologies in Section \ref{sec:setup} into layman language comprehensible to human annotators.  


\subsection{Abstract Event Extraction}\label{sup:annotation_ph1}
There are five steps in this annotation phase. Steps $1$ and $2$ are key to extracting abstract events, whereas Steps $3-5$ serve as post-processing to strengthen consistency among human annotators. 

\paragraph{Step 1: Sub-clustering.}
Each annotator is presented with a set of clusters generated from an automatic clustering algorithm. Each cluster contains multiple English sentences that describe events in daily life. Each word in every sentence is lemmatized to its base form so that the tense of the sentence does not influence the judgment of meaning. For every cluster, they are required to sub-group event sentences that are semantically similar or related together. There can exist clusters in which all sentences are related to one another; in this case no sub-clustering is needed. There can also be outlier events i.e., sentences that do not belong to any sub-clusters. For a sub-cluster to exist, it must contain at least two events. If an event cannot be sub-clustered, the annotator is to classify it as an outlier.	If a sentence is lexically or grammatically erroneous that makes it unjustifiable, the annotator is also asked to highlight and correct it whenever appropriate before clustering. 


Two event sentences are considered \textit{semantically related} or \textit{similar}\footnote{We use ``='' to denote semantic similarity and ``$\ne$" to denote semantic dissimilarity between two events.} if they describe the same event, and the decision must not be affected by the information about \texttt{location} and \texttt{time}. We note there is a difference between a \texttt{state/action} actually taking place with the prospect of the \texttt{state/action} taking place. In particular, we outline $11$ scenarios where word uses convey differences in meaning.	

\begin{enumerate}
    \item single participant vs. group of participants e.g., \emph{a person be playing in the park} $\ne$ \emph{a person and another person be playing in the park.}
    
    \item affirmation vs. negation e.g., \emph{a person be asleep}  $\ne$ \emph{a person do not sleep.}

    \item present vs. future tense e.g., \emph{a person go to sleep} $\ne$ \emph{a person will go to sleep.}

    \item ability e.g., \emph{a person do not eat} $\ne$ \emph{a person cannot eat.}

    \item intention or desire e.g., \emph{a person do not eat} $\ne$ \emph{a person do not want to eat.}

    \item deduction or possibility e.g., \emph{it rain} $\ne$ \emph{it may rain.}. 
    
    \item obligation, advice or prohibition	e.g., \emph{a person do not eat} $\ne$ \emph{a person should not eat}.

    \item offers, effort or decision e.g., \emph{a person help another person} $\ne$ \emph{a person offer to assist another person}; \emph{a person go to the gym} $\ne$ \emph{a person decide to go to the gym.}

    \item location as object. In some cases, the object receives an action from the verb refers to a place or location e.g., \emph{a person clean a place}. Here \textit{room} is considered an (spatial) item being taken action on and similar to any other items such as cup or a table $\rightarrow$ \emph{a person clean a place} \textcolor{red}{=} \emph{a person clean something.}

    \item multiple actions. Some sentences describe two actions happening at the same time e.g., \emph{a person take something and leave}. In order to evaluate its meaning, one  must select one of them to the key action. The key action is the action that is described by most of other events in the same cluster. This means that if most of the other events are about \textit{someone leaving somewhere}, the \textit{leave} action should be focused instead of \textit{take} action. 

    \item continuous vs. simple tense. Some sentences describe actions in the continuous state e.g., \emph{a person be go home}. We ignore the continuous state of the action and consider them equivalent to the action described simple tense $\rightarrow$  \emph{a person be go home} \textcolor{red}{=} \emph{a person go home.}	
\end{enumerate}

	
\paragraph{Step 2 : Topic identification.}

In this step, the annotator asked to identify the topic for every cluster or sub-cluster formed. The topic must first be an event, therefore it must contain at least two components: \texttt{participant(s)} and \texttt{action}. The topic must be specific about the state or action that takes place. At the same time, the topic must be written in a way that makes it general or abstract enough to include all event sentences.	Whenever possible, it is acceptable to use the most representative event sentence in a cluster as the topic.	

\paragraph{Intermediate processing.} In Steps $1$ and $2$, we divide the collection of clusters into $7$ batches. Each of the batch contains $60$ clusters and 
two workers are asked to annotate one same batch of clusters. This results in one cluster having two annotation results. Subsequently, an algorithm is run to automatically unify the results from two annotators. For every cluster in the original data, the algorithm starts by randomly selecting an event as a centroid. It then forms a sub-cluster around the centroid that contains all other events that are considered by both annotators to be semantically related to the centroid. The topic assigned to that sub-cluster is presented in the format \texttt{TOPIC : [Text 1] / [Text 2]} where \texttt{[Text 1]} is the topic assigned to events in this sub-cluster by the first annotator and \texttt{[Text 2]} is the topic assigned to them by the second annotator. Repeat the process with the other events until all instances are processed. Thereafter, any event that is not assigned to any cluster will exist as a stand-alone instance and temporarily be considered an outlier. 

The next steps focus on resolving the disagreements from two annotation results, which includes \textbf{Topic alignment} and \textbf{Outliers processing}. We assume that a sub-cluster is properly annotated if it (1) contain at least $2$ instances and (2) no annotators consider that sub-cluster to be an outlier.

\paragraph{Step 3: Topic alignment.}
Every cluster is now annotated with two topics. If both topics describe the same event, the annotator is asked to choose either or the one more representative. Otherwise, choose the one that fits most of the sentences in the cluster. If the chosen topic is already assigned to some previous cluster, merge the current cluster into that cluster. If at least one of the topics is Outliers (i.e., at least one annotator considers the sub-cluster as Outliers), temporarily view them as Outliers. 

\paragraph{Step 4: Outliers processing.}
The annotator moves on to process the outliers. For any event that is assigned by only one of the previous annotators to be outliers while assigned by the other to be associated some existing sub-cluster, the annotator is asked to merge it into the assigned sub-cluster if the event can be represented by the topic of that sub-cluster; otherwise, keep it as an outlier. For any event that is agreed by both annotators to be an outlier, the current annotator is asked to re-examine it for possible assignment to any existing sub-cluster. The merging decision must be again based on the conditions described in Step $1$. Any remaining stand-alone instances are discarded. 

\paragraph{Step 5: Topic matching.} This step aims to correct for potential mis-clustering from the automatic procedure. We obtain the outlier events and attempt re-categorize them into the post-annotated clustering results from all above steps. For each outlier, we present the annotators with a set of candidate clusters to which adding the outlier would not violate causal consistency. We ask them to select one cluster with whose topic the outlier is most semantically similar. The rules to determine semantic similarity of a sentence pair follows from Step 1. It is possible that there is no topic that matches the outlier. If there is any topic that is a word-by-word exact match, that topic must be selected. We also add another rule that requires the annotators to select the topic with the same level of abstraction (generality) or concreteness (specificity) as the outlier event, since there are some topics that are abstract or concrete versions of other topics. More specifically, if the outlier is concrete but the concrete topic is not presented for selection, select the abstract topic. If the outlier is abstract but the abstract topic is not presented for selection, the concrete topic must \underline{not} be selected. 

\subsection{Causal Relations Discovery}\label{sup:annotation_ph2}
The annotator is tasked with evaluating candidate pairs of clusters to determine whether a cause-and-effect relationship exists between them, based on their respective topics. Since each cluster's topic represents an event abstraction, and in essence, an event itself, the decision on causal relation hinges on whether the two topics describe causally linked events. Based on the cause-effect definition in Section \ref{sec:setup}, we provide them with the following criteria to guide their decision about whether an event $A$ causes another event $B$:
\begin{enumerate}[leftmargin=5.5mm]
    \item a causal relation must be temporal, but a temporal relation is \underline{not} always causal;

    \item the action/state of $A$ directly leads to the action/state of $B$ i.e., there must be no intermediate events or if there is one, it should be extremely rare in real-world scenarios;
    
    \item an event $B$ would not occur if $A$ did not occur.
\end{enumerate}

Initially, the workers provide non-contextual annotations based solely on their commonsense understanding of the abstractions. A relation is deemed valid if the annotator can envision a plausible scenario in daily life where the situation occurs frequently, commonly, and is highly likely. If no such scenario comes to mind, the clusters are considered unrelated. In the subsequent step, we identify the highly disagreed pairs, where the three annotators each make distinct decisions regarding causality i.e., $A$ causes $B$, $B$ causes $A$, $A$ and $B$ are unrelated. For these pairs, workers are presented with contextual information from stories in \texttt{GLUCOSE} and asked to reconsider their decisions. The final determination of the relationship is made through majority voting.




\section{Clustering Algorithm}\label{sup:clustering}
Our clustering algorithm, named \texttt{PIVOT}, is inspired by the pivoting algorithm proposed in \citet{fukunaga2019lp}. The \texttt{PIVOT} algorithm first randomly selects a pair of cause-effect events and then, for each of them, find its most similar neighbors against a threshold of $70\%$. We repeat the process for the remaining event mentions, while excluding the previously assigned events. The initial results are passed to the following process to remove self-loop and bi-directions. We remove clusters with fewer than $10$ samples and maximum pairwise similarity is less than $50\%$. Each cluster can now be considered a node in a graph and we use \texttt{GLUCOSE} to recover the causal relations among them to construct a temporary causal graph. 

\paragraph{Ablation study.} The main motivation behind \texttt{PIVOT} algorithm is to ensure the initial graph is mostly acyclic while avoiding any sub-optimality produced from post-processing. To validate whether \texttt{PIVOT} is most effective in ensuring causal consistency, we conduct an ablation study against popular clustering algorithms, including \texttt{OPTICS} \cite{ankerst1999optics}, \texttt{LOUVAIN} \cite{blondel2008fast} and \texttt{LEIDEN} \cite{traag2019louvain} algorithms, where \texttt{LOUVAIN} and \texttt{LEIDEN} were proposed for community detection problems. The criteria for selecting these clustering algorithms include: 
(1) scalability to medium-to-large-sized data, (2) ability to accommodate custom affinity matrix and (3) high cluster homogeneity score. Table \ref{tab:clustering} further reports the quality of the algorithms under analysis, which shows that our \texttt{PIVOT} algorithm yields the most desirable performance. 

 
\paragraph{Notations.} We use lower case letters (i.e., $v$) to denote single event, capital letters (i.e., $V$) for cluster of events, and blackboard bold letter (i.e., $\mathbb{V}$) for set of clusters. We let $\mathcal{D}$ denote the dataset of causal event mentions; $x \rightarrow y$ indicates event $x$ is a cause of event $y$; $x \leftarrow y$ indicates event $x$ is an effect of event $y$; $x \leftrightarrow y$ indicates $x$ and $y$ are causally related (either cause or effect). We also define the similarity between an event $y$ and cluster $V$ as the average of similarity between $y$ and every event $x$ in $V$
$$S_{yV} = \frac{1}{|V|} \sum_{x \in V} S_{xy},$$
where $S_{xy}$ is the similarity score between two events according to Eq. (\ref{eq:sim}).


\paragraph{Performance metrics.} In the following, we describe the unsupervised performance metrics to assess clustering algorithms in Table \ref{tab:clustering}. Given a set of clusters $\mathbb{C}$, let $\boldsymbol{A}$ be the matrix where $\boldsymbol{A}_{ij}$ is the number of events in cluster $C_i \in \mathbb{C} $ is the cause of any event in the cluster $C_j \in \mathbb{C}$. Recall that in this stage the causal relations between events are extracted from \texttt{GLUCOSE} dataset. A cluster $A$ is said to cause another cluster $B$ if \underline{at least one} event mentions in cluster $A$ causes any other event mentions in cluster $B$, according to the cause-effect definition in Section \ref{sec:setup}. 


\begin{enumerate}
    \item \textit{Self-loop ratio}: Proportion of clusters in which the events are either cause or effect of each other. 
    $$\frac{1}{|\mathbb{C}|} \sum_{i=1}^{|\mathbb{C}|} \frac{\boldsymbol{A}_{ii}}{2 |C_i|}.$$

    \item \textit{Bi-directional ratio:} Proportion of cluster pairs that are both cause and effect of one another.  
    $$\frac{2}{|\mathbb{C}|^2 - |\mathbb{C}|} \sum_{i=1}^{|\mathbb{C}|-1} \sum_{j=i+1}^{|\mathbb{C}|} \frac{\min (\boldsymbol{A}_{ij}, \boldsymbol{A}_{ji})}{\max (\boldsymbol{A}_{ij}, \boldsymbol{A}_{ji})}.$$

    \item \textit{Silhouette coefficient \citep{rousseeuw1987silhouettes}:}  Measure of how similar an instance is to its own cluster (cohesion) compared to other clusters (separation). A high value indicates that the object is well matched to its own cluster and poorly matched to neighboring clusters.  
    $$\frac{1}{|\mathcal{D}|} \sum_{x \in \mathcal{D}} \frac{a_x - b_x}{1 - \min(a_x,b_x)},$$
    
    where $a_x$ is the mean similarity between event $x$ and all other events in the same cluster; $b_x$ is the mean similarity between event $x$ and all other events in the next nearest cluster.

    \item \textit{Homogeneity score:} Average pairwise similarity of events in a cluster.
    $$\frac{1}{|\mathbb{C}|} \sum_{i=1}^{|\mathbb{C}|} \frac{2}{|C_i|^2-|C_i|}\sum_{x,y \in C_i, x \ne y}S_{xy},$$
    
    where $S_{xy}$ is the similarity score between two events according to Eq. (\ref{eq:sim}).
\end{enumerate}

Table \ref{tab:clustering_abs} reports the numerical results for the experiment on Abstract Event Identification in Section \ref{sec:abstraction_exp}. For the supervised metrics, we refer readers to \href{https://scikit-learn.org/stable/modules/clustering.html}{\texttt{scikit-learn}'s} documentation for the precise formulations and implementations of \textit{Adjusted Rand Index} \citep{steinley2004properties} and \textit{Normalized Mutual Information} \citep{vinh2009information}. 

\begin{table*}[hbt!]
\centering
% \resizebox{\columnwidth}{!}{
\begin{tabular}{l|r r r r}
\toprule
Metrics       & \textbf{LOUVAIN} & \textbf{LEIDEN} & \textbf{OPTICS} & \textbf{PIVOT}   \\
\midrule
Bi-directional ratio $\downarrow$               & $0.179$    & $0.162$   & $0.011$   & $\mathbf{0.004}$  \\
Self-loop ratio    $\downarrow$                & $0.252$    & $0.361$   & $0.007$   & $\mathbf{0.001}$  \\
Silhouette coefficient (Euclidean) $\uparrow$ & $-0.120$   & $-0.137$  & $-0.252$  & $\mathbf{-0.015}$ \\
Silhouette coefficient (Cosine) $\uparrow$    & $-0.234$   & $-0.262$  & $-0.392$  & $\mathbf{-0.036}$ \\
Homogeneity score $\uparrow$  & $0.506$  & $0.577$  & $0.810$   & $\mathbf{0.907}$ \\ 
\bottomrule
\end{tabular}
% }
\caption{Evaluation of alternative clustering algorithms. \textbf{Bold} indicates best performance. $\uparrow$ Higher is better. $\downarrow$ Lower is better.} \label{tab:clustering}
\end{table*}


\begin{table*}[hbt!]
    \centering
   % \resizebox{\columnwidth}{!}{
\begin{tabular}{l|r r r r}
\toprule
Metrics       & \textbf{LOUVAIN} & \textbf{LEIDEN} & \textbf{OPTICS} & \textbf{PIVOT}$^{(*)}$   \\
\hline
\multicolumn{5}{c}{\cellcolor[HTML]{C0C0C0}\textbf{Generalizations from \texttt{GPT-4o-mini}}} \\ \toprule
% Silhouette coefficient (Euclidean) $\uparrow$               & $-0.0139$    & $0.0005$   & $-0.1321$   & $\mathbf{0.0173}$          \\
% Silhouette coefficient (Cosine)   $\uparrow$            & $-0.0531$    & $-0.0255$   & $-0.2168$   & $\mathbf{0.0137}$  \\
Adjusted rand index $\uparrow$               & $0.016$    & $0.018$   & $0.001$   & $\mathbf{0.168}$  \\
Normalized mutual information $\uparrow$               & $0.450$    & $0.463$   & $0.384$   & $\mathbf{0.784}$  \\

\hline
\multicolumn{5}{c}{\cellcolor[HTML]{C0C0C0}\textbf{Generalizations from \texttt{GLUCOSE}}} \\ \toprule
% Silhouette coefficient (Euclidean) $\uparrow$               & $0.1315$    & $\mathbf{0.1331}$   & $0.0430$   & $0.0946$          \\
% Silhouette coefficient (Cosine)   $\uparrow$            & $0.1946$    & $\mathbf{0.1976}$   & $0.0691$   & $0.1302$  \\
Adjusted rand index $\uparrow$               & $0.042$    & $0.045$   & $0.011$   & $\mathbf{0.347}$  \\
Normalized mutual information $\uparrow$               & $0.635$    & $0.639$   & $0.699$   & $\mathbf{0.869}$  \\

\bottomrule
\end{tabular}
% }
    \caption{Experimental results of using automatic clustering for identifying abstractions using generalizations by \texttt{ChatGPT} and human-annotated generalizations from \texttt{GLUCOSE}.  (*) In this experiment, we use the original implementation of the \texttt{PIVOT} algorithm in \citet{fukunaga2019lp}. \textbf{Bold} indicates best performance.}
    \label{tab:clustering_abs}
\end{table*}

% Note that the Silhouette coefficients are different from Table \ref{tab:clustering} since we do not consider causal relations during clustering, which are assumed unknown.



% \section{Causal Network Visualization Tool}
% 
% \subsection{Causal Relationship Network Visualization Tool}

% \begin{figure*}[t!]
%     \centering
%     \efbox{\includegraphics[width=\linewidth]{figures/causal_network_visualization.png}}
% \caption{\textit{Causal Network Visualization} tool interface showcasing a complex network of causal relationships among event abstractions.}
% \label{fig:causal_network_visualization}
% \end{figure*}

The \textit{Causal Network Visualization} tool is an explorative attempt at enhancing the comprehension of complex causal relationships within textual data. Built on React\footnote{\url{reactjs.org}}, TypeScript\footnote{\url{typescriptlang.org}}, Sigma.js\footnote{\url{sigmajs.org}}, and leveraging state-of-the-art NLP techniques, this web-based application facilitates the clustering of sentences and the identification of causal relationships between them. The layout of the network utilizes \texttt{nx.spring\_layout}, a force-directed layout algorithm provided by networkX\footnote{\url{networkx.org}}, to arrange the nodes in a manner that reflects the complexity of the causal relationships. Key features of the tool include categorization of nodes into "keys" and "categories" for enhanced navigation, node size differentiation based on between-ness centrality to signify importance, and robust search and filtering options that improve usability. Additionally, the application supports customization for different datasets and visual styles, offering a versatile platform for academic research in understanding the intricate web of cause-and-effect relationships. Our tool is licensed under the MIT License\footnote{\url{opensource.org/licenses/MIT}.}.

\Vy{Provide like to visualization here}

\section{Statistical Causal Discovery}\label{sup:causal_discovery}

\section{Background}\label{sec:backgrnd}

\subsection{Cold Start Latency and Mitigation Techniques}

Traditional FaaS platforms mitigate cold starts through snapshotting, lightweight virtualization, and warm-state management. Snapshot-based methods like \textbf{REAP} and \textbf{Catalyzer} reduce initialization time by preloading or restoring container states but require significant memory and I/O resources, limiting scalability~\cite{dong_catalyzer_2020, ustiugov_benchmarking_2021}. Lightweight virtualization solutions, such as \textbf{Firecracker} microVMs, achieve fast startup times with strong isolation but depend on robust infrastructure, making them less adaptable to fluctuating workloads~\cite{agache_firecracker_2020}. Warm-state management techniques like \textbf{Faa\$T}~\cite{romero_faa_2021} and \textbf{Kraken}~\cite{vivek_kraken_2021} keep frequently invoked containers ready, balancing readiness and cost efficiency under predictable workloads but incurring overhead when demand is erratic~\cite{romero_faa_2021, vivek_kraken_2021}. While these methods perform well in resource-rich cloud environments, their resource intensity challenges applicability in edge settings.

\subsubsection{Edge FaaS Perspective}

In edge environments, cold start mitigation emphasizes lightweight designs, resource sharing, and hybrid task distribution. Lightweight execution environments like unikernels~\cite{edward_sock_2018} and \textbf{Firecracker}~\cite{agache_firecracker_2020}, as used by \textbf{TinyFaaS}~\cite{pfandzelter_tinyfaas_2020}, minimize resource usage and initialization delays but require careful orchestration to avoid resource contention. Function co-location, demonstrated by \textbf{Photons}~\cite{v_dukic_photons_2020}, reduces redundant initializations by sharing runtime resources among related functions, though this complicates isolation in multi-tenant setups~\cite{v_dukic_photons_2020}. Hybrid offloading frameworks like \textbf{GeoFaaS}~\cite{malekabbasi_geofaas_2024} balance edge-cloud workloads by offloading latency-tolerant tasks to the cloud and reserving edge resources for real-time operations, requiring reliable connectivity and efficient task management. These edge-specific strategies address cold starts effectively but introduce challenges in scalability and orchestration.

\subsection{Predictive Scaling and Caching Techniques}

Efficient resource allocation is vital for maintaining low latency and high availability in serverless platforms. Predictive scaling and caching techniques dynamically provision resources and reduce cold start latency by leveraging workload prediction and state retention.
Traditional FaaS platforms use predictive scaling and caching to optimize resources, employing techniques (OFC, FaasCache) to reduce cold starts. However, these methods rely on centralized orchestration and workload predictability, limiting their effectiveness in dynamic, resource-constrained edge environments.



\subsubsection{Edge FaaS Perspective}

Edge FaaS platforms adapt predictive scaling and caching techniques to constrain resources and heterogeneous environments. \textbf{EDGE-Cache}~\cite{kim_delay-aware_2022} uses traffic profiling to selectively retain high-priority functions, reducing memory overhead while maintaining readiness for frequent requests. Hybrid frameworks like \textbf{GeoFaaS}~\cite{malekabbasi_geofaas_2024} implement distributed caching to balance resources between edge and cloud nodes, enabling low-latency processing for critical tasks while offloading less critical workloads. Machine learning methods, such as clustering-based workload predictors~\cite{gao_machine_2020} and GRU-based models~\cite{guo_applying_2018}, enhance resource provisioning in edge systems by efficiently forecasting workload spikes. These innovations effectively address cold start challenges in edge environments, though their dependency on accurate predictions and robust orchestration poses scalability challenges.

\subsection{Decentralized Orchestration, Function Placement, and Scheduling}

Efficient orchestration in serverless platforms involves workload distribution, resource optimization, and performance assurance. While traditional FaaS platforms rely on centralized control, edge environments require decentralized and adaptive strategies to address unique challenges such as resource constraints and heterogeneous hardware.



\subsubsection{Edge FaaS Perspective}

Edge FaaS platforms adopt decentralized and adaptive orchestration frameworks to meet the demands of resource-constrained environments. Systems like \textbf{Wukong} distribute scheduling across edge nodes, enhancing data locality and scalability while reducing network latency. Lightweight frameworks such as \textbf{OpenWhisk Lite}~\cite{kravchenko_kpavelopenwhisk-light_2024} optimize resource allocation by decentralizing scheduling policies, minimizing cold starts and latency in edge setups~\cite{benjamin_wukong_2020}. Hybrid solutions like \textbf{OpenFaaS}~\cite{noauthor_openfaasfaas_2024} and \textbf{EdgeMatrix}~\cite{shen_edgematrix_2023} combine edge-cloud orchestration to balance resource utilization, retaining latency-sensitive functions at the edge while offloading non-critical workloads to the cloud. While these approaches improve flexibility, they face challenges in maintaining coordination and ensuring consistent performance across distributed nodes.



\begin{figure*}[hbt!]
    \centering
    \includegraphics[width=\linewidth]{figures/CD.pdf}
    \caption{SHD \textbf{(left)} and F1 score \textbf{(right)} of estimated DAGs from statistical structure learning methods. \underline{Lower} SHD is better. \underline{Higher} F1 is better.}
    \label{fig:ssl}
\end{figure*}


\paragraph{Experiments.} We here discuss how \texttt{ACCESS} is used to assess to what extent the statistical structure learning methods is applicable to recover causal relations among event abstractions. As illustrated in Figure \ref{fig:main}, after extracting abstractions, one can build representations for abstract events in the original corpus and apply structure learning on top of such data for full graph discovery. A simple representation is the co-occurrence matrix size $(\# stories \times \# abstractions)$ where each entry takes a binary value indicating whether an abstraction has any of its mentions appearing in a story. This means each abstraction is now considered as a Bernoulli random variable and the task of causal discovery is to recover the underlying SCM where the structural functions are commonly non-convex. 

Due to the limited scalability of existing statistical algorithms, we resort to learning sub-graphs by setting thresholds to select nodes that appear frequently while ensuring that the true graph is acyclic. Specifically, our selected sub-graphs are composed of edges where both nodes are adjacent to at least one other node, and each node corresponds to an abstraction whose occurrences exceed a certain frequency threshold. In our experiment, we set thresholds for document frequency within  $\{25, 30, 35, 40, 45\}$, resulting in sub-graphs with $5, 7, 16, 19, 45$ nodes. The experiments are run on $5$ CPU cores.    


We experiment with popular constraint-based and score-based algorithms. We select those that are scalable and capable of capturing non-linear causal relationships without relying on specific model forms such as additive noise. In this paper, we report the results for the following algorithms: 


\begin{itemize}
    \item \texttt{PC} algorithm \citep{spirtes1991algorithm}: A classic approach based on conditional independence tests, for which we run two kinds of tests: Chi-squared and G-squared. 
    \item \texttt{DAG-GNN} \citep{yu2019dag}: DAG structure learning with graph neural networks.
    \item \texttt{GAE} \citep{ng2019graph}: This method utilizes gradient descent and graph auto-encoders to model non-linear causal relationships.
    \item \texttt{CORL} \citep{wang2021ordering}: A reinforcement learning-based algorithm with flexible score functions with enhanced efficiency.
\end{itemize}

Besides the above methods, we have also tested \texttt{NOTEARS} \citep{zheng2020learning}, a popular score-based algorithm and its more efficient variant \texttt{GOLEM} \citep{ng2020role}. However, they both unfortunately fail to recover any edges across all settings. To ensure consistency in implementation and evaluation, we utilize the standardized framework provided by \href{https://github.com/huawei-noah/trustworthyAI/tree/master/gcastle}{\texttt{gCastle}} \citep{zhang2021gcastle}. 
As for evaluation metrics, we report the structured Hamming distance (SHD), which quantifies the smallest number of edge additions, deletions, and reversals required to transform the recovered DAG into the true one. Additionally, we assess classification accuracy using the F1 score. Ideally, we aim for a lower normalized Hamming distance and a higher F1 score. Figure \ref{fig:ssl} reports the SHD and F1 score of the estimated DAGs from these methods. 

It is seen the methods achieve relatively low accuracy on our benchmark causal graphs, which are sparse. As the SHD scores are much higher than the graph size, these model tend to predict plenty of edges, most of which are incorrect due to the low F1 scores. Scalability remains a serious challenge to statistical structure learning. As the graph scales up to $45$ nodes, their performance further deteriorates significantly, where most of them of them even fail to recover any edges. It is noteworthy that the representation power of the input data also affects the causal discovery performance. It is very likely that the co-occurrence matrix is not sufficiently expressive to capture the causal knowledge. This motivates a dedicated line of research into abstract causal representation learning. 


\section{GLUCOSE-QA Reasoning}  \label{sup:reasoning}
We here provide the prompts for LLMs in Tables \ref{tab:prompt_causal_discovery}-\ref{tab:prompt_abs}. Tables \ref{tab:examples_specific_qa}-\ref{tab:examples_cot_step} present illustrative examples of the responses from LLMs across our QA tasks. 





\definecolor{titlecolor}{rgb}{0.9, 0.5, 0.1}
\definecolor{anscolor}{rgb}{0.2, 0.5, 0.8}
\definecolor{labelcolor}{HTML}{48a07e}
\begin{table*}[h]
	\centering
	
 % \vspace{-0.2cm}
	
	\begin{center}
		\begin{tikzpicture}[
				chatbox_inner/.style={rectangle, rounded corners, opacity=0, text opacity=1, font=\sffamily\scriptsize, text width=5in, text height=9pt, inner xsep=6pt, inner ysep=6pt},
				chatbox_prompt_inner/.style={chatbox_inner, align=flush left, xshift=0pt, text height=11pt},
				chatbox_user_inner/.style={chatbox_inner, align=flush left, xshift=0pt},
				chatbox_gpt_inner/.style={chatbox_inner, align=flush left, xshift=0pt},
				chatbox/.style={chatbox_inner, draw=black!25, fill=gray!7, opacity=1, text opacity=0},
				chatbox_prompt/.style={chatbox, align=flush left, fill=gray!1.5, draw=black!30, text height=10pt},
				chatbox_user/.style={chatbox, align=flush left},
				chatbox_gpt/.style={chatbox, align=flush left},
				chatbox2/.style={chatbox_gpt, fill=green!25},
				chatbox3/.style={chatbox_gpt, fill=red!20, draw=black!20},
				chatbox4/.style={chatbox_gpt, fill=yellow!30},
				labelbox/.style={rectangle, rounded corners, draw=black!50, font=\sffamily\scriptsize\bfseries, fill=gray!5, inner sep=3pt},
			]
											
			\node[chatbox_user] (q1) {
				\textbf{System prompt}
				\newline
				\newline
				You are a helpful and precise assistant for segmenting and labeling sentences. We would like to request your help on curating a dataset for entity-level hallucination detection.
				\newline \newline
                We will give you a machine generated biography and a list of checked facts about the biography. Each fact consists of a sentence and a label (True/False). Please do the following process. First, breaking down the biography into words. Second, by referring to the provided list of facts, merging some broken down words in the previous step to form meaningful entities. For example, ``strategic thinking'' should be one entity instead of two. Third, according to the labels in the list of facts, labeling each entity as True or False. Specifically, for facts that share a similar sentence structure (\eg, \textit{``He was born on Mach 9, 1941.''} (\texttt{True}) and \textit{``He was born in Ramos Mejia.''} (\texttt{False})), please first assign labels to entities that differ across atomic facts. For example, first labeling ``Mach 9, 1941'' (\texttt{True}) and ``Ramos Mejia'' (\texttt{False}) in the above case. For those entities that are the same across atomic facts (\eg, ``was born'') or are neutral (\eg, ``he,'' ``in,'' and ``on''), please label them as \texttt{True}. For the cases that there is no atomic fact that shares the same sentence structure, please identify the most informative entities in the sentence and label them with the same label as the atomic fact while treating the rest of the entities as \texttt{True}. In the end, output the entities and labels in the following format:
                \begin{itemize}[nosep]
                    \item Entity 1 (Label 1)
                    \item Entity 2 (Label 2)
                    \item ...
                    \item Entity N (Label N)
                \end{itemize}
                % \newline \newline
                Here are two examples:
                \newline\newline
                \textbf{[Example 1]}
                \newline
                [The start of the biography]
                \newline
                \textcolor{titlecolor}{Marianne McAndrew is an American actress and singer, born on November 21, 1942, in Cleveland, Ohio. She began her acting career in the late 1960s, appearing in various television shows and films.}
                \newline
                [The end of the biography]
                \newline \newline
                [The start of the list of checked facts]
                \newline
                \textcolor{anscolor}{[Marianne McAndrew is an American. (False); Marianne McAndrew is an actress. (True); Marianne McAndrew is a singer. (False); Marianne McAndrew was born on November 21, 1942. (False); Marianne McAndrew was born in Cleveland, Ohio. (False); She began her acting career in the late 1960s. (True); She has appeared in various television shows. (True); She has appeared in various films. (True)]}
                \newline
                [The end of the list of checked facts]
                \newline \newline
                [The start of the ideal output]
                \newline
                \textcolor{labelcolor}{[Marianne McAndrew (True); is (True); an (True); American (False); actress (True); and (True); singer (False); , (True); born (True); on (True); November 21, 1942 (False); , (True); in (True); Cleveland, Ohio (False); . (True); She (True); began (True); her (True); acting career (True); in (True); the late 1960s (True); , (True); appearing (True); in (True); various (True); television shows (True); and (True); films (True); . (True)]}
                \newline
                [The end of the ideal output]
				\newline \newline
                \textbf{[Example 2]}
                \newline
                [The start of the biography]
                \newline
                \textcolor{titlecolor}{Doug Sheehan is an American actor who was born on April 27, 1949, in Santa Monica, California. He is best known for his roles in soap operas, including his portrayal of Joe Kelly on ``General Hospital'' and Ben Gibson on ``Knots Landing.''}
                \newline
                [The end of the biography]
                \newline \newline
                [The start of the list of checked facts]
                \newline
                \textcolor{anscolor}{[Doug Sheehan is an American. (True); Doug Sheehan is an actor. (True); Doug Sheehan was born on April 27, 1949. (True); Doug Sheehan was born in Santa Monica, California. (False); He is best known for his roles in soap operas. (True); He portrayed Joe Kelly. (True); Joe Kelly was in General Hospital. (True); General Hospital is a soap opera. (True); He portrayed Ben Gibson. (True); Ben Gibson was in Knots Landing. (True); Knots Landing is a soap opera. (True)]}
                \newline
                [The end of the list of checked facts]
                \newline \newline
                [The start of the ideal output]
                \newline
                \textcolor{labelcolor}{[Doug Sheehan (True); is (True); an (True); American (True); actor (True); who (True); was born (True); on (True); April 27, 1949 (True); in (True); Santa Monica, California (False); . (True); He (True); is (True); best known (True); for (True); his roles in soap operas (True); , (True); including (True); in (True); his portrayal (True); of (True); Joe Kelly (True); on (True); ``General Hospital'' (True); and (True); Ben Gibson (True); on (True); ``Knots Landing.'' (True)]}
                \newline
                [The end of the ideal output]
				\newline \newline
				\textbf{User prompt}
				\newline
				\newline
				[The start of the biography]
				\newline
				\textcolor{magenta}{\texttt{\{BIOGRAPHY\}}}
				\newline
				[The ebd of the biography]
				\newline \newline
				[The start of the list of checked facts]
				\newline
				\textcolor{magenta}{\texttt{\{LIST OF CHECKED FACTS\}}}
				\newline
				[The end of the list of checked facts]
			};
			\node[chatbox_user_inner] (q1_text) at (q1) {
				\textbf{System prompt}
				\newline
				\newline
				You are a helpful and precise assistant for segmenting and labeling sentences. We would like to request your help on curating a dataset for entity-level hallucination detection.
				\newline \newline
                We will give you a machine generated biography and a list of checked facts about the biography. Each fact consists of a sentence and a label (True/False). Please do the following process. First, breaking down the biography into words. Second, by referring to the provided list of facts, merging some broken down words in the previous step to form meaningful entities. For example, ``strategic thinking'' should be one entity instead of two. Third, according to the labels in the list of facts, labeling each entity as True or False. Specifically, for facts that share a similar sentence structure (\eg, \textit{``He was born on Mach 9, 1941.''} (\texttt{True}) and \textit{``He was born in Ramos Mejia.''} (\texttt{False})), please first assign labels to entities that differ across atomic facts. For example, first labeling ``Mach 9, 1941'' (\texttt{True}) and ``Ramos Mejia'' (\texttt{False}) in the above case. For those entities that are the same across atomic facts (\eg, ``was born'') or are neutral (\eg, ``he,'' ``in,'' and ``on''), please label them as \texttt{True}. For the cases that there is no atomic fact that shares the same sentence structure, please identify the most informative entities in the sentence and label them with the same label as the atomic fact while treating the rest of the entities as \texttt{True}. In the end, output the entities and labels in the following format:
                \begin{itemize}[nosep]
                    \item Entity 1 (Label 1)
                    \item Entity 2 (Label 2)
                    \item ...
                    \item Entity N (Label N)
                \end{itemize}
                % \newline \newline
                Here are two examples:
                \newline\newline
                \textbf{[Example 1]}
                \newline
                [The start of the biography]
                \newline
                \textcolor{titlecolor}{Marianne McAndrew is an American actress and singer, born on November 21, 1942, in Cleveland, Ohio. She began her acting career in the late 1960s, appearing in various television shows and films.}
                \newline
                [The end of the biography]
                \newline \newline
                [The start of the list of checked facts]
                \newline
                \textcolor{anscolor}{[Marianne McAndrew is an American. (False); Marianne McAndrew is an actress. (True); Marianne McAndrew is a singer. (False); Marianne McAndrew was born on November 21, 1942. (False); Marianne McAndrew was born in Cleveland, Ohio. (False); She began her acting career in the late 1960s. (True); She has appeared in various television shows. (True); She has appeared in various films. (True)]}
                \newline
                [The end of the list of checked facts]
                \newline \newline
                [The start of the ideal output]
                \newline
                \textcolor{labelcolor}{[Marianne McAndrew (True); is (True); an (True); American (False); actress (True); and (True); singer (False); , (True); born (True); on (True); November 21, 1942 (False); , (True); in (True); Cleveland, Ohio (False); . (True); She (True); began (True); her (True); acting career (True); in (True); the late 1960s (True); , (True); appearing (True); in (True); various (True); television shows (True); and (True); films (True); . (True)]}
                \newline
                [The end of the ideal output]
				\newline \newline
                \textbf{[Example 2]}
                \newline
                [The start of the biography]
                \newline
                \textcolor{titlecolor}{Doug Sheehan is an American actor who was born on April 27, 1949, in Santa Monica, California. He is best known for his roles in soap operas, including his portrayal of Joe Kelly on ``General Hospital'' and Ben Gibson on ``Knots Landing.''}
                \newline
                [The end of the biography]
                \newline \newline
                [The start of the list of checked facts]
                \newline
                \textcolor{anscolor}{[Doug Sheehan is an American. (True); Doug Sheehan is an actor. (True); Doug Sheehan was born on April 27, 1949. (True); Doug Sheehan was born in Santa Monica, California. (False); He is best known for his roles in soap operas. (True); He portrayed Joe Kelly. (True); Joe Kelly was in General Hospital. (True); General Hospital is a soap opera. (True); He portrayed Ben Gibson. (True); Ben Gibson was in Knots Landing. (True); Knots Landing is a soap opera. (True)]}
                \newline
                [The end of the list of checked facts]
                \newline \newline
                [The start of the ideal output]
                \newline
                \textcolor{labelcolor}{[Doug Sheehan (True); is (True); an (True); American (True); actor (True); who (True); was born (True); on (True); April 27, 1949 (True); in (True); Santa Monica, California (False); . (True); He (True); is (True); best known (True); for (True); his roles in soap operas (True); , (True); including (True); in (True); his portrayal (True); of (True); Joe Kelly (True); on (True); ``General Hospital'' (True); and (True); Ben Gibson (True); on (True); ``Knots Landing.'' (True)]}
                \newline
                [The end of the ideal output]
				\newline \newline
				\textbf{User prompt}
				\newline
				\newline
				[The start of the biography]
				\newline
				\textcolor{magenta}{\texttt{\{BIOGRAPHY\}}}
				\newline
				[The ebd of the biography]
				\newline \newline
				[The start of the list of checked facts]
				\newline
				\textcolor{magenta}{\texttt{\{LIST OF CHECKED FACTS\}}}
				\newline
				[The end of the list of checked facts]
			};
		\end{tikzpicture}
        \caption{GPT-4o prompt for labeling hallucinated entities.}\label{tb:gpt-4-prompt}
	\end{center}
\vspace{-0cm}
\end{table*}

\begin{table*}[hbt!]
\centering
\resizebox{\linewidth}{!}{%
\begin{tabular}{l|p{12cm}}
\toprule
Story & In a store, two women were arguing, and Howard wanted to intervene. He attempted to get them to stop talking, but it didn't work. So, he stepped in between them, which caused them to cease their fighting. \\
\midrule
Specific Question & What could be the cause of the event \textit{`howard wants to help the women'}? \\
\midrule
Abstract Question & The question describes an event where \textit{`a person hears something in a place'}. What could be the effect of the event?\\
\midrule
Choices & \begin{tabular}[c]{@{}l@{}}0: "Two women fights each other.",\\ 1: "He went in between them.",\\ 2: "Two women were fighting in a store.",\\ 3: "They stopped.",\\ 4: "Howard wanted to help."\\ 5: "He tried telling them to stop but it did not work."\end{tabular} \\ \midrule
Causal Graph (CG) & \textit{a person have a fight with another person} $\rightarrow$ \textit{a person want to stop another person} \\ \midrule
Correct Answers & 0, 2 \\ \midrule
\texttt{GPT-4o-mini} Answers & 2, 4 \\ \midrule
\texttt{GPT-4o-mini} Answers w/ CG & 0, 2 \\ \midrule
\texttt{Llama3.1-8B} Answers & 0, 1 \\ \midrule
\texttt{Llama3.1-8B} Answers w/ CG & 0, 2, 4 \\ 
\bottomrule
\end{tabular}%
}
\caption{Examples of multi-choice Specific-QA reasoning in \texttt{GPT-4o-mini} and \texttt{Llama3.1-8B}.}
\label{tab:examples_specific_qa}
\end{table*}


\begin{table*}[hbt!]
\centering
\resizebox{\linewidth}{!}{%
\begin{tabular}{l|p{12cm}}
\toprule
Story & His cousins were scheduled to visit later that day, so his mom had him clean in the morning, shop for groceries in the afternoon, and get ready in the evening. Eventually, his cousins arrived at his house. \\
\midrule
Abstract Question & The question describes an event where \textit{`a person are coming to a place (that is another person house)'}. What could be the effect of the event?\\
\midrule
Choices & \begin{tabular}[c]{@{}l@{}}0: "His cousins were coming later too his house.",\\ 1: "He get groceries in the afternoon.",\\ 2: "His mom made him clean all morning.",\\ 3: "His cousins came to his house.",\\ 4: "He get ready in the evening."\end{tabular} \\ \midrule
Causal Graph (CG) & \textit{a person come to another person 's place} $\rightarrow$ \textit{a person clean something} \\ \midrule
Correct Answers & 1, 2, 4 \\ \midrule
\texttt{GPT-4o-mini} Answers & 0, 3 \\ \midrule
\texttt{GPT-4o-mini} Answers w/ CG & 0, 2 \\ \midrule
\texttt{Llama3.2-3B} Answers & 0, 3 \\ \midrule
\texttt{Llama3.2-3B} Answers w/ CG & 1, 3 \\ 
\bottomrule
\end{tabular}%
}
\caption{Examples of multi-choice Abstract-QA reasoning in \texttt{GPT-4o-mini} and \texttt{Llama3.2-3B}.}
\label{tab:examples_abstract_qa_1}
\end{table*}

\begin{table*}[hbt!]
\centering
\resizebox{\linewidth}{!}{%
\begin{tabular}{l|p{12cm}}
\toprule
Story & Felix wanted to visit Disney World. One day, he won two tickets and invited his friend Alissa. However, Alissa disliked Disney, so Felix ended up going by himself. \\
\midrule
Abstract Question & The question describes an event where \textit{`a person invited another person'}. What could be the cause of the event?\\
\midrule
Choices & \begin{tabular}[c]{@{}l@{}}0: "Alissa hated disney.",\\ 1: "Felix wanted to go to disney world.",\\ 2: "One day he won two tickets for entry.",\\ 3: "He invited his friend Alissa.",\\ 4: "He ended up going alone."\end{tabular} \\ \midrule
Causal Graph (CG) & \textit{a person want to go to a place} $\rightarrow$ \textit{a person give another person an invitation to a place} \\ \midrule
Correct Answers & 1, 2 \\ \midrule
\texttt{GPT-4o-mini} Answers & 0, 1, 3 \\ \midrule
\texttt{GPT-4o-mini} Answers w/ CG & 1, 2 \\ \midrule
\texttt{Llama2-7B} Answers & 1, 2 \\ \midrule
\texttt{Llama2-7B} Answers w/ CG & 1, 2 \\ 
\bottomrule
\end{tabular}%
}
\caption{Examples of multi-choice Abstract-QA reasoning in \texttt{GPT-4o-mini} and \texttt{Llama2-7B}.}
\label{tab:examples_abstract_qa_2}
\end{table*}
\begin{table*}[hbt!]
\centering
\resizebox{\linewidth}{!}{%
\begin{tabular}{l|p{12cm}}
\toprule
Story & He wanted toast, so he got some bread and put it in the toaster. When it popped out and landed on the floor, he ate it anyway. \\
\midrule
Abstract Question & The question describes an event where \textit{`a person got another thing (that is an ingredient in another thing'}. What could be the cause of the event?\\
\midrule
Choices & \begin{tabular}[c]{@{}l@{}}0: "He ate it anyway.",\\ 1: "He put it in the toaster.",\\ 2: "He got some bread.",\\ 3: "It shot out of the toaster and onto the floor.",\\ 4: "He was making toast."\end{tabular} \\ \midrule
% Causal Graph (CG) & \textit{a person make food} $\rightarrow$ \textit{a person get ingredient} \\ \midrule
Correct Answers & 4 \\ 
\midrule
\texttt{Llama3.2-3B} Answers (zero-shot) & 1, 2 \\ 
\texttt{Llama3.2-3B} Answers & 1, 4 \\ 
\texttt{Llama3.2-3B} Answers + CG & 1, 4 \\ 
\midrule
\texttt{Llama3.1-8B} Answers(zero-shot) & 1, 3 \\ 
\texttt{Llama3.1-8B} Answers  & 2 \\ 
\texttt{Llama3.1-8B} Answers + CG & 4 \\ 
\midrule
\texttt{Llama2-7B} Answers (zero-shot) & 1, 2 \\ 
\texttt{Llama2-7B} Answers & 4 \\ 
\texttt{Llama2-7B} Answers + CG & 1, 4 \\ 
\bottomrule
\end{tabular}%
}
\caption{Examples of multi-choice Abstract-QA reasoning with bi-level COT.}
\label{tab:examples_cot}
\end{table*}

\begin{table*}[hbt!]
\centering
\resizebox{\linewidth}{!}{%
\begin{tabular}{l|p{12cm}}
\toprule
Story & He wanted toast, so he got some bread and put it in the toaster. When it popped out and landed on the floor, he ate it anyway. \\
\midrule
Abstract Question & The question describes an event where \textit{`a person got another thing (that is an ingredient in another thing'}. What could be the cause of the event?\\
\midrule
Choices & \begin{tabular}[c]{@{}l@{}}0: "He ate it anyway.",\\ 1: "He put it in the toaster.",\\ 2: "He got some bread.",\\ 3: "It shot out of the toaster and onto the floor.",\\ 4: "He was making toast."\end{tabular} \\ \midrule
Prompt & The event \textit{`a person got another thing (that is an ingredient in another thing'} is described by one of the sentences in the story context. First identify that part
of the story. Then retrieve the event mentioned in the story that is a
corresponding cause/effect.\\ 
\midrule
% Causal Graph (CG) & \textit{a person make food} $\rightarrow$ \textit{a person get ingredient} \\ \midrule
Correct Answers & 4 \\ 
\midrule
Rationale & To solve this question, let's break down the steps: \\
& 1. Identify the sentence describing `another thing which is an ingredient in another thing': - The relevant sentence from the given text is not explicitly stated but rather implied through analysis. In this case, what seems like `he made breakfast food'. In other words - Toaster = Breakfast Food; Bread + Toast = Ingredients for the meal at hand.  Therefore the first item on our consideration has been identified as number four because if you are going into kitchen then most likely, your objective would be about cooking something.\\ 
& 2. Now find any events occurring after items were used during said activity using index numbers provided against them starting their respective sequence no. This means looking back towards end, start at \#5 because there will definitely exist only those indexed things found before each index respectively being associated w/ this particular action taking place afterward within same timeframe i.e., related actions happening after certain things are prepared here. Considering these points: We look forward now...4 provides our initial object `the ingredients'. This makes us say - That must come after getting things ready prior than eating. Therefore we also know - getting stuff does indeed precede putting things inside the device whose result gets eaten later.\\ 
& So while option zero fits too since all meals need ending act (eating) coming second to doing preparatory tasks involving having desired objects (`bread').. But finally- lets verify... As shown below, the correct answer(s): 4, 1\\ 
\bottomrule
\end{tabular}%
}
\caption{Example of multi-choice Abstract-QA reasoning with bi-level COT in \texttt{Llama3.2-3B}.}
\label{tab:examples_cot_step}
\end{table*}









\end{document}


