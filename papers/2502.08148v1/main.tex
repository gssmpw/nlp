\documentclass[11pt]{article}
\usepackage{latex/acl}
\usepackage{times}
\usepackage{latexsym}
\usepackage[T1]{fontenc}
\usepackage[utf8]{inputenc}
\usepackage{microtype}
\usepackage{inconsolata}
\usepackage{algorithm,algpseudocode}
\renewcommand{\algorithmicrequire}{\textbf{Input:}}
\renewcommand{\algorithmicensure}{\textbf{Output:}}
\usepackage{amsmath}
\usepackage{amsfonts}
\usepackage{color,soul}
\usepackage{enumitem}
\usepackage{graphicx}
\usepackage{colortbl}
\usepackage{xcolor}
\usepackage{subfigure}
\usepackage{booktabs}
\usepackage[normalem]{ulem}
\useunder{\uline}{\ul}{}
\usepackage{efbox,graphicx}
\efboxsetup{linecolor=black,linewidth=2pt}

%%%%% NEW MATH DEFINITIONS %%%%%

% \usepackage{amsmath,amsfonts,bm}
\usepackage{amsmath,amsfonts}

\usepackage{pifont}


\newcommand{\R}{\mathbb{R}}


\def\va{{\mathbf{a}}}
\def\vg{{\mathbf{g}}}

% Sets
\def\sR{\mathbb{R}}
\def\sC{\mathbb{C}}
\def\sZ{\mathbb{Z}}
\def\sN{\mathbb{N}}
\def\sQ{\mathbb{Q}}

\def\sS{\mathcal{S}}



% Vectors
\def\vzero{{\mathbf{0}}}
\def\vone{{\mathbf{1}}}
\def\vmu{{\mathbf{\mu}}}
\def\vtheta{{\mathbf{\theta}}}
\def\va{{\mathbf{a}}}
\def\vb{{\mathbf{b}}}
\def\vc{{\mathbf{c}}}
\def\vd{{\mathbf{d}}}
\def\ve{{\mathbf{e}}}
\def\vf{{\mathbf{f}}}
\def\vg{{\mathbf{g}}}
\def\vh{{\mathbf{h}}}
\def\vi{{\mathbf{i}}}
\def\vj{{\mathbf{j}}}
\def\vk{{\mathbf{k}}}
\def\vl{{\mathbf{l}}}
\def\vm{{\mathbf{m}}}
\def\vn{{\mathbf{n}}}
\def\vo{{\mathbf{o}}}
\def\vp{{\mathbf{p}}}
\def\vq{{\mathbf{q}}}
\def\vr{{\mathbf{r}}}
\def\vs{{\mathbf{s}}}
\def\vt{{\mathbf{t}}}
\def\vu{{\mathbf{u}}}
\def\vv{{\mathbf{v}}}
\def\vw{{\mathbf{w}}}
\def\vx{{\mathbf{x}}}
\def\vy{{\mathbf{y}}}
\def\vz{{\mathbf{z}}}
\def\vzeta{{\mathbf{\zeta}}}

% Matrix
\def\mA{{\mathbf{A}}}
\def\mB{{\mathbf{B}}}
\def\mC{{\mathbf{C}}}
\def\mD{{\mathbf{D}}}
\def\mE{{\mathbf{E}}}
\def\mF{{\mathbf{F}}}
\def\mG{{\mathbf{G}}}
\def\mH{{\mathbf{H}}}
\def\mI{{\mathbf{I}}}
\def\mJ{{\mathbf{J}}}
\def\mK{{\mathbf{K}}}
\def\mL{{\mathbf{L}}}
\def\mM{{\mathbf{M}}}
\def\mN{{\mathbf{N}}}
\def\mO{{\mathbf{O}}}
\def\mP{{\mathbf{P}}}
\def\mQ{{\mathbf{Q}}}
\def\mR{{\mathbf{R}}}
\def\mS{{\mathbf{S}}}
\def\mT{{\mathbf{T}}}
\def\mU{{\mathbf{U}}}
\def\mV{{\mathbf{V}}}
\def\mW{{\mathbf{W}}}
\def\mX{{\mathbf{X}}}
\def\mY{{\mathbf{Y}}}
\def\mZ{{\mathbf{Z}}}
\def\mBeta{{\mathbf{\beta}}}
\def\mPhi{{\mathbf{\Phi}}}
\def\mLambda{{\mathbf{\Lambda}}}
\def\mSigma{{\mathbf{\Sigma}}}


% Expectation
% \def\eE{\mathop{\mathbb{E}}\limits}
\def\eE{\mathbb{E}}

% Probability
\def\pP{\mathbb{P}}

% Tilde
\def\tf{\tilde{f}}
\def\tS{\tilde{S}}
\def\wtF{\widetilde{\mathcal{F}}}
\def\whR{\widehat{R}}
\def\tvx{\tilde{\mathbf{x}}}
\def\ty{\tilde{y}}


\def\defeq{\overset{\textup{def}}{=}}
% \def\defeq{\overset{.}{=}}
\def\defone{\overset{\text{\ding{172}}}{=}}
\def\deftwo{\overset{\text{\ding{173}}}{=}}
\def\leqone{\overset{\text{\ding{172}}}{\leq}}
\def\leqtwo{\overset{\text{\ding{173}}}{\leq}}
\def\leqthree{\overset{\text{\ding{174}}}{\leq}}
\def\leqfour{\overset{\text{\ding{175}}}{\leq}}
\def\eqone{\overset{\text{\ding{172}}}{=}}
\def\eqtwo{\overset{\text{\ding{173}}}{=}}
\def\eqthree{\overset{\text{\ding{174}}}{=}}
\def\eqfour{\overset{\text{\ding{175}}}{=}}
\def\geqfive{\overset{\text{\ding{176}}}{\geq}}
\newcommand{\Vy}[1]{\textcolor{red}{Vy:~#1}}
\newcommand{\lizhen}[1]{{\textcolor{brown}{Lizhen:~#1}}}

\title{ACCESS : A Benchmark for \\ Abstract Causal Event Discovery and Reasoning}


\author{Vy Vo \quad Lizhen Qu \quad Tao Feng \quad Yuncheng Hua \quad Xiaoxi Kang\\
{\bf Songhai Fan \quad Tim Dwyer \quad Lay-Ki Soon \quad Gholamreza Haffari} \\
Monash University, Australia \\ 
\texttt{\{firstname.lastname\}@monash.edu}
}


\begin{document}
\maketitle
\begin{abstract}
  Identifying cause-and-effect relationships is critical to understanding real-world dynamics and ultimately causal reasoning. Existing methods for identifying event causality in NLP, including those based on Large Language Models (LLMs), exhibit difficulties in out-of-distribution settings due to the limited scale and heavy reliance on lexical cues within available benchmarks. Modern benchmarks, inspired by probabilistic causal inference, have attempted to construct causal graphs of events as a robust representation of causal knowledge, where \texttt{CRAB} \citep{romanou2023crab} is one such recent benchmark along this line. In this paper, we introduce \texttt{ACCESS}, a benchmark designed for discovery and reasoning over abstract causal events. Unlike existing resources, \texttt{ACCESS}  focuses on causality of everyday life events on the abstraction level. We propose a pipeline for identifying abstractions for event generalizations from \texttt{GLUCOSE} \citep{mostafazadeh-etal-2020-glucose}, a large-scale dataset of implicit commonsense causal knowledge, from which we subsequently extract $1,4$K causal pairs. Our experiments highlight the ongoing challenges of using statistical methods and/or LLMs for automatic abstraction identification and causal discovery in NLP. Nonetheless, we demonstrate that the abstract causal knowledge provided in \texttt{ACCESS} can be leveraged for enhancing QA reasoning performance in LLMs.
\end{abstract}



% \iftaclpubformat
\section{Introduction}
\section{Introduction}
\label{sec:intro}
% Image editing methods in diffusion models depend on user-defined control directions - users can unlock their creativity using these methods by specifying the desired manipulation through prompts~\cite{gandikota2023concept}, reference images~\cite{ruiz2022dreambooth, kumari2022customdiffusion, gal2022image, chen2024trainingfreeregionalpromptingdiffusion}, or attribute vectors~\cite{parmar2023zero,hertz2022prompt}. In this work, we ask a fundamentally different question: \emph{Can we automatically discover the underlying visual structure of a concept within diffusion model's knowledge?} %Rather than requiring user-specified controls, we aim to decompose the model's internal knowledge into meaningful directions.

% This question touches on a fundamental limitation in how we interact with diffusion models. Current control methods ~\cite{zhang2023addingconditionalcontroltexttoimage, gandikota2023concept, ye2023ipadaptertextcompatibleimage,ye2023ipadaptertextcompatibleimage, hertz2024stylealignedimagegeneration, li2023photomaker, shi2024instantbooth, chen2024trainingfreeregionalpromptingdiffusion} require users to specify their desired manipulations in advance, limiting interactive creativity. This contrasts with natural human artistic workflows, where creators dynamically explore creative ideas while jointly refining them toward meaningful artistic outcomes~\cite{hoffmann2016modeling}. This synergy between specification and exploration is not new to generative models. Early GAN architectures naturally developed disentangled latent spaces that enabled continuous\cite{harkonen2020ganspace,radford2015unsupervised, wu2021stylespace, shen2020interfacegan}, compositional control over generated images. Users could explore these spaces to discover interesting variations that would be difficult to describe in words~\cite{wu2021stylespace}, then combine them to achieve their creative goals~\cite{grabe2022towards}. 


% While diffusion models have largely superseded GANs in conditional image synthesis~\cite{dhariwal2021diffusion},  their underlying structure remains less understood. Diffusion models achieve remarkable diversity through high-dimensional latents, unlike GANs' compact latent spaces.  With a single prompt, diffusion models can generate radically different variations through different random initializations of input noise. We ask - Is it possible to discover interpretable structure within this vast space of variations?

Text-to-image diffusion models are capable of generating remarkable visual variations from a single prompt through different random initializations. However, this vast creative potential remains largely opaque to users---while we can generate diverse images, we lack understanding of the underlying structure of these variations. This presents a fundamental challenge: how can we discover and expose the latent visual capabilities encoded within these models?

\let\thefootnote\relax \footnote{$^{*}$Correspondence to \texttt{gandikota.ro@northeastern.edu}}

The challenge touches on a key limitation in how we interact with diffusion models today. Current control methods require users to explicitly specify their desired edits in advance through prompts~\cite{gandikota2023concept}, reference images~\cite{zhang2023addingconditionalcontroltexttoimage, chen2024trainingfreeregionalpromptingdiffusion, ruiz2022dreambooth,kumari2022customdiffusion, Ryu_lora, hu2021lora}, or attribute vectors~\cite{ye2023ipadaptertextcompatibleimage, hertz2024stylealignedimagegeneration, li2023photomaker, shi2024instantbooth,parmar2023zero,hertz2022prompt}. That contrasts sharply with natural human creative workflows, where artists dynamically explore creative ideas and jointly refine them toward meaningful artistic outcomes~\cite{hoffmann2016modeling}. The need for pre-specified controls creates a barrier between users and the full creative potential of these models.

Interestingly, earlier generative models like GANs~\cite{gans,karras2019style,brock2018large} naturally developed more interpretable internal structures. Their compact latent spaces often exhibited emergent disentanglement~\cite{harkonen2020ganspace,radford2015unsupervised, wu2021stylespace, shen2020interfacegan}, enabling continuous and compositional control over generated images. Users could explore these spaces to discover interesting variations that would be difficult to describe in words~\cite{wu2021stylespace}, then combine them to achieve their creative goals~\cite{grabe2022towards}.

Diffusion models have largely superseded GANs in conditional image synthesis~\cite{dhariwal2021diffusion}, achieving greater diversity through much higher-dimensional latents. And yet an understanding of the underlying structure of these larger latent spaces has remained elusive. In this work, we ask a fundamental question: \emph{Can we automatically discover the visual structure within a diffusion model's knowledge of a concept?} Rather than requiring user-specified controls, we aim to decompose the model's internal representations into expressive directions that users can explore and combine.

To address these needs, we present \textbf{SliderSpace}, a framework that brings systematic explorability to diffusion models. Given just a text prompt, SliderSpace discovers a canonical set of meaningful, diverse, and controllable directions within the model's knowledge of that concept. Each direction is implemented as a low-rank adapter~\cite{hu2021lora} that can be scaled and composed with others, allowing users to explore and smoothly combine different aspects of variation, as shown in Figure~\ref{fig:intro}.

We ground SliderSpace discovery in three key requirements for meaningful decomposition of a diffusion model's visual manifold: 
\begin{enumerate}
    \item \textbf{Unsupervised Discovery:} The decomposition process should emerge from the intrinsic structure of the model's learned representation, rather than being guided by predefined attributes. This ensures we capture the true topology of the model's knowledge space rather than projecting our assumptions onto it.
    
    \item \textbf{Semantic Orthogonality:} Each discovered control must represent a distinct semantic direction. This is enforced in a semantic feature space, like CLIP, where every slider has an orthogonal effect in embeddings. This prevents discovering multiple controls that create similar semantic effects, making the system more efficient and easier.
    
    \item \textbf{Distribution Consistency:} Directions must induce consistent transformations across both random seeds and prompt variations. 
\end{enumerate}

These requirements naturally lead to our proposed framework, which we formalize in Section~\ref{sec:method}. As we show in our experiments, SliderSpace is architecture-agnostic, working with both conventional U-Net based models like Stable Diffusion~\cite{rombach2022high, rombach2022sd20, podell2023sdxl, turbo, dmd} and recent transformer-based architectures like Flux~\cite{flux}.

We demonstrate the expressiveness of SliderSpace through three applications: First, we show how SliderSpace can decompose high-level concepts into diverse and expressive components, revealing the natural axes of variation in the model's understanding. Second, we explore artistic style variation, where SliderSpace discovers directions that match or exceed the diversity of manually curated artist lists while being judged more useful by human evaluators. Finally, we show how SliderSpace can help reverse the mode collapse commonly observed in distilled diffusion models, restoring diversity while maintaining generation speed.

Beyond providing practical creative control, SliderSpace opens new avenues for understanding and utilizing the latent capabilities of diffusion models. By mapping these models' visual potential into intuitive, composable directions, we take a step toward making their creative possibilities more accessible and interpretable to users.

% Image editing methods in diffusion models unlock the creativity of users. In this work we ask an alternate question: \emph{Can we organize and expose what of the diffusion model is already capable of?}.
% Existing methods for controlling image generation typically require users to manually specify edit directions for desired changes. This process is time-consuming, requires technical expertise, and limits the spontaneity of the creative process. For instance, if a user wants to adjust the smile of a generated person, they must explicitly request this edit, often through imprecise prompt engineering or model fine-tuning. This approach of predefined controls or manual specifications restricts users from fully exploring the latent capabilities of the model. There may be interesting stylistic variations or attributes that the model can generate, but users have no easy way to discover or utilize these.

% Natural visual disentanglement was an emergent property in the latent space of Generative Adversarial Models (GANs) \cite{harkonen2020ganspace,radford2015unsupervised, wu2021stylespace, shen2020interfacegan}. In particular, it has been observed that StyleGAN~\cite{karras2019style} stylespace neurons offer detailed control over many meaningful aspects of images that would be difficult to describe in words~\cite{wu2021stylespace}. However, diffusion models do not share such a compact latent space~\cite{park2023unsupervised}; and efforts to uncover such a space in the semantic embeddings of the text conditioning have met with limited success \nik{Nick - is there a specific citation you were thinking about?}.

% In this work we introduce \textbf{SliderSpace}, which takes a step towards uncovering an analogous low dimensional representation of diffusion models' visual breadth; in essence treating the diffusion model as many generators sharing parameters, where a particular generator is defined by a specific prompt. For a given prompt we sample many random seeds (and optionally prompt expansions using an LLM), generate the corresponding images, and apply an off the shelf feature extractor (in this work CLIP, but our method can be applied to any differentiable feature extractor). We use PCA to analyze these features, and for each of the leading $k$ principal components we train a LoRA \cite{} which causes the diffusion model to produces images which increase the feature magnitude along that component when passed back through the same feature extractor. This leads to a 'Slider' for each principal component, because each LoRA can be scaled and applied to the original diffusion model, continuously varying those visual features in the generated results (as measured, in our case, by CLIP).

% There are many other works that enhance the controllability of diffusion models. One common approach is enabling users to add spatial constraints to a generation either manually, or via a reference image \cite{zhang2023addingconditionalcontroltexttoimage, chen2024trainingfreeregionalpromptingdiffusion}, a second is leveraging more abstract embeddings (e.g. identity, style) extracted from a reference image \cite{ye2023ipadaptertextcompatibleimage, hertz2024stylealignedimagegeneration, li2023photomaker, shi2024instantbooth}, a third is finetuning a foundation model to better generate a concept important to the user \cite{ruiz2022dreambooth, kumari2022customdiffusion, Ryu_lora, hu2021lora}, and a fourth (most relevant to this work) is finding low-rank adaptors of the model based on a prompt or small training set which can be scaled to provide continous control over one aspect of generated image (e.g. night vs day, basic vs luxury, etc.) \cite{gandikota2023concept}. SliderSpace is complementary to all of these methods and offers something distinct. All of the other methods we are aware require the user (and / or model designer) to know in advance what type of control they want. In contrast SliderSpace assists users in discovering and controlling hidden capabilities present in the diffusion model's distribution of possible generations.

%We propose that truly intuitive creative control in a text-to-image model should meet three key criteria: \emph{discoverability}, \emph{intuitiveness}, and \emph{specificity}. The model should reveal controllable attributes that may not be immediately obvious, offer controls that are easy to understand and manipulate, and ensure each control affects a distinct attribute of the generated image.

% We demonstrate the utility and power of SliderSpace using three applications built on top of SDXL-DMD \cite{dmd}, because its fast generation speed lends itself well to the continuous control offered by SliderSpace.

% First, we study concept decomposition (Section \ref{sec:concept_exp}), where we learn sliders for a specific concept (e.g. 'monster', 'waterfall', 'car'). Through quantitative metrics of diversity and text alignment we demonstrate that the learned sliders dramatically boost the diversity of generations when randomly applied without harming text alignment; we also ask humans to qualitatively judge these results in a user study where they find the SliderSpace results to be more 'Diverse', 'Useful', and 'Creative' than our baselines.

% Second, we attempt to compare the automatic discoveries of SliderSpace to a large scale manual study of artistic styles (Section \ref{sec:art_exp}), open-sourced by ParrotZone \cite{parrotzone}. In this study SDXL was prompted with over 4300 artist names,  and based on visual inspection the cases of successful stylistic mimicry recorded. Quantitatively SliderSpace more closely matches the distribution of artistic variation discovered by ParrotZone than other baselines, and in our user studies was judged to be significantly more 'Diverse' and 'Useful' than the baselines. To our surprise humans even judged SliderSpace results to be slightly more 'Diverse' than the results generated by the manually discovered artist names of \cite{parrotzone}.

% Third, we attempt to use SliderSpace to reverse the mode collapse commonly observed in distilled few-step diffusion models relative to the original teacher model (Section \ref{sec:diverse_exp}). We quantitatively demonstrate that applying SliderSpace to SDXL-DMD leads to more closely matching the distribution of images by the original teacher, SDXL.

%Through extensive experiments on various state-of-the-art text-to-image models, we demonstrate that SliderSpace significantly enhances user control and creative expression in AI-assisted image generation tasks. Our method enables a range of applications, including concept decomposition and control, diversity improvement in generated images, customization dissection and edits, and the exploration of artistic styles inherent in the model.

% SliderSpace goes beyond providing a practical tool for enhanced creative control. By mapping the visual potential of diffusion models it can open new avenues for generative creativity and deepens our understanding of each model's hidden potential.



\section{Causal Event Abstraction}\label{sec:setup}
% 3 sections; 1. lab experiments, 2. combinations, and 3. evaluation/results

\section{Experimental Setup}

\color{red}wrote this whole section (esp eval) using the word "treatment" either need to define it or (preferably) change it\color{black}

With the goal of better understanding the ability of multi-agent LLM systems to model prosocial human behavior, we simulate three different types of experiments: (1) simulations which aim to directly capture the effect of a treatment observed in a previous lab experiment with human subjects; (2) simulations which combine treatments/effects from multiple previous lab experiments with human subjects; and (3) select "out-of-the-lab" real-world scenarios we use as case studies.

For all experiments, we use OpenAI's GPT. Specifically, we use GPT4 (model gpt-4o-mini), which has been shown to have the ability to interpret inherently human concepts such as equity and also scores  well on a variety of standardized tests, ranging from the Bar Exam to the GRE \cite{openai2023gpt4}.

\subsection{Research Questions}

We specifically address the following three research questions:

\bigskip

\begin{itemize}[itemindent=2.5em]
  \item[\underline{\textbf{RQ1}}] Can multi-agent LLM system simulations replicate behaviors observed in PGG lab experiments with human subjects?
\begin{itemize} [leftmargin=0.5in]
\item[\underline{a.}] Does priming LLM agents via game name replicate the effect of priming humans via game name?
\item[\underline{b.}] Does introducing transparency in contribution between LLM agents replicate the effect of introducing transparency in contribution between humans?
\item[\underline{c.}] Does varying the endowments of LLM agents replicate the effect of varying endowments of humans?
\end{itemize}
\end{itemize}

\bigskip
\color{blue}

\begin{itemize}[itemindent=2.5em]
  \item[\underline{\textbf{RQ1}}] \textit{Can multi-agent LLM system simulations replicate behaviors observed in PGG lab experiments with human subjects?} We answer this research question via three experiments comparing results from simulations with LLM-agents to results from lab experiments with human subjects:
  
  \begin{itemize} [leftmargin=0.5in]
    \item[\underline{1.}] Priming via Game Name
    \item[\underline{2.}] Introducing Transparency of Contributions
    \item[\underline{3.}] Varying Endowments
  \end{itemize}
\end{itemize}

\color{black}

\bigskip

\begin{itemize}[itemindent=2.5em]

  \item[\underline{\textbf{RQ2.}}] Can multi-agent LLM system simulations replicate human behavior extrapolated from combining the behaviors observed in multiple lab experiments with human subjects?
\begin{itemize}[leftmargin=0.5in]
    \item[\underline{a.}] Does priming LLM agents with sentences known to increase/decrease generosity in humans have that effect in simulations of the PGG?
\item[\underline{b.}] Does the effect of priming LLM agents via game name in multi-round PGGs fade over time, like priming has been observed to fade over time in other experiments with human subjects?
\end{itemize}
\end{itemize}

  \bigskip
\color{blue}
\begin{itemize}[itemindent=2.5em]

  \item[\underline{\textbf{RQ2.}}] \textit{Can multi-agent LLM system simulations replicate human behavior extrapolated from combining the behaviors observed in multiple lab experiments with human subjects?} We answer this research question via two experiments combining treatments and expected effects from multiple lab experiments with human subjects:
  \begin{itemize}[leftmargin=0.5in]
    \item[\underline{1.}] Using a priming methodology from other cooperation games in the PGG
    \item[\underline{2.}] Measuring the change in effect of priming over time, using results of priming over time from a different competition game
  \end{itemize}
\end{itemize}
\color{black}

\bigskip

\begin{itemize}
  \item[\underline{\textbf{RQ3.}}] What are required mechanisms for simulations to
inform policy-makers regarding real-world situations to test policies that encourage human collaboration?
\end{itemize}

\bigskip

\color{blue}

\begin{itemize}
  \item[\underline{\textbf{RQ3.}}] \textit{What are required mechanisms for simulations to
inform policy-makers regarding real-world situations to test policies that encourage human collaboration?} We answer this research question via three real-world case studies towards identifying necessary mechanisms:
\begin{itemize}[leftmargin=0.5in]
    \item[\underline{1.}] A classroom setting with varying late-assignment policies
    \item[\underline{2.}] A store parking lot where shoppers need to return shopping carts to designated areas
    \item[\underline{3.}] Graffitti?
  \end{itemize}
\end{itemize}

\color{black}

\bigskip

We expect generally positive results for the first two research questions, given other work has shown the ability of LLMs to replicate lab experiments that are both published and unpublished \cite{hewitt2024predicting}, but still verify this in the context of the PGG towards specifically understanding the abilities of multi-agent LLM systems to simulate prosocial behavior. \color{red}The third research question is more open-ended - we use case studies of two real-world scenarios towards identifying two mechanisms required for simulations to show expected outcomes, a process which can be continued to be used by policy makers. As policy makers consider the specific scenarios which they need to simulate, they may realize and implement new mechanisms required to enable their simulation.\color{black}

\subsection{Multi-Agent-LLM System}

For our experiments, we adapt a previous introduced agent architecture for simulating social emergent behavior \cite{park2023generativeagentsinteractivesimulacra}. In this architecture, the administrator provides each agent with a name, public and private biographies, instructions (called directives), and an initial plan, which consists of a description, stop condition, and a location. The architecture also allows for the creation of rooms, between which agents are able to move. 

\subsubsection{System Setup for Public Goods Game Experiments}

In PGG experiments, there is an agent named and directed to act as the moderator. The remaining agents (varies from three to four depending upon the experiment) are given arbitrary alphabetical names (Alice, Bob, Casey, and David). Each agent's initial endowment is specified in their public biography, which all other agents can see. If there is a priming condition, it is included in the private biography as to expose the desired agent to the priming without leakage to other agents. Communication is not inherently turn-taking in the architecture; so we have to specify the order and method of contribution to player agents via their directives and initial plan. We initialize two separate rooms - the Game Room and the Moderation Room. Player agents are by default in the game room, then travel to the moderation room to tell the moderator their contribution once the player specified as being before them returns to game room from the moderation room after making their contribution. The moderator remains in the moderation room other than to announce payoffs after all players have made contributions. To realize transparency, we simply remove the moderation room, and have all agents make their contributions publicly in the game room. For the experiment with multiple rounds, we use nearly the same instruction set, other than specifying to the moderator and player agents via directives to repeat the same process for multiple rounds.

The thoughts and actions of each player (and the moderator) are displayed in separate output logs, each representing one of the agents involved in the simulation. In each of the output logs, the agent waits for the condition required (the return of the player before them), then acts on the plan of moving to the moderation room and making a contribution by speaking with the moderator. The moderator stores each contribution to memory in order to later announce payoffs. We extract the contribution amount of each agent using a script on each agent's output logs, identifying the section in which the agent speaks in the moderation room and taking the monetary amount in that sentence. We also manually verified these contribution amounts to ensure accuracy.

\subsubsection{System Setup for Real-World Case Studies}

\color{red}We use two real-world settings as case studies to identify mechanisms needed to enable informative simulations. For these case studies, we run several simulations with slightly varying conditions to generally understand the ways in which agents act, and then use that to create a mechanism which enables a behavior we identify as missing from a combination of general knowledge and informative studies.

The first is a classroom setting, where we specifically focus the impact of various late policies (no penalty, full penalty, and penalty by days late) on student communicative and collaborative behaviors. We define a professor agent (named the Professor) and three student agents with arbitrary names (Alice, Bob, and Casey). The professor agent is given instructions to announce a late policy to the class, and then begin assigning assignments. In student agents public biography, they are given one of three "personalities" - an overachiever, a hard-worker, and a balanced individual. In the private biographies of the agents, we add various perturbations to student lives that would reasonably affect their ability to turn in assignments on time, such as having a midterm in another class, having special difficulty with a particular assignment, or having a busy week in general. We do not need to implement a rigid turn-taking guideline, however we do generally add a framework within which agents act - the professor agent announces assignments on a fixed frequency in the classroom, between which student agents "work" on assignments and consider how many days late they will need to submit it based on the other factors in their lives.

Second, we simulate shoppers in a parking lot who can either return or not return their shopping cart. We define five shopper agents with arbitrary names (Alice, Bob, Casey, David, and Emery). We add various perturbations to shopper lives that affect their ability to return their shopping cart \cite{scientificamerican2020shoppingcarts} in their private biographies. Shopper agents by default have blank public biographies; but if perturbations added to the shopper's private biography would reasonably be extrapolated other people in real life (such as having a child), we also include it in the shopper's public biography. Shopper agents are not given directives, but have an initial plan to decide whether or not to return their shopping cart, and then to act on it. We also try changing the environment by adding agents who are employees of the store tasked with taking the shopping carts from the parking lot back into the store. These agents are given the name "Employee," have public biographies which indicate that they are a store employee, have an initial plan (and related directives) to move carts, and otherwise have all other fields left blank.\color{black}

\subsection{Evaluation}

\subsubsection{Replicating Observed Outcomes from Past Public Goods Game Experiments}

For experiments of the first type, we run simulations of the PGG using treatments from published lab experiments with human subjects, checking whether the effect of said treatments on multi-agents LLMs in simulation in similar to that of on humans. The treatments and effects we replicate from previous lab experiments are listed in Table ~\ref{labexperiments}.

\begin{center}
\begin{table}[H]
\setlength{\abovecaptionskip}{0pt}
\setlength{\belowcaptionskip}{10pt}
\begin{tabular}{|>{\columncolor{gray!20}\centering\arraybackslash}m{0.5cm}|p{6cm}|p{6cm}|} 
 \hline
 \rowcolor{gray!20} \textbf{Exp.} & \multicolumn{1}{c|}{\textbf{Treatment}} & \multicolumn{1}{c|}{\textbf{Expected Result}} \\ [0.5ex] 
 \hline\hline
 \textbf{1} & \textbf{Priming via Game Name:} Participants are presented the PGG as a game called either the "Taxation Game" or "Teamwork Game." & Participants presented the game as the "Taxation Game" contribute less than participants presented the "Teamwork Game." \cite{Eriksson_Strimling_2014} \\ 
 \hline
 \textbf{2} & \textbf{Transparency of Contribution:} Participants know what others contribute to the public pool (under standard conditions, contributions are made privately with respect to other players). & Participants make larger contributions when participants are aware of each other's contributions compared to when contributions are made privately without other's knowledge. \cite{transparencytwo}\\
 \hline
 \textbf{3} & \textbf{Variations in Endowment:} Participants start the PGG with different endowment amounts. & "Rich" participants contribute more when other participants are also "rich." \cite{HARGREAVESHEAP20164}  \\
 \hline
\end{tabular}
\caption{\label{labexperiments}Table of treatments and expected results (effects) from previous lab experiments of the PGG with human subjects. Actual experiments that are drawn from are cited in the second column.}
\end{table}
\end{center}




For these experiments, we simulate one-shot PGGs with three or four players, measuring whether the impact the treatments have on player contributions is similar to what has been observed in experiments with human subjects (i.e., the expected result listed). For the first experiment, we run ten simulations of the PGG with four players, two under each priming condition, and compare the average contribution amount of players from either group. For the second experiment, we also run ten simulations of the PGG with four players: five simulations with transparency of contributions and five without. We compare the average contribution amount of players in either of the two conditions. In the first and second experiments, players are all endowed with \$20. For the third experiment, we run twenty simulations with three players: fifteen simulations where all players have the same initial endowment (five where that endowment is \$20, five where it is \$50, and five where it is \$80) and five simulations where players have different initial endowments (namely, \$20, \$50, and \$80). We compare the average contributions of players with each of the three endowments in the fixed and the varied endowment conditions (i.e., we compare how much a player with \$X contributes in the condition where everyone has \$X vs. the varied condition). In all three experiments, 1.6 times the amount of the public pool is split evenly amongst the players as their payoff - players are made aware of this to inform the way in which they act, but the simulations are of one-shot PGGs.

\subsubsection{Verifying Outcomes from Combinations of Past Lab Experiments}

We also design two additional experiments based on a combination of treatments and effects from multiple lab experiments with human subjects. The first takes a priming methodology and effect from experiments with human subjects of different cooperative games than the PGG to see if the effect holds for the PGG. The second uses a priming methodology used on the one-shot PGG and measures if it has the same effect over multiple rounds similar to the way in which priming affects different economic games than the PGG. The treatments and effects in the combined experiments we design are listed in Table ~\ref{combinedexperiments}.

\begin{center}
\begin{table}[H]
\setlength{\abovecaptionskip}{0pt}
\setlength{\belowcaptionskip}{10pt}
\begin{tabular}{|>{\columncolor{gray!20}\centering\arraybackslash}m{0.5cm}|p{6cm}|p{6cm}|} 
 \hline
 \rowcolor{gray!20} \textbf{Exp.} & \multicolumn{1}{c|}{\textbf{Treatment}} & \multicolumn{1}{c|}{\textbf{Expected Result}} \\ [0.5ex] 
 \hline\hline
 \textbf{1} & \textbf{Priming via Sentences:} Participants are presented with 5 sentences alluding to either "unity" or "proportionality" concepts. & Participants presented with sentences alluding to "unity" contribute more than participants primed with "proportionality". \cite{6d947858-4a17-3462-9053-fc55b58ffee1, moralsmatter} \\ 
 \hline
 \textbf{2} & \textbf{Priming via Game Name over Multiple Rounds:} Participants are presented the PGG as a game called either the "Taxation Game" or "Teamwork Game" and play for five rounds. & The effect of priming should fade over time; although there should still be a disparity between participants in either group's initial contributions, it should lessen over time. \cite{Eriksson_Strimling_2014, JIMENEZJIMENEZ201594} \\
 \hline
\end{tabular}
\caption{\label{combinedexperiments}Table of treatments and expected results (effects) from combining previous lab experiments with human subjects. Actual experiments that are drawn from are cited in the second column.}
\end{table}
\end{center}

For both experiments, all players are endowed with \$20. For the first experiment, we run ten simulations a one-shot PGG with four players, two under each priming condition ("unity" and "proportionality"), and compare the average contribution amount of players from either group. We prime each player by exposing them to five sentences alluding to each concept, taken from the lab experiment \cite{moralsmatter}. In this experiment, 1.6 times the amount of the public pool is split evenly amongst the players as their payoff - players are made aware of this to inform the way in which they act, but the simulations are of one-shot PGGs. For the second experiment, we run ten simulations of a PGG with five rounds and four players, two under each priming condition ("Taxation" and "Teamwork"). In this experiment, payoffs are computed the same as in the previous experiment (namely, 1.6 times the amount of the public pool split evenly amongst the players). Since there are multiple rounds, the payoff and amount not contributed are summed to determine each players endowment for the next round. Within each priming condition, we compare the average contribution in the first round compared to the fifth round. We also compare the contributions between priming conditions, checking that there was an initial disparity that ultimately begins to fade.

\color{red}

\subsubsection{Real-World Case Studies}

SEE RESULTS, THERE'S A BIG BULLET POINT LIST OF EVAL / RESULTS COMBINED I'M TRYING TO FIGURE OUT HOW TO SEPARATE. 

TALK ABOUT HOW THESE CONNECT TO PROSOCIAL SITUATIONS HERE?

\color{black}


\section{The ACCESS Benchmark}
\section{Causal IL as CMRs}\label{sec:method}

In this section, we demonstrate that performing causal IL in our framework is possible using trajectory histories as instruments. In the next step, we show that the problem can be described as CMRs and propose an effective algorithm to solve it.

The typical target for IL would be the expert policy $\pi_E$ itself. However, since the expert has access to information, namely $u^o_t$, which the imitator does not, the best thing an imitator can do is to learn a history-dependent policy $\pi_h$ that is the closest to the expert. A natural choice is the conditional expectation of $\pi_E(s_t,u^o_t)$ on the history $h_t$:
\begin{align}
\pi_h(h_t)\coloneqq \expectE_{\probP(u^o_t\mid h_t)}[\pi_E(s_t,u^o_t)]=\expectE[\pi_E(s_t,u^o_t)\mid h_t],\nonumber
\end{align}
% where $p(u^o_t\mid h_t)$ is a distribution over expert-observable confounders and captures the information about $u^o_t$ can be inferred from the trajectory history. 
because the conditional expectation minimizes the least squares criterion~\citep{hastie01statisticallearning} and $\pi_h$ is the best predictor of $\pi_E$ given $h_t$. In $\pi_h$, the distribution $\probP(u^o_t\mid h_t)$ captures the information about $u^o_t$ that can be inferred from trajectory histories.
\begin{remark}
\emph{Learning $\pi_h$ is not trivial. Policies learnt naively using behaviour cloning (i.e., $\expectE[a_t\mid h_t]$) fail to match $\pi_E$. In view of~\cref{eq:action}, we have that
\begin{align} 
\expectE[a_t\mid h_t]&=\expectE[\pi_E(s_t,u^o_t) \mid h_{t}]+\expectE[u^\epsilon_t\mid h_{t}]\nonumber\\
&=\pi_h(h_t)+\expectE[u^\epsilon_t\mid h_{t}],\label{eq:history_policy}
\end{align}
where $\expectE[u^\epsilon_t\mid h_{t}]\neq 0$ due to the spurious correlation between $u^\epsilon_t$ and the trajectory history $h_t$. As a result, $\expectE[a_t\mid h_t]$ becomes biased, which can lead to arbitrarily worse performance compared to $\pi_E$.   }
\end{remark}

\vspace{-5pt}
\paragraph{Derivation of CMRs.} 
Leveraging the confounding horizon from Assumption~\ref{assump:horizon}, it becomes possible to break the spurious correlation using the independence of $u^\epsilon_t$ and $u^\epsilon_{t-k}$. We propose to use the $k$-step trajectory history $h_{t-k}=(s_{1},a_{1},...,s_{t-k})$ as an instrument for the current state $s_t$. Taking the expectation conditional on $h_{t-k}$ in~\cref{eq:history_policy} yields
\begin{align*}
    \expectE[a_t\mid h_{t-k}] & = \expectE\left[\expectE[a_t\mid h_{t}]\mid h_{t-k}\right] \\ & = \expectE[\pi_h(h_t)\mid h_{t-k}]+\expectE[\expectE[u^\epsilon_t\mid h_{t}]\mid h_{t-k}] \\
    & = \expectE[\pi_h(h_t) \mid h_{t-k}]+\expectE[u^\epsilon_t\mid h_{t-k}]
\end{align*}
where we use the fact that $h_{t-k}$ is $\sigma(h_t)$-measurable because $h_{t-k}\subseteq h_t$. Next, recall that $u^\epsilon_t\indep u^\epsilon_{t-k}$ by Assumption~\ref{assump:horizon}, which implies $u^\epsilon_t\indep h_{t-k}$, so that % Hence, since $\expectE[u^\epsilon_t] = 0$, we obtain
\begin{align}
    \expectE[a_t\mid h_{t-k}] &= \expectE[\pi_h(h_t) \mid h_{t-k}]+\expectE[u^\epsilon_t]\nonumber\\
    &=\expectE[\pi_h(h_t) \mid h_{t-k}].
\end{align}

As a result, the problem of learning $\pi_h$ reduces to solving for $\pi_h$ that satisfies the following identity
\begin{align}
    \expectE[a_t-\pi_h(h_t)\mid h_{t-k}]=0,\label{eq:CMR}
\end{align}
which is a CMR problem as defined in~\cref{sec:cmr}. In this case, both $a_t$ and $h_t$ are observed in the confounded expert demonstrations, and $h_{t-k}$ acts as the instrument. 

To make sure the instrument $h_{t-k}$ is valid, we check that it satisfies the conditions of~\cref{assump:iv}. Firstly, we have checked that $u^\epsilon_t\indep h_{t-k}$. Secondly, the environment and the expert policy are non-trivial, which means $\probP(h_t\mid h_{t-k})$ is not constant in $h_{t-k}$. Finally, $h_{t-k}$ indeed only affects $a_t$ through $s_t$ by the Markovian property. However, the strength of the instrument, which informally represents the correlation between the instrument $h_{t-k}$ and $h_t$, plays an important role in how well we can identify $\pi_h(h_t)$ by solving the CMRs in~\cref{eq:CMR}. In particular, we see that, as the confounding horizon $k$ increases, the correlation between $h_{t-k}$ and $h_t$ weakens and $h_{t-k}$ becomes a weaker instrument. This means that it is less able to identify $\pi_h$ via the CMR in~\cref{eq:CMR} and the final learnt imitator will have poorer performance. This is confirmed theoretically in Proposition~\ref{prop:ill-posed} and experimentally in~\cref{sec:exps}, and we will formalise this notion of instrument strength in~\cref{sec:theory}.


% Note this problem is equivalent to solving an IV regression on~\cref{eq:history_policy}, where $Y=\expectE[a_t\lvert h_t]$, $f(x)=\pi_h(h_t)$, $\epsilon=\expectE[u^\epsilon_t$ and the instrument $Z=h_{t-k}$.




\subsection{Practical Algorithms for Solving the CMRs}

\begin{algorithm}[tb]
   \caption{DML-IL}
   \label{alg:DML-IL}
\begin{algorithmic}[1]
   \STATE {\bfseries input} Dataset $\dataset_E$ of expert demonstrations, Confounding noise horizon $k$
   \STATE Initialize the roll-out model $\hat{M}$ as a Gaussian mixture model\label{algo:roll_out_1}
    \REPEAT
   \STATE Sample $(h_{t},a_t)$ from data $\dataset_E$
   \STATE Fit the roll-out model $(h_t,a_t)\sim\hat{M}(h_{t-k})$ to maximize the log likelihood 
\UNTIL{convergence}\label{algo:roll_out_2}
   \STATE Initialize the expert model $\hat \pi_h$ as a neural network
   \REPEAT
   % \FOR{$k=1$ {\bfseries to} $K$}
   \STATE Sample $h_{t-k}$ from $\dataset_E$
   \STATE Generate $\hat{h}_t$ and $\hat{a}_t$ using the roll-out model $\hat{M}$
   \STATE Update $\hat \pi_h$ to minimise the loss $\ell:= \norm{\hat{a}_t - \hat{\pi}_h (\hat h_t)}_2$
   % \ENDFOR
    \UNTIL{convergence}
    \STATE {\bfseries return} A history-dependent imitator policy $\hat{\pi}_h$
\end{algorithmic}
\end{algorithm}

There are various techniques~\citep{Shao2024,Bennett2019,Xu2020,Dikkala2020} for solving the CMRs $\expectE[a_t\lvert h_{t-k}]=\expectE[\pi_h(h_t) \lvert h_{t-k}]$. Here, the \textit{CMR error} that we aim to minimise is given by 
\begin{align*}
\sqrt{\expectE\big[\expectE[a_t-\hat{\pi}_h(h_t)\lvert h_{t-k}]^2\big]}=\norm{\expectE[a_t-\hat{\pi}_h(h_t)\lvert h_{t-k}]}_{2}.    
\end{align*}
In~\cref{alg:DML-IL}, we introduce DML-IL, an algorithm adapted from the IV regression algorithm DML-IV~\citep{Shao2024}\footnote{DML stands for double machine learning~\citep{Chernozhukov2018Double}, which is a statistical technique to ensure fast convergence rate for two-step regression, as is the case in~\cref{alg:DML-IL}.}, which solves our CMRs by minimising the CMR error. The first part of the algorithm (line 3-7) learns a roll-out model $\hat{M}$ that generates a trajectory $k$ steps ahead given $h_{t-k}$. Then, the roll-out model $\hat{M}$ is used to train the policy model $\hat{\pi}_h$ (line 8-13). $\hat{\pi}_h$ takes the generated trajectory $\hat{h}_t$ from $\hat{M}(h_{t-k})$ as inputs, and minimises the mean squared error to the next action. Using generated trajectories is crucial in breaking the spurious correlation caused by $u^\epsilon_t$ between past states and actions, and using the trajectory history before $h_{t-k}$ allows the imitator to infer information about $u^o_t$.

DML-IL can also be implemented with $K$-fold cross-fitting, where the dataset is partitioned into $K$ folds, with each fold alternately used to train $\hat{\pi}_h$ and the remaining folds to train $\hat{M}$. This ensures unbiased estimation and improves the stability of training. The base IV algorithm DML-IV with $K$-fold cross-fitting is theoretically shown to converge at the rate of $O(N^{-1/2})$~\citep{Shao2024}, where $N$ is the sample size, under regularity conditions. DML-IL with $K$-fold cross-fitting (see~\cref{appendix:dmlil} for details) will thus inherit this convergence rate guarantee. 

Note that~\cref{alg:DML-IL} requires the confounding noise horizon $k$ as input. While the exact value of $k$ can be difficult to obtain in reality, any upper bound $\bar{k}$ of $k$ is sufficient to guarantee the correctness of ~\cref{alg:DML-IL}, since $h_{t-\bar{k}}$ is also a valid instrument. Ideally, we would like a data-driven approach to determine $k$. Unfortunately, it is generally intractable to empirically verify whether $h_{t-k}$ is a valid instrument from a static dataset, especially the unconfounded instrument condition (i.e., $h_{t-k}\indep u^\epsilon_t$). Therefore, we rely on the user to provide a sensible choice of $\bar{k}$ based on the environment that does not substantially overestimate $k$.


\subsection{Theoretical Analysis}\label{sec:theory}

% \begin{align}
% p(u_t\lvert do(a_{t-k+1}),...,do(a_{t-1}),s_{t-k+1},...,s_{t-1})&\propto p(h_t)p_{\mu_0}(s_{t-k+1})\prod_{i=t-k+1}^{t-1} \transitions(s_{i+1}\lvert s_i,a_i,u_i)
% \end{align}

% since $$(u_t\indep a_{(t-k+1)...(t-1)} \lvert s_{(t-k+1)...(t_1)})_{\mathcal{G}_{\underline{a{(t-k+1)...(t-1)}}}}$$
% on the causal graph $\mathcal{G}_{\underline{a{(t-k+1)...(t-1)}}}$ where the arrows going into $a_{(t-k+1)...(t-1)}$ are removed.



In this section, we derive theoretical guarantees for our algorithm, focusing on the imitation gap and its relationship with existing work.


On a high level, in order to bound the imitation gap of the learnt policy $\hat{\pi}_h$, i.e., $J(\pi_E)-J(\hat{\pi}_h)$, we need to control:
\begin{enumerate}
    \item[($i$)] The amount of information about the hidden confounders that can be inferred from trajectory histories;
    \item[($ii$)] The ill-posedness (or identifiability) of the set of CMRs, which intuitively measures the strength of the instrument $h_{t-k}$;
    \item[($iii$)] The disturbance of the confounding noise to the states and actions at test time.
\end{enumerate}
These factors are all determined by the environment and the expert policy. To control ($i$), we measure how much information about $u^o_t$ is captured by the trajectory history $h_t$ by analysing the Total Variation (TV) distance between the distribution of $u^o_t$ and $\expectE[u^o_t\lvert h_t]$ along the trajectories of $\pi_E$. To control ($ii$) and ($iii$), we need to introduce the following two key concepts.

\begin{definition}[The ill-posedness of CMRs~\citep{Dikkala2020,Chen2012}]

Given the derived CMRs in~\cref{eq:CMR}, for a policy $\pi\in\Pi$, $\norm{\pi_E-\pi}_2$ is the root mean squared error to the expert and $\norm{\expectE[a_t-\pi(s_t)\lvert s_{t-k}]}_2$ is the CMR error we aim to minimise. Then, the \emph{ill-posedness} $\ill(\Pi,k)$ of the policy space with confounding noise horizon $k$ is given by
\begin{align*}
    \ill(\Pi,k)=\sup_{\pi\in\Pi} \frac{\norm{\pi_E-\pi}_{2}}{\norm{\expectE[a_t-\pi(h_t)\lvert h_{t-k}]}_{2}}.
\end{align*}
\end{definition}
The ill-posedness $\ill(\Pi,k)$ measures the strength of the instrument where a higher $\ill(\Pi,k)$ indicates a weaker instrument. It bounds the ratio between the learning error of the imitator following our CMR objective and its $L_2$ error to the expert policy. 

As discussed previously, intuitively, the strength of the instrument would decrease as the confounding horizon $k$ increases. This is in fact true and is confirmed by the following proposition. The proof is deferred to~\cref{appendix:prop}. 
\begin{proposition}\label{prop:ill-posed}
The ill-posedness $\ill(\Pi,k)$ is monotonically increasing as the confounded horizon $k$ increases.
\end{proposition}

Next, we introduce the notion of c-TV stability.
\begin{definition}[c-total variation stability~\citep{Bassily2021,Swamy2022_temporal}]
Let $P(X)$ be the distribution of a random variable $X:\Omega\rightarrow \mathcal{X}$. $P(X)$ is c-TV stable if for $a_1,a_2\in \mathcal{X}$ and $\Delta>0$,
\begin{align*}
\norm{a_1-a_2}\leq\Delta \implies \delta_{TV}(a_1+X,a_2+X)\leq c\Delta.
\end{align*}
where $\norm{\cdot}$ is some norm defined on $\mathcal{X}$ and $\delta_{TV}$ is the total variation distance.
\end{definition}
A wide range of distributions are c-TV stable. For example, standard normal distributions are $\frac{1}{2}$-TV stable. We apply this notion to the distribution over $u^\epsilon_t$ to bound the disturbance it induces in the trajectory and the expected return.

With the notion of ill-posedness and c-TV stability, we can now analyse and upper bound the imitation gap $J(\pi_E)-J(\hat{\pi}_h)$ by controlling the three components $(i)-(iii)$ discussed above. 
% We present the main result for this paper, where t
The full proof is deferred to~\cref{appendix:gap}.

\begin{theorem}[Imitation Gap Bound]\label{thm:gap}
Let $\hat{\pi}_h$ be the learnt policy with CMR error $\epsilon$ and let $\ill(\Pi,k)$ be the ill-posedness of the problem. Assume that $\delta_{TV}(u^o_t,\expectE_{\pi_E}[u^o_t\lvert h_t])\leq\delta$ for $\delta\in\realNumber^+$, $P(u^\epsilon_t)$ is c-TV stable and $\pi_E$ is deterministic. Then, the imitation gap is upper bounded by 
\begin{align*}
    J(\pi_E)-J(\hat{\pi}_h)\leq T^2\big(c\epsilon\ill(\Pi,k)+2\delta\big)=\mathcal{O}\big(T^2(\delta+\epsilon)\big).
\end{align*}
\end{theorem}
This upper bound scales at the rate of $T^2$, which aligns with the expected behaviour of imitation learning without an interactive expert~\citep{Ross2010}.
Next, we show that the upper bounds on the imitation gap from prior work~\citep{Swamy2022_temporal, Swamy2022} are special cases of
% of  subsumed by the unifying causal IL framework introduced in Section~\ref{sec:setting} are special cases of 
Theorem~\ref{thm:gap}. The proofs are deferred to~\cref{appendix:corollaries}.
\begin{corollary}\label{corollary:noUo}
In the special case that $u^o_t = 0$, i.e., there are no expert-observable confounders, or $u^o_t=\expectE_{\pi_E}[u^o_t\lvert h_t]$, i.e., $u^o_t$ is $\sigma(h_t)$ measurable (all information about $u^o_t$ is contained in the history), the imitation gap is upper bounded by
\begin{align*}
    J(\pi_E)-J(\hat{\pi}_h)\leq T^2\big(c\epsilon\ill(\Pi,k)\big)=\mathcal{O}\big(T^2\epsilon\big),
\end{align*}
which coincides with Theorem 5.1 of~\citet{Swamy2022_temporal}.
\end{corollary}

When there are no hidden confounders, i.e, $u^\epsilon_t=0$, our framework is reduced to that of~\citet{Swamy2022}. However, \citet{Swamy2022} provided an abstract bound that directly uses the supremum of key components in the imitation gap over all possible Q functions to bound the imitation gap. We further extend and concretise the bound using the learning error $\epsilon$ and the TV distance bound $\delta$ instead of relying on the suprema.


\begin{corollary}\label{corollary:unconfounded}
In the special case that $u^\epsilon_t=0$, if the learnt policy has optimisation error $\epsilon$,  the imitation gap is upper bounded by
\begin{align*}
    J(\pi_E)-J(\hat{\pi}_h)\leq T^2\left(\frac{2}{\sqrt{\dim(A)}}\epsilon+2\delta \right),
\end{align*}
which is a concrete bound that extends the abstract bound in Theorem 5.4 of~\cite{Swamy2022}.
\end{corollary}

\begin{remark}
\emph{If both $u^\epsilon_t$ and $u^o_t$ are zero, we then recover the classic setting of IL without confounders~\citep{Ross2010}, and the imitation gap bound is $T^2\epsilon$, where $\epsilon$ is the optimisation error of the algorithm.}
\end{remark}


\section{Experiments}\label{sec:exp}
\section{Experiment}
In this section, we conduct extensive experiments to evaluate the performance of various LLMs on our Hellaswag-Pro benchmark. Our study is guided by three key research questions:
\textbf{RQ1}: How do different LLMs perform across all variants?
\textbf{RQ2}: What is the relative difficulty of different variants?
\textbf{RQ3}: How robust are LLMs to diverse prompts during evaluation?

\subsection{Experiment Setup} 
\subsubsection{Model Selection and Implementation Details}
We select 41 representative commercial and open-source models, including English LLMs, such as GPT-4o, Claude-3.5-Sonnet, Gemini-1.5-Pro,Mistral series, Llama3 series and Chinese LLMs, like Qwen-Max,  Qwen2.5 series, InternLM-2.5 series, Yi-1.5 series, Baichuan-2 series and DeepSeek series.

We integrate both Chinese HellaSwag and HellaSwagPro into the lm-evaluation-harness platform. For the open-source models, we use the default settings of lm-evaluation-harness: do\_sample is set to false and the temperature is set to the default value of the hugging-face library. For the closed-source models, we set the temperature to 0.7. In addition, we set the maximum output length to 1024.

\subsubsection{Prompt Strategy}
Taking into account the influence of language and shot, we design 9 prompting strategies, including Direct, CN-CoT, EN-CoT, CN-XLT and EN-XLT. The last four setups include both zero-shot and few-shot variants.\footnote{
For open-source models, Direct adopts an approach similar to the official implementation of HellaSwag, computing the log-likelihood for each option and selecting the one with the highest log-likelihood. And we report normalized accuracy that accounts for the impact of option length. Other prompting strategies use a generation setup and report accuracy based on exact match.}
\textbf {(1)Direct}: LLMs makes the selection directly without any CoT process.
\textbf{(2)CN-CoT}: LLMs performs CoT in Chinese, regardless of dataset language.
\textbf{(3)EN-CoT}: Similar to CN-CoT, but CoT is conducted in English. 
\textbf{(4)CN-XLT}: LLMs are instructed to first translate English questions and options to Chinese, and then reason in Chinese.
\textbf{(5)EN-XLT}: Similar to CN-XLT, but translates from Chinese dataset to English and reasons in English. 

%\textbf {CN-CoT}: LLMs perform Chinese reasoning and then output the answer and 3 shots are provided.
%\textbf {CN-CoT}: Similar as CNCoTFewShot without any shots.
%\textbf {EN-CoT}: The reasoning process in English is executed and then the answer is output and 3 shots are provided.
%\textbf {CN-XLT}: Inspired by this, we instruct LLMs to translate questions in Chinese and then output the answer after performing reasoning in Chinese too. And 3 shots are provided.
%\textbf {EN-XLT}: Inspired by this, we instruct LLMs to translate questions in Englsih and then output the answer after performing reasoning in Englsih too. Three shots are provided.

\subsubsection{Evaluation metric}

To comprehensively evaluate the robustness of each LLM, we consider four metrics: 
% Original Accuracy (\textbf{OA}), Average Robust Accuracy (\textbf{ARA}), Robust Loss Accuracy (\textbf{RLA}), and  Consistent Robust Accuracy (\textbf{CRA}).
\noindent %
\textbf{- Original Accuracy (OA)} measures accuracy on original problems.
\begin{equation}\label{eq1}
OA=\frac{\sum_{(x, y) \in D} \mathds{1}[L M(x), y]}{|D|}.
\end{equation}
\noindent %
\textbf{- Average Robust Accuracy  (ARA)} represents average accuracy across all variants, gauging overall performance on the robustness tasks.
\begin{equation}\label{eq2}
ARA=\frac{\sum_{\left(x^{\prime}, y^{\prime}\right) \in D_{R}} \mathds{1}\left(L M\left(x^{\prime}, y^{\prime}\right)\right.}{\left|D_{R}\right|}.
\end{equation}

\noindent %
\textbf{- Robust Loss Accuracy (RLA)} is the difference between ARA and OA, indicating performance degradation on robustness data versus original data.
%\begin{tiny}
%\begin{equation}\label{eq3}
%RLA=\frac{\sum_{\left(x^{\prime}, y^{\prime}\right) \in D_{R}} %\mathds{1}\left(L M\left(x^{\prime}, y^{\prime}\right)\right.}{\left|D_{R}\right|}-\frac{\sum_{(x, y) \in D}\mathds{1}[L M(x), y]}{|D|}
%\end{equation}
%\end{tiny}
\begin{equation}\label{eq3}
RLA= OA - ARA.
\end{equation}
\noindent %
\textbf{- Consistent Robust Accuracy (CRA)} shows accuracy when the model correctly answers both original and variant data, reflecting the model do understand the problem.
% consistency in problem-solving.
\begin{equation}\label{eq4}
CRA=\frac{\sum_{x, y, x^{\prime}, y^{\prime}}\mathds{1}[L M(x), y] \cdot \mathds{1}[L M(x^{\prime}), y^{\prime}]}{\left|D_{R}\right|}.
\end{equation}
For all equation above, $D$ denotes the original dataset, where $x$ represents the input question and options, and $y$ represents the correct label, while $D_{R}$ is the robust dataset with $x^{\prime}$ and $y^{\prime}$ representing similar to $x$ and $y$.


\begin{table*}[ht]
\centering
\setlength{\tabcolsep}{5pt}
% \footnotesize
\scalebox{0.6}{
% Please add the following required packages to your document preamble:
% \usepackage{multirow}
% \usepackage[table,xcdraw]{xcolor}
% Beamer presentation requires \usepackage{colortbl} instead of \usepackage[table,xcdraw]{xcolor}
% Please add the following required packages to your document preamble:
% \usepackage{multirow}
% \usepackage[table,xcdraw]{xcolor}
% Beamer presentation requires \usepackage{colortbl} instead of \usepackage[table,xcdraw]{xcolor}
\begin{tabular}{ccccccccccccc}
\hline
\multicolumn{1}{c|}{{ }}& \multicolumn{4}{c|}{Chinese}& \multicolumn{4}{c|}{English}& \multicolumn{4}{c}{AVG}\\ \cline{2-13} 
\multicolumn{1}{c|}{\multirow{-2}{*}{{ Model}}} & { OA(\%)$\uparrow$}& { ARA(\%)$\uparrow$} & {RLA(\%)$\downarrow$}& \multicolumn{1}{l|}{{CRA(\%)$\uparrow$}} & { OA(\%)$\uparrow$}& { ARA(\%)$\uparrow$} & { RLA(\%)$\downarrow$}& \multicolumn{1}{l|}{{CRA(\%)$\uparrow$}} & {OA(\%)$\uparrow$}& { ARA(\%)$\uparrow$} & {RLA(\%)$\downarrow$}& { CRA(\%)$\uparrow$} \\ \hline
\multicolumn{1}{c|}{{ Human}} & 96.41& 97.79& -1.38 & \multicolumn{1}{l|}{92.03}& 95.56& 96.04& -0.48 & \multicolumn{1}{l|}{90.02}& 95.99 & 96.92 & -0.93& 91.03 \\ \hline
\multicolumn{13}{c}{\textit{Close-source LLMs}}\\ 
\multicolumn{1}{c|}{{ GPT-4o}}& { 91.37} & { 81.97} & { 9.40}& \multicolumn{1}{l|}{{ 75.55}} & { \textbf{88.63}} & { \textbf{70.17}} & { \textbf{18.46}} & \multicolumn{1}{l|}{{ \textbf{63.06}}} & { 90.00} & { \textbf{76.07}} & { \textbf{13.93}} & { \textbf{69.31}} \\
\multicolumn{1}{c|}{{ Claude3.5}}& { \textbf{95.37}} & { 80.15} & { 15.22} & \multicolumn{1}{l|}{{ 75.04}} & { 85.11} & { 66.02} & { 19.08} & \multicolumn{1}{l|}{{ 57.20}} & { 90.24} & { 73.09} & { 17.15} & { 66.12} \\
\multicolumn{1}{c|}{{ Gemini-1.5-Pro}}& { 90.62} & { 78.36} & { 12.26} & \multicolumn{1}{l|}{{ 70.48}} & { 87.75} & { 60.74} & { 27.01} & \multicolumn{1}{l|}{{ 58.27}} & { 89.19} & { 69.55} & { 19.63} & { 64.38} \\
\multicolumn{1}{c|}{{ Qwen-Max}}& { 93.50} & { \textbf{84.82}} & { \textbf{8.68}}& \multicolumn{1}{l|}{{ \textbf{78.91}}} & { 87.60} & { 62.61} & { 24.99} & \multicolumn{1}{l|}{{ 59.65}} & { \textbf{90.55}} & { 73.72} & { 16.83} & { 69.28} \\ \hline
\multicolumn{13}{c}{\textit{Chinese open-source LLMs}} \\ 
\multicolumn{1}{c|}{{ Qwen2.5-0.5B}}& { 60.75} & { 45.18} & { \textbf{15.57}} & \multicolumn{1}{l|}{{ 28.70}} & { 49.50} & { 38.21} & { \textbf{11.29}} & \multicolumn{1}{l|}{{ 20.57}} & { 55.13} & { 41.70} & { \textbf{13.43}} & { 24.64} \\
\multicolumn{1}{c|}{{ Qwen2.5-1.5B}}& { 63.25} & { 46.16} & { 17.09} & \multicolumn{1}{l|}{{ 29.89}} & { 56.88} & { 39.57} & { 17.30} & \multicolumn{1}{l|}{{ 23.48}} & { 60.06} & { 42.87} & { 17.20} & { 26.69} \\
\multicolumn{1}{c|}{{ Qwen2.5-3B}}& { 67.50} & { 48.75} & { 18.75} & \multicolumn{1}{l|}{{ 33.79}} & { 61.75} & { 39.98} & { 21.77} & \multicolumn{1}{l|}{{ 25.75}} & { 64.63} & { 44.37} & { 20.26} & { 29.77} \\
\multicolumn{1}{c|}{{ Qwen2.5-7B}}& { 67.63} & { 50.59} & { 17.04} & \multicolumn{1}{l|}{{ 35.62}} & { 65.63} & { 43.93} & { 21.70} & \multicolumn{1}{l|}{{ 30.77}} & { 66.63} & { 47.26} & { 19.37} & { 33.20} \\
\multicolumn{1}{c|}{{ Qwen2.5-14B}} & { 69.00} & { 51.41} & { 17.59} & \multicolumn{1}{l|}{{ 35.84}} & { 68.50} & { 45.20} & { 23.30} & \multicolumn{1}{l|}{{ 32.12}} & { 68.75} & { 48.30} & { 20.45} & { 33.98} \\
\multicolumn{1}{c|}{{ Qwen2.5-32B}} & { 69.75} & { 53.11} & { 16.64} & \multicolumn{1}{l|}{{ 37.54}} & { 70.00} & { 46.10} & { 23.90} & \multicolumn{1}{l|}{{ 32.68}} & { 69.88} & { 49.61} & { 20.27} & { 35.11} \\
\multicolumn{1}{c|}{{ Qwen2.5-72B}} & { \textbf{70.87}} & { \textbf{54.75}} & { 16.12} & \multicolumn{1}{l|}{{ \textbf{39.64}}} & { \textbf{72.00}} & { \textbf{47.75}} & { 24.25} & \multicolumn{1}{l|}{{\textbf{ 35.12}}} & { \textbf{71.44}} & { \textbf{51.25}} & {20.19} & { \textbf{37.38}} \\ \hdashline[0.5pt/5pt]
\multicolumn{1}{c|}{{ Baichuan2-7B}}& { 67.00} & { 46.16} & { 20.84} & \multicolumn{1}{l|}{{ 31.50}} & { 60.62} & { 39.04} & { 21.58} & \multicolumn{1}{l|}{{ 25.21}} & { 63.81} & { 42.60} & { 21.21} & { 28.36} \\
\multicolumn{1}{c|}{{ Baichua2-13B}}& { 69.13} & { 46.98} & { 22.15} & \multicolumn{1}{l|}{{ 33.45}} & { 64.62} & { 38.82} & { 25.80} & \multicolumn{1}{l|}{{ 26.07}} & { 66.88} & { 42.90} & { 23.97} & { 29.76} \\ \hdashline[0.5pt/5pt]
\multicolumn{1}{c|}{{ DeepSeek-7B}} & { 68.13} & { 47.96} & { 20.17} & \multicolumn{1}{l|}{{ 33.30}} & { 63.38} & { 40.39} & { 22.99} & \multicolumn{1}{l|}{{ 26.70}} & { 65.76} & { 44.18} & { 21.58} & { 30.00} \\
\multicolumn{1}{c|}{{ DeepSeek-67B}}& { 71.50} & { 49.21} & { 22.29} & \multicolumn{1}{l|}{{ 35.89}} & { 71.37} & { 40.63} & { 30.75} & \multicolumn{1}{l|}{{ 29.71}} & { 71.44} & { 44.92} & { 26.52} & { 32.80} \\ \hdashline[0.5pt/5pt]
\multicolumn{1}{c|}{{ InternLM2.5-1.8B}}& { 61.62} & { 42.07} & { 19.55} & \multicolumn{1}{l|}{{ 26.99}} & { 55.37} & { 38.46} & { 16.91} & \multicolumn{1}{l|}{{ 22.61}} & { 58.50} & { 40.27} & { 18.23} & { 24.80} \\
\multicolumn{1}{c|}{{ InternLM2.5-7B}}& { 67.25} & { 49.77} & { 17.48} & \multicolumn{1}{l|}{{ 34.57}} & { 69.50} & { 40.89} & { 28.61} & \multicolumn{1}{l|}{{ 29.75}} & { 68.38} & { 45.33} & { 23.04} & { 32.16} \\
\multicolumn{1}{c|}{{ InternLM2.5-20B}} & { 67.37} & { 48.08} & { 19.29} & \multicolumn{1}{l|}{{ 33.21}} & { 73.62} & { 41.11} & { 32.51} & \multicolumn{1}{l|}{{ 31.23}} & { 70.50} & { 44.60} & { 25.90} & { 32.22} \\ \hdashline[0.5pt/5pt]
\multicolumn{1}{c|}{{ Yi-1.5-6B}} & { 67.00} & { 49.59} & { 17.41} & \multicolumn{1}{l|}{{ 34.27}} & { 64.38} & { 39.37} & { 25.01} & \multicolumn{1}{l|}{{ 26.62}} & { 65.69} & { 44.48} & { 21.21} & { 30.45} \\
\multicolumn{1}{c|}{{ Yi-1.5-9B}} & { 68.50} & { 50.18} & { 18.32} & \multicolumn{1}{l|}{{ 35.55}} & { 66.37} & { 39.58} & { 26.79} & \multicolumn{1}{l|}{{ 27.48}} & { 67.44} & { 44.88} & { 22.56} & { 31.52} \\
\multicolumn{1}{c|}{{ Yi-1.5-34B}}& { 71.00} & { 52.23} & { 18.77} & \multicolumn{1}{l|}{{ 38.09}} & { 71.00} & { 40.75} & { 30.25} & \multicolumn{1}{l|}{{ 29.91}} & { 71.00} & { 46.49} & { 24.51} & { 34.00} \\ \hline
\multicolumn{13}{c}{\textit{English open-source LLMs}} \\ 
\multicolumn{1}{c|}{{ Llama3-8B}} & { 59.13} & { 46.62} & { 12.51} & \multicolumn{1}{l|}{{ 28.23}} & { 66.25} & { 40.21} & { 26.04} & \multicolumn{1}{l|}{{ 27.34}} & { 62.69} & { 43.42} & { 19.27} & { 27.79} \\
\multicolumn{1}{c|}{{ Llama3-70B}}& { 65.75} & { 48.63} & { 17.12} & \multicolumn{1}{l|}{{ 32.70}} & { \textbf{72.50}} & { 41.27} & { 31.23} & \multicolumn{1}{l|}{{\textbf{ 30.63}}} & {\textbf{ 69.13}} & { 44.95} & { 24.18} & { 31.67} \\ \hdashline[0.5pt/5pt]
\multicolumn{1}{c|}{{ Mistral-7B-v0.2}} & { 57.75} & { 46.25} & { \textbf{11.50}} & \multicolumn{1}{l|}{{ 27.57}} & { 67.50} & { \textbf{41.52}} & { 25.98} & \multicolumn{1}{l|}{{ 28.93}} & { 62.63} & { 43.88} & { 18.74} & { 28.25} \\
\multicolumn{1}{c|}{{ Mixtral-8x7B-v0.1}} & { 63.62} & { 46.80} & { 16.82} & \multicolumn{1}{l|}{{ 30.82}} & { 69.75} & { 41.21} & { 28.54} & \multicolumn{1}{l|}{{ 29.39}} & { 66.69} & { 44.01} & { 22.68} & { 30.11} \\
\multicolumn{1}{c|}{{ Mixtral-8x22B-v0.1}}& { 66.00} & {\textbf{ 50.73}} & { 15.27} & \multicolumn{1}{l|}{{ \textbf{34.32}}} & { 72.12} & { 41.25} & { 30.87} & \multicolumn{1}{l|}{{ 30.61}} & { 69.06} & { \textbf{45.99}} & { 23.07} & { \textbf{32.47}} \\ \hdashline[0.5pt/5pt]
\multicolumn{1}{c|}{{ Gemma-2-2B}}& { 61.88} & { 45.38} & { 16.51} & \multicolumn{1}{l|}{{ 29.02}} & { 59.62} & { 39.13} & { \textbf{20.50}} & \multicolumn{1}{l|}{{ 24.88}} & { 60.75} & { 42.25} & {\textbf{ 18.50}} & { 26.95} \\
\multicolumn{1}{c|}{{ Gemma-2-9B}}& { \textbf{69.13}} & { 46.75} & { 22.38} & \multicolumn{1}{l|}{{ 33.29}} & { 64.88} & { 39.80} & { 25.08} & \multicolumn{1}{l|}{{ 26.91}} & { 67.01} & { 43.28} & { 23.73} & { 30.10} \\
\multicolumn{1}{c|}{{ Gemma-2-27B}} & { 63.38} & { 48.52} & { 14.86} & \multicolumn{1}{l|}{{ 31.96}} & { 71.88} & { 40.91} & { 30.97} & \multicolumn{1}{l|}{{ 30.25}} & { 67.63} & { 44.71} & { 22.92} & { 31.11} \\ \hline
\end{tabular}
}
\caption{TODO: bolded is not result. Results of existing LLMs on our HellaSwag-Pro dataset using \textbf{Direct} prompt. ``AVG'' indicates the average performance of each model on Chinese and English parts of the dataset.
The best results for each metric in each model category are \textbf{bolded}. }
\label{tab:main experiment.}
\end{table*}

\subsection{Model Performance (RQ1)}
\paragraph{Overall Performance}
Table \ref{tab:main experiment.} provides a comprehensive evaluation of various LLMs across four performance metrics\footnote{The results of instruct/chat models of Qwen2.5, Llama3 and Mixtral latest series are shown in Appendix.}. The main observations are as follow:
\begin{itemize}[leftmargin=*,topsep=0pt]
% \setlength{}{0}
    \item Upon evaluating all available models, we found that all performed well in overall accuracy (e.g., GPT-4 scored 90.00 in AVG OA, Claude 3.5 scored 90.24 in AVG OA). However, all models struggled with variations of the questions, as evidenced by a positive RLA value for each model. In contrast, humans received a negative RLA value, suggesting that the question variants were not more challenging than the originals. This disparity further illustrates that current LLMs lack a true understanding of the reasoning process and can easily be misled by question variants.
    \item When comparing open-source and close-source models, the close-source models demonstrate stronger capabilities in both OA and ARA scores, similar to most existing benchmarks. Overall, the RLA values for close-source models are also smaller, indicating that they are more robust in commonsense reasoning tasks compared to open-source models.
    \item When we compare models within the same series (e.g., Qwen, Llama), we observe that larger models often achieve higher scores on OA, ARA, and CRA. However, they are also more susceptible to variations, i.e., they have higher RLA values, a phenomenon particularly evident in English datasets. We attribute this phenomenon to the fact that larger models, compared to smaller ones, may have memorized more data, allowing them to rely on memorization to solve some problems more easily and making them more prone to the influence of variations~\cite{}.
\end{itemize}
% 1. When evaluating all available models, We find although 
% 2. When comparing the opensource LLMs and close source LLMs, 
% 3. When looking into each serious details
% \noindent
% \textbf{Overall Model Performance.}
% 1. close-source > open-source 2. the large the better 3. all have a performance decline when meeting varients.

% To evaluate the performance of various models, we observed patterns consistent with current mainstream trends: closed-source models generally outperform open-source models across metrics. 
% For instance, the closed-source model GPT-4o achieved scores of 90.00 in OA, 76.07 in ARA, and 69.31 in CRA, whereas the open-source model Qwen2.5-72B scored 71.44, 51.25, and 37.38, respectively. 
% Furthermore, within each model series, performance tends to improve with larger model sizes. 
% Nevertheless, even the strongest closed-source models struggle with variations in questions, as indicated by positive values in RLA for all models. In contrast, human performance yields a negative RLA value, highlighting that current LLMs do not genuinely grasp the reasoning process and are prone to falling into traps set by question variants. 
% This suggests that there is still significant room for improvement in developing models that can robustly understand and reason through complex linguistic challenges.
% It reveals a consistent pattern across Chinese, English, and average scores, with close-sourced LLMs generally outperforming open-sourced models. 
% However, all models exhibit a significant drop in performance when faced with robust variants, as indicated by RLA and CRA. Among closed-source models, GPT-4o demonstrates the highest ARA of 76.07\% in average scores, demonstrating its overwhelming superiority. Among open-sourced models, larger models tend to perform better, with Qwen2.5-72B achieving the highest OA (71.44\%) and ARA (51.25\%) in the average scores. However, even these top performers still struggle with robustness, as evidenced by the substantial RLA of 13.93\% for GPT-4o and 20.19\% for Qwen2.5-72B. Interestingly, some English open-sourced models, such as Llama3-70B and Mixtral-8x22B-v0.1, show competitive performance in English tasks but lag in Chinese tasks, highlighting the importance of language-specific training.

% \noindent
% \textbf{Chinese Models vs English Models.}
% Chinese models generally demonstrate higher OA in Chinese tasks compared to English tasks, with Qwen-Max achieving 93.50\% OA in Chinese versus 87.60\% in English. Conversely, English models tend to perform better in English tasks, exemplified by Llama3-70B's 72.50\% OA in English compared to 65.75\% in Chinese. 
% However, both Chinese and English models exhibit important drops in ARA across languages, indicating challenges in maintaining performance when faced with variations. This trend suggests that while models may excel in their primary language, they struggle with robustness across linguistic boundaries. 
% Notably, larger models tend to achieve higher ARA scores but also experience more substantial RLA, as seen with Qwen2.5-0.5B (41.70\% ARA, 13.43\% RLA in total) and Qwen2.5-72B (51.25\% ARA, 20.19\% RLA in total). 
% This pattern indicates that while increased model size enhances overall performance, it doesn't necessarily improve robustness proportionally. 
% The discrepancy between OA and ARA across languages underscores the need for improved cross-lingual robustness in language models, particularly as they scale in size and capability.


% \noindent
% \textbf{Comparison between Chinese and English datasets.}
% Generally, models demonstrate higher accuracy on the Chinese dataset compared to the English one, as evidenced by the consistently higher OA, ARA and CRA scores. For instance, GPT-4o achieves an OA of 91.37\%, an ARA of 81.97\% , an CRA of 75.55\% on the Chinese dataset, compared to 88.63\% and 70.17\% respectively on the English dataset. This trend is observed across most models, suggesting that the Chinese dataset is easier than English one. Moreover, the RLA values are typically lower for Chinese, indicating smaller performance drops when dealing with robust variants of Chinese questions. For example, Qwen-Max shows an RLA of 8.68\% for Chinese versus 24.99\% for English, highlighting a more consistent performance in Chinese. The CRA scores further reinforce this observation, with models generally maintaining higher consistency in correct answers for both original and variant Chinese questions.
% We attribute this phenomenon to the fact that blablabla

\noindent
\textbf{Reasoning Transferable Capability.}
% 为了进一步
To further analyze whether the model can transfer reasoning ability from the original question to its variant, Figure \ref{consis} presents the distribution of model performance on the original question and variant pairs. For all models, the pairs of (HellaSwag \ding{51} HellaSwag-Pro \ding{55}) occupy a significant proportion, indicating a challenge in transferring reasoning capabilities for current LLMs to more complex scenarios. Looking deeply, closed-source models like GPT-4 and Qwen-Max achieve around a 69\% portion of (HellaSwag \ding{51} HellaSwag-Pro \ding{51}) and a 3\% portion of (HellaSwag \ding{55} HellaSwag-Pro \ding{55}), while in contrast, open-source models struggle with around a 30\% portion of (HellaSwag \ding{51} HellaSwag-Pro \ding{51}) and a 20\% portion of (HellaSwag \ding{55} HellaSwag-Pro \ding{55}), further indicating the robustness of reasoning abilities in closed-source models.
% If a model can get both the original question and the variant right, we consider it to have transferable reasoning ability. Table \ref{consis} presents the distribution of model performance on the original question and variant pairs. Among all models, the pairs of (HellaSwag \ding{51}HellaSwag-Pro \ding{55}) account for a considerable proportion, i 
% The closed-source models like GPT-4o and Qwen-Max achieve around 69\% portion of (HellaSwag \ding{51}HellaSwag-Pro \ding{51}) and 3\% portion of (HellaSwag \ding{55} HellaSwag-Pro \ding{55}), indicating stronger reasoning transfer ability than other models. In contrast, open-source models struggle more, with around 30\% portion of (HellaSwag \ding{51}HellaSwag-Pro \ding{51}) and 20\% portion of (HellaSwag \ding{55} HellaSwag-Pro \ding{55}). 
% A notable trend is observed among the Qwen2.5 series, where increasing model size from 7B to 72B parameters correlates with improved performance on correct answers for both datasets (33.20\% to 37.38\%) and decreased failure rates (17.69\% to 14.7\%). It underscores the importance of model size in commonsense reasoning tasks.

\begin{figure}[t]
\centering
\setlength{\abovecaptionskip}{0.1cm}
\setlength{\belowcaptionskip}{0cm}
\includegraphics[width=\linewidth,scale=1.00]{images/consis.pdf}
\caption{Analysis of the transferable ability of model reasoning based on question pair performance. The green part, where both the original and the variant data are right, represents the transferable performance of model reasoning.}
\label{consis}
\vspace{-15pt}
\end{figure}

\begin{figure*}[ht]
\centering
\setlength{\abovecaptionskip}{0.1cm}
\setlength{\belowcaptionskip}{0cm}
\includegraphics[width=\linewidth,scale=1.00]{images/xing.pdf}
\caption{The impact of different few-shot prompts on model performance. With - as the separator, the first two parts of the legend represent the prompt name, and the third part represents the language of the dataset.}
\label{xing}
\vspace{-15pt}
\end{figure*}

\begin{figure}[ht]
\centering
\setlength{\abovecaptionskip}{0.1cm}
\setlength{\belowcaptionskip}{0cm}
\includegraphics[width=1.05\linewidth,scale=1.05]{images/zhu.pdf}
\caption{The RLA Distribution for 7 variants of commonsense reasoning. Parts below the 0 axis indicate that the model’s performance on the variant is improved compared to the original problem.}
\label{fig:zhu}
\vspace{-15pt}
\end{figure}


\subsection{Variant Analysis (RQ2)}
To further analyze the impact of different variants, we assessed the contribution of each variant to the RLA score. A higher contribution indicates that the model is more likely to make errors in that type. Figure~\ref{fig:zhu} presents the overall results, and the key observations are as follows:
\begin{itemize}[leftmargin=*]
    \item For problem restatement, causal inference, and sentence ordering, these three categories are the least challenging. Almost all models, particularly the close-source and Qwen series models, perform well on these variants, indicating that current LLMs can effectively handle these forms and we do not pay more attention on this kind of varients.
    \item For reverse conversion and critical testing, these two varients each contribute about 10\% to the RLA score. This indicates that current LLMs struggle to fully generalize to these simple scenarios, possibly because these types of questions are not commonly encountered, and reaserchers should pay some attention to this type of varients.
    \item For negative transformation and scenario refinement, this are the two most difficult tasks, with negative transformation being particularly challenging. For almost all models, these two varients accounts for more than 50\% of the RLA score. This may be due to intuitively counterintuitive questions—such as the use of "will not"  or counterfactual scenarios in scenario refinement. These setups are less common in LLM training data and cannot be easily tackled through memory alone. Only those LLMs which truely understand the question could answer the varient correctly, wihch better reflect the true performance of the model.. In the future, researchers should focus more on enhancing LLM's capability to address such types of questions.
\end{itemize}

% 1. Problem restCausal Inference 
% To further analysis the impact of different varients, we further 
% Figure \ref{fig: zhu} presents a comprehensive analysis of various LLMs' performance across different variant types. Negative transformation emerges as the most challenging task for all models, with scores consistently above 50.00\% and peaking at 78.38\% for Gemini-1.5-Pro. Conversely, problem restatement appears to be the least challenging, with most models scoring in the negative range. Intriguingly, smaller models like Qwen2.5-0.5B demonstrate unexpected strengths in certain areas, such as sentence sorting (7.75\%), outperforming some larger counterparts. A detailed analysis of each variant type follows.

% \noindent
% \textbf{Causal inference.} In this category, scores vary widely from -4.73\% for Qwen-Max to 12.25\% for Baichuan2-13B, illustrating differing degrees of sensitivity to causal reasoning among the models. Smaller models, such as Qwen2.5-0.5B and Qwen2.5-1.5B, achieve better scores, indicating relatively stronger robustness in causal reasoning. Conversely, larger models, like Baichuan2-13B, have higher scores, suggesting greater sensitivity to the challenges of inferring causality.

% \noindent
% \textbf{Critical testing.} Larger models, including Qwen2.5-72B and DeepSeek-67B, exhibit higher RLA scores of 30.50\% and 31.37\%, respectively, suggesting increased sensitivity when dealing with incomplete key information. In contrast, GPT-4o achieves the lowest score, highlighting its superior robustness in critical reasoning. This trend indicates that more complex models might struggle to handle incomplete contexts, underscoring potential areas for improvement in sophisticated architectures.

% \noindent
% \textbf{Negative transformation.} This aspect remains consistently challenging for all models, with scores ranging from 48.88\% to 78.38\%. Advanced commercial models like Gemini-1.5-Pro and Claude-3.5 also score higher (78.38\% and 76.43\%, respectively), indicating a prevalent sensitivity issue in reasoning processes when handling negations, irrespective of model size or architecture.

% \noindent
% \textbf{Problem restatement.} The negative values in this category for nearly all models suggest it is not particularly challenging. This is surprising, given that previous models were quite sensitive to sentence representation.

% \noindent
% \textbf{Reverse conversion.} This variation, which involves swapping the roles of the question and answer, seems to specifically impact larger models. For example, Qwen2.5-72B and DeepSeek-67B exhibit higher RLA scores of 24.38\% and 27.43\%, respectively, indicating heightened sensitivity to reverse reasoning compared to their performance on original questions.

% \noindent
% \textbf{Scenario refinement.} The scores range from 16.06\% for Gemma-2-2B to 32.56\% for Qwen2.5-72B, with larger models displaying more sensitivity in adapting to counterfactual predictions. This suggests that larger models may rely more heavily on general commonsense rather than flexibly adapting to specific contexts. Consequently, increased model complexity might adversely affect adaptability to scenario changes, underscoring the need for enhanced flexibility in advanced models.

% \noindent
% \textbf{Sentence sorting.} This category exhibits the most varied results across models. Some larger models like DeepSeek-67B and InternLM2.5-20B display higher scores (26.69\% and 26.68\%), indicating sensitivity, while others like Qwen2.5-72B and Gemini-1.5-Pro excel with lower scores (-9.88\% and -1.07\%, respectively). This suggests that sentence sorting ability may depend more on specific training approaches rather than being solely contingent on model size.


\subsection{Prompt Robustness (RQ3)}
% To investigate how prompt  influence our benchmark, we apply sereral prompt strategy on our datasets and showcase the average performance of all models on different kind of prompt strategies.
% Table~\ref{prompt} illustrates the final results. For both Chinese and English datasets, CN LLMs achieve the highest performance using CN-CoT-Few-Shot, followed closely by EN-CoT-Few-Shot, with overall performance scores of 67.36\% and 67.03\%, respectively. In contrast, English LLMs perform best with the EN-CoT-Few-Shot, reaching 67.55\% on the Chinese dataset and 60.36\% on the English dataset.
% Contrary to previous findings, translating the dataset to the model's advantage language before performing reasoning does not enhance performance. Moreover, Figure~\ref{xing} also shows the similar phenomenon. Conducting CoT reasoning in the model’s advantage language generally leads to better outcomes compared to Direct. Additionally, increasing the number of shots consistently improves performance across most configurations, highlighting the benefits of exposing models to multiple examples. 
To explore the impact of various prompt strategies on our benchmarks, we evaluated several approaches across our datasets and present the average performance of all models using different prompting techniques. Table~\ref{prompt} summarizes the results. For both Chinese and English datasets, Chinese LLMs performed best with the CN-CoT-Few-Shot strategy, followed closely by EN-CoT-Few-Shot, achieving overall scores of 67.36\% and 67.03\%, respectively. Conversely, English LLMs showed optimal performance with the EN-CoT-Few-Shot approach, attaining 67.55\% on the Chinese dataset and 60.36\% on the English dataset.
Besides, translating datasets into the model's native language before reasoning did not enhance performance. This phenomenon is further illustrated in Figure~\ref{xing}. Conducting CoT reasoning in the model's native language generally yields better results compared to direct reasoning. Furthermore, increasing the number of examples (shots) consistently boosts performance across most configurations, emphasizing the advantages of exposing models to multiple examples.
% Overall, the interaction between question language, prompt language, and the number of shots underscores the importance of aligning these factors to optimize task performance and robustness in LLMs.



% Please add the following required packages to your document preamble:
% \usepackage{multirow}
% Please add the following required packages to your document preamble:
% \usepackage{multirow}
\begin{table}[t]
\setlength{\tabcolsep}{8pt}
% \footnotesize
\scalebox{0.65}{
\begin{tabular}{c|l|lll}
\hline
\multicolumn{1}{l|}{Dataset}  & Prompt  & CN LLMs & EN LLMs &  LLMs \\ \hline
\multirow{7}{*}{\begin{tabular}[c]{@{}c@{}}Chinese\\ HellaSwag-Pro\end{tabular}} & Direct  & 48.95& 41.16& 45.06  \\
& CN-CoT-Few  & \textbf{71.04}& 51.90& 61.47  \\
& EN-CoT-Few  & 70.95& \textbf{67.55}& \textbf{69.25}  \\
& EN-XLT-Few  & 41.48& 28.69& 35.09  \\
& CN-CoT-Zero & 44.82& 23.89& 34.36  \\
& EN-CoT-Zero & 45.38& 31.39& 38.39  \\
& EN-XLT-Zero & 28.57& 12.93& 20.75  \\ \hline
\multirow{7}{*}{\begin{tabular}[c]{@{}c@{}}English\\ HellaSwag-Pro\end{tabular}} & Direct  & 47.46& 40.66& 44.06  \\
& CN-CoT-Few  & \textbf{63.67}& 47.24& 55.46  \\
& EN-CoT-Few  & 63.12& \textbf{60.36}& \textbf{61.74}  \\
& CN-XLT-Few  & 48.77& 16.61& 32.69  \\
& CN-CoT-Zero & 34.89& 18.25& 26.57  \\
& EN-CoT-Zero & 42.41& 31.03& 36.72  \\
& CN-XLT-Zero & 16.36& 11.22& 13.79  \\ \hline
\multirow{9}{*}{HellaSwag-Pro}& Direct  & 48.21& 40.91& 44.83  \\
& CN-CoT-Few  & \textbf{67.36}& 49.57& 58.46  \\
& EN-CoT-Few  & 67.03& \textbf{63.95}& \textbf{65.49}  \\
& CN-XLT-Few  & 59.91& 34.26& 47.08  \\
& EN-XLT-Few  & 52.30& 44.52& 48.41  \\
& CN-CoT-Zero & 39.86& 21.07& 30.46  \\
& EN-CoT-Zero & 43.90& 31.21& 37.55  \\
& CN-XLT-Zero & 30.59& 17.55& 24.07  \\
& EN-XLT-Zero & 35.49& 21.98& 28.74  \\ \hline
\end{tabular}
}
\caption{Average ARA of all open-source models on different prompts. CN-LLMs contains 17 LLMs, and EN-LLMs contains 7 LLMs. The bast results for each dataset are \textbf{bolded}.}
\label{prompt}
\end{table}





\section{Conclusion}
This paper introduces ACCESS, a benchmark for abstract causal event discovery and reasoning. We present a pipeline that combines automatic methods and human crowd-sourcing to extract $1,494$ causal relations among $725$ abstract events. We demonstrate that incorporating causal knowledge from our benchmark leads to improvements in QA reasoning tasks for LLMs. However, we also highlight challenges in automatic event abstraction identification and causal discovery, where in the latter, the popular statistical algorithms perform poorly in recovering our sub-graphs of fewer than $50$ nodes.  Our empirical evidence also suggests that LLMs are not ready to perform causal inference effectively due to the lack of effective acquisition of two critical sub-processes: abstract reasoning and causal discovery. This underscores the need for future research to equip the models with these essential skills for achieving true causal reasoning.

\section*{Limitations}
Our benchmark is built upon \texttt{GLUCOSE} \citep{mostafazadeh-etal-2020-glucose} whose scope is limited to everyday children's stories. Acknowledging this limitation, we propose a reproducible data construction pipeline applicable for curating diverse corpora of event causality.  Since \texttt{ACCESS} primarily addresses commonsense knowledge in real-world events, it is susceptible to biases regarding the judgement of semantic similarity and cause-and-effect relation of events. To mitigate this issue, our first effort is at every phase, to employ automatic methods alongside with human annotation, based on a set of objective definitions and criteria about events, abstractions and event causality. In the event abstraction phase, we specifically provide the annotators with a list of common scenarios (though non-exhaustive) indicating when the semantics of two expressions are considered similar of different to reduce potential biases.  Regarding the subjectivity in human causal judgment, while we focus on non-contextual causal commonsense knowledge, we leverage contextual signals in the original corpus whenever necessary to objectively guide the annotators' decisions on the causal relations. Due to the resource constraints, our causal graph is sparse and limited in size, which however still presents a challenge for statistical structure learning as well as LLMs on causal discovery tasks. One critical drawback in the experiment with statistical methods lies in the representation power of the co-occurrence matrix, which underscores the need for further research on representation learning of abstractions in language domain. As above, future works could also explore other resources to enlarge our causal graph and expand the coverage of real-world data. Such a causal graph could further be leveraged for causal inference according the engine described by Pearl \citep{pearl2009causality}, which seeks to answer causal queries across the three rungs of the Ladder of Causation i.e., associational (Rung 1), interventional (Rung 2), and counterfactual (Rung 3).

\section*{Ethics Statement}
To address potential misuse and uphold fairness and inclusivity, we discuss several ethical considerations for \texttt{ACCESS}. Firstly, it is crucial to clarify that the resources provided in this work are solely intended for research purposes. The narrative scenarios within \texttt{ACCESS} should not be utilized for insults, slander, or any other malicious purposes. Users are expected to adhere to the highest ethical standards, ensuring responsible and transparent use in line with ethical research practices. The creators of the dataset hold no responsibility for misuse or misinterpretation, and all necessary measures have been taken to respect privacy and ensure informed consent during the data collection process. Secondly, it is imperative to acknowledge the mental well-being of annotators during the data annotation process. Prior to data collection, this study underwent a thorough review and approval process by an internal review board. We require each annotator to take a break every two hours or whenever they feel uncomfortable.

\section*{Acknowledgment}
This material is based on research sponsored by DARPA under agreement number HR001122C0029. The U.S. Government is authorized to reproduce and distribute reprints for Governmental purposes notwithstanding any copyright notation thereon This work is also partially supported by the DARPA Assured Neuro Symbolic Learning and Reasoning (ANSR) program under award number FA8750-23-2-1016.

\bibliography{ref}
\clearpage

\appendix

\section{Related Work}
\section{Related Work}
\label{sec:related_work}

The original investigation \cite{gibson1979ecological} on the relationship between visual perception and human action defines \emph{affordance} as the opportunities for interaction with the surrounding environment. Behavioral studies on regular and cognitively impaired persons have shown evidence that perception results in both visual and motor signals in the human brain. An extended study \cite{anderson2002attentional} shows that visual attention to the spatial characteristics of the perceived objects initiates automatic motor signals for different actions. In computer vision, human affordance learning involves novel pose prediction such that the estimated pose represents a valid human action within the scene context. The task is fundamental to many problems requiring robust semantic reasoning about the environment, such as human motion synthesis \cite{wang2021scene} and scene-aware human pose generation \cite{wang2017binge, roy2016multi, zhang2022inpaint, yao2023scene}.

Earlier methods of affordance learning have explored knowledge mining \cite{zhu2014reasoning} and multimodal feature cues \cite{roy2016multi} to address the problem. In \cite{zhu2014reasoning}, the authors use a Markov Logic Network for constructing a knowledge base by extracting several object attributes from different image and metadata sources, which can perform various downstream visual inference tasks without any additional classifier, including zero-shot affordance prediction. In \cite{roy2016multi}, the authors use depth map, surface normals, and segmentation map as multimodal cues to train a multi-scale convolutional neural network (CNN) for scene-level semantic label assignment associated with specific human actions. In \cite{do2018affordancenet}, the authors design a multi-branch end-to-end CNN with two separate pathways for object detection and affordance label assignment to achieve high real-time inference throughput. Researchers \cite{chuang2018learning} have also explored socially imposed constraints for affordance learning. In \cite{chuang2018learning}, the authors propose a graph neural network (GNN) to propagate contextual scene information from egocentric views for action-object affordance reasoning.

Probabilistic modeling of scene-aware human motion generation also involves semantic reasoning of human interaction with the environment. Initial works on human motion synthesis have taken different architectural approaches, such as sequence-to-sequence models \cite{barsoum2018hp}, generative adversarial networks (GAN) \cite{barsoum2018hp, cai2018deep, yang2018pose}, graph convolutional networks (GCN) \cite{yan2019convolutional}, and variational autoencoders (VAE) \cite{guo2020action2motion}. However, these methods have mostly ignored the role of environmental semantics. Due to potential uncertainty in human motion, in a recent approach \cite{wang2021scene}, the authors address such motion synthesis with a GAN conditioned on scene attributes and motion trajectory to predict probable body pose dynamics.

One key challenge of human affordance generation in 2D scenes is the lack of large-scale datasets with rich pose annotations. In \cite{wang2017binge}, the authors compile the only public dataset of annotated human body poses in complex 2D indoor scenes by extracting frames from sitcom videos. Aiming to generate a contextually valid human affordance at a user-defined location, the authors propose sampling the scale and deformation parameters for an existing human pose template using a VAE conditioned on the localized image patches as scene context. In \cite{zhang2022inpaint}, the authors introduce a two-stage GAN architecture for achieving a similar goal by estimating the affine bounding box parameters to localize a probable human in the scene and then generating a potential body pose at that location. The method uses the input scene, corresponding depth, and segmentation maps as semantic guidance. In \cite{yao2023scene}, the authors propose a transformer-based approach with knowledge distillation for generating human affordances in 2D indoor scenes.



\section{Data Annotation Pipeline}\label{sup:annotation}
% \begin{table*}
% \centering
% \begin{tabular}{p{2.5cm} p{4.2cm}| p{2.5cm} p{5cm}}
% \toprule
% \multicolumn{2}{c}{\textbf{Cause event}} & \multicolumn{2}{c}{\textbf{Effect event}} \\
% \midrule
% Abstraction & Generalizations & Abstraction & Generalizations \\
% \midrule
% \textit{a person} & a person need money &  \textit{a person get} & a person take up a job  \\
% \textit{need money} & a person need cash &  \textit{a job} & a person get a good job  \\
%  & a person need to get money &  & a person get a job at a place  \\

% \midrule
% \textit{a person win} & a person win the contest & \textit{a person} & a person be celebrate an occasion  \\
%  & a person win something &  \textit{celebrate} & a person have a celebration  \\
 
% & a person end up winning &   & a person celebrate something  \\
% \midrule
% \textit{a person fall} & a person fall down & \textit{a person feel} & a person be in pain  \\
%  & a person fall to the floor &  \textit{pain} & a person experience pain in a body part  \\
 
% & a person fall on the ground &  & a person 's body be in pain \\
% \bottomrule
% \end{tabular}
% \caption{Examples of event causality on the abstraction and generalization level.}\label{tab:example}
% \end{table*}

We recruit in total $13$ university students in Malaysia aged $20-30$. The total hours are  $329.7$, where the hourly rate is RM$20$ (Malaysian ringgit), which is higher than the minimum wage of RM$7.1$.  

As for the annotation guidelines, we translate the technical terminologies in Section \ref{sec:setup} into layman language comprehensible to human annotators.  


\subsection{Abstract Event Extraction}\label{sup:annotation_ph1}
There are five steps in this annotation phase. Steps $1$ and $2$ are key to extracting abstract events, whereas Steps $3-5$ serve as post-processing to strengthen consistency among human annotators. 

\paragraph{Step 1: Sub-clustering.}
Each annotator is presented with a set of clusters generated from an automatic clustering algorithm. Each cluster contains multiple English sentences that describe events in daily life. Each word in every sentence is lemmatized to its base form so that the tense of the sentence does not influence the judgment of meaning. For every cluster, they are required to sub-group event sentences that are semantically similar or related together. There can exist clusters in which all sentences are related to one another; in this case no sub-clustering is needed. There can also be outlier events i.e., sentences that do not belong to any sub-clusters. For a sub-cluster to exist, it must contain at least two events. If an event cannot be sub-clustered, the annotator is to classify it as an outlier.	If a sentence is lexically or grammatically erroneous that makes it unjustifiable, the annotator is also asked to highlight and correct it whenever appropriate before clustering. 


Two event sentences are considered \textit{semantically related} or \textit{similar}\footnote{We use ``='' to denote semantic similarity and ``$\ne$" to denote semantic dissimilarity between two events.} if they describe the same event, and the decision must not be affected by the information about \texttt{location} and \texttt{time}. We note there is a difference between a \texttt{state/action} actually taking place with the prospect of the \texttt{state/action} taking place. In particular, we outline $11$ scenarios where word uses convey differences in meaning.	

\begin{enumerate}
    \item single participant vs. group of participants e.g., \emph{a person be playing in the park} $\ne$ \emph{a person and another person be playing in the park.}
    
    \item affirmation vs. negation e.g., \emph{a person be asleep}  $\ne$ \emph{a person do not sleep.}

    \item present vs. future tense e.g., \emph{a person go to sleep} $\ne$ \emph{a person will go to sleep.}

    \item ability e.g., \emph{a person do not eat} $\ne$ \emph{a person cannot eat.}

    \item intention or desire e.g., \emph{a person do not eat} $\ne$ \emph{a person do not want to eat.}

    \item deduction or possibility e.g., \emph{it rain} $\ne$ \emph{it may rain.}. 
    
    \item obligation, advice or prohibition	e.g., \emph{a person do not eat} $\ne$ \emph{a person should not eat}.

    \item offers, effort or decision e.g., \emph{a person help another person} $\ne$ \emph{a person offer to assist another person}; \emph{a person go to the gym} $\ne$ \emph{a person decide to go to the gym.}

    \item location as object. In some cases, the object receives an action from the verb refers to a place or location e.g., \emph{a person clean a place}. Here \textit{room} is considered an (spatial) item being taken action on and similar to any other items such as cup or a table $\rightarrow$ \emph{a person clean a place} \textcolor{red}{=} \emph{a person clean something.}

    \item multiple actions. Some sentences describe two actions happening at the same time e.g., \emph{a person take something and leave}. In order to evaluate its meaning, one  must select one of them to the key action. The key action is the action that is described by most of other events in the same cluster. This means that if most of the other events are about \textit{someone leaving somewhere}, the \textit{leave} action should be focused instead of \textit{take} action. 

    \item continuous vs. simple tense. Some sentences describe actions in the continuous state e.g., \emph{a person be go home}. We ignore the continuous state of the action and consider them equivalent to the action described simple tense $\rightarrow$  \emph{a person be go home} \textcolor{red}{=} \emph{a person go home.}	
\end{enumerate}

	
\paragraph{Step 2 : Topic identification.}

In this step, the annotator asked to identify the topic for every cluster or sub-cluster formed. The topic must first be an event, therefore it must contain at least two components: \texttt{participant(s)} and \texttt{action}. The topic must be specific about the state or action that takes place. At the same time, the topic must be written in a way that makes it general or abstract enough to include all event sentences.	Whenever possible, it is acceptable to use the most representative event sentence in a cluster as the topic.	

\paragraph{Intermediate processing.} In Steps $1$ and $2$, we divide the collection of clusters into $7$ batches. Each of the batch contains $60$ clusters and 
two workers are asked to annotate one same batch of clusters. This results in one cluster having two annotation results. Subsequently, an algorithm is run to automatically unify the results from two annotators. For every cluster in the original data, the algorithm starts by randomly selecting an event as a centroid. It then forms a sub-cluster around the centroid that contains all other events that are considered by both annotators to be semantically related to the centroid. The topic assigned to that sub-cluster is presented in the format \texttt{TOPIC : [Text 1] / [Text 2]} where \texttt{[Text 1]} is the topic assigned to events in this sub-cluster by the first annotator and \texttt{[Text 2]} is the topic assigned to them by the second annotator. Repeat the process with the other events until all instances are processed. Thereafter, any event that is not assigned to any cluster will exist as a stand-alone instance and temporarily be considered an outlier. 

The next steps focus on resolving the disagreements from two annotation results, which includes \textbf{Topic alignment} and \textbf{Outliers processing}. We assume that a sub-cluster is properly annotated if it (1) contain at least $2$ instances and (2) no annotators consider that sub-cluster to be an outlier.

\paragraph{Step 3: Topic alignment.}
Every cluster is now annotated with two topics. If both topics describe the same event, the annotator is asked to choose either or the one more representative. Otherwise, choose the one that fits most of the sentences in the cluster. If the chosen topic is already assigned to some previous cluster, merge the current cluster into that cluster. If at least one of the topics is Outliers (i.e., at least one annotator considers the sub-cluster as Outliers), temporarily view them as Outliers. 

\paragraph{Step 4: Outliers processing.}
The annotator moves on to process the outliers. For any event that is assigned by only one of the previous annotators to be outliers while assigned by the other to be associated some existing sub-cluster, the annotator is asked to merge it into the assigned sub-cluster if the event can be represented by the topic of that sub-cluster; otherwise, keep it as an outlier. For any event that is agreed by both annotators to be an outlier, the current annotator is asked to re-examine it for possible assignment to any existing sub-cluster. The merging decision must be again based on the conditions described in Step $1$. Any remaining stand-alone instances are discarded. 

\paragraph{Step 5: Topic matching.} This step aims to correct for potential mis-clustering from the automatic procedure. We obtain the outlier events and attempt re-categorize them into the post-annotated clustering results from all above steps. For each outlier, we present the annotators with a set of candidate clusters to which adding the outlier would not violate causal consistency. We ask them to select one cluster with whose topic the outlier is most semantically similar. The rules to determine semantic similarity of a sentence pair follows from Step 1. It is possible that there is no topic that matches the outlier. If there is any topic that is a word-by-word exact match, that topic must be selected. We also add another rule that requires the annotators to select the topic with the same level of abstraction (generality) or concreteness (specificity) as the outlier event, since there are some topics that are abstract or concrete versions of other topics. More specifically, if the outlier is concrete but the concrete topic is not presented for selection, select the abstract topic. If the outlier is abstract but the abstract topic is not presented for selection, the concrete topic must \underline{not} be selected. 

\subsection{Causal Relations Discovery}\label{sup:annotation_ph2}
The annotator is tasked with evaluating candidate pairs of clusters to determine whether a cause-and-effect relationship exists between them, based on their respective topics. Since each cluster's topic represents an event abstraction, and in essence, an event itself, the decision on causal relation hinges on whether the two topics describe causally linked events. Based on the cause-effect definition in Section \ref{sec:setup}, we provide them with the following criteria to guide their decision about whether an event $A$ causes another event $B$:
\begin{enumerate}[leftmargin=5.5mm]
    \item a causal relation must be temporal, but a temporal relation is \underline{not} always causal;

    \item the action/state of $A$ directly leads to the action/state of $B$ i.e., there must be no intermediate events or if there is one, it should be extremely rare in real-world scenarios;
    
    \item an event $B$ would not occur if $A$ did not occur.
\end{enumerate}

Initially, the workers provide non-contextual annotations based solely on their commonsense understanding of the abstractions. A relation is deemed valid if the annotator can envision a plausible scenario in daily life where the situation occurs frequently, commonly, and is highly likely. If no such scenario comes to mind, the clusters are considered unrelated. In the subsequent step, we identify the highly disagreed pairs, where the three annotators each make distinct decisions regarding causality i.e., $A$ causes $B$, $B$ causes $A$, $A$ and $B$ are unrelated. For these pairs, workers are presented with contextual information from stories in \texttt{GLUCOSE} and asked to reconsider their decisions. The final determination of the relationship is made through majority voting.




\section{Clustering Algorithm}\label{sup:clustering}
Our clustering algorithm, named \texttt{PIVOT}, is inspired by the pivoting algorithm proposed in \citet{fukunaga2019lp}. The \texttt{PIVOT} algorithm first randomly selects a pair of cause-effect events and then, for each of them, find its most similar neighbors against a threshold of $70\%$. We repeat the process for the remaining event mentions, while excluding the previously assigned events. The initial results are passed to the following process to remove self-loop and bi-directions. We remove clusters with fewer than $10$ samples and maximum pairwise similarity is less than $50\%$. Each cluster can now be considered a node in a graph and we use \texttt{GLUCOSE} to recover the causal relations among them to construct a temporary causal graph. 

\paragraph{Ablation study.} The main motivation behind \texttt{PIVOT} algorithm is to ensure the initial graph is mostly acyclic while avoiding any sub-optimality produced from post-processing. To validate whether \texttt{PIVOT} is most effective in ensuring causal consistency, we conduct an ablation study against popular clustering algorithms, including \texttt{OPTICS} \cite{ankerst1999optics}, \texttt{LOUVAIN} \cite{blondel2008fast} and \texttt{LEIDEN} \cite{traag2019louvain} algorithms, where \texttt{LOUVAIN} and \texttt{LEIDEN} were proposed for community detection problems. The criteria for selecting these clustering algorithms include: 
(1) scalability to medium-to-large-sized data, (2) ability to accommodate custom affinity matrix and (3) high cluster homogeneity score. Table \ref{tab:clustering} further reports the quality of the algorithms under analysis, which shows that our \texttt{PIVOT} algorithm yields the most desirable performance. 

 
\paragraph{Notations.} We use lower case letters (i.e., $v$) to denote single event, capital letters (i.e., $V$) for cluster of events, and blackboard bold letter (i.e., $\mathbb{V}$) for set of clusters. We let $\mathcal{D}$ denote the dataset of causal event mentions; $x \rightarrow y$ indicates event $x$ is a cause of event $y$; $x \leftarrow y$ indicates event $x$ is an effect of event $y$; $x \leftrightarrow y$ indicates $x$ and $y$ are causally related (either cause or effect). We also define the similarity between an event $y$ and cluster $V$ as the average of similarity between $y$ and every event $x$ in $V$
$$S_{yV} = \frac{1}{|V|} \sum_{x \in V} S_{xy},$$
where $S_{xy}$ is the similarity score between two events according to Eq. (\ref{eq:sim}).


\paragraph{Performance metrics.} In the following, we describe the unsupervised performance metrics to assess clustering algorithms in Table \ref{tab:clustering}. Given a set of clusters $\mathbb{C}$, let $\boldsymbol{A}$ be the matrix where $\boldsymbol{A}_{ij}$ is the number of events in cluster $C_i \in \mathbb{C} $ is the cause of any event in the cluster $C_j \in \mathbb{C}$. Recall that in this stage the causal relations between events are extracted from \texttt{GLUCOSE} dataset. A cluster $A$ is said to cause another cluster $B$ if \underline{at least one} event mentions in cluster $A$ causes any other event mentions in cluster $B$, according to the cause-effect definition in Section \ref{sec:setup}. 


\begin{enumerate}
    \item \textit{Self-loop ratio}: Proportion of clusters in which the events are either cause or effect of each other. 
    $$\frac{1}{|\mathbb{C}|} \sum_{i=1}^{|\mathbb{C}|} \frac{\boldsymbol{A}_{ii}}{2 |C_i|}.$$

    \item \textit{Bi-directional ratio:} Proportion of cluster pairs that are both cause and effect of one another.  
    $$\frac{2}{|\mathbb{C}|^2 - |\mathbb{C}|} \sum_{i=1}^{|\mathbb{C}|-1} \sum_{j=i+1}^{|\mathbb{C}|} \frac{\min (\boldsymbol{A}_{ij}, \boldsymbol{A}_{ji})}{\max (\boldsymbol{A}_{ij}, \boldsymbol{A}_{ji})}.$$

    \item \textit{Silhouette coefficient \citep{rousseeuw1987silhouettes}:}  Measure of how similar an instance is to its own cluster (cohesion) compared to other clusters (separation). A high value indicates that the object is well matched to its own cluster and poorly matched to neighboring clusters.  
    $$\frac{1}{|\mathcal{D}|} \sum_{x \in \mathcal{D}} \frac{a_x - b_x}{1 - \min(a_x,b_x)},$$
    
    where $a_x$ is the mean similarity between event $x$ and all other events in the same cluster; $b_x$ is the mean similarity between event $x$ and all other events in the next nearest cluster.

    \item \textit{Homogeneity score:} Average pairwise similarity of events in a cluster.
    $$\frac{1}{|\mathbb{C}|} \sum_{i=1}^{|\mathbb{C}|} \frac{2}{|C_i|^2-|C_i|}\sum_{x,y \in C_i, x \ne y}S_{xy},$$
    
    where $S_{xy}$ is the similarity score between two events according to Eq. (\ref{eq:sim}).
\end{enumerate}

Table \ref{tab:clustering_abs} reports the numerical results for the experiment on Abstract Event Identification in Section \ref{sec:abstraction_exp}. For the supervised metrics, we refer readers to \href{https://scikit-learn.org/stable/modules/clustering.html}{\texttt{scikit-learn}'s} documentation for the precise formulations and implementations of \textit{Adjusted Rand Index} \citep{steinley2004properties} and \textit{Normalized Mutual Information} \citep{vinh2009information}. 

\begin{table*}[hbt!]
\centering
% \resizebox{\columnwidth}{!}{
\begin{tabular}{l|r r r r}
\toprule
Metrics       & \textbf{LOUVAIN} & \textbf{LEIDEN} & \textbf{OPTICS} & \textbf{PIVOT}   \\
\midrule
Bi-directional ratio $\downarrow$               & $0.179$    & $0.162$   & $0.011$   & $\mathbf{0.004}$  \\
Self-loop ratio    $\downarrow$                & $0.252$    & $0.361$   & $0.007$   & $\mathbf{0.001}$  \\
Silhouette coefficient (Euclidean) $\uparrow$ & $-0.120$   & $-0.137$  & $-0.252$  & $\mathbf{-0.015}$ \\
Silhouette coefficient (Cosine) $\uparrow$    & $-0.234$   & $-0.262$  & $-0.392$  & $\mathbf{-0.036}$ \\
Homogeneity score $\uparrow$  & $0.506$  & $0.577$  & $0.810$   & $\mathbf{0.907}$ \\ 
\bottomrule
\end{tabular}
% }
\caption{Evaluation of alternative clustering algorithms. \textbf{Bold} indicates best performance. $\uparrow$ Higher is better. $\downarrow$ Lower is better.} \label{tab:clustering}
\end{table*}


\begin{table*}[hbt!]
    \centering
   % \resizebox{\columnwidth}{!}{
\begin{tabular}{l|r r r r}
\toprule
Metrics       & \textbf{LOUVAIN} & \textbf{LEIDEN} & \textbf{OPTICS} & \textbf{PIVOT}$^{(*)}$   \\
\hline
\multicolumn{5}{c}{\cellcolor[HTML]{C0C0C0}\textbf{Generalizations from \texttt{GPT-4o-mini}}} \\ \toprule
% Silhouette coefficient (Euclidean) $\uparrow$               & $-0.0139$    & $0.0005$   & $-0.1321$   & $\mathbf{0.0173}$          \\
% Silhouette coefficient (Cosine)   $\uparrow$            & $-0.0531$    & $-0.0255$   & $-0.2168$   & $\mathbf{0.0137}$  \\
Adjusted rand index $\uparrow$               & $0.016$    & $0.018$   & $0.001$   & $\mathbf{0.168}$  \\
Normalized mutual information $\uparrow$               & $0.450$    & $0.463$   & $0.384$   & $\mathbf{0.784}$  \\

\hline
\multicolumn{5}{c}{\cellcolor[HTML]{C0C0C0}\textbf{Generalizations from \texttt{GLUCOSE}}} \\ \toprule
% Silhouette coefficient (Euclidean) $\uparrow$               & $0.1315$    & $\mathbf{0.1331}$   & $0.0430$   & $0.0946$          \\
% Silhouette coefficient (Cosine)   $\uparrow$            & $0.1946$    & $\mathbf{0.1976}$   & $0.0691$   & $0.1302$  \\
Adjusted rand index $\uparrow$               & $0.042$    & $0.045$   & $0.011$   & $\mathbf{0.347}$  \\
Normalized mutual information $\uparrow$               & $0.635$    & $0.639$   & $0.699$   & $\mathbf{0.869}$  \\

\bottomrule
\end{tabular}
% }
    \caption{Experimental results of using automatic clustering for identifying abstractions using generalizations by \texttt{ChatGPT} and human-annotated generalizations from \texttt{GLUCOSE}.  (*) In this experiment, we use the original implementation of the \texttt{PIVOT} algorithm in \citet{fukunaga2019lp}. \textbf{Bold} indicates best performance.}
    \label{tab:clustering_abs}
\end{table*}

% Note that the Silhouette coefficients are different from Table \ref{tab:clustering} since we do not consider causal relations during clustering, which are assumed unknown.



% \section{Causal Network Visualization Tool}
% 
% \subsection{Causal Relationship Network Visualization Tool}

% \begin{figure*}[t!]
%     \centering
%     \efbox{\includegraphics[width=\linewidth]{figures/causal_network_visualization.png}}
% \caption{\textit{Causal Network Visualization} tool interface showcasing a complex network of causal relationships among event abstractions.}
% \label{fig:causal_network_visualization}
% \end{figure*}

The \textit{Causal Network Visualization} tool is an explorative attempt at enhancing the comprehension of complex causal relationships within textual data. Built on React\footnote{\url{reactjs.org}}, TypeScript\footnote{\url{typescriptlang.org}}, Sigma.js\footnote{\url{sigmajs.org}}, and leveraging state-of-the-art NLP techniques, this web-based application facilitates the clustering of sentences and the identification of causal relationships between them. The layout of the network utilizes \texttt{nx.spring\_layout}, a force-directed layout algorithm provided by networkX\footnote{\url{networkx.org}}, to arrange the nodes in a manner that reflects the complexity of the causal relationships. Key features of the tool include categorization of nodes into "keys" and "categories" for enhanced navigation, node size differentiation based on between-ness centrality to signify importance, and robust search and filtering options that improve usability. Additionally, the application supports customization for different datasets and visual styles, offering a versatile platform for academic research in understanding the intricate web of cause-and-effect relationships. Our tool is licensed under the MIT License\footnote{\url{opensource.org/licenses/MIT}.}.

\Vy{Provide like to visualization here}

\section{Statistical Causal Discovery}\label{sup:causal_discovery}

\section{Basic Background: Supervised Learning and the PAC Model}
\label{sec:background}

At this point almost everyone has heard of machine learning (ML). Anyone likely to stumble upon this article will have also heard of its most influential special case, supervised learning, and those theoretically inclined will also be familiar with the PAC model. Nonetheless, I will set the stage by  recapping the basics.

\subsection{Basics of Supervised Learning}%Let's set the stage in any case

\emph{Supervised Learning} is the task of ``coming up'' with a function $f: \X \to \Y$ to ``explain'' or ``fit'' a sequence of input/output examples   $(x_1,y_1), \ldots, (x_n,y_n)$, with $x_i \in \X$ and $y_i \in \Y$.  Here $\X$ is a \emph{data domain} consisting of \emph{datapoints} $x \in \X$, $\Y$ is a \emph{label set} consisting of \emph{labels} $y \in \Y$, and the sequence $(x_1,y_1),\ldots,(x_n,y_n)$ is the \emph{training data} consisting of \emph{labeled examples (a.k.a. samples)}~$(x_i,y_i)$.  I~will refer to the chosen function $f$ as a \emph{predictor}, and to $n$ as the \emph{sample size}. A \emph{learning algorithm} takes as input training data, and outputs (some representation of) a predictor $f \in \Y^\X$.\footnote{Note that this describes the usual \emph{batch}, a.k.a.~\emph{offline}, setting of supervised learning. I do not discuss other paradigms such as online or active learning in this article.} 



Success in supervised learning is defined as \emph{generalization} to  future examples: For a typical \emph{test example}  $(x_{\tst},y_{\tst})$, the predicted label $y'_{\tst}=f(x_{\tst})$ should ``equal'' $y_{\tst}$, perhaps approximately. We usually assume the test example is drawn from the same  ``source'' as the training data  --- commonly, i.i.d.~from the same distribution. The quality of the prediction is quantified by $\ell(y'_{\tst},y_{\tst})$, where $\ell:~\Y~\times~\Y \to \RR_{\geq 0}$ is a \emph{loss function} chosen as part of the problem definition. Common loss functions include the 0-1 loss $\ell_{0-1}(y',y) = [y' \neq y]$ for \emph{classification} problems,\footnote{The notation $[P]$ denotes $1$ when predicate $P$ is true, and denotes $0$ when $P$ is false.} as well as the absolute loss $|y'-y|$ or squared loss $(y'-y)^2$ for \emph{regression problems} featuring $\Y  \sse \RR$.

Nontrivial generalization properties are typically only possible if one assumes something about the data.\footnote{The need for such an assumption is formalized by the  \emph{no free lunch theorems} of supervised learning \cite{wolpert_connection_1992,wolpert_lack_1996,schaffer_conservation_1994}.} The Bayesian approach to  machine learning, common in many applications, assumes some parametric form for the distribution generating the data, and postulates a prior on the parameters. This is not the approach I will take in this article. Instead, I will focus on the frequentist --- and some would say ``worst-case'' or ``adversarial'' ---  approach that is common in the computational learning theory community, embodied by the PAC model. Here we assume that the (training and test) data can be explained, perhaps approximately, by a function in some ``simple enough to learn'' class of functions $\H \sse \Y^\X$, often called the \emph{hypotheses}. Equivalently, we  seek a predictor which explains the unseen data roughly  as well as the best hypothesis $h^* \in \H$, whether or not we assume that $h^*$ itself provides a perfect explanation.



 \paragraph{Common Algorithmic Templates.} Perhaps the best known general-purpose supervised learning algorithm is \emph{empirical risk minimization (ERM)}, which chooses as its predictor a hypothesis $f \in \H$ minimizing $\frac{1}{n} \sum_{i=1}^n \ell(f(x_i),y_i)$ --- a quantity called the \emph{training error}, \emph{empirical error}, or \emph{empirical risk} of $f$. %\footnote{When multiple hypotheses minimize the empirical risk, we assume ERM breaks ties arbitrarily.}
A common template for generalizing ERM involves adding a \emph{regularization term} $\psi(f)$ to the  objective function, typically chosen to measure some notion of ``hypothesis complexity.'' An algorithm instantiating this template is known as a \emph{structural risk minimizer (SRM)}, and chooses as its predictor the hypothesis $f \in \H$ minimizing the \emph{structural risk} $\frac{1}{n} \sum_{i=1}^n \ell(f(x_i),y_i) + \psi(f)$. Other well-known algorithms, such as gradient descent and its variations,  can frequently be interpreted as approximate implementations of ERM or SRM.


\paragraph{Proper vs Improper Learning.} A learning algorithm is said to be \emph{proper} if its predictor $f$ is always chosen from the hypothesis class, i.e., $f \in \H$, otherwise it is said to be \emph{improper}. ERM  is an example of a proper learning algorithm, as are SRM algorithms of the form described above.  In the \emph{proper regime} of learning, algorithms are required to be proper. This article will be concerned with the more flexible \emph{improper regime} (a.k.a \emph{representation-independent learning}), where no such constraint is placed on the learner. In other words, all we care about is predictive power at test time, rather than any insights derived from the functional form or representation of the predictor~itself.


\subsection{The PAC Model}
A standard mathematical setup for evaluation of supervised learning algorithms, at least in the theoretical computer science community, is Valiant's \emph{Probably Approximately Correct (PAC) model} of learning (see e.g.~\cite{kearns_introduction_1994,mohri_foundations_2018}). Here, we assume there is an unknown distribution $\D$ on $\X \times \Y$ from which training and test data are  drawn.  Specifically, the labeled datapoints of the training set  $(x_1,y_1), \ldots, (x_n,y_n)$, as well as the test data  $(x_\tst,y_\tst)$, are i.i.d.~from $\D$. Often it is assumed that $\D$ lies in some class of distributions of interest. The \emph{true expected loss}, or simply \emph{loss}, of a predictor $f: \X \to \Y$ is the expected loss it incurs on draws from $\D$, written $L_\D(f) = \Ex_{(x,y) \sim \D} \ell(f(x),y)$.


There are two main ``settings'' in PAC learning. The  \emph{realizable setting} only requires that the data be perfectly explained by some hypothesis in $\H$. More generally, the \emph{agnostic setting} makes no assumption relating the data to the hypotheses, but shifts the goalposts as necessary to allow nontrivial guarantees: the expected loss at test time is evaluated only ``relative'' to that of the best hypothesis $h^* \in \H$. There are other settings which make more nuanced assumptions, such as $\D$ being of a particular parametric form or its support living in some (unknown) lower-dimensional space, etc. I will mostly discuss the realizable and agnostic settings in this article, those being the simplest and most studied from a theoretical perspective. %TODO:We will briefly discuss other settings in Section ??

The PAC model demands high probability guarantees of learners, in the worst case over distributions of interest. Consider first the realizable setting, where $\D$ is such that $\min_{h \in \H} L_{\D}(h) = 0$. A PAC learner has \emph{error} $\epsilon=\epsilon(n)$ and \emph{confidence} $\delta=\delta(n)$ if, when training data consists of $n$ i.i.d~samples from a realizable distribution $\D$, it produces a predictor $f$  satisfying $L_\D(f) \leq \epsilon$ with probability at least $1-\delta$. In the agnostic setting, where $\D$ can be arbitrary, we require $L_\D(f) - \min_{h \in \H} L_\D(h) \leq \epsilon$ with probability $1-\delta$.

In both the realizable and agnostic settings, we look for PAC learners with small $\epsilon$ and $\delta$ as a function of the sample size $n$. An equivalent perspective looks at the sample complexity $m(\epsilon,\delta)$, which is the minimum sample size which guarantees error  at most $\epsilon$ with probability at least $1-\delta$. We say a problem is \emph{PAC learnable} if its PAC sample complexity is finite whenever $\epsilon,\delta > 0$.

For most PAC learning problems, learnability and sample complexity are characterized in terms of a  ``dimension'' of the hypothesis class. Most prominently this is the \emph{VC dimension} for binary classification, the \emph{fat shattering dimension} for agnostic regression, and the \emph{DS dimension} for multiclass classification (see \cite{anthony_neural_1999,daniely_optimal_2014,brukhim_characterization_2022}). Treatment of these is beyond the scope of this article. The unfamiliar reader need not worry, however,  as dimensions will feature only tangentially in our~discussion.




%\paragraph{Learning settings: Realizable, Agnostic, etc.} In learning theory, evaluating a supervised learning algorithm requires specifying a data model and an objective. We will leave the details of the data model flexible for now, to allow for both the PAC model and the adversarial transductive model. Nonetheless we will describe two variations, which we call ``settings'', which cut across different models. The  \emph{realizable setting}  requires only that the data be perfectly explained by some hypothesis $h \in \H$ --- i.e., there exists a hypothesis which is guaranteed to suffer a loss of $0$ on training and test data. The performance of the learning algorithm is its expected loss at test time for some ``worst case'' realizable instance. More generally, the \emph{agnostic setting} makes no assumption relating the data to the hypotheses, but shifts the goalposts as necessary to allow nontrivial guarantees: the expected loss at test time is evaluated only ``relative'' to that of the best hypothesis $h^* \in \H$, again for some ``worst case'' instance. There are other settings which make more nuanced assumptions about the data, such as it is drawn from a distribution of a particular parametric form, or that it lives in some (unknown) lower-dimensional space, etc. We will mostly discuss the realizable and agnostic settings, those being the simplest and most studied from a theoretical perspective.




%%% Local Variables:
%%% mode: latex
%%% TeX-master: "learning_matching"
%%% End:


\begin{figure*}[hbt!]
    \centering
    \includegraphics[width=\linewidth]{figures/CD.pdf}
    \caption{SHD \textbf{(left)} and F1 score \textbf{(right)} of estimated DAGs from statistical structure learning methods. \underline{Lower} SHD is better. \underline{Higher} F1 is better.}
    \label{fig:ssl}
\end{figure*}


\paragraph{Experiments.} We here discuss how \texttt{ACCESS} is used to assess to what extent the statistical structure learning methods is applicable to recover causal relations among event abstractions. As illustrated in Figure \ref{fig:main}, after extracting abstractions, one can build representations for abstract events in the original corpus and apply structure learning on top of such data for full graph discovery. A simple representation is the co-occurrence matrix size $(\# stories \times \# abstractions)$ where each entry takes a binary value indicating whether an abstraction has any of its mentions appearing in a story. This means each abstraction is now considered as a Bernoulli random variable and the task of causal discovery is to recover the underlying SCM where the structural functions are commonly non-convex. 

Due to the limited scalability of existing statistical algorithms, we resort to learning sub-graphs by setting thresholds to select nodes that appear frequently while ensuring that the true graph is acyclic. Specifically, our selected sub-graphs are composed of edges where both nodes are adjacent to at least one other node, and each node corresponds to an abstraction whose occurrences exceed a certain frequency threshold. In our experiment, we set thresholds for document frequency within  $\{25, 30, 35, 40, 45\}$, resulting in sub-graphs with $5, 7, 16, 19, 45$ nodes. The experiments are run on $5$ CPU cores.    


We experiment with popular constraint-based and score-based algorithms. We select those that are scalable and capable of capturing non-linear causal relationships without relying on specific model forms such as additive noise. In this paper, we report the results for the following algorithms: 


\begin{itemize}
    \item \texttt{PC} algorithm \citep{spirtes1991algorithm}: A classic approach based on conditional independence tests, for which we run two kinds of tests: Chi-squared and G-squared. 
    \item \texttt{DAG-GNN} \citep{yu2019dag}: DAG structure learning with graph neural networks.
    \item \texttt{GAE} \citep{ng2019graph}: This method utilizes gradient descent and graph auto-encoders to model non-linear causal relationships.
    \item \texttt{CORL} \citep{wang2021ordering}: A reinforcement learning-based algorithm with flexible score functions with enhanced efficiency.
\end{itemize}

Besides the above methods, we have also tested \texttt{NOTEARS} \citep{zheng2020learning}, a popular score-based algorithm and its more efficient variant \texttt{GOLEM} \citep{ng2020role}. However, they both unfortunately fail to recover any edges across all settings. To ensure consistency in implementation and evaluation, we utilize the standardized framework provided by \href{https://github.com/huawei-noah/trustworthyAI/tree/master/gcastle}{\texttt{gCastle}} \citep{zhang2021gcastle}. 
As for evaluation metrics, we report the structured Hamming distance (SHD), which quantifies the smallest number of edge additions, deletions, and reversals required to transform the recovered DAG into the true one. Additionally, we assess classification accuracy using the F1 score. Ideally, we aim for a lower normalized Hamming distance and a higher F1 score. Figure \ref{fig:ssl} reports the SHD and F1 score of the estimated DAGs from these methods. 

It is seen the methods achieve relatively low accuracy on our benchmark causal graphs, which are sparse. As the SHD scores are much higher than the graph size, these model tend to predict plenty of edges, most of which are incorrect due to the low F1 scores. Scalability remains a serious challenge to statistical structure learning. As the graph scales up to $45$ nodes, their performance further deteriorates significantly, where most of them of them even fail to recover any edges. It is noteworthy that the representation power of the input data also affects the causal discovery performance. It is very likely that the co-occurrence matrix is not sufficiently expressive to capture the causal knowledge. This motivates a dedicated line of research into abstract causal representation learning. 


\section{GLUCOSE-QA Reasoning}  \label{sup:reasoning}
We here provide the prompts for LLMs in Tables \ref{tab:prompt_causal_discovery}-\ref{tab:prompt_abs}. Tables \ref{tab:examples_specific_qa}-\ref{tab:examples_cot_step} present illustrative examples of the responses from LLMs across our QA tasks. 




\begin{tcolorbox}[title={The Prompt used for Translation}]
You are a highly skilled translator tasked with translating various types of content from English into \{\{ language \}\}. Follow these instructions carefully to complete the translation task.

You will receive a user-bot conversation in XML format. Please follow a three-step translation process:

\begin{enumerate}
  \item \textbf{Initial Translation:} Translate the input content into \{\{ language \}\}, preserving the original intent and keeping the original paragraph and text format unchanged. Do not delete or omit any content, and ensure that all original Markdown elements (e.g., images, code blocks) are preserved.
  \item \textbf{Reflection and Feedback:} Carefully review both the source text and your translation. Provide constructive criticism and specific suggestions to improve the translation in terms of:
    \begin{enumerate}[label=(\roman*)]
      \item \textbf{Accuracy:} Correct errors of addition, mistranslation, omission, or untranslated text.
      \item \textbf{Fluency:} Apply \{\{ language \}\} grammar, spelling, and punctuation rules while avoiding unnecessary repetitions.
      \item \textbf{Style:} Ensure that the translation reflects the style of the source text and considers any relevant cultural context.
    \end{enumerate}
  \item \textbf{Refinement:} Based on your reflections, refine and polish your translation.
  \item \textbf{Fallback:} If you are not confident in translating the conversation, please return ``\texttt{<stop></stop>}''.
\end{enumerate}

\bigskip
\textbf{Output:}

For each step of the translation process, output your results within the appropriate XML tags as follows:
\begin{verbatim}
<step1_initial_translation>
[Insert your initial translation here]
</step1_initial_translation>

<step2_reflection>
[Insert your reflection on the translation, including a list 
of specific, helpful, and constructive suggestions for 
improvement. Each suggestion should address a specific 
part of the translation.]
</step2_reflection>

<step3_refined_translation>
[Insert your refined and polished translation here]
</step3_refined_translation>
\end{verbatim}

Ensure that your final translation in step 3 accurately reflects the original meaning while sounding natural in \{\{ language \}\}.

Here is the original conversation:
\label{box:trans_prompt}
\end{tcolorbox}


\begin{table*}[hbt!]
\centering
\resizebox{\linewidth}{!}{%
\begin{tabular}{l|p{12cm}}
\toprule
Story & In a store, two women were arguing, and Howard wanted to intervene. He attempted to get them to stop talking, but it didn't work. So, he stepped in between them, which caused them to cease their fighting. \\
\midrule
Specific Question & What could be the cause of the event \textit{`howard wants to help the women'}? \\
\midrule
Abstract Question & The question describes an event where \textit{`a person hears something in a place'}. What could be the effect of the event?\\
\midrule
Choices & \begin{tabular}[c]{@{}l@{}}0: "Two women fights each other.",\\ 1: "He went in between them.",\\ 2: "Two women were fighting in a store.",\\ 3: "They stopped.",\\ 4: "Howard wanted to help."\\ 5: "He tried telling them to stop but it did not work."\end{tabular} \\ \midrule
Causal Graph (CG) & \textit{a person have a fight with another person} $\rightarrow$ \textit{a person want to stop another person} \\ \midrule
Correct Answers & 0, 2 \\ \midrule
\texttt{GPT-4o-mini} Answers & 2, 4 \\ \midrule
\texttt{GPT-4o-mini} Answers w/ CG & 0, 2 \\ \midrule
\texttt{Llama3.1-8B} Answers & 0, 1 \\ \midrule
\texttt{Llama3.1-8B} Answers w/ CG & 0, 2, 4 \\ 
\bottomrule
\end{tabular}%
}
\caption{Examples of multi-choice Specific-QA reasoning in \texttt{GPT-4o-mini} and \texttt{Llama3.1-8B}.}
\label{tab:examples_specific_qa}
\end{table*}


\begin{table*}[hbt!]
\centering
\resizebox{\linewidth}{!}{%
\begin{tabular}{l|p{12cm}}
\toprule
Story & His cousins were scheduled to visit later that day, so his mom had him clean in the morning, shop for groceries in the afternoon, and get ready in the evening. Eventually, his cousins arrived at his house. \\
\midrule
Abstract Question & The question describes an event where \textit{`a person are coming to a place (that is another person house)'}. What could be the effect of the event?\\
\midrule
Choices & \begin{tabular}[c]{@{}l@{}}0: "His cousins were coming later too his house.",\\ 1: "He get groceries in the afternoon.",\\ 2: "His mom made him clean all morning.",\\ 3: "His cousins came to his house.",\\ 4: "He get ready in the evening."\end{tabular} \\ \midrule
Causal Graph (CG) & \textit{a person come to another person 's place} $\rightarrow$ \textit{a person clean something} \\ \midrule
Correct Answers & 1, 2, 4 \\ \midrule
\texttt{GPT-4o-mini} Answers & 0, 3 \\ \midrule
\texttt{GPT-4o-mini} Answers w/ CG & 0, 2 \\ \midrule
\texttt{Llama3.2-3B} Answers & 0, 3 \\ \midrule
\texttt{Llama3.2-3B} Answers w/ CG & 1, 3 \\ 
\bottomrule
\end{tabular}%
}
\caption{Examples of multi-choice Abstract-QA reasoning in \texttt{GPT-4o-mini} and \texttt{Llama3.2-3B}.}
\label{tab:examples_abstract_qa_1}
\end{table*}

\begin{table*}[hbt!]
\centering
\resizebox{\linewidth}{!}{%
\begin{tabular}{l|p{12cm}}
\toprule
Story & Felix wanted to visit Disney World. One day, he won two tickets and invited his friend Alissa. However, Alissa disliked Disney, so Felix ended up going by himself. \\
\midrule
Abstract Question & The question describes an event where \textit{`a person invited another person'}. What could be the cause of the event?\\
\midrule
Choices & \begin{tabular}[c]{@{}l@{}}0: "Alissa hated disney.",\\ 1: "Felix wanted to go to disney world.",\\ 2: "One day he won two tickets for entry.",\\ 3: "He invited his friend Alissa.",\\ 4: "He ended up going alone."\end{tabular} \\ \midrule
Causal Graph (CG) & \textit{a person want to go to a place} $\rightarrow$ \textit{a person give another person an invitation to a place} \\ \midrule
Correct Answers & 1, 2 \\ \midrule
\texttt{GPT-4o-mini} Answers & 0, 1, 3 \\ \midrule
\texttt{GPT-4o-mini} Answers w/ CG & 1, 2 \\ \midrule
\texttt{Llama2-7B} Answers & 1, 2 \\ \midrule
\texttt{Llama2-7B} Answers w/ CG & 1, 2 \\ 
\bottomrule
\end{tabular}%
}
\caption{Examples of multi-choice Abstract-QA reasoning in \texttt{GPT-4o-mini} and \texttt{Llama2-7B}.}
\label{tab:examples_abstract_qa_2}
\end{table*}
\begin{table*}[hbt!]
\centering
\resizebox{\linewidth}{!}{%
\begin{tabular}{l|p{12cm}}
\toprule
Story & He wanted toast, so he got some bread and put it in the toaster. When it popped out and landed on the floor, he ate it anyway. \\
\midrule
Abstract Question & The question describes an event where \textit{`a person got another thing (that is an ingredient in another thing'}. What could be the cause of the event?\\
\midrule
Choices & \begin{tabular}[c]{@{}l@{}}0: "He ate it anyway.",\\ 1: "He put it in the toaster.",\\ 2: "He got some bread.",\\ 3: "It shot out of the toaster and onto the floor.",\\ 4: "He was making toast."\end{tabular} \\ \midrule
% Causal Graph (CG) & \textit{a person make food} $\rightarrow$ \textit{a person get ingredient} \\ \midrule
Correct Answers & 4 \\ 
\midrule
\texttt{Llama3.2-3B} Answers (zero-shot) & 1, 2 \\ 
\texttt{Llama3.2-3B} Answers & 1, 4 \\ 
\texttt{Llama3.2-3B} Answers + CG & 1, 4 \\ 
\midrule
\texttt{Llama3.1-8B} Answers(zero-shot) & 1, 3 \\ 
\texttt{Llama3.1-8B} Answers  & 2 \\ 
\texttt{Llama3.1-8B} Answers + CG & 4 \\ 
\midrule
\texttt{Llama2-7B} Answers (zero-shot) & 1, 2 \\ 
\texttt{Llama2-7B} Answers & 4 \\ 
\texttt{Llama2-7B} Answers + CG & 1, 4 \\ 
\bottomrule
\end{tabular}%
}
\caption{Examples of multi-choice Abstract-QA reasoning with bi-level COT.}
\label{tab:examples_cot}
\end{table*}

\begin{table*}[hbt!]
\centering
\resizebox{\linewidth}{!}{%
\begin{tabular}{l|p{12cm}}
\toprule
Story & He wanted toast, so he got some bread and put it in the toaster. When it popped out and landed on the floor, he ate it anyway. \\
\midrule
Abstract Question & The question describes an event where \textit{`a person got another thing (that is an ingredient in another thing'}. What could be the cause of the event?\\
\midrule
Choices & \begin{tabular}[c]{@{}l@{}}0: "He ate it anyway.",\\ 1: "He put it in the toaster.",\\ 2: "He got some bread.",\\ 3: "It shot out of the toaster and onto the floor.",\\ 4: "He was making toast."\end{tabular} \\ \midrule
Prompt & The event \textit{`a person got another thing (that is an ingredient in another thing'} is described by one of the sentences in the story context. First identify that part
of the story. Then retrieve the event mentioned in the story that is a
corresponding cause/effect.\\ 
\midrule
% Causal Graph (CG) & \textit{a person make food} $\rightarrow$ \textit{a person get ingredient} \\ \midrule
Correct Answers & 4 \\ 
\midrule
Rationale & To solve this question, let's break down the steps: \\
& 1. Identify the sentence describing `another thing which is an ingredient in another thing': - The relevant sentence from the given text is not explicitly stated but rather implied through analysis. In this case, what seems like `he made breakfast food'. In other words - Toaster = Breakfast Food; Bread + Toast = Ingredients for the meal at hand.  Therefore the first item on our consideration has been identified as number four because if you are going into kitchen then most likely, your objective would be about cooking something.\\ 
& 2. Now find any events occurring after items were used during said activity using index numbers provided against them starting their respective sequence no. This means looking back towards end, start at \#5 because there will definitely exist only those indexed things found before each index respectively being associated w/ this particular action taking place afterward within same timeframe i.e., related actions happening after certain things are prepared here. Considering these points: We look forward now...4 provides our initial object `the ingredients'. This makes us say - That must come after getting things ready prior than eating. Therefore we also know - getting stuff does indeed precede putting things inside the device whose result gets eaten later.\\ 
& So while option zero fits too since all meals need ending act (eating) coming second to doing preparatory tasks involving having desired objects (`bread').. But finally- lets verify... As shown below, the correct answer(s): 4, 1\\ 
\bottomrule
\end{tabular}%
}
\caption{Example of multi-choice Abstract-QA reasoning with bi-level COT in \texttt{Llama3.2-3B}.}
\label{tab:examples_cot_step}
\end{table*}









\end{document}


