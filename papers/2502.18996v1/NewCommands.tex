\DeclarePairedDelimiter\ceil{\lceil}{\rceil}
\DeclarePairedDelimiter\floor{\lfloor}{\rfloor}

% trig
\newcommand{\eexpo}[1]{\exp \left(  #1 \right) }
\newcommand{\esine}[1]{\sin \left(  #1 \right) }
\newcommand{\ecosi}[1]{\cos \left(  #1 \right) }
\newcommand{\esinen}[2]{\sin^{#2} \left(  #1 \right) }
\newcommand{\ecosin}[2]{\cos^{#2} \left(  #1 \right) }
\newcommand{\elogtwo}[1]{\log_{2} \left(  #1 \right) }
\newcommand{\elogten}[1]{\log_{10} \left(  #1 \right) }
\newcommand{\eerf}[1]{ \text{erf} \left(  #1 \right) }

\newcommand{\eprob}[1]{Pr\left(  #1 \right)}
\newcommand{\epdf}[2]{f_{#1}\left(  #2 \right)}
\newcommand{\eprobcond}[2]{Pr\left(  #1 \mid  #2 \right)}

\newcommand{\eave}[2]{\mathbb{E}_{#1} \left\lbrace #2 \right\rbrace}
\newcommand{\eabsn}[2]{\left|  #1 \right| ^{#2}}
\newcommand{\eangle}[1]{\angle\left(  #1 \right)}
\newcommand{\eherm}[1]{\left(  #1 \right)^{H}}
\newcommand{\econj}[1]{\left( #1 \right)^{*}}
\newcommand{\etras}[1]{\left( #1 \right)^{\top}}
\newcommand{\einv}[1]{\left( #1 \right)^{-1}}
\newcommand{\epow}[2]{\left(  #1 \right) ^{#2}}
\newcommand{\eelem}[3]{\left[  #1 \right]^{#2}_{#3} }
\newcommand{\eceil}[1]{\ceil*{#1}}
\newcommand{\efloor}[1]{\floor*{#1}}
\newcommand{\edet}[1]{\text{det}\left( #1 \right)}
\newcommand{\evec}[1]{\text{vec}\left( #1 \right)}
\newcommand{\eadj}[1]{\text{adj}\left( #1 \right)}
\newcommand{\ediag}[1]{\text{diag}\left( #1 \right)}
\newcommand{\ediagn}[2]{\text{diag}_{#2}\left( #1 \right)}
\newcommand{\etra}[1]{\text{tr}\left( #1 \right)}
\newcommand{\enormn}[3]{\left| \left| #1\right| \right| ^{#2}_{#3}}

\newcommand{\einnp}[2]{\left\langle #1,#2 \right\rangle }
\newcommand{\einnpt}[3]{\left\langle #1,#2 \right\rangle _{#3} }

\newcommand{\eparder}[2]{\frac{\partial}{\partial #1}#2}
\newcommand{\eparfra}[2]{\frac{\partial #1}{\partial #2}}
\newcommand{\ediff}[1]{\text{d}#1}

\newcommand{\esizecv}[1]{\in\mathbb{C}^{#1}}
\newcommand{\esizerv}[1]{\in\mathbb{R}^{#1}}
\newcommand{\esizecm}[2]{\in\mathbb{C}^{#1\times#2}}
\newcommand{\esizerm}[2]{\in\mathbb{R}^{#1\times#2}}


\newcommand{\egausdr}[2]{\mathcal{N}\left( #1,#2\right) }
\newcommand{\egausd}[2]{\mathcal{CN}\left( #1,#2\right) }
\newcommand{\eexpod}[1]{\text{Exp}\left( #1\right) }
\newcommand{\egammd}[2]{\Gamma\left( #1,#2\right) }
\newcommand{\eunifc}[2]{\mathcal{U}\left[  #1,#2 \right]  }

\newcommand{\eexpabstwo}[1]{\mathbb{E} \left\lbrace  \left| #1 \right|^2 \right\rbrace}
\newcommand{\eexpabsn}[2]{\mathbb{E} \left\lbrace  \left| #1 \right|^{#2} \right\rbrace}
\newcommand{\eexpnortwo}[1]{\mathbb{E} \left\lbrace  \left\lVert #1 \right\rVert^2_{2} \right\rbrace}
\newcommand{\eexp}[1]{\mathbb{E} \left\lbrace #1 \right\rbrace}
\newcommand{\evar}[1]{\text{Var} \left\lbrace #1 \right\rbrace}

\newcommand{\emod}[2]{\text{mod}\left(#1, #2\right) }
\newcommand{\emoddft}[2]{\left(#1\right)_{#2} }

\newcommand{\eeye}[1]{\mathbf{I}_{#1}}
\newcommand{\ezeros}[2]{\mathbf{0}_{\left( #1 \times #2\right) }}
\newcommand{\eones}[2]{\mathbf{1}_{\left( #1 \times #2\right) }}

\newcommand{\ereal}[1]{\Re \left\lbrace #1 \right\rbrace}
\newcommand{\eimag}[1]{\Im \left\lbrace #1 \right\rbrace}

\newcommand{\emax}[1]{\text{max}\left(#1\right) }
\newcommand{\emin}[1]{\text{min}\left(#1\right) }

\newcommand{\emaxs}[2]{\max_{#1}\left(#2\right) }
\newcommand{\emins}[2]{\min_{#1}\left(#2\right) }

\newcommand{\eoptmax}[1]{\underset{#1}{\text{max} }}
\newcommand{\eoptmin}[1]{\underset{#1}{\text{min} }}

\newcommand{\eds}[1]{\left[  #1 \right] }
\newcommand{\ecs}[1]{\left(   #1 \right)  }
\newcommand{\ebs}[1]{\left\lbrace    #1 \right\rbrace   }

\newcommand{\ecomplexity}[1]{\mathcal{O}\left( #1\right) }

\newcommand{\edft}[2]{\text{DFT}_{#2}\left\lbrace  #1\right\rbrace  }
\newcommand{\eidft}[2]{\text{IDFT}_{#2}\left\lbrace  #1\right\rbrace  }

\def\mydate{\leavevmode\hbox{\the\year/\twodigits\month/\twodigits\day}}
\def\twodigits#1{\ifnum#1<10 0\fi\the#1}


%%%%%% Nuevo comando para insertar Figura: ejemplo:   \eAddFig{nombre}{Tamaño}{Label}{Caption}
\newcommand{\eAddFig}[4]{
	\begin{figure}[!t]
		\centering
		\includegraphics[width=#2\linewidth]{#1}
		\vspace{-3mm}
		\caption{#4}
		\label{#3}
	\end{figure}
}

\newcommand{\eAddTwoColFig}[4]{
	\begin{figure*}[!t]
		\centering
		\includegraphics[width=#2\linewidth]{#1}
		\vspace{-2mm}
		\caption{#4}
		\label{#3}
	\end{figure*}
}

\newcommand{\bluecomment}[1]{\textcolor{blue}{#1}}
%\newcommand{\bluecomment}[1]{\textcolor{black}{#1}}
\newcommand{\pp}[1]{\textcolor{red}{#1}}
