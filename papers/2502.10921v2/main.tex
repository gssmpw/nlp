%%
%% This is file `sample-sigconf.tex',
%% generated with the docstrip utility.
%%
%% The original source files were:
%%
%% samples.dtx  (with options: `all,proceedings,bibtex,sigconf')
%% 
%% IMPORTANT NOTICE:
%% 
%% For the copyright see the source file.
%% 
%% Any modified versions of this file must be renamed
%% with new filenames distinct from sample-sigconf.tex.
%% 
%% For distribution of the original source see the terms
%% for copying and modification in the file samples.dtx.
%% 
%% This generated file may be distributed as long as the
%% original source files, as listed above, are part of the
%% same distribution. (The sources need not necessarily be
%% in the same archive or directory.)
%%
%%
%% Commands for TeXCount
%TC:macro \cite [option:text,text]
%TC:macro \citep [option:text,text]
%TC:macro \citet [option:text,text]
%TC:envir table 0 1
%TC:envir table* 0 1
%TC:envir tabular [ignore] word
%TC:envir displaymath 0 word
%TC:envir math 0 word
%TC:envir comment 0 0
%%
%%
%% The first command in your LaTeX source must be the \documentclass
%% command.
%%
%% For submission and review of your manuscript please change the
%% command to \documentclass[manuscript, screen, review]{acmart}.
%%
%% When submitting camera ready or to TAPS, please change the command
%% to \documentclass[sigconf]{acmart} or whichever template is required
%% for your publication.
%%
%%

\documentclass[sigconf]{acmart}
\usepackage[utf8]{inputenc}
\usepackage{algorithmic}
\usepackage{graphicx}
\usepackage{textcomp}
\usepackage{multirow}
\usepackage{mdframed}
\usepackage{enumitem}
\usepackage{booktabs}
\usepackage{comment}
\usepackage{url}
\usepackage{tikzsymbols}
\usepackage{siunitx}
\usepackage{tabularx}
\usepackage{colortbl}
\usepackage{color,soul}
\usepackage{bm}
\usepackage{courier}
\usepackage{xcolor}
\usepackage{mathtools}
\usepackage{makecell}
\usepackage{dirtytalk}
\usepackage{lscape} 
\usepackage{array} 
\usepackage{balance}
\definecolor{Sepia}{RGB}{112, 66, 20} % Defining a sepia-like color
\usepackage[most]{tcolorbox}     % Required package for custom boxes

% Define the gray box style
\newmdenv[
  linecolor=gray!50,         % Border color
  backgroundcolor=gray!10,   % Background color
  linewidth=2pt,             % Border width
  roundcorner=5pt,           % Border corner radius
]{graybox}

%%
%% \BibTeX command to typeset BibTeX logo in the docs
\AtBeginDocument{%
  \providecommand\BibTeX{{%
    Bib\TeX}}}

%% Rights management information.  This information is sent to you
%% when you complete the rights form.  These commands have SAMPLE
%% values in them; it is your responsibility as an author to replace
%% the commands and values with those provided to you when you
%% complete the rights form.
%\setcopyright{acmlicensed}
%\copyrightyear{2018}
%\acmYear{2018}
%\acmDOI{XXXXXXX.XXXXXXX}

%% These commands are for a PROCEEDINGS abstract or paper.
%\acmConference[Conference acronym 'XX]{Make sure to enter the correct
%  conference title from your rights confirmation email}{June 03--05,
%  2018}{Woodstock, NY}
%%
%%  Uncomment \acmBooktitle if the title of the proceedings is different
%%  from ``Proceedings of ...''!
%%
%%\acmBooktitle{Woodstock '18: ACM Symposium on Neural Gaze Detection,
%%  June 03--05, 2018, Woodstock, NY}
%\acmISBN{978-1-4503-XXXX-X/18/06}


%%
%% Submission ID.
%% Use this when submitting an article to a sponsored event. You'll
%% receive a unique submission ID from the organizers
%% of the event, and this ID should be used as the parameter to this command.
%%\acmSubmissionID{123-A56-BU3}

%%
%% For managing citations, it is recommended to use bibliography
%% files in BibTeX format.
%%
%% You can then either use BibTeX with the ACM-Reference-Format style,
%% or BibLaTeX with the acmnumeric or acmauthoryear sytles, that include
%% support for advanced citation of software artefact from the
%% biblatex-software package, also separately available on CTAN.
%%
%% Look at the sample-*-biblatex.tex files for templates showcasing
%% the biblatex styles.
%%

%%
%% The majority of ACM publications use numbered citations and
%% references.  The command \citestyle{authoryear} switches to the
%% "author year" style.
%%
%% If you are preparing content for an event
%% sponsored by ACM SIGGRAPH, you must use the "author year" style of
%% citations and references.
%% Uncommenting
%% the next command will enable that style.
%%\citestyle{acmauthoryear}


\newif\ifcomment
\commenttrue
%\commentfalse
\ifcomment
\newcommand{\gs}[1]{{\bf\textcolor{red}{GS: #1}}}
\newcommand{\shiza}[1]{{\bf\textcolor{teal}{Shiza: #1}}}
\else
\newcommand{\gs}[1]{}
\newcommand{\shiza}[1]{}
\fi
\newcommand{\descr}[1]{\smallskip\noindent\textbf{#1}}


%%
%% end of the preamble, start of the body of the document source.
\begin{document}

%%
%% The "title" command has an optional parameter,
%% allowing the author to define a "short title" to be used in page headers.
\title{Evolving Hate Speech Online: An Adaptive Framework for Detection and Mitigation}

%%
%% The "author" command and its associated commands are used to define
%% the authors and their affiliations.
%% Of note is the shared affiliation of the first two authors, and the
%% "authornote" and "authornotemark" commands
%% used to denote shared contribution to the research.
\author{Shiza Ali}
\email{shiza@uw.edu}
%\orcid{1234-5678-9012}
%\author{G.K.M. Tobin}
%\authornotemark[1]
%\email{webmaster@marysville-ohio.com}
\affiliation{%
  \institution{University of Washington}
  \city{Bothell}
  \state{Washington}
  \country{USA}
}
\author{Jeremy Blackburn}
\email{jblackbu@binghamton.edu}
\affiliation{%
  \institution{Binghamton University}
  \city{Binghamton}
  \state{New York}
  \country{USA}
}

\author{Gianluca Stringhini}
\email{gian@bu.edu}
\affiliation{%
  \institution{Boston University}
  \city{Boston}
  \state{Massachusets}
  \country{USA}}


%%
%% By default, the full list of authors will be used in the page
%% headers. Often, this list is too long, and will overlap
%% other information printed in the page headers. This command allows
%% the author to define a more concise list
%% of authors' names for this purpose.
\renewcommand{\shortauthors}{Ali et al.}

%%
%% The abstract is a short summary of the work to be presented in the
%% article.
\begin{abstract}

The proliferation of social media platforms has led to an increase in the spread of hate speech, particularly targeting vulnerable communities.
Unfortunately, existing methods for automatically identifying and blocking toxic language rely on pre-constructed lexicons, making them reactive rather than adaptive.
As such, these approaches become less effective over time, especially when new communities are targeted with slurs not included in the original datasets.
To address this issue, we present an adaptive approach that uses word embeddings to update lexicons and develop a hybrid model that adjusts to emerging slurs and new linguistic patterns.
This approach can effectively detect toxic language, including intentional spelling mistakes employed by aggressors to avoid detection.
Our hybrid model, which combines BERT with lexicon-based techniques, achieves an accuracy of ~95\% for most state-of-the-art datasets. Our work has significant implications for creating safer online environments by improving the detection of toxic content and proactively updating the lexicon.\\
\noindent \textbf{Content Warning:} \textit{This paper contains examples of hate speech that may be triggering.}
\end{abstract}

%%
%% The code below is generated by the tool at http://dl.acm.org/ccs.cfm.
%% Please copy and paste the code instead of the example below.
%%
\begin{CCSXML}
<ccs2012>
   <concept>
       <concept_id>10002978.10003029.10003032</concept_id>
       <concept_desc>Security and privacy~Social aspects of security and privacy</concept_desc>
       <concept_significance>500</concept_significance>
       </concept>
   <concept>
       <concept_id>10003120.10003130.10011762</concept_id>
       <concept_desc>Human-centered computing~Empirical studies in collaborative and social computing</concept_desc>
       <concept_significance>500</concept_significance>
       </concept>
 </ccs2012>
\end{CCSXML}

\ccsdesc[500]{Security and privacy~Social aspects of security and privacy}
\ccsdesc[500]{Human-centered computing~Empirical studies in collaborative and social computing}




%%
%% Keywords. The author(s) should pick words that accurately describe
%% the work being presented. Separate the keywords with commas.
\keywords{toxicity, social media, adaptive moderation, hybrid model, social computing}

%\received{20 February 2007}
%\received[revised]{12 March 2009}
%\received[accepted]{5 June 2009}

%%
%% This command processes the author and affiliation and title
%% information and builds the first part of the formatted document.
\maketitle

People engage in activities in online forums to exchange ideas and express diverse opinions. Such online activities can evolve and escalate into binary-style debates, pitting one person against another~\cite{sridhar_joint_2015}. Previous research has shown the potential benefits of debating in online forums such as enhancing deliberative democracy~\cite{habermas_theory_1984, semaan_designing_2015, baughan_someone_2021} and debaters' critical thinking skills~\cite{walton_dialogue_1989, tanprasert_debate_2024}. For example, people who hold conflicting stances can help each other rethink from a different perspective. However, research has also shown that such debates could result in people attacking each other using aggressive words, leading to depressive emotions~\cite{shuv-ami_new_2022}. Hatred could spread among various groups debating different topics~\cite{iandoli_impact_2021, nasim_investigating_2023, vasconcellos_analyzing_2023, qin_dismantling_2024}, such as politics, sports, and gender.

In recent years, people have integrated Generative AI (GenAI) into various writing tasks, such as summarizing~\cite{august_know_2024}, editing~\cite{li_value_2024}, creative writing~\cite{chakrabarty_help_2022, li_value_2024, yang_ai_2022, yuan_wordcraft_2022}, as well as constructing arguments~\cite{jakesch_co-writing_2023, li_value_2024} and assisting with online discussions~\cite{lin_case_2024}. This raises new concerns in online debates. For example, an internally synthesized algorithm of Large Language Models (LLMs) could produce hallucinations~\cite{fischer_generative_2023, razi_not_2024}, which may act as a catalyst for the spread of misinformation in online forums~\cite{fischer_generative_2023}. In addition, GenAI could introduce biased information to forum members~\cite{razi_not_2024}, which may intensify pre-existing debates. Moreover, integrating GenAI into various writing scenarios may also result in weak insights~\cite{hadan_great_2024}, raising concerns about the impact of GenAI on the ecology of online forum debates.

Given these concerns, this study aims to explore how people use GenAI to engage in debates in online forums. The integration of GenAI is not only reshaping everyday writing practices but also has the potential to redefine the online argument-making paradigm. Previous research has demonstrated the potential of co-writing with GenAI, focusing primarily on its influence on individual writing tasks~\cite{august_know_2024, chakrabarty_help_2022, jakesch_co-writing_2023, li_value_2024, yang_ai_2022}. However, the use of GenAI in the context of online debates, which combine elements of both confrontation and collaboration among remote members, remains underexplored. To explore it, we created an online forum for participants to engage in debates with the assistance of ChatGPT (GPT-4o) (\autoref{fig1}). This study enables us to closely observe how people make arguments and analyze their process data of using GenAI. We will examine three research questions to understand how the use of GenAI shapes debates in online forums: 

\begin{itemize}
\item{\textbf{RQ1}: How do people who participate in a debate on online forums collaborate with GenAI in making arguments?}

\item{\textbf{RQ2}: What patterns of arguments emerge when collaborating with GenAI to participate in a debate on online forums?}

\item{\textbf{RQ3}: How does the use of GenAI for making arguments change when a new member joins an existing debate in online forums?}
\end{itemize}

Given the universality and accessibility of debate topics, we chose one that is widely recognized and able to spark intense debates: soccer, which is regarded as the world's most popular sport~\cite{stolen_physiology_2005}. Building on this topic, we selected "Messi vs. Ronaldo: Who is better?" as the case for our study because it has been an enduring and heated debate among soccer fans. We created a small online forum as the platform for AI-mediated debates, particularly focusing on the debates among members and their interactions with ChatGPT. This approach enables more detailed observation and analysis of the entire process while fostering a nuanced understanding. The study consists of two parts: a one-on-one turn-based debate and a three-person free debate. In the first part, two participants, one supporting Messi and the other Ronaldo, took turns sharing their points of view to challenge each other through forum posts, mirroring the polarized debates that are omnipresent online. In the second part, a new participant joined the ongoing debate, and three participants were allowed to post freely without turn-based restriction, reflecting the spontaneous and unstructured nature of debates on social media. After the two-part study, semi-structured interviews were conducted to explore the participants' experiences. The researchers then applied content analysis and thematic analysis, triangulating the data from forum posts, ChatGPT records, and interview transcripts.

We found that participants prompted ChatGPT for aggressive responses, trying to tailor ChatGPT to fit the debate scenario. While ChatGPT provided participants with statistics and examples, it also led to the creation of similar posts. Furthermore, participants' posts contained logical fallacies such as hasty generalizations, straw man arguments, and ad hominem attacks. Participants reduced the use of ChatGPT to foster better human-human communication when a new member joined an ongoing debate midway. This work highlights the importance of examining how polarized forum members collaborated with GenAI to engage in online debates, aiming to inspire broader implications for socially oriented applications of GenAI.
\section{Related work}
\label{sec:related_work}
Anomaly detection, also known as outlier detection or novelty detection, is an important problem that has been studied within diverse research areas and application domains~\citep{chandola2009anomaly,chalapathy2019deep}. The problem of traffic AD bares similarities with the disciplines of robot AD and AD for surveillance cameras. In this section, we briefly review the related research and introduce common techniques in ensemble deep learning.

Recent research efforts have made noteworthy progress in developing learning-based AD algorithms for robots and mechanical systems. \cite{malhotra2016lstm} introduces an LSTM-based encoder-decoder scheme for multi-sensor AD (EncDec-AD) that learns to reconstruct normal data and uses reconstruction error to detect anomalies. \cite{park2018multimodal} proposes an LSTM-based variational autoencoder (VAE) that fuses sensory signals and reconstructs their expected distribution. The detector then reports an anomaly when a reconstruction-based anomaly score is higher than a state-based threshold. \cite{feng2022unsupervised} attacks multimodal AD with missing sources at any modality. A group of autoencoders (AEs) first restore missing sources to construct complete modalities, and then a skip-connected AE reconstructs the complete signal. Although similar in ideas, these approaches were proposed for low-dimensional signals (e.g., accelerations and pressures) and have not shown effective on high-dimensional data (e.g., images).

AD for robot navigation often involves complex perception signals from cameras and LiDARs in order to understand the environment. \cite{ji2020multi} proposes a supervised VAE (SVAE) model, which utilizes the representational power of VAE for supervised learning tasks, to identify anomalous patterns in 2D LiDAR point clouds during robot navigation. The predictive model proposed in LaND~\citep{kahn2021land} takes as input an image and a sequence of future control actions to predict probabilities of collision for each time step within the prediction horizon. \cite{schreiber2023attentional} further enhances the robot perception capability with the fusion of RGB images and LiDAR point clouds using an attention-based recurrent neural network, achieving improved AD performance on field robots. Different from these supervised-learning-based methods, \cite{wellhausen2020safe} uses normalizing flow models to learn distributions of normal samples of multimodal images, in order to realize safe robot navigation in novel environments. However, driving scenarios have additional complexities than field environments. While road environments are more structured than field environments, additional hazards arise from the presence of and interactions between dynamic road participants, which pose extra challenges on AD algorithms.

Another widely explored research area that is relevant to our work is AD for surveillance cameras, which mainly focuses on detecting the start and end time of anomalous events within a video. Under the category of frame-level methods, \cite{hasan2016learning} proposes a convolutional autoencoder to detect anomalous events by reconstructing stacked images. \cite{chong2017abnormal} and~\cite{luo2017remembering} extend such an idea by learning spatial features and the temporal evolution of the spatial features separately using convolution layers and ConvLSTM layers~\citep{shi2015convolutional}, respectively. Instead of reconstructing frames, \cite{liu2018future} trains a fully convolutional network to predict future frames based on past observations and uses the Peak Signal to Noise Ratio of the predicted frame as the anomaly score. \cite{gong2019memorizing} develops an autoencoder with a memory module, called memory-augmented autoencoder, to limit the generalization capability of the network on reconstructing anomalies. To focus more on small anomalous regions, patch-level methods generate the anomaly score of a frame as the max pooling of patch errors in the image rather than the averaged pixel error used in frame-level methods~\citep{wang2023memory}. In addition, object-level approaches have also been explored, which often focus on modeling normal object motions either through extracted features (e.g., human skeletons)~\citep{morais2019learning} or raw pixel values within bounding boxes~\citep{liu2021hybrid}. Although these methods have achieved promising results on surveillance cameras, the performance is often compromised in egocentric driving scenarios due to moving cameras and complex scenes~\citep{yao2022dota}.

In the domain of traffic AD in first-person videos, pioneering works borrow ideas from surveillance camera applications and detect abnormality by reconstructing motion features at frame level~\citep{yuan2016anomaly}. However, to overcome the issues introduced by rapid motions of cameras and thus backgrounds, object-centric methods are becoming increasingly popular. One of the most representative works by~\cite{yao2019unsupervised} proposes a recurrent encoder-decoder framework to predict future trajectories of an object in the image plane based on the object's past trajectories, spatiotemporal features, and ego motions. The accuracy and consistency of the predictions are then used to generate anomaly scores. One critical problem of such a method is the inevitable miss detection in the absence of traffic participants. As a result, ensemble methods emerge recently to combine the strengths of frame-level and object-centric methods. For example, ~\cite{yao2022dota} fuses the object location prediction model with the frame prediction model to achieve all-scenario detection capability and ~\cite{fang2022traffic} monitors the temporal consistency of frames, object locations, and spatial relation structures of scenes for AD. In this work, we derive each module and the corresponding learning objective in the ensemble based on a comprehensive analysis on anomaly patterns in egocentric driving videos. In particular, an interaction module is introduced to monitor anomalous interactions between road participants. The scores from each module are then fed as observations of a Kalman filter, from which the final anomaly score is obtained.

In this paper, we introduce an ensemble of detectors to capture different classes of anomalies.
We take inspiration from recent advances in ensemble deep learning, which aims to improve the generalization performance of a learning system by combing several individual deep learning models~\citep{ganaie2022ensemble} and has been applied to different application domains, such as speech recognition~\citep{li2017semi},  image classification~\citep{wang2020particle}, forecasting~\citep{singla2022ensemble}, and fault diagnosis~\citep{wen2022new}. Out of different classes of ensemble deep learning approaches, the most similar work to Xen is the heterogeneous ensemble (HEE), in which the components are trained on the same dataset but use different algorithms/architectures~\citep{li2018heterogeneous,tabik2020mnist}. However, each component in an HEE is usually trained with an identical learning objective, while each expert in Xen is assigned with different learning tasks. In terms of result fusion strategies, unweighted model averaging is one of the most popular approaches in the literature~\citep{ganaie2022ensemble}, which simply averages the outcomes of the base learners to get the final prediction of the ensemble model. By contrast, we exploit a Kalman filter to further combat the noise in scores from different components in a time-series task, and the unweighted model averaging can be viewed as a special case of such a method at a point along the time axis.


\begin{table}[t]
    \centering
    \begin{tabular}{lccccc}
    \toprule
         \textbf{Dataset} & \textbf{GUI Types} & \textbf{Number of GUIs} & \textbf{Detections} & \textbf{Descriptions} & \textbf{Scanpaths} \\
    \midrule
         MUD~\cite{mud} & Mobile (Android) & $\sim$18,000 & \cmark & \xmark & \xmark \\  
         RICO~\cite{rico} & Mobile (Android) & $\sim$72,000 & \cmark & \xmark & \xmark \\
         RICO-Semantic~\cite{rico_semantic} & Mobile (Android) & $\sim$72,000 & \cmark & \xmark & \xmark \\
         Clay~\cite{li2022learning} & Mobile (Android) & $\sim$60,000 & \cmark & \xmark & \xmark \\
          ENRICO~\cite{enrico} & Mobile (Android) & 1,460 & \cmark & \xmark & \xmark \\ 
          VINS~\cite{vins} & Mobile (Android + IOS) & 4,543 & \cmark & \xmark & \xmark \\ 
          AMP~\cite{zhang2021screen} & Mobile  (IOS) (not publicly available) & $\sim$77,000 & \cmark & \xmark & \xmark \\ 
          Webzeitgeist~\cite{kumar2013webzeitgeist} & Webpages & 103,744 & \cmark & \xmark & \xmark \\
         WebUI~\cite{webui} & Webpages & $\sim$350,000 & \cmark & \xmark & \xmark \\  
         UEyes~\cite{ueyes} & Mobile, Webpages, Poster, Desktop & 1,980 & \xmark & \xmark & \cmark \\
         \midrule
         \bf Ours & Webpages, Mobile (Android + IOS) & $\sim$72,500 & \cmark & \cmark & \textcolor{munsell}{Predicted} \\        
    \bottomrule
    \end{tabular}
    \caption{Comparison of GUI datasets: Our dataset uniquely combines mobile and webpage GUIs with comprehensive annotations and predicted scanpaths.
}
    \label{tab:datasets_related}
\end{table}

\section{Dataset}

\paragraph{Prior Datasets}
No existing datasets can be directly used to train a model conditioned on prompts, wireframes, and visual flow directions. 
Several datasets have been collected to support GUI tasks, as shown in \autoref{tab:datasets_related}. The AMP dataset, comprising 77,000 high-quality screens from 4,068 iOS apps with human annotations~\cite{zhang2021screen}, is not publicly available. On the other hand, the largest publicly available dataset, Rico~\cite{rico}, includes 72,000 app screens from 9,700 Android apps and has been a primary resource for GUI understanding despite its inherent noise. To address its limitation, the Clay dataset~\cite{li2022learning} was created by denoising Rico using a pipeline of automated machine learning models and human annotators to provide more accurate element labels. Enrico~\cite{enrico} further cleaned and annotated Rico but ultimately contains only a small set of high-quality GUIs. MUD~\cite{mud} offers a dataset featuring modern-style Android GUIs. The VINS dataset~\cite{vins} focuses on GUI element detection and was created by manually capturing screenshots from various sources, including both Android and iOS GUIs. Additionally, Webzeitgeist~\cite{kumar2013webzeitgeist} used automated crawling to mine design data from 103,744 webpages, associating web elements with properties such as HTML tags, size, font, and color. Similarly, WebUI~\cite{webui} provides a large-scale collection of website data. None of these datasets include both mobile GUIs and webpages and have included visual flow information in the datasets.
UEyes~\cite{ueyes} is the first mixed GUI-type eye tracking dataset with ground-truth scanpaths, although it lacks element labels.

%In our work, we create a comprehensive high-quality dataset that includes both mobile GUIs (Android and iOS) and webpages by cleaning and combining GUIs from Enrico~\cite{enrico}, VINS~\cite{vins}, and WebUI~\cite{webui}. For each GUI, we further generate segmentation maps with GUI element labels, apply LLaVA-Next~\cite{liu2024llavanext} to generate description, and use EyeFormer~\cite{eyeformer}, a state-of-the-art scanpath prediction model, to generate scanpaths.
To address these limitations, we construct a large-scale high-quality dataset of mixed mobile UIs and webpages, including about 72,500 GUI screenshots, along with their wireframes with labeled GUI elements, descriptions, and scanpaths. This dataset is designed to support the training of generative AI models, filling a gap in existing public datasets by providing not only GUI images but also detailed descriptions, element labels, and visual interaction flows. 

\paragraph{GUI Screenshots} Our dataset integrates and cleans GUI data from Enrico~\cite{enrico}, VINS~\cite{vins}, and WebUI~\cite{webui}. The Enrico dataset contains 1,460 Android mobile GUIs, while VINS includes 4,543 Android and iOS GUIs. WebUI is a large-scale webpage dataset consisting of approximately 350,000 GUI screenshots with corresponding HTML code. For WebUI, the original dataset includes screenshots for different resolutions, leading to many similar screenshots for each webpage.
We retained only the 1920 x 1080 resolution screenshots to avoid redundant images from different resolutions. We further refined these three datasets by removing abstract, non-graphic wireframe GUIs, duplicates, and GUIs with fewer than three elements.  The final dataset consists of 66,796 webpages and 5,634 mobile GUIs.

\paragraph{Wireframes with GUI Element Labels} 

For mobile GUIs, we selected the Enrico and VINS datasets for their well-labeled GUI element bounding boxes. To further refine these annotations, we applied the UIED~\cite{uied} model, which detects and refines GUI element bounding boxes. We manually verified and corrected the results for accuracy. For the WebUI dataset, each element has multiple labels. We filtered the original element labels to keep only the most relevant label for each element. We standardized the labels across mobile and webpage elements, mapping them to nine common types: `Button', `Text', Image', `Icon', `Navigation Bar', `Input Field', `Toggle', `Checkbox', and `Scroll Element'. Using these refined bounding boxes and labels, we generated wireframes with GUI element labels.

\paragraph{Descriptions} For each GUI, we employed the LLaVA-Next~\cite{liu2024llavanext} vision-language model to generate both concise and detailed descriptions.

\paragraph{Scanpaths} Finally, we used EyeFormer~\cite{eyeformer}, the state-of-the-art scanpath prediction model, to predict scanpaths for each GUI. While using real scanpaths recorded by eye trackers would provide more accurate data, this process is highly time-consuming. Alternative proxies, such as webcams or cursor movements, do not capture the same cognitive processes as actual eye movements, making them less suitable for our purposes.
\section{Methodology}
In this section, we discuss our adaptive method for updating hate speech lexicons in detail as well as the machine learning approaches we use.
We adopt both traditional supervised-learning approaches and deep-learning models to compare the accuracy of detecting hate speech using the seed lexicons and the updated lexicons. Figure~\ref{fig:pipeline} provides a high-level description of our adaptive approach to hate speech detection.
Later we propose a novel hybrid approach to hate speech detection that utilizes both lexicon-based and unsupervised learning approaches.

 \begin{figure*}[t!]
 	\centering
 	% \includegraphics[width=0.9\textwidth]{Figures/P4.png}
    \includegraphics[width=0.70\textwidth]{Figures/adaptive-toxicity-pipeline.pdf}
 	\caption{Architecture of our adaptive hate speech detection system.}
 	\label{fig:pipeline}
 \end{figure*}

\subsection{Step 1: Identifying Candidate New Toxic Words}
The first step in our pipeline after data collection is to update the lexicons.
The goal of this step is to pinpoint harmful words that are utilized similarly to already established toxic words so that they are relevant to the piece of text in which we are determining hate speech.
For example, to avoid censorship users use the word ``ducking'' instead of ``fucking''~\cite{theverge2023}.
We label all the seed lexicons as $S_{lexicons}$ and the updated lexicons as $U_{lexicons}$ respectively.
The updated lexicons contain both the seed lexicons as well as the new lexicons that we find using word-embedding models. Note that word embedding models allow us to identify \textit{contextually similar words} that may or may not be synonymous.

%\shiza{Seed Lexicons -- ($S_{lexicons}$) \\
%Updated Lexicons -- ($U_{lexicons}$) \\
%Word Embedding Lexicons -- ($W_{lexicons}$)}

To this end, we adopt a \textit{similarity-based} approach to find new toxic words.
For each word appearing in our dataset, we compute its vector embedding.
We test different approaches to find similar words, i.e., Word2Vec~\cite{tensorflowWord2vec}, GloVe~\cite{glove}, and more modern word embedding techniques like BERT~\cite{bert}. %In the following list, we go through each word embedding model in more detail.

% is more detail required
%\begin{itemize}
    %\item \textbf{Word2vec:} Word2Vec, introduced by Mikolov et al.~\cite{mikolov2013efficient} transforms words into continuous vector representations, to enable mathematical operations on words. %For our research, we employ the Word2Vec algorithm to learn distributed representations of words in the corpus.
    %By training the model on the preprocessed dataset, we obtain dense vector representations that capture the semantic relationships between words.
    %Finally, we utilize the trained Word2Vec model to compute the similarity between the seed lexicons ($S_{lexicons}$) and other words in the corpus using cosine distance and update our list of lexicons.
    
    %\item \textbf{GloVe:} GloVe (Global Vectors for Word Representation) is another word embedding model introduced by Pennington et. al.~\cite{pennington2014glove} that can capture both syntactic and semantic information.
    %GloVe learns word representations by leveraging co-occurrence statistics derived from large text corpora.
    %We use the GloVe algorithm to learn word embeddings from the preprocessed posts corpus.
    %The co-occurrence statistics derived from the social media posts dataset enable GloVe to generate distributed word representations that capture the semantic relationships between words.
    %Finally, we employ the learned GloVe embeddings to calculate word similarities using cosine similarity for all the seed lexicons ($S_{lexicons}$).
    
    %\item \textbf{BERT:} Similarly, BERT (Bidirectional Encoder Representations from Transformers), by Devlin et al.~\cite{devlin2018bert}, also captures contextualized word representations but its ability to capture bidirectional context allows it to comprehend word semantics more accurately.
    %For our research, we fine-tune BERT's pre-trained model using our posts dataset.
    %This fine-tuning process allows BERT to adapt to the specific characteristics of Twitter data.
    %Subsequently, we extract informative word embeddings from BERT's contextualized representations of the social media posts corpus and calculate word similarities.
    
%\end{itemize}

By going through this process, we were able to pinpoint words that are \textit{``similar,''} meaning they are utilized within comparable situations.
We determine the similarity between the word embeddings using cosine similarity. \textcolor{black}{We used a cosine similarity threshold of $\geq 0.75$ to identify new toxic words, after empirically testing thresholds of 0.7, 0.75, and 0.8. The threshold of 0.75 struck an optimal balance, generating a diverse set of new words while minimizing redundancy in the lexicon. All flagged words were manually labeled, and 36\% were excluded as non-toxic or irrelevant.}

\noindent\textbf{Graph-based Similarity Approach:} We incorporate a graph-based method using the Louvain algorithm~\cite{blondel2008fast} to assess word similarity through embeddings. Graphs are constructed based on word embeddings, connecting words if the cosine similarity of their vector representations surpasses a predefined threshold of $\geq 0.75$~\cite{zannettou2020quantitative}.
However, this approach exhibits limitations in performance. The method tends to generate numerous false positives, primarily due to its inherent lack of context specificity found in graph-based similarity methods.

The output of this phase is a set of words that we call updated lexicons ($U_{lexicons}$) that are \textit{likely} to be toxic for our specific dataset.

\subsection{Step 2(a): Testing The Updated Lexicons Using Traditional Machine Learning Models}

To test our adaptive approach to hate speech detection we use traditional machine learning models provided by Davidson et. al.~\cite{davidson2017automated}.
We choose Linear Support Vector Machine (SVM)~\cite{cortes1995support}, Random Forest (RF)~\cite{breiman2001random}, and Logistic Regression (LR)~\cite{hosmer2013applied} as traditional classification approaches.
We use the average accuracy of the models, F1-measure, and class-specific precision and recall to evaluate our models on the test sets.
We use grid search and stratified $k$-fold cross-validation ($k=10$) to tune the hyper-parameters during the training and validation phases.

\subsection{Step 2(b): Hybrid Approach For Hate Speech Detection}

The lexicon-based approach relies solely on a predefined list of toxic terms or phrases, which may not capture the evolving nature of hate speech or account for contextual nuances.
It can struggle to identify hate speech that does not precisely match the terms in the lexicon.
On the other hand, BERT models, while effective at capturing contextual information and semantic relationships, may require significant amounts of labeled data for fine-tuning and can be computationally expensive.

By combining the two approaches, we can leverage the strengths of both.
%The lexicon-based method provides a strong foundation for identifying explicit toxic words, while BERT models enhance the detection capability by capturing contextual cues and understanding the complexities of language.
To do this we used the \textit{Lexical Substitution} method for incorporating the hate speech lexicons as features.
We used the set of lexicons to generate additional features for the input text, which are then used as input to the BERT model.
Our method involves enhancing the input embeddings with hate speech lexicons, which are then passed through the pre-trained BERT classification model to get the prediction. Specifically, we tokenize the input text using the BERT tokenizer and then generate binary features for each word or phrase in the lexicon that appears in the input text.
We also augment the input features with binary flags to indicate the presence or absence of each hate speech lexicon in the post.
We do this by first tokenizing the post using the BERT tokenizer and then adding an extra feature vector of 0s and 1s to represent the presence or absence of each hate speech lexicon.
Then, we concatenate the feature vector with the BERT embeddings and pass it through the model. We utilize BERT-based models, like BERT-base~\cite{devlin2018bert}, BERT-large~\cite{devlin2018bert}, and RoBERTa~\cite{liu2019roberta} and state-of-the-art pre trained BERT-model for hatespeech detection Detoxify~\cite{Detoxify}, BERT-HateXplain~\cite{Mathew_Saha_Yimam_Biemann_Goyal_Mukherjee_2021} and HurtBERT~\cite{hurtbert2020} in our analysis and approach development.
\section{Results and Evaluation}

In this section, we first present the results of each of the previous models using our dataset and compare the results that these models give using the updated lexicons (RQ1).
We also present the results of our hybrid approach that predicted whether a post contains hate speech or not using a modified version of BERT (RQ2). 

\subsection{Data Preprocessing}
To test our approach we use the publicly available datasets. To evaluate our models, we employ average model accuracy, F1-measure, as well as class-specific precision and recall.
While the accuracy and F1 values offer a broad overview of the model's performance, the precision and recall scores for each class provide more specific information.

\subsection{RQ1: Evaluating Our Adaptive Approach for Lexicon Improvement}
\textcolor{black}{To evaluate the effectiveness of our adaptive lexicon approach, we used the models provided by~\citeauthor{davidson2017automated}\cite{davidson2017automated}for several reasons. First, these models are well-established baselines in the field, frequently cited and used for benchmarking new approaches to hate speech detection. Second, they rely on traditional lexicon-based methods, making them ideal candidates to demonstrate the improvements achieved by our adaptive lexicon updates. Our goal for RQ1 is to \textit{validate} that our lexicon updating method enhances the performance of existing models by aligning them with evolving language trends.
We tested the models provided by Davidson et al.~\cite{davidson2017automated} using the new dataset from Founta et al.~\cite{founta2018large}. We found that the accuracy dropped from the originally reported ~90\% during training to 76\% in our tests. This indicates that language evolves over time and that toxic lexicons must be updated to remain effective for detecting toxic language. Next, we utilized the same models with newer datasets but incorporated updated lexicons to validate our approach. We implemented and evaluated the Support Vector Machine (SVM) and Random Forest (RF) classifiers provided by Davidson et al. to detect hate speech, using the 100,000 social media posts from Founta et al.~\cite{founta2018large} as training and testing datasets.}

\begin{table}
\caption{Model performance across different word embedding lexicons for traditional models.}
\label{tab:supervisedPerformance}
\begin{center}
\small
\begin{tabular}{|l|l|lllll|}
\hline
\textbf{Features} & \textbf{Lexicon Size} & \multicolumn{1}{l|}{\textbf{Class}} & \multicolumn{1}{l|}{\textbf{Prec.}} & \multicolumn{1}{l|}{\textbf{Rec.}} & \multicolumn{1}{l|}{\textbf{F1}} & \textbf{Accr.} \\ \hline
\textbf{Linear SVM} &  &  &  &  &  &  \\ \hline
\multirow{2}{*}{$S_{lexicons}$} & \multirow{2}{*}{749} & \multicolumn{1}{l|}{Hate} & \multicolumn{1}{l|}{0.69} & \multicolumn{1}{l|}{0.77} & \multicolumn{1}{l|}{0.73} & \multirow{2}{*}{0.76} \\ \cline{3-6}
 &  & \multicolumn{1}{l|}{Normal} & \multicolumn{1}{l|}{0.73} & \multicolumn{1}{l|}{0.65} & \multicolumn{1}{l|}{0.69} &  \\ \hline
\multirow{2}{*}{$U_{Word2Vec}$} & \multirow{2}{*}{1006} & \multicolumn{1}{l|}{Hate} & \multicolumn{1}{l|}{0.89} & \multicolumn{1}{l|}{0.68} & \multicolumn{1}{l|}{0.81} & \multirow{2}{*}{0.77} \\ \cline{3-6}
 &  & \multicolumn{1}{l|}{Normal} & \multicolumn{1}{l|}{0.77} & \multicolumn{1}{l|}{0.99} & \multicolumn{1}{l|}{0.87} &  \\ \hline
\multirow{2}{*}{$U_{GloVe}$} & \multirow{2}{*}{1010} & \multicolumn{1}{l|}{Hate} & \multicolumn{1}{l|}{0.83} & \multicolumn{1}{l|}{0.77} & \multicolumn{1}{l|}{0.80} & \multirow{2}{*}{0.82} \\ \cline{3-6}
 &  & \multicolumn{1}{l|}{Normal} & \multicolumn{1}{l|}{0.70} & \multicolumn{1}{l|}{0.76} & \multicolumn{1}{l|}{0.73} &  \\ \hline
\multirow{2}{*}{$U_{BERT}$} & \multirow{2}{*}{1433} & \multicolumn{1}{l|}{Hate} & \multicolumn{1}{l|}{0.90} & \multicolumn{1}{l|}{0.70} & \multicolumn{1}{l|}{0.79} & \multirow{2}{*}{0.82} \\ \cline{3-6}
 &  & \multicolumn{1}{l|}{Normal} & \multicolumn{1}{l|}{0.74} & \multicolumn{1}{l|}{0.92} & \multicolumn{1}{l|}{0.82} &  \\ \hline
\textbf{Random Forest} &  &  &  &  &  &  \\ \hline
\multirow{2}{*}{$S_{lexicons}$} & \multirow{2}{*}{749} & \multicolumn{1}{l|}{Hate} & \multicolumn{1}{l|}{0.74} & \multicolumn{1}{l|}{0.74} & \multicolumn{1}{l|}{0.74} & \multirow{2}{*}{0.79} \\ \cline{3-6}
 &  & \multicolumn{1}{l|}{Normal} & \multicolumn{1}{l|}{0.62} & \multicolumn{1}{l|}{0.62} & \multicolumn{1}{l|}{0.62} &  \\ \hline
\multirow{2}{*}{$U_{Word2Vec}$} & \multirow{2}{*}{1006} & \multicolumn{1}{l|}{Hate} & \multicolumn{1}{l|}{0.90} & \multicolumn{1}{l|}{0.70} & \multicolumn{1}{l|}{0.79} & \multirow{2}{*}{0.82} \\ \cline{3-6}
 &  & \multicolumn{1}{l|}{Normal} & \multicolumn{1}{l|}{0.74} & \multicolumn{1}{l|}{0.92} & \multicolumn{1}{l|}{0.82} &  \\ \hline
\multirow{2}{*}{$U_{GloVe}$} & \multirow{2}{*}{1010} & \multicolumn{1}{l|}{Hate} & \multicolumn{1}{l|}{0.94} & \multicolumn{1}{l|}{0.68} & \multicolumn{1}{l|}{0.79} & \multirow{2}{*}{0.83} \\ \cline{3-6}
 &  & \multicolumn{1}{l|}{Normal} & \multicolumn{1}{l|}{0.76} & \multicolumn{1}{l|}{0.96} & \multicolumn{1}{l|}{0.85} &  \\ \hline
\multirow{2}{*}{$U_{BERT}$} & \multirow{2}{*}{1433} & \multicolumn{1}{l|}{Hate} & \multicolumn{1}{l|}{0.86} & \multicolumn{1}{l|}{0.93} & \multicolumn{1}{l|}{0.89} & \multirow{2}{*}{\textbf{0.85}} \\ \cline{3-6}
 &  & \multicolumn{1}{l|}{Normal} & \multicolumn{1}{l|}{0.91} & \multicolumn{1}{l|}{0.84} & \multicolumn{1}{l|}{0.87} & \\ \hline
\end{tabular}%
\end{center}
\end{table}


\textcolor{black}{Table~\ref{tab:supervisedPerformance} presents the performance metrics of traditional machine learning models using different feature sets, which include lexicons derived from various word embedding models (Word2Vec, GloVe, and BERT). Overall, we find that the Random Forest model with lexicons updated through BERT achieves the highest accuracy at 0.85, outperforming other classifiers. When using only the seed lexicons $S_{lexicons}$, accuracy is lower compared to the updated lexicons generated by the word embedding models. Additionally, the model demonstrates strong class-specific precision and recall. For hate speech, recall (0.93) exceeds precision (0.86), while for normal content, precision (0.91) is higher than recall (0.84).}

\subsection{RQ2: Evaluating Our Hybrid Approach to Risk Detection}

In this section, we evaluate six different BERT-based models: BERT-base~\cite{devlin2018bert}, BERT-large~\cite{devlin2018bert}, RoBERTa~\cite{liu2019roberta}, and modified pre-trained BERT models for hate speech detection, including Detoxify~\cite{Detoxify}, BERT-HateXplain~\cite{Mathew_Saha_Yimam_Biemann_Goyal_Mukherjee_2021}, and HurtBERT~\cite{hurtbert2020}. These models are tested on six different test sets, as described in section~\ref{testset}.

For each BERT-based model, we evaluate performance across six different test sets. Table~\ref{tab:BERTPerformance} summarizes the performance metrics of these models using three feature sets: without lexicons ($W$), with seed lexicons ($S_{lexicons}$), and with the best-performing lexicons derived from BERT ($U_{BERT}$). Overall, we find that Detoxify and BERT-HateXplain outperform the other BERT models.

\begin{table*}[ht]
\centering
\caption{Performance of different BERT-based models for hate speech detection using different feature sets.}
\label{tab:BERTPerformance}
\small
\begin{tabular}{||c|c|c|c|c|c|c||c|c|c|c|c|c||}
\hline
\multirow{3}{*}{\textbf{TestSet}} & \multicolumn{6}{c||}{\textbf{BERT Base}} & \multicolumn{6}{c||}{\textbf{BERT Large}} \\ \cline{2-13} 
 & \multicolumn{2}{c|}{$W$} & \multicolumn{2}{c|}{$S_{lexicons}$} & \multicolumn{2}{c||}{$U_{BERT}$} & \multicolumn{2}{c|}{$W$} & \multicolumn{2}{c|}{$S_{lexicons}$} & \multicolumn{2}{c||}{$U_{BERT}$} \\ \cline{2-13}
 & \textbf{F1} & \textbf{Accr.} & \textbf{F1} & \textbf{Accr.} & \textbf{F1} & \textbf{Accr.} & \textbf{F1} & \textbf{Accr.} & \textbf{F1} & \textbf{Accr.} & \textbf{F1} & \textbf{Accr.} \\ \hline
 
\citeauthor{davidson2017automated}\cite{davidson2017automated} & 0.68 & 0.69 & 0.68 & 0.69 & 0.71 & 0.78 & 0.68 & 0.69 & 0.65 & 0.72 & 0.74 & 0.78 \\ \hline
\citeauthor{founta2018large}\cite{founta2018large} & 0.68 & 0.68 & 0.73 & 0.75 & 0.72 & 0.75 & 0.68 & 0.67 & 0.71 & 0.72 & 0.81 & 0.81 \\ \hline
Implicit Hate~\cite{elsherief-etal-2021-latent} & 0.67 & 0.72 & 0.79 & 0.70 & 0.79 & 0.70 & 0.67 & 0.72 & 0.79 & 0.70 & 0.79 & 0.71 \\ \hline
HateCheck~\cite{rottger-etal-2021-hatecheck} & 0.69 & 0.78 & 0.86 & 0.80 & 0.87 & 0.80 & 0.70 & 0.78 & 0.86 & 0.80 & 0.86 & 0.80 \\ \hline
ToxicSpan~\cite{pavlopoulos-etal-2022-acl} & 0.74 & 0.83 & 0.87 & 0.87 & 0.89 & 0.84 & 0.74 & 0.83 & 0.89 & 0.87 & 0.89 & 0.85 \\ \hline
ToxiGen~\cite{hartvigsen-etal-2022-toxigen} & 0.73 & 0.81 & 0.81 & 0.85 & 0.89 & 0.85 & 0.74 & 0.85 & 0.89 & 0.85 & 0.87 & 0.86 \\ \hline

\multicolumn{13}{c}{} \\ \hline
\multirow{3}{*}{\textbf{TestSet}} & \multicolumn{6}{c||}{\textbf{RoBERTa}} & \multicolumn{6}{c||}{\textbf{Detoxify}} \\ \cline{2-13} 
 & \multicolumn{2}{c|}{$W$} & \multicolumn{2}{c|}{$S_{lexicons}$} & \multicolumn{2}{c||}{$U_{BERT}$} & \multicolumn{2}{c|}{$W$} & \multicolumn{2}{c|}{$S_{lexicons}$} & \multicolumn{2}{c||}{$U_{BERT}$} \\ \cline{2-13}
 & \textbf{F1} & \textbf{Accr.} & \textbf{F1} & \textbf{Accr.} & \textbf{F1} & \textbf{Accr.} & \textbf{F1} & \textbf{Accr.} & \textbf{F1} & \textbf{Accr.} & \textbf{F1} & \textbf{Accr.} \\ \hline
 
\citeauthor{davidson2017automated}\cite{davidson2017automated} & 0.74 & 0.78 & 0.74 & 0.79 & 0.71 & 0.78 
                             & 0.81 & 0.79 & 0.85 & 0.86 & \textbf{0.84} & \textbf{0.88} \\ \hline
                             
\citeauthor{founta2018large}\cite{founta2018large} & 0.76 & 0.79 & 0.75 & 0.82 & 0.74 & 0.81
                       & 0.84 & 0.87 & 0.91 & 0.92 & \textbf{0.91} & \textbf{0.94} \\ \hline
                       
Implicit Hate~\cite{elsherief-etal-2021-latent} & 0.73 & 0.72 & 0.79 & 0.70 & 0.79 & 0.70 
                                                & 0.83 & 0.82 & 0.79 & 0.80 & \textbf{0.89} & \textbf{0.91} \\ \hline
                                                
HateCheck~\cite{rottger-etal-2021-hatecheck} & 0.73 & 0.76 & 0.79 & 0.79 & 0.79 & 0.80
                                             & 0.80 & 0.88 & 0.94 & 0.90 & 0.96 & 0.91 \\ \hline
                                             
ToxicSpan~\cite{pavlopoulos-etal-2022-acl} & 0.76 & 0.83 & 0.87 & 0.87 & 0.84 & 0.84 
                                           & 0.84 & 0.84 & 0.89 & 0.87 & 0.89 & 0.85 \\ \hline
                                           
ToxiGen~\cite{hartvigsen-etal-2022-toxigen} & 0.83 & 0.81 & 0.81 & 0.85 & 0.89 & 0.85 
                                            & 0.84 & 0.85 & 0.89 & 0.85 & 0.87 & 0.86 \\ \hline

\multicolumn{13}{c}{} \\ \hline
\multirow{3}{*}{\textbf{TestSet}} & \multicolumn{6}{c||}{\textbf{HurtBERT}} & \multicolumn{6}{c||}{\textbf{BERT-HateXplain}} \\ \cline{2-13} 
 & \multicolumn{2}{c|}{$W$} & \multicolumn{2}{c|}{$S_{lexicons}$} & \multicolumn{2}{c||}{$U_{BERT}$} & \multicolumn{2}{c|}{$W$} & \multicolumn{2}{c|}{$S_{lexicons}$} & \multicolumn{2}{c||}{$U_{BERT}$} \\ \cline{2-13}
 & \textbf{F1} & \textbf{Accr.} & \textbf{F1} & \textbf{Accr.} & \textbf{F1} & \textbf{Accr.} & \textbf{F1} & \textbf{Accr.} & \textbf{F1} & \textbf{Accr.} & \textbf{F1} & \textbf{Accr.} \\ \hline
 
\citeauthor{davidson2017automated}\cite{davidson2017automated} & 0.81 & 0.81 & 0.89 & 0.89 & 0.81 & 0.85 
                             & 0.85 & 0.82 & 0.86 & 0.89 & \textbf{0.84} & 0.83 \\ \hline
                             
\citeauthor{founta2018large}\cite{founta2018large} & 0.81 & 0.82 & 0.83 & 0.85 & 0.84 & 0.85 
                       & 0.84 & 0.84 & 0.84 & 0.87 & 0.82 & 0.85 \\ \hline
                       
Implicit Hate~\cite{elsherief-etal-2021-latent} & 0.73 & 0.76 & 0.79 & 0.79 & 0.79 & 0.81 & 0.73 & 0.77 & 0.80 & 0.82 & 0.80 & 0.84 \\ \hline

HateCheck~\cite{rottger-etal-2021-hatecheck} & 0.76 & 0.78 & 0.86 & 0.91 & 0.87 & 0.91 & 0.79 & 0.81 & 0.86 & 0.92 & \textbf{0.86} & \textbf{0.93} \\ \hline

ToxicSpan~\cite{pavlopoulos-etal-2022-acl} & 0.84 & 0.83 & 0.87 & 0.87 & 0.89 & 0.93 & 0.84 & 0.83 & 0.89 & 0.87 & \textbf{0.89} & \textbf{0.95} \\ \hline

ToxiGen~\cite{hartvigsen-etal-2022-toxigen} & 0.83 & 0.86 & 0.91 & 0.95 & 0.92 & 0.95 & 0.84 & 0.85 & 0.89 & 0.95 & \textbf{0.92} & \textbf{0.96} \\ \hline


\end{tabular}
\end{table*}




\section{Interesting Case Studies}
To gain further insights into the performance of our hybrid model, we conduct an in-depth qualitative analysis.
We found that aggressors employ various sneaky methods to \textit{conceal} slurs and hate speech, often making it challenging to detect and address.
Here are some categories that encompass these tactics:
\begin{itemize}[leftmargin=*]
    \item \textbf{Introducing new hate speech lexicons:} As online platforms implement measures to combat hate speech, aggressors adapt by using alternative terms, neologisms, or coded language to express their hateful ideas without triggering automated filters or detection systems making it difficult for outsiders or automated tools to immediately recognize the underlying hate speech. For example, the word ``shitskins'' (Example 1) and ``salads'' (Example 2) are used as hate words in the following social media posts.

    \vspace{0.1in}
    \begin{graybox}
    \textbf{Example 1: }``Ive seen videos of Muslim shitskins dividing a single person into multiple pieces.''\\
    \textbf{Example 2: }``of course I'm over the limit I'm on a night out you fucking salads''
    \end{graybox}
    \vspace{0.1in}

    \item \textbf{Spelling Errors:} Aggressors intentionally misspell words related to hate speech or use deliberate variations in spelling to bypass content filters. In Examples 3, 4, and 5 we illustrate some of the spelling errors made.

    \vspace{0.1in}
    \begin{graybox}
    \textbf{Example 3: }``Y'all \textbf{niggaz} evil af''\\
    \textbf{Example 4: }``If you'll see me holding up my middle finger to the world. \textbf{Fck} ur ribbons and ur pearls.'' \\
    \textbf{Example 5: }``This shit got me \textbf{fuckin} CRYINGG!! Cuz the \textbf{lil nigga} aint even want this stupid cut just look \@ his face''
    \end{graybox}
    \vspace{0.1in}
    
    \item \textbf{Adding Punctuation:} Another tactic employed by aggressors is the insertion of special characters or punctuation marks within offensive words or slurs to obscure or obfuscate the offensive language. For example, adding an apostrophe like ``nas.ty'' (Example 6) or an underscore like ``x\_x'' (Example 7).

    \vspace{0.1in}
    \begin{graybox}
    \textbf{Example 6: }''I just like \textbf{nas.ty} shit men`` \\
    \textbf{Example 7: }''When u pounding the \textbf{x\_x} like u don't wna``
    \end{graybox}
    \vspace{0.1in}

    \item \textbf{Implied Hate:} Aggressors often resort to implied hate, where they use veiled language (Example 8), innuendos, sarcasm (Example 9), or ambiguous statements (Example 10) to convey discriminatory or hateful ideas indirectly.
    \vspace{0.1in}
    \begin{graybox}
    \textbf{Example 8: }``To bad u couldn't box the hell out I'd be even prouder'' \\
    \textbf{Example 9: }``You are a chicken nugget and soy milk'' \\
    \textbf{Example 10: }``Latina backwards spells crazy as hell in 2 languages''
    \end{graybox}

\end{itemize}

Our findings reveal that our model exhibits a higher proficiency in identifying instances of hate speech when substitute lexicons are employed especially to bypass already in place moderation systems.

\subsection{Comparing Our Hybrid Approach with State-of-The-Art Moderate Hate Speech API}

Moderate Hate Speech API~\cite{moderatehatespeech} is a Google Cloud service that helps identify and moderate hate speech.
It can be used to moderate content in a variety of applications, including social media platforms, forums, and news websites.
For each detected hate speech token, the API returns a confidence score which indicates how likely it is that the token is hate speech. However, it is important to note that the API is not perfect. It sometimes misidentifies content as hate speech, and it can also sometimes fail to identify hate speech as reported on its website~\cite{moderatehatespeech}. We find that our model detects hate and toxicity towards vulnerable populations especially women and the black community.

We use this API to compare our hybrid model.
We use the 76,378 unlabeled posts for this purpose.
We find that our model detects 663 posts as hate speech out of 76,378 posts whereas Moderate Hate Speech API detects 678 posts as hate speech. Our model detects 65 different posts than Moderate Hate Speech API, and upon manual analysis, we find that most of the posts that our model detected contained new toxic lexicons for example ``sigma'' (Example 11), ``karen,'' ``thot''(Example 12),  etc.

\vspace{0.1in}
\begin{graybox}
\textbf{Example 11: }``typical sigma behavior''\\
\textbf{Example 12: }``i am your local thot''
\end{graybox}
\vspace{0.1in}

There were other examples where the API failed where harsher emotions or words were used for example in Example 13 ``liberal Stalinists'' is used negatively:

\vspace{0.1in}
\begin{graybox}
\textbf{Example 13: }``So, for the first time ever since 2017, America is a communist nation again. liberal Stalinists!!''
\end{graybox}
\vspace{0.1in}

There were other cases where sexual harassment towards women was missed by the API for example (Examples 14 and 15):

\vspace{0.1in}
\begin{graybox}
\textbf{Example 14: }``I just wanna be a good bun, having someone clip a leash to my collar and take me for a walk, letting anyone who asks fuck and breed me, then getting headpats and scritches after''\\
\textbf{Example 15: }``Anyone else wanna help me breed her..''
\end{graybox}
\vspace{0.1in}

On the other hand, our model performs poorly when the hate speech lexicons were not part of the initial diagnosis, for example in the following post (Example 16) the lexicons ``xenophobes'' and ``halfwits'' were not part of the toxic lexicon list and hence this post is not flagged by our model but was detected by Moderate Hate Speech API.
The Moderate Hate Speech API detects 80 different posts than our model. 

\vspace{0.1in}
\begin{graybox}
\textbf{Example 16: }``RT @username: @username My business is in services. Xenophobes and halfwits like yourself destroyed the EU side of my business.''
\end{graybox}
\vspace{0.1in}

However, we also find that Moderate Hate Speech is biased towards black people (This has also been confirmed in the documentation of this API~\cite{moderatehatespeech}), for example, the following posts (Examples 17 and 18) from our dataset are labeled as hate speech by this model, however upon manual analysis, we can see that they are clearly not hate speech.

\vspace{0.1in}
\begin{graybox}
\textbf{Example 17: }``@username: Did you know a disabled Black woman invented the walker, toilet paper holder, and sanitary belt?''\\
\textbf{Example 18: }``@username: the older black generation be saying some questionable things.''
\end{graybox}
\vspace{0.1in}

We discover that our hybrid model goes beyond existing approaches by addressing the dynamic nature of language, adapting to new vocabulary, and evolving linguistic patterns. It also helps identify toxicity towards vulnerable populations that were not mentioned in the original lexicon dataset. 
\section{Discussion and limitations}
The results presented in Section~\ref{sec:experiments} show that compared to baseline methods, Xen provides a more effective solution that can enable an autonomous car to detect on-road anomalies in diverse driving scenarios using a single monocular camera. Despite the advantages, our work also encompasses several limitations.

Object detection plays an important role in Xen, as the interaction and behavior expert work under the assumption that anomalous objects can be reliably detected for trajectory reconstruction and prediction, respectively. However, the perception capability of monocular cameras is largely limited when the visibility is poor, such as at night and under inclement weather conditions. To alleviate the issue, camera-LiDAR fusion has been proposed and shown more effective than unimodal approaches in computer vision tasks~\citep{cui2021deep,chen2017multi,sindagi2019mvx}. With an additional sensor modality, object detection and thus anomaly detection can be made more robust in different environments. Furthermore, as noted in Section~\ref{subsec:bm}, perspective projection onto the image plane distorts the motion characteristics of objects (e.g., a proximate object appears to move faster than a distant object even though the two objects have the same speed in reality), which challenges efficient modeling of normal motion patterns. 3D object detection enabled by point clouds from LiDAR has the potential to resolve the issue by projecting bounding boxes to bird's eye view (BEV) and thus eliminating the negative effect of perspective projection on learning object motions.

Another common failure case of Xen results from large scene motions in normal scenarios, e.g., when the ego car executes an aggressive lane change or moves fast in complex urban areas. Frame prediction becomes difficult in such cases due to large motions of the ego car, and the resulting increase of score is indistinguishable from that caused by an anomaly. It has been shown recently in video prediction literature that camera poses are helpful in rendering high-quality images~\citep{ak2021robust}. As a result, given that additional onboard vehicle state is available, ego motions can be exploited to create a more robust scene expert. Another similar issue that can cause false positives in Xen is discussed in supplemental materials.

Anomaly detection is an active research topic both in robotics and computer vision. At a more general level, we hope that the analysis in this work, especially those in Section~\ref{sec:overview}, provides insights on a unified framework for anomaly detection in related areas. More specifically, an anomaly detector can be designed based on Figure~\ref{fig:anomaly-patterns} with necessary modifications for different applications. For example, for AD on field robots which operate in autonomous farms without human labors, only the edge between the ego agent and the environment needs to be monitored as the robot often performs a task individually; for AD with surveillance cameras, the two edges with one of the ends being the ego agent can be ignored as the surveillance camera is fixed and will never participate in an anomaly; and for AD on mobile robots that navigate through human crowds, the whole graph needs to be considered if non-ego involved anomalies also affect robot decisions. With the high-level framework determined, each expert can then be designed specifically for each type of edge based on the characteristics of different anomalies. Validating the generalization capability of Xen in other application domains is left as future work.

Another possible direction is to evaluate the efficacy of more complex architectures, such as foundation models, for anomaly detection. Large visual language models (LVLMs) have been shown powerful in a variety of application areas, including image captioning, content generation, and conversational AI~\citep{jiang2024effectiveness}. In the domain of autonomous driving, LVLMs have also been explored for tasks of visual question-answering~\citep{xu2024drivegpt4}, trajectory prediction~\citep{wu2023language}, path planning~\citep{mao2023language}, and decision-making and control~\citep{wen2023road}. These recent research advancements suggest that incorporating foundation models into on-road anomaly detection is a promising direction.

Although powerful, LVLMs are currently limited in efficiency due to billions of parameters~\citep{brohan2022rt,brohan2023rt,padalkar2023open}. Furthermore, proprietary models, such as GPT-4V, must be queried over the cloud, further increasing inference time~\citep{achiam2023gpt}. To ensure both accuracy and efficiency of on-road anomaly detection with limited onboard resources, a combination of LVLMs and lightweight models is necessary. One integration method is to retrieve intermediate embeddings of images through the LVLM, which can then be provided as an additional context to lightweight anomaly detectors for inference. The embeddings from the LVLM can be updated periodically for efficiency. Such a method, however, requires access to the hidden states of the LVLM, which most proprietary models do not allow. Alternatively, LVLMs can be used as an additional anomaly detection expert, which can then be incorporated into Xen through the Kalman filter. While the three original experts update the system states of Kalman filter at a high frequency, the LVLM can be queried at a low frequency and updates the system states asynchronously in a similar manner. With such an approach, we are able to benefit from both the efficiency of lightweight models and the accuracy and generalization ability of LVLMs.
\section{Conclusion}
This paper examined the challenges of authoring site-specific outdoor AR experiences, which are often constrained by incomplete and outdated world representations and limited access to evolving real-world conditions. Our formative study revealed that developers and designers frequently encounter these limitations, necessitating costly and time-consuming on-site visits to capture environmental details, assess user flow, and ensure contextual relevance. Based on these insights, we identified key requirements for integrating real-world context into remote authoring workflows, leading to the development of \SystemName, an asymmetric collaborative authoring system that facilitates synchronous collaboration between \exsitu (i.e., \textit{off-site}, \textit{remote}) developers and \insitu (i.e., \textit{on-site}) collaborators.

Our exploratory user study demonstrated that this approach mitigates key challenges by enhancing confidence in authored results, stimulating engagement and creativity, and enabling direct iterative refinements informed by up-to-date environmental data. At the same time, our findings highlight multitasking demands as a challenge in synchronous collaboration and emphasize the need for a balanced integration of synchronous and asynchronous workflows. Situating these findings within the broader landscape of AR authoring and remote collaboration, we provided recommendations for future work. We hope this research motivates further exploration of methods for building, testing, and evaluating site-specific AR experiences, contributing to the future of immersive and contextually grounded interactive applications.


\bibliographystyle{ACM-Reference-Format}
% This must be in the first 5 lines to tell arXiv to use pdfLaTeX, which is strongly recommended.
\pdfoutput=1
% In particular, the hyperref package requires pdfLaTeX in order to break URLs across lines.

\documentclass[11pt]{article}

% Change "review" to "final" to generate the final (sometimes called camera-ready) version.
% Change to "preprint" to generate a non-anonymous version with page numbers.
\usepackage{acl}

% Standard package includes
\usepackage{times}
\usepackage{latexsym}

% Draw tables
\usepackage{booktabs}
\usepackage{multirow}
\usepackage{xcolor}
\usepackage{colortbl}
\usepackage{array} 
\usepackage{amsmath}

\newcolumntype{C}{>{\centering\arraybackslash}p{0.07\textwidth}}
% For proper rendering and hyphenation of words containing Latin characters (including in bib files)
\usepackage[T1]{fontenc}
% For Vietnamese characters
% \usepackage[T5]{fontenc}
% See https://www.latex-project.org/help/documentation/encguide.pdf for other character sets
% This assumes your files are encoded as UTF8
\usepackage[utf8]{inputenc}

% This is not strictly necessary, and may be commented out,
% but it will improve the layout of the manuscript,
% and will typically save some space.
\usepackage{microtype}
\DeclareMathOperator*{\argmax}{arg\,max}
% This is also not strictly necessary, and may be commented out.
% However, it will improve the aesthetics of text in
% the typewriter font.
\usepackage{inconsolata}

%Including images in your LaTeX document requires adding
%additional package(s)
\usepackage{graphicx}
% If the title and author information does not fit in the area allocated, uncomment the following
%
%\setlength\titlebox{<dim>}
%
% and set <dim> to something 5cm or larger.

\title{Wi-Chat: Large Language Model Powered Wi-Fi Sensing}

% Author information can be set in various styles:
% For several authors from the same institution:
% \author{Author 1 \and ... \and Author n \\
%         Address line \\ ... \\ Address line}
% if the names do not fit well on one line use
%         Author 1 \\ {\bf Author 2} \\ ... \\ {\bf Author n} \\
% For authors from different institutions:
% \author{Author 1 \\ Address line \\  ... \\ Address line
%         \And  ... \And
%         Author n \\ Address line \\ ... \\ Address line}
% To start a separate ``row'' of authors use \AND, as in
% \author{Author 1 \\ Address line \\  ... \\ Address line
%         \AND
%         Author 2 \\ Address line \\ ... \\ Address line \And
%         Author 3 \\ Address line \\ ... \\ Address line}

% \author{First Author \\
%   Affiliation / Address line 1 \\
%   Affiliation / Address line 2 \\
%   Affiliation / Address line 3 \\
%   \texttt{email@domain} \\\And
%   Second Author \\
%   Affiliation / Address line 1 \\
%   Affiliation / Address line 2 \\
%   Affiliation / Address line 3 \\
%   \texttt{email@domain} \\}
% \author{Haohan Yuan \qquad Haopeng Zhang\thanks{corresponding author} \\ 
%   ALOHA Lab, University of Hawaii at Manoa \\
%   % Affiliation / Address line 2 \\
%   % Affiliation / Address line 3 \\
%   \texttt{\{haohany,haopengz\}@hawaii.edu}}
  
\author{
{Haopeng Zhang$\dag$\thanks{These authors contributed equally to this work.}, Yili Ren$\ddagger$\footnotemark[1], Haohan Yuan$\dag$, Jingzhe Zhang$\ddagger$, Yitong Shen$\ddagger$} \\
ALOHA Lab, University of Hawaii at Manoa$\dag$, University of South Florida$\ddagger$ \\
\{haopengz, haohany\}@hawaii.edu\\
\{yiliren, jingzhe, shen202\}@usf.edu\\}



  
%\author{
%  \textbf{First Author\textsuperscript{1}},
%  \textbf{Second Author\textsuperscript{1,2}},
%  \textbf{Third T. Author\textsuperscript{1}},
%  \textbf{Fourth Author\textsuperscript{1}},
%\\
%  \textbf{Fifth Author\textsuperscript{1,2}},
%  \textbf{Sixth Author\textsuperscript{1}},
%  \textbf{Seventh Author\textsuperscript{1}},
%  \textbf{Eighth Author \textsuperscript{1,2,3,4}},
%\\
%  \textbf{Ninth Author\textsuperscript{1}},
%  \textbf{Tenth Author\textsuperscript{1}},
%  \textbf{Eleventh E. Author\textsuperscript{1,2,3,4,5}},
%  \textbf{Twelfth Author\textsuperscript{1}},
%\\
%  \textbf{Thirteenth Author\textsuperscript{3}},
%  \textbf{Fourteenth F. Author\textsuperscript{2,4}},
%  \textbf{Fifteenth Author\textsuperscript{1}},
%  \textbf{Sixteenth Author\textsuperscript{1}},
%\\
%  \textbf{Seventeenth S. Author\textsuperscript{4,5}},
%  \textbf{Eighteenth Author\textsuperscript{3,4}},
%  \textbf{Nineteenth N. Author\textsuperscript{2,5}},
%  \textbf{Twentieth Author\textsuperscript{1}}
%\\
%\\
%  \textsuperscript{1}Affiliation 1,
%  \textsuperscript{2}Affiliation 2,
%  \textsuperscript{3}Affiliation 3,
%  \textsuperscript{4}Affiliation 4,
%  \textsuperscript{5}Affiliation 5
%\\
%  \small{
%    \textbf{Correspondence:} \href{mailto:email@domain}{email@domain}
%  }
%}

\begin{document}
\maketitle
\begin{abstract}
Recent advancements in Large Language Models (LLMs) have demonstrated remarkable capabilities across diverse tasks. However, their potential to integrate physical model knowledge for real-world signal interpretation remains largely unexplored. In this work, we introduce Wi-Chat, the first LLM-powered Wi-Fi-based human activity recognition system. We demonstrate that LLMs can process raw Wi-Fi signals and infer human activities by incorporating Wi-Fi sensing principles into prompts. Our approach leverages physical model insights to guide LLMs in interpreting Channel State Information (CSI) data without traditional signal processing techniques. Through experiments on real-world Wi-Fi datasets, we show that LLMs exhibit strong reasoning capabilities, achieving zero-shot activity recognition. These findings highlight a new paradigm for Wi-Fi sensing, expanding LLM applications beyond conventional language tasks and enhancing the accessibility of wireless sensing for real-world deployments.
\end{abstract}

\section{Introduction}

In today’s rapidly evolving digital landscape, the transformative power of web technologies has redefined not only how services are delivered but also how complex tasks are approached. Web-based systems have become increasingly prevalent in risk control across various domains. This widespread adoption is due their accessibility, scalability, and ability to remotely connect various types of users. For example, these systems are used for process safety management in industry~\cite{kannan2016web}, safety risk early warning in urban construction~\cite{ding2013development}, and safe monitoring of infrastructural systems~\cite{repetto2018web}. Within these web-based risk management systems, the source search problem presents a huge challenge. Source search refers to the task of identifying the origin of a risky event, such as a gas leak and the emission point of toxic substances. This source search capability is crucial for effective risk management and decision-making.

Traditional approaches to implementing source search capabilities into the web systems often rely on solely algorithmic solutions~\cite{ristic2016study}. These methods, while relatively straightforward to implement, often struggle to achieve acceptable performances due to algorithmic local optima and complex unknown environments~\cite{zhao2020searching}. More recently, web crowdsourcing has emerged as a promising alternative for tackling the source search problem by incorporating human efforts in these web systems on-the-fly~\cite{zhao2024user}. This approach outsources the task of addressing issues encountered during the source search process to human workers, leveraging their capabilities to enhance system performance.

These solutions often employ a human-AI collaborative way~\cite{zhao2023leveraging} where algorithms handle exploration-exploitation and report the encountered problems while human workers resolve complex decision-making bottlenecks to help the algorithms getting rid of local deadlocks~\cite{zhao2022crowd}. Although effective, this paradigm suffers from two inherent limitations: increased operational costs from continuous human intervention, and slow response times of human workers due to sequential decision-making. These challenges motivate our investigation into developing autonomous systems that preserve human-like reasoning capabilities while reducing dependency on massive crowdsourced labor.

Furthermore, recent advancements in large language models (LLMs)~\cite{chang2024survey} and multi-modal LLMs (MLLMs)~\cite{huang2023chatgpt} have unveiled promising avenues for addressing these challenges. One clear opportunity involves the seamless integration of visual understanding and linguistic reasoning for robust decision-making in search tasks. However, whether large models-assisted source search is really effective and efficient for improving the current source search algorithms~\cite{ji2022source} remains unknown. \textit{To address the research gap, we are particularly interested in answering the following two research questions in this work:}

\textbf{\textit{RQ1: }}How can source search capabilities be integrated into web-based systems to support decision-making in time-sensitive risk management scenarios? 
% \sq{I mention ``time-sensitive'' here because I feel like we shall say something about the response time -- LLM has to be faster than humans}

\textbf{\textit{RQ2: }}How can MLLMs and LLMs enhance the effectiveness and efficiency of existing source search algorithms? 

% \textit{\textbf{RQ2:}} To what extent does the performance of large models-assisted search align with or approach the effectiveness of human-AI collaborative search? 

To answer the research questions, we propose a novel framework called Auto-\
S$^2$earch (\textbf{Auto}nomous \textbf{S}ource \textbf{Search}) and implement a prototype system that leverages advanced web technologies to simulate real-world conditions for zero-shot source search. Unlike traditional methods that rely on pre-defined heuristics or extensive human intervention, AutoS$^2$earch employs a carefully designed prompt that encapsulates human rationales, thereby guiding the MLLM to generate coherent and accurate scene descriptions from visual inputs about four directional choices. Based on these language-based descriptions, the LLM is enabled to determine the optimal directional choice through chain-of-thought (CoT) reasoning. Comprehensive empirical validation demonstrates that AutoS$^2$-\ 
earch achieves a success rate of 95–98\%, closely approaching the performance of human-AI collaborative search across 20 benchmark scenarios~\cite{zhao2023leveraging}. 

Our work indicates that the role of humans in future web crowdsourcing tasks may evolve from executors to validators or supervisors. Furthermore, incorporating explanations of LLM decisions into web-based system interfaces has the potential to help humans enhance task performance in risk control.






\section{Related Work}
\label{sec:relatedworks}

% \begin{table*}[t]
% \centering 
% \renewcommand\arraystretch{0.98}
% \fontsize{8}{10}\selectfont \setlength{\tabcolsep}{0.4em}
% \begin{tabular}{@{}lc|cc|cc|cc@{}}
% \toprule
% \textbf{Methods}           & \begin{tabular}[c]{@{}c@{}}\textbf{Training}\\ \textbf{Paradigm}\end{tabular} & \begin{tabular}[c]{@{}c@{}}\textbf{$\#$ PT Data}\\ \textbf{(Tokens)}\end{tabular} & \begin{tabular}[c]{@{}c@{}}\textbf{$\#$ IFT Data}\\ \textbf{(Samples)}\end{tabular} & \textbf{Code}  & \begin{tabular}[c]{@{}c@{}}\textbf{Natural}\\ \textbf{Language}\end{tabular} & \begin{tabular}[c]{@{}c@{}}\textbf{Action}\\ \textbf{Trajectories}\end{tabular} & \begin{tabular}[c]{@{}c@{}}\textbf{API}\\ \textbf{Documentation}\end{tabular}\\ \midrule 
% NexusRaven~\citep{srinivasan2023nexusraven} & IFT & - & - & \textcolor{green}{\CheckmarkBold} & \textcolor{green}{\CheckmarkBold} &\textcolor{red}{\XSolidBrush}&\textcolor{red}{\XSolidBrush}\\
% AgentInstruct~\citep{zeng2023agenttuning} & IFT & - & 2k & \textcolor{green}{\CheckmarkBold} & \textcolor{green}{\CheckmarkBold} &\textcolor{red}{\XSolidBrush}&\textcolor{red}{\XSolidBrush} \\
% AgentEvol~\citep{xi2024agentgym} & IFT & - & 14.5k & \textcolor{green}{\CheckmarkBold} & \textcolor{green}{\CheckmarkBold} &\textcolor{green}{\CheckmarkBold}&\textcolor{red}{\XSolidBrush} \\
% Gorilla~\citep{patil2023gorilla}& IFT & - & 16k & \textcolor{green}{\CheckmarkBold} & \textcolor{green}{\CheckmarkBold} &\textcolor{red}{\XSolidBrush}&\textcolor{green}{\CheckmarkBold}\\
% OpenFunctions-v2~\citep{patil2023gorilla} & IFT & - & 65k & \textcolor{green}{\CheckmarkBold} & \textcolor{green}{\CheckmarkBold} &\textcolor{red}{\XSolidBrush}&\textcolor{green}{\CheckmarkBold}\\
% LAM~\citep{zhang2024agentohana} & IFT & - & 42.6k & \textcolor{green}{\CheckmarkBold} & \textcolor{green}{\CheckmarkBold} &\textcolor{green}{\CheckmarkBold}&\textcolor{red}{\XSolidBrush} \\
% xLAM~\citep{liu2024apigen} & IFT & - & 60k & \textcolor{green}{\CheckmarkBold} & \textcolor{green}{\CheckmarkBold} &\textcolor{green}{\CheckmarkBold}&\textcolor{red}{\XSolidBrush} \\\midrule
% LEMUR~\citep{xu2024lemur} & PT & 90B & 300k & \textcolor{green}{\CheckmarkBold} & \textcolor{green}{\CheckmarkBold} &\textcolor{green}{\CheckmarkBold}&\textcolor{red}{\XSolidBrush}\\
% \rowcolor{teal!12} \method & PT & 103B & 95k & \textcolor{green}{\CheckmarkBold} & \textcolor{green}{\CheckmarkBold} & \textcolor{green}{\CheckmarkBold} & \textcolor{green}{\CheckmarkBold} \\
% \bottomrule
% \end{tabular}
% \caption{Summary of existing tuning- and pretraining-based LLM agents with their training sample sizes. "PT" and "IFT" denote "Pre-Training" and "Instruction Fine-Tuning", respectively. }
% \label{tab:related}
% \end{table*}

\begin{table*}[ht]
\begin{threeparttable}
\centering 
\renewcommand\arraystretch{0.98}
\fontsize{7}{9}\selectfont \setlength{\tabcolsep}{0.2em}
\begin{tabular}{@{}l|c|c|ccc|cc|cc|cccc@{}}
\toprule
\textbf{Methods} & \textbf{Datasets}           & \begin{tabular}[c]{@{}c@{}}\textbf{Training}\\ \textbf{Paradigm}\end{tabular} & \begin{tabular}[c]{@{}c@{}}\textbf{\# PT Data}\\ \textbf{(Tokens)}\end{tabular} & \begin{tabular}[c]{@{}c@{}}\textbf{\# IFT Data}\\ \textbf{(Samples)}\end{tabular} & \textbf{\# APIs} & \textbf{Code}  & \begin{tabular}[c]{@{}c@{}}\textbf{Nat.}\\ \textbf{Lang.}\end{tabular} & \begin{tabular}[c]{@{}c@{}}\textbf{Action}\\ \textbf{Traj.}\end{tabular} & \begin{tabular}[c]{@{}c@{}}\textbf{API}\\ \textbf{Doc.}\end{tabular} & \begin{tabular}[c]{@{}c@{}}\textbf{Func.}\\ \textbf{Call}\end{tabular} & \begin{tabular}[c]{@{}c@{}}\textbf{Multi.}\\ \textbf{Step}\end{tabular}  & \begin{tabular}[c]{@{}c@{}}\textbf{Plan}\\ \textbf{Refine}\end{tabular}  & \begin{tabular}[c]{@{}c@{}}\textbf{Multi.}\\ \textbf{Turn}\end{tabular}\\ \midrule 
\multicolumn{13}{l}{\emph{Instruction Finetuning-based LLM Agents for Intrinsic Reasoning}}  \\ \midrule
FireAct~\cite{chen2023fireact} & FireAct & IFT & - & 2.1K & 10 & \textcolor{red}{\XSolidBrush} &\textcolor{green}{\CheckmarkBold} &\textcolor{green}{\CheckmarkBold}  & \textcolor{red}{\XSolidBrush} &\textcolor{green}{\CheckmarkBold} & \textcolor{red}{\XSolidBrush} &\textcolor{green}{\CheckmarkBold} & \textcolor{red}{\XSolidBrush} \\
ToolAlpaca~\cite{tang2023toolalpaca} & ToolAlpaca & IFT & - & 4.0K & 400 & \textcolor{red}{\XSolidBrush} &\textcolor{green}{\CheckmarkBold} &\textcolor{green}{\CheckmarkBold} & \textcolor{red}{\XSolidBrush} &\textcolor{green}{\CheckmarkBold} & \textcolor{red}{\XSolidBrush}  &\textcolor{green}{\CheckmarkBold} & \textcolor{red}{\XSolidBrush}  \\
ToolLLaMA~\cite{qin2023toolllm} & ToolBench & IFT & - & 12.7K & 16,464 & \textcolor{red}{\XSolidBrush} &\textcolor{green}{\CheckmarkBold} &\textcolor{green}{\CheckmarkBold} &\textcolor{red}{\XSolidBrush} &\textcolor{green}{\CheckmarkBold}&\textcolor{green}{\CheckmarkBold}&\textcolor{green}{\CheckmarkBold} &\textcolor{green}{\CheckmarkBold}\\
AgentEvol~\citep{xi2024agentgym} & AgentTraj-L & IFT & - & 14.5K & 24 &\textcolor{red}{\XSolidBrush} & \textcolor{green}{\CheckmarkBold} &\textcolor{green}{\CheckmarkBold}&\textcolor{red}{\XSolidBrush} &\textcolor{green}{\CheckmarkBold}&\textcolor{red}{\XSolidBrush} &\textcolor{red}{\XSolidBrush} &\textcolor{green}{\CheckmarkBold}\\
Lumos~\cite{yin2024agent} & Lumos & IFT  & - & 20.0K & 16 &\textcolor{red}{\XSolidBrush} & \textcolor{green}{\CheckmarkBold} & \textcolor{green}{\CheckmarkBold} &\textcolor{red}{\XSolidBrush} & \textcolor{green}{\CheckmarkBold} & \textcolor{green}{\CheckmarkBold} &\textcolor{red}{\XSolidBrush} & \textcolor{green}{\CheckmarkBold}\\
Agent-FLAN~\cite{chen2024agent} & Agent-FLAN & IFT & - & 24.7K & 20 &\textcolor{red}{\XSolidBrush} & \textcolor{green}{\CheckmarkBold} & \textcolor{green}{\CheckmarkBold} &\textcolor{red}{\XSolidBrush} & \textcolor{green}{\CheckmarkBold}& \textcolor{green}{\CheckmarkBold}&\textcolor{red}{\XSolidBrush} & \textcolor{green}{\CheckmarkBold}\\
AgentTuning~\citep{zeng2023agenttuning} & AgentInstruct & IFT & - & 35.0K & - &\textcolor{red}{\XSolidBrush} & \textcolor{green}{\CheckmarkBold} & \textcolor{green}{\CheckmarkBold} &\textcolor{red}{\XSolidBrush} & \textcolor{green}{\CheckmarkBold} &\textcolor{red}{\XSolidBrush} &\textcolor{red}{\XSolidBrush} & \textcolor{green}{\CheckmarkBold}\\\midrule
\multicolumn{13}{l}{\emph{Instruction Finetuning-based LLM Agents for Function Calling}} \\\midrule
NexusRaven~\citep{srinivasan2023nexusraven} & NexusRaven & IFT & - & - & 116 & \textcolor{green}{\CheckmarkBold} & \textcolor{green}{\CheckmarkBold}  & \textcolor{green}{\CheckmarkBold} &\textcolor{red}{\XSolidBrush} & \textcolor{green}{\CheckmarkBold} &\textcolor{red}{\XSolidBrush} &\textcolor{red}{\XSolidBrush}&\textcolor{red}{\XSolidBrush}\\
Gorilla~\citep{patil2023gorilla} & Gorilla & IFT & - & 16.0K & 1,645 & \textcolor{green}{\CheckmarkBold} &\textcolor{red}{\XSolidBrush} &\textcolor{red}{\XSolidBrush}&\textcolor{green}{\CheckmarkBold} &\textcolor{green}{\CheckmarkBold} &\textcolor{red}{\XSolidBrush} &\textcolor{red}{\XSolidBrush} &\textcolor{red}{\XSolidBrush}\\
OpenFunctions-v2~\citep{patil2023gorilla} & OpenFunctions-v2 & IFT & - & 65.0K & - & \textcolor{green}{\CheckmarkBold} & \textcolor{green}{\CheckmarkBold} &\textcolor{red}{\XSolidBrush} &\textcolor{green}{\CheckmarkBold} &\textcolor{green}{\CheckmarkBold} &\textcolor{red}{\XSolidBrush} &\textcolor{red}{\XSolidBrush} &\textcolor{red}{\XSolidBrush}\\
API Pack~\cite{guo2024api} & API Pack & IFT & - & 1.1M & 11,213 &\textcolor{green}{\CheckmarkBold} &\textcolor{red}{\XSolidBrush} &\textcolor{green}{\CheckmarkBold} &\textcolor{red}{\XSolidBrush} &\textcolor{green}{\CheckmarkBold} &\textcolor{red}{\XSolidBrush}&\textcolor{red}{\XSolidBrush}&\textcolor{red}{\XSolidBrush}\\ 
LAM~\citep{zhang2024agentohana} & AgentOhana & IFT & - & 42.6K & - & \textcolor{green}{\CheckmarkBold} & \textcolor{green}{\CheckmarkBold} &\textcolor{green}{\CheckmarkBold}&\textcolor{red}{\XSolidBrush} &\textcolor{green}{\CheckmarkBold}&\textcolor{red}{\XSolidBrush}&\textcolor{green}{\CheckmarkBold}&\textcolor{green}{\CheckmarkBold}\\
xLAM~\citep{liu2024apigen} & APIGen & IFT & - & 60.0K & 3,673 & \textcolor{green}{\CheckmarkBold} & \textcolor{green}{\CheckmarkBold} &\textcolor{green}{\CheckmarkBold}&\textcolor{red}{\XSolidBrush} &\textcolor{green}{\CheckmarkBold}&\textcolor{red}{\XSolidBrush}&\textcolor{green}{\CheckmarkBold}&\textcolor{green}{\CheckmarkBold}\\\midrule
\multicolumn{13}{l}{\emph{Pretraining-based LLM Agents}}  \\\midrule
% LEMUR~\citep{xu2024lemur} & PT & 90B & 300.0K & - & \textcolor{green}{\CheckmarkBold} & \textcolor{green}{\CheckmarkBold} &\textcolor{green}{\CheckmarkBold}&\textcolor{red}{\XSolidBrush} & \textcolor{red}{\XSolidBrush} &\textcolor{green}{\CheckmarkBold} &\textcolor{red}{\XSolidBrush}&\textcolor{red}{\XSolidBrush}\\
\rowcolor{teal!12} \method & \dataset & PT & 103B & 95.0K  & 76,537  & \textcolor{green}{\CheckmarkBold} & \textcolor{green}{\CheckmarkBold} & \textcolor{green}{\CheckmarkBold} & \textcolor{green}{\CheckmarkBold} & \textcolor{green}{\CheckmarkBold} & \textcolor{green}{\CheckmarkBold} & \textcolor{green}{\CheckmarkBold} & \textcolor{green}{\CheckmarkBold}\\
\bottomrule
\end{tabular}
% \begin{tablenotes}
%     \item $^*$ In addition, the StarCoder-API can offer 4.77M more APIs.
% \end{tablenotes}
\caption{Summary of existing instruction finetuning-based LLM agents for intrinsic reasoning and function calling, along with their training resources and sample sizes. "PT" and "IFT" denote "Pre-Training" and "Instruction Fine-Tuning", respectively.}
\vspace{-2ex}
\label{tab:related}
\end{threeparttable}
\end{table*}

\noindent \textbf{Prompting-based LLM Agents.} Due to the lack of agent-specific pre-training corpus, existing LLM agents rely on either prompt engineering~\cite{hsieh2023tool,lu2024chameleon,yao2022react,wang2023voyager} or instruction fine-tuning~\cite{chen2023fireact,zeng2023agenttuning} to understand human instructions, decompose high-level tasks, generate grounded plans, and execute multi-step actions. 
However, prompting-based methods mainly depend on the capabilities of backbone LLMs (usually commercial LLMs), failing to introduce new knowledge and struggling to generalize to unseen tasks~\cite{sun2024adaplanner,zhuang2023toolchain}. 

\noindent \textbf{Instruction Finetuning-based LLM Agents.} Considering the extensive diversity of APIs and the complexity of multi-tool instructions, tool learning inherently presents greater challenges than natural language tasks, such as text generation~\cite{qin2023toolllm}.
Post-training techniques focus more on instruction following and aligning output with specific formats~\cite{patil2023gorilla,hao2024toolkengpt,qin2023toolllm,schick2024toolformer}, rather than fundamentally improving model knowledge or capabilities. 
Moreover, heavy fine-tuning can hinder generalization or even degrade performance in non-agent use cases, potentially suppressing the original base model capabilities~\cite{ghosh2024a}.

\noindent \textbf{Pretraining-based LLM Agents.} While pre-training serves as an essential alternative, prior works~\cite{nijkamp2023codegen,roziere2023code,xu2024lemur,patil2023gorilla} have primarily focused on improving task-specific capabilities (\eg, code generation) instead of general-domain LLM agents, due to single-source, uni-type, small-scale, and poor-quality pre-training data. 
Existing tool documentation data for agent training either lacks diverse real-world APIs~\cite{patil2023gorilla, tang2023toolalpaca} or is constrained to single-tool or single-round tool execution. 
Furthermore, trajectory data mostly imitate expert behavior or follow function-calling rules with inferior planning and reasoning, failing to fully elicit LLMs' capabilities and handle complex instructions~\cite{qin2023toolllm}. 
Given a wide range of candidate API functions, each comprising various function names and parameters available at every planning step, identifying globally optimal solutions and generalizing across tasks remains highly challenging.



\section{Preliminaries}
\label{Preliminaries}
\begin{figure*}[t]
    \centering
    \includegraphics[width=0.95\linewidth]{fig/HealthGPT_Framework.png}
    \caption{The \ourmethod{} architecture integrates hierarchical visual perception and H-LoRA, employing a task-specific hard router to select visual features and H-LoRA plugins, ultimately generating outputs with an autoregressive manner.}
    \label{fig:architecture}
\end{figure*}
\noindent\textbf{Large Vision-Language Models.} 
The input to a LVLM typically consists of an image $x^{\text{img}}$ and a discrete text sequence $x^{\text{txt}}$. The visual encoder $\mathcal{E}^{\text{img}}$ converts the input image $x^{\text{img}}$ into a sequence of visual tokens $\mathcal{V} = [v_i]_{i=1}^{N_v}$, while the text sequence $x^{\text{txt}}$ is mapped into a sequence of text tokens $\mathcal{T} = [t_i]_{i=1}^{N_t}$ using an embedding function $\mathcal{E}^{\text{txt}}$. The LLM $\mathcal{M_\text{LLM}}(\cdot|\theta)$ models the joint probability of the token sequence $\mathcal{U} = \{\mathcal{V},\mathcal{T}\}$, which is expressed as:
\begin{equation}
    P_\theta(R | \mathcal{U}) = \prod_{i=1}^{N_r} P_\theta(r_i | \{\mathcal{U}, r_{<i}\}),
\end{equation}
where $R = [r_i]_{i=1}^{N_r}$ is the text response sequence. The LVLM iteratively generates the next token $r_i$ based on $r_{<i}$. The optimization objective is to minimize the cross-entropy loss of the response $\mathcal{R}$.
% \begin{equation}
%     \mathcal{L}_{\text{VLM}} = \mathbb{E}_{R|\mathcal{U}}\left[-\log P_\theta(R | \mathcal{U})\right]
% \end{equation}
It is worth noting that most LVLMs adopt a design paradigm based on ViT, alignment adapters, and pre-trained LLMs\cite{liu2023llava,liu2024improved}, enabling quick adaptation to downstream tasks.


\noindent\textbf{VQGAN.}
VQGAN~\cite{esser2021taming} employs latent space compression and indexing mechanisms to effectively learn a complete discrete representation of images. VQGAN first maps the input image $x^{\text{img}}$ to a latent representation $z = \mathcal{E}(x)$ through a encoder $\mathcal{E}$. Then, the latent representation is quantized using a codebook $\mathcal{Z} = \{z_k\}_{k=1}^K$, generating a discrete index sequence $\mathcal{I} = [i_m]_{m=1}^N$, where $i_m \in \mathcal{Z}$ represents the quantized code index:
\begin{equation}
    \mathcal{I} = \text{Quantize}(z|\mathcal{Z}) = \arg\min_{z_k \in \mathcal{Z}} \| z - z_k \|_2.
\end{equation}
In our approach, the discrete index sequence $\mathcal{I}$ serves as a supervisory signal for the generation task, enabling the model to predict the index sequence $\hat{\mathcal{I}}$ from input conditions such as text or other modality signals.  
Finally, the predicted index sequence $\hat{\mathcal{I}}$ is upsampled by the VQGAN decoder $G$, generating the high-quality image $\hat{x}^\text{img} = G(\hat{\mathcal{I}})$.



\noindent\textbf{Low Rank Adaptation.} 
LoRA\cite{hu2021lora} effectively captures the characteristics of downstream tasks by introducing low-rank adapters. The core idea is to decompose the bypass weight matrix $\Delta W\in\mathbb{R}^{d^{\text{in}} \times d^{\text{out}}}$ into two low-rank matrices $ \{A \in \mathbb{R}^{d^{\text{in}} \times r}, B \in \mathbb{R}^{r \times d^{\text{out}}} \}$, where $ r \ll \min\{d^{\text{in}}, d^{\text{out}}\} $, significantly reducing learnable parameters. The output with the LoRA adapter for the input $x$ is then given by:
\begin{equation}
    h = x W_0 + \alpha x \Delta W/r = x W_0 + \alpha xAB/r,
\end{equation}
where matrix $ A $ is initialized with a Gaussian distribution, while the matrix $ B $ is initialized as a zero matrix. The scaling factor $ \alpha/r $ controls the impact of $ \Delta W $ on the model.

\section{HealthGPT}
\label{Method}


\subsection{Unified Autoregressive Generation.}  
% As shown in Figure~\ref{fig:architecture}, 
\ourmethod{} (Figure~\ref{fig:architecture}) utilizes a discrete token representation that covers both text and visual outputs, unifying visual comprehension and generation as an autoregressive task. 
For comprehension, $\mathcal{M}_\text{llm}$ receives the input joint sequence $\mathcal{U}$ and outputs a series of text token $\mathcal{R} = [r_1, r_2, \dots, r_{N_r}]$, where $r_i \in \mathcal{V}_{\text{txt}}$, and $\mathcal{V}_{\text{txt}}$ represents the LLM's vocabulary:
\begin{equation}
    P_\theta(\mathcal{R} \mid \mathcal{U}) = \prod_{i=1}^{N_r} P_\theta(r_i \mid \mathcal{U}, r_{<i}).
\end{equation}
For generation, $\mathcal{M}_\text{llm}$ first receives a special start token $\langle \text{START\_IMG} \rangle$, then generates a series of tokens corresponding to the VQGAN indices $\mathcal{I} = [i_1, i_2, \dots, i_{N_i}]$, where $i_j \in \mathcal{V}_{\text{vq}}$, and $\mathcal{V}_{\text{vq}}$ represents the index range of VQGAN. Upon completion of generation, the LLM outputs an end token $\langle \text{END\_IMG} \rangle$:
\begin{equation}
    P_\theta(\mathcal{I} \mid \mathcal{U}) = \prod_{j=1}^{N_i} P_\theta(i_j \mid \mathcal{U}, i_{<j}).
\end{equation}
Finally, the generated index sequence $\mathcal{I}$ is fed into the decoder $G$, which reconstructs the target image $\hat{x}^{\text{img}} = G(\mathcal{I})$.

\subsection{Hierarchical Visual Perception}  
Given the differences in visual perception between comprehension and generation tasks—where the former focuses on abstract semantics and the latter emphasizes complete semantics—we employ ViT to compress the image into discrete visual tokens at multiple hierarchical levels.
Specifically, the image is converted into a series of features $\{f_1, f_2, \dots, f_L\}$ as it passes through $L$ ViT blocks.

To address the needs of various tasks, the hidden states are divided into two types: (i) \textit{Concrete-grained features} $\mathcal{F}^{\text{Con}} = \{f_1, f_2, \dots, f_k\}, k < L$, derived from the shallower layers of ViT, containing sufficient global features, suitable for generation tasks; 
(ii) \textit{Abstract-grained features} $\mathcal{F}^{\text{Abs}} = \{f_{k+1}, f_{k+2}, \dots, f_L\}$, derived from the deeper layers of ViT, which contain abstract semantic information closer to the text space, suitable for comprehension tasks.

The task type $T$ (comprehension or generation) determines which set of features is selected as the input for the downstream large language model:
\begin{equation}
    \mathcal{F}^{\text{img}}_T =
    \begin{cases}
        \mathcal{F}^{\text{Con}}, & \text{if } T = \text{generation task} \\
        \mathcal{F}^{\text{Abs}}, & \text{if } T = \text{comprehension task}
    \end{cases}
\end{equation}
We integrate the image features $\mathcal{F}^{\text{img}}_T$ and text features $\mathcal{T}$ into a joint sequence through simple concatenation, which is then fed into the LLM $\mathcal{M}_{\text{llm}}$ for autoregressive generation.
% :
% \begin{equation}
%     \mathcal{R} = \mathcal{M}_{\text{llm}}(\mathcal{U}|\theta), \quad \mathcal{U} = [\mathcal{F}^{\text{img}}_T; \mathcal{T}]
% \end{equation}
\subsection{Heterogeneous Knowledge Adaptation}
We devise H-LoRA, which stores heterogeneous knowledge from comprehension and generation tasks in separate modules and dynamically routes to extract task-relevant knowledge from these modules. 
At the task level, for each task type $ T $, we dynamically assign a dedicated H-LoRA submodule $ \theta^T $, which is expressed as:
\begin{equation}
    \mathcal{R} = \mathcal{M}_\text{LLM}(\mathcal{U}|\theta, \theta^T), \quad \theta^T = \{A^T, B^T, \mathcal{R}^T_\text{outer}\}.
\end{equation}
At the feature level for a single task, H-LoRA integrates the idea of Mixture of Experts (MoE)~\cite{masoudnia2014mixture} and designs an efficient matrix merging and routing weight allocation mechanism, thus avoiding the significant computational delay introduced by matrix splitting in existing MoELoRA~\cite{luo2024moelora}. Specifically, we first merge the low-rank matrices (rank = r) of $ k $ LoRA experts into a unified matrix:
\begin{equation}
    \mathbf{A}^{\text{merged}}, \mathbf{B}^{\text{merged}} = \text{Concat}(\{A_i\}_1^k), \text{Concat}(\{B_i\}_1^k),
\end{equation}
where $ \mathbf{A}^{\text{merged}} \in \mathbb{R}^{d^\text{in} \times rk} $ and $ \mathbf{B}^{\text{merged}} \in \mathbb{R}^{rk \times d^\text{out}} $. The $k$-dimension routing layer generates expert weights $ \mathcal{W} \in \mathbb{R}^{\text{token\_num} \times k} $ based on the input hidden state $ x $, and these are expanded to $ \mathbb{R}^{\text{token\_num} \times rk} $ as follows:
\begin{equation}
    \mathcal{W}^\text{expanded} = \alpha k \mathcal{W} / r \otimes \mathbf{1}_r,
\end{equation}
where $ \otimes $ denotes the replication operation.
The overall output of H-LoRA is computed as:
\begin{equation}
    \mathcal{O}^\text{H-LoRA} = (x \mathbf{A}^{\text{merged}} \odot \mathcal{W}^\text{expanded}) \mathbf{B}^{\text{merged}},
\end{equation}
where $ \odot $ represents element-wise multiplication. Finally, the output of H-LoRA is added to the frozen pre-trained weights to produce the final output:
\begin{equation}
    \mathcal{O} = x W_0 + \mathcal{O}^\text{H-LoRA}.
\end{equation}
% In summary, H-LoRA is a task-based dynamic PEFT method that achieves high efficiency in single-task fine-tuning.

\subsection{Training Pipeline}

\begin{figure}[t]
    \centering
    \hspace{-4mm}
    \includegraphics[width=0.94\linewidth]{fig/data.pdf}
    \caption{Data statistics of \texttt{VL-Health}. }
    \label{fig:data}
\end{figure}
\noindent \textbf{1st Stage: Multi-modal Alignment.} 
In the first stage, we design separate visual adapters and H-LoRA submodules for medical unified tasks. For the medical comprehension task, we train abstract-grained visual adapters using high-quality image-text pairs to align visual embeddings with textual embeddings, thereby enabling the model to accurately describe medical visual content. During this process, the pre-trained LLM and its corresponding H-LoRA submodules remain frozen. In contrast, the medical generation task requires training concrete-grained adapters and H-LoRA submodules while keeping the LLM frozen. Meanwhile, we extend the textual vocabulary to include multimodal tokens, enabling the support of additional VQGAN vector quantization indices. The model trains on image-VQ pairs, endowing the pre-trained LLM with the capability for image reconstruction. This design ensures pixel-level consistency of pre- and post-LVLM. The processes establish the initial alignment between the LLM’s outputs and the visual inputs.

\noindent \textbf{2nd Stage: Heterogeneous H-LoRA Plugin Adaptation.}  
The submodules of H-LoRA share the word embedding layer and output head but may encounter issues such as bias and scale inconsistencies during training across different tasks. To ensure that the multiple H-LoRA plugins seamlessly interface with the LLMs and form a unified base, we fine-tune the word embedding layer and output head using a small amount of mixed data to maintain consistency in the model weights. Specifically, during this stage, all H-LoRA submodules for different tasks are kept frozen, with only the word embedding layer and output head being optimized. Through this stage, the model accumulates foundational knowledge for unified tasks by adapting H-LoRA plugins.

\begin{table*}[!t]
\centering
\caption{Comparison of \ourmethod{} with other LVLMs and unified multi-modal models on medical visual comprehension tasks. \textbf{Bold} and \underline{underlined} text indicates the best performance and second-best performance, respectively.}
\resizebox{\textwidth}{!}{
\begin{tabular}{c|lcc|cccccccc|c}
\toprule
\rowcolor[HTML]{E9F3FE} &  &  &  & \multicolumn{2}{c}{\textbf{VQA-RAD \textuparrow}} & \multicolumn{2}{c}{\textbf{SLAKE \textuparrow}} & \multicolumn{2}{c}{\textbf{PathVQA \textuparrow}} &  &  &  \\ 
\cline{5-10}
\rowcolor[HTML]{E9F3FE}\multirow{-2}{*}{\textbf{Type}} & \multirow{-2}{*}{\textbf{Model}} & \multirow{-2}{*}{\textbf{\# Params}} & \multirow{-2}{*}{\makecell{\textbf{Medical} \\ \textbf{LVLM}}} & \textbf{close} & \textbf{all} & \textbf{close} & \textbf{all} & \textbf{close} & \textbf{all} & \multirow{-2}{*}{\makecell{\textbf{MMMU} \\ \textbf{-Med}}\textuparrow} & \multirow{-2}{*}{\textbf{OMVQA}\textuparrow} & \multirow{-2}{*}{\textbf{Avg. \textuparrow}} \\ 
\midrule \midrule
\multirow{9}{*}{\textbf{Comp. Only}} 
& Med-Flamingo & 8.3B & \Large \ding{51} & 58.6 & 43.0 & 47.0 & 25.5 & 61.9 & 31.3 & 28.7 & 34.9 & 41.4 \\
& LLaVA-Med & 7B & \Large \ding{51} & 60.2 & 48.1 & 58.4 & 44.8 & 62.3 & 35.7 & 30.0 & 41.3 & 47.6 \\
& HuatuoGPT-Vision & 7B & \Large \ding{51} & 66.9 & 53.0 & 59.8 & 49.1 & 52.9 & 32.0 & 42.0 & 50.0 & 50.7 \\
& BLIP-2 & 6.7B & \Large \ding{55} & 43.4 & 36.8 & 41.6 & 35.3 & 48.5 & 28.8 & 27.3 & 26.9 & 36.1 \\
& LLaVA-v1.5 & 7B & \Large \ding{55} & 51.8 & 42.8 & 37.1 & 37.7 & 53.5 & 31.4 & 32.7 & 44.7 & 41.5 \\
& InstructBLIP & 7B & \Large \ding{55} & 61.0 & 44.8 & 66.8 & 43.3 & 56.0 & 32.3 & 25.3 & 29.0 & 44.8 \\
& Yi-VL & 6B & \Large \ding{55} & 52.6 & 42.1 & 52.4 & 38.4 & 54.9 & 30.9 & 38.0 & 50.2 & 44.9 \\
& InternVL2 & 8B & \Large \ding{55} & 64.9 & 49.0 & 66.6 & 50.1 & 60.0 & 31.9 & \underline{43.3} & 54.5 & 52.5\\
& Llama-3.2 & 11B & \Large \ding{55} & 68.9 & 45.5 & 72.4 & 52.1 & 62.8 & 33.6 & 39.3 & 63.2 & 54.7 \\
\midrule
\multirow{5}{*}{\textbf{Comp. \& Gen.}} 
& Show-o & 1.3B & \Large \ding{55} & 50.6 & 33.9 & 31.5 & 17.9 & 52.9 & 28.2 & 22.7 & 45.7 & 42.6 \\
& Unified-IO 2 & 7B & \Large \ding{55} & 46.2 & 32.6 & 35.9 & 21.9 & 52.5 & 27.0 & 25.3 & 33.0 & 33.8 \\
& Janus & 1.3B & \Large \ding{55} & 70.9 & 52.8 & 34.7 & 26.9 & 51.9 & 27.9 & 30.0 & 26.8 & 33.5 \\
& \cellcolor[HTML]{DAE0FB}HealthGPT-M3 & \cellcolor[HTML]{DAE0FB}3.8B & \cellcolor[HTML]{DAE0FB}\Large \ding{51} & \cellcolor[HTML]{DAE0FB}\underline{73.7} & \cellcolor[HTML]{DAE0FB}\underline{55.9} & \cellcolor[HTML]{DAE0FB}\underline{74.6} & \cellcolor[HTML]{DAE0FB}\underline{56.4} & \cellcolor[HTML]{DAE0FB}\underline{78.7} & \cellcolor[HTML]{DAE0FB}\underline{39.7} & \cellcolor[HTML]{DAE0FB}\underline{43.3} & \cellcolor[HTML]{DAE0FB}\underline{68.5} & \cellcolor[HTML]{DAE0FB}\underline{61.3} \\
& \cellcolor[HTML]{DAE0FB}HealthGPT-L14 & \cellcolor[HTML]{DAE0FB}14B & \cellcolor[HTML]{DAE0FB}\Large \ding{51} & \cellcolor[HTML]{DAE0FB}\textbf{77.7} & \cellcolor[HTML]{DAE0FB}\textbf{58.3} & \cellcolor[HTML]{DAE0FB}\textbf{76.4} & \cellcolor[HTML]{DAE0FB}\textbf{64.5} & \cellcolor[HTML]{DAE0FB}\textbf{85.9} & \cellcolor[HTML]{DAE0FB}\textbf{44.4} & \cellcolor[HTML]{DAE0FB}\textbf{49.2} & \cellcolor[HTML]{DAE0FB}\textbf{74.4} & \cellcolor[HTML]{DAE0FB}\textbf{66.4} \\
\bottomrule
\end{tabular}
}
\label{tab:results}
\end{table*}
\begin{table*}[ht]
    \centering
    \caption{The experimental results for the four modality conversion tasks.}
    \resizebox{\textwidth}{!}{
    \begin{tabular}{l|ccc|ccc|ccc|ccc}
        \toprule
        \rowcolor[HTML]{E9F3FE} & \multicolumn{3}{c}{\textbf{CT to MRI (Brain)}} & \multicolumn{3}{c}{\textbf{CT to MRI (Pelvis)}} & \multicolumn{3}{c}{\textbf{MRI to CT (Brain)}} & \multicolumn{3}{c}{\textbf{MRI to CT (Pelvis)}} \\
        \cline{2-13}
        \rowcolor[HTML]{E9F3FE}\multirow{-2}{*}{\textbf{Model}}& \textbf{SSIM $\uparrow$} & \textbf{PSNR $\uparrow$} & \textbf{MSE $\downarrow$} & \textbf{SSIM $\uparrow$} & \textbf{PSNR $\uparrow$} & \textbf{MSE $\downarrow$} & \textbf{SSIM $\uparrow$} & \textbf{PSNR $\uparrow$} & \textbf{MSE $\downarrow$} & \textbf{SSIM $\uparrow$} & \textbf{PSNR $\uparrow$} & \textbf{MSE $\downarrow$} \\
        \midrule \midrule
        pix2pix & 71.09 & 32.65 & 36.85 & 59.17 & 31.02 & 51.91 & 78.79 & 33.85 & 28.33 & 72.31 & 32.98 & 36.19 \\
        CycleGAN & 54.76 & 32.23 & 40.56 & 54.54 & 30.77 & 55.00 & 63.75 & 31.02 & 52.78 & 50.54 & 29.89 & 67.78 \\
        BBDM & {71.69} & {32.91} & {34.44} & 57.37 & 31.37 & 48.06 & \textbf{86.40} & 34.12 & 26.61 & {79.26} & 33.15 & 33.60 \\
        Vmanba & 69.54 & 32.67 & 36.42 & {63.01} & {31.47} & {46.99} & 79.63 & 34.12 & 26.49 & 77.45 & 33.53 & 31.85 \\
        DiffMa & 71.47 & 32.74 & 35.77 & 62.56 & 31.43 & 47.38 & 79.00 & {34.13} & {26.45} & 78.53 & {33.68} & {30.51} \\
        \rowcolor[HTML]{DAE0FB}HealthGPT-M3 & \underline{79.38} & \underline{33.03} & \underline{33.48} & \underline{71.81} & \underline{31.83} & \underline{43.45} & {85.06} & \textbf{34.40} & \textbf{25.49} & \underline{84.23} & \textbf{34.29} & \textbf{27.99} \\
        \rowcolor[HTML]{DAE0FB}HealthGPT-L14 & \textbf{79.73} & \textbf{33.10} & \textbf{32.96} & \textbf{71.92} & \textbf{31.87} & \textbf{43.09} & \underline{85.31} & \underline{34.29} & \underline{26.20} & \textbf{84.96} & \underline{34.14} & \underline{28.13} \\
        \bottomrule
    \end{tabular}
    }
    \label{tab:conversion}
\end{table*}

\noindent \textbf{3rd Stage: Visual Instruction Fine-Tuning.}  
In the third stage, we introduce additional task-specific data to further optimize the model and enhance its adaptability to downstream tasks such as medical visual comprehension (e.g., medical QA, medical dialogues, and report generation) or generation tasks (e.g., super-resolution, denoising, and modality conversion). Notably, by this stage, the word embedding layer and output head have been fine-tuned, only the H-LoRA modules and adapter modules need to be trained. This strategy significantly improves the model's adaptability and flexibility across different tasks.


\section{Experiment}
\label{s:experiment}

\subsection{Data Description}
We evaluate our method on FI~\cite{you2016building}, Twitter\_LDL~\cite{yang2017learning} and Artphoto~\cite{machajdik2010affective}.
FI is a public dataset built from Flickr and Instagram, with 23,308 images and eight emotion categories, namely \textit{amusement}, \textit{anger}, \textit{awe},  \textit{contentment}, \textit{disgust}, \textit{excitement},  \textit{fear}, and \textit{sadness}. 
% Since images in FI are all copyrighted by law, some images are corrupted now, so we remove these samples and retain 21,828 images.
% T4SA contains images from Twitter, which are classified into three categories: \textit{positive}, \textit{neutral}, and \textit{negative}. In this paper, we adopt the base version of B-T4SA, which contains 470,586 images and provides text descriptions of the corresponding tweets.
Twitter\_LDL contains 10,045 images from Twitter, with the same eight categories as the FI dataset.
% 。
For these two datasets, they are randomly split into 80\%
training and 20\% testing set.
Artphoto contains 806 artistic photos from the DeviantArt website, which we use to further evaluate the zero-shot capability of our model.
% on the small-scale dataset.
% We construct and publicly release the first image sentiment analysis dataset containing metadata.
% 。

% Based on these datasets, we are the first to construct and publicly release metadata-enhanced image sentiment analysis datasets. These datasets include scenes, tags, descriptions, and corresponding confidence scores, and are available at this link for future research purposes.


% 
\begin{table}[t]
\centering
% \begin{center}
\caption{Overall performance of different models on FI and Twitter\_LDL datasets.}
\label{tab:cap1}
% \resizebox{\linewidth}{!}
{
\begin{tabular}{l|c|c|c|c}
\hline
\multirow{2}{*}{\textbf{Model}} & \multicolumn{2}{c|}{\textbf{FI}}  & \multicolumn{2}{c}{\textbf{Twitter\_LDL}} \\ \cline{2-5} 
  & \textbf{Accuracy} & \textbf{F1} & \textbf{Accuracy} & \textbf{F1}  \\ \hline
% (\rownumber)~AlexNet~\cite{krizhevsky2017imagenet}  & 58.13\% & 56.35\%  & 56.24\%& 55.02\%  \\ 
% (\rownumber)~VGG16~\cite{simonyan2014very}  & 63.75\%& 63.08\%  & 59.34\%& 59.02\%  \\ 
(\rownumber)~ResNet101~\cite{he2016deep} & 66.16\%& 65.56\%  & 62.02\% & 61.34\%  \\ 
(\rownumber)~CDA~\cite{han2023boosting} & 66.71\%& 65.37\%  & 64.14\% & 62.85\%  \\ 
(\rownumber)~CECCN~\cite{ruan2024color} & 67.96\%& 66.74\%  & 64.59\%& 64.72\% \\ 
(\rownumber)~EmoVIT~\cite{xie2024emovit} & 68.09\%& 67.45\%  & 63.12\% & 61.97\%  \\ 
(\rownumber)~ComLDL~\cite{zhang2022compound} & 68.83\%& 67.28\%  & 65.29\% & 63.12\%  \\ 
(\rownumber)~WSDEN~\cite{li2023weakly} & 69.78\%& 69.61\%  & 67.04\% & 65.49\% \\ 
(\rownumber)~ECWA~\cite{deng2021emotion} & 70.87\%& 69.08\%  & 67.81\% & 66.87\%  \\ 
(\rownumber)~EECon~\cite{yang2023exploiting} & 71.13\%& 68.34\%  & 64.27\%& 63.16\%  \\ 
(\rownumber)~MAM~\cite{zhang2024affective} & 71.44\%  & 70.83\% & 67.18\%  & 65.01\%\\ 
(\rownumber)~TGCA-PVT~\cite{chen2024tgca}   & 73.05\%  & 71.46\% & 69.87\%  & 68.32\% \\ 
(\rownumber)~OEAN~\cite{zhang2024object}   & 73.40\%  & 72.63\% & 70.52\%  & 69.47\% \\ \hline
(\rownumber)~\shortname  & \textbf{79.48\%} & \textbf{79.22\%} & \textbf{74.12\%} & \textbf{73.09\%} \\ \hline
\end{tabular}
}
\vspace{-6mm}
% \end{center}
\end{table}
% 

\subsection{Experiment Setting}
% \subsubsection{Model Setting.}
% 
\textbf{Model Setting:}
For feature representation, we set $k=10$ to select object tags, and adopt clip-vit-base-patch32 as the pre-trained model for unified feature representation.
Moreover, we empirically set $(d_e, d_h, d_k, d_s) = (512, 128, 16, 64)$, and set the classification class $L$ to 8.

% 

\textbf{Training Setting:}
To initialize the model, we set all weights such as $\boldsymbol{W}$ following the truncated normal distribution, and use AdamW optimizer with the learning rate of $1 \times 10^{-4}$.
% warmup scheduler of cosine, warmup steps of 2000.
Furthermore, we set the batch size to 32 and the epoch of the training process to 200.
During the implementation, we utilize \textit{PyTorch} to build our entire model.
% , and our project codes are publicly available at https://github.com/zzmyrep/MESN.
% Our project codes as well as data are all publicly available on GitHub\footnote{https://github.com/zzmyrep/KBCEN}.
% Code is available at \href{https://github.com/zzmyrep/KBCEN}{https://github.com/zzmyrep/KBCEN}.

\textbf{Evaluation Metrics:}
Following~\cite{zhang2024affective, chen2024tgca, zhang2024object}, we adopt \textit{accuracy} and \textit{F1} as our evaluation metrics to measure the performance of different methods for image sentiment analysis. 



\subsection{Experiment Result}
% We compare our model against the following baselines: AlexNet~\cite{krizhevsky2017imagenet}, VGG16~\cite{simonyan2014very}, ResNet101~\cite{he2016deep}, CECCN~\cite{ruan2024color}, EmoVIT~\cite{xie2024emovit}, WSCNet~\cite{yang2018weakly}, ECWA~\cite{deng2021emotion}, EECon~\cite{yang2023exploiting}, MAM~\cite{zhang2024affective} and TGCA-PVT~\cite{chen2024tgca}, and the overall results are summarized in Table~\ref{tab:cap1}.
We compare our model against several baselines, and the overall results are summarized in Table~\ref{tab:cap1}.
We observe that our model achieves the best performance in both accuracy and F1 metrics, significantly outperforming the previous models. 
This superior performance is mainly attributed to our effective utilization of metadata to enhance image sentiment analysis, as well as the exceptional capability of the unified sentiment transformer framework we developed. These results strongly demonstrate that our proposed method can bring encouraging performance for image sentiment analysis.

\setcounter{magicrownumbers}{0} 
\begin{table}[t]
\begin{center}
\caption{Ablation study of~\shortname~on FI dataset.} 
% \vspace{1mm}
\label{tab:cap2}
\resizebox{.9\linewidth}{!}
{
\begin{tabular}{lcc}
  \hline
  \textbf{Model} & \textbf{Accuracy} & \textbf{F1} \\
  \hline
  (\rownumber)~Ours (w/o vision) & 65.72\% & 64.54\% \\
  (\rownumber)~Ours (w/o text description) & 74.05\% & 72.58\% \\
  (\rownumber)~Ours (w/o object tag) & 77.45\% & 76.84\% \\
  (\rownumber)~Ours (w/o scene tag) & 78.47\% & 78.21\% \\
  \hline
  (\rownumber)~Ours (w/o unified embedding) & 76.41\% & 76.23\% \\
  (\rownumber)~Ours (w/o adaptive learning) & 76.83\% & 76.56\% \\
  (\rownumber)~Ours (w/o cross-modal fusion) & 76.85\% & 76.49\% \\
  \hline
  (\rownumber)~Ours  & \textbf{79.48\%} & \textbf{79.22\%} \\
  \hline
\end{tabular}
}
\end{center}
\vspace{-5mm}
\end{table}


\begin{figure}[t]
\centering
% \vspace{-2mm}
\includegraphics[width=0.42\textwidth]{fig/2dvisual-linux4-paper2.pdf}
\caption{Visualization of feature distribution on eight categories before (left) and after (right) model processing.}
% 
\label{fig:visualization}
\vspace{-5mm}
\end{figure}

\subsection{Ablation Performance}
In this subsection, we conduct an ablation study to examine which component is really important for performance improvement. The results are reported in Table~\ref{tab:cap2}.

For information utilization, we observe a significant decline in model performance when visual features are removed. Additionally, the performance of \shortname~decreases when different metadata are removed separately, which means that text description, object tag, and scene tag are all critical for image sentiment analysis.
Recalling the model architecture, we separately remove transformer layers of the unified representation module, the adaptive learning module, and the cross-modal fusion module, replacing them with MLPs of the same parameter scale.
In this way, we can observe varying degrees of decline in model performance, indicating that these modules are indispensable for our model to achieve better performance.

\subsection{Visualization}
% 


% % 开始使用minipage进行左右排列
% \begin{minipage}[t]{0.45\textwidth}  % 子图1宽度为45%
%     \centering
%     \includegraphics[width=\textwidth]{2dvisual.pdf}  % 插入图片
%     \captionof{figure}{Visualization of feature distribution.}  % 使用captionof添加图片标题
%     \label{fig:visualization}
% \end{minipage}


% \begin{figure}[t]
% \centering
% \vspace{-2mm}
% \includegraphics[width=0.45\textwidth]{fig/2dvisual.pdf}
% \caption{Visualization of feature distribution.}
% \label{fig:visualization}
% % \vspace{-4mm}
% \end{figure}

% \begin{figure}[t]
% \centering
% \vspace{-2mm}
% \includegraphics[width=0.45\textwidth]{fig/2dvisual-linux3-paper.pdf}
% \caption{Visualization of feature distribution.}
% \label{fig:visualization}
% % \vspace{-4mm}
% \end{figure}



\begin{figure}[tbp]   
\vspace{-4mm}
  \centering            
  \subfloat[Depth of adaptive learning layers]   
  {
    \label{fig:subfig1}\includegraphics[width=0.22\textwidth]{fig/fig_sensitivity-a5}
  }
  \subfloat[Depth of fusion layers]
  {
    % \label{fig:subfig2}\includegraphics[width=0.22\textwidth]{fig/fig_sensitivity-b2}
    \label{fig:subfig2}\includegraphics[width=0.22\textwidth]{fig/fig_sensitivity-b2-num.pdf}
  }
  \caption{Sensitivity study of \shortname~on different depth. }   
  \label{fig:fig_sensitivity}  
\vspace{-2mm}
\end{figure}

% \begin{figure}[htbp]
% \centerline{\includegraphics{2dvisual.pdf}}
% \caption{Visualization of feature distribution.}
% \label{fig:visualization}
% \end{figure}

% In Fig.~\ref{fig:visualization}, we use t-SNE~\cite{van2008visualizing} to reduce the dimension of data features for visualization, Figure in left represents the metadata features before model processing, the features are obtained by embedding through the CLIP model, and figure in right shows the features of the data after model processing, it can be observed that after the model processing, the data with different label categories fall in different regions in the space, therefore, we can conclude that the Therefore, we can conclude that the model can effectively utilize the information contained in the metadata and use it to guide the model for classification.

In Fig.~\ref{fig:visualization}, we use t-SNE~\cite{van2008visualizing} to reduce the dimension of data features for visualization.
The left figure shows metadata features before being processed by our model (\textit{i.e.}, embedded by CLIP), while the right shows the distribution of features after being processed by our model.
We can observe that after the model processing, data with the same label are closer to each other, while others are farther away.
Therefore, it shows that the model can effectively utilize the information contained in the metadata and use it to guide the classification process.

\subsection{Sensitivity Analysis}
% 
In this subsection, we conduct a sensitivity analysis to figure out the effect of different depth settings of adaptive learning layers and fusion layers. 
% In this subsection, we conduct a sensitivity analysis to figure out the effect of different depth settings on the model. 
% Fig.~\ref{fig:fig_sensitivity} presents the effect of different depth settings of adaptive learning layers and fusion layers. 
Taking Fig.~\ref{fig:fig_sensitivity} (a) as an example, the model performance improves with increasing depth, reaching the best performance at a depth of 4.
% Taking Fig.~\ref{fig:fig_sensitivity} (a) as an example, the performance of \shortname~improves with the increase of depth at first, reaching the best performance at a depth of 4.
When the depth continues to increase, the accuracy decreases to varying degrees.
Similar results can be observed in Fig.~\ref{fig:fig_sensitivity} (b).
Therefore, we set their depths to 4 and 6 respectively to achieve the best results.

% Through our experiments, we can observe that the effect of modifying these hyperparameters on the results of the experiments is very weak, and the surface model is not sensitive to the hyperparameters.


\subsection{Zero-shot Capability}
% 

% (1)~GCH~\cite{2010Analyzing} & 21.78\% & (5)~RA-DLNet~\cite{2020A} & 34.01\% \\ \hline
% (2)~WSCNet~\cite{2019WSCNet}  & 30.25\% & (6)~CECCN~\cite{ruan2024color} & 43.83\% \\ \hline
% (3)~PCNN~\cite{2015Robust} & 31.68\%  & (7)~EmoVIT~\cite{xie2024emovit} & 44.90\% \\ \hline
% (4)~AR~\cite{2018Visual} & 32.67\% & (8)~Ours (Zero-shot) & 47.83\% \\ \hline


\begin{table}[t]
\centering
\caption{Zero-shot capability of \shortname.}
\label{tab:cap3}
\resizebox{1\linewidth}{!}
{
\begin{tabular}{lc|lc}
\hline
\textbf{Model} & \textbf{Accuracy} & \textbf{Model} & \textbf{Accuracy} \\ \hline
(1)~WSCNet~\cite{2019WSCNet}  & 30.25\% & (5)~MAM~\cite{zhang2024affective} & 39.56\%  \\ \hline
(2)~AR~\cite{2018Visual} & 32.67\% & (6)~CECCN~\cite{ruan2024color} & 43.83\% \\ \hline
(3)~RA-DLNet~\cite{2020A} & 34.01\%  & (7)~EmoVIT~\cite{xie2024emovit} & 44.90\% \\ \hline
(4)~CDA~\cite{han2023boosting} & 38.64\% & (8)~Ours (Zero-shot) & 47.83\% \\ \hline
\end{tabular}
}
\vspace{-5mm}
\end{table}

% We use the model trained on the FI dataset to test on the artphoto dataset to verify the model's generalization ability as well as robustness to other distributed datasets.
% We can observe that the MESN model shows strong competitiveness in terms of accuracy when compared to other trained models, which suggests that the model has a good generalization ability in the OOD task.

To validate the model's generalization ability and robustness to other distributed datasets, we directly test the model trained on the FI dataset, without training on Artphoto. 
% As observed in Table 3, compared to other models trained on Artphoto, we achieve highly competitive zero-shot performance, indicating that the model has good generalization ability in out-of-distribution tasks.
From Table~\ref{tab:cap3}, we can observe that compared with other models trained on Artphoto, we achieve competitive zero-shot performance, which shows that the model has good generalization ability in out-of-distribution tasks.


%%%%%%%%%%%%
%  E2E     %
%%%%%%%%%%%%


\section{Conclusion}
In this paper, we introduced Wi-Chat, the first LLM-powered Wi-Fi-based human activity recognition system that integrates the reasoning capabilities of large language models with the sensing potential of wireless signals. Our experimental results on a self-collected Wi-Fi CSI dataset demonstrate the promising potential of LLMs in enabling zero-shot Wi-Fi sensing. These findings suggest a new paradigm for human activity recognition that does not rely on extensive labeled data. We hope future research will build upon this direction, further exploring the applications of LLMs in signal processing domains such as IoT, mobile sensing, and radar-based systems.

\section*{Limitations}
While our work represents the first attempt to leverage LLMs for processing Wi-Fi signals, it is a preliminary study focused on a relatively simple task: Wi-Fi-based human activity recognition. This choice allows us to explore the feasibility of LLMs in wireless sensing but also comes with certain limitations.

Our approach primarily evaluates zero-shot performance, which, while promising, may still lag behind traditional supervised learning methods in highly complex or fine-grained recognition tasks. Besides, our study is limited to a controlled environment with a self-collected dataset, and the generalizability of LLMs to diverse real-world scenarios with varying Wi-Fi conditions, environmental interference, and device heterogeneity remains an open question.

Additionally, we have yet to explore the full potential of LLMs in more advanced Wi-Fi sensing applications, such as fine-grained gesture recognition, occupancy detection, and passive health monitoring. Future work should investigate the scalability of LLM-based approaches, their robustness to domain shifts, and their integration with multimodal sensing techniques in broader IoT applications.


% Bibliography entries for the entire Anthology, followed by custom entries
%\bibliography{anthology,custom}
% Custom bibliography entries only
\bibliography{main}
\newpage
\appendix

\section{Experiment prompts}
\label{sec:prompt}
The prompts used in the LLM experiments are shown in the following Table~\ref{tab:prompts}.

\definecolor{titlecolor}{rgb}{0.9, 0.5, 0.1}
\definecolor{anscolor}{rgb}{0.2, 0.5, 0.8}
\definecolor{labelcolor}{HTML}{48a07e}
\begin{table*}[h]
	\centering
	
 % \vspace{-0.2cm}
	
	\begin{center}
		\begin{tikzpicture}[
				chatbox_inner/.style={rectangle, rounded corners, opacity=0, text opacity=1, font=\sffamily\scriptsize, text width=5in, text height=9pt, inner xsep=6pt, inner ysep=6pt},
				chatbox_prompt_inner/.style={chatbox_inner, align=flush left, xshift=0pt, text height=11pt},
				chatbox_user_inner/.style={chatbox_inner, align=flush left, xshift=0pt},
				chatbox_gpt_inner/.style={chatbox_inner, align=flush left, xshift=0pt},
				chatbox/.style={chatbox_inner, draw=black!25, fill=gray!7, opacity=1, text opacity=0},
				chatbox_prompt/.style={chatbox, align=flush left, fill=gray!1.5, draw=black!30, text height=10pt},
				chatbox_user/.style={chatbox, align=flush left},
				chatbox_gpt/.style={chatbox, align=flush left},
				chatbox2/.style={chatbox_gpt, fill=green!25},
				chatbox3/.style={chatbox_gpt, fill=red!20, draw=black!20},
				chatbox4/.style={chatbox_gpt, fill=yellow!30},
				labelbox/.style={rectangle, rounded corners, draw=black!50, font=\sffamily\scriptsize\bfseries, fill=gray!5, inner sep=3pt},
			]
											
			\node[chatbox_user] (q1) {
				\textbf{System prompt}
				\newline
				\newline
				You are a helpful and precise assistant for segmenting and labeling sentences. We would like to request your help on curating a dataset for entity-level hallucination detection.
				\newline \newline
                We will give you a machine generated biography and a list of checked facts about the biography. Each fact consists of a sentence and a label (True/False). Please do the following process. First, breaking down the biography into words. Second, by referring to the provided list of facts, merging some broken down words in the previous step to form meaningful entities. For example, ``strategic thinking'' should be one entity instead of two. Third, according to the labels in the list of facts, labeling each entity as True or False. Specifically, for facts that share a similar sentence structure (\eg, \textit{``He was born on Mach 9, 1941.''} (\texttt{True}) and \textit{``He was born in Ramos Mejia.''} (\texttt{False})), please first assign labels to entities that differ across atomic facts. For example, first labeling ``Mach 9, 1941'' (\texttt{True}) and ``Ramos Mejia'' (\texttt{False}) in the above case. For those entities that are the same across atomic facts (\eg, ``was born'') or are neutral (\eg, ``he,'' ``in,'' and ``on''), please label them as \texttt{True}. For the cases that there is no atomic fact that shares the same sentence structure, please identify the most informative entities in the sentence and label them with the same label as the atomic fact while treating the rest of the entities as \texttt{True}. In the end, output the entities and labels in the following format:
                \begin{itemize}[nosep]
                    \item Entity 1 (Label 1)
                    \item Entity 2 (Label 2)
                    \item ...
                    \item Entity N (Label N)
                \end{itemize}
                % \newline \newline
                Here are two examples:
                \newline\newline
                \textbf{[Example 1]}
                \newline
                [The start of the biography]
                \newline
                \textcolor{titlecolor}{Marianne McAndrew is an American actress and singer, born on November 21, 1942, in Cleveland, Ohio. She began her acting career in the late 1960s, appearing in various television shows and films.}
                \newline
                [The end of the biography]
                \newline \newline
                [The start of the list of checked facts]
                \newline
                \textcolor{anscolor}{[Marianne McAndrew is an American. (False); Marianne McAndrew is an actress. (True); Marianne McAndrew is a singer. (False); Marianne McAndrew was born on November 21, 1942. (False); Marianne McAndrew was born in Cleveland, Ohio. (False); She began her acting career in the late 1960s. (True); She has appeared in various television shows. (True); She has appeared in various films. (True)]}
                \newline
                [The end of the list of checked facts]
                \newline \newline
                [The start of the ideal output]
                \newline
                \textcolor{labelcolor}{[Marianne McAndrew (True); is (True); an (True); American (False); actress (True); and (True); singer (False); , (True); born (True); on (True); November 21, 1942 (False); , (True); in (True); Cleveland, Ohio (False); . (True); She (True); began (True); her (True); acting career (True); in (True); the late 1960s (True); , (True); appearing (True); in (True); various (True); television shows (True); and (True); films (True); . (True)]}
                \newline
                [The end of the ideal output]
				\newline \newline
                \textbf{[Example 2]}
                \newline
                [The start of the biography]
                \newline
                \textcolor{titlecolor}{Doug Sheehan is an American actor who was born on April 27, 1949, in Santa Monica, California. He is best known for his roles in soap operas, including his portrayal of Joe Kelly on ``General Hospital'' and Ben Gibson on ``Knots Landing.''}
                \newline
                [The end of the biography]
                \newline \newline
                [The start of the list of checked facts]
                \newline
                \textcolor{anscolor}{[Doug Sheehan is an American. (True); Doug Sheehan is an actor. (True); Doug Sheehan was born on April 27, 1949. (True); Doug Sheehan was born in Santa Monica, California. (False); He is best known for his roles in soap operas. (True); He portrayed Joe Kelly. (True); Joe Kelly was in General Hospital. (True); General Hospital is a soap opera. (True); He portrayed Ben Gibson. (True); Ben Gibson was in Knots Landing. (True); Knots Landing is a soap opera. (True)]}
                \newline
                [The end of the list of checked facts]
                \newline \newline
                [The start of the ideal output]
                \newline
                \textcolor{labelcolor}{[Doug Sheehan (True); is (True); an (True); American (True); actor (True); who (True); was born (True); on (True); April 27, 1949 (True); in (True); Santa Monica, California (False); . (True); He (True); is (True); best known (True); for (True); his roles in soap operas (True); , (True); including (True); in (True); his portrayal (True); of (True); Joe Kelly (True); on (True); ``General Hospital'' (True); and (True); Ben Gibson (True); on (True); ``Knots Landing.'' (True)]}
                \newline
                [The end of the ideal output]
				\newline \newline
				\textbf{User prompt}
				\newline
				\newline
				[The start of the biography]
				\newline
				\textcolor{magenta}{\texttt{\{BIOGRAPHY\}}}
				\newline
				[The ebd of the biography]
				\newline \newline
				[The start of the list of checked facts]
				\newline
				\textcolor{magenta}{\texttt{\{LIST OF CHECKED FACTS\}}}
				\newline
				[The end of the list of checked facts]
			};
			\node[chatbox_user_inner] (q1_text) at (q1) {
				\textbf{System prompt}
				\newline
				\newline
				You are a helpful and precise assistant for segmenting and labeling sentences. We would like to request your help on curating a dataset for entity-level hallucination detection.
				\newline \newline
                We will give you a machine generated biography and a list of checked facts about the biography. Each fact consists of a sentence and a label (True/False). Please do the following process. First, breaking down the biography into words. Second, by referring to the provided list of facts, merging some broken down words in the previous step to form meaningful entities. For example, ``strategic thinking'' should be one entity instead of two. Third, according to the labels in the list of facts, labeling each entity as True or False. Specifically, for facts that share a similar sentence structure (\eg, \textit{``He was born on Mach 9, 1941.''} (\texttt{True}) and \textit{``He was born in Ramos Mejia.''} (\texttt{False})), please first assign labels to entities that differ across atomic facts. For example, first labeling ``Mach 9, 1941'' (\texttt{True}) and ``Ramos Mejia'' (\texttt{False}) in the above case. For those entities that are the same across atomic facts (\eg, ``was born'') or are neutral (\eg, ``he,'' ``in,'' and ``on''), please label them as \texttt{True}. For the cases that there is no atomic fact that shares the same sentence structure, please identify the most informative entities in the sentence and label them with the same label as the atomic fact while treating the rest of the entities as \texttt{True}. In the end, output the entities and labels in the following format:
                \begin{itemize}[nosep]
                    \item Entity 1 (Label 1)
                    \item Entity 2 (Label 2)
                    \item ...
                    \item Entity N (Label N)
                \end{itemize}
                % \newline \newline
                Here are two examples:
                \newline\newline
                \textbf{[Example 1]}
                \newline
                [The start of the biography]
                \newline
                \textcolor{titlecolor}{Marianne McAndrew is an American actress and singer, born on November 21, 1942, in Cleveland, Ohio. She began her acting career in the late 1960s, appearing in various television shows and films.}
                \newline
                [The end of the biography]
                \newline \newline
                [The start of the list of checked facts]
                \newline
                \textcolor{anscolor}{[Marianne McAndrew is an American. (False); Marianne McAndrew is an actress. (True); Marianne McAndrew is a singer. (False); Marianne McAndrew was born on November 21, 1942. (False); Marianne McAndrew was born in Cleveland, Ohio. (False); She began her acting career in the late 1960s. (True); She has appeared in various television shows. (True); She has appeared in various films. (True)]}
                \newline
                [The end of the list of checked facts]
                \newline \newline
                [The start of the ideal output]
                \newline
                \textcolor{labelcolor}{[Marianne McAndrew (True); is (True); an (True); American (False); actress (True); and (True); singer (False); , (True); born (True); on (True); November 21, 1942 (False); , (True); in (True); Cleveland, Ohio (False); . (True); She (True); began (True); her (True); acting career (True); in (True); the late 1960s (True); , (True); appearing (True); in (True); various (True); television shows (True); and (True); films (True); . (True)]}
                \newline
                [The end of the ideal output]
				\newline \newline
                \textbf{[Example 2]}
                \newline
                [The start of the biography]
                \newline
                \textcolor{titlecolor}{Doug Sheehan is an American actor who was born on April 27, 1949, in Santa Monica, California. He is best known for his roles in soap operas, including his portrayal of Joe Kelly on ``General Hospital'' and Ben Gibson on ``Knots Landing.''}
                \newline
                [The end of the biography]
                \newline \newline
                [The start of the list of checked facts]
                \newline
                \textcolor{anscolor}{[Doug Sheehan is an American. (True); Doug Sheehan is an actor. (True); Doug Sheehan was born on April 27, 1949. (True); Doug Sheehan was born in Santa Monica, California. (False); He is best known for his roles in soap operas. (True); He portrayed Joe Kelly. (True); Joe Kelly was in General Hospital. (True); General Hospital is a soap opera. (True); He portrayed Ben Gibson. (True); Ben Gibson was in Knots Landing. (True); Knots Landing is a soap opera. (True)]}
                \newline
                [The end of the list of checked facts]
                \newline \newline
                [The start of the ideal output]
                \newline
                \textcolor{labelcolor}{[Doug Sheehan (True); is (True); an (True); American (True); actor (True); who (True); was born (True); on (True); April 27, 1949 (True); in (True); Santa Monica, California (False); . (True); He (True); is (True); best known (True); for (True); his roles in soap operas (True); , (True); including (True); in (True); his portrayal (True); of (True); Joe Kelly (True); on (True); ``General Hospital'' (True); and (True); Ben Gibson (True); on (True); ``Knots Landing.'' (True)]}
                \newline
                [The end of the ideal output]
				\newline \newline
				\textbf{User prompt}
				\newline
				\newline
				[The start of the biography]
				\newline
				\textcolor{magenta}{\texttt{\{BIOGRAPHY\}}}
				\newline
				[The ebd of the biography]
				\newline \newline
				[The start of the list of checked facts]
				\newline
				\textcolor{magenta}{\texttt{\{LIST OF CHECKED FACTS\}}}
				\newline
				[The end of the list of checked facts]
			};
		\end{tikzpicture}
        \caption{GPT-4o prompt for labeling hallucinated entities.}\label{tb:gpt-4-prompt}
	\end{center}
\vspace{-0cm}
\end{table*}
% \section{Full Experiment Results}
% \begin{table*}[th]
    \centering
    \small
    \caption{Classification Results}
    \begin{tabular}{lcccc}
        \toprule
        \textbf{Method} & \textbf{Accuracy} & \textbf{Precision} & \textbf{Recall} & \textbf{F1-score} \\
        \midrule
        \multicolumn{5}{c}{\textbf{Zero Shot}} \\
                Zero-shot E-eyes & 0.26 & 0.26 & 0.27 & 0.26 \\
        Zero-shot CARM & 0.24 & 0.24 & 0.24 & 0.24 \\
                Zero-shot SVM & 0.27 & 0.28 & 0.28 & 0.27 \\
        Zero-shot CNN & 0.23 & 0.24 & 0.23 & 0.23 \\
        Zero-shot RNN & 0.26 & 0.26 & 0.26 & 0.26 \\
DeepSeek-0shot & 0.54 & 0.61 & 0.54 & 0.52 \\
DeepSeek-0shot-COT & 0.33 & 0.24 & 0.33 & 0.23 \\
DeepSeek-0shot-Knowledge & 0.45 & 0.46 & 0.45 & 0.44 \\
Gemma2-0shot & 0.35 & 0.22 & 0.38 & 0.27 \\
Gemma2-0shot-COT & 0.36 & 0.22 & 0.36 & 0.27 \\
Gemma2-0shot-Knowledge & 0.32 & 0.18 & 0.34 & 0.20 \\
GPT-4o-mini-0shot & 0.48 & 0.53 & 0.48 & 0.41 \\
GPT-4o-mini-0shot-COT & 0.33 & 0.50 & 0.33 & 0.38 \\
GPT-4o-mini-0shot-Knowledge & 0.49 & 0.31 & 0.49 & 0.36 \\
GPT-4o-0shot & 0.62 & 0.62 & 0.47 & 0.42 \\
GPT-4o-0shot-COT & 0.29 & 0.45 & 0.29 & 0.21 \\
GPT-4o-0shot-Knowledge & 0.44 & 0.52 & 0.44 & 0.39 \\
LLaMA-0shot & 0.32 & 0.25 & 0.32 & 0.24 \\
LLaMA-0shot-COT & 0.12 & 0.25 & 0.12 & 0.09 \\
LLaMA-0shot-Knowledge & 0.32 & 0.25 & 0.32 & 0.28 \\
Mistral-0shot & 0.19 & 0.23 & 0.19 & 0.10 \\
Mistral-0shot-Knowledge & 0.21 & 0.40 & 0.21 & 0.11 \\
        \midrule
        \multicolumn{5}{c}{\textbf{4 Shot}} \\
GPT-4o-mini-4shot & 0.58 & 0.59 & 0.58 & 0.53 \\
GPT-4o-mini-4shot-COT & 0.57 & 0.53 & 0.57 & 0.50 \\
GPT-4o-mini-4shot-Knowledge & 0.56 & 0.51 & 0.56 & 0.47 \\
GPT-4o-4shot & 0.77 & 0.84 & 0.77 & 0.73 \\
GPT-4o-4shot-COT & 0.63 & 0.76 & 0.63 & 0.53 \\
GPT-4o-4shot-Knowledge & 0.72 & 0.82 & 0.71 & 0.66 \\
LLaMA-4shot & 0.29 & 0.24 & 0.29 & 0.21 \\
LLaMA-4shot-COT & 0.20 & 0.30 & 0.20 & 0.13 \\
LLaMA-4shot-Knowledge & 0.15 & 0.23 & 0.13 & 0.13 \\
Mistral-4shot & 0.02 & 0.02 & 0.02 & 0.02 \\
Mistral-4shot-Knowledge & 0.21 & 0.27 & 0.21 & 0.20 \\
        \midrule
        
        \multicolumn{5}{c}{\textbf{Suprevised}} \\
        SVM & 0.94 & 0.92 & 0.91 & 0.91 \\
        CNN & 0.98 & 0.98 & 0.97 & 0.97 \\
        RNN & 0.99 & 0.99 & 0.99 & 0.99 \\
        % \midrule
        % \multicolumn{5}{c}{\textbf{Conventional Wi-Fi-based Human Activity Recognition Systems}} \\
        E-eyes & 1.00 & 1.00 & 1.00 & 1.00 \\
        CARM & 0.98 & 0.98 & 0.98 & 0.98 \\
\midrule
 \multicolumn{5}{c}{\textbf{Vision Models}} \\
           Zero-shot SVM & 0.26 & 0.25 & 0.25 & 0.25 \\
        Zero-shot CNN & 0.26 & 0.25 & 0.26 & 0.26 \\
        Zero-shot RNN & 0.28 & 0.28 & 0.29 & 0.28 \\
        SVM & 0.99 & 0.99 & 0.99 & 0.99 \\
        CNN & 0.98 & 0.99 & 0.98 & 0.98 \\
        RNN & 0.98 & 0.99 & 0.98 & 0.98 \\
GPT-4o-mini-Vision & 0.84 & 0.85 & 0.84 & 0.84 \\
GPT-4o-mini-Vision-COT & 0.90 & 0.91 & 0.90 & 0.90 \\
GPT-4o-Vision & 0.74 & 0.82 & 0.74 & 0.73 \\
GPT-4o-Vision-COT & 0.70 & 0.83 & 0.70 & 0.68 \\
LLaMA-Vision & 0.20 & 0.23 & 0.20 & 0.09 \\
LLaMA-Vision-Knowledge & 0.22 & 0.05 & 0.22 & 0.08 \\

        \bottomrule
    \end{tabular}
    \label{full}
\end{table*}




\end{document}



\end{document}
\endinput
%%
%% End of file `sample-sigconf.tex'.
