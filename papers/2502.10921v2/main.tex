%%
%% This is file `sample-sigconf.tex',
%% generated with the docstrip utility.
%%
%% The original source files were:
%%
%% samples.dtx  (with options: `all,proceedings,bibtex,sigconf')
%% 
%% IMPORTANT NOTICE:
%% 
%% For the copyright see the source file.
%% 
%% Any modified versions of this file must be renamed
%% with new filenames distinct from sample-sigconf.tex.
%% 
%% For distribution of the original source see the terms
%% for copying and modification in the file samples.dtx.
%% 
%% This generated file may be distributed as long as the
%% original source files, as listed above, are part of the
%% same distribution. (The sources need not necessarily be
%% in the same archive or directory.)
%%
%%
%% Commands for TeXCount
%TC:macro \cite [option:text,text]
%TC:macro \citep [option:text,text]
%TC:macro \citet [option:text,text]
%TC:envir table 0 1
%TC:envir table* 0 1
%TC:envir tabular [ignore] word
%TC:envir displaymath 0 word
%TC:envir math 0 word
%TC:envir comment 0 0
%%
%%
%% The first command in your LaTeX source must be the \documentclass
%% command.
%%
%% For submission and review of your manuscript please change the
%% command to \documentclass[manuscript, screen, review]{acmart}.
%%
%% When submitting camera ready or to TAPS, please change the command
%% to \documentclass[sigconf]{acmart} or whichever template is required
%% for your publication.
%%
%%

\documentclass[sigconf]{acmart}
\usepackage[utf8]{inputenc}
\usepackage{algorithmic}
\usepackage{graphicx}
\usepackage{textcomp}
\usepackage{multirow}
\usepackage{mdframed}
\usepackage{enumitem}
\usepackage{booktabs}
\usepackage{comment}
\usepackage{url}
\usepackage{tikzsymbols}
\usepackage{siunitx}
\usepackage{tabularx}
\usepackage{colortbl}
\usepackage{color,soul}
\usepackage{bm}
\usepackage{courier}
\usepackage{xcolor}
\usepackage{mathtools}
\usepackage{makecell}
\usepackage{dirtytalk}
\usepackage{lscape} 
\usepackage{array} 
\usepackage{balance}
\definecolor{Sepia}{RGB}{112, 66, 20} % Defining a sepia-like color
\usepackage[most]{tcolorbox}     % Required package for custom boxes

% Define the gray box style
\newmdenv[
  linecolor=gray!50,         % Border color
  backgroundcolor=gray!10,   % Background color
  linewidth=2pt,             % Border width
  roundcorner=5pt,           % Border corner radius
]{graybox}

%%
%% \BibTeX command to typeset BibTeX logo in the docs
\AtBeginDocument{%
  \providecommand\BibTeX{{%
    Bib\TeX}}}

%% Rights management information.  This information is sent to you
%% when you complete the rights form.  These commands have SAMPLE
%% values in them; it is your responsibility as an author to replace
%% the commands and values with those provided to you when you
%% complete the rights form.
%\setcopyright{acmlicensed}
%\copyrightyear{2018}
%\acmYear{2018}
%\acmDOI{XXXXXXX.XXXXXXX}

%% These commands are for a PROCEEDINGS abstract or paper.
%\acmConference[Conference acronym 'XX]{Make sure to enter the correct
%  conference title from your rights confirmation email}{June 03--05,
%  2018}{Woodstock, NY}
%%
%%  Uncomment \acmBooktitle if the title of the proceedings is different
%%  from ``Proceedings of ...''!
%%
%%\acmBooktitle{Woodstock '18: ACM Symposium on Neural Gaze Detection,
%%  June 03--05, 2018, Woodstock, NY}
%\acmISBN{978-1-4503-XXXX-X/18/06}


%%
%% Submission ID.
%% Use this when submitting an article to a sponsored event. You'll
%% receive a unique submission ID from the organizers
%% of the event, and this ID should be used as the parameter to this command.
%%\acmSubmissionID{123-A56-BU3}

%%
%% For managing citations, it is recommended to use bibliography
%% files in BibTeX format.
%%
%% You can then either use BibTeX with the ACM-Reference-Format style,
%% or BibLaTeX with the acmnumeric or acmauthoryear sytles, that include
%% support for advanced citation of software artefact from the
%% biblatex-software package, also separately available on CTAN.
%%
%% Look at the sample-*-biblatex.tex files for templates showcasing
%% the biblatex styles.
%%

%%
%% The majority of ACM publications use numbered citations and
%% references.  The command \citestyle{authoryear} switches to the
%% "author year" style.
%%
%% If you are preparing content for an event
%% sponsored by ACM SIGGRAPH, you must use the "author year" style of
%% citations and references.
%% Uncommenting
%% the next command will enable that style.
%%\citestyle{acmauthoryear}


\newif\ifcomment
\commenttrue
%\commentfalse
\ifcomment
\newcommand{\gs}[1]{{\bf\textcolor{red}{GS: #1}}}
\newcommand{\shiza}[1]{{\bf\textcolor{teal}{Shiza: #1}}}
\else
\newcommand{\gs}[1]{}
\newcommand{\shiza}[1]{}
\fi
\newcommand{\descr}[1]{\smallskip\noindent\textbf{#1}}


%%
%% end of the preamble, start of the body of the document source.
\begin{document}

%%
%% The "title" command has an optional parameter,
%% allowing the author to define a "short title" to be used in page headers.
\title{Evolving Hate Speech Online: An Adaptive Framework for Detection and Mitigation}

%%
%% The "author" command and its associated commands are used to define
%% the authors and their affiliations.
%% Of note is the shared affiliation of the first two authors, and the
%% "authornote" and "authornotemark" commands
%% used to denote shared contribution to the research.
\author{Shiza Ali}
\email{shiza@uw.edu}
%\orcid{1234-5678-9012}
%\author{G.K.M. Tobin}
%\authornotemark[1]
%\email{webmaster@marysville-ohio.com}
\affiliation{%
  \institution{University of Washington}
  \city{Bothell}
  \state{Washington}
  \country{USA}
}
\author{Jeremy Blackburn}
\email{jblackbu@binghamton.edu}
\affiliation{%
  \institution{Binghamton University}
  \city{Binghamton}
  \state{New York}
  \country{USA}
}

\author{Gianluca Stringhini}
\email{gian@bu.edu}
\affiliation{%
  \institution{Boston University}
  \city{Boston}
  \state{Massachusets}
  \country{USA}}


%%
%% By default, the full list of authors will be used in the page
%% headers. Often, this list is too long, and will overlap
%% other information printed in the page headers. This command allows
%% the author to define a more concise list
%% of authors' names for this purpose.
\renewcommand{\shortauthors}{Ali et al.}

%%
%% The abstract is a short summary of the work to be presented in the
%% article.
\begin{abstract}

The proliferation of social media platforms has led to an increase in the spread of hate speech, particularly targeting vulnerable communities.
Unfortunately, existing methods for automatically identifying and blocking toxic language rely on pre-constructed lexicons, making them reactive rather than adaptive.
As such, these approaches become less effective over time, especially when new communities are targeted with slurs not included in the original datasets.
To address this issue, we present an adaptive approach that uses word embeddings to update lexicons and develop a hybrid model that adjusts to emerging slurs and new linguistic patterns.
This approach can effectively detect toxic language, including intentional spelling mistakes employed by aggressors to avoid detection.
Our hybrid model, which combines BERT with lexicon-based techniques, achieves an accuracy of ~95\% for most state-of-the-art datasets. Our work has significant implications for creating safer online environments by improving the detection of toxic content and proactively updating the lexicon.\\
\noindent \textbf{Content Warning:} \textit{This paper contains examples of hate speech that may be triggering.}
\end{abstract}

%%
%% The code below is generated by the tool at http://dl.acm.org/ccs.cfm.
%% Please copy and paste the code instead of the example below.
%%
\begin{CCSXML}
<ccs2012>
   <concept>
       <concept_id>10002978.10003029.10003032</concept_id>
       <concept_desc>Security and privacy~Social aspects of security and privacy</concept_desc>
       <concept_significance>500</concept_significance>
       </concept>
   <concept>
       <concept_id>10003120.10003130.10011762</concept_id>
       <concept_desc>Human-centered computing~Empirical studies in collaborative and social computing</concept_desc>
       <concept_significance>500</concept_significance>
       </concept>
 </ccs2012>
\end{CCSXML}

\ccsdesc[500]{Security and privacy~Social aspects of security and privacy}
\ccsdesc[500]{Human-centered computing~Empirical studies in collaborative and social computing}




%%
%% Keywords. The author(s) should pick words that accurately describe
%% the work being presented. Separate the keywords with commas.
\keywords{toxicity, social media, adaptive moderation, hybrid model, social computing}

%\received{20 February 2007}
%\received[revised]{12 March 2009}
%\received[accepted]{5 June 2009}

%%
%% This command processes the author and affiliation and title
%% information and builds the first part of the formatted document.
\maketitle

In recent years, there has been a notable increase in the development and research of tethered UAVs, reflecting a growing interest in their diverse applications. One of the main motivations is to carry out long-term missions with aerial vehicles, as these present significant challenges due to the limitations of current battery solutions \cite{robotics12040117}. A UAV tethered to a UGV is an interesting configuration, as the UGV can power the UAV through the tether for longer times given the higher payload of the former.  %According to this, an interesting configuration to allow long-duration flights of a UAV is a tethered robot configuration in which a UGV is tied to the UAV and powering it. 
This introduces a paradigm in robotic collaboration, offering distinct advantages over traditional standalone systems by combining the strengths of each of the robotic agents \cite{MooreIROS2018}. %As we venture UGV tied to UAV into scenarios requiring heightened enhanced situational awareness involving an extended operational endurance, the tethered approach proves invaluable, due to the capability to provide energy to the UAV, thus increasing fly time \cite{6961531}. 
When deploying a UGV tethered to a UAV in scenarios requiring increased situational awareness and extended operational endurance, the tethered configuration can become even more invaluable, not only providing the UAV with power to significantly extend its flight time \cite{6961531},  %In this way, the cable plays an important role in providing 
but also with safe high-bandwidth communications \cite{850822,9202196}. 

However, the tethering mechanism introduces several challenges, particularly in modeling the hanging tether state \cite{XiaoSSRR2018}. Unlike standalone systems, where each vehicle operates independently, the tether requires intricate and permanent coordination between the UGV and the UAV. Understanding and managing the state of the tether becomes a critical aspect, which requires sophisticated algorithms and real-time processing capabilities \cite{9561062}. 

\begin{figure}
  \includegraphics[width=0.2\textwidth]{Figures/setup1.png}
  \hfill
  \includegraphics[width=0.2\textwidth]{Figures/setup2.png}
  \caption{Simplified 2D sketch showing an example for motion planning of a tethered UAV-UGV with a hanging tether. (Left) Initial robots and tether configuration, and UAV goal (red circle). (Right) Sequence of robots positions and tether length to reach the given goal. Notice how the goal cannot be reached by means of a taut tether, a hanging tether must be considered in this case.}
  \label{fig:planning-setup}
\end{figure}

The state of the tether has traditionally been analyzed through parameterization, an approach that employs equations to represent its physical behavior, especially the catenary curve \cite{BOOKOFCURVES}. Numerous methodologies, with the aim of simplifying this process, approximate the tether as a straight line \cite{autonomousvisual}\cite{framworktether}\cite{uavfire}. This straight-line approximation is only suitable in scenarios where there is a direct line of sight between the tether endpoints, and thus it inherently restricts the exploratory range of the UAV.

In general, hanging-tether approaches allow UAVs to access a broader range of areas compared to straight tether setups; see Fig. \ref{fig:planning-setup} for an example. This concept has been explored by incorporating tether parameterization into localization or planning processes. For instance, Lima and Pereira \cite{9476778} use the catenary equation to determine the UAV's position.  % This concept has been explored by incorporating tether parameterization into the localization or planning processes, such as in the work conducted by Lima and Pereira \cite{9476778}, where using the catenary equation is feasible to find the UAV position. 
Similarly, in \cite{9364354}, the focus is on computing the state of a catenary tether to localize two UAVs attached at each end. This setup is specifically designed to suspend an object, providing a novel approach to object manipulation using UAVs while maintaining a constant tether length. Another interesting application of the catenary model is presented in \cite{LARANJEIRA2020107018} for underwater operations, where the catenary is used to monitor the status of a cable connected to an \emph{N}-number of ROVs (Remotely Operated Vehicles) performing exploration tasks, also with a constant tether length.

In \cite{8848946}, the parameterization of the tether is used in the localization and control stages to perform two autonomous motion primitives, reactive feedback-based position control and model-predictive feedforward velocity control, but is not used in the planning stage. An interesting approach is presented in \cite{drones7020073}, where a tied unmanned aerial vehicle (TUAV), named ``Oxpecke'', was designed for the inspection of stone-mine pillars. This system uses a sweeping (lawnmower) pattern path planning method intended to map and inspect an entire rectangular area, such as the surface of a pillar. However, the surface to inspect is simple (a rectangle), and the tether length is not directly included in the path planning.

%A general approach about the consideration of the tether in the planning stage is introduced in \cite{battocletti2024entanglementdefinitionstetheredrobots}, where the authors present the definition of tether entanglement problems. Specifically, it addresses the challenges posed by the presence of a tether, including the geometric constraints on the robot's motion due to the finite tether length. For that, different constraints are considered in the planning stage. However, the method is too general and mainly tested in ground points, so UAV implementation are not considering, and algo this method allow tether contact with the floor while entanglement desnt exit.

A comprehensive approach to incorporate a tether in the planning stage is presented in \cite{battocletti2024entanglementdefinitionstetheredrobots}, where the authors define the challenges associated with tether entanglement. Specifically, this work addresses the constraints imposed by the tether on the motion of the robot, particularly the limitations arising from the finite length of the tether. Various constraints are integrated into the planning stage to account for these challenges. However, the proposed method is limited and mainly focused on ground applications, 
thus limiting its applicability to UAVs. Additionally, the approach allows for tether contact with the ground, as long as it does not result in entanglement.

On the other hand, \cite{capitán2024efficientstrategypathplanning} focuses on the development of a path planning strategy for marsupial robotic systems composed of a UGV tethered to a UAV. The article introduces a sequential planning strategy called MASPA (Marsupial Sequential Path-Planning Approach), which allows calculating collision-free 3D trajectories for the tethered UAV-UGV system in complex scenarios, for which the UGV advances to a point where the UAV executes the take-off and then advances to a desired point. This method considers both the geometric limitations imposed by obstacles and the cable and the properties of the joint motion of both robots. A novel algorithm, the PVA (Polygonal Visibility Algorithm), is also presented to identify feasible take-off points and solve visibility problems for the UAV in a three-dimensional space. Despite the novelty of the approach, it is not able to consider coordinated planning of the UGV and the UAV at the same time.

In \cite{smartinezr2023}, the catenary approximation is used to parameterize the state of the tether and plan a collision-free trajectory, in which the UAV must achieve objectives using a hanging tether. However, using the catenary equation, the planning process becomes a time-consuming task, allowing only offline computations. %which makes the planning process to be carried out offline.

%This paper will focus on reducing the complexity associated with the calculation of the variable length hanging tether. We will propose an approach that efficiently calculates the tether state with a minimum representation error concerning the real state, and integrating it into a trajectory planning algorithm for a UGV-UAV tethered team. To this end, we test our approach in the motion planning method for a mobile UGV-UAV tethered system presented in \cite{smartinezr2023}, which is based on two stages. The first stage computes a free-collision path planning for UAV, UGV, and tether, using the RRT* algorithm. The second stage corresponds to a trajectory planning method based on nonlinear optimization that considers smoothness, speed, acceleration limitations of the UGV and UAV, and optimizes the tether configuration to maximize the distance from obstacles. %Unfortunately, considering the real catenary curve in the planner could make it computationally demanding, as shown in our previous work \cite{martinez2021optimization}. In it, we manage to design, implement and test in experiments a two-step optimized planner which considers the catenary shape. For this reason, we propose to approximate the shape of the tether as a parabola without affecting the safety of the planning system and making use of its simpler description to speed-up the computation of optimal paths.

%Our approach is based on the motion planning mentioned above due to the robustness of computed trajectories. Thus, we include in the first stage, a decision problem to set the initial tether length, to quickly obtain a collision-free state for the whole system. Furthermore, we propose a new planner-state parameterization and replace the use of the catenary equation with a parabola equation for estimating the shape of the tether. Thus, the main contributions of the article are:
%\begin{itemize}
%
%\item In Planner Stage: Solving the decision problem to find a collision-free parabola curve instead of the traditional catenary curve. This change allows the RRT* (Rapidly-exploring Random Tree) planner to calculate trajectories faster and more efficiently, since it avoids the computational complexity associated with the calculation of the catenary. The parabolic curve simplifies the collision decision process and increases the success rate in three-dimensional environments with obstacles.
%
%\item  In Optimizer stage: This stage introduces a direct parameterization of the tether in the trajectory state function, which includes the parameters of the curve (parabola or catenary) in the system state vector. This allows a more accurate evaluation of geometric constraints (such as distance to obstacles) and reduces the optimization time by up to an order of magnitude compared to previous methods, achieving safer and smoother trajectories for the UAV-UGV system.
%\end{itemize}

This paper focuses on reducing the complexity associated with the calculation of the variable length hanging tether. %The paper builds on the previous work of the authors \cite{smartinezr2023}, extending it with a new approach that efficiently calculates the tether state with a minimum representation error related to the actual state and a new parameterization of the tether curve in the trajectory optimizer for faster computation. Thus, 
The main contributions are listed below.

\begin{itemize}
    \item A new method for efficient computation of a collision-free catenary curve based on the parabola approximation. This paper proposes using the parabola curve to model the hanging tether curve, detailing the full pipeline, including the computation of the final catenary model. This method reduces the execution time of the path planner to great extent, since it avoids the computational complexity associated with the calculation of the catenary model for tether collision detection. This model also increases the feasibility of the trajectory planner approach, reaching an averaged 98\% of feasibility in the validation scenarios. 

    \item A direct parameterization of the tether in the trajectory state definition, which includes the parameters of the curve (parabola or catenary) in the system state vector. This allows a more accurate evaluation of geometric constraints (such as distance to obstacles) and reduces the optimization time \rev{by more than an order of magnitude} compared to previous methods, achieving safer and smoother trajectories for the UAV-UGV system. \rev{Such improvement opens the door to apply the proposed method to real-time local re-planning.}
\end{itemize}

%The experimental results will show how this new parameterization boosts the computation, while the parabola model will clearly improve the feasibility of the method over the catenary. 

%Thus, we include in the first stage, a decision problem to set the initial tether length, to quickly obtain a collision-free state for the whole system. Additionally, we replace the traditional catenary equation with a parabolic approximation to estimate the tether shape more efficiently. In the second stage of nonlinear optimization stage, we further simplify the process by parameterizing the tether instead of relying on the catenary model. This approach not only streamlines the representation of the curve but also facilitates more straightforward and efficient gradient calculations during optimization.

The paper is structured as follows. In Section \ref{sec:overview}, we show the general problem to be solved, whereas Section \ref{sec:approach} formalizes the solutions proposed. Section \ref{sec:path_planning} details the implementation of the solution within the planning stage. In Section \ref{sec:optimization_process}, we describe how curve parameterization is utilized to enhance the optimization process for trajectory computation. The experimental results are discussed in Section \ref{sec:experiments}. Finally, the paper is concluded in Section \ref{sec:conclusions}.

\section{Related work}
\label{sec:related_work}
Anomaly detection, also known as outlier detection or novelty detection, is an important problem that has been studied within diverse research areas and application domains~\citep{chandola2009anomaly,chalapathy2019deep}. The problem of traffic AD bares similarities with the disciplines of robot AD and AD for surveillance cameras. In this section, we briefly review the related research and introduce common techniques in ensemble deep learning.

Recent research efforts have made noteworthy progress in developing learning-based AD algorithms for robots and mechanical systems. \cite{malhotra2016lstm} introduces an LSTM-based encoder-decoder scheme for multi-sensor AD (EncDec-AD) that learns to reconstruct normal data and uses reconstruction error to detect anomalies. \cite{park2018multimodal} proposes an LSTM-based variational autoencoder (VAE) that fuses sensory signals and reconstructs their expected distribution. The detector then reports an anomaly when a reconstruction-based anomaly score is higher than a state-based threshold. \cite{feng2022unsupervised} attacks multimodal AD with missing sources at any modality. A group of autoencoders (AEs) first restore missing sources to construct complete modalities, and then a skip-connected AE reconstructs the complete signal. Although similar in ideas, these approaches were proposed for low-dimensional signals (e.g., accelerations and pressures) and have not shown effective on high-dimensional data (e.g., images).

AD for robot navigation often involves complex perception signals from cameras and LiDARs in order to understand the environment. \cite{ji2020multi} proposes a supervised VAE (SVAE) model, which utilizes the representational power of VAE for supervised learning tasks, to identify anomalous patterns in 2D LiDAR point clouds during robot navigation. The predictive model proposed in LaND~\citep{kahn2021land} takes as input an image and a sequence of future control actions to predict probabilities of collision for each time step within the prediction horizon. \cite{schreiber2023attentional} further enhances the robot perception capability with the fusion of RGB images and LiDAR point clouds using an attention-based recurrent neural network, achieving improved AD performance on field robots. Different from these supervised-learning-based methods, \cite{wellhausen2020safe} uses normalizing flow models to learn distributions of normal samples of multimodal images, in order to realize safe robot navigation in novel environments. However, driving scenarios have additional complexities than field environments. While road environments are more structured than field environments, additional hazards arise from the presence of and interactions between dynamic road participants, which pose extra challenges on AD algorithms.

Another widely explored research area that is relevant to our work is AD for surveillance cameras, which mainly focuses on detecting the start and end time of anomalous events within a video. Under the category of frame-level methods, \cite{hasan2016learning} proposes a convolutional autoencoder to detect anomalous events by reconstructing stacked images. \cite{chong2017abnormal} and~\cite{luo2017remembering} extend such an idea by learning spatial features and the temporal evolution of the spatial features separately using convolution layers and ConvLSTM layers~\citep{shi2015convolutional}, respectively. Instead of reconstructing frames, \cite{liu2018future} trains a fully convolutional network to predict future frames based on past observations and uses the Peak Signal to Noise Ratio of the predicted frame as the anomaly score. \cite{gong2019memorizing} develops an autoencoder with a memory module, called memory-augmented autoencoder, to limit the generalization capability of the network on reconstructing anomalies. To focus more on small anomalous regions, patch-level methods generate the anomaly score of a frame as the max pooling of patch errors in the image rather than the averaged pixel error used in frame-level methods~\citep{wang2023memory}. In addition, object-level approaches have also been explored, which often focus on modeling normal object motions either through extracted features (e.g., human skeletons)~\citep{morais2019learning} or raw pixel values within bounding boxes~\citep{liu2021hybrid}. Although these methods have achieved promising results on surveillance cameras, the performance is often compromised in egocentric driving scenarios due to moving cameras and complex scenes~\citep{yao2022dota}.

In the domain of traffic AD in first-person videos, pioneering works borrow ideas from surveillance camera applications and detect abnormality by reconstructing motion features at frame level~\citep{yuan2016anomaly}. However, to overcome the issues introduced by rapid motions of cameras and thus backgrounds, object-centric methods are becoming increasingly popular. One of the most representative works by~\cite{yao2019unsupervised} proposes a recurrent encoder-decoder framework to predict future trajectories of an object in the image plane based on the object's past trajectories, spatiotemporal features, and ego motions. The accuracy and consistency of the predictions are then used to generate anomaly scores. One critical problem of such a method is the inevitable miss detection in the absence of traffic participants. As a result, ensemble methods emerge recently to combine the strengths of frame-level and object-centric methods. For example, ~\cite{yao2022dota} fuses the object location prediction model with the frame prediction model to achieve all-scenario detection capability and ~\cite{fang2022traffic} monitors the temporal consistency of frames, object locations, and spatial relation structures of scenes for AD. In this work, we derive each module and the corresponding learning objective in the ensemble based on a comprehensive analysis on anomaly patterns in egocentric driving videos. In particular, an interaction module is introduced to monitor anomalous interactions between road participants. The scores from each module are then fed as observations of a Kalman filter, from which the final anomaly score is obtained.

In this paper, we introduce an ensemble of detectors to capture different classes of anomalies.
We take inspiration from recent advances in ensemble deep learning, which aims to improve the generalization performance of a learning system by combing several individual deep learning models~\citep{ganaie2022ensemble} and has been applied to different application domains, such as speech recognition~\citep{li2017semi},  image classification~\citep{wang2020particle}, forecasting~\citep{singla2022ensemble}, and fault diagnosis~\citep{wen2022new}. Out of different classes of ensemble deep learning approaches, the most similar work to Xen is the heterogeneous ensemble (HEE), in which the components are trained on the same dataset but use different algorithms/architectures~\citep{li2018heterogeneous,tabik2020mnist}. However, each component in an HEE is usually trained with an identical learning objective, while each expert in Xen is assigned with different learning tasks. In terms of result fusion strategies, unweighted model averaging is one of the most popular approaches in the literature~\citep{ganaie2022ensemble}, which simply averages the outcomes of the base learners to get the final prediction of the ensemble model. By contrast, we exploit a Kalman filter to further combat the noise in scores from different components in a time-series task, and the unweighted model averaging can be viewed as a special case of such a method at a point along the time axis.


\begin{table}[t]
    \centering
    \begin{tabular}{lccccc}
    \toprule
         \textbf{Dataset} & \textbf{GUI Types} & \textbf{Number of GUIs} & \textbf{Detections} & \textbf{Descriptions} & \textbf{Scanpaths} \\
    \midrule
         MUD~\cite{mud} & Mobile (Android) & $\sim$18,000 & \cmark & \xmark & \xmark \\  
         RICO~\cite{rico} & Mobile (Android) & $\sim$72,000 & \cmark & \xmark & \xmark \\
         RICO-Semantic~\cite{rico_semantic} & Mobile (Android) & $\sim$72,000 & \cmark & \xmark & \xmark \\
         Clay~\cite{li2022learning} & Mobile (Android) & $\sim$60,000 & \cmark & \xmark & \xmark \\
          ENRICO~\cite{enrico} & Mobile (Android) & 1,460 & \cmark & \xmark & \xmark \\ 
          VINS~\cite{vins} & Mobile (Android + IOS) & 4,543 & \cmark & \xmark & \xmark \\ 
          AMP~\cite{zhang2021screen} & Mobile  (IOS) (not publicly available) & $\sim$77,000 & \cmark & \xmark & \xmark \\ 
          Webzeitgeist~\cite{kumar2013webzeitgeist} & Webpages & 103,744 & \cmark & \xmark & \xmark \\
         WebUI~\cite{webui} & Webpages & $\sim$350,000 & \cmark & \xmark & \xmark \\  
         UEyes~\cite{ueyes} & Mobile, Webpages, Poster, Desktop & 1,980 & \xmark & \xmark & \cmark \\
         \midrule
         \bf Ours & Webpages, Mobile (Android + IOS) & $\sim$72,500 & \cmark & \cmark & \textcolor{munsell}{Predicted} \\        
    \bottomrule
    \end{tabular}
    \caption{Comparison of GUI datasets: Our dataset uniquely combines mobile and webpage GUIs with comprehensive annotations and predicted scanpaths.
}
    \label{tab:datasets_related}
\end{table}

\section{Dataset}

\paragraph{Prior Datasets}
No existing datasets can be directly used to train a model conditioned on prompts, wireframes, and visual flow directions. 
Several datasets have been collected to support GUI tasks, as shown in \autoref{tab:datasets_related}. The AMP dataset, comprising 77,000 high-quality screens from 4,068 iOS apps with human annotations~\cite{zhang2021screen}, is not publicly available. On the other hand, the largest publicly available dataset, Rico~\cite{rico}, includes 72,000 app screens from 9,700 Android apps and has been a primary resource for GUI understanding despite its inherent noise. To address its limitation, the Clay dataset~\cite{li2022learning} was created by denoising Rico using a pipeline of automated machine learning models and human annotators to provide more accurate element labels. Enrico~\cite{enrico} further cleaned and annotated Rico but ultimately contains only a small set of high-quality GUIs. MUD~\cite{mud} offers a dataset featuring modern-style Android GUIs. The VINS dataset~\cite{vins} focuses on GUI element detection and was created by manually capturing screenshots from various sources, including both Android and iOS GUIs. Additionally, Webzeitgeist~\cite{kumar2013webzeitgeist} used automated crawling to mine design data from 103,744 webpages, associating web elements with properties such as HTML tags, size, font, and color. Similarly, WebUI~\cite{webui} provides a large-scale collection of website data. None of these datasets include both mobile GUIs and webpages and have included visual flow information in the datasets.
UEyes~\cite{ueyes} is the first mixed GUI-type eye tracking dataset with ground-truth scanpaths, although it lacks element labels.

%In our work, we create a comprehensive high-quality dataset that includes both mobile GUIs (Android and iOS) and webpages by cleaning and combining GUIs from Enrico~\cite{enrico}, VINS~\cite{vins}, and WebUI~\cite{webui}. For each GUI, we further generate segmentation maps with GUI element labels, apply LLaVA-Next~\cite{liu2024llavanext} to generate description, and use EyeFormer~\cite{eyeformer}, a state-of-the-art scanpath prediction model, to generate scanpaths.
To address these limitations, we construct a large-scale high-quality dataset of mixed mobile UIs and webpages, including about 72,500 GUI screenshots, along with their wireframes with labeled GUI elements, descriptions, and scanpaths. This dataset is designed to support the training of generative AI models, filling a gap in existing public datasets by providing not only GUI images but also detailed descriptions, element labels, and visual interaction flows. 

\paragraph{GUI Screenshots} Our dataset integrates and cleans GUI data from Enrico~\cite{enrico}, VINS~\cite{vins}, and WebUI~\cite{webui}. The Enrico dataset contains 1,460 Android mobile GUIs, while VINS includes 4,543 Android and iOS GUIs. WebUI is a large-scale webpage dataset consisting of approximately 350,000 GUI screenshots with corresponding HTML code. For WebUI, the original dataset includes screenshots for different resolutions, leading to many similar screenshots for each webpage.
We retained only the 1920 x 1080 resolution screenshots to avoid redundant images from different resolutions. We further refined these three datasets by removing abstract, non-graphic wireframe GUIs, duplicates, and GUIs with fewer than three elements.  The final dataset consists of 66,796 webpages and 5,634 mobile GUIs.

\paragraph{Wireframes with GUI Element Labels} 

For mobile GUIs, we selected the Enrico and VINS datasets for their well-labeled GUI element bounding boxes. To further refine these annotations, we applied the UIED~\cite{uied} model, which detects and refines GUI element bounding boxes. We manually verified and corrected the results for accuracy. For the WebUI dataset, each element has multiple labels. We filtered the original element labels to keep only the most relevant label for each element. We standardized the labels across mobile and webpage elements, mapping them to nine common types: `Button', `Text', Image', `Icon', `Navigation Bar', `Input Field', `Toggle', `Checkbox', and `Scroll Element'. Using these refined bounding boxes and labels, we generated wireframes with GUI element labels.

\paragraph{Descriptions} For each GUI, we employed the LLaVA-Next~\cite{liu2024llavanext} vision-language model to generate both concise and detailed descriptions.

\paragraph{Scanpaths} Finally, we used EyeFormer~\cite{eyeformer}, the state-of-the-art scanpath prediction model, to predict scanpaths for each GUI. While using real scanpaths recorded by eye trackers would provide more accurate data, this process is highly time-consuming. Alternative proxies, such as webcams or cursor movements, do not capture the same cognitive processes as actual eye movements, making them less suitable for our purposes.
\section{Methodology}
In this section, we discuss our adaptive method for updating hate speech lexicons in detail as well as the machine learning approaches we use.
We adopt both traditional supervised-learning approaches and deep-learning models to compare the accuracy of detecting hate speech using the seed lexicons and the updated lexicons. Figure~\ref{fig:pipeline} provides a high-level description of our adaptive approach to hate speech detection.
Later we propose a novel hybrid approach to hate speech detection that utilizes both lexicon-based and unsupervised learning approaches.

 \begin{figure*}[t!]
 	\centering
 	% \includegraphics[width=0.9\textwidth]{Figures/P4.png}
    \includegraphics[width=0.70\textwidth]{Figures/adaptive-toxicity-pipeline.pdf}
 	\caption{Architecture of our adaptive hate speech detection system.}
 	\label{fig:pipeline}
 \end{figure*}

\subsection{Step 1: Identifying Candidate New Toxic Words}
The first step in our pipeline after data collection is to update the lexicons.
The goal of this step is to pinpoint harmful words that are utilized similarly to already established toxic words so that they are relevant to the piece of text in which we are determining hate speech.
For example, to avoid censorship users use the word ``ducking'' instead of ``fucking''~\cite{theverge2023}.
We label all the seed lexicons as $S_{lexicons}$ and the updated lexicons as $U_{lexicons}$ respectively.
The updated lexicons contain both the seed lexicons as well as the new lexicons that we find using word-embedding models. Note that word embedding models allow us to identify \textit{contextually similar words} that may or may not be synonymous.

%\shiza{Seed Lexicons -- ($S_{lexicons}$) \\
%Updated Lexicons -- ($U_{lexicons}$) \\
%Word Embedding Lexicons -- ($W_{lexicons}$)}

To this end, we adopt a \textit{similarity-based} approach to find new toxic words.
For each word appearing in our dataset, we compute its vector embedding.
We test different approaches to find similar words, i.e., Word2Vec~\cite{tensorflowWord2vec}, GloVe~\cite{glove}, and more modern word embedding techniques like BERT~\cite{bert}. %In the following list, we go through each word embedding model in more detail.

% is more detail required
%\begin{itemize}
    %\item \textbf{Word2vec:} Word2Vec, introduced by Mikolov et al.~\cite{mikolov2013efficient} transforms words into continuous vector representations, to enable mathematical operations on words. %For our research, we employ the Word2Vec algorithm to learn distributed representations of words in the corpus.
    %By training the model on the preprocessed dataset, we obtain dense vector representations that capture the semantic relationships between words.
    %Finally, we utilize the trained Word2Vec model to compute the similarity between the seed lexicons ($S_{lexicons}$) and other words in the corpus using cosine distance and update our list of lexicons.
    
    %\item \textbf{GloVe:} GloVe (Global Vectors for Word Representation) is another word embedding model introduced by Pennington et. al.~\cite{pennington2014glove} that can capture both syntactic and semantic information.
    %GloVe learns word representations by leveraging co-occurrence statistics derived from large text corpora.
    %We use the GloVe algorithm to learn word embeddings from the preprocessed posts corpus.
    %The co-occurrence statistics derived from the social media posts dataset enable GloVe to generate distributed word representations that capture the semantic relationships between words.
    %Finally, we employ the learned GloVe embeddings to calculate word similarities using cosine similarity for all the seed lexicons ($S_{lexicons}$).
    
    %\item \textbf{BERT:} Similarly, BERT (Bidirectional Encoder Representations from Transformers), by Devlin et al.~\cite{devlin2018bert}, also captures contextualized word representations but its ability to capture bidirectional context allows it to comprehend word semantics more accurately.
    %For our research, we fine-tune BERT's pre-trained model using our posts dataset.
    %This fine-tuning process allows BERT to adapt to the specific characteristics of Twitter data.
    %Subsequently, we extract informative word embeddings from BERT's contextualized representations of the social media posts corpus and calculate word similarities.
    
%\end{itemize}

By going through this process, we were able to pinpoint words that are \textit{``similar,''} meaning they are utilized within comparable situations.
We determine the similarity between the word embeddings using cosine similarity. \textcolor{black}{We used a cosine similarity threshold of $\geq 0.75$ to identify new toxic words, after empirically testing thresholds of 0.7, 0.75, and 0.8. The threshold of 0.75 struck an optimal balance, generating a diverse set of new words while minimizing redundancy in the lexicon. All flagged words were manually labeled, and 36\% were excluded as non-toxic or irrelevant.}

\noindent\textbf{Graph-based Similarity Approach:} We incorporate a graph-based method using the Louvain algorithm~\cite{blondel2008fast} to assess word similarity through embeddings. Graphs are constructed based on word embeddings, connecting words if the cosine similarity of their vector representations surpasses a predefined threshold of $\geq 0.75$~\cite{zannettou2020quantitative}.
However, this approach exhibits limitations in performance. The method tends to generate numerous false positives, primarily due to its inherent lack of context specificity found in graph-based similarity methods.

The output of this phase is a set of words that we call updated lexicons ($U_{lexicons}$) that are \textit{likely} to be toxic for our specific dataset.

\subsection{Step 2(a): Testing The Updated Lexicons Using Traditional Machine Learning Models}

To test our adaptive approach to hate speech detection we use traditional machine learning models provided by Davidson et. al.~\cite{davidson2017automated}.
We choose Linear Support Vector Machine (SVM)~\cite{cortes1995support}, Random Forest (RF)~\cite{breiman2001random}, and Logistic Regression (LR)~\cite{hosmer2013applied} as traditional classification approaches.
We use the average accuracy of the models, F1-measure, and class-specific precision and recall to evaluate our models on the test sets.
We use grid search and stratified $k$-fold cross-validation ($k=10$) to tune the hyper-parameters during the training and validation phases.

\subsection{Step 2(b): Hybrid Approach For Hate Speech Detection}

The lexicon-based approach relies solely on a predefined list of toxic terms or phrases, which may not capture the evolving nature of hate speech or account for contextual nuances.
It can struggle to identify hate speech that does not precisely match the terms in the lexicon.
On the other hand, BERT models, while effective at capturing contextual information and semantic relationships, may require significant amounts of labeled data for fine-tuning and can be computationally expensive.

By combining the two approaches, we can leverage the strengths of both.
%The lexicon-based method provides a strong foundation for identifying explicit toxic words, while BERT models enhance the detection capability by capturing contextual cues and understanding the complexities of language.
To do this we used the \textit{Lexical Substitution} method for incorporating the hate speech lexicons as features.
We used the set of lexicons to generate additional features for the input text, which are then used as input to the BERT model.
Our method involves enhancing the input embeddings with hate speech lexicons, which are then passed through the pre-trained BERT classification model to get the prediction. Specifically, we tokenize the input text using the BERT tokenizer and then generate binary features for each word or phrase in the lexicon that appears in the input text.
We also augment the input features with binary flags to indicate the presence or absence of each hate speech lexicon in the post.
We do this by first tokenizing the post using the BERT tokenizer and then adding an extra feature vector of 0s and 1s to represent the presence or absence of each hate speech lexicon.
Then, we concatenate the feature vector with the BERT embeddings and pass it through the model. We utilize BERT-based models, like BERT-base~\cite{devlin2018bert}, BERT-large~\cite{devlin2018bert}, and RoBERTa~\cite{liu2019roberta} and state-of-the-art pre trained BERT-model for hatespeech detection Detoxify~\cite{Detoxify}, BERT-HateXplain~\cite{Mathew_Saha_Yimam_Biemann_Goyal_Mukherjee_2021} and HurtBERT~\cite{hurtbert2020} in our analysis and approach development.
%%%%%%%%%%%%%%%%%%%%%%%%%%%%%%%%%%%%%%%%%%%%%%%%%%%%%%%%%%%%%%%%%%%%%%%%%%%%%%%%
\section{RESULTS}
%\subsection{Surface Results}
%\hl{Have more detailed discussion about specific surfaces - consider classifying as uniform, partially uniform, etc.}
The translation results of 0ML, 1ML, and 2ML are compared on 4 surfaces. %The overall trends of planar movement of the three prototypes on the four surfaces are briefly discussed. 
%We remind the reader that for each prototype on a given surface, three trials were performed. %The table below shows the total average displacement and rotation for all three runs of a SoRo on a given surface, with \textbf{bolded} numbers representing the greatest quantity for a given surface and \textit{italicized} numbers representing the lowest value.
%A short discussion is had on the effect of adding compliant microspines onto the tips of this specific MTA SoRo design.  Additionally, we discuss the coupling that occurs between translation and rotation with this given three-limb SoRo design.
\Fig \ref{fig:data} shows the average displacement and standard deviation of the 3 trials per prototype. 

The concrete surface is level with a uniform distribution of asperities. Here, both 1ML and 2ML outperform 0ML in terms of displacement. In fact, 1ML had an average displacement of 39.63cm which is over 30 times as much as the 1.32cm that 0ML traveled. 2ML traveled 14.13cm, nearly 11 times as much as 0ML. 2ML had higher consistency across trials, having a standard deviation of 0.76 whereas 0ML's standard deviation was 0.83.

The partially uniform brick contains gaps in between bricks that can cause microspines to become stuck. This was observed to happen randomly across all trials. However, in every instance where this occurred, the microspine that was caught on 1ML or 2ML was able to wiggle free within a few seconds and continue moving. Even with these obstacles, both 1ML traveling 15cm and 2ML traveling 10.52cm had far greater average displacement than 0ML, roughly 25 times and 18 times more, respectively. This was the only surface where 0ML had the lowest standard deviation, and this is due to the microspine limbs on the other two robots randomly getting stuck when crossing over brick perimeters. Additionally, this surface resulted in the lowest overall movement of 0ML, with an average displacement of 0.59cm. %The displacement was so consistent because the robot had difficulty overcoming the terrain at all.

The granular, compact sand was not entirely level and various pebbles, holes, insects, and small sticks were scattered around the testing area. This surface was difficult to overcome for all three robots as the compact sand was covered in a loose, granular top layer that was easy to become partially submerged in. Due to this, the average displacement is far lower on this surface. The two microspine limbs on 2ML seemed to dig itself deeper, resulting in average displacement of only 0.46cm and the lowest standard deviation. 1ML still outperformed 0ML with over 3 times increased displacement, traveling 2.53cm compared to 0.82cm.


\begin{figure}[!h]
    \centering
    \includegraphics[width=\linewidth]{Figures/micro_dataV3.png}
    \caption{The average displacement data per prototype and surface combination. The repeatability is shown in error bars.}
    \label{fig:data}
\end{figure}


\begin{figure}[!h]
    \centering
    \includegraphics[width=\linewidth]{Figures/del_x_del_yV4.png}
    \caption{Gait analysis. a) Every $\Delta$X and $\Delta$Y position per gait on concrete and brick. b) The average absolute $\Delta$X and $\Delta$Y displacement data per gait on concrete and brick.}
    \label{fig:delta}
\end{figure}

The forest floor surface was completely non-uniform with highly varying terrain. Specifically, the prototypes had to first overcome a large, 4" tall tree root and then move through leaves, acorns, and other tree debris. All SoRos were able to successfully traverse over the cylinder-like tree root due to the conformable nature of the soft limbs, highlighted in \Fig \ref{fig:tree}. However, only the 1ML and 2ML were able to navigate through the thick tree debris after making it over the large tree root; 0ML became stuck at the base of the tree root in each of its three trials. This is exhibited in the average displacement of 13.74cm for 0ML, 23.95cm for 1ML, and 21.32cm for 2ML. The forest floor was critical for testing as it was the most unstructured of the four experimental surfaces and showcases the benefits of the soft limbs paired with the added gripping stability of the microspine array.

For each of the 36 trials, the robots move for 60 gaits, resulting in a total of 2,160 gaits or 720 gaits per robot. Each gait results in relative change in position in the robot coordinate system $\Delta X, \Delta Y$. The locomotion consistency can be analyzed by examining the 180 poses on a given surface per prototype, such as those shown in \Fig \ref{fig:delta}a). The average displacement per gait ($\Delta X,\Delta Y)$  as well as the standard deviation over all gaits per prototype/surface combination is visualized in \Fig \ref{fig:delta}b). We only analyze the uniform/partially uniform surfaces as the gait-to-gait movement is much less consistent due to non-uniformity and randomness for the other two surfaces. On concrete, the average gait displacement per gait of both 1ML $(0.65cm,0.44cm)$ and 2ML $(0.20cm,0.15cm)$ is greater than 0ML $(0.07cm,0.06cm)$. On brick, both 1ML $(0.22cm,0.19cm)$ and 2ML $(0.13cm,0.15cm)$ outperform 0ML $(0.02cm,0.01cm)$. The relative standard deviation (standard deviation/mean) for 1ML and 2ML is lower than that for  0ML, indicating greater grip stability through improved repeatability.

Examples of a single trial of each prototype row on each surface column is visualized in \Fig \ref{fig:surfaces} with additional data in the accompanying video. The starting and end points are green and blue dots, and the path of traversal is a red, gradient line. On all surfaces, 1ML interacts with the environment significantly more than the baseline 0ML, resulting in greater overall movement. On every surface except for compact sand, 2ML outperforms 0ML. On concrete and compact sand, 2ML had the lowest standard deviation and on the forest floor, 1ML had the lowest standard deviation. This indicates the addition of microspine arrays also increasing the consistency and repeatability of planar locomotion with SoRos. 
%Line graphs plot the displacement of each run per SoRo on a given surface, resulting in four total graphs.  Discrepancies of consistency and improved(?) locomotion are shown.




\begin{figure*}[htbp]
    \centering
    \includegraphics[width=0.9\textwidth]{Figures/qual_results2V2.png}
    \caption{Experimental results for 0ML, 1ML, and 2ML on concrete, brick, compact sand, and a forest floor. Rows represent the different surfaces increasing in unstructured nature with the three different prototypes distinguished by columns.}
    \label{fig:surfaces}
\end{figure*}



% \begin{figure*}[h]
%     \centering
%     \begin{subfigure}{0.3\linewidth}
%          \centering
%          \includegraphics[width=\textwidth]{Figures/Sequence 02_crop.png}
%          %\caption{0ML Concrete}
%          \label{fig:base}\vspace{-9pt}
%      \end{subfigure}
%      \begin{subfigure}{0.3\linewidth}
%          \centering
%          \includegraphics[width=\textwidth]{Figures/Sequence 5_crop.png}
%          %\caption{1ML Concrete}
%          \label{fig:one}\vspace{-9pt}
%      \end{subfigure}
%      \begin{subfigure}{0.3\linewidth}
%          \centering
%          \includegraphics[width=\textwidth]{Figures/Sequence 9_crop.png}
%          %\caption{2ML Concrete}
%          \label{fig:two}\vspace{-9pt}
%      \end{subfigure}
%      \begin{subfigure}{0.3\linewidth}
%          \centering
%          \includegraphics[width=\textwidth]{Figures/Sequence 11_crop.png}
%          %\caption{0ML Brick}
%          \label{fig:base}\vspace{-9pt}
%      \end{subfigure}
%      \begin{subfigure}{0.3\linewidth}
%          \centering
%          \includegraphics[width=\textwidth]{Figures/Sequence 20_crop.png}
%          %\caption{1ML Brick}
%          \label{fig:one}\vspace{-9pt}
%      \end{subfigure}
%      \begin{subfigure}{0.3\linewidth}
%          \centering
%          \includegraphics[width=\textwidth]{Figures/Sequence 15_crop.png}
%          %\caption{2ML Brick}
%          \label{fig:two}\vspace{-9pt}
%      \end{subfigure}
%      \begin{subfigure}{0.3\linewidth}
%          \centering
%          \includegraphics[width=\textwidth]{Figures/Sequence 21_crop.png}
%          %\caption{0ML Sand}
%          \label{fig:base}\vspace{-9pt}
%      \end{subfigure}
%      \begin{subfigure}{0.3\linewidth}
%          \centering
%          \includegraphics[width=\textwidth]{Figures/Sequence 26_crop.png}
%          %\caption{1ML Sand}
%          \label{fig:one}\vspace{-9pt}
%      \end{subfigure}
%      \begin{subfigure}{0.3\linewidth}
%          \centering
%          \includegraphics[width=\textwidth]{Figures/Sequence 25_crop.png}
%          %\caption{2ML Sand}
%          \label{fig:two}\vspace{-9pt}
%      \end{subfigure}
%      \begin{subfigure}{0.3\linewidth}
%          \centering
%          \includegraphics[width=\textwidth]{Figures/Sequence 46_crop.png}
%          %\caption{0ML Forest}
%          \label{fig:base}
%      \end{subfigure}
%      \begin{subfigure}{0.3\linewidth}
%          \centering
%          \includegraphics[width=\textwidth]{Figures/Sequence 40b_crop.png}
%          %\caption{1ML Forest}
%          \label{fig:one}
%      \end{subfigure}
%      \begin{subfigure}{0.3\linewidth}
%          \centering
%          \includegraphics[width=\textwidth]{Figures/Sequence 44_crop.png}
%          %\caption{2ML Forest}
%          \label{fig:two}
%      \end{subfigure}
%     \caption{Experimental results for 0ML, 1ML, and 2ML on concrete, brick, compact sand, and a forest floor. Rows represent the different surfaces with the three different prototypes distinguished by columns.}
%     \label{fig:surfaces}
% \end{figure*}

% %\begin{table}
% %\caption{Test}
% \begin{center}
% \begin{tabular}{|c|c|c|c|} 
%  \hline
%  Robot & Terrain & $\Delta$ Displacement & $\Delta$ Rotation \\ %[0.5ex] 
%  \hline\hline
%  0ML & Concrete & \textit{1.32} & \textit{17.98} \\ 
%  \hline
%  1ML & Concrete & \textbf{39.63} & \textbf{131.30} \\
%  \hline
%  2ML & Concrete & 14.13 & 68.70 \\
%  \hline\hline
%  0ML & Brick & \textit{0.59} & \textit{2.32} \\ 
%  \hline
%  1ML & Brick & \textbf{15.00} & \textbf{77.25} \\
%  \hline
%  2ML & Brick & 10.52 & 37.16 \\
%  \hline\hline
%  0ML & Compact Sand & 0.82 & \textit{3.36} \\ 
%  \hline
%  1ML & Compact Sand & \textbf{2.53} & \textbf{49.88} \\
%  \hline
%  2ML & Compact Sand & \textit{0.46} & 4.82 \\
%  \hline\hline
%  0ML & Forest Floor & \textit{13.74} & 69.96 \\ 
%  \hline
%  1ML & Forest Floor & \textbf{23.95} & \textit{34.85} \\
%  \hline
%  2ML & Forest Floor & 21.32 & \textbf{158.77} \\
%  \hline %[1ex] 
% %\caption{Table Results}
% \end{tabular}
% \end{center}
% %\end{table}

% %\begin{center}
% %\begin{tabular}{|c|c|c|c|c|c|c|c|c|c|c|c|c|}
% %\hline
% %\multicolumn{13}{|c|}{Rotation Gait} \\% 
% %\hline
% %\multicolumn{1}{|c|} {Incline} & %
% %\multicolumn{3}{|c|} {Wood} & 
% %\multicolumn{3}{|c|} {White Board} & 
% %\multicolumn{3}{|c|} {Black Mat} & 
% %\multicolumn{3}{|c|} {Carpet} \\ \hline
% %$0^{\circ}$ & & & & \hspace{0.3cm} & \hspace{0.3cm} & \hspace{0.3cm} & \hspace{0.2cm} & %\hspace{0.2cm} & \hspace{0.2cm} & & &\\ \hline
% %$5^{\circ}$ & & & & & & & & & & & &\\ \hline
% %$10^{\circ}$ & & & & & & & & & & & &\\ \hline
% %$15^{\circ}$ & & & & & & & & & & & &\\ \hline
% %$20^{\circ}$ & & & & & & & & & & & &\\ \hline
% %$25^{\circ}$ & & & & & & & & & & & &\\ \hline
% %$30^{\circ}$ & & & & & & & & & & & &\\ \hline
% %\end{tabular}
% %\end{center}
% % \begin{table}
% % \begin{center}
% % \begin{tabular}{|c|c|c|c|c|c|c|c|c|c|c|c|c|}
% % \hline
% % \multicolumn{13}{|c|}{Translation Gait} \\% 
% % \hline
% % \multicolumn{1}{|c|} {Incline} & %
% % \multicolumn{3}{|c|} {Wood} & 
% % \multicolumn{3}{|c|} {White Board} & 
% % \multicolumn{3}{|c|} {Black Mat} & 
% % \multicolumn{3}{|c|} {Carpet}\\\hline
% % $0^{\circ}$ & & & & \hspace{0.3cm} & \hspace{0.3cm} & \hspace{0.3cm} & \hspace{0.2cm} & \hspace{0.2cm} & \hspace{0.2cm} & & &\\\hline
% % $3^{\circ}$ & & & & & & & & & & & &\\\hline
% % \end{tabular}
% % \caption{Average Displacement}
% % \label{table:1}
% % \end{center}
% % \end{table}

% %\pagebreak
% \begin{figure*}[h]
%      \centering
%      \begin{subfigure}{0.32\textwidth}
%          \centering
%          \includegraphics[width=\textwidth]{Figures/rubber_stddev.png}
%          %\caption{Black Mat}
%          \label{fig:black stddev}
%      \end{subfigure}
%      \begin{subfigure}{0.32\textwidth}
%          \centering
%          \includegraphics[width=\textwidth]{Figures/smooth_whiteboard_stddev.png}
%          %\caption{Whiteboard}
%          \label{fig:white stddev}
%      \end{subfigure}
%      \begin{subfigure}{0.32\textwidth}
%          \centering
%          \includegraphics[width=\textwidth]{Figures/porous_wood_stddev.png}
%          %\caption{Wood}
%          \label{fig:wood stddev}
%      \end{subfigure}
%           \begin{subfigure}{0.32\textwidth}
%          \centering
%          \includegraphics[width=\textwidth]{Figures/rubber_disp_rot.png}
%          \caption{Rubber Mat}
%          \label{fig:black disp vs rot}
%      \end{subfigure}
%      \begin{subfigure}{0.32\textwidth}
%          \centering
%          \includegraphics[width=\textwidth]{Figures/whiteboard_disp_rot.png}
%          \caption{Whiteboard}
%          \label{fig:white disp vs rot}
%      \end{subfigure}
%      \begin{subfigure}{0.32\textwidth}
%          \centering
%          \includegraphics[width=\textwidth]{Figures/wood_disp_rot.png}
%          \caption{Wood}
%          \label{fig:wood disp vs rot}
%      \end{subfigure}
%      \begin{subfigure}{0.4\textwidth}
%          \centering
%          \includegraphics[width=\textwidth]{Figures/graph_legend.png}
%      \end{subfigure}
%     \caption{The displacement and rotation results of the translation gait trials ($49.5\sec/\mathrm{trial}$) for the three surfaces and three spine configurations. The increased engagement with the surfaces is observed through coupled increase in translation-rotation quantities.}
%     \label{fig:disp vs rot}
% \end{figure*}
% %%%%%%%%%%%%%%%%%%%%%%%%%%%%%%%%%%%%%%%%%%%%%%%%%%%%%%%%%%%%%%%%%%%%%%%%%%%%%%%%
% The results for the planar locomotion tests on the three experimental surfaces - rubber mat, smooth whiteboard, and porous wood - are plotted as displacement vs. rotation in \Fig \ref{fig:disp vs rot}. For each surface, 20 trials per spine configuration are performed resulting in a total of 180 trials. Each trial comprised of 45 gaits with total duration of $1.1\sec\times45=49.5\sec$ .The reasoning behind the large number of tests is to investigate the consistency of movement of SoRos with and without microspines.

% The baseline SoRos without spine endcaps travel less distance than both spine configurations on all three surfaces. Additionally, there is a large variance in displacement on the smooth surface. The inward microspines increase overall displacement on all three surfaces, and display the least amount of rotation on average on the smooth and porous surfaces. Finally, the SoRos equipped with the directional microspine endcaps have the greatest displacement on the rubber and smooth surfaces.

% On rough surfaces containing asperities, microspines increase overall distance traversed, confirming what is seen in literature. On smooth surfaces where microspines historically have not been utilized, we find that they increase locomotion consistency. By running multiple tests for each set of parameters, we provide a more accurate picture of SoRo locomotion, microspine effectiveness, and the need for repeatability - an under-reported challenge in the field of soft robotics. In all tested scenarios, there is improvement in translation when microspine endcaps are present. For this specific translation gait, there is improved translation performance, but the cumulative rotation has high variance. This can be attributed to the fact that this particular gait is not optimal for all given surfaces. As mentioned earlier, the gait is optimal for a rubber mat with no spines, not for other surfaces and configurations. However, it can be concluded that there is a consistent increase in engagement of the robot with the surfaces by observing the increase in coupled translation-rotation quantities.
\section{Interesting Case Studies}
To gain further insights into the performance of our hybrid model, we conduct an in-depth qualitative analysis.
We found that aggressors employ various sneaky methods to \textit{conceal} slurs and hate speech, often making it challenging to detect and address.
Here are some categories that encompass these tactics:
\begin{itemize}[leftmargin=*]
    \item \textbf{Introducing new hate speech lexicons:} As online platforms implement measures to combat hate speech, aggressors adapt by using alternative terms, neologisms, or coded language to express their hateful ideas without triggering automated filters or detection systems making it difficult for outsiders or automated tools to immediately recognize the underlying hate speech. For example, the word ``shitskins'' (Example 1) and ``salads'' (Example 2) are used as hate words in the following social media posts.

    \vspace{0.1in}
    \begin{graybox}
    \textbf{Example 1: }``Ive seen videos of Muslim shitskins dividing a single person into multiple pieces.''\\
    \textbf{Example 2: }``of course I'm over the limit I'm on a night out you fucking salads''
    \end{graybox}
    \vspace{0.1in}

    \item \textbf{Spelling Errors:} Aggressors intentionally misspell words related to hate speech or use deliberate variations in spelling to bypass content filters. In Examples 3, 4, and 5 we illustrate some of the spelling errors made.

    \vspace{0.1in}
    \begin{graybox}
    \textbf{Example 3: }``Y'all \textbf{niggaz} evil af''\\
    \textbf{Example 4: }``If you'll see me holding up my middle finger to the world. \textbf{Fck} ur ribbons and ur pearls.'' \\
    \textbf{Example 5: }``This shit got me \textbf{fuckin} CRYINGG!! Cuz the \textbf{lil nigga} aint even want this stupid cut just look \@ his face''
    \end{graybox}
    \vspace{0.1in}
    
    \item \textbf{Adding Punctuation:} Another tactic employed by aggressors is the insertion of special characters or punctuation marks within offensive words or slurs to obscure or obfuscate the offensive language. For example, adding an apostrophe like ``nas.ty'' (Example 6) or an underscore like ``x\_x'' (Example 7).

    \vspace{0.1in}
    \begin{graybox}
    \textbf{Example 6: }''I just like \textbf{nas.ty} shit men`` \\
    \textbf{Example 7: }''When u pounding the \textbf{x\_x} like u don't wna``
    \end{graybox}
    \vspace{0.1in}

    \item \textbf{Implied Hate:} Aggressors often resort to implied hate, where they use veiled language (Example 8), innuendos, sarcasm (Example 9), or ambiguous statements (Example 10) to convey discriminatory or hateful ideas indirectly.
    \vspace{0.1in}
    \begin{graybox}
    \textbf{Example 8: }``To bad u couldn't box the hell out I'd be even prouder'' \\
    \textbf{Example 9: }``You are a chicken nugget and soy milk'' \\
    \textbf{Example 10: }``Latina backwards spells crazy as hell in 2 languages''
    \end{graybox}

\end{itemize}

Our findings reveal that our model exhibits a higher proficiency in identifying instances of hate speech when substitute lexicons are employed especially to bypass already in place moderation systems.

\subsection{Comparing Our Hybrid Approach with State-of-The-Art Moderate Hate Speech API}

Moderate Hate Speech API~\cite{moderatehatespeech} is a Google Cloud service that helps identify and moderate hate speech.
It can be used to moderate content in a variety of applications, including social media platforms, forums, and news websites.
For each detected hate speech token, the API returns a confidence score which indicates how likely it is that the token is hate speech. However, it is important to note that the API is not perfect. It sometimes misidentifies content as hate speech, and it can also sometimes fail to identify hate speech as reported on its website~\cite{moderatehatespeech}. We find that our model detects hate and toxicity towards vulnerable populations especially women and the black community.

We use this API to compare our hybrid model.
We use the 76,378 unlabeled posts for this purpose.
We find that our model detects 663 posts as hate speech out of 76,378 posts whereas Moderate Hate Speech API detects 678 posts as hate speech. Our model detects 65 different posts than Moderate Hate Speech API, and upon manual analysis, we find that most of the posts that our model detected contained new toxic lexicons for example ``sigma'' (Example 11), ``karen,'' ``thot''(Example 12),  etc.

\vspace{0.1in}
\begin{graybox}
\textbf{Example 11: }``typical sigma behavior''\\
\textbf{Example 12: }``i am your local thot''
\end{graybox}
\vspace{0.1in}

There were other examples where the API failed where harsher emotions or words were used for example in Example 13 ``liberal Stalinists'' is used negatively:

\vspace{0.1in}
\begin{graybox}
\textbf{Example 13: }``So, for the first time ever since 2017, America is a communist nation again. liberal Stalinists!!''
\end{graybox}
\vspace{0.1in}

There were other cases where sexual harassment towards women was missed by the API for example (Examples 14 and 15):

\vspace{0.1in}
\begin{graybox}
\textbf{Example 14: }``I just wanna be a good bun, having someone clip a leash to my collar and take me for a walk, letting anyone who asks fuck and breed me, then getting headpats and scritches after''\\
\textbf{Example 15: }``Anyone else wanna help me breed her..''
\end{graybox}
\vspace{0.1in}

On the other hand, our model performs poorly when the hate speech lexicons were not part of the initial diagnosis, for example in the following post (Example 16) the lexicons ``xenophobes'' and ``halfwits'' were not part of the toxic lexicon list and hence this post is not flagged by our model but was detected by Moderate Hate Speech API.
The Moderate Hate Speech API detects 80 different posts than our model. 

\vspace{0.1in}
\begin{graybox}
\textbf{Example 16: }``RT @username: @username My business is in services. Xenophobes and halfwits like yourself destroyed the EU side of my business.''
\end{graybox}
\vspace{0.1in}

However, we also find that Moderate Hate Speech is biased towards black people (This has also been confirmed in the documentation of this API~\cite{moderatehatespeech}), for example, the following posts (Examples 17 and 18) from our dataset are labeled as hate speech by this model, however upon manual analysis, we can see that they are clearly not hate speech.

\vspace{0.1in}
\begin{graybox}
\textbf{Example 17: }``@username: Did you know a disabled Black woman invented the walker, toilet paper holder, and sanitary belt?''\\
\textbf{Example 18: }``@username: the older black generation be saying some questionable things.''
\end{graybox}
\vspace{0.1in}

We discover that our hybrid model goes beyond existing approaches by addressing the dynamic nature of language, adapting to new vocabulary, and evolving linguistic patterns. It also helps identify toxicity towards vulnerable populations that were not mentioned in the original lexicon dataset. 
\section{Discussion}
In this section, we discuss the key implications of our findings based on our two overarching research questions. Overall, our findings open up interesting opportunities for future research and implications for the industry as a whole.

\subsection{Resilience Of Adaptive Hate Speech Detection Against Poisoning Attacks - RQ1}
Lexicon-based approaches to hate speech detection systems are prone to poisoning attacks.
In a poisoning attack, an adversary intentionally uses safe words in place of toxic words that can cause the model to produce incorrect or biased outputs.
For example, in 2016, 4Chan's /pol/ launched a deliberate attack against Google's Perspective API via the so-called ``Operation Google''~\cite{hine2017kek,operationgoogle}. This attack was designed to poison models by replacing slurs with the names of various tech companies.
For example, instead of saying a slur for a black person, you would say ``Google,'' or instead of a slur for a Jew, you would say ``Skype'' etc.
Poisoning attacks can be challenging to mitigate because they exploit vulnerabilities in the training process of machine learning models.
However, our proposed system can be used to expose it specifically because our goal is to discover how toxic behavior and aggression attacks change over time.
The basic core of our approach is to identify words that are used in a \textit{similar fashion} as known toxic and toxic ones.
We adopt a similarity-based approach thus for each word appearing in our dataset we calculate its vector embedding, extracted from the models built as part of the previous step. We compare this vector with the vector embeddings for all the words in our seed dataset. If the vector for a word has a high similarity (e.g., cosine similarity) with a known toxic word, it is very likely that this word is itself toxic – this is because the two words are used in similar contexts on social media. The output of this phase is a set of words that are likely to be toxic, or used in a toxic way.

\subsection{Hybrid Approach to Hate Speech Detection - RQ2}
By combining the strengths of lexicon-based detection as well as BERT methodologies into a hybrid model we can effectively identify and analyze hate speech in various domains with improved accuracy and contextual understanding. The lexicon-based analysis component leverages pre-defined word lists and sentiment analysis techniques to identify toxic words and sentiments associated with them.
This approach provides a good foundation for detecting explicit risk indicators and capturing straightforward and easily identifiable risk factors.
It allows for quick identification of keywords and phrases commonly associated with risk, enabling efficient detection in real-time scenarios.
On the other hand, the BERT approach, which utilizes a deep learning neural network model, brings contextual understanding and semantic analysis to the hybrid system. BERT enables the model to comprehend the context and nuances of language, capturing the subtleties and complexities of risk factors that may not be explicitly expressed.
This contextual understanding helps the hybrid model to identify implicit risks, detect sarcasm, and recognize risks that might be disguised through various linguistic techniques.
The combination of these two approaches creates a comprehensive risk detection system that combines the advantages of both methods. Secondly, our model also detects implicit hate found in most text online. Unlike explicit hate speech, which uses overtly offensive words or phrases, implied hate speech is more subtle and can be embedded within seemingly innocuous language. Our hybrid model mitigates this limitation by leveraging BERT's contextual understanding.

\subsection{Limitations and Future Work}
\textcolor{black}{In our research, we propose an adaptive methodology to detect toxic language through the utilization of word embeddings. However, it is important to acknowledge that our hybrid approach, despite its numerous strengths, does possess certain limitations.
One notable limitation lies within the lexicon-based analysis employed in our methodology itself. However, our approach reduces this dependency by employing adaptive techniques, allowing for the detection of new toxic words with minimal manual input. This significantly enhances scalability compared to traditional lexicon-based methods. Future work could explore removing older lexicons, and real-time language monitoring to fully automate lexicon updates and improve adaptability to evolving language trends.
Our evaluation also primarily focuses on English-language content, which is a limitation given the global nature of online hate speech. While this allowed us to deeply analyze our approach within a single language, adapting the method to multilingual contexts is crucial for real-world applicability. Hate speech varies significantly across languages and cultures, both in content and contextual nuances. Future work will explore the use of multilingual embeddings (e.g., mBERT, XLM-R) and cross-lingual transfer learning to adapt the approach to other languages. Additionally, we aim to incorporate culturally diverse datasets and expert input to address cross-cultural variations in hate speech detection. Furthermore, it is worth mentioning that recent limitations imposed on using Twitter's APIs have impacted the availability and accessibility of data for research purposes. These limitations may pose challenges in acquiring the necessary data for training and evaluating our model, however, our approach can be mapped to other text-based social media applications especially Threads which is a Meta-owned platform similar in design to Twitter.}




\section{Conclusion}
In summary, our work takes an adaptive approach to advance hate speech detection approaches on social media.
First, we present our adaptive method to update hate speech lexicons.
We test our approach on existing lexicon-based machine learning models and show that the updated lexicons are better at detecting hate speech.
Then we introduce our hybrid approach that combines the powers of lexicon-based hate speech detection with that of BERT-based models.

\bibliographystyle{ACM-Reference-Format}
\documentclass{MITstyle}

%\usepackage[table]{xcolor}
\usepackage{chngcntr}
\usepackage{hyperref}
\usepackage{microtype}

\title{A Lightweight and Extensible Cell Segmentation and Classification Model for Whole Slide Images}

\author{Nikita Shvetsov~$^{1, }$\footnote{Correspondence e-mail: nikita.shvetsov@uit.no}, Thomas K. Kilvaer~$^{2, 3}$, Masoud Tafavvoghi~$^{4}$, Anders Sildnes~$^{1}$, \\ Kajsa Møllersen~$^{4}$, Lill-Tove Rasmussen Busund~$^{5, 6}$, Lars Ailo Bongo~$^{1}$ \\
%
\vspace{1em} % Space between authors and afilliations
%
\normalfont{\small $^{1}$Department of Computer Science, UiT The Arctic University of Norway}\\
\normalfont{\small $^{2}$Department of Oncology, University Hospital of North Norway}\\
\normalfont{\small $^{3}$Department of Clinical Medicine, UiT The Arctic University of Norway}\\
\normalfont{\small $^{4}$Department of Community Medicine, UiT The Arctic University of Norway}\\
\normalfont{\small $^{5}$Department of Medical Biology, UiT The Arctic University of Norway} \\
\normalfont{\small $^{6}$Department of Clinical Pathology, University Hospital of North Norway} %\vspace{2em}
}

\begin{document}
\maketitle

\section*{Abstract}

% \begin{abstract}
% Developing clinically useful cell-level analysis tools in digital pathology remains challenging due to limitations in dataset granularity, inconsistent annotations, computational demands of advanced models, and difficulties in integrating new technologies into clinical workflows. To address these challenges, we propose a multi-faceted solution that enhances data quality, model performance, and usability to create a lightweight and extensible cell segmentation and classification model.

% First, we update data labels by employing a cross-relabeling process that refines the labels of two existing datasets, PanNuke and MoNuSAC, to create a new unified dataset with enhanced granularity, encompassing seven distinct cell types. Second, we leverage the H-Optimus foundation model as a fixed encoder to improve feature representation for simultaneous cell segmentation and classification tasks. Third, to address the computational demands of foundation models, we employ knowledge distillation to reduce model size and complexity while maintaining comparable performance. Finally, to facilitate integration into clinical workflows, we integrate the distilled model into the QuPath software, a widely used open-source platform in digital pathology.

% Our results demonstrate improvements in cell segmentation and classification performance using the H‑Optimus-based model compared to a CNN-based model. Specifically, the average $R^2$ improved from 0.575 to 0.871, and the average $PQ$ score improved from 0.450 to 0.492, indicating better alignment with actual cell counts and enhanced segmentation and classification quality. Furthermore, the distilled student model maintains performance comparable to the larger foundation model while reducing the parameter count by a factor of 48.
% Overall, by reducing computational complexity and integrating it into existing workflows, the proposed approach may significantly impact diagnostic processes, reduce the workload of pathologists, and contribute to improved patient outcomes. Though our approach shows potential enhancements in efficiency and usability of cell segmentation and classification models in digital pathology, extensive validation is needed to deploy these models in clinical practice.
% \end{abstract}

%%% shortened abstract
\begin{abstract}
Developing clinically useful cell-level analysis tools in digital pathology remains challenging due to limitations in dataset granularity, inconsistent annotations, high computational demands, and difficulties integrating new technologies into workflows. To address these issues, we propose a solution that enhances data quality, model performance, and usability by creating a lightweight, extensible cell segmentation and classification model. 

First, we update data labels through cross-relabeling to refine annotations of PanNuke and MoNuSAC, producing a unified dataset with seven distinct cell types. Second, we leverage the H-Optimus foundation model as a fixed encoder to improve feature representation for simultaneous segmentation and classification tasks. Third, to address foundation models' computational demands, we distill knowledge to reduce model size and complexity while maintaining comparable performance. Finally, we integrate the distilled model into QuPath, a widely used open-source digital pathology platform. 

Results demonstrate improved segmentation and classification performance using the H-Optimus-based model compared to a CNN-based model. Specifically, average $R^2$ improved from 0.575 to 0.871, and average $PQ$ score improved from 0.450 to 0.492, indicating better alignment with actual cell counts and enhanced segmentation quality. The distilled model maintains comparable performance while reducing parameter count by a factor of 48. By reducing computational complexity and integrating into workflows, this approach may significantly impact diagnostics, reduce pathologist workload, and improve outcomes. Although the method shows promise, extensive validation is necessary prior to clinical deployment.
\end{abstract}
\clearpage

\section{Introduction}
In digital pathology, accurate segmentation and classification of cells are crucial for many diagnostic, prognostic, and predictive analyses \cite{Jaber_Beziaeva_etal._2019,Lin_Pan_etal._2022,Park_Ock_etal._2022,Shen_Choi_etal._2024}. Nowadays, developments in computational pathology offer multiple solutions \cite{H._Qu_P._Wu_etal._2020,Javed_Mahmood_etal._2020} to utilize cell-level datasets to train machine learning models that solve these problems. The quality and specificity of training datasets are critical for robust and accurate models. Adhering to the principle of "garbage in, garbage out", it is essential to ensure that these datasets are extensively and accurately labeled with distinct classes that reflect the diverse biological characteristics of different cell types. Unfortunately, the number of open-source datasets comprising such high-quality annotations is limited. Existing cell segmentation datasets \cite{Gamper_Koohbanani_etal._2019,Graham_Vu_etal._2019,Verma_Kumar_etal._2021} may offer extensive annotations for certain cell types while providing more general labels for others. For example, in PanNuke, which is one of the largest open-source datasets comprising labeled cells, various types of morphologically and functionally different inflammatory cells like macrophages and lymphocytes are clustered in a broad "inflammatory" class. Consequently, these classes are frequently omitted from analyses or aggregated into broader meta-classes \cite{Gamper_Koohbanani_etal._2020} and likely interfere with other cell classes included in the dataset. This and similar inconsistencies in annotation granularity limit the ability of machine learning models to learn the comprehensive and nuanced features necessary for accurate cell segmentation and classification. To address these challenges, methods for refining and standardizing dataset annotations are essential to enhance the quality of training data.

A complementary approach to mitigate the absence of high-quality training data is the use of foundation models. Foundation models as encoders are defined as large-scale, versatile networks pre-trained on vast, diverse datasets using self-supervised learning, contrasting with convolutional neural network (CNN) pre-trained encoders that rely on supervised learning with labeled data. In practice, foundation models leverage enormous amounts of weakly or unlabeled data from millions of whole slide images (WSIs) and employ self-attention mechanisms to capture long-range dependencies and global context \cite{Chen_Ding_etal._2024,Saillard_Jenatton_etal._2024,Vorontsov_Bozkurt_etal._2024,Xu_Usuyama_etal._2024}. As a consequence, foundation models are able to produce transferable feature representations across different cell types and tissue environments. The feature representations can be leveraged by decoder networks to produce segmentation masks and pixel-level classifications. Because foundation models have comprehensive feature representations, they can be effectively fine-tuned using much smaller amounts of cell-level data compared to the large datasets needed to train models from scratch. Furthermore, foundation models incorporate adversarial training elements or contrastive learning \cite{Chen_Ding_etal._2024,Xu_Usuyama_etal._2024}, enhancing their resilience and adaptability by exposing them to challenging and varied scenarios during training. This may result in more generalizable models, often making them well-suited for diverse and complex tasks in digital pathology.

Despite the inherent advantages of foundation models, their deployment for practical use faces its own obstacles. In particular, they require substantial computational power, financial investments and rigorous testing to ensure reliability and efficacy for a given task \cite{Akkus_Dangott_etal._2022,Dragomir_Cocuz_etal._2022,Go_2022,Jafri_Farooqui_etal._2024}. Moreover, while foundation models enhance feature representation and performance, they depend on the quality of available annotations for decoder fine-tuning and, like any other model, cannot resolve existing inconsistencies or ambiguities in data labels. Therefore, there remains a critical need for solutions that address both data quality and practical deployment considerations.
Further, integrating new technologies into existing clinical workflows often encounters resistance, as it necessitates adjustments to established diagnostic processes. So, there is a need to develop solutions that could be integrated into current practices, minimizing the burden on medical professionals to adopt new tools \cite{King_Williams_etal._2023}.

Existing solutions \cite{Goldsborough_Philps_etal._2024,Hörst_Rempe_etal._2024}, while addressing some aspects of these challenges, fall short in providing a comprehensive approach. To address the data quality and clinical deployment issues, we propose a multi-faceted solution that encompasses data refinement, model optimization, and integration with existing pathology tools (\hyperref[fig:fig1]{Figure 1}). The outcome is a lightweight cell segmentation and classification model that can be integrated into digital pathology workflows for practical clinical use.

\begin{figure}[h!]
    \centering
    \includegraphics[width=\textwidth, height=0.82\textheight, keepaspectratio]{images/Figure_1.pdf}
    \caption{Overview of the proposed solution, including 1) Data refinement using cross-relabeling, 2) Teacher model development and fine tuning, 3) Student model optimization with knowledge distillation and 4) Student model and QuPath integration}
    \label{fig:fig1}
\end{figure}
\clearpage

Our approach begins with preparing the data for the fine-tuning and training of the machine learning models. We create a refined dataset, acquired via cross-relabeling two cell-level datasets, enhancing annotation specificity and consistency of the labeled data. Subsequently, we create a cell segmentation and classification model based on the foundation model. We leverage the foundation model as a fixed encoder and fine-tune a decoder using the refined dataset to improve generalization across diverse tissue- and cell types.
To ensure that the model remains lightweight and deployable in a possibly resource-constrained environment, we employ knowledge distillation to approximate the functionality of the foundation model. Finally, to facilitate the practical application of our model in digital pathology workflows, we integrate it with the QuPath \cite{Bankhead_Loughrey_etal._2017} application. Each methodological component contributes to the overarching goal of enhancing model performance, generalizability, and usability in clinical settings.

The primary contributions of this paper are:
\begin{enumerate}
    \item \textit{Data labels refinement through cross-relabeling:}
    
    We propose a new method for refining labels of cell-level datasets through cross-relabeling. This method employs classification models to re-label broad and ambiguous instances, resulting in a more diverse dataset. Our evaluation demonstrates that these classification models achieve high accuracy on test subsets, indicating the reliability of the method for label refinement.

    \item \textit{Enhanced model performance via foundation models:}
    
    We employ a foundation model as a feature extractor for the cell segmentation and classification task. In comparison with training a CNN model from scratch, the foundation model backbone only needs fine-tuning, which significantly reduces training time, computational resources and data requirements. We show that using a foundation model encoder leads to better performance in cell segmentation and classification networks than using a CNN-based encoder. This improvement may enable the model to generalize more effectively across various tissue types and imaging methods.
    
    \item \textit{Model optimization through knowledge distillation:}
    
    We show that a smaller student model trained using knowledge distillation on the refined dataset obtained via our cross-relabeling approach from a foundation model achieves comparable performance in cell segmentation and quantification tasks. As a result, this model is more suitable for deployment in environments without high-performance computing resources.
    
    \item \textit{Integration with QuPath:}
    
    We integrate the distilled cell segmentation and classification model into QuPath, a widely used open-source digital pathology platform, to accelerate clinical adaptation by enabling pathologists to more easily incorporate advanced computational tools into their existing workflows.
\end{enumerate}

Through these methodological steps, we aim to bridge the gap between advanced machine learning techniques and practical clinical applications, making accurate and efficient digital pathology accessible in a broader range of healthcare settings.

\section{Refining Existing Datasets Using Cross-Relabeling}
To address the limitations of sparse and ambiguous labeling of cell-level datasets, we propose a generalizable cross-relabeling strategy that can be applied to any dataset containing broadly categorized or imprecisely labeled cell types. This approach involves training and subsequently leveraging classification models to refine broad categories into more specific or biologically relevant classes.
When applied to cell-level data, the methodology includes extracting individual cell images from the dataset patches, preprocessing these images to standardize the size and accommodate partial cells, and then training deep learning classifiers capable of distinguishing between the finer cell subtypes within the coarser categories. 
To illustrate our approach, we focus on the PanNuke \cite{Gamper_Koohbanani_etal._2020, Gamper_Koohbanani_etal._2019} and MoNuSAC \cite{Verma_Kumar_etal._2021} datasets that we have used to train models for cell quantification in our previous works \cite{Shvetsov_Grønnesby_etal._2022,Shvetsov_Sildnes_etal._2024}. We find that for better cell differentiation we have to introduce more granular labels. PanNuke includes a broad classification of "inflammatory" cells, encompassing lymphocytes, macrophages, and neutrophils. Each cell type differs significantly in structure, function, and clinical relevance. Conversely, MoNuSAC uses the label "epithelial" for a class that comprises both benign epithelial cells and malignant neoplastic cells. This practice makes it challenging to differentiate between benign and malignant epithelial cells in the dataset, which is a critical distinction when identifying tumor areas within tissue samples. To address these issues, we implement a cross-relabeling strategy as shown in \hyperref[fig:fig2]{Figure 2}. The key components are two classification models: one is trained on singular cell images from PanNuke data to classify the epithelial meta-class into epithelial and neoplastic classes. The other is trained on MoNuSAC to refine the inflammatory class into lymphocytes, neutrophils, and macrophages.

\begin{figure}[h!]
    \centering
    \includegraphics[width=\textwidth]{images/Figure_2.pdf}
    \caption{Refined dataset generation via cross relabeling}
    \label{fig:fig2}
\end{figure}

The refining approach consists of three consecutive steps. The first is the preprocessing step, in which we extract individual cells from both datasets (\hyperref[fig:fig3]{Figure 3}). The specifics of PanNuke and MoNuSAC patch preparation before cell preprocessing are provided in \hyperref[chap:S1]{Appendix S1}.

\begin{figure}[h!]
    \centering
    \includegraphics[width=\textwidth]{images/Figure_3.pdf}
    \caption{Cell instances preprocessing including (1) cell map extraction, (2) bounding box delineation, (3) adjusting cell boxes and (4) cropping and resizing of cell images}
    \label{fig:fig3}
\end{figure}

During preprocessing, we extract cell type maps from the ground truth label mask and calculate bounding boxes around each cell instance. To accommodate partial cells at patch borders, a common issue in cropped patch images, we employ mirror padding and extend the field of view of the cell label by 15 pixels to capture adjacent cells. We then crop and resize the identified regions to $64 \times 64$ pixels using bicubic interpolation.

The preprocessed PanNuke dataset comprises 68,031 neoplastic and 23,207 epithelial cell images, while MoNuSAC comprises  33,104 lymphocytes, 1,252 neutrophils, and 1,695 macrophages, which we subsequently use in training cell classification models and classifying the cell image data \hyperref[fig:S2]{Appendix Figure S2 (1)}. 

The next step is to train two distinct ResNet50-based classifiers tailored to address the specific labeling challenges inherent in each dataset. We use ResNet50 for classification models due to its proven effectiveness for image classification tasks in histopathology \cite{pan2022reviewmachinelearningapproaches}, and its compatibility with small images. For the PanNuke dataset, we design the classifier, trained on MoNuSAC data, to disaggregate the heterogeneous "inflammatory" cell category into distinct subtypes: lymphocytes, macrophages, and neutrophils. Similarly, for the MoNuSAC dataset, the classifier is trained on PanNuke data and distinguishes between benign and malignant epithelial cells within the overarching "epithelial" label. By applying these targeted classifiers to their respective datasets, we assign more specific labels to individual cell instances, thus enabling us to create a unified dataset.
To ensure a balanced representation of classes, we train both models on datasets that had been equalized to match the size of the least represented class. Thus, we obtain datasets comprising 23,207 samples per class for PanNuke and 1,252 samples per class for MoNuSAC data. Next, we partition both of them into training (70\%), validation (20\%), and testing (10\%) subsets. To mitigate the risk of overfitting, we use a single dropout layer with a rate of p=0.5 in both models and data augmentation using randomized color perturbations, rotation, and horizontal and vertical flipping. We employ AdamW optimizer and the cross-entropy loss function for the training criterion.

To evaluate the two trained models, we measure the classification accuracy on the respective test subsets. The accuracies on the test subset for both classifiers are presented in \hyperref[tab:1]{Table 1}. The PanNuke model achieves an average accuracy of 93.57\%, with higher accuracy for neoplastic cells (96.06\%) compared to epithelial cells (86.26\%). The confusion matrix in Figure A3.1 shows that the model predominantly distinguishes accurately between epithelial and neoplastic tissues, with a substantial number of correct classifications and relatively few misclassifications. The MoNuSAC model demonstrates an average accuracy of 98.92\%, excelling in classifying lymphocytes (99.67\%) and macrophages (94.12\%), with lower performance for neutrophils (85.71\%). The confusion matrix in Figure A3.2 shows that the model identifies lymphocytes and performs reasonably well with macrophages and neutrophils.

\begin{table}[h!]
\renewcommand{\arraystretch}{1.5}
  \centering
  \caption{Cell classification results for PanNuke and MoNuSAC trained models (CI 95\%).}
  \label{tab:1}
  \begin{tabular}{|l|c|c|}
   \hline
   %\rowcolor{gray!30}
    Accuracy               & PanNuke model              & MoNuSAC model              \\
    \hline
    Average      & 0.936 (0.931--0.941)         & 0.989 (0.986--0.993)        \\
    \hline
    Neoplastic   & 0.961 (0.956--0.965)         & -                          \\
    \hline
    Epithelial   & 0.863 (0.849--0.877)         & -                          \\
    \hline
    Lymphocytes  & -                          & 0.997 (0.995--0.999)        \\
    \hline
    Neutrophils  & -                          & 0.857 (0.796--0.918)        \\
    \hline
    Macrophages  & -                          & 0.941 (0.906--0.976)        \\
    \hline
  \end{tabular}
\end{table}

Finally, during the last step, we use the model trained on PanNuke data for epithelial cells in MoNuSAC and the model trained on MoNuSAC for the inflammatory cells class in PanNuke. Specifically, we use classifier models to relabel epithelial cells in MoNuSAC and inflammatory cells in PanNuke data. Then we combine cells with refined labels and the rest of the cells in both datasets to create a refined dataset (\hyperref[fig:S2]{Appendix Figure S2 (2)}). The process of relabeling cells and visualizing them on a patch is shown in \hyperref[fig:fig4]{Figure 4}. The cell counts in the refined dataset are provided in \hyperref[tab:S4]{Appendix Table S4}.

\begin{figure}[h!]
    \centering
    \includegraphics[width=\textwidth, height=0.42\textheight, keepaspectratio]{images/Figure_4.pdf}
    \caption{Cell relabeling procedure for epithelial and inflammatory cell classes}
    \label{fig:fig4}
\end{figure}

%\hfill

Relabeling and combining datasets have been explored in a prior study \cite{Parulekar_Kanwat_etal._2023}, where consecutive fine-tuning on multiple datasets was employed to account for hierarchical class label structures. While the method presented in \cite{Parulekar_Kanwat_etal._2023} is intuitive, it often lacks consistency and requires multiple fine-tuning runs, which can be cumbersome and time-consuming. 
In contrast, cross-relabeling simplifies this process by using specialized classification models tailored to each dataset's specific labeling challenges. This approach provides better transparency and produces a unified dataset encompassing seven distinct cell types across multiple tissue samples, enhancing data diversity for further model training or fine-tuning.

Despite these improvements, cross-relabeling does not entirely resolve issues related to poor labeling quality or the amount of labeled data. Specifically, our results show lower accuracies persist for underrepresented classes, such as macrophages, which may stem from a limited sample availability and intrinsic challenges in distinguishing these cells based solely on H\&E staining. Furthermore, while our method enhances label specificity, it relies on the initial quality of the broad labels; thus, any fundamental inaccuracies in the original annotations can propagate through the relabeling process. Addressing the overall problem of limited data labels may require integrating additional data sources or utilizing complementary immunohistochemical staining methods.
Although the reported performance metrics are obtained from evaluations on the native test sets of each dataset, it is important to note that the primary application of these classifiers is to perform cross-relabeling, where a model trained on one dataset (e.g., PanNuke) is applied to another (e.g., MoNuSAC) and vice versa. We acknowledge that a more systematic evaluation of cross-dataset generalization is needed and could be performed in future work.

Overall, the refined dataset produced by our approach can enhance the supervised training or fine-tuning of cell segmentation and classification models, especially those that utilize pre-trained foundation models to improve feature extraction robustness. In addition, these models can detect nuanced classes that enable researchers to conduct more detailed analyses of biological processes in computational pathology.

\section{Foundation models for robust cell segmentation and classification}

Accurate cell segmentation and classification in digital pathology are hindered by limited labeled data and the fact that conventional CNNs are unable to capture global contextual information due to their local receptive field constraints \cite{Gheflati_Rivaz_2022,Yang_Marcus_etal.}. Traditional approaches in cell quantification have predominantly relied on CNN encoders, such as ResNet50, given their proven effectiveness in semantic segmentation tasks \cite{Deshmane_2023,Graham_Vu_etal._2019,Mukasheva_Koishiyeva_etal._2024,Stringer_Wang_etal._2021}. However, approaches that include fine-tuning of pretrained CNNs, data augmentation, and stain normalization to partially increase data variability and address staining differences often fail to achieve the necessary generalization and robustness across diverse tissue types and staining conditions \cite{G._Wang_W._Li_etal._2018,Gao_Bagci_etal._2018,Karim_El_Khoury_Martin_Fockedey_etal._2021}.

To overcome these challenges, we leverage an encoder-decoder network that uses a foundation model as the encoder and a CNN upsampling decoder (\hyperref[fig:fig5]{Figure 5}) for simultaneous cell segmentation and classification in 2D patches extracted from WSIs. Foundation models with transformer-based architectures are viable alternatives to CNN-based encoders \cite{Shamshad_Khan_etal._2023,Sourget_2023}. They enable the creation of more advanced architectures that can decode or transform learned features more effectively \cite{Chen_Duan_etal._2023,Cheng_Misra_etal._2022,Xie_Wang_etal._2021}.

\begin{figure}[h!]
    \centering
    \includegraphics[width=\textwidth]{images/Figure_5.pdf}
    \caption{UNETR-like model with foundational model as backbone}
    \label{fig:fig5}
\end{figure}

By utilizing a transformer-based encoder, we incorporate global contextual information into the feature extraction process, which is a key advantage of such architectures \cite{Chen_Lu_etal._2021}. This foundation model integration facilitates accurate pixel-wise segmentation and classification without the need for extensive encoder training, thereby potentially improving generalization across varied cellular structures and tissue types.
In our implementation, we employ a modified UNETR \cite{Hatamizadeh_Tang_etal._2021} architecture that combines a vision transformer (ViT) \cite{Dosovitskiy_Beyer_etal._2021} encoder with a CNN-based decoder. The encoder utilizes the pretrained H-Optimus foundation model, which contains 1.1 billion parameters and is trained on over 500,000 H\&E stained WSIs \cite{Saillard_Jenatton_etal._2024}. We extract outputs from four evenly spaced transformer blocks $Z_i$, where $i \in [1, 14, 26, 38]$, to serve as residual connections for the CNN decoder. We select these blocks based on our observation that features from non-adjacent levels of the encoder lead to better overall performance on the test subset.

The CNN decoder upsamples the feature representations, acquired from the transformer blocks, to generate an intermediate vector that is handled by two task-specific layers that generate cell segmentation and classification masks. The first task-specific layer is the ‘Cellpose head’,  which is used to delineate cell instances. The layer generates horizontal and vertical gradient maps to form vector fields that are refined through gradient tracking in a post-processing step using the Cellpose algorithm \cite{Stringer_Wang_etal._2021}, known for its efficacy in cell segmentation tasks and generalizability across multiple domains \cite{Pachitariu_Stringer_2022,Stringer_Pachitariu_2024}. The second task-specific layer is the "Cell type head", which assigns labels to individual pixels. In the post-processing step, we determine the output classification label of each segmented cell instance by majority voting over the labeled pixels that comprise the cell in the segmentation map.

To evaluate model performance and measure the impact of adding a foundation model as backbone, we compare it to a ResNet50-based model. ResNet50 is a widely used solution for encoders in segmentation architectures in the medical domain \cite{Deshmane_2023,Graham_Vu_etal._2019,Mukasheva_Koishiyeva_etal._2024,Stringer_Wang_etal._2021}. For the H-Optimus-based model, we utilize frozen weights for the encoder and only fine-tune the decoder to take advantage of the extensive pre-training of the foundation model. For the ResNet50-based model we start with ImageNet \cite{Deng_Dong_etal.} weights and train both encoder and decoder parts. Hyperparameters for the training step are set to be identical, where possible, for comparable evaluation. 
For this evaluation, we deliberately use the PanNuke dataset to provide a standardized and controlled comparison between the H‑Optimus and ResNet50-based models (\hyperref[fig:S2]{Appendix Figure S2 (3)}). Specifically, we use two of the default PanNuke dataset splits (66\%) for training and validation, and reserve the third split (33\%) for testing.

To address the challenge of cell class imbalance in the PanNuke dataset, which is a common characteristic in most cell-level H\&E patch datasets, both models’ training processes employ a weighted loss function comprising cross-entropy and focal loss \cite{Lin_Goyal_etal._2018}. The focal loss component is adjusted with coefficients derived from each cell class' instance frequency, emphasizing learning from underrepresented classes and enhancing the model's sensitivity to rare but significant cellular patterns. The cross-entropy loss is augmented with spectral decoupling regularization \cite{Pezeshki_Kaba_etal._2021,Pohjonen_Stürenberg_etal._2022} and spatially varying label smoothing \cite{Islam_Glocker_2021}, which potentially stabilizes training and improves generalization in case of complex tissue morphologies. For optimization, we employ the \textit{AdamW} \cite{Loshchilov_Hutter_2019} to counter unbalanced class scenarios, with cosine annealing learning rate scheduler.

We utilize the scikit-learn library \cite{Van_der_Walt_Schönberger_etal._2014} and HoVer-Net \cite{Graham_Vu_etal._2019} implementations of $R^2$ (the coefficient of determination) and $PQ$ (panoptic quality) to evaluate our experiments. Complete mathematical formulations and detailed explanations of these metrics are provided in \hyperref[chap:S5]{Appendix S5}. To compute confidence intervals, we use nonparametric bootstrapping, where after calculating the metric on the full sample, we generated 1000 bootstrap replicates by resampling with replacement and then determined the 95\% confidence intervals as the 2.5th and 97.5th percentiles of the resulting empirical distribution.

%\hfill

The model comparisons are summarized in \hyperref[tab:2]{Table 2}. The H‑Optimus-based model achieves higher $R^2$ across all cell classes compared to the ResNet50-based model, which means that its predictions are more closely aligned with the PanNuke cell counts, indicating a stronger correlation with the observed data. Notably, the improvement of $R^2_{dead}$ may be an indicator of better global contextual representations provided by the foundation model backbone. In terms of segmentation and classification quality combined, measured by the PQ score, the H‑Optimus-based model demonstrates notable improvements across most cell classes. Overall, the average $R^2$ improved from 0.575 to 0.871, while the average $PQ$ score improved from 0.450 to 0.492, demonstrating better performance of the H-Optimus-based model.

\begin{table}[h!]
\renewcommand{\arraystretch}{1.5}
  \centering
  \caption{Cell quantification metrics for baseline and proposed models (CI 95\%).}
  \label{tab:2}
  \begin{tabular}{|l|c|c|}
    \hline
    %\rowcolor{gray!30}
    Metric             & Resnet50-based            & H-optimus-based              \\
    \hline
    $R^2_{neoplastic}$    & 0.681 (0.576--0.769)       & \textbf{0.941 (0.917--0.960)} \\
    \hline
    $R^2_{inflammatory}$  & 0.863 (0.778--0.903)       & \textbf{0.949 (0.918--0.966)} \\
    \hline
    $R^2_{connective}$    & 0.600 (0.488--0.698)       & 0.609 (0.436--0.772)          \\
    \hline
    $R^2_{dead}$          & 0.097 (-11.389--0.669)     & 0.925 (0.404--0.982)          \\
    \hline
    $R^2_{epithelial}$    & 0.635 (0.490--0.747)       & \textbf{0.930 (0.886--0.964)} \\
    \hline
    $PQ_{neoplastic}$       & 0.517 (0.499--0.535)       & \textbf{0.589 (0.575--0.604)} \\
    \hline
    $PQ_{inflammatory}$     & 0.455 (0.429--0.482)       & \textbf{0.528 (0.507--0.549)} \\
    \hline
    $PQ_{connective}$       & 0.416 (0.400--0.431)       & \textbf{0.451 (0.436--0.465)} \\
    \hline
    $PQ_{dead}$             & 0.374 (0.342--0.408)       & 0.292 (0.209--0.365)          \\
    \hline
    $PQ_{epithelial}$       & 0.488 (0.460--0.519)       & \textbf{0.599 (0.579--0.618)} \\
    \hline
  \end{tabular}
\end{table}

Our results  show that integrating the H‑Optimus foundation model within the UNETR architecture enhances the model's ability to segment and classify cells across diverse tissues from PanNuke data. The pretrained transformer encoder provides robust feature representations, resulting in higher average $R^2$ and $PQ$ scores compared to the CNN-based model. This leads to more reliable cell quantification and more accurate downstream analysis. Additionally, the streamlined fine-tuning process reduces computational overhead and training time, making the model more adaptable for new data.

Despite these advancements, the foundation model-based approach does not fully resolve all challenges related to cell segmentation and classification. We observe lower metric scores for underrepresented classes in the training data. Furthermore, foundation models typically encompass billions of parameters, resulting in substantial computational and memory requirements. It therefore poses challenges for deployment in resource-constrained environments, limiting their practical applicability in certain clinical settings.

\section{Model optimization via Knowledge Distillation}

To address the limitations posed by the extensive size of foundation models, we implement knowledge distillation — a model compression technique that leverages the teacher-student paradigm \cite{Hinton_Vinyals_etal._2015}. By training a smaller, more efficient student model to replicate the output of a larger, pre-trained teacher model, we retain performance while significantly reducing the model's complexity and resource requirements (\hyperref[fig:fig6]{Figure 6}).

\begin{figure}[h!]
    \centering
    \includegraphics[width=\textwidth, height=0.45\textheight, keepaspectratio]{images/Figure_6.pdf}
    \caption{Knowledge distillation framework for training a student model using a pre-trained teacher}
    \label{fig:fig6}
\end{figure}

We employ knowledge distillation to compress the H‑Optimus-based teacher model into a more efficient student model. The teacher model is the modified UNETR architecture with the H‑Optimus foundation model described in the previous chapter. The student model is based on a UNet architecture augmented with residual connections and incorporates a smaller ViT encoder with 9 million parameters \cite{Steiner_Kolesnikov_etal._2022,Wightman_2019}. 

First, we fine-tune the teacher model using the refined dataset from the cross-relabeling procedure (Section 2). Initially we train the decoder of the teacher model while keeping the encoder weights frozen. We split the refined dataset into train (70\%), validation (20\%) and test (10\%) subsets (\hyperref[fig:S2]{Appendix Figure S2 (4)}). During fine-tuning, we use the train and validation subsets, while leaving the test subset for model evaluation. We set the training procedure and model hyperparameters to be identical to those that were used to demonstrate the utility of foundation models for the simultaneous cell segmentation and classification task.

Next, we perform knowledge distillation from teacher to student using the refined dataset used to fine-tune the teacher model. The student model is trained to replicate the teacher model's outputs. We utilize a specialized loss function that aligns the student's predicted probability distribution with the teacher's, incorporating the teacher's class probability distribution derived from the output. Following the methodology of Hinton et al. \cite{Hinton_Vinyals_etal._2015}, we experiment with various hyperparameter settings for the temperature ($T$) and the balancing coefficients ($\alpha$ and $\beta$) in the loss function. We vary $T$ from 1 to 20 and adjust $\alpha$ and $\beta$ to balance the distillation and student losses. Through iterative tuning and evaluation, we identify that setting $T=14$, $\alpha=0.3$, and $\beta=0.7$ yields a configuration that converges and closely approximates the teacher model's performance during training.

Finally, we assess the performance of both models using the $R^2$ and $PQ$ (defined in \hyperref[chap:S5]{Appendix S5}) on the test set of the refined dataset (\hyperref[tab:3]{Table 3}). We observe that the 95\% confidence intervals overlap for most cell types, so we cannot claim statistically significant performance differences between the teacher and student models. One exception appears in the neoplastic class. The teacher model produces an $R^2$ of 0.919, while the student model shows an $R^2$ of 0.852. In addition, the student model achieves higher $PQ$ values for the neoplastic and connective classes, though the confidence intervals show overlap.

\begin{table}[h!]
\renewcommand{\arraystretch}{1.5}
  \centering
  \caption{Cell quantification metrics for teacher and distilled student models (CI 95\%).}
  \label{tab:3}
  \begin{tabular}{|l|c|c|}
    \hline
    %\rowcolor{gray!30}
    Metric & Teacher & Student \\
    \hline
    $R^2_{neoplastic}$    & \textbf{0.919} (0.898--0.939) & 0.852 (0.800--0.891) \\
    \hline
    $R^2_{lymphocyte}$    & 0.969 (0.956--0.977)         & 0.969 (0.956--0.978) \\
    \hline
    $R^2_{connective}$    & 0.694 (0.548--0.809)         & 0.618 (0.469--0.741) \\
    \hline
    $R^2_{dead}$          & 0.755 (0.400--0.908)         & 0.424 (0.100--0.731) \\
    \hline
    $R^2_{epithelial}$    & 0.922 (0.870--0.958)         & 0.843 (0.738--0.917) \\
    \hline
    $R^2_{macrophage}$    & 0.384 (-0.369--0.724)        & 0.704 (0.352--0.859) \\
    \hline
    $R^2_{neutrofil}$     & 0.854 (0.578--0.929)         & 0.833 (0.502--0.925) \\
    \hline
    $PQ_{neoplastic}$       & 0.581 (0.569--0.593)         & 0.601 (0.588--0.613) \\
    \hline
    $PQ_{lymphocyte}$       & 0.536 (0.520--0.553)         & 0.563 (0.544--0.579) \\
    \hline
    $PQ_{connective}$       & 0.436 (0.421--0.451)         & 0.457 (0.441--0.474) \\
    \hline
    $PQ_{dead}$             & 0.272 (0.235--0.315)         & 0.279 (0.201--0.369) \\
    \hline
    $PQ_{epithelial}$       & 0.522 (0.500--0.545)         & 0.530 (0.506--0.555) \\
    \hline
    $PQ_{macrophage}$       & 0.524 (0.459--0.588)         & 0.474 (0.405--0.543) \\
    \hline
    $PQ_{neutrofil}$        & 0.541 (0.490--0.592)         & 0.565 (0.522--0.607) \\
    \hline
  \end{tabular}
\end{table}


We further decompose the $PQ$ metric into its $SQ$ and $DQ$ components (\hyperref[tab:S6]{Appendix Table S6}). Both models produce nearly identical $SQ$ values, which indicates that they predict instance boundaries with similar precision. Although the student model shows some improvement in $DQ$ scores for certain classes, the confidence intervals overlap and do not confirm a statistically significant difference.

We observe that the student and teacher models yield comparable detection performance despite the student model using a much smaller and simpler architecture. A model with fewer parameters reduces the risk of overfitting when training data are scarce relative to the model’s complexity \cite{Farias_Ludermir_etal._2022}. The knowledge distillation process also encourages the student model to focus on the most generalizable detection features learned from the teacher. These factors enable the student model to achieve similar detection performance across different cell types.

Additionally, considering the model sizes reported in \hyperref[tab:4]{Table 4}, the distilled model achieves a significant reduction compared to the teacher model, with a 48-fold decrease in parameter count and a 5.5-fold reduction in on-disk size. In inference mode, the teacher model requires 16 GB of VRAM for a batch size of 32, while the distilled model only needs 3 GB of VRAM for the same batch size. These reductions make the distilled model significantly more practical for fine-tuning and deployment in resource-constrained environments.

\begin{table}[h!]
\renewcommand{\arraystretch}{1.5}
  \centering
  \caption{Parameter counts and size of teacher and distilled model}
  \label{tab:4}
  \adjustbox{max width=\textwidth}{%
  \begin{tabular}{|l|c|c|c|}
    \hline
    %\rowcolor{gray!30}
    Metric & H-optimus-based (Teacher) & mobileViT-based (Student) & Magnitude of difference \\
    \hline
    Parameters count       & 1,158,917,906   & \textbf{24,093,393}   & \textbf{48x}  \\
    \hline
    Estimated Total Size (MB) & 87,912       & \textbf{15,935}    & \textbf{5.5x} \\
    \hline
  \end{tabular}%
}
\end{table}

%\hfill

With recent advancements in complex network architectures and the use of pretrained encoders to achieve state-of-the-art performance \cite{Baumann_Dislich_etal._2024,Hörst_Rempe_etal._2024} in cell segmentation and classification tasks, model size, computational complexity, and processing times have increased. This limits the scalability and accessibility of these models. As we demonstrate, this may be mitigated using knowledge distillation. Studies in the field of natural language processing have demonstrated the efficacy of knowledge distillation in retaining the capabilities of the teacher model while achieving significant reductions in size and complexity \cite{Huangpu_Gao_2024,Sun_Yu_etal.}. 

We demonstrate the feasibility of knowledge distillation in digital pathology, specifically for cell segmentation and classification tasks. Moreover, we achieve this performance while also significantly reducing the parameter count. In addressing the challenge of knowledge transfer, we found that distillation from a transformer-based model to a smaller transformer is more straightforward than attempting to map transformer features to CNN blocks. In our experiments, using a CNN-based network as a student results in worse cell quantification performance due to the structural constraints of CNN feature space dimensions. 

Although our primary approach relies on a transformer-based student model that performs well, it can be further optimized to incorporate advantages from CNN architectures. For example, employing alternative techniques such as using ViT adapters \cite{Chen_Duan_etal._2023} or $1 \times 1$ convolutions to adjust feature map sizes may be beneficial for harnessing CNN advantages like enhanced local feature extraction. Moreover, if additional performance improvements are desired, the process can be further enhanced by applying supplementary knowledge distillation techniques, such as self-distillation \cite{Zhang_Song_etal._2019} or online distillation \cite{Houyon_Cioppa_etal._2023}.

Despite these promising results, further validation on independent datasets is necessary to fully understand the model's limitations. Underrepresented classes may pose challenges when addressing complex cases. Pathologists need to validate these models to adopt them in clinical settings. While the distilled models are smaller and more deployable, a technological gap persists because pathologists traditionally rely on established methods for inspecting WSIs and diagnosing diseases. Addressing the complexities involved in deploying models for inference and supporting pathologists in adopting new tools is essential for integrating these models into clinical workflows.

\section{Model integration with QuPath}
Digital pathology tools with graphical user interfaces are essential for visualizing and analyzing WSIs. To make our student model useful in clinical pathology workflows, it needs to be integrated into a tool that enables inspecting regions, creating annotations, and providing quantitative analyses of biomarkers. Therefore, we integrate the trained student model from the previous chapter into the QuPath open‑source platform \cite{Bankhead_Loughrey_etal._2017}. QuPath provides the required annotation, visualization, and analysis tools to interpret complex histological data, including workflows for cell segmentation, classification, and quantification (\hyperref[fig:fig7]{Figure 7}). 

\begin{figure}[h!]
    \centering
    \includegraphics[width=\textwidth]{images/Figure_7.pdf}
    \caption{Visualization of model-generated cell quantification annotations (left) and the corresponding unannotated slide (right) in QuPath}
    \label{fig:fig7}
\end{figure}

To identify the regions in a WSI critical for prognosticating tumor development, such as specific tumor areas or border regions without overlapping healthy tissue, the pathologist uses QuPath to outline these regions. Then, the pathologist initiates a cell segmentation and classification script through the QuPath interface for the selected regions. The resulting annotations and quantified cell information are then directly overlaid onto the WSI in the QuPath interface. Additional design and implementation details are in \hyperref[chap:S7]{Appendix S7}. 

Two common approaches for integrating deep learning models into QuPath are Java‑based native QuPath extensions \cite{Goldsborough_Philps_etal._2024} and the execution of RESTful API requests to a model server coupled with handling the response via an extension, as demonstrated in the application of cell segmentation models applied to immunofluorescence images \cite{Sugawara_2023}. While the community is actively working on these integration strategies, there is currently no universal solution that fully addresses all integration and performance requirements.

Extensions may offer better integration with QuPath, allowing slightly improved performance and more widespread usage of the built-in QuPath models, but they lack the flexibility to customize models and modify their behavior. For example, the newest version of QuPath includes models such as StarDist \cite{Weigert_Schmidt} and InstanSeg \cite{Goldsborough_Philps_etal._2024} that can perform cell segmentation. Both models pose limitations when applied to simultaneous cell segmentation and classification. StarDist performs well only on convex, round shapes by design, whereas some neoplastic, inflammatory, and connective cells exhibit complex and non-convex shapes. InstanSeg provides only semantic segmentation without assigning classes to the segmented cells.

%\hfill

In contrast, our approach offers an alternative integration strategy. It utilizes the paquo library to directly interact with QuPath’s internal application programming interface from within Python. This enables data exchange and processing without the need for intermediate conversion steps and provides greater control over model customization, retraining, and the incorporation of custom processing steps.

The integration of our custom model with QuPath underscores its potential to significantly enhance the diagnostic process by reducing the time burden on pathologists and enabling them to focus on more complex interpretative tasks using familiar software. Leveraging a tool that is already well-established among pathologists increases the likelihood of its adoption into daily clinical workflows. The quantitative data generated through the automated workflow is critical for both clinical decision-making and research, facilitating more accurate biomarker analysis, enabling robust statistical evaluations, and supporting hypothesis generation and testing. Additionally, by streamlining cell segmentation and classification, the tool enhances the scalability and reproducibility of pathological assessments, ultimately contributing to improved diagnostic accuracy and patient outcomes.

\section{Conclusion and future work}

In this study, we address critical challenges in digital pathology and tackle the usability and deployment issues of the developed models in standard computing environments without the need for high-performance computing systems. Our multi-faceted approach encompasses data refinement through cross-relabeling, leveraging foundation models for robust cell segmentation and classification, optimizing model performance via knowledge distillation, and integrating the optimized model into the QuPath software for practical application. This approach is used to construct a capable, versatile, and adjustable model for cell segmentation and classification, with enhanced performance and usability.

\begin{sloppypar}
While our approach shows potential in the field of computational pathology, certain limitations persist. 
For example, our implementation currently exhibits lower performance in detecting macrophages. 
This serves as an instance of the broader challenge of accurately identifying complex cell types. In order to address this issue, extending our approach to incorporate additional data sources, exploring alternative modeling approaches, and integrating other imaging modalities such as immunohistochemical staining may help improve detection accuracy. Moreover, although the distilled model reduces computational demands, integrating advanced deep learning models into clinical practice requires addressing technological gaps and potential resistance to adopting new tools within established diagnostic processes.
\end{sloppypar}

Future work could focus on several key areas to refine the proposed approach and facilitate its adoption in clinical environments. Enhancing the cell-relabeling process with additional datasets \cite{Graham_Jahanifar_etal._2021} could improve the representation of underrepresented cell types and enhance overall model performance. Also, incorporating additional data sources, such as multi-modal imaging or complementary staining methods, may address limitations related to cell type differentiation and class imbalance. Exploring other foundation models \cite{Vorontsov_Bozkurt_etal._2024,Zimmermann_Vorontsov_etal._2024} or introducing additional modalities \cite{Ding_Wagner_etal._2024,Vaidya_Zhang_etal._2025} may provide alternative architectures better suited to specific tasks or offer improved efficiency. Implementing more complex knowledge distillation techniques \cite{Houyon_Cioppa_etal._2023,Zhang_Song_etal._2019} could further optimize the model's performance and adaptability. Additionally, deeper integration with QuPath or other digital pathology software could provide pathologists more control over cell quantification analysis directly within the QuPath interface, thereby increasing accessibility and usability. Such enhancements would not only refine model performance but also ensure greater adaptability and scalability within various clinical environments. Finally, extensive validation of the model by pathologists and benchmarking against independent datasets are essential steps toward establishing the model's reliability and fostering confidence in its clinical utility.

\section*{Acknowledgments} 
This work was funded in part by the Research Council of Norway grant no. 309439 SFI Visual Intelligence, and the North Norwegian Health Authority grant no. HNF1521-20.

\bibliographystyle{IEEEtran}
\begin{sloppypar}
\begin{thebibliography}{99}

\bibitem{chaplot2020neural} Chaplot, Devendra Singh, et al. "Neural topological slam for visual navigation." Proceedings of the IEEE/CVF conference on computer vision and pattern recognition. 2020.

\bibitem{maksymets2021thda} Maksymets, Oleksandr, et al. "Thda: Treasure hunt data augmentation for semantic navigation." Proceedings of the IEEE/CVF International Conference on Computer Vision. 2021.

\bibitem{mezghan2022memory} Mezghan, Lina, et al. "Memory-augmented reinforcement learning for image-goal navigation." 2022 IEEE/RSJ International Conference on Intelligent Robots and Systems (IROS). IEEE, 2022.

\bibitem{al2022zero} Al-Halah, Ziad, Santhosh Kumar Ramakrishnan, and Kristen Grauman. "Zero experience required: Plug \& play modular transfer learning for semantic visual navigation." Proceedings of the IEEE/CVF Conference on Computer Vision and Pattern Recognition. 2022.

\bibitem{ye2021auxiliary} Ye, Joel, et al. "Auxiliary tasks and exploration enable objectgoal navigation." Proceedings of the IEEE/CVF international conference on computer vision. 2021.

\bibitem{chaplot2020object} Chaplot, Devendra Singh, et al. "Object goal navigation using goal-oriented semantic exploration." Advances in Neural Information Processing Systems 33 (2020)

\bibitem{ramakrishnan2022poni} Ramakrishnan, Santhosh Kumar, et al. "Poni: Potential functions for objectgoal navigation with interaction-free learning." Proceedings of the IEEE/CVF Conference on Computer Vision and Pattern Recognition. 2022.

\bibitem{ramrakhya2022habitat} Ramrakhya, Ram, et al. "Habitat-web: Learning embodied object-search strategies from human demonstrations at scale." Proceedings of the IEEE/CVF Conference on Computer Vision and Pattern Recognition. 2022.

\bibitem{mousavian2019visual} Mousavian, Arsalan, et al. "Visual representations for semantic target driven navigation." 2019 International Conference on Robotics and Automation (ICRA). IEEE, 2019.

\bibitem{dhariwal2021diffusion} Dhariwal, Prafulla, and Alexander Nichol. "Diffusion models beat gans on image synthesis." Advances in neural information processing systems 34 (2021)

\bibitem{ho2022classifier} Ho, Jonathan, and Tim Salimans. "Classifier-free diffusion guidance." arXiv preprint arXiv:2207.12598 (2022).

\bibitem{nichol2021glide} Nichol, Alex, et al. "Glide: Towards photorealistic image generation and editing with text-guided diffusion models." arXiv preprint arXiv:2112.10741 (2021)

\bibitem{brooks2023instructpix2pix} Brooks, Tim, Aleksander Holynski, and Alexei A. Efros. "Instructpix2pix: Learning to follow image editing instructions." Proceedings of the IEEE/CVF Conference on Computer Vision and Pattern Recognition. 2023.

\bibitem{fu2023guiding} Fu, Tsu-Jui, et al. "Guiding instruction-based image editing via multimodal large language models." arXiv preprint arXiv:2309.17102 (2023).

\bibitem{geng2024instructdiffusion} Geng, Zigang, et al. "Instructdiffusion: A generalist modeling interface for vision tasks." Proceedings of the IEEE/CVF Conference on Computer Vision and Pattern Recognition. 2024.

\bibitem{zhou2024minedreamer} Zhou, Enshen, et al. "Minedreamer: Learning to follow instructions via chain-of-imagination for simulated-world control." arXiv preprint arXiv:2403.12037 (2024).

\bibitem{zhou2023esc} Zhou, Kaiwen, et al. "Esc: Exploration with soft commonsense constraints for zero-shot object navigation." International Conference on Machine Learning. PMLR, 2023.

\bibitem{yu2023l3mvn} Yu, Bangguo, Hamidreza Kasaei, and Ming Cao. "L3mvn: Leveraging large language models for visual target navigation." 2023 IEEE/RSJ International Conference on Intelligent Robots and Systems (IROS). IEEE, 2023.

\bibitem{gadre2023cows} Gadre, Samir Yitzhak, et al. "Cows on pasture: Baselines and benchmarks for language-driven zero-shot object navigation." Proceedings of the IEEE/CVF Conference on Computer Vision and Pattern Recognition. 2023.

\bibitem{shah2023navigation} Shah, Dhruv, et al. "Navigation with large language models: Semantic guesswork as a heuristic for planning." Conference on Robot Learning. PMLR, 2023.

\bibitem{cai2024bridging} Cai, Wenzhe, et al. "Bridging zero-shot object navigation and foundation models through pixel-guided navigation skill." 2024 IEEE International Conference on Robotics and Automation (ICRA). IEEE, 2024.

\bibitem{yu2023co} Yu, Bangguo, Hamidreza Kasaei, and Ming Cao. "Co-NavGPT: Multi-robot cooperative visual semantic navigation using large language models." arXiv preprint arXiv:2310.07937 (2023).

\bibitem{wu2024voronav} Wu, Pengying, et al. "Voronav: Voronoi-based zero-shot object navigation with large language model." arXiv preprint arXiv:2401.02695 (2024).

\bibitem{qin2023mp5} Qin, Yiran, et al. "Mp5: A multi-modal open-ended embodied system in minecraft via active perception." arXiv preprint arXiv:2312.07472 (2023).

\bibitem{du2024learning} Du, Yilun, et al. "Learning universal policies via text-guided video generation." Advances in Neural Information Processing Systems 36 (2024).

\bibitem{ajay2024compositional} Ajay, Anurag, et al. "Compositional foundation models for hierarchical planning." Advances in Neural Information Processing Systems 36 (2024).

\bibitem{liang2024skilldiffuser} Liang, Zhixuan, et al. "Skilldiffuser: Interpretable hierarchical planning via skill abstractions in diffusion-based task execution." Proceedings of the IEEE/CVF Conference on Computer Vision and Pattern Recognition. 2024.

\bibitem{heusel2017gans} Heusel, Martin, et al. "Gans trained by a two time-scale update rule converge to a local nash equilibrium." Advances in neural information processing systems 30 (2017).

\bibitem{zhang2018unreasonable} Zhang, Richard, et al. "The unreasonable effectiveness of deep features as a perceptual metric." Proceedings of the IEEE conference on computer vision and pattern recognition. 2018.

\bibitem{brown2020language} Brown, Tom B. "Language models are few-shot learners." arXiv preprint arXiv:2005.14165 (2020).

\bibitem{podell2023sdxl} Podell, Dustin, et al. "Sdxl: Improving latent diffusion models for high-resolution image synthesis." arXiv preprint arXiv:2307.01952 (2023).

\bibitem{brohan2022rt} Brohan, Anthony, et al. "Rt-1: Robotics transformer for real-world control at scale." arXiv preprint arXiv:2212.06817 (2022).

\bibitem{brohan2023rt} Brohan, Anthony, et al. "Rt-2: Vision-language-action models transfer web knowledge to robotic control." arXiv preprint arXiv:2307.15818 (2023).

\bibitem{li2024manipllm} Li, Xiaoqi, et al. "Manipllm: Embodied multimodal large language model for object-centric robotic manipulation." Proceedings of the IEEE/CVF Conference on Computer Vision and Pattern Recognition. 2024.

\bibitem{shah2023vint} Shah, Dhruv, et al. "ViNT: A foundation model for visual navigation." arXiv preprint arXiv:2306.14846 (2023).

\bibitem{liu2024visual} Liu, Haotian, et al. "Visual instruction tuning." Advances in neural information processing systems 36 (2024).

\bibitem{hu2021lora} Hu, Edward J., et al. "Lora: Low-rank adaptation of large language models." arXiv preprint arXiv:2106.09685 (2021).

\bibitem{qin2023supfusion} Qin, Yiran, et al. "SupFusion: Supervised LiDAR-camera fusion for 3D object detection." Proceedings of the IEEE/CVF International Conference on Computer Vision. 2023.

\bibitem{qin2024worldsimbench} Qin, Yiran, et al. "Worldsimbench: Towards video generation models as world simulators." arXiv preprint arXiv:2410.18072 (2024).

\bibitem{yu2025gamefactory} Yu, Jiwen, et al. "GameFactory: Creating New Games with Generative Interactive Videos." arXiv preprint arXiv:2501.08325 (2025).

\bibitem{zhou2024code} Zhou, Enshen, et al. "Code-as-Monitor: Constraint-aware Visual Programming for Reactive and Proactive Robotic Failure Detection." arXiv preprint arXiv:2412.04455 (2024).

\bibitem{zhang2024ad} Zhang, Zaibin, et al. "AD-H: Autonomous Driving with Hierarchical Agents." arXiv preprint arXiv:2406.03474 (2024).

\bibitem{wang2024toward} Wang, Chaoqun, et al. "Toward Accurate Camera-based 3D Object Detection via Cascade Depth Estimation and Calibration." arXiv preprint arXiv:2402.04883 (2024).

\bibitem{huang2024story3d} Huang, Yuzhou, et al. "Story3d-agent: Exploring 3d storytelling visualization with large language models." arXiv preprint arXiv:2408.11801 (2024).

\bibitem{savinov2018semi} Savinov, Nikolay, Alexey Dosovitskiy, and Vladlen Koltun. "Semi-parametric topological memory for navigation." arXiv preprint arXiv:1803.00653 (2018).

\bibitem{majumdar2022zson} Majumdar, Arjun, et al. "Zson: Zero-shot object-goal navigation using multimodal goal embeddings." Advances in Neural Information Processing Systems 35 (2022): 32340-32352.

\bibitem{yadav2023offline} Yadav, Karmesh, et al. "Offline visual representation learning for embodied navigation." Workshop on Reincarnating Reinforcement Learning at ICLR 2023. 2023.

\bibitem{yadav2023ovrl} Yadav, Karmesh, et al. "Ovrl-v2: A simple state-of-art baseline for imagenav and objectnav." arXiv preprint arXiv:2303.07798 (2023).

\bibitem{sun2024fgprompt} Sun, Xinyu, et al. "FGPrompt: fine-grained goal prompting for image-goal navigation." Advances in Neural Information Processing Systems 36 (2024).

\bibitem{zhu2017target} Zhu, Yuke, et al. "Target-driven visual navigation in indoor scenes using deep reinforcement learning." 2017 IEEE international conference on robotics and automation (ICRA). IEEE, 2017.

\bibitem{koh2024generating} Koh, Jing Yu, Daniel Fried, and Russ R. Salakhutdinov. "Generating images with multimodal language models." Advances in Neural Information Processing Systems 36 (2024).

\bibitem{krantz2022instance} Krantz, Jacob, et al. "Instance-specific image goal navigation: Training embodied agents to find object instances." arXiv preprint arXiv:2211.15876 (2022).

\bibitem{schulman2017proximal} Schulman, John, et al. "Proximal policy optimization algorithms." arXiv preprint arXiv:1707.06347 (2017).

\bibitem{anderson2018evaluation} Anderson, Peter, et al. "On evaluation of embodied navigation agents." arXiv preprint arXiv:1807.06757 (2018).

\bibitem{lin2024navcot} Lin, Bingqian, et al. "NavCoT: Boosting LLM-Based Vision-and-Language Navigation via Learning Disentangled Reasoning." arXiv preprint arXiv:2403.07376 (2024).

\bibitem{NavGPT} Zhou, Gengze, Yicong Hong, and Qi Wu. "Navgpt: Explicit reasoning in vision-and-language navigation with large language models." Proceedings of the AAAI Conference on Artificial Intelligence.

\bibitem{hahn2021no} Hahn, Meera, et al. "No rl, no simulation: Learning to navigate without navigating." Advances in Neural Information Processing Systems 34 (2021): 26661-26673.

\bibitem{li2025t2isafety} Li, Lijun, et al. "T2ISafety: Benchmark for Assessing Fairness, Toxicity, and Privacy in Image Generation." arXiv preprint arXiv:2501.12612 (2025).

\bibitem{an2024agfsync} An, Jingkun, et al. "AGFSync: Leveraging AI-Generated Feedback for Preference Optimization in Text-to-Image Generation." arXiv preprint arXiv:2403.13352 (2024).


\end{thebibliography}
\end{sloppypar}

\clearpage
\beginsupplement
\section*{Appendix}
\renewcommand{\thesubsection}{S\arabic{subsection}}

\subsection{\label{chap:S1}PanNuke and MoNuSAC preprocessing}
The PanNuke dataset comprises a set of 7,901 RGB patches, each with dimensions of $256 \times 256$ pixels, which we set as the standard patch size for our analysis. In contrast, the MoNuSAC dataset encompasses 294 images of heterogeneous dimensions. To standardize the MoNuSAC images with our experiments, we implement a standardization protocol. Specifically, for images exceeding the dimensions of $256 \times 256$ pixels, we segment them into equal-sized patches and apply mirror padding to the remaining portions to avoid information loss at the peripherals. Patches with dimensions less than $128 \times 128$ pixels are excluded from the dataset due to the insufficient resolution to capture relevant cellular details. For patches where either dimension falls between 128 and 256 pixels, we employ upsampling to achieve the standard patch size. As a result, we obtain a total of 2,823 RGB patches derived from the MoNuSAC dataset for subsequent analysis. For additional details on the MoNuSAC data preparation process, refer to the source code \cite{Shvetsov_2025a}.
\clearpage

\subsection{\label{chap:S2}Data usage for the methodology}

\counterwithin{figure}{subsection}
\renewcommand{\thefigure}{S\arabic{subsection}}

\begin{figure}[h!]
    \centering
    \includegraphics[width=\textwidth, height=0.85\textheight, keepaspectratio]{images/A2.pdf}
    \caption{Overview of the methodology for cross-labeling, dataset refinement, and model comparison. (1) Cross-relabeling - training and testing cell classification models, (2) Cross-relabeling - using cell classification models to create refined dataset, (3) Fine-tuning and training models for comparison, (4) Student knowledge distillation with refined dataset}
    \label{fig:S2}
\end{figure}
\clearpage

\subsection{\label{chap:S3}Confusion matrices for classification models}
\counterwithin{figure}{subsection}
\renewcommand{\thefigure}{S\arabic{subsection}.\arabic{figure}}

\begin{figure}[h!]
    \centering
    \includegraphics[width=\textwidth, height=0.4\textheight, keepaspectratio]{images/A3_1.pdf}
    \caption{Confusion matrix for PanNuke trained model}
    \label{fig:S3.1}
\end{figure}

\begin{figure}[h!]
    \centering
    \includegraphics[width=\textwidth, height=0.4\textheight, keepaspectratio]{images/A3_2.pdf}
    \caption{Confusion matrix for MoNuSAC trained model}
    \label{fig:S3.2}
\end{figure}

\clearpage

\subsection{\label{chap:S4}Datasets cell counts}

\counterwithin{table}{subsection}
\renewcommand{\thetable}{S\arabic{subsection}}

\begin{table}[h!]
\renewcommand{\arraystretch}{2.0}
\centering
\caption{\label{tab:S4}Cell counts for PanNuke, MoNuSAC and refined datasets. Numbers in parentheses indicate preprocessed cell counts for cell classifier models training and testing.}
%\adjustbox{max width=\textwidth}{%
\begin{tabular}{|l|c|c|c|}
\hline
%\rowcolor{gray!30}
Cell type & PanNuke & MoNuSAC & Refined \\
\hline
Neoplastic & 77,403 (68,031) & - & 105,451 \\
\hline
Epithelial & 26,572 (23,207) & - & 29,926 \\
\hline
Epithelial (benign and malignant) & - & 31,402 & - \\
\hline
Inflammatory & 32,276 & - & - \\
\hline
Lymphocytes & - & 37,045 (33,104) & 65,275 \\
\hline
Neutrophils & - & 1,355 (1,252) & 3,833 \\
\hline
Macrophage & - & 1,842 (1,695) & 3,410 \\
\hline
Dead & 2,908 & - & 2,908 \\
\hline
Connective & 50,585 & - & 50,585 \\
\hline
\end{tabular}
%
%}
\end{table}



\clearpage

\subsection{\label{chap:S5}Definition of validation metrics}
\counterwithin{equation}{subsection}
\renewcommand{\theequation}{\arabic{equation}}

\subsubsection{\label{chap:S5.1}R\textsuperscript{2}}
The coefficient of determination, denoted as $R^2$, is a statistical measure that represents the proportion of variance in the dependent variable that is predictable from the independent variables. In the context of cell quantification in pathology, $R^2$ is used to assess how well the predicted quantities of different cell types in a patch align with the actual quantities observed in the ground truth data, with higher values representing more accurate quantification. $R^2$ is defined as
\begin{equation*}
R^2 = 1 - \frac{\sum_{i=1}^n (y_i - \hat{y}_i)^2}{\sum_{i=1}^n (y_i - \bar{y})^2},
\end{equation*}
where $y_i$ represents the actual number of cells of a specific type in the $i$-th image, $\hat{y}_i$ represents the predicted number of cells of that type in the $i$-th image, $\bar{y}$ is the mean of the actual numbers across all images, and $n$ is the total number of images in the dataset.

The $R^2$ metric has a range of $(-\infty, 1]$. An $R^2$ of 1 indicates perfect prediction, where all predicted values exactly match the actual values. An $R^2$ of 0 suggests that the model explains none of the variability of the response data around its mean. If $R^2$ is negative, it indicates that the model performs worse than a model that simply predicts the mean of the actual values for all observations.

\subsubsection{\label{chap:S5.2}PQ}
Panoptic Quality ($PQ$) is a comprehensive metric used to evaluate the performance of segmentation models in tasks that require both instance segmentation and classification. $PQ$ provides a single score that encapsulates both the detection accuracy (i.e., how many objects were correctly identified) and the segmentation quality (i.e., how accurately the objects' boundaries were delineated). This metric is particularly useful in multiclass scenarios where each pixel is classified into distinct categories, such as different cell types in pathology images.

$PQ$ is calculated as the product of two terms: Detection Quality ($DQ$) and Segmentation Quality ($SQ$). It can be expressed as
\begin{equation*}
PQ = DQ \cdot SQ,
\end{equation*}
where
\begin{equation*}
DQ = \frac{TP}{TP + 0.5\, FP + 0.5\, FN},
\end{equation*}
\begin{equation*}
SQ = \frac{\sum_{(p, g) \in \mathcal{M}} IoU(p, g)}{TP}.
\end{equation*}
In these formulas, $TP$ denotes the number of correctly matched instances between ground truth and prediction, $FP$ denotes the predicted instances that have no corresponding ground truth, $FN$ denotes the ground truth instances that were not detected, $IoU(p, g)$ is the Intersection over Union for a pair of matched instances $p$ (prediction) and $g$ (ground truth), and $\mathcal{M}$ is the set of matched pairs.

The $PQ$ metric is calculated for each class and is averaged across classes to provide a global performance measure.

The $PQ$ score has a range of $[0, 1.0]$, where a higher score indicates better performance in both detecting and segmenting the instances correctly. A $PQ$ of 1 signifies perfect identification and segmentation of all instances, whereas a $PQ$ of 0 indicates that no instances were correctly identified and segmented.

\clearpage

\subsection{\label{chap:S6}Segmentation and Detection quality metrics for teacher and student models}

\begin{table}[h!]
\renewcommand{\arraystretch}{2.0}
\centering
\caption{Segmentation and detection quality for student and teacher models (CI 95\%)}
\label{tab:S6}
%\adjustbox{max width=\textwidth}{%
\begin{tabular}{|l|c|c|}
\hline
%\rowcolor{gray!30}
Metric & Teacher & Student \\
\hline
$SQ_{neoplastic}$ & 0.819 (0.815--0.823) & 0.824 (0.819--0.828) \\
\hline
$SQ_{lymphocyte}$ & 0.795 (0.788--0.802) & 0.790 (0.783--0.796) \\
\hline
$SQ_{connective}$ & 0.770 (0.762--0.776) & 0.780 (0.772--0.786) \\
\hline
$SQ_{dead}$ & 0.659 (0.623--0.688) & 0.657 (0.624--0.695) \\
\hline
$SQ_{epithelial}$ & 0.780 (0.770--0.790) & 0.788 (0.779--0.797) \\
\hline
$SQ_{macrophage}$ & 0.788 (0.760--0.810) & 0.757 (0.730--0.783) \\
\hline
$SQ_{neutrofil}$ & 0.782 (0.761--0.801) & 0.775 (0.759--0.792) \\
\hline
$DQ_{neoplastic}$ & 0.706 (0.692--0.719) & 0.727 (0.712--0.741) \\
\hline
$DQ_{lymphocyte}$ & 0.675 (0.656--0.698) & 0.713 (0.691--0.734) \\
\hline
$DQ_{connective}$ & 0.566 (0.546--0.584) & 0.583 (0.565--0.602) \\
\hline
$DQ_{dead}$ & 0.410 (0.361--0.465) & 0.435 (0.306--0.561) \\
\hline
$DQ_{epithelial}$ & 0.668 (0.639--0.694) & 0.673 (0.644--0.702) \\
\hline
$DQ_{macrophage}$ & 0.657 (0.583--0.727) & 0.615 (0.531--0.703) \\
\hline
$DQ_{neutrofil}$ & 0.691 (0.625--0.753) & 0.729 (0.679--0.778) \\
\hline
\end{tabular}
%
%}
\end{table}

\clearpage

\subsection{\label{chap:S7}QuPath integration method}
We adopt an integration strategy leveraging the paquo \cite{Bayer_AG} library, a Python package that enables direct interaction with QuPath’s internal API, thereby facilitating seamless data exchange without intermediate conversion steps. The data processing pipeline (\hyperref[fig:S7]{Appendix Figure S7}) begins with the acquisition of WSIs and their associated annotations from QuPath, which are represented as Shapely \cite{Gillies_Wel_etal._2024} polygons. Utilizing paquo, we directly read, create, and modify these annotations and detections within a QuPath project in the Python environment. Images are then cropped using these polygons and processed by cell segmentation and classification models employing standard vision processing toolkits such as OpenCV, pyvips, and PyTorch. Additionally, QuPath employs Groovy scripts to initiate a Python process that starts the entire pipeline from QuPath graphical interface: fetching polygons, extracting images from them, and running deep learning model inference on the cropped images. 
The results are returned to QuPath, leveraging paquo's Python bindings to manipulate QuPath data while minimizing the computational overhead typically associated with cross-environment communication.

\counterwithin{figure}{subsection}
\renewcommand{\thefigure}{S\arabic{subsection}}

\begin{figure}[h!]
    \centering
    \includegraphics[width=\textwidth]{images/A7.pdf}
    \caption{QuPath integration workflow using Python environment}
    \label{fig:S7}
\end{figure}

Compared to traditional workflows that involve exporting annotations as GeoJSON, classifying them in Python, and reimporting them into QuPath, our approach offers several advantages. We eliminate the need to switch between programming languages, providing a cohesive and streamlined development process entirely within QuPath software and removing the necessity to use other tools. Meanwhile, we avoid storing annotations as intermediate JSON files unless required for external use or archiving. By conducting the entire inference and post-processing workflow within the Python environment, we leverage the power and flexibility of Python libraries for image processing and machine learning. This approach also enables adjustments to any set of labels and models, thereby improving its applicability.

%\hfill

The distilled model and QuPath integration code are packaged into a Docker container, enabling streamlined execution with the Docker engine. Detailed integration code and deployment instructions can be found in the GitHub repository \cite{Shvetsov_2025b}.

Despite these benefits, we acknowledge that the paquo library is a proof‑of‑concept project in its early development stage and has not been tested across all versions of QuPath.

\clearpage

\subsection{\label{chap:S8}Data and code availability statement}
All datasets, models, and code used in this study are publicly available and can be obtained from the repositories listed below. 
The PanNuke \cite{Gamper_Koohbanani_etal._2019} and MoNuSAC \cite{Verma_Kumar_etal._2021} datasets are publicly accessible, and download information along with detailed descriptions can be found in their respective articles. Preprocessing scripts for PanNuke and MoNuSAC data, as well as individual cell extraction scripts, are available on GitHub \cite{Shvetsov_2025a}. The H-Optimus foundation model used in our experiments can be downloaded from the HuggingFace repository \cite{hoptimus2024}, and model information is available on GitHub \cite{Saillard_Jenatton_etal._2024}. In addition, the integration code for QuPath and the distilled model packaged in a Docker container are provided in the repository \cite{Shvetsov_2025b}, and paquo Python library is available from the authors GitHub repository \cite{Bayer_AG}.
\clearpage

\end{document}



\end{document}
\endinput
%%
%% End of file `sample-sigconf.tex'.
