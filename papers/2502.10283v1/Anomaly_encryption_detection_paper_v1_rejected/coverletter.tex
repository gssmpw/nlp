\documentclass{letter}

\usepackage[left=1in,right=1in,top=1in,bottom=1in]{geometry}
\pagenumbering{gobble}

\begin{document}
\hfill March 17, 2023

Dear Editors and Reviewers,

We wish to submit the manuscript entitled “Anomaly Detection with Encrypted Control” to be considered for publication in the IEEE Control Systems Letters (L-CSS), together with a presentation at the 62nd IEEE Conference on Decision and Control 2023. 

We investigate how to use the properties of homomorphic encryption based on the Learning with Errors problem to construct an explicit method for anomaly detection in an encrypted control system that does not require additional resources from the detector's side, such as the secret key or secure, multi-round communications with the plant. Specifically, we ask if a statistical hypothesis test can be set up and only use encrypted signals to reject the hypothesis that the system is not under attack. By transforming an ensemble of encrypted signals in a particular way, we can bypass the encryption and perform detection on the entire data set. Several key insights and results that enable the detection are derived, and are highlighted as follows:

\begin{enumerate}
   \item We devise an integrated controller and anomaly detector whose settings, specifically the false alarm rate, are equally simple to specify as in conventional anomaly detectors that have previously been treated in the literature.
   
	\item We show that the power of the statistical test depends on the noise in the encrypted signals that are used for detection. Furthermore, we show that this noise depends on the encryption and controller settings, and we show how the noise can be reduced by solving a vector reduction problem.

	\item We show that by recognizing the temporal nature of the dynamical system, we can apply the classical Lenstra-Lenstra-Lov\'ascz algorithm in a particular way to find specific approximate solutions to the vector reduction problem which correspond to performing anomaly detection using only recent samples.
	
	\item Several trade-offs appear when implementing the integrated controller and anomaly detector. We choose to focus on the trade-off between noise amplification and how many samples to use for the detection, which is determined by the solutions to the vector reduction problem. Numerically, we observe that depending on when the attack was initialized, there might be an optimal choice of a number of samples.
	
\end{enumerate}

The work we present in this submission is completely novel, although some inspiration to our solution approach is drawn from our previous work entitled ``Model-free Undetectable Attacks on Linear Systems Using LWE-based Encryption", published in IEEE Control Systems Letters in 2023. In that paper, we considered an attacker who tries, from encrypted input-output data, to learn how to attack a system that has an anomaly detector that knows the secret key. Although a vector reduction problem also appears in that paper, it has a different structure and thus requires a different approach compared to the one presented here. The major novelty in our current work is that we are able to side-step solving the Learning with Errors Problem, which would break the encryption, and still extract sufficient information to be able to perform detection. Another novelty is the way we apply the Lenstra-Lenstra-Lov\'ascz algorithm, because we know that the encrypted messages are generated from a dynamical system, we are able to give slightly better guarantees of the approximate solution compared to the ordinary vector reduction problem.

The keywords associated with our submission are the following: Fault detection, Control over communications, and Quantized systems.

Thank you for your consideration of this manuscript. We are looking forward to hearing from you.

Yours sincerely,

Rijad Alisic, Junsoo Kim, and Henrik Sandberg

\end{document}