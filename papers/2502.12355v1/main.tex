% \documentclass[conference]{IEEEtran}
\documentclass[letterpaper, 10 pt, conference]{ieeeconf} 
\IEEEoverridecommandlockouts
% The preceding line is only needed to identify funding in the first footnote. If that is unneeded, please comment it out.
\usepackage{cite}
\usepackage{comment}
\usepackage{amsmath,amssymb,amsfonts}
% \usepackage{caption}
\usepackage{tabularx} 
\usepackage{algorithmic}
\usepackage{graphicx}
\usepackage{textcomp}
\usepackage{xcolor}
\usepackage{multirow}
\usepackage{etoolbox}
\usepackage{balance}
\usepackage[nolist,nohyperlinks]{acronym}
\usepackage{tikz}
\usepackage{dsfont}
% \usepackage{subfigure}
% \usepackage{hyperref}       % hyperlinks
\usepackage{url}            % simple URL typesetting
\usepackage{booktabs}       % professional-quality tables
\usepackage{nicefrac}       % compact symbols for 1/2, etc
\usepackage{microtype}      % microtypography
\usepackage{xcolor}         % colors
% \usepackage{cleveref}

% \usepackage{todonotes}
\makeatletter
\patchcmd{\@makecaption}
  {\scshape}
  {}
  {}
  {}
\makeatother

\def\BibTeX{{\rm B\kern-.05em{\sc i\kern-.025em b}\kern-.08em
    T\kern-.1667em\lower.7ex\hbox{E}\kern-.125emX}}
\usetikzlibrary{
  shapes,arrows,automata,fit,backgrounds,calc,positioning,patterns,
  decorations.pathreplacing,
  decorations.pathmorphing
}

\begin{acronym}
\acro{OC}{optimal control}
\acro{LQR}{linear quadratic regulator}
\acro{MAV}{micro aerial vehicle}
\acro{IMAV}{insect-scale micro aerial vehicle}
\acro{GPS}{guided policy search}
\acro{UAV}{unmanned aerial vehicle}
\acro{MPC}{model predictive control}
\acro{BC}{behavior cloning}
\acro{DR}{domain randomization}
\acro{IL}{imitation learning}
\acro{DAgger}{dataset aggregation}
\acro{MDP}{Markov decision process}
\acro{CoM}{Center of Mass}
\acro{SD}{standard deviation}
\acro{MSE}{mean squared error}
\acro{RMSE}{root mean square error}
\acro{NN}{neural network}
\acro{UKF}{unscented Kalman filter}
\acro{PPO}{proximal policy optimization}
\acro{LfD}{learning from demonstration}
\acro{RL}{reinforcement learning}
\acro{PPO}{proximal policy optimization}
\acro{DEA}{dielectric elastomer actuator}
\acro{LSTM}{long short-term memory}
\acro{MoI}{moment of inertia}
\end{acronym}

\def\-{\raisebox{.75pt}{-}}
    
\begin{document}

% set space above/below the equations
\setlength{\abovedisplayskip}{3pt}
\setlength{\belowdisplayskip}{3pt}

% \title{\vspace*{-0mm} \LARGE \bf Delay-Compensated Deep Reinforcement Learning for an Insect-Scale Aerial Robot Powered by Soft Artificial Muscles \vspace*{-0mm}

\title{\vspace*{-0mm} \LARGE \bf Hovering Flight of Soft-Actuated Insect-Scale Micro Aerial Vehicles using Deep Reinforcement Learning \vspace*{-0mm}

\author{Yi-Hsuan Hsiao$^\star$, Wei-Tung Chen$^\star$, Yun-Sheng Chang$^\star$, Pulkit Agrawal, and YuFeng Chen$^\dagger$}

% \author{Anonymous Authors}
\thanks{$^\star$These authors contributed to this work equally. $^\dagger$Corresponding author.}

\thanks{These authors are with Massachusetts Institute of Technology (MIT), Cambridge, MA, USA (email: \texttt{yhhsiao, weitung, yschang, pulkitag, yufengc@mit.edu}).}

\thanks{This work was partially supported by the National Science Foundation (FRR-2202477 and FRR-2236708), and the Research Laboratory of Electronics, MIT (2244181). Any opinions, findings, and conclusions or recommendations expressed in this material are those of the authors and do not necessarily reflect the views of the National Science Foundation. Yi-Hsuan Hsiao is supported by the Mathworks Engineering Fellowship.}

}

\maketitle

\begin{abstract}

Soft-actuated insect-scale micro aerial vehicles (IMAVs) pose unique challenges for designing robust and computationally efficient controllers. At the millimeter scale, fast robot dynamics ($\sim$ms), together with system delay, model uncertainty, and external disturbances significantly affect flight performances. Here, we design a deep reinforcement learning (RL) controller that addresses system delay and uncertainties. To initialize this neural network (NN) controller, we propose a modified behavior cloning (BC) approach with state-action re-matching to account for delay and domain-randomized expert demonstration to tackle uncertainty. Then we apply proximal policy optimization (PPO) to fine-tune the policy during RL, enhancing performance and smoothing commands. In simulations, our modified BC substantially increases the mean reward compared to baseline BC; and RL with PPO improves flight quality and reduces command fluctuations. We deploy this controller on two different insect-scale aerial robots that weigh 720 mg and 850 mg, respectively. The robots demonstrate multiple successful zero-shot hovering flights, with the longest lasting 50 seconds and root-mean-square errors of 1.34 cm in lateral direction and 0.05 cm in altitude, marking the first end-to-end deep RL-based flight on soft-driven IMAVs.

\end{abstract}

% \begin{IEEEkeywords}
% component, formatting, style, styling, insert
% \end{IEEEkeywords}

% ===============================
% ===============================
% ===============================
% ===============================

\section{Introduction}
Inspired by the exquisite maneuverability of natural insects, the robotics community has developed \acp{IMAV} that weigh less than a gram and are capable of stable hovering \cite{ma2013controlled,chukewad2021robofly,bena2023high,chen2019controlled}. Among these platforms, a class of soft-actuated \acp{IMAV} \cite{chen2019controlled} has gained particular attention due to their resilience to collisions \cite{chen2021collision}. Driven by muscle-like \acp{DEA}, these soft-actuated sub-gram robots can absorb external impacts, highlighting the potential of IMAV applications such as assisted pollination in unstructured environments. 

Despite demonstrating unique capabilities, these DEA-driven \acp{IMAV} face distinct challenges in flight controller design due to their soft actuation. First, while the soft robots exhibit excellent impact resistance, they respond more slowly and require real-time communication between the off-board sensing, power, and control subsystems. Prior work \cite{chen2019controlled} reported a 15- to 20-ms system delay that is contributed by the soft actuation and the communication between the robot and external apparatus. Such delay is critical to \acp{IMAV}, which have fast body dynamics in the millisecond range. Traditional model-based flight controllers\cite{chen2019controlled}\cite{chen2021collision} mitigate this issue by setting non-aggressive control gains; however, this method greatly reduces the closed-loop control performance, making it difficult to track aggressive and long trajectories, like the ones shown on larger scale flying robots \cite{song2023reaching}.
%\cite{tagliabue2019robust} added a predictive element to the \ac{UKF} to alleviate the delay; nonetheless, this approach is computationally expensive, making it infeasible for computation-constrained onboard microcontrollers in the future. 

The second control challenge comes from model uncertainty and large external disturbances. The fabrication of soft IMAVs requires manual assembly,
%of many microrobotic components. While the robot's moment of inertia is calculated using computer-aided-design (CAD) models, we estimate 
which leads to a 10-20$\%$ error in the estimation of the robot's \ac{MoI}. 
%due to fabrication and assembly imprecision. 
In addition, the soft actuators require 1.4- to 2-kV voltage for flights, so these DEA-driven IMAVs are tethered to a bundle of wires for offboard power during their aerial maneuvers. These wires contribute to position and attitude-dependent disturbances that are difficult to model.

% The third challenge involves IMAV’s limited lifetime \cite{hsiao2023heading}. Flapping-wing IMAVs operate in the 150 – 500 Hz range, which creates large stress on actuators, transmissions, and wings. Most IMAVs require frequent tuning or repair after a sequence of short flights that span tens of seconds. The mapping between driving commands and lift force needs to be updated after each repair

% robot photo
\begin{figure}[t]
\centering
\centerline{\includegraphics[width=0.485\textwidth]{figures/robot_v8.pdf}}
\vspace{-2.5mm}
\caption{An image of a 720-mg eight-wing micro-aerial-robot (left) and an 850-mg four-wing micro-aerial-robot (right) both driven by DEAs. The robot consists of either a 3D-printed or a carbon fiber airframe that connects four modules. Each module has a DEA, transmissions, wing hinges, and wings. The robot requires external systems for sensing, control, and power.  }
%An image of a 680 mg micro-aerial-robot driven by dielectric elastomer actuators. The robot has 4 modules that are assembled into an inclined airframe. Each module has an airframe, a dielectric elastomer actuator (DEA), two linear four-bar transmissions, wing hinges, and wings. The robot requires external power supply through its tethers. }
\label{fig:robot_photo}
\vspace{-6mm}
\end{figure}

% Early works \cite{ma2013controlled,chen2019controlled} on IMAV flight control developed model-based strategies to address these challenges. System identification methods \cite{finio2011system} were applied to estimate robot parameters by performing a sequence of short open-loop attempts prior to the closed-loop flights. 
% %Based on these results, short closed-loop flights are conducted to fit controller parameters. This process aims to minimize the robot operation time and it has demonstrated good hovering flight results. However, the
% This tuning process requires hours of a researcher's input, and it needs to be repeated every time a robot is repaired. 

\begin{figure*}[t]
    \centering
    \scalebox{0.95}{\begin{figure*}[!t]
    \centering
    \includegraphics[width=0.8\linewidth]{imgs/main/img-overview.pdf}
    \caption{\textbf{Overview of \method.} Given the input view, we first estimate the depth information from a depth estimator and then perform pixel splatting to generate splatted views as diffusion conditioning. In our splatting-guided diffusion, we fine-tune the latent U-Net for aligned synthesis, producing consistent novel views while correcting splatting errors. Meanwhile, a texture bridge module is designed to aggregate the encoder features for high-fidelity synthesis.}
    \label{fig:main-overview}
\end{figure*}
}
    \vspace*{-3mm}
    \caption{Overview of our proposed controller design. First, from a model-based controller, $\pi_{e_i}$, a set of expert demonstrations is generated with randomized domain parameters. Then, we re-match the delayed state with the action to account for system delay. We implement behavior cloning to initialize a neural network controller. Next, in the RL phase, the control policy is fine-tuned with PPO to improve performance and reduce driving command fluctuations. Finally, the controller is integrated into the Matlab Simulink Real-Time environment for demonstrating robot hovering flight.}
    \label{fig:overview}
    \vspace*{-6mm}
\end{figure*}

To address these challenges, learning-based methods were applied in several prior studies \cite{perez2015model}\cite{de2021efficient}.  In a previous work \cite{perez2015model}, researchers designed controllers with fixed structures and then used learning methods to identify the control parameters based on flight experiments. Another work \cite{de2021efficient} combined model-based and model-free methods to design a hovering flight controller. The simulations show substantial improvement, but the 1.5-second real-world flight demonstrations suffer a large position error of over 10 cm. This result shows that the unaccounted-for \textit{Sim2Real} gap has a substantial influence on the flight performance of \acp{IMAV} and highlights the difficulties of bringing the controller from simulation to real-world \acp{IMAV}, emphasizing the importance of incorporating model uncertainty and system delay into the simulator.



Another work \cite{tagliabue2023robust} presented a cascaded control architecture, which connects a positional \ac{NN} controller to a model-based attitude controller. % \cite{lee2010geometric}. %trained a \ac{NN} based on a carefully tuned controller. 
The researchers used supervised learning to train a \ac{NN} with expert demonstrations from a hand-tuned model predictive controller (MPC). While the flight results show hovering capability, this supervised learning method relies heavily on the performance of the expert controller and may lead to sub-optimal behavior as the time horizon increases \cite{song2023reaching}.  
%Specifically, the researchers implemented a robust-tube \ac{MPC} to generate expert demonstrations and then trained an \ac{NN} to improve the computation efficiency. This method showed an over 60$\%$ improvement in flight accuracy, which represented a substantial improvement over purely model-based controllers.
% However, this controller did not account for system delay and model uncertainty explicitly and still required careful manual tuning of the model-based attitude controller. 

Here, we propose a deep \ac{RL} approach for controlling \acp{IMAV} and demonstrate stable real-world flights on these soft-actuated platforms. In contrast to previous work \cite{de2021efficient}, we successfully bridge the \textit{Sim2Real} gap and showcase translational flight performance from the simulator to multiple real-world platforms at the insect scale. Compared to \cite{tagliabue2023robust}, our method replaces the entire control scheme with a single \ac{NN} controller and is trained through unsupervised learning (\ac{RL}) to seek optimal performance without relying on hand-tuned expert demonstrations.

Our new approach has two main features: 1) we explicitly account for the system delay of soft IMAVs during the initialization of \ac{NN} by using state-action re-matching in \ac{BC}, and also incorporate this delay in the simulator for \ac{RL} with \ac{PPO}. 2) we randomize domain parameters, including mass, \ac{MoI}, and external disturbances, in both \ac{BC} and \ac{RL} phases to improve controller robustness against system uncertainty. 
%We further implement \ac{PPO} \cite{schulman2017proximal} in the \ac{RL} stage to improve driving command smoothness.

These design choices result in a new flight controller that is resilient to system delay and model uncertainty on soft-actuated \ac{IMAV}. We deploy this new type of controller on two distinct DEA-driven IMAVs (Fig. 1) and evaluate their performance. Our results demonstrate multiple zero-shot hovering flights for both robots, marking the first successful deep \ac{RL} flights at the insect scale.  The longest flight we conducted lasts 50 seconds and achieves lateral and altitude \acp{RMSE} of 1.34 cm and 0.05 cm that outperform the state-of-the-art robots of similar scale \cite{bena2023high}\cite{kim2023laser}. By explicitly accounting for system delay and model uncertainty, we achieved substantial flight performance improvement with deep reinforcement learning, representing a significant step toward unlocking the full potential of fast dynamics in soft-driven \acp{IMAV}.
%Since this controller does not require manual parameter tuning, it has the potential to be deployed on multiple IMAVs, representing a first step towards achieving scalable swarm flights. 



% ===============================
% ===============================
% ===============================
% ===============================

\section{Controller design}

In this section, we describe our flight controller design under the deep \ac{RL} framework. First, we define the robot states and controller actions, together with the dynamics and flight simulator. Next, we develop a modified BC method to initialize the \ac{NN}, which accounts for the system delay and uncertainties by using state-action re-matching and domain-randomized expert demonstrations. Finally, we design a reward function and use \ac{RL} with \ac{PPO} to further optimize the policy and improve driving commands' smoothness. The high-level design of our learning-based controller is illustrated in Fig. \ref{fig:overview}. 

% ===============================

\subsection{States \& Actions} \label{sec:state_action}
The states of our robot include positions $\mathbf{p}=[x,  y, z]^T$, velocities $\mathbf{v}=[\dot{x}, \dot{y}, \dot{z}]^T$, rotational angles represented by quaternions $\mathbf{q}=[q_x, q_y, q_z, q_w]^T$, and angular velocities $\boldsymbol{\omega}=[p, q, r]^T$. The state vector is expressed as: 
\begin{equation*}
    \mathbf{s} = [\; x, \; y, \; z, \; q_x, \; q_y, \; q_z, \; q_w,\; \dot{x}, \; \dot{y}, \; \dot{z}, \; p, \; q, \; r \;]^T,
\end{equation*}
where $\mathbf{p}$, $\mathbf{v}$, and $\mathbf{q}$ are in the world-fixed frame and $\boldsymbol{\omega}$ is on the body frame. The action of the robot is defined as 
\begin{equation*}
    \mathbf{a} = [\;F, \;\tau_x, \;\tau_y\;]^T,
\end{equation*}
where $F$ represents the total thrust force along the body z-axis; $\tau_x$ and $\tau_y$ are the torques with respect to the body x-axis and body y-axis. While our robot cannot generate yaw torque ($\tau_z$), other works \cite{chen2019controlled}\cite{chen2021collision} have shown that hovering flight does not require yaw control authority. 

% ===============================

\subsection{Robot Dynamics \& Simulator}\label{subsec:dynamics_simulator}
The simulator for RL is constructed based on the 6-DOF rigid body dynamics.
% The RL framework requires a flight simulator for training the control policy. We construct a model to simulate insect-scale flapping-wing flight.
Compared to existing UAV simulators, our model accounts for yaw motion damping, external disturbances, and actuation delay.

We aim to develop the \ac{NN} controller in a near-zero yawing condition to simplify training and use the method in \cite{tagliabue2023robust} to re-map actions to the correct body frame. The robot yawing dynamics is thus intentionally constrained by a large damping term $-k_y\Dot{r}$. In addition, the power tethers create external force disturbance ($\mathbf{F}_{dist}$) and torque disturbances ($\tau_{dist,x}$,  $\tau_{dist,y}$) on the robot. The robot dynamics is described by: 
\begin{equation}
    \mathbf{\dot{p}}= \mathbf{v} \label{eq:dy_p2v},
\end{equation}
\begin{equation} 
    \mathbf{\dot{v}} = (\mathbf{R} 
    \begin{bmatrix}
    \;0, \! \! & \! \! 0, \! \! & \! \! F \; 
    \end{bmatrix}^T \!+
    \begin{bmatrix}
    \;0, \! \! & \! \!  0, \! \! & \! \!  \-mg\;
    \end{bmatrix}^T \!+ \mathbf{F}_{dist})/m\label{eq:dy_v2a},
\end{equation}
\begin{equation} \label{eq:dy_ang2angv}
    \mathbf{\dot{q}} = (\mathbf{q}\otimes[ \;0,\;\boldsymbol{\omega}^T\;]^T)/2,
\end{equation}
\begin{equation} \label{eq:dy_angv2angacc}
    \boldsymbol{\dot{\omega}} = 
    \mathbf{J}^{-1}(\-\boldsymbol{\omega}
    \times \! \mathbf{J} 
    \boldsymbol{\omega} + 
    \begin{bmatrix}
        \tau_x \! +\!  \tau_{dist,x}, \! \! \! & \! \!  \tau_y \! +\!  \tau_{dist,y}, \! \! \! & \! \!  \-k_y\dot{r}
    \end{bmatrix}^T),
\end{equation}
where $\mathbf{R}$ is the rotational matrix, $\mathbf{J}$ is the diagonalized moment of inertia tensor, and $\otimes$ is an operator representing quaternion multiplication.

The force generation in the flapping-wing systems, which involves unsteady aerodynamics and the complex underactuated hinge and wing motion, is challenging to model analytically. The inclusion of soft actuator dynamics further complicates this problem. Fortunately, the wing inertia is orders of magnitude smaller than that of the entire robot, allowing the time-averaged lift force to be used for modeling force and torque generation with a slight delay \cite{ma2013controlled}. This delay is typically less than 4 ms, with the overall system delay being dominated by the communication between various instruments required to operate the soft actuators.

To model the overall system delay \cite{tan2018sim2real}, we specify a delay time $d$. The action $\mathbf{a}_t$, which is computed at a time $t$, would be executed on the robot at the time $t+d$; the compact form of robot dynamics can be expressed as 
\begin{equation}\label{eq:delay_dynamics}
    \mathbf{\dot{s}}_{t+d} = f(\mathbf{s}_{t+d}, \, \mathbf{a}_t),
\end{equation}
where the function $f$ represents the nonlinear robot dynamics described in Eq. \eqref{eq:dy_p2v}-\eqref{eq:dy_angv2angacc}.
To solve Eq. \eqref{eq:delay_dynamics} in discrete time, we use the forward Euler method with a step size of 1 ms.

% ===============================

\subsection{Modified Behavior Cloning for Controller Initialization}\label{sec:bc_method}

To initialize the \ac{NN} controller, we design a modified BC approach that accounts for model uncertainty and system delay. %Similar to that of existing BC methods, 
We first generate expert demonstrations using a model-based controller \cite{chirarattananon2014single} with randomized robot parameters and disturbances to improve robustness. Then, we re-match the state-action pairs to account for system delay (Fig. \ref{fig:re-match}, upper right). Ultimately, we utilize supervised learning to train an \ac{NN} controller that imitates the expert demonstrations.

% ===============================



\subsubsection{Domain-Randomized Expert Demonstration}
\label{subsec:dr-demo}
To bridge the discrepancy between simulation and real-world environments (\textit{Sim2Real} gap), we randomize three types of domain parameters, including robot parameters $\mathcal{R} = \{ m, I_{xx}, I_{yy} \}$, environmental disturbance parameters $\mathcal{E} = \{ \mathbf{F}_{dist}, \tau_{dist,x}, \tau_{dist,y}$\}, and delay $d$, where $I_{xx}$ and $I_{yy}$ are \ac{MoI} with respect to body x-axis and y-axis.

We specify a value range for each parameter that corresponds to fabrication variation and model uncertainty. For instance, we choose $I_{xx}$ from the interval $[0.75I_{xx}, 1.25I_{xx}]$. Then we pick several $\phi_i$ where each of them is a mapping function from the set $\mathcal{P}$ ($\mathcal{P} \!=\! \mathcal{R} \cup \mathcal{E} \cup \{ d \}$) to their corresponding values.
% Since the expert model takes the robot parameters into account, we can instantiate multiple expert models $\pi_{\psi_i}$ for each $i$, where $\psi_i$ is a projection of $\phi_i$ that keeps the robot parameters.
The closed-loop dynamics under the expert policy $\pi_{e}$ in the environment parameterized by $\phi_i$ becomes
\begin{equation}\label{eq:closedloop_dynamics}
    \mathbf{\dot{s}} = f_{\phi_i}(\mathbf{s}, \, \pi_{e}(\mathbf{s})).
\end{equation}
Based on this domain-randomized closed-loop dynamics, we roll out trajectories with various initial states, $\mathbf{s}_0$, to create expert demonstrations for the training data set.

By incorporating the disturbances (Eq. \ref{eq:dy_p2v}-\ref{eq:dy_angv2angacc}), which push the robot slightly away from the nominal trajectory, we can efficiently sample more states and generate corresponding expert demonstrations during the rollout (Eq. \ref{eq:closedloop_dynamics}) without using iterative \ac{DAgger}\cite{ross2011reduction}. 


\begin{figure}[t]
\centering
\centerline{\includegraphics[width=0.485\textwidth]{figures/re-match_v1.pdf}}
\vspace{-2.5mm}
\caption{Workflow of State-Action Re-matching. The expert demonstration is first rolled out in the undelayed simulator; then, we offset the state-action pairs by $d$ and have $(\mathbf{s}_t,\mathbf{a}_{t+d})$ as a pair for supervised learning to clone the delay-compensated controller. The policy then goes through PPO fine-tuning and is deployed to the real-world environment.}
\label{fig:re-match}
\vspace{-6mm}
\end{figure}
% ===============================

\begin{figure*}[t]
    \centering
    \centerline{\includegraphics[width=0.995\textwidth]{figures/BC_sim_v2.pdf}}
    \vspace{-2mm}
    \caption{Simulation results of behavior cloning. (a) Comparison of the baseline method, the method with state-action re-matching, and the method with both state-action re-matching and domain randomization. Colored boxes show 25$\%$, 50$\%$, and 75$\%$ percentiles and the black bars show non-outlier minimum and maximum. Dots are outliers that are 1.5 interquartile range (IQR) away from the top or bottom of the box. (b) Comparison of controller performance as the randomization range increases. (c) Controller performance as a function of training data set size.}
    \label{fig:bc}
    \vspace{-1mm}
\end{figure*}

% ===============================

\begin{figure*}[t]
    \centering
    \centerline{\includegraphics[width=0.995\textwidth]{figures/PPO_train_v2.pdf}}
    \vspace{-2mm}
    \caption{
    Simulation results of before and after PPO fine-tuning. (a) shows the training curve of the PPO with respect to the chosen reward function. The dark blue line shows the median rewards and the light blue shaded region represents two standard deviations away from the median. (b) displays the performance improvement in simulation after PPO fine-tuning. (c-d) compare the aggressiveness of command before and after PPO, the fluctuation in command is greatly reduced after PPO fine-tuning.
    }
    \label{fig:finetune}
    \vspace{-6mm}
\end{figure*}

% ===============================

\subsubsection{State-Action Re-Matching}\label{subsubsec:rematching}
To account for the system delay, we first turn off delay in the simulator to obtain an undelayed ideal demonstration $\mathcal{T}_{\text{ud}} = \{\mathbf{s}_0, \mathbf{a}_0, ..., \mathbf{s}_t , \mathbf{a}_t , ..., \mathbf{s}_{t+d} , \mathbf{a}_{t+d} ,..., \mathbf{s}_{T} \}$ with a model-based controller (Fig. \ref{fig:re-match}, upper left). 
%We generate multiple demonstrations, $\mathcal{T}_{\text{ud}} ^i$, with various initial conditions, $\mathbf{s}_0^i$. 

Next, we "re-match" the state-action pairs.
%to obtain a delay-compensated state-action data set $\mathcal{D}$ from the undelayed demonstrations, $\mathcal{T}_{\text{ud}}$.
We recognize that in the real world, an action $\mathbf{a}_t$ would be executed at $\mathbf{s}_{t+d}$ due to system delay (Fig. \ref{fig:re-match}, lower left). Therefore, to have the optimal action $\mathbf{a}_{t+d}$ being executed at $\mathbf{s}_{t+d}$, %we need to command the action ahead of time. Specifically, 
the controller needs to generate $\mathbf{a}_{t+d}$ at $\mathbf{s}_t$. Hence, we re-match the state-action pairs with respect to time and choose $(\mathbf{s}_{t},\mathbf{a}_{t+d})$ as a pair in the training data set $\mathcal{D}$ (Fig. \ref{fig:re-match}, upperleft), where $\mathcal{D}$ is defined as
\begin{equation*}
    \mathcal{D} = 
    \{(\mathbf{s}_{t},\mathbf{a}_{t+d})| \mathbf{s}_t, \mathbf{a}_{t+d}\in \mathcal{T}_{\text{ud}} \text{ where } 0 \le t < T - d \}.
\end{equation*}

\subsubsection{Behavior Cloning (Supervised Learning)}
With the re-matched dataset, $\mathcal{D}$, the \ac{NN} controller $\pi_{\theta}$ can be initialized through \ac{BC} by solving the following optimization equation 
\begin{equation*}
    \max_\theta \sum_{(\mathbf{s},\mathbf{a}) \in \mathcal{D} } \log \pi_\theta(\mathbf{a}|\mathbf{s}).
\end{equation*}
This cloned policy, $\pi_\theta$, has already accounted for the delay from the soft actuators.

% ===============================

\subsection{Reward Function for Reinforcement Learning}\label{sec:reward_func}
To further optimize the \ac{BC} policy, $\pi_\theta$, with \ac{RL}, we design a reward function that considers both states and actions. The state-dependent objective, $r_{s}(\mathbf{s})$, aims to minimize the distance between the current state and the setpoint (origin). To intuitively assign rewards, we use Euler angles ($\phi,\theta,\psi$) retrieved from $\mathbf{q}$. The reward function involves bringing positions ($\mathbf{p}$), velocities ($\mathbf{v}$), two Euler angles ($\phi$ and $\theta$), and two angular velocities ($p$ and $q$) to zero. Since we cannot control the body yaw rate, we do not assign rewards on the states $\psi$ and $r$. The reward function on states is defined as: 
\begin{align*}
r_{s}(\mathbf{s}) =  -(\; &k_p||\mathbf{p}||^2 + k_{e}(||\phi||^2 + ||\theta||^2) + k_v||\mathbf{v}||^2 \\
&+ k_{\omega }(||p||^2 + ||q||^2)\; ),
\end{align*}
where $k_p$, $k_{e}$, $k_v$, and $k_{\omega}$ are hyperparameters that determine the relative weight of each state reward. To specify the action rewards, we penalize aggressive (fluctuating) control outputs and their deviations from the nominal action. The reward functions are defined as 
\begin{align*}
    r_{f}(\mathbf{a}) =  -( \; & k_{ff}||F_t-F_{t-1}||^2 + k_{\tau_xf}||\tau_{x,t}-\tau_{x,t-1}||^2 \\
    &+ k_{\tau_yf}||\tau_{y,t}-\tau_{y,t-1}||^2 \;), \; \forall t \in (0,T],
\end{align*}
\begin{align*}
    r_{n}(\mathbf{a}) = -(k_f||F-F_n||^2 + k_{\tau_x}||\tau_{x}||^2 + k_{\tau_y}||\tau_{y}||^2 ),
\end{align*}
where $k_{ff}$, $k_{\tau_xf}$, $k_{\tau_yf}$, $k_f$, $k_{\tau_x}$, and $k_{\tau_y}$ are hyperparameters that determine the relative weight of each action reward, and $F_n$ is the nominal thrust at the hovering state. The action reward is given by $r_{a}(\mathbf{a}) = r_{f}(\mathbf{a}) + r_{n}(\mathbf{a}).$ The total reward function sums contributions from states and actions:  
\begin{equation*}
    r(\mathbf{s},\mathbf{a}) = r_{s}(\mathbf{s}) + r_{a}(\mathbf{a}).
\end{equation*}

% ===============================

\subsection{Reinforcement Learning through PPO}\label{sec:PPO}
We utilize reinforcement learning to further optimize the policy $\pi_\theta$ that is initialized by modified \ac{BC}. Specifically, we choose to train our policy in the delayed simulator described in Sec. \ref{subsec:dynamics_simulator} (Eq. \ref{eq:delay_dynamics}), whose domain is parameterized by set $\mathcal{P}$ in Sec. \ref{subsec:dr-demo} with various initial conditions, $\mathbf{s}_0$. The \acf{PPO} is implemented to update the policy with the reward function defined in Sec. \ref{sec:reward_func}.

The main challenge in implementing \ac{PPO} involves setting appropriate hyperparameter values in the reward function. Quadrotor-like aerial robots are at least 4th-order systems, where the effects of commanded torques would appear in positional states after being integrated four times. This property makes positional rewards extremely sparse in the policy optimization formulation. As a result, although achieving position control is our primary objective, we still assign rewards to intermediate states such as velocity, Euler angles, and body angular velocity. 

In addition to setting the state rewards, it is also crucial to set appropriate action reward hyperparameters. The actions generated through the \ac{BC} policy are non-smooth \cite{zhao2023learning} because \ac{BC} learns discrete state-action pairs without considering the continuity in time. While fluctuating actions could generate reasonable results in simulation, they are harmful to the lifetime of robotic hardware \cite{hsiao2023heading}. Hence, we assign hyperparameters to the action reward function while minimizing the negative impact on control effectiveness.

The selection of hyperparameter values is guided by an initial educated estimation, followed by empirical tuning. First, we select state-dependent hyperparameters ($k_p$, $k_e$, $k_v$, and $k_\omega$) to ensure each type of state contributes to the reward with comparable magnitude. For instance, position $y = 0.015$ (m) and angular velocity $q = 2$ (rad/s) are considered similarly favorable, so the corresponding $k_p$ and $k_\omega$ are adjusted to ensure they are weighted similarly in the reward function. Action-dependent hyperparameters are chosen in a similar manner. Subsequently, multiple sets of estimated hyperparameters are employed during \ac{PPO} training, and the resulting \ac{NN} policies are evaluated in the simulator. Based on these evaluations, minor adjustments are made iteratively, followed by additional rounds of \ac{PPO} training. 

% This iterative process is completed in less than a week using an M1 MacBook Air and resulted in a set of hyperparameters capable of producing a satisfactory hovering policy.

% ===============================

\subsection{Deployment on Soft-Actuated Robot}
\label{subsec:deployment}
After performing \ac{PPO} fine-tuning, we deploy the \ac{RL}-trained policy, $\pi_{\theta'}$, on our customized experimental setup that runs Matlab Simulink Real-Time at 1 kHz. The flight arena consists of a commercial motion tracking system (Vicon), a specialized controller (Speedgoat), and high-voltage amplifiers (Trek). We built a 720-mg eight-wing IMAV and an 850-mg four-wing IMAV; both of them have four soft actuators (DEAs).


% ===============================
% ===============================
% ===============================
% ===============================

\begin{figure*}
    \centering
    \centerline{\includegraphics[width=0.995\textwidth]{figures/PPO_flight_v3.pdf}}
    \vspace{-2.5mm}\caption{A successful hovering flight performed by the deep reinforcement learning controller on a 720-mg soft-actuated IMAV. (a) A sequence of composite images illustrating a 2-second hovering flight. (b) Tracked robot lateral position, altitude, and the commanded thrust force.}
    \label{fig:PPO_flight}
    \vspace{-6mm}
\end{figure*}

\section{Results}

We conduct simulations and flight experiments to evaluate controller effectiveness. In simulations, the modified \ac{BC} improves the reward (median) by 75$\%$ compared to the baseline BC. The \ac{PPO} further enhances the reward (median) by 62$\%$ and reduces the thrust fluctuation by 51$\%$, which is crucial for performing experimental validation on real-world hardware. In flight experiments, we achieve multiple zero-shot stable hovering where our position errors outperform the state-of-the-art long endurance flights on IMAVs.

% ===============================

\subsection{Simulation Results}\label{subsec:sim_results}

To train the policy, we use a \ac{NN} with 2 hidden layers, 32 neurons per layer, and $\arctan$ as the activation function. The training and simulations are conducted in the \texttt{Gym} environment on an M1 MacBook Air, with \ac{BC} taking approximately 5 minutes and \ac{PPO} around 15 minutes per iteration. The iterative hyperparameter tuning process for PPO (Sec. \ref{sec:PPO}) is completed in less than a week and results in a set of hyperparameters capable of producing a satisfactory hovering policy.

To compare the \ac{NN} performance in simulation, we run each type of controller in the delayed simulator for 5 seconds and repeat 100 times with different robot parameters $\mathcal{R}$, environmental disturbances $\mathcal{E}$, delays $d$, and initial conditions $\mathbf{s}_0$. 

First, we %perform simulations to 
evaluate the effectiveness of the two techniques: state-action re-matching and domain-randomized expert demo. We compare the performance of three behavior cloned controllers: 1) baseline \ac{BC} (non-randomized expert demo and no re-matching); 2) \ac{BC} with state-action re-matching (non-randomized expert demo); and 3) \ac{BC} with both state-action re-matching and domain-randomized expert demonstrations. In Fig. \ref{fig:bc}a, the baseline BC policy returns the lowest median reward of -238.2, while the policy trained on state-action re-matching (without randomized domains) scores -201.2, indicating that the proposed re-matching method improves controller performance in a delayed environment. The policy trained with the randomized domains (with re-matching), achieves the best median reward of -59.8. This result showcases that randomizing parameters at the behavior cloning stage enhances the robustness of the controller to accommodate more model uncertainty.

% ===============================

% \begin{figure*}
%     \centering
%     \centerline{\includegraphics[width=0.995\textwidth]{figures/failed_v2.pdf}}
%     \vspace*{-2.5mm}\caption{A failed hovering experiment performed by a baseline BC controller. (a) A composite image of the robot flight. The robot quickly loses stability and drifts away from the hovering setpoint. (b) The tracked robot attitude shows large body oscillation. (c-e) The tracked robot x, y, and z positions. The shaded regions represent the time interval before the controller is switched off for the safety reasons.}
%     \label{fig:BC_flight}
%     \vspace{-6mm}
% \end{figure*}

% ===============================

We also vary the parameter range in our \ac{DR} implementation and evaluate its influence on controller performance. Fig. \ref{fig:bc}b shows that the mean reward remains similar (within 20$\%$ change) despite having large changes in parameter range. This result implies that larger parameter variation can lead to higher tolerance to model uncertainty without substantially sacrificing performance.

Furthermore, we investigate the learning convergence rate. Fig. \ref{fig:bc}c shows the controller performance improves as the number of training state-action pairs increases; specifically, the median rewards for 2,000, 10,000, and 20,000 state-action pairs are -409.8, -134.9, and -66.1, respectively. However, further increasing the number of training pairs has a diminishing effect on the rewards, as 40,000 and 100,000 pairs yield rewards of -65.9 and -64.7, respectively, an improvement of only 0.2 for additional 20,000 data points. This result shows training the policy only requires approximately 20,000 state-action pairs to converge, equivalent to 20 seconds of expert flight demonstrations in the simulation. 

% ===============================




% ===============================

In addition, we investigate the performance of PPO fine-tuning and its influence on smoothing the control policy at the \ac{RL} stage. Fig. \ref{fig:finetune}a shows the mean reward increases as the number of agent-environment interactions increases. Fig. \ref{fig:finetune}b demonstrates that the fine-tuned  \ac{RL} policy, $\pi_{\theta'}$, is improved through \ac{PPO} by 62$\%$ (median reward). Fig. \ref{fig:finetune}c and \ref{fig:finetune}d compare the commanded thrust before and after PPO fine-tuning. The command fluctuation reduces by 51$\%$ without reducing trajectory tracking accuracy, making it more preferable to be deployed on real-world hardware.

% ===============================



\subsection{Experimental Flight Results}\label{sec:result_flight}

% First, we conduct a flight experiment with the baseline BC controller (without re-matching and domain randomization). Fig. \ref{fig:BC_flight} and Supplementary video part 1 shows an unsuccessful flight with the baseline BC where the robot drifts more than 20 cm from the hovering set point within 1 second. The robot experiences large body oscillation (Fig. \ref{fig:BC_flight}b), and it cannot maintain constant position and attitude (Fig. \ref{fig:BC_flight}c-e) tracking. This flight result resembles tuning flights performed by a model-based controller, where the robot parameters have not been accurately identified.

\begin{figure*}
    \centering
    \centerline{\includegraphics[width=0.995\textwidth]{figures/new_robot_flight_v3.pdf}}
    \vspace*{-2.5mm}\caption{Successful hovering flights performed by the deep reinforcement learning controller on an 850-mg soft-actuated four-wing IMAV. (a) A sequence of composite images illustrating a 50-second hovering flight. The blue dots in the images indicate the setpoint (origin) of the robot. (b)-(c) Tracked robot lateral position and altitude. (b) Three 10-second hovering flights. The light colors represent repeating flights. (c) The 50-second hovering flight.}
    \label{fig:new_flight}
    \vspace{-6mm}
\end{figure*}

To evaluate the effectiveness of state-action re-matching and \ac{DR} in bridging the \textit{Sim2Real} gap, we deploy the trained policy on two distinct soft-actuated robots. The results demonstrate successful and stable hovering flights on both real-world platforms.

We first conduct a flight test on a 720-mg eight-wing \ac{IMAV} (Fig. \ref{fig:robot_photo}, left) to evaluate the reinforcement learning policy after PPO fine-tuning (Fig. \ref{fig:PPO_flight}). The soft-actuated robot achieves a zero-shot stable hovering with small lateral drift, marking the first successful deployment of deep \ac{RL} control on an insect-scale flapping-wing soft robot (Supplementary Video - Flight Video 1). Fig. \ref{fig:PPO_flight} illustrates the tracked position and commanded thrust, with position errors comparable to other work \cite{chen2019controlled}\cite{chen2021collision}. The commanded thrust is also reasonably smooth, highlighting the effectiveness of PPO fine-tuning.
%Specifically, the \acp{RMSE} of lateral position, altitude, and attitude are 3.06 cm, 0.18 cm, and 2.26 degrees for the \ac{RL} policy (Fig. \ref{fig:PPO_flight}b-d). 
% Fig. \ref{fig:PPO_flight}e-g also show \ac{RL} policy's commanded torques, thrust force, and the driving voltages to the four DEAs. In the hovering phase, the voltage amplitude oscillation is within 20 V, which is comparable to finely-tuned model-based controllers.

We then attempt to fly a four-wing robot (Fig. \ref{fig:robot_photo} right), which has an \ac{MoI} approximately six times smaller than the eight-wing version on the y-axis, making it even more challenging to control. To adapt to this design, we adjust only the robot parameters $\mathcal{R}$ for both \ac{BC} and \ac{PPO}, keeping other parameters unchanged for \ac{NN} controller training. Using the trained policy, we achieve three consecutive 10-second flights (Fig. \ref{fig:new_flight}b and Supplementary Video - Flight Video 2) with lateral position and altitude \acp{RMSE} of 0.97–1.58 cm and 0.10–0.12 cm, respectively (error calculated after a 1-second takeoff stage to allow altitude to converge). To further evaluate the policy’s reliability, we conduct an extended 50-second flight—longer than any other reported flight at the insect scale \cite{hsiao2023heading}. During this flight, the lateral position and altitude \acp{RMSE} are 1.34 cm and 0.05 cm, respectively (Fig. \ref{fig:new_flight}a,c and Supplementary Video - Flight Video 3, error calculated after a 1-second takeoff).

Compared with other long hovering flights ($>$ 10 s), the position errors of these successful flight attempts on the four-wing soft-actuated robot are smaller than those reported in state-of-the-art  \ac{IMAV} studies \cite{ bena2023high}\cite{ chen2021collision}\cite{kim2023laser}.

%, but still slightly larger than that in and \cite{tagliabue2023robust}. The main advantage of this work is that there is no need for robot calibrations. For instance, Fig. \ref{fig:PPO_flight}b shows the positional drifting in the x direction is more substantial than that in y. This may be caused by the miss alignment between the thrust vector and robot body z-axis, which can also be observed from the non-zero mean pitch angle during the hovering phase (Fig. \ref{fig:PPO_flight}d). Compared to prior methods, the time saving from manual calibration is well worthy of incurring a small positional error.


% ===============================




% ===============================
% ===============================

% \section{Discussion}

%Compared to prior works on IMAV flights  \cite{chen2021collision,tagliabue2023robust}, this paper reports the first controller that does not require manual parameter tuning. For instance, in the recent paper \cite{tagliabue2023robust}, researchers performed nine closed-loop trimming flights before achieving stable hovering. 

% In contrast, with the proposed state-action re-matching and domain randomization, we demonstrate a stable hovering flight at the very first attempt. We also verify that without domain randomization, the controller becomes unstable. This result shows an RL-trained policy can account for large uncertainty that model-based controllers have not been able to address. 
% It is crucial for enabling future work on swarm flight, where tuning every robot before flight would be tedious.




% ===============================
% ===============================
% ===============================
% ===============================


\section{Discussion \& Conclusion}

In this work, we develop a deep reinforcement learning-based controller for insect-scale aerial robots. We address the challenges of system delay and model uncertainty by initializing the policy through the state-action re-matching method with domain-randomized demonstrations. Then, we apply PPO in the reinforcement learning stage to improve flight performance and reduce driving command fluctuation. The simulation results show that the proposed techniques for BC can effectively improve the mean reward, and PPO fine-tuning reduces variations of thrust. Most importantly, we deploy this controller on a 720-mg and an 850-mg soft-actuated IMAV and demonstrate a 50-second hovering flight with lateral position and altitude error of 1.34 cm and 0.05 cm, respectively.

Achieving insect-like locomotion on a soft-actuated \ac{IMAV} requires complex planning and high-rate feedback control, yet we are limited to lightweight onboard microprocessors with constrained computational capacity. Deep reinforcement learning is an ideal solution, as a small (32x32) multi-layer perceptron can run efficiently on hardware of this size, and the neural network can learn an optimal policy over long time horizons through deep \ac{RL} \cite{song2023reaching}.

Unlike most \ac{BC} methods that rely on \ac{DAgger} \cite{ross2011reduction} to enhance neural network robustness after cloning, the proposed modified \ac{BC} method incorporates disturbances, $\mathcal{E}$, directly during the expert demonstration stage. This early introduction allows us to sample more diverse state-action pairs without \ac{DAgger}, making demonstration generation computationally efficient and reducing dataset creation time to under a minute. Including supervised learning, the modified BC approach initializes a delay-compensated \ac{NN} ready for PPO in less than 5 minutes (on an M1 MacBook Air).

Compared to \cite{tagliabue2023robust}, the proposed modified \ac{BC} could generate demonstrations more efficiently for long trajectories. For dynamically feasible trajectories \cite{sun2022comparative}, model predictive controls (MPC) \cite{tagliabue2023robust} is often computationally intensive and time-consuming. In contrast, the proposed modified \ac{BC} method, which involves no optimization, could efficiently initialize a \ac{NN} and leave the long-horizon optimization to the \ac{RL} stage \cite{song2023reaching}. At the same time, the policy obtained through deep \ac{RL} would not be limited by the sub-optimal demonstration from the hand-tuned MPC.

An essential aspect of this work is bridging the \textit{Sim2Real} gap. For the first time, an insect-scale robot achieves stable hovering flight using model-free deep \ac{RL}, demonstrating that controllers proven effective in the simulator can reliably translate to real-world soft-actuated robots. This milestone not only validates the robustness of our approach but also paves the way for testing more control strategies safely and efficiently within the simulation environment.

While this work focuses on hovering flights, it represents an intermediate step toward achieving insect-like agile maneuverability. By harnessing the potential of unsupervised deep reinforcement learning, the neural network controller can be trained on more complex tasks in simulation—such as wall perching \cite{chirarattananon2016perching}, inverted ceiling landings \cite{habas2023inverted}, and aggressive trajectory following \cite{song2023reaching}—and perform these challenging maneuvers on real-world IMAVs in the near future.

% There are several limitations that could be addressed in future works. 
%For instance, we only demonstrate flights for 2 seconds, which are much shorter than prior works on model-based methods \cite{hsiao2023heading}, and we only apply zero-mean disturbances, $\mathcal{E}$, in the simulation, which is not realistic in real-world environments. To bridge this gap, we can implement adaptive policies \cite{zhao2023efficient} or \ac{LSTM} architectures \cite{jitosho2023reinforcement} to improve controller robustness against time-varying disturbances and extend flight time.

% In addition, this controller is currently limited to performing hovering flight, which is not difficult for model-based controllers. In the future, we aim to achieve more challenging maneuvers on our \ac{IMAV}, such as wall perching \cite{chirarattananon2016perching}, inverted landing on ceiling \cite{habas2023inverted}, and aggressive trajectory following \cite{song2023reaching}.

\vspace*{1mm}

% \section*{Acknowledgement}
% This research was initiated as the course project of 6.8200 Computational Sensorimotor Learning under the guidance of Prof. Pulkit Agrawal at MIT. 

\balance
\bibliographystyle{IEEEtran}
% \bibliography{reference.bib}
\documentclass{article}
\usepackage{amsmath}
\usepackage{authblk}
\usepackage{cite}
\usepackage{graphicx}

\newtheorem{lemma}{Lemma}

\title{Tensor Evolution: A Framework for Fast Evaluation of Tensor Computations using Recurrences}

\author[1]{Javed Absar}
\author[2]{Samarth Narang }
\author[3]{Muthu Baskaran }
\affil[1]{Qualcomm Technologies International}
\affil[2]{Qualcomm Technologies, Inc.}
\affil[3]{Qualcomm Technologies, Inc.}

\begin{document}
\maketitle

\begin{abstract}
This paper introduces a new mathematical framework for analysis and optimization of tensor
expressions within an enclosing loop. Tensors are multi-dimensional arrays of values.
They are common in high performance computing (HPC)  and machine learning
\cite{TensorComprehension} domains. Our framework extends Scalar Evolution
\cite{scev_j} -- an important optimization pass implemented in both LLVM and GCC -- to tensors.
Scalar Evolution (SCEV) relies on the theory of `Chain of Recurrences' \cite{Zima} for its
mathematical underpinnings. We use the same theory for \textbf{Tensor Evolution} (TeV).
While some concepts from SCEV map easily to TeV -- e.g. element-wise operations;
tensors introduce new operations \cite{Attention, mlir,TVM,Glow} such as concatenation,
slicing, broadcast, reduction, and reshape which have no equivalent in scalars and SCEV.
Not all computations are amenable to TeV analysis but it can play a part in the optimization
and analysis parts of ML and HPC compilers. Also, for many mathematical/compiler ideas,
applications may go beyond what was initially envisioned, once others build on it and
take it further. We hope for a similar trajectory for the tensor-evolution concept.
\end{abstract}

\section{Introduction}
In this work, we introduce \textbf{Tensor Evolution} (TeV), a framework designed for analysis and optimization
 of tensor computations within loops. TeV provides a mathematical framework and mechanism to trace tensor
  values as they evolve over iterations of the enclosing loops. This approach enables optimizations that
   simplify loop-carried tensor computations to reduce computational overhead.

As machine learning (ML) models grow in complexity and scale, the demand for efficient computation has
spurred the development of specialized ML compilers, such as MLIR\cite{mlir}, Glow\cite{Glow}, and
TVM\cite{TVM}. These compilers convert computation in frameworks such as PyTorch and TensorFlow into
IR (intermediate representation) graphs where nodes represent operations
(e.g., matrix multiplications, additions, etc.), and edges represent tensors.
While traditional compiler optimizations, such as Constant Propagation, Loop Strength Reduction,
and Loop Invariant Code Motion have been somewhat adapted to these graph compilers,
more advanced transformations for tensor operations remain unexplored. Notably,
scalar evolution which is widely successful in traditional compilers such as LLVM for
analyzing scalar variables in loops lacks a counterpart in tensor operations within deep learning models or HPC. 

The paper is organized as follows: To explain TeV, we first introduce the basic concepts of recurrences
 and SCEV. Then, we present a formal definition of TeV. We outline a set of lemmas (re-write rules)
  that simplify TeV expressions. Next, we provide a worked-out example. Finally, we survey related
   work and conclude the paper.

Please note that part of this work was previously presented at the LLVM Developer
Summit \cite{tev_j} in the form of `Technical Talk`.

\section{Chain of Recurrences and Scalar Evolution}
Consider the function $S(n) = 1+2+3...+n$. We know that $S(n)=n(n+1)/2$, and this closed form can help
 calculate $S$ at any point, bypassing the need for a long chain of sequential additions.
  `Chain of Recurrences' (CR) is essentially a more complex algebraic generalization of this
concept \cite{ZimaRealPaper}.CRs are a compact representation of a series of computations. 
They provide a set of lemmas to simplify CR expressions and offer a closed-form function to evaluate
 these representations at any fixed point. Scalar Evolution (SCEV), implemented in LLVM and GCC,
  incorporates a simplified form of CR. To understand CR and SCEV, let us consider an example.


% ---    SCEV code examples  ---
\begin{verbatim}
    foo(...) { 
      int f = k0;
      for (int i = 0; i < n; ++i) {
         ... 
         f = f + k1;
      }
      ... = f // loop exit  value of `f`
\end{verbatim}

Here the value of $f$ can be expressed as a `basic recurrence'($BR$) $f = \{k_0, +, k_1\}_i$. 
In this $BR$ notation, $f$ has initial value $k_0$ and in each iteration of the loop with 
induction variable $i$, $f$ increases by $k_1$. In compiler terminology, $i$ is a basic 
induction variable, while $f$ is a derived induction variable. To help appreciate the algebraic
 power of recurrences, we show a few lemmas on basic recurrence.
    
% ---  Basic Recurrence Lemmas ---

\section{Basic Recurrence Lemmas}
Let $f = \{a, \odot, b\}$ represent a basic recurrence, where $\odot$ denotes an operator (e.g., addition or multiplication), and let $c$ be a constant. The following lemmas express  properties of basic recurrences:
    
    \begin{lemma}[Addition of a Constant]
    For a constant $c$ added to a basic recurrence:
    \[
    c + \{a, +, b\} \implies \{c + a, +, b\}.
    \]
    \end{lemma}
    
    \begin{lemma}[Multiplication by a Constant]
    For a constant $c$ multiplied with a basic recurrence:
    \[
    c \cdot \{a, +, b\} \implies \{c \cdot a, +, c \cdot b\}.
    \]
    \end{lemma}
    
    \begin{lemma}[Exponentiation by a Constant]
    For a constant $c$ raised to the power of a basic recurrence:
    \[
    c^{\{a, +, b\}} \implies \{c^a, \cdot, c^b\}.
    \]
    \end{lemma}
    
    \section*{Combining Basic Recurrences}
    
    Let $f = \{a, \odot, b\}$ and $g = \{c, \odot, d\}$ represent two basic recurrences. It can be proven that:
    
    \begin{lemma}[Addition of Two Recurrences]
    For the addition of two basic recurrences:
    \[
    \{a, +, b\} + \{c, +, d\} \implies \{a + c, +, b + d\}.
    \]
    \end{lemma}
    
    \begin{lemma}[Multiplication of Two Recurrences]
    For the multiplication of two basic recurrences:
    \[
    \{a, +, b\} \cdot \{c, +, d\} \implies \{ac, +, \{a, \odot, b\}d + \{c, \odot, d\}b + bd\}.
    \]
    \end{lemma}
    
    \section*{Chain of Recurrences}    
    A chain of recurrences (CR) is defined recursively as a function over $N$ with
     constants $\{c_0, c_1, \dots, c_{k-1}\}$, a function $f_k$, and an operator $\odot$ (either $+$ or $\cdot$). 
    
    \begin{lemma}[Chain of Recurrences]
    A chain of recurrences $\phi(i)$ is defined as:
    \[
    \phi(i) = \{c_0, \odot, \{c_1, \odot, c_2, \dots, \odot, f_k\}\}(i).
    \]
    \end{lemma}
    
    In practice, the nested braces are omitted, and the chain of recurrences
    simply written out as (LLVM SCEV prints out in same format):
    \[
    \phi = \{c_0, \odot, c_1, \odot, c_2, \dots, \odot, f_k\}.
    \]
    
    Consider the CR $\phi = \{7, +, 6, +, 10, +, 6\}$. This CR can be simplified to the closed expression:
    \[
    \phi(i) = i^3 + 2i^2 + 3i + 7.
    \]
    
    
This closed expression can be derived by applying the recurrence lemmas outlined
above and is described in detail in \cite{ZimaRealPaper, Zima, OlafRecurrence}.
The application of chains of recurrences in LLVM, including their usage for 
induction variables, is discussed in \cite{scev_j}.
    
    % ---  Tensor Evolution   ---
    \section{Tensor Evolution}
    Both Scalar Evolution and Tensor Evolution are founded on the principle of `compact representation',
     which enables computational reducibility \cite{Wolfram}—the ability to compute the value at a distant
      iteration point directly, without sequentially processing each iteration of the loop.
    
    \subsection{Code Example}
    Consider the following code -    
    \begin{verbatim}
        def forward(a: torch.Tensor, x: torch.Tensor) -> torch.Tensor:
            for _ in range(15):
                x = a + x 
            return x
    \end{verbatim}
    
    
    By analyzing the code and understanding the loop's behavior, we observe that the
     return (final) value is  $a+15x$, where $x$ is the initial input tensor value.
      A tensor can be viewed as a collection of scalar values. Thus, we can see the 
      connection between what we refer to as \textit{Tensor Evolution} and the concepts discussed
       in the previous section—\textit{Scalar Evolution} and \textit{Chain of Recurrences}.

       
    % -- Tensor Evolution - Basic Formulation ---
    \subsection{Basic Formulation}    
    \begin{figure}[h!]
    \centering
    \includegraphics[width=0.75\linewidth]{evol_basic.png}
    \caption{Basic Tensor Evolution}
    \label{fig:basic_TEV}
    \end{figure}
 

Given a constant or loop-invariant tensor $T_c$,  a function $\tau_1$ over the natural
numbers $N$ that produces a tensor of same shape as $T_c$, and an element-wise operator
$\odot$, then the basic evolution of the tensor value represented by tuple
$\tau=\{T_c, \odot, \tau_1\}$ is defined as a function $\tau(i)$ over $N$
(see Fig. \ref{fig:basic_TEV}):
    
\begin{equation}
\{T_c, \odot, \tau_1\}(i) = T_c \odot \tau_1(1) \odot \tau_1(2) ... \odot \tau_1(i)
\end{equation}
    
    For instance, given a zeros tensor $T_0$, a ones tensor $T_1$, and the operator `+'
     representing tensor addition, the tensor evolution expression $\tau=\{T_0, +, T_1\}$
      produces the sequence of tensors $T_0$, $T_1$, ..., $T_k$ for subsequent values
       of $\tau(i)$. Note in this case, the function $\tau_1(i) = T_i, \forall i \in N$.
    
    Next, given loop-invariant tensors $T_{c_0}, T_{c_1}, T_{c_2}, ...,T_{c_{i-1}}$,
     a function $\tau_k$ defined over $N$, and operators $\odot_{1}, \odot_{2}, ..., \odot_k$,
      a chain of evolution of the tensor value is represented by the tuple 
    
    \begin{equation}
    \tau = \{T_{c_0}, \odot_{1},T_{c_1}, \odot_{2}, ..., \odot_{k}, \tau_{k} \}    
    \end{equation}
    
    which is defined recursively as 
    \begin{equation}
    \tau(i) = \{T_{c_0}, \odot_{1}, \{T_{c_1}, \odot_{2}, ..., \odot_{k}, \tau_{k} \} \}(i)
    \end{equation}
    
    The operators $\odot_{i}$ could all be the same (e.g., $+$ or $*$) or different. Additionally,
     if we drop the braces and understand that the application of evolution proceeds 
     from right to left, the chain of evolution can be rewritten simply as: 
    
    \begin{equation}
    \tau(i) = \{T_{c_0}, \odot_{1}, T_{c_1}, \odot_{2}, ..., \odot_{k}, \tau_{k} \}(i)
    \end{equation}
    
    % -- Tensor Evolution Lemmas --
    \section{Tensor Evolution Lemmas}
    Next, we develop a set of lemmas for Tensor Evolution. These lemmas are algebraic 
    properties (re-write rules) designed to simplify computations involving tensor evolution expressions.
    
    \iffalse
    \begin{figure}[h!]
    \centering
      \includegraphics[scale=0.75]{images/evol_add_basic.png}
      \caption{Adding a loop invariant tensor `K' to a TeV expression}
    \end{figure}
    \fi

    % ADD CONSTANT TO TeV    
    % \setcounter{lemma}{0}
    \begin{lemma}{Add or multiply TeV by a loop invariant tensor}
    \label{lem::evol_add_basic}
    \newline Let $\{A,\odot,\tau\}$ be a tensor evolution expression where $\odot$ is either element wise add, subtract or multiply (see Fig. \ref{lem::evol_add_basic}).
    And, let $K$ be a loop invariant tensor. Then, 
    \[
    K+\{A, +, \tau \} \Rightarrow \{K+A, +, \tau \}
    \]
    \[
    K *\{A,+,\tau\} \Rightarrow \{K*A,+,K*\tau \}
    \]
    \end{lemma}
    
    
    % RESHAPE LEMMA
    \begin{lemma}{Reshape of tensor evolution expression}
    \label{lem::Reshape}
    \newline Let $\{A,\odot,\tau\}$ be a tensor evolution expression where $\odot$ is an element-wise operator.
    Let $\mathcal{R}(A)$ represent the reshaping of the tensor 
    A, which includes the permutation of dimensions and the ordinary transpose.
    Then,
    \[
     \mathcal{R}(\{A, \odot, \tau \}) 
     \Rightarrow
     \{\mathcal{R}(A), \odot, \mathcal{R}(\tau)\}
    \]
    \end{lemma}
    
    % SLICE LEMMA
    
    
    \begin{lemma}{Slice of tensor evolution expression}
    \label{lem::Slice}
    \newline Let $\{A,\odot,\tau\}$ be a tensor evolution expression where $\odot$ is an element-wise operator.
    Let $\mathcal{S}(A)$ represent slicing of the tensor, where slicing selects a subset of elements to form the resulting tensor.
    Then,
    \[
     \mathcal{S}(\{A, \odot, \tau \}) 
     \Rightarrow 
     \{\mathcal{S}(A), \odot, \mathcal{S}(\tau)\}
    \]
    \end{lemma}
    
    % CONCAT LEMMA
    
    \begin{lemma}{Concatenation of TeV expressions}
    \label{lem::Concat}
    \newline Let $\{A,\odot,\tau_1\}$ and $\{B,\odot,\tau_2\}$ be two tensor evolution expressions
    where $\odot$ is an element-wise operator.
    Let $\mathcal{C}(T_1,T_2)$ represent concatenation of two tensors.
    Then,
    \[
     \mathcal{C}(\{A,\odot,\tau_1\}, \{B,\odot,\tau_2\})
     \Rightarrow 
     \{\mathcal{C}(A,B),\odot,\mathcal{C}(\tau_1,\tau_2)\} 
    \]
    \end{lemma}
    
    % ADD TWO TeVS TO SSA VALUE    
    \begin{figure}[h!]
    \centering
      \includegraphics[width=0.75\linewidth]{evol_add_chain.png}
      \caption{Adding two TeV expressions}
      \label{fig::evol_add_chain}
    \end{figure}
    

    
    \begin{lemma}{Adding two TeV Expressions}
    \label{lem::evol_add_chain}
    \newline Let $\{A,\odot,\tau_1\}$ and $\{B,\odot,\tau_2\}$ be two tensor evolution expressions (see Fig. \ref{fig::evol_add_chain}). Then, 
    \[
    \{A, +, \tau_1 \} + \{B, +, \tau_2\} \Rightarrow \{A+B, +, \tau_1 + \tau_2\}
    \]
    \end{lemma}
    
    % Multiple TWO TeVS 
    
    \begin{lemma}{Multiply two TeV Expressions}
    \label{lem::evol_mul_chain}
    \newline Let $\{A,\odot,\tau_1\}$ and $\{B,\odot,\tau_2\}$ be two tensor evolution expressions. Then, 
    \[
    \{A, +, \tau_1 \} * \{B, +, \tau_2\} \Rightarrow \big\{AB, +, \tau_1\{B, +, \tau_2\} + \tau_2\{A, +, \tau_1 \} + \tau_1\tau_2\big\}
    \]
    \end{lemma}
    
    % ---    Evolution of chain  ---
   
    \begin{lemma}{TeV input to a loop variant tensor}
    \label{lem::evol_mem_chain}
    \newline Let $X$ be a loop variant tensor with an initial value of $A$. At each iteration,
     its value evolves as $A + \{B,\odot_2,\tau_2\}$, meaning the TeV $\{B,\odot_2,\tau_2\}$ is 
     added to the current value of $X$. This is represented as 
    $\{B, +, \tau_2\} \xrightarrow[+]{} A$. Then,
    \newline
    \[
    \{B, +, \tau_2\} \xrightarrow[+]{} A \Rightarrow \{A, +, B, +, \tau\}
    \]
    In other words, this results in a chain of evolution of depth two.
    \end{lemma}
    
    There are additional lemmas based on variations in composition and operators beyond those
     discussed above. However, the lemmas presented so far are already sufficient to address
      a range of interesting problems.
    % ---  Table of  TeV Lemmas  ---
    \begin{table}[h!]
    \centering
    \begin{tabular}{|c|c|c|}
    \hline
    \textbf{Operator} & \textbf{Tensor Expression} & \textbf{Rewrite-rule} \\ \hline
    reshape & $\mathcal{R}(\{A, +, \tau\})$ & $\{\mathcal{R}(A), +, \mathcal{R}(\tau)\}$ \\ 
            & $\mathcal{R}(\{A, \ast, \tau\})$ & $\{\mathcal{R}(A), \ast, \mathcal{R}(\tau)\}$ \\ \hline
    slice & $\mathcal{S}(\{A, +, \tau\})$ & $\{\mathcal{S}(A), +, \mathcal{S}(\tau)\}$ \\ 
          & $\mathcal{S}(\{A, \ast, \tau\})$ & $\{\mathcal{S}(A), \ast, \mathcal{S}(\tau)\}$ \\ \hline
    broadcast & $\mathcal{B}(\{A, +, \tau\})$ & $\{\mathcal{B}(A), +, \mathcal{B}(\tau)\}$ \\ 
              & $\mathcal{B}(\{A, \ast, \tau\})$ & $\{\mathcal{B}(A), \ast, \mathcal{B}(\tau)\}$ \\ \hline
    concat & $\mathcal{C}(\{\{A, +, \tau_1\}, \{B, \odot, \tau_2\}\})$ & $\{\mathcal{C}(A, B), +, \mathcal{C}(\tau_1, \tau_2)\}$ \\ 
           & $\mathcal{C}(\{\{A, \ast, \tau_1\}, \{B, \odot, \tau_2\}\})$ & $\{\mathcal{C}(A, B), \ast, \mathcal{C}(\tau_1, \tau_2)\}$ \\ \hline
    log & $\log(\{A, \ast, \tau\})$ & $\{\log(A), +, \log(\tau)\}$ \\ \hline
    exponent & $\exp(\{A, +, \tau\})$ & $\{\exp(A), \ast, \exp(\tau)\}$ \\ \hline
    add const & $K + \{A, +, \tau\}$ & $\{K + A, +, \tau\}$ \\ \hline
    multiply by const & $K \ast \{A, +, \tau\}$ & $\{K \ast A, +, K \ast \tau\}$ \\ \hline
    add two TeV & $\{A, +, \tau_1\} + \{B, +, \tau_2\}$ & $\{A + B, +, \tau_1 + \tau_2\}$ \\ \hline
    inject TeV into TeV & $\{B, +, \tau_2\} \to A$ & $\{A, +, B, +, \tau\}$ \\ \hline
    \end{tabular}
    \caption{Useful Lemmas (re-write Rules) for Tensor Evolution Evaluation.}
    \label{table:tev}
    \end{table}
    
    \section{Application of Tensor Evolution}
    
    Consider the following code which operates on tensors -
    \begin{verbatim}
        def forward(a: torch.Tensor, x: torch.Tensor) -> torch.Tensor:
          y = torch.zeros_like(x)
          for _ in range(15):
            x = x + a
            z = x[1,:]
            y = y + z
          return y
    \end{verbatim}
    The interpretation of how the tensor values are evolving across iterations of the loop is shown in \ref{fig:TEV_Code_Example}. 
    
    \begin{figure}[h!]
    \centering
      \includegraphics[width=\linewidth]{simplify_ex.png}
      \caption{Calculate tensor evolution expressions of given code.}
      \label{fig:TEV_Code_Example}
    \end{figure}
    

    Now, using the re-write rules of Tensor Evolution that were laid out before, we can construct and simplify the TeV expressions.
    
    \begin{equation}
    \begin{cases}
    Y_k = \left\{  Y_0, +, \mathcal{S}( \left\{ X_0, +, A\right\})    \right\}_k \\
    Y_k= \left\{  Y_0, +,  \left\{\mathcal{S}(X_0), +, \mathcal{S}(A)\right\}    \right\}_k \\
    Y_k= \left\{  Y_0, +,  \mathcal{S}(X_0), +, \mathcal{S}(A) \right\}_k \\
    Y_k= Y_0 + k * \mathcal{S}(X_0) + k*(k+1)/2 *  \mathcal{S}(A)    
    \end{cases}
    \end{equation}
    
    Based on the above simplifications (rewrites) the same code can be re-written now as 
    \begin{verbatim}
    def forward(self, a: torch.Tensor, x:torch.Tensor)->torch.Tensor: 
      y = torch.zeros_like(x)
      return y + 15*x[1,:] + 15*(15+1)/2*a[1,:]
    \end{verbatim}
    
    \section{Background Art and Related Work}
    Tensor Comprehensions (TC) \cite{TensorComprehension} is another representation for
     computations on tensors, based on generalized Einstein notation. For example, let $X$ and $Y$
      be two input tensors, and $Z$ the output tensor.  The TC statement $Z(r) += X(r,c) * Y(c)$ 
      introduces two index variables, $r$ and $c$. Their ranges are inferred from their indexing 
      of tensors $X$ and $Y$. Since $c$ appears only on the right-hand side, the computation reduces
       over $c$ when storing values into $Z$. TC provides a compact representation of computations. 
       However, it does not address the \textit{computational reducibility} aspect of the problem.
        In contrast, SCEV and TeV include algebraic rewrite rules that simplify computations. 
        While not all computations are reducible, areas of reducibility often exist in general code,
         and both TeV and SCEV are specifically designed to identify and simplify these reducible computations.
    
    As mentioned earlier, Tensor Evolution is based on the concept of Chain of Recurrences (CR).
     The paper by \cite{Zima} on CR develops rewrite rules for long chains and complex operations,
      including trigonometric functions and exponents of exponent operations. Compilers such as LLVM 
      and GCC implement CR as Scalar Evolution (SCEV), but only for simple operations like addition
       and subtraction, and limited to two levels of recurrence. While extending beyond this is 
       mathematically possible, it is not practical for general-purpose compilers and has limited
        usability in real-world applications.
     
    Many Domain-Specific Languages (DSLs) and compilers have been developed for machine 
    learning (ML) and high-performance computing (HPC). DSLs leverage domain-specific constructs
     to capture parallelism, locality, and other application-specific characteristics. 
     An example is Halide \cite{HALIDE}, a DSL for image processing. Halide defines its inputs
      as images over an infinite range, whereas Tensor Comprehensions (TC) sets fixed sizes for
       each dimension using range inference. In ML applications, most computations are performed 
       on fixed-size tensors with higher temporal locality than images. Additionally, TC is less
        verbose than Halide, as Halide carries the syntactic burden of pre-declaring stage names 
        and free variables.
    
    The polyhedral framework is another abstraction used for the analysis and transformation of loop nests.
     One of the first high-level language polyhedral compilers was the R-Stream 3.0 Compiler \cite{Lethin2008RStream3C}
      and several tools and libraries have been developed to harness its benefits, such as GCC (Graphite) and LLVM
       (Polly). Polyhedral techniques have also been adapted for domain-specific purposes. Examples include 
       PolyMage \cite{Polymage}, designed for image processing pipelines, and PENCIL \cite{PENCIL}, 
       an approach for constructing parallelizing compilers for DSLs.


    \section{Conclusion}       
    In this paper, we presented a new mathematical framework—Tensor Evolution (TeV).
     TeV extends the theory of Chain of Recurrences (CR) to the evolution of tensor values
      within loops found in machine learning (ML) and high-performance computing (HPC) graphs.
       While CR was originally developed for scalar values and successfully implemented in LLVM and GCC as Scalar Evolution (SCEV), TeV introduces a novel approach by generalizing these concepts to tensors.
    We demonstrated the usefulness of TeV in specific contexts through illustrative code examples.
     The mathematical framework we propose has potential applications in general ML and HPC
      scenarios when further extended and applied creatively. For example, SCEV is widely
       used in optimizations like loop strength reduction (LSR) and loop-exit value computation.
        Similarly, TeV could be leveraged in cases where parts of a tensor evolve predictably,
         enabling the effective application of recurrence lemmas to optimize computations.

\bibliography{reference}{}
\bibliographystyle{unrst}
    
\end{document}

\end{document}
