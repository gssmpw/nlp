\section{Related Work\label{sec_related}
}
\subsection{Deep Learning-based Time-series Prediction}
Advancements in deep learning have led to the development of time-series forecasting models, such as LSTM \cite{Hochreiter_1997} and Transformer \cite{Vaswani_2017}, that provide more accurate predictions than other techniques. The method of calibrating time-series data is similar to that of forecasting time-series. Therefore, it is essential to evaluate existing deep learning-based forecasting models to identify an appropriate technique for calibrating data acquired from low-cost sensors.


Several studies have aimed to minimize the necessary hardware resources for prediction. In particular, variations of the LSTM model, such as GRU \cite{Cho_2014} and C-LSTM \cite{Wang_2018}, have been proposed to improve the speed of predictions by modifying the structure of the traditional LSTM. Other methods, such as those proposed by \cite{Hoyer_2022} and \cite{Chen_2022}, have been developed to enhance efficiency without changing the LSTM structure. For Transformer-based models, new architectures such as Informer \cite{Zhou_2020}, Reformer \cite{Kitaev_2020}, Performer \cite{Choromanski_2021}, Pyraformer \cite{Liu_2021}, and Ecoformer \cite{Liu_2022} have been proposed to improve the time and space complexity of the attention mechanism. These studies enable the calibration of low-cost sensors using deep learning models on reduced hardware resources. However, compared to the linear model, the resource utilization is considerably higher, but the improvement in calibration accuracy is not significant.

Some studies, such as Phased LSTM \cite{Neil_2016} and THP \cite{Zuo_2020}, have focused on event time-series. Forecasting events is crucial since the data collected from sensors exhibits non-periodic characteristics. However, they are not suitable for low-cost sensor calibration due to high computational complexity.

\subsection{Hybrid Machine Learning}
In the field of machine learning, the term \textit{hybrid machine learning} is used in two distinct contexts. First, it refers to a methodology used to construct a model by simultaneously combining two or more techniques or models, also known as ensemble learning \cite{Oliveira_2014}. This technique exhibits powerful performance for task prediction \cite{Stefenon_2022}, and various ensemble methods are employed for predictions \cite{Gastinger_2021, Zhang_2022, Cawood_2021}. These hybrid models are mainly based on machine learning or statistical techniques \cite{Gastinger_2021, Zhang_2022}, but ensembles using deep learning models are also conducted \cite{Cawood_2021}. Their hybrid methods demonstrate higher accuracy compared to using a single model, however, the use of multiple models leads to slower inference speeds and more hardware utilization, which poses challenges for use in IoT-controlled devices.

Second, hybrid machine learning is a technology that enables the creation of models applicable in multiple environments \cite{Liu_2020}. Most studies focus on reducing model complexity \cite{Liu_2020, Asutkar_2023} or excluding certain operations to support inference on various hardware platforms \cite{David_2020, Aydonat_2017}. However, this lightweight approach does not improve inference speed to the same extent as the linear model, as both training and inference methods remain unchanged. In contrast, SenDaL operates two separate methods, namely bottom-up training and top-down inference, resulting in a significant improvement in latency and energy consumption during the inference process.


\subsection{Data-driven Robotics and Soft-Sensors}
The use of data-driven approaches in robotics has been gradually increasing, since they have been shown to improve performance \cite{Barclay_2022}, such as enhancing the capability of robots to acquire knowledge \cite{Serkan_2019} or improve the accuracy of tasks, such as robot grasping \cite{Gupta_2018, Kleeberger_2020}. However, these methods cannot be applied to low-cost sensors, as they are primarily designed for use in high-cost sensors or robot components.


In soft-sensors, data-driven approaches are used in a wide range of applications, from industry-scale environments to various tasks suitable for IoT environments, using diverse deep learning models \cite{Ke_2017, LoyBenitez_2020, Sun_2021}. Various deep learning models have been used for soft-sensor \cite{Sun_2021}, but LSTM-based models are mainly used in their domains \cite{Ke_2017, LoyBenitez_2020}. Moreover, research has focused on monitoring the sensor reliability or calibrating data based on multiple low-cost sensors \cite{Cheng_2020, Cheng_2022, Zuniga_2022}. Nevertheless, these methods are not sufficiently simple for increasing the accuracy of a single low-cost sensor and do not improve the latency because they do not consider real-time inference. In summary, there is currently no research that comprehensively considers all of the accuracy, execution speed, and energy consumption of a model using only a single low-cost sensor.