\documentclass[letterpaper, 10pt, conference]{ieeeconf}

\makeatletter
\let\IEEEproof\proof
\let\IEEEendproof\endproof
\let\proof\@undefined
\let\endproof\@undefined
\makeatother

\IEEEoverridecommandlockouts  % used for \thanks

\usepackage{cite}
\usepackage{amsmath,amssymb,amsfonts}
\usepackage{algorithmic}
\usepackage{graphicx}
\usepackage{textcomp}
\usepackage{xcolor}
\usepackage{booktabs}
\usepackage{multirow}
\usepackage{xspace}  % used for \xspace in macros (smart spacing)
\usepackage{amsthm}
\usepackage{amsfonts}
\usepackage{mathtools}  % used for \DeclarePairedDelimiter
\usepackage{bm}  % used for bold and prettier greek letters with \bm
\usepackage{paralist}
\usepackage{soul}
\usepackage[nolist,nohyperlinks]{acronym}
\begin{acronym}
\acro{gan}[GANs]{Generative Adversarial Networks}
\acro{rl}[RL]{Reinforcement Learning}
\acro{pae}[PAE]{Periodic Autoencoder}
\acro{fld}[FLD]{Fourier Latent Dynamics}
\acro{ppo}[PPO]{Proximal Policy Optimization}
\acro{fft}[FFT]{Fast Fourier Transform}
\acro{pca}[PCA]{Principal Component Analysis}
\acro{dfm}[DFM]{Deep Fourier Mimic}
\acro{dof}[DoF]{Degrees of Freedom}
\acro{mlp}[MLPs]{Multi-Layer Perceptrons}
\end{acronym}


\usepackage[hidelinks,bookmarks]{hyperref} 

\usepackage{multicol}
\usepackage{citesort}
\usepackage[textsize=tiny]{todonotes}
\usepackage{siunitx}
\usepackage{relsize}
\theoremstyle{definition}
\newtheorem{definition}{Definition}
\newtheorem{theorem}{Theorem}
\newtheorem{lemma}{Lemma}
\newtheorem{proposition}{Proposition}
\theoremstyle{remark}
\newtheorem{remark}{Remark}

\newcommand{\vectorproj}[2][]{\textit{proj}_{\vect{#1}}\vect{#2}}



%%% METADATA %%%%%%%%%%%%%%%%%%%%%%%%%%%%%%%%%%%%%%%%%%%%%%%%%%%
% TITLE
\title{\LARGE \bf Safe Quadrotor Navigation using Composite Control Barrier Functions}
%Safe Geometric Control of Multirotors in highly cluttered Environments}
% AUTHOR
\author{Marvin Harms, Martin Jacquet, Kostas Alexis
%=======
    % AFFILIATIONS
	\thanks{Autonomous Robots Lab, Norwegian University of Science and Technology (NTNU), Trondheim, Norway,
    {\tt \footnotesize
        \href{mailto:marvin.c.harms@ntnu.no}{marvin.c.harms@ntnu.no}}
    }
    \thanks{This work was supported by the European Commission Horizon Europe grants DIGIFOREST (EC 101070405) and SPEAR (EC 101119774).}
}

\begin{document}
\maketitle

\begin{abstract}
This paper introduces a safety filter to ensure collision avoidance for multirotor aerial robots. The proposed formalism leverages a single Composite Control Barrier Function from all position constraints acting on a third-order nonlinear representation of the robot's dynamics. We analyze the recursive feasibility of the safety filter under the composite constraint and demonstrate that the infeasible set is negligible. The proposed method allows computational scalability against thousands of constraints and, thus, complex scenes with numerous obstacles. We experimentally demonstrate its ability to guarantee the safety of a quadrotor with an onboard LiDAR, operating in both indoor and outdoor cluttered environments against both naive and adversarial nominal policies.
\end{abstract}


% =============================================================================================
% =============================================================================================
% =============================================================================================
\section{Introduction}

\section{Introduction}

\begin{figure*}
    \centering
    \includegraphics[width=\textwidth]{figures/Introduction.pdf}
    \caption{Showing the novel problem statement applied to traffic prediction use case. Multiple unstructured observations from the past are used to reconstruct a hidden traffic state from which a full traffic state is forecast with a set of query locations. }
    \label{fig:intro}
\end{figure*}

% Was sagen denn die anderen warum Traffic Prediction gut ist? 
Forecasting the traffic in the near future is an important task for city management.
Data from the near past is used to predict future traffic states with spatio-temporal Graph Neural Networks \cite{bui22}.
Accurate prediction provides the opportunity to optimize traffic flow, reduce traffic jams and increase air quality \cite{Po19}.

% Wieso ist Sparsity in allen Dimensionen wichtig.
While traffic prediction relies on the availability of data from traffic sensors, there exists a plethora of reasons why sensors may stop working temporarily, such as simple errors, energy saving, or overloaded communication systems.
Considering small- or medium-sized cities, the coverage of sensors may be low because the sensors are too expensive or not available.
Also, the sensors are typically static and do not adapt to changes in the traffic flow (e.g. caused by a construction site), which motivates moving sensors that for example could be mounted on cars. 
However, both missing and moving sensors introduce sparsity, since measurements may not be available for all locations at all times.
This sparsity must be explicitly addressed in traffic prediction for a realistic application scenario, which is illustrated in figure \ref{fig:intro}.
From one hour of data on Sunday morning, only few observations of the traffic state are available at each timestep.
The number of observations may differ throughout the observed time and the observation itself can be distributed arbitrarily in the city. 
We assume a relatively low number of sensors to account for resource saving and sensor failure in our proposed framework SUSTeR.
The task is to predict the dense traffic state one timestep after the observations at all possible sensor locations.
We study this problem on the traffic dataset Metr-LA and PEMS-BAY to test our assumption that only a fraction of the sensor values would be enough for good predictions.
By modifying an existing traffic dataset, we are able to compare our results from very sparse observations to the bottom line with all information available.
A successful study will provide insights in how sensors in new cities can be reduced before installing them and further mobile sensors would save more resources and are able to adapt to new traffic situations.
We argue that in order to be adaptable to other cities and changes in traffic flows, prior information like the road network should be neglected and just the sparse observations considered.
This comes with the added benefit of making our solution applicable in regions where no openly available road network is maintained or pathways change frequently (e.g. flood areas, animal observations). 


The aforementioned problem is novel and more challenging than the commonly considered traffic prediction problem, since there exist very few observations in each input sample.
Current works for the traffic prediction problem do not consider any missing values. \cite{Li2021, Shao22}
A common method among state of the art approaches is the usage of Graph Neural Networks on graphs that model the sensor network \cite{bui22}.
The values of a sensor are applied to the same graph node for each timestep which prohibits any non-stationary sensors . 
With fixed sensor locations, the resulting sensor network is highly correlated with the road network.
Streets connecting two intersections with sensors should be also an interesting point for correlations in the sensor network.
However, variable observations and high temporal sparsity rates can not be modeled adequately in a static network.
We show in our experiments that the road network has only a small influence on the traffic predictions.

Besides the traffic prediction for future timesteps, some works explore the field of traffic speed imputation \cite{Cini22, Cuza22} where missing sensor values are predicted.
But the amount of missing values is assumed to be at most 80\%, which on average are still over 40 given sensors in each timestep in the Metr-LA dataset with a total of 207 sensors.
We consider up to 99.9\% missing values which are on average 2.4 observations in each timestep that are used as input.
Such high sparsity rates drastically decrease the chance that multiple values are present in one input sample from the same sensor location, which makes it challenging to recognize and learn temporal correlations for each location on its own.

High sparsity rates (>95\%) result in few sensor values, but if a reconstruction of the traffic state would be possible, we question if spatio-temporal graphs require nodes for each sensor.
In SUSTeR we utilize only a small amount of graph nodes for the encoding of information and do not relate such nodes to the sensor network.
We call this the hidden graph (see figure \ref{fig:intro}), which is still able to reconstruct the complete traffic state.
Due to the reduced number of nodes SUSTeR achieves faster runtimes, as shown in the experiments.
This hidden graph is not embedded directly in the spatial domain, which is why the assignment of observations, as well as the querying of the future traffic, is done with an encoder and a decoder, implemented as neural networks.
The decoding from the hidden graph to future values depends on a set of query locations.
Figure \ref{fig:intro} shows the query locations as given from outside and in combination with the reconstructed traffic state the future values are predicted.

To construct the hidden graph we encode observations from each timestep into from multiple graphs, one for each timestep. 
The graphs are created in a residual style and information is added to the node embeddings from the previous timesteps.
We choose this method to incorporate all timesteps equally into the hidden state because the redundant information along the past is non-existing for high sparsity rates.
From the sequence of graphs where our framework inserted the observations step by step we apply STGCN \cite{Yu18}, an algorithm for traffic prediction to find and learn the spatio-temporal correlations on our small number of graph nodes.
The first future timestep of the STGCN is our hidden graph in which the traffic state is reconstructed. 

% Recent work has an implicit embedding of the graph nodes into the spatial domain as the assignment from the sensor to graph node is fixed one by one.
% Because the graph has the same structure as the road network spatio-temporal correlations can be learned between those sensors.
% We reduce the number of nodes and use a non-linear assignment learned data-driven from the observations.

We find in the experiments that SUSTeR outperforms the plain STGCN and modern traffic prediction frameworks like D2STGNN for high sparsity rates $(\geq 99\%)$.
This is equivalent to only $0.2$ to $2.4$ observation for each timestep on average.
SUSTeR uses fewer parameters than the baselines and can train faster and with less training data.
Our main contributions can be summarized as follows:
\begin{itemize}
    \item We introduce a sparse and unstructured variant of the traffic prediction problem with sparsity in all dimensions. The sensors report only a fraction of their values and are arbitrarily distributed in the spatial domain.
    \item We propose SUSTeR, a framework around the STGCN architecture, which maps sparse observations onto a dense hidden graph to reconstruct the complete traffic state.
    Our code is available at github.\footnote{https://github.com/ywoelker/SUSTeR}
    \item We conducts experiments that show that SUSTeR outperforms the baselines in very sparse situations ($\geq 95\%$) and has a competitive performance in low sparsity rates.
    % \item SUSTeR trains a third faster than the next competitor.
\end{itemize}


% =============================================================================================
% =============================================================================================
% =============================================================================================
\section{Preliminaries}\label{sec:preliminaries}
\subsection{Notation}
\begin{tabbing}
 \hspace*{2.2cm} \= \kill
  $\mathbf{x} \in \mathcal{X},  \mathbf{u} \in \mathcal{U}$ \>  state and input vectors \\[0.5ex]
  %\mathbf{u} \in \mathcal{U}$ \>  input vector \\[0.5ex]
  $\mathcal{L}_f h, \mathcal{L}_g h \mathbf{u}$ \>  Lie derivatives of $h$ along $f$, $g \mathbf{u}$ \\[0.5ex] 
  $\frac{d}{dt} V$, $\nabla_\mathbf{x} V$ \>  time derivative of $V$, gradient of $V$ w.r.t. $\mathbf{x}$ \\[0.5ex] 
  %$\nabla_\mathbf{x} V$ \>  gradient of $V$ w.r.t. $\mathbf{x}$ \\[0.5ex] 
  $\mathbf{a} \cdot \mathbf{b}$, $\mathbf{a} \times \mathbf{b}$ \>  dot product, cross product of $\mathbf{a}$ and $\mathbf{b}$ \\[0.5ex]  
  $\|\cdot\|$ \>  Euclidean norm \\[0.5ex] 
  $[\mathbf{a}]_{\times}$ \>  skew-symmetric matrix associated with $\mathbf{a}$ \\[0.5ex]
\end{tabbing}

\subsection{Control Barrier Functions}
Let us consider the general, control affine system

\small
\begin{equation}\label{system}
    \frac{d}{dt} \mathbf{x} = f(\mathbf{x}) + g(\mathbf{x}) \mathbf{u},
\end{equation}
\normalsize
where $\mathbf{x} \in \mathcal{X} \subseteq \mathbb{R}^n$ and $\mathbf{u} \in \mathcal{U} \subseteq \mathbb{R}^m$. We now consider a continuously differentiable function $h: \mathcal{X} \rightarrow \mathbb{R}$ with the property $\{\mathbf{x} | \frac{dh}{d\mathbf{x}}(\mathbf{x}) = 0\} \cap \{\mathbf{x} | h(\mathbf{x}) = 0\} = \emptyset$, which describes the superlevel set $\mathcal{X}_\text{safe} = \{\mathbf{x} \in \mathcal{X}  :  h(\mathbf{x}) \geq 0 \}$.

%\iffalse
\begin{definition}[Control Invariant Set]
    A set $\mathcal{X}_\text{safe}$ is a forward control invariant set for the system \eqref{system}, if for any $\mathbf{x}({t_0}) \in \mathcal{X}_\text{safe}$ there exists at least one input trajectory $\mathbf{u}(t) \in \mathcal{U}$ such that $\mathbf{x}({t}) \in \mathcal{X}_\text{safe} \quad \forall t\geq t_0$ under the system \eqref{system}.
\end{definition}
%\fi

\begin{definition}[Control Barrier Function \cite{ames2019control}]\label{def:cbf}
    Let $\mathcal{X}_\text{safe}$ be the $0$-superlevel set of a continuously differentiable function
    $h: \mathcal{X} \rightarrow \mathbb{R}$ with the property that $\mathcal{X}_\text{safe} = \{ \mathbf{x} | h(\mathbf{x})\geq 0 \}$. Then, $h$ is a \ac{cbf} if there exists an extended class $\mathcal{K}_\infty$ function $\alpha(\cdot)$ such that for the system \eqref{system} it holds:

    \small
    \begin{equation}\label{CBF_lie_condition}
        \underset{\mathbf{u} \in \mathcal{U}}{\text{sup}} [\mathcal{L}_f h(\mathbf{x}) + \mathcal{L}_g h(\mathbf{x}) \mathbf{u}] \geq -\alpha(h(\mathbf{x}))
    \end{equation}
    \normalsize
    for all $\mathbf{x} \in \mathcal{X}_\text{safe}$. Further, an input $\mathbf{u}$ is considered safe with respect to a valid CBF, if it satisfies \eqref{CBF_lie_condition}.
\end{definition}

From the above, it becomes clear that $\mathcal{X}_\text{safe}$ is a control invariant set for the system \eqref{system} subject to any control law satisfying \eqref{CBF_lie_condition}. See \cite{ames2016control} for a proof. Condition \eqref{CBF_lie_condition} requires a nonzero Lie derivative, e.g. $\mathcal{L}_g h(\mathbf{x}) \neq 0$ in general, restricting the use of \acp{cbf} to functions $h$ of relative degree~1. Systematic approaches to construct a CBF with relative degree greater than one include exponential control barrier functions \cite{ECBF}, high-order control barrier functions \cite{HOCBF} and backstepping control barrier functions \cite{BCBF}.

\subsection{Exponential Control Barrier Functions}
Consider a safety metric $h(\mathbf{x})$ of uniform relative degree $r\geq1$. Repeatedly differentiating $h(\mathbf{x})$ with respect to time results in terms $\mathcal{L}_g \mathcal{L}_f^{i} h(\mathbf{x})$ which are equal to zero for $i<r-1$ due to the relative degree assumption. However, by using negative constants (poles) $p_i<0$ and defining the series of functions $\nu_i: \mathcal{X} \rightarrow \mathbb{R}$ and corresponding superlevel sets $C_i$ for $i \in \{0,1, ...,r\}$ as

\small
\begin{align*}
    \nu_0(\mathbf{x}) &= h(\mathbf{x}), & C_0 &= \{ \mathbf{x} : \nu_0(\mathbf{x}) \geq 0 \}, \\
    \nu_1(\mathbf{x}) &= \dot{\nu}_0(\mathbf{x}) - p_1 \nu_0(\mathbf{x}), & C_1 &= \{ \mathbf{x} : \nu_1(\mathbf{x}) \geq 0 \}, \\
    &\vdots & &\vdots \\
    \nu_r(\mathbf{x}) &= \dot{\nu}_{r-1}(\mathbf{x}) - p_r \nu_{r-1}(\mathbf{x}), & C_r &= \{ \mathbf{x} : \nu_r(\mathbf{x}) \geq 0 \},
\end{align*}
\normalsize
it is shown in \cite{ECBF} that the following theorem holds

\begin{theorem}\cite{ames2019control}\label{thm1}
If $C_r$ is forward-invariant and $\mathbf{x}_0 \in \bigcap_{i=0}^r C_i$ then $\mathcal{C}_0$ is forward-invariant.
\end{theorem}
Note that Theorem \ref{thm1} additionally requires conditions of the initial state $\mathbf{x}_0$ to hold in addition to the invariance of $\mathcal{C}_r$ to ensure invariance of $\mathcal{C}_0$. If the function $\nu_r(\mathbf{x})$ is a \ac{cbf}, then $h(\mathbf{x})$ is said to be an exponential \ac{cbf} (ECBF).

\subsection{Composite Control Barrier Functions}
Enforcing multiple safety constraints simultaneously on a system can lead to practical challenges, especially if the number of constraints considered becomes large. In \cite{compositeCBFames}, a method to construct a single CBF as a logical AND-OR composition from multiple, different safety constraints through a softmin/softmax is described. This approach was also described in \cite{compositeCBFhoagg} for high-order systems with mixed relative degree. For the sake of brevity, we recite the method focusing on ECBFs. We consider a set of functions $\{h_i(\mathbf{x})\}_{i=1}^N$, where $h_i$ is an ECBF of relative degree $r_i$. Furthermore, we assume the safe set $\mathcal{S} = \bigcap_{i=1}^{N} \{\mathbf{x} : h_i(\mathbf{x}) \geq 0\}$ is nonempty. Defining the intermediate high-order functions

\small
\begin{equation}\label{ccbf1}
    \nu_{i,j+1} = \mathcal{L}_f \nu_{i,j} - p_{i,j} \nu_{i,j} ,
\end{equation}
\normalsize
where $\nu_{i,0} = h_i$, $j \in \{0,1, \hdots, r_{i-1} \}$ and $p_{i,j}<0$, we define the sets

\small
\begin{equation}\label{ccbf2}
    \mathcal{C}_{i,j} = \{\mathbf{x} : \nu_{i,j}(\mathbf{x}) \geq 0\} \quad \text{,} \quad C_i = \bigcap_{j=0}^{r_i-1} \mathcal{C}_{i,j}.
\end{equation}
\normalsize
By application of Theorem \ref{thm1} and Definition~\ref{def:cbf} we arrive at the following lemma:
\begin{lemma}\label{lemma1}
If $\mathbf{x}_0 \in \mathcal{C}_i$ and $\nu_{i,r_{i-1}}$ satisfies $\|\mathcal{L}_g \nu_{i,r_{i-1}}(\mathbf{x})\|>0~ \forall \mathbf{x} \in \mathcal{X}$, then the condition with $\alpha > 0$

\small
\begin{equation}\label{ccbf3}
    \mathcal{L}_f \nu_{i,r_{i-1}}(\mathbf{x}) + \mathcal{L}_g \nu_{i,r_{i-1}}(\mathbf{x}) \mathbf{u} \geq -\alpha \nu_{i,r_{i-1}}(\mathbf{x}) \quad \forall t \geq t_0
\end{equation}
\normalsize
implies $\mathbf{x} \in \mathcal{C}_i \quad~ \forall t \geq t_0$ and thus $h_i(\mathbf{x})\geq0$. Also, $\nu_{i,r_{i-1}}$ is a CBF of relative degree~1.
\end{lemma}

Our objective is to satisfy all constraints jointly, e.g. $\mathbf{x} \in \mathcal{S}$, which can be achieved by ensuring $\mathbf{x} \in \mathcal{C} = \bigcap_{i=i}^{N} \mathcal{C}_{i}$. In~\cite{compositeCBFames}, the intersection set $\mathcal{C}$ is compactly expressed through the $\min$ operation as $\{\mathbf{x} : \nu_{i,j}(\mathbf{x}) \geq 0\} = \{\mathbf{x} : \min_i \nu_{i,r_{i}-1}(\mathbf{x}) \geq 0\}$. Correspondingly, a new function using the soft minimum is defined in \cite{compositeCBFames} as a smooth under-approximation of the set $\mathcal{C}$ as

\begin{equation}\label{ccbf}
    h(\mathbf{x}) = - \frac{1}{\kappa} \log \sum_i e^{-\kappa \nu_{i,r_{i-1}}(\mathbf{x})},
\end{equation}
where we have $\{\mathbf{x} : h(\mathbf{x}) \geq 0\} \subseteq \mathcal{C}$ with the parameter $\kappa > 0$. The Lie derivatives of $h(\mathbf{x})$ are given by the functions

\small
\begin{equation}
    \mathcal{L}_f h(\mathbf{x}) = \sum_i \lambda_i(\mathbf{x}) \mathcal{L}_f h_i(\mathbf{x}) \text{,} \quad
    \mathcal{L}_g h(\mathbf{x}) = \sum_i \lambda_i(\mathbf{x}) \mathcal{L}_g h_i(\mathbf{x}),
\end{equation}
\normalsize
where $\lambda_i(\mathbf{x}) = e ^ {-\kappa (h_i(\mathbf{x}) - h(\mathbf{x}) )}$.

\begin{theorem}
The function given by \eqref{ccbf} is a CBF for the system \eqref{system} if and only if~\cite{compositeCBFames}

\small
\begin{equation}\label{theorem3_eq}
    \sum_i \lambda_i(\mathbf{x}) \mathcal{L}_g h_i(\mathbf{x}) = 0 \Longrightarrow \sum_i \lambda_i(\mathbf{x}) (\mathcal{L}_f h_i(\mathbf{x}) + \alpha  h_i(\mathbf{x})) \geq 0
\end{equation}
\normalsize
holds for $\alpha > 0$ and for all $\mathbf{x} \in \mathcal{X}$.
\end{theorem}

\begin{remark}\label{rmrk1}
By Lemma \ref{lemma1}, we have that $\nu_{i,r_{i-1}}(\mathbf{x})$ is a CBF and thus \eqref{theorem3_eq} trivially holds for $\lambda_p = 1$, $\lambda_i = 0 \quad \forall i \in \{1, \hdots ,N \} \setminus p$. This can be achieved by letting $\kappa \rightarrow \infty$ whenever $\nu_{p,r_{i-1}}(\mathbf{x}) < \nu_{i,r_{i-1}}(\mathbf{x}) \quad \forall i \in \{1, \hdots ,N \} \setminus p$. In practice, a large value of $\kappa$ is generally chosen to avoid evaluation of \eqref{theorem3_eq}. 
\end{remark}
        


    


% =============================================================================================
% =============================================================================================
% =============================================================================================

% =============================================================================================
% =============================================================================================
% =============================================================================================
\section{Composite CBF for Multirotor Collision Avoidance}\label{sec:approach}
\section{Method}


\subsection{Generative Topology Optimization}


\paragraph{Definitions.}
Let $\mathcal{X} \subset \mathbb{R}^{d_x}$
be the domain of interest in which there is a shape $\Omega \subset \mathcal{X}$. 
Let $\mathcal{Z} \subset \mathbb{R}^{d_z}$ be a discrete or continuous modulation space, where each $\mathbf{z} \in \mathcal{Z}$ parametrizes a shape $\Omega_\mathbf{z}$. The possibly infinite set of all shapes is denoted by $\Omega_\mathcal{Z} = \{ \Omega_\mathbf{z} | \mathbf{z} \in \mathcal{Z} \}$.
The modulation vectors $\{\mathbf{z}_{i} \}$ are either elements in a fixed, finite set $\mathcal{Z}$ or sampled according to a continuous probability distribution $p(\mathcal{Z})$ on an interval $[a, b]^n \subset \mathbb{R}^n$. For brevity, we will use $\mathbf{z} \sim p(\mathcal{Z})$ for both of these cases. \newline
For density representations a shape $\Omega_\mathbf{z}$ is defined as the set of points with a density greater than the level $\tau \in [0,1]$, formally $\Omega_\mathbf{z} = \{\mathbf{x} \in \mathcal{X} \mid \rho_\mathbf{z}(\mathbf{x}) > \tau \}$.
While some priors work on occupancy networks \citep{Occupancy_Networks} treat $\tau$ as a tunable hyperparameter, we follow \citet{fenitop} and fix the level at $\tau = 0.5$. 
We model the density $\rho_\mathbf{z}(\mathbf{x}) = f_\theta(\mathbf{x}, \mathbf{z})$, corresponding to the modulation vector $\mathbf{z}$ at a point $\mathbf{x}$ using a neural network $f_\theta$, with $\theta$ being the learnable parameters. 

\begin{algorithm}[tb]
   \caption{GenTO}
   \label{alg:GenTO}
\begin{algorithmic}
   \STATE {\bfseries Input:} 
   parameterized density $f_\theta$, 
   vector of mesh points $\mathbf{x_i} \in \mathbb{R}^{d_x}$,
   $\beta$ and annealing factor $\Delta_\beta$,
   number of shapes per iteration $k$,
   learning rate scheduler $\gamma(t)$,
   iterations $T$
   \FOR{$t=1$ {\bfseries to} $T$}
   \STATE $\mathbf{z_j} \sim p(\mathcal{Z})$ \hfill\COMMENT{sample modulation vectors}
   \STATE $\Tilde{\rho}_{z_j} \leftarrow f_\theta(\mathbf{x_i}, \mathbf{z_j})$  \hfill\COMMENT{net forward all shapes $j$}
   \STATE $\rho_j \leftarrow H(\Tilde{\rho}_{z_j}, \beta)$ \hfill\COMMENT{Heaviside contrast filter}
   \STATE $C, V, \frac{\partial C}{\partial \rho}, \frac{\partial V}{\partial \rho} \leftarrow \text{FEM}(\rho_j)$ \hfill\COMMENT{solver step}
   \STATE $\frac{\partial \rho}{\partial \theta} \leftarrow \text{backward}(f_\theta, \rho_j)$ \hfill\COMMENT{autodiff backward}
   \STATE $\nabla_\theta C \leftarrow \frac{\partial C}{\partial \rho} \frac{\partial \rho}{\partial \theta} $ \hfill\COMMENT{compliance gradient}
    \STATE $\nabla_\theta V \leftarrow \lambda_V \frac{\partial C}{\partial \rho} \frac{\partial \rho}{\partial \theta} $ \hfill\COMMENT{volume gradient}
   \STATE $L_\text{G} \leftarrow \lambda_\text{interface} L_\text{interface} + ...$ \ \hfill\COMMENT{GINN constraints}
   \STATE $\Delta \theta \leftarrow  \text{ALM} \left( \nabla_\theta C, \nabla_\theta V, \nabla_\theta L_\text{G} \right) $ \hfill\COMMENT{ALM}
   \STATE $\theta \leftarrow \theta - \gamma(t) \Delta \theta $ \hfill\COMMENT{parameter update}
   \STATE $\beta \leftarrow \beta \Delta_\beta$ \hfill\COMMENT{annealing}
   \ENDFOR
\end{algorithmic}
\end{algorithm}

\paragraph{GenTO}
aims to solve a constrained optimization problem with the objective of minimizing the expected compliance of multiple shapes subject to a volume constraint on each.
At the core of our method is the diversity constraint $\delta(\Omega_\mathcal{Z})$, defined over multiple shapes to make them less similar. This leads to the following constrained optimization problem:

\begin{equation}
\begin{aligned}
\label{eq:GenTo}
\min : & \quad \mathbb{E}_{\mathbf{z} \sim p(\mathcal{Z})} \left[ C(\rho_{\mathbf{z}}) \right]
\\
\text{s.t.} : & \quad V_{\mathbf{z}} = \int_\mathcal{X} \mathbf{\rho}_{\mathbf{z}}(\mathbf{x}) d\mathbf{x} \leq V^*\\
& \quad 0 \leq \mathbf{\rho}_{\mathbf{z}}(\mathbf{x}) \leq 1 \\
& \quad \delta^* \leq \delta(\Omega_\mathcal{Z})
\end{aligned}
\end{equation}

where $V^*$ and $\delta^*$ are target volumes and diversities, respectively.
GenTO updates the density fields of multiple shapes iteratively. In each iteration, the density distribution of each shape is computed and the resulting densities are passed to a finite element method (FEM) solver. 
The FEM solver calculates the compliance loss $C$ and gradients $\nabla C$ using the adjoint equation instead of differentiating through the solver.
To accelerate the computation, parallelization of the FEM solver is employed across multiple CPU cores, with a separate process dedicated to each shape.
\\ 
The diversity loss (see Section \ref{sec:diversity}) is computed on the GPU, along with any optional geometric losses similar to GINNs (see Section \ref{sec:geom_constraints}).
Gradients for these losses are obtained via automatic differentiation, based on which GenTO updates the network parameters $\theta$.
\\
We use the adaptive augmented Lagrangian method (ALM) \citep{basir2023adaptive} to automatically balance multiple loss terms. 
Additionally, we gradually increase the sharpness parameter $\beta$ of the Heaviside filter (Equation \ref{eq:heaviside}) which acts as annealing. This outline of the GenTO method is summarized in Algorithm \ref{alg:GenTO}.











\subsection{Diversity}
\label{sec:diversity}

The diversity loss, defined in Equation \ref{eq:div}, requires the definition of a dissimilarity measure between a pair of shapes. 
Generally, the dissimilarity between two shapes can be based on either the volume of a shape $\Omega$ or its boundary $\partial \Omega$.
GINNs \citep{GINNs} use boundary dissimilarity, which is easily optimized for SDFs since the value at a point is the distance to the zero level set. However, this dissimilarity measure is not applicable to our density field shape representation.
Hence we propose a boundary dissimilarity based on the chamfer discrepancy in the 
We also developed a volume-based dissimilarity (detailed in Appendix \ref{subsec:volume_dissimilarity}), however we focus on boundary diversity due to its promising results in early experiments.



\paragraph{Diversity on the boundary via differentiable chamfer discrepancy.}
To define the dissimilarity on the boundaries of a pair of shapes $(\partial \Omega_1, \partial \Omega_2)$, we use the one-sided chamfer discrepancy (CD):
\begin{align}
\label{eq:1_sided_chamfer}
    \text{CD}(\partial \Omega_1, \partial \Omega_2) = \frac{1}{\left| \partial \Omega_1 \right|} \sum_{x \in \partial \Omega_1} \min_{\Tilde{x} \in \partial \Omega_2} ||x-\Tilde{x}||_2
\end{align}
where $x$ and $\Tilde{x}$ are sampled points on the boundaries.
To use the CD as a loss, it must be differentiable w.r.t. the network parameters $\theta$.
However, the chamfer discrepancy $\text{CD}(\partial\Omega_1, \partial\Omega_2)$ depends only on the boundary points $x_i \in \partial \Omega$ which only depend on $f_\theta (x_i)$ implicitly.
Akin to prior work \citep{chen_converting_2010, Berzins2023bs, mehta_level_2022}, we apply the chain rule and use the level-set equation to derive
\isformat{icml}

    \begin{align}
        \frac{\partial \text{CD}}{\partial \theta} &= \frac{\partial \text{CD}}{\partial x} \frac{\partial x}{\partial y} \frac{\partial y}{\partial \theta} \nonumber \\ 
        &= \frac{\partial \text{CD}}{\partial x} \frac{\nabla_x f_\theta}{|\nabla_x f_\theta|^2} \frac{\partial y}{\partial \theta} \label{eq:level_set}
    \end{align}
\else
    \begin{align}
    \label{eq:level_set}
        \frac{\partial \text{CD}}{\partial \theta} = \frac{\partial \text{CD}}{\partial x} \frac{\partial x}{\partial y} \frac{\partial y}{\partial \theta} = \frac{\partial \text{CD}}{\partial x} \frac{\nabla_x f_\theta}{|\nabla_x f_\theta|^2} \frac{\partial y}{\partial \theta} 
    \end{align}
\fi
where $y = \rho$ is the density field in our case.
We detail this derivation in Appendix \ref{subsec:chamfer_diversity}.


\paragraph{Finding surface points on density fields.}
Finding surface points for densities is substantially harder than for SDFs.
For an SDF $f$, surface points can be obtained by flowing randomly initialized points along the gradient $\nabla f$ to the boundary at $f=0$. 
However, density fields $g$ do not satisfy the eikonal equation $|\nabla g| \neq 1$, causing gradient flows to easily get stuck in local minima. 
To overcome this challenge, we employ a robust algorithm that relies on dense sampling and binary search, as detailed in Algorithm \ref{alg:boundary_points}.



\subsection{Formalizing geometric constraints}
\label{sec:geom_constraints}

The compliance and volume losses are computed by the FEM solver on a discrete grid.
In addition, we use geometric constraints similar to GINNs.
This leverages the continuous field representation and allows learning finer details where necessary, e.g. at the interfaces.
In principle, these constraints could also be imposed via the TO framework, by defining solid regions ($\rho=1$ values) to specific mesh cells.
However, this comes at the cost of requiring increased mesh resolution, which raises the computational cost for the CPU-based FEM solver.
Instead, we adapt the geometric constraints of GINNs to operate with a general level set $\tau$ and reflect the change from an SDF to a density representation.
The applied constraints and further details are in Table \ref{tab:constraints} and Appendix \ref{app:constraints}. Crucially, these are computed in parallel on the GPU, accelerating the optimization process.




% =============================================================================================
% =============================================================================================
% =============================================================================================

\section{Evaluation Studies}\label{sec:evaluation}
\begin{figure}
    \centering
    %\includegraphics[width=\linewidth]{figures/CBF_elektro.eps}
    \includegraphics[width=\linewidth]{figures/CBF_elektro_final.pdf}
    \vspace{-6ex}
    \caption{\small Constraint values (top) and velocity error w.r.t. the reference (center) and filter-induced deviation (bottom) during experiment A.}
    \label{fig:CBF_elektro}
    \vspace{-3ex}
\end{figure}

\subsection{Computational Scalability Study}
To evaluate the scalability of the proposed approach to cluttered environments with many obstacles, we evaluate the run times of the equations \eqref{composition1}, \eqref{composite_cbf} and \eqref{composite_invariance} for different environment sizes. Specifically, we ablate the required time for a single computation of $h(\mathbf{x})$, $\mathcal{L}_f h(\mathbf{x})$ and $\mathcal{L}_g h(\mathbf{x})$ in \eqref{composite_invariance} over different numbers of randomly placed obstacles. This is performed for the cases of computation of $\mathcal{L}_f h(\mathbf{x})$, $\mathcal{L}_g h(\mathbf{x})$ via the analytic gradients and via automatic gradient computation. Also, we evaluate this on a commercial laptop PC on CPU (Intel i7) and GPU (Nvidia RTX 3070Ti). All computations are performed in PyTorch. The processing times are displayed in table \ref{simulation_table}. Note that for fewer obstacles ($<10^3$) the processing time on CPU is lower than on GPU, while for more than $10^3$ obstacles, the computation benefits from the parallelism on the GPU.
These results show that the approach enables constructing the composite \ac{cbf} at high rates for environments with thousands of obstacles.

\begin{table}[t]
    \caption{Composition time of \ac{cbf} and Lie derivatives in milliseconds.}
    \label{simulation_table}    
    \centering
    \begin{tabular}{@{}llcccccc@{}}
        \toprule
        & & $10^1$ & $10^2$ & $10^3$ & $5 \cdot 10^3$ & $10^4$ \\
        \midrule
        \multirow{2}{*}{CPU} & Analytic & \textbf{0.264} & \textbf{0.279} & \textbf{0.485} & 2.575 & 3.353 \\
                             & Numeric  & 0.910 & 0.996 & 1.589 & 5.169 & 6.655 \\
        \midrule
        \multirow{2}{*}{GPU} & Analytic & 0.614 & 0.544 & 0.711 & \textbf{0.772} & \textbf{0.908} \\
                             & Numeric  & 2.606 & 2.703 & 2.760 & 2.653 & 2.624 \\
        \bottomrule
    \end{tabular}
    \vspace{-2ex}
    \iffalse
    \centering
    \begin{tabular}{ |p{0.5cm}|p{0.8cm}||p{0.8cm}|p{0.8cm}|p{0.8cm}|p{0.8cm}|p{0.8cm}|  }
     % \hline
     % \multicolumn{4}{|c|}{success rate [\%]} \\
     \hline
      \multicolumn{2}{|c|}{$N$} & $10^1$ &  $10^2$ & $10^3$ & $5\cdot10^3$ & $10^4$\\
     \hline
     CPU  & analytic  & \textbf{0.264} & \textbf{0.279} & \textbf{0.485} & 2.575 & 3.353\\
     \hline
     CPU  & numeric & 0.910 & 0.996 & 1.589 & 5.169 & 6.655\\
     \hline
     GPU  & analytic  & 0.614 & 0.544 & 0.711 & \textbf{0.772} & \textbf{0.908}\\
     \hline
     GPU  & numeric & 2.606 & 2.703 & 2.760 & 2.653 & 2.624\\
     \hline
    \end{tabular}
    \vspace{-4ex}
    \fi
\end{table}

\begin{figure*}[ht]
    \centering
    %\includegraphics[width=\linewidth]{figures/mission_dragvoll.eps}
    \includegraphics[width=\linewidth]{figures/mission_dragvoll_final.pdf}
    \caption{\small Aggregated map and path of experiment B. The mission starts at the cyan circle on the right. The quadrotor receives an adversarial velocity reference that actively tries to collide with obstacles. The red arrows depict this reference velocity for some selected time instances, A-D, which are reported in Fig. \ref{fig:CBF_dragvoll}.}
    \label{fig:mission_dragvoll}
    \vspace{-4ex}
\end{figure*}

\begin{figure}
    \centering
    %\includegraphics[width=\linewidth]{figures/CBF_dragvoll.eps}
    \includegraphics[width=\linewidth]{figures/CBF_dragvoll_final.pdf}
    \vspace{-6ex}
    \caption{\small Constraint values during experiment B.}
    \label{fig:CBF_dragvoll}
    \vspace{-3.5ex}
\end{figure}

\subsection{Experimental Implementation}
The experiments relied on a custom-built quadrotor from \cite{harms2024neural} with dimensions $0.52\times 0.52\times0.31~\textrm{m}$ and a takeoff mass of $2.58~\textrm{kg}$. The system integrates PX4-based autopilot avionics for low-level control, together with an NVIDIA Orin NX single-board computer, as well as an Ouster OS0-64 LiDAR and a VectorNav VN-100 IMU used for odometry estimation as in~\cite{khattak2020complementary}. The mapping method \cite{voxblox} is then used to create a voxel-grid representation of the environment at a resolution of 20cm. The closest 400 occupied voxels are then used as the obstacles $x_i$ in \eqref{position_cbf} and updated at 10Hz.
The proposed safe control law \eqref{ref_controller} is implemented in Python, where we performed a minor adjustment by using the adaptive position control law introduced in~\cite{TalINDI}. This attenuates tracking errors introduced by modelling uncertainties in the thrust coefficient. The composition of the composite \ac{cbf} and its Lie derivatives (equations \eqref{composition1} - \eqref{composite_cbf}) uses PyTorch and the safety QP \eqref{OCP} is implemented in Casadi~\cite{Andersson_Casadi} using the QP solver qpOASES~\cite{Ferreau_qpOASES}. Furthermore, an estimate of the current thrust $T_\text{est}$ estimate is obtained by low-pass filtering the previous thrust commands. As a control output, the vector $[\Omega^T,T_\text{est}+ \tau \Delta t ]^T$ is sent to the autopilot, where $\Delta t$ is the time interval between control updates.
The safety filter and reference controller run on the Orin NX CPU with an update rate of 100Hz with the parameters listed in Tab. \ref{tab:parameters}.
Experiments A and B presented hereafter can be seen in the supplementary video.

\subsection{Experiment A}
In the first experiment, the quadrotor receives a constant reference velocity of $[1,0,0]\unit{m/s}$  and reference height of $1.3\unit{m}$ in an obstacle-filled hallway. The cluttered environment forces the safety filter to become active during many short periods, deflecting the quadrotor away from obstacles (points A and B), above an obstacle (point C), while finally reaching a dead-end where it remains in hover (point D). The trajectory and environment are shown in Fig. \ref{fig:mission_elektro}. The resulting set function values are plotted in Fig. \ref{fig:CBF_elektro}. It can be seen that all original constraints and higher order sets are satisfied over the entire mission since the minimum over constraints of $\nu_{0,i}$, $\nu_{1,i}$, $\nu_{2,i}$ is positive over the entire mission. The \ac{cbf} value is mostly positive but displays slight crossings of $h(\mathbf{x})=0$. These minor violations are small and are expected in a real system due to modelling errors and time-varying observations. The velocity tracking errors induced by the safety filter are also visualized in Fig. \ref{fig:CBF_elektro}, showing clearly that the corrections induced by the safety filter induce tracking errors. It can also be seen from Fig. \ref{fig:CBF_elektro} that the safety filter acts on roll-rate ($p$), pitch-rate ($q$) and thrust-rate ($\tau$) to achieve safe actions. The results demonstrate the ability of the proposed safe control law to enforce constraint satisfaction over the entire mission.

\begin{table}[t]
    \centering
    \caption{Parameter values used in the experiments.}
    \label{tab:parameters}
    \setlength{\tabcolsep}{5.5pt} % Adjust column spacing
    \renewcommand{\arraystretch}{1} % Keep standard row spacing
    \begin{tabular}{c|cccccccc}
        \toprule
        \textbf{Parameter} & $p_0$ & $p_1$ & $\alpha_1$ & $\gamma$ & $\kappa$ & $\epsilon$ & $\alpha_2$  & $\epsilon_T$ \\
        \midrule
        \textbf{Value} & -3 & -2 & 1 & 40 & 20 & 0.5 & 5 & 7.5 \\
        \bottomrule
    \end{tabular}
    \vspace{-4ex}
\end{table}

\subsection{Experiment B}
The second experiment takes place in a forest with tree trunks
%, twigs, branches 
and foliage as natural obstacles. The reference velocity in this experiment is given by an adversarial operator, which intentionally attempts to collide the quadrotor with the surroundings. Horizontal and vertical references are given to provoke collisions with various trees and with the ground. A slight wind was present during the experiment, adding unmodelled disturbances. The trajectory and environment are shown in Fig. \ref{fig:mission_dragvoll}. The figure highlights four instances during the experiment where the unsafe references of the operator are corrected by the safety filter. The \ac{cbf} values are plotted in Fig. \ref{fig:CBF_dragvoll}, showing that the original constraints are all satisfied during the entire experiment. However, the value of $h_1$ often drops below the zero line due to the present disturbances. This experiment demonstrates the applicability of the proposed method in realistic challenging conditions.

% =============================================================================================
% =============================================================================================
% =============================================================================================
\section{Conclusions}\label{sec:conclusions}
This work presented a novel approach for safe navigation of multirotors in unknown, cluttered environments.
The proposed safety filter leverages a Composite \ac{cbf} formulation for synthesizing a single \ac{cbf} from an arbitrary number of 1D collision constraints representing point-wise obstacles.
The resulting \ac{cbf} is both computationally scalable, and is shown to be recursively feasible, except for a zero-volume set of infeasible configurations.
The proposed method is validated in two hardware experiments, in varying conditions, against an environment-agnostic and an adversarial policy.
Future work includes extending the method to a robust \ac{cbf} design to explicitly account for modelling uncertainty, and investigating approaches to also account for constraints on the control input, while retaining feasibility.

\bibliographystyle{IEEEtran}
\bibliography{references}



\end{document}
