\documentclass[letterpaper, 10pt, conference]{ieeeconf}

\makeatletter
\let\IEEEproof\proof
\let\IEEEendproof\endproof
\let\proof\@undefined
\let\endproof\@undefined
\makeatother

\IEEEoverridecommandlockouts  % used for \thanks

\usepackage{cite}
\usepackage{amsmath,amssymb,amsfonts}
\usepackage{algorithmic}
\usepackage{graphicx}
\usepackage{textcomp}
\usepackage{xcolor}
\usepackage{booktabs}
\usepackage{multirow}
\usepackage{xspace}  % used for \xspace in macros (smart spacing)
\usepackage{amsthm}
\usepackage{amsfonts}
\usepackage{mathtools}  % used for \DeclarePairedDelimiter
\usepackage{bm}  % used for bold and prettier greek letters with \bm
\usepackage{paralist}
\usepackage{soul}
\usepackage[nolist,nohyperlinks]{acronym}
\newacronym{rl}{RL}{Reinforcement Learning}
\newacronym{drl}{DRL}{Deep Reinforcement Learning}
\newacronym{mdp}{MDP}{Markov Decision Process}
\newacronym{ppo}{PPO}{Proximal Policy Optimization}
\newacronym{sac}{SAC}{Soft Actor-Critic}
\newacronym{epvf}{EPVF}{Explicit Policy-conditioned Value Function}
\newacronym{unf}{UNF}{Universal Neural Functional}
\usepackage[hidelinks,bookmarks]{hyperref} 

\usepackage{multicol}
\usepackage{citesort}
\usepackage[textsize=tiny]{todonotes}
\usepackage{siunitx}
\usepackage{relsize}
\theoremstyle{definition}
\newtheorem{definition}{Definition}
\newtheorem{theorem}{Theorem}
\newtheorem{lemma}{Lemma}
\newtheorem{proposition}{Proposition}
\theoremstyle{remark}
\newtheorem{remark}{Remark}

\newcommand{\vectorproj}[2][]{\textit{proj}_{\vect{#1}}\vect{#2}}



%%% METADATA %%%%%%%%%%%%%%%%%%%%%%%%%%%%%%%%%%%%%%%%%%%%%%%%%%%
% TITLE
\title{\LARGE \bf Safe Quadrotor Navigation using Composite Control Barrier Functions}
%Safe Geometric Control of Multirotors in highly cluttered Environments}
% AUTHOR
\author{Marvin Harms, Martin Jacquet, Kostas Alexis
%=======
    % AFFILIATIONS
	\thanks{Autonomous Robots Lab, Norwegian University of Science and Technology (NTNU), Trondheim, Norway,
    {\tt \footnotesize
        \href{mailto:marvin.c.harms@ntnu.no}{marvin.c.harms@ntnu.no}}
    }
    \thanks{This work was supported by the European Commission Horizon Europe grants DIGIFOREST (EC 101070405) and SPEAR (EC 101119774).}
}

\begin{document}
\maketitle

\begin{abstract}
This paper introduces a safety filter to ensure collision avoidance for multirotor aerial robots. The proposed formalism leverages a single Composite Control Barrier Function from all position constraints acting on a third-order nonlinear representation of the robot's dynamics. We analyze the recursive feasibility of the safety filter under the composite constraint and demonstrate that the infeasible set is negligible. The proposed method allows computational scalability against thousands of constraints and, thus, complex scenes with numerous obstacles. We experimentally demonstrate its ability to guarantee the safety of a quadrotor with an onboard LiDAR, operating in both indoor and outdoor cluttered environments against both naive and adversarial nominal policies.
\end{abstract}


% =============================================================================================
% =============================================================================================
% =============================================================================================
\section{Introduction}
\section{Introduction}

% \textcolor{red}{Still on working}

% \textcolor{red}{add label for each section}


Robot learning relies on diverse and high-quality data to learn complex behaviors \cite{aldaco2024aloha, wang2024dexcap}.
Recent studies highlight that models trained on datasets with greater complexity and variation in the domain tend to generalize more effectively across broader scenarios \cite{mann2020language, radford2021learning, gao2024efficient}.
% However, creating such diverse datasets in the real world presents significant challenges.
% Modifying physical environments and adjusting robot hardware settings require considerable time, effort, and financial resources.
% In contrast, simulation environments offer a flexible and efficient alternative.
% Simulations allow for the creation and modification of digital environments with a wide range of object shapes, weights, materials, lighting, textures, friction coefficients, and so on to incorporate domain randomization,
% which helps improve the robustness of models when deployed in real-world conditions.
% These environments can be easily adjusted and reset, enabling faster iterations and data collection.
% Additionally, simulations provide the ability to consistently reproduce scenarios, which is essential for benchmarking and model evaluation.
% Another advantage of simulations is their flexibility in sensor integration. Sensors such as cameras, LiDARs, and tactile sensors can be added or repositioned without the physical limitations present in real-world setups. Simulations also eliminate the risk of damaging expensive hardware during edge-case experiments, making them an ideal platform for testing rare or dangerous scenarios that are impractical to explore in real life.
By leveraging immersive perspectives and interactions, Extended Reality\footnote{Extended Reality is an umbrella term to refer to Augmented Reality, Mixed Reality, and Virtual Reality \cite{wikipediaExtendedReality}}
(XR)
is a promising candidate for efficient and intuitive large scale data collection \cite{jiang2024comprehensive, arcade}
% With the demand for collecting data, XR provides a promising approach for humans to teach robots by offering users an immersive experience.
in simulation \cite{jiang2024comprehensive, arcade, dexhub-park} and real-world scenarios \cite{openteach, opentelevision}.
However, reusing and reproducing current XR approaches for robot data collection for new settings and scenarios is complicated and requires significant effort.
% are difficult to reuse and reproduce system makes it hard to reuse and reproduce in another data collection pipeline.
This bottleneck arises from three main limitations of current XR data collection and interaction frameworks: \textit{asset limitation}, \textit{simulator limitation}, and \textit{device limitation}.
% \textcolor{red}{ASSIGN THESE CITATION PROPERLY:}
% \textcolor{red}{list them by time order???}
% of collecting data by using XR have three main limitations.
Current approaches suffering from \textit{asset limitation} \cite{arclfd, jiang2024comprehensive, arcade, george2025openvr, vicarios}
% Firstly, recent works \cite{jiang2024comprehensive, arcade, dexhub-park}
can only use predefined robot models and task scenes. Configuring new tasks requires significant effort, since each new object or model must be specifically integrated into the XR application.
% and it takes too much effort to configure new tasks in their systems since they cannot spawn arbitrary models in the XR application.
The vast majority of application are developed for specific simulators or real-world scenarios. This \textit{simulator limitation} \cite{mosbach2022accelerating, lipton2017baxter, dexhub-park, arcade}
% Secondly, existing systems are limited to a single simulation platform or real-world scenarios.
significantly reduces reusability and makes adaptation to new simulation platforms challenging.
Additionally, most current XR frameworks are designed for a specific version of a single XR headset, leading to a \textit{device limitation} 
\cite{lipton2017baxter, armada, openteach, meng2023virtual}.
% and there is no work working on the extendability of transferring to a new headsets as far as we know.
To the best of our knowledge, no existing work has explored the extensibility or transferability of their framework to different headsets.
These limitations hamper reproducibility and broader contributions of XR based data collection and interaction to the research community.
% as each research group typically has its own data collection pipeline.
% In addition to these main limitations, existing XR systems are not well suited for managing multiple robot systems,
% as they are often designed for single-operator use.

In addition to these main limitations, existing XR systems are often designed for single-operator use, prohibiting collaborative data collection.
At the same time, controlling multiple robots at once can be very difficult for a single operator,
making data collection in multi-robot scenarios particularly challenging \cite{orun2019effect}.
Although there are some works using collaborative data collection in the context of tele-operation \cite{tung2021learning, Qin2023AnyTeleopAG},
there is no XR-based data collection system supporting collaborative data collection.
This limitation highlights the need for more advanced XR solutions that can better support multi-robot and multi-user scenarios.
% \textcolor{red}{more papers about collaborative data collection}

To address all of these issues, we propose \textbf{IRIS},
an \textbf{I}mmersive \textbf{R}obot \textbf{I}nteraction \textbf{S}ystem.
This general system supports various simulators, benchmarks and real-world scenarios.
It is easily extensible to new simulators and XR headsets.
IRIS achieves generalization across six dimensions:
% \begin{itemize}
%     \item \textit{Cross-scene} : diverse object models;
%     \item \textit{Cross-embodiment}: diverse robot models;
%     \item \textit{Cross-simulator}: 
%     \item \textit{Cross-reality}: fd
%     \item \textit{Cross-platform}: fd
%     \item \textit{Cross-users}: fd
% \end{itemize}
\textbf{Cross-Scene}, \textbf{Cross-Embodiment}, \textbf{Cross-Simulator}, \textbf{Cross-Reality}, \textbf{Cross-Platform}, and \textbf{Cross-User}.

\textbf{Cross-Scene} and \textbf{Cross-Embodiment} allow the system to handle arbitrary objects and robots in the simulation,
eliminating restrictions about predefined models in XR applications.
IRIS achieves these generalizations by introducing a unified scene specification, representing all objects,
including robots, as data structures with meshes, materials, and textures.
The unified scene specification is transmitted to the XR application to create and visualize an identical scene.
By treating robots as standard objects, the system simplifies XR integration,
allowing researchers to work with various robots without special robot-specific configurations.
\textbf{Cross-Simulator} ensures compatibility with various simulation engines.
IRIS simplifies adaptation by parsing simulated scenes into the unified scene specification, eliminating the need for XR application modifications when switching simulators.
New simulators can be integrated by creating a parser to convert their scenes into the unified format.
This flexibility is demonstrated by IRIS’ support for Mujoco \cite{todorov2012mujoco}, IsaacSim \cite{mittal2023orbit}, CoppeliaSim \cite{coppeliaSim}, and even the recent Genesis \cite{Genesis} simulator.
\textbf{Cross-Reality} enables the system to function seamlessly in both virtual simulations and real-world applications.
IRIS enables real-world data collection through camera-based point cloud visualization.
\textbf{Cross-Platform} allows for compatibility across various XR devices.
Since XR device APIs differ significantly, making a single codebase impractical, IRIS XR application decouples its modules to maximize code reuse.
This application, developed by Unity \cite{unity3dUnityManual}, separates scene visualization and interaction, allowing developers to integrate new headsets by reusing the visualization code and only implementing input handling for hand, head, and motion controller tracking.
IRIS provides an implementation of the XR application in the Unity framework, allowing for a straightforward deployment to any device that supports Unity. 
So far, IRIS was successfully deployed to the Meta Quest 3 and HoloLens 2.
Finally, the \textbf{Cross-User} ability allows multiple users to interact within a shared scene.
IRIS achieves this ability by introducing a protocol to establish the communication between multiple XR headsets and the simulation or real-world scenarios.
Additionally, IRIS leverages spatial anchors to support the alignment of virtual scenes from all deployed XR headsets.
% To make an seamless user experience for robot learning data collection,
% IRIS also tested in three different robot control interface
% Furthermore, to demonstrate the extensibility of our approach, we have implemented a robot-world pipeline for real robot data collection, ensuring that the system can be used in both simulated and real-world environments.
The Immersive Robot Interaction System makes the following contributions\\
\textbf{(1) A unified scene specification} that is compatible with multiple robot simulators. It enables various XR headsets to visualize and interact with simulated objects and robots, providing an immersive experience while ensuring straightforward reusability and reproducibility.\\
\textbf{(2) A collaborative data collection framework} designed for XR environments. The framework facilitates enhanced robot data acquisition.\\
\textbf{(3) A user study} demonstrating that IRIS significantly improves data collection efficiency and intuitiveness compared to the LIBERO baseline.

% \begin{table*}[t]
%     \centering
%     \begin{tabular}{lccccccc}
%         \toprule
%         & \makecell{Physical\\Interaction}
%         & \makecell{XR\\Enabled}
%         & \makecell{Free\\View}
%         & \makecell{Multiple\\Robots}
%         & \makecell{Robot\\Control}
%         % Force Feedback???
%         & \makecell{Soft Object\\Supported}
%         & \makecell{Collaborative\\Data} \\
%         \midrule
%         ARC-LfD \cite{arclfd}                              & Real        & \cmark & \xmark & \xmark & Joint              & \xmark & \xmark \\
%         DART \cite{dexhub-park}                            & Sim         & \cmark & \cmark & \cmark & Cartesian          & \xmark & \xmark \\
%         \citet{jiang2024comprehensive}                     & Sim         & \cmark & \xmark & \xmark & Joint \& Cartesian & \xmark & \xmark \\
%         \citet{mosbach2022accelerating}                    & Sim         & \cmark & \cmark & \xmark & Cartesian          & \xmark & \xmark \\
%         ARCADE \cite{arcade}                               & Real        & \cmark & \cmark & \xmark & Cartesian          & \xmark & \xmark \\
%         Holo-Dex \cite{holodex}                            & Real        & \cmark & \xmark & \cmark & Cartesian          & \cmark & \xmark \\
%         ARMADA \cite{armada}                               & Real        & \cmark & \xmark & \cmark & Cartesian          & \cmark & \xmark \\
%         Open-TeleVision \cite{opentelevision}              & Real        & \cmark & \cmark & \cmark & Cartesian          & \cmark & \xmark \\
%         OPEN TEACH \cite{openteach}                        & Real        & \cmark & \xmark & \cmark & Cartesian          & \cmark & \cmark \\
%         GELLO \cite{wu2023gello}                           & Real        & \xmark & \cmark & \cmark & Joint              & \cmark & \xmark \\
%         DexCap \cite{wang2024dexcap}                       & Real        & \xmark & \cmark & \xmark & Cartesian          & \cmark & \xmark \\
%         AnyTeleop \cite{Qin2023AnyTeleopAG}                & Real        & \xmark & \xmark & \cmark & Cartesian          & \cmark & \cmark \\
%         Vicarios \cite{vicarios}                           & Real        & \cmark & \xmark & \xmark & Cartesian          & \cmark & \xmark \\     
%         Augmented Visual Cues \cite{augmentedvisualcues}   & Real        & \cmark & \cmark & \xmark & Cartesian          & \xmark & \xmark \\ 
%         \citet{wang2024robotic}                            & Real        & \cmark & \cmark & \xmark & Cartesian          & \cmark & \xmark \\
%         Bunny-VisionPro \cite{bunnyvisionpro}              & Real        & \cmark & \cmark & \cmark & Cartesian          & \cmark & \xmark \\
%         IMMERTWIN \cite{immertwin}                         & Real        & \cmark & \cmark & \cmark & Cartesian          & \xmark & \xmark \\
%         \citet{meng2023virtual}                            & Sim \& Real & \cmark & \cmark & \xmark & Cartesian          & \xmark & \xmark \\
%         Shared Control Framework \cite{sharedctlframework} & Real        & \cmark & \cmark & \cmark & Cartesian          & \xmark & \xmark \\
%         OpenVR \cite{openvr}                               & Real        & \cmark & \cmark & \xmark & Cartesian          & \xmark & \xmark \\
%         \citet{digitaltwinmr}                              & Real        & \cmark & \cmark & \xmark & Cartesian          & \cmark & \xmark \\
        
%         \midrule
%         \textbf{Ours} & Sim \& Real & \cmark & \cmark & \cmark & Joint \& Cartesian  & \cmark & \cmark \\
%         \bottomrule
%     \end{tabular}
%     \caption{This is a cross-column table with automatic line breaking.}
%     \label{tab:cross-column}
% \end{table*}

% \begin{table*}[t]
%     \centering
%     \begin{tabular}{lccccccc}
%         \toprule
%         & \makecell{Cross-Embodiment}
%         & \makecell{Cross-Scene}
%         & \makecell{Cross-Simulator}
%         & \makecell{Cross-Reality}
%         & \makecell{Cross-Platform}
%         & \makecell{Cross-User} \\
%         \midrule
%         ARC-LfD \cite{arclfd}                              & \xmark & \xmark & \xmark & \xmark & \xmark & \xmark \\
%         DART \cite{dexhub-park}                            & \cmark & \cmark & \xmark & \xmark & \xmark & \xmark \\
%         \citet{jiang2024comprehensive}                     & \xmark & \cmark & \xmark & \xmark & \xmark & \xmark \\
%         \citet{mosbach2022accelerating}                    & \xmark & \cmark & \xmark & \xmark & \xmark & \xmark \\
%         ARCADE \cite{arcade}                               & \xmark & \xmark & \xmark & \xmark & \xmark & \xmark \\
%         Holo-Dex \cite{holodex}                            & \cmark & \xmark & \xmark & \xmark & \xmark & \xmark \\
%         ARMADA \cite{armada}                               & \cmark & \xmark & \xmark & \xmark & \xmark & \xmark \\
%         Open-TeleVision \cite{opentelevision}              & \cmark & \xmark & \xmark & \xmark & \cmark & \xmark \\
%         OPEN TEACH \cite{openteach}                        & \cmark & \xmark & \xmark & \xmark & \xmark & \cmark \\
%         GELLO \cite{wu2023gello}                           & \cmark & \xmark & \xmark & \xmark & \xmark & \xmark \\
%         DexCap \cite{wang2024dexcap}                       & \xmark & \xmark & \xmark & \xmark & \xmark & \xmark \\
%         AnyTeleop \cite{Qin2023AnyTeleopAG}                & \cmark & \cmark & \cmark & \cmark & \xmark & \cmark \\
%         Vicarios \cite{vicarios}                           & \xmark & \xmark & \xmark & \xmark & \xmark & \xmark \\     
%         Augmented Visual Cues \cite{augmentedvisualcues}   & \xmark & \xmark & \xmark & \xmark & \xmark & \xmark \\ 
%         \citet{wang2024robotic}                            & \xmark & \xmark & \xmark & \xmark & \xmark & \xmark \\
%         Bunny-VisionPro \cite{bunnyvisionpro}              & \cmark & \xmark & \xmark & \xmark & \xmark & \xmark \\
%         IMMERTWIN \cite{immertwin}                         & \cmark & \xmark & \xmark & \xmark & \xmark & \xmark \\
%         \citet{meng2023virtual}                            & \xmark & \cmark & \xmark & \cmark & \xmark & \xmark \\
%         \citet{sharedctlframework}                         & \cmark & \xmark & \xmark & \xmark & \xmark & \xmark \\
%         OpenVR \cite{george2025openvr}                               & \xmark & \xmark & \xmark & \xmark & \xmark & \xmark \\
%         \citet{digitaltwinmr}                              & \xmark & \xmark & \xmark & \xmark & \xmark & \xmark \\
        
%         \midrule
%         \textbf{Ours} & \cmark & \cmark & \cmark & \cmark & \cmark & \cmark \\
%         \bottomrule
%     \end{tabular}
%     \caption{This is a cross-column table with automatic line breaking.}
% \end{table*}

% \begin{table*}[t]
%     \centering
%     \begin{tabular}{lccccccc}
%         \toprule
%         & \makecell{Cross-Scene}
%         & \makecell{Cross-Embodiment}
%         & \makecell{Cross-Simulator}
%         & \makecell{Cross-Reality}
%         & \makecell{Cross-Platform}
%         & \makecell{Cross-User}
%         & \makecell{Control Space} \\
%         \midrule
%         % Vicarios \cite{vicarios}                           & \xmark & \xmark & \xmark & \xmark & \xmark & \xmark \\     
%         % Augmented Visual Cues \cite{augmentedvisualcues}   & \xmark & \xmark & \xmark & \xmark & \xmark & \xmark \\ 
%         % OpenVR \cite{george2025openvr}                     & \xmark & \xmark & \xmark & \xmark & \xmark & \xmark \\
%         \citet{digitaltwinmr}                              & \xmark & \xmark & \xmark & \xmark & \xmark & \xmark &  \\
%         ARC-LfD \cite{arclfd}                              & \xmark & \xmark & \xmark & \xmark & \xmark & \xmark &  \\
%         \citet{sharedctlframework}                         & \cmark & \xmark & \xmark & \xmark & \xmark & \xmark &  \\
%         \citet{jiang2024comprehensive}                     & \cmark & \xmark & \xmark & \xmark & \xmark & \xmark &  \\
%         \citet{mosbach2022accelerating}                    & \cmark & \xmark & \xmark & \xmark & \xmark & \xmark & \\
%         Holo-Dex \cite{holodex}                            & \cmark & \xmark & \xmark & \xmark & \xmark & \xmark & \\
%         ARCADE \cite{arcade}                               & \cmark & \cmark & \xmark & \xmark & \xmark & \xmark & \\
%         DART \cite{dexhub-park}                            & Limited & Limited & Mujoco & Sim & Vision Pro & \xmark &  Cartesian\\
%         ARMADA \cite{armada}                               & \cmark & \cmark & \xmark & \xmark & \xmark & \xmark & \\
%         \citet{meng2023virtual}                            & \cmark & \cmark & \xmark & \cmark & \xmark & \xmark & \\
%         % GELLO \cite{wu2023gello}                           & \cmark & \xmark & \xmark & \xmark & \xmark & \xmark \\
%         % DexCap \cite{wang2024dexcap}                       & \xmark & \xmark & \xmark & \xmark & \xmark & \xmark \\
%         % AnyTeleop \cite{Qin2023AnyTeleopAG}                & \cmark & \cmark & \cmark & \cmark & \xmark & \cmark \\
%         % \citet{wang2024robotic}                            & \xmark & \xmark & \xmark & \xmark & \xmark & \xmark \\
%         Bunny-VisionPro \cite{bunnyvisionpro}              & \cmark & \cmark & \xmark & \xmark & \xmark & \xmark & \\
%         IMMERTWIN \cite{immertwin}                         & \cmark & \cmark & \xmark & \xmark & \xmark & \xmark & \\
%         Open-TeleVision \cite{opentelevision}              & \cmark & \cmark & \xmark & \xmark & \cmark & \xmark & \\
%         \citet{szczurek2023multimodal}                     & \xmark & \xmark & \xmark & Real & \xmark & \cmark & \\
%         OPEN TEACH \cite{openteach}                        & \cmark & \cmark & \xmark & \xmark & \xmark & \cmark & \\
%         \midrule
%         \textbf{Ours} & \cmark & \cmark & \cmark & \cmark & \cmark & \cmark \\
%         \bottomrule
%     \end{tabular}
%     \caption{TODO, Bruce: this table can be further optimized.}
% \end{table*}

\definecolor{goodgreen}{HTML}{228833}
\definecolor{goodred}{HTML}{EE6677}
\definecolor{goodgray}{HTML}{BBBBBB}

\begin{table*}[t]
    \centering
    \begin{adjustbox}{max width=\textwidth}
    \renewcommand{\arraystretch}{1.2}    
    \begin{tabular}{lccccccc}
        \toprule
        & \makecell{Cross-Scene}
        & \makecell{Cross-Embodiment}
        & \makecell{Cross-Simulator}
        & \makecell{Cross-Reality}
        & \makecell{Cross-Platform}
        & \makecell{Cross-User}
        & \makecell{Control Space} \\
        \midrule
        % Vicarios \cite{vicarios}                           & \xmark & \xmark & \xmark & \xmark & \xmark & \xmark \\     
        % Augmented Visual Cues \cite{augmentedvisualcues}   & \xmark & \xmark & \xmark & \xmark & \xmark & \xmark \\ 
        % OpenVR \cite{george2025openvr}                     & \xmark & \xmark & \xmark & \xmark & \xmark & \xmark \\
        \citet{digitaltwinmr}                              & \textcolor{goodred}{Limited}     & \textcolor{goodred}{Single Robot} & \textcolor{goodred}{Unity}    & \textcolor{goodred}{Real}          & \textcolor{goodred}{Meta Quest 2} & \textcolor{goodgray}{N/A} & \textcolor{goodred}{Cartesian} \\
        ARC-LfD \cite{arclfd}                              & \textcolor{goodgray}{N/A}        & \textcolor{goodred}{Single Robot} & \textcolor{goodgray}{N/A}     & \textcolor{goodred}{Real}          & \textcolor{goodred}{HoloLens}     & \textcolor{goodgray}{N/A} & \textcolor{goodred}{Cartesian} \\
        \citet{sharedctlframework}                         & \textcolor{goodred}{Limited}     & \textcolor{goodred}{Single Robot} & \textcolor{goodgray}{N/A}     & \textcolor{goodred}{Real}          & \textcolor{goodred}{HTC Vive Pro} & \textcolor{goodgray}{N/A} & \textcolor{goodred}{Cartesian} \\
        \citet{jiang2024comprehensive}                     & \textcolor{goodred}{Limited}     & \textcolor{goodred}{Single Robot} & \textcolor{goodgray}{N/A}     & \textcolor{goodred}{Real}          & \textcolor{goodred}{HoloLens 2}   & \textcolor{goodgray}{N/A} & \textcolor{goodgreen}{Joint \& Cartesian} \\
        \citet{mosbach2022accelerating}                    & \textcolor{goodgreen}{Available} & \textcolor{goodred}{Single Robot} & \textcolor{goodred}{IsaacGym} & \textcolor{goodred}{Sim}           & \textcolor{goodred}{Vive}         & \textcolor{goodgray}{N/A} & \textcolor{goodgreen}{Joint \& Cartesian} \\
        Holo-Dex \cite{holodex}                            & \textcolor{goodgray}{N/A}        & \textcolor{goodred}{Single Robot} & \textcolor{goodgray}{N/A}     & \textcolor{goodred}{Real}          & \textcolor{goodred}{Meta Quest 2} & \textcolor{goodgray}{N/A} & \textcolor{goodred}{Joint} \\
        ARCADE \cite{arcade}                               & \textcolor{goodgray}{N/A}        & \textcolor{goodred}{Single Robot} & \textcolor{goodgray}{N/A}     & \textcolor{goodred}{Real}          & \textcolor{goodred}{HoloLens 2}   & \textcolor{goodgray}{N/A} & \textcolor{goodred}{Cartesian} \\
        DART \cite{dexhub-park}                            & \textcolor{goodred}{Limited}     & \textcolor{goodred}{Limited}      & \textcolor{goodred}{Mujoco}   & \textcolor{goodred}{Sim}           & \textcolor{goodred}{Vision Pro}   & \textcolor{goodgray}{N/A} & \textcolor{goodred}{Cartesian} \\
        ARMADA \cite{armada}                               & \textcolor{goodgray}{N/A}        & \textcolor{goodred}{Limited}      & \textcolor{goodgray}{N/A}     & \textcolor{goodred}{Real}          & \textcolor{goodred}{Vision Pro}   & \textcolor{goodgray}{N/A} & \textcolor{goodred}{Cartesian} \\
        \citet{meng2023virtual}                            & \textcolor{goodred}{Limited}     & \textcolor{goodred}{Single Robot} & \textcolor{goodred}{PhysX}   & \textcolor{goodgreen}{Sim \& Real} & \textcolor{goodred}{HoloLens 2}   & \textcolor{goodgray}{N/A} & \textcolor{goodred}{Cartesian} \\
        % GELLO \cite{wu2023gello}                           & \cmark & \xmark & \xmark & \xmark & \xmark & \xmark \\
        % DexCap \cite{wang2024dexcap}                       & \xmark & \xmark & \xmark & \xmark & \xmark & \xmark \\
        % AnyTeleop \cite{Qin2023AnyTeleopAG}                & \cmark & \cmark & \cmark & \cmark & \xmark & \cmark \\
        % \citet{wang2024robotic}                            & \xmark & \xmark & \xmark & \xmark & \xmark & \xmark \\
        Bunny-VisionPro \cite{bunnyvisionpro}              & \textcolor{goodgray}{N/A}        & \textcolor{goodred}{Single Robot} & \textcolor{goodgray}{N/A}     & \textcolor{goodred}{Real}          & \textcolor{goodred}{Vision Pro}   & \textcolor{goodgray}{N/A} & \textcolor{goodred}{Cartesian} \\
        IMMERTWIN \cite{immertwin}                         & \textcolor{goodgray}{N/A}        & \textcolor{goodred}{Limited}      & \textcolor{goodgray}{N/A}     & \textcolor{goodred}{Real}          & \textcolor{goodred}{HTC Vive}     & \textcolor{goodgray}{N/A} & \textcolor{goodred}{Cartesian} \\
        Open-TeleVision \cite{opentelevision}              & \textcolor{goodgray}{N/A}        & \textcolor{goodred}{Limited}      & \textcolor{goodgray}{N/A}     & \textcolor{goodred}{Real}          & \textcolor{goodgreen}{Meta Quest, Vision Pro} & \textcolor{goodgray}{N/A} & \textcolor{goodred}{Cartesian} \\
        \citet{szczurek2023multimodal}                     & \textcolor{goodgray}{N/A}        & \textcolor{goodred}{Limited}      & \textcolor{goodgray}{N/A}     & \textcolor{goodred}{Real}          & \textcolor{goodred}{HoloLens 2}   & \textcolor{goodgreen}{Available} & \textcolor{goodred}{Joint \& Cartesian} \\
        OPEN TEACH \cite{openteach}                        & \textcolor{goodgray}{N/A}        & \textcolor{goodgreen}{Available}  & \textcolor{goodgray}{N/A}     & \textcolor{goodred}{Real}          & \textcolor{goodred}{Meta Quest 3} & \textcolor{goodred}{N/A} & \textcolor{goodgreen}{Joint \& Cartesian} \\
        \midrule
        \textbf{Ours}                                      & \textcolor{goodgreen}{Available} & \textcolor{goodgreen}{Available}  & \textcolor{goodgreen}{Mujoco, CoppeliaSim, IsaacSim} & \textcolor{goodgreen}{Sim \& Real} & \textcolor{goodgreen}{Meta Quest 3, HoloLens 2} & \textcolor{goodgreen}{Available} & \textcolor{goodgreen}{Joint \& Cartesian} \\
        \bottomrule
        \end{tabular}
    \end{adjustbox}
    \caption{Comparison of XR-based system for robots. IRIS is compared with related works in different dimensions.}
\end{table*}



% =============================================================================================
% =============================================================================================
% =============================================================================================
\section{Preliminaries}\label{sec:preliminaries}
\subsection{Notation}
\begin{tabbing}
 \hspace*{2.2cm} \= \kill
  $\mathbf{x} \in \mathcal{X},  \mathbf{u} \in \mathcal{U}$ \>  state and input vectors \\[0.5ex]
  %\mathbf{u} \in \mathcal{U}$ \>  input vector \\[0.5ex]
  $\mathcal{L}_f h, \mathcal{L}_g h \mathbf{u}$ \>  Lie derivatives of $h$ along $f$, $g \mathbf{u}$ \\[0.5ex] 
  $\frac{d}{dt} V$, $\nabla_\mathbf{x} V$ \>  time derivative of $V$, gradient of $V$ w.r.t. $\mathbf{x}$ \\[0.5ex] 
  %$\nabla_\mathbf{x} V$ \>  gradient of $V$ w.r.t. $\mathbf{x}$ \\[0.5ex] 
  $\mathbf{a} \cdot \mathbf{b}$, $\mathbf{a} \times \mathbf{b}$ \>  dot product, cross product of $\mathbf{a}$ and $\mathbf{b}$ \\[0.5ex]  
  $\|\cdot\|$ \>  Euclidean norm \\[0.5ex] 
  $[\mathbf{a}]_{\times}$ \>  skew-symmetric matrix associated with $\mathbf{a}$ \\[0.5ex]
\end{tabbing}

\subsection{Control Barrier Functions}
Let us consider the general, control affine system

\small
\begin{equation}\label{system}
    \frac{d}{dt} \mathbf{x} = f(\mathbf{x}) + g(\mathbf{x}) \mathbf{u},
\end{equation}
\normalsize
where $\mathbf{x} \in \mathcal{X} \subseteq \mathbb{R}^n$ and $\mathbf{u} \in \mathcal{U} \subseteq \mathbb{R}^m$. We now consider a continuously differentiable function $h: \mathcal{X} \rightarrow \mathbb{R}$ with the property $\{\mathbf{x} | \frac{dh}{d\mathbf{x}}(\mathbf{x}) = 0\} \cap \{\mathbf{x} | h(\mathbf{x}) = 0\} = \emptyset$, which describes the superlevel set $\mathcal{X}_\text{safe} = \{\mathbf{x} \in \mathcal{X}  :  h(\mathbf{x}) \geq 0 \}$.

%\iffalse
\begin{definition}[Control Invariant Set]
    A set $\mathcal{X}_\text{safe}$ is a forward control invariant set for the system \eqref{system}, if for any $\mathbf{x}({t_0}) \in \mathcal{X}_\text{safe}$ there exists at least one input trajectory $\mathbf{u}(t) \in \mathcal{U}$ such that $\mathbf{x}({t}) \in \mathcal{X}_\text{safe} \quad \forall t\geq t_0$ under the system \eqref{system}.
\end{definition}
%\fi

\begin{definition}[Control Barrier Function \cite{ames2019control}]\label{def:cbf}
    Let $\mathcal{X}_\text{safe}$ be the $0$-superlevel set of a continuously differentiable function
    $h: \mathcal{X} \rightarrow \mathbb{R}$ with the property that $\mathcal{X}_\text{safe} = \{ \mathbf{x} | h(\mathbf{x})\geq 0 \}$. Then, $h$ is a \ac{cbf} if there exists an extended class $\mathcal{K}_\infty$ function $\alpha(\cdot)$ such that for the system \eqref{system} it holds:

    \small
    \begin{equation}\label{CBF_lie_condition}
        \underset{\mathbf{u} \in \mathcal{U}}{\text{sup}} [\mathcal{L}_f h(\mathbf{x}) + \mathcal{L}_g h(\mathbf{x}) \mathbf{u}] \geq -\alpha(h(\mathbf{x}))
    \end{equation}
    \normalsize
    for all $\mathbf{x} \in \mathcal{X}_\text{safe}$. Further, an input $\mathbf{u}$ is considered safe with respect to a valid CBF, if it satisfies \eqref{CBF_lie_condition}.
\end{definition}

From the above, it becomes clear that $\mathcal{X}_\text{safe}$ is a control invariant set for the system \eqref{system} subject to any control law satisfying \eqref{CBF_lie_condition}. See \cite{ames2016control} for a proof. Condition \eqref{CBF_lie_condition} requires a nonzero Lie derivative, e.g. $\mathcal{L}_g h(\mathbf{x}) \neq 0$ in general, restricting the use of \acp{cbf} to functions $h$ of relative degree~1. Systematic approaches to construct a CBF with relative degree greater than one include exponential control barrier functions \cite{ECBF}, high-order control barrier functions \cite{HOCBF} and backstepping control barrier functions \cite{BCBF}.

\subsection{Exponential Control Barrier Functions}
Consider a safety metric $h(\mathbf{x})$ of uniform relative degree $r\geq1$. Repeatedly differentiating $h(\mathbf{x})$ with respect to time results in terms $\mathcal{L}_g \mathcal{L}_f^{i} h(\mathbf{x})$ which are equal to zero for $i<r-1$ due to the relative degree assumption. However, by using negative constants (poles) $p_i<0$ and defining the series of functions $\nu_i: \mathcal{X} \rightarrow \mathbb{R}$ and corresponding superlevel sets $C_i$ for $i \in \{0,1, ...,r\}$ as

\small
\begin{align*}
    \nu_0(\mathbf{x}) &= h(\mathbf{x}), & C_0 &= \{ \mathbf{x} : \nu_0(\mathbf{x}) \geq 0 \}, \\
    \nu_1(\mathbf{x}) &= \dot{\nu}_0(\mathbf{x}) - p_1 \nu_0(\mathbf{x}), & C_1 &= \{ \mathbf{x} : \nu_1(\mathbf{x}) \geq 0 \}, \\
    &\vdots & &\vdots \\
    \nu_r(\mathbf{x}) &= \dot{\nu}_{r-1}(\mathbf{x}) - p_r \nu_{r-1}(\mathbf{x}), & C_r &= \{ \mathbf{x} : \nu_r(\mathbf{x}) \geq 0 \},
\end{align*}
\normalsize
it is shown in \cite{ECBF} that the following theorem holds

\begin{theorem}\cite{ames2019control}\label{thm1}
If $C_r$ is forward-invariant and $\mathbf{x}_0 \in \bigcap_{i=0}^r C_i$ then $\mathcal{C}_0$ is forward-invariant.
\end{theorem}
Note that Theorem \ref{thm1} additionally requires conditions of the initial state $\mathbf{x}_0$ to hold in addition to the invariance of $\mathcal{C}_r$ to ensure invariance of $\mathcal{C}_0$. If the function $\nu_r(\mathbf{x})$ is a \ac{cbf}, then $h(\mathbf{x})$ is said to be an exponential \ac{cbf} (ECBF).

\subsection{Composite Control Barrier Functions}
Enforcing multiple safety constraints simultaneously on a system can lead to practical challenges, especially if the number of constraints considered becomes large. In \cite{compositeCBFames}, a method to construct a single CBF as a logical AND-OR composition from multiple, different safety constraints through a softmin/softmax is described. This approach was also described in \cite{compositeCBFhoagg} for high-order systems with mixed relative degree. For the sake of brevity, we recite the method focusing on ECBFs. We consider a set of functions $\{h_i(\mathbf{x})\}_{i=1}^N$, where $h_i$ is an ECBF of relative degree $r_i$. Furthermore, we assume the safe set $\mathcal{S} = \bigcap_{i=1}^{N} \{\mathbf{x} : h_i(\mathbf{x}) \geq 0\}$ is nonempty. Defining the intermediate high-order functions

\small
\begin{equation}\label{ccbf1}
    \nu_{i,j+1} = \mathcal{L}_f \nu_{i,j} - p_{i,j} \nu_{i,j} ,
\end{equation}
\normalsize
where $\nu_{i,0} = h_i$, $j \in \{0,1, \hdots, r_{i-1} \}$ and $p_{i,j}<0$, we define the sets

\small
\begin{equation}\label{ccbf2}
    \mathcal{C}_{i,j} = \{\mathbf{x} : \nu_{i,j}(\mathbf{x}) \geq 0\} \quad \text{,} \quad C_i = \bigcap_{j=0}^{r_i-1} \mathcal{C}_{i,j}.
\end{equation}
\normalsize
By application of Theorem \ref{thm1} and Definition~\ref{def:cbf} we arrive at the following lemma:
\begin{lemma}\label{lemma1}
If $\mathbf{x}_0 \in \mathcal{C}_i$ and $\nu_{i,r_{i-1}}$ satisfies $\|\mathcal{L}_g \nu_{i,r_{i-1}}(\mathbf{x})\|>0~ \forall \mathbf{x} \in \mathcal{X}$, then the condition with $\alpha > 0$

\small
\begin{equation}\label{ccbf3}
    \mathcal{L}_f \nu_{i,r_{i-1}}(\mathbf{x}) + \mathcal{L}_g \nu_{i,r_{i-1}}(\mathbf{x}) \mathbf{u} \geq -\alpha \nu_{i,r_{i-1}}(\mathbf{x}) \quad \forall t \geq t_0
\end{equation}
\normalsize
implies $\mathbf{x} \in \mathcal{C}_i \quad~ \forall t \geq t_0$ and thus $h_i(\mathbf{x})\geq0$. Also, $\nu_{i,r_{i-1}}$ is a CBF of relative degree~1.
\end{lemma}

Our objective is to satisfy all constraints jointly, e.g. $\mathbf{x} \in \mathcal{S}$, which can be achieved by ensuring $\mathbf{x} \in \mathcal{C} = \bigcap_{i=i}^{N} \mathcal{C}_{i}$. In~\cite{compositeCBFames}, the intersection set $\mathcal{C}$ is compactly expressed through the $\min$ operation as $\{\mathbf{x} : \nu_{i,j}(\mathbf{x}) \geq 0\} = \{\mathbf{x} : \min_i \nu_{i,r_{i}-1}(\mathbf{x}) \geq 0\}$. Correspondingly, a new function using the soft minimum is defined in \cite{compositeCBFames} as a smooth under-approximation of the set $\mathcal{C}$ as

\begin{equation}\label{ccbf}
    h(\mathbf{x}) = - \frac{1}{\kappa} \log \sum_i e^{-\kappa \nu_{i,r_{i-1}}(\mathbf{x})},
\end{equation}
where we have $\{\mathbf{x} : h(\mathbf{x}) \geq 0\} \subseteq \mathcal{C}$ with the parameter $\kappa > 0$. The Lie derivatives of $h(\mathbf{x})$ are given by the functions

\small
\begin{equation}
    \mathcal{L}_f h(\mathbf{x}) = \sum_i \lambda_i(\mathbf{x}) \mathcal{L}_f h_i(\mathbf{x}) \text{,} \quad
    \mathcal{L}_g h(\mathbf{x}) = \sum_i \lambda_i(\mathbf{x}) \mathcal{L}_g h_i(\mathbf{x}),
\end{equation}
\normalsize
where $\lambda_i(\mathbf{x}) = e ^ {-\kappa (h_i(\mathbf{x}) - h(\mathbf{x}) )}$.

\begin{theorem}
The function given by \eqref{ccbf} is a CBF for the system \eqref{system} if and only if~\cite{compositeCBFames}

\small
\begin{equation}\label{theorem3_eq}
    \sum_i \lambda_i(\mathbf{x}) \mathcal{L}_g h_i(\mathbf{x}) = 0 \Longrightarrow \sum_i \lambda_i(\mathbf{x}) (\mathcal{L}_f h_i(\mathbf{x}) + \alpha  h_i(\mathbf{x})) \geq 0
\end{equation}
\normalsize
holds for $\alpha > 0$ and for all $\mathbf{x} \in \mathcal{X}$.
\end{theorem}

\begin{remark}\label{rmrk1}
By Lemma \ref{lemma1}, we have that $\nu_{i,r_{i-1}}(\mathbf{x})$ is a CBF and thus \eqref{theorem3_eq} trivially holds for $\lambda_p = 1$, $\lambda_i = 0 \quad \forall i \in \{1, \hdots ,N \} \setminus p$. This can be achieved by letting $\kappa \rightarrow \infty$ whenever $\nu_{p,r_{i-1}}(\mathbf{x}) < \nu_{i,r_{i-1}}(\mathbf{x}) \quad \forall i \in \{1, \hdots ,N \} \setminus p$. In practice, a large value of $\kappa$ is generally chosen to avoid evaluation of \eqref{theorem3_eq}. 
\end{remark}
        


    


% =============================================================================================
% =============================================================================================
% =============================================================================================

% =============================================================================================
% =============================================================================================
% =============================================================================================
\section{Composite CBF for Multirotor Collision Avoidance}\label{sec:approach}

\section{Problem Definition}
\label{sec:problem}


Our problem is a variant of the multivariate time series prediction that is commonly considered for traffic forecasting. 
We formulate a more general problem that can handle high sparsity and is not restricted by spatial structures such as roads. 
This flexible formulation enables the usage of unstructured observations that are moving and are inconsistent in time. 
Important for a proper reconstruction within sparse and unstructured observations is the detection of latent dependencies in a series of observations which can be applied across the sparse samples.
There is a high chance that those dependencies will only be partially part of a single sparse training sample which is in contrast to the common traffic prediction.

Let $\mathcal{S}$ be a spatial domain, which can be continuous or discrete, and $\mathcal{T} = \{t_0, \dots, t_{m-1}, t_{m}\}$ a temporal domain with $m+1$ discrete timesteps.
Then let $\mathbf{O}$ be a set of observations in $\mathcal{S}$ throughout the first $m$ timesteps $\mathcal{T}_{obs} = \mathcal{T} \setminus \{t_{m}\}$.
Each observation is a tuple of a time $t$, a spatial position $s$, and a vector of the observed values $\bm{y} \in \mathbb{R}^{d_f}$ with dimension $d_f \in \mathbb{N}$ :
\begin{equation}
    \mathbf{O} \subseteq \mathcal{T}_{obs} \times \mathcal{S} \times \mathbb{R}^{d_f}
\end{equation}
The high flexibility of the problem is expressed by the unstructured observations, a varying amount of observations for each timestep $t$, and observation positions which can change for each $t$.
High sparsity within the spatial and temporal domain can be modeled by this problem definition because it is possible to have few observations in a region that do not have to appear in any other past or future timestep. 

The goal is to predict for a subset of query locations $Q \subseteq \mathcal{S}$ the future values $\hat{\bm{y}} \in \mathbb{R}^{d_f}$ for the timestep $t_m \in \mathcal{T}$.
Therefore we learn a function $F: \mathcal{P}(\bm{O}) \times \mathcal{S} \to \mathbb{R}^{d_f}$ which predicts the future timestep from the observations $\mathbf{O}$ and the target position $s$. 
\begin{align}
    \hat{\bm{y}} = F(\bm{O}, s) ,& &  \forall \, s \in Q    
\end{align}
Note that $Q$ can be either a set of independent points of interest or a regular grid as frequently considered in previous works.
The problem challenges possible solutions to gather the information of the observations scattered across the timesteps $\mathcal{T}_{obs}$ and combine them to a reconstructed dense traffic state.
The combination is required because the observations can be arbitrarily scattered in space and in time.
Specifically, in the case of very sparse observations, a possible solution will have to learn spatio-temporal correlations, which are only partly represented in a single sample $o$ or are split between multiple samples and then fused by training over the complete data set.
For such a problem it is most important to fuse the data from previous sparse times steps for the prediction in $t_m$. 
While other problem definitions use nearly complete traffic states at each timestep as input, this problem definition forces the algorithm to exploit the temporal structures even more when the data is sparse.

Within common traffic prediction methods the road network and the information about the fixed sensor locations are often utilized, which we count as prior knowledge because the information is not dependent on the observations.
From a broader perspective such a usage of the road network to design or guide the graph graph construction as in \cite{Yu18, Zhou20} is an example of great informed machine learning algorithms \cite{vonrueden2023}.
Furthermore, our definition targets a more general case, where observations are not restricted to roads, but distributed in space without prior knowledge of the spatial structure. 
The road network is, at least in Static Graph solutions (see section \ref{sec:related}), explicitly incorporated, because it is freely available prior information.
Due to these requirements the solutions are strongly tied to the car traffic prediction task as other traffic predictions like ships or planes are not restricted in this way.
Considering traffic prediction in smaller cities with a lot fewer sensors on the road, we could handle moving sensors, e.g. sensors installed in cars.
This introduces a new complexity to the problem definition because hidden spatial structures have to be learned additionally.
% Without the power of the stationary sensors most recent works are not able to model all possible positions in a spatial domain as single graph nodes because there exist continuous spatial domains.
% We argue that such small changes require a more general problem definition as we introduced.



\section{Method}
\label{sec:method}


Our goal is to infer from sparse observations which are spatially distributed a complete traffic state also for spatial locations where no observations are available.
Therefore we need to learn and utilize spatio-temporal correlations that describe the state of unobserved regions from just a few present observations.
In section \ref{subsec:imagine} we describe how we achieve a latent dense spatial description for a traffic state represented by the input sample. 
The following section \ref{subsec:merge} describes how we create a modeled graph from these latent descriptions and use state-of-the-art methods to predict a future latent description. 
Finally, we query our latent description with query locations to receive values of interest, which is described in section \ref{subsec:query}.

\begin{figure*}
    \centering    \includegraphics[width=\textwidth]{figures/Architecture.pdf}
    \caption{Framework of SUSTeR with an observation encoding, a residual architecture for hidden traffic state reconstruction with a variable amount of observations and a decoding from the dense hidden traffic state into the original space.}
    \label{fig:architecture}
\end{figure*}



\subsection{Reconstruction of Traffic State}
\label{subsec:imagine}

We want SUSTeR to learn complex spatio-temporal correlations across multiple input samples which we aim to achieve with explicit spatio-temporal correlations.
This is important because due to the sparseness of the observations those correlations are not completely represented in a single input sample.
A graph is a good representation of locations connected by spatio-temporal correlations as it was done in many works before \cite{Zhou20, Yu18, Shao22}.
In early approaches each sensor location was modeled as a single graph node \cite{Li2018} while later work grouped similar locations to districts to learn more abstract correlations \cite{Li2021}. 
We extend this idea and use a fixed number of graph nodes set $V$. 
However, these do not correspond to a specific location or sensor and the information of the observations can be assigned freely to those nodes making the usage as flexible as possible.
Such a strategy gives the framework a higher degree of freedom to find similar regions with similar correlations and assign those to the same node.
From the point of our introduced problem in section \ref{sec:problem}, the observations can have variable locations and their positions can be unique throughout the entire data, which is why we need such a flexible solution.
Each node $v \in V$ is assigned to a row in $X_{i} \in \mathbb{R}^{|V| \times d_e}, \, t_i \in \mathcal{T}_{obs}$ which contains the embedding vector with $d_e \in \mathbb{N} $ dimensions and is used to encode latent information to all spatial points connected to this graph node.
The connectivity between the introduced graph nodes is described further in section \ref{subsec:merge}.

To tackle the problem of distributed correlations in the training data we introduce a context that can be used to remember similarities across training samples.
Context information in a spatio-temporal learning setup describes the overall environment.
For example, at Sunday 08:00 am the relevant context could be the weekend.
The temporal context models a broader influence on the traffic state which can differ greatly between weekends and weekdays.
We model the context as a function $C_{\theta}: \mathcal{T} \to \mathbb{R}^{\mid V \mid \times d_e}$ that maps the temporal context information of $t_0$, to our graph nodes as a bootstrap of the node embedding for estimating the initial traffic state $\widebar{X}$.
\begin{equation}
    \widebar{X} = C_{\theta}(t_0)
    \label{eq:mean_state}
\end{equation}
We mark all functions with learnable weight with $\theta$ representing the weights.
The context $\widebar{X}$ is consecutively assimilated to the following observations $\mathbf{O}$, as illustrated in figure \ref{fig:architecture}.
We use a residual structure to enrich $\widebar{X}$ in each timestep with the observations, adding information and reducing uncertainty.
With the function $enc_{\theta}: \mathbf{O} \times \mathbb{R}^{|V| \times d_e} \to \mathbb{R}^{|V| \times d_e}$ we compute the change $\Delta X$ that a given observation $o$ imparts on the previous traffic state $X$:
\begin{equation}
    \Delta X = enc_{\theta}(o, X) = sample_{\theta}(s) \cdot inf_{\theta}(o, X)^T,
\end{equation}
where $sample_{\theta}: \mathcal{S} \to \mathbb{R}^{|V|}$ creates a one-hot assignment to select a graph node, and $inf_{\theta}: \mathbf{O} \times \mathbb{R}^{|V| \times d_e} \to \mathbb{R}^{d_e}$ contains the residual information.
Please note the observations within a single timestep can not influence each other but can utilize the past encodings.
This allows for an architecture handling various amounts of observations throughout the time in a single sample.
% to make the assignment from an observation to the nodes interpretable.
% The multiplication between the output of $sample_{\theta}$ and $inf_{\theta}$ works as an assignment from the observation to a single graph node.
% In such way we create an assignment to a single graph node and the information propagation between those nodes is the task of an existing spatio-temporal correlation mining algorithm.

% This is similar to human intuition when driving through a city and an anomaly from the mean traffic state is observed, we already change our expectation in remote districts which could be influenced by our observation.
% With each new observation our uncertainty will decrease for certain areas which is determined by the location and context information of the observation.
% In contrast to an ego perspective, in this example our method can process multiple unrelated observations in the same timestep, which could result from traffic cameras or social media.
% From the design of our framework we do not connect observations in the same timestep because the framework should relay on past reconstructed traffic states and not have expectations on simultaneously occurring observations.
%\todo{Finden wir noch einen Platz?}

% One of our advantages is that this approach can handle varying sizes of $\mathbf{O}_i = \{o \mid t_i \in \mathcal{T}_{obs}\}$ which is the subset of observation at the same time $t_i$.
%\todo{Introdcution?}
%because we can not know if observations contain redundant information.
% A sum of redundant information can increase the confidence of the information expressed through the change $\Delta X$.
From the temporal node changes $\Delta X$ we create a sequence of graph node embeddings $(X_0, \dots, X_{m-1})$, which contains the residual information accumulated over all observations.
The state changes $\Delta X$ from all timesteps are aggregated as follows:
\begin{align}
X_i = 
\begin{cases}
    \widebar{X} + \,\, \mathlarger{\sum}\limits_{j=0}^{i-1} \, X_j + \mathlarger{\sum}\limits_{o \in O_i} enc_{\theta}(o, X_{i-1}) & i > 0\\[15pt]
    \widebar{X} + \mathlarger{\sum}\limits_{o \in O_i} enc_{\theta}(o, \widebar{X}) & i = 0
\end{cases}
\label{eq:Vi}
\end{align}
% Note that the function $enc_{\theta}$ has information available about the previous changes to the context information in $\widebar{X}$.



\subsection{Merging Graph Information}
\label{subsec:merge}

From the node embeddings $(X_{0}, \dots, X_{m-1})$ we define a sequence of graphs $\bm{G} = (G_{0}, \dots, G_{m-1})$, which have a common set of nodes $V$, and a common adjacency matrix $A$ representing weighted edges:
\begin{align}
    G_i = \left(V, A, X_i \right), & &  A \in \mathbb{R}^{|V| \times |V|}
\end{align}
In the following, we discuss how to learn $A$ from the node embeddings.
We use a self-adaptive adjacency matrix for SUSTeR, where $A$ is learned by the architecture itself because recent work showed that an adaptive adjacency matrix outperforms static road network adjacency \cite{Lan22, Wu2019, Bai20}.
Intuitively, two nodes in a spatio-temporal graph should have a strong edge when they are strongly correlated.
We adopt the idea from Bai et al. \cite{Bai20} and calculate the connectivity directly as the Laplacian from the similarity of the node embeddings.
To compute the Laplacian we use the last element of the node embedding sequence $X_{m-1}$, as it contains all the accumulated information by the nature of its construction (Eq. \ref{eq:Vi}). 
One could argue that for each graph $G_i$ the Laplacian from the node embedding $X_i$ could be used but we see the connectivity of the nodes and therefore the flow of information dependent on the overall situation which can only be represented by the finished accumulation of information.
The Laplacian $\mathcal{L}$ is computed as:
\begin{equation}
    \mathcal{L} = D^{-\frac{1}{2}} A D^{-\frac{1}{2}} = softmax\left(ReLU \left(X^{\,}_{m-1} \cdot X_{m-1}^T \right) \right)
\end{equation}

The resulting sequence of graphs $\bm{G}$ encodes all the information from the sparse observations including spatio-temporal correlations.
Note that the graphs in the final sequence only differ in their node embeddings $X_i$.

Because we transferred the sparse representation of our problem to a sequence of spatio-temporal graphs $\bm{G}$, we support any spatio-temporal graph neural network $\mathcal{X}_{\theta}$ for the aggregation of the correlation graphs, such as STGCN \cite{Yu18}, D2STGNN \cite{Shao22}, MegaCRNN \cite{jiang23}, Wavenet \cite{Wu2019}, etc.
In SUSTeR we use STGCN to provide the future graph $G_m$ at timestep $t_m$:
\begin{equation}
    G_m = \mathcal{X}_{\theta}(G_{0}, \dots, G_{m-1})
\end{equation}
The final graph $G_m = (V, A, X_m)$ encodes the reconstructed traffic state with fewer graph nodes than sensors.


\subsection{Querying of Locations}
\label{subsec:query}

A set of query locations $Q \subseteq \mathcal{S}$ can be chosen freely for traffic prediction.
The decoder function $dec_{\theta}: \mathbb{R}^{|V| \times d_e} \times \mathcal{S} \to \mathbb{R}^{d_f}$ predicts the next value $\hat{\bm{y}}$ for any location $s \in \mathcal{S}$.
\begin{align}
    \hat{\bm{y}} = dec_{\theta}(X_m, s),& &  \forall s \in Q
\end{align}
In the case of sparse traffic prediction, we choose the position of all traffic sensors from the original traffic datasets in order to evaluate the accuracy on ground truth.




% =============================================================================================
% =============================================================================================
% =============================================================================================

\section{Evaluation Studies}\label{sec:evaluation}
\begin{figure}
    \centering
    %\includegraphics[width=\linewidth]{figures/CBF_elektro.eps}
    \includegraphics[width=\linewidth]{figures/CBF_elektro_final.pdf}
    \vspace{-6ex}
    \caption{\small Constraint values (top) and velocity error w.r.t. the reference (center) and filter-induced deviation (bottom) during experiment A.}
    \label{fig:CBF_elektro}
    \vspace{-3ex}
\end{figure}

\subsection{Computational Scalability Study}
To evaluate the scalability of the proposed approach to cluttered environments with many obstacles, we evaluate the run times of the equations \eqref{composition1}, \eqref{composite_cbf} and \eqref{composite_invariance} for different environment sizes. Specifically, we ablate the required time for a single computation of $h(\mathbf{x})$, $\mathcal{L}_f h(\mathbf{x})$ and $\mathcal{L}_g h(\mathbf{x})$ in \eqref{composite_invariance} over different numbers of randomly placed obstacles. This is performed for the cases of computation of $\mathcal{L}_f h(\mathbf{x})$, $\mathcal{L}_g h(\mathbf{x})$ via the analytic gradients and via automatic gradient computation. Also, we evaluate this on a commercial laptop PC on CPU (Intel i7) and GPU (Nvidia RTX 3070Ti). All computations are performed in PyTorch. The processing times are displayed in table \ref{simulation_table}. Note that for fewer obstacles ($<10^3$) the processing time on CPU is lower than on GPU, while for more than $10^3$ obstacles, the computation benefits from the parallelism on the GPU.
These results show that the approach enables constructing the composite \ac{cbf} at high rates for environments with thousands of obstacles.

\begin{table}[t]
    \caption{Composition time of \ac{cbf} and Lie derivatives in milliseconds.}
    \label{simulation_table}    
    \centering
    \begin{tabular}{@{}llcccccc@{}}
        \toprule
        & & $10^1$ & $10^2$ & $10^3$ & $5 \cdot 10^3$ & $10^4$ \\
        \midrule
        \multirow{2}{*}{CPU} & Analytic & \textbf{0.264} & \textbf{0.279} & \textbf{0.485} & 2.575 & 3.353 \\
                             & Numeric  & 0.910 & 0.996 & 1.589 & 5.169 & 6.655 \\
        \midrule
        \multirow{2}{*}{GPU} & Analytic & 0.614 & 0.544 & 0.711 & \textbf{0.772} & \textbf{0.908} \\
                             & Numeric  & 2.606 & 2.703 & 2.760 & 2.653 & 2.624 \\
        \bottomrule
    \end{tabular}
    \vspace{-2ex}
    \iffalse
    \centering
    \begin{tabular}{ |p{0.5cm}|p{0.8cm}||p{0.8cm}|p{0.8cm}|p{0.8cm}|p{0.8cm}|p{0.8cm}|  }
     % \hline
     % \multicolumn{4}{|c|}{success rate [\%]} \\
     \hline
      \multicolumn{2}{|c|}{$N$} & $10^1$ &  $10^2$ & $10^3$ & $5\cdot10^3$ & $10^4$\\
     \hline
     CPU  & analytic  & \textbf{0.264} & \textbf{0.279} & \textbf{0.485} & 2.575 & 3.353\\
     \hline
     CPU  & numeric & 0.910 & 0.996 & 1.589 & 5.169 & 6.655\\
     \hline
     GPU  & analytic  & 0.614 & 0.544 & 0.711 & \textbf{0.772} & \textbf{0.908}\\
     \hline
     GPU  & numeric & 2.606 & 2.703 & 2.760 & 2.653 & 2.624\\
     \hline
    \end{tabular}
    \vspace{-4ex}
    \fi
\end{table}

\begin{figure*}[ht]
    \centering
    %\includegraphics[width=\linewidth]{figures/mission_dragvoll.eps}
    \includegraphics[width=\linewidth]{figures/mission_dragvoll_final.pdf}
    \caption{\small Aggregated map and path of experiment B. The mission starts at the cyan circle on the right. The quadrotor receives an adversarial velocity reference that actively tries to collide with obstacles. The red arrows depict this reference velocity for some selected time instances, A-D, which are reported in Fig. \ref{fig:CBF_dragvoll}.}
    \label{fig:mission_dragvoll}
    \vspace{-4ex}
\end{figure*}

\begin{figure}
    \centering
    %\includegraphics[width=\linewidth]{figures/CBF_dragvoll.eps}
    \includegraphics[width=\linewidth]{figures/CBF_dragvoll_final.pdf}
    \vspace{-6ex}
    \caption{\small Constraint values during experiment B.}
    \label{fig:CBF_dragvoll}
    \vspace{-3.5ex}
\end{figure}

\subsection{Experimental Implementation}
The experiments relied on a custom-built quadrotor from \cite{harms2024neural} with dimensions $0.52\times 0.52\times0.31~\textrm{m}$ and a takeoff mass of $2.58~\textrm{kg}$. The system integrates PX4-based autopilot avionics for low-level control, together with an NVIDIA Orin NX single-board computer, as well as an Ouster OS0-64 LiDAR and a VectorNav VN-100 IMU used for odometry estimation as in~\cite{khattak2020complementary}. The mapping method \cite{voxblox} is then used to create a voxel-grid representation of the environment at a resolution of 20cm. The closest 400 occupied voxels are then used as the obstacles $x_i$ in \eqref{position_cbf} and updated at 10Hz.
The proposed safe control law \eqref{ref_controller} is implemented in Python, where we performed a minor adjustment by using the adaptive position control law introduced in~\cite{TalINDI}. This attenuates tracking errors introduced by modelling uncertainties in the thrust coefficient. The composition of the composite \ac{cbf} and its Lie derivatives (equations \eqref{composition1} - \eqref{composite_cbf}) uses PyTorch and the safety QP \eqref{OCP} is implemented in Casadi~\cite{Andersson_Casadi} using the QP solver qpOASES~\cite{Ferreau_qpOASES}. Furthermore, an estimate of the current thrust $T_\text{est}$ estimate is obtained by low-pass filtering the previous thrust commands. As a control output, the vector $[\Omega^T,T_\text{est}+ \tau \Delta t ]^T$ is sent to the autopilot, where $\Delta t$ is the time interval between control updates.
The safety filter and reference controller run on the Orin NX CPU with an update rate of 100Hz with the parameters listed in Tab. \ref{tab:parameters}.
Experiments A and B presented hereafter can be seen in the supplementary video.

\subsection{Experiment A}
In the first experiment, the quadrotor receives a constant reference velocity of $[1,0,0]\unit{m/s}$  and reference height of $1.3\unit{m}$ in an obstacle-filled hallway. The cluttered environment forces the safety filter to become active during many short periods, deflecting the quadrotor away from obstacles (points A and B), above an obstacle (point C), while finally reaching a dead-end where it remains in hover (point D). The trajectory and environment are shown in Fig. \ref{fig:mission_elektro}. The resulting set function values are plotted in Fig. \ref{fig:CBF_elektro}. It can be seen that all original constraints and higher order sets are satisfied over the entire mission since the minimum over constraints of $\nu_{0,i}$, $\nu_{1,i}$, $\nu_{2,i}$ is positive over the entire mission. The \ac{cbf} value is mostly positive but displays slight crossings of $h(\mathbf{x})=0$. These minor violations are small and are expected in a real system due to modelling errors and time-varying observations. The velocity tracking errors induced by the safety filter are also visualized in Fig. \ref{fig:CBF_elektro}, showing clearly that the corrections induced by the safety filter induce tracking errors. It can also be seen from Fig. \ref{fig:CBF_elektro} that the safety filter acts on roll-rate ($p$), pitch-rate ($q$) and thrust-rate ($\tau$) to achieve safe actions. The results demonstrate the ability of the proposed safe control law to enforce constraint satisfaction over the entire mission.

\begin{table}[t]
    \centering
    \caption{Parameter values used in the experiments.}
    \label{tab:parameters}
    \setlength{\tabcolsep}{5.5pt} % Adjust column spacing
    \renewcommand{\arraystretch}{1} % Keep standard row spacing
    \begin{tabular}{c|cccccccc}
        \toprule
        \textbf{Parameter} & $p_0$ & $p_1$ & $\alpha_1$ & $\gamma$ & $\kappa$ & $\epsilon$ & $\alpha_2$  & $\epsilon_T$ \\
        \midrule
        \textbf{Value} & -3 & -2 & 1 & 40 & 20 & 0.5 & 5 & 7.5 \\
        \bottomrule
    \end{tabular}
    \vspace{-4ex}
\end{table}

\subsection{Experiment B}
The second experiment takes place in a forest with tree trunks
%, twigs, branches 
and foliage as natural obstacles. The reference velocity in this experiment is given by an adversarial operator, which intentionally attempts to collide the quadrotor with the surroundings. Horizontal and vertical references are given to provoke collisions with various trees and with the ground. A slight wind was present during the experiment, adding unmodelled disturbances. The trajectory and environment are shown in Fig. \ref{fig:mission_dragvoll}. The figure highlights four instances during the experiment where the unsafe references of the operator are corrected by the safety filter. The \ac{cbf} values are plotted in Fig. \ref{fig:CBF_dragvoll}, showing that the original constraints are all satisfied during the entire experiment. However, the value of $h_1$ often drops below the zero line due to the present disturbances. This experiment demonstrates the applicability of the proposed method in realistic challenging conditions.

% =============================================================================================
% =============================================================================================
% =============================================================================================
\section{Conclusions}\label{sec:conclusions}
This work presented a novel approach for safe navigation of multirotors in unknown, cluttered environments.
The proposed safety filter leverages a Composite \ac{cbf} formulation for synthesizing a single \ac{cbf} from an arbitrary number of 1D collision constraints representing point-wise obstacles.
The resulting \ac{cbf} is both computationally scalable, and is shown to be recursively feasible, except for a zero-volume set of infeasible configurations.
The proposed method is validated in two hardware experiments, in varying conditions, against an environment-agnostic and an adversarial policy.
Future work includes extending the method to a robust \ac{cbf} design to explicitly account for modelling uncertainty, and investigating approaches to also account for constraints on the control input, while retaining feasibility.

\bibliographystyle{IEEEtran}
\bibliography{references}



\end{document}
