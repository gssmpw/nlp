\begin{figure}
    \centering
    %\includegraphics[width=\linewidth]{figures/CBF_elektro.eps}
    \includegraphics[width=\linewidth]{figures/CBF_elektro_final.pdf}
    \vspace{-6ex}
    \caption{\small Constraint values (top) and velocity error w.r.t. the reference (center) and filter-induced deviation (bottom) during experiment A.}
    \label{fig:CBF_elektro}
    \vspace{-3ex}
\end{figure}

\subsection{Computational Scalability Study}
To evaluate the scalability of the proposed approach to cluttered environments with many obstacles, we evaluate the run times of the equations \eqref{composition1}, \eqref{composite_cbf} and \eqref{composite_invariance} for different environment sizes. Specifically, we ablate the required time for a single computation of $h(\mathbf{x})$, $\mathcal{L}_f h(\mathbf{x})$ and $\mathcal{L}_g h(\mathbf{x})$ in \eqref{composite_invariance} over different numbers of randomly placed obstacles. This is performed for the cases of computation of $\mathcal{L}_f h(\mathbf{x})$, $\mathcal{L}_g h(\mathbf{x})$ via the analytic gradients and via automatic gradient computation. Also, we evaluate this on a commercial laptop PC on CPU (Intel i7) and GPU (Nvidia RTX 3070Ti). All computations are performed in PyTorch. The processing times are displayed in table \ref{simulation_table}. Note that for fewer obstacles ($<10^3$) the processing time on CPU is lower than on GPU, while for more than $10^3$ obstacles, the computation benefits from the parallelism on the GPU.
These results show that the approach enables constructing the composite \ac{cbf} at high rates for environments with thousands of obstacles.

\begin{table}[t]
    \caption{Composition time of \ac{cbf} and Lie derivatives in milliseconds.}
    \label{simulation_table}    
    \centering
    \begin{tabular}{@{}llcccccc@{}}
        \toprule
        & & $10^1$ & $10^2$ & $10^3$ & $5 \cdot 10^3$ & $10^4$ \\
        \midrule
        \multirow{2}{*}{CPU} & Analytic & \textbf{0.264} & \textbf{0.279} & \textbf{0.485} & 2.575 & 3.353 \\
                             & Numeric  & 0.910 & 0.996 & 1.589 & 5.169 & 6.655 \\
        \midrule
        \multirow{2}{*}{GPU} & Analytic & 0.614 & 0.544 & 0.711 & \textbf{0.772} & \textbf{0.908} \\
                             & Numeric  & 2.606 & 2.703 & 2.760 & 2.653 & 2.624 \\
        \bottomrule
    \end{tabular}
    \vspace{-2ex}
    \iffalse
    \centering
    \begin{tabular}{ |p{0.5cm}|p{0.8cm}||p{0.8cm}|p{0.8cm}|p{0.8cm}|p{0.8cm}|p{0.8cm}|  }
     % \hline
     % \multicolumn{4}{|c|}{success rate [\%]} \\
     \hline
      \multicolumn{2}{|c|}{$N$} & $10^1$ &  $10^2$ & $10^3$ & $5\cdot10^3$ & $10^4$\\
     \hline
     CPU  & analytic  & \textbf{0.264} & \textbf{0.279} & \textbf{0.485} & 2.575 & 3.353\\
     \hline
     CPU  & numeric & 0.910 & 0.996 & 1.589 & 5.169 & 6.655\\
     \hline
     GPU  & analytic  & 0.614 & 0.544 & 0.711 & \textbf{0.772} & \textbf{0.908}\\
     \hline
     GPU  & numeric & 2.606 & 2.703 & 2.760 & 2.653 & 2.624\\
     \hline
    \end{tabular}
    \vspace{-4ex}
    \fi
\end{table}

\begin{figure*}[ht]
    \centering
    %\includegraphics[width=\linewidth]{figures/mission_dragvoll.eps}
    \includegraphics[width=\linewidth]{figures/mission_dragvoll_final.pdf}
    \caption{\small Aggregated map and path of experiment B. The mission starts at the cyan circle on the right. The quadrotor receives an adversarial velocity reference that actively tries to collide with obstacles. The red arrows depict this reference velocity for some selected time instances, A-D, which are reported in Fig. \ref{fig:CBF_dragvoll}.}
    \label{fig:mission_dragvoll}
    \vspace{-4ex}
\end{figure*}

\begin{figure}
    \centering
    %\includegraphics[width=\linewidth]{figures/CBF_dragvoll.eps}
    \includegraphics[width=\linewidth]{figures/CBF_dragvoll_final.pdf}
    \vspace{-6ex}
    \caption{\small Constraint values during experiment B.}
    \label{fig:CBF_dragvoll}
    \vspace{-3.5ex}
\end{figure}

\subsection{Experimental Implementation}
The experiments relied on a custom-built quadrotor from \cite{harms2024neural} with dimensions $0.52\times 0.52\times0.31~\textrm{m}$ and a takeoff mass of $2.58~\textrm{kg}$. The system integrates PX4-based autopilot avionics for low-level control, together with an NVIDIA Orin NX single-board computer, as well as an Ouster OS0-64 LiDAR and a VectorNav VN-100 IMU used for odometry estimation as in~\cite{khattak2020complementary}. The mapping method \cite{voxblox} is then used to create a voxel-grid representation of the environment at a resolution of 20cm. The closest 400 occupied voxels are then used as the obstacles $x_i$ in \eqref{position_cbf} and updated at 10Hz.
The proposed safe control law \eqref{ref_controller} is implemented in Python, where we performed a minor adjustment by using the adaptive position control law introduced in~\cite{TalINDI}. This attenuates tracking errors introduced by modelling uncertainties in the thrust coefficient. The composition of the composite \ac{cbf} and its Lie derivatives (equations \eqref{composition1} - \eqref{composite_cbf}) uses PyTorch and the safety QP \eqref{OCP} is implemented in Casadi~\cite{Andersson_Casadi} using the QP solver qpOASES~\cite{Ferreau_qpOASES}. Furthermore, an estimate of the current thrust $T_\text{est}$ estimate is obtained by low-pass filtering the previous thrust commands. As a control output, the vector $[\Omega^T,T_\text{est}+ \tau \Delta t ]^T$ is sent to the autopilot, where $\Delta t$ is the time interval between control updates.
The safety filter and reference controller run on the Orin NX CPU with an update rate of 100Hz with the parameters listed in Tab. \ref{tab:parameters}.
Experiments A and B presented hereafter can be seen in the supplementary video.

\subsection{Experiment A}
In the first experiment, the quadrotor receives a constant reference velocity of $[1,0,0]\unit{m/s}$  and reference height of $1.3\unit{m}$ in an obstacle-filled hallway. The cluttered environment forces the safety filter to become active during many short periods, deflecting the quadrotor away from obstacles (points A and B), above an obstacle (point C), while finally reaching a dead-end where it remains in hover (point D). The trajectory and environment are shown in Fig. \ref{fig:mission_elektro}. The resulting set function values are plotted in Fig. \ref{fig:CBF_elektro}. It can be seen that all original constraints and higher order sets are satisfied over the entire mission since the minimum over constraints of $\nu_{0,i}$, $\nu_{1,i}$, $\nu_{2,i}$ is positive over the entire mission. The \ac{cbf} value is mostly positive but displays slight crossings of $h(\mathbf{x})=0$. These minor violations are small and are expected in a real system due to modelling errors and time-varying observations. The velocity tracking errors induced by the safety filter are also visualized in Fig. \ref{fig:CBF_elektro}, showing clearly that the corrections induced by the safety filter induce tracking errors. It can also be seen from Fig. \ref{fig:CBF_elektro} that the safety filter acts on roll-rate ($p$), pitch-rate ($q$) and thrust-rate ($\tau$) to achieve safe actions. The results demonstrate the ability of the proposed safe control law to enforce constraint satisfaction over the entire mission.

\begin{table}[t]
    \centering
    \caption{Parameter values used in the experiments.}
    \label{tab:parameters}
    \setlength{\tabcolsep}{5.5pt} % Adjust column spacing
    \renewcommand{\arraystretch}{1} % Keep standard row spacing
    \begin{tabular}{c|cccccccc}
        \toprule
        \textbf{Parameter} & $p_0$ & $p_1$ & $\alpha_1$ & $\gamma$ & $\kappa$ & $\epsilon$ & $\alpha_2$  & $\epsilon_T$ \\
        \midrule
        \textbf{Value} & -3 & -2 & 1 & 40 & 20 & 0.5 & 5 & 7.5 \\
        \bottomrule
    \end{tabular}
    \vspace{-4ex}
\end{table}

\subsection{Experiment B}
The second experiment takes place in a forest with tree trunks
%, twigs, branches 
and foliage as natural obstacles. The reference velocity in this experiment is given by an adversarial operator, which intentionally attempts to collide the quadrotor with the surroundings. Horizontal and vertical references are given to provoke collisions with various trees and with the ground. A slight wind was present during the experiment, adding unmodelled disturbances. The trajectory and environment are shown in Fig. \ref{fig:mission_dragvoll}. The figure highlights four instances during the experiment where the unsafe references of the operator are corrected by the safety filter. The \ac{cbf} values are plotted in Fig. \ref{fig:CBF_dragvoll}, showing that the original constraints are all satisfied during the entire experiment. However, the value of $h_1$ often drops below the zero line due to the present disturbances. This experiment demonstrates the applicability of the proposed method in realistic challenging conditions.
