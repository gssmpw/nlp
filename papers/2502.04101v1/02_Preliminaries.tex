\subsection{Notation}
\begin{tabbing}
 \hspace*{2.2cm} \= \kill
  $\mathbf{x} \in \mathcal{X},  \mathbf{u} \in \mathcal{U}$ \>  state and input vectors \\[0.5ex]
  %\mathbf{u} \in \mathcal{U}$ \>  input vector \\[0.5ex]
  $\mathcal{L}_f h, \mathcal{L}_g h \mathbf{u}$ \>  Lie derivatives of $h$ along $f$, $g \mathbf{u}$ \\[0.5ex] 
  $\frac{d}{dt} V$, $\nabla_\mathbf{x} V$ \>  time derivative of $V$, gradient of $V$ w.r.t. $\mathbf{x}$ \\[0.5ex] 
  %$\nabla_\mathbf{x} V$ \>  gradient of $V$ w.r.t. $\mathbf{x}$ \\[0.5ex] 
  $\mathbf{a} \cdot \mathbf{b}$, $\mathbf{a} \times \mathbf{b}$ \>  dot product, cross product of $\mathbf{a}$ and $\mathbf{b}$ \\[0.5ex]  
  $\|\cdot\|$ \>  Euclidean norm \\[0.5ex] 
  $[\mathbf{a}]_{\times}$ \>  skew-symmetric matrix associated with $\mathbf{a}$ \\[0.5ex]
\end{tabbing}

\subsection{Control Barrier Functions}
Let us consider the general, control affine system

\small
\begin{equation}\label{system}
    \frac{d}{dt} \mathbf{x} = f(\mathbf{x}) + g(\mathbf{x}) \mathbf{u},
\end{equation}
\normalsize
where $\mathbf{x} \in \mathcal{X} \subseteq \mathbb{R}^n$ and $\mathbf{u} \in \mathcal{U} \subseteq \mathbb{R}^m$. We now consider a continuously differentiable function $h: \mathcal{X} \rightarrow \mathbb{R}$ with the property $\{\mathbf{x} | \frac{dh}{d\mathbf{x}}(\mathbf{x}) = 0\} \cap \{\mathbf{x} | h(\mathbf{x}) = 0\} = \emptyset$, which describes the superlevel set $\mathcal{X}_\text{safe} = \{\mathbf{x} \in \mathcal{X}  :  h(\mathbf{x}) \geq 0 \}$.

%\iffalse
\begin{definition}[Control Invariant Set]
    A set $\mathcal{X}_\text{safe}$ is a forward control invariant set for the system \eqref{system}, if for any $\mathbf{x}({t_0}) \in \mathcal{X}_\text{safe}$ there exists at least one input trajectory $\mathbf{u}(t) \in \mathcal{U}$ such that $\mathbf{x}({t}) \in \mathcal{X}_\text{safe} \quad \forall t\geq t_0$ under the system \eqref{system}.
\end{definition}
%\fi

\begin{definition}[Control Barrier Function \cite{ames2019control}]\label{def:cbf}
    Let $\mathcal{X}_\text{safe}$ be the $0$-superlevel set of a continuously differentiable function
    $h: \mathcal{X} \rightarrow \mathbb{R}$ with the property that $\mathcal{X}_\text{safe} = \{ \mathbf{x} | h(\mathbf{x})\geq 0 \}$. Then, $h$ is a \ac{cbf} if there exists an extended class $\mathcal{K}_\infty$ function $\alpha(\cdot)$ such that for the system \eqref{system} it holds:

    \small
    \begin{equation}\label{CBF_lie_condition}
        \underset{\mathbf{u} \in \mathcal{U}}{\text{sup}} [\mathcal{L}_f h(\mathbf{x}) + \mathcal{L}_g h(\mathbf{x}) \mathbf{u}] \geq -\alpha(h(\mathbf{x}))
    \end{equation}
    \normalsize
    for all $\mathbf{x} \in \mathcal{X}_\text{safe}$. Further, an input $\mathbf{u}$ is considered safe with respect to a valid CBF, if it satisfies \eqref{CBF_lie_condition}.
\end{definition}

From the above, it becomes clear that $\mathcal{X}_\text{safe}$ is a control invariant set for the system \eqref{system} subject to any control law satisfying \eqref{CBF_lie_condition}. See \cite{ames2016control} for a proof. Condition \eqref{CBF_lie_condition} requires a nonzero Lie derivative, e.g. $\mathcal{L}_g h(\mathbf{x}) \neq 0$ in general, restricting the use of \acp{cbf} to functions $h$ of relative degree~1. Systematic approaches to construct a CBF with relative degree greater than one include exponential control barrier functions \cite{ECBF}, high-order control barrier functions \cite{HOCBF} and backstepping control barrier functions \cite{BCBF}.

\subsection{Exponential Control Barrier Functions}
Consider a safety metric $h(\mathbf{x})$ of uniform relative degree $r\geq1$. Repeatedly differentiating $h(\mathbf{x})$ with respect to time results in terms $\mathcal{L}_g \mathcal{L}_f^{i} h(\mathbf{x})$ which are equal to zero for $i<r-1$ due to the relative degree assumption. However, by using negative constants (poles) $p_i<0$ and defining the series of functions $\nu_i: \mathcal{X} \rightarrow \mathbb{R}$ and corresponding superlevel sets $C_i$ for $i \in \{0,1, ...,r\}$ as

\small
\begin{align*}
    \nu_0(\mathbf{x}) &= h(\mathbf{x}), & C_0 &= \{ \mathbf{x} : \nu_0(\mathbf{x}) \geq 0 \}, \\
    \nu_1(\mathbf{x}) &= \dot{\nu}_0(\mathbf{x}) - p_1 \nu_0(\mathbf{x}), & C_1 &= \{ \mathbf{x} : \nu_1(\mathbf{x}) \geq 0 \}, \\
    &\vdots & &\vdots \\
    \nu_r(\mathbf{x}) &= \dot{\nu}_{r-1}(\mathbf{x}) - p_r \nu_{r-1}(\mathbf{x}), & C_r &= \{ \mathbf{x} : \nu_r(\mathbf{x}) \geq 0 \},
\end{align*}
\normalsize
it is shown in \cite{ECBF} that the following theorem holds

\begin{theorem}\cite{ames2019control}\label{thm1}
If $C_r$ is forward-invariant and $\mathbf{x}_0 \in \bigcap_{i=0}^r C_i$ then $\mathcal{C}_0$ is forward-invariant.
\end{theorem}
Note that Theorem \ref{thm1} additionally requires conditions of the initial state $\mathbf{x}_0$ to hold in addition to the invariance of $\mathcal{C}_r$ to ensure invariance of $\mathcal{C}_0$. If the function $\nu_r(\mathbf{x})$ is a \ac{cbf}, then $h(\mathbf{x})$ is said to be an exponential \ac{cbf} (ECBF).

\subsection{Composite Control Barrier Functions}
Enforcing multiple safety constraints simultaneously on a system can lead to practical challenges, especially if the number of constraints considered becomes large. In \cite{compositeCBFames}, a method to construct a single CBF as a logical AND-OR composition from multiple, different safety constraints through a softmin/softmax is described. This approach was also described in \cite{compositeCBFhoagg} for high-order systems with mixed relative degree. For the sake of brevity, we recite the method focusing on ECBFs. We consider a set of functions $\{h_i(\mathbf{x})\}_{i=1}^N$, where $h_i$ is an ECBF of relative degree $r_i$. Furthermore, we assume the safe set $\mathcal{S} = \bigcap_{i=1}^{N} \{\mathbf{x} : h_i(\mathbf{x}) \geq 0\}$ is nonempty. Defining the intermediate high-order functions

\small
\begin{equation}\label{ccbf1}
    \nu_{i,j+1} = \mathcal{L}_f \nu_{i,j} - p_{i,j} \nu_{i,j} ,
\end{equation}
\normalsize
where $\nu_{i,0} = h_i$, $j \in \{0,1, \hdots, r_{i-1} \}$ and $p_{i,j}<0$, we define the sets

\small
\begin{equation}\label{ccbf2}
    \mathcal{C}_{i,j} = \{\mathbf{x} : \nu_{i,j}(\mathbf{x}) \geq 0\} \quad \text{,} \quad C_i = \bigcap_{j=0}^{r_i-1} \mathcal{C}_{i,j}.
\end{equation}
\normalsize
By application of Theorem \ref{thm1} and Definition~\ref{def:cbf} we arrive at the following lemma:
\begin{lemma}\label{lemma1}
If $\mathbf{x}_0 \in \mathcal{C}_i$ and $\nu_{i,r_{i-1}}$ satisfies $\|\mathcal{L}_g \nu_{i,r_{i-1}}(\mathbf{x})\|>0~ \forall \mathbf{x} \in \mathcal{X}$, then the condition with $\alpha > 0$

\small
\begin{equation}\label{ccbf3}
    \mathcal{L}_f \nu_{i,r_{i-1}}(\mathbf{x}) + \mathcal{L}_g \nu_{i,r_{i-1}}(\mathbf{x}) \mathbf{u} \geq -\alpha \nu_{i,r_{i-1}}(\mathbf{x}) \quad \forall t \geq t_0
\end{equation}
\normalsize
implies $\mathbf{x} \in \mathcal{C}_i \quad~ \forall t \geq t_0$ and thus $h_i(\mathbf{x})\geq0$. Also, $\nu_{i,r_{i-1}}$ is a CBF of relative degree~1.
\end{lemma}

Our objective is to satisfy all constraints jointly, e.g. $\mathbf{x} \in \mathcal{S}$, which can be achieved by ensuring $\mathbf{x} \in \mathcal{C} = \bigcap_{i=i}^{N} \mathcal{C}_{i}$. In~\cite{compositeCBFames}, the intersection set $\mathcal{C}$ is compactly expressed through the $\min$ operation as $\{\mathbf{x} : \nu_{i,j}(\mathbf{x}) \geq 0\} = \{\mathbf{x} : \min_i \nu_{i,r_{i}-1}(\mathbf{x}) \geq 0\}$. Correspondingly, a new function using the soft minimum is defined in \cite{compositeCBFames} as a smooth under-approximation of the set $\mathcal{C}$ as

\begin{equation}\label{ccbf}
    h(\mathbf{x}) = - \frac{1}{\kappa} \log \sum_i e^{-\kappa \nu_{i,r_{i-1}}(\mathbf{x})},
\end{equation}
where we have $\{\mathbf{x} : h(\mathbf{x}) \geq 0\} \subseteq \mathcal{C}$ with the parameter $\kappa > 0$. The Lie derivatives of $h(\mathbf{x})$ are given by the functions

\small
\begin{equation}
    \mathcal{L}_f h(\mathbf{x}) = \sum_i \lambda_i(\mathbf{x}) \mathcal{L}_f h_i(\mathbf{x}) \text{,} \quad
    \mathcal{L}_g h(\mathbf{x}) = \sum_i \lambda_i(\mathbf{x}) \mathcal{L}_g h_i(\mathbf{x}),
\end{equation}
\normalsize
where $\lambda_i(\mathbf{x}) = e ^ {-\kappa (h_i(\mathbf{x}) - h(\mathbf{x}) )}$.

\begin{theorem}
The function given by \eqref{ccbf} is a CBF for the system \eqref{system} if and only if~\cite{compositeCBFames}

\small
\begin{equation}\label{theorem3_eq}
    \sum_i \lambda_i(\mathbf{x}) \mathcal{L}_g h_i(\mathbf{x}) = 0 \Longrightarrow \sum_i \lambda_i(\mathbf{x}) (\mathcal{L}_f h_i(\mathbf{x}) + \alpha  h_i(\mathbf{x})) \geq 0
\end{equation}
\normalsize
holds for $\alpha > 0$ and for all $\mathbf{x} \in \mathcal{X}$.
\end{theorem}

\begin{remark}\label{rmrk1}
By Lemma \ref{lemma1}, we have that $\nu_{i,r_{i-1}}(\mathbf{x})$ is a CBF and thus \eqref{theorem3_eq} trivially holds for $\lambda_p = 1$, $\lambda_i = 0 \quad \forall i \in \{1, \hdots ,N \} \setminus p$. This can be achieved by letting $\kappa \rightarrow \infty$ whenever $\nu_{p,r_{i-1}}(\mathbf{x}) < \nu_{i,r_{i-1}}(\mathbf{x}) \quad \forall i \in \{1, \hdots ,N \} \setminus p$. In practice, a large value of $\kappa$ is generally chosen to avoid evaluation of \eqref{theorem3_eq}. 
\end{remark}