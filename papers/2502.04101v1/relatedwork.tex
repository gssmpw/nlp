\section{Related Work}
\label{sec:relatedwork}

Due to the inherent safety concern in aerial robots, multiple prior works have considered safe control for multirotors using \acp{cbf}. One earlier work employs multiple exponential \acp{cbf} inside a safety filter \ac{qp} for a linear, second-order system to obtain safe acceleration commands for a double integrator. In \cite{cascaded_CBF}, exponential \acp{cbf} are again used to formulate a safety filter using two cascaded Quadratic Programs (QP) for thrust and torque constraints. The cascaded structure is reported to lead to feasibility issues. In \cite{range_sensing_CBF}, the authors design a safe control law for a multirotor using \acp{cbf} with configuration constraints and collision constraints. This approach only manipulates thrust for obstacle avoidance, while attitude is only altered to satisfy the configuration constraint.
The authors of \cite{cbf_potential_field_analysis} perform a comparative analysis of \acp{cbf} and potential fields for obstacle avoidance, demonstrating results in a 2D planar quadrotor using a safety filter.
In the recent work \cite{backstepping_CBF}, a \ac{cbf} backstepping approach is used to impose multiple, convex constraints for position, velocity and rates for a multirotor in simulation. In \cite{collisionConeCBF}, a novel type of \ac{cbf} is proposed for the avoidance of dynamic obstacles. However, this formulation disallows directly approaching obstacles, which leads to feasibility issues in highly cluttered environments.
In \cite{compositeCBFtemporal}, the authors propose using the recently introduced composite \acp{cbf} for control barrier synthesis in unknown environments, demonstrating the applicability in simulation on a linear, double integrator model of a quadrotor in a 2D map.
In \cite{CBF_aided_teleop}, a teleoperation scheme for safe position control of a quadrotor, modelled as a linear system, is presented, where the local occupancy map is represented as an \ac{sdf} and used as a discrete-time \ac{cbf}.
%However, the system in this work is modelled as a linear system, making the use with more dynamic vehicles problematic.
Methods for learning \acp{cbf} have also been recently proposed to achieve collision avoidance. Here, the sampling-based approach shown in \cite{dawson2022learning} demonstrate safe control for a first order systems, while the work in \cite{harms2024neural} has been applied to a quadrotor modelled as a double integrator but has yet to demonstrated for systems of order 3 and higher.