

Autonomous aerial robots, such as multirotor platforms, are widely used in scientific and industrial contexts alike. A diverse set of works has focused on solving the core technological challenges among which safe autonomous navigation in unknown environments is of paramount importance. A multitude of methods for collision avoidance exists that include reactive schemes~\cite{alejo2016reactive}, control strategies accounting for the presence of obstacles as constraints~\cite{dentler2016real,alexis2016robust}, learning~\cite{kahn2017uncertainty,kulkarni2024reinforcement}, and map-based motion planning~\cite{karaman2011sampling,allen2016real}. Despite the unprecedented progress, ensuring the safe autonomous navigation of aerial robots in complex environments remains a pertinent challenge.

Control-driven methods and safety filters represent an appealing strategy for collision-free safe flight in that they correspond to a key layer of the autonomy stack that typically runs at high update rates while integrating the system's dynamics natively. Potentially working in synergy with other approaches, such as map-based motion planning, they represent a formal last-resort safety mechanism even when higher-level methods fail, e.g. due to global map drift or planning at low rates. Successful examples of control-driven methods applied to multirotor collision avoidance are \ac{mpc} and \acp{cbf}. 

However, control-oriented methods to ensure collision avoidance of aerial robots often struggle to scale to cluttered and obstacle-rich environments due to the number or complexity of the imposed constraints, which becomes prohibitive for high-rate control onboard computationally constrained platforms~\cite{alexis2016robust}. Safety filters based on \acp{cbf} often pose a computationally cheap alternative to \ac{mpc}. Nevertheless, finding an admissible \ac{cbf} for high-order systems is a challenging task.

Aiming to overcome these challenges, the contribution of this work is two-fold. First, we propose a computationally scalable safety filter using a composite \ac{cbf} representing all collision constraints. Second, the proposed approach is formulated for a third-order nonlinear system, which allows the formulation of a safety filter without requiring dynamic inversion of the nonlinear attitude dynamics, and a better representation of the actual system constraints such as positive thrust values.
The method is first validated by demonstrating its scalability against the order of magnitude of the number of position constraints. The safety filter is then tested in hardware experiments, indoors and outdoors, both when guided by a naive policy that is agnostic to obstacles and when guided by an adversarial policy trying to collide with the environment.

The remainder of this work is structured as follows: Sec.~\ref{sec:relatedwork} presents an overview of the related literature, while Sec.~\ref{sec:preliminaries} provides an overview on \ac{cbf} theory. Sec.~\ref{sec:approach} introduces the problem statement, proposed method, and a brief analysis thereof. The experimental results are detailed in Sec.~\ref{sec:evaluation} and concluding remarks are presented in Sec.~\ref{sec:conclusions}.


\section{Related Work}\label{sec:relatedwork}

Due to the inherent safety concern in aerial robots, multiple prior works have considered safe control for multirotors using \acp{cbf}. One earlier work employs multiple exponential \acp{cbf} inside a safety filter \ac{qp} for a linear, second-order system to obtain safe acceleration commands for a double integrator. In \cite{cascaded_CBF}, exponential \acp{cbf} are again used to formulate a safety filter using two cascaded Quadratic Programs (QP) for thrust and torque constraints. The cascaded structure is reported to lead to feasibility issues. In \cite{range_sensing_CBF}, the authors design a safe control law for a multirotor using \acp{cbf} with configuration constraints and collision constraints. This approach only manipulates thrust for obstacle avoidance, while attitude is only altered to satisfy the configuration constraint.
The authors of \cite{cbf_potential_field_analysis} perform a comparative analysis of \acp{cbf} and potential fields for obstacle avoidance, demonstrating results in a 2D planar quadrotor using a safety filter.
In the recent work \cite{backstepping_CBF}, a \ac{cbf} backstepping approach is used to impose multiple, convex constraints for position, velocity and rates for a multirotor in simulation. In \cite{collisionConeCBF}, a novel type of \ac{cbf} is proposed for the avoidance of dynamic obstacles. However, this formulation disallows directly approaching obstacles, which leads to feasibility issues in highly cluttered environments.
In \cite{compositeCBFtemporal}, the authors propose using the recently introduced composite \acp{cbf} for control barrier synthesis in unknown environments, demonstrating the applicability in simulation on a linear, double integrator model of a quadrotor in a 2D map.
In \cite{CBF_aided_teleop}, a teleoperation scheme for safe position control of a quadrotor, modelled as a linear system, is presented, where the local occupancy map is represented as an \ac{sdf} and used as a discrete-time \ac{cbf}.
%However, the system in this work is modelled as a linear system, making the use with more dynamic vehicles problematic.
Methods for learning \acp{cbf} have also been recently proposed to achieve collision avoidance. Here, the sampling-based approach shown in \cite{dawson2022learning} demonstrate safe control for a first order systems, while the work in \cite{harms2024neural} has been applied to a quadrotor modelled as a double integrator but has yet to demonstrated for systems of order 3 and higher.

