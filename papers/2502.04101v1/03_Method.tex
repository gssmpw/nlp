In the following section, we propose a safety filter designed to ensure safety of the system under multiple state constraints. The intersection of control invariant set is in general not control invariant and imposing multiple \acp{cbf} across a control structure may lead to feasibility issues as reported in prior work \cite{cascaded_CBF} in the case of multirotors. Contrasting to prior work, we explicitly analyse the feasibility, showing that the infeasible set is a null-set on $\mathbb{R}^n$.

In the following, we consider the system model of a rate-controlled planar multirotor as in \cite{backstepping_CBF}, described by

\vspace{-3ex}
\small
\begin{subequations}\label{dynamics}
\begin{align}
    \dot x &= v, \label{position_dynamics} \\
    \dot v &= ge_3 - \frac{T}{m} R e_3, \label{linvel_dynamics} \\
    \dot R &= R[\Omega]_{\times}, \label{attitude_dynamics} \\
    \dot T &= \tau, \label{angvel_dynamics}
\end{align}
\end{subequations}
\normalsize
where $g$ is the intensity of gravity, and the artificial control dynamics $\dot T = \tau$ is introduced to achieve a uniform relative degree $r=3$ of any function of $x$ in the input variables $\mathbf{u} = [\Omega^T, \tau]^T$.
%This modification will later allow us to use the full set of input variables for obstacle avoidance.
\begin{figure}
    \centering
    \includegraphics[width=0.35\linewidth]{figures/conventions-crop.pdf}
    \hspace{1cm}
    \includegraphics[width=0.4\linewidth]{figures/singularity-1-crop.pdf}
    \caption{\small \textit{Left}: Coordinate conventions utilized. All equations are expressed in North-East-Down (NED) and Front-Right-Down (FRD) frame. \textit{Right}: Singular configuration example. A necessary requirement for conflicting constraints in \eqref{OCP} is that the virtual obstacle (orange) lies on the positive thrust axis (shown in red).}
    \label{fig:conventions}
    \vspace{-6ex}
\end{figure}
%
\subsection{Reference Controller}
As a reference controller, we propose a simple adaptation of the commonly used geometric controller \cite{Lee} to account for the change in system dynamics \eqref{dynamics}. The proposed geometric control law becomes

\small
\begin{subequations}
\begin{align}
    \Omega &= -k_R \frac{1}{2}[R_d^tR - R^T R_d]\textsuperscript{$\wedge$} + R^T R_d \Omega_{d}, \label{lee_attitude}\\
    T_d &= - (K_x e_x + K_v e_v + m g e_3 - m \Ddot{x})^T R e_3,\label{lee_position} \\
    \tau &= k_T (T_d - T) + \dot{T}_d, \label{lee_thrust}
\end{align}
\end{subequations}
\normalsize
where $(\cdot)$\textsuperscript{$\wedge$} is the vectorize operator. While it is straightforward to show global exponential stability of the thrust error $T_d - T$, local exponential stability of the attitude error $ \frac{1}{2}\text{tr}[\mathbb{I} - R_d^T R]$ can be shown in fashion similar to \cite{LeeProof}.

\subsection{Constraints}
In this work, we seek to enable safe control of multirotors in arbitrarily cluttered environments. To circumvent representing complex geometries by means of more elaborate functions, we choose to represent the collision constraint as a composition of multiple, simple constraints, describing the free space $\mathcal{C}_x$ as

\small
\begin{equation}\label{collision_constraint}
    \mathcal{C}_x = \bigcap_{i=1}^{N} \{x : \|x - x_i\| \geq \epsilon\} \text{,}
\end{equation}
\normalsize
where $x_i$ represents the obstacle locations and $\epsilon>0$. Furthermore, the model described above does not have any limit on the thrust value $T$. To account for the inability of most multirotors to exhibit negative thrust values, we also consider a thrust constraint set $\mathcal{C}_T$ as

\small
\begin{equation}\label{thrust_constraint}
    \mathcal{C}_T = \{ T : T \geq T_\text{min} \},
\end{equation}
\normalsize
for some positive constant $T_\text{min}\ge0$.
We do not consider additional input bounds on $\mathbf{u}$ and leave this to future work. Our goal is to ensure $x \in \mathcal{C}_x$ and $T \in \mathcal{C}_T$ at all times during operation, i.e. ensure invariance of the set $\mathcal{C}_x \cap \mathcal{C}_T$.

\subsection{Control Barrier Function Design}
To ensure $\mathbf{x} \in \mathcal{C}_x$ with obstacles located at $x_i$ for $i\in \{1, \hdots, N \}$, we define a set of functions as

\begin{equation}\label{position_cbf}
   \nu_{i,0}(\mathbf{x}) = ||(x-x_i)||^2 - \epsilon^2.
\end{equation}

Noting that $\nu_{i,0}$ have relative degree 3 in all control inputs, we define the functions, given two constants $p_0, p_1 < 0$,

\small
\begin{subequations}\label{composition1}
    \begin{equation}
        \nu_{i,1} = \mathcal{L}_f \nu_{i,0} - p_{0} \nu_{i,0},
    \end{equation}
    \begin{equation}
        \nu_{i,2} = \mathcal{L}_f \nu_{i,1} - p_{1} \nu_{i,1}.
    \end{equation}
\end{subequations}
\normalsize

Using lemma \ref{lemma1}, we note that $\nu_{i,2}$ are \acp{cbf} of relative degree 1 ($\|\mathcal{L}_g \nu_{i,2}(\mathbf{x})\|>0$ holds everywhere except for $x=x_i$, which is not a meaningful scenario). Constructing the composite \ac{cbf} as in \eqref{ccbf} with the additional saturation through the $\tanh$ function as proposed in \cite{compositeCBFames} with $\gamma>0$ yields the \ac{cbf} candidate

\small
\begin{equation}\label{composite_cbf}
    h_1(\mathbf{x}) = - \frac{\gamma}{\kappa} \log \sum_i e^{-\kappa 
\tanh{\frac{ \nu_{i,2}(\mathbf{x})}{\gamma}}}.
\end{equation}
\normalsize
To ensure invariance of the set $\mathcal{C}_x$, the standard condition 
\eqref{CBF_lie_condition} with \eqref{composite_cbf} becomes using $\alpha_1 > 0$ and $\lambda_i(\mathbf{x}) = e ^ {-\kappa (\nu_{i,2}(\mathbf{x}) - h_1(\mathbf{x}) )}$:


\small
\begin{equation}\label{composite_invariance}
\begin{split}
    \alpha_1 h_1(\mathbf{x}) +
    \underbrace{\frac{2T}{m}(x- \sum_i \frac{\lambda_i(\mathbf{x})}{\cosh^2(\frac{\nu_{i,2}(\mathbf{x})}{\gamma})} x_i)^T R [  [e_3]_\times, - e_3]}_{\mathcal{L}_g h_1(\mathbf{x})} \cdot \mathbf{u} \geq \\
    - \underbrace{ \sum_i \frac{\lambda_i(\mathbf{x})}{\cosh^2(\frac{\nu_{i,2}(\mathbf{x})}{\gamma})} (\mathcal{L}_f^3 \nu_{i,0} (\mathbf{x}) - p_1\mathcal{L}_f^2 \nu_{i,0} (\mathbf{x}) + p_1 p_0\mathcal{L}_f \nu_{i,0} (\mathbf{x}))}_{\mathcal{L}_f h_1(\mathbf{x})} \text{.}
\end{split}
\end{equation}
\normalsize
The structure of the above constraint regressor $\mathcal{L}_g h(\mathbf{x})$ allows the interpretation of a ``virtual obstacle'' at the location 

\small
\begin{equation}
    \hat{x}(\mathbf{x}) =  \sum_i \frac{\lambda_i(\mathbf{x})}{\cosh^2(\frac{\nu_{i,2}(\mathbf{x})}{\gamma})} x_i.
\end{equation}
\normalsize
The relative distance vector $(x- \hat{x}(\mathbf{x}))$ can be seen as the virtual obstacle direction, constraining the direction of the control input $\mathbf{u}$.
The virtual obstacle is therefore a weighted average of the original constraints $\nu_{i,0}$
with weighting terms $\lambda_i (\mathbf{x})$. Furthermore, the constraint regressor above is nonzero as long as $x \neq \hat{x}(\mathbf{x})$ since the column vectors of $ [[e_3]_\times, - e_3]$ form a basis of $\mathbb{R}^3$. We therefore have that $h(\mathbf{x})$ is a \ac{cbf} from (\ref{theorem3_eq}) iff $x = \hat{x}(\mathbf{x}) \Longrightarrow \mathcal{L}_f h(\mathbf{x}) + \alpha  h(\mathbf{x}) \geq 0$. While an online evaluation of this condition is impractical due to the non-fixed and arbitrary configuration of obstacles, we resort to the argument taken in Remark~\ref{rmrk1}, i.e. that for one $i$, $\underset{\kappa \rightarrow \infty}{\lim} h_1(\mathbf{x}) = \gamma \tanh({{ \nu_{i,2}(\mathbf{x})}/{\gamma}})$, while $\nu_{i,2}$ is a \ac{cbf}.

To ensure the thrust constraint by invariance of $\mathcal{C}_T$, we introduce the \ac{cbf} candidate with relative degree 1

\small
\begin{equation}\label{thrust_cbf}
    h_2(\mathbf{x}) = T - \epsilon_T.
\end{equation}
\normalsize
The corresponding invariance condition \eqref{CBF_lie_condition} becomes with  $\alpha_2 > 0$

\small
\begin{equation}\label{thrust_invariance}
     \underbrace{[\mathbb{O}^{1\times3}, 1]}_{\mathcal{L}_g h_2\mathbf{x})} \cdot \mathbf{u} \geq \underbrace{-\alpha_2 h_2 (\mathbf{x}) - \mathcal{L}_f h_2(\mathbf{x})}_{b_2(\mathbf{x})}.
\end{equation}
\normalsize
Looking at the constraint regressor, we see that $\mathcal{L}_g h(\mathbf{x})\neq 0 \quad \forall \mathbf{x} \in \mathcal{X}$, so \eqref{thrust_cbf} indeed is a \ac{cbf}.

We now proceed with a feasibility analysis to show that the stacking of the constraints only has an insignificant violation space, which is given by a zero-volume null set.



\subsection{Recursive Feasibility}

We analyze the feasibility of the constraints \eqref{composite_invariance} and \eqref{thrust_invariance} by considering the constraint regressors $\mathcal{L}_g h_1(\tilde{\mathbf{x}}) = [l_1(\mathbf{x}), l_2(\mathbf{x}), 0, l_3(\mathbf{x})]$ and $\mathcal{L}_g h_2(\mathbf{x})$. To result in conflicting constraints, we require $\mathcal{L}_g h_1(\mathbf{x}) \cdot \mathcal{L}_g h_2(\mathbf{x})^T <0$ which is true only when 

\small
\begin{equation}\label{infeasibility1}
    \frac{2T}{m}(x- \hat{x}(\mathbf{x}))^T R e_3 >0 \text{,}
\end{equation}
\normalsize
if $T > 0$. Thus, the virtual obstacle must be 'above' the quadrotor. Furthermore, the constraints can only conflict if

\small
\begin{equation}\label{infeasibility2}
    (x- \hat{x}(\mathbf{x})) \times R e_3 = 0^{3 \times 1}
\end{equation}
\normalsize
since otherwise the constraint sensitivity on roll and pitch in \eqref{composite_invariance} is nonzero. The conflicting cases (denoted as $\tilde{\mathbf{x}}$) therefore require that the thrust vector $-T e_3$ and the virtual obstacle-relative vector $(x- \hat{x}(\mathbf{x}))$ are colinear. In this case, we have $\mathcal{L}_g h_1(\tilde{\mathbf{x}}) = -\beta \mathcal{L}_g h_2(\tilde{\mathbf{x}})$ for $\beta\geq 0$ and thus $l_1(\mathbf{x})=0$, $l_2(\mathbf{x})=0$. This configuration is illustrated in Fig. \ref{fig:conventions}.


By studying the sensitivity of constraint \eqref{infeasibility2} for the idealized case (Remark~\ref{rmrk1}), we arrive at the following result:
\begin{proposition}
    The set of states $\mathcal{X}_\text{singular} \subset \mathcal{X}$ satisfying equation \eqref{infeasibility2} is a zero-volume set in $\mathcal{X}$ for $x \neq \hat{x}(\mathbf{x})$ when. That is $\mathcal{X}_\text{singular}$ is a null set.
\end{proposition}

%\textit{Proof:} 
\begin{proof}
Assume we have a solution $\tilde{\mathbf{x}} \in \mathcal{X}_\text{singular}$, denoted by $(x- \hat{x}(\tilde{\mathbf{x}})) \times R e_3 = q = 0^{3 \times 1}$.
For $\mathcal{X}_\text{singular}$ to be a zero volume set, we require $\frac{d}{dt} q \neq 0^{3 \times 1}$. Evaluating $\nabla_{r_{13}} q$, $\nabla_{r_{23}} q$ and $\nabla_{r_{33}} q$ with $R e_3 = [r_{13}, r_{23}, r_{33}]^T$ and $\kappa \rightarrow \infty$, we get 

\small
\begin{equation}\nonumber
\nabla_{r_{i3}} q = -e_i \times (x - \hat{x}(\mathbf{x})), \quad \forall i \in \{1,2,3\},
%\begin{split}
%    &\nabla_{r_{13}} q = -\begin{bmatrix}
%        1&  0&   0
%    \end{bmatrix}^T
%    \times (x - \hat{x}(\mathbf{x}))\\
%    &\nabla_{r_{23}} q = -\begin{bmatrix}
%        0&  1&   0
%    \end{bmatrix}^T
%    \times (x - \hat{x}(\mathbf{x}))\\
%    &\nabla_{r_{33}} q = -\begin{bmatrix}
%        0&  0&   1
%    \end{bmatrix}^T
%    \times (x - \hat{x}(\mathbf{x}))
%\end{split}
\end{equation}
\normalsize
which is nonzero if and only if $x \neq \hat{x}(\mathbf{x})$. In this case, one can always choose $\Omega$ such that $\frac{d}{dt} q \neq 0^{3 \times 1}$. This holds on $\mathcal{X}$ by defining $\nabla_{r} q$ on either side in discontinuities of $\hat{x}(\mathbf{x})$.
\end{proof}

The above result is important as it states that all possibly infeasible configurations form a null set, that can be left instantaneously by a suitable control action. In the infeasible set one of the constraints \eqref{composite_invariance} or \eqref{thrust_invariance} must be slackened to retain feasibility. 

\iffalse
This property is somewhat analogous to the stability result of the attitude controller \eqref{lee_attitude}, where almost global exponential stability of the attitude error is shown, except for a set of specific configurations. 
\fi
\iffalse
We now study the attractiveness of these configurations.
Next, we define the Lyapunov-like functions $\mathcal{V}$ as
\begin{equation}
    \mathcal{V}_(\mathbf{x})  = l_1(\mathbf{x})^2 + l_2(\mathbf{x})^2,
\end{equation}

where $\mathcal{V}(\tilde{\mathbf{x}}) = 0$. To avoid possibly infeasible configurations $\tilde{\mathbf{x}}$, the origin $\mathcal{V}(\tilde{\mathbf{x}}) = 0$ should be unstable when the constraint \eqref{composite_invariance} is active. This translates to
\begin{equation}
    \dot{\mathcal{V}} = \underbrace{\mathcal{L}_f \mathcal{V}}_{\text{autonomous}} + \underbrace{\mathcal{L}_g \mathcal{V} \mathbf{u}}_{\text{controlled}}  \geq 0
\end{equation}
While the above equation does not hold in general, we show that the desired property holds for the controlled dynamics, e.g. $\mathcal{L}_g \mathcal{V} \mathbf{u}  \geq 0$.
This is valuable as it implies that any control action enforced by \eqref{composite_invariance} also does "as good as possible" to avoid possible infeasible configurations.
\fi

\begin{figure*}[t]
    \centering
    %\includegraphics[width=\linewidth]{figures/mission_elektro.eps}
    \includegraphics[width=\linewidth]{figures/mission_elektro_final.pdf}
    \caption{ \small Aggregated map and path of experiment A. An example of the obstacle map used by the safety filter is highlighted in green. The mission starts at the cyan circle on the right, receiving a constant velocity reference of $1 m/s$ in the positive $X$ direction, shown as a blue arrow. The norm of the intervention (cost of the \ac{qp} \eqref{OCP}) is color-coded into the path, highlighting the areas where the safety filter becomes active. Time instances marked A-D are reported in Fig. \ref{fig:CBF_elektro} for visualizing the corresponding numerical values.}
    \label{fig:mission_elektro}
    \vspace{-2ex}
\end{figure*}

\subsection{Safety Filter}
To integrate both constraints \eqref{composite_invariance} and \eqref{thrust_invariance} to jointly enforce invariance of $\mathcal{C}_x \cap \mathcal{C}_T $, we propose a safety filter, where the joint satisfaction of all constraints can be guaranteed everywhere except in the zero volume set $\mathcal{X}_\text{singular}$. Furthermore, the proposed architecture makes use of both thrust and attitude for obstacle avoidance and can easily be applied to in highly cluttered environments with hundreds of obstacles at a low computational cost.

The safe control law uses the proposed geometric tracking controller \eqref{lee_attitude} - \eqref{lee_position} as a nominal control law, on top of which a reactive safety filter as in \cite{backstepping_CBF} is applied. The control law $\mathbf{u}_\text{safe}$ is sequentially computed as 

\small
\begin{equation}\label{ref_controller}
\mathbf{u}_\text{ref} = 
\begin{bmatrix}
    \Omega \\ 
    \tau
\end{bmatrix}
= 
\begin{bmatrix}
    -k_R \frac{1}{2}[R_d^tR - R^T R_d]\textsuperscript{$\wedge$} + R^T R_d \Omega_{d} \\ 
    k_T (T_d - T) + \dot{T}_d
\end{bmatrix},
\end{equation}
\vspace{-2ex}
\begin{subequations}\label{OCP}
    \begin{align}
    \mathbf{u}_\text{safe} &= \underset{\mathbf{u} \in \mathcal{U}}{\text{argmin}} \; (\mathbf{u} - \mathbf{u}_\text{ref})^T P (\mathbf{u} - \mathbf{u}_\text{ref}) \\
    \text{s.t.} \quad 
    &\begin{bmatrix} \label{stacked_constraints}
        \mathcal{L}_g h_1(\mathbf{x}) \\
        \mathcal{L}_g h_2(\mathbf{x})
    \end{bmatrix}
    \mathbf{u} \geq 
    \begin{bmatrix}
        -\mathcal{L}_f h_1(\mathbf{x}) -\alpha_1 h_1(\mathbf{x})\\
        b_2(\mathbf{x})
    \end{bmatrix},
    \end{align}
\end{subequations}
\normalsize
where we simply stacked constraints \eqref{composite_invariance} and \eqref{thrust_invariance} in a single \ac{qp}. The safety filter thus corrects the commands of the reference controller to satisfy the invariance constraints. The resulting output matches the reference value if the constraints allow, and is minimally altered while satisfying the constraints otherwise.

