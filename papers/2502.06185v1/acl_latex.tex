% This must be in the first 5 lines to tell arXiv to use pdfLaTeX, which is strongly recommended.
\pdfoutput=1
% In particular, the hyperref package requires pdfLaTeX in order to break URLs across lines.

\documentclass[11pt]{article}

% Change "review" to "final" to generate the final (sometimes called camera-ready) version.
% Change to "preprint" to generate a non-anonymous version with page numbers.
\usepackage{acl}

% Standard package includes
\usepackage{times}
\usepackage{latexsym}
\usepackage[raggedrightboxes]{ragged2e}
\usepackage{nicematrix}
% For proper rendering and hyphenation of words containing Latin characters (including in bib files)
\usepackage[T1]{fontenc}
% For Vietnamese characters
% \usepackage[T5]{fontenc}
% See https://www.latex-project.org/help/documentation/encguide.pdf for other character sets

% This assumes your files are encoded as UTF8
\usepackage[utf8]{inputenc}

% This is not strictly necessary, and may be commented out,
% but it will improve the layout of the manuscript,
% and will typically save some space.
\usepackage{microtype}

% This is also not strictly necessary, and may be commented out.
% However, it will improve the aesthetics of text in
% the typewriter font.
\usepackage{inconsolata}
\usepackage{graphicx}
\usepackage{booktabs}
\usepackage{tabularx}
\usepackage{multirow}
\usepackage{subcaption}
\usepackage{listings}
\usepackage{xcolor,colortbl}
\usepackage{transparent}  
\usepackage{nicematrix,tikz}
\usepackage[raggedrightboxes]{ragged2e}
\usepackage{enumitem}
% If the title and author information does not fit in the area allocated, uncomment the following
%
%\setlength\titlebox{<dim>}
%
% and set <dim> to something 5cm or larger.

\title{Discourse-Driven Evaluation: Unveiling Factual Inconsistency in Long Document Summarization}

% Author information can be set in various styles:
% For several authors from the same institution:
% \author{Author 1 \and ... \and Author n \\
%         Address line \\ ... \\ Address line}
% if the names do not fit well on one line use
%         Author 1 \\ {\bf Author 2} \\ ... \\ {\bf Author n} \\
% For authors from different institutions:
% \author{Author 1 \\ Address line \\  ... \\ Address line
%         \And  ... \And
%         Author n \\ Address line \\ ... \\ Address line}
% To start a separate ``row'' of authors use \AND, as in
% \author{Author 1 \\ Address line \\  ... \\ Address line
%         \AND
%         Author 2 \\ Address line \\ ... \\ Address line \And
%         Author 3 \\ Address line \\ ... \\ Address line}

\author{Yang Zhong \\
  Department of Computer Science \\
 University of Pittsburgh \\
  \texttt{yaz118@pitt.edu} \\\And
  Diane Litman \\
  Department of Computer Science and LRDC \\
 University of Pittsburgh \\
  \texttt{dlitman@pitt.edu} \\}

\begin{document}
\maketitle
\begin{abstract}
Detecting factual inconsistency for long document summarization remains challenging, given the complex structure of the source article and long summary length. In this work, we study factual inconsistency errors and connect them with a line of discourse analysis. We find that errors are more common in complex sentences and are associated with several discourse features. We propose a framework that decomposes long texts into discourse-inspired chunks and utilizes discourse information to better aggregate sentence-level scores predicted by natural language inference models. Our approach shows improved performance on top of different model baselines over several evaluation benchmarks, covering rich domains of texts, focusing on long document summarization. This underscores the significance of
incorporating discourse features
in developing models for scoring summaries for long document factual inconsistency.
\end{abstract}

\section{Introduction}

Current state-of-the-art summarization systems can generate fluent summaries; however, their ability to produce factually consistent summaries that adhere to the source content or world knowledge remains questionable. This phenomenon is known as \textbf{factual inconsistency}, one type of ``hallucination'' problem \cite{maynez-etal-2020-faithfulness, zhang2023languagemodelhallucinationssnowball, cao-wang-2021-cliff,kryscinski-etal-2020-evaluating, goyal-durrett-2021-annotating, cao-etal-2022-hallucinated}.  A rigorous line of research approaches this problem by developing models to detect unfaithful summary content, including utilizing pre-trained models such as natural language inference (NLI)  \cite{ kryscinski-etal-2020-evaluating, laban2022summac, zha-etal-2023-alignscore} and question answering (QA) \cite{scialom-etal-2021-questeval, fabbri-etal-2022-qafacteval} models. Such approaches are tested on rich benchmark datasets, such as \textsc{True} \cite{honovich-etal-2022-true}, \textsc{Summac} \cite{laban2022summac}, and \textsc{AggreFact} \cite{tang-etal-2023-understanding}, etc. 

However, such benchmark datasets only include short documents (< 1000 words) and summaries with a few sentences. While the methods mentioned above perform well with short texts, they struggle with longer documents \cite{schuster-etal-2022-stretching}. Recent NLI work addresses this by selecting the input and breaking down the summary. Lengthy summaries are split into individual sentences or more minor atomic claims, while small chunks of the source document are extracted as premises. This approach reduces the task to multiple short evaluations, which are then aggregated to provide a summary-level label \cite{zha-etal-2023-alignscore, zhang-etal-2024-fine, scire-etal-2024-fenice, yang2024fizz}. 

Out of the existing NLI-based methods, \textsc{AlignScore} demonstrated superior performance on multiple benchmarks. It breaks the input document into continuous chunks of text to tackle the input restriction. However, this exhaustive approach may break the structure of the context (section and paragraph split), thus reducing the chances that the summary sentence can be correctly verified with its factual consistency.  On the other hand, most factuality evaluation metrics aggregate the sentence-level aligning scores through averaging or selecting the minimum, disregarding that sentences are not equally important \cite{krishna-etal-2023-longeval}. For instance, people can remember the big picture more easily but struggle to retain low-level details when retelling a story. The natural questions would be: do system-generated summaries carry a similar pattern? If so, how can we utilize the text organization information to help detect the inconsistencies between the summary and the source document? 

In this work, we study the factual inconsistency problem through the lens of discourse analysis. By analyzing the structure (here we use Rhetorical Structure Theory (RST) \cite{MANNTHOMPSON+1988}) of the original articles and the summaries, we uncover the importance of preserving the article structure and studying the connections between discourse structure and the factual consistency of model-generated summaries. Our analysis shows that complex sentences built by multiple elementary discourse units (EDUs, the basic units used in the discourse theory) have a higher chance of containing errors, and we also find several discourse features connected to the factual consistency of summary sentences.

Motivated by the analyses mentioned above, we propose a new evaluation method, \textsc{StructScore}, based on the NLI-based approaches to better detect factual inconsistency. Our algorithm includes two steps:  (1) leveraging the discourse information when aggregating the sentence-level alignment scores of the target summary and (2) decomposing the long input article into multiple discourse-inspired chunks. We tested our proposed approach on multiple document summarization benchmarks, including \textsc{AggreFact-FtSOTA} split, \textsc{DiverSumm}, \textsc{LongSciVerify}, \textsc{LongEval}, and a non-scientific domain dataset \textsc{LegalSumm} with a focus on long document summarization. Our proposed approach obtained a performance gain on multiple tasks.\footnote{Our models and model outputs are publicly available at \url{https://github.com/cs329yangzhong/discourse-driven-summary-factuality-evaluation}} 

To sum up, two research questions are addressed:
1. How and what discourse features are connected to the factual inconsistency evaluation?
2. Can our discourse-inspired approach improve the detection performance on long document summarization?




\section{Related Work}

\paragraph{Factual Inconsistency Detection in Long Document Summarization}

%Despite the numerous datasets released in the news domain \cite{kryscinski-etal-2020-evaluating, cao-wang-2021-cliff, goyal-durrett-2021-annotating, laban2022summac, tang-etal-2023-understanding}, 
Research on automatic factual inconsistency evaluation metrics and resources for long document summarization is limited.
Recently, \citet{Koh2022AnES} surveyed the progress of long document summarization evaluation and called for better metrics and corpora to evaluate long document summaries. \citet{koh-etal-2022-far} released annotated model-generated summaries assessing factual consistency at the
\textbf{sentence} and \textbf{summary} levels for GovReport \cite{huang-etal-2021-efficient} and arXiv \cite{cohan-etal-2018-discourse}. Furthermore, \citet{bishop-etal-2024-longdocfactscore-evaluating} and \citet{zhang-etal-2024-fine} introduced benchmarks of \textsc{LongSciVerify} and \textsc{DiverSumm} that cover diverse domains respectively, and further proposed different frameworks to utilize the context of source sentences for evaluating the factual consistency of generated summaries. However, their approaches relied on extracting context through computing similarities with the summary sentence. The summary-level score is a simple average of all sentence-level predictions. \textit{Our work analyzed a subset of \textsc{DiverSumm} and \textsc{AggreFact} \cite{tang-etal-2023-understanding} that have sentence-level factual inconsistency types and introduced a generalizable approach to better detect such inconsistency errors across domains.}


% \paragraph{Inconsistency Detection with Source Selection and Atomic Claim Extraction} 



\paragraph{Aggregation of Sentence-level Evaluations}
Text summaries are usually composed of multiple sentences. Most factual inconsistency evaluation metrics first compute the sentence-level scores for individual summaries, then aggregate them by either \textbf{soft aggregation} in computing the 
\textbf{unweighted-average} \cite{ zha-etal-2023-alignscore, glover-etal-2022-revisiting,scire-etal-2024-fenice,zhang-etal-2024-fine} or  \textbf{hard aggregation} with the minimum score \cite{ schuster-etal-2022-stretching,yang2024fizz}. ver, these approaches were primarily validated on older benchmarks, consisting of shorter texts (a few hundred input words and summaries of 2-3 sentences). There lacks a systematic study in the context of long document summarization. \textit{Our work dives into the discourse structure of system-generated summaries with span/sentence-level factuality annotations. We introduce a discourse-inspired re-weighting algorithm to calibrate the scores.}
\vspace{-1mm}

\paragraph{Discourse-assisted Text Summarization} Discourse factors have been known to play an important role in the summarization task \cite{ono-etal-1994-abstract, Marcu1998tobuild, kikuchi-etal-2014-single, xu-etal-2020-discourse, hewett-stede-2022-extractive, pu-etal-2023-incorporating}. \citet{louis-etal-2010-discourse} conducted comprehensive experiments to examine the power of different discourse features for context selection. We carry a similar analysis but focus on summary sentences that contain factual inconsistency errors. On adjusting the weight of EDUs, \citet{huber-etal-2021-w} proposed a weighted RST style discourse framework that derives the discourse units' continuous weights from auxiliary summarization task \cite{xiao-etal-2021-predicting}. Differently, our re-weighting algorithm is built on top of the trained parser's parsed discourse tree and applies to the final aggregation of scores. \textit{To the best of our knowledge, our work is the first that studies the connections between RST discourse structure and the factual consistency of model-generated summaries.}




\begin{table*}
\label{datasetdescription}
\centering
\caption{Overview of the evaluation dataset, where M denotes malware and B denotes benign applications.}
\begin{tabular}{c|c|c|c|c|c|c|c|c} 
\hline
           & \textbf{Time Interval} & \textbf{Sample Size} & \begin{tabular}[c]{@{}c@{}}\textbf{The number of}\\\textbf{Existing family}\end{tabular} & \begin{tabular}[c]{@{}c@{}}\textbf{The number of}\\\textbf{New family}\end{tabular} & \textbf{Packed} & \textbf{Malicious} & \textbf{Benign} & \textbf{M/(M+B)\%}  \\ 
\hline
Test set 1 & 2020.05 - 2021.01 & 3015  & 21                                                               & 24                                                          & 18     & 284       & 2731   & 9.42                       \\ 
\hline
Test set 2 & 2021.01 - 2021.12 & 3015      & 28                                                               & 32                                                          & 30     & 298       & 2717   & 9.88                      \\ 
\hline
Test set 3 & 2021.12 - 2023.12 & 3016      & 34                                                               & 36                                                          & 40     & 302       & 2714   & 10.01                       \\
\hline                  
\end{tabular}
\end{table*}

% \input{Section/Motivations}
We begin by presenting the two main assumptions we will make to analyze \Cref{alg:uSCG,alg:SCG}. The first is an assumption on the Lipschitz-continuity of $\nabla f$ with respect to the norm $\|\cdot\|_{\ast}$ restricted to $\mathcal{X}$. We do not assume this norm to be Euclidean which means our results apply to the geometries relevant to training neural networks.
\begin{assumption}\label{asm:Lip} The gradient $\nabla f$ is $L$-Lipschitz with $L \in (0,\infty)$, i.e.,
    \begin{equation}
    \|\nabla f(x) - \nabla f(x)\|_{\ast}
    \leq
    L\|x-y\|
    \quad \forall x,y \in \mathcal X.
    \end{equation}
Furthermore, $f$ is bounded below by $\fmin$.
\end{assumption}
Our second assumption is that the stochastic gradient oracle we have access to is unbiased and has a bounded variance, a typical assumption in stochastic optimization.
\begin{assumption}\label{asm:stoch}
The stochastic gradient oracle $\nabla f(\cdot,\xi):\mathcal X\rightarrow \mathbb{R}^d$ satisfies.
    \begin{assnum}
        \item \label{asm:stoch:unbiased}
            Unbiased:
            \(%
                \mathbb{E}_{\xi}\left[\nabla f(x,\xi)\right] = \nabla f(x) \quad \forall x \in \mathcal X
            \).%
        \item  \label{asm:stoch:var}
            Bounded variance:\\
            \(%
                \mathbb{E}_{\xi}\left[\|\nabla f(x,\xi)-\nabla f(x)\|_2^2\right] \leq \sigma^2  \quad \forall x \in \mathcal X,\sigma\geq 0
            \).%
    \end{assnum}
\end{assumption}

With these assumptions we can state our worst-case convergence rates, first for \Cref{alg:uSCG} and then for \Cref{alg:SCG}. 

\looseness=-1To bridge the gap between theory and practice, we investigate these algorithms when run with a \emph{constant} stepsize $\gamma$, which depends on the specified horizon $n\in\mathbb{N}^*$, and momentum which is either constant $\alpha\in(0,1)$ (except for the first iteration where we take $\alpha=1$ by convention) or \emph{vanishing} $\alpha_k\searrow 0$. The exact constants for the rates can be found in the proofs in \Cref{app:analysis}; we try to highlight the dependence on the parameters $L$ and $\rho$, which correspond to the natural geometry of $f$ and $\mathcal{D}$, explicitly here. Our rates are non-asymptotic and use big O notation for brevity.

\begin{toappendix}
\label{app:analysis}
In this section we present the proofs of the main convergence results of the paper as well as some intermediary lemmas that we will make use of along the way. Throughout this section, we adopt the notation:
\begin{align*}
\text{(stochastic gradient estimator error)} && \lambda^k &:= d^k-\nabla f(x^k) \\
\text{(diameter of $\mathcal{D}$ in $\ell_2$ norm)} && D_2 &:= \max_{x,y\in\mathcal{D}}\norm{x-y}_2 \\
\text{(radius of $\mathcal{D}$ in $\ell_2$ norm)} && \rho_2 &:= \max_{x\in\mathcal{D}}\norm{x}_2 \\
\text{(norm equivalence constant)} && \zeta &:= \max_{x\in\mathcal{X}}\frac{\norm{x}_{\ast}}{\norm{x}_2} \\
\text{(Lipschitz constant of $\nabla f$ with respect to $\norm{\cdot}_{2}$)} && L_2 &:= \inf \{M>0\colon \forall x,y\in\mathcal{X}, \norm{\nabla f(x)-\nabla f(y)}_{2}\leq M\norm{x-y}_{2}\}
\end{align*}
We analyze each algorithm separately, although the analysis is effectively unified between the two, modulo constants. This is done in \Cref{subsec:uSCG,subsec:SCG}, respectively. Our convergence analysis proceeds in three steps: we begin by establishing a template descent inequality for each algorithm via the descent lemma. Next, we analyze the behavior of the second moment of the error $\mathbb{E}[\norm{\lambda^k}_{2}^2]$ under different choices for $\alpha$. Then, we combine these results to derive a convergence rate. Finally, we note that when analyzing algorithms with constant momentum, we will still always take $\alpha=1$ on the first iteration $k=1$.

\subsection{Convergence analysis of \ref{eq:uSCG}}\label{subsec:uSCG}
We begin with the analysis of \Cref{alg:uSCG} by establishing a generic template inequality for the dual norm of the gradient at iteration $k$. This inequality holds regardless of whether the momentum $\alpha_k$ is constant or vanishing, as long as it remains in $(0,1]$.
\begin{lemma}[\ref{eq:uSCG} template inequality]
\label{lem:uSCGtemplate1}
    Suppose \Cref{asm:Lip} holds. Let $n\in\mathbb{N}^*$ and consider the iterates $\{x^{k}\}_{k=1}^n$ generated by \Cref{alg:uSCG} with a constant stepsize $\gamma>0$.
    Then we have
    \begin{equation}
        \mathbb{E}[\norm{\nabla f(\bar{x}^n)}_2^2]\leq \frac{\mathbb{E}[f(x^{1})-\fmin]}{\rho\gamma n} +\frac{L\rho\gamma}{2} + \frac{1}{n}\left(\frac{\rho_2}{\rho}+\zeta\right)\sum\limits_{k=1}^n\sqrt{\mathbb{E}[\norm{\lambda^{k}}_2^2]}.
    \end{equation}
\end{lemma}
\begin{proof}
    Under \Cref{asm:Lip}, we can use the descent lemma for the function $f$ at the points $x^{k}$ and $x^{k+1}$ to get, for all $k\in\{1,\ldots,n\}$,
    \begin{equation}\label{eq:lem:uSCGtemplate1:first2}
        \begin{aligned}
            f(x^{k+1})&\leq f(x^{k})+ \langle \nabla f(x^{k}),x^{k+1}-x^{k}\rangle +\tfrac{L}{2}\norm{x^{k+1}-x^{k}}^{2}
            \\
            &= f(x^{k})+\langle \nabla f(x^{k})-d^{k},x^{k+1}-x^{k}\rangle + \langle d^{k},x^{k+1}-x^{k}\rangle+\tfrac{L}{2}\norm{x^{k+1}-x^{k}}^{2}
            \\
            &= f(x^{k})+\gamma \langle \nabla f(x^{k})-d^{k},\lmo (d^{k})\rangle+\gamma \langle d^{k},\lmo(d^{k})\rangle +\tfrac{L\gamma^{2}}{2}\norm{\lmo(d^{k})}^{2}
            \\
            &\leq f(x^{k})+\gamma \rho_{2}\norm{\lambda^{k}}_{2}+\gamma \langle d^{k},\lmo(d^{k})\rangle +\tfrac{L\gamma^{2}}{2}\rho^{2},
        \end{aligned}
    \end{equation}
    the final step employing Cauchy-Schwarz, the definition of $\lambda^k$, and the definition of $\rho_2$ as the radius of $\mathcal{D}$ in the $\norm{\cdot}_2$ norm.
    By definition of the dual norm we have, for all $u\in\mathcal{X}$,
    \begin{equation*}
        \|u\|_{\ast} = \max\limits_{v\colon \|v\|\leq 1}\langle u,v\rangle = \max_{v\in\mathcal{D}}\langle u,\tfrac{1}{\rho}v\rangle= -\langle u, \tfrac{1}{\rho}\lmo(u)\rangle
    \end{equation*}
    which means that, for all $k\in\{1,\ldots,n\}$,
    \begin{equation*}
        \gamma \langle d^k, \lmo(d^k)\rangle = \gamma\rho\langle d^k,\tfrac{1}{\rho}\lmo(d^k)\rangle = -\gamma\rho\|d^k\|_{\ast}.
    \end{equation*}
    Plugging this expression for $\gamma\langle d^k,\lmo(d^k)\rangle$ into \eqref{eq:lem:uSCGtemplate1:first2} gives, for all $k\in\{1,\ldots,n\}$,
    \begin{equation*}
        \begin{aligned}
            f(x^{k+1})
                &\leq f(x^{k})+\gamma \rho_{2}\norm{\lambda^{k}}_{2}-\gamma\rho\|d^k\|_{\ast} +\tfrac{L\gamma^{2}}{2}\rho^{2}\\
                &= f(x^{k})+\gamma \rho_{2}\norm{\lambda^{k}}_{2}-\gamma\rho\|d^k - \nabla f(x^k) + \nabla f(x^k)\|_{\ast} +\tfrac{L\gamma^{2}}{2}\rho^{2}\\
                &\stackrel{\text{(a)}}{\leq} f(x^{k})+\gamma \rho_{2}\norm{\lambda^{k}}_{2} +\gamma\rho\|\lambda^k\|_{\ast} -\gamma\rho\|\nabla f(x^k)\|_{\ast} +\tfrac{L\gamma^{2}}{2}\rho^{2}\\
                &\stackrel{\text{(b)}}{\leq} f(x^{k})+\gamma (\rho_{2}+\zeta\rho)\norm{\lambda^{k}}_{2}-\gamma\rho\|\nabla f(x^k)\|_{\ast} +\tfrac{L\gamma^{2}}{2}\rho^{2},
        \end{aligned}
    \end{equation*}
    applying the reverse triangle inequality in (a) while (b) stems from the definition of $\zeta$.
    By rearranging terms and taking expectations, we get
    \begin{equation*}
        \begin{aligned}
            \gamma\rho\mathbb{E}[\norm{\nabla f(x^k)}_{\ast}]
                &\leq \mathbb{E}[f(x^{k})-f(x^{k+1})] + \gamma\left(\rho_2+\zeta\rho\right)\mathbb{E}[\norm{\lambda^{k}}_2] +\frac{L\rho^2\gamma^2}{2}.
        \end{aligned}
    \end{equation*}
    Summing this from $k=1$ to $n$ and dividing by $\gamma\rho n$ we get
    \begin{equation*}
        \begin{aligned}
            \mathbb{E}[\norm{\nabla f(\bar{x}^n)}_{\ast}]
                &= \frac{1}{n}\sum\limits_{k=1}^n\mathbb{E}[\norm{\nabla f(x^k)}_{\ast}]\\
                &\leq \frac{\mathbb{E}[f(x^{1})-f(x^{n+1})]}{\rho\gamma n} +\frac{L\rho\gamma}{2} + \frac{1}{n}\left(\frac{\rho_2}{\rho}+\zeta\right)\sum\limits_{k=1}^n\mathbb{E}[\norm{\lambda^{k}}_2]\\
                &\stackrel{\text{(a)}}{\leq} \frac{\mathbb{E}[f(x^{1})-\fmin]}{\rho\gamma n} +\frac{L\rho\gamma}{2} + \frac{1}{n}\left(\frac{\rho_2}{\rho}+\zeta\right)\sum\limits_{k=1}^n\mathbb{E}[\norm{\lambda^{k}}_2]\\
                &\stackrel{\text{(b)}}{\leq} \frac{\mathbb{E}[f(x^{1})-\fmin]}{\rho\gamma n} +\frac{L\rho\gamma}{2} + \frac{1}{n}\left(\frac{\rho_2}{\rho}+\zeta\right)\sum\limits_{k=1}^n\sqrt{\mathbb{E}[\norm{\lambda^{k}}_2^2]},
        \end{aligned}
    \end{equation*}
    using the definition of $\fmin$ for (a) and Jensen's inequality for (b).
\end{proof}

At this point, we need to determine the growth of the induced error captured by the quantity $\norm{\lambda^{k}}_2^2$. To estimate this, we first use a recursion relating $\mathbb{E}[\norm{\lambda^{k}}_2^2]$ and $\mathbb{E}[\norm{\lambda^{k-1}}_2^2]$ adapted from the proof in \citet[Lem. 6]{mokhtari2020stochastic} and then we prove a bound on the decay of $\norm{\lambda^k}_2^2$ for \Cref{alg:uSCG}.
\begin{lemma}[Linear recursive inequality for $\mathbb{E}\norm{\lambda^k}_2^2$]\label{lem:uSCGerror}
    Suppose \Cref{asm:Lip,asm:stoch} hold. Let $n\in\mathbb{N}^*$ and consider the iterates $\{x_k\}_{k=1}^n$ generated by \Cref{alg:uSCG} with a constant stepsize $\gamma>0$. Then, for all $k\in\{1,\ldots,n
    \}$,
    \begin{equation*}
        \mathbb{E}[\norm{\lambda^k}_2^2] \leq \left(1-\frac{\alpha_k}{2}\right)\mathbb{E}[\norm{\lambda^{k-1}}_2^2] + \frac{2L_2^2\rho_2^2\gamma^2}{\alpha_k} + \alpha_k^2\sigma^2.
    \end{equation*}
\end{lemma}
\begin{proof}
    The proof is a straightforward adaptation of the arguments laid out in \citet[Lem. 6]{mokhtari2020stochastic}, which in fact do not depend on convexity nor on the choice of stepsize. Let $n\in\mathbb{N}^*$ and $k\in\{1,\ldots,n\}$, then
    \begin{equation*}
        \begin{aligned}
            \norm{\lambda^k}_2^2
                &= \norm{\nabla f(x^k) - d^{k}}_2^2\\
                &= \norm{\nabla f(x^k) - \alpha_k \nabla f(x^k,\xi_k) - (1-\alpha_k)d^{k-1}}_2^2\\
                &= \norm{\alpha_k\left(\nabla f(x^k) - \nabla f(x^k,\xi_k)\right) +(1-\alpha_k)\left(\nabla f(x^{k})-\nabla f(x^{k-1})\right) - (1-\alpha_k)\left(d^{k-1} - \nabla f(x^{k-1})\right)}_2^2\\
                &= \alpha_k^2\norm{\nabla f(x^k) - \nabla f(x^k,\xi_k)}_2^2 + (1-\alpha_k)^2\norm{\nabla f(x^k)-\nabla f(x^{k-1})}_2^2\\
                    &\quad\quad + (1-\alpha_k)^2\norm{\nabla f(x^{k-1})-d^{k-1}}_2^2\\
                    &\quad\quad +2\alpha_k(1-\alpha_k)\langle\nabla f(x^{k-1})-\nabla f(x^{k-1},\xi_{k-1}), \nabla f(x^k)-\nabla f(x^{k-1})\rangle\\
                    &\quad\quad +2\alpha_k(1-\alpha_k)\langle \nabla f(x^k)-\nabla f(x^k,\xi_k), \nabla f(x^{k-1})-d^{k-1}\rangle\\
                    &\quad\quad +2(1-\alpha_k)^2\langle \nabla f(x^k)-\nabla f(x^{k-1}),\nabla f(x^{k-1}) - d^{k-1}\rangle.
        \end{aligned}
    \end{equation*}
    Taking the expectation conditioned on the filtration $\mathcal{F}_k$ generated by the iterates until $k$, i.e., the sigma algebra generated by $\{x_1,\ldots,x_k\}$, which we denote using $\mathbb{E}_k[\cdot]$, and using the unbiased property in \Cref{asm:stoch}, we get,
    \begin{equation*}
        \begin{aligned}
            \mathbb{E}_k[\norm{\lambda^k}_2^2]
                &= \alpha_k^2\mathbb{E}_k[\norm{\nabla f(x^k)-\nabla f(x^k,\xi_k)}_2^2] + (1-\alpha_k)^2\norm{\nabla f(x^k)-\nabla f(x^{k-1})}_2^2\\
                    &\quad\quad + (1-\alpha_k)^2\norm{\lambda^{k-1}}_2^2 + 2(1-\alpha_k)^2\langle \nabla f(x^k)-\nabla f(x^{k-1}),\lambda^{k-1}\rangle.
        \end{aligned}
    \end{equation*}
    From this expression we can estimate,
    \begin{equation*}
        \begin{aligned}
            \mathbb{E}_k[\norm{\lambda^k}_2^2]
                &\stackrel{\text{(a)}}{\leq} \alpha_k^2\sigma^2 + (1-\alpha_k)^2\norm{\nabla f(x^{k})-\nabla f(x^{k-1})}_2^2 + (1-\alpha_k)^2\norm{\lambda^{k-1}}_2^2 + 2(1-\alpha_k)^2\langle \nabla f(x^k)-\nabla f(x^{k-1}),\lambda^{k-1}\rangle\\
                &\stackrel{\text{(b)}}{\leq} \alpha_k^2\sigma^2 + (1-\alpha_k)^2\norm{\nabla f(x^{k})-\nabla f(x^{k-1})}_2^2 + (1-\alpha_k)^2\norm{\lambda^{k-1}}_2^2\\
                    &\quad\quad + (1-\alpha_k)^2\left(\tfrac{\alpha_k}{2}\norm{\nabla f(x^k)-\nabla f(x^{k-1})}_2^2+\tfrac{2}{\alpha_k}\norm{\lambda^{k-1}}_2^2\right)\\
                 &\stackrel{\text{(c)}}{\leq} \alpha_k^2\sigma^2 + (1-\alpha_k)^2L_2^2\norm{x^k-x^{k-1}}_2^2 + (1-\alpha_k)^2\norm{\lambda^{k-1}}_2^2 + (1-\alpha_k)^2\left((\tfrac{\alpha_k}{2})L_2^2\norm{x^k-x^{k-1}}_{2}^2+\tfrac{2}{\alpha_k}\norm{\lambda^{k-1}}_2^2\right)\\
                 &\stackrel{\text{(d)}}{\leq} \alpha_k^2\sigma^2 + (1-\alpha_k)^2L_2^2\rho_2^2\gamma^2 + (1-\alpha_k)^2\norm{\lambda^{k-1}}_2^2 + (1-\alpha_k)^2\left((\tfrac{\alpha_k}{2})L_2^2\rho_2^2\gamma^2+\tfrac{2}{\alpha_k}\norm{\lambda^{k-1}}_2^2\right)\\
                 &\stackrel{\text{(e)}}{\leq} \alpha_k^2\sigma^2 + (1+\tfrac{\alpha_k}{2})(1-\alpha_k)L_2^2\rho_2^2\gamma^2 + (1+\tfrac{2}{\alpha_k})(1-\alpha_k)\norm{\lambda^{k-1}}_2^2,
        \end{aligned}
    \end{equation*}
    using the bounded variance property from \Cref{asm:stoch} for (a), Young's inequality with parameter $\alpha_k/2>0$ for (b), the Lipschitz property of $f$ under norm $\|\cdot\|_2$ for (c), the update definition from \Cref{alg:uSCG} for (d), and the fact that $1-\alpha_k < 1$ for (e).
    To complete the proof, we note that
    \begin{equation*}
        (1+\tfrac{2}{\alpha_k})(1-\alpha_k)\leq \tfrac{2}{\alpha_k}\quad\text{and}\quad(1-\alpha_k)(1+\tfrac{\alpha_k}{2})\leq (1-\tfrac{\alpha_k}{2})
    \end{equation*}
    which, applied to the previous inequality and taking total expectations, yields
    \begin{equation*}
        \mathbb{E}[\norm{\lambda^k}_2^2] \leq \left(1-\frac{\alpha_k}{2}\right)\mathbb{E}[\norm{\lambda^{k-1}}_2^2] + \alpha_k^2\sigma^2 + \frac{2L_2^2\rho_2^2\gamma^2}{\alpha_k}.
    \end{equation*}
\end{proof}

\subsubsection{Constant $\alpha$}

\begin{lemma}
    Suppose \Cref{asm:Lip,asm:stoch} hold. Let $n \in \mathbb{N}^*$ and consider the iterates $\{x^k\}_{k=1}^n$ generated by \Cref{alg:uSCG} with constant stepsize $\gamma >0$ and constant momentum $\alpha\in(0,1)$ with the exception of the first iteration, where we take $\alpha=1$.
    Then, we have for all $k\in\{1,\ldots,n\}$
    \begin{equation*}
        \begin{aligned}
            \sqrt{\mathbb{E}[\norm{\lambda^k}_2^2]}
                &\leq \frac{\sqrt{2}L_2\rho_2\gamma}{\alpha} + \left(\sqrt{\alpha} + \left(\sqrt{1-\frac{\alpha}{2}}\right)^k\right)\sigma.
        \end{aligned}
    \end{equation*}
\end{lemma}
\begin{proof}
    Let $n\in\mathbb{N}^*$, $k\in\{1,\ldots,n\}$, and invoke \Cref{lem:uSCGerror} to get
    \begin{equation*}
        \mathbb{E}[\norm{\lambda^k}_2^2] \leq \left(1-\frac{\alpha}{2}\right)\mathbb{E}[\norm{\lambda^{k-1}}_2^2] + \frac{2L_2^2\rho_2^2\gamma^2}{\alpha} + \alpha^2\sigma^2.
    \end{equation*}
    Applying \Cref{lem:recursive_geometric} with $\beta = \frac{\alpha}{2}$ and $\eta = \frac{2L_2^2\rho_2^2\gamma^2}{\alpha}+\alpha^2\sigma^2$ gives directly
    \begin{equation*}
        \begin{aligned}
            \mathbb{E}[\norm{\lambda^k}_2^2]
                &\leq \frac{2L_2^2\rho_2^2\gamma^2}{\alpha^2} + \alpha\sigma^2 + \left(1-\frac{\alpha}{2}\right)^k\mathbb{E}[\norm{\lambda^1}_2^2]\\
                &\leq \frac{2L_2^2\rho_2^2\gamma^2}{\alpha^2} + \left(\alpha + \left(1-\frac{\alpha}{2}\right)^k\right)\sigma^2
        \end{aligned}
    \end{equation*}
    after using \Cref{asm:stoch} in the final inequality.
    Taking square roots and upper boudning then yields
    \begin{equation*}
        \begin{aligned}
            \sqrt{\mathbb{E}[\norm{\lambda^k}_2^2]}
                &\leq \frac{\sqrt{2}L_2\rho_2\gamma}{\alpha} + \left(\sqrt{\alpha} + \left(\sqrt{1-\frac{\alpha}{2}}\right)^k\right)\sigma.
        \end{aligned}
    \end{equation*}
\end{proof}

\end{toappendix}

\begin{lemmarep}[{Convergence rate for \ref{eq:uSCG} with constant $\alpha$}]\label{lem:uSCGrate1}
    Suppose \Cref{asm:Lip,asm:stoch} hold. Let $n\in\mathbb{N}^*$ and consider the iterates $\{x^k\}_{k=1}^n$ generated by \Cref{alg:uSCG} with constant stepsize $\gamma = \frac{1}{\sqrt{n}}$ and constant momentum $\alpha\in(0,1)$.
    Then, it holds that
    \begin{equation*}
        \mathbb{E}[\norm{\nabla f(\bar{x}^n)}_{\ast}] \leq O\left(\tfrac{L\rho}{\sqrt{n}}+\sigma\right).
    \end{equation*}
\end{lemmarep}
\begin{appendixproof}
    Let $n\in\mathbb{N}^*$; we will first invoke \Cref{lem:uSCGtemplate1} and then we will estimate the error terms inside using \Cref{lem:uSCGerror} under \Cref{asm:Lip,asm:stoch}.
    As shown in \Cref{lem:uSCGtemplate1},
    \begin{equation}\label{eq:uSCGrate1}
        \begin{aligned}
            \mathbb{E}[\norm{\nabla f(\bar{x}^n)}_2^2]
                &\leq \frac{\mathbb{E}[f(x^{1})-\fmin]}{\rho\gamma n} +\frac{L\rho\gamma}{2n} + \frac{1}{n}\left(\frac{\rho_2}{\rho}+\zeta\right)\sum\limits_{k=1}^n\sqrt{\mathbb{E}[\norm{\lambda^{k}}_2^2]}.
            \end{aligned}
    \end{equation}
    By \Cref{lem:uSCGerror} with \Cref{lem:recursive_geometric}, we get
    \begin{equation*}
        \sqrt{\mathbb{E}[\norm{\lambda^k}_2^2]}
            \leq \frac{\sqrt{2}L_2\rho_2\gamma}{\alpha} + \left(\sqrt{\alpha} + \left(\sqrt{1-\frac{\alpha}{2}}\right)^k\right)\sigma
    \end{equation*}
    which, if we sum from $k=1$ to $n$, gives us
    \begin{equation*}
        \sum\limits_{k=1}^n\sqrt{\mathbb{E}[\norm{\lambda^k}_2^2]}
            \leq n\frac{\sqrt{2}L_2\rho_2\gamma}{\alpha} + \left(n\sqrt{\alpha} + \frac{\sqrt{1-\frac{\alpha}{2}}}{1-\sqrt{1-\frac{\alpha}{2}}}\right)\sigma.
    \end{equation*}
    Plugging this estimate into \Cref{eq:uSCGrate1} gives
    \begin{equation}\label{eq:uSCGfinalineq}
        \begin{aligned}
            \mathbb{E}[\norm{\nabla f(\bar{x}^n)}_2^2]
                &\leq \frac{\mathbb{E}[f(x^{1})-\fmin]}{\rho\gamma n} +\frac{L\rho\gamma}{2} + \frac{1}{n}\left(\frac{\rho_2}{\rho}+\zeta\right)\sum\limits_{k=1}^n\mathbb{E}[\norm{\lambda^{k}}_2]\\
                &\leq \frac{\mathbb{E}[f(x^{1})-\fmin]}{\rho\gamma n} +\frac{L\rho\gamma}{2} + \frac{1}{n}\left(\frac{\rho_2}{\rho}+\zeta\right)\left(n\frac{\sqrt{2}L_2\rho_2\gamma}{\alpha} + \left(n\sqrt{\alpha} + \frac{\sqrt{1-\frac{\alpha}{2}}}{1-\sqrt{1-\frac{\alpha}{2}}}\right)\sigma\right)\\
                &= \frac{\mathbb{E}[f(x^{1})-\fmin]}{\rho\gamma n} +\frac{L\rho\gamma}{2} + \left(\frac{\rho_2}{\rho}+\zeta\right)\left(\frac{\sqrt{2}L_2\rho_2\gamma}{\alpha} + \left(\sqrt{\alpha} + \frac{\sqrt{1-\frac{\alpha}{2}}}{n(1-\sqrt{1-\frac{\alpha}{2}})}\right)\sigma\right).
        \end{aligned}
    \end{equation}
    Finally, by substituting $\gamma = \frac{1}{\sqrt{n}}$ and noting $f(x^{n+1}) \geq \fmin$ we arrive at
    \begin{equation*}
        \begin{aligned}
            \mathbb{E}[\norm{\nabla f(\bar{x}^n)}_{\ast}]
                &\leq \frac{\mathbb{E}[f(x^{1})-\fmin]}{\sqrt{n}\rho} +\frac{L\rho}{2\sqrt{n}} + \left(\frac{\rho_2}{\rho}+\zeta\right)\left(\frac{\sqrt{2}L_2\rho_2}{\alpha\sqrt{n}} + \left(\sqrt{\alpha} + \frac{\sqrt{1-\frac{\alpha}{2}}}{n(1-\sqrt{1-\frac{\alpha}{2}})}\right)\sigma\right)\\
                &= O\left(\frac{1}{\sqrt{n}} + \sigma\right).
        \end{aligned}
    \end{equation*}
\end{appendixproof}

\begin{toappendix}

\subsubsection{Vanishing $\alpha_k$}\label{subsec:uSCGvanishing}

\begin{lemma}[Bound on the gradient error with vanishing $\alpha$]
\label{lem:uSCGerrorbound}
    Suppose \Cref{asm:Lip,asm:stoch} hold. Let $n\in\mathbb{N}^*$ and consider the iterates $\{x_{k}\}_{k=1}^n$ generated by \Cref{alg:uSCG}
    with a constant stepsize $\gamma$ satisfying
    \begin{equation}
        \frac{1}{2 n^{3/4}}<\gamma <\frac{1}{n^{3/4}}.
    \end{equation}
    Moreover, consider momentum which vanishes $\alpha_{k}= \frac{1}{\sqrt{k}}$. Then, for all $k\in\{1,\ldots,n\}$ the following holds
     \begin{equation}
            \mathbb{E}[\norm{\lambda^{k}}_{2}^{2}]\leq \frac{4\sigma^2+8L_2^2\rho_2^2}{\sqrt{k}}.
    \end{equation}
\end{lemma}

\begin{proof}
    Let $k\in\{1,\ldots,n\}$, then by invoking the recursive inequality obtained in \Cref{lem:uSCGerror} for $\mathbb{E}[\norm{\lambda^k}_2^2]$ we have,
    \begin{equation}
        \mathbb{E}[\norm{\lambda^k}^{2}_{2}]\leq \left(1-\frac{\alpha_{k}}{2}\right)\mathbb{E}[\norm{\lambda^{k-1}}^{2}_{2}]+\alpha_{k}^{2}\sigma^{2}+\frac{2L_2^2\rho_2^2\gamma^2}{\alpha_{k}}.
        \end{equation}
        Using the particular choice of $\gamma$ given in the statement of the lemma,
        \begin{equation}
            \frac{1}{2 n^{3/4}}<\gamma <\frac{1}{n^{3/4}},
        \end{equation}
        as well as the choice of $\alpha_k$ and the fact that $n\geq k$, we get
    \begin{align*}
        \mathbb{E}[\norm{\lambda^k}_2^{2}]
            &\leq \bigg(1-\frac{\alpha_{k}}{2} \bigg)\mathbb{E}[\norm{\lambda^{k-1}}_2^{2}]+\alpha_{k}^{2}\sigma^{2}+\frac{2L_2^2\rho_2^2}{\alpha_{k}n^{3/2}}\\
            &\leq \bigg(1-\frac{\alpha_{k}}{2} \bigg)\mathbb{E}[\norm{\lambda^{k-1}}_2^{2}]+\alpha_{k}^{2}\sigma^{2}+\frac{2L_2^2\rho_2^2}{\alpha_{k}k^{3/2}}\\
            &=\bigg(1-\frac{1}{2\sqrt{k}}\bigg)\mathbb{E}[\norm{\lambda^{k-1}}_2^{2}]+\frac{\sigma^{2}}{k}+\frac{2L_2^2\rho_2^2}{k}\\
            &= \bigg(1-\frac{1}{2\sqrt{k}}\bigg)\mathbb{E}[\norm{\lambda^{k-1}}_2^{2}]+\frac{\sigma^{2}+2L_2^2\rho_2^2}{k}.
        \end{align*}
    Then, by applying \Cref{lem:recursivevanishing} with $u^k = \mathbb{E}[\norm{\lambda^k}_2^2]$ and $c=\sigma^2+2L_2^2\rho_2^2$ we readily obtain
    \begin{equation}
        \mathbb{E}[\norm{\lambda^{k}}_{2}^{2}]\leq \frac{4\sigma^2+8L_2^2\rho_2^2}{\sqrt{k}}
    \end{equation}
    since $Q$ as defined in \Cref{lem:recursivevanishing} is given by $Q = \max\{\mathbb{E}[\norm{\lambda^1}_2^2], 4\sigma^2+8L_2^2\rho_2^2\} \leq 4\sigma^2+8L_2^2\rho_2^2$, which concludes our result.
\end{proof}

Combining these results yields our accuracy guarantees for \Cref{alg:uSCG} with vanishing $\alpha_k$, presented in the next lemma.
\end{toappendix}

\begin{lemmarep}[{Convergence rate for \ref{eq:uSCG} with vanishing $\alpha_k$}]
    Suppose that \Cref{asm:Lip,asm:stoch} hold. Let $n\in\mathbb{N}^*$ and consider the iterates $\{x^{k}\}_{k=1}^n$ generated by \Cref{alg:uSCG} with a constant stepsize $\gamma$ satisfying $\frac{1}{2n^{3/4}}<\gamma <\frac{1}{n^{3/4}}$ and vanishing momentum $\alpha_{k}=\tfrac{1}{\sqrt{k}}$. Then, it holds that
    \begin{equation*}
        \mathbb{E}[\|\nabla f(\bar{x}^n)\|_{\ast}] = O\left(\tfrac{1}{n^{1/4}} + \tfrac{L\rho}{n^{3/4}}\right).
    \end{equation*}
\end{lemmarep}
\begin{appendixproof}
    Let $n\in\mathbb{N}^*$, $k\in\{1,\ldots,n\}$; by combining \Cref{lem:uSCGtemplate1} and \Cref{lem:uSCGerrorbound} we have
    \begin{equation}\label{eq:pre_rate}
        \begin{aligned}
            \mathbb{E}[\|\nabla f(\bar{x}^n)\|_{\ast}]
                &\stackrel{\text{\eqref{lem:uSCGtemplate1}}}{\leq} \frac{2\mathbb{E}[f(x^1)-\fmin]}{\rho n^{1/4}} + \frac{2(\rho_2 + \zeta\rho)\sum_{k=1}^n\sqrt{\mathbb{E}[\norm{\lambda^k}_2^2]}}{\rho n} + \frac{L\rho}{n^{3/4}}\\
                &\stackrel{\text{\eqref{lem:uSCGerrorbound}}}{\leq} \frac{2\mathbb{E}[f(x^1)-\fmin]}{\rho n^{1/4}} + \frac{2(\rho_2 + \zeta\rho)\sqrt{4\sigma^2+8L_2^2\rho_2^2}\sum_{k=1}^{n}\frac{1}{k^{1/4}}}{\rho n}  + \frac{L\rho}{n^{3/4}}\\
                &\leq \frac{2\mathbb{E}[f(x^1)-\fmin]}{\rho n^{1/4}} + \frac{2(\rho_2 + \zeta\rho)\sqrt{4\sigma^2+8L_2^2\rho_2^2}\sum_{k=1}^{n}\frac{1}{k^{1/4}}}{\rho n}  + \frac{L\rho}{n^{3/4}}.
        \end{aligned}
    \end{equation}
    Using the integral test and noting that $x\mapsto \tfrac{1}{x^{1/4}}$ is decreasing on $\mathbb{R}_+$, we can upper bound the sum in the right hand side as
    \begin{equation*}
        \sum_{k=1}^{n}\frac{1}{k^{1/4}}\leq 1 + \int_{1}^{n}\frac{1}{x^{3/4}}dx=1+\frac{4}{3}[x^{3/4}]^{n}_1=1+\frac{4}{3}(n^{3/4}-1) = \frac{4}{3}n^{3/4}-\frac{1}{3}\leq \frac{4}{3}n^{3/4}.
    \end{equation*}
    Inserting the above estimation into \eqref{eq:pre_rate} we arrive at
    \begin{align*}
        \mathbb{E}[\|\nabla f(\bar{x}^n)\|_{\ast}] &\leq \frac{2\mathbb{E}[f(x^1)-\fmin]}{\rho n^{1/4}}+ \frac{8 n^{3/4}(\rho_2 + \zeta\rho)\sqrt{4\sigma^2+8L_2^2\rho_2^2}}{3\rho n}  + \frac{L\rho}{n^{3/4}}\\
        &= \frac{2\mathbb{E}[f(x^1)-\fmin]+ \tfrac{8}{3}(\rho_2 + \zeta\rho)\sqrt{4\sigma^2+8L_2^2\rho_2^2}}{\rho n^{1/4}} + \frac{L\rho}{n^{3/4}}\\
        &= O\left(\frac{1}{n^{1/4}}+\frac{L\rho}{n^{3/4}}\right)
    \end{align*}
    which is the claimed result.
\end{appendixproof}

\begin{toappendix}

\subsection{Convergence analysis of \ref{eq:SCG}}\label{subsec:SCG}

In this section we will analyze the worst-case convergence rate of \Cref{alg:SCG}. To do this, we will prove bounds on the expectation of the so-called Frank-Wolfe gap, $\max\limits_{u\in\mathcal{D}} \langle \nabla f(x), x-u\rangle$, which ensures criticality for the constrained optimization problem over $\mathcal{D}$, i.e., for $x^\star\in\mathcal{D}$
\begin{equation*}
    0 = \nabla f(x^\star) + \mathrm{N}_{\mathcal{D}}(x^\star) \iff \max\limits_{u\in\mathcal{D}} \langle \nabla f(x^\star), x^\star-u\rangle \leq 0
\end{equation*}
where $\mathrm{N}_{\mathcal{D}}$ is the normal cone to the set convex $\mathcal{D}$.

This next lemma characterizes the descent of \Cref{alg:SCG} for any stepsize $\gamma$ and momentum $\alpha_k$ in $(0,1]$.
\begin{lemma}[{Nonconvex analog \citet[Lem. 2]{mokhtari2020stochastic}}]
    \label{lem:commondescent}
    Suppose \Cref{asm:Lip} holds.
    Let $n\in\mathbb{N}^*$ and consider the iterates $\{x_k\}_{k=1^n}$ generated by \Cref{alg:SCG} with constant stepsize $\gamma\in(0,1]$.
    Then, for all $k\in\{1,\ldots,n\}$, for all $u\in \mathcal{D}$, it holds
    \begin{equation}
        \gamma \mathbb{E}[\langle \nabla f(x^k), x^k-u\rangle] \leq \mathbb{E}[f(x^k) - f(x^{k+1})] + D_2\gamma \sqrt{\mathbb{E}[\| \lambda^k\|_2^2]} + 2L\rho^2\gamma^2.
    \end{equation}
\end{lemma}
\begin{proof}
    Let $n\in\mathbb{N}^*$, then by \Cref{asm:Lip} we can apply the descent lemma for the function $f$ at the points $x^k$ and $x^{k+1}$ to get, for all $k\in\{1,\ldots,n\}$,
    \begin{equation*}
        \begin{aligned}
            f(x^{k+1})
                &\leq f(x^k) + \langle \nabla f(x^k), x^{k+1}-x^k\rangle + \tfrac{L}{2}\|x^{k+1}-x^k\|^2\\
                &= f(x^k) + \langle d^k, x^{k+1}-x^k\rangle + \langle \lambda^k, x^{k+1}-x^k\rangle + \tfrac{L}{2}\|x^{k+1}-x^k\|^2\\
                &= f(x^k) + \gamma\langle d^k, \lmo(d^k)-x^k\rangle + \gamma \langle \lambda^k, \lmo(d^k)-x^k\rangle + \tfrac{L}{2}\gamma^2\|\lmo(d^k)-x^k\|^2\\
                &\stackrel{\text{(a)}}{\leq} f(x^k) + \gamma\langle d^k, u-x^k\rangle + \gamma \langle \lambda^k, \lmo(d^k)-x^k\rangle + \tfrac{L}{2}\gamma^2\|\lmo(d^k)-x^k\|^2\\
                &= f(x^k) + \gamma\langle -\lambda^k, u-x^k\rangle + \gamma \langle \nabla f(x^k), u-x^k\rangle + \gamma \langle \lambda^k, \lmo(d^k)-x^k\rangle + \tfrac{L}{2}\gamma^2\|\lmo(d^k)-x^k\|^2\\
                &= f(x^k) + \gamma \langle \nabla f(x^k), u-x^k\rangle + \gamma \langle \lambda^k, \lmo(d^k)-u\rangle + \tfrac{L}{2}\gamma^2\|\lmo(d^k)-x^k\|^2\\
                &\stackrel{\text{(b)}}{\leq} f(x^k) + \gamma \langle \nabla f(x^k), u-x^k\rangle + \gamma \langle \lambda^k, \lmo(d^k)-u\rangle + 2L\rho^2\gamma^2,
        \end{aligned}
    \end{equation*}
    using the optimality of $\lmo(d^k)$ for the linear minimization subproblem for (a) and the $2\rho$ upper bound on $\|\lmo(d^k)-x^k\|$ for (b).
    Rearranging and estimating we find, for all $k\in\{1,\ldots,n\}$, for all $u\in\mathcal{D}$,
    \begin{equation*}
        \begin{aligned}
            \gamma\langle \nabla f(x^k),x^k-u\rangle
                &\stackrel{\text{(a)}}{\leq} f(x^k) - f(x^{k+1}) + \gamma \| \lambda^k\|_2 \|\lmo(d^k)-u\|_2 + \tfrac{L}{2}\gamma^2\|\lmo(d^k)-x^k\|^2\\
                &\stackrel{\text{(b)}}{\leq} f(x^k) - f(x^{k+1}) + D_2 \gamma \| \lambda^k\|_2  + 2L\rho^2\gamma^2
        \end{aligned}
    \end{equation*}
    where we have used the Cauchy-Schwarz inequality in (a) and and bounded $\|\lmo(d^k)-x^k\|_2$ using the diameter of the set $\mathcal{D}$ with respect to the Euclidean norm, denoted $D_2$, in (b).
    Taking the expectation of both sides and applying Jensen's inequality we finally arrive, for all $k\in\{1,\ldots,n\}$, for all $u\in\mathcal{D}$,
    \begin{equation*}
        \begin{aligned}
            \gamma\mathbb{E}[\langle \nabla f(x^k),x^k-u\rangle]
                &\leq \mathbb{E}[f(x^k) - f(x^{k+1})] + D_2 \gamma \mathbb{E}[\| \lambda^k\|_2] + 2L\rho^2\gamma^2\\
                &\leq \mathbb{E}[f(x^k) - f(x^{k+1})] + D_2 \gamma \sqrt{\mathbb{E}[\| \lambda^k\|_2^2]} + 2L\rho^2\gamma^2.
        \end{aligned}
    \end{equation*}
\end{proof}

\subsubsection{\ref{eq:SCG} with constant $\alpha$}\label{subsec:SCGconstant}
\begin{lemma}\label{lem:SCGconstanterror}
    Suppose \Cref{asm:Lip,asm:stoch} hold. Let $n\in\mathbb{N}^*$ and consider the iterates $\{x^k\}_{k=1}^n$ generated by \Cref{alg:SCG} with constant stepsize $\gamma=\tfrac{1}{\sqrt{n}}$ and constant momentum $\alpha \in(0,1)$ with the exception of the first iteration, where we take $\alpha=1$. Then we have
    \begin{equation*}
        \mathbb{E}[\norm{\lambda^k}_2^2] \leq 4L_2^2D_2^2\frac{\gamma^2}{\alpha^2} + \left(2\alpha + \left(1-\frac{\alpha}{2}\right)^k\right)\sigma^2.
    \end{equation*}
\end{lemma}
\begin{proof}
    Under \Cref{asm:Lip,asm:stoch}, Lemma 1 in \citet{mokhtari2020stochastic} yields, after taking expectations, for all $k\in\{1,\ldots,n\}$
    \begin{equation*}
        \mathbb{E}[\| \lambda^{k+1}\|_2^2] \leq (1-\frac{\alpha_{k+1}}{2})\mathbb{E}[\| \lambda^k\|_2^2] + \sigma^2\alpha_{k+1}^2 + 2L_2^2D_2^2\frac{\gamma^2}{\alpha_{k+1}}.
    \end{equation*}
    Taking $\gamma$ and $\alpha$ to be constant we get
    \begin{equation*}
        \mathbb{E}[\| \lambda^{k+1}\|_2^2] \leq (1-\frac{\alpha}{2})\mathbb{E}[\| \lambda^k\|_2^2] + \sigma^2\alpha^2 + 2L_2^2D_2^2\frac{\gamma^2}{\alpha}.
    \end{equation*}
    Applying \Cref{lem:recursive_geometric} to the above with $u^k =\mathbb{E}[\| \lambda^{k+1}\|_2^2]$, $\beta = \frac{\alpha}{2}$, and $\eta = \sigma^2\alpha^2 + 2L_2^2D_2^2\frac{\gamma^2}{\alpha}$ we obtain
    \begin{equation*}
        \begin{aligned}
            \mathbb{E}[\norm{\lambda^{k}}_2^2]
                &\leq 2\alpha\sigma^2 + 4L_2^2D_2^2\frac{\gamma^2}{\alpha^2} + \left(1-\frac{\alpha}{2}\right)^k\mathbb{E}[\norm{\lambda^{1}}_2^2]\\
                &\leq 4L_2^2D_2^2\frac{\gamma^2}{\alpha^2} + \left(2\alpha + \left(1-\frac{\alpha}{2}\right)^k\right)\sigma^2
        \end{aligned}
    \end{equation*}
    with the final inequality following by the variance bound in \Cref{asm:stoch}.
\end{proof}

\end{toappendix}

These results show that, in the worst-case, running \Cref{alg:uSCG} with constant momentum $\alpha$ guarantees faster convergence but to a noise-dominated region with radius proportional to $\sigma$. In contrast, running \Cref{alg:uSCG} with vanishing momentum $\alpha_k$ is guaranteed to make the expected dual norm of the gradient small but at a slower rate. \Cref{alg:SCG} exhibits the analogous behavior, as we show next.

Before stating the results for \Cref{alg:SCG}, we emphasize that they are with \emph{constant} stepsize $\gamma$, which is atypical for conditional gradient methods. However, like most conditional gradient methods, we provide a convergence rate on the so-called Frank-Wolfe gap which measures criticality for the constrained optimization problem over $\mathcal{D}$. 

Finally, we remind the reader that the iterates of \Cref{alg:SCG} are always feasible for the set $\mathcal{D}$ by the design of the update and convexity of the norm ball $\mathcal{D}$.
\begin{lemmarep}[{Convergence rate for \ref{eq:SCG} with constant $\alpha$}]
    Suppose \Cref{asm:Lip,asm:stoch} hold. Let $n\in\mathbb{N}^*$ and consider the iterates $\{x^k\}_{k=1}^n$ generated by \Cref{alg:SCG} with constant stepsize $\gamma=\tfrac{1}{\sqrt{n}}$ and constant momentum $\alpha \in(0,1)$. Then, for all $u\in\mathcal{D}$, it holds that
    \begin{equation*}
        \begin{aligned}
            \mathbb{E}[\langle \nabla f(\bar{x}^n), \bar{x}^n-u\rangle] = O\left(\tfrac{L\rho^2}{\sqrt{n}} + \sigma\right).
        \end{aligned}
    \end{equation*}
\end{lemmarep}
\begin{appendixproof}
    Let $n\in\mathbb{N}^*$ and let $k\in\{1,\ldots,n\}$.
    By \Cref{asm:Lip}, we can invoke \Cref{lem:commondescent} to get, for all $k\in\{1,\ldots,n\}$, for all $u\in\mathcal{D}$,
    \begin{equation*}
        \gamma \mathbb{E}[\langle \nabla f(x^k), x^k-u\rangle]
            \leq \mathbb{E}[f(x^k) - f(x^{k+1})] + D_2\gamma \sqrt{\mathbb{E}[\| \lambda^k\|_2^2]} + 2L\rho^2\gamma^2.
    \end{equation*}
    Since \Cref{asm:stoch} holds, we can then invoke \Cref{lem:SCGconstanterror} and apply this to the above. This gives, for all $u\in\mathcal{D}$
    \begin{equation*}
        \begin{aligned}
            \gamma\mathbb{E}[\langle \nabla f(x^k),x^k-u\rangle]
                &\leq \mathbb{E}[f(x^k) - f(x^{k+1})] + 2L\rho^2\gamma^2 + D_2\gamma \sqrt{4L_2^2D_2^2\frac{\gamma^2}{\alpha^2} + \left(2\alpha + \left(1-\frac{\alpha}{2}\right)^k\right)\sigma^2}\\
                &\leq \mathbb{E}[f(x^k) - f(x^{k+1})] + 2L\rho^2\gamma^2 + 2L_2D_2^2\frac{\gamma^2}{\alpha} + D_2\gamma \left(\sqrt{2\alpha} + \left(\sqrt{1-\frac{\alpha}{2}}\right)^k\right)\sigma.
        \end{aligned}
    \end{equation*}
    Summing from $k=1$ to $n$ then dividing by $n\gamma$ we find, for all $u\in\mathcal{D}$,
    \begin{equation}\label{eq:SCGfinalineq}
        \begin{aligned}
            \mathbb{E}[\langle \nabla f(\bar{x}^n), \bar{x}^n-u\rangle]
                &=\frac{1}{n}\sum\limits_{k=1}^n\mathbb{E}[\langle \nabla f(x^k),x^k-u\rangle]\\
                &\stackrel{\text{(a)}}{\leq} \frac{\mathbb{E}[f(x^1) - f(x^{n+1})]}{\gamma n} + 2L\rho^2\gamma + 2L_2D_2^2\frac{\gamma}{\alpha} + D_2 \left(\sqrt{2\alpha} + \frac{1}{n}\sum\limits_{k=1}^n\left(\sqrt{1-\frac{\alpha}{2}}\right)^k\right)\sigma\\
                &\stackrel{\text{(b)}}{\leq} \frac{\mathbb{E}[f(x^1) - f(x^{n+1})]}{\gamma n} + 2L\rho^2\gamma + 2L_2D_2^2\frac{\gamma}{\alpha} + D_2 \left(\sqrt{2\alpha} + \frac{\sqrt{1-\frac{\alpha}{2}}}{n\left(1-\sqrt{1-\frac{\alpha}{2}}\right)}\right)\sigma\\
                &\stackrel{\text{(c)}}{\leq} \frac{\mathbb{E}[f(x^1) - \fmin]}{\gamma n} + 2L\rho^2\gamma + 2L_2D_2^2\frac{\gamma}{\alpha} + D_2 \left(\sqrt{2\alpha} + \frac{\sqrt{1-\frac{\alpha}{2}}}{n\left(1-\sqrt{1-\frac{\alpha}{2}}\right)}\right)\sigma,
        \end{aligned}
    \end{equation}
    applying the subadditivity of the square root for (a), geometric series due to $\sqrt{1-\frac{\alpha}{2}}\in (0,1)$ for (b), and the definition of $\fmin$ for (c).
    Taking $\gamma = \frac{1}{\sqrt{n}}$ then gives the final result, for all $u\in\mathcal{D}$,
    \begin{equation*}
        \begin{aligned}
            \mathbb{E}[\langle \nabla f(\bar{x}^n), \bar{x}^n-u\rangle]
                &\leq \frac{\mathbb{E}[f(x^1) - \fmin]}{\sqrt{n}} + \frac{2L\rho^2}{\sqrt{n}} + \frac{2L_2D_2^2}{\alpha\sqrt{n}} + D_2 \left(\sqrt{2\alpha} + \frac{\sqrt{1-\frac{\alpha}{2}}}{n\left(1-\sqrt{1-\frac{\alpha}{2}}\right)}\right)\sigma
                &= O\left(\frac{L\rho^2}{\sqrt{n}}+\sigma\right).
        \end{aligned}
    \end{equation*}
\end{appendixproof}

\begin{toappendix}
\subsubsection{\ref{eq:SCG} with vanishing $\alpha$}\label{subsec:SCGvanishing}
We now proceed to analyze the convergence of \Cref{alg:SCG} with vanishing $\alpha_k$.
The next lemma provides an estimation on the decay of the second moment of the noise $\lambda^k$.
\begin{lemma}[Bound on the gradient error with vanishing $\alpha$ \Cref{alg:SCG}]\label{lem:SCG_vanishing_error}
    Suppose \Cref{asm:Lip,asm:stoch} hold. Let $n\in\mathbb{N}^*$ and consider the iterates $\{x_{k}\}_{k=1}^n$ generated by \Cref{alg:SCG}
    with a constant stepsize $\gamma$ satisfying
    \begin{equation}
        \frac{1}{2 n^{3/4}}<\gamma <\frac{1}{n^{3/4}}.
    \end{equation}
    Moreover, consider vanishing momentum $\alpha_{k}= \frac{1}{\sqrt{k}}$. Then, for all $k\in\{1,\ldots,n\}$ the following holds
    \begin{equation}
            \mathbb{E}[\norm{\lambda^{k}}_{2}^{2}]\leq \frac{4\sigma^2+8L_2^2D_2^2}{\sqrt{k}}.
    \end{equation}
\end{lemma}
\begin{proof}
    Under \Cref{asm:Lip,asm:stoch}, we have the following recursion from Lemma 1 in \citet{mokhtari2020stochastic} after taking expectations, for all $k\in\mathbb{N}^*$,
    \begin{equation*}
        \mathbb{E}[\| \lambda^{k+1}\|_2^2] \leq (1-\frac{\alpha_{k+1}}{2})\mathbb{E}[\| \lambda^k\|_2^2] + \sigma^2\alpha_{k+1}^2 + 2L_2^2D_2^2\frac{\gamma^2}{\alpha_{k+1}}.
    \end{equation*}
    Comparing with the bound in \Cref{lem:uSCGerrorbound}, we see the only difference is the change of the constant $D_2^2$ by $\rho_2^2$. Repeating the argument in \Cref{lem:uSCGerrorbound}, the desired claim is directly obtained with $D_2^2$ in place of $\rho_2^2$, with the constant $Q = \max\{\mathbb{E}[\norm{\lambda^1}_2^2], 4\sigma^2+8L_2^2D_2^2\} \leq 4\sigma^2+8L_2^2D_2^2$ since $\mathcal{E}[\norm{\lambda^1}_2^2]\leq \sigma^2$ by \Cref{asm:stoch}.
\end{proof}

\end{toappendix}

\begin{lemmarep}[Convergence rate for \ref{eq:SCG} with vanishing $\alpha_k$]\label{lem:frankwolfe_rate}
    Suppose \Cref{asm:Lip,asm:stoch} hold. Let $n\in\mathbb{N}^*$ and consider the iterates $\{x^k\}_{k=1}^n$ generated by \Cref{alg:SCG} with a constant stepsize $\gamma$ satisfying $\tfrac{1}{2n^{3/4}}<\gamma<\tfrac{1}{n^{3/4}}$ and vanishing momentum $\alpha_k = \frac{1}{\sqrt{k}}$. Then, for all $u\in\mathcal{D}$, it holds that
    \begin{equation*}
        \mathbb{E}[\langle \nabla f(\bar{x}^n), \bar{x}^n-u\rangle] = O\left(\tfrac{1}{n^{1/4}} + \tfrac{L\rho^2}{n^{3/4}}\right).
    \end{equation*}
\end{lemmarep}
\begin{appendixproof}
    Let $n\in\mathbb{N}^*$ and $k\in\{1,\ldots,n\}$. By \Cref{asm:Lip}, we can invoke \Cref{lem:commondescent} to get,
    \begin{equation*}
        \begin{aligned}
            \gamma\mathbb{E}[\langle \nabla f(x^k),x^k-u\rangle]
                &\leq \mathbb{E}[f(x^k) - f(x^{k+1})] + D_2 \gamma \sqrt{\mathbb{E}[\| \lambda^k\|_2^2]} + 2L\rho^2\gamma^2.
        \end{aligned}
    \end{equation*}
    Applying the estimate given in \Cref{lem:SCG_vanishing_error} to the above we get
    \begin{equation*}
        \begin{aligned}
            \gamma\mathbb{E}[\langle \nabla f(x^k),x^k-u\rangle]
                &\leq \mathbb{E}[f(x^k) - f(x^{k+1})] + D_2 \gamma \sqrt{\frac{4\sigma^2+8L_2^2D_2^2}{\sqrt{k}}} + 2L\rho^2\gamma^2\\
                &= \mathbb{E}[f(x^k) - f(x^{k+1})] + D_2 \sqrt{4\sigma^2+8L_2^2D_2^2} \gamma \frac{1}{k^{1/4}} + 2L\rho^2\gamma^2.
        \end{aligned}
    \end{equation*}
    Summing from $k=1$ to $n$ and then dividing by $n\gamma$ we find, for all $u\in\mathcal{D}$,
    \begin{equation*}
        \begin{aligned}
            \mathbb{E}[\langle \nabla f(\bar{x}^n),\bar{x}^n-u\rangle]
                &= \frac{1}{n}\sum\limits_{k=1}^n\mathbb{E}[\langle \nabla f(x^k),x^k-u\rangle]\\
                &\stackrel{\text{(a)}}{\leq} \frac{\mathbb{E}[f(x^1) - f(x^{n+1})]}{n\gamma} + \frac{D_2\sqrt{4\sigma^2+8L_2^2D_2^2}}{n}\sum\limits_{k=1}^n\frac{1}{k^{1/4}} + 2L\rho^2\gamma\\
                &\stackrel{\text{(b)}}{\leq} \frac{\mathbb{E}[f(x^1) - f(x^{n+1})]}{n\gamma} + \frac{4D_2\sqrt{4\sigma^2+8L_2^2D_2^2}n^{3/4}}{3n} + 2L\rho^2\gamma\\
                &= \frac{\mathbb{E}[f(x^1) - f(x^{n+1})]}{n\gamma} + \frac{4D_2\sqrt{4\sigma^2+8L_2^2D_2^2}}{3n^{1/4}} + 2L\rho^2\gamma,
        \end{aligned}
    \end{equation*}
    using division by $\gamma n$ for (a) and the integral test with decreasing function $x\mapsto \frac{1}{x^{1/4}}$ for (b).
    Using the definition of $\fmin$ and estimating $n\gamma > \tfrac{n^{1/4}}{2}$ and $\gamma < \frac{1}{n^{3/4}}$ gives
    \begin{equation*}
        \begin{aligned}
            \mathbb{E}[\langle \nabla f(\bar{x}^n),\bar{x}^n-u\rangle]
                &\leq \frac{2\mathbb{E}[f(x^1) - \fmin]}{n^{1/4}} + \frac{4D_2\sqrt{4\sigma^2+8L_2^2D_2^2}}{3n^{1/4}} + \frac{2L\rho^2}{n^{3/4}}\\
                &= O\left(\frac{1}{n^{1/4}} + \frac{L\rho^2}{n^{3/4}}\right).
        \end{aligned}
    \end{equation*}
\end{appendixproof}
\begin{insightbox}[label={insight:convergence}]
For both algorithms, our worst-case analyses for constant momentum suggest that tuning $\alpha$ requires balancing two effects. Making $\alpha$ smaller helps eliminate a constant term that is proportional to the noise level $\sigma$. However, if $\alpha$ becomes too small, it amplifies an $O(1/\sqrt{n})$ term and an $O(\sigma/n)$ term. The stepsize $\gamma$ must also align with the choice of momentum $\alpha$; for vanishing $\alpha_k$ the theory suggests a smaller constant stepsize like $\gamma=\tfrac{3}{4(n^{3/4})}$ to ensure convergence.
\end{insightbox}
\begin{toappendix}

\subsection{Averaged LMO Directional Descent (ALMOND)}\label{subsec:almond}
In this section we present a variation on \Cref{alg:uSCG} that computes the $\lmo$ directly on the stochastic gradient oracle and then does averaging. This is in contrast to how we have presented \Cref{alg:uSCG} which first does averaging (aka momentum) with the stochastic gradient oracle and then computes the $\lmo$. 
A special case of this algorithm is the Normalized SGD based algorithm of \citet{zhao2020stochastic} when the set $\mathcal{D}$ is with respect to the Euclidean norm. 
In contrast with \Cref{alg:uSCG}, the method relies on large batches, since the noise is not controlled by the momentum parameter $\alpha$ due to the bias introduced by the $\lmo$.

\begin{algorithm}
\caption{Averaged LMO directioNal Descent (ALMOND)}
\label{alg:ALMOND}
\textbf{Input:} Horizon $n$, initialization $x^1 \in \mathcal X$, $d^0 = 0$, momentum $\alpha \in (0,1)$, stepsize $\gamma \in (0,1)$
\begin{algorithmic}[1]
    \For{$k = 1, \dots, n$}
        \State Sample $\xi_{k}\sim \mathcal P$
        \State $d^{k} \gets \alpha \lmo(\nabla f(x^{k}, \xi_{k})) + (1 - \alpha)d^{k-1}$
        \State $x^{k+1} \gets x^k + \gamma d^k$
    \EndFor
    \State Choose $\bar{x}^n$ uniformly at random from $\{x^1, \dots, x^n\}$
    \item[\algfont{Return}] $\bar{x}^n$
\end{algorithmic}
\end{algorithm}

\begin{lemmarep}
    Suppose \Cref{asm:Lip,asm:stoch} hold. Let $n\in\mathbb{N}^*$ and consider the iterates $\{x_k\}_{k=1}^n$ generated by \Cref{alg:ALMOND} with stepsize $\gamma = \frac{1}{\sqrt{n}}$. Then, it holds
    \begin{equation*}
        \mathbb{E}[\norm{\nabla f(\bar{x}^n)}_{\ast}] \leq \frac{\mathbb{E}[f(x^1)-\fmin]}{\rho\sqrt{n}} + \frac{L(1-\alpha)\rho}{\alpha\sqrt{n}} + \frac{L\rho}{2\sqrt{n}} + 2\mu\sigma = O\left(\tfrac{1}{\sqrt{n}}\right) + 2\mu\sigma
    \end{equation*}
    where\footnote{Alternatively, instead of invoking the constant $\mu$ we could make an assumption that the gradient oracle has bounded variance measured in the norm $\norm{\cdot}_{\ast}$.} $\mu = \max\limits_{x\in\mathcal{X}}\frac{\norm{x}_\ast}{\norm{x}_{2}}$.
\end{lemmarep}
\begin{proof}
    Let $n\in\mathbb{N}^*$ and denote $z^{k} = \tfrac{1}{\alpha}x^k-\tfrac{1-\alpha}{\alpha}x^{k-1}$ with the convention that $x_0 = x_1$ so that $z_1 = x_1$ and, for all $k\in\{1,\ldots,n\}$,
    \begin{equation*}
        \begin{aligned}
            z^{k+1} - z^k
                &= \frac{1}{\alpha}x^{k+1}-\frac{1-\alpha}{\alpha}x^{k}-\frac{1}{\alpha}x^{k}+\frac{1-\alpha}{\alpha}x^{k-1}= \frac{1}{\alpha}\left(\gamma d^{k} - \gamma (1-\alpha)d^{k-1}\right)= \gamma\lmo(g^k).
        \end{aligned}
    \end{equation*}
    Applying the descent lemma for $f$ at the points $z^{k+1}$ and $z^k$ gives
    \begin{equation}\label{eq:nsgd_descent1}
        \begin{aligned}
            f(z^{k+1})
                &\leq f(z^{k}) + \langle \nabla f(z^k), z^{k+1}-z^k\rangle +\frac{L}{2}\norm{z^{k+1}-z^k}^2\\
                &= f(z^{k}) + \gamma\langle \nabla f(z^k), \lmo(g^k)\rangle +\frac{L\gamma^2}{2}\norm{\lmo(g^k)}^2\\
                &= f(z^{k}) + \gamma\left(\langle \nabla f(z^k)-\nabla f(x^k), \lmo(g^k)\rangle + \langle \nabla f(x^k) - g^k,\lmo(g^k)\rangle +\langle g^k,\lmo(g^k)\rangle\right) +\frac{L\gamma^2}{2}\norm{\lmo(g^k)}^2\\
                &= f(z^{k}) + \gamma\left(\langle \nabla f(z^k)-\nabla f(x^k), \lmo(g^k)\rangle + \langle \nabla f(x^k) - g^k,\lmo(g^k)\rangle -\rho\norm{g^k}_{\ast}\right) +\frac{L\gamma^2}{2}\norm{\lmo(g^k)}^2\\
                &\stackrel{\text{(a)}}{\leq} f(z^{k}) + \gamma\left(\left(\norm{\nabla f(z^k)-\nabla f(x^k)}_{\ast} + \norm{\nabla f(x^k) - g^k}_{\ast}\right)\norm{\lmo(g^k)} -\rho\norm{g^k}_{\ast}\right) +\frac{L\gamma^2}{2}\norm{\lmo(g^k)}^2\\
                &\stackrel{\text{(b)}}{\leq} f(z^{k}) + \gamma\left(\rho\left(\norm{\nabla f(z^k)-\nabla f(x^k)}_{\ast} + \norm{\nabla f(x^k) - g^k}_{\ast}\right) -\rho\norm{g^k}_{\ast}\right) +\frac{L\rho^2\gamma^2}{2}\\
                &\stackrel{\text{(c)}}{\leq} f(z^{k}) + \gamma\left(\rho\left(L\norm{z^k-x^k} + \norm{\nabla f(x^k) - g^k}_{\ast}\right) -\rho\norm{g^k}_{\ast}\right) +\frac{L\rho^2\gamma^2}{2},
        \end{aligned}
    \end{equation}
    applying H\"{o}lder's inequality with norm $\norm{\cdot}_{\ast}$ for (a), the radius $\rho$ of $\mathcal{D}$ for (b), and \Cref{asm:Lip} for (c).
    We note that
    \begin{equation*}
        x^{k+1}-x^{k} = \gamma d^k = \gamma\left((1-\alpha) d^{k-1}+\alpha\lmo(g^k)\right) = \alpha\gamma \lmo(g^k) + (1-\alpha)\gamma\left(\frac{x^k-x^{k-1}}{\gamma}\right)=\alpha\gamma\lmo(g^k)+(1-\alpha)(x^{k}-x^{k-1})
    \end{equation*}
    which we can use to bound
    \begin{equation*}
        \norm{x^{k}-x^{k-1}} \leq (1-\alpha)\norm{x^k-x^{k-1}} + \alpha\gamma\norm{\lmo(g^k)} \leq (1-\alpha)\norm{x^k-x^{k-1}} + \alpha\rho\gamma \leq \frac{\alpha\rho\gamma}{(1-\alpha)}.
    \end{equation*}
    We then have
    \begin{equation*}
        \norm{z^k-x^k} = \frac{(1-\alpha)}{\alpha}\norm{x^k-x^{k-1}}\leq \frac{(1-\alpha)\rho\gamma}{\alpha}
    \end{equation*}
    by using the definition of the update and the $\lmo$, which can be plugged into \eqref{eq:nsgd_descent1} to get
    \begin{equation}
        \begin{aligned}
            \rho\gamma\norm{g^k}_{\ast}
                &\leq f(z^k) - f(z^{k+1}) + \gamma\rho\left(L\norm{z^k-x^k} + \norm{\nabla f(x^k)-g^k}_{\ast}\right) + \frac{L\rho^2\gamma^2}{2}\\
            \implies \norm{g^k}_{\ast}
                &\stackrel{\text{(a)}}{\leq} \frac{f(z^k)-f(z^{k+1})}{\rho\gamma} + L\norm{z^k-x^k} + \norm{\nabla f(x^k)-g^k}_{\ast} + \frac{L\rho\gamma}{2}\\
                &\stackrel{\text{(b)}}{\leq} \frac{f(z^k)-f(z^{k+1})}{\rho\gamma} + \frac{L(1-\alpha)\rho\gamma}{\alpha} + \norm{\nabla f(x^k)-g^k}_{\ast} + \frac{L\rho\gamma}{2}\\
            \implies \norm{\nabla f(x^k)}_{\ast}
                &\stackrel{\text{(c)}}{\leq} \frac{(f(z^k)-f(z^{k+1})}{\rho\gamma} + \frac{L(1-\alpha)\rho\gamma}{\alpha} + 2\norm{\nabla f(x^k)-g^k}_{\ast} + \frac{L\rho\gamma}{2}
        \end{aligned}
    \end{equation}
    where (a) is the result of dividing both sides by $\rho\gamma$, (b) is the result of bounding $\norm{z^k-x^k}$, and (c) follows by the reverse triangle inequality after adding and subtracting $\nabla f(x^k)$ in the norm on the left hand side.
    Taking expectations, using \Cref{asm:stoch} and the constant $\mu = \max\limits_{x\in\mathcal{X}}\frac{\norm{x}_{\ast}}{\norm{x}_2}$, it holds
    \begin{equation*}
        \mathbb{E}[\norm{\nabla f(x^k)-g^k}_{\ast}]\leq \mu\mathbb{E}[\norm{\nabla f(x^k)-g^k}_{2}]\leq \mu\sqrt{\mathbb{E}[\norm{\nabla f(x^k)-g^k}_{2}^2]}\leq \mu\sigma
    \end{equation*}
    which we can sum from $k=1$ to $n$ to obtain
    \begin{equation*}
        \sum\limits_{k=1}^n\mathbb{E}[\norm{\nabla f(x^k)}_{\ast}] \leq \frac{\mathbb{E}[f(z^0)-f(z^{n+1})]}{\rho\gamma} + \frac{nL(1-\alpha)\rho\gamma}{\alpha} + 2n\mu\sigma + \frac{nL\rho\gamma}{2}.
    \end{equation*}
    Diving both sides by $n$ and then plugging in $\gamma = \frac{1}{\sqrt{n}}$ yields the desired final result.
\end{proof}

\subsection{Linear recursive inequalities}
We now present two elementary lemmas that establish bounds for linear recursive inequalities. These results are essential for analyzing the convergence behavior of our stochastic gradient estimator, particularly when examining the error term $\mathbb{E}[\norm{\lambda^k}_2^2]$.
\begin{lemma}[Linear recursive inequality with constant coefficients]\label{lem:recursive_geometric}
    Let $n>1$ and consider $\{u_k\}_{k=1}^n\in\mathbb{R}_+^n$ a sequence of nonnegative real numbers satisfying, for all $k\in\{2,\ldots,n\}$,
    \begin{equation*}
        u^k\leq (1-\beta) u^{k-1} + \eta
    \end{equation*}
    with $\eta>0$ and $\beta\in(0,1)$.
    Then, for all $k\in\{2,\ldots,n\}$, it holds
    \begin{equation*}
        u^k\leq \frac{\eta}{\beta} + (1-\beta)^ku^1.
    \end{equation*}
\end{lemma}
\begin{proof}
    We prove the claim by induction on $k$. For the base case $k=2$ we find
    \begin{equation*}
        u^2 \leq (1-\beta)u^1 + \eta \leq \frac{\eta}{\beta} + (1-\beta)u^1
    \end{equation*}
    since $\beta<1$.
    Assume now for some $k\in\{2,\ldots,n\}$ that the claim holds. Then, by the assumed recursive inequality on $\{u_i\}_{i=1}^n$, we have
    \begin{equation*}
        u^{k+1} \leq (1-\beta)u^k + \eta \leq (1-\beta)\left(\frac{\eta}{\beta} + (1-\beta)^ku^1\right) + \eta = (1-\beta)^{k+1}u^1 + \left(\frac{1-\beta}{\beta} + 1\right)\eta = (1-\beta)^{k+1}u^1 + \frac{\eta}{\beta}
    \end{equation*}
    and thus the desired claim holds by induction.
\end{proof}

The first lemma establishes a geometric decay bound for sequences with constant momentum. The following lemma extends this analysis to the case of variable coefficients, which we will use when we analyze \Cref{alg:uSCG} and \Cref{alg:SCG} with vanishing momentum $\alpha_k$.

\begin{lemma}[Linear recursive inequality with vanishing coefficients]\label{lem:recursivevanishing}   
    Let $\{u^k\}_{k\in\mathbb{N}^*}$ be a sequence of nonnegative real numbers satisfying, for all $k\in\mathbb{N}^*$, the following recursive inequality
    \begin{equation*}
        u^k\leq \left(1-\frac{1}{2\sqrt{k}}\right)u^{k-1} + \frac{c}{k}
    \end{equation*}
    where $c>0$ is constant.
    Then, the sequence $\{u^k\}_{k\in\mathbb{N}^*}$ satisfies, for all $k\in\mathbb{N}^*$,
    \begin{equation*}
        u^k \leq \frac{Q}{\sqrt{k}}
    \end{equation*}
    with $Q=\max\{u^1, 4c\}$.
\end{lemma}
\begin{proof}
    We prove the claim by induction. For $k=1$ the inequality holds by the definition of $Q$, since
    \begin{equation*}
        u^1 \leq Q = \frac{Q}{\sqrt{1}}.
    \end{equation*}
    Let $k>1$ and assume that
    \begin{equation*}
        u^{k-1}\leq\frac{Q}{\sqrt{k-1}}.
    \end{equation*}
    Then, by the assumed recursive inequality for $u^k$, we have
    \begin{equation}\label{eq:recursive_ineq2}
        \begin{aligned}
            u^{k}
                &\leq \left(1-\frac{1}{2\sqrt{k}}\right)u^{k-1} + \frac{c}{k}\\
                &\leq \left(1-\frac{1}{2\sqrt{k}}\right)\frac{Q}{\sqrt{k-1}} + \frac{c}{k}.
        \end{aligned}
    \end{equation}
    Since $k>1$, we can estimate
    \begin{equation*}
        \frac{1}{\sqrt{k-1}} = \frac{\sqrt{k}}{\sqrt{k(k-1)}} = \frac{1}{\sqrt{k}}\sqrt{\frac{k}{k-1}} = \frac{1}{\sqrt{k}}\sqrt{1 + \frac{1}{k-1}} \leq \frac{1}{\sqrt{k}}\left(1 + \frac{1}{2(k-1)}\right)
    \end{equation*}
    which, when applied to \eqref{eq:recursive_ineq2}, gives
    \begin{equation}\label{eq:recursive_ineq3}
        u^k\leq \left(1-\frac{1}{2\sqrt{k}}\right)\left(1+\frac{1}{2(k-1)}\right)\frac{Q}{\sqrt{k}} + \frac{c}{k}.
    \end{equation}
    Furthermore, as $k>1$, we also have
    \begin{equation*}
        \left(1-\frac{1}{2\sqrt{k}}\right)\left(1+\frac{1}{2(k-1)}\right)\leq \left(1-\frac{1}{4\sqrt{k}}\right).
    \end{equation*}
    Applying the above to \eqref{eq:recursive_ineq3} gives
    \begin{equation*}
        \begin{aligned}
            u^k
                &\leq \left(1-\frac{1}{4\sqrt{k}}\right)\frac{Q}{\sqrt{k}}+\frac{c}{k}\\
                &= \frac{Q}{\sqrt{k}} + \frac{c-Q/4}{k}\\
                &\leq \frac{Q}{\sqrt{k}}
        \end{aligned}
    \end{equation*}
    with the last inequality following since $Q\geq 4c$.
    The desired claim is therefore obtained by induction.
\end{proof}

\end{toappendix}


\begin{figure*}
	\centering
	\includegraphics[width = \linewidth]{figure/AgentArena.pdf}
	\caption{\textbf{Stock Trading Workflow in \textit{Agent Trading Arena}.} 
	\textbf{Top:} Workflow of a trading day, including preparation, trading, and post-trading reflection. Agents discuss insights in the chat pool, analyze market trends, execute trades, and refine strategies based on performance.  
	\textbf{Bottom:} Example of agents' interactions in the chat pool and dynamic strategy updates.}
	\label{fig:AgentArena}
	\vspace{-3pt}
\end{figure*}

\section{Proposed Method}

% 核心部分visual representation,

To mitigate the influence of human prior knowledge and memory, we designed a closed-loop economic system~\citep{guo2024economics} called the \textit{Agent Trading Arena}, a zero-sum game simulating complex, quantitative real-world scenarios. The simulation workflow is illustrated in \autoref{fig:AgentArena} and further detailed in \autoref{appendix_arena}. In the \textit{Agent Trading Arena}, agents can invest in assets, earn dividends from holding assets, and pay daily expenses using virtual currency. The agent with the highest total return wins the game.

\subsection{Agent Trading Arena}

\paragraph{Structure of Agent Trading Arena.} 

To eliminate external knowledge biases, asset prices are determined by a bid-ask system, reflecting the prices at which buyers and sellers are willing to transact. The system evolves solely based on agents' actions and interactions, without external influences. This design ensures that the outcomes of agents' actions are not immediately apparent but unfold gradually, influenced by other agents' decisions.

To encourage active participation, a dividend mechanism is introduced. There are two primary sources of income in this system: capital gains from asset price differentials and dividends from holding assets. Dividends for each asset are distributed according to a predefined ratio, serving as an implicit anchor for asset prices. Agents holding more low-cost assets receive higher dividends. To prevent passive asset holding until the end of the game, agents must pay a daily capital cost proportional to their total wealth. These expenses are offset by asset dividends, and only agents with sufficient low-cost assets can cover costs. Under the pressure of significant daily expenses, agents must act swiftly and strategically, triggering frequent trades and price fluctuations to stimulate market activity. This dynamic mechanism ensures fairness in the zero-sum game while preventing agents from relying on fixed strategies to find optimal solutions.

\vspace{-3pt}

\paragraph{Agents Learn and Compete in Arena.}

The zero-sum game structure is crucial to eliminating the possibility of a universally optimal strategy. In fixed scenarios with a static optimal solution, agents could rely on predefined rules or memory-based approaches, bypassing adaptive decision-making. The zero-sum game ensures that there is no universally correct solution, with outcomes evolving dynamically based on agent interactions and competition. This design forces agents to continually adapt, learn from feedback, and develop context-dependent strategies, promoting deeper environmental exploration and preventing reliance on static or memory-driven solutions.

In the \textit{Agent Trading Arena}, agents are unaware of implicit rules, except for the objective to maximize their virtual wealth throughout the simulation. To win this zero-sum game, agents must effectively learn from experience, decipher hidden game rules, and develop strategies to counter competitors. This requires the ability to comprehend numerical feedback, formulate enduring strategies, and make informed decisions. Unlike other mathematical reasoning problems, the results of their actions unfold gradually and dynamically. Moreover, agents are easily misled by erroneous information from competitors, hindering their ability to discern strategic cues from competitors' textual data. Importantly, agents remain unaware of these implicit rules, so applying real-world knowledge does not benefit their performance. Therefore, agents must rely on experiential learning to decipher the hidden game rules and ultimately achieve victory.

\subsection{Types of Numerical Data Input}

\paragraph{Limitations of Textual Numerical Data.}

In the \textit{Agent Trading Arena}, the generated stock data is stored in numerical format. When used directly as input to an LLM, the models often struggle to interpret numerical data accurately or make sound decisions. To mitigate this, we convert the data into textual formats~\citep{numerical_text, long_text}, enhancing semantic features and clarifying output requirements to improve the models' understanding. During interactions, the LLMs process stock prices, trading volumes, and market indices presented as textual numerical data.

\begin{figure*}
	\centering
	\includegraphics[width = \linewidth]{figure/v_t.pdf}
	\caption{\textbf{Textual and Visual Representations of Corresponding Inputs and Outputs.} The left images display the agent’s Buy and Sell trading records, daily trade prices, and K-line charts for three stocks. The output from visual inputs (bottom right) captures overall stock trends and long-term behavior, while the output from textual inputs (top right) focuses on specific current prices.}
	\label{textual_visualized}
	\vspace{-3pt}
\end{figure*}

However, this textual approach reveals significant limitations. While the data is presented clearly, LLMs tend to focus excessively on specific values rather than identifying long-term trends or global patterns. They also struggle with understanding correlative relations and percentage changes, limiting their ability to assess differences and identify connections between data points. When analyzing time-series data with complex patterns, LLMs often fixate on individual data points, overlooking overarching relations. This issue is evident in the analysis output in the top-right corner of \autoref{textual_visualized}, where LLMs' focus on individual values impedes their ability to generalize, reducing their capacity to extract meaningful global insights.

Additionally, LLMs often overemphasize recent data while undervaluing historical information, even when prompted to consider its importance. This prevents them from effectively integrating past data and recognizing long-term patterns, complicating their understanding of numerical relations and trends. These challenges highlight the need for improved mechanisms to process numerical relations, identify global trends, and derive deeper insights from textual numerical data.

\vspace{-3pt}

\paragraph{Potential of Visual Numerical Data.}

Since textual numerical data often leads LLMs to focus on local details while neglecting broader relations, we investigated whether visual representations, such as scatter plots, line charts, and bar charts, could help LLMs better understand overall trends, similar to human reasoning. Thus, we transition from textual numerical data inputs to visualized formats ~\citep{storyllava}. As demonstrated in the bottom-right corner of \autoref{textual_visualized}, visual representations enable LLMs to more effectively grasp global trends, patterns, and relations that are often difficult to discern from textual numerical data alone.

These findings highlight the advantages of structured, visual numerical data, indicating that this format allows LLMs to more intuitively and comprehensively understand complex data, better capturing overall fluctuations, whereas text tends to focus on local details. By combining visualization and textual representations, LLMs not only overcome the challenges of relations in time-series data but also demonstrate better performance in identifying long-term trends and global patterns, while still attending to local details.

\subsection{Reflection Module}

We propose a strategy distillation method, illustrated in \autoref{fig:reflection}, that delivers real-time feedback to LLMs by analyzing both descriptive textual and visual numerical data. This enables the generation of new strategies and optimization of action plans. The approach allows agents to evaluate their results, refine strategies, and adapt continuously based on feedback. The process begins with assessing the day’s trajectory memory and associated strategies using an evaluation function. The strategic generation process leverages contrastive analysis of peak and nadir performers from the evaluation phase, creating bidirectional learning signals that inform subsequent iterations. This iterative cycle ensures continuous strategy evolution, fostering sustained improvement in decision-making.

\begin{figure}[t]
	\centering
	\includegraphics[width = \linewidth]{figure/reflection.pdf}
	\caption{\textbf{Design of the Reflection Module.} The process evaluates daily trajectory memory and strategies (top right), then generates new strategies (center) based on evaluation, environmental feedback (bottom right), and feedback from the 5 top- and bottom-performing strategies. Stock visualization (bottom left) enhances reflection, driving continuous improvement.}
	%The process evaluates daily trajectory memory and strategies, generating new strategies based on positive and negative feedback from the top- and bottom-performing strategies. Stock visualizations (bottom left) further enhance the reflection process, reinforcing continuous strategy refinement.}
	\label{fig:reflection}
	\vspace{-3pt}
\end{figure}

% We propose a strategy distillation method, illustrated in \autoref{fig:reflection}, that provides real-time feedback to LLMs by analyzing both descriptive textual and visualized numerical data. This enables the generation of new strategies and the optimization of action plans. The approach allows agents to assess their results, refine strategies, and continuously adapt based on feedback. The process begins by evaluating the day's trajectory memory and associated strategies using an evaluation function. From this assessment, new strategies are generated by selecting the top-performing and lowest-performing strategies, offering both positive and negative feedback. This iterative cycle ensures continuous strategy evolution, driving sustained improvement in decision-making.

The reflection module plays a crucial role in refining strategies by offering real-time feedback. It analyzes both descriptive textual and visual numerical data to generate new strategies and optimize action plans. Within the \textit{Agent Trading Arena}, the reflection module is triggered regularly to consolidate daily trading records and evaluate the effectiveness of strategies, refining both successful and unsuccessful experiences to guide future decisions. Ineffective strategies are stored in a strategy library for future reference, allowing agents to review and learn from past experiences. Further details can be found in \autoref{appendix_arena}.

\section{Experimental Details}
\newcommand{\dashrule}[1][black]{%
  \color{#1}\rule[\dimexpr.5ex-.2pt]{1pt}{.4pt}\xleaders\hbox{\rule{1pt}{0pt}\rule[\dimexpr.5ex-.2pt]{3pt}{.4pt}}\hfill\kern0pt%
}

We adapt mainstream evaluation setups for each benchmark. For \textsc{DiverSumm}, we apply an 80/20 test/dev split by stratifying the labels for each subtask. For \textsc{AggreFact}, we use their released val/test split. For \textsc{LongSciVerify}, \textsc{LongEval} and \textsc{LegalSumm}, we use them as test sets.

\newcommand{\gray}[1]{\textcolor{gray}{#1}}
\newcolumntype{?}{!{\vrule width 1pt}}
\begin{table*}[t!]
% \small
\scriptsize
\centering
\setlength{\tabcolsep}{3pt}
% \setlength{\tabcolsep}{6pt}

\begin{tabular}{llll?llllc|l?ll?c|c}
 \toprule
\multirow{2}{*}{\shortstack{\textbf{ID} }} & \multirow{2}{*}{\shortstack{\textbf{Evaluation Model} }} & \multicolumn{2}{c}{\textsc{\textbf{AggreFact}}} & \multicolumn{6}{c}{\textsc{\textbf{DiverSumm}}} &  \multicolumn{2}{c}{\textsc{\textbf{LSV}}} & \textsc{\textbf{LongEval}} & \textsc{\textbf{LegalS}}\\
\cmidrule(lr){3-4}\cmidrule{5-10} \cmidrule{11-14} 

&& XSM\textsubscript{AG} & CND\textsubscript{AG} &MNW  & QMS  & GOV 
 & AXV    & CSM & \textit{Macro-} & PUB & AXV & PUB &  \\
 & evaluation metric & \multicolumn{2}{c}{\textit{AUC}}  & \multicolumn{5}{c}{\textit{AUC}} & \textit{AVG} & \multicolumn{2}{c}{\textit{Kendal's $\tau$}} & \textit{Kendal's $\tau$} & AUC\\
& avg src. len & 360.54 & 518.85 & 669.20 & 1138.72   & 2008.16 & 4406.99 & 4612.40 & -- & 3776.80 & 6236.40 & 3158.35 & 2873.87  \\
% \midrule



\midrule

\multicolumn{12}{l}{\textit{\textbf{Baselines}}}\\
\midrule
% \RowStyle{\color{gray}}
\textbf{1} & \textbf{\textsc{LongDocFactScore}} & 50.47 &   65.27& \cellcolor{green!10}{61.20} & 40.69 & 83.52 & 65.36 & 60.06 & 62.17 &\cellcolor{green!10}{61.0}  & \cellcolor{green!10}{61.0} & 29.0 & 60.19 \\


\textbf{2} & \textbf{\textsc{MiniCheck-FT5}} & 75.04 &  72.62 & 48.68 & 45.31 & 70.26 & 61.77 & 52.93 & 55.79& 26.5 & 38.1 & 17.4 & 61.33\\


\textbf{3} &\textbf{GPT4o} & 75.36 & 70.47 &51.11 &\cellcolor{green!10}{70.22} & {86.81} & 67.78 & 61.53 & 67.49& 54.7 & 51.8 & \underline{51.2} & 67.71\\

\textbf{4} &\textbf{BeSpoke-MC-7B} & \cellcolor{green!10}{83.56} & 71.38  & 55.38 & \underline{65.42} &  82.83 & 75.07 & 63.43 & \underline{68.42} & 55.1 & \underline{57.9} & \cellcolor{green!10}{58.1} & 55.81 \\ 
\midrule

\multicolumn{12}{l}{\textit{{Apply our approach with different \textbf{baselines}(\textit{$\uparrow$} means improved the performance compared to the baseline with significance.)}}}\\

\midrule
% \RowStyle{\color{gray}}
\textbf{5} & \textbf{\textsc{AlignScore}} & {75.66} & 69.50 & 46.74 & 56.48 & {87.02} & 77.46 & 61.03 & 65.75 & 54.9 & 53.9 & 36.9 & {73.52}\\

 6 &   \hspace{3mm}+ re-weighting  & 75.67 & 69.20 &45.33 & 53.95 & 87.29$\uparrow$ & 81.15$\uparrow$ & 60.55 & 65.65 & 53.0 &  54.3$\uparrow$ & 34.8 & \cellcolor{green!10}{76.57}$\uparrow$\\
 \cmidrule{2-14}
 7 &\hspace{3mm}\textsc{+ Lv1 segment} & 76.23$\uparrow$ & 69.25\textsuperscript{$\dagger$} & 45.86\textsuperscript{$\dagger$}  & {61.25}$\uparrow$   & {86.74}\textsuperscript{$\dagger$} & 79.47$\uparrow$ & \underline{64.15}$\uparrow$ & 67.49$\uparrow$  & 51.9 & 52.8 & 43.6$\uparrow$ & 59.43\\
8& \hspace{3mm}\textsc{StructS-Lv1} & 76.20$\uparrow$ & 69.03 & 46.21\textsuperscript{$\dagger$}  & 60.06$\uparrow$ & 86.04  & {82.78}$\uparrow$   & \cellcolor{green!10}{64.47}$\uparrow$  & 67.91$\uparrow$  &  50.4 & 53.9\textsuperscript{$\dagger$} & 43.4$\uparrow$ & 59.81 \\
\cmidrule{2-14}

9& \hspace{3mm}\textsc{+ Lv2 segment}  & 74.27 & 70.30$\uparrow$  & 46.03\textsuperscript{$\dagger$}  & 55.74  & 85.10  & 76.79  & 63.11$\uparrow$  & 65.35 &  58.1$\uparrow$ & 51.1 & {43.9}$\uparrow$ & 67.05\\
10& \hspace{3mm}\textsc{StructS-Lv2}  &  74.28 & 69.85$\uparrow$&  45.33 & 51.86 & 85.65 & 80.00$\uparrow$ & 63.59$\uparrow$ & 65.29& 55.3$\uparrow$ & 54.1$\uparrow$ &  43.7$\uparrow$  & 64.00\\
 
\midrule
\midrule
% \textsc{InFusE}

% \RowStyle{\color{gray}}
\textbf{11}& \textbf{\textsc{MC-FT5 (sent)}} &  {79.62} & 70.95 & \underline{57.67} & 60.66&  83.24 & 78.66 & 59.74 & 67.99 & 55.7 & 52.7 & 30.2 & 61.14\\
12&\hspace{3mm}+ re-weighting & \underline{79.73} & 70.76\textsuperscript{$\dagger$} & 56.79 & 60.36\textsuperscript{$\dagger$}  & 84.75$\uparrow$ & 79.38$\uparrow$ & 60.06$\uparrow$ & {68.27}$\uparrow$  & 52.8 & 55.1$\uparrow$ & 31.4$\uparrow$  & 59.81\\
\cmidrule{2-14}
13&\hspace{3mm}\textsc{+ Lv1 segment} & 77.84 & \cellcolor{green!10}{73.48$\uparrow$} & 44.80 & 61.10$\uparrow$ & \underline{87.50}$\uparrow$ &\underline{85.22}$\uparrow$ & 63.59$\uparrow$ & \cellcolor{green!10}68.44$\uparrow$   & 57.5$\uparrow$ & 51.4 &  33.0$\uparrow$ & 68.95$\uparrow$ \\
14&\hspace{3mm}\textsc{StructS-Lv1} & 76.75 & \cellcolor{green!10}{73.40$\uparrow$}& 38.45 & 60.66\textsuperscript{$\dagger$}  & \cellcolor{green!10}88.05$\uparrow$ & \cellcolor{green!10}{86.32}$\uparrow$ & 63.11$\uparrow$ & 67.31 &  56.2$\uparrow$ & 53.8$\uparrow$ & 30.7$\uparrow$ & 72.57$\uparrow$\\
\cmidrule{2-14}
15&\hspace{3mm}\textsc{+ Lv2 segment} & 73.70&  72.30$\uparrow$ & 47.80 & 57.53 & 86.26$\uparrow$& 83.73$\uparrow$ & 62.07$\uparrow$ & 67.48 & 56.0$\uparrow$ & 52.9$\uparrow$&  35.6$\uparrow$ & 72.57$\uparrow$\\
16&\hspace{3mm}\textsc{StructS-Lv2} & 71.31 & 72.30$\uparrow$ & 41.27 & 59.02 & 87.16$\uparrow$ & 84.78$\uparrow$ & 61.75$\uparrow$ & 66.80 & 53.4 & 54.2$\uparrow$ & 33.0$\uparrow$ & \underline{73.71}$\uparrow$\\
\midrule 
\midrule
% \RowStyle{\color{gray}}
\textbf{17}&\textbf{\textsc{InFusE}} &68.48 & {72.52} & 54.14 & 39.64 & 84.41 & 68.13 & 57.82 & 60.83 &  \underline{59.4} & 55.9 & 36.9 & 63.43\\
18&+ re-weighting& 67.30 & {72.37} & 53.44 & 40.54$\uparrow$ & 84.68$\uparrow$ & 74.31$\uparrow$ & 59.82$\uparrow$ & 62.56$\uparrow$ &  58.3 & {56.3}$\uparrow$ &  34.6 & 66.29$\uparrow$\\

\bottomrule

\end{tabular}
\caption{Results for all summarization tasks in \textsc{AggreFact-FtSOTA} (\textsc{AggreFact}), \textsc{DiverSumm}, \textsc{LongSciVerify} (LSV), \textsc{LongEval} and \textsc{LegalSumm} (LegalS).  In \textsc{DiverSumm},  CSM, MNW, QMS, AXV, and GOV refer to ChemSum, MultiNews, QMSUM, ArXiv, and
GovReport. We also report the macro-average of \textsc{DiverSumm} AUC. We highlight the \colorbox{green!10}{best} performed approach where multiple greens indicate systems indistinguishable from the best
according to a paired bootstrap test with p-value < 0.05, and the \underline{second-best} system for each column. The seven baseline models are \textbf{bolded}. Cells with \textsuperscript{$\dagger$} mean the result is {indistinguishable} from the raw baseline according to the bootstrap test. We report the average of 3 runs for GPT4o, given the randomness in LLM inference. 
%We highlight highest scores and scores significantly different from FULLDOC, SUMMACvariants and SENTLI approaches (at p < .05). 
%For \textsc{AggreFact}, we report the overall ROCAUC on XSum and CNN/DM, respectively.
}\label{tab:aggrefact_diversum_res}
\end{table*}

\paragraph{Baselines}
One of our baselines is \textbf{\textsc{AlignScore}} \cite{zha-etal-2023-alignscore}, an NLI-based metric that computes the aggregated inference score between a source article and generated summaries. We included \textbf{\textsc{Infuse}} \cite{zhang-etal-2024-fine}, which sets the SOTA on \textsc{DiverSumm},  \textbf{\textsc{MiniCheck FT5}} (MiniCheck-FlanT5 checkpoints) \cite{tang2024minicheck} that is a best-performing non-LLM fact-checker over multiple benchmarks, and \textbf{\textsc{LongDocFactScore}} \cite{bishop-etal-2024-longdocfactscore-evaluating} which claimed to work well on factuality validation of lengthy scientific article summaries. Our experiment notes that \textsc{MiniCheck} did not work well over long summaries due to its design objectives of short-statement fact-checking. We thus introduce \textbf{\textsc{MC-FT5 (SENT)}}, which computes the individual summary sentences' scores using \textsc{MiniCheck} and reports their average as the final summary score. We additionally include the \textbf{GPT4o} \cite{openai2024gpt4technicalreport} as the LLM fact-checker, using a prompt adopted from \citet{tang2024minicheck} (see Table \ref{tab:gpt4o_prompt} in Appendix \ref{appendix:implementation_details}). Lastly, we include \textbf{Llama-3.1-BeSpoke-MiniCheck-7B (BeSpoke-MC-7B)}\footnote{\url{https://huggingface.co/bespokelabs/Bespoke-MiniCheck-7B}}, the SOTA fact-checking model on the LLM-AggreFact benchmark \cite{tang2024minicheck}. Unless otherwise noted, we reran the baseline models on our datasets using the original authors' released code and checkpoints. Implementation details are provided in Appendix \ref{appendix:implementation_details}.
%\vspace{-3mm}
\paragraph{Our Approach}

We re-utilized  baseline models to compute the scores between context chunks and summary sentences, including \textsc{AlignScore} \cite{zha-etal-2023-alignscore}, \textsc{MiniCheck-FT5} (SENT) and \textsc{InfUsE} \cite{zhang-etal-2024-fine}, and experimented with below settings to apply our proposed approaches:
\setlist{nolistsep}
\begin{itemize}
\itemsep0em 
    \item + re-weighting: we apply the discourse-inspired re-weighting algorithm to adjust the sentence-level scores. We tune the factor $\alpha$ on height-subtree weighting as 1 over the validation set of \textsc{DiverSumm} and apply it to other benchmark datasets.
    \item + LvN \textsc{Segment}: Instead of using the default chunking approach, we segmented the source documents with the algorithms introduced in Sec. \ref{sec:source_segment} with different levels of granularity. 
    \item \textsc{StructS}-LvN: Combining top two methods.
\end{itemize}

The reweighting and segmentation can not be applied to \textsc{LongDocFactScore}, as it produced negative scores on all enumeration of source-target sentence pairs, which does not utilize the structural information. \textsc{InfUsE} utilizes the ranked list of entailment scores for all document sentences associated with each summary sentence. Thus, the segmentation approach does not affect.  
\vspace{-2.5mm}
\paragraph{Evaluation Metrics}

For experiments with {\textsc{AggreFact-FtSOTA}, \textsc{DiverSumm}} and \textsc{LegalSumm}, following \citet{laban2022summac, zhang-etal-2024-fine}, we adopt ROCAUC which measures classification
performance with varied thresholds as our evaluation metric.
On \textsc{LongSciVerify} and \textsc{LongEval}, we report Kendall's Tau $\tau$, following \citet{bishop-etal-2024-longdocfactscore-evaluating}.

\section{Results}\label{sec:result}
% \subsection{Factual Inconsistency Evaluation}\label{sec:main_result}
\textbf{Overall Performance} Table \ref{tab:aggrefact_diversum_res} presents our main results with detailed setups. Overall,  our proposed approach (with different combinations of re-weighting and segmentation settings) achieves the best or second best across \textsc{AggreFact}, most of \textsc{DiverSumm} and \textsc{LegalSumm} (\textsc{LegalS}).
Compared to top-performed LLM-based models (rows 3,4), our approach outperforms in 7 out of 11 datasets, with significant improvements on GOV, AXV, CSM, and \textsc{LegalSumm}.\footnote{More discussions on strong baselines in Appendix \ref{appendix:more_discussion}.}
The rest of the section addresses the following research questions: \textbf{RQ1:} Can the re-weighting algorithm help improve the models' performance?  \textbf{RQ2}: How does source document segmentation impact factual inconsistency detection? \textbf{RQ3}: How does combining both in \textsc{StructScore} perform?
\vspace{-1mm}
\paragraph{RQ1.}\label{sec:abalation_diversumm} \textit{We observe that the re-weighting algorithm improves prediction performance on different baselines (rows 5-6, 11-12, 17-18).} For long source documents, the re-weighting approach consistently improves or closely matches GOV, AXV, CSM splits in \textsc{DiverSumm} and the AXV split in \textsc{LongSciVerify} (LSV-AXV) and \textsc{LegalS} performance. Noticeably, \textsc{AlignScore} with reweighting scored the best on LegalS. On the other hand,  for both XSM and CND in \textsc{AggreFact-FtSOTA}, the re-weighting algorithm does not help much. We posit that the short summary length (1-3 sentences) has minimally structured information, so the scores will not change much. For MNW and QMS, the short summaries in QMS (averaging 3 sentences) reduce the effectiveness of the re-weighting algorithm. Moreover, MNW's non-factual sentences often receive high prediction scores, which our re-weighting approach tends to amplify, leading to a drop in performance. We also observe a slight performance drop on LSV-PUB and \textsc{LongEval-PUB} for \textsc{AlignScore} and \textsc{InfUsE}, potentially due to the different document structure of scientific articles from the medical domain. These observations also suggest potential future work for a dynamic weighting algorithm based on the document structure and domain knowledge.
In Table \ref{tab:diversumm_ablation}, we ablate the two discourse factors from the re-weighting algorithm with our best baseline MC-FT5 (SENT) on a subset of long datasets, noticing both features are helpful, and the improvement in adding subtree height is greater.\footnote{We include a more complete table in Appendix \ref{appendix:ablation_study}.}
\begin{table}[ht!]
\scriptsize
\centering
% \setlength{\tabcolsep}{1pt}
\begin{NiceTabular}{lcccc}
 \toprule
\multirow{1}{*}{\shortstack{\textbf{Model} }}
 &   \textbf{GOV}  & \textbf{AXV}  & \textbf{CSM} & \textbf{LSV-AXV} \\
% avg src. len &  4612.4 & 1138.7 & 4407.0 & 2008.2 & 669.2 \\
\midrule


{MC-FT5 (SENT)} & 83.24 & 78.66 & 59.74 & 52.73 \\

{\hspace{3mm}\textit{+ subtree height}} & 84.55 & 79.09 & 60.55 & 55.08 \\
{\hspace{3mm}\textit{+ depth score}} & 83.65 & 78.90 & 59.90 & 53.80 \\
 re-weighting  &  84.75 & 79.38 & 60.06& 55.08 \\


\bottomrule
\end{NiceTabular}
\caption{Ablation results on a subset of datasets from \textsc{DiverSumm} and \textsc{LongSciVerify}, the top and bottom rows are rows 11 and 12 in Table \ref{tab:aggrefact_diversum_res}.}\label{tab:diversumm_ablation}
\end{table}

\vspace{-1mm}
\paragraph{RQ2.} \textit{Applying document and discourse-structure-inspired approaches enhances performance across different baselines on long document summarization tasks.} We start by applying the level-1 and level-2 segmentation to preserve the document structures while segmenting at higher levels. For example, MC-FT5 (SENT) with \textsc{Lv1 Segment} (row 13) obtains the highest macro-average AUC on \textsc{DiverSumm}, a trend also observed with \textsc{ALignScore}. Specifically, comparing row 11 and row 13, the Lv1 \textsc{Segment} improved the model's performance on 7 of 8 long datasets from QMS to \textsc{LegalS} (i.e. 78.66 -> 85.22  and 83.24 -> 87.50  on AXV and GOV). However, the effect of fine-grained segmentation can vary depending on the document's length and structure. For instance, \textsc{AlignScore} in row 9 with Lv2 segment obtained better performance than Lv1 on LSV-PUB but was worse on QMS.
%We also observe that \textsc{Lv2 Segment} can lead to slight performance declines on multiple long document summarization datasets compared to \textsc{Lv1 Segment}, such as GOV, AXV and LSV-PUB for MC-FT5 (SENT) (row 14 vs. row 12),  as well as shorter datasets like XSM in \textsc{AggreFact-FtSOTA}) (74.27 in row 8 vs. 75.66 in row 4 and 73.70 vs. 79.62 in rows 14 and 10, respectively) for both models. 
\vspace{-1mm}
\paragraph{RQ3.} \textit{Combining both approaches is not universally beneficial across all scenarios}. When both individual approaches contribute positively, the combined \textsc{StructS} generally achieves better performance, as seen in row 8 on AXV, CSM, and row 14 on AXV. However, when one component causes a performance drop, combining both often leads to weaker overall performance than the stronger component alone. For instance, on GOV, row 8 performs worse than row 5, likely due to the segmentation in row 7, making the model less accurate. Similarly, row 14 performs slightly better than row 11 on LSV-PUB, but row 13's improvement does not translate into better performance gains when combined with row 12.  Differences in evaluation metrics (AUC vs. correlation) and dataset sizes may also have influenced these outcomes (i.e., row 14 does not improve much on \textsc{LongEval}-PUB while rows 12 and 13 have larger gains).

% We find that combining both re-weighting and source segmentation can further enhance the performance when neither approach hurts the performance much. For example,  \textsc{StructS-Lv1} on \textsc{AlignScore} obtains the best macro-average performance on \textsc{DiverSumm} compared to all its variants (rows 4-9), particularly on AXV and CSM tasks. This improvement persisted in rows 10-13. However, we observed that \textsc{StructS} occasionally underperformed compared to re-weighting and segmentation-only approaches on \textsc{LSV} and \textsc{LongEval}, as well as on GOV and \textsc{AggreFact}. In some cases, low performance in one component was amplified, leading to worse overall results when combined (e.g., on QMS, rows 5, 6, and 7 are lower than row 4). Differences in evaluation metrics (AUC vs. correlation) and dataset sizes may also have influenced these outcomes.
%We attribute this to our re-weighting algorithm being fine-tuned on \textsc{DiverSumm}’s development split, which has a different structure.





% \begin{table}[]
% \scriptsize
% \centering
% % \setlength{\tabcolsep}{1pt}
% \begin{NiceTabular}{l|ccc}
%  \toprule
% \textbf{Evaluation Model} & \textbf{LSV} & \textbf{LSV} & \textbf{LE} \\
% & -PUB & -AXV & -PUB \\
% avg src. len & 6515 & 3209 & 3193\\
% \midrule
% \multicolumn{4}{l}{\textit{Baselines}}\\
% \midrule
% LongDocFACTScore & \colorbox{green!10}{0.61} & \colorbox{green!10}{0.61} & 0.29 \\
% \textsc{InfUsE} & \underline{0.594} &0.559  & 0.369\\ 

% \textsc{MiniCheck-FT5}   & 0.174&0.381& 0.265 \\

% \textsc{MC-FT5 (sent)} & 0.557 & 0.527 & 0.302 \\
% \textsc{AlignScore} &  0.549 &  0.539 & 0.369\\
% GPT4o & 0.547 & 0.518 & \colorbox{green!10}{0.512}\\
% \midrule
% \multicolumn{4}{l}{\textit{Apply Discourse-inspired Re-weighting}}\\
% \midrule
% \textsc{InfUsE}-weighted & 0.587& \underline{0.574}& 0.343\\ 
% \textsc{AlignScore}-weighted & & 0.543 &   \\ 
% \textsc{MC-FT5 (sexnt)}-weighted &0.556 & 0.543 & 0.304 \\
% \midrule 
% \multicolumn{4}{l}{\textit{Ours with AlignScore as backbone}}\\
% \midrule 
% % \RowStyle{\color{gray}}

%  \multicolumn{4}{@{}c@{}}{\makebox[\linewidth]{\dashrule[black]}} \\
%   \multicolumn{4}{l}{\textit{{w/ Source Segmentation}}}\\
  
% \textsc{Segment}-LV.1 & 0.519 & {0.528} & {0.436} \\
% \textsc{Segment}-LV.2 & {0.581} & 0.511 &  {0.439} \\
% \textsc{Segment}-LV.3 & 0.549 &  0.513 & 0.423 \\
% \midrule
% \multicolumn{3}{l}{\textit{{w/ Source Segmentation + Discourse-inspired Re-weighting}}}\\
% StructS-LV.1 & 0.515 & {0.543} & {0.439} \\
% StructS-LV.2 & 0.568 & 0.526 &  \underline{0.444} \\
% StructS-LV.3 & 0.547 &  0.520  & 0.434 \\



% % \midrule

% \bottomrule
% \end{NiceTabular}
% \caption{Kendall’s Tau ($\tau$) correlations between the human factual consistency annotations and automatic
% metrics for LongSciVerify datasets (\textbf{LSV-PubMed/ArXiv}) and LongEval PubMed (LE PUB) dataset. We highlight the \colorbox{green!10}{best}
% performance for each dataset the second-best is \underline{underlined}.}\label{tab:longdocfactscore}
% \end{table}







\colorlet{green30}{green!20}
\colorlet{green50}{green!30}
\colorlet{green70}{green!40}
\colorlet{red30}{red!20}
\colorlet{red50}{red!30}
\colorlet{red70}{red!40}




% \subsection{Computational Cost Comparison}
% Following \citet{tang2024minicheck}, We compare the computational cost of different approaches and LLMs on our test split of \textsc{DiverSumm}. For most models, we
% use our hardware and convert the prediction
% time on our GPUs to the equivalent cost of using
% cloud computing services (see Appendix \ref{appendix:machine_config} for
% details). For GPT4o, we compute the costs of corresponding API calls. Results
% are shown in Table \ref{tab:computation_cost}. We see that non-LLM approaches have much lower inference costs while maintaining better or comparable performances on long document summarizations. 

% \begin{table}[]
%     \centering
%     \scriptsize
%     \begin{tabular}{l|c}
%     \toprule
%     \textbf{Model} & \textbf{Cost (\$)} \\
%     \midrule
%         \textsc{AlignScore} &  \$0.3\\
%         \textsc{StructScore} & \$0.3-\$0.6 \\
%         \textsc{LongDocFactScore} & \$0.3\\
%         \textsc{MC-FT5 (SENT)} &  \$3.2\\
%         \midrule
%          GPT4o&  \$15\\
%          \bottomrule
%     \end{tabular}
%     \caption{Comparison of models' inference cost (around \$0.5/GPU-hr) to the API model GPT4o on our test split of \textsc{DiverSumm}.}
%     \label{tab:computation_cost}
% \end{table}
% % \begin{table*}[th]
% % \footnotesize
% % \centering
% % \setlength{\tabcolsep}{2pt}
% % \begin{NiceTabular}{l|cccccccccc|c}
% %  \toprule
% % \textbf{Model} & Podcast& BillSum &SAMSum &News& Sales C& Sales E &Shkspr &SciTLDR &QMSum &ECTSum &Avg. (↓) \\
% % avg src len &1075.9 & 1202.5 & 121.0 & 506.6 & 403.9 & 371.5 &898.5 & 163.8 & 624.8 & 154.6 \\
% % \midrule 
% % \RowStyle{\color{gray}}
% % DAE &54.9 &55.1 &59.5 &61.7 &50.8 &55.0 &54.5 &55.2& 52.0 &58.6 &55.7 \\
% % \RowStyle{\color{gray}}
% % SummaC &58.5 &55.7 &54.7 &62.1& 59.0& 57.7& 59.3 &59.7 &56.6 &64.4 &58.8 \\
% % \RowStyle{\color{gray}}
% % QAFactEval& 64.0& 54.4 &66.3& 74.6& 68.5& 64.2 &61.9& 67.5& 62.4& 72.9 &65.7 \\

% % \RowStyle{\color{gray}}
% % PaLM2-bison &66.0& 62.0 &69.0& 68.4& 74.5 &68.1 &61.6& 78.1 &70.2 &72.3 &69.0 \\
% % \RowStyle{\color{gray}}
% % Dav003 &65.7 &59.9& 67.5& 71.2 &78.8 &69.4& 69.6& 74.4 &72.2 &77.9 &70.7 \\
% % \RowStyle{\color{gray}}
% % GPT3.5-turbo &68.4 &63.6 &69.1 &74.5 &79.7 &65.5 &68.1& 75.6 &69.2& 78.9 &71.3\\
% % \RowStyle{\color{gray}}
% % GPT4 &83.3& 71.1& 82.9 &83.3& 87.6& 80.1& 84.6& 82.4 &80.4 &88.0 &82.4 \\
% % \midrule 
% % \multicolumn{12}{c}{\textit{Our Scores}}\\
% % \midrule
% % Alignscore &  75.9  & 59.3  & 78.4  & 75.1  & 86.0  & 79.0  & 67.2  & 77.4  & 75.4  & 85.6  & 75.9 \\
% % STRUCTSCORE\_lvl1 & 75.0  & 58.8  & 73.0  & 70.8  & 85.6  & 77.8  & 66.8  & 75.0  & 71.3  & 76.3  & 73.0 \\
% % STRUCTSCORE\_lvl2 & 69.2  & 58.3  & 66.7  & 72.6  & 85.2  & 77.9  & 65.0  & 74.9  & 69.2  & 76.0  & 71.5 \\
% % STRUCTSCORE\_lvl3 & 69.0  & 59.1  & 66.3  & 71.7  & 81.4  & 75.1  & 64.9  & 62.4  & 72.1  & 66.2  & 68.8 \\
% % \midrule
% % \multicolumn{12}{c}{\textit{Upper Bound}}\\
% % \midrule

% % \RowStyle{\color{gray}}
% % GPT4 Oracle &90.2 &85.5& 86.3 &88.3& 91.1 &83.5 &96.6& 86.3 &89.9 &91.7& 88.9\\
% % \RowStyle{\color{gray}}
% % Human Perf.& 90.8 &87.5 &89.4 &90.0& 91.8 &87.4& 96.9& 89.3 &90.7 &95.4 &90.9 \\
% % \bottomrule

% % \end{NiceTabular}
% %  \caption{Results for all summarization tasks in the \textsc{SummEdits} dataset. We report the balanced accuracy score.}
% %  \end{table*}

 

 
% % \begin{table*}[t]
% % \small
% % \centering
% % \setlength{\tabcolsep}{4pt}
% % \begin{NiceTabular}{llcccc|cccc}
% %  \toprule
% % \multirow{4}{*}{{\parbox{1cm}{\textbf{Model Type}}}}  & \multirow{4}{*}{\parbox{1.5cm}{\textbf{Evaluation Models
% % }}} & \multicolumn{4}{c}{Sentence-Level (BAcc $\uparrow$)} & \multicolumn{4}{c}{Summary-Level (BAcc $\uparrow$)}\\
% % \cmidrule(lr){3-6} \cmidrule(lr){7-10}
% % && \multicolumn{2}{c}{MediaSum} & \multicolumn{2}{c}{MeetingBank} & \multicolumn{2}{c}{MediaSum} & \multicolumn{2}{c}{MeetingBank}\\
% % \cmidrule(lr){3-4} \cmidrule(lr){5-6} \cmidrule(lr){7-8}\cmidrule(lr){9-10}
% % &  & Main & Marginal & Main & Marginal& Main & Marginal& Main & Marginal\\
% % \midrule 
% % - &Baseline & 50.0 &50.0& 50.0 &50.0 &50.0 &50.0 &50.0 &50.0 \\
% % \midrule
% % \multirow{4}{*}{\parbox{0.8cm}{\textbf{Non-LLM}}} & 
% % \RowStyle{\color{gray}}
% % \textsc{SummaC-ZS} & 66.1&  73.9&  63.9&  80.6 & 62.7&  64.1 & 58.1 & 72.4 \\
% % \RowStyle{\color{gray}}
% % & \textsc{SummaC-CV} &67.6 &73.0& 62.6 &77.3 &61.2 &66.5 &52.4 &72.9 \\
% % \RowStyle{\color{gray}}
% % & \textsc{QAFactEval} &53.9 &74.0 & 58.0 &75.8 & 61.4 &74.2 &55.1 &68.2 \\
% % \RowStyle{\color{gray}}
% % & \textsc{AlignScore} & 69.2 & 76.2 & 61.2  &78.6  &65.5 & 72.1 & 63.4 & 71.8 \\
% % \cmidrule(lr){2-10}
% % & \textsc{AlignScore}* & 68.1 & 75.5 & 60.4  &78.8  &62.8 & 71.6 & 64.1 & 72.1 \\
% % \midrule
% % \multirow{4}{*}{\parbox{0.8cm}{\textbf{Open
% % Source
% % LLM}}}&
% % \RowStyle{\color{gray}}
% % Vicuna-13B &54.0 &54.8 &49.6 &61.9 &55.6 &59.1 &51.2 &59.2 \\
% % \RowStyle{\color{gray}}
% % &Vicuna-33B &51.0 &51.1 &53.6& 48.4 &52.5 &53.4 &53.2 &51.0 \\
% % \RowStyle{\color{gray}}
% % &WizardLM-13B &59.8 &53.5 &58.8 &56.6 &57.0 &54.5 &54.6& 58.0 \\
% % \RowStyle{\color{gray}}
% % &WizardLM-30B &54.5 &53.9 &53.5 &53.4 &53.3& 54.4 &53.0 &53.2 \\
% % \midrule 
% % \multirow{2}{*}{\parbox{0.8cm}{\textbf{Prop.
% % LLM}}} & 
% % \RowStyle{\color{gray}}
% % { GPT-3.5-Turbo} &  61.6 & 68.9 & 56.0&  65.0 & 59.6 & 65.8&  63.2&  65.7\\
% % \RowStyle{\color{gray}}
% % & GPT-4 & 64.9 & 80.2 & 67.5 & 90.3 & 63.7 & 78.9 & 74.7 & 83.1 \\
% % \midrule 
% % \multirow{2}{*}{\parbox{1.5cm}{\textbf{Ours}}} & \textsc{AlignScore-utter}  & 64.6 & 73.3 & 58.2 & 79.3 & 61.1 & 71.6 & 56.6 & 66.7\\
% % &
% % \\ 
% % \\ 
% % \midrule
% % \end{NiceTabular}
% % \caption{\textbf{Sentence-level and summary-level balanced accuracy (BAcc) for factual consistency evaluators on
% % the test set of \textsc{TofuEval}}. Most scores are taken from \citet{Tang2024TofuEvalEH}. * means a reproduced score evaluation. For LLM-based methods, summary-level labels are aggregated sentence-level
% % labels, as it achieves better performance than directly predicting consistency labels on whole summaries. Note that a baseline method that always predicts inconsistent or consistent
% % achieves 50\% balanced accuracy}
% % \end{table*}




% \input{Section/Experimental_Analysis}
Software development is increasingly conceived as a collaboration activity between developers and AIs. Indeed, IDEs already implement features to enable interactive development, with AI suggesting implementations that are reused by developers.

Although multiple studies show this interaction can be successful, there is still limited understanding of how the models must be configured and used in the context of code generation tasks. This study addresses this gap, systematically investigating the impact of several key parameters, including the repeated submission of a prompt to accommodate for the non-deterministic nature of the models.

Our study reveals several key findings about the usage of ChatGPT. In particular, we discovered how creativity, although up to a limited extent, is useful to increase the range of methods whose code can be generated correctly. A major role is played by parameter top-p, which is commonly underrated, and instead has a major impact on the correctness of the results, with lower values producing better results. Finally, prompts should be submitted multiple times, with $5$ repetitions combined with a temperature of $1.2$ resulting in an effective configuration in our experiments.  

Future work concerns two main research directions. One is about replicating this experiment with other AI assistants, to validate our findings in multiple contexts. The second research direction concerns finding strategies to deal with the need to submit the same prompt multiple times to obtain a useful result, and thus developing approaches able to select or merge multiple responses automatically. 
\section{Limitations} \label{sec:limitations}

While the above results demonstrate an important step toward flexible and robust humanoid locomotion, our proposed technique is not a panacea. 
%
Both HLIP and CI-MPC require parameter tuning, and their combination only increases the complexity of this process. While we used only one set of parameters for all the experiments, we did find that some parameters induced sharp tradeoffs. For example, a lower weight on base orientation tracking gave more natural-looking gaits, but reduced push recovery performance.
%


Our CI-MPC implementation uses significantly simplified collision geometries. This enables fast solve times, but precludes behaviors that involve contact away from the hands and the feet. As a result, the robot is not able to automatically recover from a fall. Furthermore, our CI-MPC solver's performance is reliant on smooth collision geometries, as sharp corners introduce problematic discontinuous gradients. 
%
Similarly, self-collisions present a major failure mode in the current implementation. Adding self-collision constraints either in the optimization problem \cite{grandia2021multi} or with a high order control barrier function \cite{khazoom2024tailoring, ames2019control, singletary2021safety} presents an obvious next step for improving reliability.

Finally, there are instances in which HLIP's suggested contact sequence guides the robot in an unhelpful direction. For example, if the robot is standing and pushed to the left, HLIP might suggest lifting the right leg, depending on the timing of the gait cycle. This could be mitigated with a richer reduced-order model, but illustrates a trade-off inherent to guiding whole-body behaviors with a reduced-order model.

% \section*{Acknowledgements}


% Bibliography entries for the entire Anthology, followed by custom entries
%\bibliography{anthology,custom}
% Custom bibliography entries only
\bibliography{custom}

\appendix

% \section{Example Appendix}
% \label{sec:appendix}
\newpage


\newpage
\clearpage
\section{Discourse Analyses}
\subsection{Short Summary Analysis}\label{appendix:short_analysis}
\begin{table}[h!]
\scriptsize
\setlength\columnsep{1pt}
\begin{NiceTabular}{l|ccc}

\toprule
\textbf{Dataset} & \textbf{Size} &  \textbf{Gran} & \textbf{Error Tag} \\
\midrule

\textsc{AGU}\_\textsc{Cliff} &  300 & word & intrin./extrin./other/wld. knowl.\\
\textsc{AGU}\_Goyal'22 & 150   & span & intrins./extrin./other \\
\bottomrule
\end{NiceTabular}
\caption{Statistics of Sent/Span-level factual inconsistency datasets \textsc{AggreFact-Unified} (AGU) \cite{tang-etal-2023-understanding}. We report the size of doc-summary pairs (Size), the granularity of annotation (Gran), and the error labels (Error Tag).  }\label{tab:token_label_short}
\end{table}
We also conduct a discourse analysis on \textsc{AggreFac-United} \cite{tang-etal-2023-understanding}, as shown in Table \ref{tab:token_label_short}. This dataset includes BART and Pegasus summaries from \textsc{Cliff} \cite{cao-wang-2021-cliff} and Goyal'21 \cite{goyal-durrett-2021-annotating}.\footnote{\textsc{AggreFact-Unified} (\textsc{AGU\_Cliff}) includes additional error types such as \textit{comments}, \textit{other errors: noise, grammar} and \textit{world knowledge} (wld. knowl.)}  In the Goyal22 split of AGGREFACT-UNITED,
a total of 61 errors were detected. Intrinsic errors
are found to appear more often in satellite EDUs (18/31) with the attribution relation. Regarding extrinsic errors, the nucleus EDUs take the majority. We further analyzed the CLIFF dataset \cite{cao-wang-2021-cliff}, where span-level annotations of faithful errors are available. Out of 600 sentences, the parser failed to parse 131 summaries, likely due to their short lengths and simplistic structures. Therefore, our analysis focused on the 469 summaries that were successfully parsed. We observed that Elementary Discourse Units (EDUs) containing errors are more likely to appear at the bottom of the discourse tree.  These findings are similar to the long summary analysis in \S \ref{sec:discourse_analysis}. %pecifically, the relations labeled as "Background" and "Elaboration" appear more frequently in instances of intrinsic errors, while "Attribution," "Background," and "Elaboration" are predominant in summaries with extrinsic and worldknowledge errors.




% \subsection{Discourse Relations in RST}\label{appendix:discourse_relations}
% We include the complete list of coarse-grained and fine-grained relation classes in the RST Discourse Treebank in Table \ref{tab:discourse_relations}, as summarized in \cite{feng_2015}.

\subsection{Discourse Features}
\label{appendix:disocurse_feature_details}

Following prior work \cite{louis-etal-2010-discourse}, we analyze the nucleus-satellite penalty score (Ono penalty) \cite{ono-etal-1994-abstract}, the maximum depth (Depth score) \cite{Marcu1998tobuild}, and the promotion-based score \cite{Marcu1998tobuild} for sentence level. The penalty/score for
a sentence is computed as the maximum of the
penalties/scores of its constituent EDUs.
For the normalized version, instead of following \citet{louis-etal-2010-discourse}, who normalized them by the number of words in the
document, we opt to divide the scores by the maximum depth of the discourse tree, which similarly alleviates the scores' dependencies on document length. Below, we provide one example demonstrating the computation of each score (borrowed from \citet{louis-etal-2010-discourse}) and will release our code for reproduction purposes.

\begin{figure}[t!]
    \centering
    \includegraphics[width=0.9\linewidth]{Figs/discourse_example.jpg}
    \caption{RST for the example sentence, and the salient units (promotion
set) of each text span are shown above the horizontal line, which represents the span.The example is taken from \citet{louis-etal-2010-discourse}.}
    \label{fig:discourse_example}
\end{figure}

\subsubsection{Example}
Here, we re-utilize the example from \citet{louis-etal-2010-discourse}, which is part of the RSTDT \cite{carlson2002rst} in Figure \ref{fig:discourse_example}, which contains four EDUs.

\textit{1. [Mr. Watkins said] 2. [volume on Interprovincial’s system is down about 2\% since January] 3. [and is expected to
fall further,] 4. [making expansion unnecessary until perhaps
the mid-1990s.]} 
\paragraph{Nucleaus-Satellite Penalty (Ono Penalty)}\cite{ono-etal-1994-abstract}: The spans of individual EDUs are represented
at the leaves of the tree. At the root of the tree, the
span covers the entire text. The path from EDU 1
to the root contains one satellite node. It is, therefore, assigned a penalty of 1. Paths to the root from
all other EDUs involve only nucleus nodes; subsequently, these EDUs do not incur any penalty. Thus, the Ono Penalty scores for EDU 1 to 4 are [1, 0, 0, 0].

\paragraph{Maximum Depth Score}
Below we cite the original texts from \cite{louis-etal-2010-discourse}.
\begin{quote}


\citet{Marcu1998tobuild} proposed the method to utilize the nucleus-satellite distinction, rewarding nucleus status instead of penalizing the satellite. He introduced the notion of \textit{promotion set}, consisting of
salient/important units of a text span. The nucleus is denoted as the more salient unit in the full span of
a mono-nuclear relation (i.e., in Elaboration, the satellite unit is to elaborate on the key information of the nucleus. Thus, the latter is more salient). In a multinuclear relation,
all the nuclei are salient units of the larger span.

For example, in Figure \ref{fig:discourse_example}, EDUs 2 and 3 participate in a multinuclear (List) relation. As a result,
both EDUs 2 and 3 appear in the promotion set of
their combined span (2-3). The salient units (promotion
set) of each text span are shown above the horizontal line which represents the span. At the leaves,
salient units are the EDUs themselves.


For the purpose of identifying important content, units in the promotion sets of nodes close to
the root are hypothesized to be more important
than those at lower levels. The highest promotion of an EDU occurs at the node closest to the
root, which contains that EDU in its promotion set.
The depth of the tree from the highest promotion
is assigned as the score for that EDU. Hence, the
closer to the root an EDU is promoted, the better
its score. Since EDUs 2, 3 and 4 are promoted all
the way up to the root of the tree, the score assigned to them is equal to 4, the total depth of the
tree. EDU 1 receives a depth score of 3.
\end{quote}

Thus, the final maximum depth score based on the promotion set for EDUs 1-4 are [3, 4, 4, 4]. 

\paragraph{Promotion Score}
In the same example, while EDUs 2, 3, and 4 all have a depth score of 4, EDUs 2 and 3 are promoted to the root from a greater depth than EDU 4. To account for the difference, \citet{Marcu1998tobuild} further introduced the promotion score, which is a measure of the number
of levels over which an EDU is promoted. For instance, EDU 2 is promoted by three levels, while EDU 4 is promoted by two levels. Thus,
EDUs 2 and 3 receive a promotion score of 3, while the score of EDU 4 is only 2. EDU 1, given that it is never promoted received scores of 0. 


\paragraph{Discourse Tree Computation}
In Section \ref{sec:discourse_analysis} Table \ref{tab: subtree_struct}, we compute the tree depth as follows.
We use a string-matching system to construct a dictionary that aligns annotated sentences with EDU segments. For instance, in Figure \ref{fig:discourse_example}, the sentence is mapped to EDUs 1-4. We then compute the maximum depth of the discourse tree from the root node to the lowest leaf node, which would be 3 in this case. However, there may be cases where sentences are segmented into EDUs that are not gathered into a single node in the parsed discourse tree. In such instances, we employ the methods described in Section \ref{sec:discourse_analysis} to approximate the depth.

\section{LegalSumm Dataset}\label{appendix:legalsumm_detail}
We utilized a subset of the \textbf{CanLII Dataset} \cite{xu2021}, which consisted of 1,049 legal opinion documents with expert-written summaries.\footnote{Data obtained through an agreement with CanLII
\url{https://www.canlii.org/en/}}. We followed the setting from \citet{elaraby-etal-2024-adding}, where we consider the output of three different abstractive models in our annotation process: (1) \textbf{Finetuned LED-base} \cite{elaraby-litman-2022-arglegalsumm} which finetuned
the pre-trained longformer-encoder-decoder \cite{beltagy2020longformer} (LED) on the CanLII cases without additional information about the argument structure
of the document (2) \textbf{arg-LED-base}, which utilizes the LED model but includes the information about the argument units (Issues, Reasons, and Conclusions) in its training phase, and (3) \textbf{arg-aug-LED-base}, a model introduced in \citet{elaraby-etal-2023-towards} that can select a summary from multiple augmented versions of generated summaries based on its overlap with the input case’s predicted argument roles. 

\paragraph{Annotation Details}
We conducted evaluations with two voluntary legal experts from the research group, all of whom hold a J.D. degree and
possess at least four years of experience
in providing professional legal services. For each summary, the annotators are asked to select from four choices justifying the factual consistency of the model-generated summary with the reference summary and source article. They are also encouraged to provide free-text rationales justifying their selections.

To guarantee the quality of annotation, we conducted multiple
sessions with annotators to refine the guidelines and continuously monitor the agreements. Ultimately, the first author and the two annotators held in-person sessions to resolve label inconsistencies. The labels remained unresolved in two cases as the annotators identified differing yet reasonable interpretations of the instructions.  We thus retain the average scores as is. To distinguish summaries with severe or moderate factual inconsistencies from those without, we computed the average of the two annotators' ratings and rounded based on a threshold of 2.
The annotation guideline is included in Figure \ref{fig:annotation_guideline}.

\begin{figure*}
    \centering
    \includegraphics[width=0.9\linewidth]{Figs/annotation.001.jpeg}
    \caption{The annotation interface for LegalSumm. The left panel displays the instructions and the content to be annotated. Annotators are then prompted to select one of four options, as shown in the right panel.}
    \label{fig:annotation_guideline}
\end{figure*}

\section{Discourse Analysis on Fine-grained Error Types}\label{sec:appendix_discourse_test}
\begin{table*}[h!]
    \centering
    \scriptsize
    \begin{NiceTabular}{l|c|c|c|c|c|c|c|c}
    \toprule
    \textbf{RST features}     & \textbf{GramE} & \textbf{LinkE} & \textbf{OutE} & \textbf{EntE} & \textbf{PredE} & \textbf{CorefE} & \textbf{CircE} & \textbf{ALL Errors}\\
    Count & (83) & (35) & (48) & (117) & (15) & (9) & (13) & (320)\\
    
    \midrule
      Ono penalty   &  -1.166\textsuperscript{}& 1.855\textsuperscript{}& 0.621\textsuperscript{}& 1.647\textsuperscript{}& 0.730\textsuperscript{}& 0.215\textsuperscript{}& 1.627\textsuperscript{}& 1.606 (0.1089)\\
      Depth score& -5.218\textsuperscript{**}& -7.381\textsuperscript{**}& -4.628\textsuperscript{**}& -3.252\textsuperscript{**}& -2.002\textsuperscript{}& 0.214\textsuperscript{}& -0.565\textsuperscript{}& -8.249 (0.0000)\\
    Promotion score  & -6.519\textsuperscript{**}& -0.971\textsuperscript{}& -0.440\textsuperscript{}& 1.734\textsuperscript{}& -0.195\textsuperscript{}& 2.613\textsuperscript{*}& 0.629\textsuperscript{}& -0.828 (0.4083)\\
    \midrule
    Normalized penalty  &  -1.742\textsuperscript{}& 3.051\textsuperscript{**}& 0.695\textsuperscript{}& 1.990\textsuperscript{*}& 0.673\textsuperscript{}& -0.002\textsuperscript{}& 0.493\textsuperscript{}& 2.160 (0.0314)\\
      Normalized depth score & -6.689\textsuperscript{**}& -6.043\textsuperscript{**}& -4.823\textsuperscript{**}& -3.307\textsuperscript{**}& -1.731\textsuperscript{}& -0.153\textsuperscript{}& -1.986\textsuperscript{}& -9.084 (0.0000)\\
        Normalized promotion score &  -5.754\textsuperscript{**}& 0.487\textsuperscript{}& -0.322\textsuperscript{}& 1.796\textsuperscript{}& -0.087\textsuperscript{}& 2.206\textsuperscript{}& -0.218\textsuperscript{}& -0.303 (0.7617)  \\
    \bottomrule
      
    \end{NiceTabular}
    \caption{Two-sided t-test statistic of significant RST-based
features comparing unfaithful sentences to faithful
ones in \textsc{DiverSumm} annotated split. We report the
test statistics and significance levels. For fine-grained errors, we report the significant level in * (0.01 <= p-value <=0.05) and ** (p-value <=0.01). For All errors, we report the p-value in parenthesis.}
    \label{tab:rst_result_all_errors}
\end{table*}



\paragraph{Error Types} Relation
Error (PreE) is when the predicate in a summary sentence is inconsistent with respect to the document. Entity Error (EntE) is when the primary
arguments of the predicate are incorrect. Circumstance Error (CircE) is when the predicate’s circumstantial information (i.e., name or time) is wrong.
Co-reference error (CorefE) is when there is a pronoun or reference with an incorrect or non-existing
antecedent. Discourse Link Error (LinkE) is when
multiple sentences are incorrectly linked. Out of
Article Error (OutE) is when the piece of summary
contains information not present in the document.
Grammatical Error (GramE) indicates the existence
of unreadable sentences due to grammatical errors.


% \paragraph{Error Examples}

\paragraph{Fine-grained Error Analysis} In Table \ref{tab:rst_result_all_errors}, we demonstrate the breakdowns of fine-grained error types and report the t-test results on different discourse features. 


    



\section{Example of Segmentation Failures}\label{appendix:segmentation_examples}
This section includes one example of the \textsc{AlignScore}'s chunking method that failed to preserve the document structure, while our discourse-inspired chunk addresses it.

For example, as shown in Figure \ref{fig:sub1}, the original document contains two consecutive sentences: "To determine the extent ..." and "To develop the SMS" (highlighted in the orange box). These sentences are meant to be read together and should not be separated.  However, the default chunking approach in \textsc{AlignScore} and \textsc{MiniCheck} breaks this continuity by placing them in two separate chunks, given the former chunk is large enough. On the contrary, our approach maintains the structural integrity of the documents, keeping the sentences connected as intended. Similarly, in Figure \ref{fig:sub2}, the conclusion section is separated into two chunks by the default chunking approach, while our method maintains them in a single chunk.

\begin{figure*}[]
\centering
    \begin{subfigure}[b]{1\textwidth}
        \centering
        \includegraphics[width=1\textwidth]{Figs/segmentation_example.001.jpeg}
        \caption{Example from GovReport of \textsc{DiverSumm}.}\label{fig:sub1}
    \end{subfigure}%
    \caption{Example of segmentation failures, left is the output of chunking method used in \textsc{AlignScore} and \textsc{MiniCheck}, right is the segments produced by our segmentation method.}\label{fig:segmentation_fault1}
\end{figure*}

\begin{figure*}[]
    
    \begin{subfigure}[b]{1\textwidth}
        \centering
        \includegraphics[width=1\textwidth]{Figs/segmentation_example2.001.jpeg}
        \caption{Example from ArXiv of \textsc{DiverSumm}.}\label{fig:sub2}
    \end{subfigure}
    \caption{Example of segmentation failures, left is the output of chunking method used in \textsc{AlignScore} and \textsc{MiniCheck}, right is the segments produced by our segmentation method.}\label{fig:segmentation_fault2}
\end{figure*}




% We demonstrate all the prompt templates used, including ``Aims and Conditions Recognition'', ``Prune by Semantics'' and ``Generate Final Answer With LLMs'', as shown in Figure~\ref{fig:extract aims and conditions template}, Figure~\ref{fig:prune by semantics template}, and Figure~\ref{fig:generate answer template}.

\newpage
\clearpage
\section{Implementation Details}\label{appendix:implementation_details}

\subsection{GPT4o Prompts}
We include our prompt for zero-shot factual consistency evaluation in Table \ref{tab:gpt4o_prompt}.
\begin{table*}
\scriptsize
    \begin{center}
        \begin{tabular}{l}
        \toprule
             Determine whether the provided claims are consistent with the corresponding document. Consistency in this context\\
             implies that
    all information presented in the claim is substantiated by the document.  If not, it should be \\
    considered inconsistent.\\
    \\
    Document: [DOCUMENT]\\
    Claims: [CLAIMS] \\
    Please assess the claim’s consistency with the document by responding with either "yes" or "no".\\
    The CLAIMs are ordered in the format of a dictionary, with \{ index: CLAIM \}. You will need to return the result in JSON format.\\
    For instance, for a CLAIMs list of 4 items, you should return \{0:yes/no, 1:yes/no, ...., 3:yes/no\}.\\
\\
    ANSWER: \\
            \bottomrule
        \end{tabular}
        \caption{Zero-shot factual consistency evaluation prompt for GPT4o.}\label{tab:gpt4o_prompt}
    \end{center}
\end{table*}

\subsection{Baselines}
\paragraph{AlignScore} (model size 355M) \cite{zha-etal-2023-alignscore} is an entailment-based model that has been trained on data from a wide range of tasks such as NLI, QA, and fact verification tasks. It divides the source document into a set of sequential chunks at sentence boundaries. For a multi-sentence summary, it predicts the max scoring value of all combinations of source chunk and target sentence, then returns the unweighted average of all sentences as the summary prediction. We follow the original setting by setting chunk size at 350 tokens and use the default model alingsocre\_large ckpt. The model outputs a score between 0 and 1. We conduct experiments on top of their released codebase \url{https://github.com/yuh-zha/AlignScore}.
\paragraph{MiniCheck-FT5} (model size 770M) \cite{tang2024minicheck} is an entailment-based fact checker built on flan-t5-large. It has been further fine-tuned on 21K datapoints from the ANLI dataset \cite{nie2019adversarial} and 35k synthesized data points generated in \cite{tang2024minicheck} on the tasks to predict whether a given claim is supported by a document. We follow the authors's setting and set the chunk size to 500 tokens using white space splitting. The output score is between 0 and 1. We use the released code repo from \url{https://github.com/Liyan06/MiniCheck}.

\paragraph{LongDocFactScore} \cite{bishop-etal-2024-longdocfactscore-evaluating} is a reference-free framework for assessing factual consistency. It splits source documents and the generated summary into sentences, then computes the pair-wise similarities by computing the cosine similarities of sentences (they use the sentence-transformers library initialized with the bert-base-nmli-mean-tokens model). Afterward, for each individual summary sentence,  K most similar source sentences are picked.  The method extracts the neighboring source document sentences of the selected sentences as context, then applies a metric  BARTScore to evaluate the score between source context and summary sentences. The overall summary score is an unweighted average of all sentences. We follow the authors' parameters setting and utilize their released code repo from \url{https://github.com/jbshp/LongDocFACTScore}.

\paragraph{InfUSE} (model size 60M) \citet{zhang-etal-2024-fine} uses a variable premise size and breaks the summary into sentences or shorter hypotheses. Instead of fixing the source context, it retrieves the best possible context to assess the faithfulness of an individual summary sentence by applying an NLI model to successive expansions of the document sentences. Similar to prior approaches, it outputs an entailment score for each summary sentence, and the summary-level score is the unweighted average. We follow their settings on \textsc{InfUsE} with summary sentences instead of \textsc{InfUsE\textsubscript{SUB}} as the authors only released the code for the former model. \textsc{InfUsE} outputs scores in the range 0-1. We use the author's released codebase from \url{https://github.com/HJZnlp/Infuse}.
\paragraph{GPT4o}
 We used the version of gpt-4o-2024-05-13; we set max\_tokens 100, sampling temperature at 0.7, and  top\_p as 1.0. We call the OpenAI API from \url{https://openai.com/api}. Given the lengthy summary, we prompted the LLM to assign a binary label (yes/no) to assess individual summary sentences' consistency with the original article. Then, we reported the percentile of ``yes'' answers as the summary-level rating. 

 \paragraph{BeSpoke-MC-7B}
 We harnessed the SOTA Llama-3.1-Bespoke-MiniCheck-7B (BeSpoke-MC-7B) released by Bespoke Labs. The model is fine-tuned from ``internlm/internlm2\_5-7b-chat'' \cite{cai2024internlm2} on the combination of 35K data points following the approach in MiniCheck \cite{tang2024minicheck}. We use the suggested code repo from \url{https://huggingface.co/bespokelabs/Bespoke-MiniCheck-7B}. To calculate the AUC score, we employed the raw probabilities returned by the code to determine sentence-level ratings, and we calculated the summary-level score as the unweighted average across all sentences.



\subsection{Machine Configuration for Models}\label{appendix:machine_config}
We use up to 4 NVIDIA RTX 5000 GPUs, each equipped with 16 GB VRAM,  for model inferences on our hardware. According to Lambda\footnote{\url{https://lambdalabs.com/service/gpu-cloud}} (RTX5000 is depreciated), a single NVIDIA Quadro RTX 6000 (the closest to our setting) GPU costs \$0.5 per hour and has 24 GB VRAM. Additionally, we loaded the Bespoke-MC-7B model on a single NVIDIA L40S GPU with 48 GB of VRAM, provided by the Pitt CRC computing cluster.

\section{Experimental Results}
\subsection{Discussion on Performance Compared to Strong Baselines}\label{appendix:more_discussion}
Our primary analysis focuses on discussing how the proposed approach can improve different baselines (we utilized three backbone baselines: rows 5, 11, and 17 with their improved versions) in Table \ref{tab:aggrefact_diversum_res}. We observe several baselines obtained the best performance on certain tasks and provide a more careful justification below: 

\paragraph{}
While the improvements may appear marginal in some baseline models, they are statistically significant and consistent across multiple datasets. The capabilities of baseline models and the characteristics of testbeds can also affect performance. For instance, as noted in Section \ref{sec:result}, dialogue-based inputs in QMS limit the effectiveness of discourse parsing (RQ2), while short summaries like XSUM minimize the impact of reweighting (RQ1). On longer datasets like AXV and CSM, gains are more substantial, with improvements of up to 7 points (row 14 vs. row 11 in AXV). This is comparable to, or even more significant than, prior work \cite{zhang-etal-2024-fine}, and it is common to observe varying levels of performance gains across different tasks \cite{tang-etal-2023-understanding,tang2024minicheck}.

\paragraph{LongDocFactScore} (LDFS) introduced the LongSciVerify (LSV) dataset (PUB and AXV), using a different annotation method by subsampling three sentences with human annotations for factuality. We conjecture this may lead to less accurate summary-level labels, favoring their metric, which utilizes the top-k sentence-level scores. Meanwhile, LDFS underperformed compared to most other baselines on AggreFact, QMS, AXV (from \textsc{DiverSumm}), and LongEval-PUB. In contrast, our approach outperformed LongDocFactScore on most other benchmarks (e.g., 86.32 vs. 65.36 on AXV), suggesting our approach is more robust and capable of handling different long document summarization datasets. While each baseline may excel in specific tasks, a more robust benchmarking dataset could better ensure fair comparisons for future research.

\paragraph{GPT-4o} GPT4o is utilized as a comparison between the SOTA LLMs (GPT4o models have unknown sizes but could be greater than known open-sourced LLMs with up to 405B) and our lightweight model (770M), which in the usual case, the LLMs can outperform baselines by noticeable margins \cite{tang2024minicheck}). In Table \ref{tab:aggrefact_diversum_res}, regarding the long document summarization datasets (from GOV in DiverSumm to LegalSumm), our models (rows 12, 13) outperformed GPT4o in 5 out of 6 test sets (the only exception is LongEval PUB). This confirmed that the discourse-inspired approaches are beneficial.

\paragraph{BeSpoke-MC-7B} is claimed to be the best fact-checking model publicly available on the LLM-AggreFact benchmark, which outperformed many other LLMs with bigger sizes. Compared to our proposed models, it performed better on QMS, XSM\textsubscript{AG}, LSV-AXV, and had the best performance on LongEval-PUB (similar pattern to GPT-4o). However, on other benchmarks, our discourse approaches still demonstrate their benefits (i.e., on \textsc{LegalSumm}, AlignScore + reweighting obtained 76.57 while BeSpoke-MC-7B only scored 55.81). 


\subsection{Ablation Study}\label{appendix:ablation_study}
Table \ref{tab:larger_ablation} presents the ablation results of different discourse features on our baselines. We cover the long document summarization tasks starting from QMS in Table \ref{tab:aggrefact_diversum_res}.

\begin{table*}[ht!]
\small
\centering
% \setlength{\tabcolsep}{1pt}
\begin{NiceTabular}{lccccccc}
 \toprule
\multirow{1}{*}{\shortstack{\textbf{Model} }}
 &  \textbf{QMS} &  \textbf{GOV}  & \textbf{AXV}  & \textbf{CSM} & \textbf{LSV-PUB} & \textbf{LSV-AXV} & \textbf{LE-PUB}\\
% avg src. len &  4612.4 & 1138.7 & 4407.0 & 2008.2 & 669.2 \\
\midrule


{MC-FT5 (SENT)} &  60.66 & 83.24 & 78.66 & 59.74 & 55.7 & 52.7 & 30.2 \\

{\hspace{3mm}\textit{+ subtree height}} & 60.21 & 84.55 & 79.09 & 60.55& 53.6 & 55.1 & 30.4\\
{\hspace{3mm}\textit{+ depth score}} & 60.51 & 83.65 & 78.90 & 59.90 & 55.7 & 53.8 & 33.3  \\
 re-weighting  & 60.36 & 84.75 & 79.38 & 60.06& 52.8 & 55.1 & 31.4 \\
\midrule 
\midrule
{AlignScore} & 56.48 & 87.02 & 77.46 & 61.03 & 54.9 & 53.9 & 36.9 \\

{\hspace{3mm}\textit{+ subtree height}} &52.91 & 87.29 & 81.15 &  60.47 & 51.7 & 55.4 & 34.1\\
{\hspace{3mm}\textit{+ depth score}} & 56.63 & 87.29 & 77.66 & 60.30 & 54.3 & 52.4 & 36.6  \\
 re-weighting  & 53.95 & 87.29 & 81.15 & 60.55 & 53.0 & 54.3 & 34.8 \\

\bottomrule
\end{NiceTabular}
\caption{Ablation results on long document datasets from \textsc{DiverSumm}, \textsc{LongSciVerify} and \textsc{LongEval}.}\label{tab:larger_ablation}
\end{table*}


% \section{LegalSumm Annotation Guideline}


\end{document}
