\section{Related Work}
In this section, we addressed several topics closely related to our research objective, including (1) the role and objectives of PE in K-12 settings, (2) recent developments in AI applications for student learning and teacher support, and (3) the importance of teacher involvement in designing AI-integrated curricula.

\subsection{Physical Education in K-12} 
According to the Society of Health and Physical Educators (SHAPE) in the United States, PE is an ``academic discipline designed to provide students with a planned, sequential, standards-based K-12 program of curricula and instruction to develop motor skills, knowledge, and behaviors for active living, physical fitness, sportsmanship, self-efficacy, and emotional intelligence'' ~\cite{america2015essential}. According to UNESCO's worldwide survey of school PE ~\cite{hardman2014unesco}, PE is typically a required or standard component of the K-12 curriculum in most countries during the compulsory education phase, reflecting the importance of PE in promoting physical activity and health among youth. Despite some regional and national variations, the general global trend indicates that both elementary and secondary schools allocate approximately 100 minutes per week~for~PE~classes.

The core curriculum models of PE encompass three main areas: movement education, which emphasizes basic motor skills as prerequisites for lifelong physical activity; sports education, which helps students become proficient in diverse sports; and fitness education, which promotes personal fitness and health maintenance ~\cite{2013Etsb}. In practice, many PE classes primarily endeavor to immerse students in a variety of physical exercises and sports ~\cite{chen2018toward, castelli2015contextualizing, Bar1995Health}. However, these classes also serve as platforms for honing social skills, thus bolstering social competence ~\cite{pangrazi2003physical, dishman2015motivation}.
Ultimately, PE aims to promote the holistic development of students, encompassing physical, mental, emotional, and cognitive abilities ~\cite{2013Etsb}. This comprehensive approach to student development is globally shared as a core objective in K-12 PE ~\cite{lynch2019physical}, reflecting the multifaceted nature of PE and its importance in overall student growth.

Given the multifaceted objectives of K-12 PE, the scope of content that PE teachers are expected to address in their classes is broad. Usually, nationwide standard curricula provide a guiding framework, but they are not binding. As a result, the overall curriculum structure is often left entirely to the discretion of PE teachers ~\cite{landi2021physical}. Additionally, the nature of PE, rooted in physical activity across large open spaces such as pools, soccer fields, gymnasiums, and outdoor courts, demands both specialized equipment and extended instructional time ~\cite{cothran2014classroom, Bevans2010Physical}. Furthermore, given the heterogeneous skill sets and abilities of many students, stringent control measures are crucial to ensure students' safety ~\cite{chepyator2003pre, o1994rules, siedentop2022introduction}. Consequently, achieving optimal PE outcomes is inherently challenging and depends significantly on PE teachers' classroom management and pedagogical strategies ~\cite{Rimmer1989Confrontation}. Specifically, students' PE learning outcomes largely hinge on their teachers' ability to effectively utilize space and resources and select instructional strategies that fit the classroom context and student characteristics ~\cite{Powell2014Teachers, sieberer2016effective}. Therefore, to address these challenges and support PE teachers in delivering high-quality education, it would be essential to actively involve them in the design and development of technologies, particularly when integrating advanced systems such as AI.



\subsection{AI for Education and Sports} 
Artificial Intelligence in Education (AIED) offers a wide range of opportunities for improving education quality and enhancing learning outcomes ~\cite{crompton2021potential, crompton2020psychological, chen2020artificial}. AIED applications broadly focus on supporting both students and teachers by providing tailored educational experiences and addressing various learning needs.

A large body of research has focused on enhancing student learning. For example, intelligent tutoring systems have been extensively researched, with the goal of providing personalized learning paths based on real-time analysis of learner behavior and performance ~\cite{vanlehn2011relative, ma2014intelligent}. Other notable applications include real-time analysis of student progress ~\cite{mirzaeian2016learning, roschelle2020ai}, and personalized feedback system on student work ~\cite{foster2019barriers, zhu2020effect}.

Simultaneously, AIED provides comprehensive support for instructional tasks, enhancing teachers' capabilities and efficiency. Research on automated summative assessment, for instance, focuses on improving the accuracy and reliability of AI-based automatic scoring systems ~\cite{hsu2021attitudes}. These research areas extend into classroom monitoring and orchestration, investigating how AI can effectively support teacher decision-making in complex classroom environments ~\cite{song2021review, vanlehn2021can, lee2024investigating}.

While AI applications have been extensively integrated into lecture-based classes in STEM disciplines ~\cite{bates2020can, kabudi2021ai}, there have also been some initiatives to employ data mining for sports training and skill analysis ~\cite{pan2019big}, as well as deep learning approaches to assess the quality of training ~\cite{wang2022evaluation} in PE contexts. For example, recent studies have demonstrated the effective use of AI technologies in training dance skills ~\cite{wang2024artificial, feng2022automatic} and rehabilitation ~\cite{jacob2021ai, vourganas2020individualised}. Building on these advancements, AI technologies integrated with tangible user interfaces~\cite{bakker2012embodied}—offer intuitive methods to enhance sports training by combining tactile and visual senses ~\cite{gallud2022use, nacher2015game}. 
% Similarly, embodied user interfaces foster spatial understanding and enhance interaction by connecting visual and verbal information in educational and training contexts ~\cite{skulmowski2016embodied}.

Despite these promising applications, the scope of AI integration into general K-12 PE classes remains limited ~\cite{edtech_ai_k12_2023}. Current implementations often emphasize the development of specific sports skills rather than comprehensively addressing the multifaceted educational goals and the broader context of the PE classroom. To bridge this gap, this paper aims to explore opportunities for integrating AI into PE classes by identifying unique needs and challenges through focus group ideation workshops involving PE teachers.


\subsection{Challenges in Integrating AI into K-12: Importance of Teacher Involvement}
AI is driving educational advancements ~\cite{alberola2016artificial, chakroun2019artificial, park2016teachers} by promoting tailored learning experiences for individual students ~\cite{bernacki2020towards, samani2017bridging} and providing diverse ways to amplify the capabilities of educators ~\cite{ZHANG2021100025, edwards2018not}. Nevertheless, the integration of AI in educational settings presents complex challenges and considerations, given the multitude of stakeholders involved in the educational ecosystem. In particular, issues surrounding the collection, use, and protection of student data have persisted, especially concerning cybersecurity threats  ~\cite{grayson1978education, huang2023ethics}. Thus, students keep demanding transparency and control over their data usage, extending beyond mere information provision to actual data management rights ~\cite{jones2019learning, slade2014student, sun2019s}.

Also, recent AIED research consistently underscores the absence of a pedagogical context ~\cite{Chen2020application, hinojo2019artificial, zawacki2019systematic, tang2023trends}. AIED innovations often remain in the experimental or research phase. As a result, a significant gap exists between the potential capabilities of AIED and its actual implementation in educational settings ~\cite{bates2020can, kabudi2021ai}. Moreover, the successful adoption of AIED technologies depends on their alignment with learning objectives, educators' instructional strategies, and expectations regarding the role of such technologies ~\cite{Ertmer2012Teacher, Mama2013Developing}. Therefore, in order to effectively implement AIED systems, it is imperative to co-construct a ``clear and specific framework for innovation'' ~\cite{roschelle2006co} through the active involvement of key stakeholders such as teachers and reflect their educational practices and beliefs in the design processes ~\cite{Ertmer2012Teacher, Mama2013Developing}.

Therefore, an increasing number of AIED projects involved various stakeholders, such as teachers and students, in the design and development processes of educational technologies. This approach focuses on integrating the tacit knowledge from participants' real-life experiences with the knowledge of the researchers ~\cite{bodker2022participatory, mckercher2022beyond, schuler1993participatory}. Particularly, a recent study highlighted the importance of teachers' perspectives in incorporating AI training and tools into K-12 curriculum ~\cite{Lin2021Engaging}. For example, collaboration with teachers on the development of science education curriculum resulted in the successful implementation of the program ~\cite{durall2019co}. In light of this, this study engaged PE teachers in focus group ideation workshops to envision AI-integrated PE classes by incorporating their needs, challenges, and lived experiences.



\begin{table*}
  \centering
  \caption{Participant Demographic and Background Information}
  \label{tab:participant_info}
  \begin{tabular}{p{1.5cm} p{1cm} p{1.2cm} p{1.7cm} p{4.2cm} p{3cm}}
    \toprule
    \textbf{Group} & \textbf{ID} & \textbf{Gender} & \textbf{Teaching \newline Experience} & 
    \textbf{General Attitude Toward AI \newline(1: negative - 5: positive)} & 
    \textbf{Experience \newline Teaching with AI} \\
    \midrule
    \multirow{2}{*}{Group 1} 
      & P1  & M & 25 & 3.17 & No \\
      & P2  & M & 7  & 3.75 & No \\
    \midrule
    \multirow{2}{*}{Group 2} 
      & P3  & F & 18 & 3.42 & No \\
      & P4  & M & 10 & 4.33 & No \\
    \midrule
    \multirow{3}{*}{Group 3} 
      & P5  & M & 24 & 3.83 & Yes \\
      & P6  & M & 7  & 3.08 & No \\
      & P7  & F & 1  & 3.83 & No \\
    \midrule
    \multirow{3}{*}{Group 4} 
      & P8  & M & 9  & 3.25 & No \\
      & P9  & M & 10 & 3.83 & No \\
      & P10 & M & 11 & 3.75 & No \\
    \midrule
    \multirow{2}{*}{Group 5}
      & P11 & F & 8  & 4.08 & No \\
      & P12 & M & 9  & 3.58 & Yes \\
    \midrule
    \multirow{3}{*}{Group 6} 
      & P13 & M & 5  & 4.33 & Yes \\
      & P14 & M & 9  & 3.25 & No \\
      & P15 & F & 1  & 3.67 & No \\
    \midrule
    \multirow{2}{*}{Group 7} 
      & P16 & M & 5  & 3.25 & No \\
      & P17 & F & 2  & 3.75 & No \\
    \bottomrule
  \end{tabular}
  \Description{This table displays the participant demographic and background information of 17 participants. The data present the participants’ gender, teaching experience, general attitudes toward AI, and experience teaching with AI.}
\end{table*}