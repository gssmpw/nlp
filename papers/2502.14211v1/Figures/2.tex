\begin{figure*}[t]
\noindent\fbox{
\parbox{.98\textwidth}{
\color{teal}{
Your task is to generate the instruction <INS>. Below are some previous instructions with their scores. The score ranges from 0 to 100.

}

\vspace{1em}
\color{darkgreen}{
text:

The following are multiple-choice questions (with answers) about professional medicine. Please choose the correct answer from 'A', 'B', 'C' or 'D'.

score:

61

\vspace{1em}

text:

Below is a multiple-choice question related to professional medicine. Review the question carefully and select the correct answer from the options 'A', 'B', 'C', or 'D'. 

score:

63

\vspace{1em}
(… more instructions and scores …)
\vspace{1em}
}

\color{teal}{
The following exemplars show how to apply your text: you replace <INS> in each input with your text, then read the input and give an output. We say your output is wrong if your output is different from the given output, and we say your output is correct if they are the same.
}
\color{brown}{
\vspace{1em}

input:

Q: A 13-month-old child is brought to the emergency department because of urticaria, swelling of the lips, and difficulty breathing immediately after eating an egg. A potential risk for hypersensitivity reaction is posed by vaccination against which of the following illnesses?
A: Hepatitis
B: Influenza
C: Pertussis
D: Poliomyelitis

A: <INS>

output:

B

\vspace{1em}
(… more exemplars …)
\vspace{1em}
}

\color{teal}{
Generate an instruction that is different from all the instructions <INS> above, and has a higher score than all the instructions <INS> above. The instruction should begin with <INS> and end with </INS>. The instruction should be concise, effective, and generally applicable to all problems above.}
}
}
\caption{Illustration of prompt optimization using instruction-tuned GPT-4 in MMLU professional medicine. The generated answer is inserted at the beginning of ``A:'' in the scorer LLM output, marked by <INS>. The \textcolor{darkgreen}{green} text shows case-score pairs, the \textcolor{brown}{brown} text outlines the optimization task and output format, and the \textcolor{teal}{teal} text provides reference-instructions.
}
\label{fig:reference_prompt_example_GPT-4}
\end{figure*}
