
% \begin{figure*}[htb!]
% \noindent\fbox{
% \parbox{.98\textwidth}{
% \color{teal}{
% I have some texts along with their corresponding scores. The texts are arranged in ascending order based on their scores, where higher scores indicate better quality.

% }

% \vspace{1em}
% \color{darkgreen}{
% text:

% The following are multiple-choice questions (with answers) about professional medicine. Please choose the correct answer from 'A', 'B', 'C' or 'D'.

% score:

% 61

% \vspace{1em}

% text:

% Below is a multiple-choice question related to professional medicine. Review the question carefully and select the correct answer from the options 'A', 'B', 'C', or 'D'. 

% score:

% 63

% \vspace{1em}
% (… more instructions and scores …)
% \vspace{1em}
% }

% \color{teal}{
% The following exemplars show how to apply your text: you replace <INS> in each input with your text, then read the input and give an output. We say your output is wrong if it is different from the given output, and we say your output is correct if they are the same.
% }
% \color{brown}{
% \vspace{1em}

% input:

% Q: A 13-month-old child is brought to the emergency department because of urticaria, swelling of the lips, and difficulty breathing immediately after eating an egg. A potential risk for hypersensitivity reaction is posed by vaccination against which of the following illnesses?
% A: Hepatitis
% B: Influenza
% C: Pertussis
% D: Poliomyelitis

% A: <INS>

% output:

% B

% \vspace{1em}
% (… more exemplars …)
% \vspace{1em}
% }

% \color{teal}{
% Write your new text that is different from the old ones and has a high score. Write the text in square brackets.}
% }
% }
% \caption{Illustration of prompt optimization using instruction-tuned Palm2 Text-bison in MMLU professional medicine. The generated answer is inserted at the beginning of ``A:'' in the scorer LLM output, marked by <INS>. The \textcolor{darkgreen}{green} text shows case-score pairs, the \textcolor{brown}{brown} text outlines the optimization task and output format, and the \textcolor{teal}{teal} text provides reference-instructions.
% }
% \label{fig:reference_prompt_example}
% \end{figure*}

% \begin{figure*}[htb!]
% \centering
% \begin{tcolorbox}[width=0.98\textwidth, colframe=black, boxrule=0.5pt, arc=0mm, colback=white,
%   left=4pt, right=4pt, top=4pt, bottom=4pt]
% \small 

% {\color{teal}
% I have some prompts along with their corresponding scores. The prompts are arranged in ascending order based on their scores, where higher scores indicate better quality.
% }

% \medskip

% {\color{darkgreen}
% \textbf{Prompt 1:}

% The following are multiple-choice questions about professional medicine. Please choose the correct answer from 'A', 'B', 'C', or 'D'.

% \textbf{Score:} 61

% \medskip

% \textbf{Prompt 2:}

% Below is a multiple-choice question related to professional medicine. Review the question carefully and select the correct answer from the options 'A', 'B', 'C', or 'D'.

% \textbf{Score:} 63

% \medskip

% \textbf{Prompt 3:}

% You will be presented with a professional medical multiple-choice question. Please read carefully and choose the correct option from 'A', 'B', 'C', or 'D'.

% \textbf{Score:} 65

% \medskip

% (… more prompts and scores …)
% }

% \medskip

% {\color{teal}
% The following exemplars show how to apply your prompt: you replace \texttt{<INS>} in each input with your prompt, then read the input and provide the correct answer. We say your output is correct if it matches the given output, and incorrect if it does not.

% Please write a new prompt that is different from the previous ones and aims to achieve a higher score. Write your new prompt in square brackets [ ].
% }

% \medskip

% {\color{brown}
% \textbf{Input:}

% \textbf{A:} \texttt{<INS>}

% Q: A 13-month-old child is brought to the emergency department because of urticaria, swelling of the lips, and difficulty breathing immediately after eating an egg. A potential risk for hypersensitivity reaction is posed by vaccination against which of the following illnesses?
% A: Hepatitis
% B: Influenza
% C: Pertussis
% D: Poliomyelitis



% \medskip

% \textbf{Expected Output:}

% B

% \medskip

% (… more examples …)
% }
% \end{tcolorbox}
% \caption{An illustration of the reference-prompt used in the POI method. The \textcolor{darkgreen}{green} text shows previously generated prompts and their scores, the \textcolor{brown}{brown} text provides example inputs and expected outputs for the optimization task, and the \textcolor{teal}{teal} text contains instructions for generating a new prompt.
% }
% \label{fig:reference_prompt_example}
% \end{figure*}


\begin{figure*}[t] \noindent\fbox{ \parbox{\textwidth}{ \color{teal}{ I have some instructions along with their corresponding scores. The instructions are arranged in ascending order based on their scores, where higher scores indicate better quality. }

\vspace{1em} \color{darkgreen}{ text:

The following are multiple choice questions (with answers) about professional medicine. Please choose the correct answer from "A", "B", "C", or "D".

score:

61

\vspace{1em}

text:

Below is a multiple-choice question related to professional medicine. Review the question carefully and select the correct answer from the options "A", "B", "C", or "D".

score:

63

\vspace{1em} (… more instructions and scores …) \vspace{1em} }

\color{teal}{ The following examples demonstrate how to apply your instruction: you replace <INS> in each input with your instruction, then read the input and provide an output. Your output is considered correct if it matches the given output. }

\color{brown}{ \vspace{1em}

input:

Q: A 13-month-old child is brought to the emergency department because of urticaria, swelling of the lips, and difficulty breathing immediately after eating an egg. A potential risk for hypersensitivity reaction is posed by vaccination against which of the following illnesses?
A: Hepatitis
B: Influenza
C: Pertussis
D: Poliomyelitis

A: <INS>

output:

B

\vspace{1em} (… more examples …) \vspace{1em} }

\color{teal}{ Please write a new instruction that is different from the ones given and aims for the highest possible score. Write your instruction in square brackets. } } } \caption{\small An example of the meta-prompt for prompt optimization using instruction-tuned models on the MMLU professional medicine dataset. The generated instruction is inserted at the position marked by <INS> in the input. The \textcolor{darkgreen}{dark green} text displays instruction-score pairs; the \textcolor{brown}{brown} text provides examples of how to apply the instruction; the \textcolor{teal}{teal} text contains the meta-instructions.} \label{fig
} \end{figure*}


% \begin{figure*}[htb!]
% \centering
% \begin{tcolorbox}[width=0.98\textwidth, colframe=black, boxrule=0.5pt, arc=0mm, colback=white,
%   left=4pt, right=4pt, top=4pt, bottom=4pt]
% \small % 使用较小的字体大小
% {\color{teal}
% I have some texts along with their corresponding scores. The texts are arranged in ascending order based on their scores, where higher scores indicate better quality.
% }

% \medskip

% {\color{darkgreen}
% \textbf{prompt:} The following are multiple-choice questions (with answers) about professional medicine. Please choose the correct answer from 'A', 'B', 'C' or 'D'.

% \textbf{score:} 61

% \medskip

% \textbf{prompt:} Below is a multiple-choice question related to professional medicine. Review the question carefully and select the correct answer from the options 'A', 'B', 'C', or 'D'.

% \textbf{score:} 63

% \medskip

% (… more instructions and scores …)
% }

% \medskip

% {\color{teal}
% The following exemplars show how to apply your text: you replace \texttt{<INS>} in each input with your text, then read the input and give an output. We say your output is wrong if it is different from the given output, and we say your output is correct if they are the same.
% }

% \medskip

% {\color{brown}
% \textbf{Input:}

% Q: A 13-month-old child is brought to the emergency department because of urticaria, swelling of the lips, and difficulty breathing immediately after eating an egg. A potential risk for hypersensitivity reaction is posed by vaccination against which of the following illnesses?
% A: Hepatitis
% B: Influenza
% C: Pertussis
% D: Poliomyelitis

% \textbf{A:} \texttt{<INS>}

% \textbf{Output:} B

% (… more exemplars …)
% }

% \medskip

% {\color{teal}
% Write your new text that is different from the old ones and has a high score. Write the text in square brackets.
% }
% \end{tcolorbox}
% \caption{Illustration of prompt optimization using instruction-tuned Palm2 Text-bison in MMLU professional medicine. The generated answer is inserted at the beginning of ``A:'' in the scorer LLM output, marked by \texttt{<INS>}. The \textcolor{darkgreen}{green} text shows case-score pairs, the \textcolor{brown}{brown} text outlines the optimization task and output format, and the \textcolor{teal}{teal} text provides reference instructions.
% }
% \label{fig:reference_prompt_example}
% \end{figure*}



% \begin{figure*}[htb!]
% \centering
% \begin{tcolorbox}[width=0.98\textwidth, colframe=black, boxrule=0.5pt, arc=0mm, colback=white]
% {\color{teal}
% I have some texts along with their corresponding scores. The texts are arranged in ascending order based on their scores, where higher scores indicate better quality.
% }

% \medskip

% {\color{darkgreen}
% \textbf{text:}

% The following are multiple-choice questions (with answers) about professional medicine. Please choose the correct answer from 'A', 'B', 'C' or 'D'.

% \textbf{score:}

% 61

% \medskip

% \textbf{text:}

% Below is a multiple-choice question related to professional medicine. Review the question carefully and select the correct answer from the options 'A', 'B', 'C', or 'D'. 

% \textbf{score:}

% 63

% \medskip

% (… more instructions and scores …)
% }

% \medskip

% {\color{teal}
% The following exemplars show how to apply your text: you replace \texttt{<INS>} in each input with your text, then read the input and give an output. We say your output is wrong if it is different from the given output, and we say your output is correct if they are the same.
% }

% \medskip

% {\color{brown}
% \textbf{input:}

% Q: A 13-month-old child is brought to the emergency department because of urticaria, swelling of the lips, and difficulty breathing immediately after eating an egg. A potential risk for hypersensitivity reaction is posed by vaccination against which of the following illnesses?

% A: Hepatitis

% B: Influenza

% C: Pertussis

% D: Poliomyelitis

% \textbf{A:} \texttt{<INS>}

% \textbf{output:}

% B

% \medskip

% (… more exemplars …)
% }

% \medskip

% {\color{teal}
% Write your new text that is different from the old ones and has a high score. Write the text in square brackets.
% }
% \end{tcolorbox}
% \caption{Illustration of prompt optimization using instruction-tuned Palm2 Text-bison in MMLU professional medicine. The generated answer is inserted at the beginning of ``A:'' in the scorer LLM output, marked by \texttt{<INS>}. The \textcolor{darkgreen}{green} text shows case-score pairs, the \textcolor{brown}{brown} text outlines the optimization task and output format, and the \textcolor{teal}{teal} text provides reference instructions.
% }
% \label{fig:reference_prompt_example}
% \end{figure*}
