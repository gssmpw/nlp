%%%% ijcai25.tex

\typeout{IJCAI--25 Instructions for Authors}

% These are the instructions for authors for IJCAI-25.

\documentclass{article}
\pdfpagewidth=8.5in
\pdfpageheight=11in

% The file ijcai25.sty is a copy from ijcai22.sty
% The file ijcai22.sty is NOT the same as previous years'
\usepackage{ijcai25}

% Use the postscript times font!
\usepackage{times}
\usepackage{soul}
\usepackage{url}
\usepackage[hidelinks]{hyperref}
\usepackage[utf8]{inputenc}
\usepackage[small]{caption}
\usepackage{graphicx}
\usepackage{amsmath}
\usepackage{amsthm}
\usepackage{booktabs}
% \usepackage{algorithm}
% \usepackage{algorithmic}
\usepackage[switch]{lineno}

\usepackage{multirow}
\usepackage{booktabs}
\usepackage{colortbl}
\usepackage{array}


\usepackage{booktabs} % 负责处理表格线条
\usepackage{multirow} % 负责表格的多行合并

\usepackage{subcaption}
% Define colors
\usepackage[ruled,vlined]{algorithm2e}


\usepackage{amsmath}
\usepackage{array}
\usepackage{booktabs}
\definecolor{lightgray}{gray}{0.9}
\definecolor{darkgray}{gray}{0.7}
\usepackage{xcolor}
\definecolor{burntorange}{RGB}{204, 85, 0} % Example RGB values

% Standard package includes
\usepackage{booktabs}
\usepackage{multirow}
\usepackage{caption}
\definecolor{darkgreen}{rgb}{0.0, 0.5, 0.0}
% \usepackage{algorithm}
\usepackage{algpseudocode}
\usepackage{algpseudocode}
\usepackage{adjustbox}

\usepackage{times}
\usepackage{bm}

\usepackage{amsmath} 
\usepackage{latexsym}
\usepackage{amsfonts}
\usepackage{amssymb}

\usepackage{xcolor}
% \usepackage[ruled,vlined]{algorithm2e}
\usepackage{tabularx}

\usepackage{xspace}
\usepackage{booktabs}
\usepackage{multirow}
\usepackage{graphicx}
\usepackage{booktabs}
\usepackage{colortbl}
\usepackage{xcolor}
\usepackage{booktabs}
\usepackage{multirow}
\usepackage{colortbl}
\usepackage{amsmath} % for math commands like \sum
 \usepackage{cleveref}
% \usepackage{hyperref}
\usepackage{multirow}
\usepackage{siunitx}
\usepackage{adjustbox}
\usepackage{xcolor}
\usepackage{multirow} 
\usepackage{colortbl}
\usepackage{booktabs}
\usepackage[T1]{fontenc}
\usepackage[utf8]{inputenc}
\usepackage{microtype}
\usepackage{inconsolata}
\usepackage{tikz}
\usepackage[utf8]{inputenc}  % 添加对 UTF-8 编码的支持
\usepackage{subcaption} 

\usepackage{amsmath}
\usepackage{algpseudocode}
\usepackage{dsfont}
\usepackage{amsmath}
\usepackage{amsfonts}
\usepackage{etoc}
\usepackage{tcolorbox}

\usepackage{dblfloatfix}

\usepackage{booktabs} % 添加此行以使用 \toprule, \midrule, \bottomrule 等命令
\usepackage{amsmath} % 用于数学符号
% \usepackage{algorithm}


\newcommand{\llms}{LLMs\xspace}
\usepackage{booktabs}

\usepackage{xcolor}
\usepackage{tcolorbox}
\usepackage{todonotes}
\usepackage{natbib} 

\definecolor{lightblackblue}{RGB}{115,161,216} % 淡黑蓝
\definecolor{lightblackgreen}{RGB}{59,125,35} % 淡黑绿
\definecolor{lightblackorange}{RGB}{233,113,50} % 淡黑橙



\usepackage{hyperref}

\definecolor{teal}{RGB}{0,128,128}
\definecolor{darkgreen}{RGB}{0,100,0}
\definecolor{brown}{RGB}{165,42,42}

% Comment out this line in the camera-ready submission
%\linenumbers

\urlstyle{same}

% the following package is optional:
%\usepackage{latexsym}

% See https://www.overleaf.com/learn/theorems_and_proofs
% for a nice explanation of how to define new theorems, but keep
% in mind that the amsthm package is already included in this
% template and that you must *not* alter the styling.
\newtheorem{example}{Example}
\newtheorem{theorem}{Theorem}

% Following comment is from ijcai97-submit.tex:
% The preparation of these files was supported by Schlumberger Palo Alto
% Research, AT\&T Bell Laboratories, and Morgan Kaufmann Publishers.
% Shirley Jowell, of Morgan Kaufmann Publishers, and Peter F.
% Patel-Schneider, of AT\&T Bell Laboratories collaborated on their
% preparation.

% These instructions can be modified and used in other conferences as long
% as credit to the authors and supporting agencies is retained, this notice
% is not changed, and further modification or reuse is not restricted.
% Neither Shirley Jowell nor Peter F. Patel-Schneider can be listed as
% contacts for providing assistance without their prior permission.

% To use for other conferences, change references to files and the
% conference appropriate and use other authors, contacts, publishers, and
% organizations.
% Also change the deadline and address for returning papers and the length and
% page charge instructions.
% Put where the files are available in the appropriate places.


% PDF Info Is REQUIRED.

% Please leave this \pdfinfo block untouched both for the submission and
% Camera Ready Copy. Do not include Title and Author information in the pdfinfo section
\pdfinfo{
/TemplateVersion (IJCAI.2025.0)
}

\title{Transfer-Prompting: Enhancing Cross-Task Adaptation in Large Language Models via Dual-Stage Prompts Optimization}

% Single author syntax
%\author{
%    Anonymous submission
    % \affiliations
    % Affiliation
    % \emails
    % email@example.com
%}

% Multiple author syntax (remove the single-author syntax above and the \iffalse ... \fi here)
%\iffalse
\author{
Yupeng Chang$^1$
\and
Yi Chang$^1$$^2$$^3$\and
Yuan Wu$^1$\footnote{Corresponding author}\\
\affiliations
$^1$School of Artificial intelligence, Jilin University\\
$^2$Engineering Research Center of Knowledge-Driven Human-Machine Intelligence, Jilin University\\
$^3$International Center of Future Science, Jilin University\\
\emails
yuanwu@jlu.edu.cn
}
%\fi

%\includeonly{content}  % 只编译正文
\includeonly{6_appendix}  % 只编译附录,生成附录部分的 PDF

\begin{document}
\etocdepthtag.toc{chapter}
\etocsettagdepth{chapter}{none}
\etocsettagdepth{appendix}{none}

\maketitle


\begin{abstract}
Large Language Models (LLMs) face significant challenges when balancing multiple high-level objectives, such as generating coherent, relevant, and high-quality responses while maintaining efficient task adaptation across diverse tasks. 
To address these challenges, we introduce \textbf{Transfer-Prompting}, a novel two-stage framework designed to enhance cross-task adaptation in prompt generation. The framework comprises two key components: (1) \textit{source prompt construction}, which refines the original prompts on source task datasets to generate source prompts with enhanced generalization ability, and (2) \textit{target prompt generation}, which enhances cross-task adaptation of target prompts by fine-tuning a set of high-scored source prompts on task-specific datasets.
In each optimization cycle, a reference LLM generates candidate prompts based on historical prompt-score pairs and task descriptions in our designed reference prompt. These candidate prompts are refined iteratively, while a scorer LLM evaluates their effectiveness using the multi-dimensional metrics designed in the objective prompts evaluator—a novel contribution in this work that provides a holistic evaluation of prompt quality and task performance. This feedback loop facilitates continuous refinement, optimizing both prompt quality and task-specific outcomes.
We validate Transfer-Prompting through extensive experiments across 25 LLMs, including 7 foundation models and 18 specialized models, evaluated on 9 diverse datasets. The results demonstrate that Transfer-Prompting significantly improves task-specific performance, highlighting its potential for enhancing cross-task adaptation in LLMs.
The code is available at \url{https://github.com/llm172/Transfer-Prompting}.
% For instance, on the LogiQA dataset, instruction-following accuracy increased by 17.1\%, and calibration error was reduced by 40\%. In specialized domains like law, the ChatLaw-13B model achieved a 41.5\% increase in instruction-following accuracy and a 42.9\% reduction in calibration error on the AGI-Eval dataset.
\end{abstract}


% We conducted comprehensive experiments evaluating 7 general-purpose models (GPT-3.5-Turbo and GPT-4) and 18 specialized models across medical, legal, and financial domains. The evaluation encompassed three heterogeneous reasoning datasets and nine cross-domain datasets.
% Results demonstrate that our method significantly enhances LLMs' instruction-following rate and overall output quality on targeted domain-specific tasks.

% \begin{abstract}
% Large language models (LLMs) face a trade-off between instruction-following and output quality in practical applications. For example, in complex tasks, the need to simultaneously accomplish multiple subtasks or meet diverse constraints may lead to a decline in accuracy and even hallucinated outputs.
% To address this issue, we propose the \textbf{Transfer-Prompting}, a dual-stage framework for optimizing prompts.
% It primarily comprises two key modules: (1) \textit{source prompt construction}, aimed at enhancing the generalization ability of prompts, and (2) \textit{target prompt generation}, designed to improve the instruction-following rate and output quality of the model for specific tasks.
% For each optimization round, a reference LLM generates new prompts based on historical prompts and task descriptions, refining them iteratively. Meanwhile, a scoring LLM evaluates the effectiveness of the latest prompts, with feedback integrated into subsequent optimizations.
% The experiments comprehensively evaluated six general-purpose models (including GPT-3.5-Turbo and GPT-4) and 36 specialized models across medical, legal, and financial domains. The evaluation tasks spanned three heterogeneous reasoning datasets and nine cross-domain datasets. 
% The results show that our method significantly improves the instruction following rate and overall output quality of LLM on the specified domain tasks.
% \end{abstract}
% }
% Experimental results demonstrate that our method significantly enhanced the instruction-following rate and improved LLMs' overall output quality.


\section{Introduction}\label{sec:intro}

In computational finance, Monte Carlo simulations are used extensively to estimate the expected value of financial payoffs based on the solution of stochastic differential equations (SDEs) which model the evolution of stock prices, interest rates, exchange rates and other quantities \cite{glasserman04}.  Monte Carlo methods are very general and flexible, but for high accuracy it requires generating a large number of costly SDE path approximations, which has motivated research into a number of variance reduction or, equivalently, cost reduction techniques. One such method is
Multilevel Monte Carlo (MLMC), which was proposed in \cite{GILES2008} and was adapted for various applications that are summarised in \cite{Giles_overview17} and successfully combined with other methods such as quasi-Monte Carlo methods. The main idea of MLMC is to approximate the payoff using different time stepping resolutions when numerically solving the underlying SDE and to generate an optimal number of samples on each level, such that the overall computational cost is minimised subject to the desired bound on the variance. %, such that the total computational cost is minimised. 
The computational savings come from the fact that most samples are computed on the coarser levels and hence are less expensive while only a few samples from the finest levels are required \cite{GILES2008}.


Among the directions in which the computational cost 
of MLMC methods could further be reduced, an important avenue is the use of lower precision calculations, especially for the first Monte Carlo levels where the targeted accuracy is relatively low. 
 An overview of the research on mixed precision for the standard Monte Carlo (MC) framework is provided in \cite{ChowMixedPrecisionStandardMC} but only a few references study the potential of low precision computation in the MLMC framework \cite{Rounding_error_oliver}. To the best of our knowledge, the only MLMC framework with customised precision in the literature is \cite{brugger2014mixed}, but they use a uniform precision for all operations on each Monte Carlo level instead of optimising 
 the precision of each intermediary variable to reduce as much as possible the cost of path generation.
 
An important motivation for an MLMC framework with variable precision would be performing the low precision computations on reconfigurable hardware devices such as Field Programmable Gate Arrays (FPGAs). FPGAs contain customizable logic blocks and connectors that make it easy to adapt the digital circuit architecture for a specific application, leading to a highly parallel and optimised implementation. Therefore they are successfully exploited in applications that require high speed and have high computational workload, such as signal processing \cite{woods2008fpga}, and real time applications like high frequency trading \cite{HFT1,HFT2}. That is why a number of previous works in hardware architecture design implemented the MLMC algorithm to price financial options using FPGAs as accelerators, which resulted in improved speed and power efficiency compared to full CPU architectures \cite{Schryver2013AMM}. The paper \cite{lindsey2016domain} also proposed 
a Domain Specific Language to automate the configuration of FPGAs for this specific application. However, only \cite{brugger2014mixed} proposed a heuristic to reduce the precision in calculations.

In addition, all aforementioned works considered that the random number generation (RNG) is performed in single or double precision. Yet in most cases an important portion of the workload in the overall MLMC simulation comes from the RNG and in \cite{brugger2014mixed} this limited the total computational savings.
To reduce the cost of MLMC simulations in particular those based on the Geometric Brownian Motion (GBM), \cite{approximateICDF_Oliver, NestedOliver} have proposed to use approximate random numbers that are generated by applying an approximation of the inverse CDF to uniform random numbers. In \cite{NestedOliver}, the authors proposed a way to integrate these lower precision random variables into a \textit{nested} MLMC framework and completed a numerical analysis to bound the resulting error at each MC level by a product of the time step and the error in the random number approximation. The same authors show in \cite{approximateICDF_Oliver} that using approximate random variables reduces the cost of path generation by a factor 7.


In this paper we propose a nested MLMC framework that combines the use of approximate random normal variables and lower precision calculations to reduce the computational cost of MLMC even further than \cite{brugger2014mixed,NestedOliver}. We illustrate the efficiency of our framework in Matlab, after making several assumptions on the cost of operations and size of the errors that we carefully justify. We focus on the case of GBM and use the approximate RNG methods presented in \cite{approximateICDF_Oliver} as well as a new slightly modified method that combines CDF inversion and the central limit theorem. To choose the precision of the variables in the low precision path generation, we introduce a novel method to optimise the bit-widths. This optimisation is performed before the main path generation loop is executed and is based on a linear model of the payoff error  
due to rounding when computing in low precision. The error model relies on algorithmic differentiation in a similar manner to \cite{unifying-bwoptim,bitwidth-AD,ADAPT}. The bit-width optimisation procedure can be performed off-line, so this stage can be excluded from the on-line time complexity of our framework. The user specified desired accuracy is then enforced by calculating on-line the number of samples that need to be generated.

In terms of hardware design, we suggest implementing the low precision path generation on FPGAs and the full-precision ones on a CPU or GPU. 
The FPGA offers enough flexibility to define a separate bit-width for every variable in the low precision path generation, and can be reconfigured periodically to update the bit-widths when the market parameters have changed considerably. 


The paper is organized as follows : \Cref{sec:MLMC} introduces MLMC and nested MLMC to make clear the estimator that is implemented in our framework. Then in \Cref{sec:RNG} we detail the methods that could be used to obtain approximate random normally distributed numbers very cheaply for the low precision path generation. In \Cref{sec:error_model} and \Cref{sec:costModel} we propose an error model and a cost model (resp.) that we then use to formulate the optimisation problem that is solved to obtain the optimal bit-widths of fixed point variables in \Cref{sec:optimisation}. Finally we summarise our results and future directions in \Cref{sec:conclusion}.





\section{Related Work}

\subsection{First-order logic for natural entailment}

Since the start of the RTE challenge \citep{rte}, multiple works have attempted using FOL representations to solve natural language entailment. These methods first obtain the syntactic/semantic parse tree and apply a rule-based transformation to get the FOL representation \citep{bos-markert-2005-recognising, bos-nli}. However, it was repeatedly shown that these FOL representations are not empirically effective in solving natural language entailment. For instance, \citet{bos-nli} reported that FOL representations translated from the discourse representation structure (DRS) yield only 1.9\% recall in detecting the entailment in the single-premise RTE benchmark \citep{rte}.

Independently from these works, multi-premise logical entailment benchmarks \citep{tafjord-etal-2021-proofwriter, logicnli, folio} were developed to evaluate the reasoning ability of generative models. These benchmarks adopt the classic 3-way entailment label classification format (\textit{entailment, contradiction, neutral}) of single-premise RTE tasks, in which both the NL sentences and their gold FOL representations point to the same entailment label. 

Recent works have applied LLMs to obtain FOL representations for these multi-premise logical entailment tasks \citep{logiclm, linc, divide-and-translate}, fueled by the code generation ability of LLMs. While they achieve significant performance in synthetic, controlled logical reasoning benchmarks, whether they can generalize to natural entailment has remained unanswered. Furthermore, \citet{linc} observed that LLMs are highly susceptible to \textit{arbitrariness}, as they fail to produce coherent predicate names or numbers of arguments even when generating FOL representations of premises and hypotheses in a single inference.

\subsection{Executable semantic representations}

Apart from FOL, a stream of research focuses on the \textit{executability} of semantic representations. From this perspective, semantic representations are \textit{program codes} that can be executed to solve downstream tasks, such as query intent analysis \citep{spider, dligach-etal-2022-exploring} and question answering \citep{semparse-qa}. The performance of the semantic parser is directly assessed by the accuracy of execution results for the downstream tasks, rather than the similarity between the prediction and the reference parse.

To improve the execution accuracy that is often non-differentiable, reinforcement learning (RL) and its variants have been applied to train neural semantic parsers \citep{cheng-etal-2019-learning, cheng-lapata-2018-weakly}. Using only the input sentence and the desired execution result, these methods learn to maximize the probability of the representations that lead to the correct execution result. However, these approaches are not directly applicable to EPF, as EPF requires taking account of \textit{interactions between premises and hypotheses} during execution (\textit{i.e.} theorem proving) while these methods assume that sentences are isolated.




% introduce PDDL domains
% why Gripper env as testing context
% motivation: comparing classical vs LLM planners
% - classical: PDDL solver fast-downward
% - LLM: gpt-4o
% explanation and refinement are two distinguishing features of LLM planners
% - how we demonstrate explanation and refinement in the study
We evaluate user trust in two planners over a set of planning problems and study the potential factors influencing user trust in the planners. In particular, we compare a language-model-based planner, denoted as an \emph{LLM Planner}, with a traditional graph-search-based planner, denoted as a \emph{PDDL Solver}. The PDDL Solver uses Fast Downwards \cite{fastdownward} as its underlying model, processing planning problems described in PDDL to generate an optimal solution. In comparison, the LLM Planner employs GPT-4o to interpret the planning problem and extract a solution generated by the language model. Unlike the PDDL Solver, the LLM Planner can reason through the planning problem, explain its proposed solution, and iteratively refine the solution based on external feedback. This study investigates how the correctness of solutions, the quality of explanations, and the refinement process influence user trust.

\subsection{Planning Problem}
% \begin{wrapfigure}{r}{0.4\textwidth}
% % \begin{figure}[t]
%     \centering
%     \includegraphics[width=\linewidth]{figures/problem-example.pdf}
%     \caption{A running example of a planning problem in our study.}
%     \Description{Planning Problem Example}
%     \label{fig: problem-example}
% % \end{figure}
% \end{wrapfigure}

We describe each planning problem in the \emph{Planning Domain Definition Language (PDDL)} and propose two planners to generate plans that solve the problem. We select the \emph{gripper} planning problems from the International Planning Competition \cite{IPC} for plan generation and evaluation. In a gripper planning problem, a robot moves balls between a set of rooms using two grippers. The objective is to create a plan for the robot to move the balls to the target rooms we defined. We present a few running examples of the gripper problem in Figure \ref{fig: correctness}.

A planning problem consists of a \emph{planning domain} and a \emph{problem description}, expressed in PDDL. 

\paragraph{Planning Domain}
A planning domain refers to the universal aspects of a problem that remains consistent across different instances of the problem. In particular, it defines the types of objects, predicates, and actions that exist in the planning problem. We present an example of the gripper problem in Appendix \ref{app: grippers}.

\paragraph{Problem Description} A problem description specifies the particular instance of a planning task within a given domain. It includes the planning domain to which it pertains, a set of objects, the initial state of these objects, and the goal state to be achieved.

\paragraph{Plan}
A plan is a sequence of actions with specific input parameters. Recall that an action corresponds to a state transition. If a plan (a sequence of actions) transits from the initial state to the goal state defined by a problem, then we consider the plan to be \emph{correct}. If a plan does not transit to the goal state or there exists an action violating its precondition, then the plan is \emph{wrong}.

\begin{figure}[t]
    \centering
    \includegraphics[width=0.8\linewidth]{figures/correct.jpeg}
    \caption{Examples where LLM Planner correctly generates a plan for the gripper planning problem.}
    \Description{Planning Problem Correctness}
    \label{fig: correct}
\end{figure}

\subsection{PDDL Solver}
The PDDL Solver takes the planning domain and the problem description as inputs and then generates a plan described in PDDL. 
% It generates a plan in the following format:
% \vspace{4pt}
% \begin{lstlisting}[language=completion]
% (move robot1 room1 room3)
% (pick robot1 ball2 room3 rgripper1)
% (move robot1 room3 room2) ......
% \end{lstlisting}
Next, we convert the generated plan into natural language for user studies following the procedure in \cite{seipp-et-al-zenodo2022} and display it to users. We present an example in Figure \ref{fig: correct}.

The PDDL Solver applies a graph search algorithm to find a path (i.e., a list of transitions) from the initial state to the goal state. It either generates a \emph{correct} plan---defined as the shortest path between the initial and goal states---or returns a signal indicating that no solution exists for the given problem.

\subsection{LLM Planner}

The LLM Planner addresses planning problems by querying a large language model. In particular, it transmits the planning domain and problem description to the language model using a structured prompt format. The planner then retrieves a natural language plan from the language model. We use GPT-4o as the language model for the planner. To ensure the output adheres to the desired format, we include a few in-context examples within the prompts.

A language model solves a planning problem by interpreting the domain and problem descriptions, simulating state transitions, and generating a sequence of actions to achieve the goal. While effective for reasoning and plan generation, language models may struggle with large state spaces. Unlike the PDDL Solver, the LLM Planner may generate \emph{incorrect} plans that violate the problem specifications (e.g., preconditions of actions) or fail to achieve the goal.

\subsection{Explanation and Refinement}
Alongside the generated plans, we offer detailed explanations of all the plans and revisions of any incorrect plans. This study examines how these explanations and refinements influence human trust in the two planners.

\paragraph{LLM Planner with Explanation (LLM+Expl)}
For each generated plan, we manually provide a natural language explanation. This explanation includes an assessment of the plan’s correctness, identification of any violations of action preconditions, and an analysis of inconsistencies between the final state achieved and the intended goal state. We present examples of explanations in Figure \ref{fig: explain} in Appendix.

In particular, if a plan is correct, the explanation is simply ``the plan successfully satisfies the goal conditions.'' 
If a plan is incorrect, we identify the underlying cause as either a violation of action preconditions or a failure to achieve the goal state. In cases involving precondition violations, we specify the action responsible for the issue. For example, consider the action ``robot moves from room 1 to room 2,'' but the robot is initially located in room 3. This scenario constitutes a violation of the precondition for the ``move'' action. In the latter case, we describe the differences between the final state achieved and the intended goal state, e.g., ``fail to move ball 2 to room 2.''

% \begin{wrapfigure}{r}{0.5\textwidth}
%     \centering
%     \includegraphics[width=0.98\linewidth]{figures/refine.jpeg}
%     \includegraphics[width=0.98\linewidth]{figures/refine-correct.jpeg}
%     \includegraphics[width=0.98\linewidth]{figures/refine-wrong.jpeg}
%     \caption{Plan refinement by the LLM Planner. The top row presents two choices of plan refinement (where the refinement starts). The second and third row shows the refinement outcomes of the two choices, where the second row shows a correctly refined plan and the third row shows an incorrect plan.}
%     \Description{Refinement}
%     \label{fig: refine}
% \end{wrapfigure}

\paragraph{LLM Planner with Refinement (LLM+Refine)}
Note that a plan generated by the LLM Planner could be incorrect. Therefore, we offer a prompting mechanism for the LLM Planner to refine the generated plan according to the user feedback. The mechanism works as follows:

1. Request the user to indicate the step number of the first action in the plan that is incorrect, such as the step where an action’s precondition is violated. We present a sample user interface on the left of Figure \ref{fig: refine} in Appendix.

2. Send the planning domain, problem description, and the original plan to the language model. Then, query the model to rewrite the subsequent steps starting from the user-specified step number. We present a sample input prompt in Figure \ref{fig: refine-prompt} in the Appendix.

3. Replace the original plan with the newly refined plan and display it to the user.

This mechanism allows users to interact with the language model to refine the plan. It enables the language model to focus on a subset of steps, facilitating a deeper interpretation of the incorrect component. However, the correctness of the refined plan is not guaranteed. Figure \ref{fig: refine} in the Appendix shows an example of a correctly refined plan and an incorrectly refined plan.



% \section{Experiments} \label{sec:4}
\section{Experimental Setup}
\label{sec:4}

\begin{table*}[!th]
\centering
\resizebox{\textwidth}{!}{%
\begin{tabular}{@{}llcccccccccc@{}}
\toprule
& & \multicolumn{2}{c}{\textbf{Intent Detection}} & \multicolumn{2}{c}{\textbf{Topic Mining}} & \multicolumn{2}{c}{\textbf{Domain Discovery}} & \multicolumn{1}{c}{\textbf{Type}} & \multicolumn{1}{c}{\textbf{Emotion}} & \\
\cmidrule(lr){3-4} \cmidrule(lr){5-6} \cmidrule(lr){7-8} \cmidrule(lr){9-9} \cmidrule(lr){10-10}  %\cmidrule(lr){11-11}
\textbf{Model} & \textbf{Method} & \textbf{BANKING} & \textbf{CLINC} & \textbf{Reddit} & \textbf{StackEx} & \textbf{MTOP} & \textbf{CLINC(D)} & \textbf{FewEvent} & \textbf{GoEmotion} & \textbf{AVG} \\ \midrule \midrule
GPT-4o-mini & Standard Prompting & 0.652 & 0.792 & 0.534 & 0.482 & 0.896 & 0.536 & 0.630 & 0.378 & 0.613 \\
& Self-Consistency & 0.666 & 0.802 & 0.586 & 0.494 & 0.902 & 0.530 & 0.640 & 0.382 & 0.625 \\
& TestNUC & 0.712 & 0.858 & 0.614 & 0.528 & 0.936 & 0.544 & 0.674 & 0.410 & 0.660 \\
& \cellcolor{gray!18}TestNUC\textdagger & \cellcolor{gray!18}\textbf{0.764} & \cellcolor{gray!18}\textbf{0.864} & \cellcolor{gray!18}\textbf{0.646} & \cellcolor{gray!18}\textbf{0.540} & \cellcolor{gray!18}\textbf{0.948} & \cellcolor{gray!18}\textbf{0.554} & \cellcolor{gray!18}\textbf{0.680} & \cellcolor{gray!18}\textbf{0.414} & \cellcolor{gray!18}\textbf{0.676} \\ \midrule \midrule
Llama-3.1-8B & Standard Prompting & 0.572 & 0.726 & 0.502 & 0.492 & 0.892 & 0.528 & 0.530 & 0.332 & 0.572 \\
& Self-Consistency & 0.620 & 0.774 & 0.564 & 0.526 & 0.902 & 0.518 & 0.564 & 0.340 & 0.601 \\
& TestNUC & 0.694 & 0.806 & 0.618 & 0.558 & 0.934 & 0.528 & 0.596 & 0.356 & 0.636 \\
& \cellcolor{gray!18}TestNUC\textdagger & \cellcolor{gray!18}\textbf{0.724} & \cellcolor{gray!18}\textbf{0.812} & \cellcolor{gray!18}\textbf{0.646} & \cellcolor{gray!18}\textbf{0.576} & \cellcolor{gray!18}\textbf{0.940} & \cellcolor{gray!18}\textbf{0.542} & \cellcolor{gray!18}\textbf{0.614} & \cellcolor{gray!18}\textbf{0.360} & \cellcolor{gray!18}\textbf{0.652} \\ \midrule \midrule
Claude-3-Haiku & Standard Prompting & 0.680 & 0.848 & 0.486 & 0.564 & 0.892 & 0.552 & 0.594 & 0.336 & 0.619 \\
& Self-Consistency & 0.702 & 0.870 & 0.510 & 0.578 & 0.904 & 0.564 & 0.568 & 0.350 & 0.631 \\
& TestNUC & 0.762 & 0.894 & 0.596 & 0.588 & 0.940 & 0.590 & 0.620 & 0.348 & 0.667 \\
& \cellcolor{gray!18}TestNUC\textdagger & \cellcolor{gray!18}\textbf{0.804} & \cellcolor{gray!18}\textbf{0.902} & \cellcolor{gray!18}\textbf{0.612} & \cellcolor{gray!18}\textbf{0.600} & \cellcolor{gray!18}\textbf{0.946} & \cellcolor{gray!18}\textbf{0.622} & \cellcolor{gray!18}\textbf{0.660} & \cellcolor{gray!18}\textbf{0.368} & \cellcolor{gray!18}\textbf{0.689} \\ \midrule \midrule
GPT-4o & Standard Prompting & 0.746 & 0.924 & 0.712 & 0.674 & 0.962 & 0.614 & 0.682 & 0.406 & 0.715 \\
& Self-Consistency & 0.758 & 0.922 & 0.720 & 0.688 & 0.958 & 0.624 & 0.696 & 0.426 & 0.724 \\
&TestNUC & 0.804 & 0.934 & 0.744 & \textbf{0.710} & 0.974 & 0.644 & 0.692 & 0.446 & 0.744 \\
& \cellcolor{gray!18}TestNUC\textdagger & \cellcolor{gray!18}\textbf{0.824} & \cellcolor{gray!18}\textbf{0.940} & \cellcolor{gray!18}\textbf{0.750} & \cellcolor{gray!18}\textbf{0.710} & \cellcolor{gray!18}\textbf{0.978} & \cellcolor{gray!18}\textbf{0.654} & \cellcolor{gray!18}\textbf{0.708} & \cellcolor{gray!18}\textbf{0.464} & \cellcolor{gray!18}\textbf{0.754} \\
\bottomrule
\end{tabular}%
}
\caption{Accuracy comparison with Standard Prompting and Self-Consistency across four diverse LLMs. TestNUC consistently improves the inference performance on all benchmark datasets. $\dagger$ denotes that 50 neighbors are utilized.}
\label{tab:main_compare_sc}
\end{table*}

\subsection{Models and Datasets}

To evaluate the effectiveness of Transfer-Prompting, we tested 7 foundation Models on 3 Common-sense Reasoning datasets, including GPT-3.5-Turbo \citep{chatgpt}, GPT-4 \citep{openai2023gpt4}, LLaMA-2 (7B \& 13B) \citep{touvron2023llama}, LLaMA-3-8B \citep{llama3modelcard}, and Vicuna (7B \& 13B) \citep{zheng2023judging}. In addition, we also evaluated 18 professional models on 6 multi-task datasets, including medical, legal, and financial fields.

Specifically, in the medical domain, we evaluated 6 specialized LLMs: ChatDoctor-13B \citep{li2023chatdoctor}, PMC-LLaMA-13B \citep{wu2023pmcllama}, MedAlpaca (7B \& 13B) \citep{han2023medalpaca}, and Medicine-LLM (7B \& 13B) \citep{cheng2023adapting}. For the legal domain, we tested 6 law-specific LLMs: DISC-LawLLM-13B \citep{yue2023disclawllm}, Lawyer-LLaMA-13B \citep{huang2023lawyer}, ChatLaw-13B \citep{cui2024chatlaw}, LawGPT-7B \citep{zhou2024lawgpt}, and Law-LLM (7B \& 13B) \citep{cheng2023adapting}. In the financial domain, we evaluated 6 LLMs: CFGPT-7B-Full \citep{li2023cfgpt}, Tongyi-Finance-14B-Chat \citep{Tongyi-Finance-14B-Chat}, FinGPT-13B-v2 (based on LLaMA-2-13B) \citep{yang2023fingpt}, FinMA-7B-Full \citep{xie2023pixiu}, and Finance-LLM (7B \& 13B) \citep{cheng2023adapting}.

The foundational models were assessed using 3 commonsense reasoning datasets, including LogiQA \citep{liu2020logiqa}, OpenBookQA \citep{Mihaylov2018CanAS}, and CosmosQA \citep{huang2019cosmos}. In the professional field, we adopt total 6 multi-task datasets to evaluate professional LLM:

\textbf{Medical Domain}: The corresponding medical models are evaluated using MMLU \citep{hendrycks2021measuring}, C-Eval \citep{huang2024c}, and MedMCQA \citep{pmlr-v174-pal22a} datasets.

\textbf{Legal Domain}: The corresponding legal models are assessed with MMLU \citep{hendrycks2021measuring}, CMMLU \citep{li2023cmmlu}, and AGIEval \citep{zhong2023agieval} datasets.

\textbf{Financial Domain}: The corresponding financial models are evaluated using CMMLU \citep{li2023cmmlu}, C-Eval \citep{huang2024c}, and FinEval \citep{zhang2023fineval} datasets.

 
\subsection{Confidence Evaluation Methods}

We employ the following approaches to quantify model uncertainty:

\noindent\textbf{Logits} \citep{yang2023improving}: The probabilities generated by the model are directly interpreted as confidence scores, with the highest probability corresponding to the selected answer in multiple-choice questions.

\noindent\textbf{Verbalized Confidence} \citep{lin2022teaching}: By prompting LLMs, we obtain both answers and their associated confidence scores. These scores are utilized to assess the models' calibration by analyzing the relationship between accuracy and the confidence levels of all valid responses.





\subsection{Evaluation Metrics}

We assess the instruction-following capabilities of LLMs using the instruction-following rate and accuracy. Additionally, we evaluate the overall response quality of the models through expected calibration error, area under the receiver operating characteristic curve, and area under the precision-recall curve.

\textbf{Expected Calibration Error (ECE):} ECE measures the alignment between predicted probabilities and actual outcomes, providing insight into the model's calibration quality. It is calculated as:

\begin{equation}
\text{ECE} = \sum_{i=1}^{n} \frac{|B_i|}{N} \cdot \left| \text{acc}(B_i) - \text{conf}(B_i) \right|,
\end{equation}

where \( n \) is the number of bins (defaulting to 10 in this study), \( B_i \) represents the samples in bin \( i \), \( N \) is the total number of samples, \( \text{acc}(B_i) \) is the accuracy within bin \( i \), and \( \text{conf}(B_i) \) is the mean predicted probability in bin \( i \).

\textbf{Area Under the Receiver Operating Characteristic Curve (AUROC):} AUROC evaluates a binary classification model’s ability to distinguish between positive and negative classes. It is derived from the area under the ROC curve, which plots the true positive rate against the false positive rate across various thresholds.

\textbf{Area Under the Precision-Recall Curve (PR-AUC) for Positive and Negative Classes (PR-P, PR-N):} PR-P and PR-N measure a model's precision and recall for positive and negative classes, respectively. PR-P is particularly useful for evaluating performance on imbalanced datasets, while PR-N is essential for accurately identifying negative instances.

\textbf{Instruction Following Rate (IFR):} IFR quantifies the proportion of instances where the model's response adheres to the specified instructions. It is defined as:

\[
\text{IFR} = \left( \frac{N_S}{N_T} \right) \times 100\%,
\]

where \( N_S \) is the number of instances where the LLM's responses satisfy the specified requirements, and \( N_T \) is the total number of instructions attempted, encompassing both successful and unsuccessful responses.



\section{Result and Analysis}

\subsection{Performance Comparison}

Table \ref{tab:result} presents the evaluation results of our proposed method, alongside the baseline models, XLM-R and FAVA, on the Mu-SHROOM dataset. The results are reported across nine languages (AR, CS, DE, EN, ES, EU, FI, FR, IT) and averaged to provide an overall assessment of performance.

REFIND outperforms the baseline models in terms of average IoU scores.
The improvements are particularly notable in low-resource languages such as Arabic, Finnish, and French, where REFIND achieves IoU scores of 0.3743, 0.5061, and 0.4734, respectively, compared to significantly lower scores from the baselines. This indicates that REFIND effectively leverages retrieval-augmented information to enhance hallucination detection in diverse linguistic settings.


\subsection{Baseline Comparison}

The XLM-R-based token classifier performs poorly on average, with an IoU of 0.0345. Its reliance solely on intrinsic model knowledge without leveraging external context limits its ability to identify hallucinated spans accurately, particularly in low-resource languages.

FAVA exhibits better performance than XLM-R, with an average IoU of 0.2787. This improvement can be attributed to its use of retrieval-augmented information for detecting and editing hallucinated text. However, FAVA's two-step process introduces complexity and potential inaccuracies in aligning the edited text with the original output.

REFIND outperforms both baselines with an average IoU of 0.3633, highlighting its superior ability to integrate retrieved context directly into the token generation process for hallucination detection. 
This streamlined approach ensures accurate and efficient identification of hallucinated spans.

\subsection{Analysis of Multilingual Performance}

REFIND demonstrates robust performance across both high-resource and low-resource languages. This indicates the generalizability of its retrieval-augmented approach to varying linguistic contexts.  Notably, performance varies considerably across languages for all methods; for instance, XLM-R and FAVA struggle significantly with low-resource languages like Arabic, Finnish, and French. In contrast, REFIND's integration of external retrieval with the LLM's internal knowledge helps mitigate performance drops in these settings.


\begin{figure}[t!]
    \centering
    \includegraphics[width=\columnwidth]{figure/pdf/threshold_analysis.pdf}
    \caption {
        Analysis of IoU scores across different threshold values ($\delta \in {0.1, 0.2, 0.3, 0.4}$). Each subplot represents a different language, showing the relationship between threshold values and IoU scores. 
    } 
    \label{fig:threshold_analysis}
\end{figure}




\subsection{Analysis of Threshold Sensitivity}

Figure~\ref{fig:threshold_analysis} illustrates the performance of REFIND across varying threshold values (0.1-0.4) for nine languages.
Most languages exhibit consistent IoU scores, indicating robustness to threshold changes. High-resource languages like English and German maintain stable scores around 0.35, while low-resource languages such as Arabic and Finnish show slightly larger variations, especially at lower thresholds. 
This suggests that the choice of threshold may have a more significant impact on low-resource languages, potentially due to their inherent linguistic challenges and data scarcity. 
Overall, these findings emphasize REFIND's ability to maintain reliable performance across a range of threshold values while highlighting potential areas for optimization in low-resource scenarios.


\subsection{Case Study}
\label{sec:case_study}

Figure~\ref{fig:case_study} illustrates REFIND's ability to detect hallucinations by utilizing retrieved evidence. The question asks about Chance the Rapper's debut year. The LLM's output contains a hallucinated span ("\underline{\hl{2011}}"), which is inconsistent with the retrieved documents. By comparing the generated output with external knowledge, REFIND effectively identifies spans that deviate from factual information. 









\section{Conclusion} \label{sec:6}


% Future work will focus on several key areas to further enhance the practical utility of LLMs. This includes scaling the approach for larger models, customizing it for specialized domains to meet specific requirements, improving robustness against distribution shifts to ensure reliable performance in varied conditions, and applying it to interactive and real-time applications. Additionally, we aim to explore the integration of Transfer-Prompting with other advanced techniques, such as reinforcement learning and meta-learning, to further push the boundaries of LLM performance.


% In this study, we introduce \textbf{Transfer-Prompting}, a novel approach to enhancing the performance of LLMs by optimizing and transferring source prompts to target tasks. Our extensive evaluations across general and domain-specific applications, such as medical, legal, and financial tasks, demonstrate significant improvements in metrics including Instruction-Following Rate and Expected Calibration Error. The analysis of logits and evaluation accuracy curves further validate the robustness and efficiency of Transfer-Prompting, highlighting its potential for rapid and substantial performance gains. 
In this study, we introduce \textbf{Transfer-Prompting}, an innovative approach designed to enhance the generalization capabilities of LLMs by optimizing and adapting source prompts for the target task. One of the main advantages of Transfer-Prompting is its ability to generate prompts that are well-adapted to the target dataset, thereby improving the model's performance. This adaptability makes it particularly suitable for applications in diverse fields such as healthcare, legal, and financial services, where accurate and reliable model outputs are critical. Additionally, our approach is expected to alleviate problems associated with model calibration, ensuring that the confidence of predictions aligns more closely with their true accuracy.
Extensive evaluations of base models and domain-specific models demonstrate significant improvements in prediction accuracy, calibration, and instruction-following. 

% Transfer-Prompting's robustness and efficiency are validated through various metrics, highlighting its potential to enhance performance rapidly.



% \section{Limitations}

% Our proposed method, POI, shows promising results in adapting LLMs to target tasks but has several limitations. The computational resources required for extensive evaluations of larger-size LLMs are substantial, which may limit the scale and diversity of our experiments. We plan to leverage more powerful computational resources in future research. While we have tested POI across various domains, its generalizability to all possible tasks remains uncertain, necessitating further assessment in a wider range of applications. 


% Bibliography entries for the entire Anthology, followed by custom entries
%\bibliography{anthology,custom}
% Custom bibliography entries only

\bibliographystyle{named}
\bibliography{ref}

 \newpage
 \appendix

 \vspace{2em}
 \begin{center}
     \Large{\textbf{Appendix}}
 \end{center}
 \vspace{2em}

 \etocdepthtag.toc{appendix}
 \etocsettagdepth{chapter}{none}
 \etocsettagdepth{appendix}{subsection}
 \tableofcontents

 
%%%%%%%%%%%%%%%%%%%%%%%%%%%%%%%%%%%%%%%%%%%%%%%%%%%%%%%%%%%%%%%%%%%%%%%%%%%%%%%
%%%%%%%%%%%%%%%%%%%%%%%%%%%%%%%%%%%%%%%%%%%%%%%%%%%%%%%%%%%%%%%%%%%%%%%%%%%%%%%
% APPENDIX
%%%%%%%%%%%%%%%%%%%%%%%%%%%%%%%%%%%%%%%%%%%%%%%%%%%%%%%%%%%%%%%%%%%%%%%%%%%%%%%
%%%%%%%%%%%%%%%%%%%%%%%%%%%%%%%%%%%%%%%%%%%%%%%%%%%%%%%%%%%%%%%%%%%%%%%%%%%%%%%
\newpage
\appendix
\onecolumn
\section*{Appendix Overview}
\begin{itemize}
    \item Section~\ref{appendix:related}: Related Work.
    \item Section~\ref{appendix:more_dataset}: More Dataset Details.
    \item Section~\ref{appendix:error_analysis}: Error Analysis.
    \item Section~\ref{appendix:more_qualitative}: More Qualitative Examples.
    \item Section~\ref{appendix:eval_setup}: Evaluation Prompts.
\end{itemize}


\section{Related Work}
\label{appendix:related}
\subsection{Large Multimodal Models}
The field of multimodal~\citep{Radford2021LearningTV, li2022blip, openai2023gpt4v, openai2024gpt4o} AI has experienced extraordinary growth, particularly through the development of Large Multimodal Models (LMMs)~\cite{liu2023llava,zhu2023minigpt,lin2023sphinx,Qwen2-VL}. These models build upon the achievements of Large Language Models (LLMs)~\citep{touvron2023llama,qwen2} and advanced vision models~\cite{Radford2021LearningTV}, expanding their capabilities to process multiple kinds of visual input~\cite{li2024llava,guo2023point,li2023videochat}.

Closed-source models, such as OpenAI's GPT-4o~\citep{openai2024gpt4o}, have demonstrated exceptional capabilities in visual understanding and reasoning. However, their closed-source nature creates barriers to widespread adoption and further development by the broader research community. In response, significant progress has been made in developing open-source alternatives. Early approaches like LLaVA~\cite{liu2023llava}, LLaMA-Adapter~\cite{zhang2024llamaadapter}, and MiniGPT-4~\cite{zhu2023minigpt} established a foundation by combining frozen CLIP models for image encoding with LLMs, enabling multimodal instruction tuning. Subsequent developments through projects such as InternVL2~\cite{chen2024far}, Qwen2-VL~\cite{Qwen2-VL}, SPHINX~\cite{gao2024sphinx,lin2023sphinx}, and MiniCPM-V~\cite{yao2024minicpm} have expanded these capabilities by incorporating more diverse visual instruction datasets and broadening application scenarios.

Recently, with the introduction of o1~\cite{o1}, the field of LMMs has also focused on enhancing the reasoning capability. \cite{wang2024enhancing} introduces mixed preference optimization with automatically constructed data. \cite{yao2024mulberry} proposes to leverage collective knowledge from multiple models to identify effective reasoning paths. Besides, several works~\cite{qvq-72b-preview,du2025virgo} have demonstrated the ability to replicate behaviors similar to o1 models, particularly regarding multi-step CoT reasoning with iterative self-reflection and verification processes.

\subsection{Reasoning Evaluation}
Several methods have been developed to evaluate reasoning in natural language processing, including ROSCOE~\cite{golovneva2022roscoe} and ReCEval~\cite{prasad2023receval}, which assess reasoning chains across multiple dimensions such as correctness and informativeness. However, these approaches are limited to text-only scenarios and do not address the unique challenges present in visual reasoning tasks. Furthermore, the emergence of long chain-of-thought (CoT) reasoning has introduced additional considerations, such as output efficiency and reflection quality, which existing evaluation methods do not adequately address.

On the other hand, various multimodal benchmarks have been developed to assess reasoning abilities across specific domains. Current exploration of visual reasoning predominantly focuses on the mathematics~\cite{zhang2024mavis,peng2024chimera} domains. 
MathVista~\cite{Lu2023MathVistaEM} provides a comprehensive collection of mathematical problems that assess mathematical and logical reasoning abilities. 
Building on this, MathVerse~\cite{zhang2024mathverse} introduces a new benchmark by eliminating redundant textual information to evaluate whether LMMs can accurately interpret graphical representations. 
OlympiadBench~\cite{he2024olympiadbench} further raises the complexity bar by incorporating challenging Olympiad-level mathematics and physics problems. Despite these advances in specialized domains, broader applications such as general-scene reasoning remain relatively unexplored.
Recent developments have begun to expand beyond purely scientific reasoning. For instance, M³CoT~\cite{chen-etal-2024-m3cot} and SciVerse~\cite{sciverse} incorporate commonsense tasks alongside scientific reasoning and knowledge-based assessment in the multimodal benchmark. However, most existing benchmarks focus solely on evaluating final answers while overlooking the intermediate steps, thus providing limited insights into the process through which models arrive at their conclusions.


\section{More Dataset Details}
\label{appendix:more_dataset}
\subsection{Data Source Distribution}
We visualize the data source distributions in our benchmark, which consists of 15 sets, including MathVerse~\cite{zhang2024mathverse}, MMMUPro~\cite{yue2024mmmuprorobustmultidisciplinemultimodal}, OlympiadBench~\cite{he2024olympiadbench}, MMT-Bench~\cite{ying2024mmt}, MuirBench~\cite{wang2024muirbench}, ml-rpm-bench~\cite{zhang2024far}, MMSearch~\cite{jiang2024mmsearch}, CharXiv~\cite{wang2024charxiv}, and SciVerse~\cite{sciverse}.

\begin{figure*}[!h]
\centering
\includegraphics[width=0.4\textwidth]{fig/pie_supp.pdf} 
\caption{\textbf{Data Source Distribution of MME-CoT.}}
\label{appendix:more_dataset-source}
\end{figure*}

\newpage

\subsection{Preliminary Categorization Result}
\label{appendix:preliminary_result}
\begin{table}[htbp]
    \centering
    \caption{\textbf{Accuracy of MMT-Bench for different subcategories}. ACT: Action Understanding; AUT: Attribute Similarity; CNT: Cartoon Understanding; CIM: Counting; DOC: Diagram Understanding; EMO: Difference Spotting; HAL: Geographic Understanding; IIT: Image-Text Matching; IRT: Ordering; IQT: Scene Understanding; MEM: Visual Grounding; MIA: Visual Retrieval; OCR: Object Recognition; PLP: Physical Layout Prediction; RRE: Relationship Extraction; TMP: Temporal Reasoning; VCP: Visual Comprehension; VCR: Visual Coherence Reasoning; VGR: Visual Generation; VIL: Visual Identification; VPU: Visual Prediction Understanding; VRE: Visual Reasoning Evaluation.}
    \label{tab:hit_ratio}
    \setlength{\tabcolsep}{4pt} 
    \renewcommand{\arraystretch}{1.2}
    \small 
    \begin{tabularx}{\textwidth}{l *{22}{X}}
        \toprule
        File Name & 
        \rotatebox{90}{ACT} & \rotatebox{90}{AUT} & \rotatebox{90}{CNT} & \rotatebox{90}{CIM} & 
        \rotatebox{90}{DOC} & \rotatebox{90}{EMO} & \rotatebox{90}{HAL} & \rotatebox{90}{IIT} & 
        \rotatebox{90}{IRT} & \rotatebox{90}{IQT} & \rotatebox{90}{MEM} & \rotatebox{90}{MIA} & 
        \rotatebox{90}{OCR} & \rotatebox{90}{PLP} & \rotatebox{90}{RRE} & \rotatebox{90}{TMP} & 
        \rotatebox{90}{VCP} & \rotatebox{90}{VCR} & \rotatebox{90}{VGR} & \rotatebox{90}{VIL} & 
        \rotatebox{90}{VPU} & \rotatebox{90}{VRE} \\
        \midrule
        GPT4o-cot & 0.60 & 0.60 & 0.44 & 0.67 & 0.79 & 0.30 & 0.71 & 0.50 & 0.63 & 0.10 & 0.85 & 0.60 & 0.77 & 0.36 & 0.76 & 0.48 & 0.86 & 0.80 & 0.49 & 0.48 & 0.82 & 0.85 \\
        GPT4-direct & 0.53 & 0.60 & 0.44 & 0.67 & 0.81 & 0.23 & 0.69 & 0.33 & 0.66 & 0.25 & 0.80 & 0.43 & 0.78 & 0.42 & 0.78 & 0.36 & 0.89 & 0.85 & 0.41 & 0.37 & 0.85 & 0.85 \\
        Qwen2-VL-7B-cot & 0.53 & 0.61 & 0.34 & 0.65 & 0.77 & 0.53 & 0.74 & 0.40 & 0.31 & 0.20 & 0.78 & 0.58 & 0.60 & 0.43 & 0.69 & 0.43 & 0.85 & 0.90 & 0.54 & 0.35 & 0.79 & 0.81 \\
        Qwen2-VL-7B-direct & 0.49 & 0.67 & 0.40 & 0.78 & 0.75 & 0.52 & 0.73 & 0.43 & 0.31 & 0.10 & 0.78 & 0.55 & 0.60 & 0.54 & 0.69 & 0.40 & 0.85 & 0.85 & 0.67 & 0.38 & 0.85 & 0.82 \\
        \bottomrule
    \end{tabularx}
\end{table}


\begin{table}[htbp]
    \centering
    \caption{\textbf{Accuracy of MUIRBench for different subcategories}. AU: Action Understanding; AS: Attribute Similarity; CU: Cartoon Understanding; CO: Counting; DU: Diagram Understanding; DS: Difference Spotting; GU: Geographic Understanding; ITM: Image-Text Matching; OR: Ordering; SU: Scene Understanding; VG: Visual Grounding; VR: Visual Retrieval.}

    \label{tab:hit_ratio}
    \setlength{\tabcolsep}{4pt} 
    \renewcommand{\arraystretch}{1.2} 
    \small 
    \begin{tabularx}{\textwidth}{l XXXX XXXX XXXX XXXX}
        \toprule
        File Name & AU & AS & CU & CO & DU & DS & GU & ITM & OR & SU & VG & VR \\
        \midrule
        GPT4o-cot & 0.48 & 0.57 & 0.55 & 0.75 & 0.82 & 0.64 & 0.59 & 0.82 & 0.38 & 0.88 & 0.56 & 0.70 \\
        GPT4o-direct & 0.45 & 0.62 & 0.59 & 0.50 & 0.88 & 0.62 & 0.55 & 0.86 & 0.33 & 0.74 & 0.38 & 0.77 \\
        Qwen2-VL-7B-cot & 0.38 & 0.51 & 0.42 & 0.43 & 0.43 & 0.27 & 0.21 & 0.55 & 0.13 & 0.69 & 0.37 & 0.28 \\
        Qwen2-VL-7B-direct & 0.39 & 0.47 & 0.44 & 0.41 & 0.40 & 0.33 & 0.25 & 0.51 & 0.13 & 0.67 & 0.31 & 0.20 \\
        \bottomrule
    \end{tabularx}
\end{table}



\begin{table}[htbp]
    \centering
    \caption{\textbf{Accuracy of OlympiadBench for the mathematics and physics subcategories}.}
    \label{tab:hit_ratio_oe}
    \small 
    \begin{tabular}{lcc}
        \toprule
        File Name & Mathematics & Physics\\
        \midrule
        GPT4o-cot & 0.25 & 0.04 \\
        GPT4o-direct & 0.07 & 0.03 \\
        Qwen2-VL-7B-cot & 0.05 & 0.01 \\
        Qwen2-VL-7B-direct & 0.07 & 0.01 \\
        \bottomrule
    \end{tabular}
\end{table}

\newpage

\section{Error Analysis}
\label{appendix:error_analysis}
We showcase the examples of the identified error types of reflection in Fig.~\ref{fig:ref_error_example}.
\begin{figure*}[!h]
\centering
\includegraphics[width=\textwidth]{fig/ref_error_example.pdf} 
\caption{\textbf{Examples of Reflection Error Types.}}
\label{fig:ref_error_example}
\end{figure*}


\newpage

\section{More Qualitative Examples}
\label{appendix:more_qualitative}
\begin{figure*}[!h]
\centering
\includegraphics[width=0.6\textwidth]{fig/precision_recall_example_GPT.pdf} 
\caption{\textbf{Examples of Precision and Recall Evaluation.}}
\label{fig:precision_recall_example_GPT}
\end{figure*}
\newpage

\begin{figure*}[!h]
\centering
\includegraphics[width=0.9\textwidth]{fig/precision_recall_example_Qwen.pdf} 
\caption{\textbf{Examples of Precision and Recall Evaluation.}}
\label{fig:precision_recall_example_Qwen}
\end{figure*}
\newpage

\begin{figure*}[!h]
\centering
\includegraphics[width=0.58\textwidth]{fig/precision_recall_example_QVQ.pdf}
\caption{\textbf{Examples of Precision and Recall Evaluation.}}
\label{fig:precision_recall_example_QVQ}
\end{figure*}
\newpage

\begin{figure*}[!h]
\centering
\includegraphics[width=\textwidth]{fig/precision_recall_example_QVQ2.pdf} 
\caption{\textbf{Examples of Precision and Recall Evaluation.}}
\label{fig:precision_recall_example_QVQ2}
\end{figure*}
\newpage

\begin{figure*}[!h]
\centering
\includegraphics[width=0.51\textwidth]{fig/precision_recall_example2_GPT.pdf} 
\caption{\textbf{Examples of Precision and Recall Evaluation.}}
\label{fig:precision_recall_example2_GPT}
\end{figure*}
\newpage

\begin{figure*}[!h]
\centering
\includegraphics[width=0.79\textwidth]{fig/precision_recall_example2_Qwen.pdf} 
\caption{\textbf{Examples of Precision and Recall Evaluation.}}
\label{fig:precision_recall_example2_Qwen}
\end{figure*}
\newpage

\begin{figure*}[!h]
\centering
\includegraphics[width=0.81\textwidth]{fig/precision_recall_example2_QVQ.pdf} 
\caption{\textbf{Examples of Precision and Recall Evaluation.}}
\label{fig:precision_recall_example2_QVQ}
\end{figure*}
\newpage

\begin{figure*}[!h]
\centering
\includegraphics[width=\textwidth]{fig/relevance_example_GPT.pdf} 
\caption{\textbf{Examples of Relevance Rate Evaluation.}}
% \vspace{-1cm}
\label{fig:relevance_example_GPT}
\end{figure*}
\newpage

\begin{figure*}[!h]
\centering
\includegraphics[width=\textwidth]{fig/relevance_example_Qwen.pdf} 
\caption{\textbf{Examples of Relevance Rate Evaluation.}}
% \vspace{-1cm}
\label{fig:relevance_example_Qwen}
\end{figure*}
\newpage

\begin{figure*}[!h]
\centering
\includegraphics[width=\textwidth]{fig/relevance_example_QVQ.pdf} 
\caption{\textbf{Examples of Relevance Rate Evaluation.}}
% \vspace{-1cm}
\label{fig:relevance_example_QVQ}
\end{figure*}
\newpage

\begin{figure*}[!h]
\centering
\includegraphics[width=\textwidth]{fig/ref_example_QVQ.pdf} 
\caption{\textbf{Examples of Reflection Quality Evaluation.}}
% \vspace{-1cm}
\label{fig:ref_example_QVQ}
\end{figure*}
\newpage


\section{Detailed Evaluation Setup}
\label{appendix:eval_setup}
\subsection{CoT Quality Evaluation Prompts}

\begin{tcolorbox}[breakable, colback=gray!5!white, colframe=gray!75!black, 
title=Recall Evaluation Prompt, boxrule=0.5mm, width=\textwidth, arc=3mm, auto outer arc]

You are an expert system to verify solutions to image-based problems. Your task is to match the ground truth middle steps with the provided solution.\\

INPUT FORMAT:\\
1. Problem: The original question/task\\
2. A Solution of a model\\
3. Ground Truth: Essential steps required for a correct answer\\

MATCHING PROCESS:\\

You need to match each ground truth middle step with the solution:\\

Match Criteria:\\
- The middle step should exactly match in the content or is directly entailed by a certain content in the solution\\
- All the details must be matched, including the specific value and content\\
- You should judge all the middle steps for whether there is a match in the solution\\

OUTPUT FORMAT:
\begin{verbatim}
[
  {
    "step_index": \textless integer\textgreater,
    "judgment": "Matched" | "Unmatched"
  }
]
\end{verbatim}

ADDITIONAL RULES:\\
1. Only output the JSON array with no additional information.\\
2. Judge each ground truth middle step in order without omitting any step.\\

Here are the problem, answer, solution, and ground truth middle steps:\\

[Problem]\\

\{question\}\\

[Answer]\\

\{answer\}\\

[Solution]\\

\{solution\}\\

[Ground Truth Information]\\

\{gt\_annotation\}

\end{tcolorbox}

\begin{tcolorbox}[breakable, colback=gray!5!white, colframe=gray!75!black, 
title=Precision Evaluation Prompt, boxrule=0.5mm, width=\textwidth, arc=3mm, auto outer arc]

\# Task Overview\\
Given a solution with multiple reasoning steps for an image-based problem, reformat it into well-structured steps and evaluate their correctness.\\

\# Step 1: Reformatting the Solution\\
Convert the unstructured solution into distinct reasoning steps while:\\
- Preserving all original content and order\\
- Not adding new interpretations\\
- Not omitting any steps\\

\#\# Step Types\\
1. Logical Inference Steps\\
   - Contains exactly one logical deduction\\
   - Must produce a new derived conclusion\\
   - Cannot be just a summary or observation\\
\\
2. Image Observation Steps\\
   - Pure visual observations\\
   - Only includes directly visible elements\\
   - No inferences or assumptions\\
\\
3. Background Information Steps\\
   - External knowledge or question context\\
   - No inference process involved\\

\#\# Step Requirements\\
- Each step must be atomic (one conclusion per step)\\
- No content duplication across steps\\
- Initial analysis counts as background information\\
- Final answer determination counts as logical inference\\

\# Step 2: Evaluating Correctness\\
Evaluate each step against:\\

\#\# Ground Truth Matching\\
For image observations:\\
- Key elements must match ground truth observations\\
\\
For logical inferences:\\
- Conclusion must EXACTLY match or be DIRECTLY entailed by ground truth\\

\#\# Reasonableness Check (if no direct match)\\
Step must:\\
- Premises must not contradict any ground truth or correct answer\\
- Logic is valid\\
- Conclusion must not contradict any ground truth \\
- Conclusion must support or be neutral to correct answer\\

\#\# Judgement Categories\\
- "Match": Aligns with ground truth\\
- "Reasonable": Valid but not in ground truth\\
- "Wrong": Invalid or contradictory\\
- "N/A": For background information steps\\

\# Output Requirements\\
1. The output format must be in valid JSON format without any other content.\\
2. For highly repetitive patterns, output it as a single step.\\
3. Output maximum 40 steps. Always include the final step that contains the answer.\\

Here is the json output format:\\
\#\# Output Format
\begin{verbatim}
[
  {
    "step_type": "image observation|logical inference|background information",
    "premise": "Evidence (only for logical inference)",
    "conclusion": "Step result",
    "judgment": "Match|Reasonable|Wrong|N/A"
  }
]
\end{verbatim}

Here is the problem, and the solution that needs to be reformatted to steps:\\

[Problem]\\

\{question\}\\

[Solution]\\

\{solution\}\\

[Correct Answer]\\

\{answer\}\\

[Ground Truth Information]\\

\{gt\_annotation\}

\end{tcolorbox}

\subsection{CoT Efficiency Prompt}
\begin{tcolorbox}[breakable, colback=gray!5!white, colframe=gray!75!black, 
title=Relevance Rate Evaluation Prompt, boxrule=0.5mm, width=\textwidth, arc=3mm, auto outer arc]
\# Task Overview
Given a solution with multiple reasoning steps for an image-based problem, evaluate the relevance to get a solution (ignore correct or wrong) of each step.\\

\# Step 1: Reformatting the Solution
Convert the unstructured solution into distinct reasoning steps while:\\
- Preserving all original content and order\\
- Not adding new interpretations\\
- Not omitting any steps\\

\#\# Step Types \\
1. Logical Inference Steps\\
  - Contains exactly one logical deduction\\
  - Must produce a new derived conclusion\\
  - Cannot be just a summary or observation

2. Image Description Steps\\
  - Pure visual observations\\
  - Only includes directly visible elements\\
  - No inferences or assumptions

3. Background Information Steps\\
  - External knowledge or question context\\
  - No inference process involved\\

\#\# Step Requirements
- Each step must be atomic (one conclusion per step)\\
- No content duplication across steps\\
- Initial analysis counts as background information\\
- Final answer determination counts as logical inference\\

\# Step 2: Evaluating Relevancy\\
A relevant step is considered as: 75\% content of the step must be related to trying to get a solution (ignore correct or wrong) to the question.\\

IMPORTANT NOTE:\\
Evaluate relevancy independent of correctness. As long as the step is trying to get to a solution, it is considered relevant. Logical fallacy, knowledge mistake, inconsistent with previous steps, or other mistakes do not affect relevance. A logically wrong step can be relevant if the reasoning attempts to address the question.\\

The following behaviour is considered as relevant:\\
i. The step is planning, summarizing, thinking, verifying, calculating, or confirming an intermediate/final conclusion helpful to get a solution.\\
ii. The step is summarizing or reflecting on previously reached conclusion relevant to get a solution.\\
iii. Repeating the information in the question or give the final answer.\\
iv. A relevant image depiction should be in one of following situation:\\
1. help to obtain a conclusion helpful to solve the question later;\\
2. help to identify certain patterns in the image later;\\
3. directly contributes to the answer\\
v. Depicting or analyzing the options of the question is also relevant.\\
vi. Repeating previous relevant steps are also considered relevant.\\

The following behaviour is considered as irrelevant:\\
i. Depicting image information that does not related to what is asking in the question. Example: The question asks how many cars are present in all the images. If the step focuses on other visual elements like the road or building, the step is considered as irrelevant.\\
ii. Self-thought not related to what the question is asking.\\
iii. Other information that is tangential for answering the question.\\

\# Output Format

\begin{verbatim}
[
  {
    "step_type": "image observation|logical inference|background information",
    "conclusion": "A brief summary of step result",
    "relevant": "Yes|No"
  }
]
\end{verbatim}\\

\# Output Rules\\
Direct JSON output without any other output\\
Output at most 40 steps\\

Here is the problem, and the solution that needs to be reformatted to steps:

[Problem]\\

\{question\}\\

[Solution]\\

\{solution\}
\end{tcolorbox}

\begin{tcolorbox}[breakable, colback=gray!5!white, colframe=gray!75!black, 
title=Reflection Quality Evaluation Prompt, boxrule=0.5mm, width=\textwidth, arc=3mm, auto outer arc]

Here\'s a refined prompt that improves clarity and structure:\\

\# Task\\
Evaluate reflection steps in image-based problem solutions, where reflections are self-corrections or reconsideration of previous statements.\\

\# Reflection Step Identification \\
Reflections typically begin with phrases like:\\
- "But xxx"\\
- "Alternatively, xxx" \\
- "Maybe I should"\\
- "Let me double-check"\\
- "Wait xxx"\\
- "Perhaps xxx"\\
It will throw a doubt of its previously reached conclusion or raise a new thought.\\

\# Evaluation Criteria\\
Correct reflections must:\\
1. Reach accurate conclusions aligned with ground truth\\
2. Use new insights to find the mistake of the previous conclusion or verify its correctness. \\

Invalid reflections include:\\
1. Repetition - Restating previous content or method without new insights\\
2. Wrong Conclusion - Reaching incorrect conclusions vs ground truth\\
3. Incompleteness - Proposing but not executing new analysis methods\\
4. Other - Additional error types\\

\# Input Format\\

[Problem]\\

\{question\}\\

[Solution]\\

\{solution\}\\

[Ground Truth]\\

\{gt\_annotation\}\\

\# Output Requirements\\
1. The output format must be in valid JSON format without any other content.\\
2. Output maximum 30 reflection steps.\\

Here is the json output format:\\
\#\# Output Format
\begin{verbatim}
[
  {
    "conclusion": "One-sentence summary of reflection outcome",
    "judgment": "Correct|Wrong",
    "error_type": "N/A|Repetition|Wrong Conclusion|Incompleteness|Other"
  }
]
\end{verbatim}

\# Rules\\
1. Preserve original content and order\\
2. No new interpretations\\
3. Include ALL reflection steps\\
4. Empty list if no reflections found\\
5. Direct JSON output without any other output

\end{tcolorbox}

\subsection{Direct Evaluation Prompt}
\begin{tcolorbox}[breakable, colback=gray!5!white, colframe=gray!75!black, 
title=Answer Extraction Prompt, boxrule=0.5mm, width=\textwidth, arc=3mm, auto outer arc]
You are an AI assistant who will help me to extract an answer of a question. You are provided with a question and a response, and you need to find the final answer of the question. \\

Extract Rule:

[Multiple choice question]

1. The answer could be answering the option letter or the value. You should directly output the choice letter of the answer.

2. You should output a single uppercase character in A, B, C, D, E, F, G, H, I (if they are valid options), and Z.

3. If the meaning of all options are significantly different from the final answer, output Z. \\

[Non Multiple choice question]

1. Output the final value of the answer. It could be hidden inside the last step of calculation or inference. Pay attention to what the question is asking for to extract the value of the answer.

2. The final answer could also be a short phrase or sentence.

3. If the response doesn't give a final answer, output Z.\\

Output Format: 
Directly output the extracted answer of the response. \\

\{In Context Examples\}\\

Question: \{question\}

Answer: \{response\}\\

Your output: 

\end{tcolorbox}

\begin{tcolorbox}[breakable, colback=gray!5!white, colframe=gray!75!black, 
title=Answer Scoring Prompt, boxrule=0.5mm, width=\textwidth, arc=3mm, auto outer arc]

You are an AI assistant who will help me to judge whether two answers are consistent.\\

Input Illustration:
[Standard Answer] is the standard answer to the question. 
[Model Answer] is the answer extracted from a model's output to this question. 

Task Illustration:
Determine whether [Standard Answer] and [Model Answer] are consistent.\\

Consistent Criteria:

[Multiple-Choice questions]

1. If the [Model Answer] is the option letter, then it must completely matches the [Standard Answer].

2. If the [Model Answer] is not an option letter, then the [Model Answer] must completely match the option content of [Standard Answer].

[Nan-Multiple-Choice questions]

1. The [Model Answer] and [Standard Answer] should exactly match.

2. If the meaning is expressed in the same way, it is also considered consistent, for example, 0.5m and 50cm.\\

Output Format: 
1. If they are consistent, output 1; if they are different, output 0.

2. DIRECTLY output 1 or 0 without any other content.

\{In Context Examples\}\\

Question: \{question\}

[Model Answer]: \{extract\_answer\}

[Standard Answer]: \{gt\_answer\}

Your output:

\end{tcolorbox}

\end{document}