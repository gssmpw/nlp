\section{Related Work}
\label{sec:2}
% \textbf{Evaluation of the instruction following and overall output quality of LLMs.} LLMs have shown impressive capabilities, but their predictions often come with varying degrees of uncertainty, making calibration essential for reliable outputs. To address this, various methods have been proposed. \cite{kuleshov2018accurate} introduce a recalibration technique that adjusts confidence scores to better match empirical accuracy without altering the model's architecture or training data. Building on this, \cite{zhang2017mixup} propose mixup training, which enhances calibration by using convex combinations of inputs and labels during training. \cite{guo2017calibration} provide a comprehensive analysis of calibration errors and introduce metrics such as Expected Calibration Error (ECE) and Maximum Calibration Error (MCE), facilitating precise comparisons between models. For LLMs, \cite{desai2020calibration} apply temperature scaling, tailored specifically for the complexity of language models. More recent advances include an ensemble approach by \cite{zhao2021calibrate}, combining multiple models to achieve a calibrated consensus. Additionally, \cite{tian2023just} propose directly querying LLMs to assess their confidence in responses, while \cite{he2023investigating} evaluate LLM calibration using metrics like ECE, AUROC, and AUPRC. Finally, \cite{lyu2024calibrating} introduce coherence sampling, a novel approach contributing to the ongoing refinement of LLM calibration.

% \textbf{Prompt Engineering and Optimization.} Recent advances in prompt engineering have significantly enhanced the efficiency and efficacy of interactions with large language models (LLMs). Key developments include few-shot and zero-shot learning, which utilize minimal examples to guide models, thereby reducing reliance on extensive labeled datasets. In-context learning, introduced by \cite{brown2020language}, enables models to adapt to tasks using input prompts alone, without requiring parameter updates. Automated prompt generation methods, such as those explored by \cite{liu2023pre}, employ algorithms like reinforcement learning to discover optimal prompts, thus enhancing model performance. Furthermore, \cite{schick2020exploiting} demonstrated that pre-training models with specific prompt formats can improve generalization across diverse tasks.

% In terms of prompt optimization, recent studies have highlighted the effectiveness of LLMs as prompt optimizers. For instance, \cite{ma2024large} showed that LLMs can effectively refine prompts, boosting performance across various tasks. Similarly, \cite{yang2024largelanguagemodelsoptimizers} reported that optimized prompts generated by LLMs can outperform manually crafted ones through the Optimization by PROmpting (OPRO) method. Additionally, \cite{li2023robust} identified vulnerabilities in prompt optimization techniques due to distribution shifts, such as subpopulation shifts, and proposed Generalized Prompt Optimization (GPO) to enhance the generalization capabilities of LLMs.