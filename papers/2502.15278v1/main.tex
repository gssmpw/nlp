%%%%%%%% ICML 2025 EXAMPLE LATEX SUBMISSION FILE %%%%%%%%%%%%%%%%%

\documentclass{article}

% Recommended, but optional, packages for figures and better typesetting:
\usepackage{microtype}
\usepackage{graphicx}
\usepackage{subfigure}
\usepackage{booktabs} % for professional tables
\usepackage{subcaption}
\usepackage{float}  

% hyperref makes hyperlinks in the resulting PDF.
% If your build breaks (sometimes temporarily if a hyperlink spans a page)
% please comment out the following usepackage line and replace
% \usepackage{icml2025} with \usepackage[nohyperref]{icml2025} above.
\usepackage{hyperref}
\newcommand{\llv}[1]{\ifshowcmts {\color{orange}[lv: #1]} \fi}
\newcommand{\liu}[1]{\ifshowcmts {\color{blue}[liu: #1]} \fi}
\newcommand{\jin}[1]{\ifshowcmts {\color{red}[jin: #1]} \fi}

% Attempt to make hyperref and algorithmic work together better:
\newcommand{\theHalgorithm}{\arabic{algorithm}}

% Use the following line for the initial blind version submitted for review:
%\usepackage{icml2025}

% If accepted, instead use the following line for the camera-ready submission:
\usepackage[accepted]{icml2025}

% For theorems and such
\usepackage{amsmath}
\usepackage{amssymb}
\usepackage{mathtools}
\usepackage{amsthm}
\usepackage[most]{tcolorbox}
\usepackage{multirow}

% if you use cleveref..
\usepackage[capitalize,noabbrev]{cleveref}

%%%%%%%%%%%%%%%%%%%%%%%%%%%%%%%%
% THEOREMS
%%%%%%%%%%%%%%%%%%%%%%%%%%%%%%%%
\theoremstyle{plain}
\newtheorem{theorem}{Theorem}[section]
\newtheorem{proposition}[theorem]{Proposition}
\newtheorem{lemma}[theorem]{Lemma}
\newtheorem{corollary}[theorem]{Corollary}
\theoremstyle{definition}
\newtheorem{definition}[theorem]{Definition}
\newtheorem{assumption}[theorem]{Assumption}
\theoremstyle{remark}
\newtheorem{remark}[theorem]{Remark}

% Todonotes is useful during development; simply uncomment the next line
%    and comment out the line below the next line to turn off comments
%\usepackage[disable,textsize=tiny]{todonotes}
\usepackage[textsize=tiny]{todonotes}


% The \icmltitle you define below is probably too long as a header.
% Therefore, a short form for the running title is supplied here:
%\icmltitlerunning{Submission and Formatting Instructions for ICML 2025}

\begin{document}

\twocolumn[
\icmltitle{CopyJudge: Automated Copyright Infringement Identification and Mitigation in Text-to-Image Diffusion Models} % or CopyJudge: Automated Copyright Infringement Identification and Mitigation in Text-to-Image Diffusion with Large Vision-Language Models

% It is OKAY to include author information, even for blind
% submissions: the style file will automatically remove it for you
% unless you've provided the [accepted] option to the icml2025
% package.

% List of affiliations: The first argument should be a (short)
% identifier you will use later to specify author affiliations
% Academic affiliations should list Department, University, City, Region, Country
% Industry affiliations should list Company, City, Region, Country

% You can specify symbols, otherwise they are numbered in order.
% Ideally, you should not use this facility. Affiliations will be numbered
% in order of appearance and this is the preferred way.
\icmlsetsymbol{equal}{*}

\begin{icmlauthorlist}
\icmlauthor{Shunchang Liu}{equal,epfl}
\icmlauthor{Zhuan Shi}{equal,epfl}
\icmlauthor{Lingjuan Lyu}{sony}
\icmlauthor{Yaochu Jin}{wlu}
\icmlauthor{Boi Faltings}{epfl}
%\icmlauthor{}{sch}
%\icmlauthor{}{sch}
\end{icmlauthorlist}

\icmlaffiliation{epfl}{EPFL, Switzerland}
\icmlaffiliation{sony}{Sony AI}
\icmlaffiliation{wlu}{Westlake University, China}

\icmlcorrespondingauthor{Shunchang Liu}{shunchang.liu@epfl.ch}
\icmlcorrespondingauthor{Zhuan Shi}{zhuan.shi@epfl.ch}
\icmlcorrespondingauthor{Boi Faltings}{boi.faltings@epfl.ch}
 

% You may provide any keywords that you
% find helpful for describing your paper; these are used to populate
% the "keywords" metadata in the PDF but will not be shown in the document
\icmlkeywords{Machine Learning, ICML}

\vskip 0.3in
]

% this must go after the closing bracket ] following \twocolumn[ ...

% This command actually creates the footnote in the first column
% listing the affiliations and the copyright notice.
% The command takes one argument, which is text to display at the start of the footnote.
% The \icmlEqualContribution command is standard text for equal contribution.
% Remove it (just {}) if you do not need this facility.

%\printAffiliationsAndNotice{}  % leave blank if no need to mention equal contribution
\printAffiliationsAndNotice{\icmlEqualContribution} % otherwise use the standard text.

Large language model (LLM)-based agents have shown promise in tackling complex tasks by interacting dynamically with the environment. 
Existing work primarily focuses on behavior cloning from expert demonstrations and preference learning through exploratory trajectory sampling. However, these methods often struggle in long-horizon tasks, where suboptimal actions accumulate step by step, causing agents to deviate from correct task trajectories.
To address this, we highlight the importance of \textit{timely calibration} and the need to automatically construct calibration trajectories for training agents. We propose \textbf{S}tep-Level \textbf{T}raj\textbf{e}ctory \textbf{Ca}libration (\textbf{\model}), a novel framework for LLM agent learning. 
Specifically, \model identifies suboptimal actions through a step-level reward comparison during exploration. It constructs calibrated trajectories using LLM-driven reflection, enabling agents to learn from improved decision-making processes. These calibrated trajectories, together with successful trajectory data, are utilized for reinforced training.
Extensive experiments demonstrate that \model significantly outperforms existing methods. Further analysis highlights that step-level calibration enables agents to complete tasks with greater robustness. 
Our code and data are available at \url{https://github.com/WangHanLinHenry/STeCa}.
\section{Introduction}

Despite the remarkable capabilities of large language models (LLMs)~\cite{DBLP:conf/emnlp/QinZ0CYY23,DBLP:journals/corr/abs-2307-09288}, they often inevitably exhibit hallucinations due to incorrect or outdated knowledge embedded in their parameters~\cite{DBLP:journals/corr/abs-2309-01219, DBLP:journals/corr/abs-2302-12813, DBLP:journals/csur/JiLFYSXIBMF23}.
Given the significant time and expense required to retrain LLMs, there has been growing interest in \emph{model editing} (a.k.a., \emph{knowledge editing})~\cite{DBLP:conf/iclr/SinitsinPPPB20, DBLP:journals/corr/abs-2012-00363, DBLP:conf/acl/DaiDHSCW22, DBLP:conf/icml/MitchellLBMF22, DBLP:conf/nips/MengBAB22, DBLP:conf/iclr/MengSABB23, DBLP:conf/emnlp/YaoWT0LDC023, DBLP:conf/emnlp/ZhongWMPC23, DBLP:conf/icml/MaL0G24, DBLP:journals/corr/abs-2401-04700}, 
which aims to update the knowledge of LLMs cost-effectively.
Some existing methods of model editing achieve this by modifying model parameters, which can be generally divided into two categories~\cite{DBLP:journals/corr/abs-2308-07269, DBLP:conf/emnlp/YaoWT0LDC023}.
Specifically, one type is based on \emph{Meta-Learning}~\cite{DBLP:conf/emnlp/CaoAT21, DBLP:conf/acl/DaiDHSCW22}, while the other is based on \emph{Locate-then-Edit}~\cite{DBLP:conf/acl/DaiDHSCW22, DBLP:conf/nips/MengBAB22, DBLP:conf/iclr/MengSABB23}. This paper primarily focuses on the latter.

\begin{figure}[t]
  \centering
  \includegraphics[width=0.48\textwidth]{figures/demonstration.pdf}
  \vspace{-4mm}
  \caption{(a) Comparison of regular model editing and EAC. EAC compresses the editing information into the dimensions where the editing anchors are located. Here, we utilize the gradients generated during training and the magnitude of the updated knowledge vector to identify anchors. (b) Comparison of general downstream task performance before editing, after regular editing, and after constrained editing by EAC.}
  \vspace{-3mm}
  \label{demo}
\end{figure}

\emph{Sequential} model editing~\cite{DBLP:conf/emnlp/YaoWT0LDC023} can expedite the continual learning of LLMs where a series of consecutive edits are conducted.
This is very important in real-world scenarios because new knowledge continually appears, requiring the model to retain previous knowledge while conducting new edits. 
Some studies have experimentally revealed that in sequential editing, existing methods lead to a decrease in the general abilities of the model across downstream tasks~\cite{DBLP:journals/corr/abs-2401-04700, DBLP:conf/acl/GuptaRA24, DBLP:conf/acl/Yang0MLYC24, DBLP:conf/acl/HuC00024}. 
Besides, \citet{ma2024perturbation} have performed a theoretical analysis to elucidate the bottleneck of the general abilities during sequential editing.
However, previous work has not introduced an effective method that maintains editing performance while preserving general abilities in sequential editing.
This impacts model scalability and presents major challenges for continuous learning in LLMs.

In this paper, a statistical analysis is first conducted to help understand how the model is affected during sequential editing using two popular editing methods, including ROME~\cite{DBLP:conf/nips/MengBAB22} and MEMIT~\cite{DBLP:conf/iclr/MengSABB23}.
Matrix norms, particularly the L1 norm, have been shown to be effective indicators of matrix properties such as sparsity, stability, and conditioning, as evidenced by several theoretical works~\cite{kahan2013tutorial}. In our analysis of matrix norms, we observe significant deviations in the parameter matrix after sequential editing.
Besides, the semantic differences between the facts before and after editing are also visualized, and we find that the differences become larger as the deviation of the parameter matrix after editing increases.
Therefore, we assume that each edit during sequential editing not only updates the editing fact as expected but also unintentionally introduces non-trivial noise that can cause the edited model to deviate from its original semantics space.
Furthermore, the accumulation of non-trivial noise can amplify the negative impact on the general abilities of LLMs.

Inspired by these findings, a framework termed \textbf{E}diting \textbf{A}nchor \textbf{C}ompression (EAC) is proposed to constrain the deviation of the parameter matrix during sequential editing by reducing the norm of the update matrix at each step. 
As shown in Figure~\ref{demo}, EAC first selects a subset of dimension with a high product of gradient and magnitude values, namely editing anchors, that are considered crucial for encoding the new relation through a weighted gradient saliency map.
Retraining is then performed on the dimensions where these important editing anchors are located, effectively compressing the editing information.
By compressing information only in certain dimensions and leaving other dimensions unmodified, the deviation of the parameter matrix after editing is constrained. 
To further regulate changes in the L1 norm of the edited matrix to constrain the deviation, we incorporate a scored elastic net ~\cite{zou2005regularization} into the retraining process, optimizing the previously selected editing anchors.

To validate the effectiveness of the proposed EAC, experiments of applying EAC to \textbf{two popular editing methods} including ROME and MEMIT are conducted.
In addition, \textbf{three LLMs of varying sizes} including GPT2-XL~\cite{radford2019language}, LLaMA-3 (8B)~\cite{llama3} and LLaMA-2 (13B)~\cite{DBLP:journals/corr/abs-2307-09288} and \textbf{four representative tasks} including 
natural language inference~\cite{DBLP:conf/mlcw/DaganGM05}, 
summarization~\cite{gliwa-etal-2019-samsum},
open-domain question-answering~\cite{DBLP:journals/tacl/KwiatkowskiPRCP19},  
and sentiment analysis~\cite{DBLP:conf/emnlp/SocherPWCMNP13} are selected to extensively demonstrate the impact of model editing on the general abilities of LLMs. 
Experimental results demonstrate that in sequential editing, EAC can effectively preserve over 70\% of the general abilities of the model across downstream tasks and better retain the edited knowledge.

In summary, our contributions to this paper are three-fold:
(1) This paper statistically elucidates how deviations in the parameter matrix after editing are responsible for the decreased general abilities of the model across downstream tasks after sequential editing.
(2) A framework termed EAC is proposed, which ultimately aims to constrain the deviation of the parameter matrix after editing by compressing the editing information into editing anchors. 
(3) It is discovered that on models like GPT2-XL and LLaMA-3 (8B), EAC significantly preserves over 70\% of the general abilities across downstream tasks and retains the edited knowledge better.
\section{Related Work}
\label{related}
\textbf{Copyright infringement in text-to-image models.}
Recent research \cite{carlini2023extracting, somepalli2023diffusion, somepalli2023understanding, gu2023memorization, wang2024replication, wen2024detecting, chiba2024probabilistic} highlights the potential for copyright infringement in text-to-image models. These models are trained on vast datasets that often include copyrighted material, which could inadvertently be memorized by the model during training \cite{vyas2023provable, ren2024copyright}. Additionally, several studies have pointed out that synthetic images generated by these models might violate IP rights due to the inclusion of elements or styles that resemble copyrighted works \cite{poland2023generative, wang2024stronger}. Specifically, models like stable diffusion \cite{Rombach_2022_CVPR} may generate images that bear close resemblances to copyrighted artworks, thus raising concerns about IP infringement \cite{shi2024rlcp}. 

\textbf{Image infringement detection and mitigation.}
The current mainstream infringing image detection methods primarily measure the distance or invariance in pixel or embedding space \cite{carlini2023extracting, somepalli2023diffusion, shi2024rlcp, wang2021bag, wang2024image}. For example, \citeauthor{carlini2023extracting} uses the $L_2$ norm to retrieve memorized images. \citeauthor{somepalli2023diffusion} use SSCD \cite{pizzi2022self}, which learns the invariance of image transformations to identify memorized prompts by comparing the generated images with the original training ones. \citeauthor{zhang2018unreasonable} compare image similarity using the LPIPS distance, which aligns with human perception but has limitations in capturing certain nuances. \citeauthor{wang2024image} transform the replication level of each image replica pair into a probability density function. Studies \cite{wen2024detecting, wang2024evaluating} have shown that these methods have lower generalization capabilities and lack interpretability because they do not fully align with human judgment standards. For copyright infringement mitigation, the current approaches mainly involve machine unlearning to remove the model's memory of copyright information \cite{bourtoule2021machine, nguyen2022survey, kumari2023ablating, zhang2024forget} or deleting duplicated samples from the training data \cite{webster2023duplication, somepalli2023understanding}. These methods often require additional model training. On the other hand, \citeauthor{wen2024detecting} have revealed the overfitting of memorized samples to specific input texts and attempt to modify prompts to mitigate the generation of memorized data. Similarly, \citeauthor{wang2024evaluating} leverage LVLMs to detect copyright information and use this information as negative prompts to suppress the generation of infringing images.














\section{LVLM for Infringement Identification}
\label{identification}
\begin{figure*}[!ht]
    \centering
    \includegraphics[width=\linewidth]{figs/1.pdf}
    \vspace{-6mm}
    \caption{CopyJudge: an automated abstraction-filtering-comparison framework for image copyright infringement identification. This LVLM-based framework automatically decides whether a generated image infringes copyright by combining copyright expression extraction with infringement determination through multi-LVLM debate.}
    \label{fig:iden}
    \vspace{-3mm}
\end{figure*}

\subsection{Problem Formulation}
Our goal is to determine whether an image infringes the copyright of a known copyrighted image. Based on U.S. law \cite{roth_greeting_cards_wikipedia} and similar laws in other countries, given an image $x$ created with access to the copyrighted image $x_{cr}$, if $x$ and $x_{cr}$ exhibit \textit{substantial similarity}, then $x$ is deemed to infringe the copyright of $x_{cr}$. 
% That is,
% \begin{equation}
%     \text{IsInfringement}(x) = \text{IsSubstantiallySimilar}
% (x, x_\text{cr}),
% \end{equation}
% where $x \in \text{Access}(x_\text{cr})$.

Motivated by this, we aim to establish a substantial similarity identification model $f$, which takes 
$x$ and $x_{cr}$ as inputs and outputs a similarity score $s$. When $s$ exceeds a threshold $\gamma$, we determine that $x$ infringes on $x_{cr}$. This can be defined as:
\begin{equation}
\text{IsInfringement}(x) = \mathbb{I}(f(x, x_{{cr}}) > \gamma),
\end{equation}
where $x$ represents the AI-generated image, and $x_{cr}$ is the corresponding copyrighted image. 

% However, there is no unified understanding of the definition of substantial similarity. In this paper, we focus on three aspects of similarity:
% \begin{itemize}
%     \item \textbf{Memorization}: \citeauthor{carlini2023extracting} found that generative models may almost entirely replicate content from training data, which we consider the strictest form of copyright infringement.
%     \item \textbf{Multilevel Similarity}: \citeauthor{wang2024image} classified image similarity into six levels (0-5) through manual labeling, with varying degrees of similarity between different levels. We adopt this approach and consider a similarity level of 4 or higher as infringement.
%     \item \textbf{IP Resemblance}: We refer to \citeauthor{wang2024evaluating}'s definition, which considers similar copyrighted characters appearing in the image.
% \end{itemize}

\subsection{Abstraction-Filtering-Comparison Framework}
For the process of identifying substantial similarity, we refer to the \textit{abstraction-filtering-comparison} test method \cite{abramson2002promoting}, which has been widely adopted in practical court rulings on infringement cases, and propose an automated infringement identification framework using large vision-language models, as seen in Figure \ref{fig:iden}. In the \textit{copyright expression extraction stage}, we break down images into different elements (such as composition and color patterns), and filter out non-copyrightable parts, leaving copyrighted portions to assess substantial similarity. In the next \textit{copyright infringement determination} stage, multiple LVLMs debate and score the similarity of images given the copyrighted elements, with a final decision made by a meta-judge LVLM based on their consensus. Human priors are injected into the models through few-shot demonstrations to better align with human preferences. 

\textbf{Copyright expression extraction via image-to-text abstraction and filtration.} The process of distinguishing between the fundamental ideas and the specific expressions of an image is a crucial step in determining copyright protection. The core idea of our method is to break down the image into different layers or components, in order to examine the true copyright elements.

First, during the abstraction phase, the image is analyzed and decomposed into its fundamental building blocks. This involves identifying the core elements that contribute to the overall meaning or aesthetic of the image, such as composition, themes, color palette, or other unique visual elements. We can implement this using an LVLM $\pi_{abs}$, defined as:
\begin{equation}
\pi_{abs}(x, x_{{cr}}, p_{abs}) \to (z, z_{cr}),
\end{equation}
where $z$ and $z_\text{cr}$ represent the expressions of $x$ and $x_{{cr}}$ in text after decoupling, respectively.
%\begin{tcolorbox}[colback=gray!10, colframe=gray!80, rounded corners, boxrule=0.1mm]
%\small \textbf{Abstraction Prompt}: ``Please help decompose the given two images into their essential elements and abstract concepts. Identify key components such as composition, themes, color palette, textures, and any unique visual elements. Describe each component in detail."
%\end{tcolorbox}
The goal is to abstract away the superficial features of the image that do not hold significant creative value and instead focus on the underlying concepts that convey the essence of the work.

The next step is filtering. At this stage, elements of the image that are not eligible for copyright protection are removed from consideration. These can include generic concepts, common patterns, functional aspects, or elements derived from public domain sources.
%\begin{tcolorbox}[colback=gray!10, colframe=gray!80, rounded corners, boxrule=0.1mm]
%\small \textbf{Filtering Prompt}: ``Based on the image decomposition, please identify which elements are likely unprotectable under copyright law. These could include generic concepts, common patterns, functional aspects, or elements derived from public domain sources. Then, specify the remaining unique elements that are original and creative."
%\end{tcolorbox}
For example, standard design patterns or commonly used motifs in artwork may not be deemed original enough to warrant protection under copyright law. This process could be defined as:
\begin{equation}
\pi_{fil}(z, z_{{cr}}, p_{fil}) \to (z^c, z^c_{cr}),
\end{equation}
where $z^c$ and $z^c_{cr}$ are the filtered copyright expressions, and $\pi_{fil}$ is another independent LVLM. Filtering helps ensure that only the truly creative, original aspects of the image are preserved for comparison.

The following step is to conduct a comparison of the remaining abstracted elements to assess the degree of similarity. This process helps determine whether the image in question constitutes a derivative work or if it has enough original expression to qualify for copyright protection. To ensure the reliability of the results, we used multi-LVLM debates to perform the comparison and make the final infringement determination.

\textbf{Copyright infringement determination via multi-LVLM comparison.} 
Many studies \cite{du2023improving, chan2023chateval, lakara2024mad, liu2024groupdebate} have shown that multi-agent debate can effectively improve the reliability of responses generated by large models. At this stage, we utilize $N$ LVLMs $\pi_i (i = 1, 2, …, N)$ to communicate with each other and evaluate overall similarity. In addition, to align with human judgment preferences, we employ few-shot in-context learning \cite{dong2022survey, agarwal2024many} by presenting multiple pairs of images scored by humans as references. Specifically, for a single agent $\pi_i$, given inputs including $x$, $x_{cr}$, the filtered copyright expressions $z^c$, $z^c_{cr}$,  an instruction $p_i$, and the set of human reference images $D_h$ and their corresponding score set $S_h$, the agent is required to output a score $s_i \in [0,1]$, confidence $c_i \in [0,1]$, and supporting rationale $r_i$. Specifically, the process can be represented as:
\begin{equation}
\pi_i(x, x_{{cr}}, z^c, z^c_{cr}, p_i, D_h, S_h) \to (s_i, c_i, r_i).
\end{equation}
Following \cite{du2023improving}, we adopt a fully connected synchronous communication debate approach, where each LVLM receives the responses ($s$, $c$, $r$) from the other $N-1$ LVLMs before making the next judgment. This creates a dynamic feedback loop that strengthens the reliability and depth of the analysis, as models adapt their evaluations based on new insights presented by their peers. Each LVLM can adjust its score based on the responses from the other LVLMs or keep it unchanged. We use the following consistency judgment criterion: 
\begin{equation}
\left| s_i - s_j \right| \leq \alpha \quad \forall i, j \in \{1, 2, \dots, N\}.
\end{equation}
If the difference in scores between all LVLMs is less than $\alpha$, we consider that all models have reached a consensus. Additionally, to avoid the models getting stuck in a meaningless loop, we set the maximum number of debate rounds to $M$.

After the debate, the agreed-upon results will be input into an independent meta-judge LVLM $\pi_{f}$, which synthesizes the results to give the final score on whether substantial similarity has occurred, defined as: 
\begin{equation}
\pi_{f}(x, x_{{cr}}, z^c, z^c_{cr}, p_{f}, D_h, S_h, S_m, C_m, R_m) \to (s_{f}, c_{f}, r_{f}),
\end{equation}
where $S_m$, $C_m$, and $R_m$ represent the set of scores, confidence levels, and rationales from $N$ LVLMs after reaching consensus in the $m$-th ($m \leq M$) debate. By combining the strengths of individual agents and iterative debating, the approach could achieve a reliable assessment of visual similarity.
Furthermore, we can determine whether the generated image constitutes infringement based on whether the final similarity score exceeds a specific threshold $\gamma$:
\begin{equation}
\text{IsInfringement}(x) = \mathbb{I}(s_{f}) > \gamma.
\end{equation}
The whole two-stage process ensures a comprehensive and reliable evaluation by integrating multiple perspectives and rigorous analysis. The complete algorithm and instruction prompts can be found in appendix \ref{algA} and \ref{promptA}.





\section{LVLM for Infringement Mitigation}
\label{mitigation}
\begin{figure*}[!ht]
    \centering
    \includegraphics[width=\linewidth]{figs/2.pdf}
    \vspace{-6mm}
    \caption{Copyright infringement mitigation framework with controlling input prompts and latent noise. For prompt control, the defense agent iteratively optimizes prompts to mitigate copyright risks in generated images given the CopyJudge's feedback. For latent control, it integrates a RL-based defense agent with reward-guided latent sampling to reduce the predicted infringement scores.}
    \label{fig:mit}
    \vspace{-3mm}
\end{figure*}
Based on the identification results, we attempt to control the generation model to ensure its outputs do not infringe on copyright. As shown in Figure \ref{fig:mit}, we will discuss two methods separately depending on the control target: prompt control and latent control.
\subsection{Mitigation via LVLM-based Prompt Control}
\citeauthor{wen2024detecting} have proved that slightly modifying the overfitted prompts can effectively avoid generating memorized images. To achieve automated prompt modification aimed at eliminating infringement expressions, we use an LVLM as a prompt optimizer to iteratively adjust the infringing prompt until the final score falls below a threshold $\gamma$. Formally, given the source image $x_{{cr}}$, the generated image $x^t$ and the corresponding prompt $p^t$ at round $t$, control condition $p^c$, historical judgment score $s^t_{f}$, confidence $c^t_{f}$, and rationale $r^t_{f}$, the prompt modifier $\pi_{p}$ is tasked with providing the prompt for the next round, that is: 
\begin{equation}
\pi_{p}(x^t, x_{{cr}}, p^t, p^c, s^t_{f}, c^t_{f}, r^t_{f}) \to p^{t+1}.
\end{equation}
Here, $p^c$ requires the modifier to alter the infringing expression while preserving the original expression as much as possible to avoid generating meaningless images. This mitigation strategy does not require any knowledge of text-to-image models, making it suitable for general black-box scenarios. The algorithm and instruction prompts can be found in appendix \ref{algB} and \ref{promptB}.

\subsection{Mitigation via RL-based Latent Control}
For diffusion models, the output is influenced not only by the prompt but also by the latent noise. Latent noise represents encoded representations of the input that capture essential features in a lower-dimensional space. These latent variables guide the generation process, affecting the finer details of the resulting image. In this section, we propose a reinforcement learning (RL)-based latent control method to mitigate copyright infringement in diffusion-based generative models. Our method involves training an agent to search the input latent variables that yield lower infringement scores, ensuring that the generated outputs do not violate copyright.

Specifically, for latent variable \( z \), we define a policy \( \pi_{\omega} \) parameterized by \( \omega \), allowing us to sample latent noise \( \epsilon \sim \pi_{\omega}(z) \), which follows a Gaussian distribution. The sampled noise \( \epsilon \) is then passed through the pre-trained diffusion decoder \( f \) to produce the image \( x = f(z, \epsilon) \).

To assess the copyright infringement potential of the generated image, we employ our CopyJudge to obtain the infringement score $s_{f}$. Based on this score, we define a reward function:

\begin{equation}
    R(z) = -\log(s_{f}).
\end{equation}

This reward is designed to penalize outputs with higher infringement scores, thus encouraging the generation of non-infringing content. We optimize the parameters \( \omega \) by maximizing the expected reward, \( L(\omega) \), defined as:

\begin{equation}
    L(\omega) = \mathbb{E}_{z \sim \pi_{\omega}}[R(z)].
\end{equation}

The gradient of this objective is computed using the REINFORCE rule \cite{williams1992simple}, which is given by:

\begin{equation}
\nabla_{\omega}L(\omega) = \mathbb{E}_{z \sim \pi_{\omega}}[\nabla_{\omega}\log(\pi_{\omega}) R(z)].
\end{equation}

During the training process, the latent variable $z$ is updated according to the following rule:
\begin{equation}
 z' = z + \beta \epsilon, \quad \epsilon \sim \pi_{\omega}(z),
\end{equation}
where $\beta$ is the step size. We further conduct normalization for the latent variables to maintain stability and prevent extreme deviations. This  RL-based approach allows the agent to explore variations in the latent space, thereby improving its ability to generate non-infringing content. The detailed algorithm can be found in appendix \ref{algB}.

\section{Experiments}
\label{sec:sec-Experiment}

In this section, we present a comprehensive evaluation of our proposed algorithms against three different baseline algorithms on both synthetic and real-world datasets. 



\subsection{Baseline Algorithms}


\begin{figure*}[t]
	\centering
		\begin{tabular}{cccc}
	 			\multicolumn{4}{c}{\hspace{-8mm} \includegraphics[height=2.8mm]{figures/socod_legend.eps}}   \\[-1mm]
                 \includegraphics[height=26mm]{figures/uniform-max-error.eps} &
			 \includegraphics[height=26mm]{figures/normal-max-error.eps} &
			\ \includegraphics[height=26mm]{figures/multi-modal-max-error.eps} &
			 \includegraphics[height=26mm]{figures/hpmax-max-error.eps}
			\\[-3mm]
             (a) Uniform Random &
                 (b) Random Noisy &
			 (c) Multimodal Data &
			 (d) HPMaX  \\[-1mm]
		\end{tabular}
		\vspace{-3mm}
		\caption{Maximum Sketch Size vs. Maximum Error.} \label{fig:max-error}
		\vspace{-1mm}
\end{figure*}

\begin{figure*}[t]
	\centering
 \vspace{-2mm}
		\begin{tabular}{cccc}
	 	%		\multicolumn{4}{c}{\hspace{-8mm} \includegraphics[height=2.8mm]{figures/socod_legend.pdf}}   \\
			\includegraphics[height=26mm]{figures/uniform-avg-error.eps} &
             \includegraphics[height=26mm]{figures/normal-avg-error.eps} &
			 \includegraphics[height=26mm]{figures/multi-modal-avg-error.eps} &
			 \includegraphics[height=26mm]{figures/hpmax-avg-error.eps}
			\\[-3mm]
                 (a) Uniform Random &
                (b) Random Noisy &
			 (c) Multimodal Data &
			 (d) HPMaX  \\[-1mm]
		\end{tabular}
		\vspace{-3mm}
		\caption{Maximum Sketch Size vs. Average Error.} \label{fig:avg-error}
		% \vspace{-1mm}
\end{figure*}

\begin{figure*}[t]
	\centering
 \vspace{-2mm}
		\begin{tabular}{cccc}
	 	%		\multicolumn{4}{c}{\hspace{-8mm} \includegraphics[height=2.8mm]{figures/socod_legend.pdf}}   \\
			 \includegraphics[height=26mm]{figures/uniform-sketch-size.eps} &
            		 \includegraphics[height=26mm]{figures/normal-sketch-size.eps} &
			 \includegraphics[height=26mm]{figures/multi-modal-sketch-size.eps} &
			 \includegraphics[height=26mm]{figures/hpmax-sketch-size.eps}
			\\[-3mm]
             (a) Uniform Random &
                (b) Random Noisy &
			 (c) Multimodal Data &
			(d) HPMaX  \\[-1mm]
		\end{tabular}
		\vspace{-2mm}
		\caption{$\log_{10}(1/\epsilon)$ vs. Maximum Sketch Size ($\log_{10}(\frac{1}{0.25}) \approx 0.6$, \text{and} $\log_{10}(\frac{1}{0.016}) \approx 1.8$).}\label{fig:sketch-size}
		\vspace{-1mm}
\end{figure*}

\htitle{Sampling method.} For the AMM problem, the algorithm samples a small proportion of the matrices. Specifically, each pair of columns $(x_i,y_i)$ is assigned to a priority $\rho=u^{1/(\xinorm\yinorm)}$, where $u$ is uniformly sampled from the interval $(0,1)$ \cite{efraimidis2006weighted}. This priority-based sampling strategy ensures that columns with larger norms are more likely to be selected, thereby preserving the most significant contributions to the matrix product. To achieve an $\epsilon$-approximation guarantee, the algorithm requires $O(\frac{1}{\epsilon^2})$ independent samples selected based on the highest priorities \cite{drineas2006fast, efraimidis2006weighted}. To extend the priority sampling on the sliding window, we use the technique from \cite{babcock2001sampling}, leading to a space complexity of $O(\frac{d_x+d_y}{\epsilon^2}\log{N})$ for the normalized model and $O(\frac{d_x+d_y}{\epsilon^2}\log{NR})$ for general unnormalized model.

\htitle{DI-COD.} DI-COD applied the Dyadic Interval approach \cite{arasu2004approximate} to Co-Occurring Directions, maintaining a hierarchical structure with $L=\log{\frac{R}{\epsilon}}$ parallel levels, each of which contains a dynamic number of blocks. For $i$-th level, the window is segmented into at most $2^{L-i+1}$ block, and each block maintains a COD sketch. The space cost for DI-COD is $O(\frac{(d_x+d_y)R}{\epsilon}\log^2{\frac{R}{\epsilon}})$. 

\htitle{EH-COD.} Exponential Histogram Co-occurring Directions (EH-COD) combines the Exponential Histograms technique \cite{DatarGIM02} and incorporates the COD algorithm for efficiently approximating matrix multiplication within the sliding window model. The space cost for EH-COD is $O(\frac{d_x+d_y}{\epsilon^2}\log{\epsilon NR})$.



\subsection{Experiments Setup}
\htitle{Datasets.} Experiments are conducted on both synthetic and real-world datasets widely used in matrix multiplication \cite{YaoLCWC24,YeLZ16,MrouehMG17,GhashamiDP14,KangKK20}. All datasets are \emph{unnormalized}.
The details are listed below:
\begin{itemize}[leftmargin=10pt]
\item \textbf{Uniform Random \cite{YaoLCWC24,YeLZ16}.} We generate two random matrices: one of size $2000 \times 10000$ and another of size $1000 \times 10000$. The entries of both matrices are drawn uniformly at random from the interval $[0, 1)$. The window size for this dataset is $N = 4000$.
    
\item \textbf{Random Noisy \cite{MrouehMG17,GhashamiDP14}.} 
We generate the input matrix $\boldsymbol{X}^T = \boldsymbol{SDU} + \boldsymbol{F} / \zeta \in \mathbb{R}^{n \times d_x}$. Here, the term $\boldsymbol{SDU}$ represents an $m$-dimensional signal, while the other part $\boldsymbol{F} / \zeta$ is a Gaussian noise matrix, with scalar parameter $\zeta$ controlling the noise-to-signal ratio.
Specifically, $\boldsymbol{S}\in\mathbb{R}^{n\times m}$ is a random matrix where each entry is drawn from a standard normal distribution. $\boldsymbol{D}\in\mathbb{R}^{m\times m}$ is a diagonal matrix with entries $\boldsymbol{D}_{i,i}=1-(i-1)/m$, and $\boldsymbol{U}\in\mathbb{R}^{m\times d_x}$ is a random rotation which represents the row space of the signal and satisfies that $\boldsymbol{U}^T\boldsymbol{U}=I_m$. $\boldsymbol{F}$ is again a Gaussian matrix with each entries generated i.i.d. from a normal distribution $N(0,1)$. Matrix $\boldsymbol{Y}$ is generated in the same manner as $\boldsymbol{X}$. We set $d_x = 2000$, $d_y=1000$, $m = 400$, $\zeta = 100$, and the window size $N = 4000$.

 \item \textbf{Multimodal Data \cite{MrouehMG17}.} We study the empirical performance of the algorithms in approximating correlation between images and captions. Following \cite{MrouehMG17}, we consider Microsoft COCO dataset \cite{LinMBHPRDZ14}. For visual features we use the residual CNN Resnet101 \cite{HeZRS16} to generate a feature vector of dimension $d_x = 2048$ for each picture. For text we use the Hierarchical Kernel Sentence Embedding \cite{mroueh2015asymmetrically}, resulting in a feature vector of dimensions $d_y = 3000$. We construct the matrices $\boldsymbol{X}$ and $\boldsymbol{Y}$ with sizes $2048 \times 123287$ and $3000 \times 123287$, respectively, where each column represents a feature vector. The window size is set to $N = 10000$.

 \item \textbf{HPMaX \cite{KangKK20}:} We also include the dataset HPMaX, which is used to test the performance of heterogenous parallel algorithms for matrix multiplication. In this dataset, both of $\boldsymbol{X}$ and $\boldsymbol{Y}$  have size of $16384\times 32768$. The window size $N$ is $10000$.
\end{itemize}


\htitle{Evaluation Metrics.} Recall that our \oursolution achieves the optimal space complexity while providing an $\epsilon$-approximation guarantee. Therefore, we design the experiments to explicitly demonstrate the trade-off between space consumption and empirical accuracy across different datasets. Specifically, we tune the parameters of each algorithm and report both the maximum sketch size and the empirical relative correlation error.  
\begin{itemize}[topsep=0.5mm, partopsep=0pt, itemsep=0pt, leftmargin=10pt] 
    \item \textbf{Maximum sketch size}. This metric is measured by the \textit{maximum} number of column vectors maintained by a matrix sketching algorithm. The maximum sketch size metric represents the peak space cost of a matrix sketching algorithm. 
    \item \textbf{Relative correlation error}. This metric is used to assess the approximation quality of the output matrices. It is defined as $\left\| \boldsymbol{X}_W \boldsymbol{Y}_W^T - \boldsymbol{A}_W \boldsymbol{B}_W^T\right\|_2 /\left\|\boldsymbol{X}_W\right\|_F\left\|\boldsymbol{Y}_W\right\|_F$, where $\boldsymbol{X}_W$ and $\boldsymbol{X}_W$ denotes the exact matrices covered by the current window, and $\boldsymbol{A}_W$ and $\boldsymbol{B}_W$ denotes sketch matrices for $\boldsymbol{X}_W\boldsymbol{Y}_W^T$. 
\end{itemize} 





\subsection{Experimental Results}
We first adjust the error parameter $\epsilon$ for each algorithm to analyze the trade-off between space efficiency and empirical accuracy. Generally, when the error parameter $\epsilon$ decreases, the maximum sketch size increases. As shown in Figures~\ref{fig:max-error}--\ref{fig:avg-error}, we report the maximum sketch size, as well as the maximum and average relative correlation errors, for each algorithm. Both the x-axis and y-axis are displayed on a logarithmic scale to encompass the wide range of values. 

First, we observe that the curve representing our solution \oursolution consistently resides in the lower-left corner compared to other baselines, in terms of both maximum and average errors.  This implies that for a given space cost (i.e., maximum sketch size), our \oursolution consistently produces matrices with much lower correlation errors. Therefore, our solution demonstrates a superior space-error trade-off, aligning with its optimal space complexity as discussed in Section~\ref{sec:unnormalized-setting}. Second, on certain datasets (e.g., Multimodal Data and HPMax), the second-best algorithm, EH-COD, produces results comparable to our solution when the maximum sketch size is small (i.e., the error parameter $\epsilon$ is large). However, the gap between the two curves widens as the maximum sketch size increases (i.e., the error parameter $\epsilon$ decreases). This is also aligned with the theoretical result that the suboptimal space complexity $O(\frac{d_x + d_y}{\epsilon^2}\log{\epsilon NR})$ of EH-COD is outperformed by our optimal complexity $O(\frac{d_x + d_y}{\epsilon}\log{R})$. Finally, we observe that the EH-COD baseline performs better than the DI-COD baseline in almost all cases, which aligns with the observations in \cite{YaoLCWC24}.

Then, we examine the impact of the error parameter on the space cost of each algorithm. We vary the parameter $\epsilon$ and report the maximum value of sketch size. The results are shown in Figure~\ref{fig:sketch-size}.  The curve of our solution \oursolution consistently remains the lowest. This indicates that, for a given error parameter, \oursolution requires the least space, thereby confirming the conclusion of space optimality. One may note that as $\log_{10}(1/\epsilon)$ increases, the space growth of the EH-COD algorithm gradually slows down. This occurs because, as $\epsilon$ decreases, the storage capacity of EH-COD increases, and the entire sketch becomes sufficient to store the entire window without significant COD compression operations. Consequently, the maximum sketch size approaches the window size.


In summary, when space is the primary concern, our \oursolution is the preferred choice, delivering the best accuracy under space constraints compared to all competitors.

\section{Conclusion}

In this paper, we introduce STeCa, a novel agent learning framework designed to enhance the performance of LLM agents in long-horizon tasks. 
STeCa identifies deviated actions through step-level reward comparisons and constructs calibration trajectories via reflection. 
These trajectories serve as critical data for reinforced training. Extensive experiments demonstrate that STeCa significantly outperforms baseline methods, with additional analyses underscoring its robust calibration capabilities.








\newpage

\bibliography{main}
\bibliographystyle{icml2025}


%%%%%%%%%%%%%%%%%%%%%%%%%%%%%%%%%%%%%%%%%%%%%%%%%%%%%%%%%%%%%%%%%%%%%%%%%%%%%%%
%%%%%%%%%%%%%%%%%%%%%%%%%%%%%%%%%%%%%%%%%%%%%%%%%%%%%%%%%%%%%%%%%%%%%%%%%%%%%%%
% APPENDIX
%%%%%%%%%%%%%%%%%%%%%%%%%%%%%%%%%%%%%%%%%%%%%%%%%%%%%%%%%%%%%%%%%%%%%%%%%%%%%%%
%%%%%%%%%%%%%%%%%%%%%%%%%%%%%%%%%%%%%%%%%%%%%%%%%%%%%%%%%%%%%%%%%%%%%%%%%%%%%%%
\newpage
\appendix
\onecolumn
\section{Hard Threshold of EAC}\label{threshhold}
In constructing a weighted-gradient saliency map, the value of \(\gamma\) determines the number of the dimensions we select where important feature anchors are located. As the value of \(\gamma\) increases, the number of selected dimensions decreases, requiring the editing information to be compressed into a smaller space during the compression process. 
During compression, it is desired for the compression space to be as small as possible to preserve the general abilities of the model. However, reducing the compression space inevitably increases the loss of editing information, which reduces the editing performance of the model.
Therefore, to ensure editing performance in a single editing scenario, different values of \(\gamma\) are determined for various models, methods, and datasets. Fifty pieces of knowledge were randomly selected from the dataset, and reliability, generalization, and locality were measured after editing. The averages of these metrics were then taken as a measure of the editing performance of the model.
Table~\ref{value} presents the details of \(\gamma\), while Table~\ref{s} illustrates the corresponding editing performance before and after the introduction of EAC. $P_{x}$ denotes the value below which x\% of the values in the dataset.


\begin{table}[!htb]
\caption{The value of $\gamma$.}
\centering
\resizebox{0.45\textwidth}{!}{
\begin{tabular}{lcccc}
\toprule
\textbf{Datasets} & \textbf{Model} & \textbf{ROME} & \textbf{MEMIT} \\
\midrule
\multirow{2}{*}{\textbf{ZSRE}} & GPT-2 XL & $P_{80}$ & $P_{80}$ \\
 & LLaMA-3 (8B) & $P_{90}$ & $P_{95}$ \\
\midrule
\multirow{2}{*}{\textbf{COUNTERFACT}} & GPT-2 XL & $P_{85}$ & $P_{85}$ \\
 & LLaMA-3 (8B) & $P_{95}$ & $P_{95}$ \\
\bottomrule
\end{tabular}}
\label{value}
\end{table}


\begin{table}[!htb]
\caption{The value of $\gamma$.}
\centering
\resizebox{\textwidth}{!}{%
\begin{tabular}{lccccccccccccc}
\toprule
\multirow{1}{*}{Dataset} & \multirow{1}{*}{Method} & \multicolumn{3}{c}{\textbf{GPT-2 XL}} & \multicolumn{3}{c}{\textbf{LLaMA-3 (8B)}} \\
\cmidrule(lr){3-5} \cmidrule(lr){6-8}
& & \multicolumn{1}{c}{Reliability} & \multicolumn{1}{c}{Generalization} & \multicolumn{1}{c}{Locality} & \multicolumn{1}{c}{Reliability} & \multicolumn{1}{c}{Generalization} & \multicolumn{1}{c}{Locality} \\
\midrule
\multirow{1}{*}{ZsRE} & ROME & 1.0000 & 0.9112 & 0.9661 & 1.0000 & 0.9883 & 0.9600  \\
& ROME-EAC & 1.0000 & 0.8923 & 0.9560 & 0.9933 & 0.9733 & 0.9742  \\
\cmidrule(lr){2-8}
& MEMIT & 0.6928 & 0.5208 & 1.0000 & 0.9507 & 0.9333 & 0.9688  \\
& MEMIT-EAC & 0.6614 & 0.4968 & 0.9971 & 0.9503 & 0.9390 & 0.9767  \\
\midrule
\multirow{1}{*}{CounterFact} & ROME & 1.0000 & 0.4200 & 0.9600 & 1.0000 & 0.3600 & 0.7800  \\
& ROME-EAC & 0.9800 & 0.3800 & 0.9600 & 1.0000 & 0.3200 & 0.8800  \\
\cmidrule(lr){2-8}
& MEMIT & 0.9000 & 0.2200 & 1.0000 & 1.0000 & 0.3800 & 0.9500  \\
& MEMIT-EAC & 0.8000 & 0.1800 & 1.0000 & 1.0000 & 0.3200 & 0.9800  \\
\bottomrule
\end{tabular}%
}
\label{s}
\end{table}

\section{Optimization Details}\label{b}
ROME derives a closed-form solution to achieve the optimization:
\begin{equation}
\text{minimize} \ \| \widehat{W}K - V \| \ \text{such that} \ \widehat{W}\mathbf{k}_* = \mathbf{v}_* \ \text{by setting} \ \widehat{W} = W + \Lambda (C^{-1}\mathbf{k}_*)^T.
\end{equation}

Here \( W \) is the original matrix, \( C = KK^T \) is a constant that is pre-cached by estimating the uncentered covariance of \( \mathbf{k} \) from a sample of Wikipedia text, and \( \Lambda = (\mathbf{v}_* - W\mathbf{k}_*) / ( (C^{-1}\mathbf{k}_*)^T \mathbf{k}_*) \) is a vector proportional to the residual error of the new key-value pair on the original memory matrix.

In ROME, \(\mathbf{k}_*\) is derived from the following equation:
\begin{equation}
\mathbf{k}_* = \frac{1}{N} \sum_{j=1}^{N} \mathbf{k}(x_j + s), \quad \text{where} \quad \mathbf{k}(x) = \sigma \left( W_{f_c}^{(l^*)} \gamma \left( a_{[x],i}^{(l^*)} + h_{[x],i}^{(l^*-1)} \right) \right).
\end{equation}

ROME set $\mathbf{v}_* = \arg\min_z \mathcal{L}(z)$, where the objective $\mathcal{L}(z)$ is:
\begin{equation}
\frac{1}{N} \sum_{j=1}^{N} -\log \mathbb{P}_{G(m_{i}^{l^*}:=z))} \left[ o^* \mid x_j + p \right] + D_{KL} \left( \mathbb{P}_{G(m_{i}^{l^*}:=z)} \left[ x \mid p' \right] \parallel \mathbb{P}_{G} \left[ x \mid p' \right] \right).
\end{equation}

\section{Experimental Setup} \label{detail}

\subsection{Editing Methods}\label{EM}

In our experiments, Two popular editing methods including ROME and MEMIT were selected as baselines.

\textbf{ROME} \cite{DBLP:conf/nips/MengBAB22}: it first localized the factual knowledge at a specific layer in the transformer MLP modules, and then updated the knowledge by directly writing new key-value pairs in the MLP module.

\textbf{MEMIT} \cite{DBLP:conf/iclr/MengSABB23}: it extended ROME to edit a large set of facts and updated a set of MLP layers to update knowledge.

The ability of these methods was assessed based on EasyEdit~\cite{DBLP:journals/corr/abs-2308-07269}, an easy-to-use knowledge editing framework which integrates the released codes and hyperparameters from previous methods.

\subsection{Editing Datasets}\label{dat}
In our experiment, two popular model editing datasets \textsc{ZsRE}~\cite{DBLP:conf/conll/LevySCZ17} and \textsc{CounterFact}~\cite{DBLP:conf/nips/MengBAB22} were adopted.

\textbf{\textsc{ZsRE}} is a QA dataset using question rephrasings generated by back-translation as the equivalence neighborhood.
Each input is a question about an entity, and plausible alternative edit labels are sampled from the top-ranked predictions of a BART-base model trained on \textsc{ZsRE}.

\textbf{\textsc{CounterFact}} accounts for counterfacts that start with low scores in comparison to correct facts. It constructs out-of-scope data by substituting the subject entity for a proximate subject entity sharing a predicate. This alteration enables us to differentiate between superficial wording changes and more significant modifications that correspond to a meaningful shift in a fact. 

\subsection{Metrics for Evaluating Editing Performance}\label{Mediting performance}
\paragraph{Reliability} means that given an editing factual knowledge, the edited model should produce the expected predictions. The reliability is measured as the average accuracy on the edit case:
\begin{equation}
\mathbb{E}_{(x'_{ei}, y'_{ei}) \sim \{(x_{ei}, y_{ei})\}} \mathbf{1} \left\{ \arg\max_y f_{\theta_{i}} \left( y \mid x'_{ei} \right) = y'_{ei} \right\}.
\label{rel}
\end{equation}

\paragraph{Generalization} means that edited models should be able to recall the updated knowledge when prompted within the editing scope. The generalization is assessed by the average accuracy of the model on examples uniformly sampled from the equivalence neighborhood:
\begin{equation}
\mathbb{E}_{(x'_{ei}, y'_{ei}) \sim N(x_{ei}, y_{ei})} \mathbf{1} \left\{ \arg\max_y f_{\theta_{i}} \left( y \mid x'_{ei} \right) = y'_{ei} \right\}.
\label{gen}
\end{equation}

\paragraph{Locality} means that the edited model should remain unchanged in response to prompts that are irrelevant or the out-of-scope. The locality is evaluated by the rate at which the edited model's predictions remain unchanged compared to the pre-edit model.
\begin{equation}
\mathbb{E}_{(x'_{ei}, y'_{ei}) \sim O(x_{ei}, y_{ei})} \mathbf{1} \left\{ f_{\theta_{i}} \left( y \mid x'_{ei} \right) = f_{\theta_{i-1}} \left( y \mid x'_{ei} \right) \right\}.
\label{loc}
\end{equation}

\subsection{Downstream Tasks}\label{pro}

Four downstream tasks were selected to measure the general abilities of models before and after editing:
\textbf{Natural language inference (NLI)} on the RTE~\cite{DBLP:conf/mlcw/DaganGM05}, and the results were measured by accuracy of two-way classification.
\textbf{Open-domain QA} on the Natural Question~\cite{DBLP:journals/tacl/KwiatkowskiPRCP19}, and the results were measured by exact match (EM) with the reference answer after minor normalization as in \citet{DBLP:conf/acl/ChenFWB17} and \citet{DBLP:conf/acl/LeeCT19}.
\textbf{Summarization} on the SAMSum~\cite{gliwa-etal-2019-samsum}, and the results were measured by the average of ROUGE-1, ROUGE-2 and ROUGE-L as in \citet{lin-2004-rouge}.
\textbf{Sentiment analysis} on the SST2~\cite{DBLP:conf/emnlp/SocherPWCMNP13}, and the results were measured by accuracy of two-way classification.

The prompts for each task were illustrated in Table~\ref{tab-prompt}.

\begin{table*}[!htb]
% \small
\centering
\begin{tabular}{p{0.95\linewidth}}
\toprule

NLI:\\
\{\texttt{SENTENCE1}\} entails the \{\texttt{SENTENCE2}\}. True or False? answer:\\

\midrule

Open-domain QA:\\
Refer to the passage below and answer the following question. Passage: \{\texttt{DOCUMENT}\} Question: \{\texttt{QUESTION}\}\\

\midrule

Summarization:\\
\{\texttt{DIALOGUE}\} TL;DR:\\

\midrule


Sentiment analysis:\\
For each snippet of text, label the sentiment of the text as positive or negative. The answer should be exact 'positive' or 'negative'. text: \{\texttt{TEXT}\} answer:\\

\bottomrule
\end{tabular}
\caption{The prompts to LLMs for evaluating their zero-shot performance on these general tasks.}
\label{tab-prompt}
\end{table*}

\subsection{Hyper-parameters for Elastic Net}\label{hy}

In our experiment, we set \(\lambda = 5 \times 10^{-7} \), \(\mu = 5 \times 10^{-1} \) for GPT2-XL\cite{radford2019language} and \(\lambda = 5 \times 10^{-7} \), \(\mu = 1 \times 10^{-3} \) for LLaMA-3 (8B)\cite{llama3}.

\begin{figure*}[!hbt]
  \centering
  \includegraphics[width=0.5\textwidth]{figures/legend_edit.pdf}
  \vspace{-4mm}
\end{figure*}

\begin{figure*}[!hbt]
  \centering
  \subfigure{
  \includegraphics[width=0.23\textwidth]{figures/ROME-GPT2XL-CF-edit.pdf}}
  \subfigure{
  \includegraphics[width=0.23\textwidth]{figures/ROME-LLaMA3-8B-CF-edit.pdf}}
  \subfigure{
  \includegraphics[width=0.23\textwidth]{figures/MEMIT-GPT2XL-CF-edit.pdf}}
  \subfigure{
  \includegraphics[width=0.23\textwidth]{figures/MEMIT-LLaMA3-8B-CF-edit.pdf}}
  \caption{Edited on CounterFact, editing performance of edited models using the ROME~\cite{DBLP:conf/nips/MengBAB22} and MEMIT~\cite{DBLP:conf/iclr/MengSABB23} on GPT2-XL~\cite{radford2019language} and LLaMA-3 (8B)~\cite{llama3}, as the number of edits increases before and after the introduction of EAC.}
  \vspace{-4mm}
  \label{edit-performance-cf}
\end{figure*}

\begin{figure*}[!hbt]
  \centering
  \includegraphics[width=0.75\textwidth]{figures/legend.pdf}
  \vspace{-4mm}
\end{figure*}

\begin{figure*}[!htb]
  \centering
  \subfigure{
  \includegraphics[width=0.23\textwidth]{figures/ROME-GPT2XL-CounterFact.pdf}}
  \subfigure{
  \includegraphics[width=0.23\textwidth]{figures/ROME-LLaMA3-8B-CounterFact.pdf}}
  \subfigure{
  \includegraphics[width=0.23\textwidth]{figures/MEMIT-GPT2XL-CounterFact.pdf}}
  \subfigure{
  \includegraphics[width=0.23\textwidth]{figures/MEMIT-LLaMA3-8B-CounterFact.pdf}}
  \caption{Edited on CounterFact, performance on general tasks using the ROME~\cite{DBLP:conf/nips/MengBAB22} and MEMIT~\cite{DBLP:conf/iclr/MengSABB23} on GPT2-XL~\cite{radford2019language} and LLaMA-3 (8B)~\cite{llama3}, as the number of edits increases before and after the introduction of EAC.}
  \vspace{-4mm}
  \label{task-performance-cf}
\end{figure*}

\section{Experimental Results}\label{app}

\subsection{Results of Editing Performance}\label{cf-performance}
Applying CounterFact as the editing dataset, Figure~\ref{edit-performance-cf} presents the editing performance of the ROME~\cite{DBLP:conf/nips/MengBAB22} and MEMIT~\cite{DBLP:conf/iclr/MengSABB23} methods on GPT2-XL~\cite{radford2019language} and LLaMA-3 (8B)~\cite{llama3}, respectively, as the number of edits increases before and after the introduction of EAC.
The dashed line represents the ROME or MEMIT, while the solid line represents the ROME or MEMIT applying the EAC.


\subsection{Results of General Abilities}\label{cf-ability}
Applying CounterFact as the editing dataset, Figure~\ref{task-performance-cf} presents the performance on general tasks of edited models using the ROME~\cite{DBLP:conf/nips/MengBAB22} and MEMIT~\cite{DBLP:conf/iclr/MengSABB23} methods on GPT2-XL~\cite{radford2019language} and LLaMA-3 (8B)~\cite{llama3}, respectively, as the number of edits increases before and after the introduction of EAC. 
The dashed line represents the ROME or MEMIT, while the solid line represents the ROME or MEMIT applying the EAC.

\subsection{Results of Larger Model}\label{13 B}
To better demonstrate the scalability and efficiency of our approach, we conducted experiments using the LLaMA-2 (13B)~\cite{DBLP:journals/corr/abs-2307-09288}.
Figure~\ref{13B-edit} presents the editing performance of the ROME~\cite{DBLP:conf/nips/MengBAB22} and MEMIT~\cite{DBLP:conf/iclr/MengSABB23} methods on LLaMA-2 (13B)
~\cite{DBLP:journals/corr/abs-2307-09288}, as the number of edits increases before and after the introduction of EAC.
Figure~\ref{13B} presents the performance on general tasks of edited models using the ROME and MEMIT methods on LLaMA-2 (13B), as the number of edits increases before and after the introduction of EAC.
The dashed line represents the ROME or MEMIT, while the solid line represents the ROME or MEMIT applying the EAC.

\begin{figure*}[!hbt]
  \centering
  \includegraphics[width=0.5\textwidth]{figures/legend_edit.pdf}
  \vspace{-4mm}
\end{figure*}

\begin{figure*}[!hbt]
  \centering
  \subfigure{
  \includegraphics[width=0.23\textwidth]{figures/ROME-LLaMA2-13B-ZsRE-edit.pdf}}
  \subfigure{
  \includegraphics[width=0.23\textwidth]{figures/MEMIT-LLaMA2-13B-ZsRE-edit.pdf}}
  \subfigure{
  \includegraphics[width=0.23\textwidth]{figures/ROME-LLaMA2-13B-CF-edit.pdf}}
  \subfigure{
  \includegraphics[width=0.23\textwidth]{figures/MEMIT-LLaMA2-13B-CF-edit.pdf}}
  \caption{Editing performance of edited models using the ROME~\cite{DBLP:conf/nips/MengBAB22} and MEMIT~\cite{DBLP:conf/iclr/MengSABB23} on LLaMA-2 (13B)~\cite{DBLP:journals/corr/abs-2307-09288}, as the number of edits increases before and after the introduction of EAC.}
  \vspace{-4mm}
  \label{13B-edit}
\end{figure*}

\begin{figure*}[!hbt]
  \centering
  \includegraphics[width=0.75\textwidth]{figures/legend.pdf}
  \vspace{-4mm}
\end{figure*}

\begin{figure*}[!htb]
  \centering
  \subfigure{
  \includegraphics[width=0.23\textwidth]{figures/ROME-LLaMA2-13B-ZsRE.pdf}}
  \subfigure{
  \includegraphics[width=0.23\textwidth]{figures/MEMIT-LLaMA2-13B-ZsRE.pdf}}
  \subfigure{
  \includegraphics[width=0.23\textwidth]{figures/ROME-LLaMA2-13B-CounterFact.pdf}}
  \subfigure{
  \includegraphics[width=0.23\textwidth]{figures/MEMIT-LLaMA2-13B-CounterFact.pdf}}
  \caption{Performance on general tasks using the ROME~\cite{DBLP:conf/nips/MengBAB22} and MEMIT~\cite{DBLP:conf/iclr/MengSABB23} on LLaMA-2 (13B)~\cite{DBLP:journals/corr/abs-2307-09288}, as the number of edits increases before and after the introduction of EAC.}
  \vspace{-4mm}
  \label{13B}
\end{figure*}

\section{Analysis of Elastic Net}
\label{FT}
It is worth noting that the elastic net introduced in EAC can be applied to methods beyond ROME and MEMIT, such as FT~\cite{DBLP:conf/emnlp/CaoAT21}, to preserve the general abilities of the model.
Unlike the previously mentioned fine-tuning, FT is a model editing approach. It utilized the gradient to gather information about the knowledge to be updated and applied this information directly to the model parameters for updates.
Similar to the approaches of ROME and MEMIT, which involve locating parameters and modifying them, the FT method utilizes gradient information to directly update the model parameters for editing. Therefore, we incorporate an elastic net during the training process to constrain the deviation of the edited matrix.
Figure~\ref{ft} shows the sequential editing performance of FT on GPT2-XL and LLaMA-3 (8B) before and after the introduction of elastic net.
The dashed line represents the FT, while the solid line represents the FT applying the elastic net.
The experimental results indicate that when using the FT method to edit the model, the direct use of gradient information to modify the parameters destroys the general ability of the model. By constraining the deviation of the edited matrix, the incorporation of the elastic net effectively preserves the general abilities of the model.

\begin{figure*}[t]
  \centering
  \subfigure{
  \includegraphics[width=0.43\textwidth]{figures/legend_FT.pdf}}
\end{figure*}

\begin{figure*}[t]%[!ht]
  \centering
  \subfigure{
  \includegraphics[width=0.22\textwidth]{figures/FT-GPT2XL-ZsRE.pdf}}
  \subfigure{
  \includegraphics[width=0.22\textwidth]{figures/FT-GPT2XL-CounterFact.pdf}}
  \vspace{-2mm}
  \caption{Edited on the ZsRE or CounterFact datasets, the sequential editing performance of FT~\cite{DBLP:conf/emnlp/CaoAT21} and FT with elastic net on GPT2-XL before and after the introduction of elastic net.}
  \vspace{-2mm}
  \label{ft}
\end{figure*}


\end{document}


% This document was modified from the file originally made available by
% Pat Langley and Andrea Danyluk for ICML-2K. This version was created
% by Iain Murray in 2018, and modified by Alexandre Bouchard in
% 2019 and 2021 and by Csaba Szepesvari, Gang Niu and Sivan Sabato in 2022.
% Modified again in 2023 and 2024 by Sivan Sabato and Jonathan Scarlett.
% Previous contributors include Dan Roy, Lise Getoor and Tobias
% Scheffer, which was slightly modified from the 2010 version by
% Thorsten Joachims & Johannes Fuernkranz, slightly modified from the
% 2009 version by Kiri Wagstaff and Sam Roweis's 2008 version, which is
% slightly modified from Prasad Tadepalli's 2007 version which is a
% lightly changed version of the previous year's version by Andrew
% Moore, which was in turn edited from those of Kristian Kersting and
% Codrina Lauth. Alex Smola contributed to the algorithmic style files.
