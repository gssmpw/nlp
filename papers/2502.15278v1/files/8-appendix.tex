\section{Supplementary Information on Infringement Identification}

\subsection{Algorithm}
\label{algA}
In Algorithm \ref{alg:infringement_identification}, we present the detailed procedure for automated infringement identification based on the abstraction-filtering-comparison framework. The algorithm systematically extracts copyrightable expressions from images, filters out non-protectable elements, and employs a multi-agent debate among LVLMs to assess substantial similarity.  

\begin{algorithm}
\caption{CopyJudge: Automated Copyright Infringement Identification via Abstraction-Filtering-Comparison Framework}
\label{alg:infringement_identification}
\begin{algorithmic}[1]
\REQUIRE Image pair $(x, x_{cr})$, LVLMs $\{\pi_\text{abs}, \pi_\text{fil}, \pi_1, \dots, \pi_N, \pi_f\}$, thresholds $\alpha, \gamma$, maximum debate rounds $M$, human reference dataset $D_h$, score set $S_h$
\STATE \textbf{Stage 1: Copyright Expression Extraction}
\STATE Extract abstracted expressions: \newline
    \quad $(z, z_{cr}) \leftarrow \pi_\text{abs}(x, x_{cr}, p_\text{abs})$
\STATE Filter non-copyrightable elements: \newline
    \quad $(z^c, z^c_{cr}) \leftarrow \pi_\text{fil}(z, z_{cr}, p_\text{fil})$

\STATE \textbf{Stage 2: Copyright Infringement Determination}
\FOR{each LVLM $\pi_i$, where $i \in \{1, 2, \dots, N\}$}
    \STATE Compute similarity score, confidence, and rationale:
    \quad $(s_i, c_i, r_i) \leftarrow \pi_i(x, x_{cr}, z^c, z^c_{cr}, p_i, D_h, S_h)$
\ENDFOR

\STATE Initialize debate round counter $m \gets 1$
\WHILE{$m \leq M$}
    \FOR{each LVLM $\pi_i$}
        \STATE Receive responses $\{(s_j, c_j, r_j) | j \neq i\}$
        \STATE Update score based on peer evaluations
    \ENDFOR
    \STATE Check consensus: $\left| s_i - s_j \right| \leq \alpha, \forall i, j$
    \IF{consensus achieved}
        \STATE Break loop
    \ENDIF
    \STATE $m \gets m + 1$
\ENDWHILE

\STATE Meta-judge LVLM final decision:
\STATE $(s_f, c_f, r_f) \leftarrow \pi_f(x, x_{cr}, z^c, z^c_{cr}, p_f, D_h, S_h, S_m, C_m, R_m)$

\STATE \textbf{Determine infringement:}
\STATE $\text{IsInfringement}(x) \gets \mathbb{I}(s_f > \gamma)$

\end{algorithmic}
\end{algorithm}


\subsection{Instructive Prompts Used}
\label{promptA}
The effectiveness of the infringement identification process relies on well-designed prompts. Below are examples of instructive prompts used in our framework:

\begin{itemize}
    \item \textbf{Abstraction Prompt:}  
    ``Please help decompose the given two images into their abstract concepts. Identify key components such as composition, themes, color palette, textures, and any unique visual elements. Describe each component in detail.  
    Ensure the output follows the template format: `Image1: XXX, Image2: XXX'.''

    \item \textbf{Filtration Prompt:}  
    ``Based on the image decomposition, please identify which elements are likely unprotectable under copyright law. These could include generic concepts, common patterns, functional aspects, or elements derived from public domain sources. Then, specify the remaining unique elements that are original and creative.  
    Ensure the output follows the template format: `Image1 Unique Elements: XXX, Image2 Unique Elements: XXX'.''

    \item \textbf{Comparison and Scoring Prompt:}  
    ``Please assign an overall similarity score (0–1) between the two images based on the unique elements. The score should be higher if there are similar elements. Provide a confidence score on a scale of 0 to 1, where 1 indicates very high confidence in the evaluation. Additionally, provide a rationale for whether this case constitutes likely copyright infringement. Ensure the output follows the template format: `Score: [0-1], Confidence: [0-1], Reason: [clear and concise explanation]'.''

    \item \textbf{Agent Feedback Integration Prompt:}  
    ``The following is feedback from other agents: 
    
    \textit{Agent 1 score: S1, confidence: C1, reason: R1}  
    
    \textit{Agent 2 score: S2, confidence: C2, reason: R2}

    ……(if any)
    
    You may adjust your score based on this information or maintain your judgment. Ensure the output follows the template format:  `Score: [0-1], Confidence: [0-1], Reason: [clear and concise explanation]'.''

    \item \textbf{Final Decision Prompt:}  
    ``The original instruction is: [comparison and scoring prompt]. Below are the scores, confidence levels, and rationales from multiple judges. Please combine them to make a final decision.  
    
    Scores: [S1, S2, S3, ……(if any)]
    
    Confidences: [C1, C2, C3, ……(if any)] 
    
    Rationales: [R1, R2, R3, ……(if any)]  
    
    Ensure the output follows the template format: `Score: [0-1], Confidence: [0-1], Reason: [clear and concise explanation]'.''


\end{itemize}


\subsection{Additional Results}
\label{resultsA}
We present some successful cases of copyright infringement identified by our framework in Figure \ref{fig:app1}, while Figure \ref{fig:app2} shows some failure cases. We found that these failures mainly stem from excessive sensitivity to specific compositions and imprecise recognition of real individuals.

\begin{figure*}[!ht]
    \centering
    \includegraphics[width=\linewidth]{figs/6.pdf}
    \vspace{-6mm}
    \caption{Successful copyright infringement identification by CopyJudge.}
    \label{fig:app1}
    \vspace{-3mm}
\end{figure*}
\begin{figure*}[!ht]
    \centering
    \includegraphics[width=\linewidth]{figs/7.pdf}
    \vspace{-6mm}
    \caption{Failure case of copyright infringement identification by CopyJudge.}
    \label{fig:app2}
    \vspace{-3mm}
\end{figure*}

\subsection{Time Consumption}
\label{timeA}
The computational cost of running the CopyJudge is an important consideration. Table~\ref{tab:time} provides the average runtime per sample for different steps in the process. We call the official API for testing and find that a single API call, along with extracting useful information, takes approximately 12s. The main time overhead comes from the multi-agent debate process. We acknowledge that this could be a limitation of our approach, while it allows for higher accuracy. In the future, we aim to improve the deployment and interaction of LVLMs to achieve faster decision-making.

\begin{table}[h]
    \centering
    \begin{footnotesize}
    \begin{tabular}{c|c}
        \hline
        Process Step & Average Time (seconds) \\
        \hline
        Single GPT-4o  & 12.3 \\
        Copyright Expression Extraction (Abstraction + Filtration)  & 13.1 \\
        Copyright Infringement Determination (Multi-LVLM Comparison) & 75.6 \\
        Whole Process & 88.7 \\
        \hline
    \end{tabular}
    \caption{Time analysis of the infringement identification process.}
    \label{tab:time}
    \end{footnotesize}
\end{table}


\section{Supplementary Information on Infringement Mitigation}



\subsection{Algorithm}
\label{algB}
For infringement mitigation, we use two strategies: prompt control and latent control, corresponding to Algorithm \ref{alg:pc} and Algorithm \ref{alg:lc}, respectively.

\subsection{Instructive Prompts and IP Used}
\label{promptB}
We list the Instructive Prompts used below, and in Table \ref{tab:ip}, we list the cartoon characters and artworks used, along with their respective IP ownership by individuals/organizations.

\begin{itemize}
    \item \textbf{Prompt for Initial Attack:} ``Generating a prompt that describes the [IP type (cartoon character / artwork)] [IP Name] in the given image. Ensure the output follows the template format: `Prompt: Generate a [IP type] like [IP name]. XXX'." (For implicit infringement, we delete the IP name.)

    \item \textbf{Modifying a Prompt to Improve Similarity Score (Attack Iteration):}  
    "Adjust the parts of the original prompt of the second image that may cause expressions of distinction in the following rationale, making it more similar to the first image to achieve a higher score. Add more information about the [IP type] in Image 1, and provide more unique expressions specific to the [IP type] in Image 1. You can make any changes as long as they improve the similarity score. 
    
    Original prompt: [Original Prompt]
    
    Judgment results: score-[Score], confidence-[Confidence], reason-[Reason]
    
    Ensure the output follows the template format: `Modified Prompt: Generate a [IP type] like [IP Name]. xxx'." (For implicit infringement, we delete the IP name.)

    \item \textbf{Modifying a Prompt to Reduce Similarity Score (Defense Iteration):}  
    ``Adjust the parts of the original prompt of the second image that may cause expressions of similarity in the following rationale, making it more distinct to the first image to achieve a lower score.
    
    Original prompt: [Original Prompt]
    
   Judgment results: score-[Score], confidence-[Confidence], reason-[Reason]
   
   Try to keep the original prompt unchanged but use less unique expressions specific to Image 1. Ensure the output follows the template format:`'Modified Prompt: xxx'."

\end{itemize}

\begin{algorithm}
\caption{Copyright Infringement Mitigation via Prompt Control}
\label{alg:pc}
\begin{algorithmic}[1]
\REQUIRE Source image $x_{\text{cr}}$, generated image $x^0$, initial prompt $p^0$, control condition $p^c$, LVLM-based prompt modifier $\pi_p$, infringement identification function $f$, threshold $\gamma$, maximum iterations $T$
\STATE Initialize prompt: $p^t \gets p^0$, generated image: $x^t \gets x^0$, iteration counter: $t \gets 0$
\WHILE{$t < T$}
    \STATE Conduct infringement judgement: $s^t, c^t, r^t \gets f(x^t, x_{\text{cr}})$
    \IF{$s^t \leq \gamma$}
        \STATE Break loop
    \ENDIF
    \STATE Generate new prompt: 
    \quad $p^{t+1} \gets \pi_p(x^t, x_{\text{cr}}, p^t, p^c, s^t, c^t, r^t)$
    \STATE Generate new image using $p^{t+1}$: $x^{t+1} \gets \text{T2I}(p^{t+1})$
    \STATE Update iteration counter: $t \gets t + 1$
\ENDWHILE
\STATE \textbf{Return} final prompt $p^t$ and generated image $x^t$
\end{algorithmic}
\end{algorithm}

\begin{algorithm}
\caption{Copyright Infringement Mitigation via Latent Control}
\label{alg:lc}
\begin{algorithmic}[1]
\REQUIRE Initial latent variable $z^0$, source image $x_{\text{cr}}$, generated image $x^0$, initial prompt $p^0$, NN-based policy net $\pi_{\omega}$, infringement identification function $f$, threshold $\gamma$, learning rate $\beta$, maximum iterations $T$
\STATE Initialize latent variable: $z^t \gets z^0$, generated image: $x^t \gets x^0$, iteration counter: $t \gets 0$; Initialize policy parameters $\omega$
\WHILE{$t < T$}
    \STATE Sample latent noise: $\epsilon^t \sim \pi_{\omega}(z^t)$
    \STATE Generate image: $x^t \gets \text{T2I}(z^t, \epsilon^t, p^0)$
    \STATE Conduct infringement judgement: $s^t, c^t, r^t \gets f(x^t, x_{\text{cr}})$
    \IF{$s^t \leq \gamma$}
        \STATE Break loop
    \ENDIF
    \STATE Compute reward: $R(z^t) \gets -\log(s^t)$
    \STATE Compute gradient of the objective and update policy net parameters: 
    \quad $\nabla_{\omega}L(\omega) \gets \mathbb{E}_{z^t \sim \pi_{\omega}}[\nabla_{\omega}\log(\pi_{\omega}) R(z^t)]$
    \STATE Update latent variable: $z^{t+1} \gets z^t + \beta \epsilon^t$
    \STATE Update iteration counter: $t \gets t + 1$
\ENDWHILE
\STATE \textbf{Return} final latent variable $z^t$ and generated image $x^t$
\end{algorithmic}
\end{algorithm}

\begin{table}[h]
    \centering
    \begin{footnotesize}
    \begin{tabular}{c|c}
         \hline
    \textbf{Author/Creator} & \textbf{IP Name} \\
    \hline
    Disney & Elsa, Mickey Mouse \\
    Nickelodeon / Paramount Global & SpongeBob SquarePants \\
    Sanrio & Hello Kitty \\
    Marvel / Disney & Iron Man, Hulk, Spider-Man \\
    DC Comics / Warner Bros. & Batman \\
    Nintendo & Super Mario \\
    Akira Toriyama (Toei Animation / Shueisha) & Goku (Dragon Ball) \\
    \hline
    Leonardo da Vinci & Mona Lisa \\
    Vincent van Gogh & Starry Night \\
    Edvard Munch & The Scream \\
    Sandro Botticelli & The Birth of Venus \\
    Johannes Vermeer & The Girl with a Pearl Earring \\
    Gustav Klimt & The Kiss \\
    Eugène Delacroix & Liberty Leading the People \\
    Grant Wood & American Gothic \\
    Edward Hopper & Nighthawks \\
    Raphael & The School of Athens \\
    \hline
    \end{tabular}
    \caption{Summary of IPs by author or creator.}
    \label{tab:ip}
    \end{footnotesize}
\end{table}

\subsection{Additional Results}
\label{resultB}
\begin{figure*}[!ht]
    \centering
    \includegraphics[width=\linewidth]{figs/8.pdf}
    \vspace{-6mm}
    \caption{Generated images and corresponding prompts for copyright infringement mitigation.}
    \label{fig:app3}
    \vspace{-3mm}
\end{figure*}
Figure \ref{fig:app3} presents additional results using our prompt control and latent control for mitigating infringement. It can be observed that our approach effectively prevents the generation of infringing images while ensuring closeness to the original expressions.

\subsection{Time Consumption}
\label{timeB}
Table \ref{tab:time2} presents the time consumption for a single mitigation iteration. As shown, the majority of the time is spent on the infringement identification phase (detailed results can be found in Table \ref{tab:time}). In the future, we will focus on improving the efficiency of both identification and mitigation processes.

\begin{table}[h]
    \centering
    \begin{footnotesize}
    \begin{tabular}{c|c}
        \hline
        Process Step & Average Time (seconds) \\
        \hline
        Infringement Identification  & 88.7 \\
        Latent Control  & 92.5 \\
        Prompt Control  & 101.7\\
        Latent Control + Prompt Control & 115.5 \\
        \hline
    \end{tabular}
    \caption{Time analysis of the infringement mitigation process.}
    \label{tab:time2}
    \end{footnotesize}
\end{table}

