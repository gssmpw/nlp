\begin{abstract}
%Text-to-image models have been found to sometimes memorize and reproduce content from their training data, raising concerns about potential copyright infringement. 
%However, traditional methods, such as manually designed metrics (e.g., $L_2$ norm) or trained detection models, often lack interpretability and do not align well with judicial judgments.
Assessing whether AI-generated images are substantially similar to copyrighted works is a crucial step in resolving copyright disputes. In this paper, we propose \textit{CopyJudge}, an automated copyright infringement identification framework that leverages large vision-language models (LVLMs) to simulate practical court processes for determining substantial similarity between copyrighted images and those generated by text-to-image diffusion models. Specifically, we employ an abstraction-filtration-comparison test framework with multi-LVLM debate to assess the likelihood of infringement and provide detailed judgment rationales.
Based on the judgments, we further introduce a general LVLM-based mitigation strategy that automatically optimizes infringing prompts by avoiding sensitive expressions while preserving the non-infringing content. Besides, our approach can be enhanced by exploring non-infringing noise vectors within the diffusion latent space via reinforcement learning, even without modifying the original prompts.
Experimental results show that our identification method achieves comparable state-of-the-art performance, while offering superior generalization and interpretability across various forms of infringement, and that our mitigation method could more effectively mitigate memorization and IP infringement without losing non-infringing expressions.
%Our automated identification method, tested across various forms of infringement, achieves performance comparable to state-of-the-art methods while offering superior generalization and interpretability. Additionally, evaluations on memorization and IP infringement mitigation demonstrate that our approach more effectively eliminates infringement while keeping low alterations to the original content.

\end{abstract}