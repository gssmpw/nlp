\section{Limitations and Future Work}
\label{limitation}
While our method shows promise, its performance is currently constrained by the availability of labeled data, making it challenging to fully evaluate its alignment with human judgment. To address this, we believe that developing a more comprehensive dataset containing detailed human judgment criteria is crucial. In addition, we are looking forward to stronger attacks to test the robustness of our mitigation strategies. Improving the retrieval of infringing images will also provide valuable insights into the relationship between generative models and copyrighted content. Lastly, although our approach does not require a specific LVLM, we plan to explore other models beyond GPT in advancing copyright protection in the future.

\section{Conclusion}
\label{conclusion}
We present \textit{CopyJudge}, an innovative framework for automating the identification of copyright infringement in text-to-image diffusion models. By leveraging abstraction-filtration-comparison test and multi-LVLM debates, our approach could effectively evaluate the substantial similarity between generated and copyrighted images, providing clear and interpretable judgments. Additionally, our LVLM-based mitigation strategy helps avoid infringement by automatically optimizing prompts and exploring non-infringing latent noise vectors, while ensuring that generated images align with the user's requirements.








