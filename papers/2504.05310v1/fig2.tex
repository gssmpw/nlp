\begin{figure*}
\centering

% nDCG at top 10, 100 and 1000 results

% \includegraphics[scale=0.36]{lbplots/plt_nDCGat10.png}
% \includegraphics[scale=0.36]{lbplots/plt_nDCGat100.png}
% \includegraphics[scale=0.36]{lbplots/plt_nDCGat1000.png}

% Recall with \sys{} at first 1000 results across $t$ parameter values. Left: Small Task-oriented query set. Right: Large Task-oriented query set.
    
\includegraphics[scale=0.47]{plots/2s.png}
\includegraphics[scale=0.47]{plots/2l.png}
%\includegraphics[scale=0.55]{fig1-rbo-dl20_cropped.pdf}
\\(a.) On task-oriented small query set. \hspace{96pt} (b.) On task-oriented large query set.


\caption{Recall at first 1000 results for \sys{} across $t$ parameter values from $t=0.000$ to $t=0.040$ at the interval of $0.005$. Baseline result is at $t=0$ denoted by dot. On \qs{} benchmark - small and large query sets respectively. Robustness of \sys{} is clearly evident across seed size parameter ($t$) values on the \qs{} benchmark. Further, we also note that for the small query set BM25 and TAS-B show comparable performance. But, in case of large query set TAS-B outperforms BM25.}
\label{fig:tgraphs2}
\end{figure*}