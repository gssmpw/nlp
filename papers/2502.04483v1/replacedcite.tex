\section{Related Work}
\subsection*{3D Human Pose Estimation}
3D HPE is performed using monocular ____ or multi-view ____ camera inputs.
Monocular camera-based solutions are cost-effective, however, they present unique challenges due to their inherent depth ambiguity.
Multi-view methods resolve these ambiguities through aggregation of multiple camera but require known camera projection matrices.
%
Within these two classes, single-stage approaches approximate the 3D pose by estimating parameters of a statistical body model ____ and two-stage methods ____ employ a 2D detector to predict a 2D pose and ``lift'' it into 3D.
%
The most commonly used metrics for quantitative evaluation, MPJPE, MPVE, and PCK, essentially measure the average displacement between predicted joints and corresponding ground truth joints. 
However, they fall short in their ability to quantitatively evaluate plausible postures or motion. Our proposed metrics \metricOne\ and \metricTwo\ address this through physical simulation of the predicted pose.

\subsection*{Physics-aware 3D Human Pose Estimation}
The prior 3D HPE methods are kinematics-based, only modeling joint location and disregard underlying forces and environmental factors. 
Failure to account for dynamics results in errors such as floating, foot skating, and unrealistic postures ____.
To address these implausibilities, some works have proposed ground contact metrics to measure ground penetration ____ and foot sliding (or skating) ____.
But these metrics use handcrafted heuristics and are most reliable on a calibrated floor plane.
To measure stability of poses, other works ____ incorporate terms from biomechanics literature ____ to discern balanced stationary postures.
However, these functions are only valid for stationary poses and not a pose undergoing motion because it does not account for the velocity of the center of gravity ____.
Despite these advances, these metrics do not analyze the temporal impacts of physical implausibility or a body in motion. For a more comprehensive evaluation, we look to physical simulation as a test bed for plausible motion assessment under the effects of gravity, frictional forces, and collisions.