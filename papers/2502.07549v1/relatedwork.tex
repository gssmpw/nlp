\section{Related Work}
\subsection{Trajectory-User Linking}

Trajectory data provides unprecedented insights into human mobility patterns. Recently, TUL problem was introduced in ~\cite{TULER},  which links trajectories to their generating-users, and gradually becomes a hot topic in spatio-temporal data mining. Deep learning-based methods for TUL can be broadly categorized into two distinct modeling approaches:

(1) \textbf{\textit{Sequence-based modeling methods}} represent trajectories as time-series to address the TUL task. TULER ~\cite{TULER} utilizes RNNs to learn sequential transition patterns from trajectory data and link them to users. It first employs word embedding to learn location representations and then fed them into RNN model to capture user mobility patterns for TUL. However, standard RNN-based models face data sparsity issues due to their limitation in leveraging the unlabeled data that inherently contains rich information about user mobility patterns. In their subsequent work ~\cite{TULVAE}, TULVAE alleviates the data sparsity problem by leveraging large-scale unlabeled data and captures the hierarchical and structural semantics of trajectories through \textbf{Variational Autoencoder (VAE)}. However, it does not exploit the rich features within trajectories or consider the multi-periodicity of human mobility. DeepTUL ~\cite{deeptul} addresses this limitation by employing an attentive recurrent network to learn from historical trajectory, capturing the multi-periodicity of human mobility while mitigating data sparsity. However, these methods primarily rely on existing sequence models such as LSTM ~\cite{TULER}~\cite{TULVAE} or attention mechanism \cite{deeptul} to capture intra-trajectory information and generate trajectory representations but fail to capture the global association relationships between trajectories.

(2) \textbf{\textit{Graph-based methods}} leverage GNNs to model locations or trajectories, capturing more complex and diverse relationships. Instead of only relying on visited sequences as previous methods did, GNNTUL \cite{gnntul} constructs a check-in graph that integrates geographical and temporal information and leverages GNN to effectively model user mobility patterns. However, despite modeling transition patterns using GNN, it remains limited in capturing inter-trajectory relationships.
S2TUL \cite{s2tul} is the first attempt to incorporate trajectory-level information for modeling the TUL problem. It models the complex relationships between trajectories by constructing multiple homogeneous and heterogeneous graphs, and then passes this information through a GCN to learn trajectory representations. And then, it combines intra-trajectory and inter-trajectory information to predict the generating users of trajectories.
%Contemporaneously, ANES \cite{aspect} focuses on representation learning for social link inference based on user trajectory data. It leverages user trajectory data and bipartite graphs to learn aspect-oriented relationships between users and POIs, thereby better capturing user behavioral patterns.
Nevertheless, the aforementioned methods overlook the integration and synergy of local and global information. AttnTUL \cite{attntul} proposes a hierarchical spatio-temporal attention neural network, which simultaneously models the local and global spatio-temporal characteristics of user mobility trajectories through a GNN architecture. And it designs a hierarchical attention network to jointly encode local transition patterns and global spatial dependencies for TUL. However, due to the limitations of traditional graph structures, they can only model pairwise relationships and fail to effectively capture high-order inter-trajectory association relationships, thus limiting the representation of mobility patterns in trajectories.

\subsection{Hypergraph Learning}

Hypergraph \cite{hycn}\cite{hygnn}\cite{hgnn+} is a generalization of graph, which can naturally model complex higher-order relationships among vertices by connecting multiple vertices simultaneously through hyperedges. Due to its remarkable ability to capture higher-order relations, hypergraphs have increasingly drawn significant attention from researchers. To efficiently learn deep embeddings on higher-order graph-structured data, \cite{hycn} proposed to extend the graph neural network architecture to hypergraphs by introducing two end-to-end trainable operators, namely hypergraph convolution and hypergraph attention.

In recent years, hypergraph neural networks have been widely used in spatio-temporal data modeling and user mapping tasks. In user mapping, UMAH \cite{hguid} models social structure and user profile relationships in a unified hypergraph, learns a common subspace by preserving the hypergraph structure as well as the correspondence relations of labeled users, and facilitates user mapping across social networks based on similarities in the subspace. For spatio-temporal data modeling, STHGCN \cite{hypoi} constructs a hypergraph to model trajectory granularity information, and captures higher-order information including collaborative relationships between trajectories through hypergraph learning. These studies highlight the significant potential of hypergraph-based learning methods and offer a novel solution approach to the trajectory user linking problem.