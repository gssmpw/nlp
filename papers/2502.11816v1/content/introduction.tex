\section{Introduction}

%What is the problem 
Time series forecasting plays a pivotal role in numerous fields such as economics, climate science, healthcare and more. 
As of today, a vast majority of the proposed models in the statistics and machine learning literature assume that all variables are observed at the same regular interval. 
However, this assumption does not hold in a multitude of applications in various fields. 
In such cases, models need to be able to predict the future development of variables that are observed in irregular patterns.
When time series encompass multiple variables (channels), which are observed irregularly we refer to them as \textbf{I}rregular \textbf{M}ultivariate \textbf{T}ime \textbf{S}eries (IMTS). 
An IMTS is typically considered to have missing values because most channels are not observed simultaneously.
Hence, at a single observation time point the states of only few channels are known, while the states of the remaining channels are unknown (missing), as illustrated in \Cref{fig:time_series_curvy}.
IMTS without missing values are uncommon, as it would require an observation mechanism that can access all channels at every observation step but is somehow disabled to sample observations in a regular pattern. 
Therefore, we define IMTS with missing values as the default scenario that generalizes the rare special case of IMTS without missing values.




\begin{figure}[ht]
    \centering
    \begin{tikzpicture}[x=1cm, y=1cm]
       % % Time Axis for Regular
       % \draw[->] (0,4) -- (7.5,4) node[right]{Time};
       % 
       % \draw[dashed] (5.25,4) -- (5.25,6);
       % % Regular Time Series (shifted right by 1 unit)
       % \draw[thick, softblue, smooth] plot coordinates {
       %     (1, 4.5) (1.66, 5) (2.32, 4.7) (2.98, 4.9) (3.64, 4.3)
       %     (4.3, 4.6) (4.96, 4.2) (5.62, 4.2) (6.28, 4.4) (6.94, 4.1)
       % };
       % 
       % \draw[thick, softred, smooth] plot coordinates {
       %     (1, 5)(1.66, 4.2) (2.32, 4.2) (2.98, 4.5) (3.64, 4.6)
       %     (4.3, 4.9) (4.96, 4.7) (5.62, 4.4) (6.28, 4.3) (6.94, 4.7)
       % };
       % 
       % \draw[thick, softgreen, smooth] plot coordinates {
       %     (1, 5.5) (1.66, 5.3) (2.32, 5.1) (2.98, 5.4) (3.64, 5.6)
       %     (4.3, 5.8) (4.96, 5.5) (5.62, 5.3) (6.28, 5.1) (6.94, 5.0)
       % };
       % 
       % % Regular time points (dots for softred line)
       % \foreach \x/\y in {
       %     1/5, 1.66/4.2, 2.32/4.2, 2.98/4.5, 3.64/4.6, 4.3/4.9, 4.96/4.7} {
       %         \draw[fill=softred] (\x,\y) circle (0.07cm);
       % }
       % \foreach \x/\y in {5.62/4.4, 6.28/4.3, 6.94/4.7} {
       %         \draw[fill=softred, opacity=0.5] (\x,\y) circle (0.07cm);
       % }
       % 
       % % Regular time points (dots for softblue line)
       % \foreach \x/\y in {
       %     1/4.5, 1.66/5, 2.32/4.7, 2.98/4.9, 3.64/4.3, 4.3/4.6, 4.96/4.2} {
       %         \draw[fill=softblue] (\x,\y) circle (0.07cm);
       % }
       % \foreach \x/\y in {
       %     5.62/4.2, 6.28/4.4, 6.94/4.1} {
       %         \draw[fill=softblue,opacity=0.5] (\x,\y) circle (0.07cm);
       % }
       % 
       % % Regular time points (dots for softgreen line)
       % \foreach \x/\y in {
       %     1/5.5, 1.66/5.3, 2.32/5.1, 2.98/5.4, 3.64/5.6, 4.3/5.8, 4.96/5.5} {
       %         \draw[fill=softgreen] (\x,\y) circle (0.07cm);
       % }
       % \foreach \x/\y in {
       %      5.62/5.3, 6.28/5.1, 6.94/5.0} {
       %         \draw[fill=softgreen, opacity=0.5] (\x,\y) circle (0.07cm);
       % }
%
       % \node[left] at (1,5.5) {$c_1$};
       % \node[left] at (1,5) {$c_2$};
       % \node[left] at (1,4.5) {$c_3$};
%
       % \node[above] at (4, 3.4) {Regular Multivariate Time Series};
       % \scriptsize
       % \node[] at (2.5,6) {Observation};
       % \node[] at (2.5,5.7) {Range};
%
       % \node[] at (6.7,6) {Forecasting};
       % \node[] at (6.7,5.7) {Horizon};
%

        %------------------------ Start IMTS --------------------------------

        \normalsize
        % Time Axis for Regular
        \draw[->] (0,0.5) -- (7.5,0.5) node[right]{Time};
        
        \draw[dashed] (5.25,0.5) -- (5.25,2.3);

        % Regular Time Series (shifted down by 1 unit)
        \draw[thick, softblue, smooth] plot coordinates {
            (1, 1) (1.66, 1.5) (2.32, 1.2) (2.98, 1.4) (3.64, 0.8)
            (4.3, 1.1) (4.96, 0.7) (5.62, 0.7) (6.28, 0.9) (6.94, 0.6)
        };

        \foreach \x in  {1,1.3,3.64,2.98,3.64,4,4.96,2.32,5.62,5.9,6.94,6.28}{
            \draw (\x,0.42) -- (\x,0.58); 
        }

        
    
        \draw[thick, softred, smooth] plot coordinates {
            (1, 1.5) (1.66, 0.7) (2.32, 0.7) (2.98, 1) (3.64, 1.1)
            (4.3, 1.4) (4.96, 1.2) (5.62, 0.9) (6.28, 0.8) (6.94, 1.2)
        };

        \draw[thick, softgreen, smooth] plot coordinates {
            (1, 2) (1.66, 1.8) (2.32, 1.6) (2.98, 1.9) (3.64, 2.1)
            (4.3, 2.3) (4.96, 2) (5.62, 1.8) (6.28, 1.6) (6.94, 1.5)
        };

        % Regular time points (dots for softred line)
        \foreach \x/\y in {
            1.3/1.07, 2.32/0.7,  4.96/1.2} {
                \draw[fill=softred] (\x,\y) circle (0.07cm);
        }
        \foreach \x/\y in {6.28/0.8} {
                \draw[fill=softred, opacity=0.5] (\x,\y) circle (0.07cm);
        }

        % Regular time points (dots for softblue line)
        \foreach \x/\y in {
        2.98/1.4, 3.64/0.8, 4.0/0.93} {
                \draw[fill=softblue] (\x,\y) circle (0.07cm);
        }

        \foreach \x/\y in {5.62/0.7} {
                \draw[fill=softblue,opacity=0.5] (\x,\y) circle (0.07cm);
        }

        % Regular time points (dots for softgreen line)
        \foreach \x/\y in {
            1/2, 3.64/2.1} {
                \draw[fill=softgreen] (\x,\y) circle (0.07cm);
        }

        \foreach \x/\y in {
            5.9/1.72, 5.62/1.8, 6.94/1.5} {
                \draw[fill=softgreen, opacity=0.5] (\x,\y) circle (0.07cm);
        }

        % Labels
        \node[left] at (1,2) {$c_1$};
        \node[left] at (1,1.5) {$c_2$};
        \node[left] at (1,1) {$c_3$};

        %\node[above] at (4, -0.1) {Irregular Multivariate Time Series};
        \scriptsize
        \node[] at (2.5,2.6) {Observation};
        \node[] at (2.5,2.3) {Range};
        \node[] at (6.7,2.6) {Forecasting};
        \node[] at (6.7,2.3) {Horizon};
        \normalsize


    \end{tikzpicture}
    \caption{Example for an IMTS Forecasting Task. The observations and forecasting targets are irregularly spaced.}\label{fig:time_series_curvy}
    \Description{Depiction of an Irregularly sampled time series with three channels}
\end{figure}

%How is the problem currently solved / what are open problems
Previous studies~\cite{Che2018.Recurrent,Yalavarthi2023.Forecasting,Zhang.Irregular} have demonstrated the necessity of specifically designed IMTS forecasting models as regular time series forecasting models perform poorly when applied to IMTS.\@

Initially, researchers proposed models based on Ordinary Differential Equations (ODEs)~\cite{Chen2018.Neural,DeBrouwer2019.GRUODEBayes,Bilos2021.Neural,Schirmer2022.Modeling}. 
ODEs are intuitively well-suited for IMTS modeling, because they are used in various scientific fields
such as Physics, Chemistry and Biology to model complex dynamics in a continuous manner. 
However, learning an ODE that is defined by a neural network (Neural ODE~\cite{Chen2018.Neural}) has shown to be ineffective.
Furthermore, ODE-based models typically need a computationally expensive ODE-solver, which results in slow training and inference. 

More recent architectures encode IMTS into graphs~\cite{Yalavarthi2023.Forecasting,Zhang.Irregular}
and provide a significant lift over ODE-based models in terms of forecasting accuracy and computational efficiency. 

On the other hand, time series mixer (TS-mixer) models~\cite{Ekambaram2023.TSMixer,Chen2023.TSMixer} are the state-of-the-art models for regular multivariate time series (RMTS) forecasting, making it intriguing 
to apply that approach to IMTS.\@ 
However, TS-Mixer treats a time series as fixed-size matrix and applies fully connected layers along the time and channel dimension without modeling observation times explicitly. 
Hence, TS-Mixer is not applicable to IMTS out-of-the-box. 
Instead, it is necessary to transform each channel into a fixed-size vector, resulting in a fixed-size matrix when these channel-vectors are stacked.

%  What are we doiing now.
In this work, we introduce \model{}, a new architecture for IMTS forecasting. In order to apply mixer networks to IMTS, we implement a novel aggregation method that encodes an irregularly observed channel 
into a fixed-sized vector. For every channel, we embed each observation into a vector and aggregate these vectors with a convex sum. 
The weights for the convex sums are inferred based on observation times and the observed values. 
We yield one vector per channel and combine these channels into a matrix which we can feed into mixer-blocks to capture intra- and inter-channel dependencies.
Finally, we introduce a novel module to predict IMTS forecasting targets from the channel embeddings.

To evaluate the performance of \model{}, we conduct experiments on the evaluation tasks proposed by \citeauthor{Zhang.Irregular}~\cite{Zhang.Irregular}.
Our results show that \model{} yields the most accurate forecasts on three out of four evaluation tasks. 
On the fourth task, we establish a new state-of-the-art by evaluating an additional model (GraFITi~\cite{Yalavarthi2023.Forecasting}) on the benchmark. 

Hence, the findings of this work have a profound impact on both researchers and practitioners in the field of IMTS forecasting.
Our contributions contain the following:
\begin{itemize}
    \item We propose \model{}, the first IMTS forecasting model, that builds upon TS-mixer~\cite{Ekambaram2023.TSMixer,Chen2023.TSMixer} a simple, yet effective architecture that is well-established in RMTS forecasting.
    \item In order to make mixer networks applicable, we introduce a novel encoder that aggregates irregularly observed input channels into fixed-sized vectors.   
    \item We introduce a decoder that can predict IMTS forecasting targets based on fixed-sized vectors. 
    \item Our experiments establish a new state-of-the-art on every evaluation task from the benchmark we used.
    %In three out of four tasks our model \model{} is the new state-of-the-art model.
    \item For reproducabitly, we share our code on GitHub: \newline
    \url{https://anonymous.4open.science/r/IMTS-Mixer-D63C/}
\end{itemize}

