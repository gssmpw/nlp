\section{Method}
\Cref{fig:model} depicts an overview over \model{}.
In the subsections below we explain each module, starting with
the channel aggregation method, the key innovation of this work that is depicted in \Cref{fig:obs_enc}. 
Subsequently, we explain how we apply the mixer blocks to the aggregated channels and how we predict for the query time points.
\subsection{Observation-Encoder}\label{sec:obsenc}
To aggregate the observations of a channel in a meaningful way, we first transform observation tuples $(v_i, t_i)$ into higher dimensional vector-embeddings $h_i \in \R^D$.
From here on, we will write $v_i, t_i$ for $v_{c,i}, t_{c,i}$, if the channel $c$ is not relevant. 
First we, encode the scalar value $v$ into the hidden dimension $D$ with a linear layer:
\begin{align}
    v_i^{\text{enc}} = W^{v_\text{enc}} v_i +  b^{v_\text{enc}}, 
\end{align}
where $W^{\text{v}} \in \R^{D}$ and $b^{\text{v}} \in \R^{D}$ are trainable weights and biases.
To include information about the observation time, we multiply the encoded observation value
element-wise with a vector of the same length that is inferred by feeding the scalar observation time through a 2-layer fully connected network with relu activation function:
\begin{align}\label{eq:obs_encoding}
    t^\text{enc}_i &= W_2^{t_\text{enc}}(\relu(W_1^{t_\text{enc}}t_i + b^{t_\text{enc}}_1)) + b^{t_\text{enc}}_2 \\
    h_i &=  t^\text{enc}_i \odot v^\text{enc}_i 
\end{align}

\begin{figure}
    \centering
    \begin{tikzpicture}[x=1cm, y=1cm]
        \draw[fill=veryverylightgreen,rounded corners=3pt] (-0.6,0) rectangle (1.1,0.5); 
        \node at (0.25,0.25) {$T_c \in \R^{N_c \times 1}$};
        \draw[->] (0.25,0.5) -- (0.25,1);

        \draw[fill=veryverylightyellow,rounded corners=3pt] (1.45,0) rectangle (3.15,0.5); 
        \node at (2.3,0.25) {$V_c \in \R^{N_c \times 1}$};
        \draw[->] (2.3,0.5) -- (2.3,1);


        \draw[fill=relublue ,rounded corners=3pt] (1.45,1) rectangle (3.15,1.5); 
        \node at (2.3,1.25) {\small $\FC(1 \to D)$};
        \draw[->, rounded corners=3pt] (2.3,1.5) -- (2.3,2.75) -- (1.4,2.75);
        \draw[->, rounded corners=3pt] (2.3,1.5) -- (2.3,2.75) -- (3.35,2.75);


        %\draw[fill=pastelyellow] (-0.7,0.8) rectangle (1.2,2.6);
        %\node[right] at (-0.7,2.45) {\scriptsize{$\text{MLP}_\text{emb}$}};

        \draw[fill=relublue, opacity=1,rounded corners=3pt] (-0.6,1) rectangle (1.1,2.3); 
        \node at (0.25,1.25) {\small$\FC (1 \to D^{\prime})$};

        \draw[fill=verysoftred,rounded corners=0pt] (-0.6,1.5) rectangle (1.1,1.8); 
        \node at (0.25, 1.65) {\small{relu}};

        %\draw[fill=relublue, opacity=1,rounded corners=3pt] (-0.6,1.8) rectangle (1.1,2.3); 
        \node at (0.25,2) {\small$\FC(D^{\prime} \to D)$};
        \draw[->,rounded corners=3pt] (0.25,2.3) -- (0.25,2.75) -- (1.1,2.75);

    
        %\draw[fill=pastelyellow] (3.8,0.8) rectangle (5.7,2.6);
        %\node[right] at (3.8,2.45) {\scriptsize{$\text{MLP}_\text{w}$}};
        \draw[fill=relublue, rounded corners=3pt] (3.8,1) rectangle (5.5,2.3); 
        \node at (4.65,1.25) {\small $\FC (1 \to D^\prime)$};

        \draw[fill=verysoftred , rounded corners=0pt] (3.8,1.5) rectangle (5.5,1.8); 
        \node at (4.65, 1.65) {\small{relu}};

        %\draw[fill=relublue, rounded corners=3pt] (3.8,1.8) rectangle (5.5,2.3); 
        \node at (4.65,2) {\small $\FC (D^\prime \to D)$};
        \draw[->,rounded corners=3pt] (4.65,2.3) -- (4.65,2.75) -- (3.65,2.75);

        \node at (1.25,2.75) {\large{$\mathbf{\odot}$}};

        \node at (3.5,2.75) {\large{$\mathbf{+}$}};
        \draw[-> ] (1.25,2.9) -- (1.25,3.3);

        \draw[rounded corners=3pt,fill=pastelyellow,opacity=1] (0.25,3.3) rectangle (2.25,3.8) ; 
        \node at (1.25,3.55) {$H_c \in \R^{N_c \times D}$};

        \draw[-> ] (3.5,2.9) -- (3.5,3.3);

        \draw[rounded corners=3pt,fill=pastelyellow,opacity=1] (2.5,3.3) rectangle (4.5,3.8) ; 
        \node at (3.5,3.55) {$A_c \in \R^{N_c \times D}$};
        \draw[->] (3.5,3.8) -- (3.5,4.1) ; 

        \draw[rounded corners=3pt,fill=verysoftred] (2.75,4.1) rectangle (4.25,4.4) ; 
        \node at (3.5,4.25) {\small{softmax}};

        \draw[->,rounded corners=3pt] (0.25,0) --(0.25,-0.5) -- (4.65,-0.5) -- (4.65,1);

        \draw[->,rounded corners=3pt] (1.25,3.8) -- (1.25,4.2) -- (1.7,4.2) -- (1.7,4.7);
        \draw[->,rounded corners=3pt] (3.5,4.4) -- (3.5,4.5) -- (3.05,4.5) -- (3.05,4.7);

        \draw[rounded corners=3pt,fill=transgray,opacity=1] (1.2, 4.7) rectangle (3.4, 5.3);
        \node at (2.3,5) {\small{$\sum^{N_c}_{i=1} H_c^{(i)} \odot W_c^{(i)}$}};
        \draw[->] (2.3,5.3) -- (2.3,5.6) ;


        \draw[rounded corners=3pt,fill=pastelyellow,opacity=1] (1.3,5.6) rectangle (3.3,6.1) ; 
        \node at (2.3,5.85) {$Z_c \in \R^D$};
 




    \end{tikzpicture}
    \caption{\model{}'s channel aggregation architecture.
    For simplicity, we split a channel $X_c$ into the observation times $T_c$ and observed values $V_c$.
    For each channel, the aggregation module shown summarizes a varying number of observation tuples into a fixed-sized vector $Z_c$.
    We use $\FC(A \to B)$ to symbolize a single fully connected linear layer with A input and B output neurons.}\label{fig:obs_enc}
    \Description{An awesome model architecture}
\end{figure}

\subsection{Channel Aggregation}
After embedding each observation into a vector we aggregate the embeddings within a weighted convex sum.
The respective weights $\alpha_i$ are based on the observation time and the observed value. 
Here, we reuse the value-encoding $v^{enc}_i$ that was used to compute the observation embedding and add it to the output of another
neural network that inputs time as a scalar.
\begin{align}\label{eq:weight}
    \alpha_i = W_2^{t_\text{weight}}(\relu(W_1^{t_\text{weight}}t_i + b^{t_\text{weight}}_1)) + b^{t_\text{weight}}_2
\end{align}
We aggregate the observations embeddings of each channel to obtain the channel encoding $Z_c \in \R^D$. Specifically $Z_c$ is the encoding
of the IMTS channel $X_c$. While $X_c$ is a set of arbitrary many observation tuples, $Z_c$ is a vector of size $D$ for every channel $c$.
The computation of the weighted convex sum can be expressed with:   
\begin{align}\label{eq:agg}
    Z_c = \sum^{N_c}_{i=1} \softmax{(A_c)}_i \odot h_i
\end{align}
Here, $A_c \in \R^{N_c \times D}$ is the matrix created by concatenating the $N_c$ many $\alpha_i$-vectors as defined by \Cref{eq:obs_encoding},
with $N_c$ referring to the number of observations in channel $X_c$. The softmax function is applied 
over the rows of $A_c$ to ensure that all weights contained in the vectors $\alpha_i$ are positive and add up to one in each dimension.    

\subsection{Channel-Bias}
The proposed channel encoder shares the weights for all channels and thus only learns to embed them based on the patterns in the observation horizon. 
To learn channel specific information and distinguish the channel embeddings, we add channel biases $b_c \in R^D$. 
The final channel encoding is then computed as
\begin{align}\label{eq:bias}
    Z_c^+ = Z_c + b_c
\end{align}
This helps also for the case where a channel is unobserved in the observation horizon.
In that case, we define the channel encoding simply to be the bias term of that channel. 
In \Cref{sec:bias} we show an ablation study showing that this is sufficient for encoding the channels and no other channel specific weights are necessary.
Note also that the subsequent mixer blocks can learn channel specific patterns as well as mix information across channels.

\subsection{Mixer Blocks}
\begin{figure}
    \centering
    \begin{tikzpicture}[x=1cm, y=1cm]
        \draw[fill=pastelyellow,opacity=1, , rounded corners=5pt] (1.5,0) rectangle (3.5,0.5);
        \node at (2.5, 0.25){$\mathbf{Z^{l-1}} \in \R^{C \times D}$};

        \draw[->] (2.5,0.5) to (2.5,1);         
        \draw[fill=transgray,rounded corners=3pt] (1.5,1) rectangle (3.5,1.5);
        \draw[->] (2.5,1.5) to (2.5,1.85); 
        \node at (2.5, 1.25){Transpose};
        
        \draw [fill=verysoftred,rounded corners=5pt] (1,1.85) rectangle (4,3.05);
        \node at (2.5, 2.){\small{RMS-Norm}};
        \draw [fill=relublue,rounded corners=0pt](1,2.15) rectangle (4,2.75);
        \node at (2.5, 2.45){$\FC (C \to C)$};
        %\draw [fill=transgray, ,rounded corners=0pt] (1,2.75) rectangle (4,3.05);
        \node at (2.5, 2.9){\small{relu}};
        \draw[->] (2.5,3.05) to (2.5,3.3);

        \draw[fill=transgray,rounded corners=3pt] (1.5,3.3) rectangle (3.5,3.8);
        \node at (2.5, 3.55){Transpose};
        \draw[->] (2.5,3.8) to (2.5,4.35);

        \draw[fill=verysoftred,rounded corners=5pt] (1,4.35) rectangle (4,5.55);
        \node at (2.5, 4.5){\small{RMS-Norm}};
        \draw[fill=relublue,rounded corners=0pt] (1,4.65) rectangle (4,5.25);
        \node at (2.5, 4.9){$\FC (D \to D)$};
        %\draw[fill=transgray, ,rounded corners=0pt](1,5.25) rectangle (4,5.55);
        \node at (2.5, 5.4){\small{relu}};

        \draw[->, ] (2.5,5.55) to (2.5,6.2);
        \draw[fill=pastelyellow,opacity=1, ,rounded corners=5pt] (1.5,6.2) rectangle (3.5,6.7);
        \node at (2.5, 6.45){$\mathbf{Z^{l}} \in \R^{C \times D}$};


        \draw[->, rounded corners=5pt] (3.5,0.25) -- (4.7,0.25) -- (4.7,4.1) -- (2.5,4.1);
        \draw[->, rounded corners=5pt] (1.5,0.25) -- (0.5,0.25) -- (0.5,6) -- (2.5,6);
        \draw[->, rounded corners=5pt] (3.5,3.55) -- (4.5,3.55) -- (4.5,5.8) -- (2.5,5.8);

        \node[left] at (0.5,3) {\textbf{+}};
        \node[right] at (4.7,2.5) {\textbf{+}};
        \node[right] at (4.5,4.9) {\textbf{+}};

    \end{tikzpicture}
    \caption{Architecture of the mixer block from IMTS-Mixer.
            $\FC (A \to B) $ refers to a linear layer with $A$ input and $B$ output neurons.}\label{fig:mixer}
    \Description{An awesome model architecture}
\end{figure}
\section{Model}
\label{sec:model}
Let $[N] = \{1, 2, \dots, N \}$ be a set of $N$ agents.
We examine an environment in which a system interacts with the agents over $T$ rounds.
Every round $t\leq T$ comprises $N$ \emph{sessions}, each session represents an encounter of the system with exactly one agent, and each agent interacts exactly once with the system every round.
I.e., in each round $t$ the agents arrive sequentially. 


\paragraph{Arrival order} The \emph{arrival order} of round $t$, denoted as $\ordv_t=(\ord_t(1),\dots, \ord_t(N))$, is an element from set of all permutations of $[N]$. Each entry $q$ in $\ordv_t$ is the index of the agent that arrives in the $q^{\text{th}}$ session of round $t$.
For example, if $\ord_t(1) = 2$, then agent $2$ arrives in the first session of round $t$.
Correspondingly, $\ord_t^{-1}(i)=q$ implies that agent $i$ arrives in the $q^{\text{th}}$ session of round $t$. 

As we demonstrate later, the arrival order has an immediate impact on agent rewards. We call the mechanism by which the arrival order is set \emph{arrival function} and denote it by $\ordname$. Throughout the paper, we consider several arrival functions such as the \emph{uniform arrival} function, denoted by $\uniord$, and the \emph{nudged arrival} $\sugord$; we introduce those formally in Sections~\ref{sec:uniform} and~\ref{sec:nudge}, respectively.

%We elaborate more on this concept in Section~\ref{sec: arrival}.


\paragraph{Arms} We consider a set of $K \geq 2$ arms, $A = \{a_1, \ldots, a_K\}$. The reward of arm $a_i$ in round $t$ is a random variable $X_i^t \sim \mathcal{D}^t_i$, where the rewards $(X_i^t)_{i,t}$ are mutually independent and bounded within the interval $[0,1]$. The reward distribution $\mathcal{D}^t_i$ of arm $a_i$, $i\in [K]$ at round $t\in T$ is assumed to be non-stationary but independent across arms and rounds. We denote the realized reward of arm $a_i$ in round $t$ by $x_i^t$. We assume \emph{reward consistency}, meaning that rewards may vary between rounds but remain constant within the sessions of a single round. Specifically, if an arm $a_i$ is selected multiple times during round~$t$, each selection yields the same reward $x_i^t$, where the superscript $t$ indicates its dependence on the round rather than the session. This consistency enables the system to leverage information obtained from earlier sessions to make more informed decisions in later sessions within the same round. We provide further details on this principle in Subsection~\ref{subsec:information}.


\paragraph{Algorithms} An algorithm is a mapping from histories to actions. We typically expect algorithms to maximize some aggregated agent metric like social welfare. Let $\mathcal H^{t,q}$ denote the information observed during all sessions of rounds $1$ to $t-1$ and sessions $1$ to $q-1$ in round $t$.  The history $\mathcal H^{t,q}$ is an element from $(A \times [0,1])^{(t-1) \cdot N +q-1}$, consisting of pairs of the form (pulled arm, realized reward). Notice that we restrict our attention to \emph{anonymous} algorithms, i.e., algorithms that do not distinguish between agents based on their identities. Instead, they only respond to the history of arms pulled and rewards observed, without conditioning on which specific agent performed each action.
%In the most general case, algorithms make decisions at session $q$ of round $t$  based on the entire history $\mathcal H^{t,q}$ and the index of the arriving agent $\ord_t(q)$. %Furthermore, we sometimes assume that algorithms have Bayesian information, i.e., algorithms are aware of the distributions $(\mathcal D_i)^K_{i=1}$. 
Furthermore, we sometimes assume that algorithms have Bayesian information, meaning they are aware of the reward distributions $(\mathcal{D}^t_i)_{i,t}$. If such an assumption is required to derive a result, we make it explicit. %Otherwise, we do not assume any additional knowledge about the algorithm’s information. %This distinction allows us to analyze both general algorithms without prior distributional knowledge and specialized algorithms that leverage Bayesian information.


\paragraph{Rewards} Let $\rt{i}$ denote the reward received by agent $i \in [N]$ at round $t$, and let $\Rt{i}$ denote her cumulative reward at the end of round $t$, i.e., $\Rt{i} = \sum_{\tau=1}^{t}{r^{\tau}_{i}}$. We further denote the \emph{social welfare} as the sum of the rewards all agents receive after $T$ rounds. Formally, $\sw=\sum^{N}_{i=1}{R^T_i}$. We emphasize that social welfare is independent of the arrival order. 


\paragraph{Envy}
We denote by $\adift{i}{j}$ the reward discrepancy of agents $i$ and $j$ in round $t$; namely, $\adift{i}{j}= \rt{i} - \rt{j}$. %We call this term \omer{name??} reward discrepancy in round $t$. 
The (cumulative) \emph{envy} between two agents at round $t$ is the difference in their cumulative rewards. Formally, $\env_{i,j}^t= \Rt{i} - \Rt{j}$ is the envy after $t$ rounds between agent $i$ and $j$. We can also formulate envy as the sum of reward discrepancies, $\env_{i,j}^t= \sum^{t}_{\tau=1}{\adif{i}{j}^\tau}$. Notice that envy is a signed quantity and can be either positive or negative. Specifically, if $\env_{i,j}^t < 0$, we say that agent $i$ envies agent $j$, and if $\env_{i,j}^t > 0$, agent $j$ envies agent $i$. The main goal of this paper is to investigate the behavior of the \emph{maximal envy}, defined as
\[
\env^t = \max_{i,j \in [N]} \env^t_{i,j}.
\]
For clarity, the term \emph{envy} will refer to the maximal envy.\footnote{ We address alternative definitions of envy in Section~\ref{sec:discussion}.} % Envy can also be defined in alternative ways, such as by averaging pairwise envy across all agents. We address average envy in Section~\ref{sec:avg_envy}.}
Note that $\env_{i,j}^t$ are random variables that depend on the decision-making algorithm, realized rewards, and the arrival order, and therefore, so is $\env^t$. If a result we obtain regarding envy depends on the arrival order $\ordname$, we write $\env^t(\ordname)$. Similarly, to ease notation, if $\ordname$ can be understood from the context, it is omitted.



\paragraph{Further Notation} We use the subscript $(q)$ to address elements of the $q^{\text{th}}$ session, for $q\in [N]$.
That is, we use the notation $\rt{(q)}$ to denote the reward granted to the agent that arrives in the $q^{\text{th}}$ session of round $t$ and $\Rt{(q)}$ to denote her cumulative reward. %Additionally, we introduce the notation $\at{(q)}$ to denote the arm pulled in that session.
Correspondingly, $\sdift{q}{w} = \rt{(q)} - \rt{(w)}$ is the reward discrepancy of the agents arriving in the $q^{\text{th}}$ and $w^{\text{th}}$ sessions of round $t$, respectively. 
To distinguish agents, arms, sessions and rounds, we use the letters $i,j$ to mark agents and arms, $q,w$ for sessions, and $t,\tau$ for rounds.


\subsection{Example}
\label{sec: example}
To illustrate the proposed setting and notation, we present the following example, which serves as a running example throughout the paper.

\begin{table}[t]
\centering
\begin{tabular}{|c|c|c|c|}
\hline
$t$ (round) & $\ordv_t$ (arrival order) & $x_1^t$ & $x_2^t$ \\ \hline
1           & 2, 1                     & 0.6     & 0.92    \\ \hline
2           & 1, 2                     & 0.48    & 0.1     \\ \hline
3           & 2, 1                     & 0.15    & 0.8     \\ \hline
\end{tabular}
\caption{
    Data for Example~\ref{example 1}.
}
\label{tbl: example}
\end{table}

\begin{algorithm}[t]
\caption{Algorithm for Example~\ref{example 1}}
\label{alguni}
\DontPrintSemicolon 
\For{round $t = 1$ to $T$}{
    pull $a_{1}$ in the first session\label{alguniexample: first}\\
    \lIf{$x^t_1 \geq \frac{1}{2}$}{pull $a_{1}$ again in second session \label{alguniexample: pulling a again}}
    \lElse{pull $a_{2}$ in second session \label{alguniexample: sopt else}}
}
\end{algorithm}


\begin{example}\label{example 1}
We consider $K=2$ uniform arms, $X_1,X_2 \sim \uni{0,1}$, and $N=2$ for some $T\geq 3$. We shall assume arm decision are made by Algorithm~\ref{alguni}: In the first session, the algorithm pulls $a_1$; if it yields a reward greater than $\nicefrac{1}{2}$, the algorithm pulls it again in the second session (the ``if'' clause). Otherwise, it pulls $a_2$.



We further assume that the arrival orders and rewards are as specified in Table~\ref{tbl: example}. Specifically, agent 2 arrives in the first session of round $t=1$, and pulling arm $a_2$ in this round would yield a reward of $x^1_2 = 0.92$. Importantly, \emph{this information is not available to the decision-making algorithm in advance} and is only revealed when or if the corresponding arms are pulled.




In the first round, $\boldsymbol{\eta}^1 = \left(2,1\right)$; thus, agent 2 arrives in the first session.
The algorithm pulls arm $a_1$, which means, $a^1_{(1)} = a_1$, and the agent receives $r_{2}^1=r_{(1)}^1=x_1^1=0.6$.
Later that round, in the second session, agent 1 arrives, and the algorithm pulls the same arm again since $x^1_1 = 0.6 \geq \nicefrac{1}{2}$ due to the ``if'' clause.
I.e., $a^1_{(2)} = a_1$ and $r_{1}^1 = r_{(2)}^1 = x_1^1 = 0.6$.
Even though the realized reward of arm $a_2$ in that round is higher ($0.92$), the algorithm is not aware of that value.
At the end of the first round, $R^1_1 = R^1_{(2)} = R^1_2 = R^1_{(1)} = 0.6$. The reward discrepancy is thus $\adif{1}{2}^1 = \adif{2}{1}^1= \sdif{2}{1}^1 = 0.6 - 0.6 =0$. 



In the second round, agent 1 arrives first, followed by agent 2.
Firstly, the algorithm pulls arm $a_1$ and agent 1 receives a reward of $r_{1}^2 = r_{(1)}^2 = x_1^2 = 0.48$.
Because the reward is lower than $\nicefrac{1}{2}$, in the second session the algorithm pulls the other arm ($a^2_{(2)} = a_2$), granting agent 2 a reward of $r_{2}^2 = r_{(2)}^2 = x_2^2 = 0.1$.
At the end of the second round, $R^2_1 = R^2_{(1)} = 0.6 + 0.48 = 1.08$ and $R^2_2 = R^2_{(2)} = 0.6 + 0.1 = 0.7$. Furthermore, $\sdif{2}{1}^2 = \adif{2}{1}^2 = r^2_{2} - r^2_{1} = 0.1 - 0.48 = -0.38$.

In the third and final round, agent 2 arrives first again, and receives a reward  of $0.15$ from $a_1$. When agent 1 arrives in the second session, the algorithm pulls arm $a_2$, and she receives a reward of $0.8$. As for the reward discrepancy, $\sdif{2}{1}^3 = \adif{2}{1}^3 = r^3_{2} - r^3_{1} = 0.15 - 0.8 = -0.75$. 

Finally, agent 1 has a cumulative reward of $R^3_1 = R^3_{(2)} = 0.6 + 0.48 + 0.8 = 1.88$, whereas agent~2 has a cumulative reward of $R^3_2 = R^3_{(1)} = 0.6 + 0.1 + 0.15 = 0.85$. In terms of envy, $\env^1_{1,2}= \adif{1}{2}^1 =0$, $\env^2_{1,2}=\adif{1}{2}^1+\adif{1}{2}^2= 0.38$, and $\env^3_{1,2} = -\env^3_{2,1} = R^3_1-R^3_2 = 1.88-0.85 = 1.03$, and consequently the envy in round 3 is $\env^3 = 1.03$.
\end{example}


\subsection{Information Exploitation}
\label{subsec:information}

In this subsection, we explain how algorithms can exploit intra-round information.
Since rewards are consistent in the sessions of each round, acquiring information in each session can be used to increase the reward of the following sessions.
In other words, the earlier sessions can be used for exploration, and we generally expect agents arriving in later sessions to receive higher rewards.
Taken to the extreme, an agent that arrives after all arms have been pulled could potentially obtain the highest reward of that round, depending on how the algorithm operates.

To further demonstrate the advantage of late arrival, we reconsider Example~\ref{example 1} and Algorithm~\ref{alguni}. 
The expected reward for the agent in the first session of round $t$ is $\E{\rt{(1)}}=\mu_1=\frac{1}{2}$, yet the expected reward of the agent in the second session is
\begin{align*}
\E{\rt{(2)}}=\E{\rt{(2)}\mid X^t_1 \geq \frac{1}{2} }\prb{X^t_1 \geq \frac{1}{2}} + \E{\rt{(2)}\mid X^t_1 < \frac{1}{2} }\prb{X^t_1 < \frac{1}{2}};
\end{align*}
thus, $\E{\rt{(2)}} =\E{X^t_1\mid X^t_1 \geq \frac{1}{2} }\cdot \frac{1}{2} + \mu_2\cdot\frac{1}{2} = \frac{5}{8}$.
Consequently, the expected welfare per round is $\E{\rt{(1)}+\rt{(2)}}=1+\frac{1}{8}$, and the benefit of arriving in the second session of any round $t$ is $\E{\rt{(2)} - \rt{(1)}} = \frac{1}{8}$. This gap creates envy over time, which we aim to measure and understand.
%This discrepancy generates envy over time, and our paper aims to better understand it.
\subsection{Socially Optimal Algorithms}
\label{sec: sw}
Since our model is novel, particularly in its focus on the reward consistency element, studying social welfare maximizing algorithms represents an important extension of our work. While the primary focus of this paper is to analyze envy under minimal assumptions about algorithmic operations, we also make progress in the direction of social welfare optimization. See more details in Section~\ref{sec:discussion}.%Due to space limitations, we defer the discussion on socially optimal algorithms to  \ifnum\Includeappendix=0{the appendix}\else{Section~\ref{appendix:sociallyopt}}\fi.




% Since our model is novel and specifically the reward consistency element, it might be interesting to study social welfare optimization. While the main focus of our paper is to study envy under minimal assumptions on how the algorithm operates, we take steps toward this direction as well. Due to space limitations, we defer the discussion on socially optimal algorithms to  \ifnum\Includeappendix=0{the appendix}\else{Section~\ref{appendix:sociallyopt}}\fi.  We devise a socially optimal algorithm for the two-agent case, offer efficient and optimal algorithms for special cases of $N>2$ agents, and an inefficient and approximately optimal algorithm for any instance with $N>2$. Moreover, we address the welfare-envy tradeoff in Section~\ref{sec:extensions}.


% Social welfare, unlike envy, is entirely independent of the arrival order. While the main focus of our paper is to study envy under minimal assumptions on how the algorithm operates, socially optimal algorithms might also be of interest. Due to space limitations, we defer the discussion on socially optimal algorithms to  \ifnum\Includeappendix=0{the appendix}\else{Section~\ref{appendix:sociallyopt}}\fi. We devise a socially optimal algorithm for the two-agent case, offer efficient and optimal algorithms for special cases of $N>2$ agents, and an inefficient and approximately optimal algorithm for any instance with $N>2$. %Furthermore, we treat the welfare-envy tradeoff of the special case of Example~\ref{example 1}.




Time series mixer networks~\cite{Ekambaram2023.TSMixer,Chen2023.TSMixer} apply fully connected layers along the time and channel dimension alternately.  
They are well-established in regular time series forecasting and yield state-of-the-art forecasting despite exclusively relying on vanilla fully connected layers~\cite{Ekambaram2023.TSMixer,Chen2023.TSMixer}. 

A mixer network can only be applied to 2-D inputs where both dimensions are fixed across instances. 
Additionally, it cannot directly process \nan-values. 
While these conditions are met in regular time series, they can never be satisfied in any IMTS dataset.

However, when we aggregate each channel of an IMTS into fixed-sized vectors as we propose in the previous subsections, we encode an IMTS in matrix form, in which the rows 
directly represent the channels. As a result, we are able to harness the power of mixer networks to learn channel interactions and extract the information relevant to forecasting.   

Instead of using mixer blocks, we could also flatten the hidden state and apply fully connected layers  $\R^{CD} \to \R^{CD}$. 
However, the number of parameters (${(CD)}^2$) of such a layer would explode.
Hence, the resulting model would demand an unnecessary amount of memory while being prone to overfitting.

We define a mixer block as two fully connected layers that are applied subsequently along the hidden dimension $D$ and the input channels $C$.
To learn non-linear dependencies we add relu as an activation function to each layer. 
Aligning with TS-mixer architectures established in RMTS forecasting~\cite{Chen2023.TSMixer,Ekambaram2023.TSMixer}, we add residual connections as they are known to simplify the optimization while enhancing expressiveness~\cite{He2016.Deep}.
Furthermore, we also apply normalization before each fully connected layer as it is known to improve learning deep architectures and was also added to RMTS TS-mixer architectures~\cite{Chen2023.TSMixer,Ekambaram2023.TSMixer}.   
Specifically, we use Rooted-Mean-Square Layer Normalization (RMS-Norm)~\cite{Zhang2019.Root}. 

Theoretically, we could vary the hidden dimension $D$ after each layer as well as the number of (latent) channels $C$. The next channel-to-channel layer would simply have to adjust its input dimension to the output 
dimension of the previous channel-to-channel layer. However, the final channel-to-channel layer is restricted to have an output dimension of $C$, due to our method of utilizing the outcome of the mixer blocks for the final prediction.
We describe the respective method in the next subsection. We decide to keep the output dimension of channel-to-channel layers as $C$. 

Furthermore, we also keep the hidden dimension at $D$ except for the last layer where we allow us to choose a different output dimension $D_\text{out}$.
This decision is supposed to account for the fact that the range in which queries are given can significantly vary from the observation time. 
For example, a model might be tasked to predict a month based on two years.
As the mixer blocks are supposed to map a sequence of observations into a latent encoding of an IMTS's state during the forecasting span~\cite{Chen2023.TSMixer,Ekambaram2023.TSMixer}
the output dimension $D_\text{out}$ should match the necessary complexity. 
Staying in the given example, the aggregated channels of dimension $D$ need to contain all the relevant information 
of two years, but the channel representations emitted by the final mixer block only have to encode information that correspond to one month. 
Therefore, it would be adequate if $D_\text{out}$ is smaller than $D$.   

\model{} contains $L$ mixer blocks, and we define the output of the $l^{th}$ mixer block as:
\begin{align}
    Z^{\prime{(l)}} & = Z^{(l-1)} + \relu(W_C^{(l)} \RMS(Z^{(l-1)}) + b_C^{(l)}) \\
    Z^{(l)} & = Z^{(l-1)} +  Z^{\prime{(l)}} + \relu(W_D^{(l)} \RMS(Z^{\prime{(l)}}) + b_D^{(l)}),
\end{align}
with $W_D^{(l)} \in \R^{D\times D}$, $b_D^{(l)} \in \R^D$, $W_C^{(l)} \in \R^{C \times C}$ and $b_C^{(l)} \in \R^C$.
We use $Z^+$ from \Cref{eq:bias} as the input for the first mixer block ($Z^0 = Z^+$). \Cref{fig:mixer} summarizes the architecture of the mixer blocks as introduced in this subsection.
\subsection{Decoder}
To obtain the forecasts we need to map the hidden state of each channel to the queries and decode it into a scalar for each pair of channel and query. 
Predicting continuous queries from fixed-sized vectors that represent a complete channel is not a novel problem and has been previously solved by concatenating the time stamp to the hidden state.
The final prediction can then be inferred by a neural network~\cite{Zhang.Irregular}. 

However, we propose an alternative method, that closely relates to how we computed the embeddings for observation-tuples, but is reversely applied. 
We input the query time as a scalar into a 2-layer neural network with $D_{out}$ output neurons. 
The vector encoding of the query time is then multiplied element-wisely with the channel encoding and fed into linear layer that aggregates the vector product into a scalar.
Formally, we obtain the prediction $\hat{y}$ of channel $c$ at query time $q_{i,c}$ with:
\begin{align}
    q^{enc}_{i,c} &=  W^{q}_2 \relu(W^{q}_1  q_{i,c}  + b^{q}_{c,1}) + b^{q}_{c,2} \\
    \hat{y}_{i,c} &=  W^{\text{out}} (q^{enc}_{i,c} \odot Z^L_c)  + b^\text{out} 
\end{align}
\subsection{Sharing weights}
The modules for channel aggregation can either be shared across channels or kept separate for each channel. Sharing these modules reduces the number of parameters by a factor of $C$, though at the cost of some modeling flexibility.

Theoretically, a model must learn that a given set of observation tuples can have different implications depending on the channel. When channel aggregation and decoder weights are shared across channels, the model's ability to differentiate between channels is limited to the mixer blocks. However, this weight-sharing approach not only enhances parameter efficiency but also increases robustness against overfitting.

Based on our ablation studies, we found that the trade-off between expressiveness and robustness favors weight-sharing, at least within the evaluated setting.
Since, the channels within one dataset can already be very different, we could theoretically train these modules over different datasets enabling a form of transfer learning. 
We do not share the weights of \model{}'s decoder.