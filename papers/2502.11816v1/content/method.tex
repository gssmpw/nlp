\section{Method}
\Cref{fig:model} depicts an overview over \model{}.
In the subsections below we explain each module, starting with
the channel aggregation method, the key innovation of this work that is depicted in \Cref{fig:obs_enc}. 
Subsequently, we explain how we apply the mixer blocks to the aggregated channels and how we predict for the query time points.
\subsection{Observation-Encoder}\label{sec:obsenc}
To aggregate the observations of a channel in a meaningful way, we first transform observation tuples $(v_i, t_i)$ into higher dimensional vector-embeddings $h_i \in \R^D$.
From here on, we will write $v_i, t_i$ for $v_{c,i}, t_{c,i}$, if the channel $c$ is not relevant. 
First we, encode the scalar value $v$ into the hidden dimension $D$ with a linear layer:
\begin{align}
    v_i^{\text{enc}} = W^{v_\text{enc}} v_i +  b^{v_\text{enc}}, 
\end{align}
where $W^{\text{v}} \in \R^{D}$ and $b^{\text{v}} \in \R^{D}$ are trainable weights and biases.
To include information about the observation time, we multiply the encoded observation value
element-wise with a vector of the same length that is inferred by feeding the scalar observation time through a 2-layer fully connected network with relu activation function:
\begin{align}\label{eq:obs_encoding}
    t^\text{enc}_i &= W_2^{t_\text{enc}}(\relu(W_1^{t_\text{enc}}t_i + b^{t_\text{enc}}_1)) + b^{t_\text{enc}}_2 \\
    h_i &=  t^\text{enc}_i \odot v^\text{enc}_i 
\end{align}

\begin{figure}
    \centering
    \begin{tikzpicture}[x=1cm, y=1cm]
        \draw[fill=veryverylightgreen,rounded corners=3pt] (-0.6,0) rectangle (1.1,0.5); 
        \node at (0.25,0.25) {$T_c \in \R^{N_c \times 1}$};
        \draw[->] (0.25,0.5) -- (0.25,1);

        \draw[fill=veryverylightyellow,rounded corners=3pt] (1.45,0) rectangle (3.15,0.5); 
        \node at (2.3,0.25) {$V_c \in \R^{N_c \times 1}$};
        \draw[->] (2.3,0.5) -- (2.3,1);


        \draw[fill=relublue ,rounded corners=3pt] (1.45,1) rectangle (3.15,1.5); 
        \node at (2.3,1.25) {\small $\FC(1 \to D)$};
        \draw[->, rounded corners=3pt] (2.3,1.5) -- (2.3,2.75) -- (1.4,2.75);
        \draw[->, rounded corners=3pt] (2.3,1.5) -- (2.3,2.75) -- (3.35,2.75);


        %\draw[fill=pastelyellow] (-0.7,0.8) rectangle (1.2,2.6);
        %\node[right] at (-0.7,2.45) {\scriptsize{$\text{MLP}_\text{emb}$}};

        \draw[fill=relublue, opacity=1,rounded corners=3pt] (-0.6,1) rectangle (1.1,2.3); 
        \node at (0.25,1.25) {\small$\FC (1 \to D^{\prime})$};

        \draw[fill=verysoftred,rounded corners=0pt] (-0.6,1.5) rectangle (1.1,1.8); 
        \node at (0.25, 1.65) {\small{relu}};

        %\draw[fill=relublue, opacity=1,rounded corners=3pt] (-0.6,1.8) rectangle (1.1,2.3); 
        \node at (0.25,2) {\small$\FC(D^{\prime} \to D)$};
        \draw[->,rounded corners=3pt] (0.25,2.3) -- (0.25,2.75) -- (1.1,2.75);

    
        %\draw[fill=pastelyellow] (3.8,0.8) rectangle (5.7,2.6);
        %\node[right] at (3.8,2.45) {\scriptsize{$\text{MLP}_\text{w}$}};
        \draw[fill=relublue, rounded corners=3pt] (3.8,1) rectangle (5.5,2.3); 
        \node at (4.65,1.25) {\small $\FC (1 \to D^\prime)$};

        \draw[fill=verysoftred , rounded corners=0pt] (3.8,1.5) rectangle (5.5,1.8); 
        \node at (4.65, 1.65) {\small{relu}};

        %\draw[fill=relublue, rounded corners=3pt] (3.8,1.8) rectangle (5.5,2.3); 
        \node at (4.65,2) {\small $\FC (D^\prime \to D)$};
        \draw[->,rounded corners=3pt] (4.65,2.3) -- (4.65,2.75) -- (3.65,2.75);

        \node at (1.25,2.75) {\large{$\mathbf{\odot}$}};

        \node at (3.5,2.75) {\large{$\mathbf{+}$}};
        \draw[-> ] (1.25,2.9) -- (1.25,3.3);

        \draw[rounded corners=3pt,fill=pastelyellow,opacity=1] (0.25,3.3) rectangle (2.25,3.8) ; 
        \node at (1.25,3.55) {$H_c \in \R^{N_c \times D}$};

        \draw[-> ] (3.5,2.9) -- (3.5,3.3);

        \draw[rounded corners=3pt,fill=pastelyellow,opacity=1] (2.5,3.3) rectangle (4.5,3.8) ; 
        \node at (3.5,3.55) {$A_c \in \R^{N_c \times D}$};
        \draw[->] (3.5,3.8) -- (3.5,4.1) ; 

        \draw[rounded corners=3pt,fill=verysoftred] (2.75,4.1) rectangle (4.25,4.4) ; 
        \node at (3.5,4.25) {\small{softmax}};

        \draw[->,rounded corners=3pt] (0.25,0) --(0.25,-0.5) -- (4.65,-0.5) -- (4.65,1);

        \draw[->,rounded corners=3pt] (1.25,3.8) -- (1.25,4.2) -- (1.7,4.2) -- (1.7,4.7);
        \draw[->,rounded corners=3pt] (3.5,4.4) -- (3.5,4.5) -- (3.05,4.5) -- (3.05,4.7);

        \draw[rounded corners=3pt,fill=transgray,opacity=1] (1.2, 4.7) rectangle (3.4, 5.3);
        \node at (2.3,5) {\small{$\sum^{N_c}_{i=1} H_c^{(i)} \odot W_c^{(i)}$}};
        \draw[->] (2.3,5.3) -- (2.3,5.6) ;


        \draw[rounded corners=3pt,fill=pastelyellow,opacity=1] (1.3,5.6) rectangle (3.3,6.1) ; 
        \node at (2.3,5.85) {$Z_c \in \R^D$};
 




    \end{tikzpicture}
    \caption{\model{}'s channel aggregation architecture.
    For simplicity, we split a channel $X_c$ into the observation times $T_c$ and observed values $V_c$.
    For each channel, the aggregation module shown summarizes a varying number of observation tuples into a fixed-sized vector $Z_c$.
    We use $\FC(A \to B)$ to symbolize a single fully connected linear layer with A input and B output neurons.}\label{fig:obs_enc}
    \Description{An awesome model architecture}
\end{figure}

\subsection{Channel Aggregation}
After embedding each observation into a vector we aggregate the embeddings within a weighted convex sum.
The respective weights $\alpha_i$ are based on the observation time and the observed value. 
Here, we reuse the value-encoding $v^{enc}_i$ that was used to compute the observation embedding and add it to the output of another
neural network that inputs time as a scalar.
\begin{align}\label{eq:weight}
    \alpha_i = W_2^{t_\text{weight}}(\relu(W_1^{t_\text{weight}}t_i + b^{t_\text{weight}}_1)) + b^{t_\text{weight}}_2
\end{align}
We aggregate the observations embeddings of each channel to obtain the channel encoding $Z_c \in \R^D$. Specifically $Z_c$ is the encoding
of the IMTS channel $X_c$. While $X_c$ is a set of arbitrary many observation tuples, $Z_c$ is a vector of size $D$ for every channel $c$.
The computation of the weighted convex sum can be expressed with:   
\begin{align}\label{eq:agg}
    Z_c = \sum^{N_c}_{i=1} \softmax{(A_c)}_i \odot h_i
\end{align}
Here, $A_c \in \R^{N_c \times D}$ is the matrix created by concatenating the $N_c$ many $\alpha_i$-vectors as defined by \Cref{eq:obs_encoding},
with $N_c$ referring to the number of observations in channel $X_c$. The softmax function is applied 
over the rows of $A_c$ to ensure that all weights contained in the vectors $\alpha_i$ are positive and add up to one in each dimension.    

\subsection{Channel-Bias}
The proposed channel encoder shares the weights for all channels and thus only learns to embed them based on the patterns in the observation horizon. 
To learn channel specific information and distinguish the channel embeddings, we add channel biases $b_c \in R^D$. 
The final channel encoding is then computed as
\begin{align}\label{eq:bias}
    Z_c^+ = Z_c + b_c
\end{align}
This helps also for the case where a channel is unobserved in the observation horizon.
In that case, we define the channel encoding simply to be the bias term of that channel. 
In \Cref{sec:bias} we show an ablation study showing that this is sufficient for encoding the channels and no other channel specific weights are necessary.
Note also that the subsequent mixer blocks can learn channel specific patterns as well as mix information across channels.

\subsection{Mixer Blocks}
\begin{figure}
    \centering
    \begin{tikzpicture}[x=1cm, y=1cm]
        \draw[fill=pastelyellow,opacity=1, , rounded corners=5pt] (1.5,0) rectangle (3.5,0.5);
        \node at (2.5, 0.25){$\mathbf{Z^{l-1}} \in \R^{C \times D}$};

        \draw[->] (2.5,0.5) to (2.5,1);         
        \draw[fill=transgray,rounded corners=3pt] (1.5,1) rectangle (3.5,1.5);
        \draw[->] (2.5,1.5) to (2.5,1.85); 
        \node at (2.5, 1.25){Transpose};
        
        \draw [fill=verysoftred,rounded corners=5pt] (1,1.85) rectangle (4,3.05);
        \node at (2.5, 2.){\small{RMS-Norm}};
        \draw [fill=relublue,rounded corners=0pt](1,2.15) rectangle (4,2.75);
        \node at (2.5, 2.45){$\FC (C \to C)$};
        %\draw [fill=transgray, ,rounded corners=0pt] (1,2.75) rectangle (4,3.05);
        \node at (2.5, 2.9){\small{relu}};
        \draw[->] (2.5,3.05) to (2.5,3.3);

        \draw[fill=transgray,rounded corners=3pt] (1.5,3.3) rectangle (3.5,3.8);
        \node at (2.5, 3.55){Transpose};
        \draw[->] (2.5,3.8) to (2.5,4.35);

        \draw[fill=verysoftred,rounded corners=5pt] (1,4.35) rectangle (4,5.55);
        \node at (2.5, 4.5){\small{RMS-Norm}};
        \draw[fill=relublue,rounded corners=0pt] (1,4.65) rectangle (4,5.25);
        \node at (2.5, 4.9){$\FC (D \to D)$};
        %\draw[fill=transgray, ,rounded corners=0pt](1,5.25) rectangle (4,5.55);
        \node at (2.5, 5.4){\small{relu}};

        \draw[->, ] (2.5,5.55) to (2.5,6.2);
        \draw[fill=pastelyellow,opacity=1, ,rounded corners=5pt] (1.5,6.2) rectangle (3.5,6.7);
        \node at (2.5, 6.45){$\mathbf{Z^{l}} \in \R^{C \times D}$};


        \draw[->, rounded corners=5pt] (3.5,0.25) -- (4.7,0.25) -- (4.7,4.1) -- (2.5,4.1);
        \draw[->, rounded corners=5pt] (1.5,0.25) -- (0.5,0.25) -- (0.5,6) -- (2.5,6);
        \draw[->, rounded corners=5pt] (3.5,3.55) -- (4.5,3.55) -- (4.5,5.8) -- (2.5,5.8);

        \node[left] at (0.5,3) {\textbf{+}};
        \node[right] at (4.7,2.5) {\textbf{+}};
        \node[right] at (4.5,4.9) {\textbf{+}};

    \end{tikzpicture}
    \caption{Architecture of the mixer block from IMTS-Mixer.
            $\FC (A \to B) $ refers to a linear layer with $A$ input and $B$ output neurons.}\label{fig:mixer}
    \Description{An awesome model architecture}
\end{figure}
To illustrate equilibria and dynamics of performative prediction games, we focus on a scenario in which a \emph{duopoly} of mortgage companies, i.e. banks, compete to sell loans to customers.

\paragraph{Customer Model:} In our game, each bank is trying to attract customers from a given population $\mathcal{P}$. We model this population as comprised of individuals with a single-dimensional type: we denote individual $j$'s type as $y_j \in [0,1]$. For simplicity, we assume that \(y\) represents the customer’s probability of repaying the loan\footnote{In practice, a customer's (normalized) credit score can be interpreted as a noisy observation of $y_j$. This also corresponds to credit scores being \emph{calibrated}.}, i.e., $y_j := \P[Y_j = 1]$, where $Y_j$ is a random variable such that $Y_j = 0$ means that $j$ defaults on their loan, and $Y_j = 1$ means they repay their loan. Customer types in the population are drawn from a known distribution $D_y$ supported on $[0,1]$. 

\paragraph{Game between Banks:} Each Bank \(i \in \{1, 2\}\) selects two parameters \( (\tau_i, \gamma_i) := \theta_i\), where:
\begin{itemize}
    \item \(\tau_i \in \{\tau_l,\tau_h\}\) is the credit score threshold for approving a customer\footnote{We restrict the bank to only pick between two thresholds, $\tau_l$ and $\tau_h$. However, we highlight how our results are affected when we expand the strategy space to $n > 2$ actions in our experiments of Appendix \ref{app:3gamma}.}. Specifically, a customer $j$ with credit score \(y_j\) is approved by Bank $i$ if and only if \(y_j \geq \tau_i\);
    \item \(\gamma_i \in \{\gamma_l, \gamma_h\}\) is the interest rate offered to approved customers.
\end{itemize}
We denote as shorthand the space of allowable thresholds by $\Gamma := [0,1]$ and allowable interests rates by $\Lambda := [0,1]$. %The latter is set without loss of generality---we simply normalize the rates to be at most $1$. 
% {\color{red} Vidya: just thinking about this but is it natural to restrict interest rate to $1$? I don't think it would affect the equilibrium structure of the game but theoretically I think the interest rate could be anything in $[0,\infty)$.} {\color{green} Guanghui: Could we say something like this is without loss of generality} \gua{changed.}\juba{I think we repeated this twice, the next sentence already had this}
The loan amount is normalized to $1$ in the entire paper, without loss of generality; in this case, if a customer chooses Bank $i$, and the customer is approved by the bank at an interest rate of $\gamma_i$, the expected utility for the bank is equal to
\[
(1+\gamma_i)\cdot \P[Y_i = 1]-\P[Y_i = 0] = (1+\gamma_i)y_i-(1-y_i).
\]


%In practice, the credit score \(y\) serves as a noisy observation of the true likelihood of the customer's repayment. 

\paragraph{Banks' Utilities:} For given parameter choices \(\theta_1 = (\tau_1, \gamma_1)\) by Bank 1 and \(\theta_2 = (\tau_2, \gamma_2)\) by Bank 2, a \emph{rational} customer with credit score $y$ acts as follows:

\begin{enumerate}
    \item \textbf{Qualified for a single bank}: 
        \begin{itemize}
        \item If \(\tau_1 \leq y < \tau_2\), the customer goes to Bank 1, as the score qualifies for Bank 1 but not Bank 2. Conversely, if \(\tau_2 \leq y < \tau_1\), the customer chooses Bank 2.
    \end{itemize}
    \item \textbf{Qualified for both banks}:
     \begin{itemize}
        \item If \(\tau_1, \tau_2 \leq y\) and \(\gamma_1 < \gamma_2\), the customer selects Bank 1 for its lower interest rate. Conversely, if \(\gamma_1 > \gamma_2\), the customer chooses Bank 2.
        \item If \(\gamma_1 = \gamma_2\), the customer picks each bank with probability $1/2$. 
    \end{itemize}
    \item \textbf{Unqualified for both banks}:
    \begin{itemize}
        \item If \(y < \tau_1\) and \(y < \tau_2\), the customer is rejected by both banks.
    \end{itemize}
\end{enumerate}

The expected reward for Bank 1, denoted as \(u_1(\theta_1, \theta_2)\), can then be expressed as:
\begin{align}\label{eq:utility}
    u_1(\theta_1, \theta_2) 
    &=  \mathbb{E}_{y \sim D_y} \left[ \mathbb{I}\{\underbrace{\tau_1 \leq y < \tau_2 \ \cup \ (\tau_1, \tau_2 \leq y \ \cap \ \gamma_1 < \gamma_2)}_{\text{accepted by Bank 1}}\} \cdot \big((1+\gamma_1)y - (1-y)\big) \right] \nonumber\\
    & + \frac{1}{2} \mathbb{E}_{y \sim D_y} \left[ \mathbb{I}\{\underbrace{\tau_1, \tau_2 \leq y \ \cap \ \gamma_1 = \gamma_2}_{\text{accepted by both Banks}}\} \cdot \big((1+\gamma_1)y - (1-y)\big) \right].
\end{align}
Note that the problem is \emph{symmetric}, i.e., the utility function for Bank 2 can be derived by swapping the roles of \(\theta_1\) and \(\theta_2\). I.e., $u_2(\theta_1, \theta_2) = u_1(\theta_2, \theta_1)$. 

% If a bank only attracts customers between thresholds $\tau_a$ and $\tau_b$, for $\tau_a<\tau_b$, we call $[\tau_a,\tau_b]$ the \emph{threshold} range for that bank. For example, if Bank $1$ sets a threshold of $\tau_1$, Bank $2$ a threshold of $\tau_2 > \tau_1$, and $\gamma_1 > \gamma_2$, then Bank 1 has a threshold range of $[\tau_1,\tau_2]$, while bank $2$ has a threshold range of $[\tau_2,1]$.
% Note that the parameters set by \emph{both} banks, i.e. $(\theta_1,\theta_2)$ both influence the threshold range for each of Bank 1 and 2.  If $\tau_1>\tau_2$, $\gamma_1>\gamma_2$, then $\tau_a>\tau_b$, and the bank does not attract any customers. 
% {\color{red} is it possible for $\tau_a > \tau_b$, leading to the bank never attracting customers?} \gua{if $\gamma_1>\gamma_2$, $\tau_1>\tau_2$, then it gets no customer. I think it also makes sense.}\juba{I think we said we wanted to delete the discussion of the threshold range, no?}

% \noindent \textbf{Discrete Model}   
% We now present the discrete version of our model, where the interest rates and thresholds are selected from finite sets \(\Gamma\) and \(\Lambda\), respectively, with $\tau\in[0,1], \gamma\in[0,1]$,  for all $\tau\in\Lambda$ and $\gamma\in\Gamma$, \(|\Gamma| = n\) and \(|\Lambda| = m\). Let \(p_1, p_2 \in \Delta(\Gamma \times \Lambda)\) represent the mixed strategies of the two banks, where \(\Delta(\Gamma \times \Lambda)\) denotes the set of probability distributions over the discrete decision space \(\Gamma \times \Lambda\).


% \begin{Remark}
%    Note that our proposed problem can be reformulated as a standard multi-player performative prediction problem \citep{narang2023multiplayer}. However, in our problem, the data distribution faced by each learner breaks the Lipschitzness assumption of previous work~\citep{hardt2023performative,narang2023multiplayer}. A small modification in one of the learner's thresholds can completely change how demand is allocated across both learners, as is often the case in Bertrand-style games. 
% \end{Remark} 

% \gua{I made some changes to Remark 1, please have a look}
\begin{Remark}
   Previous works in multi-learner performative prediction~\citep{narang2023multiplayer} resort to an insensitivity assumption, i.e., the data distribution faced by each player can only changes slightly when the parameters also change slightly; formally, the data distribution faced by each player is Lipschitz in their decisions. This is immediately not true in our setting: the bank slightly changing its parameters can completely changes the demand distribution of customers it faces. Intuitively, this is because of Bertrand-competition-style effects, where if two banks have similar rates, one bank that lowers their rate by a small amount suddenly captures the entire customer demand that is eligible for that rate.%\juba{made further light edits adding intuition}
   
   In Appendix \ref{Appendix:refumulation}, we discuss this problem more carefully by reformulating our problem in the standard multi-learner performative prediction form given by~\citep{narang2023multiplayer}. We show the distribution is not Lipschitz with respect to the parameters, and thus does not satisfy the insensitivity assumption. 
%Prior work~\citep{hardt2023performative,narang2023multiplayer} showed that, for a general multi-agent performative prediction framework to work, insensitivity assumptions are needed: in the \textbf{worst case}, they can construct settings where the insensitivity assumption does not hold and simple dynamics do not converge anymore. We add nuance to this picture. We will show that our dynamics often converge, even absent insensitivity assumptions, highlighting that while the impossibility results of previous work hold in the worst case, they may not hold in the ``average case'' and especially not in problems motivated by applications. In particular, we will show convergence to a variety of equilibria of our game, and often to symmetric Nash equilibria where insensitivity is immediately violated.
     
\end{Remark}



% \paragraph{Relationship to Performative Prediction} A central point of our work is to highlight that \textcolor{red}{needs writing from intro}. We highlight how our work specifically ties to ``Performative Prediction'' below:


%\textcolor{red}{needs a definition environment}



%Here, \(\E_{\theta_1, \theta_2}\) represents the expected utility of the banks over their respective strategies \((\theta_1, \theta_2)\). These inequalities ensure that neither bank can unilaterally improve its expected utility by deviating from its mixed strategy in the equilibrium.



%and  for all $\tau\in\Gamma$, we have $\tau\in\Lambda$, $(\tau,\gamma)\in[0,1]^2$. Let $\Gamma\times\Lambda$
%In this paper, we focus on the most fundamental case, where there are two choices for each parameter: $0\leq\tau_{\ell}<\tau_{h}\leq 1$, and $0\leq \gamma_{\ell}< \gamma_{h}\leq 1$. In this case, the utility for each pair of decisions forms a $4\times4$ matrix (given in Table \ref{tab:my-table}). We consider the canonical case where $\tau_{\ell}=\frac{1}{2+\gamma_{h}}$, and $\tau_{h}=\frac{1}{2+\gamma_{\ell}}.$ Note that these are natural choices for the thresholds, in the sense that, if there is only one bank and the interest rate is set to be $\gamma$, then $\frac{1}{2+\gamma}$ is the optimal threshold corresponding to the fixed $\gamma$.


%and the thresholds are chosen in $\Lambda=\{\tau^{(1)},\dots,\tau^{(m)}\}$. Here, we only assume that, for each $\gamma\in\Gamma$, there at least exist one $\tau\in\Lambda$ such that $f(\gamma,\tau,1)>0$. Note that this is a very minor assumption, in the sense that, if for a $\gamma$ such that $f(\gamma,\tau,1)<0$ for all $\tau\in\Lambda$, then adopting this decision will lead to negative utility regardless of the opponent's decision, and thus is not an interesting case. 

%\textcolor{red}{The model section is missing the dynamic version of the game. We should clearly define the one-shot and the dynamic game}
% we only considered one-shot case in our paper




Time series mixer networks~\cite{Ekambaram2023.TSMixer,Chen2023.TSMixer} apply fully connected layers along the time and channel dimension alternately.  
They are well-established in regular time series forecasting and yield state-of-the-art forecasting despite exclusively relying on vanilla fully connected layers~\cite{Ekambaram2023.TSMixer,Chen2023.TSMixer}. 

A mixer network can only be applied to 2-D inputs where both dimensions are fixed across instances. 
Additionally, it cannot directly process \nan-values. 
While these conditions are met in regular time series, they can never be satisfied in any IMTS dataset.

However, when we aggregate each channel of an IMTS into fixed-sized vectors as we propose in the previous subsections, we encode an IMTS in matrix form, in which the rows 
directly represent the channels. As a result, we are able to harness the power of mixer networks to learn channel interactions and extract the information relevant to forecasting.   

Instead of using mixer blocks, we could also flatten the hidden state and apply fully connected layers  $\R^{CD} \to \R^{CD}$. 
However, the number of parameters (${(CD)}^2$) of such a layer would explode.
Hence, the resulting model would demand an unnecessary amount of memory while being prone to overfitting.

We define a mixer block as two fully connected layers that are applied subsequently along the hidden dimension $D$ and the input channels $C$.
To learn non-linear dependencies we add relu as an activation function to each layer. 
Aligning with TS-mixer architectures established in RMTS forecasting~\cite{Chen2023.TSMixer,Ekambaram2023.TSMixer}, we add residual connections as they are known to simplify the optimization while enhancing expressiveness~\cite{He2016.Deep}.
Furthermore, we also apply normalization before each fully connected layer as it is known to improve learning deep architectures and was also added to RMTS TS-mixer architectures~\cite{Chen2023.TSMixer,Ekambaram2023.TSMixer}.   
Specifically, we use Rooted-Mean-Square Layer Normalization (RMS-Norm)~\cite{Zhang2019.Root}. 

Theoretically, we could vary the hidden dimension $D$ after each layer as well as the number of (latent) channels $C$. The next channel-to-channel layer would simply have to adjust its input dimension to the output 
dimension of the previous channel-to-channel layer. However, the final channel-to-channel layer is restricted to have an output dimension of $C$, due to our method of utilizing the outcome of the mixer blocks for the final prediction.
We describe the respective method in the next subsection. We decide to keep the output dimension of channel-to-channel layers as $C$. 

Furthermore, we also keep the hidden dimension at $D$ except for the last layer where we allow us to choose a different output dimension $D_\text{out}$.
This decision is supposed to account for the fact that the range in which queries are given can significantly vary from the observation time. 
For example, a model might be tasked to predict a month based on two years.
As the mixer blocks are supposed to map a sequence of observations into a latent encoding of an IMTS's state during the forecasting span~\cite{Chen2023.TSMixer,Ekambaram2023.TSMixer}
the output dimension $D_\text{out}$ should match the necessary complexity. 
Staying in the given example, the aggregated channels of dimension $D$ need to contain all the relevant information 
of two years, but the channel representations emitted by the final mixer block only have to encode information that correspond to one month. 
Therefore, it would be adequate if $D_\text{out}$ is smaller than $D$.   

\model{} contains $L$ mixer blocks, and we define the output of the $l^{th}$ mixer block as:
\begin{align}
    Z^{\prime{(l)}} & = Z^{(l-1)} + \relu(W_C^{(l)} \RMS(Z^{(l-1)}) + b_C^{(l)}) \\
    Z^{(l)} & = Z^{(l-1)} +  Z^{\prime{(l)}} + \relu(W_D^{(l)} \RMS(Z^{\prime{(l)}}) + b_D^{(l)}),
\end{align}
with $W_D^{(l)} \in \R^{D\times D}$, $b_D^{(l)} \in \R^D$, $W_C^{(l)} \in \R^{C \times C}$ and $b_C^{(l)} \in \R^C$.
We use $Z^+$ from \Cref{eq:bias} as the input for the first mixer block ($Z^0 = Z^+$). \Cref{fig:mixer} summarizes the architecture of the mixer blocks as introduced in this subsection.
\subsection{Decoder}
To obtain the forecasts we need to map the hidden state of each channel to the queries and decode it into a scalar for each pair of channel and query. 
Predicting continuous queries from fixed-sized vectors that represent a complete channel is not a novel problem and has been previously solved by concatenating the time stamp to the hidden state.
The final prediction can then be inferred by a neural network~\cite{Zhang.Irregular}. 

However, we propose an alternative method, that closely relates to how we computed the embeddings for observation-tuples, but is reversely applied. 
We input the query time as a scalar into a 2-layer neural network with $D_{out}$ output neurons. 
The vector encoding of the query time is then multiplied element-wisely with the channel encoding and fed into linear layer that aggregates the vector product into a scalar.
Formally, we obtain the prediction $\hat{y}$ of channel $c$ at query time $q_{i,c}$ with:
\begin{align}
    q^{enc}_{i,c} &=  W^{q}_2 \relu(W^{q}_1  q_{i,c}  + b^{q}_{c,1}) + b^{q}_{c,2} \\
    \hat{y}_{i,c} &=  W^{\text{out}} (q^{enc}_{i,c} \odot Z^L_c)  + b^\text{out} 
\end{align}
\subsection{Sharing weights}
The modules for channel aggregation can either be shared across channels or kept separate for each channel. Sharing these modules reduces the number of parameters by a factor of $C$, though at the cost of some modeling flexibility.

Theoretically, a model must learn that a given set of observation tuples can have different implications depending on the channel. When channel aggregation and decoder weights are shared across channels, the model's ability to differentiate between channels is limited to the mixer blocks. However, this weight-sharing approach not only enhances parameter efficiency but also increases robustness against overfitting.

Based on our ablation studies, we found that the trade-off between expressiveness and robustness favors weight-sharing, at least within the evaluated setting.
Since, the channels within one dataset can already be very different, we could theoretically train these modules over different datasets enabling a form of transfer learning. 
We do not share the weights of \model{}'s decoder.