%-------------------------------------------------------------------------------------
% Dependencies
%-------------------------------------------------------------------------------------
\usepackage{comment,url,graphicx,subcaption,relsize}
\usepackage{amssymb,amsfonts,amsmath,amsthm,amscd,dsfont,mathrsfs,mathtools,nicefrac}
\usepackage{float,psfrag,epsfig,color,xcolor,url,hyperref}
% \usepackage{epstopdf,bbm,mathtools,enumitem}
\usepackage{listings}
\usepackage{color}

\definecolor{dkgreen}{rgb}{0,0.6,0}
\definecolor{gray}{rgb}{0.5,0.5,0.5}
\definecolor{mauve}{rgb}{0.58,0,0.82}

\lstset{frame=tb,
  language=Python,
  aboveskip=3mm,
  belowskip=3mm,
  showstringspaces=false,
  columns=flexible,
  basicstyle={\small\ttfamily},
  numbers=none,
  numberstyle=\tiny\color{gray},
  keywordstyle=\color{blue},
  commentstyle=\color{dkgreen},
  stringstyle=\color{mauve},
  breaklines=true,
  breakatwhitespace=true,
  tabsize=3
}

%%%%%%%%%% Page layout
% \footnotesep 14pt
% \floatsep 27pt plus 2pt minus 4pt   
% \textfloatsep 40pt plus 2pt minus 4pt
% \intextsep 27pt plus 4pt minus 4pt
% %%
% \topmargin 0.25 in
% \headsep -0.15in
% \textheight 8.5in
% \oddsidemargin -0.08in
% \evensidemargin -0.08in
% \textwidth 6.4in

%-------------------------------------------------------------------------------------
% Common differentials with a small space in front of them
%-------------------------------------------------------------------------------------
\newcommand{\dt}{\,\text{d} t}
\newcommand{\ddx}{\frac{\text{d}}{\dx}}
\newcommand{\ds}{\,\text{d} s}
\newcommand{\dx}{\,\text{d} x}
\newcommand{\dy}{\,\text{d} y}
\newcommand{\dz}{\,\text{d} z}
\newcommand{\dv}{\,\text{d} v}
\newcommand{\du}{\,\text{d} u}
\newcommand{\dw}{\,\text{d} w}
\newcommand{\dr}{\,\text{d} r}
\newcommand{\dB}{\,\text{d} B} % Brownian motion
\newcommand{\dW}{\,\text{d} W} % Wiener process
\newcommand{\dmu}{\,\text{d} \mu}
\newcommand{\dnu}{\,\text{d} \nu}
\newcommand{\domega}{\,\text{d} \omega}

%-------------------------------------------------------------------------------------
% Set notation
%-------------------------------------------------------------------------------------
\newcommand{\smiddle}{\mathrel{}|\mathrel{}} % Well-spaced \middle | symbol

%-------------------------------------------------------------------------------------
% Environment shortcuts
%-------------------------------------------------------------------------------------
% \def\balign#1\ealign{\begin{align}#1\end{align}}
% \def\baligns#1\ealigns{\begin{align*}#1\end{align*}}
% \def\balignat#1\ealign{\begin{alignat}#1\end{alignat}}
% \def\balignats#1\ealigns{\begin{alignat*}#1\end{alignat*}}
% \def\bitemize#1\eitemize{\begin{itemize}#1\end{itemize}}
% \def\benumerate#1\eenumerate{\begin{enumerate}#1\end{enumerate}}

% Align environments that use textstyle instead of displaystyle
% \newenvironment{talign*}
%  {\let\displaystyle\textstyle\csname align*\endcsname}
%  {\endalign}
% \newenvironment{talign}
%  {\let\displaystyle\textstyle\csname align\endcsname}
%  {\endalign}

\def\balignst#1\ealignst{\begin{talign*}#1\end{talign*}}
\def\balignt#1\ealignt{\begin{talign}#1\end{talign}}
%---------------------------------------------------

%-------------------------------------------------------------------------------------
%Text with quads around it
%-------------------------------------------------------------------------------------
\newcommand{\qtext}[1]{\quad\text{#1}\quad} 

%-------------------------------------------------------------------------------------
% Redefine left and right to remove initial and trailing space
%-------------------------------------------------------------------------------------
\let\originalleft\left
\let\originalright\right
\renewcommand{\left}{\mathopen{}\mathclose\bgroup\originalleft}
\renewcommand{\right}{\aftergroup\egroup\originalright}

%-------------------------------------------------------------------------------------
% Words with special symbols
%-------------------------------------------------------------------------------------
\def\Gronwall{Gr\"onwall\xspace}
\def\Holder{H\"older\xspace}
\def\Ito{It\^o\xspace}
\def\Nystrom{Nystr\"om\xspace}
\def\Schatten{Sch\"atten\xspace}
\def\Matern{Mat\'ern\xspace}

%-------------------------------------------------------------------------------------
% Smaller citations
%-------------------------------------------------------------------------------------
\def\tinycitep*#1{{\tiny\citep*{#1}}}
\def\tinycitealt*#1{{\tiny\citealt*{#1}}}
\def\tinycite*#1{{\tiny\cite*{#1}}}
\def\smallcitep*#1{{\scriptsize\citep*{#1}}}
\def\smallcitealt*#1{{\scriptsize\citealt*{#1}}}
\def\smallcite*#1{{\scriptsize\cite*{#1}}}

%-------------------------------------------------------------------------------------
% Colors
%-------------------------------------------------------------------------------------
\def\blue#1{\textcolor{blue}{{#1}}}
\def\green#1{\textcolor{green}{{#1}}}
\def\orange#1{\textcolor{orange}{{#1}}}
\def\purple#1{\textcolor{purple}{{#1}}}
\def\red#1{\textcolor{red}{{#1}}}
\def\teal#1{\textcolor{teal}{{#1}}}

%-------------------------------------------------------------------------------------
% Font styles
%-------------------------------------------------------------------------------------
\def\mbi#1{\boldsymbol{#1}} % Bold and italic (math bold italic)
\def\mbf#1{\mathbf{#1}}
\def\mbb#1{\mathbb{#1}}
\def\mc#1{\mathcal{#1}}
\def\mrm#1{\mathrm{#1}}
\def\tbf#1{\textbf{#1}}
\def\tsc#1{\textsc{#1}}
%-------------------------------------------------------------------------------------
% Bold and italic variables
%-------------------------------------------------------------------------------------
\def\mbiA{\mbi{A}}
\def\mbiB{\mbi{B}}
\def\mbiC{\mbi{C}}
\def\mbiDelta{\mbi{\Delta}}
\def\mbif{\mbi{f}}
\def\mbiF{\mbi{F}}
\def\mbih{\mbi{g}}
\def\mbiG{\mbi{G}}
\def\mbih{\mbi{h}}
\def\mbiH{\mbi{H}}
\def\mbiI{\mbi{I}}
\def\mbim{\mbi{m}}
\def\mbiP{\mbi{P}}
\def\mbiQ{\mbi{Q}}
\def\mbiR{\mbi{R}}
\def\mbiv{\mbi{v}}
\def\mbiV{\mbi{V}}
\def\mbiW{\mbi{W}}
\def\mbiX{\mbi{X}}
\def\mbiY{\mbi{Y}}
\def\mbiZ{\mbi{Z}}

%-------------------------------------------------------------------------------------
% Textstyle vs. displaystyle
%-------------------------------------------------------------------------------------
\def\textsum{{\textstyle\sum}} % Sum in textstyle form
\def\textprod{{\textstyle\prod}} % Prod in textstyle form
\def\textbigcap{{\textstyle\bigcap}} % Bigcap in textstyle form
\def\textbigcup{{\textstyle\bigcup}} % Bigcup in textstyle form

%-------------------------------------------------------------------------------------
% Mathematical sets
%-------------------------------------------------------------------------------------
\def\reals{\mathbb{R}} % Real number symbol
\def\integers{\mathbb{Z}} % Integer symbol
\def\rationals{\mathbb{Q}} % Rational numbers
\def\naturals{\mathbb{N}} % Natural numbers
\def\complex{\mathbb{C}} % Complex numbers

%-------------------------------------------------------------------------------------
% Special symbols
%-------------------------------------------------------------------------------------
\def\<{\left\langle} % Angle brackets
\def\>{\right\rangle}

\def\iff{\Leftrightarrow}
%\def\choose#1#2{\left(\begin{array}{c}{#1} \\ {#2}\end{array}\right)}
\def\chooses#1#2{{}_{#1}C_{#2}}
\def\defeq{\triangleq} % defined equal to
\def\bs{\backslash} % backslash
\def\half{\frac{1}{2}}
\def\nhalf{\nicefrac{1}{2}}
\def\textint{{\textstyle\int}} % Sum in textstyle form
\def\texthalf{{\textstyle\frac{1}{2}}}
\newcommand{\textfrac}[2]{{\textstyle\frac{#1}{#2}}}

% Semidefinite orders
\newcommand{\psdle}{\preccurlyeq}
\newcommand{\psdge}{\succcurlyeq}
\newcommand{\psdlt}{\prec}
\newcommand{\psdgt}{\succ}

% \newcommand{\vo}{\vec{o}\@ifnextchar{^}{\,}{}}

%-------------------------------------------------------------------------------------
% Vectors and matrices
%-------------------------------------------------------------------------------------
\newcommand{\boldone}{\mbf{1}} % Bold 1
\newcommand{\ident}{\mbf{I}} % Identity matrix
% \def\v#1{\mbi{#1}} % Vector notation
\def\norm#1{\left\|{#1}\right\|} % A norm with 1 argument
\newcommand{\onenorm}[1]{\norm{#1}_1} % L1 norm
\newcommand{\twonorm}[1]{\norm{#1}_2} % L2 norm
\newcommand{\infnorm}[1]{\norm{#1}_{\infty}} % Linfty norm
\newcommand{\opnorm}[1]{\norm{#1}_{\text{op}}} % Operator norm
\newcommand{\fronorm}[1]{\norm{#1}_{\text{F}}} % Frobenius norm
\newcommand{\nucnorm}[1]{\norm{#1}_{*}} % Frobenius norm
\def\staticnorm#1{\|{#1}\|} % A static norm that does not resize with input
\newcommand{\statictwonorm}[1]{\staticnorm{#1}_2} % L2 norm
\newcommand{\inner}[1]{{\langle #1 \rangle}} % inner product
\newcommand{\binner}[2]{\left\langle{#1},{#2}\right\rangle} % Inner product with expandable brackets
\def\what#1{\widehat{#1}}

\def\twovec#1#2{\left[\begin{array}{c}{#1} \\ {#2}\end{array}\right]}
\def\threevec#1#2#3{\left[\begin{array}{c}{#1} \\ {#2} \\ {#3} \end{array}\right]}
\def\nvec#1#2#3{\left[\begin{array}{c}{#1} \\ {#2} \\ \vdots \\ {#3}\end{array}\right]} % An n-vector with three arguments

% ------------------------------------------------------------------------
% Eigenvalues
% ------------------------------------------------------------------------
\def\maxeig#1{\lambda_{\mathrm{max}}\left({#1}\right)}
\def\mineig#1{\lambda_{\mathrm{min}}\left({#1}\right)}

%-------------------------------------------------------------------------------------
% Operators
%-------------------------------------------------------------------------------------
\def\Re{\operatorname{Re}} % Real part
\def\indic#1{\mathds{1}\left[{#1}\right]} % Indicator function
\def\logarg#1{\log\left({#1}\right)} % log with argument
\def\polylog{\operatorname{polylog}}
\def\maxarg#1{\max\left({#1}\right)} % max with argument
\def\minarg#1{\min\left({#1}\right)} % min with argument
\def\E{\mbb{E}} % Expectation symbol
\def\Earg#1{\E\left[{#1}\right]}
\def\Esub#1{\E_{#1}}
\def\Esubarg#1#2{\E_{#1}\left[{#2}\right]}
\def\bigO#1{\mathcal{O}\left(#1\right)} % big-oh notation
\def\littleO#1{o(#1)} % big-oh notation
\def\P{\mbb{P}} % Probability symbol
\def\Parg#1{\P\left({#1}\right)}
\def\Psubarg#1#2{\P_{#1}\left[{#2}\right]}
\DeclareMathOperator{\Tr}{Tr} % Trace
\def\T{\top} % transpose
\def\Trarg#1{\Tr\left[{#1}\right]} % Trace with argument
\def\trarg#1{\tr\left[{#1}\right]} % trace with argument
\def\Var{\mrm{Var}} % Variance symbol
\def\Vararg#1{\Var\left[{#1}\right]}
\def\Varsubarg#1#2{\Var_{#1}\left[{#2}\right]}
\def\Cov{\mrm{Cov}} % Covariance symbol
\def\Covarg#1{\Cov\left[{#1}\right]}
\def\Covsubarg#1#2{\Cov_{#1}\left[{#2}\right]}
\def\Corr{\mrm{Corr}} % Covariance symbol
\def\Corrarg#1{\Corr\left[{#1}\right]}
\def\Corrsubarg#1#2{\Corr_{#1}\left[{#2}\right]}
\newcommand{\info}[3][{}]{\mathbb{I}_{#1}\left({#2};{#3}\right)} % Information symbol
%\renewcommand{\exp}[1]{\operatorname{exp}\left(#1\right)} % Exponential
\newcommand{\staticexp}[1]{\operatorname{exp}(#1)} % An exponential with parens that do not resize with input
\newcommand{\loglihood}[0]{\mathcal{L}} % log likelihood

%-------------------------------------------------------------------------------------
% Distributions
%-------------------------------------------------------------------------------------
\def\normal{{\sf N}}
\newcommand{\Gsn}{\mathcal{N}}
\newcommand{\BeP}{\textnormal{BeP}}
\newcommand{\Ber}{\textnormal{Ber}}
\newcommand{\Bern}{\textnormal{Bern}}
\newcommand{\Bet}{\textnormal{Beta}}
\newcommand{\Beta}{\textnormal{Beta}}
\newcommand{\Bin}{\textnormal{Bin}}
\newcommand{\BP}{\textnormal{BP}}
\newcommand{\Dir}{\textnormal{Dir}}
\newcommand{\DP}{\textnormal{DP}}
\newcommand{\Expo}{\textnormal{Expo}}
\newcommand{\Gam}{\textnormal{Gamma}}
\newcommand{\GEM}{\textnormal{GEM}}
\newcommand{\HypGeo}{\textnormal{HypGeo}}
\newcommand{\Mult}{\textnormal{Mult}}
\newcommand{\NegMult}{\textnormal{NegMult}}
\newcommand{\Poi}{\textnormal{Poi}}
\newcommand{\Pois}{\textnormal{Pois}}
\newcommand{\Unif}{\textnormal{Unif}}


%-------------------------------------------------------------------------------------
% Derivative symbols
%-------------------------------------------------------------------------------------
\newcommand{\grad}{\nabla}
\newcommand{\Hess}{\nabla^2} % Hessian
\newcommand{\lapl}{\triangle} % Laplace operator / Laplacian
\newcommand{\deriv}[2]{\frac{d #1}{d #2}} % derivative
\newcommand{\pderiv}[2]{\frac{\partial #1}{\partial #2}} % partial derivative

%-------------------------------------------------------------------------------------
% Probability and statistics macros
%-------------------------------------------------------------------------------------
\newcommand{\eqdist}{\stackrel{d}{=}}
\newcommand{\todist}{\stackrel{d}{\to}}
\newcommand{\eqd}{\stackrel{d}{=}}
\def\KL#1#2{\textnormal{KL}({#1}\Vert{#2})}
\def\independenT#1#2{\mathrel{\rlap{$#1#2$}\mkern4mu{#1#2}}}
%\def\indep{\perp\!\!\!\perp} % conditional independence

%-------------------------------------------------------------------------------------
% Optimization macros
%-------------------------------------------------------------------------------------
\providecommand{\argmax}{\mathop\mathrm{arg max}} % Defining math symbols
\providecommand{\argmin}{\mathop\mathrm{arg min}}
\providecommand{\arccos}{\mathop\mathrm{arccos}}
\providecommand{\dom}{\mathop\mathrm{dom}}
\providecommand{\diag}{\mathop\mathrm{diag}}
\providecommand{\tr}{\mathop\mathrm{tr}}
%\providecommand{\abs}{\mathop\mathrm{abs}}
\providecommand{\card}{\mathop\mathrm{card}}
\providecommand{\sign}{\mathop\mathrm{sign}}
\providecommand{\conv}{\mathop\mathrm{conv}} % Convex hull
\def\rank#1{\mathrm{rank}({#1})}
\def\supp#1{\mathrm{supp}({#1})}

\providecommand{\minimize}{\mathop\mathrm{minimize}}
\providecommand{\maximize}{\mathop\mathrm{maximize}}
\providecommand{\subjectto}{\mathop\mathrm{subject\;to}}

%\renewcommand\eqref[1]{Eq.~(\ref{#1})}

\def\openright#1#2{\left[{#1}, {#2}\right)}

%-------------------------------------------------------------------------------------
% Proof environments
%-------------------------------------------------------------------------------------
\ifdefined\nonewproofenvironments\else
% The Theorems are numbered consecutively
% Lemmas are numbered by section, and observations, claims, facts, and 
% assumptions take their numbering. Propositions and definitions have their
% own numbering by section.
\ifdefined\ispres\else
% These conflict with Beamer definitions in pres mode
% \newtheorem{theorem}{Theorem}
% \newtheorem{lemma}[theorem]{Lemma}
% \newtheorem{corollary}[theorem]{Corollary}
% \newtheorem{definition}[theorem]{Definition}
% \newtheorem{fact}[theorem]{Fact}
\renewenvironment{proof}{\noindent\textbf{Proof.}\hspace*{.3em}}{\qed\\}
\newenvironment{proof-sketch}{\noindent\textbf{Proof Sketch}
  \hspace*{1em}}{\qed\bigskip\\}
\newenvironment{proof-idea}{\noindent\textbf{Proof Idea}
  \hspace*{1em}}{\qed\bigskip\\}
\newenvironment{proof-of-lemma}[1][{}]{\noindent\textbf{Proof of Lemma {#1}}
  \hspace*{1em}}{\qed\\}
\newenvironment{proof-of-theorem}[1][{}]{\noindent\textbf{Proof of Theorem {#1}}
  \hspace*{1em}}{\qed\\}
\newenvironment{proof-attempt}{\noindent\textbf{Proof Attempt}
  \hspace*{1em}}{\qed\bigskip\\}
\newenvironment{proofof}[1]{\noindent\textbf{Proof of {#1}}
  \hspace*{1em}}{\qed\bigskip\\}
\newenvironment{remark}{\noindent\textbf{Remark.}
  \hspace*{0em}}{\smallskip}%\bigskip}
\newenvironment{remarks}{\noindent\textbf{Remarks}
  \hspace*{1em}}{\smallskip}
\fi
% \newtheorem{observation}[theorem]{Observation}
% \newtheorem{proposition}[theorem]{Proposition}
% \newtheorem{claim}[theorem]{Claim}
% \newtheorem{assumption}{Assumption}
%\renewcommand{\theassumption}{\Alph{assumption}} % Set counter for assumptions
                                                 % to be alphabetical
\fi
% Makes equation numbers have (1.1) style
% \numberwithin{equation}{section}
% \numberwithin{equation}{subsection}
\makeatletter
\@addtoreset{equation}{section}
\makeatother
\def\theequation{\thesection.\arabic{equation}}

\hypersetup{
  colorlinks,
  linkcolor={red!50!black},
  citecolor={blue!50!black},
  urlcolor={blue!80!black}
}

%-------------------------------------------------------------------------------------
% Equation environments
%-------------------------------------------------------------------------------------
\newcommand{\eq}[1]{\begin{align}#1\end{align}}
\newcommand{\eqn}[1]{\begin{align*}#1\end{align*}}
\renewcommand{\Pr}[1]{\mathbb{P}\left( #1 \right)}
\newcommand{\Ex}[1]{\mathbb{E}\left[#1\right]}
\newcommand{\var}[1]{\text{Var}\left(#1\right)}
\newcommand{\ind}[1]{{\mathbbm{1}}_{\{ #1 \}} }
\newcommand{\abs}[1]{\left|#1\right|}


%-------------------------------------------------------------------------------------
% Comment environments
%-------------------------------------------------------------------------------------
\newcommand{\murat}[1]{{\color{magenta}\bf[Murat: #1]}}
\newcommand{\notate}[1]{\textcolor{blue}{\textbf{[#1]}}}

%-------------------------------------------------------------------------------------
% Lecture template
%-------------------------------------------------------------------------------------
\newcommand{\handout}[5]{
  \noindent
  \begin{center}
    \framebox{
      \vbox{
        \hbox to 5.78in { {\bf \title } \hfill #2 }
        \vspace{4mm}
        \hbox to 5.78in { {\Large \hfill #5  \hfill} }
        \vspace{2mm}
        \hbox to 5.78in { {\em #3 \hfill #4} }
      }
    }
  \end{center}
  \vspace*{4mm}
}
% \newcommand{\lecture}[4]{\handout{#1}{#2}{#3}{Scribes: #4}{Lecture #1}}
\newcommand{\homework}[4]{\handout{#1}{#2}{#3}{Collaborators: #4}{Homework #1}}
\newcommand{\project}[4]{\handout{#1}{#2}{#3}{Collaborators: #4}{Project Report: #1}}

\newcommand{\riskhat}{\hat{R}}
\newcommand{\fat}{\hat{f}}