\section{Related Work}
\label{sec:related}

David Patterson and John Hennessy coined the term \emph{reduced instruction set computer} (RISC).
The first publicly available RISC processor was the RISC I~\cite{risc:1980} \cite{risc1:patterson:1981},
designed at UCB and MIPS~\cite{MIPS:1982}, designed at Stanford University.
The IBM 801 predates the two RISC processors from academia, but details have not been disclosed
in the early 80's.

%\url{https://en.wikipedia.org/wiki/Reduced_instruction_set_computer}

The RISC-V project started with Andrew's thesis~\cite{Waterman:EECS-2016-1}.
To verify his work on defining \emph{the} RISC instruction set, the research group in Berkeley
developed several versions of a RISC-V microprocessor. The initial version was an
instruction set simulator called Spike. The hardware implementation following the simulator
was called Rocket~\cite{rocket:techrep}. Rocket is implemented in Chisel and represents a
5-stage in-order scalar pipeline. The project is also called the Rocket Chip generator, as it is a collection
of tools to generate RISC-V-based system-on-chips. Besides
the Rocket in-order pipelined core, the generator contains components
with the TileLink protocol, several tools, and Diplomacy~\cite{cook2017diplomatic}.
Diplomacy is an extension package for Chisel, providing support for two-phase
hardware elaboration. This approach enables dynamic negotiation of specific
parameters between modules.
Rocket was not intended as a reference implementation of a RISC-V microprocessor,
but is considered as a reference by many.


As Rocket was a bit too advanced for education, a group in Berkeley developed the
Sodor family of RISC-V processors.
The Sodor project\footnote{\url{https://github.com/ucb-bar/riscv-sodor}} is an open-source initiative that contains educational RISC-V processor cores.
The cores range from a single-stage, 2-stage, 3-stage, up to a classic 5-stage version.
Sodor is written in Chisel. However, some of the code is quite advanced, such as Scala code,
which is not an easy read for a beginner of computer architecture and Chisel.
Furthermore, the Sodor project is no longer self-contained. It needs the
The Chipyard SoC generator itself has an elaborate setup.
In contrast to Sodor, Wildcat is available in a standalone, and we have put effort
into producing readable code instead of showing off with clever but hard-to-read solutions.
Furthermore, Sodor cannot easily be used in an FPGA, as it depends on asynchronous memories:

\begin{quotation}
All processors talk to a simple scratchpad memory (asynchronous, single-cycle), with no backing outer memory.... Programs are loaded in via a Host-target Interface (HTIF) port (while the core is kept in reset), effectively making the scratchpads 3-port memories (instruction, data, HTIF).
\end{quotation}

YARVI (Yet Another RISC-V Implementation)~\footnote{\url{https://github.com/tommythorn/yarvi}} by Tommy Thorn was probably the first RISC-V implementation that could be synthesized into an FPGA (originally released in 2014).
The first implementation was a multi-cycle version followed by a pipelined version.
The current version of YARI, called YARVI2,  is an 8-stage pipeline with an effort on branch prediction.
The website reports a maximum clock frequency of over 100 MHz on a Cyclone~V.
YARVI, with its 8-stage in-order pipeline, shows a path for future versions of Wildcat.

PULPino~\cite{PULPino} is a 32-bit RISC-V microcontroller system developed at ETH Zurich.
It is written in SystemVerilog in a conservative style (e.g., not using structures but
individual signals.) We appreciate that project but consider it too complex for education or research.

Ibex~\cite{Ibex2017}\footnote{\url{https://github.com/lowRISC/ibex}} is a two-stage pipeline
with an additional clock cycle for memory access. Therefore, similar in design to our
Wildcat project. The pipeline can be extended with a write-back stage. The documentation
states:

\begin{quotation}
Ibex can be configured to have a third pipeline stage (Writeback) which has major effects on performance and instruction behavior. The details of its impact are not yet documented here.
\end{quotation}


PicoRV32~\footnote{\url{https://github.com/YosysHQ/picorv32}} is a RISC-V implementation optimized for small size and a high clocking frequency but not for execution speed. The application area is as an auxiliary processor in FPGA or ASIC designs. The focus on high clock frequency shall simplify the integration into existing designs without the need for clock domain crossing. The implementation is sequential, with instructions taking between 3 and 14 clock cycles.
The single Verilog file picorv32.v is about 3000 lines of Verilog code, which is not an easy read.

RISC-V processors are specialized for different domains. For example,
MINOTAuR~\cite{gruin2021speculative} is a time-predictable RISC-V core aiming to be used
in real-time systems. Although Wildcat's focus is currently as an educational RISC-V core,
we plan to add features from the T-CREST platform~\cite{t-crest:dasia:2014} to use it in real-time systems.

Ripes~\cite{ripes} is a graphical simulator for different configurations of a RISC-V pipeline.
Morten developed Ripes while he was still a student at DTU. We use Ripes to educate students in computer architecture.
Ripes seems to have become the most used RISC-V simulator to teach computer architecture.
Another RISC-V simulator is WebRISC-V~\cite{giorgi2019webriscv}. The simulator was originally
developed for MIPS and has been adapted to the RISC-V instruction set.

There are many RISC-V implementations, both open-source and commercial closed-source. Describing them all would result in a long survey paper.
A small set of application class RISC-V implementations has been compared~\cite{dorflinger2021comparative} concerning performance, area, and power.

\subsection{Source Access}

Wildcat is available in open source on GitHub:
%\emph{hidden for double blind review}
\url{https://github.com/schoeberl/wildcat}.