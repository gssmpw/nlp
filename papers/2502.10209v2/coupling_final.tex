%\documentclass[12pt, draftclsnofoot, onecolumn]{IEEEtran}
\documentclass[journal]{IEEEtran}
%\IEEEoverridecommandlockouts
% The preceding line is only needed to identify funding in the first footnote. If that is unneeded, please comment it out.
\usepackage{cite}
\usepackage{amsmath,amssymb,amsfonts,amsthm}
\usepackage{mathtools}
\usepackage{algorithmic}
%\usepackage{physics}
\usepackage{graphicx}
\usepackage{textcomp}
\usepackage[T1]{fontenc}
\usepackage{caption}
\usepackage{subcaption}
\usepackage{epstopdf}
\usepackage{overpic}
\usepackage{array}
%\usepackage{color}
\usepackage[monochrome]{color}
\usepackage{url}
\usepackage{bm}
\usepackage{bbm}

\usepackage{hyperref}
\usepackage{lipsum}
\usepackage{overpic}

\def\BibTeX{{\rm B\kern-.05em{\sc i\kern-.025em b}\kern-.08em
    T\kern-.1667em\lower.7ex\hbox{E}\kern-.125emX}}
    
\usepackage{color}
%\usepackage[monochrome]{color}
 
\newcommand{\Ttran}{\rm \scriptscriptstyle T}
\newcommand{\Htran}{\rm \scriptscriptstyle H}
%\def\Htran{\mbox{\tiny $\mathrm{H}$}}
%\def\Ttran{\mbox{\tiny $\mathrm{T}$}}
\def\CN{\mathcal{N}_{\mathbb{C}}}
\def\CW{\mathcal{W}_{\mathcal{C}}}
\def\Real{\mathbb{R}}
\def\Complex{\mathbb{C}}
\def\Integer{\mathbb{Z}}
\def\Natural{\mathbb{N}}
\def\Ex{\mathbb{E}}
\def\sinc{\mathrm{sinc}}
\def\rank{\mathrm{rank}}
\def\kron{\otimes}
\def\diag{\mathrm{diag}}
\def\tr{\mathrm{tr}} 
\def\vec{\mathrm{vec}}
\def\DoF{\mathrm{DoF}}
\def\Exp{\mathrm{Exp}}
\def\imagunit{{\sf j}} % Imaginary number

\newcommand{\argmax}[1]{{\underset{{#1}}{\mathrm{arg\,max}}}}
\newcommand{\argmin}[1]{{\underset{{#1}}{\mathrm{arg\,min}}}}
\newcommand{\vect}[1]{{\boldsymbol{#1}}}
\newcommand{\maximize}[1]{{\underset{{#1}}{\mathrm{maximize}}}}
\newcommand{\minimize}[1]{{\underset{{#1}}{\mathrm{minimize}}}}
\DeclareMathOperator*{\dprime}{\prime \prime}
%\renewcommand\qedsymbol{$\blacksquare$}

\theoremstyle{plain}
\newtheorem{example}{Example}
\newtheorem{theorem}{Theorem}
\newtheorem{definition}{Definition}
\newtheorem{lemma}{Lemma}
\newtheorem{corollary}{Corollary}
\newtheorem{remark}{Remark}
\newtheorem{proposition}{Proposition}

\captionsetup[figure]{font=small}
%\captionsetup[figure]{labelfont={bf}}
%\captionsetup[table]{labelfont={bf}}
%\captionsetup[figure]{labelformat=simple, labelsep=period}
%\captionsetup[table]{labelformat=simple, labelsep = period}

%\usepackage[margin=0.9in]{geometry}

%\IEEEoverridecommandlockouts

\graphicspath{{./images/}}

\begin{document}

%\title{%A First Glimpse at the Impact of Mutual Coupling on Holographic MIMO}
%\title{Mutual Coupling in Holographic MIMO: Physical Modeling and SNR-Limiting Regimes}
\title{Mutual Coupling in Holographic MIMO: Physical Modeling and Information-Theoretic Analysis}
%\thanks{Identify applicable funding agency here. If none, delete this.}

\definecolor{alizarin}{rgb}{0.82, 0.1, 0.26}
\definecolor{ao}{rgb}{0.0, 0.5, 0.0}
\newcommand{\angel}[1]{\noindent { {{$\blacktriangleright$ 
   {\textsf{[Angel]: {\color{red}#1}}} $\blacktriangleleft$}}}}
\newcommand{\andrea}[1]{\noindent { {{$\blacktriangleright$ 
   {\textsf{[Andrea]: {\color{ao}#1}}} $\blacktriangleleft$}}}}
   
\author{
\IEEEauthorblockN{Andrea Pizzo, \emph{Member, IEEE} and Angel Lozano, \emph{Fellow, IEEE}}
\thanks{
%\newline \indent Part of this work was presented at the IEEE Asilomar Conference on signals, Systems and Computers, Pacific Grove, CA, 2021 \cite{ASILOMAR}. 
A.~Pizzo and A.~Lozano are with the Department of Engineering, Universitat Pompeu Fabra, 08018 Barcelona, Spain (andrea.pizzo@upf.edu, angel.lozano@upf.edu).
Work supported by MICIU under the Maria de Maeztu Units of Excellence Programme (CEX2021-001195-M), by ICREA, and by AGAUR (Catalan Government).
%T. L.~Marzetta is with the Department of Electrical and Computer Engineering, New York University, 11201 Brooklyn, NY (tom.marzetta@nyu.edu).
}
}

% make the title area

\maketitle

\begin{abstract}
%This paper introduces a comprehensive framework for holographic multiantenna systems, a novel paradigm encompassing both closely spaced antennas and wide apertures relative to the wavelength. The framework is grounded in physics and inherently accounts for spatial effects, including fading correlation and mutual coupling among antennas. It shows how coupling can be harnessed to enhance channel capacity across varying signal-to-noise-ratio levels by adjusting antenna patterns to manipulate correlation. This framework provides a rigorous foundation for the design and analysis of reconfigurable communication systems that dynamically adapt to changes in the environment.
This paper presents a comprehensive framework for holographic multiantenna communication, a paradigm that integrates both wide apertures and
 closely spaced antennas relative to the wavelength. The presented framework is physically grounded, enabling information-theoretic analyses that inherently incorporate correlation and mutual coupling among the antennas. 
This establishes the combined effects of correlation and coupling on the information-theoretic performance limits across SNR levels.
% with given antenna patterns and fading selectivity.
Additionally, it reveals that, by suitably selecting the individual antenna patterns, mutual coupling can be harnessed to either reinforce or counter spatial correlations as appropriate for specific SNRs, thereby improving the performance.
\end{abstract}

\smallskip
\begin{IEEEkeywords}
Holographic MIMO, mutual coupling, fading correlation, channel capacity, SNR-limiting analysis.
\end{IEEEkeywords}

\IEEEpeerreviewmaketitle


\section{Introduction}
%\lipsum[1-3]

%%%%
%MOTIVATE HOLOGRAPHIC MIMO IN BROAD GENERALITY, BUT ALSO GIVE EXAMPLES OF SPECIFIC EMBODIMENTS WITH REFERENCES \cite{BJORNSON20193,Huang2020}
%%%%
%Multiple-input-multiple-output (MIMO) communications have reached the limits of their current evolution. % evolutionary road they have followed thus far. 


The progress of multiple-input multiple-output (MIMO) communication entails expanding the array apertures relative to the wavelength, %thereby increasing the number of spatial dimensions and enhancing angular resolvability \cite{10144733}. 
so as to %linearly
augment the number of spatial dimensions %and enhance the angular resolution
\cite{10144733}.
%spatial multiplexing and angular resolvability \cite{TseBook}.
%for spatial multiplexing and diversity
This is accomplished by physically enlarging the apertures and/or by increasing the carrier frequency, with antenna spacings at or above a half-wavelength as dictated by the channel's scattering richness \cite{BJORNSON20193} and in line with the sampling theorem \cite{PizzoTSP21}.
% the latter is nothing but an embodiment of the sampling theorem \cite{PizzoTSP22}.
%Recognizing physical constraints on finite apertures \angel{too vague}, 
However, both approaches face challenges and are ultimately curbed: exceedingly large apertures become difficult to implement and deploy, while escalating frequencies compromise the range.
% and the scattering richness (\cite{}?)---which in turn forces broader antenna spacings, hampering the very increase in dimensionality that was being sought. 
%\andrea{if we think about the DOF in paraxial settings, diminishing range enhances angular richness. This is true for LOS and NLOS scenarios with close scatterers as per the image theorem.}
%While physically large arrays face implementation and deployment challenges, %while also encountering non-stationary propagation.
%through increased path losses, signal blockage, and limited diffraction \cite{}.
Further progress may be possible by shrinking the antenna spacing below %a half-wavelength
the Nyquist distance, an idea that defines the holographic MIMO
paradigm \cite{Prather2016}. %with faster-than-Nyquist spatial sampling % offering  the following.
%{\color{blue} allowing the encoding and transmission of an arbitrarily large amount of information within a finite aperture for a given channel. Formalized for the time domain in \cite{Kempf2000}, this feature straightforwardly extends to the space domain.}
%{\color{magenta}with faster-than-Nyquist spatial sampling increasing power through higher antenna density, but adds little spatial dimensions \cite{PoonDoF}.}
%\andrea{I would emphasize the ability of coupling to create new spatial dimensions. The argument about packing information in a small interval is perfect for our other paper on superdirectivity!}
%power with given apertures thanks 
%At low SNR, it manifests through superlinear gains in array directivity along specific directions, aptly named superdirectivity \cite{Schelkunoff}.
%\andrea{I have compressed this part by removing the bullet list. The first point on array gain overlapped with superdirectivity, and \cite{Kempf2000} incorporated both. In this way, we save space for comments about pattern diversity and DOF augmentation in the introduction.}
%\angel{Ok. But the point on superdirectivity relies on coupling, which hasn't been introduced yet. This paragraph was meant to expose the classical motivation for holoMIMO, which usually disregards coupling}
%{\color{blue}Superlinear gains in array directivity along specific directions, aptly named superdirectivity \cite{Schelkunoff}.}
%Encoding and transmitting %an arbitrarily large amount of information in an arbitrarily short time interval of a 1Hz bandlimited signal
%information within arbitrarily small apertures over a given channel. Formalized for the time domain in \cite{Kempf2000}, this feature straightforwardly extends to the space domain.} 
%the following. 
%offers further enhancements by leveraging the finite aperture size relative to the wavelength
%\angel{It's a weird acronym, I'd refrain from introducing it, my impression is that we don't use the term that often.}
%\begin{itemize}
%\item
%Additional power with given apertures thanks to the increased antenna density.
%\item
%\andrea{True for uncoupled antennas. We should stress this point and maybe lower the emphasis on second-order effects.}
%\angel{Yes, very important. One possibility is to directly remove this point, since it steals the thunder of "our" DOF gains and its an effect that disappears asymptotically anyway}
 %\footnote{{\color{blue}The sampling theorem requires channels to be concentrated in both the spatial and wavenumber domains. However, the uncertainty principle limits the achievable concentration, leading to a smooth transition in the channel's support, represented by a logarithmic term in the expansion \cite{Franceschetti,PizzoWCL22}.} \angel{Still not accessible enough for an Intro. Either we find a way of explaining to a broad audience why sub-Nyquist spacing provides this log improvement, or else we remove the explanation altogether and just leave the references.}}
%{\color{red}Sublinear improvements in spatial dimensions. Indeed, channels on finite apertures cannot be overly concentrated in the wavenumber domain as per the uncertainty principle.
%The Nyquist theorem overlooks this limitation, taking the wavenumber support to be infinitely concentrated. This causes a sublinear loss in spatial dimensions, recoverable via faster-than-Nyquist sampling \cite{Franceschetti,PizzoWCL22}.}
%{\color{blue}resulting in a smooth transition of their wavenumber support. This effect is quantified by a logarithmic increase in spatial dimensions \cite{Franceschetti,PizzoWCL22}.} \angel{Better, but we're still not making the connection with the subNyquist spacing, and how it enables this log increase in dimensions}}
%
%
%the nuance with the sampling theorem being that channels cannot be simultaneously concentrated in both the spatial and wavenumber domains \cite{Franceschetti,PizzoWCL22}}
%the nuance with the sampling theorem being that channels cannot be simultaneously concentrated in both the spatial and wavenumber domains \cite{Franceschetti,PizzoWCL22}; 
%\angel{Not sure this will be enough for people not versed in the matter to understand the point your're making}
 %\cite{Landau1975}; 
%\angel{I understand you're referring to the second term in the expansion. Fine. But there's an inherent---perhaps irresolvable---tension between the fact that here you're emphasizing the finiteness of the apertures and in the paragraph right below you're emphasizing asymptotic (in the apertures) optimalities.}
%\andrea{I tried to get away with the issue by specifying that "linear" improvements are added by widening the apertures and "logarithmic" improvements are obtained from antenna densification.}
%\angel{Sure, go ahead restate "linear" up above. (I thought it unnecessary, but that's because I hadn't gotten to this point yet.) However, that doesn't solve the tension I was referring to. This improvement is only significant if the apertures are small, but that would get you in trouble for the Fourier analysis and vice versa. Just playing devil'd advocate here, better me than the reviewers :-)}
%\andrea{But I don't see an irresolvable issue here. We are being very clear in stating that gains beyond the ones provided by increasing the aperture are possible and have logarithmic (instead of linear) behavior.}
%{\color{red}Improved resolvability of rapid transients over finite apertures. Formalized for finite time intervals in \cite{Ferreira2006}, this feature straightforwardly carries over to finite apertures in the space domain.}
%{\color{magenta}Encoding of information within an arbitrarily short time interval over a prescribed bandwidth. Formalized for finite time intervals in \cite{Kempf2000}, this feature straightforwardly carries over to finite apertures in the space domain.} \angel{Perhaps it'd be better to state this directly for the space domain, then mentioning that the result has been formalized for the time domain in the reference.}
%\item \andrea{Wireless power transfer? \cite{Marzetta2019}} \angel{This last one seems out of the scope of the paper, I'd drop it}
%\andrea{I added references [5], [6] for the signal dimensionality to the spatial case (instead of time domain). Similarly, the reference [7] is about time-domain signals. As citing our SPAWC paper may be premature, I've added a comment.}
%\item
%guard against spurious components in the channel's wavevector spectrum, associated with evanescent propagation phenomena \cite{PizzoIT21}. 
%\angel{This seems like an oxymoron, as evanescent waves do not propagate. And how does faster-than-Nyquist sampling help?}
%\andrea{An enlargement of the spatial bandwidth would require an increase in sampling rate to avoid aliasing.}
%\andrea{The role of evanescent waves is unclear. Drawing from Tom's latest publication: "Just as an increased array aperture extends the Fraunhofer distance so that many practical communication situations can take place in the radiative-near-field, it also expands the reactive near-field so it can be used for some short-range systems."}
%\angel{Here I'd stick---as you do throughout the paper---to the established notion that evanescent waves do not reach the receiver. Under that premise, I can't make sense out of this bullet point...}
%\item
%{\color{blue}Superlinear gains in array directivity along specific directions, aptly named superdirectivity \cite{Schelkunoff}.}
%\end{itemize}
%{\color{blue}The nuance with the sampling theorem resides in the finiteness of the wide---but finite---apertures and practical signal-to-noise-ratio (SNR) conditions, with additional antennas enhancing both spatial resolution and received power through array gain. }
%\angel{Too much redundancy in "the finiteness of the wide---but finite---apertures". Also, perhaps these two separate points (resolution and SNR) should be dealt with in separate sentences, rather than mixed up. Perhaps even in two bullet points. This is important, as the justification of why holographic MIMO may be of interest to begin with.}
 %\angel{This last sentence is important, and it needs work. The "nuance with sampling theorem" part needs to be elaborated, and the latter part too: the log improvement in capacity is only at high SNR---at low SR that gain is linear---so perhaps it's best to articulate the gain in terms of power than capacity}
%subsumes and generalizes prior MIMO definitions, allowing for near-field operations and sub-critical antenna spacings, connoting a more comprehensive approach \cite{BJORNSON20193}. %\cite{Huang2020}

%\angel{"coupling" appears much too late in the Intro, I'd suggest you start discussing it right after the bullet points that motivate HoloMIMO. That's where people will decided whether they wanna continue reading or not. I'd push with hard with the message that it's utterly senseless to study holographic MIMO ignoring mutual coupling, and preview the highlights of its impact and the possibilities of harnessing it by means of the antenna pattern, etc}

Antennas are inherently exposed to their mutual radiations, with the intensity of the ensuing mutual coupling being determined by their radiation patterns and relative spacing \cite{JensenReview}. 
%\angel{Shouldn't the antenna pattern also be mentioned as a feature that impacts coupling?}
In particular, the coupling surges as the spacing falls %significantly below a half-wavelength, %\cite{Diamond1966}.
below the Nyquist distance, which is precisely the defining attribute of holographic MIMO. 
%\angel{Would you wanna generalize this to "fall significantly below the Nyquist distance" rather than just half-wavelength?}
Therefore, while uncoupled formulations have been valuable in understanding fading correlation among the antennas \cite{PizzoIT21,PizzoJSAC20,PizzoTWC22}, %in arbitrary environments \cite{PizzoIT21,PizzoJSAC20,PizzoTWC22},
ignoring coupling is ill-advised in holographic MIMO.
%\angel{I guess the issue here is that all your previous papers on HoloMIMO ignored coupling, so some sort of disclaimer could be good...}
%Thus, ignoring coupling in HoloMIMO is utterly senseless, as tight antenna spacings are the defining attribute of the paradigm.
%While uncoupled models have been most useful \cite{PizzoJSAC20,PizzoTWC22},
%The characterization of signal propagation that leads to these findings assumes uncoupled antennas, treating them as independent transducers \cite{PizzoJSAC20,PizzoTWC22}. 
In recognition of this reality,
this paper introduces a unifying linear system-theoretic framework that %while extending its applicability to 
intrinsically incorporates both correlation and coupling among the transmitting antennas. 
Coupling among receiving antennas would enter the formulation similarly, as it stems from the same physical phenomenon.
%\angel{Maybe say something about coupling at the receiver, and emphasize somewhere in the intro that the paper deals only with coupling at the Tx}
A fundamental analogy is uncovered between fading correlation and coupling, both manifesting as linear filtering operations, but in stark contrast. To wit, i) coupling and fading behave oppositely, with fading mapping directly onto the channel response through a spatial convolution while coupling entails a deconvolution, and ii) the wavenumber response is dictated by scattering in the case of fading, while it is tied to the antenna patterns in the case of coupling.
The two phenomena, correlation and coupling, are subsumed in a cascaded filter response. %which can be configured by adjusting the antenna spacings and patterns.
%\andrea{Fading correlation can also be adjusted to SNR conditions by changing the antenna spacing \cite{HeedongULA}. Instead, we alter the antenna pattern with the same goal of boosting capacity.}
The present study shows how coupling can be harnessed, by suitably altering the antenna pattern, to improve this cascaded filter response and thus the performance at every signal-to-noise-ratio (SNR).
Specifically matching/countering the antenna pattern to the fading spectrum whitens/colors the cascaded wavenumber response, leading to spatial decorrelation/correlation. 
%In contrast, countering the fading spectrum colors the response, enhancing correlation.
A reduced correlation is tantamount to a larger antenna spacing, effectively opening up additional spatial dimensions for multiplexing or diversity.
Antenna density is then dictated by the sampling theorem for the enlarged aperture, resulting in faster-than-Nyquist spatial sampling for the actual aperture.
The potential for such reconfigurability is backed by the latest advancements in antenna technology, including tightly coupled antennas \cite{Prather2013,Prather2017} and metasurface antennas \cite{Insang2019}.
The specific contributions of this work are as follows.
\begin{itemize}
%\item
%Introduce a rigorous formulation based on fundamental physical principles that captures the nuances and complexities of compact MIMO systems.
\item
Establish a formal relationship between coupling and correlation, % to elucidate their intrinsic properties and interactions.
subsuming and generalizing previous conceptions of mutual coupling as a mechanism for antenna decorrelation  \cite{Steyskal1990,Birtcher2006}. % Vaughan1987,Moustakas2001%between neighboring antennas
%\andrea{These papers from EM are on antenna diversity. How is this different from pattern diversity? If they represent the same thing, we should remove a reference and add another about pattern diversity from COMMS.}
%\item
%Provide a tractable and accessible presentation of both phenomena through linear system-theory and Fourier transform;
\item
%Model the channel response as a wide-sense stationary Gaussian random field, ensuring the second-order statistics align with electromagnetic principles to capture coupling and correlation effects.
Augment the Fourier analysis in \cite{PizzoIT21,PizzoJSAC20} to include mutual coupling in wide-sense stationary spatial random channels using an accessible framework based on linear system theory and Fourier transforms.
%This is facilitated through an accessible framework relying solely on linear system theory and Fourier transforms.
%Derive the Fourier spectral representation for spatially stationary channels inclusive of mutual coupling.
\item
%Leverage the channel representation to 
Derive the capacity-achieving transmit precoder at any arbitrary SNR, extending superdirective designs for low SNRs \cite{Wallace2005,Marzetta2019,Matthaiou2023}.
\item
Analyze how coupling affects power and spatial dimensionality, which dominate the channel capacity in the low- and high-SNR regimes, respectively.
%\item
%Reconcile differing perspectives on whether mutual coupling hinders \cite{Ksienski1983,Foschini} or 
%enhances \cite{Ranheim2001,Wallace2004,Nossek2010} communication at varying SNRs. 
%\andrea{The first two references are very old to make a point against more recent works. I would merge this point with the following.} %\angel{Ok}
\item
Provide additional evidence for the potential benefits of antenna densification within a fixed aperture, associated with a constructive exploitation of mutual coupling \cite{Wallace2004,Clerckx2007,Nossek2010,Masouros2013,Heath2023}, which could otherwise degrade performance  \cite{Janaswamy2002}.
%\item
%Broaden the literature's scope on mutual coupling, which has hitherto focused on antenna densification within a fixed aperture or on aperture widening with fixed antenna spacings \cite{Janaswamy2002,Wallace2005,Clerckx2007,Nossek2010,Masouros2013};
%%this framework encompasses
%holographic MIMO involves a combination of the two. \andrea{This point needs work.}
%Broaden the scope of traditional literature on mutual coupling, which focuses on bringing a fixed number of antennas closer together \cite{Wallace2004,Clerckx2007,Nossek2010}; this framework encompasses and extends these scenarios. %,Masouros2017
\end{itemize}
%Introduce a rigorous formulation based on fundamental physical principles that captures the nuances and complexities of compact MIMO systems.
%\item
%Elucidate the underlying nature of coupling and correlation, providing a deeper understanding of their interactions and impact on MIMO performance.
%\item
%Determine how and to what extent these phenomena %super-resolution effects are possible (contingent on a proper matching network)
%impact capacity in compact arrays, particularly in low and high SNR regimes.

%\angel{This subsection can be compressed considerably. Parts are redundant in light of the previous page, and parts unnecessary. Do we really need to discuss the Fourier-based transceiver architecture at all in the Intro?}
%Traditional MIMO systems are modeled as multiport networks, where each antenna serves as a port for transmitting or receiving signals.
%Interactions within the network are characterized using either a scattering \cite{Wallace2004} or impedance \cite{Nossek2010} formulation, relating radio signals or terminal voltages and currents, respectively.
%%Network behavior is described by a response matrix, which details the relationship between electrical signals entering and exiting different ports. 
%%This comprehensive model accounts for physical phenomena like coupling and correlation, all grounded in electromagnetic principles \cite{JensenReview}.
%The network behavior is described by a response matrix that accounts for all possible physical interactions, with its size growing quadratically as the number of antennas increases. 
%%in holographic MIMO, the combination of aperture widening and antenna densification causes the network size to grow quadratically with the number of antennas, requiring more scalable modeling approaches. 
%This modeling approach becomes prohibitive in holographic MIMO due to exploding complexity. 
%Here, a dense discrete array of sub-Nyquist-spaced antennas approaches a spatial current distribution as spacing shrinks, eventually creating a continuous holographic profile across the aperture.
%%A continuous aperture is envisioned as an array with continuous amplitude and phase control capabilities. 
%%Due to practical constraint imposed by its physical realizability, this will likely be implemented as a dense discrete array of sub-wavelength antennas...
%%holographic MIMO envisions an enormous number of antennas due to the conjunction of antenna densification and aperture widening, which requires more scalable modeling approaches.
%This paper conceptualizes network interactions as a linear system represented by the spatial impulse response between antennas.
%%A continuous formulation enables Fourier analysis which enhances efficiency in algorithmic implementation, and reduced complexity in transceiver architecture and channel estimation
%Adopting a linear system-theoretic approach for holographic MIMO offers advantages over its circuit-theoretic counterpart due to its connection with Fourier transforms, providing \emph{i)} enhanced efficiency in algorithmic implementation for large-scale antenna systems, and \emph{ii)} reduced complexity in transceiver architecture and channel estimation, leveraging the unitarity of Fourier transformations.

%However, recognizing that antennas in holographic MIMO arrays are inherently coupled has sparked renewed interest in studying mutual coupling effects, particularly in the contexts of beamforming \cite{Branislav2024}, matching networks \cite{Sanguinetti2024}, and spatial dimensions \cite{Sha2023}. 
Research in holographic MIMO often relies on multiport network models, where each antenna serves as a port for transmitting or receiving signals \cite{Branislav2024,Sanguinetti2024,Sha2023,Tengjiao2022,Tengjiao2023,Yongxi2024}. These models use a response matrix to capture port interactions through scattering \cite{Wallace2004} or impedance \cite{Nossek2010} formulations, relating radio signals or terminal voltages and currents among antennas.
%\andrea{The saving in computational complexity and simplification of transceiver architecture has been removed from Section~IX. Why don't we just say that instead of looking at the problem in the spatial domain through discrete convolution/deconv. we translate the problem to the wavenumber domain, with consequent simplification of the analysis?}
%However, the network size grows quadratically with the number of antennas, making this approach computationally burdensome and leading to complex architectures.
As an alternative, one can leverage that, as antenna spacings shrink, currents on the array form a quasi-continuous holographic profile, with interactions described by a spatial impulse response.
That invites the Fourier-based framework adopted by this paper.
%which offers improved algorithmic efficiency and reduced transceiver complexity compared to its circuit-theoretic counterpart.
%The model's complexity is intentionally minimized to facilitate the extraction of general insights, paving the way for further research.
%\angel{This final sentence is still too vague}
%Information is encoded directly to the antennas with impedance matching providing another layer of interaction.
Both approaches rely on impedance matching networks that mix information signals at each link end, introducing another layer of interaction. Their effects are deliberately excluded here to extract insights, leaving room for further research.

%Fourier theory is leveraged in this paper both as an analytical instrument 
%Coupling and correlation in MIMO systems is encapsulated by a multiport network description, where each antenna is modeled as a port capable of transmitting or receiving signals \cite{JensenReview}.
%characterizing their  how much of an incoming signal is reflected back, transmitted through, or coupled to other ports.
%Each antenna element has its own input/output port, where signals can be fed into or extracted from the array.
%However, as the antenna density increases, as envisioned by holographic MIMO, %the spatial channel approaches a continuous function.
%The MIMO model is derived by discretizing the continuous response at antenna locations.
%, specified by a space-variant impulse response function that complies with electromagnetic theory. 
%In this framework, coupling is incorporated as a linear filtering operation, alike correlation in \cite{PizzoJSAC20,PizzoTWC22}, that modifies the system's spectral response.
%The MIMO model is obtainable discretizing the continuous response at antenna locations, which augments the model in \cite{PizzoJSAC20,PizzoTWC22} with neglected coupling.
%whose impulse response describes the antenna skin whereon input currents flow. %specifies directionality
%the impact of scattering and antenna directionality is embedded into the spectral response 
%The asymptotically optimal (in the aperture relative to the wavelength) transceiver architecture for holographic MIMO involves performing a 2D spatial Fourier transform at each link end, mapping information symbols to the resolvable directions made available by the scattering environment for given apertures \cite{PizzoJSAC20,PizzoTWC22}. 
%This approach generalizes previous results for half-wavelength-spaced linear arrays operating in the far field \cite{Sayeed2002} to accommodate arbitrarily small antenna spacings and near-field operations. 
%say that results confirm signal processing theory for far-field operation 
%This vision is supported by related advancements in antenna technology such as those provided by 
%Fourier transforms can be efficiently and scalably implemented in analog form using optical lenses \cite{Nader2011,Sayeed2013}, tightly coupled arrays \cite{Prather2013,Prather2017}, and metasurface antennas \cite{Insang2019}. \angel{Fine, but for us the Fourier transform is merely an analytical instrument. This sentence could lead someone to think that our analysis is only relevant if you can actually implement a FT}
%paving the way for more power-efficient and scalable implementations.
%efficient circuit designs that can perform spatial Fourier transforms quickly and accurately.
%The characterization of signal propagation that leads to these findings assumes uncoupled antennas, treating them as independent transducers \cite{PizzoIT21}. 
%While uncoupled models have been most useful, 
%The emergence of holographic MIMO has sparked renewed interest in studying mutual coupling effects, 
%Conflicting perspectives in existing studies stem largely from differences in scope and assumptions, with this paper providing a comprehensive understanding that captures the nuances and complexities of holographic MIMO systems.
%, connoting the comprehensive nature of holographic MIMO
%A continuous aperture is envisioned as an array with continuous amplitude and phase control capabilities. 
% this will likely be implemented as a dense discrete array of sub-wavelength antennas...
%Our goal is to create an antenna that will generate an independent waveform for each input port from a shared aperture.
%The number of RF ports in Holographic Massive MIMO can be equal to the number of signals to spatially multiplex
%Nevertheless, current MIMO formulation may no longer capture the nuances of emerging systems, which could operate in unconventional regimes.

%\subsection{Multiport theory}

%\begin{itemize}
%\item
%Greater tractability and accessibility in modeling representation. % in terms of Fourier transform, linear system theory, and sampling theorem;
%\item
%Enhanced efficiency in algorithmic implementation, facilitating simulation of large-scale antenna systems. % through discrete Fourier transforms;
%\item
%Reduced complexity in transceiver architecture and channel estimation, leveraging the unitarity of Fourier transformations.
%\end{itemize}
%Fourier models inclusive of mutual coupling are not new in antenna theory, particularly for correcting antenna pattern distortions %accumulating at the edge of small arrays  
%in beamforming applications \cite{Steyskal1990}. 
%This reasoning aligns with the traditional conception that coupling is detrimental to communications \cite{Ksienski1983,Foschini}. 
%Instead of compensating for mutual coupling \cite{Birtcher2006}, it can be leveraged to %enhance system performance at low signal-to-noise-ratio (SNR), 
%achieve higher array gains than widely spaced arrays with the same number of antennas.
%%Similarly, coupling can boost communications at low SNR enhancing array gain \cite{Nossek2010}.
%%However, beamforming is capacity-achieving only at low signal-to-noise-ratio (SNR), with the focus shifting to spatial multiplexing as SNR increases \cite{heath_lozano_2018}.
%Similarly, properly harnessed, coupling can decorrelate signals between neighboring antennas \cite{Vaughan1987,Moustakas2001}, %\cite{Clerckx2007}
%thereby improving capacity at high signal-to-noise-ratio (SNR) \cite{Ranheim2001}.
%Altogether, coupling plays a crucial role in enhancing the capacity of MIMO systems across different SNR conditions \cite{Wallace2004,Nossek2010}. 
%This present study subsumes and generalizes previous works, connoting the comprehensive nature of holographic MIMO.

%%%%
%IMPLEMENTATION 
%%%%

%%%%
%- WARN THAT, WHILE UNCOUPLED MIMO MODELS HAVE BEEN MOST USEFUL \cite{PizzoJSAC20,PizzoIT21,PizzoTWC22}, IGNORING COUPLING BECOMES PERILOUS IN HOLOGRAPHIC MIMO.
%%%%
%Departing from conventional wisdom that antennas must be separated by half wavelength,
%intends to integrate missing aspects into current MIMO formulation.
%New paradigms such as holographic MIMO offers solutions %compensating for the enhanced path losses and lost diffraction 
%by utilizing a very large number of antennas that are spaced beneath half wavelength \cite{BJORNSON20193}. (\cite{Huang2020}?) 
%%It embodies other MIMO definitions such as massive MIMO, extra large MIMO, large intelligent surface, and gigantic MIMO.
%%The definition is very broad and embodies various nomenclature 
%The spatial channel between holographic arrays is specified by the Fourier spectral representation, function of the 2D spatial frequencies at each end of the link.
%Due to the asymptotic optimality of Fourier transform in the wide-aperture regime, 
%The asymptotically optimal transceiver architecture for holographic MIMO embeds a 2D spatial Fourier transforms at each link ends that directly relates information symbols to resolvable directions made available by the scattering environment for given apertures \cite{Sayeed2002,PizzoTWC22}.
%changes the domain of representation from spatial to wavenumber (or angular) \cite{PizzoTWC22}.
%Then, transmit symbols are mapped onto resolvable directions % whose powers can be optimized via waterfilling. 
%Precoding techniques can be used to account for channel state information (CSI)

%A continuous aperture is envisioned as an array with continuous amplitude and phase control capabilities. 
%Due to practical constraint imposed by its physical realizability, this will likely be implemented as a dense discrete array of sub-wavelength antennas...

%[SHOULDN'T THIS BE IN THE INTRODUCTION, RATHER THAN HERE?]

%- DEVELOP A THEORETICAL FRAMEWORK TO ANALYZE THE FUNDAMENTAL PERFORMANCE OF HOLOGRAPHIC MIMO WITH MUTUAL COUPLING INTRINSICALLY INCORPORATED. WE SHOULD POSITION THIS FRAMEWORK AS AN ALTERNATIVE TO THE CLASSICAL MULTIPORT NETWORK THEORY

%\subsection{Outline and Notation}

The manuscript is organized as follows.
Sec.~\ref{sec:multiport} introduces the circuit model. % in discrete- and continuous-space antenna systems. 
%specializes the multiport network modeling of a MIMO system to the considered scenario.
%physics behind plane-wave reflection off a smooth planar surface relying solely on linear system theory and Fourier transform.
Sec.~\ref{sec:coupling_iso} characterizes coupling with punctiform antennas, extended to arbitrary antennas in Sec.~\ref{sec:coupling}.
Sec.~\ref{sec:MIMO_coupled} derives the channel model with transmit coupling. 
%; the ensuing discrete model is obtained by sampling this response at antenna locations.
Sec.~\ref{sec:holo_MIMO_uncoupled} revisits the Fourier channel model for uncoupled antennas, with coupling incorporated in Sec.~\ref{sec:holo_MIMO_coupled}.
Sec.~\ref{sec:DOF_increase} evaluates the impact of coupling on DOF, with 
%Sec.~\ref{sec:spectral_efficiency}  information-theoretic performance limits. %capacity and transceiver design. 
information-theoretic performance limits studied in Sec.~\ref{sec:spectral_efficiency}.
%Sec.~\ref{sec:SNR_regimes} explore the low- and high-SNR regimes
Sec.~\ref{sec:conclusion} concludes with discussions and potential extensions.
   
\emph{Notation:} 
Lower (upper) case letters denote spatial (wavenumber) entities and calligraphic letters indicate linear operators. % and sans-serif upper case letters represent quantities influenced by coupling.
The Hilbert-Schmidt space of square-integrable functions over $\Real^3$ is denoted by $L^2$, with inner product $\langle f,g \rangle = \iiint_{-\infty}^\infty f(\vect{{\sf r}}) \overline{g(\vect{{\sf r}})} \, d\vect{{\sf r}}$ and norm $\|g\| = \sqrt{\langle g,g \rangle}$. 
Here, the adjoint $\mathcal{H}^*: L^2 \to L^2$ of an operator $\mathcal{H}: L^2 \to L^2$ is defined by the requirement $\langle f,\mathcal{H} g \rangle = \langle \mathcal{H}^* g,f \rangle$ for every $f,g$. An operator is self-adjoint when $\mathcal{H}^* = \mathcal{H}$.
%$\lfloor \cdot \rfloor$ denotes integer round-down, 
%$J_0(\cdot)$ is the Bessel function of the first kind and order $0$, $(x)^+ = \max(x,0)$, $\delta(\cdot)$ is the Dirac delta function, %, and $\lor$ is the logic `or' operation between two sets. 
%$\Real^n$ is the $n$-dimensional spaces of real-valued numbers.
%A point $\vect{r}$ in $\Real^3$ is described by its Cartesian coordinates $(x,y,z)$ with $\|\vect{r}\| = \sqrt{x^2 + y^2 + z^2}$. % the Euclidean norm.
%$\mathbbm{1}_\mathcal{X}$ is the indicator function of set $\mathcal{X} $, and $|\mathcal{X}|$ is the Lebesgue measure. In turn, $\mathcal{F}$ is the Fourier operator, $(\mathcal{F} h)(\vect{\omega}) = \int_{\Real^d} h(\vect{t}) e^{-\imagunit \vect{\omega}^{\Ttran} \vect{t}} \, d\vect{t} $, and $\mathcal{F}^{-1}$ is its inverse, %$(\mathcal{F}^{-1} g)(\vect{t}) = h(\vect{t})$.
%while ${\rm supp}(f) = \{\vect{x} : f(\vect{x}) \ne 0\}$ denotes the support of $f(\vect{x})$.
%with the shorthand notation $\mathcal{F}_1 = \mathcal{F}$ and $\mathcal{F}_1^{-1} = \mathcal{F}^{-1}$.  
%$g(\vect{x})$, $\vect{x} \in \Real^d$
In turn, 
$\mathbbm{1}_{X}$ is the indicator function of set ${X} $, $|{X}|$ is the Lebesgue measure, $\nabla$ is the gradient operator, $\nabla^2$ is the laplacian operator, ${\rm supp}(f)$ is the support of $f(\vect{\cdot})$, and $(x)^+ = \max(x,0)$. % while $o(\cdot)$ and $\mathcal{O}(\cdot)$ are Landau symbols.
%$\Re\{z\}$ denotes the real part of a complex number $z$.
Also, $\vect{x}\sim\CN(\vect{\mu},\vect{Q})$ is shorthand for circularly symmetric complex Gaussian random vectors with mean $\vect{\mu}$ and covariance $\vect{Q}$.
%$\Ex\{\vect{\cdot}\}$ denotes the expectation operator.

%\section{Multiantenna Multiport Model} \label{sec:multiport}
\section{Multiantenna Circuit Model} \label{sec:multiport}

\begin{figure}
\centering\vspace{-0.0cm}
\includegraphics[width=.8\linewidth]{multiport_simplified} 
%\caption{Multiport model of a MIMO system with $N_{\text t}$ ideal transmit antennas and $N_{\text r}$ open-circuited receive antennas.}\vspace{-0cm}
%\caption{Multiport network model.}
\caption{Multiantenna circuit model. Each transmit antenna is driven by an ideal zero-impedance current source, which is the Norton equivalent of a source with an infinite impedance in parallel. In turn, each receive antenna is connected to an infinite-impedance voltmeter, such that no current is drawn from antennas.}
%The conjunction of infinite-impedance loads at the transmit and receive ports in \eqref{input_output} yields }
\label{fig:multiport}
\end{figure}

\subsection{Mutual Impedance 
%of Time-Invariant Sources
}

For narrowband communication at frequency $\omega$, each antenna is identified by a pair
%of frequency-dependent
of complex phasors $(j_n, v_n)$, i.e., a port, characterizing the instantaneous current and voltage via
%\setcounter{equation}{11}
\begin{align} \label{phasor_port}
j_n(t) = \Re\{j_n e^{-\imagunit \omega t}\} \qquad 
v_n(t) = \Re\{v_n e^{-\imagunit \omega t}\}.
\end{align}
Circuit theory inherently accounts for the interaction among antennas. Precisely, when a current is injected at one port, voltages appear across all ports.
%Recall that closely-spaced antennas inevitably leads to a mutual coupling effect implying that the current induced at one point produces a non-zero effect in its surroundings 
This effect is measured by the mutual impedance \cite{Wallace2004}
\begin{equation} \label{mutual_coupling_coeff}
z_{nm} = \frac{v_n}{j_m}
\end{equation}
defined as the ratio of the open-circuit voltage at a port $m$ to the ideal current inserted at another port $n$, when all other ports are open-circuited (i.e., connected to an infinite load). The self-impedances correspond to $m=n$.
%When identical antennas are used, it is customary (e.g., \cite{Wallace2004,Nossek2010}) to impose a unitary self-impedance constraint, namely, $z_{nn} = 1$ for all $n$.


%\begin{figure*}[t!] 
%\begin{align}   \notag
%P_{\text c}(t)  & = \frac{1}{4} \iiint_{-\infty}^\infty d\vect{{\sf r}} \iiint_{-\infty}^\infty d\vect{{\sf s}} \Big( j_{\text{t}}^*(\vect{{\sf r}}) z_{\text{t,t}}(\vect{{\sf r}}-\vect{{\sf s}}) j_{\text{t}}(\vect{{\sf s}}) 
%+ j_{\text{t}}(\vect{{\sf r}}) z_{\text{t,t}}^*(\vect{{\sf r}}-\vect{{\sf s}}) j_{\text{t}}^*(\vect{{\sf s}})  \\& \tag{10} \label{circuit_power} \hspace{6cm}
%+ j_{\text{t}}(\vect{{\sf r}}) z_{\text{t,t}}(\vect{{\sf r}}-\vect{{\sf s}}) j_{\text{t}}(\vect{{\sf s}}) e^{-\imagunit 2\omega t} + j_{\text{t}}^*(\vect{{\sf r}}) z_{\text{t,t}}^*(\vect{{\sf r}}-\vect{{\sf s}}) j_{\text{t}}^*(\vect{{\sf s}}) e^{\imagunit 2\omega t}\Big)
%\end{align} 
%\hrule
%\begin{align}   \nonumber
%P_{\text{em}}(t)  & = -\frac{\imagunit \kappa Z_0}{4} \iiint_{-\infty}^\infty d\vect{{\sf r}}  \iiint_{-\infty}^\infty d\vect{{\sf s}} \Big( j_{\text{t}}^*(\vect{{\sf r}}) g(\vect{{\sf r}}-\vect{{\sf s}}) j_{\text{t}}(\vect{{\sf s}}) 
%+ j_{\text{t}}(\vect{{\sf r}}) g^*(\vect{{\sf r}}-\vect{{\sf s}}) j_{\text{t}}^*(\vect{{\sf s}})  \\ & \tag{11} \label{wave_power_final} \hspace{6cm}
%+ j_{\text{t}}(\vect{{\sf r}}) g(\vect{{\sf r}}-\vect{{\sf s}}) j_{\text{t}}(\vect{{\sf s}}) e^{-\imagunit 2\omega t} + j_{\text{t}}^*(\vect{{\sf r}}) g^*(\vect{{\sf r}}-\vect{{\sf s}}) j_{\text{t}}^*(\vect{{\sf s}}) e^{\imagunit 2\omega t}\Big)
%\end{align} 
%\hrule
%\end{figure*}


%\subsection{Linear Multiport Network}  \label{sec:multiport_model}
\subsection{Discrete Multiport System Model}  \label{sec:multiport_model}

%\andrea{I re-structured this part. We gave Fig.1, then \eqref{input_output} (which is a generalization), and finally \eqref{transmit_unilateral}-\eqref{open_circuit_MIMO}. Also, $\vect{Z}$ never appeared in the text.}

A system of $N_\text{t}$ transmit and $N_\text{r}$ receive
%spatially distributed
antennas can be modeled as an $(N_\text{t}+N_\text{r})$-port network, %by circuit theory 
%\cite{Nossek2010,Wallace2004,Qinlong2018}. %The block diagram is reported in .
%In its simplest form, each transmit antenna is driven by an ideal zero-impedance current source, while each receive antenna is connected to an infinite-impedance load (see Fig.~\ref{fig:multiport}) \cite{MarzettaIT}.
%{\color{blue}
%The ideal current source can be seen as the Norton's equivalent of an arbitrary circuit with an infinite-impedance load in parallel, with the latter assumption being also used at the receiver.}
%Generalization to finite impedances at transmitter and receiver requires solving circuit equations \cite{Nossek2010}.
%\angel{As a side thought, I wonder about this. How can we be sure that whatever we surmise with infinite loads is relevant to finite ones?}
%\andrea{We don't. I guess coupling effects would be smoothened out by practical impedance matching networks. It's worth investigating though.}
specified by an impedance matrix $\vect{Z}$ %$\vect{Z} \in \Complex^{(N_\text{t}+N_\text{r}) \times (N_\text{t}+N_\text{r})}$ 
of entries
%$[\vect{Z}]_{nm} = z_{nm}$ with
$z_{nm}$ \cite{Nossek2010,Wallace2004}.
%From \eqref{mutual_coupling_coeff}, an $(N_\text{t}+N_\text{r})$-dimensional impedance matrix can be assembled.
%Discriminating between transmit and receive antennas, we can 
Transmit antenna ports are indexed by $1$ to $N_\text{t}$ and receive antenna ports by $N_\text{t}+1$ to $N_\text{t}+N_\text{r}$. Defining $\vect{v}_{\text{t}} = (v_{\text{t},1}, \ldots, v_{\text{t},N_\text{t}})^{\Ttran}$ and $\vect{j}_{\text{t}} = (j_{\text{t},1}, \ldots, j_{\text{t},N_\text{t}})^{\Ttran}$ as the transmit voltage and current vectors, and $\vect{v}_{\text{r}} = (v_{\text{r},1}, \ldots, v_{\text{r},N_\text{r}})^{\Ttran}$ and $\vect{j}_{\text{r}} = (j_{\text{r},1}, \ldots, j_{\text{r},N_\text{r}})^{\Ttran}$ as their receive counterparts, 
the multiport model partitions as \cite{Nossek2010,Wallace2004} %\cite{Qinlong2018}
\begin{equation} \label{input_output}
\begin{pmatrix}
\vect{v}_{\text t} \\
\vect{v}_{\text r}
\end{pmatrix}
=
\begin{pmatrix}
\vect{Z}_{\text{t,t}} &  \vect{Z}_{\text{t,r}} \\
\vect{H} &  \vect{Z}_{\text{r,r}}
\end{pmatrix}
\begin{pmatrix}
\vect{j}_{\text t} \\
\vect{j}_{\text r}
\end{pmatrix}
\end{equation}
where $\vect{Z}_{\text{t,t}}$ and $\vect{Z}_{\text{r,r}}$ are the transmit and receive impedance matrices, respectively, and $\vect{Z}_{\text{t,r}}$ and $\vect{H}$ the transimpedance matrices.
%modeling the cross-interactions between every transmit antennas and every other receive antennas.
The latter depend on both the array geometries and the propagating medium. Due to reciprocity, a real symmetry occurs, namely $\vect{Z}_{\text{t,t}} = \vect{Z}_{\text{t,t}}^{\Ttran}$, $\vect{Z}_{\text{r,r}} = \vect{Z}_{\text{r,r}}^{\Ttran}$, and $\vect{Z}_{\text{t,r}} = \vect{H}^{\Ttran}$.
%These are real-symmetric, i.e., $z_{nm} = z_{mn}$ for all $n,m$, due to reciprocity theorem \cite{TomBook}.
%This corresponds to the standard MIMO channel matrix, namely, $\vect{Z}_{\text{r,t}} = \vect{H}_{\text{MIMO}}$.
%We will assume that antennas are calibrated opportunely that $\vect{Z}^{\Ttran} = \vect{Z}$ due to channel reciprocity. %as interactions are immaterial to an interchange of the excitation and measurement points.


%The contribution to the transmit voltages of power reradiated back by the receive antennas
%%When currents are applied at the transmit antenna ports, transmit voltages are created by radiated power and backward radiation due to part of that power being scattered by the receive antennas. The latter contribution
%is negligible as per the \emph{unilateral approximation} \cite{Wallace2004,Nossek2010}, which rests on the strong signal attenuation at radio frequencies.
%This translates to $\vect{Z}_{\text{t,r}} = \vect{0}_{N_\text{t}\times N_\text{r}}$, which, plugged into \eqref{input_output}, returns 
%%\angel{It still shocks me that, looking four lines up, this means $\vect{H}=\vect{0}$}
%%\andrea{Can't we just say that when currents are applied at the transmit antenna ports $\vect{j}_{\text r} \approx 0$? This  would preserve the symmetry of the coupling matrix in \eqref{input_output}. No assumption is made on $\vect{Z}_{\text{t,r}}$.}\angel{Ok, let's do that}
%\begin{equation} \label{transmit_unilateral}
%\vect{v}_{\text t} =  \vect{Z}_{\text{t,t}} \vect{j}_{\text{t}}.
%\end{equation}
%%Non-ideal source antenna ports are characterized by an impedance load $z_{\text t} $ such that $\vect{v}_{\text t} = \vect{v}_{\text g} - z_{\text t} \vect{j}_{\text t}$ for a source voltage $\vect{v}_{\text g} \in \Complex^{N_\text{t}}$. Thus, by virtue of \eqref{transmit_unilateral}, at the transmitter,
%%\begin{equation} \label{transmit_voltages}
%%\vect{v}_{\text t} = z_{\text t} (z_{\text t} \vect{I}_M + \vect{Z}_{\text{t,t}})^{-1} \vect{v}_{\text g}.
%%\end{equation}
%% Similarly, every receive antenna port is terminated by an impedance load $z_{\text r} $ yielding $\vect{v}_{\text r} = -z_{\text r} \vect{j}_{\text r}$. 
%%Combining \eqref{input_output} with \eqref{transmit_voltages} we obtain the (noiseless) input-output relation \cite{Qinlong2018,Wallace2004}
%%\begin{align} \label{receive_voltage}
%%\vect{v}_{\text r} & = z_{\text r} (z_{\text r} \vect{I}_N + \vect{Z}_{\text{r,r}})^{-1} \vect{Z}_{\text{r,t}} (z_{\text t} \vect{I}_M + \vect{Z}_{\text{t,t}})^{-1} \vect{v}_{\text g} \\&
%%= (\vect{I}_N + z_{\text r}^{-1} \vect{Z}_{\text{r,r}})^{-1} \vect{Z}_{\text{r,t}} (z_{\text t} \vect{I}_M + \vect{Z}_{\text{t,t}})^{-1} \vect{v}_{\text g}.
%%\end{align} 
%The conjunction of lossless transmit ports %\angel{idealized how?} 
%and open-circuit receive ports, i.e., $\vect{j}_{\text r}=0$ in \eqref{input_output}, % \angel{I guess this means $\vect{j}_{\text r}=0$? we should make it explicit} 

%In its simplest form (see Fig.~\ref{fig:multiport}), each transmit antenna is driven by an ideal zero-impedance current source, which is the Norton equivalent of a source with an infinite impedance in parallel. In turn, each receive antenna is connected to an infinite-impedance voltmeter, such that $\vect{j}_{\text r}=0$  \cite{MarzettaIT}. The conjunction of infinite-impedance loads at the transmit and receive ports in \eqref{input_output} yields 
As illustrated in Fig.~\ref{fig:multiport},
in its simplest form, \eqref{input_output} reduces to \cite{MarzettaIT}
\begin{align} \label{transmit_unilateral}
\vect{v}_{\text t} & =  \vect{Z}_{\text{t,t}} \vect{j}_{\text{t}} \\ \label{open_circuit_MIMO}
\vect{v}_{\text r} & = \vect{H} \vect{j}_{\text t},
\end{align}
%and the noiseless MIMO relationship % typically used in communication theory
%\begin{equation} \label{open_circuit_MIMO}
%\vect{v}_{\text r} = \vect{H} \vect{j}_{\text t},
%\end{equation}
with $\vect{H}$ the channel matrix.
The contribution to the transmit voltages of power reradiated back by the receive antennas is %identically
zero in \eqref{transmit_unilateral}; it
%When currents are applied at the transmit antenna ports, transmit voltages are created by radiated power and backward radiation due to part of that power being scattered by the receive antennas. The latter contribution
is negligible even with a finite-impedance receiver as per the \emph{unilateral approximation} \cite{Wallace2004,Nossek2010}, which rests on the strong signal attenuation at radio frequencies.

%Here, the input is the current vector $\vect{j}_{\text t}\in \Complex^{N_\text{t}}$ while the output is the vector of receive voltages $\vect{v}_{\text r}\in \Complex^{N_\text{r}}$.
Communication theorists tend to ignore the coupling among transmit antennas by setting $\vect{Z}_{\text{t,t}} = \vect{I}_{N_\text{t}}$ 
%\angel{We're not using $M$ in this paper} 
in \eqref{transmit_unilateral}, whereby
% that is not the case in our formulation.
%Omitting such effect would imply,
%from the voltage-current conjugate product,
the transmit circuit power is $P_{\text t} = \|\vect{j}_{\text t}\|^2$.
 %Setting $\Re\{\vect{Z}_{\text{t,t}}\} = \vect{I}_M$ (i.e., no coupling) would yield %the sum of squares
Together with \eqref{open_circuit_MIMO}, this forms the (noiseless) uncoupled MIMO channel model. While exceedingly useful, this model is not physical. %---not even for punctiform antennas.
% [DO THE IMPEDANCE LOADS PLAY ANY ROLE IN THE SEQUEL? IF NOT, I'D GET RID OF THEM, THEY ARE A DISTRACTION]


%The instantaneous power expended by the transmit circuitry is the sum of the voltage-current conjugate products \cite{Nossek2010,TomBook}
%\begin{equation} \label{power_real_discrete_sum}
%P_{\text c}(t)  = \sum_{n=0}^{N_\text{t}-1} j_{\text{t},n}(t) v_{\text{t},n}^*(t).
%\end{equation}
%Plugging \eqref{phasor_port} into the unilateral transmit model in \eqref{transmit_unilateral},
%\begin{align}  \notag
%P_{\text c}(t)  & = \frac{1}{4} \sum_{n=0}^{N_\text{t}-1} \sum_{m=0}^{N_\text{t}-1} \Big(j_{\text{t},n}^* z_{nm} j_{\text{t},m} +  j_{\text{t},n} z_{nm}^* j_{\text{t},m}^*  \\& \hspace{1cm}  \label{power_real_discrete}
%+ j_{\text{t},n} z_{nm} j_{\text{t},m} e^{-\imagunit 2 \omega t} +  j_{\text{t},n}^* z_{nm}^* j_{\text{t},m}^* e^{\imagunit 2 \omega t}\Big)
%\end{align}
%%Averaging the instantaneous complex circuit power over several sinusoidal cycles yields the time-average quantity \cite{TomBook}
%whose time-average (real) counterpart is
%\begin{align} \label{time_avg_power_electrical}
%P_{\text c} & = \lim_{T\to\infty} \frac{1}{T} \int_{-T/2}^{T/2} P_{\text c}(t) dt \\ \label{power_real_discrete_time_avg}
%& = \frac{1}{2} \sum_{n=0}^{N_\text{t}-1} \sum_{m=0}^{N_\text{t}-1} j_{\text{t},n}^* \Re\{z_{nm}\} j_{\text{t},m}
%\end{align}
%where %{\color{blue}oscillating terms were averaged out and}
%the symmetry of the transmit impedance ($z_{nm} = z_{mn}$) was exploited.


\subsection{Continuous Multiport System Model}

Let $(\hat{\vect{x}}_\text{r},\hat{\vect{y}}_\text{r},\hat{\vect{z}}_\text{r})$ and $(\hat{\vect{x}}_\text{t},\hat{\vect{y}}_\text{t},\hat{\vect{z}}_\text{t})$ be orthonormal bases describing coordinate systems locally at receiver and transmitter, respectively. With respect to the former, a point in space is represented by the vector $r_x \hat{\vect{x}}_\text{r} + r_y \hat{\vect{y}}_\text{r} + r_z \hat{\vect{z}}_\text{r}$ with coordinates $\vect{{\sf r}} = (\vect{r},r_z)$ with $\vect{r} = (r_x,r_y)$; ditto with respect to the latter. 
%{\color{red} Ambiguity between the two reference frames is resolved by the context in which they are used.}

Letting the number of transmit antennas grow unboundedly within a compact space, what arises is a continuum of ports described by the current and voltage space-time fields %$j(t,\vect{{\sf r}})$ and $v(t,\vect{{\sf r}})$
specified by their complex phasors $j(\vect{{\sf r}})$ and $v(\vect{{\sf r}})$.
%\setcounter{equation}{16}
%\begin{align} \label{phasor_field}
%j(t,\vect{{\sf r}}) = \Re\{j(\vect{{\sf r}}) e^{-\imagunit \omega t}\}, \quad 
%v(t,\vect{{\sf r}}) = \Re\{v(\vect{{\sf r}}) e^{-\imagunit \omega t}\}
%\end{align}
%with $\vect{{\sf r}} = (\vect{r},r_z) \in \Real^3$.
%{\color{blue}
%We define $V_\text{r}\in \Real^3$ and $V_\text{t}\in \Real^3$ the space regions wherein the receive field and current density are strictly nonzero.}
The interactions between $\vect{{\sf r}}$ and some other point $\vect{{\sf s}}$ are described by a complex impedance kernel
\begin{equation}
z(\vect{{\sf r}},\vect{{\sf s}}) = \frac{v(\vect{{\sf r}})}{j(\vect{{\sf s}})} ,
\end{equation}
which is the continuous counterpart to \eqref{mutual_coupling_coeff}. 
We differentiate between the transmit and receive impedance kernels, $z_{\text{t,t}}(\vect{{\sf r}},\vect{{\sf s}})$ and $z_{\text{r,r}}(\vect{{\sf r}},\vect{{\sf s}})$, and the transimpedance kernel $h(\vect{{\sf r}},\vect{{\sf s}})$.

%For any square-integrable $z_{\text{t,t}}(\vect{{\sf s}},\vect{{\sf t}})$,
For continuous apertures,
\eqref{transmit_unilateral} morphs into the convolution
\begin{equation} \label{voltage}
v_{\text t}(\vect{{\sf s}}) = (\mathcal{Z}_{\text{t,t}} j_{\text{t}})(\vect{{\sf s}}) = \iiint_{-\infty}^\infty  z_{\text{t,t}}(\vect{{\sf s}},\vect{{\sf t}}) j_{\text t}(\vect{{\sf t}}) d\vect{{\sf t}}
\end{equation}
with $\mathcal{Z}_{\text{t,t}} : L^2 \to L^2$ the operator analogue to $\vect{Z}_{\text{t,t}} : \Complex^{N_\text{t}} \to \Complex^{N_\text{t}}$. This operator is nonconjugate and self-adjoint, meaning that $z(\vect{{\sf r}},\vect{{\sf s}}) = z(\vect{{\sf s}},\vect{{\sf r}})$, consistent with the physical reciprocity of the corresponding impedance matrices. 
%measure how coupling 
%is distributed spatially on the same continuous aperture. 
%\angel{Perhaps we should finesse this: on a continuous apertures, there are no distinct antennas.}
%whose convergence is guaranteed by the Cauchy-Schwarz inequality.
%Even though it would be impossible to establish zero port currents in a real experiment, all we have to do to deduce the impedance matrix from theory is to calculate what the port volt- ages are supposed to be if all port currents, except one, happened to be zero.
%Our starting point for the analysis of both finite and infinite arrays will be the current (or equivalent‐current) distributions on an infinite array when only one element is excited. This allows the easy determination of coupling between ports and of the embedded element patterns. When a single element is excited, different current distributions are found on successive elements, with magnitudes which, most of the time, decay away from the excited element.
%Similar to \eqref{power_real_discrete_sum}, 
%[DOESN'T IT DEPEND ONLY ON THE MEDIUM, AND NOT AT ALL ON THE COUPLING?]


The transimpedance, in turn, depends on the propagation medium and is generally not self-adjoint.
From \eqref{open_circuit_MIMO}, it can be regarded as the impulse response of the uncoupled MIMO channel, being
\begin{equation} \label{convolution}
v_{\text{r}}(\vect{{\sf r}}) = (\mathcal{H} j_{\text{t}})(\vect{{\sf r}}) 
= \iiint_{-\infty}^\infty h(\vect{{\sf r}},\vect{{\sf s}}) j_{\text{t}}(\vect{{\sf s}}) d\vect{{\sf s}}
\end{equation}
%for continuous apertures,  
with $\mathcal{H} : L^2 \to L^2$ the %operator 
continuous analogue to $\vect{H} : \Complex^{N_\text{t}} \to \Complex^{N_\text{r}}$. % within $L^2$. 
Ignoring coupling at the transmitter entails $z_{\text{t,t}}(\vect{{\sf r}})$ being impulsive in \eqref{voltage}, whereby the transmit circuit power is
\begin{equation} \label{uncoupled_power}
P_\text{t} 
= \iiint_{-\infty}^\infty |j_\text{t}(\vect{{\sf s}})|^2 d\vect{{\sf s}}.
\end{equation}
%With $N_\text{t}$ antenna elements, the mutual impedance matrix $\vect{Z}_{\text{t,t}}$ is obtained by sampling the impedance kernel at the antenna locations, $[\vect{Z}_{\text{t,t}}]_{n,m} = z(\vect{{\sf r}}_n,\vect{{\sf s}}_m)$ for $n,m = 0, \ldots, N_\text{t}-1$. The same holds at the receiver for $\vect{Z}_{\text{r,r}}$ due to reciprocity.
%Under channel reciprocity, the impedance kernel is symmetric, i.e., $z(\vect{{\sf r}},\vect{{\sf s}}) = z(\vect{{\sf r}}-\vect{{\sf s}})$.
%However, our development never relies on this.


\section{Transmit Coupling Kernel \\ for Punctiform Antennas} \label{sec:coupling_iso}

%Physics dictates that different level of abstraction must yield the same result. Precisely,
For a lossless system, conservation of energy requires the transmit circuit power to equal the power
%expended by a source to drive
of the produced
%the corresponding
electromagnetic field. % for all time instant $t$. 
These powers, derived in Appendix~A for punctiform antennas, are respectively given by 
%\angel{I don't think we've defined the overbar to denote conjugation, rather you've been using an asterisk}
\begin{align}    \label{circuit_power}
{\sf P}_{\text c}  & = \frac{1}{2} \Re\left\{\iiint_{-\infty}^\infty d\vect{{\sf r}} \, \overline{j_{\text t}(\vect{{\sf r}})} \iiint_{-\infty}^\infty d\vect{{\sf s}}  \, z_{\text{t,t}}(\vect{{\sf r}},\vect{{\sf s}}) j_{\text t}(\vect{{\sf s}}) \right\}
\end{align}
and
\begin{align}   \label{wave_power_final}
{\sf P}_\text{em}  & = \frac{1}{2} \Re\left\{ - \imagunit \kappa Z_0 \iiint_{-\infty}^\infty d\vect{{\sf r}} \,  \overline{j_{\text t}(\vect{{\sf r}})} \iiint_{-\infty}^\infty d\vect{{\sf s}} \, g(\vect{{\sf r}} - \vect{{\sf s}}) j_{\text t}(\vect{{\sf s}}) \right\}
\end{align}
where ${Z_0 \approx 120 \pi}$ is the wave impedance of free space while
\begin{equation} \label{Green}
g(\vect{{\sf r}}) = \frac{e^{\imagunit \kappa \|\vect{{\sf r}}\|}}{4\pi \|\vect{{\sf r}}\|}
\end{equation}
is the Green's function and
%$\kappa = \omega/c$ is the wavenumber, given $c$ the speed of light.
$\kappa = 2\pi/\lambda$ is the wavenumber, given $\lambda$ as the wavelength.
As they coincide, the two above powers are henceforth not distinguished, but
% no distinction is made between transmit circuit power and transmit electromagnetic power, as they coincide,
 unified into ${\sf P}_{\text t}$.

\subsection{Spectral Representation of the Transmit Impedance}

%Due to the self-adjointness of the transmit impedance kernel, it is convenient to introduce the space-lag variable $\vect{{\sf v}}=(\vect{v},v_z)$, $\vect{v}=\vect{r}-\vect{s}$ and $v_z=r_z-s_z$.
Define $\vect{{\sf v}}=(\vect{v},v_z)$, with $\vect{v}=\vect{r}-\vect{s}$ and $v_z=r_z-s_z$, as the space-lag variable.
The equality between the circuit power in \eqref{circuit_power} and the electromagnetic power in \eqref{wave_power_final} %due to conservation of energy, 
yields
\begin{align} \label{impedance_kernel_iso}  
z_{\text{t,t}}(\vect{{\sf v}}) & = -\imagunit \kappa Z_0 \,  g(\vect{{\sf v}})
= -\imagunit \kappa Z_0 \frac{e^{\imagunit \kappa \|\vect{{\sf v}}\|}}{4\pi \|\vect{{\sf v}}\|}
\end{align}
for $\|\vect{{\sf v}}\|>0$.
%after recalling \eqref{Green}.
%The impedance kernel for punctiform antennas thus takes the form of a spherical wave.
The transmit impedance kernel %thus 
is space-invariant and takes the form of a spherical wave emanating from $\vect{{\sf s}}$ to any other point $\vect{{\sf r}}$.
%This is congruent with the fact that a current density can be regarded as a continuum of point sources.
 %this
 %is thus given by the spherical wave in \eqref{impedance_kernel_iso}. % at any $\|\vect{{\sf v}}\| > 0$.
 
An exact Fourier representation of the above impedance kernel %for punctiform antennas
is derived next, using Weyl's identity \cite{ChewBook}
\begin{equation} \label{Weyl}
\frac{e^{\imagunit \kappa \|\vect{{\sf v}}\|}}{\|\vect{{\sf v}}\|} =  \frac{\imagunit}{2\pi} \iint_{-\infty}^\infty  \frac{e^{\imagunit (\vect{\kappa}^{\Ttran} \vect{v} + \gamma |v_z|)}}{\gamma(\vect{\kappa})} d\vect{\kappa},
\end{equation} 
with $\gamma$ defined as
\begin{equation} \label{gamma}
\gamma(\vect{\kappa}) = 
\begin{cases}
\sqrt{\kappa^2 - \|\vect{\kappa}\|^2} & \quad \|\vect{\kappa}\|\le \kappa\\
\imagunit \sqrt{\|\vect{\kappa}\|^2 - \kappa^2} & \quad \|\vect{\kappa}\|> \kappa .
\end{cases}
\end{equation}
%{\color{blue}such that the integral above remain finite for $v_z\to\infty$ as per the radiation condition.}
Plugging \eqref{Weyl} into \eqref{impedance_kernel_iso} yields 
%the spectral representation
\begin{align}  \label{impedance_kernel_spectral}
z_{\text{t,t}}(\vect{{\sf v}}) & =  \frac{\kappa Z_0}{2}
\iint_{-\infty}^\infty  \frac{e^{\imagunit \gamma |v_z|}}{\gamma(\vect{\kappa})} e^{\imagunit \vect{\kappa}^{\Ttran} \vect{v}}  \frac{d\vect{\kappa}}{(2\pi)^2}.
\end{align}
%for any $|v_z|> 0$.
%with 
%\begin{equation}  \label{impedance_spectrum}
%Z_{\text{t,t}}(\vect{\kappa};v_z) =   .
%\end{equation}
%{\color{blue}for all $v_z$.}
%Similar to the  representation of an electromagnetic field \cite{PizzoIT21},
The spectrum contributing to the impedance is the one lying on a wavenumber hemisphere of radius $\kappa = 2\pi/\lambda$, either $\kappa_z=\gamma$ or $\kappa_z= - \gamma$ depending on the
%The choice of hemisphere is dictated by the relative
location of $\vect{{\sf r}}$ relative to $\vect{{\sf s}}$. % ensuring the convergence of \eqref{Weyl} for any $\|\vect{{\sf v}}\|>0$
 %in agreement with Sommerfeld's radiation conditions
% \cite{ChewBook,PlaneWaveBook}. 

%\subsection{Real Part of the Transmit Impedance Kernel}
\subsection{Transmit Coupling Kernel}

%[REMEMBER TO BRIEFLY MOTIVATE THE INTEREST IN THE REAL PART]
%[MAYBE WE SHOULD MOTIVATE THE REMINDER OF THIS SUBSECTION, WHICH IS A SORT OF SANITY CHECK USING ESTABLISHED RESULTS FOR THE DISCRETE CASE...]

Our attention now turns to the real part of the impedance, which is responsible for the transmit power. %(see App.~\ref{app:power}) \angel{A bit abrupt, besides pointing to the appendix, can we indicate where this comes from? I guess it's from (10).}
As shown in Appendix~B, \eqref{circuit_power} can be rewritten as % and the real symmetry of the impedance as
\begin{align}  \label{transmit_power_zeta}
{\sf P}_\text{t} & = \frac{1}{2} \iiint_{-\infty}^\infty \! d\vect{{\sf r}} \, \overline{j_{\text{t}}(\vect{{\sf r}})} \iiint_{-\infty}^\infty \! d\vect{{\sf s}} \, \Re\{z_{\text{t,t}}(\vect{{\sf r}}-\vect{{\sf s}})\} j_{\text{t}}(\vect{{\sf s}}).
\end{align}
%Indeed, replacing the impedance kernel in \eqref{circuit_power} with its expression \eqref{impedance_kernel_iso} while applying Euler’s formula,
%with $z_{\text{t,t}}(\vect{\cdot})$ defined in \eqref{impedance_kernel_iso}.
%, namely, ${\sf c}_{\text t}(\vect{{\sf v}}) = \Re\{z_{\text{t,t}}(\vect{{\sf v}})\}$, which is responsible for the radiating transmit power. % in \eqref{time_avg_power_electrical}.
%[NOT CLEAR WHAT NET POWER MEANS. IS THERE A GROSS POWER? :-) ALSO, THE EQUATION YOU REFER TO IS NOT em POWER, BUT CIRCUIT POWER. PERHAPS WE SHOULD JUST SAY "TRANSMIT POWER" HERE AND HENCEFORTH.]
%Since all punctiform antennas are identical up to a space translation
It is customary \cite{Wallace2004,Nossek2010}
%to impose the normalization %on the real part of the self-impedance, 
to express the real part of $z_{\text{t,t}}(\vect{{\sf v}})$ as
\begin{equation} \label{normalization}
\Re\{z_{\text{t,t}}(\vect{{\sf v}})\} = {\sf R} \, {\sf c}_{\text t}(\vect{{\sf v}})
\end{equation}
where ${\sf c}_{\text t}(\vect{{\sf v}})$ is the transmit coupling kernel and ${\sf R} = \kappa^2 Z_0/4\pi$ is the radiation resistance, such that ${\sf c}_{\text t}(\vect{0}) = 1$.
With that, \eqref{transmit_power_zeta} becomes
 %${\sf c}_{\text t}(\vect{0}) = 1$. %the Green's function
%["NORMALIZATION" WOULD BE BETTER THAN "CONSTRAINT"]
%\angel{Don't quite see the relationship between the antennas being identical and the normalization}
\begin{align}  \label{transmit_power}
{\sf P}_\text{t} & = \frac{{\sf R}}{2} \iiint_{-\infty}^\infty \! d\vect{{\sf r}} \, \overline{j_{\text{t}}(\vect{{\sf r}})} \iiint_{-\infty}^\infty \! d\vect{{\sf s}} \, {\sf c}_\text{t}(\vect{{\sf r}}-\vect{{\sf s}}) j_{\text{t}}(\vect{{\sf s}}),
\end{align}
%which coincides with \cite[Eq.~49]{Nossek2010}. [DOESN'T THIS REFERENCE DEAL ONLY WITH DISCRETE MODELS?]
subsuming \eqref{uncoupled_power} for ${\sf c}_{\text t}(\vect{{\sf v}}) = \delta(\vect{{\sf v}})$, which embodies the special case of no coupling at the transmitter---a case that cannot arise from physical principles. %such phenomenon 
Rather, the coupling with punctiform antennas is captured by a non-impulsive %coupling kernel 
${\sf c}_\text{t}(\vect{{\sf r}})$. 
%\angel{Not clear what "such phenomenon" refers to...}
Precisely, applying Euler's formula to \eqref{impedance_kernel_iso},
\begin{align}  \label{real_impedance_kernel_spherical}
{\sf c}_{\text t}(\vect{{\sf v}}) = \sinc \! \left ( 2 \frac{\|\vect{{\sf v}} \|} {\lambda} \right) .
\end{align}
%which is a real and even function, consistent with the symmetry of the impedance kernel.


%Punctiform antennas with 
The discretization with punctiform antennas amounts to an ideal spatial sampling of the continuous current (see Fig.~\ref{fig:impedance_corr_tot}).
From \eqref{real_impedance_kernel_spherical}, antennas with half-wavelength spacing are uncoupled whereas, for arbitrary spacing, they are in general coupled.
An analogy can be established between \eqref{real_impedance_kernel_spherical} and
%reinforces the analogy with
%can be established with 
the autocorrelation of an isotropic random channel
%in that, in both cases, the spectrum is isotropic
\cite{teal2002spatial},
%The connection between the two is drawn by the underlying physics through the Helmholtz equation, whereby both have a rotationally symmetric spectrum.
with uncoupled antennas being the analogue of antennas experiencing IID fading.
%\angel{This sentence needs some work, I don't think that "established" is what you mean.}
Indeed, as will be seen, coupling can be regarded as an additional spatial correlation---one that, unlike actual fading correlation, is not due to the angular spreading caused by scattering, but rather to the antenna structure. % and spacing.
%{\color{blue}We will return to this important aspect later on.}
 
%As of the Fourier representation, %assume that $v_z=0$.
Momentarily disregarding the $z$-component, which amounts to a translation,
%\angel{What does this entail?}
%Then,
%Alternatively, %it can be seen in the spectral representation that,impedance_kernel_spectral
for $\|\vect{\kappa}\| > \kappa$ the spectrum in \eqref{impedance_kernel_spectral} is imaginary due to %the expression for $\gamma$ in
\eqref{gamma}. 
But the Fourier transform of \eqref{real_impedance_kernel_spherical}, which is  real and even,
must be real.
Thus, retaining only the portion $\|\vect{\kappa}\|\le \kappa$ of the impedance spectrum in \eqref{impedance_kernel_spectral}, under proper normalization, % for {\color{blue}$|v_z|> 0$},
\begin{align}  \label{real_impedance_kernel}
{\sf c}_{\text t}(\vect{{\sf v}}) & =  
\frac{1}{2\pi \kappa} 
\iint_{\|\vect{\kappa}\|\le \kappa}  \frac{1}{\gamma(\vect{\kappa})} \, e^{\imagunit (\vect{\kappa}^{\Ttran} \vect{v} + \gamma |v_z|)}  d\vect{\kappa},
\end{align}
which implies the exclusion of evanescent waves; these do not contribute to the time-average transmit power
%in agreement with the bandlimited nature of wave propagation
\cite{PlaneWaveBook}.
Consequently, the region of convergence of \eqref{real_impedance_kernel} extends to all $\vect{{\sf v}}$.
%[PERHAPS WE SHOULD REMIND THE READER THAT SOME NORMALIZATION HAS TAKEN PLACE BETWEEN \eqref{impedance_kernel_spectral} AND \eqref{real_impedance_kernel}]
The above integration is intended on a $\kappa$-radius hemisphere, subject to the 2D parametrization
\begin{equation} \label{wavenumber_hemisphere}
(\vect{\kappa},\pm\gamma) : \{\|\vect{\kappa}\|\le \kappa\} \to \{\|\vect{\kappa}\|^2+\gamma^2=\kappa^2\}
\end{equation}
whose Jacobian, already reflected in \eqref{real_impedance_kernel}, is proportional to $\sqrt{\| \nabla \gamma\|^2 + 1}= 1/\gamma$ \cite{PizzoJSAC20}.
The upper and lower hemisphere supports in \eqref{real_impedance_kernel} respectively relate to the causal and anticausal parts of ${\sf c}_{\text t}(\vect{{\sf v}})$ along the $z$-axis.
The notion of causality applies here in the spatial domain and requires that $v_z>0$, which maps to $r_z > s_z$ $\forall s_z$; see Fig.~\ref{fig:impedance_corr_tot} (top). 
%\andrea{I'd keep the "causal" and "anticausal" notation (in place of "upgoing" and "downgoing"), but maybe we need to add a line to differentiate w.r.t. the time domain.}
%Both are obtainable by shifting ${\sf c}_{\text t}(\vect{v},0)$
%at the reference plane $v_z=0$
%along the $z$-axis, due to the term $e^{\imagunit \gamma |v_z|}$.

%\andrea{Dispensable}
%The self-adjointness of ${\sf c}_{\text t}(\vect{{\sf v}})$, related to the physical reciprocity of the channel, is preserved in \eqref{real_impedance_kernel} by virtue of the even symmetry of $\gamma$ in \eqref{gamma}.
%, up to a constant factor due to renormalization.
%We hasten to emphasize that our derivation does not rely on the far-field approximation, used in \cite{Nossek2010}. This leads to the general expression in \eqref{impedance_kernel_spectral} for the complex impedance kernel, rather than only its real part.
%[FINE, BUT HOW DOES IT MATTER?]

%Notably, the uniform coupling model in \cite[Sec.~III.C]{Nossek2010} corresponds to $A(\theta,\phi) = 1$ over the entire angular horizon, which requires a constant, unitary spectrum $A(\vect{{\sf k}})$ over the sphere $\|\vect{{\sf k}}\|=\kappa$.
%Unlike that formula, \eqref{impedance_matrix} does not refer to just the real part of the impedance matrix but it also includes its imaginary part.
%The correctness of our result can be proved by looking at the continuous formula in \eqref{impedance_kernel_linear}.  
%\eqref{impedance_kernel} provides with a general integral expression of the real-part of the impedance kernel that is valid regardless the type of interaction among different radiating elements at source. This is modeled by the plane-wave spectrum $A(\vect{k})$ through the mutual coupling response $\alpha(\vect{{\sf r}})$ via \eqref{source_spectrum}.
%Similarly to \eqref{input_field}, there exist a Fourier plane-wave representation of the impedance kernel in terms of the coupling spectrum $A(\vect{k})$. It follows that radiating elements inside an electric source are \emph{always} coupled each other, even under isotropic coupling; a property that is reminiscent of spatial correlation in stationary random channels being always correlated in space, even when the scattering shows no angular selectivity \cite{PizzoIT21,PizzoJSAC20}. 

\subsection{Isotropic Coupling}

%The denominator of the spectrum in \eqref{real_impedance_kernel} appears due to a parametrization of the $\kappa$-radius wavenumber hemispheres \cite{PizzoJSAC20}.
%upper $\mathbb{S}_+\subset \Real^3$ and lower $\mathbb{S}_-\subset \Real^3$
%defined by the graph
%$(\vect{k}, \pm \gamma) : \{\|\vect{k}\|\le \kappa\} \to \mathbb{S}_\pm$, whose Jacobian is proportional to $\sqrt{\| \nabla \gamma\|^2 + 1}= 1/\gamma$ \cite{PizzoJSAC20}. 
%Recalling that it is $\mathbb{S}_+$ for $v_z>0$ and $\mathbb{S}_-$ for $v_z<0$,
%Hence, \eqref{real_impedance_kernel} can be rewritten as a 3D inverse transform
%\begin{align}  \label{real_impedance_kernel_full}
%\Re\{z_{\text{t,t}}(\vect{{\sf v}})\} & =  
%\iiint_{\|\vect{{\sf k}}\| = \kappa} Z_{\text{t,t}}(\vect{{\sf k}}) e^{\imagunit \vect{{\sf k}}^{\Ttran} \vect{{\sf v}}}  \frac{d\vect{{\sf k}}}{(2\pi)^3},
%\end{align}
%of the constant isotropic spectrum $Z_{\text{t,t}}(\vect{{\sf k}})$. 
%this parametrization 
%\andrea{I would simply refer to as "coupling kernel" since it has been defined already above Eq.19.}
The %transmit impedance kernel real part 
coupling kernel
in \eqref{real_impedance_kernel_spherical} can be written as the Fourier transform
\begin{align}  \label{real_impedance_kernel_full}
{\sf c}_{\text t}(\vect{{\sf v}}) & = 
  \iiint_{-\infty}^\infty {\sf C}_{\text t}(\vect{{\sf k}}) \, e^{\imagunit \vect{{\sf k}}^{\Ttran} \vect{{\sf v}}} \frac{d\vect{{\sf k}}}{(2\pi)^3} 
\end{align}
of the spectrum
\begin{equation} \label{spectrum_impedance_isotropic}
{\sf C}_{\text t}(\vect{{\sf k}}) = \frac{4\pi^2}{\kappa} \, \delta \! \left(\|\vect{{\sf k}}\|^2 - \kappa^2 \right)
\end{equation}
given $\vect{{\sf k}} = (\vect{\kappa},\kappa_z) \in \Real^3$ the wavevector. The normalization ensures, as advanced, that ${\sf c}_{\text t}(\vect{{\sf 0}})=1$.
With punctiform antennas, therefore, a physically meaningful impedance must have 
a real part whose
%an isotropic
%a spectrum of the form in \eqref{spectrum_impedance_isotropic}, 
spectrum lives on the skin of the wavenumber sphere as per \eqref{spectrum_impedance_isotropic}.
%as derivable from the laws of propagation involving punctiform antennas.
This is incompatible with any model ignoring mutual coupling, as a constant spectrum $\forall \vect{{\sf k}} \in \Real^3$ would be required for ${\sf c}_{\text t}(\vect{{\sf v}}) = \delta(\vect{{\sf v}})$.
%This corresponds to a constant impedance over the entire angular support, i.e., an isotropic spectrum.
%This is  , thus it constitutes the fundamental building block of an electromagnetic source. 

%[I SEE TWO DIFFERENT ASPECTS OF () THAT ARE RELEVANT. FIRST, THAT IT IS IS ISOTROPIC, SECOND, IT LIVES ON THE SKIN OF THE SPHERE, BY CONSTRAINT. IT'S THE SECOND ONE THAT IS MOST DISTINCTIVE, BECAUSE A SPECTRUM CONSTANT EVERYWHERE WOULD ALSO BE ISOTROPIC.]
%{\color{blue}[Andrea: I see the impedance spectrum being spherical as a necessary and sufficient condition. Every other spectrums are just non physical, so they do not exist.]}

\begin{figure}
\centering\vspace{-0.0cm}
\includegraphics[width=.8\linewidth]{impedance_corr_tot} 
\caption{Mutual coupling between spatially causal antennas centered at $\vect{{\sf s}}$ and $\vect{{\sf r}}$. Top: the coupling between punctiform antennas takes the form of a spherical wave from $\vect{{\sf s}}$ to $\vect{{\sf r}}$. Bottom: physical antennas with arbitrary responses, ${\sf a}_\text{t}(\bm{\cdot})$; the coupling arises from the superposition of spherical waves emitted from points $\vect{{\sf p}}$ on the transmitting antenna's skin and received at points $\vect{{\sf q}}$ on the receiving antenna's skin.}\vspace{-0cm}
\label{fig:impedance_corr_tot}
\end{figure}

Once more, an analogy can be drawn between mutual coupling and autocorrelation. % specialized here to punctiform antennas.
Precisely, \eqref{spectrum_impedance_isotropic} is isomorphic with \cite[Eq.~27]{PizzoJSAC20}, the latter being the power spectral density of an %omnidirectional
isotropic random channel fading in 3D.
%[HMMM. NOT SURE I SEE THE ANALOGY FULLY. THE POWER SPECTRAL DENSITY IS A FUNCTION OF AZIMUTH AND ELEVATION ONLY.
%WHAT WOULD BE THE ANALOGOUS OF $\kappa$ IN IT?]
%{\color{blue}[Andrea: To support the analogy, I moved the reference you mentioned in Sec. IV.B and mine here, the latter dealing with the wavenumber spectrum.}


%\section{MIMO model}
\section{Transmit Coupling Kernel \\ for Physical Antennas} \label{sec:coupling}



%\begin{figure}[t!]
%     \centering
%     \begin{subfigure}{\columnwidth}
%         \centering
%         \includegraphics[width=.85\columnwidth]{impedance_corr_punctiform}
%\caption{Punctiform antennas.}
%         \label{fig:impedance_corr_punctiform}
%     \end{subfigure}
%     \vfill
%     \begin{subfigure}{\columnwidth}
%         \centering
%         \includegraphics[width=.85\columnwidth]{impedance_corr}
%         \caption{Physical antennas.}
%         \label{fig:impedance_corr}
%     \end{subfigure}
%       \caption{Spatial causality. \andrea{New plots!} \andrea{I'd explain how coupling is computed graphically in the caption.}}
%        \label{fig:Discretization}
%\end{figure}

%The discretization with punctiform antennas amounts to an ideal spatial sampling of the continuous current density.
%, namely $j_\text{t}(\vect{{\sf r}}) = \sum_{m=1}^{N_\text{t}} j_{\text{t},m} \, \delta(\vect{{\sf r}} - \vect{{\sf r}}_m)$.}  
%Here, the analogy is drawn by an antenna device.
%However,
Realizable antennas are non-infinitesimal, which corresponds to sampling with a non-impulsive response \cite{Unser1994}. 
%{\color{blue}${\sf a}_{\text t}(\vect{{\sf s}},\vect{{\sf s}}-\vect{{\sf r}}) \neq \delta(\vect{{\sf r}})$} \cite{Unser1994}. 
This is a %space-variant
real function %${\sf a}_{\text t}(\vect{{\sf r}},\vect{{\sf r}}+\vect{{\sf v}})$ 
that physically describes the antenna skin whereon currents may exist. It specifies the directivity pattern, with narrower patterns requiring a larger antenna structure as per the uncertainty principle.
We henceforth assume all antennas are identical with space-invariant response ${\sf a}_{\text t}(\vect{{\sf v}}) \in L^2$ 
and a corresponding 3D spectrum ${\sf A}_\text{t}(\vect{{\sf k}})$.
%\andrea{$\alpha(\vect{{\sf r}})$ must be connected to the physical size of an antenna. It cannot be described by an arbitrary Fourier spectrum.}
Then, the complex current density with physical antennas is specified by the convolution
\begin{align}  \label{current_LSI_filtering_discrete}
j^\prime_\text{t}(\vect{{\sf r}}) = \iiint_{-\infty}^\infty  j_\text{t}(\vect{{\sf s}}) \, {\sf a}_\text{t}(\vect{{\sf r}}-\vect{{\sf s}}) \, d\vect{{\sf s}}
\end{align}    
%with ${\sf a}_\text{t}(\vect{{\sf r}})$ the response of an individual antenna and 
with $j_\text{t}(\vect{{\sf r}})$ the current corresponding to punctiform antennas.
Replacing the current in \eqref{wave_power_final} with \eqref{current_LSI_filtering_discrete} returns the transmit electromagnetic power 
\begin{align}  \nonumber
{\sf P}_\text{em}  & = \frac{1}{2} \Re \Big\{ - \imagunit \kappa Z_0  \iiint_{-\infty}^\infty d\vect{{\sf r}} \,  \overline{j_\text{t}(\vect{{\sf r}})}
\iiint_{-\infty}^\infty d\vect{{\sf s}} \, j_\text{t}(\vect{{\sf s}}) \\& \hspace{0.5cm}  \label{aa} 
\cdot \iiint_{-\infty}^\infty d\vect{{\sf p}} 
  \iiint_{-\infty}^\infty d\vect{{\sf q}}  \,    {\sf a}_\text{t}(\vect{{\sf p}}-\vect{{\sf r}}) g(\vect{{\sf p}} - \vect{{\sf q}}) {\sf a}_\text{t}(\vect{{\sf q}}-\vect{{\sf s}})   \Big\}.
\end{align}
Equated to the circuit power in \eqref{circuit_power}, %as per conservation of energy,
the above yields the impedance kernel with physical antennas
\begin{align} \nonumber
z_{\text{t,t}}(\vect{{\sf v}})  & = -\imagunit \kappa Z_0  \iiint_{-\infty}^\infty d\vect{{\sf p}} \,  {\sf a}_\text{t}(\vect{{\sf p}}-\vect{{\sf r}})  \\& \hspace{1cm}  \label{bb} 
\cdot \iiint_{-\infty}^\infty d\vect{{\sf q}}  \,  g(\vect{{\sf p}}-\vect{{\sf q}}) {\sf a}_\text{t}(\vect{{\sf q}}+\vect{{\sf v}}-\vect{{\sf r}}),
\end{align}
as a function of the space lag, $\vect{{\sf v}} = \vect{{\sf r}} - \vect{{\sf s}}$.
This kernel, which reduces to \eqref{impedance_kernel_iso} for ${\sf a}_\text{t}(\vect{{\sf p}}) = \delta(\vect{{\sf p}})$,
 takes the form of a superposition of spherical waves between any two points $\vect{{\sf p}}$ and $\vect{{\sf q}}$ on the antenna skins, with centroids respectively at $\vect{{\sf r}}$ and $\vect{{\sf s}}$, as illustrated in Fig.~\ref{fig:impedance_corr_tot}.
Reciprocity holds in \eqref{bb} due to Green's function symmetry, $g(\vect{{\sf p}}-\vect{{\sf q}}) = g(\vect{{\sf q}} - \vect{{\sf p}})$. % $\forall \vect{{\sf p}}, \vect{{\sf q}}$.

%\subsection{Non-Isotropic Coupling}
\subsection{Spectral Representation of the Transmit Impedance Kernel}

Consider an antenna physically confined to a sphere of radius $r_0\ge0$, as shown in Fig.~\ref{fig:impedance_corr_tot} (bottom).
%{\color{red}let $(\hat{\vect{x}}_{\text{t}^\prime},\hat{\vect{y}}_{\text{t}^\prime},\hat{\vect{z}}_{\text{t}^\prime})$ describe a coordinate system with origin at the antenna centroid.}
%Rotating the coordinate system such that the $z$-axis points to the centroid of any adjacent antenna, we have that $|v_z| \ge 2r_0$ as antennas cannot overlap spatially.
Then, centroids of adjacent antennas must be separated by at least $2 r_0$ as antennas cannot overlap spatially.
This inherent causality and the antenna directionality are captured by the Fourier representation of the transmit impedance kernel, derived in Appendix~C as
\begin{align}    \nonumber
z_{\text{t,t}}(\vect{{\sf v}})  & =  \frac{\kappa Z_0}{2} 
\iint_{-\infty}^\infty \frac{d\vect{\kappa}}{(2\pi)^2} \,  \frac{e^{\imagunit \vect{\kappa}^{\Ttran} \vect{v}}}{\gamma(\vect{\kappa})}  \\ \label{impedance_kernel_spectral_noniso} 
& \hspace{2cm} \cdot
\begin{cases} \displaystyle
|{\sf A}_\text{t}^+(\vect{\kappa})|^2  \, e^{\imagunit \gamma v_z} & \quad v_z\ge0  \\ \displaystyle
|{\sf A}_\text{t}^-(\vect{\kappa})|^2  \, e^{-\imagunit \gamma v_z} & \quad v_z<0
\end{cases} 
\end{align} 
for $\|\vect{{\sf v}}\| > 2 r_0$, where ${\sf A}_\text{t}^\pm(\vect{\kappa}) = {\sf A}_\text{t}(\vect{\kappa},\pm \gamma)$ is the antenna pattern obtained evaluating the 3D spectrum at $\kappa_z = \pm \gamma$.
%These spectra, which need not be rotationally symmetric, specify non-isotropic coupling in the upgoing/downgoing direction.
These patterns specify the non-isotropic coupling in the causal/anticausal direction.
%\andrea{Earlier we refer to the causal and anticausal part or the wavenumber hemisphere. The nomenclature "upgoing/downgoing" has never been defined.}
%\angel{You explain what upgoing/downgoing means later in Sec. VI; that text should be moved here}
%\andrea{I moved the text in Sec. VI here. However, I just found out that the same explanation is provided above Eq. (27).}
%the antenna spectrum contributing to the impedance is the one lying solely on the upper-lower hemisphere in \eqref{wavenumber_hemisphere}.

For $r_0\to 0$, the antenna becomes infinitesimal and ${\sf a}_{\text t}(\vect{{\sf r}}) \to \delta(\vect{{\sf r}})$ whereby \eqref{impedance_kernel_spectral_noniso} correctly reduces to \eqref{impedance_kernel_spectral}. %implying ${\sf c}_{\text t}(\vect{0})=1$.
As in \eqref{impedance_kernel_spectral}, the symmetry of $z_{\text{t,t}}(\vect{{\sf v}})$ is established for any real-valued ${\sf a}_{\text t}(\vect{{\sf r}})$, as %the antenna spectrum satisfies
%the conjugate symmetry (Hermitian) property,
${\sf A}_{\text t}^\pm(-\vect{\kappa}) = \overline{{\sf A}^\pm_{\text t}(\vect{\kappa})}$ due to Hermitian symmetry.

\subsection{Non-Isotropic Coupling}

Excluding the evanescent portion of the spectrum from \eqref{impedance_kernel_spectral_noniso}, the Fourier representation in \eqref{real_impedance_kernel} generalizes to %, for non-punctiform antennas and $\|\vect{{\sf v}}\| \ge 2 r_0$, to
%reads as
\begin{align}  \label{real_impedance_kernel_antenna}
{\sf c}_{\text t}(\vect{{\sf v}}) & =  
\frac{1}{2\pi \kappa} \iint_{\|\vect{\kappa}\|\le \kappa} \!\!\!\!\!\! d\vect{\kappa} \, \frac{e^{\imagunit \vect{\kappa}^{\Ttran} \vect{v}}}{\gamma(\vect{\kappa})} 
\cdot
\begin{cases} \displaystyle
|{\sf A}_{\text t}^+(\vect{\kappa})|^2 \, e^{\imagunit \gamma v_z}   & \; v_z\ge 0 \\\displaystyle
|{\sf A}_{\text t}^-(\vect{\kappa})|^2 \, e^{-\imagunit  \gamma v_z}  & \; v_z< 0
\end{cases} 
\end{align}
with 
\begin{align} \label{norm_A_spectrum}
1  & = \frac{1}{4 \pi \kappa}  \iint_{\|\vect{\kappa}\| \le \kappa} \frac{|{\sf A}_\text{t}^+(\vect{\kappa})|^2  + |{\sf A}_\text{t}^-(\vect{\kappa})|^2}{\gamma(\vect{\kappa})} \, d\vect{\kappa},
\end{align}
which translates the normalization ${\sf c}_{\text t}(\vect{{\sf 0}}) = 1$ to the spectral domain (see Appendix~D).
%By means of \eqref{delta_nonlin}, the normalization \eqref{norm_A_spectrum} can be rewritten as $\iiint_{-\infty}^\infty {\sf C}_\text{t}(\vect{{\sf k}}) {d\vect{{\sf k}}}/{(2\pi)^3}  = 1$ with a spectrum
Invoking the identity
\begin{equation} \label{delta_nonlin}
\delta(\|\vect{{\sf k}}\|^2 - \kappa^2) = \frac{\delta(\kappa_z - \gamma) + \delta(\kappa_z + \gamma)}{2\gamma(\vect{\kappa})},
\end{equation}
the normalization \eqref{norm_A_spectrum} becomes $\iiint_{-\infty}^\infty {\sf C}_\text{t}(\vect{{\sf k}}) {d\vect{{\sf k}}}/{(2\pi)^3}  = 1$ with a spectrum
\begin{equation} \label{spectrum_impedance} 
{\sf C}_{\text t}(\vect{{\sf k}}) = \frac{4 \pi^2}{\kappa} \, |{\sf A}_{\text{t}}(\vect{{\sf k}})|^2 \, \delta(\|\vect{{\sf k}}\|^2 - \kappa^2),
\end{equation}
which generalizes \eqref{spectrum_impedance_isotropic} (i.e., ${\sf A}_{\text{t}}(\vect{{\sf k}})=1$) to %a multiantenna system with 
%non-punctiform 
physical antennas.
%Leveraging this linear system-theoretic interpretation
%Since a uniform scale of the Cartesian axes amounts to a scalar Jacobian and a scalar multiplication of $A_{\text t}(\vect{{\sf k}})$ is immaterial as per the normalization in \eqref{coupling_norm}, we consider  and regard 
%The angular selectivity of the coupling is imposed directly by
%is not rotationally symmetric \angel{It could still be, perhaps we should say "it need not be rotationally symmetric"} 
%but exhibits a directionality factor that
In the analogy between coupling and spatial correlation,
\eqref{spectrum_impedance} is isomorphic with the power spectral density of a stationary random channel fading \cite[Eq.~14]{PizzoJSAC20}.
%{\color{blue}where $A_{\text{t}}(\vect{{\sf k}})$ was replaced by its 3D Fourier transform while using $\iiint_{-\infty}^\infty  \delta(\|\vect{{\sf k}}\|^2 - \kappa^2) e^{\imagunit \vect{{\sf k}}^{\Ttran} \vect{{\sf r}}} \frac{d\vect{{\sf k}}}{(2\pi)^3} = \sinc(2 \|\vect{{\sf r}}\|/\lambda)$.}
%Substituting \eqref{spectrum_impedance} into \eqref{real_impedance_kernel_full} while using \eqref{delta_nonlin} and integrating over $\kappa_z$, the normalization entails
%\begin{align}
%1 & = \frac{4\pi^2}{\kappa} \iiint_{-\infty}^\infty |A_{\text{t}}(\vect{{\sf k}})|^2 \, \delta(\|\vect{{\sf k}}\|^2 - \kappa^2) \frac{d\vect{{\sf k}}}{(2\pi)^3} \\& \label{coupling_norm_wavenumber}
%= \frac{1}{4\pi \kappa} \iint_{\|\vect{\kappa}\|\le \kappa} \!\! \frac{|A_{\text t}^+(\vect{\kappa})|^2 + |A_{\text t}^-(\vect{\kappa})|^2}{\gamma} \, d\vect{{\sf k}},
%\end{align}

 
\subsection{Transmit Power Spectral Density}


%\begin{figure}
%\centering\vspace{-0.0cm}
%\includegraphics[width=.9\linewidth]{coupling_LSI} 
%\caption{Linear system-theoretic interpretation of multiantenna signal generation.}\vspace{-0cm}
%\label{fig:coupling_LSI}
%\end{figure}

%[WOULDN'T IT MAKE MORE SENSE FOR THIS TO BE A SUBSECTION OF THE PREVIOUS SECTION?]
%[PERHAPS THIS POINT SHOULD BE MADE EARLIER, AS SOON AS THE TWO POWERS ARE EQUATED]
 % under conservation of energy
%Time-averaging the instantaneous power in \eqref{circuit_power}, the transmit power emerges as
 % as per \eqref{time_avg_power_electrical}.
%, i.e, $P_\text{t} = \lim_{T\to\infty} \frac{1}{T} \int_{T/2}^{T/2} P_{\text c}(t) dt/T$.
%\begin{align}    \label{time_avg_power_electrical}
%P_\text{t}  & = \lim_{T\to\infty} \frac{1}{T} \int_{-T/2}^{T/2} P_{\text c}(t) dt \\ \label{circuit_power_avg}
%& = \frac{R}{2} \iiint_{-\infty}^\infty \! d\vect{{\sf r}} \, j_{\text{t}}^*(\vect{{\sf r}}) \iiint_{-\infty}^\infty \! d\vect{{\sf s}} \, {\sf c}_{\text t}(\vect{{\sf r}},\vect{{\sf s}}) j_{\text{t}}(\vect{{\sf s}}) ,
%\end{align}
%where the self-adjointness of $z_{\text{t,t}}(\vect{{\sf r}},\vect{{\sf s}})$
%%\angel{something's missing here...} %{\color{blue}oscillating terms were averaged out and}
% was invoked.
%applied a change of variables in the second integration of \eqref{circuit_power_avg_0} and exploited the self-adjointness of $z_{\text{t,t}}(\vect{{\sf r}},\vect{{\sf s}})$.
%exploiting the self-adjointness of the transmit impedance kernel \angel{point already made right above} and 
%Replacing ${\sf c}_{\text t}(\vect{{\sf r}},\vect{{\sf s}})$ by
Plugging \eqref{real_impedance_kernel_antenna} into \eqref{transmit_power}, the transmit power becomes
\begin{align}  \nonumber
{\sf P}_\text{t} & = \frac{{\sf R}}{4\pi \kappa}
\iint_{\|\vect{\kappa}\|\le \kappa}  \!  \bigg(\frac{|J_{\text t}^+(\vect{\kappa})|^2 |{\sf A}_\text{t}^+(\vect{\kappa})|^2}{\gamma(\vect{\kappa})}  \\ \label{time_avg_power_final} 
& \hspace{3cm} 
+ \frac{|J_{\text t}^-(\vect{\kappa})|^2 |{\sf A}_\text{t}^-(\vect{\kappa})|^2}{\gamma(\vect{\kappa})}\bigg)  \, d\vect{\kappa}
\end{align}
where $J_{\text t}^\pm(\vect{\kappa}) = J_{\text t}(\vect{\kappa},\pm \gamma)$ with
%[MIGHT BE MORE CONSISTENT TO KEEP THE DEPENDENCE ON $\vect{k}, \gamma$, EVEN IF $\vect{{\sf k}}$ INCORPORATES THE TWO]
\begin{equation} \label{3Dcurrent_spectrum}
J_{\text t}(\vect{{\sf k}}) = \iiint_{-\infty}^\infty  j_{\text t}(\vect{{\sf s}}) \, e^{-\imagunit \vect{{\sf k}}^{\Ttran} \vect{{\sf s}}} d\vect{{\sf s}}
\end{equation}
%is the source's spectrum obtained via a 3D Fourier transform of $j_{\text t}(\vect{{\sf s}})$ evaluated at $\kappa_z = \pm \gamma$.
the 3D spectrum of the source.
%Interestingly, \eqref{time_avg_power_final} equals the radiated power evaluated on any infinite $z$-oriented planar slab bounding an acoustic source of density $j(\vect{{\sf r}})$ \cite[Eq.~3.243]{PlaneWaveBook}. The equivalence hinges on the scalar Helmholtz equation that unites both theories \cite{MarzettaIT}. \angel{Perhaps this analogy can be dispensed with, as we're pressed for room?} 
Here, $|J_{\text t}^+(\vect{\kappa})|^2$ and $|J_{\text t}^-(\vect{\kappa})|^2$ are associated with the power flow along the $z$-axis, respectively causal and anticausal, measured on any $z$-plane.
%\andrea{See similar comment below Eq.28.}
%{\color{blue}Also worthwhile is that scalar electromagnetic fields qualitatively behave as  acoustic fields; both obey the Helmholtz equation in \eqref{Helmholtz} \cite{ChewBook,MarzettaIT}.
%Physically, an acoustic source dynamically injects air into its surroundings according to a space-time density $j_{\text t}(t,\vect{{\sf r}})$ (in 1/s), which measures the cubic meters per second of injected air per unit volume \cite{TomBook}. This process creates an excess of pressure field $e_{\text t}(t,\vect{{\sf r}})$ (in Joule/m$^3$).
%The (real) power passing through an infinite $z$-plane bounding the source correctly yields \eqref{time_avg_power_final} \cite[Eq.~3.244]{PlaneWaveBook}. Hence, our development complies with wave propagation theory.}

%Correctly, this shows no dependence on the $z$ coordinate due to a field migration into a lossless environment. 
%Invoking the identity
%\begin{equation} \label{delta_nonlin}
%\delta(\|\vect{{\sf k}}\|^2 - \kappa^2) = \frac{\delta(\kappa_z - \gamma) + \delta(\kappa_z + \gamma)}{2\gamma}
%\end{equation}
By means of \eqref{delta_nonlin}, we can rewrite  \eqref{time_avg_power_final} as 
\begin{equation} \label{power_psd}
{\sf P}_\text{t} =  \iiint_{-\infty}^{\infty} {\sf S}_{\text t}(\vect{{\sf k}}) \frac{d\vect{{\sf k}}}{(2\pi)^3}
\end{equation}
given the %impulsive
power spectral density
%\begin{align}  \label{psd_Dirac}
%S_{\text t}(\vect{{\sf k}}) & = \frac{4\pi^2}{\kappa} |J_{\text t}(\vect{{\sf k}})|^2 \delta(\|\vect{{\sf k}}\|^2 - \kappa^2),
%\end{align}
%which can be rewritten as
%Normalizing \eqref{spectrum_impedance_isotropic} to have a unitary real self-impedance yields $Z_{\text{t,t}}(\vect{{\sf k}}) = \frac{4\pi^2}{\kappa} \delta(\|\vect{{\sf k}}\|^2 - \kappa^2)$, we rewrite \eqref{psd_Dirac} exactly as
%\begin{align}  \label{psd_Dirac_LSI}
%{\sf S}_{\text t}(\vect{{\sf k}}) & = \frac{4\pi^2 R}{\kappa} \, |J_{\text t}(\vect{{\sf k}})|^2 \, \delta(\|\vect{{\sf k}}\|^2 - \kappa^2). 
%\end{align}
\begin{align}  \label{psd_Dirac_LSI_noniso}
{\sf S}_{\text t}(\vect{{\sf k}}) & =  \frac{4 \pi^2 {\sf R}}{\kappa} \, |{\sf A}_{\text t}(\vect{{\sf k}})|^2 |J_{\text t}(\vect{{\sf k}})|^2 \delta(\|\vect{{\sf k}}\|^2 - \kappa^2)
 \\&  \label{psd_Dirac_LSI_noniso_filter}
= {\sf R} \, |J_{\text t}(\vect{{\sf k}})|^2 {\sf C}_\text{t}(\vect{{\sf k}}),
\end{align}
after substituting \eqref{spectrum_impedance}.
%\angel{Likewise for these last 3 equations, is this something we actually use later?}
%\angel{Same question as posed for \eqref{psd_Dirac_LSI}: is this something we use later, or is it dispensable?}
%
%{\color{blue}The above relation describes the LSI system in Fig.~\ref{fig:coupling_LSI}, where an input ${\sf C}_{\text t}(\vect{{\sf k}})$ is multiplied by the wavenumber response $J_{\text t}(\vect{{\sf k}})$. The input is generated by the Helmholtz equation in \eqref{Helmholtz} governing electromagnetic wave propagation. 
%As seen, neglecting coupling means ignoring the spectral constraint imposed by physics through \eqref{spectrum_impedance_isotropic}.}
%[THIS COMES A BIT OUT OF THE BLUE HERE, HELP ME SEE WHERE IT COMES FROM...]
%is reminiscent of the power spectral density of a stationary electromagnetic random field \cite{PizzoJSAC20}.
%Both obey the Helmholtz equation governing scalar wave propagation that applies a spherical constraint on the radiated field and its second order statistics, in the case of a random channel.
%From \eqref{time_avg_power_final}, the transmit power radiated by a system of physical antennas reads as
%\begin{align}  \notag
%P_\text{t}  & =  \frac{R}{4\pi \kappa}
%\iint_{\|\vect{\kappa}\|\le \kappa}  \frac{d\vect{\kappa}}{\gamma} \Big(|J_{\text t}^+(\vect{\kappa})|^2 |A_{\text t}^+(\vect{\kappa})|^2 \\ & \hspace{3cm}
% +  |J_{\text t}^-(\vect{\kappa})|^2 |A_{\text t}^-(\vect{\kappa})|^2\Big),
% \label{time_avg_power_final_antenna} 
%\end{align}
%from which the power spectral density in \eqref{power_psd} of the non-isotropic coupling process equals
%The power spectral density of the non-isotropic coupling process equals
% The above formula correctly returns \eqref{psd_Dirac_LSI} with punctiform antennas (i.e., ${\sf A}_{\text{t}}(\vect{{\sf k}})=1$).
%Likewise, \eqref{psd_Dirac_LSI} and \eqref{psd_Dirac_LSI_noniso} coincide for rotationally symmetric sources, when ${\sf a}_{\text t}(\vect{{\sf r}}) = {\sf a}_{\text t}(\|\vect{{\sf r}}\|)$, also encompassing punctiform sources, whereby ${\sf A}_{\text t}(\vect{{\sf k}}) = {\sf A}_{\text t}(\kappa)$ as per the wavenumber constraint in \eqref{wavenumber_hemisphere} onto the hemispheres of radius $\kappa=2\pi/\lambda$.
Rotationally symmetric sources, punctiform included, are indistinguishable in propagation as they generate the same power density.
Specifically, due to rotational symmetry, ${\sf a}_{\text t}(\vect{{\sf r}}) = {\sf a}_{\text t}(\|\vect{{\sf r}}\|)$, leading to ${\sf A}_{\text t}(\vect{{\sf k}}) = 1$ as per the wavenumber constraint in \eqref{wavenumber_hemisphere} %on the hemispheres of radius $\kappa=2\pi/\lambda$.
%Specialized to spherical antennas as depicted in Fig.~\ref{fig:source}, it is modeled by \cite{Marzetta_superdirectivity}
%%that lives on a sphere of radius $r>0$. Such a source, depicted in Fig.~\ref{fig:source}, is modeled by \cite{Marzetta_superdirectivity}
%\begin{equation}
%{\sf a}_{\text t}(\vect{{\sf r}}) = \frac{1}{2 \pi r_0} \delta(r_x^2+r_y^2+r_z^2-r_0^2)
%\end{equation} 
%to which corresponds the spectrum
%\begin{equation}
%A_{\text t}(\vect{{\sf k}}) = \frac{\sin(\|\vect{{\sf k}}\| r_0)}{\|\vect{{\sf k}}\| r_0},
%\end{equation} 
%that is constant and given by $A_{\text t}(\vect{{\sf k}}) = \sinc(2 r_0/\lambda)$, when observed over the wavenumber hemispheres of radius $\kappa=2\pi/\lambda$ in \eqref{wavenumber_hemisphere}.
and the normalization in \eqref{norm_A_spectrum}.
%Any scalar multiplication is immaterial to \eqref{psd_Dirac_LSI_noniso} because of the normalization.

%From \eqref{normalization}, the normalization entails
%\begin{align} \label{coupling_notm_space}
%1 & = \frac{4\pi^2}{\kappa} \iiint_{-\infty}^\infty |A_{\text{t}}(\vect{{\sf k}})|^2 \, \delta(\|\vect{{\sf k}}\|^2 - \kappa^2) \frac{d\vect{{\sf k}}}{(2\pi)^3}.
%%\\& \label{coupling_notm_space}
%%= \frac{1}{4\pi \kappa} \iint_{\|\vect{\kappa}\|\le \kappa} \!\! \frac{|A_{\text t}^+(\vect{\kappa})|^2 + |A_{\text t}^-(\vect{\kappa})|^2}{\gamma} \, d\vect{{\sf k}},
%%{\color{blue}= \frac{4\pi^2}{\kappa} \iiint_{-\infty}^\infty d\vect{{\sf r}} \iiint_{-\infty}^\infty  d\vect{{\sf s}} \,{\sf a}_{\text{t}}(\vect{{\sf r}}) \, \sinc\left(\frac{2 \|\vect{{\sf r}}-\vect{{\sf s}}\|}{\lambda}\right) \, \alpha^*_{\text{t}}(\vect{{\sf s}})}
%\end{align}
%Given this, we henceforth differentiate between isotropic and non-isotropic coupling. 
%\andrea{I typically refer to a punctiform antenna, but an isotropic antenna would lead to the same results (up to a constant factor)}
%\andrea{Isn't $\|\vect{{\sf v}}\| > 2r$ for the correlation to be physical? With the condition in \eqref{real_impedance_kernel_antenna}, antennas on the same horizontal plane could be overlapped.}
%\angel{Only if the antennas are, not only confined to a sphere, but themselves spherical. It's tricky. But we can just strengthen the assumptions if we do wanna restrict ourselves to $\|\vect{{\sf v}}\| > 2r$.}
%\andrea{I used with Tom a similar assumption when deriving the field outside a volumetric source of current. We never specified the shape of such volume.}

%%where all constant factors have been subsumed into $A_{\text t}^\pm(\vect{\kappa})$, satisfying
%where $A_{\text t}^\pm(\vect{\kappa})$ satisfy
%\begin{align} \label{coupling_norm}
%1& = \frac{1}{2\pi \kappa} \iint_{\|\vect{\kappa}\|\le \kappa} \!\! \frac{|A_{\text t}^+(\vect{\kappa})|^2 + |A_{\text t}^-(\vect{\kappa})|^2}{\gamma} \, d\vect{\kappa}. %\\ \notag
%%& = \frac{1}{2\pi} \int_{0}^{2\pi} \! \left( \!\int_{0}^{\pi/2} \!\!\! |A_{\text t}^+(\theta_{\text t},\phi_{\text t})|^2 \!+\! \int_{\pi/2}^{\pi} \!\!\! |A_{\text t}^-(\theta_{\text t},\phi_{\text t})|^2 \!\right) \sin \theta_{\text t}   d\theta_{\text t} d\phi_{\text t},
%\end{align}
%as per the change of variable in \eqref{wavenumber_spherical}.
%The impedance kernel in (34) fully describes the electromagnetic interference generated by an antenna on its neigh- bors. According to the cascade of LSI filtering in Fig. 2, this interference is created by the combined effect of wave propagation and antenna response.

%Notably, the impedance outside a sphere of radius double the largest linear antenna dimension is completely determined by the antenna response inside the sphere. This is reminiscent of the eigenmode decomposition of an electromagnetic field radiated by a current density \cite{PlaneWaveBook}. 
%Indeed,
%{\color{blue}Here, antennas add another element of realism into the model generalizing the theory developed for arrays of punctiform elements.}
%Alternatively, one can rigorously arrive at \eqref{real_impedance_kernel_antenna} by computing the circuit power and the electromagnetic power generated by a system of $N_\text{t}$ identical yet arbitrary antennas, while letting $N_\text{t}$ grow indefinitely within a compact space. For the electromagnetic power, the current density must be convolved with the antenna response ${\sf a}_\text{t}(\vect{{\sf r}})$.

%\section{Transmit Array Gain and Superdirectivity} \label{sec:superdirectivity}
%
%\angel{We should streamline the terminology a bit. Do we differentiate between array gain and array directivity?}
%
%%\andrea{In the end, we could merge this section to the spectral efficiency's. But I decided to keep them separate for now.}
%Letting the receiver be located in the right \angel{upper?} $z$ half-space of a transmitter, the power in \eqref{time_avg_power_final_antenna} is the upgoing contribution\footnote{% Note that \eqref{time_avg_power_final_antenna} should be used in lieu of \eqref{power_antenna_2D} 
%Propagation could occur in both $z$ half-spaces, with the power then split between upgoing and downgoing contributions; this would be the case with a volumetric transmitter or back-to-back planar transmitters. Here, only the upgoing propagation is studied.}
%%From \eqref{time_avg_power_final_antenna}, the transmit power through the upper hemisphere specified by $\theta_{\text t} = [0,\pi/2]$ and $\phi_{\text t}\in[0,2\pi)$ reads
%\begin{align}   \label{time_avg_power_final_antenna_angle}
%P_\text{t}  & =  
% \frac{1}{2\pi}
%\int_{0}^{\pi/2} \!\! \int_{0}^{2\pi} |F_{\text t}(\theta_{\text t},\phi_{\text t})|^2 \sin \theta_{\text t}  \,  d\theta_{\text t} \, d\phi_{\text t},
%\end{align}
%where  
%\begin{equation} \label{farfield_pattern}
%F_{\text t}(\theta_{\text t},\phi_{\text t}) = \sqrt{R} \, J_{\text t}^+(\theta_{\text t},\phi_{\text t}) A_{\text t}^+(\theta_{\text t},\phi_{\text t})
%\end{equation}
%is the array's far-field radiation pattern, obtained from the current-antenna spectrum product with $\vect{\kappa}$ expressed in terms of $\theta_{\text t}$ and $\phi_{\text t}$ according to \eqref{wavenumber_spherical}.
%This change of variables entails deforming the infinite $z$-plane over which transmit power is calculated in \eqref{time_avg_power_final_antenna} into a hemisphere of infinite radius.
%%(Near-field spectra appear in the computation of the far-field spectrum in \eqref{farfield_pattern} because, in free space, the transmit power can be calculated through any closed surface enclosing the transmitter. Such integration surface can be deformed into a hemisphere of infinite radius.)
%%\angel{Perhaps this clarification will do more harm than good, I'd probably remove it}
%
%For a planar transmitter, the angular current spectrum is given by \eqref{3Dcurrent_spectrum} via \eqref{wavenumber_spherical}, and  \eqref{farfield_pattern} thus becomes
%\begin{equation} \label{farfield_pattern_planar}
%F_{\text t}(\theta_{\text t},\phi_{\text t}) = \sqrt{R}  \iint_{-\infty}^\infty  j_{\text t}(\vect{s}) A_{\text t}^+(\theta_{\text t},\phi_{\text t}) e^{-\imagunit \vect{\kappa}^{\Ttran}\!(\theta_{\text t},\phi_{\text t}) \vect{s}} d\vect{s}
%\end{equation} 
%where $\vect{\kappa}(\theta_{\text t},\phi_{\text t}) = \kappa \hat{\vect{r}}$ is a unit vector pointing %to the far-field observation point
%in the direction $(\theta_{\text t},\phi_{\text t})$.
%%Sampled, it yields
%%Discretizing it by means of a Riemann sum \cite{HeedongIRS}, 
%%\angel{This sentence seems to say you're gonna approximate the integral by a sum, but that's not quite it}
%%\andrea{See also \eqref{convolution_sampled}.}
%Sampling at the antenna locations,
%\begin{align} 
%F_{\text t}(\theta_{\text t},\phi_{\text t}) & = \sqrt{R}
% \sum_{n=1}^{N_{\text t}}  j_{{\text t},n} A_{\text t}^+(\theta_{\text t},\phi_{\text t}) e^{-\imagunit \vect{\kappa}^{\Ttran}(\theta_{\text t},\phi_{\text t}) \vect{s}_n} \\ \label{farfield_pattern_planar_array}
%& = \sqrt{R} \,  \vect{a}(\theta_{\text t},\phi_{\text t})^{\Htran} \vect{j}_{\text t} ,
%\end{align} 
%where $\vect{a}(\theta_{\text t},\phi_{\text t})$ is the array response in \eqref{Gavi} while $[\vect{j}_{\text{t}}]_m = j_{\text{t}}(\vect{s}_m,0)$ with $\vect{s}_m$ the location of the $m$th transmit antenna.
%%, given $\mathbb{A}_{\text t}$ as the transmitter's aperture.
%
%
%%Also, we recognize the integral at the denominator of \eqref{directivity_current} being the transmit power $P_{\text t}$, which is obtainable from \eqref{power_antenna_2D} after changing the integration variables to spherical coordinates.
%%For a discrete array source, it leads to \eqref{time_avg_power_discrete}.
%Substituting \eqref{farfield_pattern_planar_array} into \eqref{time_avg_power_final_antenna_angle} while invoking \eqref{impedance_matrix_general} leads to the transmit power
%\begin{align}     \label{time_avg_power_discrete}
%P_\text{t} & = R \cdot \vect{j}_{\text{t}}^{\Htran} \vect{{\sf C}}_{\text{t}} \vect{j}_{\text{t}} 
%%= \vect{j}_{\text{t}}^{\Htran} \Re\{\vect{Z}_{\text{t,t}}\} \vect{j}_{\text{t}},
%\end{align}
%%with $\vect{Z}_{\text{t,t}} = R \vect{{\sf C}}_{\text{t}}$ the transmit impedance matrix 
%with $\vect{{\sf C}}_{\text{t}}$ the coupling matrix in \eqref{impedance_matrix_general}.
%%where $[\vect{{\sf C}}_\text{t}]_{n,m} = \frac{\mathbb{A}}{N_\text{t}} {\sf c}_{\text t}(\vect{r}_n-\vect{s}_m,0)$, with a slight abuse of notation due to a different normalization of $\vect{{\sf C}}_\text{t}$ compared to \eqref{impedance_matrix_general}.
%
%\subsection{Directivity of Lossless Antenna Arrays}
%
%
%
%The array's directivity
%%quantifies the portion of transmit power allocated on every direction $(\theta,\phi)$. Precisely, it is defined as
%is the ratio of the radiation intensity (power per unit solid angle) to its average over all directions, %\cite{PlaneWaveBook}, namely
%\begin{align}  \label{directivity_current}
%D(\theta_{\text t},\phi_{\text t},j_{\text t}) & = \frac{|F_{\text t}(\theta_{\text t},\phi_{\text t})|^2}{\frac{1}{2\pi} \int_{0}^{\pi/2} \! \int_{0}^{2\pi} |F_{\text t}(\theta_{\text t},\phi_{\text t})|^2 \sin \theta_{\text t} d\theta_{\text t} d\phi_{\text t}}.
%%\\& \label{directivity_current} = \frac{|\vect{a}^{\Htran}(\theta_{\text t},\phi_{\text t}) \vect{j}_{\text t}|^2}{\vect{j}_{\text{t}}^{\Htran} \vect{{\sf C}}_{\text{t}} \vect{j}_{\text{t}}}
%\end{align}
%%where $\vect{{\sf C}}_{\text t}$ is derivable from \eqref{impedance_matrix} after replacing the array response $\vect{a}_0(\theta_{\text t},\phi_{\text t})$ with the more general $\vect{a}(\theta_{\text t},\phi_{\text t})$ in \eqref{Gavi}.
%Plugging \eqref{farfield_pattern_planar_array} and \eqref{time_avg_power_discrete} into \eqref{directivity_current}, the latter can be rewritten as the generalized Rayleigh quotient 
%\begin{align}  \label{directivity_current_discrete} 
%D(\theta_{\text t},\phi_{\text t},\vect{j}_{\text t}) = \frac{|\vect{a}(\theta_{\text t},\phi_{\text t})^{\Htran} \vect{j}_{\text t}|^2}{\vect{j}_{\text{t}}^{\Htran} \vect{{\sf C}}_{\text{t}} \vect{j}_{\text{t}}},
%\end{align}
%%where $\vect{{\sf C}}_{\text t}$ is the positive-definite matrix in \eqref{impedance_matrix_general}.
%which is influenced by the coupling among the array antennas along every direction $(\theta_{\text t},\phi_{\text t})$.
%%\angel{Perhaps soften this statement, people may object to the directivity "measuring" the coupling; it's influenced by it.}
%
%
%%Leveraging the positive-definiteness of $\vect{{\sf C}}_{\text{t}}$,
%%{\color{blue}The superscript $^\star$ is used next to distinguish the current vector attaining the maximum directivity on every direction.}
%Applying to vectors $\vect{{\sf C}}_{\text{t}}^{-1/2} \vect{a}(\theta_{\text t},\phi_{\text t})$ and $\vect{{\sf C}}_{\text{t}}^{1/2} \vect{j}_{\text t}$ the Cauchy-Schwartz inequality, the maximum directivity w.r.t. every possible $\vect{j}_{\text t}$ is seen to be 
%%Calling $\vect{x}= \vect{{\sf C}}_{\text t}^{1/2} \vect{j}_{\text t}$, \eqref{directivity_current_discrete} rewrites as the Rayleigh quotient 
%%\begin{equation} \label{directivity_theta_phi}
%%D(\theta_{\text t},\phi_{\text t}) = \frac{\vect{x}^{\Htran} \vect{A}(\theta_{\text t},\phi_{\text t}) \vect{x}}{\vect{x}^{\Htran} \vect{x}},
%%\end{equation}
%%where $\vect{A}(\theta_{\text t},\phi_{\text t}) = \vect{{\sf C}}_{\text t}^{-1/2} \vect{a}(\theta_{\text t},\phi_{\text t}) \vect{a}^{\Htran}(\theta_{\text t},\phi_{\text t}) \vect{{\sf C}}_{\text t}^{-1/2}$ is a Hermitian matrix.
%%By inspection of \eqref{directivity_theta_phi}, if $\vect{x}$ is an eigenvector of $\vect{A}(\cdot,\cdot)$, then the directivity $D(\cdot,\cdot)$ is the corresponding eigenvalue. Thus, its maximum reads
%\begin{equation} \label{array_gain}
%D^\star(\theta_{\text t},\phi_{\text t}) = \vect{a}(\theta_{\text t},\phi_{\text t})^{\Htran} \vect{{\sf C}}_{\text{t}}^{-1} \vect{a}(\theta_{\text t},\phi_{\text t}),
%\end{equation}
%attained by, up to a factor,
%\begin{equation} \label{optimal_current_directivity}
%\vect{j}_{\text t}^\star(\theta_{\text t},\phi_{\text t}) = \vect{{\sf C}}_{\text{t}}^{-1} \vect{a}(\theta_{\text t},\phi_{\text t})
%\end{equation}
%for every $(\theta_{\text t},\phi_{\text t})$.
%Remarkably, \eqref{array_gain} coincides with its counterpart in \cite[Eq.~72]{Nossek2010} for a practical multiport multiantenna system with antennas connected to an impedance load and impedance matching networks at both ends of the link; thus, the maximum directivity is independent of the transmit and receive circuitry. 
%As the antennas are pulled apart, $\vect{{\sf C}}_\text{t} \to \vect{I}_{N_{\text t}}$ and $D^\star(\theta_{\text t},\phi_{\text t}) \to N_{\text t} \, |A_{\text t}(\theta_{\text t},\phi_{\text t})|^2$, which is the classical result in the absence of coupling.
%%which is a coherent superposition of all antenna gains along a direction $(\theta_{\text t},\phi_{\text t})$.
%%In turn, with punctiform antennas, the gain reduces to a constant $N_{\text t}$, for every $(\theta_{\text t},\phi_{\text t})$.
%
%%The maximum directivity on every direction $(\theta_{\text t},\phi_{\text t})$ is achievable provided the array is driven by the current in \eqref{optimal_current_directivity},
%% with the lossy transmit impedance in \eqref{input_power_loss}.  
%%of $P_{\text{d}} = R_{\text{d}} \, \|\vect{j}_{\text{t}}\|^2$,
%%\begin{equation} \label{array_gain_lossy}
%%D_\eta^\star(\theta_{\text t},\phi_{\text t},\eta) = \vect{a}(\theta_{\text t},\phi_{\text t})^{\Htran} \left(\vect{{\sf C}}_{\text{t}} + \frac{R_\text{d}}{R} \vect{I}_{N_\text{t}}\right)^{-1} \vect{a}(\theta_{\text t},\phi_{\text t}),
%%\end{equation}
%%attained by (up to a constant factor)
%%\begin{equation} \label{optimal_current_directivity_lossy}
%%\vect{j}_{\text t}^\star(\theta_{\text t},\phi_{\text t},\eta) = \left(\vect{{\sf C}}_{\text{t}} + \frac{R_\text{d}}{R} \vect{I}_{N_\text{t}}\right)^{-1} \vect{a}(\theta_{\text t},\phi_{\text t})
%%\end{equation}
%%for every $(\theta_{\text t},\phi_{\text t})$. 
%%yet
%The truly large gains occur when $\vect{{\sf C}}_{\text t}$ is ill-conditioned, implying an array driven by potentially enormous currents. 
%Then, \eqref{optimal_current_directivity} provides an unstable solution that is sensitive to small changes in its entries, say in the antenna excitations and positions \cite{Morgan55}. 
%Robustness is gained as the lossless antenna ports considered thus far are replaced by their lossy brethren, which makes high \angel{super?} directivity accessible in practice \cite{Nossek2010,Marzetta_superdirectivity}.
%
%\subsection{Lossy Antennas and Superdirectivity}
%
%\begin{figure*}[t!]
%\centering
%\begin{subfigure}[b]{.3\linewidth}
%   \includegraphics[width=\columnwidth]{D_d2}
%   \caption{$d_\text{t}=\lambda/2$}
%   \label{fig:D_d2} 
%\end{subfigure}
%\begin{subfigure}[b]{.3\linewidth}
%   \includegraphics[width=\columnwidth]{D_d4}
%   \caption{$d_\text{t}=\lambda/4$}
%   \label{fig:D_d4} 
%\end{subfigure}
%\begin{subfigure}[b]{.3\linewidth}
%   \includegraphics[width=\columnwidth]{D_d10}
%   \caption{$d_\text{t}=\lambda/10$}
%   \label{fig:D_d10}
%\end{subfigure}
%\caption{Normalized $D^\star(\theta_{\text t},\phi_{\text t},\eta)$ in dB
%%along every direction
%with $\eta=0.9$ for distinct antenna spacings and aperture $L_{\text t} = 4\lambda$.}
%\label{fig:directivity_3D_d} 
%\end{figure*}
%
%\begin{figure}
%\centering
%\begin{subfigure}[b]{0.999\columnwidth}
%   \includegraphics[width=1\linewidth]{directivity_d_rho_frontfire}
%   \caption{Broadside direction ($\theta_\text{t}=0$)}
%   \label{fig:directivity_d_rho_frontfire} 
%\end{subfigure}
%\begin{subfigure}[b]{0.999\columnwidth}
%   \includegraphics[width=1\linewidth]{directivity_d_rho_endfire}
%   \caption{Endfire, northeast direction ($\theta_\text{t}=\pi/2$, $\phi_\text{t}=\pi/4$)}
%   \label{fig:directivity_d_rho_endfire}
%\end{subfigure}
%\caption{Normalized $D^\star$ on the broadside and endfire directions as a function of $d_\text{t}/\lambda$ for various $\eta$. The array is a UPA with punctiform antennas and aperture $L_\text{t}=4 \lambda$.}
%%\angel{It's actually the "maximum" gain on every direction, yes?}
%\label{fig:directivity_d_rho} 
%\end{figure}
%
%
%\begin{figure*}[t!]
%\centering
%\begin{subfigure}[b]{.3\linewidth}
%   \includegraphics[width=\columnwidth]{D_L5}
%   \caption{$L_\text{t}=4 \lambda$}
%   \label{fig:D_L5} 
%\end{subfigure}
%\begin{subfigure}[b]{.3\linewidth}
%   \includegraphics[width=\columnwidth]{D_L15}
%   \caption{$L_\text{t}=12 \lambda$}
%   \label{fig:D_L15} 
%\end{subfigure}
%\begin{subfigure}[b]{.3\linewidth}
%   \includegraphics[width=\columnwidth]{D_L30}
%   \caption{$L_\text{t}=20 \lambda$}
%   \label{fig:D_L30}
%\end{subfigure}
%\caption{Normalized $D^\star(\theta_{\text t},\phi_{\text t},\eta)$ in dB
%%along every direction
%with $\eta=0.9$ for distinct apertures and antenna spacing $d_{\text t} = \lambda/2$.}
%\label{fig:directivity_3D_L} 
%\end{figure*}
%
%\begin{figure}
%\centering
%\begin{subfigure}[b]{0.999\columnwidth}
%   \includegraphics[width=1\linewidth]{directivity_L_rho_frontfire}
%   \caption{Broadside direction ($\theta=0$)}
%   \label{fig:directivity_L_rho_frontfire}  
%\end{subfigure}
%\begin{subfigure}[b]{0.999\columnwidth}
%   \includegraphics[width=1\linewidth]{directivity_L_rho_endfire}
%   \caption{Endfire, northeast direction ($\theta=\pi/2$, $\phi=\pi/4$)}
%   \label{fig:directivity_L_rho_endfire}
%\end{subfigure}
%\caption{Normalized $D^\star$ on the broadside and endfire directions as a function of $L/\lambda$ for various $\eta$. The array is a UPA with punctiform antennas spaced by $d_\text{t}=\lambda/2$. \angel{If the normalization is by the directivity with uncoupled antennas, shouldn't the "uncoupled" curve be flat?}}
%%\angel{It's actually the "maximum" gain on every direction, yes?} }
%\label{fig:directivity_L_rho}  
%\end{figure}
%
%
%
%%So far, transmit antennas were modeled as lossless port devices.
%When antennas are driven by a current, part of the power is dissipated as heat in proportion to a resistance $R_{\text{d}}$.
%%
%Such dissipation amounts to an additional term in \eqref{time_avg_power_discrete}, subsumed by the augmented coupling matrix \cite[Eq.~109]{Nossek2010}
%\begin{equation} \label{input_power_loss}
%%P_\text{t} = R \cdot \vect{j}_{\text{t}}^{\Htran} \left(\vect{{\sf C}}_{\text{t}} + \eta \vect{I}_{N_\text{t}}\right) \vect{j}_{\text{t}} = \vect{j}_{\text{t}}^{\Htran} \Re\{\vect{Z}_{\text{t,t}}\} \vect{j}_{\text{t}},
%\vect{{\sf C}}_{\text{t}}(R_\text{d}/R) = \vect{{\sf C}}_{\text{t}} + \frac{R_\text{d}}{R} \vect{I}_{N_\text{t}},
%\end{equation}
%with $0 < R_{\text{d}}/R_{\text{t}} \le 1$ the ratio of the power dissipated by an array to the power radiated by such array when the antennas are uncoupled. % by virtue of \eqref{time_avg_power_discrete}.
%Notably, $R_{\text{d}}/R_{\text{t}}$ directly relates to the \emph{antenna radiation efficiency} \cite{BalanisBook}
%\begin{equation} \label{efficiency}
%\eta = \frac{1}{1+R_{\text{d}}/R_{\text{t}}} \qquad\quad 0 <  \eta < 1,
%\end{equation}
%whereby \eqref{input_power_loss} can be rewritten as 
%\begin{equation} \label{lossy_impedance}
%%P_\text{t} = R \cdot \vect{j}_{\text{t}}^{\Htran} \left(\vect{{\sf C}}_{\text{t}} + \eta \vect{I}_{N_\text{t}}\right) \vect{j}_{\text{t}} = \vect{j}_{\text{t}}^{\Htran} \Re\{\vect{Z}_{\text{t,t}}\} \vect{j}_{\text{t}},
%\vect{{\sf C}}_{\text{t}}(\eta) = \vect{{\sf C}}_{\text{t}} + \left(\frac{1}{\eta} - 1\right) \vect{I}_{N_\text{t}},
%\end{equation}
%with $\vect{{\sf C}}_{\text{t}}(1) $ corresponding to lossless antennas.
%For conciseness, the dependance on $\eta$ is henceforth omitted and $\vect{{\sf C}}_{\text{t}}$ is used to represent the entire form in \eqref{lossy_impedance}.
%%\angel{Confusing, I'm not sure how to interpret $\vect{{\sf C}}_{\text {t}}$ henceforth.}
%%values required for an efficient transmission. %$R_{\text{t}}/R_{\text{d}}\to\infty$ resulting in a lossless transmitter.
%%\angel{Yet in the remainder of this paragraph $\vect{{\sf C}}_{\text{t}}$ still denotes only the restricted form.}
%Antenna losses act through $\eta$ as a physical regularization, improving the conditioning of the coupling matrix and rendering the inversion well behaved \cite{Wallace2004}.
%%, lossless antenna ports are replaced by their lossy counterparts, each specified by
%%Precisely, let $\lambda_n(\vect{{\sf C}}_\text{t}) > 0$ be the eigenvalues of $\vect{{\sf C}}_{\text{t}}$ ordered from largest to smallest and $\vect{u}_n(\vect{{\sf C}}_{\text{t}})$ the corresponding eigenvectors, $n=1, \dots, N_\text{t}$. 
%The eigenvectors are not altered, namely $\vect{u}_n(\vect{{\sf C}}_{\text{t}}) = \vect{u}_n(\vect{{\sf C}}_{\text{t}}(1))$, whereas the eigenvalues of $\vect{{\sf C}}_{\text{t}}$ are found as $\lambda_n(\vect{{\sf C}}_\text{t}) = ({1}/{\eta} - 1) + \lambda_n(\vect{{\sf C}}_\text{t}(1))$.
%%\angel{Doesn't seem right, the eigenvalues}
%Thus, \eqref{array_gain} can be rewritten as
%\begin{equation} \label{array_gain_SVD}
%D^\star(\theta_{\text t},\phi_{\text t},\eta) = \sum_{n=1}^{N_\text{t}} \frac{|\vect{a}(\theta_{\text t},\phi_{\text t})^{\Htran} \vect{u}_n(\vect{{\sf C}}_{\text{t}})|^2}{\frac{1}{\eta} - 1 + \lambda_n(\vect{{\sf C}}_\text{t}(1))},
%\end{equation}
%where the dependance of the maximum directivity on $\eta$ has been made explicit. 
%It thus follows from \eqref{array_gain_SVD} that regularization offers a trade-off between maximum
%%\angel{maximum achievable, I guess?} 
%directivity, for $\eta = 1$, and physical realizability, corresponding to $\eta < 1$ and a surging dissipated power \cite{Nossek2010conf}.
%High directivity is attained in those directions $(\theta,\phi)$ for which the array vector $\vect{a}(\theta,\phi)$ aligns with any eigenvector of $\vect{{\sf C}}_{\text{t}}$ associated with an eigenvalue below some arbitrarily small threshold. 
%Rigorously, letting
%\begin{equation} \label{n_DOF}
%{\sf n}_\epsilon(\vect{{\sf C}}_\text{t}) = \min\{n: \lambda_n(\vect{{\sf C}}_{\text{t}}) \le \epsilon, \epsilon >0\}
%\end{equation}
%be the effective rank of $\vect{{\sf C}}_\text{t}$ within an $\epsilon$ accuracy,
%superdirectivity may be obtained in those directions $(\theta,\phi)$ for which $\vect{a}(\theta,\phi)$ lies in a subspace spanned by $\{\vect{u}_{{\sf n}_\epsilon}(\vect{{\sf C}}_\text{t}), \dots, \vect{u}_{N_\text{t}}(\vect{{\sf C}}_\text{t})\}$.
%%\angel{A bit fuzzy this subspace, since there's only 1 smallest eigenvalue. Guess you loosely refer to the subspace spanned by the eigenvalues below some threshold, or by a certain number of eigenvalues counting from small to large?}
%The angular span 
%%\angel{angular set?} 
%over which superdirectivity is observable widens with the dimension of the corresponding subspace, and it decreases with increasing ${\sf n}_\epsilon(\vect{{\sf C}}_\text{t})$.
%%The likelihood \angel{Likelihood? You mean how difficult it is?} of such an event increases with the dimension of the corresponding subspace.
%
%For a ULA specifically, as the antenna spacing shrinks with fixed $N_{\text t}$, superdirectivity arises about the endfire direction, with the directivity known to approach an unnormalized value of $N_{\text t}^2$ as the spacing vanishes, for $\eta= 1$ \cite{Nossek2010,Best2005}. 
%% \angel{I guess we don't know the counterpart limiting value for UPAs?}
%%It is next shown that this occurs with array densification or with a widening aperture, which is the operating regime of holographic MIMO. 
%%For a UPA, it is next shown that superdirectivity occurs with array densification or with a widening aperture, which are the operating regimes of holographic MIMO. 
%For a UPA, it is next shown how directivity is influenced by array densification or aperture widening, the two operating regimes of holographic MIMO. 
%%\angel{This sentence seems to restrict holographic to the latter, but I'd think that holographic MIMO encompasses both, no?}%Fig.~\ref{fig:eigC}.
%%Both can be studied using tools available for stationary channels, capitalizing on the isomorphism between the corresponding derivations.
%
%%\andrea{We should probably merge Sec. C with Sec. D. The numerical results in Sec. C have a limited interest and result about the space dimensionality in Sec. D requires the use of continuous apertures anyway. When re-reading this part I felt like it is disconnected from the rest.}
%%\angel{Ok}
%%
%\subsection{Array Densification and Wide Apertures} 
%
%
%
%\andrea{Is it worth considering planar arrays? The spectral concentration property also arises for linear array as the aperture increases. Besides, the superdirectivity phenomenon occurring along the main diagonal of the array is not so interesting as it retract when the aperture increases.}
%
%\angel{Planar arrays are much more relevant than linear ones under the umbrella of holographic MIMO. At the same time, I can see how linear arrays offer a simpler canvas on which to glean clean insights on how coupling arises with wide apertures. Let's discuss.}
%
%\andrea{Perhaps, we could remove Figs.~5a and~7b to shorten this section. These regimes are the least interesting as seen from Figs.~4 and~6.}
%\angel{Yes, go ahead}
%
%%Consider antennas spaced by $d_{\text t} \le \lambda/2$ for a fixed aperture $L_{\text t} = 5\lambda$.
%%; the maximum \angel{In which sense maximum?} corresponding to Nyquist sampling under omnidirectional propagation. 
%%Fig.~\ref{fig:eigC} plots the normalized eigenvalues of the {\color{blue}lossless} coupling matrix in \eqref{Toeplitz_coupling} for a squared UPA with punctiform antennas for various antenna spacings. The eigenvalues are plotted on an antenna scale for better visualization.
%%\angel{Not exactly, because that would correspond to only discrete values along $i$. Perhaps you can show the actual eigenvalues with circles on the interpolating lines you have right now?}
%Consider punctiform antennas, whereby $\vect{{\sf C}}_\text{t}=\vect{{\sf C}}_\text{iso}$ in \eqref{Toeplitz_coupling}. 
%%\angel{You must've changed notation, there's no $\vect{{\sf C}}_0$ in \eqref{Toeplitz_coupling} }
%%spaced by $d_{\text t} \le \lambda/2$ for a fixed aperture $L_{\text t} = 4\lambda$.
%Fig.~\ref{fig:directivity_3D_d} shows $D^\star(\theta_{\text t},\phi_{\text t},\eta)$ in \eqref{array_gain_SVD} on every direction %$(\theta_{\text t},\phi_{\text t})$
%for various antenna spacings, a fixed aperture $L_{\text t} = 4\lambda$, and $\eta=0.9$. \angel{I guess it's a UPA? How is it oriented?}
%%The coupling matrix in \eqref{array_gain_SVD} is chosen according to \eqref{Toeplitz_coupling}. 
%To isolate the impact of mutual coupling from anything else altering the directivity, chiefly the number of antennas,
% the diagrams are normalized by
%%the maximum directivity over all directions and considered antenna spacings. \angel{Not clear of this means a separate normalization for each antenna spacing. Either way, these plots are tricky because you wanna evidence the impact of coupling, yet as you change the antenna spacing with a fixed aperture you change the number of antennas and with that the directivity.}
%the directivity $N_\text{t}=1+ 2 L_\text{t}/\lambda=9$ of $\lambda/2$-spaced uncoupled antennas. % and  $L_{\text t} = 4\lambda$.
%%\angel{At half wavelength?}
%With $d_{\text t} = \lambda/2$, the directivity is highest on the broadside direction ($\theta_\text{t}=0$) and on the endfire plane (the plane of the array, $\theta_\text{t}=\pi/2$) along the northeast, southeast, southwest, and northwest directions, on which antennas are spaced $\lambda/ \sqrt{2}$ apart and experience coupling as per \eqref{Toeplitz_coupling}.
%Fig.~\ref{fig:D_d4} and Fig.~\ref{fig:D_d10} show how coupling builds up on the endfire plane when the antenna spacing shrinks below $d_{\text t} = \lambda/2$. 
%
%For the same setup %of Fig.~\ref{fig:directivity_3D_d},
%Figs.~\ref{fig:directivity_d_rho_frontfire} and~\ref{fig:directivity_d_rho_endfire} depict $D^\star$  along the broadside and endfire northeast directions for various $\eta$. 
%%All curves are normalized by the directivity of uncoupled antennas at half wavelength, namely $N_\text{t} = 1+ 2 L_\text{t}/\lambda$. \angel{You sure this is right? I'd think that $N_t$ depends on the antenna spacing, which varies along the horizontal axis of the plot...}
%The directivity increases monotonically with shrinking antenna spacing, with the benefit of array densification being modest for practical antenna efficiencies while taking a toll on hardware complexity and signal processing.
%%An exiguous superdirectivity is observed at $d_{\text t} = \lambda/2$ along both directions, for every practical $\eta$, with the advantage over uncoupled $\lambda/2$-spaced antennas slightly increasing with denser spacings.
%%diminishing as antenna spacing shrinks and eventually turning into a loss.} 
%%\angel{Need to rewrite this last sentence after the black solid curves have been removed...}
%
%
%Fig.~\ref{fig:directivity_3D_L} shows $D^\star$ on every direction %$(\theta_{\text t},\phi_{\text t})$
%for various apertures and $\eta=0.9$. \angel{Need to specify the half-wavelength spacing} %The plots are for UPAs with punctiform antennas, under the same normalization of Fig.~\ref{fig:directivity_3D_d}. 
%%\angel{Again, normalization should be clarified, and it's tricky to separate the effects of coupling with those of having more antennas}
%Coupling %becomes more pronounced along certain directions 
%diminishes on the endfire plane
%as equidistant antennas are added \angel{why? Doesn't this contradict Fig. 7a?}, while it is seemingly unaltered on the broadside direction. 
%%\angel{Hard to tell from the plot, in fact coupling seems to diminish on the endfire plane}
%%For instance, on the endfire plane, this is caused by an antenna spacing larger than $\lambda/2$ along the %northeast, southeast, southwest, and northwest directions 
%%corners 
% %main diagonals of the UPA---recall \eqref{Toeplitz_coupling}. 
%%{\color{blue} However, endfire superdirectivity seems to fall back \angel{fall back?} for an increasing aperture with most of the directivity being concentrated about the broadside direction.} \angel{Fig. 7 shows as much directivity endfire as broadside, no?}
%%\angel{Along the corners doesn't seem right, guess you mean along the diagonals.}
%% \angel{Also, the effect of coupling is not really evident in the figure.}
%%The directivity achieves its maximum about the broadside direction, regardless of the aperture.
%
%%A close inspection of Fig.~\ref{fig:Fig2} reveals a key difference in dense arrays vs wide-aperture arrays: the subspace dimension spanned by the eigenvalues of $\vect{{\sf C}}_{\text t}$ below an arbitrarily small threshold
%%%\angel{As earlier, the notion of "smaller eigenvalues" is somewhat fuzzy} 
%%grows unboundedly as the antenna spacing vanishes, but is bounded by \eqref{DOF_Landau} as the array aperture grows. This reflects in Fig.~\ref{fig:directivity_d_rho} vs Fig.~\ref{fig:directivity_L_rho}, with the former having a nearly omnidirectional diagram on the endfire plane while the latter exhibits directivity about the broadside direction. \angel{I don't see the points made in this last sentence. Also, aren't you comparing apples and oranges here? (endfire on one plot, broadside on the other)}
%%The array directivity diagram sharpens with the directivity of the individual antennas
%%%\angel{I can sort of guess what antenna selectivity means, but other may not} 
%%as per \eqref{DOF_Landau}. % with the largest beamwidth corresponding to isotropic antennas. 
%%%\angel{We may want to change "antenna selectivity" to something else throughout the paper. What it really means is that the individual antennas are themselves directive. You're using "directivity" extensively for the array, but you're using "selectivity" for the individual antennas, readers may wonder if there's more to that. In this last sentence for instance, what we're saying---if I understand correctly---is that "The array's directivity diagram sharpens with the directivity of the individual antennas".}
%
%Figs.~\ref{fig:directivity_L_rho_frontfire} and~\ref{fig:directivity_L_rho_endfire} show $D^\star$ along the broadside and endfire northeast directions for various $\eta$. 
%All curves are benchmarked to the directivity $N_\text{t}$ of uncoupled antennas.
%%and are normalized by $N_\text{t}=9$, corresponding to  $L_{\text t} = 4\lambda$.
%Remarkably, because of mutual coupling, array gains somewhat above $N_\text{t}$ are achievable on the broadside direction for $\eta>0.7$, robustly in the antenna aperture. 
%The emergence of superdirectivity in wide-aperture arrays is investigated next.
%%Thus, wide-aperture arrays naturally exhibit a certain superdirectivity compared to arrays with smaller apertures. %robustly against poor antenna efficiencies. 
%%imperfections of the antenna manufacturing. 
%%The latter observation may unlock the potentialities offered by holographic MIMO arrays, comprising of low-cost antennas spaced by half wavelength. \angel{Interesting indeed, but seemingly anecdotal. The gain is 2-3 dB and appears not to increase with the aperture.}
%
%%Next, the focus shifts to the spectral efficiency. That necessitates of a comprehensive model for the MIMO channel.
%% specialized to non-line-on-sight (NLOS) propagation.
%%The focus henceforth is on , as generally requiring a smaller antenna spacing than LOS \cite{PizzoWCL22}.
%%\angel{Hmm. Debatable. It's certainly true for spatial multiplexing, but need not be for beamforming.}
%%Thus,  a phenomenon that admits two interpretations:
%%Put differently, adding antennas within an array form factor does not increase the dimension of the 
%%number of eigenvalues falling below any arbitrarily low threshold surging as the array densifies. 
%%The directivity diagram flattens on the plane of the array, achieving its highest values on the end-fire plane (i.e., $\theta_{\text t}=\pi/2$) regardless of the azimuthal direction $\phi_t$.
%
%
%
%\subsection{Understanding Superdirectivity in Wide Apertures} \label{sec:superdirectivity_wide_aperture}
%
%%unveiling the fundamental limits of a multiantenna system, which are otherwise cluttered by spatial sampling and discrete representations.
%%unveiling some underlying properties of the coupling phenomenon---such as its essential dimensionality
%%Although a functional analysis implies a vanishing antenna spacing, it nicely reveals 
%As seen, superdirectivity arises when $\vect{{\sf C}}_\text{t}$ is of reduced rank, namely of effective rank ${\sf n}_\epsilon(\vect{{\sf C}}_\text{t}) < N_\text{t}$ with an $\epsilon$ accuracy. % as defined in \eqref{n_DOF}.
%This %reduced dimensionality
%is the case for $L_\text{t}/\lambda\gg 1$, as the eigenvalues of $\vect{{\sf C}}_\text{t}$ polarize into two levels %as per spectral concentration
%\cite{FranceschettiBook}; the transition within an $\epsilon$ accuracy occurs at
%%Precisely, consider the space spanned by the image of $j_\text{t}(\vect{{\sf r}}) \in L^2$ in \eqref{op_coupling} through the self-adjoint operator $\mathcal{C} : L^2 \to L^2$.
%\begin{equation} \label{effective_rank}
%{\sf n}_\epsilon(\vect{{\sf C}}_\text{t}) = {\sf n}_0(\vect{{\sf C}}_\text{t}) + o \! \left(\log \frac{L_\text{t}}{\lambda} \right),
%\end{equation}
%with the  right-hand side depending on $\epsilon$
%%that is omitted as only appearing as a pre-constant $\log(1/\epsilon)$ as $\epsilon\to 0$ of the second-order term
%only through the term
%$o \! \left(\log \frac{L_\text{t}}{\lambda} \right)$. 
%%\angel{This is fine, but you do what to mention what $\epsilon$ signifies.}
%Hence, ${\sf n}_0(\vect{{\sf C}}_\text{t})$ specifies the number of antennas that are essentially uncoupled, asymptotically in $L_\text{t}/\lambda$.
%
%For omnidirectional antennas, encompassing punctiform antennas, $A_{\text t}^+(\theta_{\text t},\phi_{\text t})$ in \eqref{farfield_pattern} is constant over the entire upper hemisphere $0 \le \theta_{\text t}\le \pi/2$ and $0\le \phi_{\text t}< 2\pi$, implying \cite{PoonDOF}
%\begin{align} \label{DOF_Landau_iso}
%{\sf n}_0(\vect{{\sf C}}_\text{t}) = \pi \left(\frac{L_\text{t}}{\lambda}\right)^{\!2}, % \equiv {\sf n}_\text{t},
%\end{align}
%for lossless antennas.
%The eigenvalue polarization with growing $L_\text{t}/\lambda$ is exemplified in Fig.~\ref{fig:Fig2} for an UPA having %of punctiform antennas and
%$d_\text{t}=\lambda/2$. 
%The figure %reveals a distinctive feature of superdirectivity in wide apertures:
%confirms that the subspace dimension spanned by the eigenvalues of $\vect{{\sf C}}_{\text t}$ below any small threshold increases with the array aperture, being bounded by \eqref{DOF_Landau_iso} for omnidirectional antennas. 
%This reflects in Fig.~\ref{fig:directivity_3D_L} exhibiting superdirectivity about the broadside direction over a beamwidth that progressively widens with $L_\text{t}/\lambda$. \angel{Don't see this in the figure...}
%%\angel{This is not really appreciated in Fig. 5, and "up to a point" is rather imprecise anyway. Perhaps remove this sentence, and the following one too, which seems unnecessary}
%%Fig.~\ref{fig:directivity_L_rho_frontfire} shows the impact of antenna efficiency on this broadside superdirectivity.
%%the latter being limited by the physical regularization applied in \eqref{lossy_impedance}.
%%\andrea{say that the beamwidth over which we observe in Fig.~\ref{fig:directivity_3D_L} superdirectivity grows with increasing $L/\lambda$ up to the point specified by ${\sf n}_0(\vect{{\sf C}}_\text{t})$.}
%
%The generalization of \eqref{DOF_Landau_iso} to arbitrary yet identical antennas gives \cite{PoonDOF,Franceschetti,PizzoWCL22} %\angel{The equation reference in the previous sentence is surely wrong. Also, a sentence cannot start with an equation number}
%\begin{align}   \label{DOF_Landau_wave}
%{\sf n}_0(\vect{{\sf C}}_\text{t}) & = L_\text{t}^{2} \iint_{{\rm supp}(|A_{\text t}^+(\vect{\kappa})|)} \frac{d\vect{\kappa}}{(2\pi)^2} \\ \label{DOF_Landau}
%& = 
%\left(\frac{L_\text{t}}{\lambda}\right)^{\!2} \!\! \iint_{{\rm supp}(|A_{\text t}^+(\theta_\text{t},\phi_\text{t})|)} \!\! \cos \theta_\text{t} \, \sin \theta_\text{t} \, d\theta_\text{t} \, d\phi_\text{t},
%\end{align}
%which follows from changing variables according to \eqref{wavenumber_spherical}, with ${\rm supp}(\cdot)$ denoting the support of a function.
%%and $d\Omega_\text{t} = \sin \theta_\text{t} \, d\theta_\text{t} \, d\phi_\text{t}$ the differential element of solid angle.}
%%\angel{Need to indicate what supp means}
%The highest value of \eqref{DOF_Landau} is attained by omnidirectional antennas, when it reduces to \eqref{DOF_Landau_iso} by virtue of
%\begin{align}  \label{DOF_Landau_omni}
%\int_{0}^{\pi/2} \!\! \int_0^{2\pi} \cos \theta_\text{t} \, \sin \theta_\text{t} \, d\theta_\text{t} \, d\phi_\text{t} = 2\pi \int_0^1 dy \,y = \pi
%\end{align}
%with substitution $y=\sin\theta_\text{t}$.
%The above surface integral computes the area projection of the omnidirectional antenna response onto the broadside direction ($\theta_{\text t}=0$). 
%More generally, for an arbitrary antenna response, \eqref{DOF_Landau} computes the projected area subtended %\angel{"suface" and "area" seem a bit redundant} 
%by $|A_{\text t}^+(\theta_{\text t},\phi_{\text t})|$ as viewed from the transmit array. It reduces to the solid angle $\iint_{{\rm supp}(|A_{\text t}^+(\theta_\text{t},\phi_\text{t})|)} \sin \theta_{\text t} \, d\theta_{\text t} d\phi_{\text t}$ subtended by the antenna response for a sufficiently narrow support that is centered about the broadside direction, whereby $\cos \theta_\text{t}\approx 1$.
%%\andrea{shall we add a picture?}
%%{\color{blue}More generally, the above solid angle is replaced by a surface integral computing the area projection of the antenna response onto the broadside direction of the array.}
%%every differential element $d\Omega_\text{t}$ partitioning the support of an antenna response needs be projected 
%%Thus, ${\sf n}_0(\vect{{\sf C}}_\text{t})$ is the product of the normalized array aperture and the shared projection of the angular response of the antenna. For a planar array perpendicular to the arbitrarily chosen $z$-axis, the latter is specified by the shared projection of the solid angle subtended by the antenna response with the planar array. 
%%\angel{Hard to interpret}
%%\angel{Again, hard to interpret}
%As \eqref{DOF_Landau_iso} upper bounds \eqref{DOF_Landau}, more directive antennas create an ampler range over which superdirectivity may arise.
%
%\begin{figure}
%\centering\vspace{-0.0cm}
%\includegraphics[width=.999\linewidth]{Fig2} 
%\caption{Normalized sorted eigenvalues of $\vect{{\sf C}}_\text{t}$ in \eqref{Toeplitz_coupling} for various apertures and punctiform antennas spaced by $d_\text{t}=\lambda/2$.}\vspace{-0cm}
%\label{fig:Fig2}
%\end{figure}
%


%\section{MIMO channel with coupling \\ at the transmitter}

%\subsection{Composite MIMO Channel Matrix}
%%Before proceeding, let us derive the exact MIMO model, which will then be used to test the accuracy of the UIU formulation.
%%\footnote{The spatial discretization of \eqref{Pt_inner_product} would yield the same result of \eqref{time_avg_power_discrete}.}
%
%Rewrite the transmit power in \eqref{time_avg_power_discrete} as
%\begin{align}   \label{time_avg_power_discrete_rewritten}   
%P_\text{t} = R \, \| \vect{{\sf j}}_\text{t}\|^2 
%\end{align}
%where $\vect{{\sf j}}_\text{t} = \vect{{\sf C}}_\text{t}^{1/2} \vect{j}_\text{t} \in \Complex^{N_\text{t}}$ is the composite current vector with
%%$\vect{{\sf C}}^{1/2}_\text{t}$ %= \vect{U}_{\text{c}} \vect{\Lambda}^{1/2}_{\text{c}}$ 
%%invertible, the square root matrix of
%$\vect{{\sf C}}_\text{t}$ in \eqref{impedance_matrix_general}.
%%, given 
%%$\vect{{\sf C}}_\text{t} = \vect{{\sf C}}_\text{t}^{\Htran/2} \vect{{\sf C}}_\text{t}^{1/2}$. In turn, $\vect{{\sf C}}^{-1/2}_\text{t}$ denotes its inverse, %such that $\vect{{\sf C}}^{-1}_\text{t} = \vect{{\sf C}}_\text{t}^{-1/2} \vect{{\sf C}}_\text{t}^{-\Htran/2}$.
%%$\vect{U}_{\text{c}}  \in \Complex^{N_\text{t} \times N_\text{t}}$ unitary and $\vect{\Lambda}_{\text{c}}  \in \Complex^{N_\text{t} \times N_\text{t}}$ diagonal with positive entries the left singular matrix and eigenvalue matrix of $\vect{{\sf C}}_\text{t}$ in \eqref{impedance_matrix_general}.
%
%As $\| \vect{j}_\text{t} \|^2$ does not provide the radiated power in \eqref{time_avg_power_discrete_rewritten}, we cannot regard \eqref{convolution_sampled} as a physical system. \angel{This seems at odds with \eqref{convolution_sampled} appearing in a subsection entitled "Physics-based channel model"}  
%However rewriting  \eqref{convolution_sampled} as
%\begin{equation} \label{convolution_coupling_sampled}
%\vect{v}_\text{r} = \vect{{\sf H}} \,\vect{{\sf j}}_\text{t}
%\end{equation}
%where %$\vect{{\sf H}} \in \Complex^{N_\text{r} \times N_\text{t}}$, given by
%\begin{equation} \label{channel_mat_c}
%\vect{{\sf H}} = \vect{H} \vect{{\sf C}}^{-1/2}_\text{t}
%\end{equation}
%is the composite channel matrix, %given $\vect{{\sf C}}_\text{t}^{-1/2}$ the inverse square root matrix of $\vect{{\sf C}}_\text{t}$.
%% = \vect{\Lambda}^{-1/2}_{\text{c}} \vect{U}_{\text{c}}^{\Htran}$.
%we have that the composite current and channel do coincide with the %more general
%circuit-theoretic formulas \cite[Eqs.~99, 101]{Nossek2010}
%%when the latter are
%specialized to the multiport settings of Fig.~\ref{fig:multiport}.
%The model in \eqref{convolution_coupling_sampled} is equivalent to that of \eqref{convolution_sampled} in the sense that both provide the same output, but only the latter is physically meaningful, 
%%However, only the squared norm associated with $\vect{j}_\vect{{\sf C}}$ describes a physical system, 
%yielding the radiated power (scaled). \angel{scaled how?}
%
%Although \eqref{channel_mat_c} captures the coupling, it is seemingly disconnected from the afore-introduced Fourier model. A seamless integration of channel characteristics and mutual coupling is presented next.

\section{MIMO Model with Transmit Coupling} \label{sec:MIMO_coupled}

\subsection{Composition of Channel and Coupling}

%\angel{While we seem to be stating this goal for this subsection, it actually happens in the next subsection. So this opening sentence seems out of place}
Let us rewrite \eqref{transmit_power} as %an inner product in $L^2$, namely
\begin{align} \label{psd_op}
{\sf P}_\text{t} = \langle\mathcal{C} j_\text{t}, j_\text{t}\rangle 
\end{align}
with $\mathcal{C}$ the %positive-definite
%Hilbert-Schmidt
operator associated with %the real-valued kernel 
the real %space-invariant
kernel
$\frac{{\sf R}}{2} {\sf c}_{\text t}(\vect{{\sf v}})$ given ${\sf c}_{\text t}(\vect{{\sf v}})$ in \eqref{real_impedance_kernel_antenna}. %For every ${\sf P}_\text{t} > 0$,
By virtue of its positive-definiteness and symmetry, $(\mathcal{C} j_\text{t})(\vect{{\sf r}}) = (\mathcal{C}^{1/2} \mathcal{C}^{1/2} j_\text{t})(\vect{{\sf r}})$
%$\forall j_\text{t}$
for some other %is the Hilbert Schmidt operator associated with $c^{-1/2}_{\text{t}}(\vect{{\sf v}})$}
real symmetric and positive-definite operator $\mathcal{C}^{1/2}$ with associated kernel $\sqrt{{\sf R}/2} \,{\sf c}^{1/2}_{\text{t}}(\vect{{\sf v}})$, such that
 \begin{align} \label{conv_ct_3D}
{\sf c}_\text{t}(\vect{{\sf v}}) = \iiint_{-\infty}^\infty \!   {\sf c}^{1/2}_{\text{t}}(\vect{{\sf t}}) {\sf c}^{1/2}_{\text{t}}(\vect{{\sf v}} - \vect{{\sf t}}) \, d\vect{{\sf t}}.
\end{align}
Then,  \eqref{psd_op} can be rewritten as
\begin{align} \label{psd_op_continue}
{\sf P}_\text{t} & 
= \langle\mathcal{C}^{1/2} \mathcal{C}^{1/2} j_\text{t}, j_\text{t}\rangle \\  \label{Pt_inner_product_half}
&= \langle  \mathcal{C}^{1/2} j_\text{t},  \mathcal{C}^{1/2} j_\text{t}\rangle  \\
%& = \langle {\sf j}_\text{t}, {\sf j}_\text{t}\rangle  \\ \label{Pt_inner_product}
& = \| {\sf j}_\text{t} \|^2,
\end{align} 
%\angel{Surely the norm above must be squared}
where we exploited the symmetry of $\mathcal{C}^{1/2}$ and defined as
\begin{equation} \label{j_C_op} 
{\sf j}_\text{t}(\vect{{\sf s}}) = (\mathcal{C}^{1/2} j_\text{t})(\vect{{\sf s}}) = \sqrt{\frac{{\sf R}}{2}} \iiint_{-\infty}^\infty  {\sf c}^{1/2}_{\text{t}}(\vect{{\sf s}}-\vect{{\sf t}}) \, j_{\text{t}}(\vect{{\sf t}}) \, d\vect{{\sf t}}
\end{equation}
the composition of current density and coupling.
%the composition of input current applied at each antenna individually and coupling among different antennas.
%obtainable from a spatial convolution of the current density $j_{\text{t}}(\vect{{\sf r}})$ and a linear and space-invariant filter with impulse response ${\sf c}^{1/2}_{\text{t}}(\vect{{\sf r}})$. 
%We denote with ${\sf C}^{1/2}_\text{t}(\vect{\kappa})$ the wavenumber response of the system.
%Specializing \eqref{convolution} to planar arrays and 
Rewriting \eqref{convolution} in terms of \eqref{j_C_op} leads to
\begin{align}  \label{MIMO_model_C_cont}
v_{\text{r}}(\vect{{\sf r}}) &
%= (\mathcal{H} j_{\text{t}})(\vect{r}) 
 %= (\mathcal{H} \mathcal{C}^{-1/2} \mathcal{C}^{-1/2} j_\text{t})(\vect{r}) \\& 
= ({\sf H} \, {\sf j}_\text{t})(\vect{{\sf r}})  
= \iiint_{-\infty}^\infty {\sf h}(\vect{{\sf r}},\vect{{\sf s}}) \, {\sf j}_\text{t}(\vect{{\sf s}}) \, d\vect{{\sf s}},
\end{align}
where ${\sf H} = \mathcal{H} \mathcal{C}^{-1/2}$ is the composition of the operator
modeling the uncoupled channel,
$\mathcal{H}$, with $\mathcal{C}^{-1/2}$, %the inverse operator of $\mathcal{C}^{1/2}$,
positive-definite and associated with the kernel $\sqrt{2/{\sf R}} \, {\sf c}^{-1/2}_{\text{t}}(\vect{{\sf v}})$.
The composition so defined associates with the space-variant kernel 
\begin{align}  
{\sf h}(\vect{{\sf r}},\vect{{\sf s}}) & = (\mathcal{C}^{-1/2} h)(\vect{{\sf r}},\vect{{\sf s}})  \\ \label{composite_channel_op}
& =  \sqrt{\frac{2}{{\sf R}}}\iiint_{-\infty}^\infty \!\!\! h(\vect{{\sf r}},\vect{{\sf t}}) \, {\sf c}^{-1/2}_{\text{t}}(\vect{{\sf t}}-\vect{{\sf s}}) \, d\vect{{\sf t}}.
\end{align}

 The invertibility of $\mathcal{C}^{1/2}$ is proven in Appendix~E;
 %It descends from the positiveness of $\mathcal{C}$ and
 it requires that $|{\sf A}_{\text{t}}(\vect{{\sf k}})|$ be strictly positive almost everywhere.
 %\angel{I know this proof is short, but I'd move it to an appendix anyway because here it breaks the flow}
  
  
%While ${\sf j}_\text{t}(\vect{{\sf r}})$ and ${\sf h}(\vect{{\sf r}},\vect{{\sf s}})$ lack meaning---unlike $j_{\text{t}}(\vect{{\sf s}})$ and $h(\vect{{\sf r}},\vect{{\sf s}})$---they represent another embodiment of the MIMO formulation that is conducive to analysis, as will be seen.
%In it, coupling is applied to the current and its inverse to the channel, with their conjunction being immaterial to the output voltage.
The model in \eqref{MIMO_model_C_cont} is equivalent to that of \eqref{convolution} in the sense that both provide the same output,
with ${\sf j}_\text{t}(\vect{{\sf r}})$ and ${\sf h}(\vect{{\sf r}},\vect{{\sf s}})$ providing an alternative embodiment of the MIMO formulation, in some ways advantageous relative to $j_{\text{t}}(\vect{{\sf s}})$ and $h(\vect{{\sf r}},\vect{{\sf s}})$.

 % but only the latter is physically meaningful.

%\begin{equation}
%\int_{t_0}^\infty dt \, e^{\imagunit \omega_0 t} e^{-\imagunit \omega t} = \frac{\imagunit}{\omega_0 - \omega} e^{\imagunit (\omega_0-\omega) t_0}
%\end{equation}
%with region of convergence $\Im\{\omega_0\}>0$,

%It is proven in App.~\ref{app:invertibility} that the existence of $\mathcal{C}^{-1/2}$ for any antenna spectrum can only be guaranteed for a planar transmitter. 
%the stability of the inverse filter is guaranteed for any spectrum bounded above \cite{Unser1994}. \angel{This previous sentence needs some work...} % whereby $S_{\text t}(\vect{{\sf k}}) \in L^2$.
%This agrees with the more general fact that a plane-wave spectrum of an electric field is determined non-uniquely by the field on any $z$-plane \cite{PlaneWaveBook,MarzettaIT}, rendering the transformation non-injective, in turn, not invertible.
% which renders the transformation non-injective \angel{what does?}.
%The same argument translates to coupling given the isomorphism between the underlying spectral representations.
%This ambiguity is resolved if antennas are deployed on a same plane allowing, e.g., for upgoing propagation, %{\color{blue}while omitting the normalization} %\angel{this sentence is a bit of a mess...}
%from which we obtain
%\begin{align}  \label{pippo}
%{\sf C}^{1/2}_\text{t}(\vect{\kappa}) & =  
%\frac{|A_{\text t}^+(\vect{\kappa})|}{\sqrt{\gamma}} \, \mathbbm{1}_{\|\vect{\kappa}\|\le \kappa}(\vect{\kappa}).
%\end{align}
%The above 2D spectrum is positive and bounded above $\forall \vect{\kappa}$. 
%The latter condition is ensured by translating the same argument above to $|A_{\text t}^+(\vect{\kappa})|$ in \eqref{time_avg_power_final_antenna} and noticing that $0 < \gamma \le \kappa$ with $\kappa = \frac{2\pi}{\lambda} < \infty$.


%A physics-based model for the small-scale fading is revisited next. 
%[DO YOU MEAN NLOS, OR RICH SCATTERING?]
%By virtue of the sampling theorem, this provides a worst case in terms of the impact of mutual coupling.
%Among all, stochastic channel models represent wave propagation into different environments allowing to draw general inferences. Our focus is on  for which a spectral representation is available . 

%To broaden the focus from power gains with single-stream transmissions to the spectral efficiency with arbitrary transmission strategies, chiefly spatial multiplexing, a comprehensive model for the MIMO channel is required.
%Building on the planar transmitter setting, %in absence of backward propagation,
%only upgoing propagation is retained.
%\angel{Perhaps we should make explicit that everything thus far has been for beamforming (rank-1) transmissions}
  
\subsection{Composite MIMO Channel Matrix}

Before proceeding, let us verify that sampling our continuous model at antenna locations yields established results for discrete arrays \cite{Nossek2010}. 
%We denote $\text{area}(\cdot)$ the Lebesgue measure on $\Real^2$.
Sampling the convolution in \eqref{MIMO_model_C_cont} yields the noiseless MIMO relationship inclusive of transmit coupling 
%%\andrea{For a discrete array, replacing the input current in \eqref{convolution} with $j_\text{t}(\vect{{\sf s}}) = \sum_m j_{\text{t}}(\vect{{\sf s}}_m) \delta(\vect{{\sf s}} - \vect{{\sf s}}_m)$ would give the same result with no approximation involved. See also \eqref{farfield_pattern_planar_array}.}
  \begin{equation} \label{convolution_coupling_sampled}
   \vect{v}_\text{r} = \vect{{\sf H}} \,\vect{{\sf j}}_\text{t},
    \end{equation}
%which is consistent with ${\sf P}_\text{t} =  \| \vect{{\sf j}}_\text{t}\|^2$,
%as obtainable from \eqref{Pt_inner_product} for a discrete-space source.
where $[\vect{{\sf H}}]_{n,m} = {\sf h}(\vect{{\sf r}}_n,\vect{{\sf s}}_m)$ and $[\vect{{\sf j}}_\text{t}]_m = {\sf j}_\text{t}(\vect{{\sf s}}_m)$ are the composite channel matrix and composite current vector, respectively. 
These are obtainable from \eqref{composite_channel_op} and \eqref{j_C_op}  as 
    \begin{align} \label{channel_samples_coupled}
\vect{{\sf H}} = \sqrt{\frac{2}{{\sf R}}} \, \vect{H} \vect{{\sf C}}_\text{t}^{-1/2}
\end{align}
and 
    \begin{align} \label{composite_current_vec}
\vect{{\sf j}}_\text{t} = \sqrt{\frac{{\sf R}}{2}} \, \vect{{\sf C}}_\text{t}^{1/2} \vect{j}_\text{t}.
\end{align}
%with 
%$\vect{{\sf C}}_\text{t}^{1/2}$ such that
%\begin{align} \label{conv_ct_matrix}
%\vect{{\sf C}}_\text{t} = \vect{{\sf C}}_\text{t}^{1/2} \vect{{\sf C}}_\text{t}^{1/2},
%\end{align}
%real and positive definite, 
%with inverse $\vect{{\sf C}}_\text{t}^{-1/2}$.
Expanding the space-lag variable in \eqref{conv_ct_3D} as $\vect{{\sf v}} = \vect{{\sf r}} - \vect{{\sf s}}$ and applying the change of variables $\vect{{\sf r}}-\vect{{\sf t}} = \vect{{\sf y}}$, 
 \begin{align}  \label{conv_ct_expanded}
{\sf c}_\text{t}(\vect{{\sf r}}-\vect{{\sf s}}) 
& = \iiint_{-\infty}^\infty \! {\sf c}^{1/2}_{\text{t}}(\vect{{\sf r}}-\vect{{\sf y}}) {\sf c}^{1/2}_{\text{t}}(\vect{{\sf y}}-\vect{{\sf s}}) \, d\vect{{\sf y}}.
\end{align} 
Sampling \eqref{conv_ct_expanded} at any two transmit antenna locations and exploiting the matrix symmetry yields
%\eqref{conv_ct_matrix} with entries
$[\vect{{\sf C}}_\text{t}]_{n,m} = {\sf c}_\text{t}(\vect{{\sf r}}_n-\vect{{\sf s}}_{m})$.     
%\begin{align}  \label{uncoupled_channel}
%[\vect{H}]_{n,m} & = \sqrt{\frac{L_{{\text t},x} L_{{\text t},y}}{N_\text{t}}} \, h(\vect{{ r}}_n,\vect{{ s}}_m) \\
%\label{uncoupled_current_vec}
%[\vect{j}_\text{t}]_m & = \sqrt{\frac{L_{{\text t},x} L_{{\text t},y}}{N_\text{t}}} \, j_\text{t}(\vect{{ s}}_m),
%\end{align}
%the latter approximately yielding the transmit power for uncoupled antennas as per \eqref{uncoupled_power}, namely $P_{\text t} \approx \|\vect{j}_{\text t}\|^2$.
%%with $[\vect{j}_\text{t}]_m = j_{\text{t}}(\vect{{\sf s}}_m)$ and $[\vect{v}_\text{r}]_n = v_{\text{r}}(\vect{{\sf r}}_n)$
%%\angel{These quantities have been defined already in earlier sections}
%while $\vect{H}$ is the channel matrix whose entries admit the
%admitting the 
%with entries $[\vect{H}]_{nm} = (L^2_\text{t}/N_\text{t}) h(\vect{r}_n,\vect{s}_m)$, for $n=1, \ldots, N_\text{r}$ and $m=1, \ldots, N_\text{t}$.
%[IT SEEMS ASYMMETRIC THAT $L_\text{t}$ APPEARS, BUT NOT $L_\text{r}$]
%The discretization of \eqref{Fourier_series} then yields the 

%whereby $\vect{{\sf C}}_\text{t}^{1/2} \vect{{\sf C}}_\text{t}^{- 1/2} = \vect{{\sf C}}_\text{t}^{-1/2} \vect{{\sf C}}_\text{t}^{1/2} \approx \vect{I}_{N_\text{t}}$ when $N_\text{t}$ is large for a fixed aperture.
%the principal square root of $\vect{{\sf C}}_\text{t}$, 
%For later use, define
%\begin{align}
%[\vect{{\sf C}}_\text{t}^{-1/2}]_{n,\ell} = \frac{L_\text{t}}{\sqrt{N_\text{t}}} \, c^{-1/2}_{\text{t}}(\vect{{\sf r}}_n-\vect{{\sf y}}_\ell)
%\end{align}
%the inverse of 

 
%    The model in \eqref{convolution_coupling_sampled} is equivalent to that of \eqref{open_circuit_MIMO} in the sense that both provide the same output, but only the latter is physically meaningful, yielding the transmit power 
%    \begin{align}  \label{time_avg_power_discrete}
%    {\sf P}_\text{t}  & %\approx R \left(\frac{L_\text{t}^2}{N_\text{t}}\right)^2  \vect{j}_\text{t}^{\Htran} \vect{{\sf C}}_\text{t} \vect{j}_\text{t} \\ 
% =  \| \vect{{\sf j}}_\text{t}\|^2
%\end{align}
%as obtainable from \eqref{Pt_inner_product} for a discrete space source.
    %%\angel{scaled how?}
    %\andrea{We could probably simply our notation even further by removing the subscript $_\text{t}$ everywhere except Sec.~II, as we only deal with coupling at the transmit side and use a Kronecker model now.}
The composite current and composite channel coincide with the %more general
    circuit-theoretic formulas \cite[Eqs.~99, 101]{Nossek2010} specialized to the multiport settings of Fig.~\ref{fig:multiport} and uncorrelated noise.
%(There is lack of reciprocity in \eqref{channel_samples_coupled}, compared to \cite[Eq.~101]{Nossek2010}, due to the absence of coupling among receive antennas.) 
The lack of reciprocity in \eqref{channel_samples_coupled}, compared to \cite[Eq.~101]{Nossek2010}, arises from the absence of coupling among receive antennas, as no current is drawn from them (see Fig.~\ref{fig:multiport}). 
%Also, the factor $1/2$ it is because your formulation is passband rather than baseband. Your powers are for phasors, so in a sense they are actually densities (Watt/Hz). Once you multiply by the bandwidth (or integrate over it), you'd get the actual power in Watts. And there's a factor of 1/2 between the passband and baseband bandwidths.
In turn, the factor $1/2$ arises from the passband formulation, in contrast with the baseband formulation in \cite{Nossek2010};
%when integrating power densities over bandwidth to yield power,
the baseband bandwidth is half the passband bandwidth.
% The former is because no current is drawn from the receive antenna ports. % each being closed on an infinite load impedance. % (see Fig.~\ref{fig:multiport}).
%\angel{Noise correlation?}
 
 
  
 %\subsection{Transmit Power by an Array Source}
%\subsection{Real Part of the Transmit Impedance Matrix}
\subsection{Transmit Coupling Matrix}

%\andrea{The discretization was scattered between isotropic and non-isotropic antennas. I merged both treatments here so we do not discretize before Sec. V.D.}
%
%\andrea{Also, we introduced a coupling matrix for both the wavenumber and angular domain. I removed the former as our goal was to compare against Nossek's formula in the angular domain.}
%
%\andrea{In the angular formulation, I did not separate the upper and lower hemisphere contributions as both the antenna angular spectrum and Nossek's formula are defined over a sphere. In the complex exponential of \eqref{Gavi}, the cosine  nicely do the job.}

%\angel{Shouldn't this, here and at other points, be wavevector rather than wavenumber?}
%\andrea{The terminology is consistent to Franceschetti's.}
Changing the domain from wavenumber to spherical, it is found that
%\footnote{[WE CAN PROBABLY DISPENSE WITH THIS FOOTNOTE] Due to the tight connection between the wavenumber and angular domain we will refer to both domains interchangeably throughout this paper. }
\begin{equation} \label{wavenumber_spherical}
\vect{\kappa}(\theta,\phi) = \left(\kappa \sin \theta \cos \phi, \kappa \sin \theta \sin \phi \right),
\end{equation}
and $\gamma(\theta) = \sqrt{\kappa^2 - \|\vect{\kappa}\|^2} = \kappa \cos \theta$, with Jacobian 
\begin{equation} \label{Jacobian}
\frac{\partial \vect{\kappa}}{ \partial(\theta,\phi)}  = \kappa^2 \cos \theta \sin \theta.
\end{equation}
With these variable changes applied to \eqref{real_impedance_kernel_antenna}, 
\begin{align} \notag
[\vect{{\sf C}}_\text{t}]_{n,m} & =  
%{\sf c}_{\text t}(\vect{{\sf r}}_n-\vect{{\sf s}}_{m}) & =  
\frac{1}{4\pi}
\int_0^\pi \int_0^{2\pi}  |{\sf A}_\text{t}(\theta_\text{t},\phi_\text{t})|^2 \\ & \hspace{1cm} \label{impedance_matrix_general} 
 \cdot a_n(\theta_{\text t},\phi_{\text t}) \overline{a_{m}(\theta_{\text t},\phi_{\text t})}
 \sin \theta_{\text t}  \,  d\theta_{\text t} \, d\phi_{\text t},
\end{align}
where the array response vector is
\begin{align}  \label{Gavi}
a_{n}(\theta_{\text t},\phi_{\text t}) & = e^{\imagunit (\vect{\kappa}^{\Ttran}(\theta_\text{t},\phi_\text{t}) \vect{r}_n + \gamma(\theta_\text{t}) r_{z,n})}
\end{align}
%and
with $\vect{\kappa}(\cdot,\cdot)$ as per \eqref{wavenumber_spherical} and with
${\sf A}_\text{t}(\cdot,\cdot)$ being either ${\sf A}_\text{t}^+(\cdot,\cdot)$ over the upper hemisphere, $\{\theta_\text{t} \in [0,\frac{\pi}{2}]\}$,
 or ${\sf A}_\text{t}^-(\cdot,\cdot)$ over the lower hemisphere, $\{\theta_\text{t} \in (\frac{\pi}{2}, \pi]\}$.
%\begin{align} \label{A_theta_phi}
%{\sf A}_\text{t}(\theta_\text{t},\phi_\text{t}) =
%\begin{cases} 
%{\sf A}_\text{t}^+(\theta_\text{t},\phi_\text{t}) & \quad (\theta_\text{t},\phi_\text{t})\in \Omega^+\\
%{\sf A}_\text{t}^-(\theta_\text{t},\phi_\text{t}) & \quad (\theta_\text{t},\phi_\text{t})\in \Omega^-
%\end{cases}
%\end{align}
%defined within $\Omega = \Omega^+ \cup \Omega^-$ given $\Omega^+ = \{\theta_\text{t} \in [0,\pi/2], \phi_\text{t}\in[0,2\pi)\}$ and $\Omega^- = \{\theta_\text{t} \in (\pi/2, \pi], \phi_\text{t}\in[0,2\pi)\}$ the unit upper and lower hemispheres.
From \eqref{norm_A_spectrum},
\begin{align} \label{norm_A_spectrum_angle}
1  & = \frac{1}{4\pi} \int_0^\pi \int_0^{2\pi}  |{\sf A}_\text{t}(\theta_{\text t},\phi_{\text t})|^2 \, \sin \theta_{\text t}  \,  d\theta_{\text t} \, d\phi_{\text t}
\end{align}
with omnidirectional coupling, including punctiform antennas, arising for ${\sf A}_\text{t}(\theta_\text{t},\phi_\text{t}) = 1$.  % $\forall (\theta_\text{t},\phi_\text{t})$.
The function $|{\sf A}_\text{t}(\theta_{\text t},\phi_{\text t})|^2$ is the %power angle spectrum (PAS).} \angel{this should be "pattern" instead of PAS} 
 power pattern of each antenna. %which expresses the transmitted power by an impulsive input current.}

%{\color{blue}For instance, the pattern of a rectangular antenna parallel to the $xy$-plane, with dimensions $L_{\text{a},x}$ and $L_{\text{a},y}$, satisfies
%\begin{align} \label{antenna_pattern_rectangular}
%{\sf A}_\text{t}(\theta_{\text t},\phi_{\text t}) \propto \sinc\!\left(\!\frac{L_{\text{a},x}}{\lambda} \sin \theta_{\text t} \cos \phi_{\text t} \!\right) \sinc\!\left(\!\frac{L_{\text{a},y}}{\lambda} \sin \theta_{\text t} \sin \phi_{\text t}\! \right)
%\end{align}
%normalized according to \eqref{norm_A_spectrum_angle}.
%Spatial causality %is determined by the minimum radius of the circle circumscribing the antenna, 
%requires that $r_0 = \max(L_{\text{a},x},L_{\text{a},y})/\sqrt{2}$. % subsuming omnidirectional patterns
%When both electrical dimensions approach zero, the antennas become punctiform.}
%, such that $\sinc(\cdot)\to 1$ $\forall (\theta_{\text t},\phi_{\text t})$.}

%at the transmit antenna locations,
%$[\vect{{\sf C}}_{\text{t}}]_{m,m^\prime} = {\sf c}_{\text t}(\vect{{\sf r}}_m-\vect{{\sf s}}_m^\prime)$ for $m,m^\prime = 1, \ldots, N_\text{t}$.
%\begin{align}
%\vect{{\sf C}}_{\text{t}} & = 
%\frac{1}{2\pi \kappa}
% \iint_{\|\vect{\kappa}\|\le \kappa}  \frac{1}{\gamma} \, \vect{a}(\vect{\kappa}) \vect{a}^{\Htran}(\vect{\kappa}) d\vect{\kappa} 
% %\\& = \iint_{-\infty}^{\infty}  |A(\theta,\phi)|^2 \vect{a}(\theta) \vect{a}^{\Ttran}(\theta) \sin(\theta) d\theta d\phi
%\end{align}
%with $\vect{a}(\vect{\kappa}) \in \Complex^{N_\text{t}}$ the array response vector,
%\begin{equation}
%[\vect{a}]_{m}(\vect{\kappa}) = 
%\begin{cases} \displaystyle
%A_{\text t}^+(\vect{\kappa}) \, e^{\imagunit (\vect{\kappa}^{\Ttran} \vect{v}_m + \gamma v_{z,m})} & v_{z,m} > 2r \\\displaystyle
%A_{\text t}^-(\vect{\kappa}) \, e^{\imagunit (\vect{\kappa}^{\Ttran} \vect{v}_m - \gamma v_{z,m})} & v_{z,m} < -2r
%\end{cases}
%\end{equation}
%measured at $\vect{{\sf v}}_m = (\vect{v}_m, |v_{z,m}|)$, $m=1, \ldots, N_\text{t}$. 
%When $v_z\ge 0$, corresponding to the upper hemisphere,


For a $z$-aligned array of punctiform antennas, %${\sf A}_\text{t}(\theta_\text{t},\phi_\text{t}) = 1$ and
%(i.e., $v_{x,n}=v_{y,n}=0$),
%exploiting the azimuthal symmetry,
\eqref{impedance_matrix_general} reduces to the impedance formula in \cite[Eq.~44]{Nossek2010}.
%Yet, the causality constraint requires $d_m > 2r$ for an antenna of radial dimension $r\ge0$. %We now connect the derived representation to other available in the research literature, such as \cite{Nossek2010}, obtainable by pursuing a different antenna theory approach. 
%Under isotropic coupling, 
%For punctiform antennas in particular,
And, for a uniform linear array (ULA) in particular,
\begin{align}  \label{Toeplitz_coupling}
[\vect{{\sf C}}_\text{t}]_{n,m}
%{\sf c}_{\text t}(\vect{{\sf r}}_n-\vect{{\sf s}}_{m}) 
= \sinc \! \left (2 \frac{\|\vect{{\sf r}}_n-\vect{{\sf s}}_{m}\|} {\lambda} \right),
\end{align}
consistent with \eqref{real_impedance_kernel_spherical}.
For a ULA, a real-symmetric Toeplitz structure emerges for the transmit coupling matrix while, for a uniform planar array (UPA), it is block-Toeplitz. % yielding a real-symmetric Toeplitz matrix.}
In general, uncoupled MIMO would require %${\sf c}_{\text t}(\vect{{\sf r}}_n-\vect{{\sf s}}_{m}) = \delta_{nm}$, 
$\vect{{\sf C}}_\text{t} = \vect{I}_{N_\text{t}}$, meaning transmit antennas that are infinitely apart from each other.
This is incongruent with the transmitter fitting in a certain form factor. 
%In fact, as soon as such a limitation is imposed, the eigenvalues of \eqref{Toeplitz_coupling} are never equal, even at a half-wavelength spacing.
%This is consistent with the fact that
%Along the diagonals of a UPA, for instance, antennas are $\lambda/ \sqrt{2}$ apart, yielding a nonzero mutual coupling along those directions.
%due to the inherent coupling among the diagonal elements of the transmit array.
%[WELL, UNCOUPLING IS ACHIEVED AS THE ANTENNA SPACING GROWS LARGE, BUT IT IS INCONGRUENT TO HAVE AN ARRAY WHOSE ANTENNAS ARE INFINITELY FAR APART]
    
\section{Fourier Model for MIMO Channels \\ Without Coupling} \label{sec:holo_MIMO_uncoupled}
%As the transmit and receive arrays progressively densify within a given form factor, 
%the composite channel matrix $\vect{H}_\vect{{\sf C}} : \Complex^{N_\text{t}} \to \Complex^{N_\text{r}}$ in \eqref{channel_mat_c_rx}  the wireless propagation becomes more accurately described by the Hilbert-Schmidt operator $\mathcal{H}_\mathcal{C}$, such that (see App.~\ref{app:functional})

\subsection{Fourier Representation of Stationary Channels}

%The Fourier plane-wave spectral representation of {\color{blue}a spatial channel without coupling is \cite{PizzoIT21} 
%\begin{align}  \label{spectral_representation_z} 
%h(\vect{{\sf r}},\vect{{\sf s}}) & = \iiiint_{-\infty}^\infty \frac{d\vect{k}}{2\pi} \frac{d\vect{\kappa}}{2\pi}  e^{\imagunit \vect{k}^{\Ttran} \vect{r}} H(\vect{k},\vect{\kappa};r_z,s_z) e^{-\imagunit \vect{\kappa}^{\Ttran} \vect{s}}
%\end{align}
%with wavenumber spectrum
%\begin{align}  \label{spectrum_z} 
%H(\vect{k},\vect{\kappa};r_z,s_z) = \frac{e^{\imagunit \gamma(\vect{k}) r_z}}{\sqrt{\kappa/2\pi}} \frac{H^{++}(\vect{k},\vect{\kappa})}{\sqrt{\gamma(\vect{k}) \gamma(\vect{\kappa})}} \frac{e^{-\imagunit \gamma(\vect{\kappa}) s_z}}{\sqrt{\kappa/2\pi}},
%\end{align}}
%given $\gamma(\vect{\cdot})$ in \eqref{gamma}, and with $H^{++}(\vect{k},\vect{\kappa})$ the %bivariate
%angular spectrum. % of the channel.
%The above expression is reminiscent of the virtual channel representation pioneered in \cite{Sayeed2002}, except that \eqref{spectral_representation_z} is also valid in the near field.
%%which is real-valued within the domain of integration.
%%; imaginary contributions are neglected, as only measurable in close proximity of source and scatterers.
%It can be regarded as the kernel representation of the
%%Hilbert-Schmidt
%operator $\mathcal{H}$ in \eqref{convolution}.

%Like the transmit impedance in \eqref{voltage}
The channel kernel in \eqref{convolution} comprises %an upgoing 
a causal
%(along the $z$-axis) 
and %a downgoing \andrea{"upgoing/downgoing"}
 an anticausal
%(along the $-z$-axis) 
component at each end of the link, resulting in four possible combinations. 
%\angel{Let's hope that readers can follow this, and reconcile with the fact that you're referring to the transimpedance, which goes directionally from one end of the link to the other.}
%For the sake of specificity, we henceforth retain only .} 
The %upgoing-upgoing \andrea{"upgoing"} 
 causal-causal component of a stationary Gaussian channel %$h(\vect{{\sf r}},\vect{{\sf s}})$
can be represented in Fourier form as \cite[Thm.~2]{PizzoIT21}
\begin{align}  \notag
h(\vect{{\sf r}},\vect{{\sf s}}) & = \iint_{\|\vect{k}\|\le\kappa} \frac{d\vect{k}}{\sqrt{2\pi \kappa}} \iint_{\|\vect{\kappa}\|\le\kappa} \frac{d\vect{\kappa}}{\sqrt{2\pi \kappa}} \, \frac{\tilde{H}^{++}(\vect{k},\vect{\kappa})}{\sqrt{\gamma(\vect{k}) \gamma(\vect{\kappa})}}  \\ & \hspace{2cm} \label{spectral_representation_z} 
 \cdot e^{\imagunit (\vect{k}^{\Ttran} \vect{r} + \gamma(\vect{k}) r_z)}  e^{-\imagunit (\vect{\kappa}^{\Ttran} \vect{s} + \gamma(\vect{\kappa}) s_z)}
\end{align}
with $\gamma(\vect{\cdot})$ in \eqref{gamma} and with $\tilde{H}^{++}(\vect{k},\vect{\kappa})$ the %the upgoing-upgoing 
spectrum,  
%\begin{align}   \label{spectrum}
%H^{++}(\vect{k},\vect{\kappa}) = %\mathbbm{1}_{\|\vect{k}\|\le\kappa}(\vect{k}) 
%\frac{\tilde{H}^{++}(\vect{k},\vect{\kappa})}{\sqrt{\gamma(\vect{k}) \gamma(\vect{\kappa})}} 
%%\mathbbm{1}_{\|\vect{\kappa}\|\le\kappa}(\vect{\kappa}), 
%\end{align}
%given $\gamma(\vect{\cdot})$ in \eqref{gamma}, and with $\tilde{H}^{++}(\vect{k},\vect{\kappa})$ the %bivariate
%angular spectrum, % of the channel.
an independent complex Gaussian process.
The dependences on $r_z$ and $s_z$ are henceforth omitted as they can be absorbed into $\tilde{H}^{++}(\vect{k},\vect{\kappa})$ and are immaterial due to the statistical equivalence %of the distributions
at different $z$-planes, namely
\begin{equation}
\tilde{H}^{++}(\vect{k},\vect{\kappa}) \sim e^{\imagunit \gamma(\vect{k}) r_z} \tilde{H}^{++}(\vect{k},\vect{\kappa}) \, e^{-\imagunit \gamma(\vect{\kappa}) s_z}.
\end{equation}
%In turn, the indicator functions in \eqref{spectrum} render $H(\vect{k},\vect{\kappa})$ spatially bandlimited, ruling out evanescent waves at either end of the link.
%The above expression is reminiscent of the virtual channel representation pioneered in \cite{Sayeed2002}, except that \eqref{spectral_representation_z} is also valid in the near field.
%which is real-valued within the domain of integration.
%; imaginary contributions are neglected, which exclude evanescent waves as only measurable in close proximity of source and scatterers.
%
%When $h(\vect{{\sf r}},\vect{{\sf s}})$ is stationary and Gaussian, $H^{++}(\vect{k},\vect{\kappa})$ must be an independent complex Gaussian process {\color{blue}within $\|\vect{k}\|\le\kappa$ and $\|\vect{\kappa}\|\le\kappa$, which exclude evanescent waves
%with 
%%independent entries
%%[REPLACE "ENTRIES", WHICH SUGGESTS A DISCRETE PROCESS, AND PROVIDE A REFERENCE] 
%an impulsive autocorrelation
%\cite[Thm.~2]{PizzoIT21}.
%Then, \eqref{spectrum_z}} can be replaced by its statistical equivalent
%{\color{blue}
%\begin{align}   \label{spectrum}
%%h(\vect{r},\vect{s}) & = \iint_{\|\vect{k}\|\le\kappa} \frac{d\vect{k}}{\sqrt{2\pi \kappa}} \iint_{\|\vect{\kappa}\|\le\kappa} \frac{d\vect{\kappa}}{\sqrt{2\pi \kappa}}  \\& \hspace{2cm}  \label{spectral_representation}e^{\imagunit \vect{k}^{\Ttran} \vect{r}} \frac{H^{++}(\vect{k},\vect{\kappa})}{\sqrt{\gamma(\vect{k}) \gamma(\vect{\kappa})}} e^{-\imagunit \vect{\kappa}^{\Ttran} \vect{s}},
%H(\vect{k},\vect{\kappa}) = \frac{\mathbbm{1}_{\|\vect{k}\|\le\kappa}(\vect{k})}{\sqrt{\kappa/2\pi}} \frac{H^{++}(\vect{k},\vect{\kappa})}{\sqrt{\gamma(\vect{k}) \gamma(\vect{\kappa})}} \frac{\mathbbm{1}_{\|\vect{\kappa}\|\le\kappa}(\vect{\kappa})}{\sqrt{\kappa/2\pi}}, 
%\end{align}}
%obtained by setting to zero the local $z$-dimensions at source and receiver
%given $\phi(\vect{r},\vect{k}) =  e^{\imagunit \vect{k}^{\Ttran} \vect{r}}$, 
%as $z$-translations are immaterial {\color{blue}for every fixed $r_z$ and $s_z$. Here, the limiting support has been converted into a functional dependence through indicator functions.}
%because
%\begin{equation}
%H^{++}(\vect{k},\vect{\kappa}) \sim e^{\imagunit \gamma(\vect{k}) r_z} H^{++}(\vect{k},\vect{\kappa}) \, e^{-\imagunit \gamma(\vect{\kappa}) s_z}.
%\end{equation}
%
%in agreement with the electromagnetic Huygen's principle \cite{ChewBook}.
%[GOOD POINT, BUT DO WE NEED TO DEFINE $\phi$? ESPECIALLY SINCE OTHER SIMILAR DISCRETE FUNCTIONS ARE DEFINED BELOW...]
%This is in agreement with the physics Huygen's source and the concept of field migration.
%For the sake of specificity, 
We adhere to a separable model satisfying %the angular power spectrum is
\begin{equation} \label{separable_corr}
\Ex\{|\tilde{H}^{++}(\vect{k},\vect{\kappa})|^2\} = \Ex\{|\tilde{H}^{+}_\text{r}(\vect{k})|^2\}
 \, \Ex\{|\tilde{H}^{+}_\text{t}(\vect{\kappa})|^2\},
\end{equation}
with $\Ex\{|\tilde{H}^{+}_\text{r}(\vect{k})|^2\}$ and $\Ex\{|\tilde{H}^{+}_\text{t}(\vect{\kappa})|^2\}$ the power angle spectra due to the separate scattering at receiver and transmitter, respectively.
%for any $\Ex\{|H^{+}_\text{r}(\vect{k})|^2\}$ and $\Ex\{|H^{+}_\text{t}(\vect{\kappa})|^2\}$.
%which is motivated by coupling being only noticeable on the transmit side for the multiport setup of Fig.~\ref{fig:multiport}, as  current does not flow into receive antennas. \angel{I found this confusing. You're making a coupling argument to justify an assumption on the channel. Also, in this section there isn't coupling.}
%{\color{blue}Similarly to \eqref{norm_A_spectrum_angle}, the PAS at either end of the link is normalized to integrate to unity over all directions, e.g., at the transmitter, \angel{Hope readers don't get confused between the wavenumber spectrum used above, and the angular spectrum used below. We're slightly abusing the notation by using the same variable to denote functions of different arguments}
These spectra are normalized so each can be interpreted as an angular distribution. For example,
\begin{align} \label{normalization_channel}
1 = \frac{1}{4\pi} \int_0^{\pi} \int_0^{2\pi} \Ex\{|\tilde{H}^{+}_\text{t}(\theta_{\text t},\phi_{\text t})|^2\}  \sin \theta_{\text t} \, d\theta_{\text t} d\phi_{\text t},
\end{align}
at the transmitter, with the variables change according to \eqref{wavenumber_spherical}.
This normalization %is consistent with the unit-power normalization of the channel fading.
bounds the channel power, scaling with the product of the transmit and receive apertures rather than the antenna count, in the limit of antenna densification \cite{PizzoTWC21}.
%The normalization, e.g.,
%\begin{align}
% 1 = \frac{1}{2\pi}  \int_0^{2\pi} \int_0^{\pi/2}  \Ex\{|\tilde{H}^{+}_\text{t}(\theta_\text{t},\phi_\text{t})|^2\}  \sin \theta_{\text t} \, d\theta_{\text t} d\phi_{\text t}
%\end{align}
%at the transmitter, with the variables change according to \eqref{wavenumber_spherical}, ensures that
%\begin{align}
%\iint_{-\infty}^\infty d\vect{r} \iint_{-\infty}^\infty d\vect{s} \, \Ex\{|h(\vect{{ r}},\vect{{ s}})|^2\} =1.
%\end{align}

%\subsection{Physics-Based Unitary-Independent-Unitary (UIU) Model}
\subsection{Karhunen–Loève Decomposition}
%\angel{Don't we wanna remove all the material on the UIU model?}

% \begin{figure} [t!]
%\centering\vspace{-0.0cm}
%\includegraphics[width=.9999\linewidth]{UPA_rotated} 
%\caption{Communication between planar holographic MIMO arrays in the presence of scattering and coupling.}\vspace{-0cm}
%\label{fig:UPA_rotated}
%\end{figure}

The spatial limitation imposed by the array apertures implies that $h(\vect{{ r}},\vect{{ s}})$ can be approximated by a finite number of coefficients representing the channel's spatial dimensionality %{\color{blue}per second per Hertz}
at either end of the link. 
%For the sake of specificity, squared arrays of dimensions $L_{\text t}$ and $L_{\text r}$ are considered; generalization of the model to rectangular apertures is provided in \cite{PizzoTWC21}.
For the sake of specificity, rectangular apertures of unnormalized dimensions $(L_{{\text t},x},L_{{\text t},y})$ at the transmitter and $(L_{{\text r},x},L_{{\text r},y})$ at the receiver are considered.
% see Fig.~\ref{fig:UPA_rotated}.
%These apertures are given by
%\begin{align}
%R & = \left[\frac{-L_{{\text r},x}}{2}, \frac{L_{{\text r},x}}{2}\right] \times \left[\frac{-L_{{\text r},y}}{2}, \frac{L_{{\text r},y}}{2}\right] \\
%T & = \left[\frac{-L_{{\text t},x}}{2}, \frac{L_{{\text t},x}}{2}\right] \times \left[\frac{-L_{{\text t},y}}{2}, \frac{L_{{\text t},y}}{2}\right] .
%\end{align}
%whose dependance on $r_z$ and $s_z$ is omitted, as they are kept fixed.
For later convenience, the shortest dimension aligns with the $x$-axis, whereby $L_{{\text r},x} \le L_{{\text r},y}$ and $L_{{\text t},x} \le L_{{\text t},y}$. 
Define 
\begin{equation} \label{n_rt}
 {\sf n}_{\text r} = %\left\lceil\pi \frac{L_{{\text r},x} L_{{\text r},y}}{\lambda^2} \right\rceil 
 \lceil \pi \det(\vect{D}_\text{r}) \rceil
 \quad \qquad 
 {\sf n}_{\text t} = %\left\lceil\pi \frac{L_{{\text t},x} L_{{\text t},y}}{\lambda^2} \right\rceil
 \lceil \pi \det(\vect{D}_\text{t}) \rceil
\end{equation}
with $\vect{D}_\text{r} = \diag(L_{{\text r},x}, L_{{\text r},y})/\lambda$ and $\vect{D}_\text{t} = \diag(L_{{\text t},x}, L_{{\text t},y})/\lambda$.
The number of spatial dimensions or degrees of freedom (DOF) %equal 
is upper bounded by its value under isotropic scattering, given by ${\sf DOF} = \min({\sf DOF}_\text{r},{\sf DOF}_\text{t})$, with \cite{PizzoTWC21}
\begin{equation} \label{DOF}
% {\sf n}_{\text t} =  m(\Lambda_{\text t}) = \pi \left(\frac{L_{\text t}}{\lambda}\right)^{\!2} \qquad
% {\sf n}_{\text r} =  m(\Lambda_{\text r}) = \pi \left(\frac{L_{\text r}}{\lambda}\right)^{\!2}
{\sf DOF}_{\text r}  = {\sf n}_{\text r} +  o \! \left({\sf n}_{\text r}\right)  \qquad 
{\sf DOF}_{\text t} = {\sf n}_{\text t} +  o \! \left({\sf n}_{\text t}\right)  
  \end{equation}
the cardinalities of the 2D lattices
\begin{align}    \label{lattice_rx}
%\Lambda_{\text r} & = \left\{\vect{i}=(i_x,i_y) \in \Integer^2: \left(\frac{i_x}{L_{{\text r},x}/\lambda}\right)^{\! 2} + \left(\frac{i_y}{L_{{\text r},y}/\lambda}\right)^{\! 2} \le 1\right\} \\  \label{lattice_tx}
%\Lambda_{\text t} & = \left\{\vect{j}=(j_x,j_y) \in \Integer^2: \left(\frac{j_x}{L_{{\text t},x}/\lambda}\right)^{\! 2} + \left(\frac{j_y}{L_{{\text t},y}/\lambda}\right)^{\! 2} \le 1\right\} . 
\Lambda_{\text r} & = \left\{\vect{i} =(i_x,i_y) \in \Integer^2 : \|\vect{D}_\text{r}^{-1} \vect{i}\| \le 1\right\} \\   \label{lattice_tx}
\Lambda_{\text t} & = \left\{\vect{j} =(j_x,j_y) \in \Integer^2 : \|\vect{D}_\text{t}^{-1} \vect{j}\| \le 1\right\}.
\end{align} 
%at transmitter and receiver, respectively.
%Information can be exchanged angularly through a finite number of essential directions that are upper-bounded by the number of DOF, %defined as
%is nonzero only within the regions spanned by the two arrays. 
%Thus, the channel response can be replicated to form a periodic function such that \eqref{spectral_representation} is approximately [IN WHICH SENSE APPROXIMATELY?] a Fourier series expansion \cite{PizzoTWC21}.
%This replication reflects to a discretization of the channel spectrum at both sides of the link.

The MIMO channel $[\vect{H}]_{n,m} = h(\vect{{ r}}_n,\vect{{ s}}_m)$ can then be approximated by the Karhunen–Lo\`eve expansion \cite{PizzoTWC21,LozanoCorrelation}
%(semi)unitary-independent-(semi)unitary model \cite{LozanoCorrelation} %\cite{chizhik2000effect} 
\begin{align} \label{Kronecker_MIMO}
\vect{H} & \approx   
\vect{U} \tilde{\vect{H}} \vect{V}^{\Htran}
\end{align} 
where $\vect{U} \in \Complex^{N_\text{r} \times {\sf n}_\text{r}}$ and $\vect{V} \in \Complex^{N_\text{t} \times {\sf n}_\text{t}}$ are the isometry Fourier eigenvector matrices with column entries 
\begin{align} \label{u_vec}
[\vect{u}_\vect{i}]_n & =   \sqrt{\frac{L_{{\text r},x} L_{{\text r},y}}{N_\text{r}}} u_\vect{i}(\vect{r}_n)   
= \frac{1}{\sqrt{N_\text{r}}}  e^{\imagunit 2\pi (\vect{D}_\text{r}^{-1} \boldsymbol{i})^{\Ttran} \vect{r}_n/\lambda}   \\ \label{v_vec}
[\vect{v}_\vect{j}]_m & =    \sqrt{\frac{L_{{\text t},x} L_{{\text t},y}}{N_\text{t}}} v_\vect{j}(\vect{s}_m)  
  = \frac{1}{\sqrt{N_\text{t}}} e^{\imagunit 2\pi (\vect{D}_\text{t}^{-1} \vect{j})^{\Ttran} \vect{s}_m/\lambda}.
\end{align}
In turn, %$\tilde{\vect{H}}$ has independent entries distributed as 
\begin{equation} \label{equiv_channel} 
\vect{\tilde{H}} = \vect{\Lambda}^{1/2}_{\text{r}} \vect{W} \vect{\Lambda}_{\text{t}}^{1/2}
\end{equation}
where $\vect{W} \in \Complex^{{\sf n}_\text{r} \times {\sf n}_\text{t}}$ has IID standard complex Gaussian entries while $\vect{\Lambda}_{\text{r}}$ and $\vect{\Lambda}_{\text{t}}$ are diagonal with entries that depend on the %separate angular power spectra at receiver and transmitter, respectively. 
power spectra at receiver and transmitter, respectively.
%are independent coefficients whose variances depend on the separate angular power spectra coupling between transmitter and receiver, at both ends of the link.
For example, at the transmitter, $[\vect{\Lambda}_{\text{t}}]_{j,j} = N_\text{t} \sigma^2_{\vect{j}}(\tilde{H}^{+}_\text{t})$ for $j = 1, \ldots, {\sf n}_{\text{t}}$ an arbitrary ordering of the lattice points within $\Lambda_{\text t}$ in \eqref{lattice_tx} \cite{PizzoTWC21}
%\andrea{Don't know how to relate the vector index $\vect{j}$, describing points on a 2D ellipse, to the corresponding 1D index $j$, without introducing a complicated multi-index notation.}
%\begin{align}  \label{variances_channel_wavenumber}
%\sigma^2_{\vect{j}}(\tilde{H}^{+}_\text{t}) & 
% = \frac{1}{2\pi \kappa} \iint_{\|\vect{\kappa}\|\le\kappa}  \!\!\!\!\!\!
%\frac{\Ex\{|\tilde{H}^{+}_\text{t}(\vect{\kappa})|^2\}}{\gamma(\vect{\kappa})} 
% \mathbbm{1}_{\|\vect{D}_\text{t} \vect{\kappa} - \kappa \vect{j}\|_\infty\le \tfrac{\kappa}{2}} \, d\vect{\kappa}   
% \end{align} 
%with $\{\|\vect{D}_\text{t} \vect{\kappa} - \kappa \vect{j}\|_\infty\le \tfrac{\kappa}{2}\}$ describing a rectangular set centered at $(\frac{2\pi}{L_{{\text t},x}} j_x,\frac{2\pi}{L_{{\text t},y}} j_y)$ and with side lengths $(\frac{2\pi}{L_{{\text t},x}},\frac{2\pi}{L_{{\text t},y}})$.
%area $\det^{-1}(\vect{D}_\text{t})$, centered at $\vect{D}_\text{t}^{-1} \vect{j}$.
%With variable changes applied to \eqref{variances_channel_wavenumber} using \eqref{wavenumber_spherical}, we have
\begin{align}  \label{variances_channel}
\sigma^2_{\vect{j}}(\tilde{H}^{+}_\text{t}) & 
 = \frac{1}{2\pi}  \iint_{\Omega^+_{\vect{j}}}  \Ex\{|\tilde{H}^{+}_\text{t}(\theta_{\text t},\phi_{\text t})|^2\}  \sin \theta_{\text t} \, d\theta_{\text t} d\phi_{\text t}
\end{align} 
given $\Omega^+_{\vect{j}}$ the spherical surface elements covering the entire upper hemisphere, each one centered at
\begin{align} \label{midpoint}
\theta_{{\text t},\vect{j}} & = \sin^{-1} \! \left(\|\vect{D}_\text{t}^{-1} \vect{j}\| \right)  \qquad \phi_{{\text t},\vect{j}} = \tan^{-1} \! \left({j_y}/{j_x}\right).
\end{align}
Under isotropic scattering, $\Ex\{|\tilde{H}^{+}_\text{t}(\theta_{\text t},\phi_{\text t})|^2\} = 1$, and \eqref{variances_channel} reduces to the solid angle subtended by $\Omega^+_{\vect{j}}$,
 \begin{align}  \label{solid_angles}
|\Omega^+_{\vect{j}}| & 
 = \frac{1}{2\pi}  \iint_{\Omega^+_{\vect{j}}} \sin \theta_{\text t} \, d\theta_{\text t} \, d\phi_{\text t}.
\end{align}
With that, $\vect{\tilde{H}}$ specifies the angular coupling between resolvable transmit and receive directions, while the left and right multiplications by $\vect{U}$ and $\vect{V}$ change the domain from angular to spatial \cite{PizzoTWC21,Sayeed2002}.

As the apertures grow electrically large, the approximation in \eqref{Kronecker_MIMO} sharpens 
and, by virtue of Mercer's theorem, converges in the mean-square sense to \eqref{spectral_representation_z}. Concurrently, the lower-order terms in  \eqref{DOF} vanish as the number of DOF firms up.
%and collectively spanning the entire upper hemisphere, namely $\mathop{\bigcup}_{\vect{j} \in \Lambda_\text{t}} \Omega^+_{\vect{j}} = \{[0,\pi/2] \times [0, 2\pi)\}$.}

%From \eqref{Kronecker_MIMO}, absent coupling, the channel power %in the absence of coupling %of the Fourier model  % approximately 
%%\angel{in which way approximate?} given by
%satisfies
%\begin{align} 
%\Ex\{\|\vect{H}\|_{\text F}^2\} & = \tr \! \left(\Ex\{\vect{H}^{\Htran} \vect{H}\}\right) \\
%& \approx N_\text{r} \, \tr \! \left(\vect{V} \vect{{\Lambda}}_\text{t} \vect{V}^{\Htran}\right) \\&  \label{channel_power_uncoupled_norm}
%= N_\text{r} N_{\text t} \mathop{\sum}_{\vect{j} \in \Lambda_\text{t}} \sigma^2_{\vect{j}}(\tilde{{H}}^{+}_\text{t})
%= N_\text{r} N_{\text t}
%\end{align}
%%as per $\|\vect{u}_\vect{i}\|^2 = \|\vect{v}_\vect{j}\|^2 = 1$.
%as per the normalization in \eqref{normalization_channel}. 
%This normalization ensures that the power remains bounded as antenna spacing shrinks, scaling with the product of the transmit and receive apertures rather than the number of antennas \cite{PizzoTWC21}.
%{\color{blue}The power remains bounded as antenna spacings shrink, scaling with the product of the transmit and receive apertures rather than the number of antennas \cite{PizzoTWC21}.} \angel{Remove stuff below, and tweak any references to it.}
%
%%% EXPLANATION: The Fourier eigenvectors are replaced by Fourier harmonics with different normalization factor, its inverse embedded  into the corresponding eigenvalues.
%
 %\mathop{\sum}_{\vect{i} \in \Lambda_\text{r}} \mathop{\sum}_{\vect{j} \in \Lambda_\text{t}}  \sigma^2_{\vect{i}\vect{j}}(\mathcal{H}) =1
%\begin{equation} \label{normalization_channel}
%\Ex\{|h(\vect{r},\vect{s})|^2\} = \mathop{\sum}_{\vect{i} \in \Lambda_\text{r}} \sigma^2_{\text{r},\vect{i}}(\mathcal{H}) \cdot
%\mathop{\sum}_{\vect{j} \in \Lambda_\text{t}} \sigma^2_{\text{t},\vect{j}}(\mathcal{H}) =1
%\end{equation} 
%\begin{align} \label{normalization_channel}
%1 = \frac{1}{2\pi}  \iint_{\Omega^+}  \Ex\{|\tilde{H}^{+}_\text{t}(\theta_{\text t},\phi_{\text t})|^2\}  \sin \theta_{\text t} \, d\theta_{\text t} d\phi_{\text t},
%\end{align}
%ensures that $\mathop{\sum}_{\vect{j} \in \Lambda_\text{t}} \sigma^2_{\vect{j}}(\tilde{H}^{+}_\text{t}) = 1$, 
%\begin{align}
%\mathop{\sum}_{\vect{j} \in \Lambda_\text{t}} \sigma^2_{\vect{j}}(\tilde{H}^{+}_\text{t}) & =  \frac{1}{2\pi \lambda}  \int_0^{2\pi} \int_0^{\pi/2}  \Ex\{|\tilde{H}^{+}_\text{t}(\theta_{\text t},\phi_{\text t})|^2\}  \sin \theta_{\text t} \, d\theta_{\text t} d\phi_{\text t} = 1 
%\end{align}
%also implying that
%\begin{equation} 
% \mathop{\sum}_{\vect{i} \in \Lambda_\text{r}} \sigma^2_{\vect{i}}(\tilde{H}^{+}_\text{r}) =
%\mathop{\sum}_{\vect{j} \in \Lambda_\text{t}} \sigma^2_{\vect{j}}(\tilde{H}^{+}_\text{t}) = 1
%\end{equation} 
% the power increases proportionally to the product of the transmit and receive apertures, namely, 
%\begin{align} \label{channel_norm}
%& \iint_{\vect{r} \in R} \iint_{\vect{s}\in T} \Ex\{|h(\vect{{ r}},\vect{{ s}})|^2\} \, d\vect{r} d\vect{s} 
%= L_{\text{r},x} L_{\text{r},y} L_{\text{t},x} L_{\text{t},y},
%\end{align}
%as per $\|u_\vect{i}\|^2 = \|v_\vect{j}\|^2 = 1$.
% as $\min(L_{{\text r},x},L_{{\text t},x})/\lambda\to\infty$ for any given $L_{\text{r},x}, L_{\text{t},x} < \infty$.
%The expansion in \eqref{Fourier_series} sharpens with the electrical apertures and becomes exact for $\min(L_{\text r},L_{\text t})/\lambda\to\infty$.
%by the normalization as per Mercer's theorem.

%independent and distributed as ${h}_{\vect{i}\vect{j}} $. 
%\angel{We need to streamline things; the relationships below were already given in the previous section, so no need to repeat them here}
%for $n=1, \ldots, N_\text{r}$ and $m=1, \ldots, N_\text{t}$.
%We adopt the convention $\vect{r}_1 = \vect{0}$ and $\vect{s}_1 = \vect{0}$ in the local reference frames.
%with transmitter and receiver centered on their first indexed antenna.
%the channel matrix satisfies
%\begin{align} \label{channel_power_uncoupled_norm}
%\sum_{n=1}^{N_\text{r}} \sum_{m=1}^{N_\text{t}} \Ex\{ |h(\vect{{ r}}_n,\vect{{ s}}_m)|^2\} = N_\text{r} N_\text{t}
%\end{align}

%Define $N_\text{min} = \min(N_\text{r},N_\text{t})$ and ${\sf n}_\text{min} = \min({\sf n}_\text{r},{\sf n}_\text{t})$.
%the minimum numbers of antennas and spherical surface elements
%\angel{changed "spherical surfaces" to "spherical surface elements"} 
%at receiver and transmitter, respectively.
%The rank of $\vect{H}$ is dictated by \eqref{MIMO_channel} as $\rank(\vect{H}) \le {\sf n}_\text{min}$, whereby ${\sf n}_\text{r}$ and ${\sf n}_\text{t}$ in \eqref{n_rt} can be regarded as the maximum number of effective antennas in a correlated channel, achievable with isotropic scattering.
%\angel{This is fine. Casual readers may see an oxymoron when speaking of "correlated channel" under "isotropic conditions", but readers having digested the paper hitherto should not have a problem. Fingers crossed.}
 %{\color{blue}More details are provided in Section~\ref{sec:DOF_HighSNR}.}
%Thus, the channel power in the absence of coupling grows in direct proportion with the antenna counts at transmitter and receiver \cite{heath_lozano_2018}.
%By normalizing \eqref{channel_power_uncoupled_norm} it is found that $\Ex\{\|\vect{{\sf H}}\|_{\text F}^2\}/\Ex\{\|\vect{H}\|_{\text F}^2\} = \mathop{\sum}_{\vect{j} \in \Lambda_\text{t}} \sigma^2_{\vect{j}}(\tilde{{\sf H}}^{+}_\text{t})$ whereby the 
%$\Ex\{\|\vect{H}\|_{\text F}^2\} = N_\text{t} N_\text{r}$.
%$\lambda_{\vect{n}\vect{m}}(\vect{H}) \sim\CN(0, N_\text{r} N_\text{t} \sigma^2_{\vect{n}\vect{m}}(\mathcal{H}))$.
%$\lambda_{\vect{n}\vect{m}}(\vect{H}) \sim\CN(0,(N_{\text r} N_{\text t}) \sigma^2_{\vect{n}\vect{m}}(\mathcal{H}))$ with $\sigma^2_{\vect{n}\vect{m}}(\mathcal{H})$ in \eqref{variances_channel}. 
%{\color{blue}Thus, \eqref{eig_discrete_cont} relates the singular values of the random matrix $\vect{H}$ to the ones of the stochastic operator $\mathcal{H}$.}
%\andrea{maybe add a comment saying that compared to MIMO the channel norm does not scale with the number of transmit antennas but rather with the unnormalized transmit aperture. The channel power is unbounded in the first case.}
%The generalization of \eqref{channel_power_uncoupled_norm} to coupled antennas is derivable from \eqref{channel_mat_c} as
%\begin{align} 
% \Ex\{\|\vect{{\sf H}}\|_{\text F}^2\}  & \approx \frac{L_{{\text t},x} L_{{\text t},y}}{N_\text{t}} \sum_{n=1}^{N_\text{r}} \sum_{m=1}^{N_\text{t}} \Ex\{ |{\sf h}(\vect{{ r}}_n,\vect{{ s}}_m)|^2\} \\
%& = N_\text{r}  L_{{\text t},x} L_{{\text t},y} \mathop{\sum}_{\vect{j} \in \Lambda_\text{t}}  \sigma^2_{\vect{j}}(\tilde{{\sf H}}_\text{t}^+)  \\ \nonumber
%& = 
%N_\text{r} L_\text{t}^2  \frac{8 \pi}{Z_0} \iint_{\Omega^+}  \frac{\Ex\{|\tilde{H}^+_\text{t}(\theta_{\text t},\phi_{\text t})|^2\}}{|{\sf A}_\text{t}^+(\theta_{\text t},\phi_{\text t})|^2} \\& \label{channel_power_coupled_norm} \hspace{2.5cm}
%\cdot \cos\theta_{\text t} \, \sin \theta_{\text t} \, d\theta_{\text t} d\phi_{\text t}
%\end{align}
%with ${\sf A}_\text{t}^+(\cdot,\cdot)$ and $\tilde{H}^+_\text{t}(\cdot,\cdot)$ satisfying respectively the normalization conditions \eqref{norm_A_spectrum_angle} and \eqref{normalization_channel}.
%while rescaling them according to \eqref{xi_coeff}. 
%where {\color{blue}\eqref{spectrum_channel_transmitter_coupling_2} with \eqref{pippo} were used}
%%while removing the indicator function (as also limiting the channel support)
%and variables were changed according to \eqref{wavenumber_spherical}. In turn, $\Omega^+_{{\text t},\vect{j}}$ is the spherical surface element in \eqref{variances_channel}. 
%For lossless antennas, $\Ex\{\|\vect{{\sf H}}\|_{\text F}^2\} = N_\text{t} N_\text{r}$ as per the normalization.


\subsection{Model Accuracy with Finite Apertures}

\begin{figure}
\centering\vspace{-0.0cm}
\includegraphics[width=.999\linewidth]{channel_var_uncoupled} 
\caption{Normalized sorted eigenvalues of the transmit correlation matrix in an isotropic channel. UPA with uncoupled antennas spaced by $\lambda/2$ and aperture $20 \lambda$. The solid line indicates the exact channel in \eqref{corr_exact}, circles denote its Fourier approximation in \eqref{tx_corr}.}
%\angel{In the figure, change "Clarke-Jakes" to "exact" (and perhaps "isotropic" to "omnidirectional", for consistency with the body of the paper)} \angel{Question: does this plot perhaps look even better in linear scale? By the way, the "dB" is not indicated on the y axis} }\vspace{-0cm}
\label{fig:channel_var_uncoupled}
\end{figure}

%\begin{align} \label{Kronecker_MIMO}
%\vect{H} & = 
%\vect{R}^{1/2}_{\text{r}} \vect{W} \vect{R}^{\Htran/2}_{\text{t}}
%\end{align}
%where $\vect{W} \in \Complex^{{\sf n}_\text{r} \times {\sf n}_\text{t}}$ has IID standard complex Gaussian entries. In turn, $\vect{R}^{1/2}_{\text{r}}$ and $\vect{R}^{1/2}_{\text{t}}$ denote the square root of the correlation matrices at receiver and transmitter, respectively. 
%For example, $\vect{R}^{1/2}_{\text{t}} = \vect{U}_{\text{t}} \vect{\Lambda}^{1/2}_{\text{t}}$
Denote the correlation between the $(n,m)$th and $(n^\prime,m^\prime)$th channel entries by \cite{LozanoCorrelation}
\begin{align} \label{corr}
R_\vect{H}(n,m;n^\prime,m^\prime) = \Ex\{[\vect{H}]_{n,m} [\vect{H}]^*_{n^\prime,m^\prime}\}.
\end{align}
Under the separable
% \angel{earlier the term we used was "separable"} 
model, \eqref{corr} reduces to the product of the marginal correlations at each end of the link, namely $R_\vect{H}(n,m;n^\prime,m^\prime) = [\vect{R}_\text{r}]_{n,n^\prime} \, [\vect{R}_\text{t}]_{m,m^\prime}$ with $\vect{R}_\text{r}$ and $\vect{R}_\text{t}$ the receive and transmit correlations, e.g.,
\begin{align} \label{tx_corr}
\vect{R}_\text{t} & = \frac{1}{N_\text{r}} \Ex\{\vect{H}^{\Htran} \vect{H}\} \approx \vect{V} \vect{\Lambda}_{\text{t}} \vect{V}^{\Htran}
\end{align}
at the transmitter, with $\vect{V}$ and $\vect{\Lambda}_\text{t}$ from % the isometry Fourier matrix and diagonal matrix in 
\eqref{Kronecker_MIMO} and \eqref{equiv_channel}, respectively.
Holographic MIMO is characterized by \cite{PizzoTWC21}
\begin{equation} \label{Nyquist_cond}
N_{\text r} \ge {\sf n}_\text{r} \qquad \text{and} \qquad N_{\text t} \ge {\sf n}_\text{t}
\end{equation}
implying that $\vect{H}$ is rank-deficient with probability $1$, namely $\rank(\vect{H}) \le \min({\sf n}_\text{r},{\sf n}_\text{t}) \le \min(N_\text{r},N_\text{t})$, 
the number of spatial dimensions being limited by the environment and array apertures, rather than by the number of antennas as in regular MIMO \cite{heath_lozano_2018}.
Then, augmenting $\vect{V}$ with its orthogonal complement $\vect{V}^\perp \in \Complex^{N_\text{t} \times N_\text{t}-{\sf n}_\text{t}}$ gives the unitary matrix $[\vect{V},\vect{V}^\perp]$ whereby 
%The accuracy of \eqref{Kronecker_MIMO} for finite apertures 
%%\angel{We should refer to the expression, I guess (84), and perhaps throw the term "approximate" here since we're referring to the series expansion} 
%\begin{align}  \label{tx_corr_expanded}
%\vect{R}_\text{t} & 
%\approx \frac{L_{{\text t},x} L_{{\text t},y}}{N_\text{t}} \, 
%\begin{bmatrix}
%\vect{V} & \vect{V}^\perp
%\end{bmatrix}
%\begin{bmatrix}
%%\begin{matrix}
%%\sigma^2_{1}(\tilde{H}^{+}_\text{t}) & & \\
%%& \ddots & \\
%%& & \sigma^2_{{\sf n}_\text{t}}(\tilde{H}^{+}_\text{t})
%%\end{matrix}
%\vect{\Lambda}_{\text{t}} & \vect{0} \\
%\vect{0} & \vect{0}
%\end{bmatrix}
%\begin{bmatrix}
%\vect{V}^{\Htran} \\
%(\vect{V}^\perp)^{\Htran}
%\end{bmatrix}
%\end{align}
the eigenvalues of the transmit correlation read %are given by
\begin{align} \label{tx_eig_uncoupled}
\vect{\lambda}(\vect{R}_\text{t}) \approx N_{\text t} \, \big(\underbrace{\sigma^2_{1}(\tilde{H}^{+}_\text{t}), \ldots, \sigma^2_{{\sf n}_\text{t}}(\tilde{H}^{+}_\text{t})}_\text{DOF}, \underbrace{0, \ldots, 0}_{N_\text{t}-{\sf n}_\text{t}} \big).
\end{align}
%$\vect{R}_{\text{t}} = \frac{1}{N_\text{r}} \Ex\{\vect{H}^{\Htran} \vect{H}\} \approx \vect{V} \vect{\Lambda}_{\text{t}} \vect{V}^{\Htran}$
Fig.~\ref{fig:channel_var_uncoupled} depicts, at the scale of ${\sf n}_\text{t}$ in \eqref{n_rt}, the sorted eigenvalues \eqref{tx_eig_uncoupled} specialized for a squared UPA of uncoupled antennas spaced by $\lambda/2$ and aperture $L_{\text{t},x} = L_{\text{t},y} = 20 \lambda$.  
%Being the focus on coupling, \angel{But there's no coupling in this section yet...} 
The curve is gauged against the eigenvalues of the exact correlation matrix, 
%which abide by \eqref{Toeplitz_coupling}, given the isomorphism between omnidirectional coupling and isotropic correlation, namely
whose entries are obtained from \eqref{spectral_representation_z} with \eqref{separable_corr} and the variable changes in \eqref{wavenumber_spherical} as
%yielding, under the separability assumption,
\begin{align}  \notag
[\vect{R}_\text{t}]_{m,m^\prime} & = \frac{1}{2\pi}  \int_0^{\pi/2} \int_0^{2 \pi}  \Ex\{|\tilde{H}^{+}_\text{t}(\theta_{\text t},\phi_{\text t})|^2\}  \\& \label{corr_exact} \hspace{1cm}
\cdot
 a_{m}(\theta_{\text t},\phi_{\text t}) \overline{a_{m^\prime}(\theta_{\text t},\phi_{\text t})} \sin \theta_{\text t} \, d\theta_{\text t} d\phi_{\text t},
\end{align}
where $a_{m}(\cdot,\cdot)$ is the array response in \eqref{Gavi}.
Isotropic scattering is considered, implying 
%$\Ex\{|\tilde{H}^{+}_\text{t}(\theta_{\text t},\phi_{\text t})|^2\} = 1$.
that \eqref{corr_exact} specializes to $[\vect{R}_\text{t}]_{m,m^\prime} = \sinc(2 \|\vect{s}_m-\vect{s}_{m^\prime}\|/\lambda)$ and the variance in \eqref{tx_eig_uncoupled} reduces to the solid angle in \eqref{solid_angles}.
%{\color{blue} and \eqref{variances_channel} reduces} to the solid angle subtended by the spherical surface element $\Omega^+_{\vect{j}}$,
% \begin{align}  \label{solid_angles}
%|\Omega^+_{\vect{j}}| & 
% = \frac{1}{2\pi}  \iint_{\Omega^+_{\vect{j}}} \sin \theta_{\text t} \, d\theta_{\text t} \, d\phi_{\text t} .
%\end{align}
%whereby the transmit eigenvalues exhibit an uneven distribution.
%For isotropic scattering, \eqref{corr_exact} specializes to $[\vect{R}_\text{t}]_{m,m^\prime} = \sinc(2 \|\vect{s}_m-\vect{s}_{m^\prime}\|/\lambda)$.
The eigenvalues exhibit a polarization at ${\sf n}_\text{t}$, justifying a low-rank representation via Fourier expansion, 
%This effect is revealed by plotting the eigenvalues in linear scale and normalizing them such that their sum equals the transmit DOF in \eqref{n_rt} \cite{Franceschetti}.
 with lower-order dependencies of the rank subsumed by the term $o({\sf n}_\text{t})$ in \eqref{DOF}.


\section{Fourier Model with Coupling at the Transmitter} \label{sec:holo_MIMO_coupled}


%Owing to \eqref{conv_ct_3D}, ${\sf c}_{\text t}(\vect{{\sf v}})$ specifies the cascade of two identical linear filtering operations, each with impulse response ${\sf c}^{1/2}_{\text{t}}(\vect{{\sf v}})$. This would map, in the wavenumber domain, to ${\sf C}^{1/2}_\text{t}(\vect{{\sf k}}) =  \sqrt{{\sf C}_{\text t}(\vect{{\sf k}})}$, thus revealing the hidden linear space-invariant nature of coupling.
%From \eqref{psd_Dirac_LSI_noniso_filter},  the output power spectral density is
%\begin{align} \label{psd_coupling_filter}
%{\sf S}_{\text t}(\vect{{\sf k}}) &  
%= {\sf R} \, |J_{\text t}(\vect{{\sf k}})|^2 \left({\sf C}^{1/2}_\text{t}(\vect{{\sf k}})\right)^{\!2} .
%\end{align}
%whereby the output power spectral density is obtainable multiplying the power spectral density of the input with the  spectral response squared of the coupling filter.
%However, ${\sf C}_{\text t}(\vect{{\sf k}})$ is an impulsive function as per \eqref{spectrum_impedance}, making its square root undefined in $L^2$.

Electromagnetic propagation in 3D is representable in a %lower-dimensional
2D form \cite{Franceschetti}.
%\angel{Not sure what this next "however" means to signify, especially in the heels of another "however"}
%However, taking the 2D Fourier transform of ${\sf c}_{\text t}(\vect{{\sf v}})$ in \eqref{real_impedance_kernel_antenna}, with $v_z$ kept fixed,
This inherent lower dimensionality is revealed by a 2D Fourier transform of ${\sf c}_{\text t}(\vect{{\sf v}})$ in \eqref{real_impedance_kernel_antenna}, with $v_z$ kept fixed,
\begin{align}  
{\sf C}_{\text t}(\vect{\kappa},v_z) &  =  \iint_{-\infty}^\infty  {\sf c}_{\text t}(\vect{v},v_z) \, e^{-\imagunit \vect{\kappa}^{\Ttran} \vect{v} } d\vect{v} \\&  \label{real_impedance_kernel_antenna_spectrum}
= \frac{1}{2\pi \kappa} \mathbbm{1}_{\|\vect{\kappa}\|\le \kappa} \frac{|{\sf A}_{\text t}^+(\vect{\kappa})|^2}{\gamma(\vect{\kappa})} \, e^{\imagunit \gamma v_z},
\end{align}
which captures the spectral behavior of coupling between $z$-planes separated by $v_z\ge0$.
This spectrum evolves along the arbitrarily chosen $z$-axis as ${\sf C}_{\text t}(\vect{\kappa},v_z) = {\sf C}_{\text t}(\vect{\kappa},0) \, e^{\imagunit \gamma v_z}$, in the half-space $v_z \ge 0$.
The limitation to $\|\vect{\kappa}\|\le \kappa$ ensures that $\gamma \in \Real$, making the above relation equivalent to a 
%translation of the spectrum. Then, the problem can be described using a 2D spectrum combined with a translation along $z$.
 2D spectrum representation with translation along $z$.
%the above relation represents a simple translation of the spectrum. Consequently, the problem can be adequately described using a 2D spectrum combined with a translation along z.
%\angel{I guess this "thus" is intended to seal that, because a 2D spectrum + translation suffices to represent things, we can limit ourselves to 2D apertures It may be too cryptic, perhaps we need an extra sentence here.} 
While a 3D formulation %has proven valuable in 
%establishing the isomorphism between 
%{\color{blue}linking coupling and correlation,} 
 revealed the isomorphism between coupling and correlation in \eqref{spectrum_impedance_isotropic} and \eqref{spectrum_impedance}, 
%\angel{clarify that this refers to earlier in the paper, or remove it} 
the framework is henceforth specialized to planar apertures, whereby ${\sf C}_{\text t}(\vect{\kappa}) = {\sf C}_{\text t}(\vect{\kappa},0)$. (Volumetric apertures can be accommodated by superimposing contributions from current densities across different $z$-planes, leveraging the linearity of wave propagation.)
%Thus, while a 3D formulation has proven useful in establishing the isomorphism between coupling and correlation, the framework is henceforth specialized to planar apertures. (Volumetric apertures can always be accommodated by superimposing the contributions from current densities at different $z$-planes, as per the linearity of propagation \cite{ChewBook}.)


\subsection{Coupling as a Linear Space-Invariant Filter}

Replacing $h(\vect{{ r}},\vect{{ s}})$ with its spectral representation in \eqref{spectral_representation_z}, with the dependences on $r_z$ and $s_z$ omitted, while building on the space invariance of ${\sf c}^{-1/2}_{\text{t}}(\vect{{ r}}-\vect{{ s}})$, the composite channel ${\sf h}(\vect{{ r}},\vect{{ s}})$ in \eqref{composite_channel_op} between continuous apertures is obtained analogously to \eqref{spectral_representation_z}, but with a coupling-inclusive spectrum that, scaled by $\sqrt{2/{\sf R}}$, equals
%\begin{align}   \label{spectrum_composite}
%%h(\vect{r},\vect{s}) & = \iint_{\|\vect{k}\|\le\kappa} \frac{d\vect{k}}{\sqrt{2\pi \kappa}} \iint_{\|\vect{\kappa}\|\le\kappa} \frac{d\vect{\kappa}}{\sqrt{2\pi \kappa}}  \\& \hspace{2cm}  \label{spectral_representation}e^{\imagunit \vect{k}^{\Ttran} \vect{r}} \frac{H^{++}(\vect{k},\vect{\kappa})}{\sqrt{\gamma(\vect{k}) \gamma(\vect{\kappa})}} e^{-\imagunit \vect{\kappa}^{\Ttran} \vect{s}},
%{\sf H}(\vect{k},\vect{\kappa}) = \frac{\mathbbm{1}_{\|\vect{k}\|\le\kappa}(\vect{k})}{\sqrt{\kappa}} \frac{{\sf H}^{++}(\vect{k},\vect{\kappa})}{\sqrt{\gamma(\vect{k}) \gamma(\vect{\kappa})}} \frac{\mathbbm{1}_{\|\vect{\kappa}\|\le\kappa}(\vect{\kappa})}{\sqrt{\kappa}}, 
%\end{align}
%where
\begin{align}  \label{spectrum_HC}
\tilde{{\sf H}}^{++}(\vect{k},\vect{\kappa}) & =  
 \tilde{H}^{++}(\vect{k},\vect{\kappa}) \,
 {\sf C}^{-1/2}_{\text{t}}(\vect{\kappa})
% = \sqrt{\frac{4\pi \kappa \gamma}{R}} \frac{\tilde{H}^{++}(\vect{k},\vect{\kappa}) }{|{\sf A}_{\text t}^+(\vect{\kappa})|}  \frac{1}{\mathbbm{1}_{\|\vect{k}\|\le\kappa}(\vect{k})}
  \end{align}
%after substituting the spectrum in \eqref{C_square_root_inv} related to the upper hemisphere. 
 with 
%\begin{align} \label{coupling_spectrum_inverse}
%{\sf C}^{-1/2}_{\text{t}}(\vect{\kappa}) = \iint_{-\infty}^\infty  {\sf c}^{-1/2}_{\text t}(\vect{v},0) \, e^{-\imagunit \vect{\kappa}^{\Ttran} \vect{v} } d\vect{v},
%\end{align}
%where the dependence of the left-hand side on $v_z$ was omitted.
\begin{align}   \label{C_square_root_inv}
{\sf C}^{-1/2}_{\text{t}}(\vect{k}) & = \frac{1}{\sqrt{{\sf C}_{\text{t}}(\vect{k})}} = 
\frac{\sqrt{\gamma(\vect{\kappa})}}{|{\sf A}_{\text t}^+(\vect{\kappa})|} \frac{\sqrt{2\pi \kappa}}{\mathbbm{1}_{\|\vect{k}\|\le\kappa}},
\end{align}
obtainable specializing \eqref{conv_ct_3D} to 2D apertures and substituting \eqref{real_impedance_kernel_antenna_spectrum}.
The invertibility of $\mathcal{C}^{1/2}$, ensuring the existence of a deconvolution $\mathcal{C}^{-1/2}$, requires the magnitude of the antenna pattern $|{\sf A}_{\text{t}}^+(\vect{\kappa})|$ to be strictly positive almost everywhere (see Appendix~E).
The stability of \eqref{C_square_root_inv} is further guaranteed for any bounded antenna pattern \cite{Unser1994}.
%given by the spectral product with the coupling filter.}
%\begin{align} \notag  
%{\sf h}(\vect{r},\vect{s}) & = \iint_{\|\vect{k}\|\le\kappa} \frac{d\vect{k}}{\sqrt{2\pi \kappa}} \iint_{\|\vect{\kappa}\|\le\kappa} \frac{d\vect{\kappa}}{\sqrt{\pi} \kappa} \\& \hspace{2cm}  \label{spectral_representation_composite}
%e^{\imagunit \vect{k}^{\Ttran} \vect{r}} \frac{{\sf H}^{++}(\vect{k},\vect{\kappa})}{\sqrt{\gamma(\vect{k}) \gamma(\vect{\kappa})}} e^{-\imagunit \vect{\kappa}^{\Ttran} \vect{s}}.
%\end{align}
%of the composite channel is  {\color{blue}the uncoupled spectrum in \eqref{spectrum}} and  \eqref{C_square_root_inv_final}, yielding
%\begin{align}  \label{spectrum_HC}
%{\sf H}^{++}(\vect{k},\vect{\kappa}) & =  H^{++}(\vect{k},\vect{\kappa})  C^{-1/2}_{\text{t}}(\vect{\kappa}). %\\ \label{spectrum_HC}
%%& =  H^{++}(\vect{k},\vect{\kappa}) {\color{blue} \frac{\sqrt{\gamma}}{|A_{\text t}^+(\vect{\kappa})|}}.
%\end{align}

In light of \eqref{spectrum_HC}, the sole effect of coupling is to alter the spectrum of the channel via a space-invariant filter---akin to the separable correlation \cite{chizhik2000effect}.
From \eqref{spectrum_HC}, %factoring out $ \sqrt{{2}/{R}}$, 
%the transmit angular power spectrum of a separable channel is
%\begin{equation} \label{separable_corr_coupling}
%\Ex\{|{\sf H}^{++}(\vect{k},\vect{\kappa})|^2\} = \Ex\{|H^{+}_\text{r}(\vect{k})|^2\}
% \, \Ex\{|{\sf H}^{+}_\text{t}(\vect{\kappa})|^2\}
%\end{equation}
%with
\begin{align} \label{spectrum_channel_transmitter_coupling}
\Ex \! \left \{ |\tilde{{\sf H}}^{+}_\text{t}(\vect{\kappa})|^2 \right \} 
%& =  \frac{2}{R} \, \Ex  \big \{|\tilde{H}^{+}_\text{t}(\vect{\kappa})|^2 \big \}\, |{\sf C}^{-1/2}_{\text t}(\vect{\kappa})|^2 \\& 
= 2 \pi \kappa \gamma(\vect{\kappa}) \, \frac{\Ex  \big \{|\tilde{H}^{+}_\text{t}(\vect{\kappa})|^2 \big \}}{|{\sf A}_{\text t}^+(\vect{\kappa})|^2}  \mathbbm{1}_{\|\vect{k}\|\le\kappa},
\end{align}
where the channel spectrum's support in \eqref{spectral_representation_z} was made explicit through an indicator function.
%For $\|\vect{\kappa}\|>\kappa$, corresponding to the evanescent portion of the spectrum, \eqref{spectrum_channel_transmitter_coupling} takes an undetermined form. 
%Regarding the channel, evanescent waves are excluded %in \eqref{spectral_representation_z}
%as they contrast with the initial stationarity assumption, which underlies Karhunen-Loève expansions of spatial random processes \cite{PizzoIT21};
%however, these waves may be nonnegligible  \cite{PlaneWaveBook}.
%Regarding coupling, the spatial bandlimitedness of the antenna pattern is due to evanescent waves not carrying real power, an aspect that will be revisited in Section~\ref{sec:tx_eig}.
This support excludes evanescent waves, which would violate the stationarity assumption in \eqref{spectral_representation_z} that underlies the Karhunen-Loève expansions, and carry no real power, as evidenced by \eqref{time_avg_power_final}.
%\angel{Can we specify where?}
%where the indicator function was omitted as %also appearing in \eqref{spectral_representation}, thereby limiting the channel support.
%{\color{blue}already limiting the channel support in \eqref{spectral_representation}.}
%\angel{I see no indicator function in \eqref{spectral_representation}}
%as only measurable in close proximity of source and scatterers

\subsection{Coupling versus Correlation} \label{sec:coupling_vs_corr}

Despite fading correlation and coupling having an isomorphic linear filtering nature, their effects appear respectively in the numerator and denominator of \eqref{spectrum_channel_transmitter_coupling}. 
%\angel{Counterintuitive, is it not?}
%\andrea{I guess an answer is that coupling applies to the current density rather than the channel itself, as for the scattering (see \eqref{psd_coupling_filter}). Thus, we cannot compare them until we translate the concept of coupling to the channel, which is done via deconvolution.}
Correlation is associated with scattering and maps directly onto the channel, while coupling arises from the interactions among currents flowing into different antennas. 
Putting them on an equal footing requires translating coupling to the channel %which is done
through the reciprocal %inversion 
in \eqref{C_square_root_inv}. 
%\angel{This statement is hidden away here, but seems important as the justification for this decomposition and whitening of the coupling that you've been introducing.}
This entails a deconvolution, as opposed to the convolution applied by scattering; while their functional dependence is isomorphic, therefore,
coupling effects run counter to those of scattering.
%This distinctive feature  of coupling allows for a clear a contrast to be made between coupling and correlation,  highlighting their remarkably different impacts on multiantenna channels. 

%\angel{Not sure that "comparison" is the right term here, I guess you're referring to putting them on an equal footing? And perhaps you should be more explicit about the deconvolution, this whole relation between correlation and coupling is one of the insights of the paper and we should take any chance to build it up}

%Thus, \eqref{Fourier_series_correlation} offers another embodiment of the time-domain result in \cite[Eq.~207]{VanTreesBook} that applies in the spatial domain to stationary random fields.


%\angel{Here we should try to leverage as much as possible the earlier UIU subsection, just incorporating the coupling.}
%\angel{This title, and the text below, seem to suggest that directionality is now introduced for the first time, but it's been in the formulation already (even if not in the examples and plots)}
%The omnidirectional coupling in \eqref{real_impedance_kernel_spherical} describes the mutual interactions between antennas with spherical symmetry. However, antennas need not be omnidirectional leading to a spatially selective coupling that is specified exactly by its Fourier plane-wave representation in \eqref{real_impedance_kernel_antenna}.

%{\color{blue}The spectrum of the composite channel is obtained analogously to \eqref{spectrum}, but parametrized by the coupling-inclusive angular spectrum in \eqref{spectrum_HC},} under proper normalization (see App.~\ref{app:normalization}), reading
%{\color{blue}
%\begin{align}   \label{spectrum_composite}
%{\sf H}(\vect{k},\vect{\kappa}) = \frac{\mathbbm{1}_{\|\vect{k}\|\le\kappa}(\vect{k})}{\sqrt{\kappa/2\pi}} \frac{{\sf H}^{++}(\vect{k},\vect{\kappa})}{\sqrt{\gamma(\vect{k}) \gamma(\vect{\kappa})}} \frac{\mathbbm{1}_{\|\vect{\kappa}\|\le\kappa}(\vect{\kappa})}{\kappa/(2\sqrt{\pi})}.
%\end{align}}
%\begin{align} \notag  
%{\sf h}(\vect{r},\vect{s}) & = \iint_{\|\vect{k}\|\le\kappa} \frac{d\vect{k}}{\sqrt{2\pi \kappa}} \iint_{\|\vect{\kappa}\|\le\kappa} \frac{d\vect{\kappa}}{\sqrt{\pi} \kappa} \\& \hspace{2cm}  \label{spectral_representation_composite}
%e^{\imagunit \vect{k}^{\Ttran} \vect{r}} \frac{{\sf H}^{++}(\vect{k},\vect{\kappa})}{\sqrt{\gamma(\vect{k}) \gamma(\vect{\kappa})}} e^{-\imagunit \vect{\kappa}^{\Ttran} \vect{s}}.
%\end{align}
%In turn, the channel within $\|\vect{r}\|_\infty \le L_{\text r}$ and $\|\vect{s}\|_\infty \le L_{\text t}$ can then be approximated by the Fourier-Stieltjes series expansion
%\begin{align} \label{Fourier_series_composite}
%h_\mathcal{C}(\vect{r},\vect{s}) & = \mathop{\sum}_{\vect{n} \in \Lambda_\text{r}}
%\mathop{\sum}_{\vect{m} \in \Lambda_\text{t}}
%\lambda_{\vect{n}\vect{m}}(\mathcal{H}_\mathcal{C}) \, u_\vect{n}(\vect{r}) v^*_\vect{m}(\vect{s}),
%\end{align}
%with $u_\vect{n}(\vect{r})$ and $v_\vect{m}(\vect{s})$ respectively the 2D Fourier basis functions at transmitter.

 
\subsection{Composite MIMO Channel Matrix}

The separable MIMO channel, including %\angel{you mean "inclusive"?} 
 transmit coupling, admits the approximate Karhunen–Lo\`eve expansion 
%Accordingly, for a separable channel, the entries of the MIMO channel between coupled antennas admit the asymptotic Karhunen–Lo\`eve decomposition
%\begin{align} \label{MIMO_channel_coupling}
%[\vect{{\sf H}}]_{n,m} & \approx \mathop{\sum}_{\vect{i} \in \Lambda_\text{r}} \mathop{\sum}_{\vect{j} \in \Lambda_\text{t}}  \tilde{{\sf h}}_{\vect{i}\vect{j}} \, [\vect{u}_\vect{i}]_n [\vect{v}_\vect{j}]_m^*,
%\end{align}
%where $\vect{u}_\vect{i}$ and $\vect{v}_\vect{j}$ are given by \eqref{u_vec} and \eqref{v_vec}.
%For a separable power spectrum, %though this need not be the case, 
%\begin{equation} \label{xi_coeff}
%\tilde{{\sf h}}_{\vect{i}\vect{j}} \sim\CN\left(0,N_{\text r} N_{\text t} \,  \sigma^2_{\vect{i}}({\sf H}_\text{r}^+) \sigma^2_{\vect{j}}({\sf H}_\text{t}^+)\right)
%\end{equation}
%are independent coefficients whose variances are given by, e.g., at the transmitter,
%%fraction of power transmitted by the source onto a solid angle $\mathbb{W}_{\boldsymbol{m}}$ and measured at the receiver over another solid angle $\mathbb{W}_{\boldsymbol{\ell}}$,
%%the angular power coupling between source and receiver. While their sum is normalized to one, the cardinality yields the number of angular sectors over which transmission can take place, namely the DOF.
%%namely,
\begin{align} \label{MIMO_channel_coupling}
\vect{{\sf H}} & \approx   
 \sqrt{\frac{2}{{\sf R}}} \, \vect{U} \vect{\tilde{{\sf H}}} \vect{V}^{\Htran}
\end{align}
where $\vect{U}$ and $\vect{V}$ are defined in \eqref{Kronecker_MIMO}. In turn, 
%$\tilde{\vect{{\sf H}}}$ has independent entries $\tilde{h}_{\vect{i}\vect{j}} \sim\CN \big(0,N_{\text r} N_{\text t} \, \sigma^2_{\vect{i}}(\tilde{H}^{+}_\text{r}) \sigma^2_{\vect{j}}(\tilde{{\sf H}}^{+}_\text{t})\big)$, namely
\begin{equation} \label{equiv_channel_coupling} 
\vect{\tilde{{\sf H}}} = \vect{\Lambda}^{1/2}_{\text{r}} \vect{W} \vect{{\sf \Lambda}}_{\text{t}}^{1/2}
\end{equation}
where $\vect{{\sf \Lambda}}_{\text{t}}$ is diagonal with entries $\{N_\text{t} \sigma^2_{j}(\tilde{{\sf H}}_\text{t}^+)\}$.
%\angel{Doesn't $\vect{{\sf \Lambda}}_{\text{t}} $ equal $ \vect{{\Lambda}}_{\text{t}}^{1/2}  \vect{V}^{\Htran}   \vect{{\sf C}}^{-1/2}   \vect{V}$?} \angel{I just tried to reconcile (48), (65), and (68), with (82)-(83). That's what I got.} 
%\andrea{I don't see that. The spectral representation for the coupled channel is the same as for the uncoupled channel but with coupling-inclusive spectrum. It is consistent with the spatial convolution in \eqref{channel_samples_coupled} translating to a spectrum multiplication in \eqref{spectrum_HC}.}
%\andrea{In order to get that you would need $\vect{V} \vect{V}^{\Htran} = \vect{I}$, which is not the case as $\vect{V}$ is semiunitary.}
%\angel{Right, my bad. I do see that through the convolution, but just going thru the various matrices it's far from obvious to see that $\vect{{\sf \Lambda}}_{\text{t}}$ would be diagonal. Oh well.}
These variances are obtained similarly to \eqref{variances_channel}, but with the coupling-inclusive spectrum, yielding
%\begin{align} \label{variances_channel_composite_wave_0}
%\sigma^2_{\vect{j}}(\tilde{{\sf H}}_\text{t}^+)   
%& = \frac{1}{2\pi\kappa} \iint_{\|\vect{\kappa}\|\le\kappa}  \!\!\!\!\!\!
%\frac{\Ex\{|\tilde{{\sf H}}^{+}_\text{t}(\vect{\kappa})|^2\}}{\gamma(\vect{\kappa})}  \mathbbm{1}_{\|\vect{D}_\text{t} \vect{\kappa} - \kappa \vect{j}\|_\infty\le \tfrac{\kappa}{2}} \, d\vect{\kappa} 
%  \\& \label{variances_channel_composite_wave}
% =   \iint_{\|\vect{\kappa}\|\le\kappa} \!\!
%  \frac{\Ex\{|\tilde{H}^{+}_\text{t}(\vect{\kappa})|^2\}}{|{\sf A}_{\text t}^+(\vect{\kappa})|^2}  
%  \mathbbm{1}_{\|\vect{D}_\text{t} \vect{\kappa} - \kappa \vect{j}\|_\infty\le \tfrac{\kappa}{2}} \, d\vect{\kappa} \\& \label{variances_channel_composite}
%  = {\color{blue}\kappa^2 }
% \iint_{\Omega^+_{\vect{j}}}  \frac{\Ex\{|\tilde{H}^+_\text{t}(\theta_{\text t},\phi_{\text t})|^2\}}{|{\sf A}_\text{t}^+(\theta_{\text t},\phi_{\text t})|^2} \,\cos\theta_{\text t} \, \sin \theta_{\text t} \, d\theta_{\text t} d\phi_{\text t}
%\end{align}
%with the variable changes in  \eqref{wavenumber_spherical}. % and using ${\sf R}=\kappa^2 Z_0/4 \pi$.
\begin{align} \label{variances_channel_composite}
\sigma^2_{\vect{j}}(\tilde{{\sf H}}_\text{t}^+)   
%& = \frac{1}{2\pi\kappa} \iint_{\|\vect{\kappa}\|\le\kappa}  \!\!\!\!\!\!
%\frac{\Ex\{|\tilde{{\sf H}}^{+}_\text{t}(\vect{\kappa})|^2\}}{\gamma(\vect{\kappa})}  \mathbbm{1}_{\|\vect{D}_\text{t} \vect{\kappa} - \kappa \vect{j}\|_\infty\le \tfrac{\kappa}{2}} \, d\vect{\kappa} 
%  \\& \label{variances_channel_composite_wave}
% =   \iint_{\|\vect{\kappa}\|\le\kappa} \!\!
%  \frac{\Ex\{|\tilde{H}^{+}_\text{t}(\vect{\kappa})|^2\}}{|{\sf A}_{\text t}^+(\vect{\kappa})|^2}  
%  \mathbbm{1}_{\|\vect{D}_\text{t} \vect{\kappa} - \kappa \vect{j}\|_\infty\le \tfrac{\kappa}{2}} \, d\vect{\kappa} \\& 
= \iint_{\Omega^+_{\vect{j}}}  \frac{\Ex\{|\tilde{H}^+_\text{t}(\theta_{\text t},\phi_{\text t})|^2\}}{|{\sf A}_\text{t}^+(\theta_{\text t},\phi_{\text t})|^2} \,\cos\theta_{\text t} \, \sin \theta_{\text t} \, d\theta_{\text t} d\phi_{\text t}
\end{align}
with the variable changes in  \eqref{wavenumber_spherical}.
%where $R=\kappa^2 Z_0/4 \pi$ was used and variables were changed according to \eqref{wavenumber_spherical}.
%Note that the transmit variances are independent of $\lambda$; the dependence of the numerator of \eqref{variances_channel_composite} on $\kappa^2$, as per the variables change, cancels out with the factor $R$ appearing at the denominator. \angel{Couldn't follow this last sentence. Which $\lambda$ is this? Also the cancellation business I couldn't follow...}
%where \eqref{spectrum_channel_transmitter_coupling} and \eqref{pippo} were used, and 
%\angel{I fear that by now readers will have forgotten what $\boldsymbol{D}_t$ is. In fact, I wonder as to whether a better way could be found of normalizing by the aperture size.}
%\andrea{Interesting comment. I wonder if normalizing the axes would do more harm than good. I'd save it for the very end.}
Comparing \eqref{variances_channel_composite} against the variances of the uncoupled model in \eqref{variances_channel}, the antenna  power pattern appears at the denominator of the composite angular spectrum and a projection arises w.r.t. the broadside direction, orthogonal to the array plane. 
%\angel{How does this projection (the cosine) emanate from the coupling?} 
This projection emanates from coupling through the deconvolution applied by \eqref{C_square_root_inv} and incorporates the impact of array orientation for a given scattering.
For example, specialized to isotropic scattering and omnidirectional antennas, \eqref{variances_channel_composite} becomes (up to a factor) the projected solid angle subtended by $\Omega^+_{\vect{j}}$ in \eqref{variances_channel}, namely
\begin{equation} \label{solid_angles_coupling}
|{\sf \Omega}^+_{\vect{j}}| = \frac{1}{\pi} \iint_{\Omega^+_{\vect{j}}}  \cos\theta_{\text t} \, \sin \theta_{\text t} \, d\theta_{\text t} d\phi_{\text t}.
\end{equation}
Coupling introduces a broadside projection, absent in the uncoupled formulation---recall \eqref{solid_angles}. It incorporates the impact of array orientation, skewing the solid angles subtended by the scattering towards the broadside direction.

% which cancels out the dependence of the denominator of \eqref{variances_channel_composite_wave_0} on $\gamma$---a dependence that originates from the structure of the uncoupled channel in \eqref{spectral_representation_z}. \angel{Uf. Hard to follow... Perhaps we focus less on the math and more on the physical interpretation of this projection?}
%\andrea{Perhaps a picture would do}

%The channel power reads
%\begin{align} 
%\Ex\{\|\vect{{\sf H}}\|_{\text F}^2\} & = \tr\left(\Ex\{\vect{{\sf H}}^{\Htran} \vect{{\sf H}}\}\right) \\ &
%%& \approx N_\text{r} \, \tr\left(\vect{V} \vect{{\sf \Lambda}}_\text{t} \vect{V}^{\Htran}\right) \\& 
%\label{channel_power_coupled_norm}
%\approx {\color{blue}\frac{2}{{\sf R}}} N_\text{r} N_{\text t} \mathop{\sum}_{\vect{j} \in \Lambda_\text{t}} \sigma^2_{\vect{j}}(\tilde{{\sf H}}^{+}_\text{t}),
%\end{align}
%which differs from the uncoupled formula \eqref{channel_power_uncoupled_norm} due to the term $\sum_{\vect{j} \in \Lambda_\text{t}} \sigma^2_{\vect{j}}(\tilde{{\sf H}}^{+}_\text{t})$, where scattering and coupling appear jointly.
%Note that despite the angular power density of the channel and antenna are individually normalized as per \eqref{normalization_channel} and \eqref{norm_A_spectrum_angle}, they appear jointly in \eqref{variances_channel_composite}. \angel{Didn't get this point, what do the normalizations have to do with these quantities appearing jointly?}
%\angel{This paragraph below is relevant, but needs a thorough revision. In particular, the second sentence seems to contradict the first one.}
%as $N_\text{t}\to\infty$ within a given footprint, the discrete eigenvalues approach the continuous' \angel{continuous'?} in \eqref{eig_discrete_cont} whereby the channel power scales with the area of the transmit and receive arrays \angel{Ok, the formula above only involved the transmit spectrum}, rather than the antenna product, which is always bounded above \cite{HeedongIRS}. \angel{Not clear what "which is always bounded above" refers to}


\subsection{Impact of Coupling on the Transmit Eigenvalues}  \label{sec:tx_eig}

\begin{figure}[t]
     \centering
     \includegraphics[width=.999\columnwidth]{channel_var_coupled_iso}
         \caption{Normalized sorted eigenvalues of the transmit correlation matrix under isotropic scattering. UPA with omnidirectional antennas ($\rho = 0.01$) spaced by $\lambda/2$ and aperture $20 \lambda$. The solid line is for the exact channel in \eqref{numerical_inv}, circles for its Fourier approximation in \eqref{tx_corr_coupled}.
         %Matching the {\color{magenta}power pattern} of each antenna to the fading spectrum} causes decorrelation.
         }
     \label{fig:channel_var_coupled_iso}
     \end{figure}
     
%Define $N_\text{min} = \min(N_\text{r},N_\text{t})$ and ${\sf n}_\text{min} = \min({\sf n}_\text{r},{\sf n}_\text{t})$, the minimum numbers of antennas and spherical surface elements at receiver and transmitter, respectively.
%The rank of $\vect{H}$ is dictated by \eqref{MIMO_channel} as $\rank(\vect{H}) \le {\sf n}_\text{min}$, whereby ${\sf n}_\text{r}$ and ${\sf n}_\text{t}$ in \eqref{n_rt} can be regarded as the maximum number of effective antennas in a correlated channel, achievable with isotropic scattering.
%Holographic MIMO is characterized by \cite{PizzoTWC21}
%\begin{equation} \label{Nyquist_cond}
%N_{\text r} \ge {\sf n}_\text{r} \qquad \text{and} \qquad N_{\text t} \ge {\sf n}_\text{t}
%\end{equation}
%implying that $\vect{H}$ is rank-deficient with probability $1$, namely $\rank(\vect{H}) \le {\sf n}_\text{min} \le N_\text{min}$, 
%the number of spatial dimensions available for communication being limited by the environment rather than by the number of antennas as in MIMO \cite{heath_lozano_2018}.
 %In \eqref{tx_corr_coupled}, the transmit eigenvalues \eqref{variances_channel_composite} are proportional to $\lambda^{-3}$, which cancels out with $R=\pi Z_0/\lambda^2$ and as per the normalization of $\tilde{H}^+_\text{t}(\cdot,\cdot)$, similarly to \eqref{channel_power_coupled_norm}.
%As for \eqref{tx_corr_coupled_exact}, 
%Then,   %in proportion to a resistance $0<R_{\text{d}} \le R_{\text{t}}$ \cite{Nossek2010}. 
%Such dissipation amounts to an additional diagonal term, subsumed by the augmented coupling matrix $\vect{{\sf C}}_\text{t}(\eta) = \vect{{\sf C}}_\text{t} + \left(1 + {1}/{\eta}\right)  \vect{I}_{N_\text{t}}$ with $0 <  \eta < 1$ the antenna radiation efficiency
%Defining %$\eta = (1+R_{\text{d}}/R_{\text{t}})^{-1}$, 
%\begin{equation} \label{input_power_loss_mat}
%\vect{{\sf C}}_\text{t}(R_{\text{d}}/R_{\text{t}}) = \vect{{\sf C}}_{\text{t}} + \frac{R_\text{d}}{R_\text{t}}  \vect{I}_{N_\text{t}} 
%\end{equation}
%with $0 < R_{\text{d}}/R_{\text{t}} \le 1$ the ratio of the power dissipated by an array to the power radiated by such array when the antennas are uncoupled. % by virtue of \eqref{time_avg_power_discrete}.
%Notably, $R_{\text{d}}/R_{\text{t}}$ directly relates to the \emph{antenna radiation efficiency} \cite{BalanisBook}
%\begin{equation} \label{efficiency_mat}
%\eta = \frac{1}{1+R_{\text{d}}/R_{\text{t}}} \qquad\quad 0 <  \eta < 1,
%\end{equation}
%whereby \eqref{input_power_loss_mat} can be rewritten as 
%\begin{equation} \label{lossy_impedance_mat}
%\vect{{\sf C}}_\text{t}(\eta) = \vect{{\sf C}}_\text{t} + \left(1 + \frac{1}{\eta}\right)  \vect{I}_{N_\text{t}} 
%\end{equation}
%and $\vect{{\sf C}}_\text{t}$ corresponding to lossless antennas.
%For conciseness, the dependance on $\eta$ is henceforth omitted and $\vect{{\sf C}}_\text{t}$ is used to represent the entire form.
%\angel{Confusing, I'm not sure how to interpret $\vect{{\sf C}}_{\text {t}}$ henceforth.}
%values required for an efficient transmission. %$R_{\text{t}}/R_{\text{d}}\to\infty$ resulting in a lossless transmitter.
%\angel{Yet in the remainder of this paragraph $\vect{{\sf C}}_{\text{t}}$ still denotes only the restricted form.}
%Antenna losses act through $\eta$ as a physical regularization, which improves the conditioning of the coupling matrix \eqref{impedance_matrix_general}, rendering the inversion well behaved \cite{Wallace2004}.

%\andrea{I can't figure out why there is a dependence of \eqref{variances_channel_composite} on the wavelength through the constant $\kappa=2\pi/\lambda$. Contrarily, the exact matrix decomposition does not show a similar dependance, as it varies accordingly to $L/\lambda$ via the transmit correlation and coupling matrices. For the purpose of showing the eigenvalues, I could remove the $\kappa^2$ term by normalizing the average channel power to one; see App.~\ref{app:normalization}. But this does not solve the problem for the computation of the ergodic capacity, where I need the unnormalized channel.}

%\angel{Any comments on the figure?} 

%The accuracy of \eqref{MIMO_channel} for finite apertures is exemplified in Fig.~\ref{fig:channel_var_coupled} under isotropic scattering, in which case the spatial correlation is known as reported in \eqref{Toeplitz_coupling}, due to the isomorphism between the omnidirectional coupling and Clarke-Jakes correlation. 
%Fig.~\ref{fig:channel_var_coupled} plots, on a DOF scale, the sorted eigenvalues of the transmit correlation that are specialized here for an ULA of uncoupled antennas spaced by $\lambda/4$ and aperture $L_\text{t} = 10 \lambda$. For the Fourier, these follows from \eqref{variances_channel} as 
%\begin{equation}
%\sigma^2_{\text{t},\vect{m}} = \frac{N_\text{t}}{(2\pi)^2} \iint_{\Omega_{{\text t},\vect{m}}} \sin \theta_{\text t} d\theta_{\text t} d\phi_{\text t}
%\end{equation}
%for $\vect{m} \in \Lambda_\text{t}$ such that $|\Lambda_\text{t}| = {\sf n}_\text{t}$. The eigenvalues of the Clarke-Jakes model exhibits a polarization at ${\sf n}_\text{t}$, as captured by the low-rank Fourier representation.

%\angel{I think that rather than "other than" you mean "besides"}
%\angel{No need to repeat all of this again, just mention that it's as before only with the coupling-inclusive quantities}
%of the MIMO channel matrix
%\begin{align} \label{semi_UIU_model_coupling}
%\vect{H}_\vect{{\sf C}} & \approx 
%\vect{U}_{\text{r}} 
%{\color{blue}\tilde{\vect{H}}_\vect{{\sf C}}}
% \vect{U}_{\text{t}}^{\Htran}
%\end{align}
%where $\vect{U}_{\text{r}} \in \Complex^{N_\text{r}\times {\sf n}_\text{r}}$ and $\vect{U}_{\text{t}} \in \Complex^{N_\text{t}\times {\sf n}_\text{t}}$ are the semiunitary Fourier eigenvector matrices {\color{blue}in \eqref{semi_UIU_model}
%while $\tilde{\vect{H}}_\vect{{\sf C}}$} contains the independent complex Gaussian entries $\{\tilde{h}_{\vect{{\sf C}},\vect{i}\vect{j}}\}$.  
%\begin{align} \label{Kronecker_MIMO_coupling}
%\vect{{\sf H}} & = 
%\vect{U} \tilde{\vect{{\sf H}}} \vect{V}^{\Htran}
%\end{align} 
%where $\vect{U}$ and $\vect{V}$ are the semiunitary Fourier eigenvector matrices in \eqref{Kronecker_MIMO}. In turn, $\tilde{\vect{{\sf H}}} \in \Complex^{{\sf n}_\text{r} \times {\sf n}_\text{t}}$ has independent entries $\{\tilde{{\sf h}}_{\vect{i}\vect{j}}\}$ distributed as in \eqref{xi_coeff}, namely
%\begin{equation} \label{equiv_channel_coupling}
%\tilde{\vect{{\sf H}}} = \vect{\Lambda}_{\text{r}}^{1/2} \vect{W} \vect{{\sf \Lambda}}_{\text{t}}^{1/2}
%\end{equation}
%where $\vect{\Lambda}_{\text{r}}$ and $\vect{W}$ are defined in \eqref{Kronecker_MIMO}, whereas $\vect{{\sf \Lambda}}_{\text{t}}$ is diagonal with nonnegative entries derived by vectorizing the variances in \eqref{variances_channel_composite} while rescaling them according to \eqref{xi_coeff}, namely $\{N_\text{t} \sigma^2_{j}({\sf H}_\text{t}^+)\}$ for $j=1, \ldots, {\sf n}_\text{t}$.
%In turn, $\tilde{\vect{R}}^{1/2}_\text{t}$ denote the square root of a matrix accounting for the combined effect of correlation and coupling at the transmitter. 


%\begin{figure}[t!]
%     \centering
%     \begin{subfigure}{\columnwidth}
%         \centering
%         \includegraphics[width=.999\columnwidth]{channel_var_coupled_iso}
%\caption{Isotropic scattering with omnidirectional antennas.}
%         \label{fig:channel_var_coupled_iso}
%     \end{subfigure}
%     \vfill
%     \vspace{.1cm}
%     \begin{subfigure}{\columnwidth}
%         \centering
%         \includegraphics[width=.999\columnwidth]{channel_var_coupled_noniso_new}
%         \caption{Selective scattering with directive antennas.}
%         \label{fig:channel_var_coupled_noniso_new}
%     \end{subfigure}
%       \caption{Normalized sorted eigenvalues of the transmit correlation matrix under isotropic and selective scattering. UPA with antennas spaced by $\lambda/4$ and aperture $20 \lambda$. Matching each antenna's spectrum to the fading spectrum causes decorrelation.}
%        \label{fig:channel_var_coupled}
%\end{figure}

%When the scattering is separable, though this need not be the case in general, 
%\begin{equation} 
%\Ex\{|\vect{H}_\text{ind}|^2\} = \lambda_\vect{n}(\vect{R}_{\text r}) \lambda_\vect{m}(\vect{R}_{\text t}) 
%\end{equation}
%where $\vect{R}_{\text t} = \vect{U}_{\vect{H},\text{t}} \vect{\Lambda}_{\vect{H},\text{t}}  \vect{U}_{\vect{H},\text{t}}^{\Htran}  \in \Complex^{N_\text{t}\times N_\text{t}}$ and $\vect{R}_{\text r} = \vect{U}_{\vect{H},\text{r}} \vect{\Lambda}_{\vect{H},\text{r}}  \vect{U}_{\vect{H},\text{r}}^{\Htran} \in \Complex^{N_\text{r}\times N_\text{r}}$ are
%the transmit and receive correlation matrices, given $\vect{\Lambda}_{\vect{H},\text{t}}$ and $\vect{\Lambda}_{\vect{H},\text{r}}$ diagonal with entries $\lambda_\vect{n}(\vect{R}_{\text r}) = N_\text{r} \lambda_{{\text r},\vect{n}}$ and $\lambda_\vect{m}(\vect{R}_{\text t}) = N_\text{t} \lambda_{{\text t},\vect{m}}$, respectively.
%%$\vect{R}_{\text t} = \diag(\vect{\sigma}_{\text t} \odot \vect{\sigma}_{\text t})$ and $\vect{R}_{\text r}= \diag(\vect{\sigma}_{\text r} \odot \vect{\sigma}_{\text r})$ with diagonal entries
%For instance, at the transmit side, 
%\begin{align} \label{variances_channel_transmit}
%\lambda_{{\text t},\vect{m}} =  \iint_{\Omega_{{\text t},\vect{m}}} \Ex\{|H(\theta_{\text t},\phi_{\text t})|^2\} \sin \theta_{\text t} \, d\theta_{\text t} d\phi_{\text t}
%\end{align}
%which is obtained by leveraging separability of the angular spectrum in \eqref{variances_channel}.
%Thus, \eqref{semi_UIU_model} specializes to \cite{chizhik2000effect} 
%\begin{align} \label{semi_UIU_model_Kronecker}
%\vect{H} & = 
%\vect{R}^{1/2}_{\text{r}} \vect{H}_\text{w} \vect{R}^{\Htran/2}_{\text{t}}
%\end{align}
%where $\vect{H}_\text{w} \in \Complex^{{\sf n}_\text{r} \times {\sf n}_\text{t}}$ has IID standard complex Gaussian entries.
%
%\angel{This section should be streamlined, it's an overly comprehensive review of known representations and doesn't shed any light on the role of coupling. Perhaps we can keep the more general representations, and remove everything after "When the scattering is separable..."}

Generalization of the transmit correlation matrix in \eqref{tx_corr} to coupled antennas yields
%The curve is benchmarked against the eigenvalues of the correlation matrix corresponding to the coupled channel in , reading 
\begin{align}  \label{tx_corr_coupled}
\vect{{\sf R}}_\text{t} & = \frac{1}{N_\text{r}} \Ex\{\vect{{\sf H}}^{\Htran} \vect{{\sf H}}\}   \approx \frac{2}{{\sf R}} \,  \vect{V} \vect{{\sf \Lambda}}_{\text{t}} \vect{V}^{\Htran} 
\end{align}
with associated eigenvalues
\begin{align} \label{eig_tx_corr_coupled}
\vect{\lambda}(\vect{{\sf R}}_\text{t}) \approx  N_\text{t} \, \big(\underbrace{\sigma^2_{1}(\tilde{{\sf H}}^{+}_\text{t}), \ldots, \sigma^2_{{\sf n}_\text{t}}(\tilde{{\sf H}}^{+}_\text{t})}_\text{DOF}, \underbrace{0, \ldots, 0}_{N_\text{t}-{\sf n}_\text{t}} \big),
\end{align}
where $N_\text{t}-{\sf n}_\text{t}$ eigenvalues are zero, as in \eqref{tx_eig_uncoupled}, due to the inherent low-rankness of the Fourier representation.


When the antenna power pattern matches the fading spectrum, coupling %can decrease spatial correlation.
causes antenna decorrelation, resulting in the most uniform distribution and strongest polarization %at ${\sf n}_\text{t}$
of the eigenvalues.
This is because fading maps to the channel via convolution, whereas coupling acts as a deconvolution 
% effectively multiplying the fading spectrum by the reciprocal of the antenna power pattern .
through the reciprocal of the pattern, as discussed in Sec.~\ref{sec:coupling_vs_corr}.
%More generally, the closer the antenna pattern matches the fading's, the more uniform the transmit eigenvalue distribution becomes.
%A similar conclusion emanates from the exact correlation matrix, 
This stronger eigenvalue polarization is evidenced by the Fourier curve in Fig.~\ref{fig:channel_var_coupled_iso}, where the sorted eigenvalues in \eqref{eig_tx_corr_coupled} are plotted for isotropic scattering and omnidirectional antennas, whereby \eqref{variances_channel_composite} reduces to the projected solid angles in \eqref{solid_angles_coupling}.
%An UPA with {\color{magenta}coupled antennas (loss factor $\rho=0.01$) spaced by} $\lambda/2$ and aperture $L_{\text{t},x} = L_{\text{t},y} = 20 \lambda$ is considered. 
%\andrea{The problem is that $\rho$ has not been defined yet, but introducing the regularization here might be too early.}

Also shown in Fig.~\ref{fig:channel_var_coupled_iso} are the exact eigenvalues of the transmit correlation, under the same normalization, derivable from \eqref{channel_samples_coupled} as
\begin{align} \label{numerical_inv}
\vect{{\sf R}}_\text{t} = \frac{2}{{\sf R}} \, \vect{{\sf C}}^{-1/2}_\text{t} \vect{R}_\text{t} \vect{{\sf C}}^{-1/2}_\text{t}
\end{align}
with $\vect{R}_\text{t}$ the correlation without coupling, given in \eqref{corr_exact}.
The potential for coupling to decorrelate the antennas can also be appreciated here. Ultimately,
if $\vect{{\sf C}}_\text{t}$ equals $\vect{R}_\text{t}$, then \eqref{numerical_inv} becomes an identity matrix.
%However, this complete decorrelation hinges on the conditioning of $\vect{{\sf C}}_\text{t}$, which worsens as antennas densify within an aperture.
However, accounting for ohmic losses in the antennas, a portion of the transmit power dissipates as heat, subsumed by the augmented matrix \cite{Nossek2010}
\begin{align} \label{coupling_loss}
\vect{{\sf C}}_\text{t}(\rho) = \vect{{\sf C}}_\text{t} + \rho \vect{I}_{N_\text{t}},
\end{align}
with $0 < \rho < 1$ the loss factor.
This loss is tantamount to a physical regularization, improving the conditioning of $\vect{{\sf C}}_\text{t}$ for a stable inversion, but at the cost of a reduced decorrelation.
 %\eqref{numerical_inv} relative to its lower-dimensional approximation in \eqref{tx_corr_coupled}.

%The impact of coupling on spatial bandwidth is discussed in Section~\ref{sec:DOF_HighSNR}, while Section~\ref{sec:antenna_densification} examines how DOF are influenced by antenna densification and losses.


% its isomorphism with correlation enables $\vect{{\sf C}}_\text{t}$ to equal $\vect{R}_\text{t}$, whereby \eqref{numerical_inv} is an identity matrix. 
%It is observed that the closer the antenna pattern matches the fading spectrum, the more uniform the transmit eigenvalue distribution.
%{\color{red}Compared to \eqref{tx_eig_uncoupled}, where $N_\text{t}-{\sf n}_\text{t}$ eigenvalues are zero, some of these eigenvalues in \eqref{eig_tx_corr_coupled} could conceivably be positive depending on the undetermined spectrum in \eqref{spectrum_channel_transmitter_coupling}. %\angel{Shouldn't this point to (94) rather than (97)?} 
%if so, this would provide additional dimensions for spatial multiplexing and diversity.
% However, these dimensions would be linked to the evanescent portion of the spectrum, contradicting the spatial stationarity assumption underlying this work. 
%\angel{Here we seem to contradict our later conclusion that coupling does not change the DOF. This is a crucial issue!}
%This property is not reflected in \eqref{tx_corr_coupled} due to the reduced rank.
%However, these dimensions would be linked to the evanescent portion of the spectrum, excluded in \eqref{spectral_representation_z} due to the spatial stationarity assumption.} % thereby contradicting the main assumption underlying the Fourier model.
 %\angel{Need to make explicit "what" about these dimensions contradicts the assumption.}

%A similar conclusion emanates from the exact correlation matrix, derivable from \eqref{channel_samples_coupled} as
%\begin{align} \label{numerical_inv}
%\vect{{\sf R}}_\text{t} = \frac{2}{{\sf R}} \, \vect{{\sf C}}^{-1/2}_\text{t} \vect{R}_\text{t} \vect{{\sf C}}^{-1/2}_\text{t}
%\end{align}
%with %{\color{blue}$\vect{{\sf C}}_\text{t}^{-1/2} = (\vect{{\sf C}}_\text{t}^{1/2})^{-1}$ and}
%$\vect{R}_\text{t}$ the correlation with uncoupled antennas in \eqref{corr_exact}. 
% and $\vect{{\sf C}}^{-1/2}_\text{t}$ %obtainable from an eigenvalue or Cholesky decomposition 
%the square root matrix of $\vect{{\sf C}}_\text{t}^{-1}$. %, such that $\vect{{\sf C}}^{-1}_\text{t} = \vect{{\sf C}}^{-1/2}_\text{t} \vect{{\sf C}}^{-1/2}_\text{t}$.
%Note that the vertical shift between the curves in Fig.~\ref{fig:channel_var_coupled} is due to different normalizations as per the extra DOF provided by \eqref{numerical_inv} relative to its lower-dimensional approximation in \eqref{tx_corr_coupled}.
%Properly harnessed, coupling creates additional DOF, its isomorphism with correlation enabling $\vect{{\sf C}}_\text{t}$ to equal $\vect{R}_\text{t}$, whereby \eqref{numerical_inv} is an identity matrix. 
%\andrea{Moved here, but I am considering removing the regularization for different reasons: i) our paper is mostly theoretical, ii) it is intended to curb superdirectivity, whose effect we have not discussed here, and iii) it does not appear in the Fourier model.}
%{\color{red}Consistently with the stationarity assumption, the undetermined eigenvalues in \eqref{eig_tx_corr_coupled}
%and eigenvalues with index larger than ${\sf n}_\text{t}$ in \eqref{numerical_inv} %and in its lower-dimensional approximation in \eqref{tx_corr_coupled},
%are henceforth set to zero, 
%%Similarly, the extra DOF provided by \eqref{numerical_inv} relative to its lower-dimensional approximation in \eqref{tx_corr_coupled} are omitted
%though they may be relevant for non-stationary channels. }
%Coupling acts as a physical precoder through $\vect{{\sf C}}_\text{t}^{-1/2}$, pre-whitening the correlation from scattering effects.

%More generally, altering the antenna power pattern to match or counter the fading spectrum, coupling can enhance or reduce spatial correlation, as shown in Fig.~\ref{fig:decorrelation_2}.

%Further discussion on  
%sampling above the Nyquist rate does not significantly enhance the field description as improvement are contingent upon observability of the evanescent spectrum \cite{PizzoTSP21}. 
%{\color{red}
%From \eqref{MIMO_channel_coupling}, $\rank(\vect{{\sf H}}) \le \min(N_\text{r},N_\text{t})$ with probability $1$;  the number of spatial dimensions is determined, besides environmental scattering, by the antenna pattern, which serves as a proxy for coupling at the transmitter.
%Further discussion on the interplay between coupling and DOF can be found in Section~\ref{sec:DOF_HighSNR}.}

% shown in Fig.~\ref{fig:channel_var_uncoupled}.
%subtended by $\Omega^+_{\vect{j}}$ in \eqref{variances_channel}, given by (up to a factor)
%\begin{equation} \label{solid_angles_coupling}
%|{\sf \Omega}^+_{\vect{j}}| = \frac{1}{\pi} \iint_{\Omega^+_{\vect{j}}}  \cos\theta_{\text t} \, \sin \theta_{\text t} \, d\theta_{\text t} d\phi_{\text t}.
%\end{equation}
%Coupling introduces a broadside projection, absent in the uncoupled formulation---recall \eqref{solid_angles}. It incorporates the impact of array orientation, skewing the solid angles subtended by the scattering towards the broadside direction.
 %area projection of the angular scattering onto the broadside direction ($\theta_{\text t}=0$). 
%projected area subtended by the power angular spectrum of the channel as viewed from the corresponding antenna array.
%It reduces to the solid angle $\iint_{{\rm supp}(|\tilde{H}^+_\text{r}|^2)} \sin \theta_{\text r} \, d\theta_{\text r} d\phi_{\
%Fig.~\ref{fig:channel_var_coupled_noniso_new} considers selective fading with a $z$-aligned von-Mises-distributed \cite{MardiaBook} power angular spectrum {\color{blue}of angular standard deviation $\sigma = 20^\circ$ and antenna pattern with the same distribution.} 
%\andrea{Removed the equation.}
%von-Mises-distributed scattering and antenna spectra with the same angular standard deviation are considered in Fig.~\ref{fig:channel_var_coupled_selectivity}.
%Specifically, selective fading with a $z$-aligned von-Mises power angular spectrum ($\beta = 8.2$, $\sigma = 20^\circ$) \cite{MardiaBook}, namely 
%\begin{align} \label{vonMises}
%\Ex \! \left \{|\tilde{H}^+_\text{t}(\theta_{\text t},\phi_{\text t})|^2 \right \} = c(\beta) \, e^{\beta \cos(\theta_{\text t})} \qquad (\theta_{\text t},\phi_{\text t}) \in \Omega^+
%\end{align}
%with $c(\beta)$ the normalization factor ensuring \eqref{normalization_channel}. 
%\angel{Couldn't follow what you mean by "does not necessarily occur"}
%leading to a more uniform eigenvalue distribution.}
%\andrea{Wanted to say that \eqref{solid_angles_coupling} is more general than just arising from isotropic scattering and omnidirectional antennas.}
%\andrea{The extra DOF are achievable upon invertibility of $\vect{{\sf C}}_\text{t}$. However, some eigenvalues (those associated with evanescent waves) are very small: same issue with superdirectivity in transmit beamforming.}

%\section{Impact of Antenna Densification} \label{sec:antenna_densification}
\section{DOF Augmentation via Mutual Coupling} \label{sec:DOF_increase}

\begin{figure}[t!]
     \centering
     \begin{subfigure}{\columnwidth}
         \centering
         \includegraphics[width=.9\columnwidth]{decorrelation_2a}
\caption{Reciprocal of the antenna pattern runs counter to the fading spectrum.}
         \label{fig:decorrelation_2a}
     \end{subfigure}
     \vfill
     \vspace{.2cm}
     \begin{subfigure}{\columnwidth}
         \centering
         \includegraphics[width=.9\columnwidth]{decorrelation_2b}
         \caption{Reciprocal of the antenna pattern matches the fading spectrum.}
         \label{fig:decorrelation_2b}
     \end{subfigure}
       \caption{{\color{blue}Having the reciprocal of the antenna power pattern counter or match the fading spectrum causes coupling to 
       reduce or enhance antenna correlation, respectively. This is because fading maps to the channel via
convolution in \eqref{spectrum_HC}, whereas coupling acts as a deconvolution through the reciprocal of the pattern.}
       For a fixed antenna count, reducing correlation appears to expand the antenna spacing, hence the aperture, while increasing correlation is interchangeable with compressing the antenna spacing, hence the aperture.}
        \label{fig:decorrelation_2}
\end{figure}


%\begin{figure}[t]
%     \centering
%     \includegraphics[width=.8\columnwidth]{decorrelation_2}
%         \caption{Having the antenna power pattern match or counter the fading spectrum respectively reduces or enhances spatial correlation. For a fixed antenna count, reducing correlation appears to expand the antenna separation, hence the aperture, while increasing correlation is interchangeable with compressing the antennas, hence the aperture.}
%     \label{fig:decorrelation_2}
%     \end{figure}
     
%We analyze the impact of antenna densification at the transmitter on ergodic capacity for a fixed aperture size of $L_{\text{t},x} = L_{\text{t},y}=8 \lambda$ under low- and high-SNR conditions.
%The antennas, oriented parallel to the $xy$-plane with dimensions $L_{\text{a},x}=L_{\text{a},y}=\lambda/5$, follow the pattern in \eqref{antenna_pattern_rectangular}.
%Fig.~\ref{fig:antenna_densification} shows ${\sf C}({\sf SNR})$ from \eqref{capacity} for ${\sf SNR} = \{-5, 20\}$~dB as a function of transmit antenna spacing $d_{\text{t},x} \in[\sqrt{2} L_{\text{a},x}, L_{\text{t},x} - \sqrt{2} L_{\text{a},x}]$, with lower and upper bounds  determined by causality and aperture size constraints, respectively. 
%The number of transmit antennas along each dimension is given by
%\begin{align}
%N_{\text{t},x} = \left\lfloor\frac{L_{\text{t},x} - \sqrt{2} L_{\text{a},x}}{d_{\text{t},x}} +1 \right\rfloor
%\end{align}
%for any given $d_{\text{t},x}$.
%The receiver uses a UPA with punctiform antennas spaced $\lambda/2$ apart and aperture $L_{\text{r},x} = L_{\text{r},y} = 8 \lambda$. 
%The results are benchmarked against the average (w.r.t. the antenna arrangements) ergodic capacity of the exact channel model in \eqref{channel_samples_coupled}, where four corner antennas are fixed, and the remaining antennas are randomly distributed within the aperture.
\subsection{Impact of Antenna Power Pattern on DOF}

%The effect of antenna densification on ergodic capacity is analyzed under low- and high-SNR conditions at the transmitter. 
%For a fixed aperture, densification is achieved by uniformly increasing antenna counts in all dimensions. 
%From \eqref{lowSNR_capacity}, low-SNR capacity improves linearly with increasing antenna densification and it is ultimately bounded by the number of antennas that can be accommodated by any fixed aperture.
%At high SNR, antenna density enters the formulation through the spatial DOF and the second-order term of the expansion in \eqref{C_SNR_high}. Precisely, the DOF are determined, besides fading selectivity, by the antenna pattern, which serves as a proxy for coupling at the transmitter, effectively mimicking  aperture, as shown in Fig.~\ref{fig:decorrelation_2}.
%additional DOF beyond the uncoupled
%The maximum achievable DOF is determined
%The spatial DOF equals the number of independent fading realizations observable by antennas within a fixed aperture.
%For uncoupled antennas, the DOF per unit space are bounded solely by correlation, inversely related to fading selectivity. With coupling, the DOF are determined, besides selectivity, by the antenna pattern, which serves as a proxy for coupling---functioning like a lens that shapes how fading is observed.
%antennas can alter the perception of  and, when properly adjusted, promote decorrelation; see Fig.~\ref{fig:channel_var_coupled_iso}.
%Properly adjusting the antenna pattern can promote decorrelation (see Fig.~\ref{fig:channel_var_coupled_iso}), essentially mimicking an aperture expansion for a fixed antenna count and increasing the DOF relative to uncoupled antennas.
The exact eigenvalues in Fig.~\ref{fig:channel_var_coupled_iso} reveal a second effect that the Fourier approximation fails to capture, namely that,
properly harnessed, coupling can create \emph{additional} DOF %for spatial multiplexing and diversity, 
beyond the uncoupled limit at ${\sf n}_{\text t}$. % {\color{magenta}as opposed to what claimed in \cite{Sha2023}.} 
%Therefore, even for holographic MIMO communications with strong mutual coupling, ignoring the mutual coupling will not bring large errors in the analysis of DOF. 
An interpretation of this phenomenon is provided in Fig.~\ref{fig:decorrelation_2}, whereby---everything else being the same---a reduced correlation is indistinguishable from a wider antenna spacing, hence an enlarged aperture for a fixed antenna count.
%generally associated with more spread antennas, the correlation shift is equivalent to an aperture widening, for a given number of antennas. 
Conversely, an increased correlation %by mismatching the antenna pattern to the fading spectrum
is tantamount to having tighter antenna spacings, making the aperture appear smaller for a fixed antenna count.
%enhancing correlation via a mismatch of the antenna pattern to the fading spectrum is akin to a shrunk aperture.
Thus, consistent with Landau's theorem, which expresses the spatial DOF as the product of spatial bandwidth and aperture \cite{Franceschetti,PizzoWCL22}, extra DOF may arise due to coupling: more or fewer DOF fit on the equivalent aperture, as it is expanded or contracted by coupling.
%This explains the DOF increase due to coupling in Fig.~\ref{fig:decorrelation_2},  
Such DOF augmentation is reflected in the horizontal stretch of the exact curve in Fig.~\ref{fig:channel_var_coupled_iso}, and also its vertical shift (due to the normalization by a larger number of positive eigenvalues.)

\subsection{Impact of Antenna Losses and Densification on DOF}

Antenna losses (in the amount of $\rho$, the loss factor) tone down the impact of the antenna pattern as a proxy for transmit coupling, limiting the extent to which the equivalent aperture expands or contracts for a fixed antenna count, with $\rho=0$ and $\rho=1$ the most and least favorable to DOF augmentation, respectively.
The sorted eigenvalues in Fig.~\ref{fig:DOF_augmentation} illustrate this for various $\rho$ and a fixed aperture, $L_{\text{t},x}=L_{\text{t},y}=10 \lambda$. Antennas are omnidirectional, matching the considered isotropic scattering.
The extra DOF brought about by coupling can be leveraged if the antenna density is sufficiently high, up to what the sampling theorem determines for the enlarged aperture. The Nyquist density thus depends on $\rho$ and on the fading spectrum. For $\rho\to 0$, each additional antenna should give rise to an equal increase in DOF, as hinted by the inversion in \eqref{numerical_inv}. However, due to finite accuracy of computer simulations, this limit is not achievable in numerical simulations.
Indeed, there is probably no limit to the theoretical DOF gain that can be achieved by such ideal antennas, but this result is of limited significance when physical antennas are concerned---antenna size constraint and losses limit the achievable gain.


%hinder the ability of each antenna to decorrelate fading, limiting the extent to which the equivalent aperture can be expanded.}
%\andrea{I tried to convey the message in the added subsection. When rho is close to 1, decorrelation elicited by coupling is poor and this maps to a curbed expansion of the equivalent aperture. Conversely, when rho = 0 (ideal antennas), full decorrelation can be achieved and the expansion is maximum, hence requiring a higher Nyquist density (see my comment above).} 
%\angel{Hmm. What you're saying is that $\rho$ affects the size of the equivalent aperture, which is fine, but this paragraph is about sampling rate}
%\angel{Need to integrate this next sentence, and reconcile with the point made in the previous subsection as to how losses improve the conditioning and facilitate the decorrelation.}

%{\color{magenta}While aligning the antenna pattern with fading may seem to reinforce correlation from environmental scattering through enhanced selectivity, it instead attenuates it.
%This occurs because fading maps to the channel via convolution, whereas coupling acts as a deconvolution 
% effectively multiplying the fading spectrum by the reciprocal of the antenna power pattern .
%through the reciprocal of the antenna pattern, as discussed in Sec.~\ref{sec:coupling_vs_corr}.} 

%\begin{figure}
%\centering\vspace{-0.0cm}
%\includegraphics[width=.999\linewidth]{DOF_densification} 
%\caption{{\color{blue}Ratio of ${\sf DOF}_\text{t}(0.5,\rho)$ in \eqref{DOF_numerical} and ${\sf n}_\text{t}$ in \eqref{n_rt} versus average antenna spacing, for various loss factors $\rho$, under isotropic scattering. UPA with omnidirectional antennas and aperture $10 \lambda$. The dashed line indicates the antenna count, $N_\text{t}$.
%For each average antenna spacing, DOF values are averaged over random antenna deployments
%%antenna arrangements
% within the array.
%% \angel{I see, the antenna spacing is randomized, hence the "average" in the caption. It's a bit weird, because everywhere in the paper the spacing is deterministic, and a random spacing does not represent actual arrays. But never mind, the plot reflects the correct behavior, too late to change it now.} %while maintaining a constant aperture.
% }}
%% \angel{See if you can clarify this a bit, not sure I get it myself. And perhaps the right place for it is the figure caption itself.}
%\label{fig:DOF_densification}
%\end{figure}

%{\color{blue}When the antenna pattern is matched to the fading spectrum,} the extra DOF brought about by coupling can be leveraged if the antenna density is sufficiently high, up to what the sampling theorem determines for the enlarged aperture. This depends on $\rho$ and on the fading spectrum.%The observability of the achievable DOF requires sampling the equivalent aperture at a sufficiently high antenna density, as per the sampling theorem.
%The Nyquist sampling is determined by $\rho$ for a given antenna pattern and fading distribution, {\color{magenta}with $\rho=0$ or $\rho=1$ corresponding to the largest and smallest modified apertures, respectively.}  
%\andrea{The Nyquist sampling rate for the equivalent aperture does not change. What changes is the sampling rate for the actual aperture as more antennas needs to be accommodated into the same footprint, hence reducing the spacing.}
%\angel{Remind me, how does $\rho$ impact the Nyquist sampling?}
 %According to the sampling theorem, however, attaining this limit is contingent upon full observability of the field within the aperture, which requires a sufficiently high antenna density.
%Thus, the optimum density that guarantees the achievable DOF is specified by the effective aperture as a function of the antenna pattern and loss factor, $\rho$. 
%Exceeding Nyquist's density for the equivalent aperture does not significantly increase DOF, but instead takes the toll on the increased correlation due to reduced antenna spacing. %reduced spatial diversity

%{\color{blue}Depicted in Fig.~\ref{fig:DOF_densification} is the DOF gain, relative to uncoupled antennas, as a function of the antenna spacing for various loss factors $\rho$ and a fixed aperture.
%Antennas are omnidirectional, matching the considered isotropic scattering.
%The DOF computation relies on the asymptotic polarization of the channel eigenvalues, $\lambda_i(\rho) = \lambda_i(\vect{{\sf H}}(\rho) \vect{{\sf H}}^{\Htran}(\rho))$, with $\vect{{\sf H}}(\rho)$ as in \eqref{channel_samples_coupled} but with the augmented coupling matrix in \eqref{coupling_loss}.
%Particularly, the DOF are computed iteratively as the smallest index $n$ such that $\lambda_n > \epsilon$, with proper normalization, namely \cite{FranceschettiBook}
%\begin{align} \label{DOF_numerical}
%{\sf DOF}_\text{t}(\epsilon,\rho) = \min\! \left\{n : \lambda_n(\rho) > \epsilon, \textstyle\sum_{i=1}^n \lambda_i(\rho) = n \right\}.
%\end{align} 
%%Since $\lambda/2$-spaced omnidirectional antennas are generally uncoupled, coupling always creates additional DOF beyond those provided by uncoupled antennas.
%DOF curves are benchmarked against the number of antennas, $N_\text{t}$, demonstrating that each additional antenna can support an equal increase in DOF in the limit $\rho\to 0$.
%Indeed, there is probably no limit to the theoretical DOF gain that can be achieved by such ideal antennas, but this result is of limited significance when physical antennas are concerned. (Antenna size constraint and losses limit the achievable gain.)
%%However, these limitations can be mitigated by increasing the electrical aperture and optimizing antenna spacing.
%}

\begin{figure}[t]
     \centering
     \includegraphics[width=.999\columnwidth]{DOF_augmentation}
         \caption{Spatial DOF augmentation for various $\rho$ under isotropic scattering for the exact correlation in \eqref{numerical_inv}. Omnidirectional antennas (decorrelating the fading), %(matched to the fading) 
         spaced by $\lambda/4$ and aperture $10 \lambda$.}
     \label{fig:DOF_augmentation}
     \end{figure}
     
%\section{Capacity Analysis}  \label{sec:spectral_efficiency}
\section{Information-Theoretic Analysis}   \label{sec:spectral_efficiency}
  
%{\color{magenta}To isolate the effect of antenna densification and losses at the transmitter, arising from having $N_\text{t} \gg {\sf n}_\text{t}$ and $\rho >0$, from SNR variations, we henceforth adhere to the Fourier model and analyze the impact of coupling across different SNR levels.}
  
Let us now turn to the information-theoretic implications of mutual coupling at different SNR levels. To isolate its effect on the eigenvalue polarization from that of having extra DOF, the latter aspect is excluded in this section and the Fourier model is applied unless otherwise stated. The impact of the additional DOF could be separately studied by replacing the actual aperture by its coupling-expanded counterpart.
  
%\angel{Couldn't the Fourier model capture the extra DOF if we change the actual aperture into an expanded one?}  
%  \andrea{Great catch! In principle, yes. Replacing the actual aperture with the equivalent aperture would also preserve the holoMIMO condition in \eqref{Nyquist_cond}, as coupling is immaterial to the maximum spatial bandwidth.}
  
 \subsection{Coupling as Physical Precoder}

Communicating entails encoding independent unit-variance symbols $\vect{s} \in\Complex^{{\sf n}_{\text t}}$ onto the current vector $\vect{j}_{\text t} = \vect{F} \vect{s} $ via the %$N_{\text t} \times {\sf n}_{\text t}$ 
%spatial
precoder $\vect{F} \in\Complex^{N_{\text t} \times {\sf n}_{\text t}}$.
%\begin{equation}
%\vect{F} = \vect{U}_\vect{F}  
%\begin{pmatrix}
%\vect{P}^{1/2} \\
%\vect{0}_{(N_{\text t} - {\sf n}_{\text t}) \times {\sf n}_{\text t}}
%\end{pmatrix}
%\vect{V}_\vect{F}^{\Htran},
%\end{equation} 
%with $\vect{V}_\vect{F} \in \Complex^{{\sf n}_{\text t} \times {\sf n}_{\text t}}$ and $\vect{U}_\vect{F} \in \Complex^{N_{\text t} \times N_{\text t}}$ unitary matrices and $\vect{P} = \diag(p_1, \ldots, p_{{\sf n}_{\text t}})$, $p_i \ge 0$ such that the average (w.r.t. the input distribution) signal-to-noise-ratio (SNR) %constraint 
%equals
%\begin{equation} \label{input_power_constraint}
%%{\rm tr} \! \left(\vect{F}^{\Htran} \vect{{\sf C}}_{\text{t}} \vect{F}\right) \le 
%{\sf SNR} = G P_\text{t}/(\sigma^2 R_\text{t}) > 0
%\end{equation}
%%%is satisfied, 
%with $G$ the large-scale channel coefficient and $\sigma^2$ the noise power.
%, and $\vect{{\sf C}}_{\text{t}}$ the lossy coupling matrix in \eqref{lossy_impedance}.
%; obtainable averaging the right-hand side of \eqref{time_avg_power_discrete} with respect to the input distribution, 
%, given ${\sf SNR} = P_\text{t}/(\sigma^2 R_\text{t})$.
%Adding noise to the model, 
%\angel{Where is the coupling hidden in this model?}
%From \eqref{time_avg_power_discrete_rewritten}--\eqref{convolution_coupling_sampled},
From \eqref{convolution_coupling_sampled} and \eqref{composite_current_vec}, the input-output relationship over a noisy channel is thus
%\begin{equation} \label{MIMO_model}
%\vect{y} = \vect{H} \vect{F} \vect{s} + \vect{n}
%\end{equation}
\begin{equation} \label{MIMO_coupling}
\vect{y} = \vect{{\sf H}} \vect{{\sf F}} \vect{s} + \vect{n} 
\end{equation}
with $\vect{{\sf H}}$ the composite MIMO channel and $\vect{n}\sim\CN(\vect{0}, \sigma^2 \vect{I}_{N_{\text r}})$ given $\sigma^2$ the noise power, which could be spatially colored if interference were present in addition to thermal noise. 
In turn,
\begin{equation} \label{FC_precoder}
\vect{{\sf F}} = \vect{{\sf C}}_\text{t}^{1/2} \vect{F} \in \Complex^{N_{\text t} \times {\sf n}_{\text t}}
\end{equation}
is the composition of precoder and coupling, subject to
\begin{equation} \label{SNR_constraint}
{\rm tr} \! \left(\vect{{\sf F}}^{\Htran} \vect{{\sf F}}\right) \le  {\sf SNR}
\end{equation}
where ${\sf SNR} = \frac{2}{{\sf R}} \frac{G {\sf P}_\text{t}}{\sigma^2}$ is the average %(w.r.t. the input distribution)
signal-to-noise-ratio,
%\begin{align} \label{input_power_constraint}
%%{\rm tr} \! \left(\vect{F}^{\Htran} \vect{{\sf C}}_{\text{t}} \vect{F}\right) \le 
%{\sf SNR} & = {\color{blue}\frac{2}{{\sf R}}} \frac{G {\sf P}_\text{t}}{\sigma^2}
%\end{align}
with $G$ the large-scale channel gain. %\angel{Above you introduced the noise as having unit variance}
The above SNR constraint %ensures the positive definiteness of the input covariance and
prevents the radiated power from growing with either $N_{\text t}$ or with ${\sf n}_{\text t}$, i.e., with the number of transmit antennas or the transmit aperture \cite{Wallace2004}. %\angel{\eqref{input_power_constraint}  is itself not a power constraint}

The effect of the mutual coupling is seen to be that of a precoder that, in tandem with the standard precoder, linearly combines the entries of the input vector. % corresponding to the information sent angularly. 
Unlike that standard precoder, though, the coupling precoder is induced by antenna interactions.
%over which no control is exerted. \angel{Maybe we should tone this down a bit, because adjusting the pattern is a form of control. }
%{\color{blue}that might not be controllable.} 
%\angel{This contradicts the point made in the Intro about reconfiguring the antenna pattern with metamaterials, etc. I'd just remove it.}
This is in agreement with previous works on mutual coupling \cite{Nossek2010,Wallace2004}.
%with $\vect{n}\sim\CN(\vect{0}, \sigma^2 \vect{I}_{N_{\text r}})$, which 
%Coupling effects in \eqref{MIMO_model} are correctly revealed by constraining the input power according to the actual physical transmit power in \eqref{input_power_loss} as seen next.


%Letting the receiver be located in the right $z$ half-space of a planar transmitter, the transmit power in \eqref{time_avg_power_final_antenna} is the upgoing contribution
%\begin{align}  \label{power_antenna_2D}
%P_\text{t} & = \iint_{\|\vect{k}\|\le \kappa} d\vect{k} \, \frac{|J_{\text t}^+(\vect{k})|^2 |A_{\text t}^+(\vect{k})|^2}{\gamma}.
%\end{align}
%(Note that \eqref{time_avg_power_final_antenna} should be used in lieu of \eqref{power_antenna_2D} when propagation occurs in both $z$ half-spaces, with the power then split between upgoing and downgoing contributions; this would be the case of a volumetric transmitter or two back-to-back planar transmitters.)
%Replacing $J_{\text t}^+(\vect{k})$ with its Fourier transform in \eqref{3Dcurrent_spectrum} while using \eqref{real_impedance_kernel_antenna} correctly returns the transmit power in \eqref{circuit_power_avg}. \angel{Why couldn't we start this subsection directly with (\eqref{circuit_power_avg}) rather than (\ref{power_antenna_2D})? The point has been made that circuit and EM powers coincide.}
%For a planar source, discretizing it by means of a Riemann sum \cite{HeedongIRS},
%\begin{align}    \label{time_avg_power_discrete}
%P_\text{t} = \vect{j}_{\text{t}}^{\Htran} \vect{{\sf C}}_{\text{t}} \vect{j}_{\text{t}}
%\end{align}
%where $[\vect{j}_{\text{t}}]_m = j_{\text{t}}(\vect{s}_m,0)$, $[\vect{{\sf C}}_\text{t}]_{n,m} = (L^2_\text{t}/N_\text{t})^2 {\sf c}_{\text t}(\vect{r}_n-\vect{s}_m,0)$, with a slight abuse of notation due to a different normalization of $\vect{{\sf C}}_\text{t}$ compared to \eqref{impedance_matrix_general}. 


%\subsection{Coupling as Physical Precoder}
%Being positive-definite, $\vect{{\sf C}}_{\text{t}} = \vect{{\sf C}}_\text{t}^{\Htran/2} \vect{{\sf C}}_\text{t}^{1/2}$ for some invertible $\vect{{\sf C}}_\text{t}^{1/2} = \vect{\Lambda}_{\vect{{\sf C}}_\text{t}}^{1/2} \vect{U}_{\vect{{\sf C}}_\text{t}}^{\Htran}$, given $\vect{U}_{\vect{{\sf C}}_\text{t}} $ unitary and $\vect{\Lambda}_{\vect{{\sf C}}_\text{t}} $ diagonal with positive entries.
%Also, from \eqref{input_power_constraint}, the SNR constraint can be rewritten as
%%\angel{I don't see where this averaging is taking place.}
%%rewriting the transmit power in \eqref{time_avg_power_discrete} as
%\begin{equation} \label{power_constraint_precoder}
%{\rm tr} \! \left( \vect{F}^{\Htran} \vect{{\sf C}}_\text{t}^{\Htran/2} \vect{{\sf C}}_\text{t}^{1/2} \vect{F}\right)
%= {\rm tr}\big(\vect{F}_{\vect{{\sf C}}} \vect{F}_{\vect{{\sf C}}}^{\Htran}\big) \le {\sf SNR}
%\end{equation}
%%\begin{align}  \label{power_constraint_MIMO}
%%% {\rm tr} \! \left(\vect{{\sf C}}_\text{t} \vect{F} \vect{F}^{\Htran} \right) 
%%% = {\rm tr}\big(\vect{F}_\vect{{\sf C}} \vect{F}_\vect{{\sf C}}^{\Htran}\big) = P_\text{t}
%%P_\text{t} =  {\rm tr}\big(\vect{F}_\vect{{\sf C}} \vect{F}_\vect{{\sf C}}^{\Htran}\big)
%%\end{align}
%where
%\begin{equation}
%\vect{F}_{\vect{{\sf C}}} =  \vect{{\sf C}}_\text{t}^{1/2} \vect{F} \in \Complex^{N_{\text t} \times {\sf n}_{\text t}}
%\end{equation}
%is the composition of precoder and coupling.
%Let us now describe the input current, channel, and noisy receive fields in terms of their coordinates on an orthonormal basis. The projections thereon return the information-theoretic MIMO model to be used for performance analysis. Sampling the continuous model at antenna locations and performing a singular value decomposition in space would lead to the same results, as long as the Nyquist condition is satisfied [12], [13]. The Fourier basis is chosen because it provides asymptotically uncorrelated channel coordinates over rectangular spaces and is amenable to an intuitive physical interpretation.
%\andrea{the precoder optimization is unclear to me. Different approaches lead to different solutions. I may have found a metric for controlling array superdirectivity (i.e., the factor $Q$) though.}

\subsection{Ergodic Capacity with Coupling at the Transmitter}% \angel{Also at the Rx, no?}}

% \begin{figure}
%\centering\vspace{-0.0cm}
%\includegraphics[width=.999\linewidth]{SEvsSNR_precoder_iso} 
%\caption{Spectral efficiency vs SNR for UPAs with punctiform antennas spaced by $d_\text{t}=d_\text{r} =\lambda/2$ and $L_\text{t}=L_\text{r}=5  \lambda$.}\vspace{-0cm}
%\label{fig:SEvsSNR}
%\end{figure}



With perfect channel-state information (CSI) at the receiver,
the mutual information %conditioned on $\vect{{\sf H}}$
%the mutual information conditioned on $\vect{H}$ and $\vect{{\sf C}}_\text{t}$
%\angel{and on $\vect{{\sf C}}_\text{t}$?} 
achieved by $\vect{s}\sim\CN(\vect{0},\vect{I}_{{\sf n}_{\text t}})$
is \cite{heath_lozano_2018} 
%[PERHAPS WE OUGHT TO MAKE EXPLICIT THAT WE ASSUME PERFECT CSI AT THE RECEIVER, WHICH IS WHAT THIS CONDITIONING EMBODIES]
%\begin{align}   \label{MI_xy}
%I(\vect{x};\vect{y}|\vect{H},\vect{{\sf C}}_\text{t}) & = \log_2 \det (\vect{I} + \vect{H} \vect{F} \vect{F}^{\Htran}  \vect{H}^{\Htran})
%\\ \notag & = \log_2 \det \! \big(\vect{I} + \vect{H} \vect{{\sf C}}_\text{t}^{-1/2} \vect{F}_{\vect{{\sf C}}} \vect{F}_{\vect{{\sf C}}}^{\Htran} \vect{{\sf C}}_\text{t}^{-\Htran/2} \vect{H}^{\Htran}  \big),   
%\end{align}
%which, in terms of the composite channel matrix
%\begin{equation} \label{channel_mat_c}
%\vect{H}_{\vect{{\sf C}}} = \vect{H} \vect{{\sf C}}_\text{t}^{-1/2}
%\end{equation}
%with $\vect{{\sf C}}_\text{t}^{-1/2} = \vect{U}_{\vect{{\sf C}}_\text{t}} \vect{\Lambda}_{\vect{{\sf C}}_\text{t}}^{-1/2}$,
%%such that $\vect{{\sf C}}_\text{t}^{-1/2} \vect{{\sf C}}_\text{t}^{1/2} = \vect{I}_{N_\text{t}}$,
%gives
\begin{align}  \label{MI_xy_composite}
I(\vect{s};\vect{y}| \vect{{\sf H}}) & 
= \log_2 \det ( \vect{I} + \vect{{\sf H}} \vect{{\sf F}} \vect{{\sf F}}^{\Htran} \vect{{\sf H}}^{\Htran} ) .
\end{align}  
%where the conditioning on $\vect{{\sf H}}$ in \eqref{MIMO_channel_coupling} subsumes the dependence on fading and coupling through the antenna power pattern and loss factor.
%The conditioning on $\vect{{\sf H}}$ subsumes the dependence on channel features and array characteristics, including coupling.
%Symmetry in \eqref{channel_mat_c} is re-established when adding mutual coupling effects at the receiver, yielding (due to reciprocity) 
%\begin{equation} \label{channel_mat_c_rx}
%\tilde{\vect{H}} = \vect{{\sf C}}_\text{r}^{-1/2} \vect{H} \vect{{\sf C}}_\text{t}^{-1/2}
%\end{equation}
%with $\vect{{\sf C}}_\text{r} $ the receive coupling matrix. \angel{Why not add coupling at both Tx and Rx from the beginning of the section?}
%Thus, \eqref{MIMO_model} yields the same mutual information as
%\begin{equation} \label{MIMO_model_C}
%\vect{y}_{\vect{{\sf C}}} = \vect{H}_{\vect{{\sf C}}} \vect{F}_{\vect{{\sf C}}} \vect{s} + \vect{n} 
%\end{equation}
%with $\vect{F}_{\vect{{\sf C}}}$ subject to \eqref{power_constraint_precoder}.
%Capacity is achieved by $\vect{s}\sim\CN(\vect{0},\vect{I}_{{\sf n}_{\text t}})$;
The rotational invariance of $\vect{s}$ renders
 the right singular vectors of the precoder immaterial, whereby %\cite{heath_lozano_2018}
%[THIS COMES FROM CSI AT THE RECEIVER, NOT THE TRANSMITTER. PERHAPS WE SHOULD HELP THE READER APPRECIATE THAT, BECAUSE OF THE CIRCULAR SYMMETRY, THE RIGHT SINGULAR VECTORS OF THE PRECODER ARE THEN IMMATERIAL]
\begin{equation} \label{precoder_coupling}
\vect{{\sf F}} = \vect{U}_{\vect{{\sf F}}} \vect{{\sf P}}^{1/2}
%\begin{pmatrix}
%\vect{P}^{1/2} \\
%\vect{0}_{(N_{\text t} - {\sf n}_{\text t}) \times {\sf n}_{\text t}}
%\end{pmatrix}
\end{equation}
with $\vect{U}_{\vect{{\sf F}}}  \in\Complex^{N_{\text t} \times {\sf n}_{\text t}}$ %isometry
and $\vect{{\sf P}} = \diag({\sf p}_1, \ldots, {\sf p}_{{\sf n}_{\text t}})$, ${\sf p}_i \ge 0$ such that \eqref{SNR_constraint} is satisfied, implying $\sum_{i=1}^{{\sf n}_{\text t}} {\sf p}_i \le {\sf SNR}$.
%$\vect{U}_{\vect{F}_\vect{{\sf C}}}  \in\Complex^{N_{\text t} \times N_{\text t}}$ unitary 
%and $\vect{P} = \diag(p_1, \ldots, p_{{\sf n}_{\text t}})$ % $p_i \ge 0$ %$i=1, \ldots, {\sf n}_{\text t}$,
%such that
%\eqref{precoder_coupling} obeys the transmit power constraint in
%\eqref{power_constraint_precoder} is satisfied.
%\angel{$\vect{U}_{\vect{F}_{\vect{{\sf C}}}}$ and $\vect{P}$ must themselves be optimized as a function of the distribution of $ \vect{H}_{\vect{{\sf C}}} $, Otherwise you don't have the capacity, but an achievable rate}
%{\color{blue}with $\vect{U}_{\vect{F}_{\vect{{\sf C}}}}$ and $\vect{P}$ that must be optimized as a function of the distribution of $ \vect{H}_{\vect{{\sf C}}}$ according to the available CSI.}

%We differentiate between a classical MIMO system with 
%an unconstrained and constrained precoder optimization. The latter case is intended for  where superdirectivity has a marginal impact, whereas 
The superscript $^\star$ is henceforth used to distinguish the capacity-achieving value of any quantity. % that relates to the precoder $\vect{F}$.
%\andrea{We should emphasize that capacity is conditioned to a given antenna pattern and that optimization is performed wrt the precoder only.}
%\angel{Why abuse of notation? Also, the star to denote optimality was used earlier already}}
%Moreover, owing to the analytical tractability of the Kronecker model,
%We adhere to the Fourier model, but use the exact channel for benchmarking numerically the derived results.
With CSI further at the transmitter, \eqref{MI_xy_composite} is maximized when
%the argument of the determinant is diagonal. Invoking the singular values decomposition of $\tilde{\vect{H}}$ in \eqref{channel_mat_c}, 
%$\tilde{\vect{H}} = \vect{U}_{\vect{H}_\vect{{\sf C}}} \vect{\Sigma}^{1/2}_{\vect{H}_\vect{{\sf C}}} \vect{V}_{\vect{H}_\vect{{\sf C}}}^{\Htran}$, $\vect{V}_{\vect{H}_\vect{{\sf C}}}\in\Complex^{N_{\text t} \times N_{\text t}}$ and $\vect{U}_{\vect{H}_\vect{{\sf C}}}\in\Complex^{N_{\text r} \times N_{\text r}}$ unitary matrices containing the left and right eigenvectors of $\vect{H}_\vect{{\sf C}}^{\Htran} \vect{H}_\vect{{\sf C}}$ and $\vect{\Sigma}_{\vect{H}_\vect{{\sf C}}} \in\Complex^{N_{\text r} \times N_{\text t}}$ the eigenvalue matrix, 
%this is achieved by
$\vect{U}_{\vect{{\sf F}}}^\star = \vect{V}_{\vect{{\sf H}}}$ \cite[Sec. 5.3]{heath_lozano_2018} with $\vect{V}_{\vect{{\sf H}}}$ the right singular vector matrix of $\vect{{\sf H}}$, %approximately given by
\begin{equation} \label{right_singular_tildeHc} 
\vect{V}_{\vect{{\sf H}}} \approx \vect{V} \vect{V}_{\vect{\tilde{{\sf H}}}} \in \Complex^{N_\text{t}\times {\sf n}_\text{t}}
\end{equation}
with $\vect{V}$ the %isometry
Fourier matrix in \eqref{Kronecker_MIMO} and $\vect{V}_{\vect{\tilde{{\sf H}}}} \in \Complex^{{\sf n}_\text{t}\times {\sf n}_\text{t}}$ the right singular vector matrix of $\vect{\tilde{{\sf H}}}$ in \eqref{equiv_channel_coupling}.
%\angel{We should elaborate, here or elsewhere, on the benefit of expressing $\vect{V}_{\vect{{\sf H}}}$ as $\vect{V} \vect{V}_{\tilde{\vect{{\sf H}}}}$. Usually the MIMO formulations are in terms of $\vect{V}_{\vect{{\sf H}}}$ directly}
By exploiting the Fourier matrix unitarity, we deviate from the usual MIMO formulation, which expresses the precoder in terms of $\vect{V}_{\vect{{\sf H}}}$ directly. % and {\color{magenta}leads to a simplified transceiver architecture.}
%This leads to a simplified transceiver architecture and a reduced-complexity channel estimation, as will be seen.
%of $\vect{H}_\vect{{\sf C}}^{\Htran} \vect{H}_\vect{{\sf C}}$.
%\begin{align}
%I\left(\vect{x};\vect{y}|\vect{H}\right) & = \log_2 \det\left(\vect{I} + \vect{\Sigma}_{\vect{H}_\vect{{\sf C}}} \vect{P} \right)
%\end{align}
%\begin{equation} \label{MIMO_equiv}
%\vect{y} = \vect{U}_{\text{r}} 
%\tilde{\vect{H}}_\vect{{\sf C}}
% \vect{U}_{\text{t}}^{\Htran} \vect{F}_{\vect{{\sf C}}} \vect{s} + \vect{n} 
%\end{equation}

%\andrea{Making explicit the dependance of the capacity on $\vect{{\sf C}}_\text{t}$ is misleading, as the coupling matrix does not formally appear in our Kronecker model. We would also need to do the same for the angular power scattering, another parameter of the model. How about using ${\sf C}(\cdot)$ in place of $C(\cdot)$ to recall that this quantity is associated with coupling?}

%\angel{Not sure what you're doing here. The ergodic capacity does not depend on $\vect{{\sf H}}$. We're expecting over it.}
%{\color{blue} Let ${\sf n}_\text{min} = \min({\sf n}_\text{r},{\sf n}_\text{t})$ be the minimum number of spherical surface elements at receiver and transmitter.}
The ergodic capacity for a specific fading spectrum and antenna pattern
is then 
%since $\vect{H}$ has independent entries with each entry being a complex Gaussian random variable, the capacity-achieving input vector is an independent unit-variance complex Gaussian symbol, i.e., $\vect{F} = \diag(\vect{p}^{1/2})$, $\vect{p} = (p_1, \ldots, p_{n_{\text t}})$ the power vector \cite{LozanoCorrelation,Veeravalli}.
%Physically, it implies that symbols transmitted at different angles are uncorrelated each other.
%\andrea{True, but the ergodic capacity depends on coupling. I am trying to find a way to reflects this dependance in the left-hand side. However, coupling manifests through different parameters based on the considered model.}
\begin{align} \label{capacity}
%{\sf C}({\color{blue}\tilde{H}_\text{r}^+,\tilde{{\sf H}}_\text{t}^+,}{\sf SNR})
{\sf C}({\sf SNR})  =  \Ex \! \left\{ \sum_{i=1}^{{\sf n}_\text{min}} \log_2 \! \left(1 + {\sf p}_i^\star \lambda_{i}(\vect{\tilde{{\sf H}}} \vect{\tilde{{\sf H}}}^{\Htran}) \right)\right\}
\end{align}
%with $N_\text{min}=\min(N_\text{r},N_\text{t})$, 
where 
\begin{equation} \label{waterfilling}
{\sf p}_i^\star = \left(\nu - \lambda_{i}^{-1}(\vect{\tilde{{\sf H}}} \vect{\tilde{{\sf H}}}^{\Htran})\right)^{\! +}
\end{equation}
with ${\sf n}_\text{min} = \min({\sf n}_\text{r},{\sf n}_\text{t})$ and $\nu$ such that $\sum_{i=1}^{{\sf n}_\text{min}} {\sf p}_i^\star = {\sf SNR}$, given $\lambda_{i}(\vect{\tilde{{\sf H}}} \vect{\tilde{{\sf H}}}^{\Htran})$ as the $i$th unordered eigenvalue of $\vect{\tilde{{\sf H}}} \vect{\tilde{{\sf H}}}^{\Htran}$. %, {\color{blue}inclusive of the antenna pattern ${\sf A}_\text{t}^+(\cdot,\cdot)$}.
% and {\color{blue}${\sf SNR} = P_\text{t}/(\sigma^2 R)$} the signal-to-average-noise-ratio (SNR).
%{\color{blue}The capacity $C({\sf SNR},\eta)$ is reported in Fig.~\ref{fig:SEvsSNR} as a function of the SNR for various antenna radiation efficiencies $\eta$ and transmit and receive UPAs consisting of punctiform antennas spaced by $d_\text{t}=d_\text{r}=\lambda/2$. The two arrays have apertures $L_\text{t}=L_\text{r}=5 \lambda$; the conjunction with antenna spacing identifies two $11\times 11$ MIMO arrays.}
%The capacity of an uncoupled MIMO system (non-physical) is included as a benchmark. 
%The gap between the coupled and uncoupled curves increases with the SNR, implying that the DOF are heavily limited by mutual coupling, even at half-wavelength spacing.
%with $\vect{P}^\star = \diag(p_1^\star, \ldots, p_{{\sf n}_{\text t}}^\star)$.
% as $\{C_{n,m}\}$ are never all equal, even under isotropic coupling, thereby reflecting the curvature of the spectral support in \eqref{spectrum_impedance_isotropic}. 
%The optimal precoder for an uncoupled system is obtained for $\vect{{\sf C}}_\text{t} = \vect{I}_{N_{\text t}}$, yet this is never the case physically.
The transmit precoder achieving \eqref{capacity} arises from \eqref{FC_precoder} after substituting \eqref{precoder_coupling} and solving for $\vect{F}$,% as
\begin{align}  \label{optimal_precoder_0}
\vect{F}^\star & =  \vect{{\sf C}}_\text{t}^{-1/2} \vect{U}_{\vect{{\sf F}}}^\star (\vect{{\sf P}}^\star)^{1/2} \\ \label{optimal_precoder}
& = \vect{{\sf C}}_\text{t}^{-1/2} \vect{V} \vect{V}_{\vect{\tilde{{\sf H}}}}
%\begin{pmatrix}
%(\vect{P}^\star)^{1/2} \\
%\vect{0}_{(N_{\text t} - {\sf n}_{\text t}) \times {\sf n}_{\text t}}
%\end{pmatrix} 
(\vect{{\sf P}}^\star)^{1/2}
\end{align}
where \eqref{right_singular_tildeHc} was used.
%It subsumes superdirective precoder designs in the far-field line-of-sight direction $(\theta_{\text t},\phi_{\text t})$, specified by the array response $\vect{a}(\cdot,\cdot)$ in \eqref{Gavi}, with $\vect{F}^\star = \sqrt{{\sf SNR}} \, \vect{{\sf C}}_\text{t}^{-1/2} \vect{a}(\theta_{\text t},\phi_{\text t})$\cite{Marzetta2019}.
%{\color{magenta}This formulation generalizes far-field line-of-sight precoder designs to near-field non-line-of-sight conditions. \angel{Don't quite see that.}
%Precisely, combining \eqref{channel_samples_coupled} with \eqref{precoder_coupling}, 
%\begin{align}
%\vect{F}^\star = \sqrt{{\sf SNR}} \, \vect{{\sf C}}_\text{t}^{-1/2} \vect{a}(\theta_{\text t},\phi_{\text t}),
%\end{align}
%with $\vect{a}(\theta_{\text t},\phi_{\text t})$ the array response in \eqref{Gavi}.} % for any $(\theta_{\text t},\phi_{\text t})$.}
In the special case that the channel is  line-of-sight, $\vect{H} = \vect{a}(\theta_{\text t},\phi_{\text t}) \vect{a}^{\Htran}(\theta_{\text t},\phi_{\text t})$ with $\vect{a}(\cdot,\cdot)$ the array response in \eqref{Gavi}, 
%It subsumes superdirective precoder designs for the direction $(\theta_{\text t},\phi_{\text t})$, defined by the line-of-sight channel $\vect{H} = \vect{a}(\theta_{\text t},\phi_{\text t}) \vect{a}^{\Htran}(\theta_{\text t},\phi_{\text t})$, with $\vect{a}(\cdot,\cdot)$ the array response in \eqref{Gavi}. 
%From \eqref{channel_samples_coupled},
the right singular vector of $\vect{H}^{\Htran} \vect{H}$ is---recalling \eqref{channel_samples_coupled}---proportional to $\vect{{\sf C}}_\text{t}^{-1/2} \vect{a}(\theta_{\text t},\phi_{\text t})$, whereby %\eqref{optimal_precoder_0} reduces to
\cite{Marzetta2019}
%generalizes far-field line-of-sight precoder designs to near-field non-line-of-sight conditions. Precisely, combining \eqref{channel_samples_coupled} with \eqref{precoder_coupling}, 
%$\vect{V}_{\vect{{\sf H}}} = \vect{{\sf C}}_\text{t}^{-1/2} \vect{a}(\theta_{\text t},\phi_{\text t})$ 
\begin{align}
\vect{F}^\star = \sqrt{{\sf SNR}} \, \vect{{\sf C}}_\text{t}^{-1} \vect{a}(\theta_{\text t},\phi_{\text t}).
\end{align}
%{\color{blue}This formulation generalizes far-field line-of-sight precoder designs, where $\vect{F}^\star = \sqrt{{\sf SNR}} \, \vect{{\sf C}}_\text{t}^{-1/2} \vect{a}(\theta_{\text t},\phi_{\text t})$, with $\vect{a}(\theta_{\text t},\phi_{\text t})$ the array response in \eqref{Gavi} for any direction $(\theta_{\text t},\phi_{\text t})$.}
%\andrea{I was pointing to superdirectivity, but it is more general than that.}

A precoder that ignored coupling would convey information spatially on a set of resolvable directions determined by the scattering environment and specified by the right singular vectors of %columns of $\vect{V}_{\vect{H}}$ associated with the ${\sf n}_{\text t}$ largest eigenvalues of 
$\vect{H}^{\Htran} \vect{H}$, namely \cite{PizzoTWC21}
 \begin{equation}
 \vect{F}^\star =  \vect{V} \vect{V}_{\vect{\tilde{H}}} (\vect{P}^\star)^{1/2}
 \end{equation}
where $\vect{P}^\star$ and $\vect{V}_{\vect{\tilde{H}}}$ are associated with $\vect{\tilde{H}}$ in \eqref{equiv_channel}, rather than $\vect{\tilde{{\sf H}}}$ in \eqref{equiv_channel_coupling}.
%Here, $\vect{V}$ performs a change of representation from the angular to the spatial domain to match the prescribed transmit antenna locations \cite{PizzoTWC21}. }
%Such a strategy, however, would ignore %array geometry and antenna structure,
%coupling, % and relative spacing, 
%\angel{Hmm. The effect of spacing on the correlation would be accounted for, not so the effect on the coupling}
% which is accounted for in \eqref{optimal_precoder}, with remarkable implications on the spatial multiplexing and spectral efficiency.
Instead, the optimal precoder in \eqref{optimal_precoder} first diagonalizes $\vect{\tilde{{\sf H}}}$, which does depend on the coupling. The result is translated to the spatial domain via $\vect{V}$
% by conveying} information on a set of directions specified by the antenna structure and environmental scattering. % while balancing off the coupling produced by proximal antennas through the pre-whitening by $\vect{{\sf C}}_\text{t}^{-1/2}$.   
%The result is 
and then whitened to remove the coupling produced by proximal transmit antennas.

%{\color{blue}To emphasize the importance of relevant variables in the analysis, the dependance of the capacity in \eqref{capacity} on all other parameters will be omitted depending on the context.}


%\angel{How does the matching of the antenna pattern to the channel that we have advocated play out here? If affects things via $\vect{{\sf C}}_\text{t}^{-1/2}$ and $\vect{\tilde{{\sf H}}}$}
%\andrea{Changing of antenna patterns impacts only $\vect{\tilde{{\sf H}}}$ (akin to the deconvolution in \eqref{channel_samples_coupled} for the exact model). Instead, the whitening through $\vect{{\sf C}}_\text{t}^{-1/2}$ appears due to the mapping in \eqref{composite_current_vec}.}


%\andrea{Shall we remove the following paragraph? It becomes less important when accounting for all DOF.}
%\angel{Sounds good, but then you should also remove the earlier comment about the simplification of channel estimation}
%{\color{red}Also worthwhile is that \eqref{optimal_precoder} requires the knowledge of the $({\sf n}_{\text r} \times {\sf n}_{\text t})$-dimensional matrix $\vect{\tilde{{\sf H}}}$ only, as $\vect{{\sf C}}_\text{t}^{-1/2}$ and $\vect{V}$ are always known a priori, for a given array geometry.
%Thus, the number of channel parameters grows in proportion with the electrical areas of the transmit and receive arrays as per \eqref{n_rt}, rather than with the number of antennas.} %\angel{Another nice insight!}

%which reduces to $\vect{F}^\star = \vect{V}_{\vect{H}} \left[(\vect{P}^\star)^{1/2}, \vect{0}_{{\sf n}_{\text t} \times (N_{\text t} - {\sf n}_{\text t})}\right]^{\Ttran}$ for an uncoupled MIMO system (i.e., $\vect{{\sf C}}_\text{t} = \vect{I}_{N_\text{t}}$), where $p_i^\star = (\nu - \lambda_{\vect{H},i}^{-1})^+$ with $\nu$ complying with the standard waterfilling criterion \cite{heath_lozano_2018}. %such that the allocated powers satisfy the usual SNR constraint \cite{heath_lozano_2018}.
%  the signal-to-average-noise-ratio per 
%single-input-single-output dimension.
 %{\color{blue}Here, the equality with the SNR is ensured by the monotonicity of the mutual information in the transmit power.}
%[THIS SNR IS A BIT ARTIFICIAL, SINCE THE CHANNEL WAS NORMALIZED AND THAT NORMALIZATION SHOULD BE REFLECTED IN THE SNR]

%Thus, the optimal holographic MIMO precoder allocates power according to a waterfilling criteria on a set of directions specified by the columns of $\vect{V}_{\tilde{\vect{H}}_\vect{{\sf C}}}$. These describe the solid angles at the transmit side that are coupled to the receiver's, as determined jointly by the antenna directivity and environmental scattering. Then, $\vect{U}_{\text{t}}$ performs a change of representation from the angular to the spatial domain to match the prescribed  antenna locations. Lastly, the physical noise produced by mutual coupling among proximal antennas is cancelled through a pre-whitening operation.



%For simplicity, assume a Kronecker structure for the channel whereby the angular kernels at each ends of the link are fully decoupled \cite{chizhik2000effect}, namely
%\begin{equation}
%H(\theta_{\text r},\phi_{\text r},\theta_{\text t},\phi_{\text t}) = 
%H_\text{r}(\theta_{\text r},\phi_{\text r}) H_\text{t}(\theta_{\text t},\phi_{\text t}),
%\end{equation}
%with $H_\text{r}(\cdot,\cdot)$ and $H_\text{t}(\cdot,\cdot)$ the corresponding kernels at the receiver and transmitter. 
%Thus, \eqref{semi_UIU_model} specializes to \cite{chizhik2000effect} 
%\begin{align} \label{semi_UIU_model_Kronecker}
%\vect{H} & = 
%\vect{R}^{1/2}_{\text{r}} \vect{H}_\text{w} \vect{R}^{\Htran/2}_{\text{t}}
%\end{align}
%where $\vect{H}_\text{w} \in \Complex^{{\sf n}_\text{r} \times {\sf n}_\text{t}}$ has IID standard complex Gaussian entries. In turn, $\vect{R}_{\text t} = \vect{U}_{\vect{H},\text{t}} \vect{\Lambda}_{\vect{H},\text{t}}  \vect{U}_{\vect{H},\text{t}}^{\Htran}  \in \Complex^{N_\text{t}\times N_\text{t}}$ and $\vect{R}_{\text r} = \vect{U}_{\vect{H},\text{r}} \vect{\Lambda}_{\vect{H},\text{r}}  \vect{U}_{\vect{H},\text{r}}^{\Htran} \in \Complex^{N_\text{r}\times N_\text{r}}$ are
%the transmit and receive correlation matrices, given $\vect{\Lambda}_{\vect{H},\text{r}}$ and $\vect{\Lambda}_{\vect{H},\text{t}}$ the ${\sf n}_\text{r}$- and ${\sf n}_\text{t}$-dimensional diagonal matrices with entries $\lambda_\vect{n}(\vect{R}_{\text r}) = N_\text{r} \lambda_{{\text r},\vect{n}}$ and $\lambda_\vect{m}(\vect{R}_{\text t}) = N_\text{t} \lambda_{{\text t},\vect{m}}$, respectively.
%For instance, at the transmit side, 
%\begin{align} \label{variances_channel_transmit}
%\lambda_{{\text t},\vect{m}} =  \iint_{\Omega_{{\text t},\vect{m}}} \Ex\{|H^+_\text{t}(\theta_{\text t},\phi_{\text t})|^2\} \, d\Omega_\text{t}.
%%\sin \theta_{\text t} \, d\theta_{\text t} d\phi_{\text t}.
%\end{align} 

%\angel{Not sure what you're trying to say here...}
%\andrea{In your book, the SNR limiting regimes are studied first for deterministic channels. Then, the obtained results are averaged statistically under the assumption that those SNR conditions hold with probability $1$ for every channel realizations.}

% \subsection{Transmit Beamforming and Superdirectivity}
%
%To glean the implication of transmit superdirectivity on \eqref{optimal_precoder},
%let us momentarily focus on a single-antenna receiver, %i.e., $N_\text{r}=1$,
%whereby ${\sf n}_\text{r}=1$ under \eqref{Nyquist_cond}.
%Then, $\vect{{\sf H}}$ morphs into a multiple-input single-output (MISO) channel specified by the row vector $\vect{{\sf h}}^{\Htran}$.
%In the low-SNR regime, the precoder beamforms along the maximum-eigenvalue eigenvector of $\vect{{\sf h}} \vect{{\sf h}}^{\Htran}$ with CSI at the transmitter whereby \eqref{optimal_precoder} reduces to
%\begin{equation} \label{MRT_coupling}
%\vect{f}^\star = \sqrt{{\sf SNR}} \, \vect{{\sf C}}_{\text t}^{-1/2} \vect{{\sf h}}
%\end{equation} 
%for each realization of $\vect{{\sf h}}$.
%Expectedly, in the absence of coupling (i.e., $\vect{{\sf C}}_{\text t} = \vect{I}_{N_\text{t}}$ and $\vect{{\sf h}} = \vect{h}$), the maximum-ratio transmission $\vect{f}^\star = \sqrt{{\sf SNR}} \, \vect{h}$ is optimum in the MISO channel with CSI at the transmitter while $\Ex\{\lambda_\text{max}(\vect{h} \vect{h}^{\Htran})\} = N_{\text t}$ as per the channel normalization \cite{heath_lozano_2018}.
%Note that the array response at the transmitter is altered by coupling, which also appears as a pre-whitening in \eqref{MRT_coupling}.
%
%In a far-field LOS scenario, the receiver appears as punctiform from the vantage of the transmitter (and vice versa), implying rank-1 transmissions along a direction $(\theta_{\text t},\phi_{\text t})$.
%
% leading to array gains higher than $N_{\text t}$ at some preferred directions, a phenomenon aptly named superdirectivity \cite{Wallace2005,Nossek2010}.
%
%Then, $\vect{\tilde{{\sf H}}}$ in \eqref{equiv_channel_coupling} becomes a scalar deterministic quantity, being $1$ as per the normalization; the array response at the transmitter is not altered by coupling.
%For a single-antenna receiver, a multiple-input-single-output (MISO) channel arises whereby $\vect{H}$ equals the transmit array response $\vect{a}^{\Ttran}$ with entries in \eqref{Gavi}.
%From \eqref{optimal_precoder} the optimal precoder reduces to
%\begin{align} 
%\vect{f}^\star(\theta_{\text t},\phi_{\text t})  & = %\sqrt{\frac{2 {\sf SNR}}{R}}  \, 
%\vect{{\sf C}}_\text{t}^{-1/2} \vect{a}(\theta_{\text t},\phi_{\text t}) %\vect{V}_{\vect{\tilde{{\sf H}}}} (\vect{{\sf P}}^\star)^{1/2}
%\end{align}
%for every $(\theta_{\text t},\phi_{\text t})$.
%
%More generally, superdirectivity remains an issue at every SNR, due to the inversion of $\vect{C}_{\text t}$ required to compute \eqref{optimal_precoder}.
%
%Without mutual coupling, maximum-ratio transmission $\vect{f}^\star = \vect{h}/\|\vect{h}\|$ is optimal, achieving $G(\vect{h}) = \|\vect{h}\|^2$ and $G = N_\text{t}$.
%More generally, higher array gains can be achieved with mutual coupling, for the same number of antennas. This is the well-known phenomenon of superdirectivity whereby, for a linear array of uniformly spaced antennas, $G(\vect{h}) \to N_\text{t}^2$ along the axis containing the array, as the spacing vanishes \cite{Wallace2005,Nossek2010}.
%
%A comparison between \eqref{optimal_precoder_miso_final} and \eqref{optimal_precoder} reveals the connection between the optimal precoder and the superdirectivity phenomenon.
%Both promise array gains when an uncontrollable level of superdirectivity is allowed. 
%Precisely, from \eqref{optimal_gain}, superdirectivity is achieved when the channel vector $\vect{h}$ lies in a subspace spanned by the eigenvectors of $\vect{C}_\text{t}$ that are associated with the smallest eigenvalues \cite{Marzetta_superdirectivity}. 
%This event becomes more likely when the dimension of this subspace grows, as in the case of spatial oversampling, shown in Fig.~\ref{fig:eigC}.


%\begin{figure}
%\centering\vspace{-0.0cm}
%\includegraphics[width=.999\linewidth]{SEvsd_10lambda_lowSNR} 
%\caption{Spectral efficiency vs antenna spacing for various $\eta$ at ${\sf SNR} = -10$ dB under isotropic scattering. UPAs with punctiform antennas and apertures $L_\text{t}=L_\text{r}=10  \lambda$}.
%%\andrea{100 Monte Carlo}}
%\vspace{-0cm}
%\label{fig:SEvsd_10lambda_lowSNR}
%\end{figure}
%
%\begin{figure}
%\centering\vspace{-0.0cm}
%\includegraphics[width=.999\linewidth]{SEvsd_10lambda_highSNR}  
%\caption{Spectral efficiency vs antenna spacing for various $\eta$ at ${\sf SNR} = 20$ dB under isotropic scattering. UPAs with punctiform antennas and apertures $L_\text{t}=L_\text{r}=10  \lambda$.}
%%\andrea{50 Monte Carlo}
%\vspace{-0cm}
%\label{fig:SEvsd_10lambda_highSNR}
%\end{figure}
%
%\begin{figure}
%\centering\vspace{-0.0cm}
%\includegraphics[width=.999\linewidth]{SEvsL_d05lambda_lowSNR} 
%\caption{{\color{blue}Spectral efficiency vs array area for various $\eta$ at ${\sf SNR} = -10$ dB under isotropic scattering. UPAs with punctiform antennas spaced by $d_\text{t}=d_\text{r} =\lambda/2$.}
%\andrea{50 Monte Carlo}}\vspace{-0cm}
%\label{fig:SEvsL_d05lambda_lowSNR}
%\end{figure}
%
%\begin{figure}
%\centering\vspace{-0.0cm}
%\includegraphics[width=.999\linewidth]{SEvsL_d05lambda_highSNR}  
%\caption{{\color{blue}Spectral efficiency vs array area for various $\eta$ at ${\sf SNR} = 20$ dB under isotropic scattering. UPAs with punctiform antennas spaced by $d_\text{t}=d_\text{r} =\lambda/2$.}
%\andrea{50 Monte Carlo}}\vspace{-0cm}
%\label{fig:SEvsL_d05lambda_highSNR}
%\end{figure}
%
%Insights on omnidirectional coupling: 
%\begin{itemize}
%\item
%antenna densification adds spatial correlation, on top of the one generated by scattering. Thus, benefits available only at low SNRs.
%\item
%spectral efficiency proportional to normalized array area ($L/\lambda \in [1,20]$), irrespective of the SNR regime. At high SNRs, SE proportional to $(L/\lambda)^2$ through ${\sf n}_\text{min}$. We would expect SE with same slope as the spatial DOF are immaterial to $\eta$. However, curves have different slopes for various $\eta$. Are these the effective DOF at ${\sf SNR} = 20$ dB? Although the DOF do not change, the "noise" threshold varies with $\eta$ \cite{Lozano_HighSNR}.
%\end{itemize}
%
%\begin{figure}
%\centering\vspace{-0.0cm}
%\includegraphics[width=.999\linewidth]{SEvsSNR_10lambda_lowSNR_alpha} %SEvsSNR 
%\caption{{\color{blue}Spectral efficiency vs SNR for various $\alpha$. Directive antennas spaced by $d_\text{t}=d_\text{r} =\lambda/2$ and apertures $L_\text{t}=L_\text{r}=10  \lambda$.
%The solid line indicates the ergodic capacity in \eqref{capacity} while the dashed lines indicate the low-SNR expansion in \eqref{lowSNR_capacity} both under isotropic scattering and $\eta=0.9$.}}\vspace{-0cm}
%\label{fig:SEvsSNR_10lambda_lowSNR_alpha}
%\end{figure}
%
%\begin{figure}
%\centering\vspace{-0.0cm}
%\includegraphics[width=.999\linewidth]{SEvsSNR_10lambda_highSNR_alpha} %SEvsSNR 
%\caption{{\color{blue}Spectral efficiency vs SNR for various $\alpha$. Directive antennas spaced by $d_\text{t}=d_\text{r} =\lambda/2$ and apertures $L_\text{t}=L_\text{r}=10  \lambda$.
%The solid line indicates the ergodic capacity in \eqref{capacity} while the dashed lines indicate the high-SNR expansion in \eqref{C_SNR_high} both under isotropic scattering and $\eta=0.9$.}}\vspace{-0cm}
%\label{fig:SEvsSNR_10lambda_highSNR_alpha}
%\end{figure}

%\andrea{Before studying the SNR-limiting regimes, why not plotting the spectral efficiency of a suboptimal precoder that ignores coupling? The comparison with an uncoupled system does not stand because the transmit power would be different (the former is not the physical transmit power).} \angel{Sure, but then you'd have to compare with the optimum precoder that accounts for coupling}

%\subsection{Coupling in the Low-SNR Regime}
%\begin{figure}
%\centering\vspace{-0.0cm}
%\includegraphics[width=.999\linewidth]{SEvsSNR_10lambda_lowSNR} %SEvsSNR 
%\caption{\angel{This figure is problematic. The low-SNR range of interest is from about 0 dB (or maybe 2-3 dB) down to about -10 dB. Comercial systems cannot operate below. What exactly is the point of the figure anyway?}Spectral efficiency vs SNR for various $\eta$. UPAs with punctiform antennas spaced by $d_\text{t}=d_\text{r} =\lambda/2$ and apertures $L_\text{t}=L_\text{r}=10  \lambda$. 
%Solid lines indicates the ergodic capacity in \eqref{capacity} while dashed lines indicate the low-SNR expansion in \eqref{lowSNR_capacity}, both under isotropic scattering.}\vspace{-0cm}
%\label{fig:SEvsSNR_10lambda_lowSNR}
%\end{figure}


%For SNR ! 0, waterfilling dictates that all the power be allocated onto the strongest sub- channel, which amounts to the precoder concentrating the power along the maximum- eigenvalue eigenvector of H⇤H. This reduces the transmission to a single beam, which is what the term “beamforming” is typically reserved for: a transmission of unit rank, Ns = 1. This definition is the one espoused in this text, regardless of whether the power actually holds an angular beam shape.
%Let us arrange the eigenvalues of $\vect{H}_{\vect{{\sf C}}} \vect{H}_\vect{{\sf C}}^{\Htran}$ in a descending order, such that $\lambda_{\vect{H}_{\vect{{\sf C}}},1} \ge \ldots \ge \lambda_{\vect{H}_{\vect{{\sf C}}},{\sf n}_\text{min}} >0$.

%At low SNR, the precoder allocates all of the power onto the eigenvector of $\vect{H}_{\vect{{\sf C}}} \vect{H}_\vect{{\sf C}}^{\Htran}$ associated with the maximum eigenvalue, yielding an equivalent single-antenna channel with signal-to-noise $\lambda_\text{max}(\vect{H}_\vect{{\sf C}}\vect{H}_\vect{{\sf C}}^{\Htran}) {\sf SNR}$. Thus, \eqref{capacity} expands as \cite[Eq. 5.38]{heath_lozano_2018}
%\begin{align}   \label{lowSNR_capacity}
%C({\sf SNR},\vect{{\sf C}}_\text{r},\vect{{\sf C}}_\text{t}) & =  \Ex\{\lambda_\text{max}(\vect{H}_\vect{{\sf C}}\vect{H}_\vect{{\sf C}}^{\Htran})\} {\sf SNR} 
% \\& \hspace{0.5cm}\notag 
% + \frac{1}{2} \Ex\{\lambda_\text{max}(\vect{H}_\vect{{\sf C}}\vect{H}_\vect{{\sf C}}^{\Htran})\} {\sf SNR}^2
%+ o({\sf SNR}^2).
%\end{align}
 
%Precisely, changing variable in \eqref{spectrum_HC} from the wavenumber to the angular domain, according to \eqref{wavenumber_spherical},} the maximum eigenvalue $\Ex\{\lambda_{\mathcal{H}_\mathcal{C},1}\}$ of $\mathcal{H}_\mathcal{C} \mathcal{H}_\mathcal{C}^*$ (averaged over all channel realizations) is upper bounded by
%\begin{equation} \label{bound_eig_HC_spectra} 
%\Ex\{\lambda_{\mathcal{H}_{\mathcal{C}},1}\} \le \frac{\max \limits_{(\theta_{\text r},\phi_{\text r},\theta_{\text t},\phi_{\text t})} \Ex\{|H^{++}(\theta_{\text r},\phi_{\text r},\theta_{\text t},\phi_{\text t})|^2\}}{\min \limits_{(\theta_{\text r},\phi_{\text r})} |A^+_{\text r}(\theta_{\text r},\phi_{\text r})|^2 \cdot \min \limits_{(\theta_{\text t},\phi_{\text t})} |A^+_{\text t}(\theta_{\text t},\phi_{\text t})|^2}.
%\end{equation}
%{\color{blue}
%More generally, when $\mathcal{H}_\mathcal{C}$ is observed over progressively larger, yet finite, apertures, $\Ex\{\lambda_{\mathcal{H}_\mathcal{C},1}\}$ is bounded by \eqref{bound_eig_HC}, as derived in App.~\ref{app:bound_eig} for $\min(L_\text{t},L_\text{r})/\lambda\gg 1$.
%In the limiting regime $\min(L_\text{t},L_\text{r})/\lambda\to \infty$ the angular sets in \eqref{bound_eig_HC} collapse into singletons correctly yielding \eqref{bound_eig_HC_spectra}.


%\begin{figure*}[t!]
%\begin{align}  \label{bound_eig_HC} \tag{116}
%\Ex\{\lambda_{\mathcal{H}_\mathcal{C},1}\} \le 
%\frac{(L_{\text r} L_{\text t})^2 \max\limits_{i=1, \ldots, {\sf n}_\text{r} {\sf n}_\text{t}} \left(\iiiint_{\Omega_i} \! \Ex\{|H^{++}(\theta_{\text r},\phi_{\text r},\theta_{\text t},\phi_{\text t})|^2\} \sin \theta_{\text r} \sin \theta_{\text t} d\theta_{\text r} d\phi_{\text r} d\theta_{\text t} d\phi_{\text t}\right)}{L_{\text r}^2 \min\limits_{j=1, \ldots, {\sf n}_\text{r}} \left( \iint_{\Omega_{{\text r},j}} |A^+_{\text r}(\theta_{\text r},\phi_{\text r})|^2 \, \sin \theta_{\text r} \, d\theta_{\text r} \, d\phi_{\text r}\right) \cdot L_{\text t}^2 \min\limits_{k=1, \ldots, {\sf n}_\text{t}} \left( \iint_{\Omega_{{\text t},k}} |A^+_{\text t}(\theta_{\text t},\phi_{\text t})|^2 \, \sin \theta_{\text t} \, d\theta_{\text t} \, d\phi_{\text t}\right)}
%\end{align}
%\hrule
%\end{figure*}


%\section{SNR-limiting Regimes} \label{sec:SNR_regimes}


%The isomorphism between coupling and correlation invites drawing insights on what unfolds in limiting SNR regimes \cite{heath_lozano_2018}. Precisely, at low SNRs, coupling is expectedly beneficial as it renders beamforming more effective along preferred directions whereas, at high SNRs, it is detrimental. 
%transmit correlations enable focusing power.
%\andrea{Something must change as coupling appears as a deconvolution with inverse spectrum.}

%{\color{blue}
%\subsection{Asymptotic Eigenvalues} \label{app:asympt_variances}
%
%A large-dimensional approximation, valid for wide apertures, is next leveraged to study the SNR-limiting regimes of the ergodic capacity.
%This approximation reveals the separate effects of array geometry, channel characteristics, and antenna directivity on the composite channel.
%
%Recalling that the size of each spherical surface element $\Omega^+_{{\text t},\vect{j}}$ decreases with growing $L_{{\text t},x}/\lambda$,  the midpoint integration rule could be resorted to for sufficiently smooth angular power spectra of the scattering and antenna in \eqref{variances_channel_composite}, yielding
%\begin{align}  \label{variances_channel_composite_asymptotic}
%\sigma^2_{\vect{j}}(\tilde{{\sf H}}_\text{t}^+)   
%%&  \approx \frac{8 \pi}{Z_0}  \frac{\Ex\{|\tilde{H}^+_\text{t}(\theta_{{\text t},\vect{j}},\phi_{{\text t},\vect{j}})|^2\}}{|{\sf A}_\text{t}^+(\theta_{{\text t},\vect{j}},\phi_{{\text t},\vect{j}})|^2}
%% \iint_{\Omega^+_{{\text t},\vect{j}}} \cos\theta_{\text t} \, \sin \theta_{\text t} \, d\theta_{\text t} d\phi_{\text t}  \\
%%& \label{variances_channel_composite_asymptotic}
%\approx \frac{8 \pi^2}{Z_0}  \frac{\Ex\{|\tilde{H}^+_\text{t}(\theta_{{\text t},\vect{j}},\phi_{{\text t},\vect{j}})|^2\}}{|{\sf A}_\text{t}^+(\theta_{{\text t},\vect{j}},\phi_{{\text t},\vect{j}})|^2} \, {\rm m}({\sf \Omega}^+_{{\text t},\vect{j}})
%\end{align}
%where ${\rm m}({\sf \Omega}^+_{{\text t},\vect{j}})$ is the projected solid angle in \eqref{solid_angles_coupling}.  
%Then, the transmit eigenvalue matrix with coupled antennas in \eqref{tx_corr_coupled} is approximately described by
%\begin{align}
%\vect{{\sf \Lambda}}_{\text{t}} \approx \vect{\mathcal{L}}_{\text{t}} \, \vect{{\sf A}}^{-1}_\text{t}
%\end{align}
%with $\vect{{\sf L}}_{\text{t}}$ and $\vect{{\sf A}}_\text{t}$ being diagonal matrices with entries $\{N_\text{t} {\rm m}({\sf \Omega}^+_{{\text t},\vect{j}}) \Ex\{|H^+_\text{t}(\theta_{{\text t},\vect{j}},\phi_{{\text t},\vect{j}})|^2\}\}$ and $\{|{\sf A}_\text{t}^+(\theta_{{\text t},\vect{j}},\phi_{{\text t},\vect{j}})|^2\}$, respectively.
%Further applying the large-dimensional approximation at the receiver, \eqref{equiv_channel_coupling} can be rewritten as
%\begin{equation} \label{equiv_channel_coupling_approx} 
%\vect{\tilde{{\sf H}}} \approx \vect{\mathcal{L}}_{\text{t}}^{1/2} \vect{W} \vect{\mathcal{L}}_{\text{t}}^{1/2} \vect{{\sf A}}^{-1/2}_\text{t}
%\end{equation}
%where $\vect{{\sf \Lambda}}_{\text{r}} \approx \vect{\mathcal{L}}_{\text{r}}$ with $\vect{\mathcal{L}}_{\text{r}}$ diagonal with coupling-agnostic entries $\{N_\text{r} {\rm m}(\Omega_{{\text r},\vect{i}}) \Ex\{|H^+_\text{r}(\theta_{{\text r},\vect{i}},\phi_{{\text r},\vect{i}})|^2\}\}$.}
%%from which \eqref{equiv_channel_coupling} becomes
%%\begin{equation} \label{equiv_channel_coupling_approx}
%%\tilde{\vect{{\sf H}}} \approx  \tilde{\vect{{\cal H}}} \vect{{\sf A}}^{-1/2}_\text{t}
%%\end{equation}
%%given $\tilde{\vect{{\cal H}}} = \vect{\mathcal{L}}_{\text{r}}^{1/2} \vect{W} \vect{\mathcal{L}}_{\text{t}}^{1/2}$. 
%%The accuracy of \eqref{equiv_channel_coupling_approx} improves with growing $\min(L_\text{r},L_\text{t})/\lambda$, for given channel and antenna spectra, and it is higher with smooth spectra, for fixed apertures. 
%%For $\min(L_\text{r},L_\text{t})/\lambda \to \infty$ and a vanishing antenna spacing, \eqref{equiv_channel_coupling_approx} eventually converges to the spectral product in \eqref{spectrum_HC}.
%% \angel{I need some perspective to appreciate the contribution of this subsection...}

\subsection{Low-SNR Regime}  

%\begin{figure*}[t]
%\begin{align}  \notag
%\Ex \! \left\{ \lambda_\text{max}\big(\tilde{\vect{{\sf H}}} \tilde{\vect{{\sf H}}}^{\Htran}\big) \right\}  & \le
%\underbrace{N_{\text r} N_{\text t} \left(  \mathop{\min}\limits_{\vect{j} \in \Lambda_\text{t}}  
% |{\sf A}_\text{t}^+(\theta_{{\text t},\vect{j}},\phi_{{\text t},\vect{j}})|^2 \right)^{\!-1}}_\text{Array gain} \, 
%\underbrace{\Ex \! \left\{\lambda_\text{max}(\vect{W} \vect{W}^{\Htran}) \right\}}_\text{Fading} 
%  \\& \hspace{4cm} \label{bound_eig_HC_discrete}  \tag{124}
%\cdot 
%\underbrace{\mathop{\max}\limits_{\vect{i} \in \Lambda_\text{r}} 
%{\rm m}({\Omega}_{{\text r},\vect{i}}) 
%\Ex \! \left\{|H^+_\text{r}(\theta_{{\text r},\vect{i}},\phi_{{\text r},\vect{i}})|^2 \right\}}_\text{Receive Correlation}
%\underbrace{\mathop{\max}\limits_{\vect{j} \in \Lambda_\text{t}}
% {\rm m}({\sf \Omega}^+_{{\text t},\vect{j}}) \,
%\Ex \! \left\{|H^+_\text{t}(\theta_{{\text t},\vect{j}},\phi_{{\text t},\vect{j}})|^2 \right\}}_\text{Transmit Correlation and Omnidirectional Coupling}
% \end{align}
% \hrule
% \end{figure*}
 

At low SNR, the precoder allocates the entire power budget to the maximal-eigenvalue eigenspace of $\tilde{\vect{{\sf H}}} \tilde{\vect{{\sf H}}}^{\Htran}$, yielding an equivalent single-antenna channel with signal-to-noise $\lambda_\text{max}(\tilde{\vect{{\sf H}}} \tilde{\vect{{\sf H}}}^{\Htran}) {\sf SNR}$.
Thus, \eqref{capacity} expands as \cite[Eq. 5.38]{heath_lozano_2018}
\begin{align}   \label{lowSNR_capacity}
{\sf C}({\sf SNR}) & =  \Ex\{\lambda_\text{max}(\tilde{\vect{{\sf H}}} \tilde{\vect{{\sf H}}}^{\Htran})\} \, {\sf SNR} 
% \\& \hspace{0.5cm}\notag 
% + \frac{1}{2} \Ex\{\lambda_\text{max}(\vect{H}_\vect{{\sf C}}\vect{H}_\vect{{\sf C}}^{\Htran})\} {\sf SNR}^2
+ o({\sf SNR}).
\end{align}
%Let us arrange the eigenvalues of $\vect{H}_{\vect{{\sf C}}} \vect{H}_\vect{{\sf C}}^{\Htran}$ in a descending order, such that $\lambda_{\vect{H}_{\vect{{\sf C}}},1} \ge \ldots \ge \lambda_{\vect{H}_{\vect{{\sf C}}},{\sf n}_\text{min}} >0$.

%{\color{blue}Let us momentarily ignore coupling whereby the SNR slope in \eqref{lowSNR_capacity} is given by $\Ex\{\lambda_\text{max}(\tilde{\vect{H}} \tilde{\vect{H}}^{\Htran})\}$. Also, specialize the maximum eigenvalue to the case where} transmitter and receiver shrink into a punctiform antenna so that ${\sf n}_\text{r}={\sf n}_\text{t}=1$ in light of {\color{blue}\eqref{Nyquist_cond}; the angular space at each ends of the link is partitioned into one solid angle.}
%{\color{blue}Then, $\tilde{h}_{1} \sim\CN(0,N_\text{r} N_\text{t})$ as per the normalization,} regardless of the scattering conditions, 
%{\color{blue}whereby
%\begin{equation} \label{fully_corr}
%\Ex\{\lambda_\text{max}(\tilde{\vect{H}} \tilde{\vect{H}}^{\Htran})\} = \Ex\{|\tilde{h}_{1}|^2\} = N_\text{r} N_\text{t}.
%\end{equation}}
%The latter value coincides to the maximum eigenvalue in \cite[Example~5.4]{heath_lozano_2018} for a fully correlated channel at both the transmitter and the receiver.
%{\color{blue}Since the SNR slope of a scalar channel is $1$, \eqref{fully_corr} represents the array gain of an $N_\text{r} \times N_\text{t}$ fully correlated uncoupled MIMO channel.}

%As the above expectation %in \eqref{lowSNR_capacity}
%makes general insights difficult,
%we confine ourselves to lossless antennas \angel{But then  why does $\eta$ appear in the bound?} and 
With a view to drawing insights, we resort to the bound derived in Appendix~F, which separates the effect of coupling from the channel characteristics and array geometry, %as reported in \eqref{bound_eig_HC_discrete}.
namely
\begin{align}  \label{bound_eig_HC_discrete}  
\Ex \! \left\{ \lambda_\text{max}\big(\tilde{\vect{{\sf H}}} \tilde{\vect{{\sf H}}}^{\Htran}\big) \right\}  & \le
\underbrace{N_{\text r} N_{\text t}}_\text{Array gain} \,  
\underbrace{\mathop{\max}\limits_{\vect{i} \in \Lambda_\text{r}} 
 \sigma^2_{\vect{i}}(\tilde{{H}}_\text{r}^+)}_\text{Receive Correlation}
  \\ & \quad \notag
\cdot 
\underbrace{\Ex \! \left\{\lambda_\text{max}(\vect{W} \vect{W}^{\Htran}) \right\}}_\text{Fading} 
\!\!\!\!\!\!
\underbrace{\mathop{\max}\limits_{\vect{j} \in \Lambda_\text{t}} \sigma^2_{\vect{j}}(\tilde{{\sf H}}_\text{t}^+).}_\text{Transmit Correlation/Coupling}
 \end{align}
Note that $ \sigma^2_{\vect{i}}(\tilde{{H}}_\text{r}^+)$ and $\sigma^2_{\vect{j}}(\tilde{{\sf H}}_\text{t}^+)$ are inversely proportional to the respective electrical apertures,
%unveiling the zooming effect of increased apertures \cite{Sayeed2002} and
which could suggest that \eqref{bound_eig_HC_discrete} vanishes for an infinite aperture.
However, the number of antennas per dimension increases with the aperture as per \eqref{Nyquist_cond}, ensuring that 
%their product remains well-defined asymptotically. \angel{Not clear what "their" refers to}
the products $N_{\text r} \sigma^2_{\vect{i}}(\tilde{{H}}_\text{r}^+)$ and $N_\text{t} \sigma^2_{\vect{j}}(\tilde{{\sf H}}_\text{t}^+)$ remain finite asymptotically.
%{\color{blue}The upper bound is generally non-achievable as it would require all diagonal entries of $\vect{{\cal L}}_\text{r}$, $\vect{{\cal L}}_\text{t}$, and $\vect{{\sf A}}_\text{t}$ be constant (see App.~\ref{app:max_eigenvalue}). However, this condition is never met, even under isotropic conditions and omnidirectional antennas. Concerning the channel, the non-linearity of the map in \eqref{wavenumber_spherical} renders the spherical surface elements uneven, whereas the $\cos(\cdot)$ at the denominator of \eqref{alpha_coeff} destroys symmetry in the coupling coefficients.
%For a scalar channel, \eqref{bound_eig_HC_discrete} coincides to \eqref{fully_corr} under the same conditions, rendering the upper bound achievable in this particular case.

The first term  in \eqref{bound_eig_HC_discrete}
is the array gain %in uncoupled full-correlation conditions,
with uncoupled omnidirectional  antennas, 
$\lambda_\text{max}\big(\tilde{\vect{H}} \tilde{\vect{H}}^{\Htran}\big) = \|\tilde{\vect{H}}\|_{\text F}^2 =  N_\text{r} N_{\text t}$.
%\begin{align} \label{array_gain}
%\lambda_\text{max}\big(\tilde{\vect{H}} \tilde{\vect{H}}^{\Htran}\big) = \|\tilde{\vect{H}}\|_{\text F}^2 =  N_\text{r} N_{\text t}
%\end{align}
%as per the channel normalization, specialized in \eqref{normalization_channel}, e.g., at the transmitter.
%Expectedly, a linear increases in low-SNR capacity is due to increased antenna density.}
%The term describing correlation represents a cost compared to the fully correlated case. Indeed, this term is less than unity as per the channel normalization; the upper limit is achievable under 
%isotropic conditions, whereby the angular power distribution is $1$, and 
%vanishing apertures to which correspond a unitary angular power distribution and a spherical surface element covering the entire upper hemisphere.
%\frac{N_\text{r}}{2\pi}  \max_{i=1, \ldots, {\sf n}_\text{r}} \iint_{\Omega_{{\text r},\vect{i}}}  \Ex\{|H^{+}_\text{r}(\theta_{\text r},\phi_{\text r})|^2\}  \sin \theta_{\text r} \, d\theta_{\text r} d\phi_{\text r} \frac{N_\text{t}}{2\pi} \max_{j=1, \ldots, {\sf n}_\text{t}} {\rm m}(\Omega^+_{{\text t},\vect{j}}) \, \Ex\{|H^+_\text{t}(\theta_{{\text t},\vect{j}},\phi_{{\text t},\vect{j}})|^2\} 
%Momentarily focusing on the numerator of \eqref{bound_eig_HC_discrete}, the one characterizing the channel, let both the transmitter and receiver shrink into a punctiform antenna so that ${\sf n}_\text{r}={\sf n}_\text{t}=1$ in light of \eqref{n_rt}. 
%Then, $\sigma^2_1(\mathcal{H}) = 1$ as per the normalization whereby $\widehat{\lambda}_1(\mathcal{H}) \sim\CN(0,1)$, regardless of the scattering conditions. 
%%The conjugate product of two independent standard complex Gaussians yields a unit-variance chi-square random variable with two degrees of freedom, namely $\widehat{\lambda}_\text{max}(\mathcal{H} \mathcal{H}^*) = \widehat{\lambda}_1(\mathcal{H} \mathcal{H}^*) \sim \chi_2^2$.
%Thus, $\Ex\{\widehat{\lambda}_\text{max}(\mathcal{H} \mathcal{H}^*)\} = \Ex\{|\widehat{\lambda}_1|^2\} = 1$ implying the numerator of \eqref{bound_eig_HC_discrete} equals $N_\text{t} N_\text{r}$. The latter value coincides to the maximum eigenvalue $\Ex\{\lambda_\text{max}(\vect{H}\vect{H}^{\Htran})\} = N_\text{t} N_\text{r}$ in \cite[Example~5.4]{heath_lozano_2018} for a fully correlated channel at both the transmitter and the receiver, arising with vanishing antenna spacing under the same normalization $\Ex\{\|\vect{H}\|_{\text F}^2\} = N_\text{t} N_\text{r}$.
The remaining terms in \eqref{bound_eig_HC_discrete} incorporate the influence of %describe how directivity is influenced by
 fading correlation and transmit coupling. Some considerations:

\begin{itemize}
\item
Expectedly, stronger correlation is beneficial at low SNR for a given number of uncoupled antennas, as preferred directions arise that beamforming can exploit \cite{heath_lozano_2018}.
%Omitting coupling in  \eqref{bound_eig_HC_discrete}, in the wide-aperture regime,
For large electrical apertures and a smooth fading spectrum,
\begin{align} \label{sigma2_asympt}
\sigma^2_{\vect{j}}(\tilde{{H}}_\text{t}^+) \approx |{\Omega}^+_{\vect{j}}| \, \Ex\{|H^{+}_\text{t}(\theta_{{\text t},\vect{j}},\phi_{{\text t},\vect{j}})|^2\}
\end{align}
after applying the midpoint integration rule to \eqref{variances_channel}, with \eqref{midpoint} and \eqref{solid_angles} substituted. Correlation increases \eqref{sigma2_asympt} and, in turn, \eqref{bound_eig_HC_discrete} as per the normalization in \eqref{normalization_channel}.
\item
%The beamforming can be further enhanced by leveraging coupling. 
%This is because channel selectivity actually hinges on the composition between scattering and coupling.
Correlation can be further enhanced by leveraging coupling.
From \eqref{variances_channel_composite}, specifically, 
\begin{align} \label{sigma2_coupling_asympt}
\sigma^2_{\vect{j}}(\tilde{{\sf H}}_\text{t}^+) \approx |{\sf \Omega}^+_{\vect{j}}| \frac{\Ex\{|H^{+}_\text{t}(\theta_{{\text t},\vect{j}},\phi_{{\text t},\vect{j}})|^2\}}{|A^{+}_\text{t}(\theta_{{\text t},\vect{j}},\phi_{{\text t},\vect{j}})|^2} ,
\end{align}
%with ${\rm m}({\sf \Omega}^+_{{\text t},\vect{j}})$ the projected solid angles in \eqref{solid_angles_coupling}.
which is maximized when the antenna pattern is lowest on directions on which the fading is strongest (see Fig.~\ref{fig:decorrelation_2b}).
%The ensuing beamforming gain must be weighted against the power loss arising from not pointing the pattern towards fading directions. 
%{\color{red}The array gain shall improve if the increase of the maximal eigenvalue brought about by a stronger correlation exceeds the reduction caused by the pattern not aiming directly at the fading directions.} (Correlation causes the maximal eigenvalue to increase relative to the rest, while a power loss diminishes all of the eigenvalues.)
%\angel{But there's a loss of power not captured by SNR, we should make sure it's not lost in the eigenvalue normalizations...}
%\andrea{Possibly. Both the functions at numerator and denominator are normalized such that they integrate to 1, but the ratio of the two functions do not. (I wonder if there is a form of Cauchy-Schwartz inequality for that: Titu's lemma?) However, we only plot normalized eigenvalues in this paper.}
%\angel{SNR losses could be absorbed into the "G" within the SNR definition. On a related note, and apologies for some of these questions, I'm just try to make sense of certain things: how are we accounting for the fact that the antenna patterns, besides coupling, also affect the uncoupled correlation?} \andrea{We cannot. Uncoupled antennas are not physically tenable as they would require a constant 3D pattern. However, antenna patterns are inherently constrained on a 3D spherical support.} \andrea{The SNR losses should be captured by the variances of Fourier coefficients.}
% (see Fig.~\ref{fig:channel_var_coupled_selectivity}).
%Thus, $|A^{+}_\text{t}(\theta_{{\text t},\vect{j}},\phi_{{\text t},\vect{j}})|^2$ can be regarded as the power density of the interference generated by coupling among neighboring antennas, with \eqref{sigma2_coupling_asympt} maximized by setting the "nulls" of this spectrum in correspondence to these preferred directions.
\item
%The lowest value in \eqref{sigma2_coupling_asympt} is attained when the antenna pattern closely matches the fading spectrum, yielding $\sigma^2_{\vect{j}}(\tilde{{\sf H}}_\text{t}^+) \approx |{\sf \Omega}^+_{\vect{j}}|$, which may be even lower than the uncoupled value in \eqref{sigma2_asympt}.
%If so, coupling decorrelates the antennas, diminishing the selectivity (see Fig.~\ref{fig:channel_var_coupled}).
Coupling can alternatively decorrelate antennas, diminishing the selectivity (see Figs.~\ref{fig:channel_var_coupled_iso} and~\ref{fig:decorrelation_2a}).
The minimum of \eqref{sigma2_coupling_asympt} is attained by antenna patterns closely matching the fading spectrum, resulting in $\sigma^2_{\vect{j}}(\tilde{{\sf H}}_\text{t}^+) \approx |{\sf \Omega}^+_{\vect{j}}|$, which may be lower than the uncoupled value in \eqref{sigma2_asympt}.
\item
The highest value in \eqref{sigma2_coupling_asympt} is dictated by the maximum selectivity. 
For a fixed aperture, a trade-off arises between array gain, increasing with antenna densification, and antenna selectivity, requiring larger structures as dictated by the uncertainty principle.
\item
As the number of antennas grows with the electrical aperture,
%\angel{You should decide between "normalized aperture" and "electrical aperture" and use only one in the entire paper. I'd favor latter, since there are multiple normalizations.}, 
a plural multiplicity of $\lambda_\text{max}(\tilde{\vect{{\sf H}}} \tilde{\vect{{\sf H}}}^{\Htran})$ arises due to the eigenvalue polarization \cite{Franceschetti,PizzoWCL22,HeedongIRS}. Then, low-SNR optimality entails multiple equal-power transmissions on each of those maximal-eigenvalue  eigenvectors.
\end{itemize}


%The lowest value in \eqref{bound_eig_HC_discrete} is attained by an isotropic channel with omnidirectional antennas,
%\begin{align} \label{lambda_max_iso_omni} 
%\Ex\{\lambda_\text{max}(\tilde{\vect{{\sf H}}} \tilde{\vect{{\sf H}}}^{\Htran})\} &  \le
%N_{\text r} N_{\text t} \, \Ex\!\left\{\lambda_\text{max}(\vect{W} \vect{W}^{\Htran}) \right\}   \\& \hspace{2cm}  \nonumber
%\cdot \max_{\vect{i} \in \Lambda_\text{r}} {\rm m}({\Omega}_{{\text r},\vect{i}}) \max_{\vect{j} \in \Lambda_\text{t}} {\rm m}({\sf \Omega}^+_{{\text t},\vect{j}})
%%\\& \hspace{1cm} \notag
%%\mathop{\max}\limits_{i=1, \ldots, {\sf n}_\text{r}} {\rm m}(\Omega_{{\text r},\vect{i}})  
%%\mathop{\max}\limits_{j=1, \ldots, {\sf n}_\text{t}}  {\rm m}(\Omega^+_{{\text t},\vect{j}})
%\end{align}
%\angel{How about the coupling?}
%\andrea{Say that coupling appears through the projection inside the solid angles..}
%{\color{blue}with ${\rm m}({\Omega}_{{\text r},\vect{i}})$ and ${\rm m}({\sf \Omega}^+_{{\text t},\vect{j}})$ the unprojected solid angles and projected solid angles in \eqref{solid_angles} and \eqref{solid_angles_coupling}, respectively. }
%found in correspondence to the spherical surface elements subtending the widest area, and fully embedding coupling in \eqref{lambda_max_iso_omni}.
%broadside directions $\theta_{{\text r},i}=0$ and $\theta_{{\text t},j}=0$

%\angel{Does this quantify what happens as we separate the omni antennas?}
 
\subsection{High-SNR Regime} \label{sec:DOF_HighSNR}

%\begin{figure}
%\centering\vspace{-0.0cm}
%\includegraphics[width=.999\linewidth]{capacity_highSNR} 
%\caption{${\sf C}({\sf SNR})$ under various conditions. UPAs with punctiform antennas spaced by $\lambda/2$ and apertures $20\lambda$ $\eta=0.01$. %Also shown is the spectral efficiency with uniform power allocation. 
%Solid lines indicate the simulations with the exact model in \eqref{channel_samples_coupled}, circles indicate simulations with the Fourier model in \eqref{MIMO_channel_coupling}, dashed line indicates high-SNR expansion in \eqref{C_SNR_high}.}
%\label{fig:capacity_highSNR}
%\end{figure}

%\begin{figure}
%\centering\vspace{-0.0cm}
%\includegraphics[width=.999\linewidth]{SEvsSNR_10lambda_highSNR} %SEvsSNR 
%\caption{Spectral efficiency vs SNR for various $\eta$. UPAs with punctiform antennas spaced by $d_\text{t}=d_\text{r} =\lambda/2$ and apertures $L_\text{t}=L_\text{r}=10  \lambda$.
%The solid line indicates the ergodic capacity while the dashed lines indicate its high-SNR expansion, both under isotropic scattering.}
%\label{fig:SEvsSNR_10lambda_highSNR}
%\end{figure}

%\begin{figure}
%\centering\vspace{-0.0cm}
%\includegraphics[width=.999\linewidth]{SEvsSNR} 
%\caption{Spectral efficiency as a function of ${\sf SNR}$ under isotropic conditions. ULAs with omnidirectional antennas ($\eta=0.99$) spaced by $d_\text{r}=\lambda/2$ and $d_\text{t}= 0.3 \lambda$ and apertures $L_\text{r} = 5 \lambda$ and $L_\text{t} = 15 \lambda$.}%\angel{Not discussed in the text, yet. Observations?}}
%\label{fig:SEvsSNR}
%\end{figure}

At high SNR, with probability 1 \cite[Eq. 5.29]{heath_lozano_2018}, 
%The high-SNR regime is characterized by a uniform power allocation for every channel realization, whereby \eqref{capacity} expands with probability 1 as \cite[Eq. 5.29]{heath_lozano_2018} 
%\angel{The capacity is indicated to have an explicit dependence on the coupling matrices, but then these are hidden on the right-hand side}
\begin{align}  \label{C_SNR_high}
%{\sf C}({\sf SNR}) & =  {\sf S}_\infty^\star(\tilde{\vect{{\sf H}}}) \left(\log_2 {\sf SNR} - {\sf L}_\infty^\star(\tilde{\vect{{\sf H}}}) \right) + \mathcal{O}(1/{\sf SNR}) 
{\sf C}({\sf SNR}) & =  {\sf DOF} \, \log_2 {\sf SNR} + \mathcal{O}(1) ,
\end{align}
%\angel{No need for outer expectation of the power offset throughout this section, it already features an expectation within}
 %${\sf S}_\infty^\star(\tilde{\vect{{\sf H}}})$ denotes the high-SNR slope in bit/s/Hz,
 where %the dependence of the right-hand side on coupling through $\tilde{\vect{{\sf H}}}$ is hidden within
 %${\sf DOF}$ is the channel's rank, %number of spatial DOF. 
% {\color{magenta}within the limit dictated by uncoupled antennas.}
 %\andrea{Talking about DOF in general may be confusing.}
 %\angel{Perhaps we can do without the start superscripts.}
% {\color{blue}Here, the dependence of the right-hand side on $N_\text{r}$ and $N_\text{t}$ is omitted as not affecting the DOF under \eqref{Nyquist_cond} and only entering the capacity through the zero-order term.}
%\begin{align}
%{\sf S}_\infty^\star(\tilde{\vect{{\sf H}}}) = \lim_{{\sf SNR} \to \infty} \frac{{\sf C}({\sf SNR})}{\log_2  {\sf SNR}},
%\end{align}
%whereas ${\sf L}_\infty^\star(\tilde{\vect{{\sf H}}})$ is the power-offset, given by
%\begin{align} \label{L_infty}
%{\sf L}_\infty^\star(\tilde{\vect{{\sf H}}}) = 
%\lim_{{\sf SNR} \to \infty} \left( \log_2  {\sf SNR} - \frac{{\sf C}({\sf SNR})}{{\sf S}_\infty^\star(\tilde{\vect{{\sf H}}})}  \right).
%%\log_2 {\sf n}_\text{min} - \frac{1}{{\sf n}_\text{min}} \Ex\left\{\sum_{i=1}^{{\sf n}_\text{min}}\log_2 \lambda_i(\vect{H}_\vect{{\sf C}} \vect{H}_\vect{{\sf C}}^{\Htran})\right\}
%\end{align}
%Here, ${\sf n}_\text{min} \log_2 {\sf SNR}$ is the high-SNR capacity of a bank of ${\sf n}_\text{min}$ parallel single-antenna channels with a signal-to-noise of ${\sf SNR}$, whereas $\Ex\{{\sf L}_\infty\}$ specifies the SNR shift, in units of $3$~dB, with respect to the baseline, average over all channel realizations \cite{Lozano_HighSNR}.
%At high SNR, coupling affects the capacity through %the prelog factor
%${\sf S}_\infty^\star$, which quantifies the number of spatial DOF, and as a pre-constant that anchors the expansion. % through the coupling matrices and the composite channel. 
(see Appendix~G) %this number boils down to the minimum spatial-spectral support of the %angular power density of the  composite channel at either end of the link, 
\begin{equation} \label{S_inf_UIU_rank}
{\sf DOF} = \min({\sf DOF}_\text{r}^\prime,{\sf DOF}_\text{t}^\prime), % = {\sf n}_\text{min}^\prime
\end{equation}
with ${\sf DOF}_{\text r}^\prime$ and ${\sf DOF}_{\text t}^\prime$ as defined in \eqref{DOF}, but with ${\sf n}_\text{r}^\prime \le {\sf n}_\text{r}$ and ${\sf n}_\text{t}^\prime \le {\sf n}_\text{t}$ incorporating the effects of fading and coupling via
%\begin{align} \label{n_r_prime}
%{\sf n}_\text{r}^\prime & \approx  {\sf n}_\text{r}  \cdot \frac{1}{\pi} \!\iint_{\|\vect{k}\| \le 1} \!\!\! \mathbbm{1}_{{\rm supp}(\Ex\{|\tilde{H}^+_\text{r}|^2\})}  d\vect{k} \\  \label{n_t_prime}
%{\sf n}_\text{t}^\prime & \approx  {\sf n}_\text{t}  \cdot \frac{1}{\pi} \!\iint_{\|\vect{\kappa}\| \le 1} \!\!\! \mathbbm{1}_{{\rm supp}(\Ex\{|\tilde{{\sf H}}^+_\text{t}|^2\})}  d\vect{\kappa}
%\end{align}
\begin{align}   \label{n_r_prime}
{\sf n}_\text{r}^\prime & =  \left\lceil {\sf n}_\text{r}  \cdot \frac{1}{\pi} \!\iint_{{\rm supp}(\Ex\{|\tilde{H}^+_\text{r}|^2\})} \!\!\!  \cos \theta_\text{r} \, \sin \theta_\text{r} \, d\theta_\text{r} \, d\phi_\text{r} \right\rceil \\  \label{n_t_prime}
{\sf n}_\text{t}^\prime & =  \left\lceil {\sf n}_\text{t}  \cdot \frac{1}{\pi} \!\iint_{{\rm supp}(\Ex\{|\tilde{{\sf H}}^+_\text{t}|^2\})} \!\!\!  \cos \theta_\text{t} \, \sin \theta_\text{t} \, d\theta_\text{t} \, d\phi_\text{t} \right\rceil.
\end{align}
Here, ${\sf n}_\text{r}$ and ${\sf n}_\text{t}$ represent the leading terms of the receive and transmit DOF under isotropic scattering and uncoupled antennas in \eqref{n_rt}, as shown by
%Under isotropic conditions, the foregoing integrals correctly yield one, by virtue of
\begin{align}  \label{DOF_Landau_omni}
\frac{1}{\pi} \int_0^{\pi/2} \int_0^{2\pi} \cos \theta \, \sin \theta \, d\theta \, d\phi = 2 \int_0^1 y \, dy = 1
\end{align}
using $y=\sin\theta$.
The terms ${\sf n}_\text{r}$ and ${\sf n}_\text{t}$ are multiplied by the respective fractions made available due to the channel selectivity and array orientation at each end of the link, which are tantamount to an angular support reduction and a broadside projection. % in the corresponding normalized surface integrals.
%\andrea{Probably worth specifying that the cos(.) terms in \eqref{n_r_prime} and \eqref{n_t_prime} are unrelated to the cos(.) appearing in the computation of the variances in \eqref{variances_channel_composite}. The latter term is due to coupling while the former terms are simply due to change of variables from wavenumber to spherical (consistent with \cite{PoonDoF}).}
We note that the $\cos(\cdot)$ terms in \eqref{n_r_prime} and \eqref{n_t_prime} result from the change of variables from wavenumber to spherical as per \eqref{wavenumber_spherical} \cite{PoonDoF}. In contrast, the same term in \eqref{variances_channel_composite} arises due to coupling.
For a narrow selectivity and an array oriented such that the scattering is broadside ($\cos \theta_\text{r} \approx 1$),
\begin{align}   \label{n_r_prime_broadside}
{\sf n}_\text{r}^\prime & \approx  {\sf n}_\text{r}  \cdot \frac{2}{\pi} \, \left|{\rm supp}(\Ex\{|\tilde{H}^+_\text{r}|^2\})\right|
\end{align}
at the receiver.
Instead, for an array oriented such that the scattering is at the endfire direction ($\cos \theta_\text{r} \approx \theta_\text{r} - \pi/2$),
\begin{align}   \label{n_r_prime_endfire}
{\sf n}_\text{r}^\prime & \approx  {\sf n}_\text{r}  \cdot \frac{1}{\pi} \!\iint_{{\rm supp}(\Ex\{|\tilde{H}^+_\text{r}|^2\})} \!\!\!  (\pi/2 - \theta_\text{r}) \, \sin \theta_\text{r} \, d\theta_\text{r} \, d\phi_\text{r}.
\end{align}
%for  the linear array is oriented such that the scattering cluster is at the broadside4, then
%over the upper hemisphere
%In addition to selectivity, the DOF reductions embed the effect of array orientations through broadside projections at both link ends. 
% through the surface integral, each computing the %area projection of the angular scattering onto the broadside direction ($\theta_{\text t}=0$). 
%projected area subtended by the power angular spectrum of the channel as viewed from the corresponding antenna array.
%It reduces to the solid angle $\iint_{{\rm supp}(|\tilde{H}^+_\text{r}|^2)} \sin \theta_{\text r} \, d\theta_{\text r} d\phi_{\text r}$ subtended by the scattering for a sufficiently narrow support that is centered about the broadside direction, whereby $\cos \theta_\text{r}\approx 1$.
%
%The impact of coupling materializes in \eqref{n_t_prime} through a spectral multiplication by $1/|{\sf A}^+_\text{t}(\vect{\cdot})|^2$ in the composite channel in \eqref{spectrum_channel_transmitter_coupling}.
%Let us focus on the transmit DOF, which are influenced by coupling through the composite channel spectrum in \eqref{spectrum_channel_transmitter_coupling}. As coupling enters their formulation through a spectral multiplication by $1/|{\sf A}^+_\text{t}(\vect{\kappa})|^2$, due to the boundedness of ${\sf A}^+_\text{t}(\vect{\cdot})$ the support remains unchanged, namely ${\rm supp}(\Ex\{|\tilde{{\sf H}}^+_\text{t}|^2\}) = {\rm supp}(\Ex\{|\tilde{{H}}^+_\text{t}|^2\})$. Coupling is thus immaterial to the spatial DOF, which
Due to the antenna pattern boundedness, the support is unchanged by coupling: ${\rm supp}(\Ex\{|\tilde{{\sf H}}^+_\text{t}|^2\}) = {\rm supp}(\Ex\{|\tilde{{H}}^+_\text{t}|^2\})$. Transmit coupling is thus immaterial to ${\sf n}^\prime_\text{t}$, which %the spatial DOF, which
%, namely
%\begin{align} \label{n_r_primeprime}
%{\sf n}_\text{t}^\prime & \approx  {\sf n}_\text{t}  \cdot \frac{1}{\pi} \!\iint_{\|\vect{\kappa}\| \le 1} \!\!\! \mathbbm{1}_{{\rm supp}(\Ex\{|\tilde{{H}}^+_\text{t}|\})}  d\vect{\kappa}
%\end{align}
hinges solely on the scattering selectivity and array orientation. (As mention at the onset of the section, DOF beyond the uncoupled limit at ${\sf n}_{\text t}$ are excluded from this analysis.) % as the focus is on the first term of Landau's formula, namely the spatial bandwidth.)
%\angel{Perhaps somewhere in this section we have to restate that we are ignoring the extra DOF, as per the Fourier model, to prevent readers from interpreting a contradiction}

 % at the transmit side. 
%\andrea{I'd remove this comment.}
%{\color{red}However, the DOF may still be influenced by the antenna pattern through the sublinear term in \eqref{DOF}.}
%\angel{Isn't the above equation the same as \eqref{n_t_prime}?}
%
%We hasten to re-emphasize that DOF beyond those specified by \eqref{S_inf_UIU_rank} are theoretically possible by expanding the spectral disk into the evanescent (non-stationary) propagation region. In this regime, coupling could be leveraged to render these extra channel dimensions usable, provided that the associated eigenvalues are not negligible. 
%
%Due to the duality of Fourier transform,
%\andrea{I'd remove the paragraph below. I guess the devil is in the definition of the rank.}
%{\color{red}The same conclusion can be drawn in the spatial domain, with the spectral product in \eqref{spectrum_HC} mapping to the discrete  convolution in \eqref{channel_samples_coupled}.  This convolution materializes as a multiplication of the uncoupled MIMO channel  by the positive-definite matrix $\vect{{\sf C}}_\text{t}^{-1/2}$, which leaves the rank unchanged \cite[Ch.~7]{HornBook}.} % it is concluded that coupling is immaterial to the DOF. 
%\angel{This paragraph seems to contradict the previous one...}
%\begin{equation}
%{\sf n}_{\text{t},\epsilon}^\prime = \min\{i: \lambda_{i}(\vect{\tilde{{\sf H}}} \vect{\tilde{{\sf H}}}^{\Htran}) \le \epsilon({\sf SNR}), \epsilon({\sf SNR}) > 0\},
%\end{equation}

%\angel{This next paragraph gives the impression that you're trying to have your cake and eat it too. I'd remove it and, for this paper, embrace the strict notion of DOF (rank). If and when we write a follow-up dealing with the power offset, we can get into the nuances of how coupling affects the DOF that are usable at some finite SNR. In fact, one can compute the power offset for different effective DOF, thereby anchoring high-SNR expansions for successive SNR intervals. This is something we played with for correlations, but didn't complete because we got busy with other stuff.}

Backing off from ${\sf SNR} \to \infty$, the spectral supports become SNR-dependent. By strengthening the eigenvalue polarization (see Fig.~\ref{fig:channel_var_coupled_iso}), coupling can increase the number of spatial dimensions that are usable at high---but finite---SNRs. 
Put differently, coupling cannot enhance the asymptotic slope of the capacity versus log-SNR function, but it can steepen that slope at high---but finite---SNRs. Besides the non-asymptotic slope, coupling is certain to affect the power offset, the zero-order term in the refinement of the expansion in \eqref{C_SNR_high}. The power offset anchors the expansion, and follow-up work is needed to quantify how coupling affects it.
%Follow-up work is needed to quantify the impact of coupling on the power offset, the zero-order term in the refinement of the expansion \eqref{C_SNR_high}.
%\angel{I modified this paragraph to avoid using "DOF" non-asymptotically, which would be inappropriate. Edit at will.}

%More specifically, waterfilling allocates power unevenly, making the above spectral supports SNR-dependent and resulting in an effective number of DOF per SNR interval. Coupling can enhance DOF usability at high SNR by strengthening eigenvalue polarization (see Figs.~\ref{fig:channel_var_coupled} and \ref{fig:channel_var_coupled_selectivity}). Follow-up work is needed to quantify the impact of coupling on the power offset, the zero-order term in the refinement of the expansion \eqref{C_SNR_high}.


%{\color{blue}
%This adjustment renders the first-order high-SNR capacity in \eqref{C_SNR_high} less dependable on the DOF and more on the SNR and the zero-order term. 
%In this context, coupling can alter the effective DOF by shaping the composite channel spectrum via \eqref{spectrum_HC}, enhancing DOF usability at a given SNR. For example, matching each antenna's spectrum to the fading spectrum creates a uniform eigenvalue distribution, effectively increasing the DOF available through scattering.
%Equivalently, in the spatial domain, the foregoing spectrum inversion corresponds to a whitening operation applied to the correlation by $\vect{{\sf C}}_\text{t}^{-1/2}$.}
%
%\angel{Perhaps we can add, in lieu of the previous paragraph, a sentence indicating that we expect the impact of coupling at high SNR to be captured by the power offset, the zero-order term in the refinement of the expansion \eqref{C_SNR_high}. This term will be the object of follow-up work.}
{\color{blue}

\subsection{Impact of Coupling on the Ergodic Capacity}

\begin{figure}
\centering\vspace{-0.0cm}
\includegraphics[width=.999\linewidth]{capacity_highSNR} 
\caption{Ergodic capacity of the exact channel versus SNR for various $\rho$ under isotropic scattering at transmitter and IID fading at receiver. UPAs with apertures $15 \lambda$ and antenna spacing $0.4 \lambda$ along each planar dimension. The transmit antennas have an omnidirectional pattern (decorrelating the fading) while the receive antennas are uncoupled.}
%Spatial DOF augmentation for various $\rho$ under isotropic scattering for the exact channel in \eqref{numerical_inv}. Omnidirectional antennas (matched to the fading) spaced by $\lambda/4$ and aperture $10 \lambda$.}
\label{fig:capacity_highSNR}
\end{figure}


To gauge the impact of eigenvalue polarization and DOF augmentation on the capacity, the exact $\vect{{\sf H}}$ in \eqref{channel_samples_coupled} is applied. 
Communication occurs between square UPAs of apertures $15 \lambda$ with antennas spaced by $0.4 \lambda$ along each planar dimension. 
IID fading at the receiver emphasizes the interplay between coupling and correlation at the transmitter, yielding, from \eqref{channel_samples_coupled},
%in addition to the receive antennas being uncoupled, resulting in a delta function impulse response. 
 \begin{align} \label{channel_samples_coupled_iidrx}
\vect{{\sf H}}(\rho) = \sqrt{\frac{2}{{\sf R}}} \, \vect{H}_\text{IID} \vect{R}_\text{t}^{1/2} \vect{{\sf C}}_\text{t}^{-1/2}(\rho)
\end{align}
where $\vect{H}_\text{IID} \in \Complex^{N_\text{r} \times N_\text{t}}$ has IID standard complex Gaussian entries. Here, recall, $\vect{R}_\text{t}$ and $\vect{{\sf C}}_\text{t}(\rho)$ denote the transmit correlation and coupling matrices, as defined in \eqref{corr_exact} and  \eqref{coupling_loss}.
Illustrated in Fig.~\ref{fig:capacity_highSNR} is the ergodic capacity of the exact model, %${\sf C}({\sf SNR},\rho)$,
obtainable from \eqref{capacity} by substituting $\vect{\tilde{{\sf H}}}$ with $\vect{{\sf H}}(\rho)$ and ${\sf n}_\text{min}$ with $N_\text{min} = \min(N_\text{r}, N_\text{t})$, as a function of ${\sf SNR}$ and for varying $\rho$.
The transmit antennas exhibit an omnidirectional pattern, leading to an optimal high-SNR design through fading decorrelation (recall Fig.~\ref{fig:decorrelation_2a}). 
The capacity curves are benchmarked against the capacity of the IID channel %, having fully decorrelated samples
and of the uncoupled yet correlated channel (i.e., $\vect{{\sf C}}_\text{t}(0) = \vect{I}_{N_\text{t}}$). 
%Compared to uncoupled antennas, \angel{not clear what "Compared to uncoupled antennas" means here} 
Coupling-induced decorrelation reduces low-SNR capacity while enhancing high-SNR capacity,  %, for a given number of antennas.
with the transition between these regions marked by the intersection of the capacity curves with the uncoupled curve in Fig.~\ref{fig:capacity_highSNR}, for every $\rho$; lower loss factors $\rho$ require a smaller SNR for this crossing and asymptotically approach the IID capacity limit as $\rho \to 0$.
Additional simulations reveal that coupling effects amplify with smaller antenna spacings.}

%Letting $\min(L_{\text{r},x},L_{\text{t},x})/\lambda\gg 1$, the eigenvalues of $\tilde{\vect{{\sf H}}} \tilde{\vect{{\sf H}}}^{\Htran}$ polarize into two levels %as per spectral concentration
%\cite{FranceschettiBook}; the transition within an $\epsilon$ accuracy occurs at
%\begin{equation} \label{effective_rank}
%\rank_\epsilon(\tilde{\vect{{\sf H}}}) = {\sf n}^\prime(\tilde{\vect{{\sf H}}}) + o \! \left(\log \frac{L_\text{t}}{\lambda} \right),
%\end{equation}
%with the  right-hand side depending on $\epsilon$
%%that is omitted as only appearing as a pre-constant $\log(1/\epsilon)$ as $\epsilon\to 0$ of the second-order term
%only through the term
%$o \! \left(\log \frac{L_\text{t}}{\lambda} \right)$.
%
%The foregoing integrals correspond to the spatial bandwidths (in units of $1/$m$^2$) of the space-variant channel \cite{FranceschettiBook}, offering an embodiment of the time-domain DOF result that applies here in the spatial domain \cite{PizzoWCL22}.

%The impact of coupling on ${\sf S}_\infty^\star(\tilde{\vect{{\sf H}}})$ is confined to the transmit side through the wavenumber support of $\Ex\{|{\sf H}^+_\text{t}(\vect{\kappa})|\}$. Recalling \eqref{spectrum_channel_transmitter_coupling} and invoking the boundedness of $A^+_\text{t}(\vect{\kappa})$ $\forall \vect{\kappa}$, the spatial bandwidth at the transmit side is not influenced by the antenna directivity, whereby coupling is immaterial to the spatial DOF, yielding
%%The antenna directionality is immaterial to the spatial DOF as the spatial bandwidth is not influenced by it. 
%%\angel{This is the first time that the notion of spatial bandwidth appears, and it does so out of the blue...}
%\begin{align}
%{\sf n}_0(\vect{{\sf R}}_\text{t}) = L_\text{t}^{2}  \iint_{{\rm supp}(\Ex\{|H^+_\text{t}|\})}  \frac{d\vect{\kappa}}{(2\pi)^2} = {\sf n}_0(\vect{R}_\text{t}).
%\end{align}
%\angel{I would streamline the material below and get quickly and without math to the identity between the DOF with and without coupling, which places the spotlight on the power offset}
%{\color{blue}
%Nevertheless, the intrinsic dependance of this metric on the accuracy level appears only as a second-order term, analogously to \eqref{effective_rank}, which progressively weakens as both the electrical apertures increase.
%In the regime $\min(L_\text{r},L_\text{t})/\lambda \gg 1$, we thus have
%\begin{equation} \label{S_inf_UIU_asympt}
%{\sf S}_\infty^\star(\tilde{\vect{{\sf H}}}) \approx {\sf n}_\text{min}^\prime = \min({\sf n}_\text{r}^\prime,{\sf n}_\text{t}^\prime)
%\end{equation}
%The impact of coupling on ${\sf S}_\infty^\star(\tilde{\vect{{\sf H}}})$ is confined to the transmit side through the wavenumber support of $\Ex\{|{\sf H}^+_\text{t}(\vect{\kappa})|\}$. Recalling \eqref{spectrum_channel_transmitter_coupling} and invoking the boundedness of $A^+_\text{t}(\vect{\kappa})$ $\forall \vect{\kappa}$, the spatial bandwidth at the transmit side is not influenced by the antenna directivity, whereby coupling is immaterial to the spatial DOF, yielding
%%The antenna directionality is immaterial to the spatial DOF as the spatial bandwidth is not influenced by it. 
%%\angel{This is the first time that the notion of spatial bandwidth appears, and it does so out of the blue...}
%\begin{align}
%{\sf n}_0(\vect{{\sf R}}_\text{t}) = L_\text{t}^{2}  \iint_{{\rm supp}(\Ex\{|H^+_\text{t}|\})}  \frac{d\vect{\kappa}}{(2\pi)^2} = {\sf n}_0(\vect{R}_\text{t}).
%\end{align}
%This observation, nonetheless, still lacks a connection between the impact of coupling on the required power for a certain capacity, otherwise captured by the power offset.

%The foregoing analysis of the high-SNR slope places the spotlight on the power offset ${\sf L}_\infty^\star$, whose computation is unfolded in App.~\ref{app:power_offset}.
%
%For ${\sf n}_\text{t}^\prime > {\sf n}_\text{r}^\prime$, 
%\begin{align}  \notag
%{\sf L}_\infty^\star(\tilde{\vect{{\sf H}}}) & \approx
% \log_2 {\sf n}_\text{r}^\prime 
%-\frac{1}{{\sf n}_\text{r}^\prime} \Ex \left\{ \log_2 \det(\vect{W}^\prime \vect{{\sf \Lambda}}_{\text{t}}^\prime  (\vect{W}^\prime)^{\Htran}) \right\} \\&  \label{L_infty_app_nr} \hspace{2cm}
%-\frac{1}{{\sf n}_\text{r}^\prime}  \log_2 \det(\vect{\Lambda}_{\text{r}}^\prime) 
%\end{align}
%For ${\sf n}_\text{t}^\prime < {\sf n}_\text{r}^\prime$, 
%\begin{align}  \notag
%{\sf L}_\infty^\star(\tilde{\vect{{\sf H}}}) & \approx
% \log_2 {\sf n}_\text{t}^\prime 
%-\frac{1}{{\sf n}_\text{t}^\prime} \Ex \left\{ \log_2 \det((\vect{W}^\prime)^{\Htran} \vect{\Lambda}_{\text{r}}^\prime  \vect{W}^\prime) \right\} \\&  \label{L_infty_app_nt} \hspace{2cm}
%-\frac{1}{{\sf n}_\text{t}^\prime} \log_2 \det(\vect{{\sf \Lambda}}_{\text{t}}^\prime) 
%\end{align}
%For ${\sf n}_\text{r}^\prime = {\sf n}_\text{t}^\prime = {\sf n}^\prime$, 
%\begin{align}  \label{L_infty_app_n}
%{\sf L}_\infty^\star(\tilde{\vect{{\sf H}}}) & \approx
% \log_2 {\sf n}^\prime 
%-\frac{1}{{\sf n}^\prime} \Ex \left\{ \log_2 \det(\vect{W}^\prime (\vect{W}^\prime)^{\Htran}) \right\} \\& \notag \hspace{1cm}
%-\frac{1}{{\sf n}^\prime}  \log_2 \det(\vect{\Lambda}_{\text{r}}^\prime) 
%- \frac{1}{{\sf n}^\prime} \log_2 \det(\vect{{\sf \Lambda}}_{\text{t}}^\prime) 
%\end{align}
%
%Focus on the power-offset related to correlation, e.g., at the receiver.
%Applying Jensen’s inequality and expanding $\vect{\Lambda}_{\text{r}}$ according to \eqref{equiv_channel},
%\begin{align}  \label{rx_offset_0}
% - \frac{1}{{\sf n}_\text{r}^\prime}  \log_2 \det(\vect{\Lambda}_{\text{r}}^\prime)  & = - \frac{1}{{\sf n}_\text{r}^\prime} \sum_{i=1}^{{\sf n}_\text{r}^\prime}  \log_2([\vect{\Lambda}_{\text{r}}^\prime]_{i,i})  \\ &  \label{rx_offset_Jensen} \hspace{.5cm}
%\ge - \log_2\left(\frac{1}{{\sf n}_\text{r}^\prime} \sum_{i=1}^{{\sf n}_\text{r}^\prime} [\vect{\Lambda}_{\text{r}}^\prime]_{i,i}\right) \\ &  \label{rx_offset_Lr} \hspace{.5cm}
% = - \log_2\left(\frac{N_\text{r}}{{\sf n}_\text{r}^\prime}  \sum_{j=1}^{{\sf n}_\text{r}^\prime} \sigma^2_i(\tilde{H}^+_\text{r})  \right)  \\& \hspace{.5cm} \label{rx_offset_int}
%= - \log_2\left(\frac{N_\text{r}}{{\sf n}_\text{r}^\prime} \right) 
%\\& \hspace{.5cm} \label{rx_offset}
% = \log_2\left(\frac{{\sf n}_\text{r}^\prime}{N_\text{r}} \right) 
%\end{align}
%as per the channel normalization in \eqref{normalization_channel}, which subsumes MIMO channels for $N_\text{r} = {\sf n}_\text{r}^\prime$ \cite{heath_lozano_2018}.
%Thus, correlation applies a positive power offset $- \frac{1}{{\sf n}_\text{r}^\prime}  \log_2 \det(\vect{\Lambda}_{\text{r}}^\prime) \ge 0$ under sub-Nyquist criterion $N_\text{r} \le {\sf n}_\text{r}^\prime \le {\sf n}_\text{r}$ (see \eqref{Nyquist_cond}).

 
%\paragraph{Power Offset}
%Concerning the latter, focus on the second term in \eqref{L_infty}
% and assume that $N_{\text r} = {\sf n}_\text{r}$ and $N_{\text t} = {\sf n}_\text{t}$ corresponding to Nyquist spatial sampling  \cite{PizzoTSP21}.  \angel{This is rather restrictive for this paper, where precisely we study coupling, and it doesn't reflect the setting of Fig. 9, wheee you make a point of the artifacts that may arise if we assume IID fading when ${\sf n}_\text{min} \le \min(N_\text{r},N_\text{t})$}
%In this case, $\vect{H}_\vect{{\sf C}} \vect{H}_\vect{{\sf C}}^{\Htran}$ is full-rank with probability $1$ such that $\det(\vect{H}_\vect{{\sf C}} \vect{H}_\vect{{\sf C}}^{\Htran})$ is suitably defined.
%Using \eqref{channel_mat_c_rx},
%\begin{align} 
%\log_2 \det(\vect{H}_\vect{{\sf C}} \vect{H}_\vect{{\sf C}}^{\Htran}) \notag
%& = \log_2 \det(\vect{H} \vect{H}^{\Htran}) - \log_2 \det(\vect{{\sf C}}_\text{t}) \\
%& \quad \label{L_infty_subs}  - \log_2 \det(\vect{{\sf C}}_\text{r}),
%\end{align}
%which, substituted into \eqref{L_infty}, yields
%\begin{align} \label{L_infty_final}
%\Ex\{{\sf L}_\infty(\vect{H}_\vect{{\sf C}})\} & = \Ex\{{\sf L}_\infty(\vect{H})\} + \Delta{\sf L}_\infty(\vect{{\sf C}}_\text{r},\vect{{\sf C}}_\text{t})
%\end{align}
%with 
%\begin{equation} \label{power_off_coupling}
%\Delta{\sf L}_\infty(\vect{{\sf C}}_\text{r},\vect{{\sf C}}_\text{t}) = \frac{\log_2 \det(\vect{{\sf C}}_\text{r}) + \log_2 \det(\vect{{\sf C}}_\text{t})}{{\sf n}_\text{min}}
%\end{equation}
%embedding the effect of coupling on the power offset.
%Without coupling, $\Delta{\sf L}_\infty(\vect{I}_{N_\text{r}},\vect{I}_{N_\text{t}}) = 0$ whereby $\Ex\{{\sf L}_\infty(\vect{H}_\vect{{\sf C}})\} = \Ex\{{\sf L}_\infty(\vect{H})\}$ as expected. 
%
%Applying Hadamard's inequality to \eqref{power_off_coupling} and invoking the monotonicity of $\log_2(\cdot)$, e.g., at the transmitter, 
%\begin{align} \label{log_det_upper}
%\log_2 \det(\vect{{\sf C}}_\text{t}) & \le \log_2 \prod_{n=1}^{N_\text{t}} [\vect{{\sf C}}_\text{t}]_{n,n} 
%%= \log_2 \left(\prod_{n=1}^{N_\text{t}} \frac{1}{\eta} \right) 
%= N_\text{t} \log_2 \frac{1}{\eta},
%\end{align}
%which follows from $[\vect{{\sf C}}_\text{t}]_{n,n} = {1}/{\eta}$ $\forall n$ as per \eqref{lossy_impedance} and the normalization in \eqref{coupling_norm}.
%For $\eta=1$, hence, $\log_2 \det(\vect{{\sf C}}_\text{t}) \le 0$ and coupling is favorable as it decreases the power offset through \eqref{power_off_coupling}.
%However, \eqref{power_off_coupling} might turn positive for values of $\eta$ below a certain threshold, with a detrimental effect on the high-SNR capacity analog to the one elicited by spatial correlation.
%
%The high-SNR regime is exemplified in Fig.~\ref{fig:SEvsSNR_10lambda_highSNR} for the same setup of Fig.~\ref{fig:SEvsSNR_10lambda_highSNR}. 
%IID fading overstates the spatial DOF, which are properly captured by the remaining curves, all having a slope of ${\sf n}_\text{min} \le \min(N_\text{r},N_\text{t})$. The DOF are not influenced by $\eta$, while the power offset is monotonic-decreasing with the antenna efficiency eventually producing a negative offset for $\eta \approx 1$. \angel{Not clear how readers can figure that out from the figure, the closest is 0.9 and by then it's still positive}




%Recall the log-sum inequality, 
%\begin{equation}
%\sum_i a_i \log_2\left(\frac{a_i}{b_i} + c\right) \ge \sum_i a_i \cdot \log_2\left(\frac{\sum_i a_i}{\sum_i b_i} + c\right),
%\end{equation}
%for nonnegative numbers $a_i$ and $b_i$ $\forall i$, 
%whereby 
%\begin{align}  \notag
%I(\vect{{\sf C}}_\text{t}) & \ge \iint_{\|\vect{\kappa}\|\le \kappa} \!\!\! |A(\vect{\kappa})|^2 \frac{d\vect{\kappa}}{(2\pi)^2} \\& \label{log_det_C_bound} \hspace{.5cm}\cdot  
%\log_2\left(\frac{\iint_{\|\vect{\kappa}\|\le \kappa} |A(\vect{\kappa})|^2 \frac{d\vect{\kappa}}{(2\pi)^2}}{\iint_{\|\vect{\kappa}\|\le \kappa} \gamma \frac{d\vect{\kappa}}{(2\pi)^2}} + \frac{1}{\eta} - 1\right).
%\end{align}
%Setting $\eta=1/2$ in \eqref{log_det_C_bound} implies $I(\vect{{\sf C}}_\text{t}) \ge 0$ $\forall A$, which means a positive power offset in light of \eqref{log_det_C_lower}.
%\begin{align}  \label{log_det_C_bound_eta}
%\iint_{\|\vect{\kappa}\|\le \kappa} \!\!\! |A(\vect{\kappa})|^2 \frac{d\vect{\kappa}}{(2\pi)^2} \cdot 
%\log_2\left(1 + \frac{\iint_{\|\vect{\kappa}\|\le \kappa} |A(\vect{\kappa})|^2 \frac{d\vect{\kappa}}{(2\pi)^2}}{\iint_{\|\vect{\kappa}\|\le \kappa} \gamma \frac{d\vect{\kappa}}{(2\pi)^2}}\right),
%\end{align}
%\begin{align} \label{log_det_C_final}
%\log_2 \det(\vect{{\sf C}}_\text{t}) \ge {\sf n}_\text{t}  \log_2\left(\frac{N_\text{t}}{L_\text{t}^2}\right) \ge 0
%\end{align}

%SOME BOUNDS:
%\begin{align}
%\log_2\det(\vect{{\sf C}}_{\text{t}}(\eta)) &= \log_2\det \left(\left(\frac{1}{\eta} - 1\right) \vect{I}_{N_\text{t}} + \vect{{\sf C}}_{\text{t}}\right) \\&
%\ge \log_2\left(\det\left(\left(\frac{1}{\eta} - 1\right) \vect{I}_{N_\text{t}}\right) + \det\left(\vect{{\sf C}}_{\text{t}}\right)\right) \\& =
%\log_2  \left( \left(\frac{1}{\eta} - 1\right)^{N_\text{t}} + \det\left(\vect{{\sf C}}_{\text{t}}\right)\right) \\&
%= 1 + \log_2  \left( \frac{1}{2}\left(\frac{1}{\eta} - 1\right)^{N_\text{t}} +  \frac{1}{2} \det\left(\vect{{\sf C}}_{\text{t}}\right)\right) \\&
%\ge 1 + \frac{N_\text{t}}{2}  \log_2\left(\frac{1}{\eta} - 1\right) + \frac{1}{2} \log_2\det\left(\vect{{\sf C}}_{\text{t}}\right)
%\end{align}
%
%applying Jensen's inequality and $\tr(\vect{{\sf C}}_{\text{t}}) = N_\text{t}$,
%\begin{align}
%\log_2 \det(\vect{{\sf C}}_\text{t}(\eta)) & = N_\text{t} \sum_{n=1}^{N_\text{t}} \frac{1}{N_\text{t}} \log_2 \left(\frac{1}{\eta} - 1 + \lambda_{\vect{{\sf C}}_{\text{t}},n}\right) \\&
%\le N_\text{t} \log_2 \left(\frac{1}{N_\text{t}} \sum_{n=1}^{N_\text{t}}  \left(\frac{1}{\eta} - 1 + \lambda_{\vect{{\sf C}}_{\text{t}},n}\right)\right) \\&
%= N_\text{t} \log_2 \left(1 + \frac{1}{N_\text{t}} \sum_{n=1}^{N_\text{t}} \left(\frac{1}{\eta} - 1 \right) \right) \\&
%= N_\text{t} \log_2 \left(\frac{1}{\eta}  \right) >0
%\end{align}
%
%Since the coupling matrices are both positive definite, e.g., at the transmitter, applying Hadamard's inequality and invoking the monotonicity of $\log_2(\cdot)$,
%\begin{align}
%\log_2 \det(\vect{{\sf C}}_\text{t}(\eta)) & \le \log_2 \prod_{n=1}^{N_\text{t}} \left([\vect{{\sf C}}_\text{t}]_{n,n} + \left(\frac{1}{\eta} - 1\right)\right) \\&
%= \log_2 \prod_{n=1}^{N_\text{t}} \frac{1}{\eta} = N_\text{t} \log_2 \frac{1}{\eta} > 0
%\end{align}
%which follows from \eqref{lossy_impedance} and $[\vect{{\sf C}}_\text{t}]_{n,n} = 1$, $\forall n$, as per the normalization in \eqref{coupling_norm}.
%It implies
%\begin{equation}
%\Delta{\sf L}_\infty(\vect{{\sf C}}_\text{r},\vect{{\sf C}}_\text{t}) = \frac{N_\text{r} + N_\text{t}}{{\sf n}_\text{min}(\vect{{\sf C}}_\text{r},\vect{{\sf C}}_\text{t})} \log_2 \frac{1}{\eta}
%\end{equation}
%
%Thus, the number of deployed antennas and their efficiency have a detrimental effect on the SNR offset, namely $\Ex\{{\sf L}_\infty(\vect{H}_\vect{{\sf C}})\} \le \Ex\{{\sf L}_\infty(\vect{H})\}$, which is similar to the one elicited by spatial correlation \cite{heath_lozano_2018}.
%Precisely, the upper bound on the cost of adding lossy antennas (i.e., such that $0 < \eta <1$) in the low-SNR regime is linear in the number of antennas and logarithmic in the antenna efficiency. 


%Thus, mutual coupling may limit the spatial DOF compared to those of an uncoupled MIMO channel.
%From \eqref{rank_HC_t}, it follows that coupling potentially limits the spatial DOF of a MIMO channel in the amount proportional to the antenna selectivity. This effect is similar to the one elicited by spatial correlation with more selective scattering corresponding to a reduced dimensionality for spatial multiplexing. 
%To prevent such limitation, the angular response of an individual antenna must be designed such that its support is not smaller than the maximum angular spread envisioned by the channel, a criterion that becomes progressively more reliable as $L_\text{t}/\lambda$ increases with fixed antenna spacing. However, this may not be the case for beamforming where selective antennas may provide an extra directivity.

%\section{Waterfilling with Electromagnetic Constraint}
%\andrea{Say that classical precoder optimization is not suitable for holographic MIMO because enormous power may be allocated on some channels (huge dissipated power that may harm the RF circuitry). 2 Tx antenna example?}
%
%\andrea{Introduce the Q factor and formulate the constrained precoder optimization problem}
% 
%%leads to the maximum-ratio transmission. 
%More generally, superdirectivity remains an issue at every SNR, due to the inversion of $\vect{{\sf C}}_{\text t}$ required to compute \eqref{optimal_precoder}.
%To curb the transmit superdirectivity effect we augment the precoder optimization problem with additional constraints on its search space.
%%The transmit array gain conditioned on the channel $\vect{h}$ is % the largest received power at the receiver,
%%%[THE EXPECTATIONS OVER $s$ ARE IMMATERIAL BECAUSE BY DEFINITION THE POWERS ARE OBTAINED BY EXPECTING OVER IT; THEY CAN BE REMOVED]
%%\begin{equation}
%%G(\vect{h})  = \max_{\|\vect{f}\|=1} \frac{P_\text{r}(\vect{h})}{P_\text{t}} ,
%%%= \frac{\max_{\|\vect{f}\|=N_\text{t}} \Ex\{P_\text{r}(\vect{f}|\vect{h})\}}{\max_{\|\vect{f}\|=N_\text{t}} \Ex\{P_\text{r}(\vect{f}|\vect{h})\}|_{N_\text{t}=1}}
%%\end{equation}
%%%where the expectation is taken with respect to all input realizations.
%%and its local-average counterpart is $G = \Ex\{G(\vect{h})\}$.
%%
%%Using \eqref{power_constraint_MIMO},
%%\begin{align} \label{received_power}
%%P_\text{r}(\vect{h}) & = |\vect{f}^{\Htran} \vect{h}|^2  = P_\text{t} \frac{|\vect{f}^{\Htran} \vect{h}|^2}{\vect{f}^{\Htran} \vect{{\sf C}}_\text{t} \vect{f}} ,
%%\end{align}
%%whose generalized Rayleigh quotient is maximized by
%%\begin{equation} \label{optimal_precoder_miso}
%%\vect{f}^\star = \frac{\vect{{\sf C}}_\text{t}^{-1} \vect{h}}{\|\vect{{\sf C}}_\text{t}^{-1} \vect{h}\|}
%%\end{equation}
%%yielding
%%\begin{equation} \label{optimal_gain}
%%G(\vect{h}) = \vect{h}^{\Htran} \vect{{\sf C}}_\text{t}^{-1} \vect{h} .
%%\end{equation}
%%%since $\vect{{\sf C}}_\text{t}$ is real and symmetric. 
%%%as $\vect{h}(\theta) = [1 e^{\imagunit \kappa d \cos(\theta)} \ldots e^{\imagunit \kappa (N_\text{t}-1) d \cos(\theta)}]^{\Ttran}$, $\|\vect{h}(\theta)\|^2=N_\text{t}$ $\forall \theta$.
%%Inverting \eqref{channel_mat_c} returns $\vect{h} = \vect{{\sf C}}_\text{t}^{1/2} \vect{h}_\vect{{\sf C}}$, which, plugged into \eqref{optimal_precoder_miso}, gives
%%\begin{equation} \label{optimal_precoder_miso_final}
%%\vect{f}^\star = \vect{{\sf C}}_\text{t}^{-1/2} \vect{h}_\vect{{\sf C}} \, \sqrt{p^\star}
%%\end{equation}
%%where $\sqrt{p^\star} = \|\vect{{\sf C}}_\text{t}^{-1/2} \vect{h}_\vect{{\sf C}}\|^{-1}$.
%%Without mutual coupling, 
%%%(i.e., $\vect{{\sf C}}_\text{t} = \vect{I}_{N_\text{t}}$),
%%maximum-ratio transmission $\vect{f}^\star = \vect{h}/\|\vect{h}\|$ is optimal, achieving $G(\vect{h}) = \|\vect{h}\|^2$ and $G = N_\text{t}$.
%%More generally, higher array gains can be achieved with mutual coupling, for the same number of antennas. This is the well-known phenomenon of superdirectivity whereby, for a linear array of uniformly spaced antennas, $G(\vect{h}) \to N_\text{t}^2$ along the axis containing the array, as the spacing vanishes \cite{Wallace2005,Nossek2010}.
%%%[ARE YOU EXPLICITLY ASSUMING A LINEAR ARRAY HERE?]
%%%[WE SHOULD HELP THE READER SEE HOW THE $N_\text{t}^2$ RESULT ARISES]
%%
%%%A comparison between \eqref{optimal_precoder_miso_final} and \eqref{optimal_precoder} reveals the connection between the optimal precoder and the superdirectivity phenomenon.
%%%Both promise array gains when an uncontrollable level of superdirectivity is allowed. 
%%%Precisely, from \eqref{optimal_gain}, superdirectivity is achieved when the channel vector $\vect{h}$ lies in a subspace spanned by the eigenvectors of $\vect{{\sf C}}_\text{t}$ that are associated with the smallest eigenvalues \cite{Marzetta_superdirectivity}. 
%%%This event becomes more likely when the dimension of this subspace grows, as in the case of spatial oversampling, shown in Fig.~\ref{fig:eigC}.
%
%
%
%
%
%The superdirectivity ratio is defined by \cite{Lee66,Wallace2005} %Butler71 Wallace2005 
%\begin{equation} \label{Qfactor_aperture}
%Q(j_{\text t}(\vect{{\sf r}})) = \frac{\iiint_{-\infty}^\infty  |j_{\text t}(\vect{{\sf r}})|^2 d\vect{{\sf r}}}{\iiint_{-\infty}^\infty d\vect{{\sf r}}  \iiint_{-\infty}^\infty d\vect{{\sf s}} \, j_{\text{t}}^*(\vect{{\sf r}}) {\sf c}_{\text t}(\vect{{\sf r}}-\vect{{\sf s}}) j_{\text{t}}(\vect{{\sf s}})},
%\end{equation}
%which is positive for every current density $j_{\text t}(\vect{{\sf r}})$, its inverse being the ratio between the transmit power and the power radiated by a hypothetical uncoupled system.
%Invoking \eqref{real_impedance_kernel_antenna}, after discretization, we have that 
%\begin{equation} \label{Qfactor_inv}
%Q^{-1}(\vect{j}_{\text t}) 
%%= \frac{P_\text{r}(\vect{h})}{P_\text{r}(\vect{h})|_{\vect{{\sf C}}_\text{t} = \vect{I}_{N_\text{t}}}} 
%= \frac{\vect{j}_{\text t}^{\Htran} \vect{{\sf C}}_\text{t} \vect{j}_{\text t}}{\vect{j}_{\text t}^{\Htran} \vect{j}_{\text t}}
%\end{equation}
%for every $\vect{j}_{\text t}$. This is the Rayleigh quotient whereby its values are bounded between \cite{Lee66}
%\begin{equation}  \label{Qfactor_bound}
%\lambda_{\vect{{\sf C}}_\text{t}, N_{\text t}} \le Q^{-1}(\vect{j}_{\text t}) \le  \lambda_{\vect{{\sf C}}_\text{t}, 1}
%%\lambda_\text{max}^{-1}(\vect{{\sf C}}_\text{t}) \le Q \le \lambda_\text{min}^{-1}(\vect{{\sf C}}_\text{t}).
%\end{equation}
%where $\lambda_{\vect{{\sf C}}_\text{t}, 1} \ge \cdots \ge \lambda_{\vect{{\sf C}}_\text{t}, N_{\text t}} > 0$ are the sorted eigenvalues of $\vect{{\sf C}}_\text{t}$.
%Denoting with $\vect{v}_{\vect{{\sf C}}_\text{t},n}$, $n=1,\ldots, N_{\text t}$, the corresponding eigenvectors, the maximum and minimum values in \eqref{Qfactor_bound} are attainable by $\vect{j}_{\text t}=\vect{v}_{\vect{{\sf C}}_\text{t},1}$ and $\vect{j}_{\text t}=\vect{v}_{\vect{{\sf C}}_\text{t},N_{\text t}}$, respectively. 
%%For punctiform antennas, $\vect{{\sf C}}_\text{t}$ is the $N_\text{t}$-dimensional $\sinc(\cdot)$ matrix.
%When the antenna spacing shrinks (array densification) or the transmit aperture increases (spectral concentration), $\lambda_{\vect{{\sf C}}_\text{t},N_{\text t}} \to 0$ that $Q(\vect{j}_{\text t}) \le \lambda^{-1}_{\vect{{\sf C}}_\text{t},N_{\text t}} \to \infty$. 
%In these cases, $Q(\vect{j}_{\text t})$ is unbounded and may assume any value, leading to superdirectivity.
%Imposing an upper constraint on $Q(\vect{j}_{\text t})$ would prevent this effect by restricting the feasibility region of possible $\vect{j}_{\text t}$. 
%
%For communications with one DOF at transmitter ${\sf n}_\text{t} = 1$, averaging each term of \eqref{Qfactor_inv} over all possible inputs, $Q(\vect{f}) = \frac{\vect{f}^{\Htran} \vect{f}}{\vect{f}^{\Htran} \vect{{\sf C}}_\text{t} \vect{f}}$ for every $\vect{f}\in \Complex^{N_\text{t}}$. Generalization to multiple DOF ${\sf n}_\text{t} \ge 1$ yields
%\begin{equation} \label{Qfactor_F}
%Q(\vect{F}) %= \frac{P_\text{r}(\vect{h})}{P_\text{r}(\vect{h})|_{\vect{{\sf C}}_\text{t} = \vect{I}_{N_\text{t}}}} 
%= \frac{\tr(\vect{F} \vect{F}^{\Htran})}{\tr(\vect{F} \vect{F}^{\Htran} \vect{{\sf C}}_\text{t})}
%\end{equation}
%for every $\vect{F} \in \Complex^{N_\text{t}\times N_\text{t}}$.
%%which rewritten in terms of $\vect{F}_\vect{{\sf C}}$, 
%%\begin{equation} \label{super_constraint}
%%Q = \frac{\tr(\vect{F}_\vect{{\sf C}} \vect{F}_\vect{{\sf C}}^{\Htran} \vect{{\sf C}}_\text{t}^{-1})}{\tr(\vect{F}_\vect{{\sf C}} \vect{F}_\vect{{\sf C}}^{\Htran})}.
%%\end{equation}
%%The usable bandwidth of an array is proportional to $1/Q$ with $Q= Q Q_\text{e}$, given $Q_\text{e}$ the quality factor of an antenna. Since $Q_\text{e}\approx 10$ for a dipole antenna, $Q$ must be one or less to be a practical constraint.
%The capacity of a MIMO system with a superdirectivity constraint was studied in \cite{Wallace2005}, which provides a heuristic solution to the problem. We deviate from \cite{Wallace2005} by adding linear constraints on the transmit precoder directly into the capacity formulation.
%Before doing that, let us introduce an even more general constraint than \eqref{Qfactor_F}, namely, the array performance indicator \cite{Butler71}
%\begin{equation} \label{metric_F}
%Q(\vect{F}) = \frac{\tr(\vect{F} \vect{F}^{\Htran} \vect{B}_{\text t})}{\tr(\vect{F} \vect{F}^{\Htran} \vect{{\sf C}}_\text{t})}
%\end{equation}
%for any positive-definite matrix $\vect{B}_{\text t} \in \Complex^{N_\text{t}\times N_\text{t}}$. Clearly, setting $\vect{B}_{\text t} = \vect{I}_{N_{\text t}}$ gives the superdirectivity ratio in \eqref{Qfactor_F}, whereas $\vect{B}_{\text t} = \vect{a}(\theta_{\text t},\phi_{\text t}) \vect{a}^{\Htran}(\theta_{\text t},\phi_{\text t})$ yields the array directivity in \eqref{directivity_current_discrete} along a certain direction $(\theta_{\text t},\phi_{\text t})$.
%
%The maximization of the conditional mutual information is formulated as
%\begin{equation} \label{P0}
%\begin{aligned}
%& \underset{\vect{F} \in \Complex^{N_\text{t}\times N_\text{t}}}{\text{max}}
%& & \log_2 \det (\vect{I} +  \vect{H}^{\Htran} \vect{H} \vect{F} \vect{F}^{\Htran} ) \\
%& \text{subject to}
%&& \vect{F} \vect{F}^{\Htran} \succeq \vect{0} \\
%& & & \tr(\vect{F} \vect{F}^{\Htran} \vect{{\sf C}}_\text{t}) \le  {\sf SNR} \\
%&&&  Q(\vect{F}) \le Q_{\text t},
%\end{aligned}
%\end{equation}
%where the inequality constraints refer to, respectively, the positive semi-definiteness of the input covariance matrix, actual radiated power in \eqref{input_power_constraint}, and array performance indicator in \eqref{metric_F}.
%Invoking the singular-value decomposition (SVD) of the channel matrix $\vect{H} = \vect{U}_\vect{H} \vect{\Lambda}^{1/2}_\vect{H} \vect{V}_\vect{H}^{\Htran}$ with $\vect{U}_{\vect{H}}\in\Complex^{N_{\text t} \times N_{\text t}}$ and $\vect{V}_{\vect{H}}\in\Complex^{N_{\text t} \times N_{\text t}}$ unitary matrices and $\vect{\Lambda}_\vect{H} = \diag(\lambda_{\vect{H},1},\ldots, \lambda_{\vect{H},N_\text{t}})^{\Ttran} \in\Complex^{N_{\text t} \times N_{\text t}}$ the diagonal matrix containing the (sorted?) eigenvalues of $\vect{H}^{\Htran} \vect{H}$, \eqref{P0} rewrites as
%\begin{equation} \label{P0_1}
%\begin{aligned}
%& \underset{\vect{F}_\vect{H} \in \Complex^{N_\text{t}\times N_\text{t}}}{\text{max}}
%& & \log_2 \det (\vect{I} + \vect{F}_\vect{H} \vect{F}_\vect{H}^{\Htran} \vect{\Lambda}_\vect{H} ) \\
%& \text{subject to}
%&& \vect{F}_\vect{H} \vect{F}_\vect{H}^{\Htran} \succeq \vect{0} \\
%& & & \tr(\vect{F}_\vect{H} \vect{F}_\vect{H}^{\Htran} \vect{{\sf C}}_{\vect{H},j}) \le c_j, \quad j=1, 2,
%%&&&  \tr(\vect{F}_\vect{H} \vect{F}_\vect{H}^{\Htran} \vect{{\sf C}}_{\vect{H},2}) \le  0,
%\end{aligned}
%\end{equation}
%in the new optimization variable
%\begin{equation} \label{precoder_Fhat}
%\vect{F}_\vect{H} = \vect{V}_{\vect{H}}^{\Htran} \vect{F},
%\end{equation}
%given $\vect{{\sf C}}_{\vect{H},1} = \vect{V}_{\vect{H}}^{\Htran} \vect{{\sf C}}_\text{t} \vect{V}_{\vect{H}}$ and $\vect{{\sf C}}_{\vect{H},2} = \vect{V}_{\vect{H}}^{\Htran} (\vect{B}_\text{t} - Q_\text{t} \vect{{\sf C}}_\text{t}) \vect{V}_{\vect{H}}$. In turn, $c_1 = {\sf SNR}$ and $c_2 = 0$.
%By means of Hadamard's inequality \cite{Yu2004}, \eqref{P0_1} is maximized by $\vect{F}_\vect{H} = \diag(\vect{p}^{1/2})$, $\vect{p}=(p_1,\ldots, p_{N_\text{t}})^{\Ttran}$, which transforms \eqref{P0_1} into
%\begin{equation} \label{P1}
%\begin{aligned}
%& \underset{\vect{p} \in \Real^{N_\text{t}}}{\text{max}}
%& & \sum_{i=1}^{N_{\text t}} \log_2\left(1 + p_i \lambda_{\vect{H},i} \right) \\
%& \text{subject to}
%& &  \vect{p} \succeq \vect{0} \\
%& & & \vect{c}_{\vect{H},j}^{\Ttran} \vect{p} \le c_j, \quad j=1, 2,
%\end{aligned}
%\end{equation}
%where $\vect{c}_{\vect{H},j} = \vect{S} \vec(\vect{{\sf C}}_{\vect{H},j})$ with $\vect{S} \in \Complex^{N_{\text t} \times N^2_{\text t}}$ a selection matrix that picks out one every $N_{\text t}$ entries of $\vec(\vect{{\sf C}}_{\vect{H},j})$, $j=1, 2$.
%\eqref{P1} is a convex optimization problem given the convexity of its objective function (with opportune sign changed) and polyhedron constraint set \cite{BoydBookCVX}. 
%%We exploit the monotonicity of the objective function of \eqref{P1} in each variable $p_i$, so that the upper inequality constraints must be satisfied with equality.
%
%
%\subsection{On the $Q_\text{t}$ Factor}
%
%\begin{figure}
%\centering\vspace{-0.0cm}
%\includegraphics[width=.55\linewidth]{constraints_uncoupled} 
%\caption{.}\vspace{-0cm}
%\label{fig:constraints_uncoupled}
%\end{figure}
%
%\begin{figure}
%\centering\vspace{-0.0cm}
%\includegraphics[width=.55\linewidth]{constraints_coupled} 
%\caption{.}\vspace{-0cm}
%\label{fig:constraints_coupled}
%\end{figure}
%
%\begin{figure}
%\centering\vspace{-0.0cm}
%\includegraphics[width=.55\linewidth]{constraints_coupled_Qfactor} 
%\caption{.}\vspace{-0cm}
%\label{fig:constraints_coupled_Qfactor}
%\end{figure}
%
%To understand the operational meaning of the array $Q_\text{t}$ factor, we will focus on the $N_\text{t}=2$ antenna scenario and study the feasibility constraint in \eqref{P1} from a geometrical perspective.
%%We now provide a geometrical interpretation of the role played by the constraints in \eqref{P1} and determine general guidances for choosing the $Q_\text{t}$ factor. For the sake of representation,  
%
%Fig.~\ref{fig:constraints_uncoupled} shows the feasibility region for the power allocation of an uncoupled MIMO system with normalized transmit power. The associated semidefinite constraint and transmit power constraint are, respectively, $p_1,p_2 \ge0$ and $p_1 + p_2 \le 1$.
%To gauge the impact of coupling on the power allocation, we assume that the right singular matrix of the channel $\vect{V}_\vect{H}$ aligns exactly to the eigenvector matrix of the coupling. This represents the best case for the arising of superdirectivity (see Sec.~\ref{sec:superdirectivity}). In such a case, $\vect{{\sf C}}_{\vect{H},1} = \diag(\lambda_{\vect{{\sf C}},1},\lambda_{\vect{{\sf C}},2})$, $\lambda_{\vect{{\sf C}},1}\ge \lambda_{\vect{{\sf C}},2} > 0$ as the coupling matrix is positive definite.
%Thus, $\vect{c}_{\vect{H},1} =(\lambda_{\vect{{\sf C}},1},\lambda_{\vect{{\sf C}},2})^{\Ttran}$ whereby the transmit power constraint transforms into $\lambda_{\vect{{\sf C}},1} p_1 + \lambda_{\vect{{\sf C}},2} p_2 \le 1$. The feasibility region for the coupled MIMO case is plotted in Fig.~\ref{fig:constraints_coupled}. 
%When the antenna spacing shrinks (array densification) or the transmit aperture increases (spectral concentration) superdirectivity arises. In Fig.~\ref{fig:constraints_coupled}, we see that if $\lambda_{\vect{{\sf C}},2} \to 0$ the power allotted to $p_2$ may be infinite. 
%To avoid this, another constraint need be added properly.
%
%To curb superdirectivity, the second constraint in \eqref{P1} is added with $\vect{B}_\text{t} = \vect{I}$ \cite{Butler71}. In the above channel settings, $\vect{{\sf C}}_{\vect{H},2} = \vect{I} - Q_\text{t} \diag(\lambda_{\vect{{\sf C}},1},\lambda_{\vect{{\sf C}},2})$, thus implying $\vect{c}_{\vect{H},2} =(1-Q_\text{t} \lambda_{\vect{{\sf C}},1},1- Q_\text{t}\lambda_{\vect{{\sf C}},2})^{\Ttran}$. The array performance constraint can then be written as $(1-Q_\text{t} \lambda_{\vect{{\sf C}},1}) p_1 + (1-Q_\text{t} \lambda_{\vect{{\sf C}},2}) p_2 \le 0$.
%Equivalently, $p_2 \le \alpha p_1$ with
%\begin{equation}
%\alpha = \frac{(Q_\text{t} \lambda_{\vect{{\sf C}},1}-1)}{(1-Q_\text{t} \lambda_{\vect{{\sf C}},2})}
%\end{equation}
%whose values are parametrized by $Q_\text{t}$. 
%Clearly, $\lambda_{\vect{{\sf C}},1}^{-1} \le Q_\text{t} \le \lambda_{\vect{{\sf C}},2}^{-1}$ by its definition \eqref{Qfactor_bound}, from which we infer that $\alpha$ is non-negative (see Fig.~\ref{fig:constraints_coupled_Qfactor}).
%The lower extreme (i.e., $Q_\text{t} = \lambda_{\vect{{\sf C}},1}^{-1}$) corresponds to $\alpha=0$ that, in turn, means $p_2=0$. The upper extreme (i.e., $Q_\text{t} = \lambda_{\vect{{\sf C}},2}^{-1}$) corresponds to $\alpha=\infty$ that, in turn, means $p_2$ may be infinite; the latter is non-achievable as $\lambda_{\vect{{\sf C}},2} \to 0$ under superdirectivity.
%The intermediate case is depicted in Fig.~\ref{fig:constraints_coupled_Qfactor} with rising $Q_\text{t}$ values allowing an increasingly high level of superdirectivity.

\section{Conclusion} \label{sec:conclusion}

%This paper has shown that holographic MIMO channels can be modeled as a linear system whose spatial impulse response is determined jointly by fading correlation and mutual coupling.
%This framework spans both antenna densification within a fixed aperture and aperture widening with fixed antenna spacings.
%It demonstrated that, through a deconvolution operation, antenna patterns can be configured to modify correlation, resulting in improved channel capacity at every SNR.
%Put differently, by means of changing antenna patterns, an informed transmitter can manipulate the channel response, which is akin to deforming the environmental scatterers.
%Precisely, antenna patterns must
%%be lowest on directions on which fading is strongest, enhancing correlation. 
%enhance/reduce fading correlation at low/high SNR improving usability of spatial dimensions made available by scattering and array apertures.
%The channel characterization that leads to these findings assumes identical antennas and stationary environments, whereby 
%This paper introduced a linear system-theoretic framework for holographic MIMO that inherently integrates fading correlation and mutual coupling. It demonstrated that

%Abstract—This paper introduces a comprehensive framework for holographic multiantenna systems, a novel paradigm that integrates both closely spaced antennas and wide apertures relative to the wavelength. The framework connects information- theoretic analysis with physical constraints by inherently incorporating fading correlation and mutual coupling among antennas. It enables the assessment of spatial effects on channel capacity across SNR levels with given antenna patterns and fading selectivity. Additionally, it demonstrates how dynamic reconfiguration of antenna patterns can harness coupling to enhance system performance in response to environmental changes.

An informed holographic MIMO transmitter can optimize its channel capacity across varying SNR levels by accounting for correlation and mutual coupling in the precoder design. The configurability of the antenna patterns would add flexibility, % to adapt the array response to the fading conditions.
%An informed transmitter can enhance the holographic MIMO capacity across varying SNR levels by adjusting its antenna patterns
%{\color{blue}as a proxy for transmit coupling}
%to manipulate correlation.
allowing the transmitter to reinforce antenna correlations at low SNR %{\color{red}---at the expense of power---} 
and 
%\angel{This worries me, because taken to the limit (pattern pointing away from the incoming signal) we'll have wonderful correlation but no power! } 
tone them down at high SNR
%Precisely, the choice of the antenna pattern should seek to reinforce
% antenna correlations at low SNR and tone them down at high SNR, 
so as to optimize the usability of the spatial dimensions provided by the scattering environment and array apertures. In essence, the adjustment of the antenna patterns mimics a reshaping of the environmental scattering. 
%{\color{blue}The model complexity is intentionally minimized to facilitate the extraction of general insights (contingent on a properly designed circuit network).}\angel{Hmm. Which aspects of model simplicity are you trying to defend here?}
%\andrea{Eqs.~4 and 5 are derived specializing Eq.~3 to the settings of Fig.~1, which omits impedance networks and simplifying load impedances at both link ends. I chose the closest configuration to an uncoupled system (like isotropic fading and IID fading).}
%\andrea{Maybe we should add somewhere that the impact of impedance matching is left for future work.}
%\angel{This are fine points, but they are not reflected in your rather vague sentence. And these points are perhaps best made in the Conclusion, as well as hinted in the Intro}

Potential directions for future work include the following.
\begin{itemize}
\item
Accounting for mutual coupling at the receiver, due to finite load impedances at those antenna ports.
\item
Incorporating matching networks into the circuit model. %and load impedances. % which were omitted in this work.
%\item
%Accounting for non-stationarities, \angel{Need to be more specific} where different deconvolutions are required due to varying behaviors within the array.
%\item
%Integrating electromagnetic constraints into the capacity formulation to curb strong coupling \angel{Unclear} (as suggested in \cite{Wallace2005}, where only a heuristic approach is provided).
\item
Evaluating the impact of coupling on the power offset, the zero-order term of the capacity expansion at high SNR. % in \eqref{C_SNR_high}. 
\item
Identifying antenna patterns that maximize channel capacity across all SNR levels, generalizing the guidelines obtained for low and high SNR.
\item
Determining the optimal antenna spacing for given losses, patterns, and scattering conditions.
\item
Generalizing the framework to wideband channels \cite{Heath2023}, addressing the superposition of time-harmonic currents and voltages across the communication bandwidth.
\end{itemize}

Also, an inherent assumption throughout the paper has been that all of the antennas within the transmit array
%---and thus their patterns and polarizations---
are identical.
By relaxing this condition, the impact of coupling could be extended to the realm of pattern and/or polarization diversity, in which antennas are purposely exposed to distinct portions of the channel spectrum and/or polarization as a mechanism to diminish their correlation \cite{1216759}.

%\angel{Add to future work: effect of extra DOF (antenna densification and losses at the transmitter) arising from having $N_\text{t} \gg {\sf n}_\text{t}$ and $\rho >0$.}
%\andrea{I'd say the mapping of antenna pattern and losses to the equivalent aperture.} \angel{Yes, ok. If you add stuff on this in the intro, then here just a quick comment}

%\angel{Add short comment on polarization}
%\andrea{Don't you think pattern diversity should go in the intro? See my comment there.}

\appendices

\begin{figure*}[t]
%\begin{align}   \nonumber
%P_{\text{em}}(t)  & = \frac{-\imagunit \kappa Z_0}{4} \iiint_{-\infty}^\infty d\vect{{\sf r}}^\prime \iiint_{-\infty}^\infty d\vect{{\sf s}}^\prime  \iiint_{-\infty}^\infty d\vect{{\sf r}}  \iiint_{-\infty}^\infty d\vect{{\sf s}} 
%\Big( j_\text{t}^*(\vect{{\sf r}}^\prime) {\sf a}_\text{t}(\vect{{\sf r}}-\vect{{\sf r}}^\prime) g(\vect{{\sf r}}-\vect{{\sf s}}) {\sf a}_\text{t}(\vect{{\sf s}}-\vect{{\sf s}}^\prime)  j_\text{t}(\vect{{\sf s}}^\prime)
%+ j_\text{t}(\vect{{\sf r}}^\prime) {\sf a}_\text{t}(\vect{{\sf r}}-\vect{{\sf r}}^\prime) \\ & \label{aa}  \tag{122}
%g^*(\vect{{\sf r}}-\vect{{\sf s}}) {\sf a}_\text{t}(\vect{{\sf s}}-\vect{{\sf s}}^\prime) j_\text{t}^*(\vect{{\sf s}}^\prime) 
%+ j_\text{t}(\vect{{\sf r}}^\prime) {\sf a}_\text{t}(\vect{{\sf r}}-\vect{{\sf r}}^\prime) g(\vect{{\sf r}}-\vect{{\sf s}}) j_\text{t}(\vect{{\sf s}}^\prime) {\sf a}_\text{t}(\vect{{\sf s}}-\vect{{\sf s}}^\prime) e^{-\imagunit 2\omega t} + 
%j_\text{t}^*(\vect{{\sf r}}^\prime) {\sf a}_\text{t}(\vect{{\sf r}}-\vect{{\sf r}}^\prime) g^*(\vect{{\sf r}}-\vect{{\sf s}}) j_\text{t}^*(\vect{{\sf s}}^\prime) {\sf a}_\text{t}(\vect{{\sf s}}-\vect{{\sf s}}^\prime) e^{\imagunit 2\omega t}\Big)
%\end{align} 
%\hrule
\begin{align}  \label{bb0} \tag{128}
z_{\text{t,t}}(\vect{{\sf r}}-\vect{{\sf s}})   
& =  \frac{\kappa Z_0}{8\pi^2} 
\iint_{-\infty}^\infty d\vect{\kappa} \, \frac{1}{\gamma(\vect{\kappa})}   \iiint_{-\infty}^\infty d\vect{{\sf p}} \, {\sf a}_\text{t}(\vect{{\sf p}}-\vect{{\sf r}}) e^{\imagunit \vect{\kappa}^{\Ttran} \vect{p}} \iiint_{-\infty}^\infty d\vect{{\sf q}}  \,  {\sf a}_\text{t}(\vect{{\sf q}}-\vect{{\sf s}}) e^{-\imagunit \vect{\kappa}^{\Ttran} \vect{q}}  e^{\imagunit \gamma |p_z-q_z|}   \\ \label{bb1} \tag{129}
& =  \frac{\kappa Z_0}{8\pi^2} 
\iint_{-\infty}^\infty d\vect{\kappa} \,  \frac{e^{\imagunit \vect{\kappa}^{\Ttran} (\vect{r}-\vect{s})}}{\gamma(\vect{\kappa})} 
  \int_{-\infty}^\infty dp_z \, \overline{{\sf A}_\text{t}(\vect{\kappa},p_z-r_z)}
   \int_{-\infty}^\infty dq_z  \,  {\sf A}_\text{t}(\vect{\kappa},q_z-s_z)  
 e^{\imagunit \gamma |p_z-q_z|} 
% \\ \label{bb2} \tag{148}
% & =  \frac{\kappa Z_0}{8\pi^2} 
%\iint_{-\infty}^\infty d\vect{\kappa} \,  \frac{e^{\imagunit \vect{\kappa}^{\Ttran} (\vect{r}-\vect{s})}}{\gamma} 
%  \int_{-\infty}^\infty dr_z^\prime \, {\sf A}_\text{t}^*(\vect{\kappa},r_z^\prime) 
%   \int_{-\infty}^\infty ds_z^\prime  \,  {\sf A}_\text{t}(\vect{\kappa},s_z^\prime)  
% e^{\imagunit \gamma |(r_z^\prime-s_z^\prime) + (r_z-s_z)|} 
 \\ \label{bb3} \tag{130}
 & =  \frac{\kappa Z_0}{8\pi^2} 
\iint_{-\infty}^\infty d\vect{\kappa} \,  \frac{e^{\imagunit \vect{\kappa}^{\Ttran} (\vect{r}-\vect{s})}}{\gamma(\vect{\kappa})} 
  \int_{r-r_0}^{r+r_0} dp_z \, \overline{{\sf A}_\text{t}(\vect{\kappa},p_z-r_z)}
   \int_{s-r_0}^{s+r_0} dq_z  \,  {\sf A}_\text{t}(\vect{\kappa},q_z-s_z)  
 e^{\imagunit \gamma |p_z-q_z|} 
 \end{align} 
 \hrule
\begin{align}   \label{z_impedance} \tag{131}
z_{\text{t,t}}(\vect{{\sf r}}-\vect{{\sf s}})  
 & =  \frac{\kappa Z_0}{2} 
\iint_{-\infty}^\infty \frac{d\vect{\kappa}}{(2\pi)^2} \,  \frac{e^{\imagunit \vect{\kappa}^{\Ttran} (\vect{r}-\vect{s})}}{\gamma(\vect{\kappa})} 
\cdot
\begin{cases} \displaystyle
|{\sf A}_\text{t}(\vect{\kappa},\gamma)|^2  \, e^{\imagunit \gamma (r_z-s_z)}, & r_z-s_z>2 r_0 \\ \displaystyle
|{\sf A}_\text{t}(\vect{\kappa},-\gamma)|^2  \, e^{-\imagunit \gamma (r_z-s_z)}, & r_z-s_z<-2 r_0
\end{cases}
\end{align} 
\hrule
\end{figure*} 

\section*{Appendix}
\subsection{Circuit Power and Electromagnetic Power} \label{app:em_circuit_power}

The instantaneous circuit power expended by the continuous multiport system is given by % the superposition of the voltage-current conjugate products,
\begin{align}   \label{power_circuit_insta}
{\sf P}_{\text c}(t)  & = \iiint_{-\infty}^\infty  j_{\text t}(t,\vect{{\sf r}}) \,v_{\text t}(t,\vect{{\sf r}}) d\vect{{\sf r}}.
\end{align} 
For a time-harmonic source, expressing the circuit quantities in terms of phasors at frequency $\omega$, we have that
\begin{align}   \label{power_circuit_insta_phasor}
{\sf P}_{\text c}(t)  & = \iiint_{-\infty}^\infty  \Re\left\{j_{\text t}(\vect{{\sf r}}) e^{-\imagunit \omega t}\right\} \, \Re\left\{v_{\text t}(\vect{{\sf r}}) e^{-\imagunit \omega t}\right\} d\vect{{\sf r}}.  
\end{align} 
Time-averaging \eqref{power_circuit_insta_phasor}, the transmit power emerges as
\begin{align}    
{\sf P}_{\text c} &= \lim_{T\to\infty} \frac{1}{T} \int_{-T/2}^{T/2} {\sf P}_{\text c}(t) \, dt \\& \label{time_avg_power_electrical}
 = \frac{\omega}{2\pi} \int_{0}^{2\pi/\omega} {\sf P}_{\text c}(t) \, dt \\ \label{circuit_power_avg}
& = \frac{1}{2} \Re\left\{ \iiint_{-\infty}^\infty  \overline{j_{\text t}(\vect{{\sf r}})} \, v_{\text t}(\vect{{\sf r}}) \, d\vect{{\sf r}}\right\}
\end{align}
after using the identity \cite[Eq.~2.347]{PlaneWaveBook}
\begin{align}
\frac{\omega}{\pi} \int_{0}^{2\pi/\omega}  \Re\left\{a_\omega e^{-\imagunit \omega t}\right\} \Re\left\{b_\omega e^{-\imagunit \omega t}\right\} \, dt = \Re\left\{\overline{a_\omega} \, b_\omega\right\}.
\end{align}
Expanding $v_{\text t}(\vect{{\sf r}})$ in \eqref{circuit_power_avg} according to \eqref{voltage} yields \eqref{circuit_power}.
% while using the unilateral transmit model in
%This provides the continuous counterpart to \eqref{power_real_discrete}.
%integrating one-half the real part of the conjugate product between voltage and current densities,
%Time-averaging, the transmit circuit power emerges as
%\setcounter{equation}{10}
%\begin{align} \label{time_avg_power_electrical}
%P_{\text c} & = \lim_{T\to\infty} \frac{1}{T} \int_{-T/2}^{T/2} P_{\text c}(t) dt \\ \label{power_real_discrete_time_avg}
%& = \frac{1}{2} \iiint_{-\infty}^\infty d\vect{{\sf r}} \iiint_{-\infty}^\infty d\vect{{\sf s}} \, j_{\text{t}}^*(\vect{{\sf r}}) {\sf c}_{\text t}(\vect{{\sf r}}-\vect{{\sf s}}) j_{\text{t}}(\vect{{\sf s}}) 
%\end{align}
%where %{\color{blue}oscillating terms were averaged out and}
%the self-adjointness of the transmit impedance was exploited and we defined ${\sf c}_{\text t}(\vect{{\sf r}}) = \Re\{z_{\text{t,t}}(\vect{{\sf r}})\}$.

%As for the electromagnetic power, %consider a polarimetric source and assume that no depolarization occurs in the environment, such that all electromagnetic quantities can be expressed as scalars. 
%Ignoring polarization, this is fully described by its complex-valued (phasor) current density $j(\vect{{\sf r}})$, $\vect{{\sf r}} = (\vect{r},r_z) \in \Real^3$, that is zero everywhere outside of a sphere of radius $r_0>0$.
%Relying on the sampling property of the Dirac delta function (also the principle of superposition of electric current in electromagnetism) \cite{ChewBook}, we can model such source as an uncountably-infinite collection of punctiform antennas (i.e., point sources), 
%An arbitrary 
%the instantaneous power density per unit volume exerted by a unipolarized space-time current density $\vect{j}_{\text t}(t,\vect{{\sf r}}) = \hat{\vect{j}} j_{\text t}(t,\vect{{\sf r}})$ on a field distribution $\vect{e}_{\text t}(t,\vect{{\sf r}}) = \hat{\vect{j}} e_{\text t}(t,\vect{{\sf r}})$ is  \cite[Eq.~2.127]{PlaneWaveBook} 
%The power density per unit volume (in W/m$^3$) exerted by a space-time current density $j_{\text t}(t,\vect{{\sf r}})$ (in C/m$^2$/s) on a field distribution $e_{\text t}(t,\vect{{\sf r}})$ (in J/C/m) is  \cite[Eq.~2.127]{PlaneWaveBook} [DO WE REALLY NEED THE UNITS?]
%\begin{equation}
%p_{\text{em}}(t,\vect{{\sf r}}) = \vect{j}_{\text t}^{\Ttran}(t,\vect{{\sf r}}) \vect{e}_{\text t}(t,\vect{{\sf r}}) = j_{\text t}(t,\vect{{\sf r}}) e_{\text t}(t,\vect{{\sf r}}).
%\end{equation}
%Thus, the instantaneous %(complex)
%electromagnetic
%power equals
%expended by this dynamical source
%is obtained by integration as
 
 %whose acceleration is dictated by the pressure gradient. %\cite[Sec.~6.20]{HildebrandBook}. 
 %From Newton’s second law of motion, the frequency-domain complex power expended by the acoustic source to generate its field is given by the flux integral \cite[Eq.~2.389]{PlaneWaveBook}
% the space-time processes at frequency $\omega$ are described by
%\setcounter{equation}{23}
%\begin{align} \label{}
%j_\alpha(t,\vect{{\sf r}}) = \Re\{j_\alpha(\vect{{\sf r}}) e^{-\imagunit \omega t}\}, \quad 
%e_\alpha(t,\vect{{\sf r}}) = \Re\{e_\alpha(\vect{{\sf r}}) e^{-\imagunit \omega t}\}
%\end{align}
%through the corresponding frequency-dependent complex phasor variables $j_\alpha(\vect{{\sf r}})$ and $e_\alpha(\vect{{\sf r}})$.
%Hence, the instantaneous electromagnetic power expended by this source to generate such field is given by \cite{TomBook}
The instantaneous electromagnetic power exerted by a unipolarized space-time current density $\vect{j}_{\text t}(t,\vect{{\sf r}}) = \hat{\vect{j}} j_{\text t}(t,\vect{{\sf r}})$ on a field distribution $\vect{e}_{\text t}(t,\vect{{\sf r}}) = \hat{\vect{j}} e_{\text t}(t,\vect{{\sf r}})$ is  \cite[Eq.~2.127]{PlaneWaveBook} 
\begin{equation}
{\sf P}_{\text{em}}(t) = \iiint_{-\infty}^\infty  j_{\text t}(t,\vect{{\sf r}}) e_{\text t}(t,\vect{{\sf r}}) d\vect{{\sf r}}
\end{equation}
and, for a time-harmonic source specifically,
\begin{align}    \label{wave_power_punctiform}
{\sf P}_\text{em}  & = \frac{1}{2} \Re\left\{ \iiint_{-\infty}^\infty  \overline{j_{\text t}(\vect{{\sf r}})} \, e_{\text t}(\vect{{\sf r}}) \, d\vect{{\sf r}}\right\}
\end{align}
which is expressed in terms of the corresponding phasors.
%Incorporating both reactive and radiating components, 
%the electromagnetic power is complex. To avoid the decay of the reactive component, $e_{\text t}(\vect{{\sf r}})$ within \eqref{wave_power_punctiform} is evaluated on a closed surface bounding the source in tight proximity. %wherein line-of-sight (LOS) propagation occurs.
%[EXPLAIN WHY WE WANNA RETAIN THE REACTIVE COMPONENT FOR NOW...]
%The reactive power component is retained for now, as some of the results to be derived are generally valid for the complex impedance kernel.
%{\color{blue}[Andrea: When I first looked at the problem I did not understand where the sinc() model came from, as I was expecting a spherical wave. Also you see the low pass filtering behavior due to cancellation of evanescent waves.]}
%[WHY DO WE NEED LOS, WOULDN'T IT SUFFICE  TO SPECIFY A CLOSED BOUNDING SURFACE IN CLOSE PROXIMITY?]
The field obeys the scalar Helmholtz equation
%The field obeys the
%%frequency-domain
% Helmholtz equation 
\cite{ChewBook,PlaneWaveBook}
\begin{equation} \label{Helmholtz}
\nabla^2 e_{\text t}(\vect{{\sf r}}) + \kappa^2 e_{\text t}(\vect{{\sf r}}) =  \imagunit \kappa Z_0 j_{\text t}(\vect{{\sf r}})
\end{equation}
where ${Z_0 \approx 120 \pi}$ is the wave impedance of free-space while 
%$\kappa = \omega/c$ is the wavenumber, given $c$ the speed of light.
$\kappa = 2\pi/\lambda$ given $\lambda$ the wavelength.
%Any source is obtainable as an integral superposition of point sources or Hertzian dipoles \cite{ChewBook}. 
%, via [REMOVE EQUATION]
%\begin{equation} \label{no_coupling}
%j_{\text t}(\vect{{\sf r}}) = \iiint_{-\infty}^\infty  \delta(\vect{{\sf s}}) j_{\text t}(\vect{{\sf r}} - \vect{{\sf s}}) d\vect{{\sf s}} ,
%\end{equation}
%which rests on the sampling property of the Dirac delta function
%(also the principle of superposition of electric current in electromagnetism)
%[DO WE EVEN NEED TO MAKE THIS POINT?]
%By virtue of \eqref{no_coupling} and the linearity of \eqref{Helmholtz}, %we have that
From the linearity of \eqref{Helmholtz},
\begin{equation} \label{field_no_coupling}
e_{\text t}(\vect{{\sf r}}) = - \imagunit \kappa Z_0 \iiint_{-\infty}^\infty  j_{\text t}(\vect{{\sf s}}) g(\vect{{\sf r}} - \vect{{\sf s}}) d\vect{{\sf s}} ,
\end{equation}
where $g(\vect{{\sf r}})$ is the Green's function \eqref{Green} solving $\nabla^2 g(\vect{{\sf r}}) + \kappa^2 g(\vect{{\sf r}}) =  - \delta(\vect{{\sf r}})$.
%No simplification can be applied here as the coupling among closely spaced antennas is representative of reactive propagation mechanisms occurring in the order of a few wavelengths or less.
%Altogether, there is a cascade of linear and space-invariant (LSI) filtering operations modeling the field generation and wave propagation in LOS conditions. 
%[THE FIRST LSI DOES NOT INVOLVE THE FIELD, BUT ONLY THE CURRENT, SO PERHAPS "CURRENT GENERATION"?]
%[AGAIN, WOULDN'T FREE SPACE SUFFICE, RATHER THAN LOS?]
Substituting \eqref{field_no_coupling} into \eqref{wave_power_punctiform} yields  \eqref{wave_power_final}.
%The above current can be seen as generated by passing a unit point-source through a  filter with impulse response $j_{\text t}(\vect{{\sf r}})$.
%The Fourier (plane wave) spectral representation of  $e(\vect{{\sf r}})$ radiated by this source outside of its physical support is  \cite{ChewBook,PlaneWaveBook}
% \begin{align}   \label{input_field_nocoupling}
%e(\vect{{\sf r}}) = 
%\begin{cases} \displaystyle
%\frac{\kappa Z_0}{2} \iint_{-\infty}^{\infty}  \frac{J_+(\vect{k})}{\gamma(\vect{k})} \, e^{\imagunit \left(\vect{k}^{\Ttran} \vect{r} + \gamma(\vect{k}) r_z\right)}  \frac{d\vect{k}}{(2\pi)^2}, \quad r_z>r_0 \\ \displaystyle
%\frac{\kappa Z_0}{2} \iint_{-\infty}^{\infty}  \frac{J_-(\vect{k})}{\gamma(\vect{k})} \, e^{\imagunit \left(\vect{k}^{\Ttran} \vect{r} - \gamma(\vect{k}) r_z\right)}  \frac{d\vect{k}}{(2\pi)^2}, \; r_z<-r_0
%\end{cases}
%\end{align}
%with $J_\pm(\vect{k})$ the source's plane-wave spectrum obtained via a 3D Fourier transform of $j(\vect{{\sf r}})$ evaluated at ${k_z = \pm \gamma(\vect{k})}$,
%\begin{equation} \label{source_spectrum}
%J_\pm(\vect{k}) = \iiint_{-\infty}^\infty j(\vect{{\sf s}}) \, e^{-\imagunit \left(\vect{k}^{\Ttran} \vect{s} \pm \gamma(\vect{k}) s_z\right)} \, d\vect{{\sf s}}
%\end{equation}
%with $\gamma(\vect{k})$ being defined as
% \begin{equation} \label{kappaz} 
%\gamma(\vect{k}) = 
%\begin{cases} \displaystyle
% \sqrt{\kappa^2 - \|\vect{k}\|^2}  & \quad \|\vect{k}\| \le \kappa \\ \displaystyle
%\imagunit  \sqrt{\|\vect{k}\|^2 - \kappa^2}  & \quad \|\vect{k}\| > \kappa
%\end{cases}
%\end{equation}

%In total, an electric source with coupled antenna elements can be regarded as a cascade of two LSI filtering operations modeling the uncoupled current and interactions among antenna elements.

%This space-invariant filtering is similar to the one elicited by a reflection off a smooth surface \cite{PizzoTWC22}. 
%%Both are of 2D nature due to the inherent lower dimensionality of wave propagation. 
%Nevertheless, the reflection phenomena is observed several wavelengths away from the transmitter that its response reduces to an impulse (i.e., no blurring). Thus, reflection solely applies a scaling coefficient that quantitatively depends on the material absorption. 
%Instead, mutual coupling occurs locally at the antennas wherein the amplitude variations of the filter response are not negligible and the convolution cannot be ignored.


%\section{} \label{app:invertibility}
%
%The spectrum ${\sf C}_{\text t}(\vect{{\sf k}})$ is obtainable by Fourier transforming \eqref{real_impedance_kernel_antenna} as ${\sf C}_{\text t}(\vect{{\sf k}}) = \iiint_{-\infty}^\infty \! d\vect{{\sf v}} \, {\sf c}_{\text t}(\vect{{\sf v}}) e^{-\imagunit \vect{{\sf k}}^{\Ttran} \vect{{\sf v}}}$. Focusing on the half-space $v_z > z_0$, $z_0\ge 0$, while momentarily omitting the normalization,
%\begin{align} 
% {\sf C}_{\text t}(\vect{{\sf k}}) & = 
%\iint_{\|\vect{\xi}\|\le \kappa} d\vect{\xi}  \frac{|A_{\text t}^+(\vect{\xi})|^2}{\gamma(\vect{\xi})} \int_{z_0}^\infty dv_z \, e^{\imagunit (\gamma(\vect{\xi}) - \kappa_z) v_z}  \\& \notag\hspace{3cm} \iint_{-\infty}^\infty \! d\vect{v} \, e^{\imagunit (\vect{\xi}-\vect{\kappa})^{\Ttran} \vect{v}}    \\& =  \label{LHS}
%\frac{|A_{\text t}^+(\vect{\kappa})|^2}{\gamma} \mathbbm{1}_{\|\vect{\kappa}\|\le \kappa}(\vect{\kappa}) \int_{z_0}^\infty dv_z \, e^{\imagunit (\gamma - \kappa_z) v_z},
%\end{align}
%where the integration limiting region has been embedded into the indicator function.
%Recalling that
%\begin{equation}
%\int_{t_0}^\infty dt \, e^{\imagunit \omega_0 t} e^{-\imagunit \omega t} = \frac{\imagunit}{\omega_0 - \omega} e^{\imagunit (\omega_0-\omega) t_0}
%\end{equation}
%with region of convergence $\Im\{\omega_0\}>0$, \eqref{LHS} reduces to
%\begin{align}  \label{C_kappa}
% {\sf C}_{\text t}(\vect{{\sf k}}) =  \frac{|A_{\text t}^+(\vect{\kappa})|^2}{\gamma}  \frac{\imagunit}{\gamma - \kappa_z} e^{\imagunit (\gamma-\kappa_z) z_0} \, \mathbbm{1}_{\|\vect{\kappa}\|\le \kappa}(\vect{\kappa})
% \end{align}
%for $\Im\{\gamma\}>0$, $\gamma$ being defined in \eqref{gamma}. The latter condition is satisfied $\forall \vect{\kappa}$  as per Sommerfeld's radiation condition, requiring $\Im\{\gamma\}>0$ in the half-space $z>0$ \cite{PlaneWaveBook,ChewBook}. The Fourier transform of the anti-causal part of ${\sf c}_{\text t}(\vect{{\sf v}})$ in \eqref{real_impedance_kernel_antenna} is obtainable in the same vein of \eqref{C_kappa} as
%\begin{align}  \label{C_kappa_minus}
% {\sf C}_{\text t}(\vect{{\sf k}}) =  \frac{|A_{\text t}^-(\vect{\kappa})|^2}{\gamma}  \frac{\imagunit}{\gamma + \kappa_z} e^{\imagunit (\gamma+\kappa_z) z_0} \, \mathbbm{1}_{\|\vect{\kappa}\|\le \kappa}(\vect{\kappa})
% \end{align}
% for $\Im\{\gamma\}<0$, obeying Sommerfeld's condition in the half-space $z<0$ \cite{PlaneWaveBook,ChewBook}. 
%%Similarly, Fourier transforming in two dimensions the right-hand side of \eqref{conv_plane_wave},
%%\begin{align}
%%& \iiint_{-\infty}^\infty \! d\vect{{\sf v}} \, {\sf c}^{1/2}_{\text{t}}(\vect{{\sf v}}) \iint_{-\infty}^\infty \! d\vect{r} \, {\sf c}^{1/2}_{\text{t}}(\vect{{\sf r}} - \vect{{\sf v}}) e^{-\imagunit \vect{\kappa}^{\Ttran} \vect{r}} \\ & =  
%%  \int_{-\infty}^\infty \! dv_z \,{\sf C}^{1/2}_\text{t}(\vect{\kappa},r_z-v_z)  \iint_{-\infty}^\infty \! d\vect{v} \, {\sf c}^{1/2}_{\text{t}}(\vect{{\sf v}}) e^{-\imagunit \vect{\kappa}^{\Ttran} \vect{v}} \\ & =   \label{RHS}
%% \int_{-\infty}^\infty \! dv_z \,{\sf C}^{1/2}_\text{t}(\vect{\kappa},r_z-v_z)  {\sf C}^{1/2}_\text{t}(\vect{\kappa},v_z),
%%\end{align}
%%which describes a convolution on the $z$ dimension.
%%Notice that the dependance of \eqref{LHS} on $r_z$ is confined to the migration term $e^{\imagunit \gamma r_z}$, which describes a delay in the dual $\kappa_z$-domain.
%%Retrieving ${\sf C}^{1/2}_\text{t}(\vect{\kappa},\kappa_z)$ from \eqref{pippo} is prohibited by the Dirac delta function. This observation agrees with the more general fact of a plane-wave spectrum being determined non-uniquely by the field on any $z$-plane \cite{PlaneWaveBook,MarzettaIT}.
%%However, for a planar transmitter, this ambiguity is resolved allowing
%% \begin{equation}  \label{C2}
%%{\sf C}^{1/2}_\text{t}(\vect{\kappa}) = \frac{|A_{\text t}^+(\vect{\kappa})|}{\sqrt{\gamma}} > 0
%%\end{equation}
%%for every $\|\vect{\kappa}\|\le \kappa$.
%%
%%Contrasting \eqref{C_kappa} and \eqref{C_kappa_minus} against \eqref{C_square_root} we infer that 
%%\begin{align}  \label{pippo}
%%{\sf C}^{1/2}_\text{t}(\vect{{\sf k}}) & =  
%%\begin{cases} \displaystyle
%%\frac{|A_{\text t}^+(\vect{\kappa})|}{\sqrt{\gamma}}  \frac{e^{\imagunit (\gamma-\kappa_z) r + \imagunit \pi/4}}{\sqrt{\gamma - \kappa_z}} \mathbbm{1}_{\|\vect{\kappa}\|\le \kappa}(\vect{\kappa}) & v_z > 2r \\\displaystyle
%%\frac{|A_{\text t}^-(\vect{\kappa})|}{\sqrt{\gamma}}  \frac{e^{\imagunit (\gamma+\kappa_z) r + \imagunit \pi/4}}{\sqrt{\gamma + \kappa_z}} \mathbbm{1}_{\|\vect{\kappa}\|\le \kappa}(\vect{\kappa}) & v_z< -2r.
%%\end{cases}
%%\end{align}
%However, ${\sf C}_{\text t}(\vect{{\sf k}})$ has an oscillatory behavior due to the complex exponentials in \eqref{C_kappa} and \eqref{C_kappa_minus}. Hence, ${\sf C}_{\text t}(\vect{{\sf k}})$ has a countably infinite number of zeros on the $k_z$-axis making $C^{-1/2}_{\text{t}}(\vect{{\sf k}})$ in \eqref{C_square_root_inv} singular.
%%Augmenting the model with receive coupling $\mathcal{C}_\text{r}$ yields the composite channel operator in \eqref{channel_op_coupling} with $\mathcal{C}^{-1/2}_\text{r} : L^2 \to L^2$ the inverse square-root operator of $\mathcal{C}_\text{r}$.
%%\begin{equation}
%%\mathcal{H}_\mathcal{C} = \mathcal{C}^{-1/2}_\text{r} \mathcal{H} \mathcal{C}^{-1/2}_\text{t}
%%\end{equation}
%% associated with a kernel $c^{-1/2}_{\text{r}}(\vect{{\sf r}})$ of spectrum $C^{-1/2}_{\text{r}}(\vect{k}) = \frac{\sqrt{\gamma}}{|A_{\text r}^+(\vect{k})|}$, given $A^+_{\text r}(\vect{k})$ the directivity spectrum of a receive antenna.}
%%, which is the continuous analogue of \eqref{channel_mat_c_rx}.}

%\section{}
%
%Sampling \eqref{j_C_op} at antenna locations and discretizing the spatial convolution \cite{HeedongIRS}, yields the composite input current vector %$\vect{j}_\vect{{\sf C}} \in \Complex^{N_{\text t}}$ 
%with entries
%\begin{equation} \label{input_vector_C}
% (\mathcal{C}^{1/2}_\text{t} j_\text{t})(\vect{s}_m) \approx \frac{{\rm m}(S_\text{t})}{N_\text{t}} 
%\sum_{m^\prime}  c^{1/2}_{\text t}(\vect{s}_m,\vect{s}_m^\prime) [\vect{j}_\text{t}]_{m^\prime}
%%[\vect{{\sf C}}^{1/2}_\text{t} \vect{j}_\text{t}]_{m}
%\end{equation}
%where $\vect{j}_\text{t} \in \Complex^{N_{\text t}}$ is the input vector and 
%% and $[\vect{{\sf C}}^{1/2}_\text{t}]_{m,m^\prime} = c^{1/2}_{\text t}(\vect{s}_m,\vect{s}_m^\prime)$ for $m,m^\prime = 1, \ldots, N_\text{t}$. In turn, 
%${\rm m}(\cdot)$ denotes the Lebesgue measure.
%Discretizing \eqref{Pt_inner_product_half} while using \eqref{input_vector_C},
%\begin{align} 
%P_\text{t} &\approx R \frac{{\rm m}(S_\text{t})}{N_\text{t}}  \sum_{m} \left|(\mathcal{C}^{1/2}_\text{t} j_\text{t})(\vect{s}_m)\right|^2 \\& \label{Pt_inner_product_discretized}
%\approx R \left(\frac{{\rm m}(S_\text{t})}{N_\text{t}}\right)^2 
%\sum_{m} \left| \sum_{m^\prime}  c^{1/2}_{\text t}(\vect{s}_m,\vect{s}_m^\prime) [\vect{j}_\text{t}]_{m^\prime} \right|^2.
%%\|\vect{{\sf C}}^{1/2}_\text{t} \vect{j}_\text{t}\|^2.
%\end{align}
%%where $[\vect{j}_\vect{{\sf C}}]_n = j_\mathcal{C}(\vect{s}_n,0)$ with $\vect{s}_n$ the location of the $n$th transmit antenna.
%
%Alternatively, rewrite the power transmitted by a discrete array in \eqref{time_avg_power_discrete} as
%\begin{align}   \label{time_avg_power_discrete_rewritten}   
%P_\text{t} & %= R \, \|\vect{{\sf C}}^{1/2}_\text{t} \vect{j}_{\text{t}}\|^2 
%= R \,  \sum_m  \left| [\vect{{\sf C}}^{1/2}_\text{t} \vect{j}_{\text{t}}]_{m}  \right|^2 = R \,  \sum_m  \left| \sum_{m^\prime}  [\vect{{\sf C}}^{1/2}_\text{t}]_{m,m^\prime} [\vect{j}_\text{t}]_{m^\prime}  \right|^2 
%\end{align}
%where $\vect{{\sf C}}^{1/2}_\text{t} \in \Complex^{N_{\text t} \times N_{\text t}}$ is the square-root of the positive definite matrix $\vect{{\sf C}}_\text{t}$, such that $\vect{{\sf C}}_\text{t} = \vect{{\sf C}}_\text{t}^{\Htran/2} \vect{{\sf C}}_\text{t}^{1/2}$ with $\vect{{\sf C}}_\text{t}^{1/2} = \vect{\Lambda}^{1/2}_{\vect{{\sf C}}_\text{t}} \vect{U}_{\vect{{\sf C}}_\text{t}}^{\Htran}$ invertible, given $\vect{U}_{\vect{{\sf C}}_\text{t}}$ unitary and $\vect{\Lambda}_{\vect{{\sf C}}_\text{t}}$ diagonal with positive entries the left singular matrix and eigenvalue matrix of $\vect{{\sf C}}_\text{t}$. 
%%$\vect{{\sf C}}^{-1/2}_\text{t} = \vect{\Lambda}_{\vect{{\sf C}}_\text{t}}^{-1/2} \vect{U}_{\vect{{\sf C}}_\text{t}}^{\Htran}$, given
%Comparing \eqref{time_avg_power_discrete_rewritten} against \eqref{Pt_inner_product_discretized} and factoring out the common term, we obtain
%\begin{equation} \label{coupling_relation}
%[\vect{{\sf C}}^{1/2}_\text{t}]_{m,m^\prime} \approx \frac{{\rm m}(S_\text{t})}{N_\text{t}} c^{1/2}_{\text t}(\vect{s}_m,\vect{s}_m^\prime)
%\end{equation}
%describing in which way the coupling square-root matrix approximates the coupling impulse response.
%From \eqref{input_vector_C}, it follows that
%\begin{equation}
%\sum_{m^\prime} [\vect{{\sf C}}^{1/2}_\text{t}]_{m,m^\prime} [\vect{j}_\text{t}]_{m^\prime} \approx (\mathcal{C}^{1/2}_\text{t} j_\text{t})(\vect{s}_m).
%\end{equation}

\subsection{Transmit Power} \label{app:power}

Rewrite \eqref{circuit_power} as an inner product in $L^2$,
\begin{align} \label{pt_conj}
{\sf P}_\text{t} & =  \Re \langle \mathcal{Z}_{\text{t,t}} j_\text{t}, j_\text{t}\rangle  = \frac{1}{2} \langle \mathcal{Z}_{\text{t,t}} j_\text{t}, j_\text{t}\rangle + \frac{1}{2} \overline{\langle \mathcal{Z}_{\text{t,t}} j_\text{t}, j_\text{t}\rangle}.
\end{align}
Then, using the conjugate symmetry of the inner product with the nonconjugate symmetry and self-adjointness of $\mathcal{Z}_{\text{t,t}}$, 
\begin{align} 
\overline{\langle \mathcal{Z}_{\text{t,t}} j_\text{t}, j_\text{t}\rangle}  =  \langle j_\text{t},  \mathcal{Z}_{\text{t,t}} j_\text{t} \rangle  
& =  \langle  {\mathcal{Z}_{\text{t,t}}}^* j_\text{t}, j_\text{t} \rangle \\ \label{conj}
& =  \langle  \overline{\mathcal{Z}_{\text{t,t}}} j_\text{t}, j_\text{t} \rangle.
\end{align}
Plugging \eqref{conj} into \eqref{pt_conj} yields, by linearity,
\begin{align} \label{Pt_operator}
{\sf P}_\text{t} & = \frac{1}{2} \langle \mathcal{Z}_{\text{t,t}} j_\text{t}, j_\text{t} \rangle + \frac{1}{2} \langle  \overline{\mathcal{Z}_{\text{t,t}}} j_\text{t}, j_\text{t} \rangle 
=  \langle \Re\{\mathcal{Z}_{\text{t,t}}\} j_\text{t}, j_\text{t} \rangle
\end{align}
with $\Re\{\mathcal{Z}_{\text{t,t}}\} = (\mathcal{Z}_{\text{t,t}} + \overline{\mathcal{Z}_{\text{t,t}}})/2$ the operator associated with the real part of the impedance kernel. 


\subsection{Transmit Impedance with Physical Antennas} \label{app:impedance_kernel} 

Replacing $g(\vect{\cdot})$ in \eqref{bb} with its spectral representation via Weyl's identity in \eqref{Weyl} yields \eqref{bb0}. Here, the $z$-axis is aligned to the axis connecting the centroids of any two antennas, as shown in Fig.~\ref{fig:impedance_corr_tot}. This transformation is allowed by the rotational invariance of \eqref{Weyl}. 
% is rotationally invariant, axes can be rotated arbitrarily, accounting for the more general case where  are not aligned with the $z$-axis. This invariance leads to the transmit impedance reported in \eqref{impedance_kernel_spectral_noniso}.
Taking the 2D Fourier transform on a plane orthogonal to $z$ gives \eqref{bb1}.
%With the change of variables $r_z^\prime  = p_z-r_z$ and $s_z^\prime  = q_z-s_z$ applied to \eqref{bb1}, we obtain \eqref{bb2}.
Reflecting the space limitation of physical antennas into the integration region leads to \eqref{bb3}. 
Assume $r_z > s_z > 0$. With $p_z = r_z-r_0$ and $q_z=s_z+r_0$, it follows that $|p_z-q_z| = |r_z-s_z-2 r_0|$. Then, $|p_z-q_z|=p_z-q_z$ for $r_z-s_z > 2 r_0$, corresponding to causal antennas; see Fig.~\ref{fig:impedance_corr_tot}. Similarly, assume now $s_z > r_z > 0$. With $p_z = r_z+r_0$ and $q_z=s_z-r_0$, it follows that $|p_z-q_z| = |r_z-s_z+2 r_0|$. Thus, $|p_z-q_z| = q_z-p_z$ for $r_z-s_z <- 2 r_0$, implying anticausal antennas.
In these regimes, \eqref{bb3} simplifies to \eqref{z_impedance} as per translation property of the Fourier transform. 
Generalizing \eqref{z_impedance} to arbitrary rotations leads to the transmit impedance reported in \eqref{impedance_kernel_spectral_noniso}.
 %Here, the $z$-axis is the one connecting the centroids of the two antennas for which the impedance expression is computed, as shown in Fig.~\ref{fig:impedance_corr}.
%The impedance kernel models the interaction between any two antennas placed at $\vect{{\sf r}}$ and $\vect{{\sf s}}$, with the spatial variables defined w.r.t. the transmit coordinate system $(\hat{\vect{x}}_{\text{t}},\hat{\vect{y}}_{\text{t}},\hat{\vect{z}}_{\text{t}})$. However, using a global coordinate system is overly restricting as ruling out antenna configurations where the condition $|r_z-s_z| > 2 r_0$ does not hold, yet antennas are physically separated. This is the case, e.g., of planar arrays where antennas deployed on the $xy$-plane, for which $r_z=s_z$.
%For the sake of generality, we assume an implicit rotational coordinate transformation was applied in \eqref{z_impedance} such that the formula was defined w.r.t. an antenna reference system $(\hat{\vect{x}}_{\text{t}^\prime},\hat{\vect{y}}_{\text{t}^\prime},\hat{\vect{z}}_{\text{t}^\prime})$, placed on the first antenna centroid and aligned such that the $z$-axis passes through the second antenna centroid, 



\subsection{Normalization of the Coupling Kernel} \label{app:normalization_impedance}

Changing of variables $\vect{{\sf t}} = \vect{{\sf p}}-\vect{{\sf r}}$ and $\vect{{\sf y}} = \vect{{\sf q}}-\vect{{\sf s}}$ in \eqref{bb} while defining $\vect{{\sf v}} = \vect{{\sf r}}-\vect{{\sf s}}$,
\setcounter{equation}{131}
\begin{align}   \label{cc}
z_{\text{t,t}}(\vect{{\sf v}})  & = -\imagunit \kappa Z_0  \iiint_{-\infty}^\infty \!\!\! d\vect{{\sf t}}  \, {\sf a}_\text{t}(\vect{{\sf t}}) \iiint_{-\infty}^\infty \!\!\! d\vect{{\sf y}} \, g(\vect{{\sf v}} + \vect{{\sf t}}-\vect{{\sf y}}) {\sf a}_\text{t}(\vect{{\sf y}}).
\end{align}
Recalling the Green's function spectrum in \cite[Eq.~2.2.23]{ChewBook},
\begin{align} \label{Green_fourier}
G(\vect{{\sf k}})  & = \frac{1}{\|\vect{{\sf k}}\|^2 - \kappa^2} = \frac{1}{(\kappa_z - \gamma)(\kappa_z + \gamma)},
\end{align}
we have that
\begin{align} \label{Green_fourier_shifted}
g(\vect{{\sf v}} + \vect{{\sf t}}-\vect{{\sf y}}) = \iiint_{-\infty}^\infty \frac{d\vect{{\sf k}}}{(2\pi)^3}  \,  G(\vect{{\sf k}}) e^{\imagunit \vect{{\sf k}}^{\Ttran} (\vect{{\sf t}}-\vect{{\sf y}})}   \, e^{\imagunit \vect{{\sf k}}^{\Ttran} \vect{{\sf v}}}.
\end{align}
Substituting \eqref{Green_fourier_shifted} into \eqref{cc} yields
\begin{align} \label{cc1}
z_{\text{t,t}}(\vect{{\sf v}})  & = -\imagunit \kappa Z_0  \iiint_{-\infty}^\infty \frac{d\vect{{\sf k}}}{(2\pi)^3}  \, |{\sf A}_\text{t}(\vect{{\sf k}})|^2 G(\vect{{\sf k}})  \, e^{\imagunit \vect{{\sf k}}^{\Ttran} \vect{{\sf v}}} 
\end{align}
which is due to Hermitian symmetry ${\sf A}_{\text t}(-\vect{{\sf k}}) = \overline{{\sf A}_{\text t}(\vect{{\sf k}})}$, as per ${\sf a}_\text{t}(\vect{{\sf r}})$ being a real-valued function. 
Comparing \eqref{cc1} against \eqref{normalization}, the real part of the transmit impedance reads as
\begin{align}
{\sf c}_\text{t}(\vect{{\sf v}})  & = \Re\left\{- \imagunit \frac{4 \pi}{\kappa}  \!\! \iiint_{-\infty}^\infty  \! \frac{d\vect{{\sf k}}}{(2\pi)^3}  \, |{\sf A}_\text{t}(\vect{{\sf k}})|^2 G(\vect{{\sf k}})  \, e^{\imagunit \vect{{\sf k}}^{\Ttran} \vect{{\sf v}}} \right\}.
\end{align}
Focusing on $\vect{{\sf v}}=\vect{{\sf 0}}$, the transmit self-impedance is given by
\begin{align} 
{\sf c}_\text{t}(\vect{{\sf 0}})  & = \Re\left\{- \frac{\imagunit}{2\pi^2 \kappa}  \iiint_{-\infty}^\infty d\vect{{\sf k}}  \, |{\sf A}_\text{t}(\vect{{\sf k}})|^2 G(\vect{{\sf k}})  \right\} \\ \label{c0}
& = \Re\left\{- \frac{\imagunit}{2\pi^2 \kappa}  \iiint_{-\infty}^\infty d\vect{{\sf k}}  \, \frac{|{\sf A}_\text{t}(\vect{{\sf k}})|^2}{(\kappa_z - \gamma)(\kappa_z + \gamma)} \right\}
\end{align}
in light of \eqref{Green_fourier}.
%While the expression in brackets appears purely imaginary, 
The integration over $\kappa_z$ reveals a nonzero real part of the impedance. Specifically, for an analytic function $f(z)$ with a simple pole $z_0 \in \Real$, crossing the singularity replaces the integral with its Cauchy principal value, yielding
 %Though the expression between brackets appears purely imaginary, integrating over $\kappa_z$ reveals that the real part of the impedance is nonzero.
%Precisely, consider the integration on the real axis of an analytic function $f(z)$ including a simple pole $z_0 \in \Real$. 
%As the integration path crosses the singularity, the integral is replaced by its Cauchy principal value, yielding
\begin{align} \label{prova}
\mathrm{PV} \int_{-\infty}^\infty dz  \,  \frac{f(z)}{z - z_0}  =  \imagunit \pi f(z_0)
\end{align}
where the integral exists.
Rewriting the integrand of \eqref{c0} as its partial fraction decomposition 
%\begin{align} \label{poles}
%\frac{|{\sf A}_\text{t}(\vect{{\sf k}})|^2}{(\kappa_z - \gamma)(\kappa_z + \gamma)} = \frac{|{\sf A}_\text{t}(\vect{{\sf k}})|^2}{2 \gamma (\kappa_z - \gamma)} + \frac{|{\sf A}_\text{t}(\vect{{\sf k}})|^2}{2 \gamma (\kappa_z + \gamma)}
%\end{align}
and integrating over $\kappa_z$,
\begin{align}  \nonumber
& \mathrm{PV}  \int_{-\infty}^\infty d\kappa_z  \, \frac{|{\sf A}_\text{t}(\vect{{\sf k}})|^2}{(\kappa_z - \gamma)(\kappa_z + \gamma)}  \\ \label{PV1}
& \hspace{.5cm} =   \mathrm{PV}
\int_{-\infty}^\infty d\kappa_z  \, \left( \frac{|{\sf A}_\text{t}(\vect{{\sf k}})|^2}{2 \gamma (\kappa_z - \gamma)} + \frac{|{\sf A}_\text{t}(\vect{{\sf k}})|^2}{2 \gamma (\kappa_z + \gamma)} \right) \\ \label{PV}
& \hspace{.5cm} = \imagunit \pi \left(\frac{|{\sf A}_\text{t}^+(\vect{\kappa})|^2}{2\gamma}  + \frac{|{\sf A}_\text{t}^-(\vect{\kappa})|^2}{2\gamma} \right)
\end{align}
with the spectra ${\sf A}_\text{t}^\pm(\vect{\cdot})$ as defined in \eqref{impedance_kernel_spectral_noniso}.
Note that the above integration requires the poles $\kappa_z = \pm \gamma$ to lie on the real axis, a condition satisfied for $\|\vect{k}\|\le\kappa$, where $\gamma \in \Real$ as per \eqref{gamma}.
%where $A_\text{t}^\pm(\vect{\kappa}) = A_\text{t}(\vect{\kappa},\pm \gamma)$.
Finally, substituting \eqref{PV} into \eqref{c0},
\begin{align} \label{c2}
{\sf c}_\text{t}(\vect{{\sf 0}})  
%&= \Re\bigg\{\frac{1}{4 \pi \kappa}  \iint_{\|\vect{\kappa}\|\le\kappa}\!\!\!\!\!\!\!\! d\vect{\kappa}  \, \frac{|{\sf A}_\text{t}^+(\vect{\kappa})|^2  + |{\sf A}_\text{t}^-(\vect{\kappa})|^2}{\gamma}  \bigg\} \\
& = \frac{1}{4 \pi \kappa}  \iint_{\|\vect{\kappa}\|\le\kappa} d\vect{\kappa}  \, \frac{|{\sf A}_\text{t}^+(\vect{\kappa})|^2  + |{\sf A}_\text{t}^-(\vect{\kappa})|^2}{\gamma(\vect{\kappa})},
\end{align}
which leads to the normalization condition in \eqref{norm_A_spectrum}.

%\section{} \label{app:normalization}
%
%Tailoring the normalization condition \eqref{normalization_channel} to the channel between coupled lossless antennas yields
%\begin{equation} \label{normalization_channel_coupled}
%\Ex\{|{\sf h}(\vect{r},\vect{s})|^2\} =1,
%\end{equation} 
%which requires the unitarity of the following integral
%\begin{align}
%1 & = \iint_{\|\vect{k}\|\le\kappa} \frac{d\vect{k}}{2\pi \kappa} \iint_{\|\vect{\kappa}\|\le\kappa} \frac{d\vect{\kappa}}{2\pi \kappa} \frac{\Ex\{|{\sf H}^{++}(\vect{k},\vect{\kappa})|^2\}}{\gamma(\vect{k}) \gamma(\vect{\kappa})}.
%%\\& 
%%= \frac{c}{(2\pi \kappa)^2}  \iint_{\|\vect{k}\|\le\kappa} d\vect{k} \iint_{\|\vect{\kappa}\|\le\kappa} d\vect{\kappa}  \frac{\Ex\{|H^{++}(\vect{k},\vect{\kappa})|^2\}}{\gamma(\vect{k}) |A_{\text t}^+(\vect{\kappa})|^2}
%\end{align}
%Replacing ${\sf H}^{++}(\vect{k},\vect{\kappa})$ with its expression in \eqref{spectrum_HC},
%\begin{align}
%1 & =  \frac{1}{(2\pi \kappa)^2} \iint_{\|\vect{k}\|\le\kappa} d\vect{k} \iint_{\|\vect{\kappa}\|\le\kappa} d\vect{\kappa}  \frac{\Ex\{|H^{++}(\vect{k},\vect{\kappa})|^2\}}{\gamma(\vect{k}) |A_{\text t}^+(\vect{\kappa})|^2} \\& \notag
%= \frac{\kappa}{(2\pi)^2} \int_{0}^{\pi/2} \!\! \int_0^{2\pi} \int_{0}^{\pi/2} \!\! \int_0^{2\pi} \frac{\Ex\{|H^{++}(\theta_{\text r},\phi_{\text r},\theta_{\text t},\phi_{\text t})|^2\}}{|A_{\text t}^+(\theta_{\text t},\phi_{\text t})|^2} \\& \hspace{2cm} \label{norm_channel_coupled}
%\cdot  \sin \theta_{\text r} \sin \theta_{\text t} \cos \theta_{\text t} \, d\theta_{\text r} d\phi_{\text r} d\theta_{\text t} d\phi_{\text t}
% \end{align}
%where we have changed variables according to \eqref{wavenumber_spherical}.
%
%Now, specializing \eqref{norm_channel_coupled} to isotropic scattering and omnidirectional antennas, respectively $\Ex\{|H^{++}(\theta_{\text r},\phi_{\text r},\theta_{\text t},\phi_{\text t})|^2\}=1$ and $|A_{\text t}^+(\theta_{\text t},\phi_{\text t})|^2=1$, while multiplying ${\sf h}(\vect{r},\vect{s})$ by a constant $c$ to ensure the normalization, we shall have
%\begin{align} \label{norm_c}
%c & = \sqrt{\frac{2}{\kappa}}
%\end{align}
%by virtue of 
%\begin{align}  \label{DOF_Landau_omni}
%\int_{0}^{\pi/2} \!\! \int_0^{2\pi} \cos \theta_\text{t} \, \sin \theta_\text{t} \, d\theta_\text{t} \, d\phi_\text{t} = 2\pi \int_0^1 dy \,y = \pi
%\end{align}
%with substitution $y=\sin\theta_\text{t}$ and $\int_{0}^{\pi/2} \!\! \int_0^{2\pi} \sin \theta d\theta d\phi = 2\pi$.
%%\begin{equation} \label{normalization_channel_coupled}
%%\Ex\{|h_\mathcal{C}(\vect{r},\vect{s})|^2\} = \mathop{\sum}_{\vect{i} \in \Lambda_\text{r}} \mathop{\sum}_{\vect{j} \in \Lambda_\text{t}}  \sigma^2_{\vect{i}\vect{j}}(\mathcal{H}_\mathcal{C}) =1,
%%\end{equation} 

\subsection{Existence of Deconvolution} \label{app:invertible_Chalf}
 
 The invertibility of $\mathcal{C}^{1/2}$ is proven next by contradiction. 
 Assume that ${\sf P}_\text{t} > 0$, thereby making $\mathcal{C}$ positive-definite as per \eqref{psd_op}, and that $|{\sf A}_{\text{t}}(\vect{{\sf k}})|$ is only nonnegative, meaning this function could be zero in a non-empty region $K\subset \Real^3$. Then, a current spectrum $J_{\text t}(\vect{{\sf k}})$ could be chosen to be zero everywhere except within $K$, whereby the associated power density ${\sf S}_{\text t}(\vect{{\sf k}})$ in \eqref{psd_Dirac_LSI_noniso} would be zero $\forall \vect{{\sf k}}$, implying ${\sf P}_\text{t} = 0$ as per \eqref{power_psd}. However, this is not allowed given that ${\sf P}_\text{t} > 0$ by hypothesis.


%\section{Coordinate Transformation} \label{app:coordinate_transf}
%
%\begin{figure}[t!]
%\centering\vspace{-0.0cm}
%\includegraphics[width=.75\linewidth]{ref_system} 
%\caption{Transmit and antenna Cartesian coordinate systems.}\vspace{-0cm}
%\label{fig:ref_system}
%\end{figure}
%
%The impedance kernel models the interaction between any two antennas placed at $\vect{{\sf r}}$ and $\vect{{\sf s}}$, with the spatial variables defined w.r.t. the transmit coordinate system $(\hat{\vect{x}}_{\text{t}},\hat{\vect{y}}_{\text{t}},\hat{\vect{z}}_{\text{t}})$. However, using a global coordinate system is overly restricting as ruling out antenna configurations where the condition $|r_z-s_z| > 2 r_0$ does not hold, yet antennas are physically separated. This is the case, e.g., of planar arrays where antennas deployed on the $xy$-plane, for which $r_z=s_z$.
%For the sake of generality, we assume an implicit rotational coordinate transformation was applied in \eqref{z_impedance} such that the formula was defined w.r.t. an antenna reference system $(\hat{\vect{x}}_{\text{t}^\prime},\hat{\vect{y}}_{\text{t}^\prime},\hat{\vect{z}}_{\text{t}^\prime})$, placed on the first antenna centroid and aligned such that the $z$-axis passes through the second antenna centroid, namely
%\begin{align} \label{eza}
%\hat{\vect{z}}_{\text{t}^\prime}(\hat{\vect{{\sf v}}}) =  \frac{\vect{{\sf v}}}{\|\vect{{\sf v}}\|} = \hat{\vect{{\sf v}}}
%\end{align}
%with $\vect{{\sf v}} = \vect{{\sf r}}-\vect{{\sf s}}$ the space-lag between corresponding centroids.
%The remaining basis vectors are arbitrarily found by ensuring orthogonality with \eqref{eza}, thus leading to the general form
% \begin{align} \label{exya}
%\hat{\vect{x}}_{\text{t}^\prime}(\hat{\vect{{\sf v}}},\hat{\vect{u}}) & =  \frac{\hat{\vect{{\sf v}}} \times \hat{\vect{u}}}{\|\hat{\vect{{\sf v}}} \times \hat{\vect{u}}\|}  \\
%\hat{\vect{y}}_{\text{t}^\prime}(\hat{\vect{{\sf v}}},\hat{\vect{u}}) & =  \frac{\hat{\vect{{\sf v}}} \times \hat{\vect{{\sf v}}} \times \hat{\vect{u}}}{\|\hat{\vect{{\sf v}}} \times \hat{\vect{{\sf v}}} \times \hat{\vect{u}}\|}
%\end{align}
%for any unit-norm vector $\hat{\vect{u}} \in \Real^3$.
%%Suppose that we transform to a new coordinate system, $x'$, $y'$, $z'$, that is obtained from the $x$, $y$, $z$ system by rotating the coordinate axes through an angle $\theta $ about the $z$-axis. See Figure A.1. Let the coordinates of a general point $P$ be  $(x,\,y,\,z)$ in the first coordinate system, and  $(x',\, y',\, z')$ in the second. According to simple trigonometry, these two sets of coordinates are related to one another via the transformation:
%Thus, the rotational coordinate transformation turning $(\hat{\vect{x}}_{\text{t}^\prime},\hat{\vect{y}}_{\text{t}^\prime},\hat{\vect{z}}_{\text{t}^\prime})$ into $(\hat{\vect{x}}_{\text{t}},\hat{\vect{y}}_{\text{t}},\hat{\vect{z}}_{\text{t}})$ is specified by the unitary rotation matrix
%\begin{align}
%\vect{\Theta}(\hat{\vect{{\sf v}}},\hat{\vect{u}}) =
%\begin{bmatrix}
%\hat{\vect{x}}_{\text{t}}^{\Ttran} \\
%\hat{\vect{y}}_{\text{t}}^{\Ttran} \\
%\hat{\vect{z}}_{\text{t}}^{\Ttran}
%\end{bmatrix}
%\begin{bmatrix}
%\hat{\vect{x}}_{\text{t}^\prime}(\hat{\vect{{\sf v}}},\hat{\vect{u}}),
%\hat{\vect{y}}_{\text{t}^\prime}(\hat{\vect{{\sf v}}},\hat{\vect{u}}),
%\hat{\vect{z}}_{\text{t}^\prime}(\hat{\vect{{\sf v}}})
%\end{bmatrix}.
%\end{align}
%Note that any translation of the antenna coordinate system cancels out when considering space-lag variables.
 
\subsection{Maximum Eigenvalue} \label{app:max_eigenvalue}

%The maximum eigenvalue of an hermitian matrix $\tilde{\vect{{\sf H}}} \tilde{\vect{{\sf H}}}^{\Htran}$ equals its spectral norm $\lambda_\text{max}(\tilde{\vect{{\sf H}}} \tilde{\vect{{\sf H}}}^{\Htran}) = \|\tilde{\vect{{\sf H}}}\|_2^2$ \cite[Ch.~5.6]{HornBook}.
%\begin{equation} \label{max_eig_norm}
%\lambda_\text{max}(\tilde{\vect{{\sf H}}} \tilde{\vect{{\sf H}}}^{\Htran}) = \max_{\|\vect{x}\|=1} \|\tilde{\vect{{\sf H}}} \, \vect{x}\|^2 = \|\tilde{\vect{{\sf H}}}\|_2^2.
%\end{equation}
For a single channel realization, the maximum eigenvalue of $\tilde{\vect{{\sf H}}}$ in \eqref{equiv_channel_coupling} equals its spectral norm. Repeatedly applying the sub-multiplicativity property of a matrix norm,
\begin{align}
\lambda_\text{max}(\tilde{\vect{{\sf H}}} \tilde{\vect{{\sf H}}}^{\Htran})
& = \|\tilde{\vect{{\sf H}}}\|_2^2 \\
& \approx \| \vect{\Lambda}_{\text{r}}^{1/2} \vect{W} \vect{{\sf \Lambda}}_{\text{t}}^{1/2} \|_2^2 \\& 
\le \|\vect{\Lambda}_{\text{r}}^{1/2} \|_2^2  \|\vect{W} \|_2^2 \| \vect{{\sf \Lambda}}_{\text{t}}^{1/2} \|_2^2   \\&
= \| \vect{\Lambda}_{\text{r}} \|_2  \|\vect{W} \|_2^2 \| \vect{{\sf \Lambda}}_{\text{t}} \|_2   \\& \label{lambda_HC}
=  \lambda_\text{max}(\vect{\Lambda}_{\text{r}}) \lambda_\text{max}(\vect{W} \vect{W}^{\Htran}) \lambda_\text{max}(\vect{{\sf \Lambda}}_{\text{t}}),
\end{align}
%where $\lambda_\text{min}(\vect{{\sf C}}_\text{t}) = \min_{k=1, \ldots, N_\text{t}} \lambda_k(\vect{{\sf C}}_\text{t})$ and 
%\begin{align}  \label{lambda_max_HH_exact}
%\lambda_\text{max}(\vect{H} \vect{H}^{\Htran}) & = \max_{i=1, \ldots, N_\text{min}} \lambda_i(\vect{H} \vect{H}^{\Htran})
%= \max_{i=1, \ldots, N_\text{min}} |\lambda_i(\vect{H})|^2.
%\end{align}
%In the regime $\min(L_\text{t},L_\text{r})/\lambda\gg1$, \eqref{lambda_max_HH_exact} is characterized by
%\begin{equation} \label{lambda_max_HH}
%\lambda_\text{max}(\vect{H} \vect{H}^{\Htran}) = N_{\text r} N_{\text t} \max_{i=1, \ldots, {\sf n}_\text{r} {\sf n}_\text{t}} |\widehat{\lambda}_i(\mathcal{H})|^2
%\end{equation}
%with $\widehat{\lambda}_i(\mathcal{H})$ independent and distributed as $\widehat{\lambda}_i(\mathcal{H})\sim\CN(0,  \sigma^2_i(\mathcal{H}))$, given $\sigma^2_i(\mathcal{H})$, $i=1, \ldots, {\sf n}_\text{r} {\sf n}_\text{t}$, the vectorization of the 4D lattice \eqref{variances_channel}.
%%, indexed by $i=(j-1){\sf n}_\text{t} + k$ with $j=1, \ldots, {\sf n}_\text{r}$ and $k=1, \ldots, {\sf n}_\text{t}$. In turn, 
%As for the coupling, the transmit contribution to a separable channel in \eqref{spectral_representation} is
%\begin{align}  
%h_\text{t}(\vect{s}) & = \frac{1}{\sqrt{4\pi \kappa}} \iint_{\|\vect{\kappa}\|\le\kappa} \frac{H^{+}_\text{t}(\vect{\kappa})}{\sqrt{\gamma}} e^{-\imagunit \vect{\kappa}^{\Ttran} \vect{s}} \, d\vect{\kappa}
%\end{align}
%with $H^{+}_\text{t}(\vect{\kappa})$ an independent complex Gaussian process.
%The transmit correlation $\rho_\text{t}(\vect{v}) = \Ex\{h_\text{t}(\vect{s}+\vect{v}) h_\text{t}^*(\vect{s})\}$ reads 
%\begin{align}   \label{corr_tx}
%\rho_\text{t}(\vect{v}) & = \frac{1}{4\pi \kappa} \iint_{\|\vect{\kappa}\|\le\kappa} \frac{\Ex\{|H^{+}_\text{t}(\vect{\kappa})|^2\}}{\gamma} e^{-\imagunit \vect{\kappa}^{\Ttran} \vect{v}} \, d\vect{\kappa}.
%\end{align}
%Invoking the isomorphism between coupling and spatial correlation, we compare \eqref{corr_tx} against \eqref{real_impedance_kernel_antenna} and approximate the latter within $\|\vect{v}\|_\infty \le L_{\text t}$ via the Fourier-Stieltjes series expansion, similar to \eqref{Fourier_series}, yielding
%\begin{align} 
%{\sf c}_\text{t}(\vect{v}) & = 
%\mathop{\sum}_{\vect{m} \in \Lambda_\text{t}}
%\lambda_{\vect{m}}(\mathcal{C}_\text{t}) \, v_\vect{m}(\vect{s}+\vect{v}) v^*_\vect{m}(\vect{s}) \\ \label{Fourier_series_corr_tx}
%& = \mathop{\sum}_{\vect{m} \in \Lambda_\text{t}}
%\sigma^2_{\vect{m}}(\mathcal{C}_\text{t}) \, e^{\imagunit 2\pi \boldsymbol{m}^{\Ttran} \vect{v}/L_{\text t}}
%\end{align}
%where $v_\vect{m}(\vect{s})$ is the 2D Fourier basis function at transmitter while $\lambda_{\vect{m}}(\mathcal{C}_\text{t}) = L^2_\text{t} \sigma^2_{\vect{m}}(\mathcal{C}_\text{t})$ with
%\begin{align}  \label{variances_corr_tx}
%\sigma^2_{\vect{m}}(\mathcal{C}_\text{t}) & 
% =
%\frac{1}{2\pi} \iint_{\Omega_{{\text t},\vect{m}}}  \Ex\{|A^{+}_\text{t}(\theta_{\text t},\phi_{\text t})|^2\}
%\sin \theta_{\text t} d\theta_{\text t} d\phi_{\text t}
%\end{align}
%with $\Omega_{{\text t},\vect{m}}$ the solid angles in \eqref{variances_channel}.
%Sampling \eqref{Fourier_series_corr_tx} according to $[\vect{{\sf C}}_{\text{t}}]_{m,m^\prime} = {\sf c}_{\text t}(\vect{r}_m-\vect{s}_m^\prime)$ for $m,m^\prime = 1, \ldots, N_\text{t}$ returns the singular-value decomposition of the coupling matrix
%\begin{align} \label{coupling_svd}
%\vect{{\sf C}}_\text{t} & = \mathop{\sum}_{\vect{m} \in \Lambda_\text{t}}  \lambda_{\vect{m}}(\vect{{\sf C}}_\text{t}) \, \vect{v}_\vect{m} \vect{v}^{\Htran}_\vect{m},
%\end{align}
%where $\lambda_{\vect{m}}(\vect{{\sf C}}_\text{t}) = N_\text{t} \sigma^2_{\vect{m}}(\mathcal{C}_\text{t})$ 
%while $\vect{v}_\vect{m}$ is the $\vect{m}$th column of the Fourier eigenvector matrix.
%Using $\lambda_\text{max}(\vect{A}^{-1}) =  {1}/{\lambda_\text{min}(\vect{A})}$ for any invertible matrix $\vect{A}$, $\lambda_\text{min}(\vect{\cdot})$ denoting the minimum eigenvalue, 
%\begin{align} \label{lambda_HC}
%\lambda_\text{max}(\tilde{\vect{{\sf H}}} \tilde{\vect{{\sf H}}}^{\Htran})
%& \approx  \lambda_\text{max}(\vect{\Lambda}_{\text{r}}) \, \lambda_\text{max}(\vect{W} \vect{W}^{\Htran}) \, \lambda_\text{max}(\vect{{\sf \Lambda}}_{\text{t}}),
%\end{align}
%where
%\begin{align}  \label{lambda_A_min}
%\lambda_\text{min}(\vect{{\sf A}}_\text{t}) & 
%= \min_{\vect{j} \in \Lambda_\text{t}} |{\sf A}_\text{t}^+(\theta_{{\text t},\vect{j}},\phi_{{\text t},\vect{j}})|^2.
%%= \frac{1}{2} \min_{j=1, \ldots, {\sf n}_\text{t}} \frac{|A_\text{t}^+(\theta_{{\text t},j},\phi_{{\text t},j})|^2}{\cos\theta_{{\text t},j}}
%\end{align}
where $\lambda_\text{max}(\vect{\Lambda}_{\text{r}})$ and $\lambda_\text{max}(\vect{{\sf \Lambda}}_{\text{t}})$ denote the maximum eigenvalues of the receive and transmit correlations, given by
\begin{align}  \label{eig_rx_max}
\lambda_\text{max}(\vect{\Lambda}_{\text{r}}) & = N_{\text{r}} \, \max_{\vect{i} \in \Lambda_\text{r}} \sigma^2_{\vect{i}}(\tilde{H}_\text{r}^+)  \\ \label{eig_tx_max}
%\max_{\vect{i} \in \Lambda_\text{r}} {\rm m}({\Omega}_{{\text r},\vect{i}}) \Ex\{|H^+_\text{r}(\theta_{{\text r},\vect{i}},\phi_{{\text r},\vect{i}})|^2\} \\ \label{eig_tx_max}
\lambda_\text{max}(\vect{{\sf \Lambda}}_{\text{t}}) & =  N_{\text{t}} \, \max_{\vect{j} \in \Lambda_\text{t}} \sigma^2_{\vect{j}}(\tilde{{\sf H}}_\text{t}^+).
%\max_{\vect{j} \in \Lambda_\text{t}} {\rm m}({\sf \Omega}^+_{{\text t},\vect{j}}) \Ex\{|H^+_\text{t}(\theta_{{\text t},\vect{j}},\phi_{{\text t},\vect{j}})|^2\}.
\end{align}
The inequality \eqref{bound_eig_HC_discrete} is derived by averaging \eqref{lambda_HC} over all channel realizations while substituting \eqref{eig_rx_max} and \eqref{eig_tx_max}.
% , given $\Omega_i = \Omega_{{\text r},j} \times \Omega_{{\text t},k}$,  
%\begin{align} \notag
%\sigma^2_{\mathcal{H},i} & = \iiiint_{\Omega_i}  \Ex\{|H(\theta_{\text r},\phi_{\text r},\theta_{\text t},\phi_{\text t})|^2\}
%\\ & \hspace{2.5cm} \label{variances_channel_op}
% \cdot \sin \theta_{\text r} \sin \theta_{\text t} \, d\theta_{\text r} d\phi_{\text r} \, d\theta_{\text t} d\phi_{\text t},
%\end{align}
%Substituting \eqref{channel_mat_c_rx} while using $\lambda_{\vect{A}^{-1},i} = \lambda^{-1}_{\vect{A},i}$, $\forall i$,
%\begin{equation} \label{lambda_H_C_rx}
%\Ex\{\lambda_{\vect{H}_{\vect{{\sf C}}},1}\} \le \frac{1}{\lambda_{\vect{{\sf C}}_\text{r},N_\text{r}} \, \lambda_{\vect{{\sf C}}_\text{t},N_\text{t}}} \Ex\{\lambda_{\vect{H},1}\}
%\end{equation}
%with $\lambda_{\vect{H},1}$ the maximum eigenvalue of $\vect{H} \vect{H}^{\Htran}$, and $\lambda_{\vect{{\sf C}}_\text{r},N_\text{r}}$ and $\lambda_{\vect{{\sf C}}_\text{t},N_\text{t}}$ the minimum eigenvalues of $\vect{{\sf C}}_\text{r}$ and $\vect{{\sf C}}_\text{t}$, respectively.
%\begin{equation} \label{lambda_H}
%\lambda^{1/2}_{\vect{H},i} \approx \frac{\sqrt{N_\text{r} N_\text{t}}}{L_\text{t} L_\text{r}} \, \lambda^{1/2}_{\mathcal{H},i}, \quad i=1, \ldots, {\sf n}_\text{r} {\sf n}_\text{t},
%\end{equation}
%where
%\begin{equation} \label{lambda_calH}
%\lambda^{1/2}_{\mathcal{H},i} \sim\CN\left(0,(L_{\text r} L_{\text t})^2 \, \Ex\{|\lambda_{\mathcal{H},i}|\}\right)
%\end{equation}
%with $\Ex\{|\lambda_{\mathcal{H},i}|\}$ the sequence obtainable vectorizing the lattice of variances in \eqref{variances_channel}. 
%Note that the approximation in \eqref{lambda_H} becomes progressively more accurate as the antenna spacing shrinks within a given aperture. 
% Also, \eqref{lambda_calH} follows from the spectral representation of a stationary random field, which is accurate when $\min(L_\text{t},L_\text{r})/\lambda\gg1$. 
 %Their conjunction identifies a dense and electrically large array, both definitions subsumed by a holographic array.
%The substitution of the latter variance's expression into \eqref{lambda_H} gives

\subsection{Spatial DOF} \label{app:DOF}

%The rank of $\vect{H}$ dictates the number of spatial dimensions available for communications over a given channel with antennas being enough separated to be considered as uncoupled. Owing to \eqref{Nyquist_cond}, it is computed as the minimum dimension of the subspaces spanned by the scattering at either ends of the link, namely $\rank(\vect{H}) \le \min({\sf n}_\text{r},{\sf n}_\text{t})$ as per \eqref{n_rt}, with equality achieved by an isotropic environment \cite{PizzoTWC22,PizzoWCL22}. 
The asymptotic slope of the spectral efficiency is \cite{LozanoCorrelation}
\begin{align}
{\sf DOF} = \lim_{{\sf SNR}\to \infty} \frac{{\sf SNR}}{\log_2 e} \, \frac{d{\sf C}({\sf SNR})}{d{\sf SNR}}.
\end{align}
Substituting ${\sf C}({\sf SNR})$ from \eqref{capacity} yields \cite[Appendix~G]{LozanoCorrelation}
\begin{equation}
{\sf DOF} = \Ex \{ \rank(\tilde{\vect{{\sf H}}} \vect{{\sf P}}^\star \tilde{\vect{{\sf H}}}^{\Htran}) \}
\end{equation}
and, as at high SNR
%Moreover, since the entries of $\tilde{\vect{{\sf H}}}$ are independent, the rows and columns of $\tilde{\vect{{\sf H}}}$ are linearly independent with probability~$1$ except for those that are identically zero. 
%At infinite SNR,
waterfilling allocates power onto every direction associated with a nonzero eigenvalue of $\tilde{\vect{{\sf H}}} \tilde{\vect{{\sf H}}}^{\Htran}$  %the rank is not affected by $\vect{{\sf P}}^\star$ and
\begin{equation} \label{S_inf_UIU}
{\sf DOF} = \Ex \! \left \{ \rank(\tilde{\vect{{\sf H}}}^\prime (\tilde{\vect{{\sf H}}}^\prime)^{\Htran}) \right \},
\end{equation}
where $\tilde{\vect{{\sf H}}}^\prime$ is the ${\sf DOF}_\text{r}^\prime \times {\sf DOF}_\text{t}^\prime$ submatrix (${\sf DOF}_\text{r}^\prime \le {\sf DOF}_\text{r}$ and ${\sf DOF}_\text{t}^\prime \le {\sf DOF}_\text{t}$) obtained by removing the rows and columns tied to zero eigenvalues of $\tilde{\vect{{\sf H}}}$.
As the entries of $\tilde{\vect{{\sf H}}}^\prime$ are independent, its rows and columns are linearly independent with probability~$1$ save for those identically zero, resulting in \eqref{S_inf_UIU_rank}.
%\begin{equation} \label{S_inf_UIU_rank_app}
%{\sf DOF} = \min({\sf n}_\text{r}^\prime,{\sf n}_\text{t}^\prime).
%\end{equation}
For the separable model in \eqref{equiv_channel_coupling}, ${\sf DOF}_\text{r}^\prime$ and ${\sf DOF}_\text{t}^\prime$ %can be treated separately and 
correspond to the cardinalities of the 2D lattices 
\begin{align} \nonumber
\Lambda_\text{r}^\prime & = \Big\{\vect{i} \in \Integer^2 :  \{\|\vect{k}\| \le \kappa\} \bigcap \left\{\|\vect{D}_\text{r} \vect{k} - \kappa \vect{i}\|_\infty\le \tfrac{\kappa}{2} \right\} \\& \hspace{3cm} \label{lattice_rx_prime}
\bigcap {\rm supp}(\Ex\{|\tilde{H}^+_\text{r}|^2\}) \neq \emptyset \Big\} \\ \nonumber
\Lambda_\text{t}^\prime & = \Big\{\vect{j} \in \Integer^2 :  \{\|\vect{\kappa}\| \le \kappa\} \bigcap \left\{\|\vect{D}_\text{t} \vect{\kappa} - \kappa \vect{j}\|_\infty\le \tfrac{\kappa}{2} \right\} \\& \hspace{3cm} \label{lattice_tx_prime}
\bigcap {\rm supp}(\Ex\{|\tilde{{\sf H}}^+_\text{t}|^2\}) \neq \emptyset \Big\},
\end{align}
where the wavenumber disk is defined by stationarity, ${\rm supp}(\cdot)$ accounts for the scattering selectivity, and $\{\|\bm{\cdot}\|_\infty \le \tfrac{\kappa}{2}\}$ 
%describes rectangular sets with dimensions $(\frac{2\pi}{L_{{\text r},x}},\frac{2\pi}{L_{{\text r},y}})$ and centered at wavenumber frequencies $(\frac{2\pi}{L_{{\text r},x}} i_x,\frac{2\pi}{L_{{\text r},y}} i_y)$, which 
models the impact of array apertures \cite{PizzoTWC21}.
Normalization by $\kappa$ keeps the DOF unchanged in \eqref{lattice_rx_prime} and \eqref{lattice_tx_prime} while emphasizing their dependance on $\lambda$. With this applied normalization, at the receiver,
as $L_{\text{r},x}/\lambda$ increases, each set contracts, and \eqref{lattice_rx_prime} approximates the product of the packing density and wavenumber support:
\begin{align} \label{n_r_prime_wave_0}
{\sf n}_\text{r}^\prime & =  \left\lceil \det(\vect{D}_\text{r}) \, \iint_{\{\|\vect{k}\| \le 1\} \, \bigcap \, {\rm supp}(\Ex\{|\tilde{H}^+_\text{r}|^2\})} \, d\vect{k} \right\rceil \\ \label{n_r_prime_wave}
& =  \left\lceil {\sf n}_\text{r} \cdot \frac{1}{\pi} \iint_{\{\|\vect{k}\| \le 1\} \, \bigcap \, {\rm supp}(\Ex\{|\tilde{H}^+_\text{r}|^2\})} \, d\vect{k} \right\rceil,
\end{align}
subsuming ${\sf n}_\text{r}$ in \eqref{n_rt} under isotropic scattering. %where ${\rm supp}(\Ex\{|\tilde{H}^+_\text{r}|^2\}) = \{\|\vect{k}\| \le 1\}$. }
The transmit formula is derived analogously to \eqref{n_r_prime_wave}, but with a coupling-inclusive power spectrum as
\begin{align} \label{n_t_prime_wave} 
{\sf n}_\text{t}^\prime & =  \left\lceil {\sf n}_\text{t} \cdot \frac{1}{\pi} \iint_{\{\|\vect{\kappa}\| \le 1\} \, \bigcap \, {\rm supp}(\Ex\{|\tilde{{\sf H}}^+_\text{t}|^2\})} \, d\vect{\kappa} \right\rceil.
\end{align}
From \eqref{n_r_prime_wave} and \eqref{n_t_prime_wave}, shifting from wavenumber representation (with axes rescaled by $\kappa$) to spherical as per \eqref{wavenumber_spherical} yields \eqref{n_r_prime} and \eqref{n_t_prime}.

%\section{Power Offset} \label{app:power_offset}
%
%Truncating the sum in \eqref{capacity} to those indices corresponding to ${\sf n}_\text{min}^\prime$ nonzero channel eigenvalues in \eqref{S_inf_UIU_rank}, the ergodic capacity can be approximated by
%\begin{align} \label{capacity_snr}
%{\sf C}({\sf SNR}) \approx  \Ex \left\{ \sum_{i=1}^{{\sf n}_\text{min}^\prime} \log_2 \! \left(1 + {\sf SNR} \, \tilde{{\sf p}}_i^\star \lambda_{i}(\tilde{\vect{{\sf H}}}^\prime (\tilde{\vect{{\sf H}}}^\prime)^{\Htran}) \right)\right\}
%\end{align}
%with $\tilde{{\sf p}}_i^\star = {\sf p}_i^\star/{\sf SNR}$ such that $\sum_{i=1}^{{\sf n}_\text{min}^\prime} \tilde{{\sf p}}_i^\star = 1$ the normalized power coefficients associated with the nonzero eigenvalues of $\tilde{\vect{{\sf H}}}^\prime$ in \eqref{S_inf_UIU}.
%In the high-SNR regime, equal power allocation is asymptotically optimal whereby $\tilde{{\sf p}}_i^\star = 1/{\sf n}_\text{min}^\prime$ for $i=1, \ldots, {\sf n}_\text{min}^\prime$, and \eqref{capacity_snr} becomes
%\begin{align} \label{capacity_snr_equal}
%{\sf C}({\sf SNR}) \approx  \Ex \left\{ \sum_{i=1}^{{\sf n}_\text{min}^\prime} \log_2 \! \left(1 + \frac{{\sf SNR}}{{\sf n}_\text{min}^\prime} \lambda_{i}(\tilde{\vect{{\sf H}}}^\prime (\tilde{\vect{{\sf H}}}^\prime)^{\Htran}) \right)\right\}.
%\end{align}
%
%For ${\sf n}_\text{t}^\prime > {\sf n}_\text{r}^\prime$, we have that ${\sf n}_\text{min}^\prime = {\sf n}_\text{r}^\prime$ and $\tilde{\vect{{\sf H}}}^\prime (\tilde{\vect{{\sf H}}}^\prime)^{\Htran}$ is nonsingular with probability $1$.
%From \eqref{capacity_snr}, while using \eqref{S_inf_UIU_rank},
%\begin{align} 
%\frac{{\sf C}({\sf SNR})}{{\sf S}_\infty^\star(\tilde{\vect{{\sf H}}})} & \approx \frac{1}{{\sf n}_\text{r}^\prime} \, \Ex \left\{ \sum_{i=1}^{{\sf n}_\text{r}^\prime} \log_2 \! \left(1 + \frac{{\sf SNR}}{{\sf n}_\text{r}^\prime} \lambda_{i}(\tilde{\vect{{\sf H}}}^\prime (\tilde{\vect{{\sf H}}}^\prime)^{\Htran}) \right)\right\} \\& \label{capacity_snr_app}
%= \Ex \left\{ \log_2 \!\left[ \prod_{i=1}^{{\sf n}_\text{r}^\prime} \left(1 + \frac{{\sf SNR}}{{\sf n}_\text{r}^\prime} \lambda_{i}(\tilde{\vect{{\sf H}}}^\prime (\tilde{\vect{{\sf H}}}^\prime)^{\Htran}) \right)\right]^{\frac{1}{{\sf n}_\text{r}^\prime} } \right\},
%\end{align}
%from which $\log_2  {\sf SNR} - \frac{{\sf C}({\sf SNR})}{{\sf S}_\infty^\star(\tilde{\vect{{\sf H}}})}$ is given by
%\begin{align}
%&  \Ex \left\{ \log_2 \frac{{\sf SNR}}{\left[ \prod_{i=1}^{{\sf n}_\text{r}^\prime} \left(1 + \frac{{\sf SNR}}{{\sf n}_\text{r}^\prime} \lambda_{i}(\tilde{\vect{{\sf H}}}^\prime (\tilde{\vect{{\sf H}}}^\prime)^{\Htran}) \right)\right]^{\frac{1}{{\sf n}_\text{r}^\prime} }} \right\} 
% \\& =
%\Ex \left\{ \log_2 \frac{1}{\left[ \prod_{i=1}^{{\sf n}_\text{r}^\prime} \left(\frac{1}{{\sf SNR}} + \frac{1}{{\sf n}_\text{r}^\prime} \lambda_{i}(\tilde{\vect{{\sf H}}}^\prime (\tilde{\vect{{\sf H}}}^\prime)^{\Htran}) \right)\right]^{\frac{1}{{\sf n}_\text{r}^\prime} }} \right\} \\& 
%= \label{papa}
%-\frac{1}{{\sf n}_\text{r}^\prime} \Ex \left\{ \log_2 \left[ \prod_{i=1}^{{\sf n}_\text{r}^\prime} \left(\frac{1}{{\sf SNR}} + \frac{1}{{\sf n}_\text{r}^\prime} \lambda_{i}(\tilde{\vect{{\sf H}}}^\prime (\tilde{\vect{{\sf H}}}^\prime)^{\Htran}) \right)\right] \right\}.
%\end{align}
%According to \eqref{L_infty}, letting ${\sf SNR} \to \infty$ in \eqref{papa},
%\begin{align} 
%{\sf L}_\infty^\star(\tilde{\vect{{\sf H}}}) & \approx  -\frac{1}{{\sf n}_\text{r}^\prime} \Ex \left\{ \log_2 \left[ \prod_{i=1}^{{\sf n}_\text{r}^\prime} \frac{1}{{\sf n}_\text{r}^\prime} \lambda_{i}(\tilde{\vect{{\sf H}}}^\prime (\tilde{\vect{{\sf H}}}^\prime)^{\Htran}) \right] \right\} \\ &  \label{L_infty_nr}
%=  \log_2 {\sf n}_\text{r}^\prime -\frac{1}{{\sf n}_\text{r}^\prime} \Ex \left\{ \log_2 \det(\tilde{\vect{{\sf H}}}^\prime (\tilde{\vect{{\sf H}}}^\prime)^{\Htran}) \right\}.
%%\\ &
%%= -\frac{1}{{\sf n}_\text{r}^\prime} \Ex \left\{ \log_2 \det(\tilde{\vect{{\sf P}}}^\star) \right\} -\frac{1}{{\sf n}_\text{r}^\prime} \log_2 \det(\vect{\Lambda}_{0_\text{r}}) \\& \hspace{1cm} - \frac{1}{{\sf n}_\text{r}^\prime} \Ex \left\{ \log_2 \det(\vect{W}_0 \vect{{\sf \Lambda}}_{0_\text{t}} \vect{W}_0^{\Htran}) \right\}
%\end{align}
%For ${\sf n}_\text{t}^\prime < {\sf n}_\text{r}^\prime$, we have that ${\sf n}_\text{min}^\prime = {\sf n}_\text{t}^\prime$ and $(\tilde{\vect{{\sf H}}}^\prime)^{\Htran} \tilde{\vect{{\sf H}}}^\prime$ is nonsingular with probability $1$.
%Then, replacing $\tilde{\vect{{\sf H}}}^\prime$ with $(\tilde{\vect{{\sf H}}}^\prime)^{\Htran}$ while interchanging ${\sf n}_\text{t}^\prime$ and ${\sf n}_\text{r}^\prime$, due to reciprocity,
%\begin{align}  \label{L_infty_nt}
%{\sf L}_\infty^\star(\tilde{\vect{{\sf H}}}) & 
%\approx \log_2 {\sf n}_\text{t}^\prime  -\frac{1}{{\sf n}_\text{t}^\prime} \Ex \left\{ \log_2 \det((\tilde{\vect{{\sf H}}}^\prime)^{\Htran} \tilde{\vect{{\sf H}}}^\prime) \right\}. 
%\end{align}
%%From \eqref{L_infty}, for ${\sf n}_\text{t}^\prime = {\sf n}_\text{r}^\prime = {\sf n}_0$ we have that
%%\begin{align} \notag
%%{\sf L}_\infty^\star(\tilde{\vect{{\sf H}}}^\prime) & 
%%= -\frac{1}{{\sf n}_0} \Ex \left\{ \log_2 \det(\tilde{\vect{{\sf P}}}^\star_0) \right\} -\frac{1}{{\sf n}_0}  \Ex \left\{ \log_2 \det(\tilde{\vect{{\cal H}}}_0^{\Htran} \tilde{\vect{{\cal H}}}_0) \right\} \\& \hspace{1cm}
%%+ \frac{1}{{\sf n}_0}  \Ex \left\{ \log_2 \det(\vect{{\sf A}}_\text{t})\right\}.
%%\end{align} 
%For ${\sf n}_\text{t}^\prime = {\sf n}_\text{r}^\prime = {\sf n}^\prime$, \eqref{L_infty_nr} and \eqref{L_infty_nt} reduce to a common form.
%%\begin{align} \label{L_infty_n}
%%{\sf L}_\infty^\star(\tilde{\vect{{\sf H}}}^\prime) & = \log_2 {\sf n}_{0} -\frac{1}{{\sf n}_\text{t}^\prime} \Ex \left\{ \log_2 \det(\tilde{\vect{{\sf H}}}^\prime^{\Htran} \tilde{\vect{{\sf H}}}^\prime) \right\}. 
%%%- \frac{1}{{\sf n}_{0}}  \log_2 \det(\vect{\Lambda}_{\text{r}}) \\& \hspace{0.5cm}  \label{L_infty_n}
%%%- \frac{1}{{\sf n}_{0}}  \log_2 \det(\vect{{\sf \Lambda}}_{\text{t}})
%%%-\frac{1}{{\sf n}_{0}} \Ex \left\{\log_2 \det(\vect{W}^{\Htran} \vect{W})\right\}.
%%\end{align}

%\section{}
%
%Sampling \eqref{j_C_op} and \eqref{composite_channel_op} at antenna locations and discretizing the spatial convolutions \cite{HeedongIRS}, yields, respectively,
%\begin{align} \label{a}
%(\mathcal{C}^{1/2}_\text{t} j_\text{t})(\vect{s}_\ell) & \approx \frac{L^2_\text{t}}{N_\text{t}}  \sum_{k} c^{1/2}_{\text t}(\vect{s}_\ell,\vect{s}_k) \vect{j}_\text{t}(\vect{s}_k) \\  \label{b}
%(\mathcal{C}^{-1/2}_\text{t} h)(\vect{r}_n,\vect{s}_\ell) &\approx \frac{L^2_\text{t}}{N_\text{t}}  \sum_{m} h(\vect{r}_n,\vect{s}_m) c^{-1/2}(\vect{s}_m,\vect{s}_\ell).
%\end{align}
%%where ${\rm m}(\cdot)$ denotes the Lebesgue measure.
%Now, repeating the same exercise to the continuous model in \eqref{MIMO_model_C_cont}, yields the output voltage  
%\begin{align}  \label{c1}
%v_{\text{r}}(\vect{r}_n) & = (\mathcal{H} \mathcal{C}^{-1/2}_\text{t} \mathcal{C}^{1/2}_\text{t} j_\text{t})(\vect{r}_n) \\& \label{c}
%\approx  \frac{L^2_\text{t}}{N_\text{t}} \sum_\ell (\mathcal{C}^{-1/2}_\text{t} h)(\vect{r}_n,\vect{s}_\ell)  (\mathcal{C}^{1/2}_\text{t} j_\text{t})(\vect{s}_\ell).
%\end{align}
%
%Comparing \eqref{c} against the receive voltage in \eqref{open_circuit_MIMO} by the $n$th antenna port, namely,
%\begin{align} \label{d}
%[\vect{v}_\text{r}]_n & %= v_\text{r}(\vect{r}_n)  
%= \vect{H} \vect{{\sf C}}_\text{t}^{-1/2} \vect{{\sf C}}_\text{t}^{1/2} \vect{j}_\text{t} \\
%& = \sum_m [\vect{H}]_{n,m} \sum_\ell [\vect{{\sf C}}_\text{t}^{-1/2}]_{m,\ell} \sum_k [\vect{{\sf C}}_\text{t}^{1/2}]_{\ell,k} j_\text{t}(\vect{s}_k)
%\end{align}
%we obtain
%\begin{align}
%[\vect{H}]_{n,m} &\approx \frac{L^2_\text{t}}{N_\text{t}} h(\vect{r}_n,\vect{s}_m) \\ \label{C_sqrt_cont_disc}
%[\vect{{\sf C}}^{\pm 1/2}_\text{t}]_{m,m^\prime} & \approx \frac{L^2_\text{t}}{N_\text{t}} c^{\pm 1/2}_{\text t}(\vect{s}_m,\vect{s}_\ell)
%\end{align}
%describing the interplay between the discrete and continuous parallel derivations.
%
%In the limit $L_\text{t}/\lambda\to \infty$, 
%\begin{align} \label{variances_channel_op_asym}
%(L_{\text r} L_{\text t})^2 \, \sigma^2_{\mathcal{H},i}  \to \Ex\{|H^{++}(\theta_{\text r},\phi_{\text r},\theta_{\text t},\phi_{\text t})|^2\},
%\end{align}
%exactly, for every tuples $(\theta_{\text r},\phi_{\text r},\theta_{\text t},\phi_{\text t})$, whereby
%\begin{equation} \label{max_eig_H}
%(L_{\text r} L_{\text t})^2 \max_{i=1, \ldots, {\sf n}_\text{r} {\sf n}_\text{t}} \sigma^2_{\mathcal{H},i} \to 
%\max \limits_{(\theta_{\text r},\phi_{\text r},\theta_{\text t},\phi_{\text t})} \Ex\{|H^{++}(\theta_{\text r},\phi_{\text r},\theta_{\text t},\phi_{\text t})|^2\}.
%\end{equation}
%Similarly, for every bounded operator $\mathcal{C}_\text{t}$,
%%\begin{align}  \label{variances_coupling_op_asym}
%%\lambda_{\mathcal{C}_\text{t},i}
%%\to  |A^+_{\text t}(\theta_{\text t},\phi_{\text t})|^2
%%\end{align}
%%for every $(\theta_{\text t},\phi_{\text t})$.
%\begin{equation} \label{max_lambda_ct}
%L_{\text t}^2 \min_{k=1, \ldots, {\sf n}_\text{t}} \sigma^2_{\mathcal{C}_\text{t},k}
%%= \min_{(\theta_{\text t},\phi_{\text t})}  \lambda_{\mathcal{C}_\text{t}}(\theta_{\text t},\phi_{\text t}) 
%= \min_{(\theta_{\text t},\phi_{\text t})} |A^+_{\text t}(\theta_{\text t},\phi_{\text t})|^2.
%\end{equation}
%%Note that \eqref{variances_channel_op_asym} and \eqref{variances_coupling_op_asym} coincide to the spectrum in \eqref{spectral_representation} and \eqref{real_impedance_kernel_antenna}, respectively, after changing of variables according to \eqref{wavenumber_spherical}.
%The asymptotic inequality \eqref{bound_eig_HC_spectra} can then be obtained by letting $\min(L_\text{t},L_\text{r})/\lambda\to \infty$ in \eqref{lambda_HC} while inserting \eqref{max_eig_H} and \eqref{max_lambda_ct},


%{\color{red}
%\section{} \label{app:stationary_time}
%
%The spectral representation of a stationary random process $h(t)$ with $L^2(\Real)$ realizations reads \cite{VanTreesBook,GrangerBook}
%\begin{equation} \label{spectral}
%h(t) = \int df \, \sqrt{S_h(f)} W(f) e^{\imagunit 2\pi f t}
%\end{equation}
%with $S_h(f)$ the power spectral density of $h(t)$ and $W(f)$ a white complex Gaussian noise process with unit variance. 
%Suppose that $S_h(f)$ is singular over a finite number of frequencies.
%Then, the discretization of \eqref{spectral} via a (midpoint) Riemann sum,
%\begin{equation}
%h(t) \approx \sum_{i} \sqrt{S_h(f_i^*)} W(f_i^*) e^{\imagunit 2\pi f_i^* t},
%\end{equation}
%with $f_i^* = (f_i+f_{i-1})/2$, would be ill-behaved due to singularities. 
%Instead, start from the Stieltjes integral \cite{VanTreesBook,GrangerBook}
%\begin{equation} \label{spectral_Stieltjes}
%h(t) = \int  dH(f) \, e^{\imagunit 2\pi f t}
%\end{equation}
%with $H(f)$ a complex Gaussian process with independent increments such that, for any integrable $S(\omega)$,
%\begin{equation}
%H(f_i) - H(f_{i-1})  \sim \CN\left(0, \int_{f_{i-1}}^{f_i} df \, S_h(f) \right),
%\end{equation}
%for every $i$. 
%The Stieltjes sum approximating \eqref{spectral_Stieltjes} for large time-bandwidth products is then
%\begin{equation}  \label{spectral_Stieltjes_sum_general}
%h(t) \approx \sum_{i}  (H(f_i) - H(f_{i-1})) e^{\imagunit 2 \pi f_i^* t}.
%\end{equation}
%%For a bandlimited process of bandwidth $2B$ that is observed over an interval of duration $T$ setting $f_i^* = i/T$ would provide an accurate representation for large $T$ \cite{FranceschettiBook}. 
%Setting $f_i = i/T$ in \eqref{spectral_Stieltjes_sum_general} yields the Fourier series expansion 
%\begin{equation} \label{spectral_Stieltjes_sum}
%h(t) \approx \sum_i  H_i \, u_i(t), \qquad t\in[0,T]
%\end{equation}
%with $u_i(t) = e^{\imagunit 2 \pi i t/T}/\sqrt{T}$ orthonormal and $H_i \sim \CN\left(0, \int_{(i-1)/T}^{i/T} df \, S_h(f) \right)$.
%The term $\sqrt{T}$ is purposely omitted from $H_i$ as power is the quantity of interest, in stead of energy, with each variance representing the expected power in the frequency interval $[(i-1)/T,i/T]$ $\forall i$. 
%
%
%Before gauging the limiting SNR regimes of the capacity, we invoke a result on the eigenvalue $\lambda_i(\mathcal{H}_T \mathcal{H}_T^*)$ of a self-adjoint operator $\mathcal{H}_T \mathcal{H}_T^*$ with random kernel $h(t)$ observed over an interval $T$: if $S_h(f)$ is the (non-singular) power spectral density of $h(t)$, then $\Ex\{\lambda_i(\mathcal{H}_T\mathcal{H}_T^*)\} \approx S_h(i/T)$ with a better approximation as the time-bandwidth product increases \cite[Ch.~3]{VanTreesBook}.
%However, for a singularly integrable $S_h(f)$, a Stieltjes-type discretization of the spectrum of $h(t)$ is used instead yielding $\Ex\{\lambda_i(\mathcal{H}_T \mathcal{H}_T^*)\} \approx \int_{(i-1)/T}^{i/T} S_h(f) \, df$ (see App.~\ref{app:stationary_time}). For asymptotically large time-bandwidth products the integration regions collapse into singletons whereby the eigenvalues of $\mathcal{H}_T \mathcal{H}_T^*$ approach $S_h(f)$ $\forall f$.
%%Since the electromagnetic spectrum in \eqref{spectral_representation} is singularly integrable \cite{PlaneWaveBook,MarzettaIT}, a spatially stationary electromagnetic random field $h(\vect{{\sf r}},\vect{{\sf s}})$ with $L^2(\Real^3 \times \Real^3)$ realizations is amenable to a discretization the like of \eqref{spectral_Stieltjes_sum}, which is reported in \eqref{Fourier_series}. 
%%With large---yet finite---time-bandwidth products, $\Ex\{\lambda_{\mathcal{H},1}(T)\} = \max_i S_h(i/T)$, following a Riemann-type of discretization of the spectrum of $h(t)$ \cite[Ch.~3]{VanTreesBook}. However, the one appearing in \eqref{bound_eig_HC} is of Stieltjes-type, which would be analogous to 
%%Momentarily focusing on the numerator of \eqref{bound_eig_HC_spectra}, the one characterizing the channel, we can
%%invoke a result on the largest eigenvalue $\lambda_{\mathcal{H},1}(T)$ of a self-adjoint operator $\mathcal{H} \mathcal{H}^*$ with random kernel $h(t)$ observed over an interval $T$: if $S_h(f)$ is the power spectral density of $h(t)$, then $\Ex\{\lambda_{\mathcal{H},1}(T)\} \le \max_{f} S_h(f)$, with equality attained by asymptotically large time-bandwidth products \cite[Ch.~3]{VanTreesBook}.
%Thus, the variance of the numerator of \eqref{spectrum_HC} 
%%Recalling \eqref{eig_discrete_cont}, \eqref{bound_eig_HC_spectra_rewritten} \angel{The first term therein?} generalizes this maximum-eigenvalue result for stationary random processes to stationary electromagnetic random fields; the connection is drawn here in the angular domain, but it also holds in the wavenumber domain in light of \eqref{wavenumber_spherical}. 
%offers another embodiment of the time-domain result that applies in the spatial domain to stationary random fields; the connection is drawn here in the wavenumber domain, but it also holds in the angular domain in light of \eqref{wavenumber_spherical}.
%%\angel{Not sure "generalize" is the right term, rather we have another embodiment of the result---or perhaps both are instances of a more general version thereof.}
%}
%
%
%{\color{blue}
%\section{} \label{app:coupling_lowSNR}
%
%Let $h(t)$ be a bandlimited signal of bandwidth $B$ with singularly-integrable spectrum $H(f)$ that is associated with a continuous operator $\mathcal{H}_T : L^2(\Real) \to L^2(0,T)$. 
%Also, let $\vect{H} \in \Complex^{N\times N}$ be the $N$-dimensional discrete counterpart of $\mathcal{H}_T$ over an interval of duration $T$.
%Then,
%\begin{equation}
%\log_2 \det(\vect{H} \vect{H}^{\Htran}) = \sum_{i=1}^{N} \log_2 \lambda_i(\vect{H}\vect{H}^{\Htran})
%\end{equation}
%Invoking the time-domain counterpart relationship between each eigenvalue of a Hibert-Schmidt operator and the corresponding one associated with the discrete matrix operator $\lambda_i(\vect{H}\vect{H}^{\Htran}) \approx \frac{N}{T} \lambda_i(\mathcal{H}_T \mathcal{H}_T^*)$ \cite{HeedongIRS} 
%\begin{equation} \label{logdet_Hvec}
%\log_2 \det(\vect{H} \vect{H}^{\Htran}) =  N \log_2 \left(\frac{N}{T}\right) + \sum_{i=1}^{N} \log_2 \lambda_i(\mathcal{H}_T \mathcal{H}_T^*).
%\end{equation}
%Similarly to Appendix~\ref{app:stationary_time}, starting from the inverse Fourier transform of an $L^2(\Real)$ signal we infer that
%\begin{equation} \label{eig_Hcal}
%\lambda_i(\vect{H}_T\vect{H}_T^{\Htran}) \approx \int_{(i-1)/T}^{i/T} H(f) \, df,
%\end{equation}
%with the approximation progressively becoming more accurate as the time-bandwidth product increases.
%Thus, plugging \eqref{eig_Hcal} into \eqref{logdet_Hvec},
%\begin{align} \label{log_det_H}  
%\log_2 \det(\vect{H} \vect{H}^{\Htran}) & 
%%= \sum_{i=1}^{N} \log_2\left(\frac{{\sf n}}{T} \, T \int_{(i-1)/T}^{i/T} \!\!\!\! H(f) \, df\right)  \\& 
%=  N \log_2 \left(\frac{N}{T}\right) + \sum_{i=1}^{N} \log_2\left( \int_{(i-1)/T}^{i/T} \!\! H(f) \, df\right).
%\end{align}
%By means of Jensen's inequality, namely
%\begin{equation}
%g\left( \frac{1}{b-a} \int_a^b f(x) \, dx \right) \ge \frac{1}{b-a} \int_a^b g(f(x)) \, dx,
%\end{equation}
%for any concave function $g(\cdot)$, it follows that
%\begin{align} \label{bound}
%\log_2\left(T \int_{(i-1)/T}^{i/T} \frac{H(f)}{T} \, df\right) \ge T \int_{(i-1)/T}^{i/T} \log_2\left(\frac{H(f)}{T}\right) \, df
%\end{align}
%for any positive spectrum $H(f)$.
%Substituting \eqref{bound} into \eqref{log_det_H} yields
%\begin{align}  \notag
%\log_2 \det(\vect{H} \vect{H}^{\Htran}) & \ge N \log_2\left(\frac{N}{T}\right)  \\ & \label{log_det_H_lower} \hspace{.0cm}
%+ T \, \sum_{i=1}^{N} \int_{(i-1)/T}^{i/T} \log_2\left(\frac{H(f)}{T}\right) \, df  \\
%& = N \log_2\left(\frac{N}{T}\right) + T \int_{-B}^{B}  \log_2\left(\frac{H(f)}{T}\right) \, df.
%\end{align}
%The lower bound in \eqref{log_det_C_lower} is derivable from \eqref{log_det_H_lower} by leveraging a space-time duality underlying wireless propagation \cite{PizzoIT21}.
%%monotone-increasing function of the length of the interval $T$.
%}

%{\color{blue}
%\section{} \label{app:rank}
%
%The inequality in \eqref{n_min} can be derived as
%\begin{align}
%{\sf n}_\text{min}
%& \le  \min(\rank(\vect{{\sf C}}_\text{r}^{-1/2}\vect{H}),\rank(\vect{{\sf C}}_\text{t}^{-1/2})) \\ 
%& \le  \min(\rank(\vect{{\sf C}}_\text{r}^{-1/2}), \rank(\vect{H}),\rank(\vect{{\sf C}}_\text{t}^{-1/2}))
% \\ 
%& =  \min(\rank((\vect{{\sf C}}_\text{r}^{-1})^{1/2}), \rank(\vect{H}),\rank((\vect{{\sf C}}_\text{t}^{-1})^{1/2})) \\ 
%& =  \min(\rank(\vect{{\sf C}}_\text{r}^{-1}),\rank(\vect{H}),\rank(\vect{{\sf C}}_\text{t}^{-1})) \\ \label{rank_HC}
%& =  \min(\rank(\vect{{\sf C}}_\text{r}),\rank(\vect{H}),\rank(\vect{{\sf C}}_\text{t})) \\
%& =  \min(N_\text{r},\rank(\vect{H}),N_\text{t}) \\
%& =  \rank(\vect{H}). 
%\end{align}
%where the inequalities follows from applying $\rank(\vect{A} \vect{B}) \le \rank(\vect{A}) \rank(\vect{B})$ to the matrix multiplications in \eqref{channel_mat_c_rx} and the remaining equalities exploit the inherent positive definiteness of the coupling matrices \cite[Ch.~7]{HornBook}.}
%then recalling that the inverse of a positive definite matrix is also positive definite, and finally using the property that a positive definite matrix and its principal square root have the same rank,


%\section{} \label{app:optimization}
%
%%The optimization problem in \eqref{P0} is non-concave as can be verified by checking the second-order condition on its objective function. For the simple real-valued scalar case with $h =1$ the second-order derivative is $2(1-f^2)/(1+f^2)^2$ which may be positive or negative within the function domain.
%%Hence, i
%%In order to solve \eqref{P0}, we invoke the identity \cite{PalomarBook}
%%\begin{equation} \label{obj}
%%\det (\vect{I} +  \vect{H}^{\Htran} \vect{H} \vect{F} \vect{F}^{\Htran} ) = - \prod_{i=1}^{N_{\text t}} \left[(\vect{I} + \vect{F}^{\Htran}  \vect{H}^{\Htran} \vect{H} \vect{F} )^{-1}\right]_{ii}
%%\end{equation}
%%and recall that the function $f(\vect{x}) = \prod_{i=1}^{N_{\text t}} x_i$ at the right-hand side of \eqref{obj} is Schur-concave when the variables $x_i$ are sorted in a descending order. We retain only the first ${\sf n}_{\text t}$ terms as ${\rm rank}(\vect{H}) \le {\sf n}_{\text t}$.
%%Then, since an increasing function in each variable of a Schur-concave function is also Schur-concave, the maximization of the objective function in \eqref{P0} is equivalent to the minimization of the convex function \cite{PalomarBook}
%%\begin{equation} \label{obj}
%%\log_2\left(\prod_{i=1}^{{\sf n}_{\text t}} \frac{1}{1 + p_i \lambda_{\vect{H},i}}\right)
%%= - \sum_{i=1}^{{\sf n}_{\text t}} \log_2\left(1 + p_i \lambda_{\vect{H},i}\right),
%%\end{equation}
%%which is attainable by $\vect{F}^\star$ given in \eqref{precoder_F}.
%%%\begin{equation} \label{precoder_F}
%%%\vect{F} = \vect{V}_{\vect{H}} 
%%%\begin{pmatrix}
%%%\vect{P}^{1/2} \\
%%%\vect{0}_{(N_{\text t} - {\sf n}_{\text t}) \times {\sf n}_{\text t}}
%%%\end{pmatrix},
%%%\end{equation}
%%%given $\vect{V}_{\vect{H}}\in\Complex^{N_{\text t} \times N_{\text t}}$ the right singular vector matrix of $\vect{H}$ and given $\lambda_{\vect{H},i}$ as the sorted eigenvalues of $\vect{H}^{\Htran} \vect{H}$.
%%As for the inequality constraints of \eqref{P0}, we rewrite them as
%%\begin{equation} \label{constraints_1}
%%\tr(\vect{F} \vect{F}^{\Htran} \vect{A}_j) \le a_j, \quad j=1, 2,
%%\end{equation}
%%where $\vect{A}_1 = \vect{{\sf C}}_{\text t}$ and $a_1 = {\sf SNR}$ and, in turn, $\vect{A}_2 = \vect{B}_{\text t} - Q_{\text t} \vect{{\sf C}}_{\text t}$ and $a_2 = 0$.
%%Using \eqref{precoder_F}, \eqref{constraints_1} can be rewritten as
%%\begin{align} \label{polyhedron_1}
%%\tr(\vect{F} \vect{F}^{\Htran} \vect{A}_j) & = \tr(\vect{P} \vect{A}_{\vect{H},j}) \le a_j, \quad j=1, 2, 
%%\end{align} 
%%where $\vect{A}_{\vect{H},j} = \hat{\vect{V}}_{\vect{H}}^{\Htran} \vect{A}_j \hat{\vect{V}}_{\vect{H}} \in \Complex^{{\sf n}_{\text t} \times {\sf n}_{\text t}}$ with $\hat{\vect{V}}_{\vect{H}} \in \Complex^{N_{\text t} \times {\sf n}_{\text t}}$ the ${\sf n}_{\text t}$-rank right singular matrix of $\vect{H}$ corresponding to the most valuable eigenvalues.
%%In turn, leveraging the identity $\tr(\vect{A}^{\Ttran} \vect{B}) = \vec(\vect{A})^{\Ttran} \vec(\vect{B})$, \eqref{polyhedron_1} can be rewritten as
%%\begin{equation} \label{polyhedron}
%%\vect{a}_{\vect{H},j}^{\Ttran} \vect{p} \le a_j, \quad j=1, 2, 
%%\end{equation}
%%where $\vect{a}_{\vect{H},j} = \vect{M} \vec(\vect{A}_{\vect{H},j}) \in \Complex^{{\sf n}_{\text t}}$ with $\vect{M} \in \Complex^{{\sf n}_{\text t} \times {\sf n}^2_{\text t}}$ a selection matrix that picks out one every ${\sf n}_{\text t}$ elements of $\vec(\vect{A}_{\vect{H},j})$. 
%%%We rewrite the constraint set \eqref{polyhedron} compactly as
%%%\begin{equation} \label{poly_compact}
%%%\vect{A}_{\vect{H}} \vect{p} \preceq \vect{a}
%%%\end{equation}
%%%where $\vect{A}_{\vect{H}}^{\Ttran} = (\vect{a}_{\vect{H},1}, \vect{a}_{\vect{H},2}) \in \Complex^{{\sf n}_{\text t} \times 2}$ and $\vect{a} = (P_{\text t}, 0)^{\Ttran}$.
%%We exploit the monotonicity of \eqref{obj} in each variable $p_i$, so that the inequality constraints in \eqref{polyhedron} must be satisfied with equality
%%\begin{equation} \label{P1}
%%\begin{aligned}
%%& \underset{\vect{p} \in \Real^{{\sf n}_\text{t}}}{\text{min}}
%%& & - \sum_{i=1}^{{\sf n}_{\text t}} \log_2\left(1 + p_i \lambda_{\vect{H},i} \right) \\
%%& \text{subject to}
%%& & \vect{a}_{\vect{H},j}^{\Ttran} \vect{p}  = a_j, \quad j=1, 2, \\
%%& & & \vect{p} \succeq \vect{0},
%%\end{aligned}
%%\end{equation}
%%which is convex given the convexity of its objective function and polyhedron constraint set \cite{BoydBookCVX}. 
%%Introducing Lagrange multipliers $\vect{\lambda}  \in \Real^{{\sf n}_{\text t}}$ for the inequality constraint, and multipliers $\bm{\nu}=(\nu_1,\nu_2)^{\Ttran} $ for the equality constraints, respectively, the KKT conditions for \eqref{P1} read
%%\begin{gather} \label{KKT} 
%% \vect{a}_{\vect{H},1}^{\Ttran} \vect{p}  = {\sf SNR}, \quad \vect{a}_{\vect{H},2}^{\Ttran} \vect{p} =0, \quad \vect{p}^\star \succeq \vect{0}, \\  \notag
%%% \nu_1^\star \ge 0, \quad \nu_2^\star \ge 0, \quad 
%% \nu_1^\star (\vect{a}_{\vect{H},1}^{\Ttran} \vect{p}^\star - {\sf SNR}) = 0, \quad \nu_2^\star \vect{a}_{\vect{H},2}^{\Ttran} \vect{p}^\star =0 \\  \notag
%% \vect{\lambda}^\star \succeq \vect{0}, \quad \lambda_i^\star p_i^\star = 0, \, i=1,\ldots, {\sf n}_{\text t}  
%%  \\  \notag 
%%- \frac{\lambda_{\vect{H},i}}{1 + p_i^\star \lambda_{\vect{H},i}} + \vect{a}_{\vect{H},i}^{\Ttran} \bm{\nu}^\star - \lambda_i^\star = 0, \, i=1,\ldots, {\sf n}_{\text t}
%%\end{gather}
%%where $a_{\vect{H},j,i}$ the $i$-th element of $\vect{a}_{\vect{H},j}$, $j=1,2$.
%%Eliminating $\lambda_i^\star$ yields
%%\begin{equation}
%%\lambda_i^\star = \vect{a}_{\vect{H},i}^{\Ttran} \bm{\nu}^\star - \frac{\lambda_{\vect{H},i}}{1 + p_i^\star \lambda_{\vect{H},i}},
%%\end{equation}
%%which plugged into $\lambda_i^\star p_i^\star = 0$, $i=1,\ldots, {\sf n}_{\text t}$, returns the nontrivial solution in \eqref{waterfilling_double}. Solving the slackness conditions in \eqref{KKT} leads to \eqref{multiplier_1} and \eqref{multiplier_2}.
%
%
%Introducing Lagrange multipliers $\vect{\lambda}  \in \Real^{N_{\text t}}$ for the positive definiteness constraints, multipliers $\vect{\nu}=(\nu_1,\nu_2)^{\Ttran}$ for the radiated power and array performance constraints, respectively, the Lagrangian associated with \eqref{P1} reads
%\begin{align} \notag
%L(\vect{p},\vect{\lambda},\vect{\nu}) & = - \sum_{i=1}^{N_{\text t}} \log_2\left(1 + p_i \lambda_{\vect{H},i} \right) - \vect{\lambda}^{\Ttran} \vect{p}  \\& + \nu_1 (\vect{c}_{\vect{H},1}^{\Ttran} \vect{p} - c_1) + \nu_2 (\vect{c}_{\vect{H},2}^{\Ttran} \vect{p} - c_2).
%\end{align}
%Then, the KKT conditions for the primal and dual optimal variables $\vect{p}^\star$, $\vect{\lambda}^\star$, and $\vect{\nu}^\star$ are
%\begin{gather} \label{KKT} 
% \vect{c}_{\vect{H},1}^{\Ttran} \vect{p}^\star \le c_1, \quad \vect{c}_{\vect{H},2}^{\Ttran} \vect{p}^\star \le c_2, \quad \vect{p}^\star \succeq \vect{0}, \\  \notag
%% \nu_1^\star \ge 0, \quad \nu_2^\star \ge 0, \quad 
% \vect{\lambda}^\star \succeq \vect{0}, \quad \lambda_i^\star p_i^\star = 0, \quad i=1,\ldots, N_{\text t}  
%  \\  \notag 
%  \vect{\nu}^\star \succeq \vect{0}, \quad \nu_j^\star (\vect{c}_{\vect{H},j}^{\Ttran} \vect{p}^\star - c_j) = 0, \quad j=1,2, \\ \notag %\quad \nu_2^\star \vect{c}_{\vect{H},2}^{\Ttran} \vect{p}^\star =0 \\  \notag
%- \frac{\lambda_{\vect{H},i}}{1 + p_i^\star \lambda_{\vect{H},i}} - \lambda_i^\star + \nu_1^\star [\vect{c}_{\vect{H},1}]_i + \nu_2^\star [\vect{c}_{\vect{H},2}]_i  = 0, \, i=1,\ldots, N_{\text t}
%\end{gather}
%where $[\vect{c}_{\vect{H},j}]_i$ the $i$th entry of $\vect{c}_{\vect{H},j}$.
%Eliminating $\lambda_i^\star$ yields
%\begin{equation}
%\lambda_i^\star = \nu_1^\star [\vect{c}_{\vect{H},1}]_i + \nu_2^\star [\vect{c}_{\vect{H},2}]_i - \frac{\lambda_{\vect{H},i}}{1 + p_i^\star \lambda_{\vect{H},i}},
%\end{equation}
%which plugged into $\lambda_i^\star p_i^\star = 0$ with $p_i^\star \ge 0$, $i=1,\ldots, N_{\text t}$, returns the nontrivial solution 
%\begin{equation} \label{waterfilling_double}
%p_i^\star = \left(\frac{1}{\nu_1^\star [\vect{c}_{\vect{H},1}]_i + \nu_2^\star [\vect{c}_{\vect{H},2}]_i} - \lambda_{\vect{H},i}^{-1} \right)^+.
%\end{equation}
%Solving the slackness conditions leads to 
%\begin{align} \label{multiplier_j}
%\sum_{i=1}^{N_{\text t}} [\vect{c}_{\vect{H},j}]_i \left(\frac{1}{\nu_1^\star [\vect{c}_{\vect{H},1}]_i + \nu_2^\star [\vect{c}_{\vect{H},2}]_i} - \lambda_{\vect{H},i}^{-1} \right)^+ = c_j 
%%\\ \label{multiplier_2}
%%\sum_{i=1}^{{\sf n}_{\text t}} a_{\vect{H},2,i} \left(\frac{1}{\vect{a}_{\vect{H},i}^{\Ttran} \bm{\nu}^\star} - \lambda_{\vect{H},i}^{-1} \right)^+ = c_2.
%\end{align}
%for $j=1,2$.
%The capacity-achieving precoder is obtainable by inverting \eqref{precoder_Fhat} and recalling the diagonal structure of $\vect{F}_\vect{H}$ as
%\begin{equation} \label{precoder_F}
%\vect{F}^\star = \vect{V}_{\vect{H}} 
%(\vect{P}^\star)^{1/2},
%\end{equation}
%where the diagonal entries of $\vect{P}^\star$ are chosen according to the above water-filling criteria.

\bibliographystyle{IEEEtran}
\bibliography{IEEEabrv,refs}

\end{document}

%% SUPERDIRECTIVITY %%

%Transmit array gain
%\begin{equation}
%A_\text{t} = \frac{\max_{\vect{f}\in\Complex^{N_\text{t}}} {\sf SNR}(\vect{f})}{\max_{\vect{f}\in\Complex^{N_\text{t}}} {\sf SNR}(\vect{f})|_{N_\text{t}=1}}
%\end{equation}
%where ${\sf SNR}(\vect{f})$ is the receive SNR with single user MISO 
%\begin{align}
%{\sf SNR}(\vect{f}) & = \frac{|\vect{f}^{\Htran} \vect{h}|^2}{\sigma^2}  = {\sf SNR} \frac{|\vect{f}^{\Htran} \vect{h}|^2}{\vect{f}^{\Htran} \vect{{\sf C}}_\text{t} \vect{f}}
%\end{align}
%with ${\sf SNR} =P_\text{t}/\sigma^2$, $\|\vect{f}\|=1$
%
%No mutual coupling (i.e., $\vect{{\sf C}}_\text{t} = \vect{I}_{N_\text{t}}$): 
%\begin{align}
%{\sf SNR}(\vect{f}) & = {\sf SNR} \frac{|\vect{f}^{\Htran} \vect{h}|^2}{\|\vect{f}\|^2}
%\end{align}
%solved by $\vect{f}^\star = \vect{h}/\|\vect{h}\|$ that ${\sf SNR}(\vect{f}^\star)/{\sf SNR} = \|\vect{h}\|^2$.
%Then 
%\begin{equation}
%A_\text{t}^\text{no-mc}(\theta) = \|\vect{h}(\theta)\|^2 = N_\text{t}
%\end{equation}
%as $\vect{h}(\theta) = [1 e^{\imagunit \kappa d \cos(\theta)} \ldots e^{\imagunit \kappa (N_\text{t}-1) d \cos(\theta)}]^{\Ttran}$, $\|\vect{h}(\theta)\|^2=N_\text{t}$ $\forall \theta$.
%
%Mutual coupling: 
%solved via generalized Rayleigh quotient
%$\vect{f}^\star = \vect{{\sf C}}_\text{t}^{-1} \vect{h}/\|\vect{{\sf C}}_\text{t}^{-1} \vect{h}\|$ that ${\sf SNR}(\vect{f}^\star)/{\sf SNR} = \vect{h}^{\Htran} \vect{{\sf C}}_\text{t}^{-1} \vect{h}$
%since $\vect{{\sf C}}_\text{t} = \vect{{\sf C}}_\text{t}^{\Ttran}$ with real entries.
%Then 
%\begin{equation}
%A_\text{t}^\text{mc}(\theta) = \vect{h}(\theta)^{\Htran} \vect{{\sf C}}_\text{t}^{-1} \vect{h}(\theta)
%\end{equation}
%When $d\to 0$, at $\theta=0$
%\begin{equation}
%A_\text{t}^\text{mc}(0) \to \vect{1}^{\Ttran} \vect{{\sf C}}_\text{t}^{-1} \vect{1} = N_\text{t}^2
%\end{equation}
%as $\vect{h}(0) = \vect{1}$