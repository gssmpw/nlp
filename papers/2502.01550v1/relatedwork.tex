\section{Related Work}
Wildfires are notoriously hard to model due to the non-linear interactions between the different Earth system processes
that affect them \cite{hantson2016status}. Weather, vegetation and humans interact in evolving ways that contribute in the
expansion or suppression of wildfires. Reichstein et~al.~\cite{reichstein2019deep} propose Deep Learning as a method to learn
in a  data-driven way these complex spatio-temporal interactions that influence wildfires. Several recent studies have used deep
learning for wildfire-related use cases~\cite{jain_review_2020}. For short-term daily predictions the temporal context is mostly
enough, with spatio-temporal models offering little to no advantage~\cite{prapas2021deep, kondylatos2022wildfire}. For longer-term
predictions, i.e. on subseasonal to seasonal scales, very few works have studied the effect of spatial and temporal
context~\cite{joshi2021improving, prapas2022deep, Prapas_2023_ICCV}.
Joshi et~al.~\cite{joshi2021improving} use monthly aggregated input to predict burned area using multi-layer neural networks.
Li et~al.~\cite{li2023attentionfire_v1} includes the temporal aspect of the input using a temporal attention network on time-series

Recent studies rely on our SeasFire Datacube~\cite{karasante2023seasfire} to train models with seasonal forecasting skill.
Prapas et~al.~\cite{prapas2022deep} use temporal snapshots of the fire drivers to predict future burned area patterns,
defining burned area forecasting as a segmentation task and demonstrating skillful forecasts even two months in advance. An
expansion of this~\cite{Prapas_2023_ICCV}, recognises the need to view the earth as a system for long-term forecasting and proposes a novel architecture that leverages teleconnection indices and global input in conjunction with those snapshots. This setup helps improve long-term skill, but ignores the temporal component of the input variables. 
Zhao et~al.~\cite{zhao2024causal} integrate causality with GNNs to explicitly model the causal mechanism among complex
variables via graph learning, and test their models in the European boreal and Mediterranean biome.
Finally, in the work of \cite{zhu2025unveiling} the authors expand on the TeleViT architecture and add a balancing term to handle the data imbalance between the burned and non-burned classes. This allows them to create a fire risk index that is better calibrated between different regions. 

Machine learning has recently made significant advances in global weather forecasting \cite{rasp2024weatherbench}, with graph neural networks showing tremendous potential to represent the Earth as a system\cite{keisler2022forecasting, lam2022graphcast}. The work of \cite{keisler2022forecasting} introduces a GNN-based model that predicts multiple atmospheric variables across various pressure levels on a global scale. 
Building upon this foundation, the GraphCast model~\cite{lam2022graphcast} presents a machine learning method that forecasts hundreds of weather variables over a 10-day period at 0.25-degree resolution globally, delivering results in under a minute.
Further advancing the GraphCast architecture, the work of~\cite{oskarsson2024probabilistic} introduces Graph-EFM, a probabilistic weather forecasting model that combines a flexible latent-variable formulation with a hierarchical graph-based framework. This model efficiently generates spatially coherent ensemble forecasts, achieving errors equivalent to or lower than comparable deterministic models while accurately capturing forecast uncertainty. These types of models  have also been shown to be effective at the temporal scales that we are interested in this study, with Fuxi-S2S \cite{chen2023fuxi} achieving skillfull forecasts at the subseasonal to seasonal scales. Our work aims to leverage the advances in data-driven weather modeling for burned area forecasting with an architecure inspired by Graphcast.