\section{Preliminaries}

We consider prediction tasks over $\calX \times \calY$, where $\calX$ represents the covariate domain and $\calY$ represents the label domain. $\calY$ can be either finite for classification tasks or infinite for regression tasks. Unless stated otherwise, we assume that the observed data is drawn from a distribution $\calD$, defined over $\calX \times \calY$.
% In this paper, we focus on a weighted variant of conformal prediction. 
We use weight functions $w: \cal X\times \cal Y \to \RR$ to model our conditional coverage guarantees. 
Let $(1-\alpha)$ be the coverage level we aim for, where $\alpha \in (0,1)$. For any weight function $w$, we measure the error of a prediction set function $\calC$ by its \emph{weighted coverage deviation}: 
\begin{align*}
    \wcovdev(\calC, \alpha, w) \coloneqq \EE[w(X,Y)(\11\{Y \in \calC(X)\}-(1-\alpha))],
\end{align*}
where the expectation is taken over the test point $(X,Y)$, drawn from $\calD$, and internal randomness of $\calC$. The bounds of $\wcovdev(\calC, \alpha, w)$, for a weight function $w$, in our results depend on the quantity 
$\|w\|_{\infty} \coloneqq \sup_{x \in \calX, y \in \calY} |w(x,y)|$.

Many conformal prediction methods require calculating quantiles of a distribution.

\begin{definition}[Quantile]
    Given a distribution $P$ defined on $\RR$, for any $\tau \in [0,1]$ the $\tau$-quantile of the distribution $P$, denoted $q$, is defined as 
    \begin{align*}
        q = \inf\{x \in \RR: \PP_{X \sim P}[ X \leq x] \geq \tau\}.
    \end{align*}
\end{definition}
Our more general framework performs quantile regression by solving a pinball loss minimization problem. By \Cref{lem: quant-reg}, in the special case where we compute the largest value in $\RR$ that minimizes the sum of pinball losses over a set of points, we obtain a $(1-\alpha)$ quantile of the set.

\begin{definition}[Pinball Loss]
    For a given target quantile $1-\alpha$, where $\alpha \in [0,1]$, predicted value $\theta \in \RR$ and score $s \in \RR$, the pinball loss is defined as 
    \begin{align*}
        \ell_\alpha(\theta,s) \coloneqq \begin{cases}
            (1-\alpha)(s-\theta), &\text{ if } s \geq \theta,\\
            \alpha(\theta-s), &\text{ if } s<\theta.
        \end{cases}
    \end{align*}
\end{definition}


\begin{lemma}[\cite{KB78}]
\label{lem: quant-reg}
    Let $\alpha$ be a parameter in $(0,1)$, and let $\{s_i\}_{i \in [n]}$ be a set of $n$ points, where for all $i \in [n]$, $s_i \in \RR$.
    Then, the largest $\theta^* \in \RR$ such that 
    \[
    \theta^* \in \arg\min_{\theta \in \RR} \sum_{i \in [n]} \ell_{\alpha}(\theta, s_i)
    \]
    is a $(1-\alpha)$-quantile of $\{s_i\}_{i \in [n]}$.
\end{lemma}
