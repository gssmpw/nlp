In the following, we present a reduction from $q$-CSP-$5$ to \AcyclicM.
In that case, if there exists a $\sO ((5 - \varepsilon)^\pw)$ algorithm for \AcyclicM,
where $\varepsilon > 0$,
then there exists a $\sO ((5 - \varepsilon)^n)$ algorithm for $q$-CSP-$5$,
for any constant $q$,
which due to \cref{thm:q_CSP_B_SETH} results in SETH failing.

Our reduction is based on the construction of ``long paths'' of \emph{Block gadgets}
that are serially connected in a path-like manner.
Each such ``path'' corresponds to a variable of the given CSP,
while each column of the construction is associated with one of its constraints.
Intuitively, our aim is to embed the $5^n$ possible variable assignments into the $5^\pw$ states of some optimal
dynamic program that would solve the problem on our constructed instance.

Below, we present a sequence of gadgets used in our reduction.
The aforementioned block gadgets,
which allow a solution to choose among $5$ reasonable choices,
are the main ingredient.
We connect these gadgets in a path-like manner that ensures that choices remain consistent throughout the construction,
and connect constraint gadgets in different ``columns'' of the constructed grid in a way that allows us to verify if
the choice made represents a satisfying assignment,
without significantly increasing the graph's pathwidth.

\begin{theoremrep}
    For any constant $\varepsilon > 0$,
    there is no $\sO((5-\varepsilon)^\pw)$ algorithm deciding \AcyclicM,
    where $\pw$ denotes the pathwidth of the input graph,
    unless the SETH is false.
\end{theoremrep}

\begin{proof}

    Fix some positive $\varepsilon > 0$ for which we want to prove the theorem.
    We will reduce $q$-CSP-$5$, for some $q$ that is a constant that only depends on $\varepsilon$,
    to {\AcyclicM} in a way that ensures that if the resulting {\AcyclicM} instance could be solved in time $\sO((5-\varepsilon)^{\pw})$,
    then we would obtain an algorithm for $q$-CSP-$5$ that would contradict the SETH due to \cref{thm:q_CSP_B_SETH}.
    To this end, let $\phi$ be an instance of $q$-CSP-$5$ of $n$ variables $X = \setdef{x_i}{i \in [n]}$ taking values over the set $[5]$
    and $m$ constraints $C = \setdef{c_j}{j \in [m]}$.
    For each constraint we are given a set of $q$ variables which are involved in this constraint and a list of satisfying assignments for these variables,
    the size of which is denoted by $s \colon C \rightarrow [5^q]$,
    i.e., $s(c_j) \leq 5^q = \bO(1)$ denotes the number of satisfying assignments for constraint $c_j$.
    We will construct in polynomial time an equivalent instance $\mathcal{I} = (G,\ell)$ of \AcyclicM,
    where $\pw(G) \leq n + \bO (1)$.

    \proofsubparagraph{OR Gadget.}
    Given two vertices $v_1$ and $v_2$, when we say that we connect them via an \emph{OR gadget} $\hat{O}(v_1,v_2)$
    we first add the edge $\{v_1,v_2\}$,
    then introduce two vertices $p^{v_1,v_2}_1$ and $p^{v_1,v_2}_2$ called the \emph{private vertices of the OR gadget},
    and lastly add edges so that $v_1$, $v_2$, and the private vertices of the gadget form a cycle;
    for an illustration see \cref{fig:acyclic:lb_or_gadget}.
    We say that $v_1,v_2,p^{v_1,v_2}_1,p^{v_1,v_2}_2$ are the \emph{vertices of the OR gadget $\hat{O}(v_1,v_2)$},
    while the \emph{edges of an OR gadget} are the edges whose endpoints both are among its vertices.

    \begin{figure}[htb]
        \centering
        \begin{tikzpicture}[scale=1, transform shape]

        %%%%%%%%%% vertices and text

        \node[vertex] (u) at (1,5) {};
        \node[] () at (0.7,5) {$v_1$};

        \node[vertex] (x1) at (1.5,5.5) {};
        \node[] () at (1.3,5.8) {$p^{v_1,v_2}_1$};
        \node[vertex] (x2) at (2,5.5) {};
        \node[] () at (2.5,5.8) {$p^{v_1,v_2}_2$};

        \node[vertex] (v) at (2.5,5) {};
        \node[] () at (2.8,5) {$v_2$};


        %%%%%%%%% edges / arcs

        \draw[] (u)--(x1)--(x2)--(v)--(u);

        \end{tikzpicture}
        \caption{OR gadget $\hat{O}(v_1,v_2)$.}
        \label{fig:acyclic:lb_or_gadget}
    \end{figure}

    \proofsubparagraph{Root vertex.}
    We start our construction by introducing a vertex $r$, which we refer to as the \emph{root vertex}.
    Furthermore, we attach a leaf $r'$ to $r$.


    \proofsubparagraph{Block and Variable Gadgets.}
    For every variable $x_i$ and every constraint $c_j$,
    construct a \emph{block gadget} $\hat{B}_{i,j}$ as depicted in \cref{fig:acyclic:lb_block_gadget}.
    In order to do so, we introduce vertices $a, a', \chi_1, \chi_2, y_1, y_2, b_1, b_2$.
    Then, we attach leaves $l_1$ and $l_2$ to $\chi_1$ and $\chi_2$ respectively.
    Next, we add edges from $a$ to $\chi_1$ and $y_1$, from $a'$ to $\chi_2$ and $y_2$,
    from the root vertex $r$ to $\chi_1$ and $\chi_2$, as well as the edge $\{b_1,b_2\}$.
    Finally, we add the OR gadgets $\hat{O}(a,b_1)$, $\hat{O}(a',b_2)$, $\hat{O}(\chi_1,\chi_2)$,
    and $\hat{O}(y_1,y_2)$.
    Next, for $j \in [m-1]$, we serially connect the block gadgets $\hat{B}_{i,j}$ and $\hat{B}_{i,j+1}$ so that the vertex $a'$ of $\hat{B}_{i,j}$ is the vertex $a$ of $\hat{B}_{i,j+1}$,
    thus resulting in $n$ ``paths'' $\hat{P}_1, \ldots, \hat{P}_n$ consisting of $m$ serially connected block gadgets, called \emph{variable gadgets}.
    For an illustration see \cref{fig:acyclic:lb_serially_connected}.

    Intuitively, for any optimal matching $M$,
    it will hold that either $a, a' \notin V_M$ or $a, a' \in V_M$.
    We map the first case with an assignment where $x_i$ receives the value $5$.
    As for when $a, a' \in V_M$, it will hold that $|V_M \cap \{\chi_1,\chi_2\}| = |V_M \cap \{y_1,y_2\}| = 1$,
    for a total of $4$ choices;
    we map each choice with an assignment where $x_i$ receives a value in $[4]$.


    \begin{figure}[ht]
        \centering
          \begin{subfigure}[b]{0.4\linewidth}
          \centering
            \begin{tikzpicture}[scale=0.75, transform shape]

            %%%%%%%%%% vertices and text

            \node[vertex] (a) at (0,5) {};
            \node[] () at (0,5.3) {$a$};

            \node[gray_vertex] (x1) at (1.5,6) {};
            \node[] () at (1.5,6.3) {$\chi_1$};

            \node[gray_vertex] (x2) at (4.5,6) {};
            \node[] () at (4.5,6.3) {$\chi_2$};

            \node[vertex] (y1) at (1.5,4) {};
            \node[] () at (1.5,4.3) {$y_1$};

            \node[vertex] (y2) at (4.5,4) {};
            \node[] () at (4.5,4.3) {$y_2$};

            \node[vertex] (a') at (6,5) {};
            \node[] () at (6,5.3) {$a'$};

            \node[vertex] (b1) at (2.5,2.5) {};
            \node[] () at (2.5,2.8) {$b_1$};
            \node[vertex] (b2) at (3.5,2.5) {};
            \node[] () at (3.5,2.8) {$b_2$};

            \node[gray_vertex] (r) at (3,7.5) {};
            \node[] () at (3,7.8) {$r$};


            %%%%%%%%% edges / arcs

            \draw[] (r)--(x1);
            \draw[] (r)--(x2);
            \draw[] (a)--(x1);
            \draw[] (x2)--(a');
            \draw[dashed] (x1)--(x2);
            \draw[] (a)--(y1);
            \draw[] (y2)--(a');
            \draw[dashed] (y1)--(y2);
            \draw[] (b1)--(b2);
            \draw[dashed] (a) edge [bend right] (b1);
            \draw[dashed] (a') edge [bend left] (b2);

            \end{tikzpicture}
            \caption{Block gadget.}
            \label{fig:acyclic:lb_block_gadget}
          \end{subfigure}
        \begin{subfigure}[b]{0.4\linewidth}
        \centering
            \begin{tikzpicture}[scale=0.5, transform shape]

            \node[vertex] (a) at (0,5) {};

            \node[gray_vertex] (x1) at (1.5,6) {};

            \node[gray_vertex] (x2) at (4.5,6) {};

            \node[vertex] (y1) at (1.5,4) {};

            \node[vertex] (y2) at (4.5,4) {};

            \node[vertex] (a') at (6,5) {};

            \node[vertex] (b1) at (2.5,2.5) {};
            \node[vertex] (b2) at (3.5,2.5) {};

            \begin{scope}[shift={(6,0)}]

                \node[gray_vertex] (x1') at (1.5,6) {};

                \node[gray_vertex] (x2') at (4.5,6) {};

                \node[vertex] (y1') at (1.5,4) {};

                \node[vertex] (y2') at (4.5,4) {};

                \node[vertex] (b1') at (2.5,2.5) {};
                \node[vertex] (b2') at (3.5,2.5) {};

                \node[vertex] (a'') at (6,5) {};

            \end{scope}


            \node[gray_vertex] (r) at (6,8.5) {};


            %%%%%%%%% edges / arcs

            \draw[] (r) edge [bend right] (x1);
            \draw[] (r) edge [bend right] (x2);
            \draw[] (r) edge [bend left] (x1');
            \draw[] (r) edge [bend left] (x2');
            \draw[] (a)--(x1);
            \draw[] (x2)--(a');
            \draw[dashed] (x1)--(x2);
            \draw[] (a)--(y1);
            \draw[] (y2)--(a');
            \draw[dashed] (y1)--(y2);
            \draw[] (b1)--(b2);
            \draw[dashed] (a) edge [bend right] (b1);
            \draw[dashed] (a') edge [bend left] (b2);
            \draw[] (a')--(x1');
            \draw[] (x2')--(a'');
            \draw[dashed] (x1')--(x2');
            \draw[] (a')--(y1');
            \draw[] (y2')--(a'');
            \draw[dashed] (y1')--(y2');
            \draw[] (b1')--(b2');
            \draw[dashed] (a') edge [bend right] (b1');
            \draw[dashed] (a'') edge [bend left] (b2');


            \end{tikzpicture}
            \caption{Serially connected block gadgets.}
            \label{fig:acyclic:lb_serially_connected}
          \end{subfigure}
        \caption{Gray vertices have a leaf attached. Dashed edges denote OR gadgets.}
        \label{fig:acyclic:lb}
        \end{figure}


        \proofsubparagraph{Constraint Gadget.}
        This gadget is responsible for determining constraint satisfaction,
        based on the choices made in the rest of the graph.
        For constraint $c_j$, construct the \emph{constraint gadget} $\hat{C}_j$ as follows.
        \begin{itemize}
            \item Introduce $s(c_j)$ vertices $v^j_1, \ldots, v^j_{s(c_j)}$ and
            for all $1 \le w_1 < w_2 \le s(c_j)$ add an OR gadget $\hat{O}(v^j_{w_1},v^j_{w_2})$.
            Furthermore, fix an arbitrary one-to-one mapping between vertices $v^j_1, \ldots, v^j_{s(c_j)}$ and the satisfying assignments of $c_j$.

            \item Introduce a vertex $v_j$ and add edges between $v^j$ and all vertices $v^j_1, \ldots, v^j_{s(c_j)}$.

            \item If variable $x_i$ is involved in the constraint $c_j$ and $v^j_w$ (for $w \in [s(c_j)$]) corresponds
            to an assignment where $x_i$ has value $p \in [5]$, consider the following cases:
            \begin{itemize}
                \item If $p = 1$,
                then add an OR gadget between $v^j_w$ and vertices $\{b_1, b_2, \chi_1, y_1\}$ of $\hat{B}_{i,j}$.

                \item If $p = 2$,
                then add an OR gadget between $v^j_w$ and vertices $\{b_1, b_2, \chi_1, y_2\}$ of $\hat{B}_{i,j}$.

                \item If $p = 3$,
                then add an OR gadget between $v^j_w$ and vertices $\{b_1, b_2, \chi_2, y_1\}$ of $\hat{B}_{i,j}$.

                \item If $p = 4$,
                then add an OR gadget between $v^j_w$ and vertices $\{b_1, b_2, \chi_2, y_2\}$ of $\hat{B}_{i,j}$.

                \item If $p = 5$,
                then add an OR gadget between $v^j_w$ and vertices $\{a, a',y_1, y_2\}$ of $\hat{B}_{i,j}$.
            \end{itemize}
        \end{itemize}


        Let graph $G_0$ correspond to the graph containing all variable gadgets $\hat{P}_i$ as well as all the constraint gadgets $\hat{C}_j$,
        for $i \in [n]$ and $j \in [m]$.
        We refer to the block gadget $\hat{B}_{i,j}$, to the variable gadget $\hat{P}_i$,
        and to the constraint gadget $\hat{C}_j$ of $G_z$
        as $\hat{B}^{G_z}_{i,j}$, $\hat{P}^{G_z}_i$, and $\hat{C}^{G_z}_j$ respectively.
        To refer to a vertex $v$ belonging to a constraint gadget $\hat{C}^{G_z}_j$, we write $v(\hat{C}^{G_z}_j)$.
        When it is clear from the context, we refer to vertex $v$ instead,
        and we use the same naming conventions for block gadgets.
        To construct graph $G$, introduce $\kappa = 4n + 1$ copies $G_1, \ldots, G_\kappa$ of $G_0$,
        such that they are connected sequentially as follows:
        for $i \in [n]$ and $j \in [\kappa - 1]$,
        the vertex $a'$ of $\hat{B}^{G_j}_{i,m}$ is the vertex $a$ of $\hat{B}^{G_{j+1}}_{i,1}$.
        Let $\mathcal{P}^i$ denote the ``path'' resulting from $\hat{P}^{G_1}_i, \ldots, \hat{P}^{G_\kappa}_i$.
        Set $\ell = 1 + \kappa \cdot \ell'$, where $\ell' =  6m n + \sum_{j=1}^m \left(4s(c_j) + \binom{s(c_j)}{2} + 1 \right)$,
        and let $(G,\ell)$ denote the instance of \AcyclicM.


        \begin{lemma}\label{lem:acyclic:lb:csp->acyclic}
            If $\phi$ is satisfiable,
            then $(G,\ell)$ is a yes-instance of \AcyclicM.
        \end{lemma}

        \begin{nestedproof}
            Let $f \colon X \to [5]$ be a satisfying assignment for $\phi$, that is, $f$ satisfies all constraints $c_j$.
            We will construct a matching $M \subseteq E(G)$ of size $|M| \geq \ell$ such that $G[V_M]$ is a forest.
            \begin{itemize}
                \item First, add to $M$ the edge $\{r,r'\}$.

                \item Then, for every OR gadget $\hat{O}(v_1,v_2)$ in $G$, include in $M$ the edge between
                its two private vertices $\{p^{v_1,v_2}_1,p^{v_1,v_2}_2\}$,
                for a total of $\kappa \cdot \left(4 m n + \sum_{j=1}^m \left(4 s(c_j) + \binom{s(c_j)}{2}\right)\right)$ such edges.

                \item Next, for every $z \in [\kappa]$ and every constraint gadget $\hat{C}^{G_z}_j$, since $f$ is a satisfying assignment,
                there exists some vertex $v^j_w$ with $w \in [s(c_j)]$ which corresponds to the restriction
                of $f$ to the $q$ variables involved in $c_j$.
                Add in $M$ the edge $\{v^j,v^j_w\}$ for a total of $\kappa \cdot m$ such edges.

                \item Finally, for every $z \in [\kappa]$ and $i \in [n]$, include in $M$ for every block gadget $\hat{B}_{i,j}$ with $j \in [m]$,
                the following edges:
                \begin{itemize}
                    \item if $f(x_i) = 1$, the edges $\{ y_2, a' \}$ and $\{ \chi_2, l_2 \}$,

                    \item if $f(x_i) = 2$, the edges $\{ y_1, a \}$ and $\{ \chi_2, l_2 \}$,

                    \item if $f(x_i) = 3$, the edges $\{ y_2, a' \}$ and $\{ \chi_1, l_1 \}$,

                    \item if $f(x_i) = 4$, the edges $\{ y_1, a \}$ and $\{ \chi_1, l_1 \}$,

                    \item if $f(x_i) = 5$, the edges $\{ b_1, b_2 \}$ and $\{ \chi_1, l_1 \}$.
                \end{itemize}
                Notice that the total number of edges added in this step is
                $\kappa \cdot 2mn$.
            \end{itemize}
            Consequently, $M$ contains exactly
            $1 + \kappa \cdot \left(6m n + \sum_{j=1}^m \left(4s(c_j) + \binom{s(c_j)}{2} + 1 \right)\right) = \ell$ edges.
            Furthermore, it is easy to see that $M$ is a matching.

            It remains to show that $G[V_M]$ is a forest.
            Consider any constraint gadget $\hat{C}^{G_z}_j$.
            Let $w \in [s(c_j)]$ such that $v^j_w (G_z) \in V_M$ is the only vertex of the constraint gadget,
            apart from $v^j (G_z)$, that belongs to $V_M$ and is not a private vertex of an OR gadget.
            We argue that there are no edges between vertices of the constraint gadgets and the block gadgets in $G[V_M]$.
            To see this, notice that since $v^j_w (G_z)$ corresponds to an assignment which is a restriction of $f$
            to the $q$ variables involved in $c_j$, then all the vertices of the block gadgets that it has an OR gadget with
            do not belong to $V_M$, since no edge incident to them belongs to $M$.

            Given the previous paragraph, it is easy to see that the vertices of the constraint gadgets belonging to $V_M$ induce acyclic components.
            Indeed, the induced graph of any constraint gadget can be obtained by a matching (edges between private vertices of OR gadgets),
            on which we introduce the edge $\{v^j,v^j_w\}$ for some $w \in [s(c_j)]$ and add edges between one vertex of a subset of the matching towards $v^j_w$.

            It remains to argue that the graph induced by $r$, $r'$, and the vertices of the block gadgets in $V_M$ is acyclic.
            First, notice that the graph induced by the vertices of any single block gadget $\hat{B}^{G_z}_{i,j}$ in $V_M$ along with $\{r,r'\}$
            is acyclic and vertices $a$ and $a'$ are disconnected.
            Furthermore, for $z_1,z_2 \in [\kappa]$, $j_1, j_2 \in [m]$, and distinct $i_1,i_2 \in [n]$,
            notice that the block gadgets $\hat{B}^{G_{z_1}}_{i_1,j_1}$ and $\hat{B}^{G_{z_2}}_{i_2,j_2}$
            are disconnected in $G[V_M] - r$, thus if $G[V_M]$ contains a cycle, then this cycle involves only block gadgets
            corresponding to a single variable $x_i$.
            Finally, for $i \in [n]$ it holds that any two sequential block gadgets (as in \cref{fig:acyclic:lb_serially_connected})
            are connected in $G - r$ via their single common vertex which appears as $a$ in the first and $a'$ in the second gadget.
            Consequently, any cycle in $G[V_M]$ involving at least two block gadgets $\hat{B}^{G_{z_1}}_{i,j_1}$ and $\hat{B}^{G_{z_2}}_{i,j_2}$
            must contain $r$, the vertex $a' (\hat{B}^{G_{z_1}}_{i,j_1})$, and the vertex $a (\hat{B}^{G_{z_2}}_{i,j_2})$.
            Given that the vertices $a$ and $a'$ of any single block gadget are disconnected in $G[V_M] - \{r\}$,
            it follows that vertices $a' (\hat{B}^{G_{z_1}}_{i,j_1})$ and $a (\hat{B}^{G_{z_2}}_{i,j_2})$
            are disconnected in $G[V_M] - \{r\}$, a contradiction.

            Consequently, it follows that $G[V_M]$ is indeed a forest and this completes the proof.
        \end{nestedproof}

        \begin{lemma}\label{lem:acyclic:lb:acyclic->csp}
            If $(G,\ell)$ is a yes-instance of \AcyclicM,
            then $\phi$ is satisfiable.
        \end{lemma}

        \begin{nestedproof}

            We start with the following claim.

            \begin{claim}\label{claim:acyclic:lb:acyclic->csp:maximum_matching}
                There exists an acyclic matching $M_{\max}$ of $G$ of maximum size such that,
                for any OR gadget $\hat{O}(v_1,v_2)$ of $G$ it holds that $|\{v_1,v_2\} \cap V_{M_{\max}}| \le 1$
                and $\{p^{v_1,v_2}_1,p^{v_1,v_2}\} \in M_{\max}$.
            \end{claim}

            \begin{claimproof}
                Let $M_{\max}$ be a maximum acyclic matching of $G$, and assume that $G$ contains an OR gadget $\hat{O}(v_1,v_2)$.
                We show that we can obtain an acyclic matching $M'_{\max}$ of $G$ of the same size
                such that $|\{v_1,v_2\} \cap V_{M'_{\max}}| \le 1$ and $\{p^{v_1,v_2}_1,p^{v_1,v_2}\} \in M'_{\max}$.
                To this end, we consider different cases depending on the vertices of the OR gadget that belong to $V_{M_{\max}}$.

                If $v_1,v_2 \notin V_{M_{\max}}$, then $\{p^{v_1,v_2}_1,p^{v_1,v_2}_2\} \in M_{\max}$ as
                $M_{\max}$ is of maximum size.

                Now assume that $|\{v_1,v_2\} \cap V_{M_{\max}}| = 1$, and without loss of generality $v_1 \in V_{M_{\max}}$ (the other case is analogous).
                Let $e \in M_{\max}$ be the edge incident to $v_1$.
                If $e \neq \{v_1,p^{v_1,v_2}_1\}$, then it holds that $\{p^{v_1,v_2}_1,p^{v_1,v_2}_2\} \in M_{\max}$ as $M_{\max}$ is of maximum size.
                Otherwise $p^{v_1,v_2}_2 \notin V_{M_{\max}}$ and the acyclic matching $M'_{\max} = (M_{\max} \setminus \{e\}) \cup \{\{p^{v_1,v_2}_1,p^{v_1,v_2}_2\}\}$ has the desired properties.

                Lastly, assume that $v_1,v_2 \in V_{M_{\max}}$.
                Notice that in that case, either $p^{v_1,v_2}_1 \notin V_{M_{\max}}$ or $p^{v_1,v_2}_2 \notin V_{M_{\max}}$, as otherwise $G[V_{M_{\max}}]$
                contains a cycle.
                If $\{v_1,v_2\} \in M_{\max}$, then $M'_{\max} = (M_{\max} \setminus \{\{v_1,v_2\}\}) \cup \{\{p^{v_1,v_2}_1,p^{v_1,v_2}_2\}\}$ has the desired properties.
                Otherwise, it holds that for either $v_1$ or $v_2$ there exists an edge $e \in M_{\max}$ incident to it
                that does not belong to the OR gadget $\hat{O}(v_1,v_2)$.
                Assume without loss of generality that $e$ is incident to $v_1$, and let $e' \in M_{\max}$ be the edge incident to $v_2$ belonging to $M_{\max}$.
                Then, the acyclic matching $M'_{\max} = (M_{\max} \setminus \{e'\}) \cup \{\{p^{v_1,v_2}_1,p^{v_1,v_2}_2\}\}$ has the desired properties.
            \end{claimproof}

            Armed with the previous claim, we can now prove the existence of an acyclic matching in $G$ with specific properties.

            \begin{claim}\label{claim:acyclic:lb:acyclic->csp:matching_properties}
                There exists an acyclic matching $M$ of $G$ such that,
                for any OR gadget $\hat{O}(v_1,v_2)$ of $G$ it holds that $|\{v_1,v_2\} \cap V_M| \le 1$ and $\{p^{v_1,v_2}_1,p^{v_1,v_2}\} \in M$.
                Furthermore, it holds that
                \begin{itemize}
                    \item $\{r,r'\} \in M$,

                    \item for all constraint gadgets $\hat{C}^{G_z}_j$ with $z \in [\kappa]$ and $j \in [m]$,
                    $\{v^j (\hat{C}^{G_z}_j),v^j_w (\hat{C}^{G_z}_j)\} \in M$ where $w \in [s(c_j)]$, and

                    \item for all block gadgets $\hat{B}^{G_z}_{i,j}$ with $z \in [\kappa]$, $j \in [m]$, and $i \in [n]$,
                    $| \{ \{\chi_1,l_1\}, \{\chi_2,l_2\} \} \cap M| = 1$ and $|\{\{ y_1, a \}, \{ y_2, a' \}, \{ b_1, b_2 \}\} \cap M| = 1$.
                \end{itemize}
            \end{claim}

            \begin{claimproof}
                Let $M_{\max}$ be an acyclic matching of $G$ of maximum size as in \cref{claim:acyclic:lb:acyclic->csp:maximum_matching},
                where $|M_{\max}| \ge \ell$ since by assumption $G$ has an acyclic matching of size at least $\ell$.
                Consider the following reduction rule.

                \proofsubparagraph{Rule~$(\dagger)$.}
                Let $\{u,v\} \in M_{\mathsf{in}}$, and let $l \neq v$ be the sole leaf of $u$ in $G$.
                Then, replace $M_{\mathsf{in}}$ with $M_{\mathsf{out}} = (M_{\mathsf{in}} \setminus \{\{u,v\}\}) \cup \{\{u,l\}\}$.

                It is easy to see that in Rule~$(\dagger)$ if $M_{\mathsf{in}}$ is an acyclic matching,
                then so is $M_{\mathsf{out}}$, while the size of the two matchings is the same.
                Let $M$ denote the acyclic matching obtained after exhaustively applying Rule~$(\dagger)$ in $M_{\max}$.
                Notice that $|M| \ge \ell$, while for any OR gadget $\hat{O}(v_1,v_2)$ of $G$ it holds that $|\{v_1,v_2\} \cap V_M| \le 1$
                and $\{p^{v_1,v_2}_1,p^{v_1,v_2}\} \in M$.

                Notice that due to Rule~$(\dagger)$, if $r \in V_M$, then $\{r,r'\} \in M$.
                Furthermore, for any edge in $M \setminus \{\{r,r'\}\}$ it holds that either both of its endpoints belong to the vertices of a block gadget,
                or they both belong to the vertices of a constraint gadget;
                that is due to the fact that any connected pair of vertices from both block and constraint gadgets is connected via an OR gadget.

                We now show that for any $z \in [\kappa]$ and $j \in [m]$,
                $M$ contains at most $4s(c_j) + \binom{s(c_j)}{2} + 1$ edges from the constraint gadget $\hat{C}^{G_z}_j$.
                It suffices to show that, apart from the edges of the OR gadgets, $M$ contains at most $1$ additional edge.
                Notice that since all vertices $\{v^j_1, \ldots, v^j_{s(c_j)}\}$ are pairwise connected via OR gadgets, at most one of them,
                say $v^j_w$ for $w \in [s(c_j)]$, may belong to $V_M$. Furthermore, the only vertex of the constraint gadget it can be matched with is $v^j$,
                consequently, the only additional edge $M$ may contain from the constraint gadget is $\{v^j,v^j_w\}$.

                Next, we show that for all $z \in [\kappa]$, $j \in [m]$, and $i \in [n]$,
                $M$ contains at most $6$ edges from the block gadget $\hat{B}^{G_z}_{i,j}$.
                It suffices to show that, apart from the edges of the OR gadgets, $M$ contains at most $2$ additional edges.
                To see this, notice that if $\chi_1 \in V_M$, then $\{\chi_1,l_1\} \in M$ and $\chi_2 \notin V_M$ (similarly for $\chi_2$).
                Furthermore, one can easily verify that $|\{\{ y_1, a \}, \{ y_2, a' \}, \{ b_1, b_2 \}\} \cap M| \le 1$.

                Recall that $|M| \ge \ell = 1 + \kappa \cdot \ell'$,
                where $\ell'= 6m n + \sum_{j=1}^m \left(4s(c_j) + \binom{s(c_j)}{2} + 1 \right)$.
                Due to the previous two paragraphs and summing over all constraint and block gadgets,
                $M$ contains at most $\kappa \cdot \ell'$ edges belonging to them.
                Consequently, it follows that the bounds obtained in the previous two paragraphs are tight for all constraint and block gadgets
                (which implies that the inclusions are as described in the statement),
                while $\{r,r'\} \in M$.
            \end{claimproof}


            Now let $M$ be an acyclic matching of $G$ as in \cref{claim:acyclic:lb:acyclic->csp:matching_properties}.
            Consider a mapping $\mu$ over the block gadgets $\hat{B}^{G_z}_{i,j}$ to $[5]$.
            In particular, for all $z \in [\kappa]$, $j \in [m]$, and $i \in [n]$, if for the edges of $\hat{B}^{G_z}_{i,j}$ it holds that
            \begin{itemize}
                \item $\{y_2,a'\}, \{\chi_2,l_2\} \in M$, then $\mu(\hat{B}^{G_z}_{i,j}) = 1$,
                \item $\{y_1,a\}, \{\chi_2,l_2\} \in M$, then $\mu(\hat{B}^{G_z}_{i,j}) = 2$,
                \item $\{y_2,a'\}, \{\chi_1,l_1\} \in M$, then $\mu(\hat{B}^{G_z}_{i,j}) = 3$,
                \item $\{y_1,a\}, \{\chi_1,l_1\} \in M$, then $\mu(\hat{B}^{G_z}_{i,j}) = 4$,
                \item $\{b_1,b_2\} \in M$, then $\mu(\hat{B}^{G_z}_{i,j}) = 5$.
            \end{itemize}
            Notice that $\mu$ is well-defined due to \cref{claim:acyclic:lb:acyclic->csp:matching_properties}.
            We will say that an \emph{inconsistency} occurs in a variable gadget $\hat{P}^{G_z}_i$ if there exist
            sequential block gadgets $\hat{B}^{G_z}_{i,j}$ and $\hat{B}^{G_z}_{i,j+1}$ (that is, vertex $a'$ of the first is vertex $a$ of the second)
            with $\mu (\hat{B}^{G_z}_{i,j}) \neq \mu(\hat{B}^{G_z}_{i,j+1})$.
            We say that $G_z$ is \emph{consistent} if no inconsistency occurs in its variable gadgets $\hat{P}^{G_z}_i$ for all $i \in [n]$.

            \begin{claim}
                There exists $\pi \in [\kappa]$ such that $G_\pi$ is consistent.
            \end{claim}

            \begin{claimproof}
                We will show that for all $i \in [n]$, at most $4$ inconsistencies occur in $\mathcal{P}^i$.
                In that case, at most $4n$ inconsistencies happen over all paths $\mathcal{P}^i$,
                and since $\kappa = 4n+1$, by pigeonhole principle there exists a $\pi \in [\kappa]$ such that no inconsistencies
                happen in the variable gadgets of $G_{\pi}$.
                To this end, fix $z \in [\kappa]$, $j \in [m-1]$, and $i \in [n]$,
                and consider the two sequential block gadgets $\hat{B}^{G_z}_{i,j}$ and $\hat{B}^{G_z}_{i,j+1}$.

                We first show that if $\{y_2 (\hat{B}^{G_z}_{i,j}) ,a' (\hat{B}^{G_z}_{i,j})\} \in M$,
                then $\{y_2 (\hat{B}^{G_z}_{i,j+1}), a' (\hat{B}^{G_z}_{i,j+1})\} \in M$.
                Indeed, since $a' (\hat{B}^{G_z}_{i,j}) = a (\hat{B}^{G_z}_{i,j+1})$,
                it follows that $\{y_1 (\hat{B}^{G_z}_{i,j+1}), a(\hat{B}^{G_z}_{i,j+1})\}, \{b_1 (\hat{B}^{G_z}_{i,j+1}),b_2 (\hat{B}^{G_z}_{i,j+1})\} \notin M$,
                and since $|\{\{ y_1, a \}, \{ y_2, a' \}, \{ b_1, b_2 \}\} \cap M| = 1$ for any block gadget,
                the statement follows.
                This implies that if $\mu (\hat{B}^{G_z}_{i,j}) \in \{1,3\}$ then $\mu (\hat{B}^{G_z}_{i,j+1}) \in \{1,3\}$.

                Next, it is easy to see that if $\{b_1 (\hat{B}^{G_z}_{i,j}) ,b_2 (\hat{B}^{G_z}_{i,j})\} \in M$,
                then $\{y_1 (\hat{B}^{G_z}_{i,j+1}), a (\hat{B}^{G_z}_{i,j+1})\} \notin M$; indeed,
                since $b_2 (\hat{B}^{G_z}_{i,j}) \in V_M$ it holds that $a' (\hat{B}^{G_z}_{i,j}) = a (\hat{B}^{G_z}_{i,j+1}) \notin V_M$.
                This implies that if $\mu (\hat{B}^{G_z}_{i,j}) = 5$ then $\mu (\hat{B}^{G_z}_{i,j+1}) \in \{1,3,5\}$.

                Now assume that $\{y_2 (\hat{B}^{G_z}_{i,j}) ,a' (\hat{B}^{G_z}_{i,j})\} , \{\chi_2 (\hat{B}^{G_z}_{i,j}) ,l_2 (\hat{B}^{G_z}_{i,j})\} \in M$.
                In that case, it holds that $\chi_1 (\hat{B}^{G_z}_{i,j+1}) \notin V_M$;
                if that were not the case,
                then $G[V_M]$ contains a cycle due to vertices $\chi_2 (\hat{B}^{G_z}_{i,j}) ,a' (\hat{B}^{G_z}_{i,j}), \chi_1 (\hat{B}^{G_z}_{i,j+1}), r$.
                Consequently, $\{\chi_2 (\hat{B}^{G_z}_{i,j+1}), l_2 (\hat{B}^{G_z}_{i,j+1})\} \in M$ follows.
                This implies that if $\mu (\hat{B}^{G_z}_{i,j}) = 1$ then $\mu (\hat{B}^{G_z}_{i,j+1}) = 1$,
                while the discussion so far implies that if $\mu (\hat{B}^{G_z}_{i,j}) = 3$ then $\mu (\hat{B}^{G_z}_{i,j+1}) \in \{1,3\}$.

                Finally, assume that $\{y_1 (\hat{B}^{G_z}_{i,j}) ,a (\hat{B}^{G_z}_{i,j})\} , \{\chi_2 (\hat{B}^{G_z}_{i,j}) ,l_2 (\hat{B}^{G_z}_{i,j})\},
                \{y_1 (\hat{B}^{G_z}_{i,j+1}) , a (\hat{B}^{G_z}_{i,j+1})\} \in M$.
                Similarly to the previous paragraph, it holds that $\chi_1 (\hat{B}^{G_z}_{i,j+1}) \notin V_M$,
                which implies that $\{\chi_2 (\hat{B}^{G_z}_{i,j+1}), l_2 (\hat{B}^{G_z}_{i,j+1})\} \in M$.
                Consequently, it holds that if $\mu (\hat{B}^{G_z}_{i,j}) = 2$ then $\mu (\hat{B}^{G_z}_{i,j+1}) \neq 4$,
                or equivalently $\mu (\hat{B}^{G_z}_{i,j+1}) \in \{1,2,3,5\}$,
                while by definition, if $\mu (\hat{B}^{G_z}_{i,j}) = 4$ then $\mu (\hat{B}^{G_z}_{i,j+1}) \in \{1,2,3,4,5\}$.

                The previous three paragraphs imply that the number of inconsistencies that occur in $\mathcal{P}^i$ is at most $4$,
                and this completes the proof.
            \end{claimproof}

            Fix $j \in [m]$ and let $f \colon X \to [5]$ be the assignment on the variables of $\phi$ where for all $i \in [n]$,
            $f(x_i) = \mu(\hat{B}^{G_\pi}_{i,j})$ (since $G_\pi$ is consistent, the choice of $j$ is irrelevant).
            We argue that $f$ satisfies all constraints $c_1, \ldots, c_m$.
            Consider a constraint $c_j$.
            Regarding the constraint gadget $\hat{C}^{G_\pi}_j$, by \cref{claim:acyclic:lb:acyclic->csp:matching_properties}
            it holds that there exists $w \in [s(c_j)]$ such that $\{v^j,v^j_w\} \in M$,
            and any vertex of $\hat{B}^{G_z}_{i,j}$ ($i \in [n]$) connected to $v^j_w$ via an OR gadget does not belong to $V_M$.
            We argue that the restriction of $f$ to the variables involved in $c_j$ corresponds to vertex $v^j_w$ in $\hat{C}^{G_\pi}_j$.
            Assume that $x_i$ is a variable that appears in the constraint $c_j$, while $v^j_w$ corresponds to an assignment where $x_i = p \in [5]$.
            Then, consider the following cases.
            \begin{itemize}
                \item If $p = 1$,
                then there exist OR gadgets between $v^j_w$ and vertices $\{b_1, b_2, \chi_1, y_1\}$ of $\hat{B}_{i,j}$.
                Consequently, $\{b_1, b_2, \chi_1, y_1\} \notin M$,
                which by \cref{claim:acyclic:lb:acyclic->csp:matching_properties} implies that $\{y_2,a'\}, \{\chi_2,l_2\} \in M$,
                thus $f(x_i) = 1$.

                \item If $p = 2$,
                then there exist OR gadgets between $v^j_w$ and vertices $\{b_1, b_2, \chi_1, y_2\}$ of $\hat{B}_{i,j}$.
                Consequently, $\{b_1, b_2, \chi_1, y_2\} \notin M$,
                which by \cref{claim:acyclic:lb:acyclic->csp:matching_properties} implies that $\{y_1,a\}, \{\chi_2,l_2\} \in M$,
                thus $f(x_i) = 2$.

                \item If $p = 3$,
                then there exist OR gadgets between $v^j_w$ and vertices $\{b_1, b_2, \chi_2, y_1\}$ of $\hat{B}_{i,j}$.
                Consequently, $\{b_1, b_2, \chi_2, y_1\} \notin M$,
                which by \cref{claim:acyclic:lb:acyclic->csp:matching_properties} implies that $\{y_2,a'\}, \{\chi_1,l_1\} \in M$,
                thus $f(x_i) = 2$.

                \item If $p = 4$,
                then there exist OR gadgets between $v^j_w$ and vertices $\{b_1, b_2, \chi_2, y_2\}$ of $\hat{B}_{i,j}$.
                Consequently, $\{b_1, b_2, \chi_2, y_2\} \notin M$,
                which by \cref{claim:acyclic:lb:acyclic->csp:matching_properties} implies that $\{y_1,a\}, \{\chi_1,l_1\} \in M$,
                thus $f(x_i) = 4$.

                \item If $p = 5$,
                then there exist OR gadgets between $v^j_w$ and vertices $\{a, a',y_1, y_2\}$ of $\hat{B}_{i,j}$.
                Consequently, $\{a,a'\} \notin M$,
                which by \cref{claim:acyclic:lb:acyclic->csp:matching_properties} implies that $\{b_1,b_2\} \in M$,
                thus $f(x_i) = 5$.
            \end{itemize}
            Therefore, the satisfying assignment of the variables appearing in $c_j$ corresponding to $v^j_w$
            is a restriction of $f$, thus $f$ satisfies $c_j$.
            Since this holds for any $j \in [m]$, $\phi$ is satisfied.
            This completes the proof.
        \end{nestedproof}


        \begin{lemma}\label{lem:acyclic:lb:pathwidth}
            It holds that $\pw(G) = n + \bO(1)$.
        \end{lemma}


        \begin{nestedproof}
            We prove the statement by providing a mixed search strategy to clear $G$ using at most this many searchers simultaneously.
            Since for the mixed search number \ms{} it holds that $\pw(G) \leq \ms(G) \leq \pw(G) + 1$,
            we show that $\ms(G) \leq n + \bO(1)$ and the statement follows.
            We start by placing a searcher on vertex $r'$ and sliding it to the root vertex $r$.

            Start with graph $G_1$.
            Place $1 + s(c_1)$ searchers to the vertices $v^j, v^j_1, \ldots, v^j_{s(c_j)}$ of $\hat{C}^{G_1}_1$
            as well as $8 s(c_1) + 2 \binom{s(c_1)}{2}$ searchers to all the private vertices belonging to OR gadgets involving the previous vertices,
            and $n$ searchers on vertices $a$ of block gadgets $\hat{B}^{G_1}_{i,1}$, for $i \in [n]$.
            Notice that the only non-cleared edges with endpoints in $\hat{C}^{G_1}_1$ are those whose other endpoint belongs to the block gadgets $\hat{B}^{G_1}_{i,1}$.
            Next we describe the procedure to clear $\hat{B}^{G_1}_{i,1}$ as well as any edge between said gadget and $\hat{C}^{G_1}_1$.
            Starting from $\hat{B}^{G_1}_{1,1}$, it suffices to move $17$ extra searchers
            to the rest of its vertices (including the private vertices of its OR gadgets),
            and removing all searchers apart from the one placed on $a'$.
            By following the same procedure, eventually all block gadgets $\hat{B}^{G_1}_{i,1}$ for $i \in [n]$ are cleared, as well as all edges incident to vertices in $\hat{C}^{G_1}_1$.

            In order to clear the rest of the graph, we first move the searchers from $\hat{C}^{G_z}_j$ to $\hat{C}^{G_z}_{j+1}$ if $j < m$
            or to $\hat{C}^{G_{z+1}}_1$ alternatively (possibly introducing new searchers if required),
            clear the latter, and then proceed by clearing the corresponding block gadgets.
            By repeating this procedure, in the end we clear all the edges of $G$ by using at most $1 + n + 17 + 1 + 9 \cdot 5^q + 2 \binom{5^q}{2} = n + \bO (1)$ searchers.
        \end{nestedproof}
%
%
        \cref{thm:acyclic:lb} now follows from \cref{lem:acyclic:lb:csp->acyclic,lem:acyclic:lb:acyclic->csp,lem:acyclic:lb:pathwidth}.
\end{proof}