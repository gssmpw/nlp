\section{Kernelization Lower Bounds}

\begin{theorem}\label{thm:induced_no_kernel}
    {\InducedM} does not admit a polynomial kernel when parameterized by the vertex cover number
    of the input graph, unless $\mathsf{NP} \subseteq \mathsf{coNP} \slash \mathsf{poly}$.
\end{theorem}

\begin{proof}
    We present a polynomial parameter transformation reducing from \kMC.
    In the latter we are given a graph $G = (V, E)$ and a partition of $V$ into $k$ independent sets $V_1, \ldots, V_k$, each of size $n$,
    and we are asked to determine whether $G$ contains a $k$-clique.
    It is known that {\kMC} parameterized by $k \log n$ does not admit a polynomial kernel
    unless $\mathsf{NP} \subseteq \mathsf{coNP} \slash \mathsf{poly}$~\cite{algorithmica/HermelinKSWW15}.

    Consider an instance $(G,k)$ of \kMC, where $V_i = \setdef{v^i_j}{j \in [n]}$ for all $i \in [k]$.
    For all $i \in [k]$ and $w \in [\log n]$,
    let $b^i_w \colon V_i \to \{ 0,1 \}$ denote the $w$-th bit in the binary expansion of $v^i_j \in V_i$.
    Construct graph $H$ as follows.
    \begin{itemize}
        \item For all $i \in [k]$ and $w \in [\log n]$, introduce a clique on vertices $g^i_{w,0}$, $g^i_{w,1}$, and $l^i_w$.
        \item For every edge $\{ v^{i_1}_{j_1}, v^{i_2}_{j_2} \} \in E(G)$,
        introduce an \emph{adjacency vertex} $e^{i_1,i_2}_{j_1,j_2}$ which is incident to all vertices
        encoding its endpoints, that is, $g^{i_1}_{w,b^{i_1}_w(v^{i_1}_{j_1})}$ and $g^{i_2}_{w,b^{i_2}_w(v^{i_2}_{j_2})}$ for all $w \in [\log n]$.
        Let $E_{i_1,i_2}$ denote the set of the adjacency vertices due to edges between vertices in
        $V_{i_1}$ and $V_{i_2}$.
        \item For all $E_{i_1,i_2} \neq \varnothing$, introduce a vertex $l_{i_1,i_2}$ which is adjacent to all vertices of $E_{i_1,i_2}$.
    \end{itemize}
    We say that a vertex $v^i_j \in V_i$ of $G$ is \emph{encoded} by vertices $\setdef{g^i_{w,b^i_w(v^i_{s(i)})}}{w \in [\log n]}$.
    This concludes the construction of graph $H$.
    Notice that its vertex cover number is at most $2k \log n + \binom{k}{2}$,
    since deleting all vertices $g^i_{w,0}$, $g^i_{w,1}$, and $l_{i_1,i_2}$ results in an independent set.
    It remains to show that $(H, \ell)$ is an equivalent instance of \InducedM, where $\ell = k \log n + |E(G)| - \binom{k}{2}$.

    For the forward direction, consider a $k$-clique of $G$ consisting of vertices
    $\mathcal{V} = \setdef{v^i_{s(i)} \in V_i}{i \in [k]} \subseteq V(G)$
    for some function $s \colon [k] \to [n]$.
    Let $S$ be composed of
    \begin{enumerate}
        \item the vertices of $H$ corresponding
        to the binary encoding of the vertices in $\mathcal{V}$,
        that is, $\setdef{g^i_{w,b^i_w(v^i_{s(i)})}}{i \in [k], \, w \in [\log n]}$,
        \item the adjacency vertices of any edge of $G$ not belonging to the $k$-clique,
        that is, for any $1 \le i_1 < i_2 \le k$,
        vertices $e^{i_1,i_2}_{j_1,j_2}$ where either $v^{i_1}_{j_1} \notin \mathcal{V}$ or $v^{i_2}_{j_2} \notin \mathcal{V}$.
    \end{enumerate}
    It holds that $|S| = k \log n + |E(G)| - \binom{k}{2} \le \ell$.
    We argue that $G-S$ is an induced matching,
    that is, $1$-regular.
    For any $1 \le i_1 < i_2 \le k$ vertex $l_{i_1,i_2}$ has only a single neighbor,
    the adjacency vertex $e^{i_1,i_2}_{s(i_1),s(i_2)}$.
    That is also the case for the latter, as $S$ contains all the vertices that encode its endpoints.
    Lastly, for all $i \in [k]$ and $w \in [\log n]$,
    any undeleted vertex $g^i_{w,z}$ with $z \in \{ 0,1 \}$ has only one neighbor,
    vertex $l^i_w$, and symmetrically for the latter.

    For the converse direction, let $S \subseteq V(G)$ with $|S| \le \ell$ such that $G-S$ is an induced matching.
    Notice that for every $i \in [k]$ and $w \in [\log n]$,
    it holds that $S \cap \{ g^i_{w,0}, g^i_{w,1}, l^i_w \} \neq \varnothing$;
    if that were not the case, then $G-S$ has a $K_3$ as a subgraph.

    \begin{claim}
        For every $1 \le i_1 < i_2 \le k$, it holds that $|E_{i_1,i_2} \setminus S| \le 1$.
    \end{claim}

    \begin{claimproof}
        Assume that this is not the case and let $u_1, u_2 \in E_{i_1,i_2} \setminus S$.
        Notice that both $u_1$ and $u_2$ are adjacent to $l_{i_1,i_2}$.
        In that case, if $l_{i_1,i_2} \notin S$, then $\deg_{G-S} (l_{i_1,i_2}) \ge 2$,
        a contradiction.
    \end{claimproof}
\end{proof}