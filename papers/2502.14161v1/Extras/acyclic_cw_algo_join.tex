\subparagraph{Joining labels with edges, $H = \eta_{i,j}(H')$.}
Before describing how to populate the table in this case,
we first define a helper function%
\footnote{We omit the use of braces in case of singletons for the sake of readability.}
$\fjoin \colon \Gamma \times \Gamma \to 2^{\Gamma \times \Gamma}$
as follows, where the row denotes the first input and the column the second:
\[
	\begin{array}{r|cccccc}
		f_{\mathsf{join}} 	& \gamma_0 							& \gamma_{\tilde{1}} 																			& \gamma_1 							& \gamma_{\tilde{2}} 				& \gamma_2 							& \gamma_{-2} \\
		\hline
		\gamma_0 			& (\gamma_0, \gamma_0) 				& (\gamma_0, \gamma_{\tilde{1}}) 																& (\gamma_0, \gamma_1) 				& (\gamma_0, \gamma_{\tilde{2}}) 	& (\gamma_0, \gamma_2) 				& (\gamma_0, \gamma_{-2}) \\
		\gamma_{\tilde{1}} 	& (\gamma_{\tilde{1}}, \gamma_0) 	& \left\{\makecell{(\gamma_{\tilde{1}}, \gamma_{\tilde{1}}) \\ (\gamma_1,\gamma_1)} \right\} 	& (\gamma_{\tilde{1}}, \gamma_1) 	& (\gamma_1, \gamma_{-2}) 			& (\gamma_{\tilde{1}}, \gamma_{-2}) & \varnothing \\
		\gamma_1 			& (\gamma_1, \gamma_0) 				& (\gamma_1, \gamma_{\tilde{1}}) 																& (\gamma_1, \gamma_1) 				& \varnothing 								& (\gamma_1, \gamma_{-2}) 	& \varnothing \\
		\gamma_{\tilde{2}} 	& (\gamma_{\tilde{2}}, \gamma_0)	& (\gamma_{-2}, \gamma_1) 																		& \varnothing 						& \varnothing 								& \varnothing 				& \varnothing \\
		\gamma_2 			& (\gamma_2, \gamma_0) 				& (\gamma_{-2}, \gamma_{\tilde{1}}) 															& (\gamma_{-2}, \gamma_1) 			& \varnothing 								& \varnothing 				& \varnothing \\
		\gamma_{-2}			& (\gamma_{-2}, \gamma_0) 			& \varnothing																					& \varnothing 						& \varnothing 								& \varnothing 				& \varnothing
	\end{array}
\]
In the following, let $\fjoin^{-1}(\alpha, \beta) = \setdef{(\alpha', \beta') \in \Gamma \times \Gamma}{(\alpha,\beta) \in \fjoin(\alpha', \beta')}$.
For $s \colon [\cw] \to \Gamma$ we set
\[
    \DPt_H[s] =
        \begin{cases}
			\DPt_{H'}[s]		&\text{if $\gamma_0 \in \{s(i), \, s(j)\}$,}\\
            \varnothing			&\text{if $\fjoin^{-1}(s(i),s(j)) = \varnothing$,}\\
            \rmc\left(
				\proj(\acjoin(\mathcal{A},\{ ( \{\{i,j\}\},0)\}), \, s^{-1}(\gamma_{-2}) \cap \{i,j\})
        	\right) 			&\text{otherwise,}
        \end{cases}
\]
where
\[
	\mathcal{A} = \bigcup_{(\alpha',\beta') \in \fjoin^{-1}(s(i),s(j))} \DPt_{H'} \Bigl[ s[i \mapsto \alpha'][j \mapsto \beta'] \Bigr].
\]

Notice that in the first case we do not need to use the operators $\acreduce$ and $\rmc$ since we update
$\DPt_H[s]$ with one table from $\DPt_{H'}$.
The second case can only be when (i) $s(i)=s(j)=\gamma_{-2}$,
or (ii) $\gamma_0 \notin \{s(i),s(j)\}$ and
$\{\gamma_{\tilde{2}},\gamma_{2}\} \cap \{s(i),s(j)\} \neq \varnothing$.
As for the third case, we have that $s(i),s(j) \in \{\gamma_{\tilde{1}}, \gamma_{1}, \gamma_{-2}\}$ and
either $s(i) \neq \gamma_{-2}$ or $s(j) \neq \gamma_{-2}$.
Intuitively, we consider the weighted partitions $(p,w) \in \mathcal{A}$ such that
$i$ and $j$ belong to different blocks of $p$,
we merge the blocks containing $i$ and $j$,
remove the elements in $s^{-1}(\gamma_{-2}) \cap \{i,j\}$ from the resulting block,
and add the resulting weighted partition to $\DPt_H[s]$.
Notice that we also have to consider the case where a new edge has been added to the partial solution due to the join operation.
Furthermore, it is not necessary to call the operator $\acreduce$ in this case since we update $\DPt_H[s]$ with at most two tables from $\DPt_{H'}$.


% \[
% 	\mathcal{A} = \begin{cases}
% 		\DPt_{H'}[s]                                                                                                            &\text{if $\gamma_{\tilde{1}} \in \{s(i), s(j)\}$ and $\gamma_{-2} \notin \{s(i), s(j)\}$,}\\
% 		\DPt_{H'}[ s[i \mapsto \gamma_{2}] ]                                                                                    &\text{if $s(i) = \gamma_{-2}$ and $s(j) = \gamma_{\tilde{1}}$,}\\
% 		\DPt_{H'}[ s[j \mapsto \gamma_{2}] ]                                                                                    &\text{if $s(i) = \gamma_{\tilde{1}}$ and $s(j) = \gamma_{-2}$,}\\
% 		\DPt_{H'}[ s[i \mapsto \gamma_{\tilde{1}}, j \mapsto \gamma_{\tilde{1}}] ] \cup \DPt_{H'}[s]                            &\text{if $s(i) = s(j) = \gamma_1$,}\\
% 		\DPt_{H'}[ s[i \mapsto \gamma_{\tilde{1}}, j \mapsto \gamma_{\tilde{2}}] ] \cup \DPt_{H'}[ s[j \mapsto \gamma_{2}] ]    &\text{if $s(i) = \gamma_1$ and $s(j) = \gamma_{-2}$,}\\
% 		\DPt_{H'}[ s[i \mapsto \gamma_{\tilde{2}}, j \mapsto \gamma_{\tilde{1}}] ] \cup \DPt_{H'}[ s[i \mapsto \gamma_{2}] ]    &\text{if $s(i) = \gamma_{-2}$ and $s(j) = \gamma_1$,}
% 	\end{cases}
% \]




\begin{lemma}\label{lemma:acyclic:cw:join-correctness}
    Let $H = \eta_{i,j}(H')$.
    Assume that for all $s' \colon [\cw] \to \Gamma$,
    $\DPt_{H'}[s']$ ac-represents $\mathcal{A}_{H'}[s']$.
    Then, for all $s \colon [\cw] \to \Gamma$,
    $\DPt_H[s]$ ac-represents $\mathcal{A}_{H}[s]$.
\end{lemma}

\begin{proof}
	First, notice that the labels of the vertices remain the same in $H$ and $H'$,
	that is, $\lab_H(v) = \lab_{H'}(v)$ for all $v \in V(H) = V(H')$.

	If $\gamma_0 \in \{s(i), s(j)\}$, then it holds that
    $(F,E_0,(p,w))$ is a solution in $\mathcal{A}_{H'}[s]$ if and only if
    $(F,E_0,(p,w))$ is a solution in $\mathcal{A}_H[s]$, i.e.,
    $\mathcal{A}_{H'}[s]=\mathcal{A}_H[s]$.

	If either $s(i)=s(j)=\gamma_{-2}$ or $\gamma_0 \notin \{s(i),s(j)\}$ and $\{\gamma_{\tilde{2}},\gamma_{2}\} \cap \{s(i),s(j)\} \neq \varnothing$,
	then the statement follows since we have $\mathcal{A}_H[s]=\varnothing$.
	In the first case, for every set $S \subseteq V(H)$ respecting Condition (1)
	we have a $C_4$ in $\CG(H[S],E_0,s)$ for any $E_0 \subseteq \setdef{\{v v_0\}}{v \in S}$
	due to the at least two $i$-vertices and the at least two $j$-vertices belonging to $S$.
	In the latter case, let $S \subseteq V(H)$ respecting Condition (1).
	Assume without loss of generality that $|S \cap \lab^{-1}_H(i)| \ge 2$
	and $S \cap \lab^{-1}_H(j) \neq \varnothing$.
	Then the graph $\CG(H[S],E_0,s)$ contains a cycle for all subsets $E_0 \subseteq \setdef{\{v v_0\}}{v \in S}$
	due to the $i$- and $j$-vertices of $S$ as well as vertex $v^+_i$.


	Now consider the case where $s(i),s(j) \in \{\gamma_{\tilde{1}}, \gamma_{1}, \gamma_{-2}\}$ with
    either $s(i) \neq \gamma_{-2}$ or $s(j) \neq \gamma_{-2}$.
	Since the used operators preserve ac-representation by \cref{lemma:acyclic:cw:framework:ac_preserve},
    it is enough to prove that $\mathcal{A}_H[s] = \DPt_H[s]$ if it holds that $\DPt_{H'}[s'] = \mathcal{A}_{H'}[s']$ for all signatures $s'$.

	\todo[inline]{Up to here.}

	\begin{claim}
		The statement holds when $\gamma_{\tilde{1}} \in \{s(i), s(j)\}$ and $\gamma_{-2} \notin \{s(i), s(j)\}$.
	\end{claim}

	\begin{claimproof}
		Assume without loss of generality that $s(i) = \gamma_{\tilde{1}}$ and $s(j) \in \{ \gamma_{\tilde{1}}, \gamma_1 \}$.

		Let $(S,M,E_0,(p,w))$ be a solution in $\mathcal{A}_H[s]$.
		We first prove that $(p,w) \in \mathcal{A}$.
		By assumption, we have that $S \cap \lab^{-1}_{H}(j) = \{v_i\}$ with $v_i \notin V_M$,
		and $S \cap \lab^{-1}_{H}(j) = \{v_j\}$.
		Since we only consider irredundant $\cw$-expressions, $\{v_i, v_j\} \notin E(H')$.
		Let $p'$ be the partition on $s^{-1}(\{\gamma_{\tilde{1}}, \gamma_1, \gamma_{\tilde{2}},\gamma_2\}) \cup \{v_0\}$ such that
		$(S,M,E_0,(p',w))$ is a candidate solution in $\mathcal{A}_{H'}[s]$.
		We claim that it is also a solution in $\mathcal{A}_{H'}[s]$.
		Since the label of any vertex is the same in both $H'$ and $H$, Condition (1) is satisfied.
		Condition (2) is also satisfied, as $\CG(H'[S],E_0,s)$ is a subgraph of $\CG(H[S],E_0,s)$,
		and since the latter is acyclic, so is the former.
		Condition (3) is also satisfied because $s(i),s(j) \in \{ \gamma_{\tilde{1}}, \gamma_1 \}$
		and for every connected component $C$ of $\CG(H'[S],E_0,s)$,
		either $C$ is a connected component of $\CG(H[S],E_0,s)$ or
		$C$ contains an $\ell$-vertex with $\ell \in \{i,j\}$.
		Therefore, $(S,M,E_0,(p',w))$ is a solution in $\mathcal{A}_{H'}[s]$.

		It remains to argue that $(p,w)$ is added in $\DPt[s]$.
		We claim that $\acy(p', \{\{i,j\}\}_{\uparrow V })$ holds with $V=s^{-1}(\gamma_{\tilde{1}},\gamma_1,\gamma_{\tilde{2}},\gamma_{2}) \cup \{v_0\}$.
		By definition of $\acy$ it is equivalent to prove that $i$ and $j$ cannot belong to a same block of $p'$.
		Assume towards a contradiction that $i$ and $j$ belong to a same block of $p'$.
		Then, vertices $v_i$ and $v_j$ belong to the same connected component of $\CG(H'[S],E_0,v_0)$.
		% Let us choose this path $P$ to be the smallest one.
		% One	first notices that $P$ cannot contain the vertex $v_i^+$ of $V_{s_H}^+$, if any, because $v_i^+$ is only adjacent to $i$-vertices.
		% Because	$V(\CG(F_H,E_0,s_H))\setminus \{v_i^+\}=V(\CG(F,E_0,s))$, we would conclude that $\CG(F,E_0,s)$ contains a cycle as $x_ix_j\in E(F)\setminus E(F_H)$, contradicting that	$(F,E_0,(p,w))$ is a solution in $\mathcal{A}_H[s]$.
		% Therefore, $\acy(p', \{\{i,j\}\}_{\uparrow V})$ holds.
		% By assumption $s(j)=\gamma_{1}$, thus $s^{-1}(\gamma_{-2})\cap\{i,j\} \subseteq \{i\}$.
		% Since $i$ and $j$ are in the same block of the partition $p\sqcup \{ \{ i,j\}\}_{\uparrow V}$, we conclude that $(p,w)\in \proj(s^{-1}(\gamma_{-2})\cap\{i,j\},\acjoin(\{(p',w)\}, \{(\{\{i,j\}\},0)\}))$.
	\end{claimproof}

	% \medskip

	% It remains to prove that each  weighted partition  $(p,w)\in\mathcal{A}$ belongs to $\mathcal{A}_H[s]$.  Let $(F_H,E^{H}_0,(p',w))$ be a solution in $\mathcal{A}_{H'}[s_H]$ so that \[ (p,w)\in \proj(s^{-1}(\gamma_{-2})\cap\{i,j\},\acjoin(\{(p',w)\}, \{ ( \{\{i,j\}\},0)\})). \]
	% Let  $F=G[V(F_H)]$. By assumption $s(j)=\gamma_{1}$, and thus $s_H(j)=\gamma_{1}$. Let $\{x_j\}:=V(F_H)\cap lab_H^{-1}(j)$, $X_i:=V(F_H)\cap lab_H^{-1}(i)$, and $E_{i,j}:=\{ x_j v \mid v\in X_i\}$. Notice that $E(F)\setminus E(F_H)= E_{i,j}$.
	% We claim that $(F,E^{H}_0, (p,w))$ is a solution in  $\mathcal{A}_H[s]$.
	% \begin{itemize}
	% 	\item  First, Condition (1) is trivially satisfied by the definition of $s_H$.
	% 	\item Secondly, $\CG(F,E^{H}_0,s)$ is a forest. Indeed, $\{i,j\}$ cannot be a block of $p'$, otherwise the $\acjoin$ operator would discard $p'$. If $|X_i|=1$,
	% 	then $\CG(F,E^{H}_0,s)=(V(\CG(F_H,E^{H}_0,s_H)),E(\CG(F_H,E^{H}_0,s_H))\cup E_{i,j})$ and it is clearly a forest.
	% 	Otherwise, if $|X_i|\geq 2$, then $\CG(F,E^{H}_0,s)$ can be obtained from $\CG(F_H,E^{H}_0,s_H)$ by
	% 	fusing the vertex $v_i^+$ and the vertex $x_j$. Clearly, this operation keeps the graph acyclic since $x_j$ and $v_i^+$ are not connected in
	% 	$\CG(F_H,E^{H}_0,s_H)$. Thus $(F,E^{H}_0,(p,w))$ satisfies Condition (2).
	% 	\item Each connected component of $\CG(F_H,E^{H}_0,s_H)$ is contained in a connected component of $\CG(F,E^{H}_0,s)$, and the
	% 	$i$-vertices are in the same connected component, in $\CG(F,E^{H}_0,s)$, as $x_j$. Therefore, Condition (3) is satisfied by $(F,E^{H}_0,(p,w))$
	% 	as $s(j)=\gamma_1$ and $s(\ell)=s_H(\ell)$ for all $\ell \in [k]\setminus \{i,j\}$.

	% 	\item Also, Condition (4) is satisfied as $p$ is then obtained from $p'$ by merging the blocks of $p'$ which contains $i$ and $j$, and by removing $i$ if
	% 	$s(i)=\gamma_{-2}$.
	% \end{itemize}
	% We can therefore conclude that $(F,E^{H}_0,(p,w))$ is a solution in $\mathcal{A}_H[s]$.

	\begin{claim}
		The statement holds when $ \{ s(i),s(j) \} = \{ \gamma_{\tilde{1}}, \gamma_{-2} \}$.
	\end{claim}

	\begin{claimproof}
		\manolis{TODO!}
	\end{claimproof}

	\begin{claim}
		The statement holds when $s(i) = s(j) = \gamma_{1}$.
	\end{claim}

	\begin{claimproof}
		\manolis{TODO!}
	\end{claimproof}

	\begin{claim}
		The statement holds when $ \{ s(i),s(j) \} = \{ \gamma_{1}, \gamma_{-2} \}$.
	\end{claim}

	\begin{claimproof}
		\manolis{TODO!}
	\end{claimproof}
	%
	This concludes the proof.
\end{proof}


