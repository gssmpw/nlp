\subsection{Algorithm}\label{subsec:induced:cw}

In this section we show the following theorem.

\begin{theorem}\label{thm:induced:cw}
    There is an algorithm that,
    given a graph $G$ and an irredundant clique-width expression $\psi$ of $G$ of width $\cw$,
    counts the number of induced matchings of $G$ of each size $\ell \in [0,n/2]$ in time $\sO(3^\cw)$.
\end{theorem}

\begin{proof}
    Before describing our algorithm we start with some definitions and notation.
    Given a tuple $\sigma$, we refer to its $w$-th entry by $\sigma(w)$.
    Let $H$ be a graph and $S \subseteq V(H)$ a subset of its vertices.
    We say that a vertex $v \in S$ is \emph{unsaturated} in $H[S]$ if its degree is $\deg_{H[S]} (v) = 0$,
    else if $\deg_{H[S]} (v) = 1$ it is \emph{saturated} in $H[S]$ instead.
    We say that $S$ is a \emph{partial matching} of $H$ if the maximum degree of $H[S]$ is at most $1$,
    in which case all of its vertices are either saturated or unsaturated in $H[S]$.
    The \emph{size of a partial matching} $S$ of $H$ is equal to the number of its vertices, that is, $|S|$.
    Finally, we say that a partial matching $S$ of $H$ is \emph{extensible} if there exists an induced matching $M$ of $G$
    such that $V_M \cap V(H) = S$.


    Let $\mathcal{H}$ denote the set of all $\cw$-labeled graphs generated by a subexpression of $\psi$.
    For $H \in \mathcal{H}$ and $i \in [\cw]$,
    we define $\sgn_{H,i} \colon V(H) \to \{0,1,2\}$ for $S \subseteq V(H)$ as follows:
    \[
        \sgn_{H,i}(S) =
            \begin{cases}
                0   &\text{if $S \cap \lab^{-1}_H (i) = \varnothing$,}\\
                1   &\text{if $S \cap \lab^{-1}_H (i) \neq \varnothing$ and $\deg_{H[S]}(v) = 1$ for all $v \in S \cap \lab^{-1}_H (i)$,}\\
                2   &\text{if $S \cap \lab^{-1}_H (i) = \{v\}$ and $\deg_{H[S]}(v) = 0$.}
            \end{cases}
    \]
    We further define the tuple $\sgn_H(S) = \left( \sgn_{H,1}(S), \ldots, \sgn_{H,\cw}(S) \right)$ to be
    the \emph{signature} of $S$ in $H$ if $\sgn_{H,i}(S)$ is well-defined for all $i \in [\cw]$.
    Notice that $\sgn_H(S)$ is only defined for a subset of the partial matchings of $H$,
    while there are at most $3^{\cw}$ different signatures.
    For improved readability, we sometimes refer to tuples $\sigma \in \{0,1,2\}^{\cw}$ as signatures.
    We first argue that for any induced matching $M$ of $G$,
    the signature of $V_M \cap V(H)$ is well-defined for all $H \in \mathcal{H}$.

    \begin{claim}\label{claim:induced:cw:well_defined}
        Let $M$ be an induced matching of $G$.
        For all $H \in \mathcal{H}$ it holds that $\sgn_H (V_M \cap V(H))$ is well-defined.
    \end{claim}

    \begin{claimproof} %
        Notice that for all $H \in \mathcal{H}$ it holds that $H[V_M \cap V(H)]$ is a subgraph of $G[V_M]$,
        therefore the maximum degree of $H[V_M \cap V(H)]$ is at most $1$.
        Consequently, $V_M \cap V(H)$ is a partial matching of $H$.
        Fix $H \in \mathcal{H}$ and let $S = V_M \cap V(H)$.
        It suffices to prove that $\sgn_{H,i}(S)$ is well-defined for all $i \in [\cw]$.

        For the sake of contradiction, assume that $\sgn_{H,i}(S)$ is not well-defined for some $i \in [\cw]$.
        Since $S$ is a partial matching of $H$, that can only be the case when $S \cap \lab^{-1}_{H}(i)$ contains at least two vertices,
        say $v,v'$, and at least one of them, say $v$, is unsaturated in $H[S]$,
        that is, $\deg_{H[S]}(v) = 0$.
        Let $u \in V_M$ such that $\{v,u\} \in M$.
        Since $\deg_{H[S]}(v) = 0$ it holds that either $u \notin V(H)$ or $\{v,u\} \notin E(H)$.
        This means that at some point further in the clique-expression there exists a join operation with
        $H_1,H_2 \in \mathcal{H}$ and $H_2 = \eta_{i',j'}(H_1)$ that adds all edges between vertices of labels $i'$ and $j'$,
        with $v,v'$ having label $i'$ and $u$ having label $j'$.
        In that case however, it holds that the degree of $u \in V_M \cap V(H_2)$ in $H_2[V_M \cap V(H_2)]$
        is at least $2$ as it is adjacent to both $v$ and $v'$, a contradiction.
    \end{claimproof}

    Our algorithm proceeds by dynamic programming along $\psi$ in a bottom-up fashion.
    In particular, for every $H \in \mathcal{H}$ it stores a table $\DPt_H[\cdot,\cdot]$ where,
    for $k \in [0,n]$ and $\sigma \in \{0,1,2\}^{\cw}$, $\DPt_H[k,\sigma]$ contains
    the number of partial matchings $S$ of $H$ of size $k$ and signature $\sigma$,
    that is, with $|S|=k$ and $\sgn_H(S) = \sigma$.
    We now proceed to describe how to populate the DP tables.

    \proofsubparagraph{Singleton $H = i(v)$.}
    For $H = i(v)$, notice that $\lab^{-1}_H(i) = \{v\}$ and $\lab^{-1}_H(w) = \varnothing$ for all $w \in [\cw] \setminus \{i\}$.
    Consequently, for $k \in [0,n]$ and $\sigma \in \{0,1,2\}^{\cw}$ we set
    \begin{equation}\label{eq:DPH-singleton}
        \DPt_H[k,\sigma] =
            \begin{cases}
                1   &\text{if $k=0$ and $\sigma(w) = 0$ for all $w \in [\cw]$,}\\
                1   &\text{if $k=1$ and $\sigma(w) = 0$ for all $w \in [\cw] \setminus \{i\}$ and $\sigma(i) = 2$,}\\
                0   &\text{otherwise}.
            \end{cases}
    \end{equation}
    Notice that all but two entries in $\DPt_H[\cdot,\cdot]$ are $0$,
    and by \cref{eq:DPH-singleton} we can fill the table $\DPt_H[\cdot,\cdot]$ in time $\sO(3^{\cw})$.


    \proofsubparagraph{Joining labels with edges, $H = \eta_{i,j}(H')$.}
For each pair of pairwise different $i,j \in [\cw]$,
we define a function $f_{i,j} \colon \{0,1,2\}^{\cw} \to \{0,1,2\}^{\cw}$.
In particular, for $\sigma' \in \{0,1,2\}^{\cw}$ we set $f_{i,j}(\sigma') = \sigma$
such that $\sigma(w) = \sigma'(w)$ for all $w \in [\cw] \setminus \{i,j\}$
and
\[
    (\sigma(i),\sigma(j)) =
        \begin{cases}
            (\sigma'(i), \, \sigma'(j)) &\text{if $0 \in \{\sigma'(i), \, \sigma'(j)\}$,}\\
            (1,1)                       &\text{if $\sigma'(i) = \sigma'(j) = 2$}.
        \end{cases}
\]
For a fixed size $k \in [0,n]$ and signature $\sigma \in \{0,1,2\}^{\cw}$ we set the value of $\DPt_H[k,\sigma]$ to be
\begin{equation}\label{eq:DPH-join}
    \DPt_H[k,\sigma] = \sum_{\sigma' \in f^{-1}_{i,j} (\sigma)} \DPt_{H'}[k,\sigma'].
\end{equation}
If $f^{-1}_{i,j}(\sigma) = \varnothing$ we set $\DPt_H[k,\sigma] = 0$.
Notice that computing $f^{-1}_{i,j}(\sigma)$ requires $\bO(\cw)$ time,
thus by \cref{eq:DPH-join} we can fill the table $\DPt_H[\cdot,\cdot]$ in time $\sO(3^{\cw})$.

\begin{lemmarep}[\appsymb]\label{lemma:induced:cw:join-correctness}
    Let $H = \eta_{i,j}(H')$.
    Assume that for all $k \in [0,n]$ and $\sigma \in \{0,1,2\}^{\cw}$,
    $\DPt_{H'}[k,\sigma]$ is equal to the number of partial matchings $S$ of $H'$ such that $|S|=k$ and $\sgn_{H'}(S) = \sigma$.
    Then, $\DPt_H[k,\sigma]$ is equal to the number of partial matchings $S$ of $H$ such that $|S|=k$ and $\sgn_H(S) = \sigma$.
\end{lemmarep}

\begin{proof}
    We show that the function $f_{i,j}$ describes precisely the effect of the operation $\eta_{i,j}$
    on the signature of a partial matching $S$.

    \begin{claim}\label{claim:induced:cw:join-correctness}
        Let $S \subseteq V(H')$ such that $\sgn_{H'}(S) = \sigma'$.
        Then, $\sgn_{H}(S) = \sigma$ if and only if $f_{i,j}(\sigma') = \sigma$.
    \end{claim}

    \begin{claimproof}
        First, notice that the labels of the vertices remain the same between $H'$ and $H$,
        therefore it holds that $\lab_H(v) = \lab_{H'}(v)$ for all $v \in V(H')$.
        This implies that $S \cap \lab^{-1}_H (w) = S \cap \lab^{-1}_{H'} (w)$ for all $w \in [\cw]$.
        Since $H[S]$ is obtained from $H'[S]$ by adding all edges between vertices of label~$i$ and~$j$,
        it follows that $\deg_{H[S]}(v) = \deg_{H'[S]}(v)$ for all $v \in S \setminus (\lab^{-1}_H(i) \cup \lab^{-1}_H(j))$,
        thus $\sgn_{H,w}(S) = \sgn_{H',w}(S)$ for all $w \in [\cw] \setminus \{i,j\}$.
        In that case, $\sgn_{H}(S)$ and $f_{i,j}(\sigma')$ agree on all labels $w \in [\cw] \setminus \{i,j\}$.
        It remains to argue for labels~$i$ and~$j$.

        Assume first that $0 \in \{\sigma'(i), \, \sigma'(j)\}$.
        It is easy to see that in that case the graphs $H[S]$ and $H'[S]$ are the same,
        consequently $\sgn_{H}(S) = \sgn_{H'}(S)$ follows, and by the definition of function $f_{i,j}$
        we have that $\sgn_{H}(S) = f_{i,j}(\sigma')$.

        Next, assume that $\sigma'(i) = \sigma'(j) = 2$.
        Then, it holds that both $S \cap \lab^{-1}_{H}(i)$ and $S \cap \lab^{-1}_{H}(i)$ are comprised of a single vertex
        that is of degree $0$ in $H'[S]$, which implies that $\sgn_{H,i}(S) = \sgn_{H,j}(S) = 1$
        since the same vertices have degree~$1$ in $H[S]$.
        Again we have that $\sgn_{H}(S) = f_{i,j}(\sigma')$.

        Lastly, assume that $0 \notin \{\sigma'(i), \, \sigma'(j)\}$ and either $\sigma'(i) \neq 2$ or $\sigma'(j) \neq 2$.
        Assume without loss of generality that $\sigma'(i) = 1$ and $\sigma'(j) \in \{1,2\}$,
        and let $v \in S \cap \lab^{-1}_{H'}(i)$ and $u \in S \cap \lab^{-1}_{H'}(j)$,
        where $\deg_{H'[S]}(v) = 1$.
        Since $\psi$ is by definition an irredundant clique-width expression,
        it follows that $\{v,u\} \notin E(H')$.
        Consequently, the operation $\eta_{i,j}$ adds the edge $\{v,u\}$ in $H$, thus $\deg_{H[S]}(v) > \deg_{H'[S]}(v) = 1$.
        In that case $\sgn_{H,i}(S)$ is undefined, as is the case for $f_{i,j}(\sigma')$.
    \end{claimproof}

    Consequently, the number of partial matchings of $H$ of size $k$ and signature $\sigma$
    is exactly equal to the sum of the number of partial matchings of $H'$ of size $k$ and signature $\sigma'$
    over all $\sigma' \in f^{-1}_{i,j}(\sigma)$.
    The statement now follows by the hypothesis for $\DPt_{H'}[\cdot,\cdot]$ and \cref{eq:DPH-join}.
\end{proof}


    \proofsubparagraph{Relabeling, $H = \rho_{i \to j}(H')$.}
For each pair of pairwise different $i,j \in [\cw]$,
we define a function $g_{i \to j} \colon \{0,1,2\}^{\cw} \to \{0,1,2\}^{\cw}$.
In particular, for $\sigma' \in \{0,1,2\}^{\cw}$ we set $g_{i \to j}(\sigma') = \sigma$
such that $\sigma(w) = \sigma'(w)$ for all $w \in [\cw] \setminus \{i,j\}$,
$\sigma(i) = 0$, and
\[
    \sigma(j) =
        \begin{cases}
            \max \{ \sigma'(i), \, \sigma'(j) \}    &\text{if $0 \in \{\sigma'(i), \, \sigma'(j)\}$,}\\
            1                                       &\text{if $\sigma'(i) = \sigma'(j) = 1$}.
        \end{cases}
\]
For a fixed size $k \in [0,n]$ and signature $\sigma \in \{0,1,2\}^{\cw}$ we set the value of $\DPt_H[k,\sigma]$ to be
\begin{equation}\label{eq:DPH-relabel}
    \DPt_H[k,\sigma] = \sum_{\sigma' \in g^{-1}_{i \to j} (\sigma)} \DPt_{H'}[k,\sigma'].
\end{equation}
If $g^{-1}_{i \to j}(\sigma) = \varnothing$ we set $\DPt_H[k,\sigma] = 0$.
Notice that computing $g^{-1}_{i \to j}(\sigma)$ requires $\bO(\cw)$ time,
thus by \cref{eq:DPH-relabel} we can fill the table $\DPt_H[\cdot,\cdot]$ in time $\sO(3^{\cw})$.

\begin{lemmarep}[\appsymb]\label{lemma:induced:cw:relabel-correctness}
    Let $H = \rho_{i \to j}(H')$.
    Assume that for all $k \in [0,n]$ and $\sigma \in \{0,1,2\}^{\cw}$,
    $\DPt_{H'}[k,\sigma]$ is equal to the number of partial matchings $S$ of $H'$ such that $|S|=k$ and $\sgn_{H'}(S) = \sigma$.
    Then, $\DPt_H[k,\sigma]$ is equal to the number of partial matchings $S$ of $H$ such that $|S|=k$ and $\sgn_H(S) = \sigma$.
\end{lemmarep}

\begin{proof}

    We show that the function $g_{i \to j}$ describes precisely the effect of the operation $\rho_{i \to j}$
    on the signature of a partial matching $S$.

    \begin{claim}\label{claim:induced:cw:relabel-correctness}
        Let $S \subseteq V(H')$ such that $\sgn_{H'}(S) = \sigma'$.
        Then, $\sgn_{H}(S) = \sigma$ if and only if $g_{i \to j}(\sigma') = \sigma$.
    \end{claim}


    \begin{claimproof}
        First, notice that the labels of all vertices with label different than~$i$ remain the same between $H'$ and $H$,
        therefore it holds that $\lab_H(v) = \lab_{H'}(v)$ for all $v \in V(H') \setminus \lab^{-1}_{H'}(i)$.
        This implies that $S \cap \lab^{-1}_H (w) = S \cap \lab^{-1}_{H'} (w)$ for all $w \in [\cw] \setminus \{i,j\}$.
        Furthermore, notice that no new edge is added in $H$, consequently it holds that $\deg_{H[S]}(v) = \deg_{H'[S]}(v)$ for all $v \in V(H)$.
        This implies that $\sgn_{H,w}(S) = \sgn_{H',w}(S)$ for all $w \in [\cw] \setminus \{i,j\}$.
        Moreover, since $H$ is obtained from $H'$ by relabeling all vertices of label~$i$ to label~$j$,
        it holds that $\lab^{-1}_H(j) = \lab^{-1}_{H'}(i) \cup \lab^{-1}_{H'}(j)$ and $\lab^{-1}_H(i) = \varnothing$,
        thus $\sgn_{H,i}(S) = 0$.
        In that case, $\sgn_{H}(S)$ and $f_{i,j}(\sigma')$ agree on all labels $w \in [\cw] \setminus \{j\}$.
        It remains to argue for label~$j$.

        Assume first that $0 \in \{\sigma'(i), \, \sigma'(j)\}$.
        In that case, it holds that either $S \cap \lab^{-1}_{H'}(i) = \varnothing$ or $S \cap \lab^{-1}_{H'}(j) = \varnothing$.
        Assume that $S \cap \lab^{-1}_{H'}(i) = \varnothing$, that is, $\sgn_{H',i}(S) = 0$,
        which implies that $S \cap \lab^{-1}_{H}(j) = S \cap \lab^{-1}_{H'}(j)$,
        thus $\sgn_{H,j}(S) = \sgn_{H',j}(S) = \max \{\sgn_{H',j}(S), \, \sgn_{H',i}(S)\}$.
        In a similar way, when $S \cap \lab^{-1}_{H'}(i) = \varnothing$ we get that $\sgn_{H',j}(S) = 0$
        and $\sgn_{H,j}(S) = \sgn_{H',i}(S) = \max \{\sgn_{H',i}(S), \, \sgn_{H',j}(S)\}$.
        Thus, in this case we have that $\sgn_{H}(S) = f_{i,j}(\sigma')$.

        Next, assume that $\sigma'(i) = \sigma'(j) = 1$.
        Then, it holds that for all $v_1 \in S \cap \lab^{-1}_{H'}(i)$ and $v_2 \in S \cap \lab^{-1}_{H'}(j)$
        we have that $\deg_{H'[S]} (v_1) = \deg_{H'[S]} (v_2) = 1$.
        Since $\lab^{-1}_H(j) = \lab^{-1}_{H'}(i) \cup \lab^{-1}_{H'}(j)$ and
        for all $v \in V(H)$ it holds that $\deg_{H[S]}(v) = \deg_{H'[S]}(v)$,
        it follows that $\sgn_{H,j}(S) = 1$.
        Again we have that $\sgn_{H}(S) = f_{i,j}(\sigma')$.

        Lastly, assume that $0 \notin \{\sigma'(i), \, \sigma'(j)\}$ and either $\sigma'(i) \neq 1$ or $\sigma'(j) \neq 1$.
        Assume without loss of generality that $\sigma'(i) = 2$ and $\sigma'(j) \in \{1,2\}$.
        In that case, it holds that $S \cap \lab^{-1}_{H'}(i) = \{v\}$ with $\deg_{H'[S]}(v) = 0$
        as well as $S \cap \lab^{-1}_{H'}(i) \neq \varnothing$.
        Consequently it holds that $S \cap \lab^{-1}_H(j)$ contains at least $2$ vertices $v_1,v_2$, with $\deg_{H[S]}(v_1) = 0$,
        thus $\sgn_{H,j}(S)$ is undefined, as is the case for $f_{i,j}(\sigma')$.
    \end{claimproof}

    Consequently, the number of partial matchings of $H$ of size $k$ and signature $\sigma$
    is exactly equal to the sum of the number of partial matchings of $H'$ of size $k$ and signature $\sigma'$
    over all $\sigma' \in g^{-1}_{i \to j}(\sigma)$.
    The statement now follows by the hypothesis for $\DPt_{H'}[\cdot,\cdot]$ and \cref{eq:DPH-relabel}.
\end{proof}



    \proofsubparagraph{Disjoint union, $H = H_1 \oplus H_2$.}
For the Union nodes, an approach similar to the previous cases would yield a total running time of $\sO(9^{\cw})$
to populate the table $\DPt_H[\cdot,\cdot]$, which can be further improved to $\sO(6^{\cw})$ by some slightly more
careful analysis.
In order to bring this down to $\sO(3^{\cw})$ we proceed in a different way
by using the techniques introduced in~\cite{mfcs/BodlaenderLRV10,tcs/CyganP10,esa/RooijBR09}.

We start with some definitions.
Let $\sigmaapx \in \{0,\tilde{1},2\}^{\cw}$ be a tuple with $\cw$ entries,
each taking values over the set $\{0,\tilde{1},2\}$.
We call any such tuple a \emph{pseudosignature} and we say that the signature $\sigma \in \{0,1,2\}^{\cw}$
is \emph{compatible} with the pseudosignature $\sigmaapx$ if for all $w \in [\cw]$ it holds that
(i) if $\sigmaapx(w) = 0$ then $\sigma(w) = 0$,
(ii) if $\sigmaapx(w) = \tilde{1}$ then $\sigma(w) \in \{0,1\}$, and
(iii) if $\sigmaapx(w) = 2$ then $\sigma(w) = 2$.
Let for $H' \in \mathcal{H}$, $\DPapx_{H'}[\cdot,\cdot]$ be the table
which in entry $\DPapx_{H'}[k,\sigmaapx]$ contains the number of partial matchings
in $H'$ of size $k$ whose signatures are compatible with $\sigmaapx$.
Our main idea is to first compute the tables $\DPapx_{H_1}[\cdot,\cdot]$ and $\DPapx_{H_2}[\cdot,\cdot]$,
then from those compute the table $\DPapx_H[\cdot,\cdot]$,
and lastly use the latter to populate the table $\DPt_H[\cdot,\cdot]$.


\proofsubparagraph{Step 1.}
Let $H_i \in \{H_1,H_2\}$.
For $\lambda \in [0,\cw]$ we say that $\sigmamixed \in \{0,\tilde{1},2\}^{\lambda} \times \{0,1,2\}^{\cw-\lambda}$ is
a \emph{$\lambda$-mixed signature}.
We say that the signature $\sigma \in \{0,1,2\}^{\cw}$ is \emph{compatible} with the $\lambda$-mixed signature $\sigmamixed$
if the following hold:
\begin{itemize}
    \item For all $w \in [\lambda]$ it holds that
        (i) if $\sigmamixed(w) = 0$ then $\sigma(w) = 0$,
        (ii) if $\sigmamixed(w) = \tilde{1}$ then $\sigma(w) \in \{0,1\}$, and
        (iii) if $\sigmamixed(w) = 2$ then $\sigma(w) = 2$.

    \item For all $w \in [\lambda+1,\cw]$, $\sigma(w) = \sigmamixed(w)$.
\end{itemize}
We proceed inductively.
For $\lambda \in [0,\cw]$ let $\DPapx_{H_i}^{\lambda}[\cdot,\cdot]$ denote a table taking values over
$[0,n] \times \{0,\tilde{1},2\}^{\lambda} \times \{0,1,2\}^{\cw-\lambda}$,
where intuitively $\DPapx_{H_i}^{\lambda}[k,\sigmamixed]$ is equal to the number of partial matchings of ${H_i}$ of size $k$
whose signature is compatible with $\sigmamixed$.
For $\lambda=0$, this is precisely the table $\DPt_{H_i}[\cdot,\cdot]$,
while for $\lambda=\cw$ this will be the table $\DPapx_{H_i}[\cdot,\cdot]$.

It suffices to explain how to compute $\DPapx_{H_i}^{\lambda+1}[\cdot,\cdot]$ from the table
$\DPapx_{H_i}^{\lambda}[\cdot,\cdot]$ in time $\sO(3^{\cw})$, and repeat the whole procedure $\cw$ times.
Let $k \in [0,n]$ and $\sigmamixed \in \{0,\tilde{1},2\}^{\lambda+1} \times \{0,1,2\}^{\cw-(\lambda+1)}$
be a $(\lambda+1)$-mixed signature.
Consider two cases.
If $\sigmamixed(\lambda+1) \in \{0,2\}$,
then set $\DPapx_{H_i}^{\lambda+1}[k,\sigmamixed] = \DPapx_{H_i}^{\lambda}[k,\sigmamixed]$.
Otherwise, it holds that $\sigmamixed(\lambda+1) = \tilde{1}$,
and let $\sigmamixed', \sigmamixed'' \in \{0,\tilde{1},2\}^{\lambda} \times \{0,1,2\}^{\cw-\lambda}$
denote the $\lambda$-mixed signatures obtained from $\sigmamixed$ by substituting the $(\lambda+1)$-th
entry with $1$ and $0$ respectively.
In that case, set
$\DPapx_{H_i}^{\lambda+1}[k,\sigmamixed] = \DPapx_{H_i}^{\lambda}[k,\sigmamixed'] + \DPapx_{H_i}^{\lambda}[k,\sigmamixed'']$.


\proofsubparagraph{Step 2.}
Utilizing the tables $\DPapx_{H_1}[\cdot,\cdot]$ and $\DPapx_{H_2}[\cdot,\cdot]$ constructed in Step 1,
along with making use of the FFT technique introduced by Cygan and Pilipczuk~\cite{tcs/CyganP10},
we compute in time $\sO(3^{\cw})$ the table $\DPapx_H[\cdot,\cdot]$,
also indexed over $[0,n] \times \{0,\tilde{1},2\}^{\cw}$.
Initially we set every entry of said table to $0$.

In the following, fix $k \in [0,n]$ and a subset $A \subseteq [\cw]$.
Let $\mathcal{S}_A \subseteq \{0,\tilde{1},2\}^{\cw}$ denote the set of pseudosignatures
such that $\sigmaapx \in \mathcal{S}_A$ iff for all $w \in [\cw]$, $\sigmaapx(w) = \tilde{1} \iff w \in \mathcal{S}_A$.
We show how to compute all entries $\DPapx_H[k,\sigmaapx]$ with $\sigmaapx \in \mathcal{S}_A$
in time $\sO(2^{\cw - |A|})$.
Since $\sum_{A \subseteq [\cw]} 2^{\cw - |A|} = 3^{\cw}$ and $k$ takes $n+1$ different
values, this suffices for the stated time bound.
Let $w_1, \ldots, w_{\cw-|A|}$ denote in increasing order the non $\tilde{1}$-valued coordinates of a pseudosignature in $\mathcal{S}_A$,
that is, for all $j \in [\cw-|A|]$ and $\sigmaapx \in \mathcal{S}_A$,
we have that $\sigmaapx(w_i) \neq \tilde{1}$.

We first fix $k_1,k_2 \in [0,k]$ such that $k_1+k_2=k$.
Notice that there are $\bO(n)$ such pairs of $k_1,k_2$.
Intuitively, $k_i$ represents the number of vertices of the partial matching of $H_i$ that account towards the
total size of the partial matching in $H$.
Next we define a \emph{binary encoding} $\bin_A \colon \mathcal{S}_A \to \{0,1\}^{\cw-|A|}$
of every pseudosignature in $\mathcal{S}_A$ such that for all $\sigmaapx \in \mathcal{S}_A$
the $j$-th bit of $\bin_A(\sigmaapx)$, denoted by $\bin_{A,j}(\sigmaapx)$,
is equal to
\[
    \bin_{A,j}(\sigmaapx) =
        \begin{cases}
            0   &\text{if $\sigmaapx(w_j) = 0$,}\\
            1   &\text{if $\sigmaapx(w_j) = 2$.}
        \end{cases}
\]
For every entry $\DPapx_{H_i}[k_i,\sigmaapx]$ with $\sigmaapx \in \mathcal{S}_A$ we introduce a monomial
$\DPapx_{H_i}[k_i,\sigmaapx] \cdot x^{\bin_A(\sigmaapx)}$,
and we set $P_{A,H_i,k_i}$ to denote the polynomial comprised of all such monomials,
that is,
\[
    P_{A,H_i,k_i} = \sum_{\sigmaapx \in \mathcal{S}_A} \DPapx_{H_i}[k_i,\sigmaapx] \cdot x^{\bin_A(\sigmaapx)}.
\]
Notice that the degree of $P_{A,H_i,k_i}$ is at most $2^{\cw-|A|}$.
Let $P_{A,H_i,k_i}^{r_i}$ denote the restriction of $P_{A,H_i,k_i}$ to monomials whose degree has
\emph{exactly} $r_i$ out of the $\cw-|A|$ bits being $1$.
Fix $r_1,r_2 \in [0,\cw-|A|]$ and multiply polynomials $P_{A,H_1,k_1}^{r_1}$ and $P_{A,H_2,k_2}^{r_2}$ in
$\sO(2^{\cw - |A|})$ time using FFT.%
\footnote{Recall that using FFT we can multiply two polynomials of degree $n$ in
$\bO(n \log n)$ time~\cite{issac/Moenck76}.}
We iterate over all $\sigmaapx \in \mathcal{S}_A$ having exactly $r_1+r_2$ bits of
$\bin_A(\sigmaapx)$ being $1$ and check what coefficient, say $s$, stands in front
of $x^{\bin_A(\sigmaapx)}$ in the resulting polynomial.
We update the value of $\DPapx_H[k,\sigmaapx]$ by adding to its previous value $+s$.
We repeat over all $r_1,r_2 \in [0,\cw-|A|]$, $k_1,k_2 \in [0,k]$, and $k \in [0,n]$,
needing in total $\sO(2^{\cw-|A|})$ time.
Finally, we fill the whole table $\DPapx_H[\cdot,\cdot]$ in time $\sO(3^{\cw})$ by repeating
over all $A \subseteq [\cw]$.


\proofsubparagraph{Step 3.}
Given $\DPapx_H[\cdot,\cdot]$, we populate the values of $\DPt_H[\cdot,\cdot]$
by repeating a procedure analogous to the one described in Step 1.
The total running time is once again $\sO(3^{\cw})$.


\begin{lemmarep}[\appsymb]\label{lemma:induced:cw:union-correctness}
    Let $H = H_1 \oplus H_2$.
    Assume that for all $i \in \{1,2\}$, $k \in [0,n]$, and $\sigma \in \{0,1,2\}^{\cw}$,
    $\DPt_{H_i}[k,\sigma]$ is equal to the number of partial matchings $S$ of $H_i$ such that $|S|=k$ and $\sgn_{H_i}(S)=\sigma$,
    Then, $\DPt_H[k,\sigma]$ is equal to the number of partial matchings $S$ of $H$ such that $|S|=k$ and $\sgn_H(S) = \sigma$.
\end{lemmarep}

\begin{proof}
    It is easy to see that, for $H_i \in \{H_1,H_2\}$,
    for all $k \in [0,n]$ and $\sigmaapx \in \{0,\tilde{1},2\}^{\cw}$,
    $\DPapx_{H_i}[k,\sigmaapx]$ is equal to the number of partial matchings $S$ of $H_i$ such that $|S|=k$ and
    $\sgn_{H_i}(S)$ is compatible with $\sigmaapx$.

    We proceed to show the correctness of Step 2 of the construction.
    Let $S$ be a partial matching of $H$ with $\sigma = \sgn_H(S)$.
    Moreover, let for $i \in \{1,2\}$,
    $S_i = S \cap V(H_i)$ with $\sigma_i = \sgn_{H_i}(S_i)$.

    \begin{claim}\label{claim:induced:cw:union-correctness}
        For all $w \in [\cw]$ it holds that
        \begin{itemize}
            \item $\sigma(w)=0 \iff \sigma_1(w)=\sigma_2(w)=0$,
            \item $\sigma(w)=1 \iff (\sigma_1(w),\sigma_2(w)) \in \{(0,1),(1,0),(1,1)\}$,
            \item $\sigma(w)=2 \iff (\sigma_1(w),\sigma_2(w)) \in \{(0,2),(2,0)\}$.
        \end{itemize}
    \end{claim}

    \begin{claimproof}
        Let $i \in \{1,2\}$.
        Notice that for all $v \in S_i$ it holds that
        $\lab_{H_i}(v) = \lab_H(v)$,
        thus $S \cap \lab^{-1}_H(w) =
        (S_1 \cap \lab^{-1}_{H_1}(w)) \cup (S_2 \cap \lab^{-1}_{H_2}(w))$.
        Furthermore, it holds that $\deg_{H_i[S_i]}(v) = \deg_{H[S]}(v)$.

        For the first item, due to the previous paragraph we have that
        if $\sigma(w) = 0$, that is, $S \cap \lab^{-1}_H(w) = \varnothing$,
        then $S_1 \cap \lab^{-1}_{H_1}(w) = S_2 \cap \lab^{-1}_{H_2}(w) = \varnothing$,
        that is, $\sigma_1(w) = \sigma_2(w) = 0$.
        It can be easily shown that the converse is also true.

        As for the second item, assume that $\sigma(w) = 1$, that is, $S \cap \lab^{-1}_H(w) \neq \varnothing$
        and for all of its vertices $v$, it holds that $\deg_{H[S]}(v)=1$.
        In that case, it follows that for all $v \in (S_1 \cap \lab^{-1}_{H_1}(w)) \cup (S_2 \cap \lab^{-1}_{H_2}(w))$,
        $\deg_{H_i[S_i]}(v)=1$, with at most one $S_i \cap \lab^{-1}_{H_i}(w)$ being empty.
        Conversely, assuming that $(\sigma_1(w),\sigma_2(w)) \in \{(0,1),(1,0),(1,1)\}$, it follows that
        for every vertex $v$ of the non-empty set $S \cap \lab^{-1}_H(w)$ it holds that $\deg_{H[S]}(v) = 1$.

        Lastly, assume that $\sigmaapx(w)=2$,
        which implies that $S \cap \lab^{-1}(w) = \{v\}$ and $\deg_{H[S]}(v)=0$.
        Assume that $v \in S_{i_1}$ and let $S_{i_2}$ denote the other set among $\{S_1,S_2\}$.
        Then, it follows that $S_{i_1} \cap \lab^{-1}_{H_{i_1}}(w) = \{v\}$ with $\deg_{H_{i_1}[S_{i_1}]}(v)=0$
        and $S_{i_2} \cap \lab^{-1}_{H_{i_2}}(w) = \varnothing$,
        implying that $\sigma_{i_1}(w) = 2$ and $\sigma_{i_2}(w) = 0$.
        It can be easily shown that the converse is also true.
    \end{claimproof}

    Recall that $\sigma \in \{0,1,2\}^{\cw}$ is compatible with $\sigmaapx \in \{0,\tilde{1},2\}^{\cw}$
    if for all $w \in [\cw]$ we have that
    (i) if $\sigmaapx(w) = 0$, then $\sigma(w) = 0$,
    (ii) if $\sigmaapx(w) = \tilde{1}$, then $\sigma(w) \in \{0,1\}$, and
    (iii) if $\sigmaapx(w) = 2$, then $\sigma(w) = 2$.
    Fix $k \in [0,n]$ and $\sigmaapx \in \{0,\tilde{1},2\}^{\cw}$, with $\sigmaapx \in \mathcal{S}_A$
    for some $A \subseteq [\cw]$.
    \cref{claim:induced:cw:union-correctness} implies that the number of partial matchings of $H$ of size $k$
    and signature compatible with $\sigmaapx$ is equal to
    the number of partial matchings of $H_1$ of size $k_1$ and signature compatible with $\sigmaapx_1$
    times the number of partial matchings of $H_2$ of size $k_2$ and signature compatible with $\sigmaapx_2$,
    summing over all $k_1+k_2=k$ and $\sigmaapx_1,\sigmaapx_2 \in \mathcal{S}_A$ such that for all $w \in [\cw]$
    (i) if $\sigmaapx(w) = 0$, then $\sigmaapx_1(w) = \sigmaapx_2(w) = 0$,
    and (ii) if $\sigmaapx(w) = 2$, then $(\sigmaapx_1(w),\sigmaapx_2(w)) \in \{(0,2),(2,0)\}$.
    Notice that the latter condition imposed on the pseudosignatures $\sigmaapx_1,\sigmaapx_2$ is equivalent to
    demanding that the number $r_1$ of $1$-bits of $\bin_A(\sigmaapx_1)$ plus the number $r_2$ of $1$-bits of $\bin_A(\sigmaapx_2)$
    is equal to the number of $1$-bits of $\bin_A(\sigmaapx)$.

    In that case, for fixed $r_1,r_2 \in [0,\cw-|A|]$, it holds that the coefficient of a
    monomial $x^{\bin_A(\sigmaapx)}$ of $P_{A,H_1,k_1}^{r_1} \cdot P_{A,H_2,k_2}^{r_2}$,
    where $\bin_A(\sigmaapx)$ has exactly $r_1+r_2$ bits set to $1$,
    is equal to the number of partial matchings of $H$ of size $k = k_1+k_2$ and signature compatible
    with $\sigmaapx \in \mathcal{S}_A$
    that occur due to the union of partial matchings of $H_1$ of size $k_1$ and signature compatible with $\sigmaapx_1$
    and partial matchings of $H_2$ of size $k_2$ and signature compatible with $\sigmaapx_2$,
    where $\sigmaapx_1, \sigmaapx_2 \in \mathcal{S}_A$, $\bin_A(\sigmaapx_i)$ has exactly $r_i$ bits set to $1$,
    and $\bin_A(\sigmaapx)$ has exactly $r_1+r_2$ bits set to $1$.
    Since we iterate over all $r_1,r_2 \in [0,\cw-|A|]$ and $k_1,k_2 \in [0,k]$,
    it holds that indeed $\DPapx_H[k,\sigmaapx]$ contains the number of partial matchings of $H$ of size $k$
    and signature that is compatible with $\sigmaapx$, for all $\sigmaapx \in \mathcal{S}_A$.
    % Lastly, iterating over the rest $k \in [0,n]$ and $A \subseteq [\cw]$ fills the whole table.

    % Let $S$ be such a partial matching of $H$, with $\sigma = \sgn_H(S)$ being compatible with $\sigmaapx$.
    % Moreover, let $S_i = S \cap V(H_i)$ for $H_i \in \{H_1,H_2\}$
    % with $\sigma_i = \sgn_{H_i}(S_i)$ for $i \in \{1,2\}$.
    % % with $k_i=|S_i| \in [0,n]$.
    % % Since $S_1 \cap S_2 = \varnothing$, it holds that $k_1+k_2 = |S| = k$.
    % % and further let $\sigmaapx_i \in \{0,\tilde{1},2\}^{\cw}$
    % % denote a pseudosignature compatible with $\sgn_{H_i}(S_i)$.


    % \begin{claim}\label{claim:induced:cw:union-correctness}
    %     For all $w \in [\cw]$ it holds that
    %     \begin{itemize}
    %         \item $\sigmaapx(w)=0$, then $\sigma_1(w)=\sigma_2(w)=0$,
    %         \item $\sigmaapx(w)=\tilde{1}$, then $\sigma_1(w), \sigma_2(w) \in \{0,1\}$,
    %         \item $\sigmaapx(w)=2$, then $(\sigma_1(w),\sigma_2(w)) \in \{(0,2),(2,0)\}$.
    %     \end{itemize}
    % \end{claim}

    % \begin{claimproof}
    %     Let $i \in \{1,2\}$.
    %     Notice that for all $v \in S_i$ it holds that
    %     $\lab_{H_i}(v) = \lab_H(v)$,
    %     thus $S \cap \lab^{-1}_H(w) =
    %     (S_1 \cap \lab^{-1}_{H_1}(w)) \cup (S_2 \cap \lab^{-1}_{H_2}(w))$.
    %     Furthermore, it holds that $\deg_{H_i[S_i]}(v) = \deg_{H[S]}(v)$.

    %     For the first item, assume that $\sigmaapx(w)=0$,
    %     which implies that $S \cap \lab^{-1}_H(w) = \varnothing$.
    %     This in turn implies that $S_1 \cap \lab^{-1}_{H_1}(w) = S_2 \cap \lab^{-1}_{H_2}(w) = \varnothing$,
    %     that is, $\sgn_{H_1,w}(S_1) = \sgn_{H_2,w}(S_2) = 0$, therefore $\sigma_1(w) = \sigma_2(w) = 0$.

    %     As for the second item, we prove by contradiction that $\sigma_1(w), \sigma_2(w) \in \{0,1\}$.
    %     Assume there exists $w \in [\cw]$ such that $\sigmaapx(w) = \tilde{1}$
    %     and, without loss of generality, $\sigma_1(w) = 2$.
    %     Then, $S \cap \lab^{-1}_H(w)$
    %     contains a vertex $v$ that is unsaturated in $H[S]$,
    %     thus $\sgn_{H,w}(w) \notin \{0,1\}$, which contradicts the compatibility of $\sgn_H(S)$ with $\sigmaapx$.

    %     Lastly, assume that $\sigmaapx(w)=2$,
    %     which implies that $S \cap \lab^{-1}(w) = \{v\}$ and $\deg_{H[S]}(v)=0$.
    %     Assume that $v \in S_{i_1}$ and let $S_{i_2}$ denote the other set among $\{S_1,S_2\}$.
    %     Then, it follows that $S_{i_1} \cap \lab^{-1}_{H_{i_1}}(w) = \{v\}$ with $\deg_{H_{i_1}[S_{i_1}]}(v)=0$
    %     and $S_{i_2} \cap \lab^{-1}_{H_{i_2}}(w) = \varnothing$,
    %     implying that $\sigma_{i_1}(w) = 2$ and $\sigma_{i_2}(w) = 0$.
    % \end{claimproof}



    As for Step 3, given all the previous discussion,
    it is easy to see that it indeed constructs the table $\DPt_H[\cdot,\cdot]$
    so that $\DPt_H[k,\sigma]$ contains the number of partial matchings of $H$ of size $k$ and signature $\sigma$.
    This completes the proof.
\end{proof}






% ----------------------------------------------------------------------------------------------------
% EXTRA STUFF REGARDING THE SIMPLE SOLUTION




% \proofsubparagraph{Disjoint union, $H = H_1 \oplus H_2$.}
% We define a function $h$ acting on pairs of signatures $\sigma'_1, \sigma'_2 \in \{0,1,2\}^{\cw}$.
% In particular, we set $h(\sigma'_1, \sigma'_2) = \sigma \in \{0,1,2\}^{\cw}$
% such that for all $w \in [\cw]$ it holds that
% \[
%     \sigma(w) =
%         \begin{cases}
%             \max \{ \sigma'_1(w), \, \sigma'_2(w) \}    &\text{if $0 \in \{\sigma'_1(w), \, \sigma'_2(w)\}$,}\\
%             1                                           &\text{if $\sigma'_1(w) = \sigma'_2(w) = 1$}.
%         \end{cases}
% \]
% We show that when given the signatures of two partial matchings $S_1 \subseteq V(H_1)$ and $S_2 \subseteq V(H_2)$,
% the function $h$ outputs exactly the signature of their union $S_1 \cup S_2$ in $H$.

% \begin{claim}\label{claim:induced:cw:union-correctness}
%     Let $S_1 \subseteq V(H_1)$ and $S_2 \subseteq V(H_2)$ such that
%     $\sgn_{H_1}(S_1) = \sigma'_1$ and $\sgn_{H_2}(S_2) = \sigma'_2$ respectively.
%     Then, $\sgn_{H}(S_1 \cup S_2) = \sigma$ if and only if $h(\sigma'_1, \sigma'_2) = \sigma$.
% \end{claim}


% \begin{claimproof}
%     Let $i \in \{1,2\}$ and $S = S_1 \cup S_2$ where by definition it holds that $S_1 \cap S_2 = \varnothing$
%     since $V(H_1) \cap V(H_2) = \varnothing$.
%     Notice that the labels of all vertices in $H_i$ remain the same between $H_i$ and $H$,
%     therefore it holds that for all $v \in V(H_i)$ we have that $\lab_H(v) = \lab_{H_i}(v)$.
%     Furthermore, for $w \in [\cw]$ it holds that
%     $S \cap \lab^{-1}_H(w) =
%     (S_1 \cup S_2) \cap (\lab^{-1}_{H_1}(w) \cup \lab^{-1}_{H_2}(w)) =
%     (S_1 \cap \lab^{-1}_{H_1}(w)) \cup (S_2 \cap \lab^{-1}_{H_2}(w))$.
%     Moreover, since $H$ is obtained from $H_1$ and $H_2$ by taking their disjoint union
%     we have that for all $v \in S \cap V(H_i)$ it holds that $\deg_{H[S]}(v) = \deg_{H_i[S_i]}(v)$.

%     Fix $w \in [\cw]$.
%     Assume first that $0 \in \{\sigma'_1(w), \, \sigma'_2(w)\}$.
%     In that case, it holds that either $S_1 \cap \lab^{-1}_{H_1}(w) = \varnothing$ or
%     $S_2 \cap \lab^{-1}_{H_2}(w) = \varnothing$.
%     If $S_1 \cap \lab^{-1}_{H_1}(w) = \varnothing$, then $\sgn_{H_1,w}(S_1) = 0$,
%     and thus $\sgn_{H,w}(S) = \sgn_{H_2,w}(S_2) = \max \{ \sgn_{H_1,w}(S_1), \, \sgn_{H_2,w}(S_2) \}$.
%     The case when $S_2 \cap \lab^{-1}_{H_2}(w) = \varnothing$ is analogous.
%     Thus, in this case we have that $\sgn_{H}(S)$ and $h(\sigma'_1, \sigma'_2)$ agree on the $w$-th coordinate.

%     Next, assume that $\sigma'_1(w) = \sigma'_2(w) = 1$.
%     In that case, we have that for all $v \in S_i \cap \lab^{-1}_{H_i}(w)$ it holds that $\deg_{H_i[S_i]}(v) = 1$.
%     Consequently, the discussion so far implies that for all
%     $v \in (S_1 \cap \lab^{-1}_{H_1}(w)) \cup (S_2 \cap \lab^{-1}_{H_2}(w)) = S \cap \lab^{-1}_H(w)$,
%     it holds that $\deg_{H[S]}(v) = 1$.
%     Thus, $\sgn_{H,w}(S) = 1$ and $\sgn_{H}(S)$ and $h(\sigma'_1, \sigma'_2)$ agree on the $w$-th coordinate.

%     Lastly, assume that $0 \notin \{\sigma'_1(w), \sigma'_2(w)\}$ and
%     either $\sigma'_1(w) \neq 1$ or $\sigma'_2(w) \neq 1$.
%     Assume without loss of generality that $\sigma'_1(w) = 2$ and $\sigma'_2(w) \in \{1,2\}$.
%     In that case, it holds that $S_1 \cap \lab^{-1}_{H_1}(w) = \{v\}$ with $\deg_{H_1[S_1]}(v) = 0$
%     as well as $S_2 \cap \lab^{-1}_{H_2}(w) \neq \varnothing$.
%     Consequently it holds that $S \cap \lab^{-1}_H(w)$ contains at least $2$ vertices $v_1,v_2$,
%     with $\deg_{H[S]}(v_1) = 0$, thus $\sgn_{H,w}(S)$ is undefined,
%     as is the case for $h(\sigma'_1, \sigma'_2)$.
% \end{claimproof}
% %
% For a fixed size $k \in [0,n]$ and signature $\sigma \in \{0,1,2\}^{\cw}$ we set the value of $\DPt_H[k,\sigma]$ to be
% \begin{equation}\label{eq:DPH-union}
%     \DPt_H[k,\sigma] =
%         \sum_{k_1+k_2=k} \:
%         \sum_{(\sigma'_1,\sigma'_2) \in h^{-1} (\sigma)}
%             \DPt_{H_1}[k_1,\sigma'_1] \cdot \DPt_{H_2}[k_2,\sigma'_2].
% \end{equation}
% If $h^{-1}(\sigma) = \varnothing$ we set $\DPt_H[k,\sigma] = 0$.


    \proofsubparagraph{Wrapping up.}
    We argue about the correctness of the algorithm by a bottom-up induction on the clique-width $\cw$-expression of~$G$.
    We claim that for all $H \in \mathcal{H}$, $k \in [0,k]$, and $\sigma \in \{0,1,2\}^{\cw}$,
    $\DPt_H[k,\sigma]$ is equal to the number of partial matchings in $H$ of size $k$ and signature $\sigma$.
    Let $H = i(v)$ be a singleton graph for some label $i \in [\cw]$.
    There are only two possible partial matchings in $H$, the first being $\{v\}$ and the second being $\varnothing$.
    Notice that then by \cref{eq:DPH-singleton}, the statement holds for singleton graphs.
    For the Join, Relabel, and Union nodes the statement holds due to
    \cref{lemma:induced:cw:join-correctness,lemma:induced:cw:relabel-correctness,lemma:induced:cw:union-correctness}.
    As for the algorithm's running time, notice that for any $\cw$-labeled graph $H \in \mathcal{H}$ arising in~$\psi$,
    the table $\DPt_H[\cdot,\cdot]$ can be filled in time~$\sO(3^{\cw})$.
    As the number of such graphs is~$\bO(|\psi|)$, this implies that our algorithm runs in time $\sO(3^{\cw})$.
    Finally, notice that for all $H \in \mathcal{H}$,
    the number of induced matchings of $H$ of size $\ell$ is equal to the sum of $\DPt_H[2\ell,\sigma]$
    over all $\sigma \in \{0,1\}^{\cw}$.
    This concludes the proof of \cref{thm:induced:cw}.
\end{proof}
