\section{Acyclic Matching}\label{sec:acyclic}

In this section we consider the {\AcyclicM} problem.
Chaudhary and Zehavi~\cite{wg/ChaudharyZ23a} recently gave an $\sO(6^{\tw})$ algorithm for this problem,
where $\tw$ denotes the treewidth of the input graph.
We first notice that, with some slightly more careful analysis of its running time,
the algorithm in fact runs in time $\sO(5^{\tw})$.
Our main contribution however is to show in \cref{subsec:acyclic:lb} that this is optimal under the pw-SETH,
even for the more restricted parameterization by pathwidth.
Lastly, in \cref{subsec:acyclic:cw} we employ the rank-based approach introduced by
Bodlaender, Cygan, Kratsch, and Nederlof~\cite{iandc/BodlaenderCKN15}
and adapted to the parameterization by clique-width by
Bergougnoux and Kant\'e~\cite{tcs/BergougnouxK19}
to develop a $2^{\bO(\cw)} n^{\bO(1)}$ algorithm for {\AcyclicM} parameterized by the clique-width $\cw$
of the input graph.


\begin{theoremrep}[\appsymb]\label{thm:acyclic:tw_algo}
    Given a graph $G$ along with a nice tree decomposition of $G$ of width $\tw$,
    there is an algorithm that computes the size of the maximum acyclic matching in $G$
    in time $\sO(5^{\tw})$.
    The algorithm is randomized, cannot give false positives,
    and may give false negatives with probability at most $1/3$.
\end{theoremrep}

\begin{proof}
    The algorithm is identical to the one of Chaudhary and Zehavi~\cite{wg/ChaudharyZ23a},
    with the only difference being that we analyze its running time slightly more carefully.
    We refer to~\cite[Section~6]{arxiv/ChaudharyZ23} for the full exposition of the algorithm.
    In the following we only sketch the main ideas and argue about the claimed running time.

    This is a standard DP algorithm over the nodes of the tree decomposition.
    In order to deal with the acyclicity constraint, the authors employ the Cut \& Count technique~\cite{talg/CyganNPPRW22},
    while fast subset convolution is used to speed up the computation in the Join nodes of the tree decomposition (see~\cite[Chapter~11]{books/CyganFKLMPPS15}).
    Observe that the $\sO(6^{\tw})$ running time claimed in~\cite{wg/ChaudharyZ23a} is solely due to the Join nodes,
    as for any other type of node $\sO(5^{\tw})$ time suffices to populate all the entries of the corresponding table.

    The key observation is that computing all entries of the table of a Join node can be done in time $\sO(5^{\tw})$.
    Following the same notation, let $\mathcal{B}_x$ denote a bag of a Join node $x$ of the tree decomposition.
    We aim to populate the table $\mathcal{A}_x[a,b,c,d,w,s]$, where $a,c \in [0,n]$, $b \in [0,n-1]$,
    $w \in [0,12n^2]$,
    and $d$ and $s$ are colorings such that $d \colon \mathcal{B}_x \to \{0,1,2\}$ and $s \colon \mathcal{B}_x \to \{0,l,r\}$
    with $d(v) = 0 \iff s(v) = 0$ for all $v \in \mathcal{B}_x$; we call such colorings $d$ and $s$ \emph{compatible}.
    Notice that the table $\mathcal{A}_x$ has $\sO(5^{\tw})$ entries.

    The authors prove that for fixed values of $a,b,c,w,s$,
    and for all colorings $d$ such that $d^{-1}(0) = R \subseteq \mathcal{B}_x$,
    one can compute the entries $\mathcal{A}_x [a,b,c,d,w,s]$ in time $2^{|\mathcal{B}_x \setminus R|} n^{\bO(1)}$.
    Consequently, summing over all subsets $R \subseteq \mathcal{B}_x$ and noticing that
    $\sum_{R \subseteq \mathcal{B}_x} 2^{|\mathcal{B}_x \setminus R|} = 3^{|\mathcal{B}_x|} \le 3^{\tw+1}$
    implies that for fixed values of $a,b,c,w,s$, the total time to compute the entries
    $\mathcal{A}_x [a,b,c,d,w,s]$ for \emph{all} colorings $d$ is $\sO(3^{\tw})$.
    Since $a,c \in [0,n]$, $b \in [0,n-1]$, $w \in [0,12n^2]$, and for every $d$ there are at most $2^{\tw}$
    compatible colorings $s$, it follows that there are $\sO(2^{\tw})$ ways to fix our parameters,
    resulting in a total time of $\sO(6^{\tw})$ to compute all entries of the table $\mathcal{A}_x$.
    Notice however that for a coloring $d$ where $d^{-1}(0) = R \subseteq \mathcal{B}_x$,
    it holds that there are at most $2^{|\mathcal{B}_x \setminus R|}$ compatible colorings $s$.
    Since $\sum_{R \subseteq \mathcal{B}_x} 2^{|\mathcal{B}_x \setminus R|} \cdot 2^{|\mathcal{B}_x \setminus R|} =
    5^{|\mathcal{B}_x|} \le 5^{\tw+1}$
    it holds that for fixed values of $a,b,c,w$,
    the total time to compute the entries of $\mathcal{A}_x [a,b,c,d,w,s]$
    for \emph{all compatible} colorings $d$ and $s$ is $\sO(5^{\tw})$,
    and thus the total time to compute all entries of the table $\mathcal{A}_x$ is $\sO(5^{\tw})$.
\end{proof}


\subsection{Lower Bound}\label{subsec:acyclic:lb}

Here we present the stated lower bound for \AcyclicM.

\begin{theoremrep}[\appsymb]\label{thm:acyclic:lb}
    For any constant $\varepsilon > 0$,
    there is no $\sO((5-\varepsilon)^\pw)$ algorithm deciding \AcyclicM,
    where $\pw$ denotes the pathwidth of the input graph,
    unless the pw-SETH is false.
\end{theoremrep}

\begin{proof}
    We present a reduction from the $4$-CSP-$5$ problem of \cref{cor:weird} to \AcyclicM.
    We are given a \textsc{CSP} instance $\psi$ whose variables take values from $[5]$,
    a partition of its variables into two sets $V_1, V_2$,
    a path decomposition $B_1, \ldots, B_t$ of $\psi$ of width $p$
    such that each bag contains $\bO(\log p)$ variables of $V_2$,
    and an injective function $b$ mapping each constraint to a bag that contains its variables.
    We want to construct an {\AcyclicM} instance $(G,\ell)$ with $\pw(G) = p + o(p)$,
    such that if $\psi$ is satisfiable, then $G$ has an acyclic matching of size $\ell$,
    while if $G$ has an acyclic matching of size $\ell$,
    $\psi$ admits a monotone satisfying multi-assignment which is consistent for $V_2$.
    We assume that the variables of $\psi$ are numbered $X = \{ x_1,\ldots, x_n \}$,
    the constraints are $c_1, \ldots, c_m$,
    and that the given path decomposition is nice.
    We construct $G$ as follows.


    \proofsubparagraph{XOR Gadget.}
    Given two vertices $v_1$ and $v_2$, when we say that we connect them via an \emph{XOR gadget} $\hat{O}(v_1,v_2)$
    we first add the edge $\{v_1,v_2\}$,
    then introduce two vertices $p^{v_1,v_2}_1$ and $p^{v_1,v_2}_2$ called the \emph{private vertices of the XOR gadget},
    and lastly add edges so that $v_1$, $v_2$, and the private vertices of the gadget form a cycle;
    for an illustration see \cref{fig:acyclic:lb_or_gadget}.
    We say that $v_1,v_2,p^{v_1,v_2}_1,p^{v_1,v_2}_2$ are the \emph{vertices of the XOR gadget $\hat{O}(v_1,v_2)$},
    while the \emph{edges of an XOR gadget} are the edges whose endpoints both are among its vertices.


    \begin{figure}[htb]
        \centering
        \begin{tikzpicture}[scale=1, transform shape]

        %%%%%%%%%% vertices and text

        \node[vertex] (u) at (1,5) {};
        \node[] () at (0.7,5) {$v_1$};

        \node[vertex] (x1) at (1.5,5.5) {};
        \node[] () at (1.3,5.8) {$p^{v_1,v_2}_1$};
        \node[vertex] (x2) at (2,5.5) {};
        \node[] () at (2.5,5.8) {$p^{v_1,v_2}_2$};

        \node[vertex] (v) at (2.5,5) {};
        \node[] () at (2.8,5) {$v_2$};


        %%%%%%%%% edges / arcs

        \draw[] (u)--(x1)--(x2)--(v)--(u);

        \end{tikzpicture}
        \caption{XOR gadget $\hat{O}(v_1,v_2)$.}
        \label{fig:acyclic:lb_or_gadget}
    \end{figure}


    \proofsubparagraph{Root vertex.}
    We start our construction by introducing a vertex $r$, which we refer to as the \emph{root vertex}.
    Furthermore, we attach a leaf $r'$ to $r$.



    \proofsubparagraph{Block and Variable Gadgets.}
    For every variable $x_i$ and every bag with $x_i \in B_j$, where $j \in [j_1,j_2]$,
    construct a \emph{block gadget} $\hat{B}_{i,j}$.
    Here we consider two cases, depending on whether $x_i \in V_1$ or $x_i \in V_2$.

    If $x_i \in V_1$, then we construct the block gadget as depicted in \cref{fig:acyclic:lb_block_gadget}.
    In order to do so, we introduce vertices $a, a', \chi_1, \chi_2, y_1, y_2, b_1, b_2$.
    Then, we attach leaves $l_1$ and $l_2$ to $\chi_1$ and $\chi_2$ respectively.
    Next, we add edges from $a$ to $\chi_1$ and $y_1$, from $a'$ to $\chi_2$ and $y_2$,
    from the root vertex $r$ to $\chi_1$ and $\chi_2$, as well as the edge $\{b_1,b_2\}$.
    Finally, we add the XOR gadgets $\hat{O}(a,b_1)$, $\hat{O}(a',b_2)$, $\hat{O}(\chi_1,\chi_2)$,
    and $\hat{O}(y_1,y_2)$.
    Next, for $j \in [j_1,j_2-1]$, we serially connect the block gadgets $\hat{B}_{i,j}$ and $\hat{B}_{i,j+1}$ so that the vertex $a'$ of $\hat{B}_{i,j}$ is the vertex $a$ of $\hat{B}_{i,j+1}$,
    thus resulting in the ``path'' $\hat{P}_i$ consisting of such serially connected block gadgets, called a \emph{variable gadget}.
    For an illustration see \cref{fig:acyclic:lb_serially_connected}.
    Intuitively, for any optimal matching $M$,
    it will hold that either $a, a' \notin V_M$ or $a, a' \in V_M$.
    In the latter case, it will hold that $|V_M \cap \{\chi_1,\chi_2\}| = |V_M \cap \{y_1,y_2\}| = 1$,
    resulting in $4$ choices for a total of $5$;
    we map each choice with an assignment where $x_i$ receives a value in $[5]$.


    \begin{figure}[ht]
        \centering
          \begin{subfigure}[b]{0.4\linewidth}
          \centering
            \begin{tikzpicture}[scale=0.75, transform shape]

            %%%%%%%%%% vertices and text

            \node[vertex] (a) at (0,5) {};
            \node[] () at (0,5.3) {$a$};

            \node[gray_vertex] (x1) at (1.5,6) {};
            \node[] () at (1.5,6.3) {$\chi_1$};

            \node[gray_vertex] (x2) at (4.5,6) {};
            \node[] () at (4.5,6.3) {$\chi_2$};

            \node[vertex] (y1) at (1.5,4) {};
            \node[] () at (1.5,4.3) {$y_1$};

            \node[vertex] (y2) at (4.5,4) {};
            \node[] () at (4.5,4.3) {$y_2$};

            \node[vertex] (a') at (6,5) {};
            \node[] () at (6,5.3) {$a'$};

            \node[vertex] (b1) at (2.5,2.5) {};
            \node[] () at (2.5,2.8) {$b_1$};
            \node[vertex] (b2) at (3.5,2.5) {};
            \node[] () at (3.5,2.8) {$b_2$};

            \node[gray_vertex] (r) at (3,7.5) {};
            \node[] () at (3,7.8) {$r$};


            %%%%%%%%% edges / arcs

            \draw[] (r)--(x1);
            \draw[] (r)--(x2);
            \draw[] (a)--(x1);
            \draw[] (x2)--(a');
            \draw[dashed] (x1)--(x2);
            \draw[] (a)--(y1);
            \draw[] (y2)--(a');
            \draw[dashed] (y1)--(y2);
            \draw[] (b1)--(b2);
            \draw[dashed] (a) edge [bend right] (b1);
            \draw[dashed] (a') edge [bend left] (b2);

            \end{tikzpicture}
            \caption{Block gadget.}
            \label{fig:acyclic:lb_block_gadget}
          \end{subfigure}
        \begin{subfigure}[b]{0.4\linewidth}
        \centering
            \begin{tikzpicture}[scale=0.5, transform shape]

            \node[vertex] (a) at (0,5) {};

            \node[gray_vertex] (x1) at (1.5,6) {};

            \node[gray_vertex] (x2) at (4.5,6) {};

            \node[vertex] (y1) at (1.5,4) {};

            \node[vertex] (y2) at (4.5,4) {};

            \node[vertex] (a') at (6,5) {};

            \node[vertex] (b1) at (2.5,2.5) {};
            \node[vertex] (b2) at (3.5,2.5) {};

            \begin{scope}[shift={(6,0)}]

                \node[gray_vertex] (x1') at (1.5,6) {};

                \node[gray_vertex] (x2') at (4.5,6) {};

                \node[vertex] (y1') at (1.5,4) {};

                \node[vertex] (y2') at (4.5,4) {};

                \node[vertex] (b1') at (2.5,2.5) {};
                \node[vertex] (b2') at (3.5,2.5) {};

                \node[vertex] (a'') at (6,5) {};

            \end{scope}


            \node[gray_vertex] (r) at (6,8.5) {};


            %%%%%%%%% edges / arcs

            \draw[] (r) edge [bend right] (x1);
            \draw[] (r) edge [bend right] (x2);
            \draw[] (r) edge [bend left] (x1');
            \draw[] (r) edge [bend left] (x2');
            \draw[] (a)--(x1);
            \draw[] (x2)--(a');
            \draw[dashed] (x1)--(x2);
            \draw[] (a)--(y1);
            \draw[] (y2)--(a');
            \draw[dashed] (y1)--(y2);
            \draw[] (b1)--(b2);
            \draw[dashed] (a) edge [bend right] (b1);
            \draw[dashed] (a') edge [bend left] (b2);
            \draw[] (a')--(x1');
            \draw[] (x2')--(a'');
            \draw[dashed] (x1')--(x2');
            \draw[] (a')--(y1');
            \draw[] (y2')--(a'');
            \draw[dashed] (y1')--(y2');
            \draw[] (b1')--(b2');
            \draw[dashed] (a') edge [bend right] (b1');
            \draw[dashed] (a'') edge [bend left] (b2');


            \end{tikzpicture}
            \caption{Serially connected block gadgets.}
            \label{fig:acyclic:lb_serially_connected}
          \end{subfigure}
        \caption{Case where $x_i \in V_1$. Gray vertices have a leaf attached, dashed edges denote XOR gadgets.}
        \label{fig:acyclic:lb}
        \end{figure}

        As for the case where $x_i \in V_2$, then the block gadget $\hat{B}_{i,j}$
        is constructed as follows.
        We introduce $6$ vertices $u^{i,j}, u^{i,j}_1, u^{i,j}_2, u^{i,j}_3, u^{i,j}_4, u^{i,j}_5$,
        we add an edge from $u^{i,j}$ to all vertices in $\setdef{u^{i,j}_k}{k \in [5]}$,
        and for all $1 \le k_1 < k_2 \le 5$ we add an XOR gadget
        $\hat{O}(u^{i,j}_{k_1},u^{i,j}_{k_2})$.
        Next, for $j \in [j_1,j_2-1]$, for all \emph{distinct} $k_1,k_2 \in [5]$,
        we add an XOR gadget $\hat{O}(u^{i,j}_{k_1},u^{i,j+1}_{k_2})$,
        and we call the \emph{variable gadget} $\hat{P}_i$ this sequence of serially connected block gadgets.



        \proofsubparagraph{Constraint Gadget.}
        This gadget is responsible for determining constraint satisfaction,
        based on the choices made in the rest of the graph.
        Let $c$ be a constraint of $\psi$, where $b(c) = j$ and
        $c$ involves variables $x_{i_1},x_{i_2},x_{i_3}, x_{i_4} \in B_j$.
        Consider the set $\mathcal{S}_c$ of the at most $5^4$ satisfying assignments of $c$.
        We construct the \emph{constraint gadget} $\hat{C}_c$ as follows.
        \begin{itemize}
            \item For each $\sigma \in \mathcal{S}_c$ construct a vertex $v_{c,\sigma}$.

            \item For distinct $\sigma_1,\sigma_2 \in \mathcal{S}_c$,
            introduce an XOR gadget $\hat{O}(v_{c,\sigma_1},v_{c,\sigma_2})$.

            \item Introduce a vertex $v_c$ and add edges between $v_c$ and all vertices $\setdef{v_{c,\sigma}}{\sigma \in \mathcal{S}_c}$.

            \item For each $\sigma \in \mathcal{S}_c$ and $\alpha \in [4]$,
            consider the following cases:
            \begin{itemize}
                \item if $x_{i_\alpha} \in V_1$ and $\sigma(x_{i_\alpha}) = 1$,
                then add an XOR gadget between $v_{c,\sigma}$ and vertices $\{b_1, b_2, \chi_2, y_2\}$ of $\hat{B}_{i,j}$,

                \item if $x_{i_\alpha} \in V_1$ and $\sigma(x_{i_\alpha}) = 2$,
                then add an XOR gadget between $v_{c,\sigma}$ and vertices $\{b_1, b_2, \chi_1, y_2\}$ of $\hat{B}_{i,j}$,

                \item if $x_{i_\alpha} \in V_1$ and $\sigma(x_{i_\alpha}) = 3$,
                then add an XOR gadget between $v_{c,\sigma}$ and vertices $\{a, a',y_1, y_2\}$ of $\hat{B}_{i,j}$,

                \item if $x_{i_\alpha} \in V_1$ and $\sigma(x_{i_\alpha}) = 4$,
                then add an XOR gadget between $v_{c,\sigma}$ and vertices $\{b_1, b_2, \chi_2, y_1\}$ of $\hat{B}_{i,j}$,

                \item if $x_{i_\alpha} \in V_1$ and $\sigma(x_{i_\alpha}) = 5$,
                then add an XOR gadget between $v_{c,\sigma}$ and vertices $\{b_1, b_2, \chi_1, y_1\}$ of $\hat{B}_{i,j}$,

                \item if $x_{i_\alpha} \in V_2$,
                then add an XOR gadget between $v_{c,\sigma}$ and all vertices in $\setdef{u^{i,j}_k}{k \in [5] \setminus \{ \sigma(x_{i_\alpha}) \}}$.
            \end{itemize}
        \end{itemize}

        This completes the construction of $G$.
        We set $L$ to be the number of XOR gadgets introduced in $G$,
        and we set $\ell = 1 + L + 2L_1 + L_2 + m$,
        where $L_i$ denotes the number of block gadgets constructed due to variables belonging to $V_i$,
        and $m$ is the number of constraints of $\psi$.
        In that case, $(G,\ell)$ is the constructed instance of \AcyclicM.



        \begin{lemma}\label{lem:acyclic:lb:csp->acyclic}
            If $\psi$ is satisfiable,
            then $G$ has an acyclic matching of size $\ell$.
        \end{lemma}

        \begin{nestedproof}
            Let $\sigma \colon X \to [5]$ be a satisfying assignment for $\psi$, that is,
            $\sigma$ satisfies all constraints $c_1, \ldots, c_m$.
            We will construct a matching $M \subseteq E(G)$ of size $|M| \geq \ell$ such that $G[V_M]$ is a forest.
            \begin{itemize}
                \item First, add to $M$ the edge $\{r,r'\}$.

                \item Then, for every XOR gadget $\hat{O}(v_1,v_2)$ in $G$, include in $M$ the edge between
                its two private vertices $\{p^{v_1,v_2}_1,p^{v_1,v_2}_2\}$.

                \item Next, for every constraint gadget $\hat{C}_c$, since $\sigma$ is a satisfying assignment,
                there exists some vertex $v_{c,\sigma_c}$ with $\sigma_c$ being the restriction of $\sigma$
                to the variables involved in $c$.
                Add in $M$ the edge $\{v_c,v_{c,\sigma_c}\}$.

                \item For each $x_i \in V_2$ and $j \in [t]$ such that $x_i \in B_j$
                we select into $M$ the edge $\{u^{i,j}, u^{i,j}_k\}$, where $\sigma(x_i) = k$.

                \item Finally, for each $x_i \in V_2$ and $j \in [t]$ such that $x_i \in B_j$,
                we include in $M$ the following edges from the block gadget $\hat{B}_{i,j}$:
                \begin{itemize}
                    \item if $\sigma(x_i) = 1$, the edges $\{ y_1, a \}$ and $\{ \chi_1, l_1 \}$,

                    \item if $\sigma(x_i) = 2$, the edges $\{ y_1, a \}$ and $\{ \chi_2, l_2 \}$,

                    \item if $\sigma(x_i) = 3$, the edges $\{ b_1, b_2 \}$ and $\{ \chi_1, l_1 \}$,

                    \item if $\sigma(x_i) = 4$, the edges $\{ y_2, a' \}$ and $\{ \chi_1, l_1 \}$,

                    \item if $\sigma(x_i) = 5$, the edges $\{ y_2, a' \}$ and $\{ \chi_2, l_2 \}$.
                \end{itemize}
            \end{itemize}
            Consequently, $M$ is a matching of size $1 + L + m + L_2 + 2L_1 = \ell$.

            We claim that $G[V_M]$ is a forest.
            Consider any constraint gadget $\hat{C}_c$,
            and let $v_{c,\sigma_c}$ such that $\{v_c, v_{c,\sigma_c}\} \in M$.
            Notice that since $\sigma_c$ is the restriction of $\sigma$ to its involved variables,
            one can easily verify that none of the vertices that $v_{c,\sigma_c}$ has an XOR gadget with belongs to $V_M$.
            Consequently, the vertices of the constraint gadgets belonging to $V_M$ induce acyclic components.

            It remains to argue that the graph induced by $r$, $r'$, and the vertices of the block gadgets in $V_M$ is acyclic.
            Notice that this is indeed the case for the block gadgets corresponding to variables in $V_2$,
            thus in the following we focus on the block gadgets corresponding to variables in $V_1$.
            Let $x_i \in V_1$ and $j \in [t]$ such that $x_i \in B_j$.
            First, notice that the graph induced by the vertices of any single block gadget $\hat{B}_{i,j}$ in $V_M$ along with $\{r,r'\}$
            is acyclic and vertices $a$ and $a'$ are disconnected.
            Furthermore, for $j_1, j_2 \in [t]$ and distinct $x_{i_1},x_{i_2} \in V_1$,
            notice that the block gadgets $\hat{B}_{i_1,j_1}$ and $\hat{B}_{i_2,j_2}$
            are disconnected in $G[V_M] - r$, thus if $G[V_M]$ contains a cycle,
            then this cycle involves only block gadgets corresponding to a single variable $x_i$.
            Finally, it holds that any two sequential block gadgets (as in \cref{fig:acyclic:lb_serially_connected})
            are connected in $G - r$ via their single common vertex which appears as $a$ in the first and $a'$ in the second gadget.
            Consequently, any cycle in $G[V_M]$ involving at least two block gadgets $\hat{B}_{i,j_1}$ and $\hat{B}_{i,j_2}$ with $j_1 < j_2$
            must contain $r$, the vertex $a'$ of $\hat{B}_{i,j_1}$, and the vertex $a$ of $\hat{B}_{i,j_2}$.
            Given that the vertices $a$ and $a'$ of any single block gadget are disconnected in $G[V_M] - r$,
            it follows that vertex $a'$ of $\hat{B}_{i,j_1}$ and $a$ of $\hat{B}_{i,j_2}$
            are disconnected in $G[V_M] - r$, thus no cycle exists in $G[V_M]$.
            This completes the proof.
        \end{nestedproof}

        \begin{lemma}
            If $G$ has an acyclic matching of size $\ell$,
            then $\psi$ admits a monotone satisfying multi-assignment which is consistent for $V_2$.
        \end{lemma}

        \begin{nestedproof}
            We start with the following claim.

            \begin{claim}\label{claim:acyclic:lb:acyclic->csp:maximum_matching}
                There exists an acyclic matching $M_{\max}$ of $G$ of maximum size such that
                for any XOR gadget $\hat{O}(v_1,v_2)$ of $G$ it holds that $|\{v_1,v_2\} \cap V_{M_{\max}}| \le 1$
                and $\{p^{v_1,v_2}_1,p^{v_1,v_2}\} \in M_{\max}$.
            \end{claim}

            \begin{claimproof}
                Let $M_{\max}$ be a maximum acyclic matching of $G$, and assume that $G$ contains an XOR gadget $\hat{O}(v_1,v_2)$.
                We show that we can obtain an acyclic matching $M'_{\max}$ of $G$ of the same size
                such that $|\{v_1,v_2\} \cap V_{M'_{\max}}| \le 1$ and $\{p^{v_1,v_2}_1,p^{v_1,v_2}\} \in M'_{\max}$.
                To this end, we consider different cases depending on the vertices of the XOR gadget that belong to $V_{M_{\max}}$.

                If $v_1,v_2 \notin V_{M_{\max}}$, then $\{p^{v_1,v_2}_1,p^{v_1,v_2}_2\} \in M_{\max}$ as
                $M_{\max}$ is of maximum size.

                Now assume that $|\{v_1,v_2\} \cap V_{M_{\max}}| = 1$, and without loss of generality $v_1 \in V_{M_{\max}}$ (the other case is analogous).
                Let $e \in M_{\max}$ be the edge incident to $v_1$.
                If $e \neq \{v_1,p^{v_1,v_2}_1\}$, then it holds that $\{p^{v_1,v_2}_1,p^{v_1,v_2}_2\} \in M_{\max}$ as $M_{\max}$ is of maximum size.
                Otherwise it holds that $p^{v_1,v_2}_2 \notin V_{M_{\max}}$ and the acyclic matching $M'_{\max} = (M_{\max} \setminus \{e\}) \cup \{\{p^{v_1,v_2}_1,p^{v_1,v_2}_2\}\}$
                has the same size.

                Lastly, assume that $v_1,v_2 \in V_{M_{\max}}$.
                Notice that in that case, either $p^{v_1,v_2}_1 \notin V_{M_{\max}}$ or $p^{v_1,v_2}_2 \notin V_{M_{\max}}$, as otherwise $G[V_{M_{\max}}]$
                contains a cycle.
                If $\{v_1,v_2\} \in M_{\max}$, then $M'_{\max} = (M_{\max} \setminus \{\{v_1,v_2\}\}) \cup \{\{p^{v_1,v_2}_1,p^{v_1,v_2}_2\}\}$
                is an acyclic matching of the same size.
                Otherwise, it holds that for either $v_1$ or $v_2$ there exists an edge $e \in M_{\max}$ incident to it
                that does not belong to the XOR gadget $\hat{O}(v_1,v_2)$.
                Assume without loss of generality that $e$ is incident to $v_1$, and let $e' \in M_{\max}$ be the edge incident to $v_2$ belonging to $M_{\max}$.
                Then, the acyclic matching $M'_{\max} = (M_{\max} \setminus \{e'\}) \cup \{\{p^{v_1,v_2}_1,p^{v_1,v_2}_2\}\}$ is of the same size.
            \end{claimproof}


            Armed with the previous claim, we can now prove the existence of an acyclic matching in $G$ with specific properties.

            \begin{claim}\label{claim:acyclic:lb:acyclic->csp:matching_properties}
                There exists an acyclic matching $M$ of maximum size of $G$ such that,
                for any XOR gadget $\hat{O}(v_1,v_2)$ of $G$ it holds that $|\{v_1,v_2\} \cap V_M| \le 1$ and $\{p^{v_1,v_2}_1,p^{v_1,v_2}\} \in M$.
                Furthermore, it holds that
                \begin{itemize}
                    \item $\{r,r'\} \in M$,

                    \item for all constraint gadgets $\hat{C}_c$,
                    there exists $\sigma \in \mathcal{S}_c$ such that $\{v_c, v_{c,\sigma_c}\} \in M$,

                    \item for all block gadgets $\hat{B}_{i,j}$,
                    \begin{itemize}
                        \item if $x_i \in V_1$ then
                        $| \{ \{\chi_1,l_1\}, \{\chi_2,l_2\} \} \cap M| =
                        |\{\{ y_1, a \}, \{ y_2, a' \}, \{ b_1, b_2 \}\} \cap M| = 1$,

                        \item otherwise if $x_i \in V_2$ there exists $k \in [5]$ such that $\{u^{i,j}, u^{i,j}_k\} \in M$.
                    \end{itemize}
                \end{itemize}
            \end{claim}

            \begin{claimproof}
                Let $M_{\max}$ be an acyclic matching of $G$ of maximum size as in
%--------------- EDO EXO KANEI PATENTA ---------------%
                \iflncs
                the previous claim,
                \else
                \cref{claim:acyclic:lb:acyclic->csp:maximum_matching},
                \fi
                where $|M_{\max}| \ge \ell$ since by assumption $G$ has an acyclic matching of size at least $\ell$.
                Consider the following reduction rule.

                \proofsubparagraph{Rule~$(\dagger)$.}
                Let $\{u,v\} \in M_{\mathsf{in}}$, and let $l \neq v$ be a leaf attached to $u$ in $G$, while $v$ is not.
                Then, replace $M_{\mathsf{in}}$ with $M_{\mathsf{out}} = (M_{\mathsf{in}} \setminus \{\{u,v\}\}) \cup \{\{u,l\}\}$.

                It is easy to see that in Rule~$(\dagger)$ if $M_{\mathsf{in}}$ is an acyclic matching,
                then so is $M_{\mathsf{out}}$, while the size of the two matchings is the same.
                Let $M$ denote the acyclic matching obtained after exhaustively applying Rule~$(\dagger)$ in $M_{\max}$.
                Notice that $|M| \ge \ell$, while for any XOR gadget $\hat{O}(v_1,v_2)$ of $G$ it holds that $|\{v_1,v_2\} \cap V_M| \le 1$
                and $\{p^{v_1,v_2}_1,p^{v_1,v_2}\} \in M$.

                Notice that due to Rule~$(\dagger)$, if $r \in V_M$, then $\{r,r'\} \in M$.
                Furthermore, for any edge in $M \setminus \{\{r,r'\}\}$ that is not due to an XOR gadget
                it holds that either both of its endpoints belong to the vertices of a block gadget,
                or they both belong to the vertices of a constraint gadget;
                that is due to the fact that any connected pair of vertices from both block and constraint gadgets
                or from different block gadgets is connected via an XOR gadget.

                Notice that in the constraint gadget $\hat{C}_c$,
                all vertices $\setdef{v_{c,\sigma}}{\sigma \in \mathcal{S}_c}$ are pairwise connected via XOR gadgets,
                thus at most one of them belongs to $V_M$.
                Along with the previous paragraph, it follows that either $V_M$ contains no vertex from $\hat{C}_c$,
                or there exists $\sigma_c \in \mathcal{S}_c$ such that $\{v_c,v_{c,\sigma_c}\} \in M$.

                Now consider the block gadget $\hat{B}_{i,j}$, where $j \in [t]$ such that $x_i \in B_j$.
                Assume first that $x_i \in V_1$,
                and notice that if $\chi_1 \in V_M$, then $\{\chi_1,l_1\} \in M$ and $\chi_2 \notin V_M$ (similarly for $\chi_2$).
                Furthermore, one can easily verify that $|\{\{ y_1, a \}, \{ y_2, a' \}, \{ b_1, b_2 \}\} \cap M| \le 1$.
                If on the other hand $x_i \in V_2$, then
                since vertices $\setdef{u^{i,j}_k}{k \in [5]}$ are pairwise connected via XOR gadgets,
                it holds that either $V_M$ contains no vertex of $\hat{B}_{i,j}$,
                or there exists $k \in [5]$ such that $\{u^{i,j}, u^{i,j}_k\} \in M$.

                Recall that $|M| \ge \ell = 1 + L + 2L_1 + L_2 + m$,
                where $L$ is the number of XOR gadgets in $G$,
                $L_i$ is the number of block gadgets constructed due to variables belonging to $V_i$,
                and $m$ is the number of constraints of $\psi$.
                Due to the discussion so far and summing over all constraint and block gadgets,
                $M$ contains at most $\ell$ edges.
                Consequently, it follows that the bounds obtained in the previous paragraphs are tight for all constraint and block gadgets,
                which implies that the inclusions are as described in the statement and $\{r,r'\} \in M$.
            \end{claimproof}


            Now let $M$ be an acyclic matching of $G$ as in
%--------------- EDO EXO KANEI PATENTA ---------------%
            \iflncs
            the previous claim.
            \else
            \cref{claim:acyclic:lb:acyclic->csp:matching_properties}.
            \fi
            Consider a multi-assignment $\sigma \colon X \times [t] \to [5]$ over all pairs $x_i \in X$ and $j \in [t]$
            with $x_i \in B_j$.
            In particular, consider two cases.
            If $x_i \in V_2$ and $\{u^{i,j}, u^{i,j}_k\} \in M$, then $\sigma(x_i,j) = k \in [5]$.
            Otherwise (i.e., $x_i \in V_1$),
            if for the edges of $\hat{B}_{i,j}$ it holds that
            \begin{itemize}
                \item $\{y_1,a\},  \{\chi_1,l_1\} \in M$, then $\sigma(x_i,j) = 1$,
                \item $\{y_1,a\},  \{\chi_2,l_2\} \in M$, then $\sigma(x_i,j) = 2$,
                \item $\{b_1,b_2\} \in M$,                then $\sigma(x_i,j) = 3$,
                \item $\{y_2,a'\}, \{\chi_1,l_1\} \in M$, then $\sigma(x_i,j) = 4$,
                \item $\{y_2,a'\}, \{\chi_2,l_2\} \in M$, then $\sigma(x_i,j) = 5$.
            \end{itemize}
            Notice that $\sigma$ is well-defined due to
%--------------- EDO EXO KANEI PATENTA ---------------%
            \iflncs
            the previous claim.
            \else
            \cref{claim:acyclic:lb:acyclic->csp:matching_properties}.
            \fi

            \begin{claim}
                It holds that $\sigma$ is monotone-increasing, as well as consistent for all $x_i \in V_2$.
            \end{claim}

            \begin{claimproof}
                Regarding the consistency for $x_i \in V_2$,
                notice that it follows due to the XOR gadgets between vertices belonging to sequential
                block gadgets $\hat{B}_{i,j}$ and $\hat{B}_{i,j+1}$.
                In the following we argue about the monotonicity.
                Assume that $x_i \in V_1$ and $j \in [j_1,j_2-1]$,
                where $B_{j_1}$ and $B_{j_2}$ denote the first and last bag of the decomposition containing $x_i$ respectively.


                We first argue that if for $\hat{B}_{i,j}$ it holds that
                $\{y_2 , a'\} \in M$,
                then for $\hat{B}_{i,j+1}$ it holds that
                $\{y_2 , a'\} \in M$.
                Indeed, since vertices $a'$ and $a$ of the first and the latter block gadget coincide,
                it follows that for $\hat{B}_{i,j+1}$ we have $\{y_1, a\}, \{b_1,b_2\} \notin M$,
                and since $|\{\{ y_1, a \}, \{ y_2, a' \}, \{ b_1, b_2 \}\} \cap M| = 1$ for any block gadget,
                the statement follows.

                Now we argue that if for $\hat{B}_{i,j}$ it holds that
                $\{y_2 , a'\}, \{\chi_2,l_2\} \in M$,
                then for $\hat{B}_{i,j+1}$ it holds that
                $\{y_2 , a'\}, \{\chi_2,l_2\} \in M$.
                In this case, due to the previous paragraph we have that in $\hat{B}_{i,j+1}$,
                $\{y_2 , a'\} \in M$ as well as $a \in V_M$,
                implying that $\chi_1 \notin V_M$ (as otherwise $G[V_M]$ has a cycle),
                which in turn implies that $\{\chi_2 , l_2\} \in M$.
                Consequently, it holds that if $\sigma (x_i,j) = 5$ then $\sigma (x_i,j+1) = 5$,
                while in conjunction with the previous paragraph we get that if $\sigma (x_i,j) = 4$
                then $\sigma (x_i,j+1) \in \{4,5\}$.

                Next, it is easy to see that if in $\hat{B}_{i,j}$ we have that $\{b_1 ,b_2 \} \in M$,
                then in $\hat{B}_{i,j+1}$ we have that $\{y_1 , a \} \notin M$;
                indeed, in $\hat{B}_{i,j}$ since $b_2 \in V_M$ it holds that $a' \notin V_M$,
                however this implies that for $\hat{B}_{i,j+1}$  it holds that $a \notin V_M$.
                Consequently, if $\sigma (x_i,j) = 3$ then $\sigma (x_i,j+1) \in \{3,4,5\}$.

                Finally assume that in $\hat{B}_{i,j}$
                we have that $\{y_2 ,a'\} , \{\chi_2 , l_2\} \in M$.
                In that case, it holds that in $\hat{B}_{i,j+1}$ we have $\chi_1 \notin V_M$
                (otherwise $G[V_M]$ contains a cycle),
                implying that $\{\chi_2, l_2\} \in M$.
                Consequently, we have that if $\sigma (x_i,j) = 2$ then
                $\sigma (x_i,j+1) \neq 1$.
            \end{claimproof}

            It remains to argue that $\sigma$ is a satisfying multi-assignment for $\psi$.
            Consider a constraint $c$ with $b(c)=j$,
            involving four variables from $V_{i_1}, V_{i_2}, V_{i_3}, V_{i_4}$.
            Let $\sigma_c \in \mathcal{S}_c$ such that $\{v_c, v_{c,\sigma_c}\} \in M$.
            We claim that $\sigma_c$ must be consistent with $\sigma$.
            Indeed, if $\sigma_c$ assigns value $k \in [5]$ to $x_i \in V_2$,
            then it follows that $(\setdef{u^{i,j}_k}{k \in [5]} \setminus \{u^{i,j}_{\sigma(x_i, j)}\}) \notin V_M$
            due to the associated XOR gadgets, thus $\{u^{i,j}, u^{i,j}_{\sigma(x_i, j)}\} \in M$.
            On the other hand, if $\sigma$ assigns value $k \in [5]$ to $x_i \in V_2$,
            then
            \begin{itemize}
                \item if $k=1$, then in $\hat{B}_{i,j}$ we have that $b_1,b_2,\chi_2,y_2 \notin V_M$, implying that $\{y_1,a\}, \{\chi_1,l_1\} \in M$,
                \item if $k=2$, then in $\hat{B}_{i,j}$ we have that $b_1,b_2,\chi_1,y_2 \notin V_M$, implying that $\{y_1,a\}, \{\chi_2,l_2\} \in M$,
                \item if $k=3$, then in $\hat{B}_{i,j}$ we have that $a,a',y_1,y_2 \notin V_M$, implying that $\{b_1,b_2\} \in M$,
                \item if $k=4$, then in $\hat{B}_{i,j}$ we have that $b_1,b_2,\chi_2,y_1 \notin V_M$, implying that $\{y_2,a'\}, \{\chi_1,l_1\} \in M$,
                \item if $k=5$, then in $\hat{B}_{i,j}$ we have that $b_1,b_2,\chi_1,y_1 \notin V_M$, implying that $\{y_2,a'\}, \{\chi_1,l_1\} \in M$.
            \end{itemize}
            This completes the proof.
        \end{nestedproof}

        Finally, to bound the pathwidth of the graph $G$,
        start with the decomposition of the primal graph of $\psi$ and
        in each $B_j$ replace each $x_i \in V_2 \cap B_j$ with all the vertices of $\hat{B}_{i,j}$.
        We further add in $B_j$ all the vertices of $\hat{B}_{i,j+1}$ for $x_i \in V_2 \cap B_{j+1}$,
        as well as the vertices $r$ and $r'$.
        For each constraint $c$, let $b(c)=j$ and add into $B_j$ all the
        vertices of $\hat{C}_c$ and any private vertex of an XOR gadget involving a vertex of $\hat{C}_c$,
        for a total of at most $1 + 5^4 + 2\binom{5^4}{2} + 8 \cdot 5^4$ vertices.
        So far each bag contains $p + \bO(\log p)$ vertices.
        To cover the remaining vertices,
        for each $j \in [t]$ we replace the bag $B_j$ with a sequence of bags such that
        all of them contain the vertices we have added to $B_j$ so far,
        the first bag contains the vertices $a$ belonging to all $\hat{B}_{i,j}$ for all $x_i \in V_1 \cap B_j$ and
        the last bag contains the vertices $a'$ belonging to all $\hat{B}_{i,j}$ for all $x_i \in V_1 \cap B_j$.
        We insert a sequence of $\bO(p)$ bags between these two,
        at each step adding all vertices apart from $a$ of a block gadget $\hat{B}_{i,j}$ and then removing
        all vertices of the same block gadget apart from $a'$, for $x_i \in V_1 \cap B_j$.
\end{proof}


\subsection{Parameterization by Clique-width}\label{subsec:acyclic:cw}

Before describing our algorithm, we first define some necessary notions.

% \begin{toappendix}

\subparagraph{Partitions.}
A \emph{partition} $p$ of a set $L$ is a collection of non-empty subsets of $L$
that are pairwise non-intersecting and such that $\bigcup_{p_i \in p} p_i = L$;
each set in $p$ is called a \emph{block} of $p$.
The set of partitions of a finite set $L$ is denoted by $\Pi(L)$,
and $(\Pi(L),\sqsubseteq)$ forms a lattice where $p \sqsubseteq q$
if for each block $p_i$ of $p$ there is a block $q_j$ of $q$ with
$p_i \subseteq q_j$.
The join operation of this lattice is denoted by $\sqcup$.
For example, we have $\{ \{ 1,2\},\{3,4\},\{5\}\} \sqcup \{\{1\},\{2,3\},\{4\},\{5\}\}= \{ \{ 1,2,3,4\}, \{5\}\}$.
Let $\block(p)$ denote the number of blocks of a partition $p$.
Observe that $\varnothing$ is the only partition of the empty set.
A \emph{weighted partition} is an element of $\Pi(L) \times \mathbb{N}$ for some finite set $L$.
For $p \in \Pi(L)$ and $X \subseteq L$, let $p_{\downarrow X} \in \Pi(X)$ be the partition
$\setdef{p_i \cap X}{p_i \in p} \setminus \{\varnothing\}$,
and for a set $Y$, let $p_{\uparrow Y} \in \Pi(L \cup Y)$ be the partition
$p \cup \left(\bigcup_{y \in Y \setminus L} \{\{y\}\}\right)$.

% \end{toappendix}


\begin{theoremrep}[\appsymb]
    There is an algorithm that,
    given an integer $\ell$ as well as a graph $G$ along with an irredundant clique-width expression $\psi$ of $G$ of width $\cw$,
    determines whether $G$ has an acyclic matching of size at least $\ell$ in time $2^{\bO(\cw)} n^{\bO(1)}$.
\end{theoremrep}

\begin{proof}
    The proof follows along the lines of the $2^{\bO(\cw)} n^{\bO(1)}$ algorithm for {\FVS} by Bergougnoux and Kant\'e~\cite{tcs/BergougnouxK19}.
    There, the authors develop a general framework, expanding upon the one of Bodlaender et al.~\cite{iandc/BodlaenderCKN15},
    in order to cope with various problems with connectivity constraints on graphs of bounded clique-width.
    Using their framework, we can perform DP over the clique-width expression and, starting from a partial solution,
    eventually build an optimal one.

    Instead of \AcyclicM, we solve the equivalent problem of asking,
    given a graph $G$ and an integer $k$, whether there exists $S \subseteq V(G)$ of size $|S| \ge k$ such that
    $G[S]$ is a forest that contains a perfect matching.
    Notice that $G$ has an acyclic matching of size $\ell$ if and only if there exists such a set $S$
    of size $|S| \ge 2\ell$.
    In the following let $\mathcal{H}$ denote the set of all $\cw$-labeled graphs generated
    by a subexpression of the given clique-width expression $\psi$ of the input graph $G$.

    For every $H \in \mathcal{H}$ and for every subset $L \subseteq \setdef{i \in [\cw]}{\lab^{-1}_H(i) \neq \varnothing}$,
    we compute a set of weighted partitions $\mathcal{A} \subseteq \Pi(L) \times \mathbb{N}$.
    Each weighted partition $(p,w) \in \mathcal{A} \subseteq \Pi(L) \times \mathbb{N}$ is intended to mean that
    there is a partial solution whose vertices $S \subseteq V(H)$ have weight $w$ and $S \cap \lab^{-1}_H(i) \neq \varnothing$
    for all labels $i \in L$ and $p$ is the transitive closure of the following equivalence relation $\sim$ on $L$:
    $i \sim j$ if there exist an $i$-vertex and a $j$-vertex in the same component of $H[S]$.
    Intuitively, for each label $i$ in $L$, we expect the $i$-vertices of $S$ to have an additional neighbor in any
    extension of $S$ into an optimum solution, therefore, we can consider all vertices of $S \cap \lab^{-1}_H(i)$ as
    one vertex in terms of connectivity.
    On the other hand, the vertices of $S \cap \lab^{-1}_H(j)$, where $j \in [\cw] \setminus L$ and $S$ contains at least one $j$-vertex,
    are expected to have no additional neighbor in any extension of $S$ into an optimum solution.
    Consequently, those vertices no longer play a role in the connectivity of the solution.
    These expectations allow us to represent the connected components of $H[S]$ by $p$.
    Our algorithm will guarantee that the weighted partitions computed from $(p,w)$ are computed accordingly to these expectations.
    It remains to deal with the acyclicity constraint, and to do this,
    we need to certify that whenever we join two weighted partitions and keep the result as a partial solution,
    it does not correspond to a partial solution with cycles.
    To this end, we employ the machinery introduced by~\cite{tcs/BergougnouxK19} and is based on~\cite{iandc/BodlaenderCKN15},
    which, for the sake of completeness, we present in \cref{sec:acyclic:cw:framework}.

    We use the weighted partitions defined in~\cite{tcs/BergougnouxK19} to represent the partial solutions.
    At each step of our algorithm we will ensure that the stored weighted partitions correspond to acyclic partial solutions.
    Since the framework of~\cite{tcs/BergougnouxK19} deals only with \emph{connected} acyclic solutions,
    as in~\cite{tcs/BergougnouxK19}, we introduce a hypothetical new vertex $v_0$ that is universal,
    and we compute a pair $(F,E_0)$ so that $F$ is a maximum induced forest of $G$ that has a perfect matching,
    $E_0$ is a subset of edges incident to $v_0$,
    and $(V(F) \cup \{v_0\}, E(F) \cup E_0)$ is a tree.

    We start with some definitions and notation.
    Let $H \in \mathcal{H}$, and consider a subset of its vertices $S \subseteq V(H)$ and
    a matching $M \subseteq E(H)$ with $V_M \subseteq S$, that is, all the vertices incident to
    edges in $M$ belong to $S$.
    In that case, we say that a vertex $v \in S$ is \emph{unsaturated} in $(S,M)$ if $v \notin V_M$,
    otherwise, i.e., if $v \in V_M$, we say that it is \emph{saturated} in $(S,M)$ instead.
    We say that such a pair $(S,M)$ is a \emph{partial solution} of $H$ if $H[S]$ is acyclic.
    Finally, we say that a partial solution $(S,M)$ of $H$ is \emph{extensible} if there exists an acyclic matching $M^*$ of
    $G$ such that $V_{M^*} \cap V(H) = S$ and $M^* \supseteq M$.

    \begin{claim}\label{claim:acyclic:cw:algorithm}
        Let $(S,M)$ be a partial solution of $H$ that is extensible,
        and let $M^*$ be an acyclic matching of $G$ such that
        $V_{M^*} \cap V(H) = S$ and $M^* \supseteq M$.
        Then, for all $i \in [\cw]$, the following hold.
        \begin{enumerate}
            \item There exists at most one unsaturated vertex of label $i$ in $(S,M)$,
            that is, $|(S \setminus V_M) \cap \lab^{-1}_H(i)| \le 1$.

            \item If $|(S \setminus V_M) \cap \lab^{-1}_H(i)| = 1$,
            then each vertex of $S \cap \lab^{-1}_H(i)$ belongs to a distinct connected component of $H[S]$.

            \item If $|S \cap \lab^{-1}_H(i)| \ge 2$,
            then the vertices of $S \cap \lab^{-1}_H(i)$ take part in at most one join operation
            with a non-empty set,
            that is, there exists at most one graph $H_2$ further in $\psi$
            such that $H_2 = \eta_{i',j'}(H_1)$ with $S \cap \lab^{-1}_H(i) = S \cap \lab^{-1}_{H_1}(i')$
            and $V_{M*} \cap \lab^{-1}_{H_1}(j') \neq \varnothing$.
        \end{enumerate}
    \end{claim}

    \begin{claimproof}
        First notice that for all $H \in \mathcal{H}$ it holds that $H[V_{M^*} \cap V(H)]$ is a subgraph of $G[V_{M^*}]$,
        therefore $H[V_{M^*} \cap V(H)]$ is acyclic.

        Towards a contradiction, let $v_1,v_2 \in (S \setminus V_M) \cap \lab^{-1}_H(i)$,
        and let $u_1,u_2 \in V_{M^*}$ such that $\{v_1,u_1\}, \{v_2,u_2\} \in M^*$.
        There are two cases.
        In the first case, at some point further in the clique-expression there exists a single join operation
        that saturates both $v_1$ and $v_2$, that is,
        there exist $H_1,H_2 \in \mathcal{H}$ and $H_2 = \eta_{i',j'}(H_1)$,
        with $v_1,v_2 \in \lab^{-1}_{H_1}(i')$ and $u_1,u_2 \in \lab^{-1}_{H_1}(j')$.
        Then however $H_2[S]$ contains a cycle, which contradicts the acyclicity of $H_2[V_{M^*} \cap V(H_2)]$
        since $S \subseteq V_{M^*} \cap V(H_2)$.
        In the second case, $v_1$ and $v_2$ are saturated by different join operations.
        This means however that after the first one, they are in the same connected component,
        thus the second join operation results in a cycle, again leading to a contradiction.
        The argument for the third item of the statement is analogous.

        Similarly, one can show that if $|(S \setminus V_M) \cap \lab^{-1}_H(i)| = 1$
        and there exist two vertices of $S \cap \lab^{-1}_H(i)$ belonging to the same connected component of $H[S]$,
        then the join operation that saturates $v$ results in a cycle, thus leading to a contradiction.
    \end{claimproof}


    We proceed by dynamic programming, aiming to construct in a bottom-up fashion all extensible partial solutions.
    In order to reduce the sizes of the sets stored in the entries of the DP tables,
    we will express the steps of the algorithm in terms of the operators on weighted partitions defined in~\cite{tcs/BergougnouxK19}.
    Let $H \in \mathcal{H}$.
    We are interested in storing \emph{ac-representative sets} of all partial solutions $(S,M)$ of $H$ that may produce a solution,
    thus we assume that the stored partial solutions satisfy the properties of \cref{claim:acyclic:cw:algorithm}.
    Consequently, it holds that if $S$ contains at least $2$ $i$-vertices, then the vertices of this label may
    partake in \emph{at most one} join operation with a non-empty set further in the clique-expression.
    Let $J \subseteq [\cw]$ denote the set of all labels from which $S$ contains at least $2$ vertices,
    and consider a partition $J_1 \uplus J_2 = J$ such that for the labels in $J_1$ there exists further in the clique-expression
    a join operation with a non-empty set while for those in $J_2$ it does not exist.
    Notice that in terms of connectivity, we can treat all vertices of $S \cap \lab^{-1}_H(i)$ for $i \in J_1$
    as one vertex, as they will all eventually end up in the same connected component of the final solution.
    On the other hand, for the vertices labeled $j \in J_2$,
    we can take into account how they affect the connectivity of the graph and subsequently ignore them,
    as no other edges will be added incident to them.
    We remark that if $J_2 = \varnothing$,
    then we can encode the connectivity of the graph by storing the partition $p \in \Pi([\cw])$ where
    $i$ and $j$ are in the same block if there are
    an $i$-vertex and a $j$-vertex in the same connected component of $H[S]$.



    Now consider the case where $H_1,H_2 \in \mathcal{H}$ with $H_2 = \eta_{i,j}(H_1)$ and $(S,M)$ is a partial solution of $H_1$,
    where $i,j \in J \subseteq [\cw]$.
    Notice that this join operation might result in cycles forming in $H_2[S]$, e.g.,
    whenever an $i$-vertex and a $j$-vertex are non-adjacent and belong to the same connected component of $H_1[S]$,
    or the number of $i$-vertices and $j$-vertices are both at least $2$ in $S$.
    Unfortunately, we are not able to handle all these cases with the operators on weighted partitions.
    To resolve the situation where an $i$-vertex and a $j$-vertex are already adjacent,
    we consider \emph{irredundant} $\cw$-expressions, i.e., whenever an operation $\eta_{i,j}$ is used there are no edges
    between $i$-vertices and $j$-vertices.
    For the other cases, we index the DP tables with total functions $s \colon [\cw] \to \Gamma$ called \emph{signatures}, that indicate, for each label $i \in [\cw]$, whether $\lab_H^{-1}(i)$ intersects $S$ and $V_M$,
    with $\Gamma = \{\gamma_0,\gamma_{\tilde{1}},\gamma_1,\gamma_{\tilde{2}},\gamma_2,\gamma_{-2}\}$.
    In particular, we say that a partial solution $(S,M)$ of $H$ is \emph{compatible} with the signature $s$ if for all $i \in [\cw]$
    it holds that
    \begin{itemize}
        \item if $s(i) = \gamma_0$, then $S \cap \lab^{-1}_H(i) = \varnothing$,

        \item if $s(i) = \gamma_{\tilde{1}}$, then $S \cap \lab^{-1}_H(i) = \{v\}$ and $v \notin V_M$,

        \item if $s(i) = \gamma_1$, then $S \cap \lab^{-1}_H(i) = \{v\}$ and $v \in V_M$,

        \item if $s(i) = \gamma_{\tilde{2}}$, then $|S \cap \lab^{-1}_H(i)| \ge 2$ and $|(S \setminus V_M) \cap \lab^{-1}_H(i)| = 1$,

        \item if $s(i) \in \{\gamma_2, \, \gamma_{-2}\}$, then $|S \cap \lab^{-1}_H(i)| \ge 2$ and $(S \setminus V_M) \cap \lab^{-1}_H(i) = \varnothing$.
    \end{itemize}
    Defining the signatures like so allows us to resolve the problem with the sets $J_1$ and $J_2$ by forcing each label
    $i$ with $s(i) = \gamma_2$ to wait for \emph{exactly} one clique-width operation $\eta_{i,\ell}$ for some $\ell \in [\cw]$ with $s(\ell) \neq \gamma_0$,
    while any label $j$ with $s(j) = \gamma_{-2}$ is forbidden to partake in any such operation.
    Notice that by definition, any label class $i$ with $s(i) = \gamma_{\tilde{2}}$ is also forced to wait for exactly $1$ such join operation,
    as no vertex is unsaturated in $G$.
    Thus, we translate all the acyclicity tests to the $\acjoin$ operation.
    The following notion of \emph{certificate graph} formalizes this requirement,
    where the vertices in $V^+_s$ represent the expected future neighbors of all the vertices in
    $S \cap \lab^{-1}_H(s^{-1}(\{\gamma_{\tilde{2}},\gamma_2\}))$.




    \begin{definition}[Certificate graph of a solution]\label{defn:certif}
        Let $H \in \mathcal{H}$, $F$ an induced forest of $H$,
        $s \colon [\cw] \to \Gamma$,
        and $E_0$ a subset of edges incident to $v_0$.
        Let $V^+_s = \setdef{v^+_i}{i \in s^{-1}(\{\gamma_{\tilde{2}},\gamma_2\})}$.
        The \emph{certificate graph of $(F,E_0)$ with respect to $s$},
        denoted by $\CG(F,E_0,s)$,
        is the graph $(V(F) \cup V_s^+ \cup \{v_0\}, E(F) \cup E_0 \cup E_s^+)$ with
        \[
            E_s^+ = \bigcup_{i \in s^{-1}(\{\gamma_{\tilde{2}},\gamma_2\})} \setdef{\{v,v^+_i\}}{v \in V(F) \cap \lab^{-1}_H(i)}.
        \]
    \end{definition}

    We are now ready to define the sets of weighted partitions whose representatives we manipulate in our dynamic programming tables.

    \begin{definition}[{Weighted partitions in $\mathcal{A}_H[s]$}]\label{defn:tabfvs}
        Let $H \in \mathcal{H}$ and consider a signature $s \colon [\cw] \to \Gamma$.
        The entries of $\mathcal{A}_H[s]$ are all weighted partitions
        $(p,w) \in \Pi(s^{-1}(\{\gamma_{\tilde{1}},\gamma_1,\gamma_{\tilde{2}},\gamma_2\}) \cup \{v_0\}) \times \mathbb{N}$
        such that there exist a partial solution $(S,M)$ of $H$ and $E_0 \subseteq \setdef{\{v_0 v\}}{v \in S}$
        so that $\wc (S) = w$, and
        \begin{enumerate}
            \item $(S, M)$ is compatible with $s$,

            \item the certificate graph $\CG(H[S],E_0,s)$ is a forest,

            \item each connected component of $\CG(H[S],E_0,s)$ has at least one vertex in
            $\lab^{-1}_H(s^{-1}(\{\gamma_{\tilde{1}},\gamma_1,\gamma_{\tilde{2}},\gamma_2\})) \cup \{v_0\}$,

            \item the partition $p$ equals $(s^{-1}(\{\gamma_{\tilde{1}},\gamma_1,\gamma_{\tilde{2}},\gamma_2\}) \cup \{v_0\})/\sim$,
            where $i \sim j$ if and only if a vertex in $S \cap \lab^{-1}_H(i)$ is connected, in $\CG(H[S],E_0,s)$,
            to a vertex in $S \cap \lab^{-1}_H(j)$; we consider $\lab^{-1}_H(v_0)=\{v_0\}$.
        \end{enumerate}
    \end{definition}
    %
    Conditions (2) and (4) guarantee that $(S \cup \{v_0\},E(H[S]) \cup E_0)$ can be extended into a tree, if any.
    They also guarantee that cycles detected through the $\acjoin$ operation correspond to cycles,
    and each cycle can be detected with it.
    In the following we call any quadruple $(S,M,E_0,(p,\wc(S)))$ a \emph{candidate solution} in $\mathcal{A}_H[s]$ if Condition (4) is satisfied,
    and if in addition Conditions (1)-(3) are satisfied, we call it a \emph{solution} in $\mathcal{A}_H[s]$.

    In that case, the size of a maximum acyclic matching of $G$ corresponds to the maximum,
    over all signatures $s \colon [\cw] \to \Gamma$
    with $s^{-1}(\{\gamma_{\tilde{1}},\gamma_{\tilde{2}},\gamma_{2}\}) = \varnothing$,
    of $\max \setdef{w}{(\{ s^{-1}(\gamma_{1}) \cup \{v_0\} \},w) \in \mathcal{A}_G[s]}$.
    Indeed, by definition, if $(\{ s^{-1}(\gamma_{1}) \cup \{v_0\} \},w)$ belongs to $\mathcal{A}_{G}[s]$,
    then there exists a partial solution $(S,M)$ of $G$ with $V_M=S$ and $\wc(S)=w$,
    as well as a set $E_0$ of edges incident to $v_0$ such that $(S \cup \{v_0\},E(H[S]) \cup E_0)$ is a tree.
    This follows from the fact that if $s^{-1}(\{\gamma_{\tilde{2}},\gamma_{2}\}) = \varnothing$,
    then we have $\CG(H[S],E_0,s) = (S \cup \{v_0\}, E(H[S]) \cup E_0)$.

    Our algorithm will store, for each $H \in \mathcal{H}$ and each $s \colon [\cw] \to \Gamma$,
    an ac-representative set $\DPt_H[s]$ of $\mathcal{A}_H[s]$.
    We are now ready to give the different steps of the algorithm, depending on the clique-width operations.
    Recall that $f[\alpha \mapsto \beta]$ denotes the function obtained from $f$
    that maps $\alpha$ to $\beta$ instead of $f(\alpha)$.

    \proofsubparagraph{Singleton $H = i(v)$.}
    For $H = i(v)$, notice that $\lab^{-1}_H(i) = \{v\}$ and $\lab^{-1}_H(w) = \varnothing$ for all $w \in [\cw] \setminus \{i\}$.
    Consequently, for $s \colon [\cw] \to \Gamma$ we set the following values,
    where $s_0$ is the signature such that $s_0(w)=\gamma_0$ for all $w \in [\cw]$.
    \[
        \DPt_H[s] =
            \begin{cases}
                \bigl\{ \bigl( \{\{v_0\} \} ,0 \bigr) \bigr\}                                                   &\text{if $s = s_0$,}\\
                \bigl\{ \bigl( \{\{i,v_0\}\}, \wc(v) \bigr), \bigl( \{\{i\},\{v_0\}\}, \wc(v) \bigr) \bigr\}    &\text{if $s = s_0 [i \mapsto \gamma_{\tilde{1}}]$,}\\
                \varnothing                                                                                     &\text{otherwise.}
            \end{cases}
    \]
    %
    Since $|V(H)|=1$, there is no partial solution $(S,M)$ intersecting $\lab^{-1}_H(i)$ on at least two vertices while $H$ has no edges,
    so the set of weighted partitions satisfying \cref{defn:tabfvs} equals the empty set for $s(i) \neq \{\gamma_0,\gamma_{\tilde{1}}\}$.
    If $s(i) = \gamma_{\tilde{1}}$, there are two possibilities, depending on whether $E_0 = \varnothing$ or
    $E_0 = \{v v_0\}$.
    We can thus conclude that $\DPt_H[s] = \mathcal{A}_H[s]$ is correctly computed.

    \proofsubparagraph{Joining labels with edges, $H = \eta_{i,j}(H')$.}
    Before describing how to populate the table in this case,
    we first define a helper function%
    \footnote{We omit the use of braces in case of singletons for the sake of readability.}
    $f \colon \Gamma \times \Gamma \to 2^{\Gamma \times \Gamma}$
    as follows, where the row denotes the first input and the column the second:
    \[
        \begin{array}{r|cccccc}
            f 	                & \gamma_0 							& \gamma_{\tilde{1}} 																			& \gamma_1 							& \gamma_{\tilde{2}} 				& \gamma_2 							& \gamma_{-2} \\
            \hline
            \gamma_0 			& (\gamma_0, \gamma_0) 				& (\gamma_0, \gamma_{\tilde{1}}) 																& (\gamma_0, \gamma_1) 				& (\gamma_0, \gamma_{\tilde{2}}) 	& (\gamma_0, \gamma_2) 				& (\gamma_0, \gamma_{-2}) \\
            \gamma_{\tilde{1}} 	& (\gamma_{\tilde{1}}, \gamma_0) 	& \left\{\makecell{(\gamma_{\tilde{1}}, \gamma_{\tilde{1}}) \\ (\gamma_1,\gamma_1)} \right\} 	& (\gamma_{\tilde{1}}, \gamma_1) 	& (\gamma_1, \gamma_{-2}) 			& (\gamma_{\tilde{1}}, \gamma_{-2}) & \varnothing \\
            \gamma_1 			& (\gamma_1, \gamma_0) 				& (\gamma_1, \gamma_{\tilde{1}}) 																& (\gamma_1, \gamma_1) 				& \varnothing 								& (\gamma_1, \gamma_{-2}) 	& \varnothing \\
            \gamma_{\tilde{2}} 	& (\gamma_{\tilde{2}}, \gamma_0)	& (\gamma_{-2}, \gamma_1) 																		& \varnothing 						& \varnothing 								& \varnothing 				& \varnothing \\
            \gamma_2 			& (\gamma_2, \gamma_0) 				& (\gamma_{-2}, \gamma_{\tilde{1}}) 															& (\gamma_{-2}, \gamma_1) 			& \varnothing 								& \varnothing 				& \varnothing \\
            \gamma_{-2}			& (\gamma_{-2}, \gamma_0) 			& \varnothing																					& \varnothing 						& \varnothing 								& \varnothing 				& \varnothing
        \end{array}
    \]
    In the following, let $f^{-1}(\alpha, \beta) = \setdef{(\alpha', \beta') \in \Gamma \times \Gamma}{(\alpha,\beta) \in f(\alpha', \beta')}$.
    For $s \colon [\cw] \to \Gamma$ we set
    \[
        \DPt_H[s] =
            \begin{cases}
                \DPt_{H'}[s]		            &\text{if $\gamma_0 \in \{s(i), \, s(j)\}$,}\\
                \varnothing			            &\text{if $f^{-1}(s(i),s(j)) = \varnothing$,}\\
                \acreduce ( \rmc(\mathcal{A}) ) &\text{otherwise,}
            \end{cases}
    \]
    where
    \[
        \mathcal{A} = \bigcup_{(\alpha',\beta') \in f^{-1}(s(i),s(j))}
            \proj \left(
                \acjoin\left( \DPt_{H'} \Bigl[ s[i \mapsto \alpha'][j \mapsto \beta'] \Bigr], \,
                \{ ( \{\{i,j\}\},0)\}\right), \,
                s^{-1}(\gamma_{-2}) \cap \{i,j\}
            \right).
    \]

    Notice that in the first case we do not need to use the operators $\acreduce$ and $\rmc$ since we update
    $\DPt_H[s]$ with one table from $\DPt_{H'}$.
    The second case can only be when (i) $s(i)=s(j)=\gamma_{-2}$,
    or (ii) $\gamma_0 \notin \{s(i),s(j)\}$ and
    $\{\gamma_{\tilde{2}},\gamma_{2}\} \cap \{s(i),s(j)\} \neq \varnothing$.
    As for the third case, we have that $s(i),s(j) \in \{\gamma_{\tilde{1}}, \gamma_{1}, \gamma_{-2}\}$ and
    either $s(i) \neq \gamma_{-2}$ or $s(j) \neq \gamma_{-2}$.
    Intuitively, we consider the weighted partitions $(p,w) \in \mathcal{A}$ such that
    $i$ and $j$ belong to different blocks of $p$,
    we merge the blocks containing $i$ and $j$,
    remove the elements in $s^{-1}(\gamma_{-2}) \cap \{i,j\}$ from the resulting block,
    and add the resulting weighted partition to $\DPt_H[s]$.
    Notice that we also have to consider whether a new edge has been added to the partial solution due to the join operation.


    \proofsubparagraph{Relabeling, $H = \rho_{i \to j}(H')$.}
    As before, we first define a helper function
    $g \colon \Gamma \times \Gamma \to 2^{\Gamma}$
    as follows, where the row denotes the first input and the column the second:
    \[
        \begin{array}{r|cccccc}
            g 					& \gamma_0 				& \gamma_{\tilde{1}} 	& \gamma_1 												& \gamma_{\tilde{2}} 	& \gamma_2 				& \gamma_{-2} \\
            \hline
            \gamma_0 			& \gamma_0 				& \gamma_{\tilde{1}} 	& \gamma_1 												& \gamma_{\tilde{2}} 	& \gamma_2 				& \gamma_{-2} \\
            \gamma_{\tilde{1}} 	& \gamma_{\tilde{1}} 	& \varnothing 			& \gamma_{\tilde{2}} 									& \varnothing 			& \gamma_{\tilde{2}} 	& \varnothing \\
            \gamma_1 			& \gamma_1				& \gamma_{\tilde{2}}	& \left\{ \makecell{ \gamma_2 \\ \gamma_{-2}} \right\}	& \gamma_{\tilde{2}} 	& \gamma_2				& \gamma_{-2} \\
            \gamma_{\tilde{2}} 	& \gamma_{\tilde{2}}	& \varnothing 			& \gamma_{\tilde{2}} 									& \varnothing 			& \gamma_{\tilde{2}} 	& \varnothing \\
            \gamma_2 			& \gamma_2 				& \gamma_{\tilde{2}} 	& \gamma_2 												& \gamma_{\tilde{2}}	& \gamma_2 				& \varnothing \\
            \gamma_{-2}			& \gamma_{-2} 			& \varnothing			& \gamma_{-2} 											& \varnothing 			& \varnothing 			& \gamma_{-2}
        \end{array}
    \]
    In the following, let $\gminusrest(\beta) = \setdef{(\alpha', \beta') \in \Gamma \times \Gamma}{\beta \in f(\alpha', \beta') \text{ and } \gamma_0 \notin \{ \alpha', \beta' \}}$.
    For $s \colon [\cw] \to \Gamma$ we set
    \[
        \DPt_H[s] =
            \begin{cases}
                \acreduce ( \rmc(\mathcal{A}_1 \cup \mathcal{A}_2 \cup \mathcal{A}_3 ) )    &\text{if $s(i) = \gamma_0$,}\\
                \varnothing                                                                 &\text{otherwise},
            \end{cases}
    \]
    where we obtain $\mathcal{A}_1,\mathcal{A}_2,\mathcal{A}_3$ as follows.


    $\mathcal{A}_1$ contains all weighted partitions corresponding to partial solutions not intersecting $\lab^{-1}_{H'}(i)$,
    which are trivially partial solutions for $H$ as well, and we have
    \[
        \mathcal{A}_1 = \DPt_{H'} \Bigl[ s[i \mapsto \gamma_0] \Bigr].
    \]

    $\mathcal{A}_2$ contains all weighted partitions corresponding to partial solutions that
    intersect $\lab^{-1}_{H'}(i)$ but not $\lab^{-1}_{H'}(j)$.
    Notice that the case where a partial solution intersects neither,
    i.e., $s(i) = s(j) = \gamma_0$, is already covered by $\mathcal{A}_1$.
    We set
    \[
        \mathcal{A}_2 =
            \begin{cases}
                \varnothing         &\text{if $s(j) = \gamma_0$,}\\
                \DPt_{H'}[s_{H'}]        &\text{if $s(j) = \gamma_{-2}$},\\
                \proj\big(
                    \acjoin( \DPt_{H'}[s_{H'}], \,
                        \{ ( \{\{i,j\}\},0) \}), \,
                    \{i\}
                \big)               &\text{otherwise},
            \end{cases}
    \]
    where $s_{H'} = s[i \mapsto s(j)][j \mapsto \gamma_0]$.

    The set $\mathcal{A}_3$ has to do with the cases where the partial solution intersects
    both $\lab^{-1}_{H'}(i)$ and $\lab^{-1}_{H'}(j)$, which implies that $s(j) \in \{ \gamma_{\tilde{2}}, \gamma_2, \gamma_{-2}\}$.
    To this end, we consider the following subcases.
    If $s(j) = \gamma_{-2}$, then
    \[
        \mathcal{A}_3 = \bigcup_{(\alpha',\beta') \in \gminusrest(\gamma_{-2})}
                \proj\Big( \DPt_{H'} \Bigl[ s[i \mapsto \alpha'][j \mapsto \beta'] \Bigr] , \, \{i,j\} \Big).
    \]
    On the other hand, if $s(j) \in \{ \gamma_{\tilde{2}}, \gamma_2 \}$, then
    \[
        \mathcal{A}_3 = \bigcup_{(\alpha',\beta') \in \gminusrest(s(j))}
                \proj\bigg(
                    \acjoin \Big(
                        \DPt_{H'} \Bigl[ s[i \mapsto \alpha'][j \mapsto \beta'] \Bigr], \,
                        \{ ( \{\{i,j\}\},0) \} \Big), \,
                    \{i\}
                \bigg).
    \]

    \proofsubparagraph{Disjoint union, $H = H_1 \oplus H_2$.}
    Consider the signatures $s, s_1, s_2$.
    We say that $s_1,s_2$ \emph{agree on $s$} if for all labels $i \in [\cw]$,
    it holds that $s(i) \in g(s_1(i), s_2(i))$.
    In that case, for $s \colon [\cw] \to \Gamma$ we set $\DPt_H[s] = \acreduce(\rmc(\mathcal{A}))$,
    where
    \[
        \mathcal{A} = \bigcup_{s_1,s_2 \text{ agree on } s}
            \acjoin \bigg(
                \proj \Big( \DPt_{H_1}[s_1], \, s^{-1}(\gamma_{-2}) \Big), \,
                \proj \Big( \DPt_{H_2}[s_2], \, s^{-1}(\gamma_{-2}) \Big)
            \bigg).
    \]
    As in~\cite{tcs/BergougnouxK19},
    we need to do the projections before the $\acjoin$ operation.

    The correctness of the algorithm follows along the lines of the {\FVS} algorithm in~\cite{tcs/BergougnouxK19}.
\end{proof}

\begin{proofsketch}
    The proof follows along the lines of the $2^{\bO(\cw)} n^{\bO(1)}$ algorithm for {\FVS} by Bergougnoux and Kant\'e~\cite{tcs/BergougnouxK19}.
    There, the authors develop a general framework, expanding upon the one of Bodlaender et al.~\cite{iandc/BodlaenderCKN15},
    in order to cope with various problems with connectivity constraints on graphs of bounded clique-width.
    Using their framework, we can perform DP over the clique-width expression and, starting from a partial solution,
    eventually build an optimal one.

    Instead of \AcyclicM, we solve the equivalent problem of asking,
    given a graph $G$ and an integer $k$, whether there exists $S \subseteq V(G)$ of size $|S| \ge k$ such that
    $G[S]$ is a forest that contains a perfect matching.
    Notice that $G$ has an acyclic matching of size $\ell$ if and only if there exists such a set $S$
    of size $|S| \ge 2\ell$.
    In the following let $\mathcal{H}$ denote the set of all $\cw$-labeled graphs generated
    by a subexpression of the given clique-width expression $\psi$ of the input graph $G$.

    In order to deal with \FVS, in~\cite{tcs/BergougnouxK19} the authors solve the \textsc{Maximum Induced Tree} problem and
    introduce a \emph{signature} on the partial solutions,
    that encodes (i) the number of vertices that a partial solution contains from each label class,
    and (ii) in case a solution contains at least $2$ vertices of the same label class,
    whether this label class will partake in a join operation with a non-empty set in the future;
    as a matter of fact, the number of such join operations can be at most $1$, as otherwise a cycle is formed.
    For our problem, we have that a partial solution in $H \in \mathcal{H}$
    is a pair $(S,M)$ such that $M$ is a matching of $H$ with $V_M \subseteq S \subseteq V(H)$.
    In that case, the signature of a partial solution is, for each label $i \in [\cw]$,
    whether $S \cap \lab^{-1}_H(i)$ is empty, a singleton, or contains at least $2$ vertices,
    and whether $(S \setminus V_M) \cap \lab^{-1}_H(i)$ is empty or not.
    We notice that in fact $|(S \setminus V_M) \cap \lab^{-1}_H(i)| \le 1$, thus leading to $6$ different states
    $\Gamma = \{\gamma_0, \gamma_{\tilde{1}}, \gamma_1, \gamma_{\tilde{2}}, \gamma_2, \gamma_{-2}\}$
    per label class, with the index denoting the number of vertices in $S \cap \lab^{-1}_H(i)$,
    the tilde denoting that $|(S \setminus V_M) \cap \lab^{-1}_H(i)| = 1$,
    and $\gamma_2$ (resp., $\gamma_{-2}$) denoting that the label class will (resp., will not)
    partake in a join operation in the future.
    This, along with the machinery of the framework of~\cite{tcs/BergougnouxK19},
    allows us to detect any possible cycles as well as forcing the final solution to be a forest containing a perfect matching.
\end{proofsketch}
