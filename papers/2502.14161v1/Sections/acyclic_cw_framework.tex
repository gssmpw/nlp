\begin{toappendix}

\section{Framework}\label{sec:acyclic:cw:framework}

For the sake of completeness, here we present the framework introduced in~\cite[Section~3]{tcs/BergougnouxK19}.

The following operation determines whether a join operation between pairs of partitions produces any cycles.

\begin{definition}
    Let $L$ be a finite set.
    We define $\acy \colon \Pi(L) \times \Pi(L) \to \{\true,\false\}$
    such that $\acy(p,q) = \true$ if and only if
    $|L| + \block(p \sqcup q) - (\block(p)+\block(q))=0$.
\end{definition}

As observed in~\cite{tcs/BergougnouxK19}, if $F_p=(L,E_p)$ and $F_q=(L,E_q)$ are forests with components
$p=\cc(F_p)$ and $q=\cc(F_q)$ respectively,
then $\acy(p,q) = \true$ if and only if $E_p \cap E_q = \varnothing$ and $(L,E_p \uplus E_q)$ is a forest.
Furthermore, the following hold.

\begin{lemma}[{\cite[Fact~3.2]{tcs/BergougnouxK19}}]\label{fact:par-rep2}
    Let $L$ be a finite set.
    For all partitions $p,q,r \in \Pi(L)$, it holds that
    $\acy(p,q) \land \acy(p \sqcup q,r) \iff
    \acy(q,r) \land \acy(p,q\sqcup r)$.
\end{lemma}


\begin{lemma}[{\cite[Fact~3.3]{tcs/BergougnouxK19}}]\label{fact:join}
    Let $L$ be a finite set.
    Let $q \in \Pi(L)$ and let $X \subseteq L$ such that no subset of $X$ is a block of $q$.
    Then, for each $p \in \Pi(L \setminus X)$,
    the following hold
    \begin{align}
        p_{\uparrow X} \sqcup q = \{L\} &\iff
        p \sqcup q_{\downarrow (L \setminus X)}
        = \{L \setminus X\},\\
        \acy(p_{\uparrow X},q)  &\iff \acy(p,q_{\downarrow (L \setminus X)}).
    \end{align}
\end{lemma}


We modify in this section the operators on weighted partitions defined in~\cite{iandc/BodlaenderCKN15} in order to express our
dynamic programming algorithms in terms of these operators, and also to deal with acyclicity.

\subparagraph{Rmc.}
Let $L$ be a finite set.
For $\mathcal{A} \subseteq \Pi(L) \times \mathbb{N}$, let
\[
    \rmc(\mathcal{A}) =
        \setdef{(p,w) \in \mathcal{A}}{\forall(p,w') \in \mathcal{A}, \, w' \leq w}.
\]
This operator, defined in~\cite{iandc/BodlaenderCKN15},
is used to remove all the partial solutions whose weights are not maximum with respect to their partitions.


\subparagraph{Ac-Join.}
Let $L'$ be a finite set.
For $\mathcal{A} \subseteq \Pi(L) \times \mathbb{N}$ and $\mathcal{B} \subseteq \Pi(L') \times \mathbb{N}$,
we define $\acjoin(\mathcal{A},\mathcal{B}) \subseteq \Pi(L \cup L') \times \mathbb{N}$ as
% {\small
\[
    \acjoin(\mathcal{A},\mathcal{B}) =
        \setdef{(p_{\uparrow L'}\sqcup q_{\uparrow L},w_1+w_2)}
            {(p,w_1) \in \mathcal{A}, \, (q,w_2) \in \mathcal{B}, \textrm{ and }
            \acy(p_{\uparrow L'},q_{\uparrow L})}.
\]%}
This operator is more or less the same as the one in~\cite{iandc/BodlaenderCKN15},
except that we incorporate the acyclicity condition.
It is used to construct partial solutions while guaranteeing the acyclicity.

\subparagraph{Project.}
For $X \subseteq L$ and $\mathcal{A} \subseteq \Pi(L) \times \mathbb{N}$,
let $\proj(\mathcal{A},X) \subseteq \Pi(L \setminus X) \times \mathbb{N}$ be
\[
    \proj(\mathcal{A},X) = \setdef{(p_{\downarrow (L \setminus X)},w)}
        {(p,w) \in \mathcal{A} \textrm{ and } \forall p_i \in p, \, (p_i \setminus X) \neq \varnothing}.
\]
This operator considers all the partitions such that no block is completely contained in $X$,
and then removes $X$ from those partitions.
We index our dynamic programming tables with functions that inform on the label classes playing
a role in the connectivity of partial solutions, and this operator is used to remove from the
partitions the label classes that are required to no longer play a role in the connectivity of
the partial solutions.
If a partition has a block fully contained in $X$, it means that this block will remain
disconnected in the future steps of our dynamic programming algorithm, and that is why
we remove such partitions (besides those with cycles).

\bigskip

One needs to perform the above operations efficiently, and this is guaranteed by the following,
which assumes that $\log(|\mathcal{A}|) \le |L|^{\bO(1)}$ for each
$\mathcal{A} \subseteq \Pi(L) \times \mathbb{N}$ (this can be established by applying the operator $\rmc$).

\begin{proposition}[{\cite[Proposition~3.4]{tcs/BergougnouxK19}}]\label{prop:op}
    The operator $\acjoin$ can be performed in time $|\mathcal{A}| \cdot |\mathcal{B}| \cdot |L \cup L'|^{\bO(1)}$
    and the size of its output is upper-bounded by $|\mathcal{A}| \cdot |\mathcal{B}|$.
    The operators $\rmc$ and $\proj$ can be performed in time $|\mathcal{A}| \cdot |L|^{\bO(1)}$,
    and the sizes of their outputs are upper-bounded by $|\mathcal{A}|$.
\end{proposition}

We now define the notion of representative sets of weighted partitions which is the same as the one in~\cite{iandc/BodlaenderCKN15},
except that we need to incorporate the acyclicity condition as for the $\acjoin$ operator above.


\begin{definition}\label{defn:par-rep}
    Let $L$ be a finite set and let $\mathcal{A} \subseteq \Pi(L) \times \mathbb{N}$.
    For $q \in \Pi(L)$, let
    \[
        \acopt(\mathcal{A},q) = \max \setdef{w}{(p,w) \in \mathcal{A}, \, p \sqcup q = \{L\}, \textrm{ and } \acy(p,q)}.
    \]
    A set of weighted partitions $\mathcal{A}' \subseteq \Pi(L) \times \mathbb{N}$ \emph{ac-represents} $\mathcal{A}$ if
    for each $q \in \Pi(L)$, it holds that $\acopt(\mathcal{A},q)=\acopt(\mathcal{A}',q)$.

    Let $Z$ and $L'$ be two finite sets.
    A function $f \colon 2^{\Pi(L) \times \mathbb{N}} \times Z \to 2^{\Pi(L') \times \mathbb{N}}$ is said to
    \emph{preserve ac-representation} if for each
    $\mathcal{A},\mathcal{A}' \subseteq \Pi(L) \times \mathbb{N}$ and $z \in Z$,
    it holds that $f(\mathcal{A}',z)$ ac-represents $f(\mathcal{A},z)$ whenever $\mathcal{A}'$ ac-represents $\mathcal{A}$.
\end{definition}


At each step of our algorithm, we will compute a small set $\mathcal{S}'$ that ac-represents the set $\mathcal{S}$ containing all the partial solutions.
In order to prove that we compute an ac-representative set of $\mathcal{S}$,
we show that $\mathcal{S} = f(\mathcal{R}_1,\ldots,\mathcal{R}_t)$ with $f$ a composition of functions that preserve ac-representation,
and $\mathcal{R}_1,\ldots,\mathcal{R}_t$ the sets of partials solutions associated with the previous steps of the algorithm.
To compute $\mathcal{S}'$, it is sufficient to compute $f(\mathcal{R}'_1,\ldots,\mathcal{R}_t')$,
where each $\mathcal{R}_i'$ is an ac-representative set of $\mathcal{R}_i$.
The following lemma guarantees that the operators we use preserve ac-representation.

\begin{lemma}[{\cite[Lemma~3.6]{tcs/BergougnouxK19}}]\label{lemma:acyclic:cw:framework:ac_preserve}
    The union of two sets in $2^{\Pi(L) \times \mathbb{N}}$ and the operators
    $\rmc$, $\proj$, and $\acjoin$ preserve ac-representation.
\end{lemma}


Lastly, it holds that, for every set $\mathcal{A} \subseteq \Pi(L) \times \mathbb{N}$, we can find,
in time $|\mathcal{A}| \cdot 2^{\bO(|L|)}$, a subset $\mathcal{A}' \subseteq \mathcal{A}$ of size at
most $|L| \cdot 2^{|L|}$ that ac-represents $\mathcal{A}$.
The constant $\omega$ denotes the matrix multiplication exponent.

\begin{theorem}[{\cite[Theorem~3.8]{tcs/BergougnouxK19}}]\label{thm:reduce1}
    There exists an algorithm $\acreduce$ that,
    given a set of weighted partitions $\mathcal{A} \subseteq \Pi(L) \times \mathbb{N}$,
    outputs in time $|\mathcal{A}| \cdot 2^{(\omega-1) \cdot |L|} \cdot |L|^{\bO(1)}$
    a subset $\mathcal{A}'$ of $\mathcal{A}$ that ac-represents $\mathcal{A}$ and
    such that $|\mathcal{A}'| \leq |L| \cdot 2^{|L|-1}$.
\end{theorem}

\end{toappendix}
