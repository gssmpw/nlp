\subsection{Parameterization by Clique-width}\label{subsec:acyclic:cw}

Before describing our algorithm, we first define some necessary notions.

% \begin{toappendix}

\subparagraph{Partitions.}
A \emph{partition} $p$ of a set $L$ is a collection of non-empty subsets of $L$
that are pairwise non-intersecting and such that $\bigcup_{p_i \in p} p_i = L$;
each set in $p$ is called a \emph{block} of $p$.
The set of partitions of a finite set $L$ is denoted by $\Pi(L)$,
and $(\Pi(L),\sqsubseteq)$ forms a lattice where $p \sqsubseteq q$
if for each block $p_i$ of $p$ there is a block $q_j$ of $q$ with
$p_i \subseteq q_j$.
The join operation of this lattice is denoted by $\sqcup$.
For example, we have $\{ \{ 1,2\},\{3,4\},\{5\}\} \sqcup \{\{1\},\{2,3\},\{4\},\{5\}\}= \{ \{ 1,2,3,4\}, \{5\}\}$.
Let $\block(p)$ denote the number of blocks of a partition $p$.
Observe that $\varnothing$ is the only partition of the empty set.
A \emph{weighted partition} is an element of $\Pi(L) \times \mathbb{N}$ for some finite set $L$.
For $p \in \Pi(L)$ and $X \subseteq L$, let $p_{\downarrow X} \in \Pi(X)$ be the partition
$\setdef{p_i \cap X}{p_i \in p} \setminus \{\varnothing\}$,
and for a set $Y$, let $p_{\uparrow Y} \in \Pi(L \cup Y)$ be the partition
$p \cup \left(\bigcup_{y \in Y \setminus L} \{\{y\}\}\right)$.

% \end{toappendix}


\begin{theoremrep}[\appsymb]
    There is an algorithm that,
    given an integer $\ell$ as well as a graph $G$ along with an irredundant clique-width expression $\psi$ of $G$ of width $\cw$,
    determines whether $G$ has an acyclic matching of size at least $\ell$ in time $2^{\bO(\cw)} n^{\bO(1)}$.
\end{theoremrep}

\begin{proof}
    The proof follows along the lines of the $2^{\bO(\cw)} n^{\bO(1)}$ algorithm for {\FVS} by Bergougnoux and Kant\'e~\cite{tcs/BergougnouxK19}.
    There, the authors develop a general framework, expanding upon the one of Bodlaender et al.~\cite{iandc/BodlaenderCKN15},
    in order to cope with various problems with connectivity constraints on graphs of bounded clique-width.
    Using their framework, we can perform DP over the clique-width expression and, starting from a partial solution,
    eventually build an optimal one.

    Instead of \AcyclicM, we solve the equivalent problem of asking,
    given a graph $G$ and an integer $k$, whether there exists $S \subseteq V(G)$ of size $|S| \ge k$ such that
    $G[S]$ is a forest that contains a perfect matching.
    Notice that $G$ has an acyclic matching of size $\ell$ if and only if there exists such a set $S$
    of size $|S| \ge 2\ell$.
    In the following let $\mathcal{H}$ denote the set of all $\cw$-labeled graphs generated
    by a subexpression of the given clique-width expression $\psi$ of the input graph $G$.

    For every $H \in \mathcal{H}$ and for every subset $L \subseteq \setdef{i \in [\cw]}{\lab^{-1}_H(i) \neq \varnothing}$,
    we compute a set of weighted partitions $\mathcal{A} \subseteq \Pi(L) \times \mathbb{N}$.
    Each weighted partition $(p,w) \in \mathcal{A} \subseteq \Pi(L) \times \mathbb{N}$ is intended to mean that
    there is a partial solution whose vertices $S \subseteq V(H)$ have weight $w$ and $S \cap \lab^{-1}_H(i) \neq \varnothing$
    for all labels $i \in L$ and $p$ is the transitive closure of the following equivalence relation $\sim$ on $L$:
    $i \sim j$ if there exist an $i$-vertex and a $j$-vertex in the same component of $H[S]$.
    Intuitively, for each label $i$ in $L$, we expect the $i$-vertices of $S$ to have an additional neighbor in any
    extension of $S$ into an optimum solution, therefore, we can consider all vertices of $S \cap \lab^{-1}_H(i)$ as
    one vertex in terms of connectivity.
    On the other hand, the vertices of $S \cap \lab^{-1}_H(j)$, where $j \in [\cw] \setminus L$ and $S$ contains at least one $j$-vertex,
    are expected to have no additional neighbor in any extension of $S$ into an optimum solution.
    Consequently, those vertices no longer play a role in the connectivity of the solution.
    These expectations allow us to represent the connected components of $H[S]$ by $p$.
    Our algorithm will guarantee that the weighted partitions computed from $(p,w)$ are computed accordingly to these expectations.
    It remains to deal with the acyclicity constraint, and to do this,
    we need to certify that whenever we join two weighted partitions and keep the result as a partial solution,
    it does not correspond to a partial solution with cycles.
    To this end, we employ the machinery introduced by~\cite{tcs/BergougnouxK19} and is based on~\cite{iandc/BodlaenderCKN15},
    which, for the sake of completeness, we present in \cref{sec:acyclic:cw:framework}.

    We use the weighted partitions defined in~\cite{tcs/BergougnouxK19} to represent the partial solutions.
    At each step of our algorithm we will ensure that the stored weighted partitions correspond to acyclic partial solutions.
    Since the framework of~\cite{tcs/BergougnouxK19} deals only with \emph{connected} acyclic solutions,
    as in~\cite{tcs/BergougnouxK19}, we introduce a hypothetical new vertex $v_0$ that is universal,
    and we compute a pair $(F,E_0)$ so that $F$ is a maximum induced forest of $G$ that has a perfect matching,
    $E_0$ is a subset of edges incident to $v_0$,
    and $(V(F) \cup \{v_0\}, E(F) \cup E_0)$ is a tree.

    We start with some definitions and notation.
    Let $H \in \mathcal{H}$, and consider a subset of its vertices $S \subseteq V(H)$ and
    a matching $M \subseteq E(H)$ with $V_M \subseteq S$, that is, all the vertices incident to
    edges in $M$ belong to $S$.
    In that case, we say that a vertex $v \in S$ is \emph{unsaturated} in $(S,M)$ if $v \notin V_M$,
    otherwise, i.e., if $v \in V_M$, we say that it is \emph{saturated} in $(S,M)$ instead.
    We say that such a pair $(S,M)$ is a \emph{partial solution} of $H$ if $H[S]$ is acyclic.
    Finally, we say that a partial solution $(S,M)$ of $H$ is \emph{extensible} if there exists an acyclic matching $M^*$ of
    $G$ such that $V_{M^*} \cap V(H) = S$ and $M^* \supseteq M$.

    \begin{claim}\label{claim:acyclic:cw:algorithm}
        Let $(S,M)$ be a partial solution of $H$ that is extensible,
        and let $M^*$ be an acyclic matching of $G$ such that
        $V_{M^*} \cap V(H) = S$ and $M^* \supseteq M$.
        Then, for all $i \in [\cw]$, the following hold.
        \begin{enumerate}
            \item There exists at most one unsaturated vertex of label $i$ in $(S,M)$,
            that is, $|(S \setminus V_M) \cap \lab^{-1}_H(i)| \le 1$.

            \item If $|(S \setminus V_M) \cap \lab^{-1}_H(i)| = 1$,
            then each vertex of $S \cap \lab^{-1}_H(i)$ belongs to a distinct connected component of $H[S]$.

            \item If $|S \cap \lab^{-1}_H(i)| \ge 2$,
            then the vertices of $S \cap \lab^{-1}_H(i)$ take part in at most one join operation
            with a non-empty set,
            that is, there exists at most one graph $H_2$ further in $\psi$
            such that $H_2 = \eta_{i',j'}(H_1)$ with $S \cap \lab^{-1}_H(i) = S \cap \lab^{-1}_{H_1}(i')$
            and $V_{M*} \cap \lab^{-1}_{H_1}(j') \neq \varnothing$.
        \end{enumerate}
    \end{claim}

    \begin{claimproof}
        First notice that for all $H \in \mathcal{H}$ it holds that $H[V_{M^*} \cap V(H)]$ is a subgraph of $G[V_{M^*}]$,
        therefore $H[V_{M^*} \cap V(H)]$ is acyclic.

        Towards a contradiction, let $v_1,v_2 \in (S \setminus V_M) \cap \lab^{-1}_H(i)$,
        and let $u_1,u_2 \in V_{M^*}$ such that $\{v_1,u_1\}, \{v_2,u_2\} \in M^*$.
        There are two cases.
        In the first case, at some point further in the clique-expression there exists a single join operation
        that saturates both $v_1$ and $v_2$, that is,
        there exist $H_1,H_2 \in \mathcal{H}$ and $H_2 = \eta_{i',j'}(H_1)$,
        with $v_1,v_2 \in \lab^{-1}_{H_1}(i')$ and $u_1,u_2 \in \lab^{-1}_{H_1}(j')$.
        Then however $H_2[S]$ contains a cycle, which contradicts the acyclicity of $H_2[V_{M^*} \cap V(H_2)]$
        since $S \subseteq V_{M^*} \cap V(H_2)$.
        In the second case, $v_1$ and $v_2$ are saturated by different join operations.
        This means however that after the first one, they are in the same connected component,
        thus the second join operation results in a cycle, again leading to a contradiction.
        The argument for the third item of the statement is analogous.

        Similarly, one can show that if $|(S \setminus V_M) \cap \lab^{-1}_H(i)| = 1$
        and there exist two vertices of $S \cap \lab^{-1}_H(i)$ belonging to the same connected component of $H[S]$,
        then the join operation that saturates $v$ results in a cycle, thus leading to a contradiction.
    \end{claimproof}


    We proceed by dynamic programming, aiming to construct in a bottom-up fashion all extensible partial solutions.
    In order to reduce the sizes of the sets stored in the entries of the DP tables,
    we will express the steps of the algorithm in terms of the operators on weighted partitions defined in~\cite{tcs/BergougnouxK19}.
    Let $H \in \mathcal{H}$.
    We are interested in storing \emph{ac-representative sets} of all partial solutions $(S,M)$ of $H$ that may produce a solution,
    thus we assume that the stored partial solutions satisfy the properties of \cref{claim:acyclic:cw:algorithm}.
    Consequently, it holds that if $S$ contains at least $2$ $i$-vertices, then the vertices of this label may
    partake in \emph{at most one} join operation with a non-empty set further in the clique-expression.
    Let $J \subseteq [\cw]$ denote the set of all labels from which $S$ contains at least $2$ vertices,
    and consider a partition $J_1 \uplus J_2 = J$ such that for the labels in $J_1$ there exists further in the clique-expression
    a join operation with a non-empty set while for those in $J_2$ it does not exist.
    Notice that in terms of connectivity, we can treat all vertices of $S \cap \lab^{-1}_H(i)$ for $i \in J_1$
    as one vertex, as they will all eventually end up in the same connected component of the final solution.
    On the other hand, for the vertices labeled $j \in J_2$,
    we can take into account how they affect the connectivity of the graph and subsequently ignore them,
    as no other edges will be added incident to them.
    We remark that if $J_2 = \varnothing$,
    then we can encode the connectivity of the graph by storing the partition $p \in \Pi([\cw])$ where
    $i$ and $j$ are in the same block if there are
    an $i$-vertex and a $j$-vertex in the same connected component of $H[S]$.



    Now consider the case where $H_1,H_2 \in \mathcal{H}$ with $H_2 = \eta_{i,j}(H_1)$ and $(S,M)$ is a partial solution of $H_1$,
    where $i,j \in J \subseteq [\cw]$.
    Notice that this join operation might result in cycles forming in $H_2[S]$, e.g.,
    whenever an $i$-vertex and a $j$-vertex are non-adjacent and belong to the same connected component of $H_1[S]$,
    or the number of $i$-vertices and $j$-vertices are both at least $2$ in $S$.
    Unfortunately, we are not able to handle all these cases with the operators on weighted partitions.
    To resolve the situation where an $i$-vertex and a $j$-vertex are already adjacent,
    we consider \emph{irredundant} $\cw$-expressions, i.e., whenever an operation $\eta_{i,j}$ is used there are no edges
    between $i$-vertices and $j$-vertices.
    For the other cases, we index the DP tables with total functions $s \colon [\cw] \to \Gamma$ called \emph{signatures}, that indicate, for each label $i \in [\cw]$, whether $\lab_H^{-1}(i)$ intersects $S$ and $V_M$,
    with $\Gamma = \{\gamma_0,\gamma_{\tilde{1}},\gamma_1,\gamma_{\tilde{2}},\gamma_2,\gamma_{-2}\}$.
    In particular, we say that a partial solution $(S,M)$ of $H$ is \emph{compatible} with the signature $s$ if for all $i \in [\cw]$
    it holds that
    \begin{itemize}
        \item if $s(i) = \gamma_0$, then $S \cap \lab^{-1}_H(i) = \varnothing$,

        \item if $s(i) = \gamma_{\tilde{1}}$, then $S \cap \lab^{-1}_H(i) = \{v\}$ and $v \notin V_M$,

        \item if $s(i) = \gamma_1$, then $S \cap \lab^{-1}_H(i) = \{v\}$ and $v \in V_M$,

        \item if $s(i) = \gamma_{\tilde{2}}$, then $|S \cap \lab^{-1}_H(i)| \ge 2$ and $|(S \setminus V_M) \cap \lab^{-1}_H(i)| = 1$,

        \item if $s(i) \in \{\gamma_2, \, \gamma_{-2}\}$, then $|S \cap \lab^{-1}_H(i)| \ge 2$ and $(S \setminus V_M) \cap \lab^{-1}_H(i) = \varnothing$.
    \end{itemize}
    Defining the signatures like so allows us to resolve the problem with the sets $J_1$ and $J_2$ by forcing each label
    $i$ with $s(i) = \gamma_2$ to wait for \emph{exactly} one clique-width operation $\eta_{i,\ell}$ for some $\ell \in [\cw]$ with $s(\ell) \neq \gamma_0$,
    while any label $j$ with $s(j) = \gamma_{-2}$ is forbidden to partake in any such operation.
    Notice that by definition, any label class $i$ with $s(i) = \gamma_{\tilde{2}}$ is also forced to wait for exactly $1$ such join operation,
    as no vertex is unsaturated in $G$.
    Thus, we translate all the acyclicity tests to the $\acjoin$ operation.
    The following notion of \emph{certificate graph} formalizes this requirement,
    where the vertices in $V^+_s$ represent the expected future neighbors of all the vertices in
    $S \cap \lab^{-1}_H(s^{-1}(\{\gamma_{\tilde{2}},\gamma_2\}))$.




    \begin{definition}[Certificate graph of a solution]\label{defn:certif}
        Let $H \in \mathcal{H}$, $F$ an induced forest of $H$,
        $s \colon [\cw] \to \Gamma$,
        and $E_0$ a subset of edges incident to $v_0$.
        Let $V^+_s = \setdef{v^+_i}{i \in s^{-1}(\{\gamma_{\tilde{2}},\gamma_2\})}$.
        The \emph{certificate graph of $(F,E_0)$ with respect to $s$},
        denoted by $\CG(F,E_0,s)$,
        is the graph $(V(F) \cup V_s^+ \cup \{v_0\}, E(F) \cup E_0 \cup E_s^+)$ with
        \[
            E_s^+ = \bigcup_{i \in s^{-1}(\{\gamma_{\tilde{2}},\gamma_2\})} \setdef{\{v,v^+_i\}}{v \in V(F) \cap \lab^{-1}_H(i)}.
        \]
    \end{definition}

    We are now ready to define the sets of weighted partitions whose representatives we manipulate in our dynamic programming tables.

    \begin{definition}[{Weighted partitions in $\mathcal{A}_H[s]$}]\label{defn:tabfvs}
        Let $H \in \mathcal{H}$ and consider a signature $s \colon [\cw] \to \Gamma$.
        The entries of $\mathcal{A}_H[s]$ are all weighted partitions
        $(p,w) \in \Pi(s^{-1}(\{\gamma_{\tilde{1}},\gamma_1,\gamma_{\tilde{2}},\gamma_2\}) \cup \{v_0\}) \times \mathbb{N}$
        such that there exist a partial solution $(S,M)$ of $H$ and $E_0 \subseteq \setdef{\{v_0 v\}}{v \in S}$
        so that $\wc (S) = w$, and
        \begin{enumerate}
            \item $(S, M)$ is compatible with $s$,

            \item the certificate graph $\CG(H[S],E_0,s)$ is a forest,

            \item each connected component of $\CG(H[S],E_0,s)$ has at least one vertex in
            $\lab^{-1}_H(s^{-1}(\{\gamma_{\tilde{1}},\gamma_1,\gamma_{\tilde{2}},\gamma_2\})) \cup \{v_0\}$,

            \item the partition $p$ equals $(s^{-1}(\{\gamma_{\tilde{1}},\gamma_1,\gamma_{\tilde{2}},\gamma_2\}) \cup \{v_0\})/\sim$,
            where $i \sim j$ if and only if a vertex in $S \cap \lab^{-1}_H(i)$ is connected, in $\CG(H[S],E_0,s)$,
            to a vertex in $S \cap \lab^{-1}_H(j)$; we consider $\lab^{-1}_H(v_0)=\{v_0\}$.
        \end{enumerate}
    \end{definition}
    %
    Conditions (2) and (4) guarantee that $(S \cup \{v_0\},E(H[S]) \cup E_0)$ can be extended into a tree, if any.
    They also guarantee that cycles detected through the $\acjoin$ operation correspond to cycles,
    and each cycle can be detected with it.
    In the following we call any quadruple $(S,M,E_0,(p,\wc(S)))$ a \emph{candidate solution} in $\mathcal{A}_H[s]$ if Condition (4) is satisfied,
    and if in addition Conditions (1)-(3) are satisfied, we call it a \emph{solution} in $\mathcal{A}_H[s]$.

    In that case, the size of a maximum acyclic matching of $G$ corresponds to the maximum,
    over all signatures $s \colon [\cw] \to \Gamma$
    with $s^{-1}(\{\gamma_{\tilde{1}},\gamma_{\tilde{2}},\gamma_{2}\}) = \varnothing$,
    of $\max \setdef{w}{(\{ s^{-1}(\gamma_{1}) \cup \{v_0\} \},w) \in \mathcal{A}_G[s]}$.
    Indeed, by definition, if $(\{ s^{-1}(\gamma_{1}) \cup \{v_0\} \},w)$ belongs to $\mathcal{A}_{G}[s]$,
    then there exists a partial solution $(S,M)$ of $G$ with $V_M=S$ and $\wc(S)=w$,
    as well as a set $E_0$ of edges incident to $v_0$ such that $(S \cup \{v_0\},E(H[S]) \cup E_0)$ is a tree.
    This follows from the fact that if $s^{-1}(\{\gamma_{\tilde{2}},\gamma_{2}\}) = \varnothing$,
    then we have $\CG(H[S],E_0,s) = (S \cup \{v_0\}, E(H[S]) \cup E_0)$.

    Our algorithm will store, for each $H \in \mathcal{H}$ and each $s \colon [\cw] \to \Gamma$,
    an ac-representative set $\DPt_H[s]$ of $\mathcal{A}_H[s]$.
    We are now ready to give the different steps of the algorithm, depending on the clique-width operations.
    Recall that $f[\alpha \mapsto \beta]$ denotes the function obtained from $f$
    that maps $\alpha$ to $\beta$ instead of $f(\alpha)$.

    \proofsubparagraph{Singleton $H = i(v)$.}
    For $H = i(v)$, notice that $\lab^{-1}_H(i) = \{v\}$ and $\lab^{-1}_H(w) = \varnothing$ for all $w \in [\cw] \setminus \{i\}$.
    Consequently, for $s \colon [\cw] \to \Gamma$ we set the following values,
    where $s_0$ is the signature such that $s_0(w)=\gamma_0$ for all $w \in [\cw]$.
    \[
        \DPt_H[s] =
            \begin{cases}
                \bigl\{ \bigl( \{\{v_0\} \} ,0 \bigr) \bigr\}                                                   &\text{if $s = s_0$,}\\
                \bigl\{ \bigl( \{\{i,v_0\}\}, \wc(v) \bigr), \bigl( \{\{i\},\{v_0\}\}, \wc(v) \bigr) \bigr\}    &\text{if $s = s_0 [i \mapsto \gamma_{\tilde{1}}]$,}\\
                \varnothing                                                                                     &\text{otherwise.}
            \end{cases}
    \]
    %
    Since $|V(H)|=1$, there is no partial solution $(S,M)$ intersecting $\lab^{-1}_H(i)$ on at least two vertices while $H$ has no edges,
    so the set of weighted partitions satisfying \cref{defn:tabfvs} equals the empty set for $s(i) \neq \{\gamma_0,\gamma_{\tilde{1}}\}$.
    If $s(i) = \gamma_{\tilde{1}}$, there are two possibilities, depending on whether $E_0 = \varnothing$ or
    $E_0 = \{v v_0\}$.
    We can thus conclude that $\DPt_H[s] = \mathcal{A}_H[s]$ is correctly computed.

    \proofsubparagraph{Joining labels with edges, $H = \eta_{i,j}(H')$.}
    Before describing how to populate the table in this case,
    we first define a helper function%
    \footnote{We omit the use of braces in case of singletons for the sake of readability.}
    $f \colon \Gamma \times \Gamma \to 2^{\Gamma \times \Gamma}$
    as follows, where the row denotes the first input and the column the second:
    \[
        \begin{array}{r|cccccc}
            f 	                & \gamma_0 							& \gamma_{\tilde{1}} 																			& \gamma_1 							& \gamma_{\tilde{2}} 				& \gamma_2 							& \gamma_{-2} \\
            \hline
            \gamma_0 			& (\gamma_0, \gamma_0) 				& (\gamma_0, \gamma_{\tilde{1}}) 																& (\gamma_0, \gamma_1) 				& (\gamma_0, \gamma_{\tilde{2}}) 	& (\gamma_0, \gamma_2) 				& (\gamma_0, \gamma_{-2}) \\
            \gamma_{\tilde{1}} 	& (\gamma_{\tilde{1}}, \gamma_0) 	& \left\{\makecell{(\gamma_{\tilde{1}}, \gamma_{\tilde{1}}) \\ (\gamma_1,\gamma_1)} \right\} 	& (\gamma_{\tilde{1}}, \gamma_1) 	& (\gamma_1, \gamma_{-2}) 			& (\gamma_{\tilde{1}}, \gamma_{-2}) & \varnothing \\
            \gamma_1 			& (\gamma_1, \gamma_0) 				& (\gamma_1, \gamma_{\tilde{1}}) 																& (\gamma_1, \gamma_1) 				& \varnothing 								& (\gamma_1, \gamma_{-2}) 	& \varnothing \\
            \gamma_{\tilde{2}} 	& (\gamma_{\tilde{2}}, \gamma_0)	& (\gamma_{-2}, \gamma_1) 																		& \varnothing 						& \varnothing 								& \varnothing 				& \varnothing \\
            \gamma_2 			& (\gamma_2, \gamma_0) 				& (\gamma_{-2}, \gamma_{\tilde{1}}) 															& (\gamma_{-2}, \gamma_1) 			& \varnothing 								& \varnothing 				& \varnothing \\
            \gamma_{-2}			& (\gamma_{-2}, \gamma_0) 			& \varnothing																					& \varnothing 						& \varnothing 								& \varnothing 				& \varnothing
        \end{array}
    \]
    In the following, let $f^{-1}(\alpha, \beta) = \setdef{(\alpha', \beta') \in \Gamma \times \Gamma}{(\alpha,\beta) \in f(\alpha', \beta')}$.
    For $s \colon [\cw] \to \Gamma$ we set
    \[
        \DPt_H[s] =
            \begin{cases}
                \DPt_{H'}[s]		            &\text{if $\gamma_0 \in \{s(i), \, s(j)\}$,}\\
                \varnothing			            &\text{if $f^{-1}(s(i),s(j)) = \varnothing$,}\\
                \acreduce ( \rmc(\mathcal{A}) ) &\text{otherwise,}
            \end{cases}
    \]
    where
    \[
        \mathcal{A} = \bigcup_{(\alpha',\beta') \in f^{-1}(s(i),s(j))}
            \proj \left(
                \acjoin\left( \DPt_{H'} \Bigl[ s[i \mapsto \alpha'][j \mapsto \beta'] \Bigr], \,
                \{ ( \{\{i,j\}\},0)\}\right), \,
                s^{-1}(\gamma_{-2}) \cap \{i,j\}
            \right).
    \]

    Notice that in the first case we do not need to use the operators $\acreduce$ and $\rmc$ since we update
    $\DPt_H[s]$ with one table from $\DPt_{H'}$.
    The second case can only be when (i) $s(i)=s(j)=\gamma_{-2}$,
    or (ii) $\gamma_0 \notin \{s(i),s(j)\}$ and
    $\{\gamma_{\tilde{2}},\gamma_{2}\} \cap \{s(i),s(j)\} \neq \varnothing$.
    As for the third case, we have that $s(i),s(j) \in \{\gamma_{\tilde{1}}, \gamma_{1}, \gamma_{-2}\}$ and
    either $s(i) \neq \gamma_{-2}$ or $s(j) \neq \gamma_{-2}$.
    Intuitively, we consider the weighted partitions $(p,w) \in \mathcal{A}$ such that
    $i$ and $j$ belong to different blocks of $p$,
    we merge the blocks containing $i$ and $j$,
    remove the elements in $s^{-1}(\gamma_{-2}) \cap \{i,j\}$ from the resulting block,
    and add the resulting weighted partition to $\DPt_H[s]$.
    Notice that we also have to consider whether a new edge has been added to the partial solution due to the join operation.


    \proofsubparagraph{Relabeling, $H = \rho_{i \to j}(H')$.}
    As before, we first define a helper function
    $g \colon \Gamma \times \Gamma \to 2^{\Gamma}$
    as follows, where the row denotes the first input and the column the second:
    \[
        \begin{array}{r|cccccc}
            g 					& \gamma_0 				& \gamma_{\tilde{1}} 	& \gamma_1 												& \gamma_{\tilde{2}} 	& \gamma_2 				& \gamma_{-2} \\
            \hline
            \gamma_0 			& \gamma_0 				& \gamma_{\tilde{1}} 	& \gamma_1 												& \gamma_{\tilde{2}} 	& \gamma_2 				& \gamma_{-2} \\
            \gamma_{\tilde{1}} 	& \gamma_{\tilde{1}} 	& \varnothing 			& \gamma_{\tilde{2}} 									& \varnothing 			& \gamma_{\tilde{2}} 	& \varnothing \\
            \gamma_1 			& \gamma_1				& \gamma_{\tilde{2}}	& \left\{ \makecell{ \gamma_2 \\ \gamma_{-2}} \right\}	& \gamma_{\tilde{2}} 	& \gamma_2				& \gamma_{-2} \\
            \gamma_{\tilde{2}} 	& \gamma_{\tilde{2}}	& \varnothing 			& \gamma_{\tilde{2}} 									& \varnothing 			& \gamma_{\tilde{2}} 	& \varnothing \\
            \gamma_2 			& \gamma_2 				& \gamma_{\tilde{2}} 	& \gamma_2 												& \gamma_{\tilde{2}}	& \gamma_2 				& \varnothing \\
            \gamma_{-2}			& \gamma_{-2} 			& \varnothing			& \gamma_{-2} 											& \varnothing 			& \varnothing 			& \gamma_{-2}
        \end{array}
    \]
    In the following, let $\gminusrest(\beta) = \setdef{(\alpha', \beta') \in \Gamma \times \Gamma}{\beta \in f(\alpha', \beta') \text{ and } \gamma_0 \notin \{ \alpha', \beta' \}}$.
    For $s \colon [\cw] \to \Gamma$ we set
    \[
        \DPt_H[s] =
            \begin{cases}
                \acreduce ( \rmc(\mathcal{A}_1 \cup \mathcal{A}_2 \cup \mathcal{A}_3 ) )    &\text{if $s(i) = \gamma_0$,}\\
                \varnothing                                                                 &\text{otherwise},
            \end{cases}
    \]
    where we obtain $\mathcal{A}_1,\mathcal{A}_2,\mathcal{A}_3$ as follows.


    $\mathcal{A}_1$ contains all weighted partitions corresponding to partial solutions not intersecting $\lab^{-1}_{H'}(i)$,
    which are trivially partial solutions for $H$ as well, and we have
    \[
        \mathcal{A}_1 = \DPt_{H'} \Bigl[ s[i \mapsto \gamma_0] \Bigr].
    \]

    $\mathcal{A}_2$ contains all weighted partitions corresponding to partial solutions that
    intersect $\lab^{-1}_{H'}(i)$ but not $\lab^{-1}_{H'}(j)$.
    Notice that the case where a partial solution intersects neither,
    i.e., $s(i) = s(j) = \gamma_0$, is already covered by $\mathcal{A}_1$.
    We set
    \[
        \mathcal{A}_2 =
            \begin{cases}
                \varnothing         &\text{if $s(j) = \gamma_0$,}\\
                \DPt_{H'}[s_{H'}]        &\text{if $s(j) = \gamma_{-2}$},\\
                \proj\big(
                    \acjoin( \DPt_{H'}[s_{H'}], \,
                        \{ ( \{\{i,j\}\},0) \}), \,
                    \{i\}
                \big)               &\text{otherwise},
            \end{cases}
    \]
    where $s_{H'} = s[i \mapsto s(j)][j \mapsto \gamma_0]$.

    The set $\mathcal{A}_3$ has to do with the cases where the partial solution intersects
    both $\lab^{-1}_{H'}(i)$ and $\lab^{-1}_{H'}(j)$, which implies that $s(j) \in \{ \gamma_{\tilde{2}}, \gamma_2, \gamma_{-2}\}$.
    To this end, we consider the following subcases.
    If $s(j) = \gamma_{-2}$, then
    \[
        \mathcal{A}_3 = \bigcup_{(\alpha',\beta') \in \gminusrest(\gamma_{-2})}
                \proj\Big( \DPt_{H'} \Bigl[ s[i \mapsto \alpha'][j \mapsto \beta'] \Bigr] , \, \{i,j\} \Big).
    \]
    On the other hand, if $s(j) \in \{ \gamma_{\tilde{2}}, \gamma_2 \}$, then
    \[
        \mathcal{A}_3 = \bigcup_{(\alpha',\beta') \in \gminusrest(s(j))}
                \proj\bigg(
                    \acjoin \Big(
                        \DPt_{H'} \Bigl[ s[i \mapsto \alpha'][j \mapsto \beta'] \Bigr], \,
                        \{ ( \{\{i,j\}\},0) \} \Big), \,
                    \{i\}
                \bigg).
    \]

    \proofsubparagraph{Disjoint union, $H = H_1 \oplus H_2$.}
    Consider the signatures $s, s_1, s_2$.
    We say that $s_1,s_2$ \emph{agree on $s$} if for all labels $i \in [\cw]$,
    it holds that $s(i) \in g(s_1(i), s_2(i))$.
    In that case, for $s \colon [\cw] \to \Gamma$ we set $\DPt_H[s] = \acreduce(\rmc(\mathcal{A}))$,
    where
    \[
        \mathcal{A} = \bigcup_{s_1,s_2 \text{ agree on } s}
            \acjoin \bigg(
                \proj \Big( \DPt_{H_1}[s_1], \, s^{-1}(\gamma_{-2}) \Big), \,
                \proj \Big( \DPt_{H_2}[s_2], \, s^{-1}(\gamma_{-2}) \Big)
            \bigg).
    \]
    As in~\cite{tcs/BergougnouxK19},
    we need to do the projections before the $\acjoin$ operation.

    The correctness of the algorithm follows along the lines of the {\FVS} algorithm in~\cite{tcs/BergougnouxK19}.
\end{proof}

\begin{proofsketch}
    The proof follows along the lines of the $2^{\bO(\cw)} n^{\bO(1)}$ algorithm for {\FVS} by Bergougnoux and Kant\'e~\cite{tcs/BergougnouxK19}.
    There, the authors develop a general framework, expanding upon the one of Bodlaender et al.~\cite{iandc/BodlaenderCKN15},
    in order to cope with various problems with connectivity constraints on graphs of bounded clique-width.
    Using their framework, we can perform DP over the clique-width expression and, starting from a partial solution,
    eventually build an optimal one.

    Instead of \AcyclicM, we solve the equivalent problem of asking,
    given a graph $G$ and an integer $k$, whether there exists $S \subseteq V(G)$ of size $|S| \ge k$ such that
    $G[S]$ is a forest that contains a perfect matching.
    Notice that $G$ has an acyclic matching of size $\ell$ if and only if there exists such a set $S$
    of size $|S| \ge 2\ell$.
    In the following let $\mathcal{H}$ denote the set of all $\cw$-labeled graphs generated
    by a subexpression of the given clique-width expression $\psi$ of the input graph $G$.

    In order to deal with \FVS, in~\cite{tcs/BergougnouxK19} the authors solve the \textsc{Maximum Induced Tree} problem and
    introduce a \emph{signature} on the partial solutions,
    that encodes (i) the number of vertices that a partial solution contains from each label class,
    and (ii) in case a solution contains at least $2$ vertices of the same label class,
    whether this label class will partake in a join operation with a non-empty set in the future;
    as a matter of fact, the number of such join operations can be at most $1$, as otherwise a cycle is formed.
    For our problem, we have that a partial solution in $H \in \mathcal{H}$
    is a pair $(S,M)$ such that $M$ is a matching of $H$ with $V_M \subseteq S \subseteq V(H)$.
    In that case, the signature of a partial solution is, for each label $i \in [\cw]$,
    whether $S \cap \lab^{-1}_H(i)$ is empty, a singleton, or contains at least $2$ vertices,
    and whether $(S \setminus V_M) \cap \lab^{-1}_H(i)$ is empty or not.
    We notice that in fact $|(S \setminus V_M) \cap \lab^{-1}_H(i)| \le 1$, thus leading to $6$ different states
    $\Gamma = \{\gamma_0, \gamma_{\tilde{1}}, \gamma_1, \gamma_{\tilde{2}}, \gamma_2, \gamma_{-2}\}$
    per label class, with the index denoting the number of vertices in $S \cap \lab^{-1}_H(i)$,
    the tilde denoting that $|(S \setminus V_M) \cap \lab^{-1}_H(i)| = 1$,
    and $\gamma_2$ (resp., $\gamma_{-2}$) denoting that the label class will (resp., will not)
    partake in a join operation in the future.
    This, along with the machinery of the framework of~\cite{tcs/BergougnouxK19},
    allows us to detect any possible cycles as well as forcing the final solution to be a forest containing a perfect matching.
\end{proofsketch}
