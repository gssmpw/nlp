\subsection{Lower Bound}\label{subsec:induced:lb}

We first state the main theorem of this section,
which we then prove in \cref{lem:induced:lb:Induced->CSP,lem:induced:lb:CSP->Induced}.

\begin{theorem}\label{thm:induced:lb}
    For all $\varepsilon > 0$, we have the following:
    there exists an algorithm deciding {\InducedM} in time $\sO((3-\varepsilon)^\pw)$,
    where $\pw$ denotes the pathwidth of the input graph,
    if and only if the pw-SETH is false.
\end{theorem}


\begin{lemmarep}[\appsymb]\label{lem:induced:lb:Induced->CSP}
    If the pw-SETH is false,
    then there exists $\varepsilon > 0$
    and an algorithm that solves {\InducedM} in time $\sO((3-\varepsilon)^\pw)$,
    where $\pw$ denotes the pathwidth of the input graph.
\end{lemmarep}


\begin{proof}
    Recall that in the \textsc{MaxW-CSP} problem we are given given a
    \textsc{CSP} instance of arity $r$ and over an alphabet $\mathcal{B}$,
    as well as a weight function $w \colon \mathcal{B} \to \mathbb{N}$ and an integer $w_t$,
    and we want to decide whether there exists a satisfying assignment
    whose total weight is at least $w_t$.
    The weight of an assignment is defined as the sum of the weights
    of the assignments of its variables.

    Given an instance $(G,\ell)$ of {\InducedM} and a path decomposition of $G=(V,E)$ of width $\pw$,
    we will reduce it to an instance $\psi$ of \textsc{MaxW-CSP} of arity $r = \bO(1)$ and over an alphabet of size $3$,
    while increasing the pathwidth by at most an additive constant.
    If the pw-SETH is false, then by~\cite[Theorem~3.2]{soda/Lampis25} there is a fast algorithm for solving
    $\psi$, which will allow us to obtain the desired running time for \InducedM.

    Suppose we have a nice path decomposition of $G$ of width $\pw$ with the bags numbered
    $B_1, \ldots, B_t$.
    Vertices of $G$ are numbered $V = \{v_1, \ldots, v_n\}$.
    We can assume that there exists an injective function $b \colon E \to [t]$
    which maps each edge of $G$ to the index of a bag that contains both of its endpoints
    ($b$ can be made injective by repeating bags if necessary).
    Furthermore, we can assume that $b$ maps edges to bags which are not introducing new vertices,
    again by repeating bags.

    We will construct a CSP instance $\psi$ over an alphabet $\mathcal{B} = \{0,1,2\}$ as follows:
    \begin{enumerate}
        \item For each $v_i \in V$ we introduce $t$ variables $x_{i,j}$ for $j \in [t]$.

        \item For each $v_i \in V$ and $j \in [t-1]$ we add the following constraints:
        \begin{itemize}
            \item $x_{i,j} \neq 2 \implies x_{i,j+1} = x_{i,j}$,
            \item $x_{i,j} = 2 \implies x_{i,j+1} \neq 0$,
            \item If no edge $e$ has $b(e) = j$ or for some edge $e$ we have
                    $b(e) = j$ but $e$ is not incident on $v_i$,
                    then we also add the constraint $x_{i,j+1}=x_{i,j}$.
        \end{itemize}

        \item For each $v_i \in V$, if the last bag containing $v_i$ is $B_j$,
        then for all $j' \in [j,t-1]$ we add the constraint $x_{i,j'+1} = x_{i,j'}$.

        \item For each $e \in E$ let $j=b(e)$ and $e = \{v_{i_1}, v_{i_2}\}$ with $v_{i_1}, v_{i_2} \in B_j$.
        We add the constraints
        \begin{itemize}
            \item $(x_{i_1,j} = 0) \lor (x_{i_2,j} = 0) \lor (x_{i_1,j} = x_{i_2,j} = 2)$,
            \item $(x_{i_1,j} = 0) \lor (x_{i_2,j} = 0) \implies \big((x_{i_1,j+1} = x_{i_1,j}) \land (x_{i_2,j+1} = x_{i_2,j})\big)$,
            \item $(x_{i_1,j} = x_{i_2,j} = 2) \implies (x_{i_1,j+1} = x_{i_2,j+1} = 1)$.
        \end{itemize}

        \item For each $v_i \in V$ we add the constraints $x_{i,t} \neq 2$ and $x_{i,1} \neq 1$.
    \end{enumerate}

    We define a weight function $w \colon \mathcal{B} \to \{0,1\}$ such that
    $w(1) = w(2) = 1$ and $w(0) = 0$.
    The target weight is $w_t = 2\ell \cdot t$,
    where $\ell$ is the target value for the given {\InducedM} instance.

    \begin{claim}
        If $G$ has an induced matching of size $\ell$,
        then $\psi$ has a satisfying assignment of weight at least $2\ell \cdot t$.
    \end{claim}

    \begin{claimproof}
        Let $M \subseteq E(G)$ be an induced matching of $G$ of size $\ell$.
        For each $v_i \notin V_M$ we assign to $x_{i,j}$ the value $0$, for all $j \in [t]$.
        As for the vertices of $V_M$, let $e \in M$ with $e = \{v_{i_1},v_{i_2}\}$ and $b(e)=j$.
        Then we assign to both $x_{i_1,j'},x_{i_2,j'}$ the value $2$ for all $j' \in [j]$,
        and the value $1$ for all $j' \in [j+1,t]$.

        It is easy to see that this assignment is of weight $|V_M| \cdot t \ge 2\ell \cdot t$.
        We argue that it is also satisfying.
        It is easy to see that the constraints introduced in Steps~2,~3, and~5 are satisfied.
        To see that the constraints introduced in Step~4 are also satisfied,
        let $e \in E$ with $e = \{v_{i_1},v_{i_2}\}$ and $b(e)=j$.
        If $e \in M$, then the corresponding constraints can be easily seen to be satisfied.
        Assume that $e \notin M$.
        Since $M$ is an induced matching, it holds that at most one endpoint of $e$ belongs to $V_M$,
        consequently it holds that $(x_{i_1,j} = 0) \lor (x_{i_2,j} = 0)$ and the constraints are once
        again satisfied.
    \end{claimproof}

    \begin{claim}
        If $\psi$ has a satisfying assignment of weight at least $2\ell \cdot t$,
        then $G$ has an induced matching of size $\ell$.
    \end{claim}

    \begin{claimproof}
        Let $i \in [n]$ such that $\psi$ assigns to $x_{i,t}$ the value $0$.
        In that case, due to the constraints in Step~$1$, it follows that $\psi$ assigns to
        $x_{i,j}$ the value $0$ for all $j \in [t]$.
        Let $I \subseteq [n]$ such that $x_{i,t}$ gets value $1$ by $\psi$;
        due to the discussion so far, it follows that $|I| \ge 2\ell$.
        It suffices to show that $G' = G[\setdef{v_i}{i \in I}]$ is an $1$-regular subgraph of $G$.
        Let $i \in I$.
        First assume that there is no edge incident to $v_i$ in $G'$.
        In that case, for any edge $e = \{ v_i, v_{i'} \}$ in $G$ it holds that
        $i' \notin I$, that is, $x_{i',j}$ receives value $0$ for all $j \in [t]$.
        Consequently, by the constraints of Steps~2,~4, and~5, $x_{i,t}$ gets value $2$, a contradiction.
        Now assume that there are more than one edges incident to $v_i$ in $G'$,
        and let $e_1,e_2 \in E$ such that $e_1 = \{v_i, v_{i_1}\}$ and $e_2 = \{v_i, v_{i_2}\}$ with $i_1,i_2 \in I$,
        while $b(e_1) = j_1$ and $b(e_2) = j_2 > j_1$.
        In that case, due to the constraints in Step~2,~3, and~4 it holds that
        $x_{i,j}$ gets value $1$ for all $j \in [j_1+1,t]$,
        in which case it follows that $x_{i,j_2} = 1$ and $x_{i_2,j_2} \neq 0$,
        falsifying the constraint $(x_{i,j_2} = 0) \lor (x_{i_2,j_2} = 0) \lor (x_{i,j_2} = x_{i_2,j_2} = 2)$,
        a contradiction.
    \end{claimproof}

    Finally, let us argue about the pathwidth of $\psi$.
    Take the path decomposition of $G$ and replace in each $B_j$ each vertex $v_i \in B_j$ with the variable $x_{i,j}$.
    For each $e \in E$ such that $e = \{v_{i_1}, v_{i_2}\}$ and $b(e)=j$ also place
    $x_{i_1,j+1}, x_{i_2,j+1}$ in $B_j$.
    This covers the constraints of Step~4,
    increasing the width by at most~2 (since $b$ is injective).
    For each $j \in [t-1]$ we insert between $B_j$ and $B_{j+1}$ a sequence of bags which start with
    $B_j$ and at each step add a variable $x_{i,j+1}$ and then remove $x_{i,j}$, one by one.
    Finally, to cover the remaining variables and constraints of Steps~2,~3, and~5,
    it suffices for each $v_i \in V$ that appears in the interval $B_{j_1}, \ldots, B_{j_2}$ to insert
    to the left of $B_{j_1}$ a sequence of bags that contain all of $B_{j_1}$ and a path decomposition of the path formed
    by $x_{i,1}, \ldots, x_{i,j_1}$, and similarly after $B_{j_2}$.
\end{proof}


For the other direction, we reuse the gadgets of the reduction presented by Lampis and Vasilakis~\cite{toct/LampisV24}
to prove that {\BDD} when $\Delta = 1$ cannot be solved in $\sO((3-\varepsilon)^{\pw})$ under the SETH;
the latter problem asks, given a graph $G$, $\Delta \ge 0$, and $k>0$,
whether there exists a set $S \subseteq V(G)$ of size at most $k$
such that the maximum degree of $G-S$ is at most $\Delta$.

\begin{lemmarep}[\appsymb]\label{lem:induced:lb:CSP->Induced}
    If there exists $\varepsilon>0$ and an algorithm that solves {\InducedM}
    in time $\sO((3-\varepsilon)^\pw)$,
    where $\pw$ denotes the pathwidth of the input graph,
    then the pw-SETH is false.
\end{lemmarep}


\begin{proof}
    We present a reduction from the $4$-CSP problem of \cref{cor:weird} to \InducedM.
    We are given a \textsc{CSP} instance $\psi$ whose variables take values from $[3]$,
    a partition of its variables into two sets $V_1, V_2$,
    a path decomposition $B_1, \ldots, B_t$ of $\psi$ of width $p$
    such that each bag contains $\bO(\log p)$ variables of $V_2$,
    and an injective function $b$ mapping each constraint to a bag that contains its variables.
    We want to construct an {\InducedM} instance $(G,\ell)$ with $\pw(G) = p + o(p)$,
    such that if $\psi$ is satisfiable, then $G$ has an induced matching of size $\ell$,
    while if $G$ has an induced matching of size $\ell$,
    $\psi$ admits a monotone satisfying multi-assignment which is consistent for $V_2$.
    We assume that the variables of $\psi$ are numbered $X = \{ x_1,\ldots, x_n \}$,
    the constraints are $c_1, \ldots, c_m$,
    and that the given path decomposition is nice.
    We construct $G$ as follows.


    \proofsubparagraph{Block and Variable Gadgets.}
    For every variable $x_i$ and every bag with $x_i \in B_j$, where $j \in [j_1,j_2]$,
    construct a \emph{block gadget} $\hat{B}_{i,j}$.
    Here we consider two cases, depending on whether $x_i \in V_1$ or $x_i \in V_2$.

    If $x_i \in V_1$ then we construct the block gadget $\hat{B}_{i,j}$ as a path on vertices $p^{i,j}_1, p^{i,j}_2,p^{i,j}_3,p^{i,j}_4$.
    Furthermore, for all $j \in [j_1,j_2-1]$, we identify the vertices $p^{i,j}_4, p^{i,j+1}_1$ so that they are the same vertex.
    The resulting path is called the \emph{variable gadget} of $x_i$, and is denoted by $\hat{P}_i$.
    See also \cref{fig:induced:lb:block:V1}.

    \begin{figure}[ht]
        \centering
        \begin{tikzpicture}[scale=0.8, transform shape]

        %%%%%%%%%% vertices and text
        \node[vertex] (vl11) at (0.5,5.5) {};
        \node[] () at (0.7,5.1) {$p^{i,j}_1$};

        \node[vertex] (vl12) at (2.5,5.5) {};
        \node[] () at (2.7,5.1) {$p^{i,j}_2$};

        \node[vertex] (vl13) at (4.5,5.5) {};
        \node[] () at (4.3,5.1) {$p^{i,j}_3$};

        \node[vertex] (vl14) at (6.5,5.5) {};
        \node[] () at (6.7,5.1) {$p^{i,j}_4 = p^{i,j+1}_1$};

        \node[vertex] (vl15) at (8.5,5.5) {};
        \node[] () at (8.7,5.1) {$p^{i,j+1}_2$};

        \node[vertex] (vl16) at (10.5,5.5) {};
        \node[] () at (10.3,5.1) {$p^{i,j+1}_3$};

        \node[] (vl17) at (12.5,5.5) {$\ldots$};


        %%%%%%%%% edges / arcs

        \draw[] (vl11)--(vl12)--(vl13)--(vl14)--(vl15)--(vl16)--(vl17);

        \end{tikzpicture}
        \caption{Part of the construction of the variable gadget $\hat{P}_i$, where $x_i \in V_1$.}
        \label{fig:induced:lb:block:V1}
    \end{figure}

    As for the case where $x_i \in V_2$, then the block gadget $\hat{B}_{i,j}$
    is a clique on vertices $p^{i,j},p^{i,j}_1,p^{i,j}_2,p^{i,j}_3$.
    Furthermore, for all $j \in [j_1,j_2-1]$ and $k \in [3]$,
    we add all edges between $p^{i,j}_k$ and $\{p^{i,j+1}_1,p^{i,j+1}_2,p^{i,j+1}_3\} \setminus \{p^{i,j+1}_k\}$,
    and we call the \emph{variable gadget} $\hat{P}_i$ this sequence of serially connected block gadgets.
    See also \cref{fig:induced:lb:block:V2}.

    \begin{figure}[ht]
        \centering
        \begin{tikzpicture}[scale=0.8, transform shape]

        %%%%%%%%%% vertices and text
        \node[vertex] (vl11) at (0.5,5.5) {};
        \node[] () at (0.6,5) {$p^{i,j}$};

        \node[vertex] (vl12) at (2.5,3.5) {};
        \node[] () at (2.6,3.1) {$p^{i,j}_1$};

        \node[vertex] (vl13) at (2.5,5.5) {};
        \node[] () at (2.2,5.8) {$p^{i,j}_2$};

        \node[vertex] (vl14) at (2.5,7.5) {};
        \node[] () at (2,7.5) {$p^{i,j}_3$};

        \node[vertex] (vl22) at (4.5,3.5) {};
        \node[] () at (4.7,3.1) {$p^{i,j+1}_1$};

        \node[vertex] (vl23) at (4.5,5.5) {};
        \node[] () at (5,5.8) {$p^{i,j+1}_2$};

        \node[vertex] (vl24) at (4.5,7.5) {};
        \node[] () at (5.3,7.5) {$p^{i,j+1}_3$};

        \node[vertex] (vl21) at (6.5,5.5) {};
        \node[] () at (6.6,5) {$p^{i,j+1}$};

        %%%%%%%%% edges / arcs

        \draw[] (vl11)--(vl12);
        \draw[] (vl11)--(vl13);
        \draw[] (vl11)--(vl14);
        \draw[] (vl12)--(vl13)--(vl14);
        \draw[] (vl12) edge [bend left] (vl14);


        \draw[] (vl21)--(vl22);
        \draw[] (vl21)--(vl23);
        \draw[] (vl21)--(vl24);
        \draw[] (vl22)--(vl23)--(vl24);
        \draw[] (vl22) edge [bend right] (vl24);


        \draw[] (vl12)--(vl23);
        \draw[] (vl12)--(vl24);

        \draw[] (vl13)--(vl22);
        \draw[] (vl13)--(vl24);

        \draw[] (vl14)--(vl22);
        \draw[] (vl14)--(vl23);

        \end{tikzpicture}
        \caption{Part of the construction of the variable gadget $\hat{P}_i$, where $x_i \in V_2$.}
        \label{fig:induced:lb:block:V2}
    \end{figure}


    \proofsubparagraph{Constraint Gadget.}
    This gadget is responsible for determining constraint satisfaction,
    based on the choices made in the rest of the graph.
    Let $c$ be a constraint of $\psi$, where $b(c) = j$ and
    $c$ involves variables $x_{i_1},x_{i_2},x_{i_3}, x_{i_4} \in B_j$.
    Consider the set $\mathcal{S}_c$ of the at most $3^4$ satisfying assignments of $c$.
    We construct the \emph{constraint gadget} $\hat{C}_c$ as follows.
    For each $\sigma \in \mathcal{S}_c$ construct a vertex $y_{c,\sigma}$.
    Additionally construct a vertex $w_c$,
    and add edges so that $w_c$ along with the vertices $y_{c,\sigma}$, for $\sigma \in \mathcal{S}_c$,
    form a clique.
    For each $\sigma \in \mathcal{S}_c$ and $\alpha \in [4]$,
    consider the following cases:
    \begin{itemize}
        \item if $x_{i_\alpha} \in V_1$ and $\sigma(x_{i_\alpha}) = 1$,
        then connect $y_{c,\sigma}$ to $p^{i,j}_3$,

        \item if $x_{i_\alpha} \in V_1$ and $\sigma(x_{i_\alpha}) = 2$,
        then connect $y_{c,\sigma}$ to $p^{i,j}_1$ and $p^{i,j}_4$,

        \item if $x_{i_\alpha} \in V_1$ and $\sigma(x_{i_\alpha}) = 3$,
        then connect $y_{c,\sigma}$ to $p^{i,j}_2$,

        \item if $x_{i_\alpha} \in V_2$ and $\sigma(x_{i_\alpha}) = k \in [3]$,
        then connect $y_{c,\sigma}$ to vertices $\{p^{i,j}_1,p^{i,j}_2,p^{i,j}_3\} \setminus \{p^{i,j}_k\}$.
    \end{itemize}

    This completes the construction of $G$.
    We set $\ell = L_1+L_2+m$, where $L_i$ denotes the number of block gadgets constructed
    due to variables belonging to $V_i$, and $m$ is the number of constraints of $\psi$.
    In that case, $(G,\ell)$ is the constructed instance of \InducedM.

    \begin{claim}
        If $\psi$ is satisfiable, then $G$ has an induced matching of size $\ell$.
    \end{claim}

    \begin{claimproof}
        Let $\sigma$ be a satisfying assignment for $\psi$.
        For each $x_i \in V_2$ we select into our solution $M$ the edge $\{p^{i,j}, p^{i,j}_{\sigma(x_i)}\}$,
        for all $j \in [t]$ with $x_i \in B_j$.
        For each $x_i \in V_1$ and $j \in [t]$ such that $x_i \in B_j$
        we select into $M$ the edge $\{p^{i,j}_{\sigma(x_i)}, p^{i,j}_{\sigma(x_i)+1}\}$.
        Finally, for each constraint $c$,
        we select into $M$ the edge $\{y_{c,\sigma_c}, w_c\}$,
        where $\sigma_c \in \mathcal{S}_c$ is the restriction of $\sigma$ to the variables of $c$.

        $M$ is a matching of size $\ell$.
        It remains to argue that it is an induced matching.
        To this end, observe that among the vertices of $V_M$,
        none of the vertices belonging to a constraint gadget
        is neighbors with vertices of $V_M$ that belong to a block gadget.
        The statement then easily follows.
    \end{claimproof}

    \begin{claim}
        If $G$ has an induced matching of size $\ell$,
        then $\psi$ admits a monotone satisfying multi-assignment which is consistent for $V_2$.
    \end{claim}

    \begin{claimproof}
        Let $M$ be an induced matching of $G$ of size at least $\ell$.
        We assume without loss of generality that $M$ does not have any edge with one
        of its endpoints belonging to a block gadget and the other to a constraint gadget.
        Indeed, assume that such an edge $e$ exists, where $y_{c,\sigma_0} \in e$ for constraint $c$.
        Then it holds that $V_M$ contains no other vertex of $\hat{C}_c$,
        as its vertices form a clique.
        Consequently, $(M \setminus \{e\}) \cup \{\{y_{c,\sigma_0, w_c}\}\}$ is also an induced matching of the same size.
        Moreover, this implies that either $M$ does not contain any edge incident to vertices of $\hat{C}_c$,
        or there exists a $\sigma_c \in \mathcal{S}_c$ such that $\{y_{c,\sigma_c}, w_c\} \in M$.

        Now consider a variable gadget $\hat{P}_i$.
        First consider the case where $x_i \in V_1$.
        It is easy to see that $M$ contains at most $1$ edge per block gadget,
        with all edges having both of their endpoints in the same block gadget.
        Now assume that $x_i \in V_2$ and $x_i \in B_j$ for all $j \in [j_1,j_2]$.
        Assume that there exists an edge $e \in M$ such that $e = \{p^{i,j}_k,p^{i,j+1}_{k'}\}$,
        with $j \in [j_1,j_2-1]$ and $k,k' \in [3]$.
        Then, it follows that no other vertex of $\hat{B}_{i,j}$ and $\hat{B}_{i,j+1}$ belongs to $V_M$,
        and exchanging $\{p^{i,j}_k,p^{i,j+1}_{k'}\}$ with edges $\{p^{i,j}_k,p^{i,j}\}$ and $\{p^{i,j+1}_{k'},p^{i,j+1}\}$
        results in an induced matching of larger size.
        In the following assume that for $M$ it holds that any edge incident to vertices of $\hat{B}_{i,j}$ has
        both of its endpoints in said gadget. Notice that $M$ contains at most one edge from each such block gadget.

        Consequently, it follows that $M$ contains at most $L_1 + L_2 + m = \ell$ edges,
        where $L_i$ denotes the number of block gadgets constructed due to variables belonging
        to $V_i$, and $m$ is the number of constraints of $\psi$.
        It follows that $|M| = \ell$, and $M$ contains exactly one edge per block gadget
        as well as exactly one edge per constraint gadget.

        Consider a multi-assignment $\sigma \colon X \times [t] \to [3]$ over all pairs $x_i \in X$ and $j \in [t]$
        with $x_i \in B_j$.
        In particular, consider two cases.
        If $x_i \in V_1$ and $\{p^{i,j}_k, p^{i,j}_{k+1}\} \in M$,
        then $\sigma(x_i,j) = k \in [3]$.
        For the case where $x_i \in V_2$, if $\{p^{i,j}_k, p^{i,j}\} \in M$,
        then $\sigma(x_i,j) = k \in [3]$.

        We argue that $\sigma$ is consistent for all $x_i \in V_2$.
        Fix $x_i \in V_2$ and $j \in [j_1,j_2-1]$, where $B_{j_1}$ and $B_{j_2}$ denote the first and last
        bag that contain $x_i$.
        Notice that $\{p^{i,j}_k, p^{i,j}\} \in M$ implies that $p^{i,j+1}_{k'} \notin V_M$
        where $k \in [3]$ and $k' \in [3] \setminus \{k\}$, thus $\sigma(x_i,j) = \sigma(x_i,j')$
        for all $j,j' \in [j_1,j_2]$.

        We next argue that $\sigma$ is monotone for $x_i \in V_1$,
        where $x_i \in B_j$ for all $j \in [j_1,j_2]$.
        Let $j \in [j_1,j_2-1]$.
        If $\{p^{i,j}_3, p^{i,j}_4\} \in M$,
        then it follows that $p^{i,j+1}_2 \notin V_M$,
        implying that $\{p^{i,j+1}_3, p^{i,j+1}_4\} \in M$.
        If on the other hand $\{p^{i,j}_2, p^{i,j}_3\} \in M$,
        then it follows that $p^{i,j+1}_1 \notin V_M$,
        implying that either $\{p^{i,j+1}_2, p^{i,j+1}_3\} \in M$ or $\{p^{i,j+1}_3, p^{i,j+1}_4\} \in M$.

        Finally, we argue that $\sigma$ is satisfying.
        Consider a constraint $c$ with $b(c)=j$,
        involving four variables from $V_{i_1}, V_{i_2}, V_{i_3}, V_{i_4}$.
        Let $\sigma_c \in \mathcal{S}_c$ such that $\{y_{c,\sigma_c},w_c\} \in M$.
        We claim that $\sigma_c$ must be consistent with our assignment.
        Indeed, if $\sigma_c$ assigns value $k \in [3]$ to $x_i \in V_2$,
        then it follows that $(\{p^{i,j}_1,p^{i,j}_2,p^{i,j}_3\} \setminus \{p^{i,j}_k\}) \notin V_M$,
        thus $\{p^{i,j}_k, p^{i,j}\} \in M$.
        On the other hand, if $\sigma_c$ assigns value $k \in [3]$ to $x_i \in V_2$,
        then
        \begin{itemize}
            \item if $k=1$, then $p^{i,j}_3 \notin V_M$, implying that $\{p^{i,j}_1,p^{i,j}_2\} \in M$,
            \item if $k=2$, then $p^{i,j}_1,p^{i,j}_4 \notin V_M$, implying that $\{p^{i,j}_2,p^{i,j}_3\} \in M$,
            \item if $k=3$, then $p^{i,j}_2 \notin V_M$, implying that $\{p^{i,j}_3,p^{i,j}_4\} \in M$.
        \end{itemize}
        This completes the proof.
    \end{claimproof}

    Finally, to bound the pathwidth of the graph $G$,
    start with the decomposition of the primal graph of $\psi$ and
    in each $B_j$ replace each $x_i \in V_2 \cap B_j$ with all the vertices of $\hat{B}_{i,j}$.
    We further add in $B_j$ all the vertices of $\hat{B}_{i,j+1}$ for $x_i \in V_2 \cap B_{j+1}$.
    For each constraint $c$, let $b(c)=j$ and add into $B_j$ all the at most $3^4 + 1$
    vertices of $\hat{C}_c$.
    So far each bag contains $p + \bO(\log p)$ vertices.
    To cover the remaining vertices,
    for each $j \in [t]$ we replace the bag $B_j$ with a sequence of bags such that
    all of them contain the vertices we have added to $B_j$ so far,
    the first bag contains $\setdef{p^{i,j}_1}{x_i \in V_1 \cap B_j}$ and
    the last bag contains $\setdef{p^{i,j}_4}{x_i \in V_1 \cap B_j}$.
    We insert a sequence of $\bO(p)$ bags between these two,
    at each step adding a vertex $p^{i,j}_{k+1}$ and then removing $p^{i,j}_k$
    in a way that covers all edges of paths due to block gadgets.
\end{proof}
