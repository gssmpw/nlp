\proofsubparagraph{Joining labels with edges, $H = \eta_{i,j}(H')$.}
For each pair of pairwise different $i,j \in [\cw]$,
we define a function $f_{i,j} \colon \{0,1,2\}^{\cw} \to \{0,1,2\}^{\cw}$.
In particular, for $\sigma' \in \{0,1,2\}^{\cw}$ we set $f_{i,j}(\sigma') = \sigma$
such that $\sigma(w) = \sigma'(w)$ for all $w \in [\cw] \setminus \{i,j\}$
and
\[
    (\sigma(i),\sigma(j)) =
        \begin{cases}
            (\sigma'(i), \, \sigma'(j)) &\text{if $0 \in \{\sigma'(i), \, \sigma'(j)\}$,}\\
            (1,1)                       &\text{if $\sigma'(i) = \sigma'(j) = 2$}.
        \end{cases}
\]
For a fixed size $k \in [0,n]$ and signature $\sigma \in \{0,1,2\}^{\cw}$ we set the value of $\DPt_H[k,\sigma]$ to be
\begin{equation}\label{eq:DPH-join}
    \DPt_H[k,\sigma] = \sum_{\sigma' \in f^{-1}_{i,j} (\sigma)} \DPt_{H'}[k,\sigma'].
\end{equation}
If $f^{-1}_{i,j}(\sigma) = \varnothing$ we set $\DPt_H[k,\sigma] = 0$.
Notice that computing $f^{-1}_{i,j}(\sigma)$ requires $\bO(\cw)$ time,
thus by \cref{eq:DPH-join} we can fill the table $\DPt_H[\cdot,\cdot]$ in time $\sO(3^{\cw})$.

\begin{lemmarep}[\appsymb]\label{lemma:induced:cw:join-correctness}
    Let $H = \eta_{i,j}(H')$.
    Assume that for all $k \in [0,n]$ and $\sigma \in \{0,1,2\}^{\cw}$,
    $\DPt_{H'}[k,\sigma]$ is equal to the number of partial matchings $S$ of $H'$ such that $|S|=k$ and $\sgn_{H'}(S) = \sigma$.
    Then, $\DPt_H[k,\sigma]$ is equal to the number of partial matchings $S$ of $H$ such that $|S|=k$ and $\sgn_H(S) = \sigma$.
\end{lemmarep}

\begin{proof}
    We show that the function $f_{i,j}$ describes precisely the effect of the operation $\eta_{i,j}$
    on the signature of a partial matching $S$.

    \begin{claim}\label{claim:induced:cw:join-correctness}
        Let $S \subseteq V(H')$ such that $\sgn_{H'}(S) = \sigma'$.
        Then, $\sgn_{H}(S) = \sigma$ if and only if $f_{i,j}(\sigma') = \sigma$.
    \end{claim}

    \begin{claimproof}
        First, notice that the labels of the vertices remain the same between $H'$ and $H$,
        therefore it holds that $\lab_H(v) = \lab_{H'}(v)$ for all $v \in V(H')$.
        This implies that $S \cap \lab^{-1}_H (w) = S \cap \lab^{-1}_{H'} (w)$ for all $w \in [\cw]$.
        Since $H[S]$ is obtained from $H'[S]$ by adding all edges between vertices of label~$i$ and~$j$,
        it follows that $\deg_{H[S]}(v) = \deg_{H'[S]}(v)$ for all $v \in S \setminus (\lab^{-1}_H(i) \cup \lab^{-1}_H(j))$,
        thus $\sgn_{H,w}(S) = \sgn_{H',w}(S)$ for all $w \in [\cw] \setminus \{i,j\}$.
        In that case, $\sgn_{H}(S)$ and $f_{i,j}(\sigma')$ agree on all labels $w \in [\cw] \setminus \{i,j\}$.
        It remains to argue for labels~$i$ and~$j$.

        Assume first that $0 \in \{\sigma'(i), \, \sigma'(j)\}$.
        It is easy to see that in that case the graphs $H[S]$ and $H'[S]$ are the same,
        consequently $\sgn_{H}(S) = \sgn_{H'}(S)$ follows, and by the definition of function $f_{i,j}$
        we have that $\sgn_{H}(S) = f_{i,j}(\sigma')$.

        Next, assume that $\sigma'(i) = \sigma'(j) = 2$.
        Then, it holds that both $S \cap \lab^{-1}_{H}(i)$ and $S \cap \lab^{-1}_{H}(i)$ are comprised of a single vertex
        that is of degree $0$ in $H'[S]$, which implies that $\sgn_{H,i}(S) = \sgn_{H,j}(S) = 1$
        since the same vertices have degree~$1$ in $H[S]$.
        Again we have that $\sgn_{H}(S) = f_{i,j}(\sigma')$.

        Lastly, assume that $0 \notin \{\sigma'(i), \, \sigma'(j)\}$ and either $\sigma'(i) \neq 2$ or $\sigma'(j) \neq 2$.
        Assume without loss of generality that $\sigma'(i) = 1$ and $\sigma'(j) \in \{1,2\}$,
        and let $v \in S \cap \lab^{-1}_{H'}(i)$ and $u \in S \cap \lab^{-1}_{H'}(j)$,
        where $\deg_{H'[S]}(v) = 1$.
        Since $\psi$ is by definition an irredundant clique-width expression,
        it follows that $\{v,u\} \notin E(H')$.
        Consequently, the operation $\eta_{i,j}$ adds the edge $\{v,u\}$ in $H$, thus $\deg_{H[S]}(v) > \deg_{H'[S]}(v) = 1$.
        In that case $\sgn_{H,i}(S)$ is undefined, as is the case for $f_{i,j}(\sigma')$.
    \end{claimproof}

    Consequently, the number of partial matchings of $H$ of size $k$ and signature $\sigma$
    is exactly equal to the sum of the number of partial matchings of $H'$ of size $k$ and signature $\sigma'$
    over all $\sigma' \in f^{-1}_{i,j}(\sigma)$.
    The statement now follows by the hypothesis for $\DPt_{H'}[\cdot,\cdot]$ and \cref{eq:DPH-join}.
\end{proof}
