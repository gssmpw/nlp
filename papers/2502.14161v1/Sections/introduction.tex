%%%%%%%%%%%%%%%%%%%%%%%%%%%%%%%%%%%%%%%%%%%%%%%%%%%%%%%%%
% --------------------- INTRODUCTION --------------------
%%%%%%%%%%%%%%%%%%%%%%%%%%%%%%%%%%%%%%%%%%%%%%%%%%%%%%%%%
\section{Introduction}\label{sec:introduction}

Given a graph $G$, a \emph{matching} $M \subseteq E(G)$ of $G$ is a subset of its edges such that
every vertex of $G$ is incident to at most one edge in $M$.
Matchings are central in Computer Science~\cite{books/LovaszP09},
and problems involving them find numerous practical applications such as matching kidney donors to recipients
and clients to server clusters to name a few~\cite{books/ws/Manlove13}.
Even though finding a matching of maximum cardinality in general graphs
is well-known to be polynomial-time solvable~\cite{focs/MicaliV80},
further demanding that the graph induced by the vertices of the matching adheres to a specific constraint
oftentimes renders the problem NP-hard.
Recently, Chaudhary and Zehavi~\cite{wg/ChaudharyZ23a} studied a plethora of such NP-hard problems
under the lens of parameterized complexity, providing a variety of algorithms and posing some open questions,
some of which we answer in this work.

We first define the problems in question.
Let $G=(V,E)$ be a graph and $M \subseteq E$ a matching of $G$, where $V_M \subseteq V$ denotes
the set of vertices of $G$ incident to edges in $M$.
We say that matching $M$ is \emph{induced} (resp., \emph{acyclic}) if $G[V_M]$ is $1$-regular (resp., acyclic).
In that case, given a graph $G$ and an integer $\ell$,
{\InducedM} (resp., \AcyclicM) asks whether $G$ has an induced (resp., acyclic) matching of size at least $\ell$.
Originally introduced as the ``risk-free'' marriage problem~\cite{ipl/StockmeyerV82},
{\InducedM} has been extensively studied due to its numerous applications (see~\cite{dam/GolumbicL00}).
This problem and its variations also have connections to various problems such as
\textsc{Maximum Feasible Subsystem}~\cite{soda/ChalermsookLN13,soda/ElbassioniRRS09}
and storylines extraction~\cite{kdd/KumarMS04}.
% maximum expanding sequence~\cite{siamcomp/BriestK11}
% It is also a subtask of finding a \emph{strong edge coloring}.
As for \AcyclicM, it was introduced by Goddard et al.~\cite{dm/GoddardHHL05}
along with some other variations of \textsc{Maximum Matching}.
Both problems are known to be NP-hard~\cite{dam/Cameron89,dm/GoddardHHL05,ipl/StockmeyerV82},
thus in this paper we examine them under the perspective of parameterized complexity.%
\footnote{We assume the reader is familiar with the basics of parameterized complexity,
as given e.g.~in~\cite{books/CyganFKLMPPS15}.}
Most relevant to us are the results concerning their complexity under structural parameterizations.
Both problems are known to be FPT by the treewidth $\tw$ of the input graph~\cite{wg/ChaudharyZ23a,tcs/HajebiJ23,dam/MoserS09},
with the state-of-the-art being due to Chaudhary and Zehavi~\cite{wg/ChaudharyZ23a}
who proposed algorithms of running time%
\footnote{Standard $\sO$ notation is used to suppress polynomial factors.}
$\sO(3^{\tw})$ and $\sO(6^{\tw})$ for {\InducedM} and {\AcyclicM} respectively.
Furthermore, the authors complement their algorithms with
$\sO((\sqrt{6}-\varepsilon)^{\pw})$ and $\sO((3-\varepsilon)^{\pw})$ SETH-based lower bounds respectively,
where $\pw$ denotes the pathwidth of the input graph.
Lastly, as mentioned in~\cite{algorithmica/KoblerR03}, {\InducedM} is expressible in MSO$_1$ logic,
thus it is FPT parameterized by clique-width due to standard metatheorems~\cite{mst/CourcelleMR00}.


\subparagraph{Our Contribution.}
We start with the {\InducedM} problem.
Our first result is to show the optimality of the $\sO(3^{\tw})$ algorithm of Chaudhary and Zehavi~\cite{wg/ChaudharyZ23a}.
As a matter of fact, for the parameterization by the pathwidth $\pw$ of the input graph,
we show that for all $\varepsilon>0$, obtaining an $\sO((3-\varepsilon)^{\pw})$ algorithm is \emph{equivalent} to
falsifying the pw-SETH, a weakening of the SETH that was recently postulated by Lampis~\cite{soda/Lampis25}.
Note that this implies that an $\sO((3-\varepsilon)^{\pw})$ algorithm would falsify not just the standard form of the SETH (for $k$-CNF formulas),
but (among others) also the SETH for circuits of depth $\varepsilon n$~\cite{arxiv/Lampis24} and the Set Cover Conjecture~\cite{CyganDLMNOPSW16}.
Our reduction follows along the lines of the one for {\BDD} by Lampis and Vasilakis~\cite{toct/LampisV24}, though adapted to the machinery of the pw-SETH.
Moving on, we consider the parameterization by the clique-width $\cw$ and develop an (optimal due to our previous result)
$\sO(3^{\cw})$ algorithm based on dynamic programming.
Although obtaining an algorithm of running time $\sO(6^{\cw})$ is quite straightforward,
improving this to our obtained $\sO(3^{\cw})$ running time requires some extra machinery.
In particular, to bypass the bottleneck of the computation time in the union nodes,
we employ ideas from~\cite{mfcs/BodlaenderLRV10,esa/RooijBR09} and consider \emph{merged states} while solving the
counting version of the problem.
Doing so allows us to subsequently use the FFT technique introduced by Cygan and Pilipczuk~\cite{tcs/CyganP10}
and populate the whole table in quasilinear time.
Next we move towards the {\AcyclicM} problem.
We first provide a more careful analysis of the recent algorithm by Chaudhary and Zehavi~\cite{wg/ChaudharyZ23a}
and show that in fact it runs in time $\sO(5^{\tw})$.
Our main result however is to show that improving over this, even for the parameterization by pathwidth,
results in the pw-SETH failing.
Finally, we consider {\AcyclicM} parameterized by clique-width.
Bergougnoux and Kant\'e~\cite{tcs/BergougnouxK19} developed a framework based on the rank-based approach of
Bodlaender, Cygan, Kratsch, and Nederlof~\cite{iandc/BodlaenderCKN15}
to deal with connectivity problems in this setting and developed algorithms with single-exponential dependence on the parameter
for various problems, including \FVS.
Adapting the latter algorithm allows us to obtain an $2^{\bO(\cw)} n^{\bO(1)}$ algorithm for \AcyclicM,
which is optimal under the ETH.


\subparagraph{Related Work.}
Both {\InducedM} and {\AcyclicM} have been extensively studied with respect to their polynomial-time
solvability in specific graph classes.
We mention in passing that both problems remain NP-hard even in very restricted graph classes;
for the first, such results are known for bipartite graphs of maximum degree $3$~\cite{ipl/Lozin02}
and planar graphs of maximum degree $4$~\cite{siamdm/KoS03},
while for the latter for planar bipartite graphs of maximum degree $3$~\cite{tcs/HajebiJ23}.
For further results in specific graph classes we refer to~\cite{tcs/BrandstadtES07,algorithmica/BrandstadtH08,dam/Cameron89,dm/CameronST03,dam/Chang03,dmtcs/EkimD13,dam/GolumbicL00,algorithmica/HabibM20,algorithmica/KlemzR22,siamdm/KoS03,algorithmica/KoblerR03,dam/KrishnamurthyS12,ipl/Lozin02,wg/Zito99}
for {\InducedM} and to~\cite{dam/BasteR18,tcs/HajebiJ23,tcs/PandaC23,dmaa/PandaP12} for \AcyclicM.
Exact exponential-time algorithms for {\InducedM} have been proposed in the literature~\cite{ol/ChangCH15,iandc/XiaoT17},
while both problems have been also studied with respect to their (in)approximability;
see~\cite{disopt/BasteFR20,soda/ChalermsookLN13,focs/ChalermsookLN13,tcs/ChlebikC06,iandc/ChlebikC08,jda/DuckworthMZ05,soda/ElbassioniRRS09,tcs/FurstLR18,dam/LinMV18,disopt/OrlovichFGZ08,tcs/Rautenbach15}
and~\cite{dm/BasteFR22,dam/BasteR18,anor/FurstR19,tcs/PandaC23} respectively.

From the viewpoint of Parameterized Complexity,
both {\InducedM} and {\AcyclicM} are W[1]-hard parameterized by the natural parameter $\ell$ even in bipartite graphs,
yet become FPT for various graph classes, including planar~\cite{tcs/DabrowskiDL13,tcs/HajebiJ23,dam/MoserS09};
as a matter of fact, in case of {\InducedM} in planar graphs, various kernelization algorithms have been developed~\cite{dam/ErmanKKW10,jcss/KanjPSX11}.
Recently, Chaudhary and Zehavi~\cite{jcss/ChaudharyZ25} also provided parameterized inapproximability results.
Below guarantee parameterizations have also been studied~\cite{jcss/ChaudharyZ25,stacs/Koana23,jda/MoserT09,tcs/XiaoK20}.

Regarding structural parameterizations, both problems are FPT parameterized by the treewidth $\tw$ of the input graph.
For \InducedM, Moser and Sikdar~\cite{dam/MoserS09} developed a $\sO(4^{\tw})$-time DP algorithm
which was subsequently improved to $\sO(3^{\tw})$ by Chaudhary and Zehavi~\cite{wg/ChaudharyZ23a}
who also showed a $\sO((\sqrt{6}-\varepsilon)^{\pw})$ lower bound under the SETH ($\pw$ denotes the pathwidth).
As for \AcyclicM, Hajebi and Javadi~\cite{tcs/HajebiJ23} noticed that the problem is expressible in MSO$_2$,
thus it is FPT parameterized by treewidth due to Courcelle's theorem~\cite{iandc/Courcelle90};
Chaudhary and Zehavi~\cite{wg/ChaudharyZ23a} later provided an explicit algorithm of complexity $\sO(6^{\tw})$
as well as a $\sO((3-\varepsilon)^{\pw})$ lower bound under the SETH.
For structural parameters apart from treewidth,
it is known that {\InducedM} is expressible in MSO$_1$ logic~\cite{algorithmica/KoblerR03},
thus FPT parameterized by clique-width due to standard metatheorems~\cite{mst/CourcelleMR00},
and {\AcyclicM} is FPT parameterized by modular-width~\cite{tcs/HajebiJ23},
while neither problem admits (under standard assumptions) a polynomial kernel parameterized by
either the vertex cover number or the distance to clique of the input graph~\cite{jcss/ChaudharyZ25,tcs/GomesMPSS23}.
