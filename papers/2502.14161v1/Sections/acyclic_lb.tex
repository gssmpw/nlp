\subsection{Lower Bound}\label{subsec:acyclic:lb}

Here we present the stated lower bound for \AcyclicM.

\begin{theoremrep}[\appsymb]\label{thm:acyclic:lb}
    For any constant $\varepsilon > 0$,
    there is no $\sO((5-\varepsilon)^\pw)$ algorithm deciding \AcyclicM,
    where $\pw$ denotes the pathwidth of the input graph,
    unless the pw-SETH is false.
\end{theoremrep}

\begin{proof}
    We present a reduction from the $4$-CSP-$5$ problem of \cref{cor:weird} to \AcyclicM.
    We are given a \textsc{CSP} instance $\psi$ whose variables take values from $[5]$,
    a partition of its variables into two sets $V_1, V_2$,
    a path decomposition $B_1, \ldots, B_t$ of $\psi$ of width $p$
    such that each bag contains $\bO(\log p)$ variables of $V_2$,
    and an injective function $b$ mapping each constraint to a bag that contains its variables.
    We want to construct an {\AcyclicM} instance $(G,\ell)$ with $\pw(G) = p + o(p)$,
    such that if $\psi$ is satisfiable, then $G$ has an acyclic matching of size $\ell$,
    while if $G$ has an acyclic matching of size $\ell$,
    $\psi$ admits a monotone satisfying multi-assignment which is consistent for $V_2$.
    We assume that the variables of $\psi$ are numbered $X = \{ x_1,\ldots, x_n \}$,
    the constraints are $c_1, \ldots, c_m$,
    and that the given path decomposition is nice.
    We construct $G$ as follows.


    \proofsubparagraph{XOR Gadget.}
    Given two vertices $v_1$ and $v_2$, when we say that we connect them via an \emph{XOR gadget} $\hat{O}(v_1,v_2)$
    we first add the edge $\{v_1,v_2\}$,
    then introduce two vertices $p^{v_1,v_2}_1$ and $p^{v_1,v_2}_2$ called the \emph{private vertices of the XOR gadget},
    and lastly add edges so that $v_1$, $v_2$, and the private vertices of the gadget form a cycle;
    for an illustration see \cref{fig:acyclic:lb_or_gadget}.
    We say that $v_1,v_2,p^{v_1,v_2}_1,p^{v_1,v_2}_2$ are the \emph{vertices of the XOR gadget $\hat{O}(v_1,v_2)$},
    while the \emph{edges of an XOR gadget} are the edges whose endpoints both are among its vertices.


    \begin{figure}[htb]
        \centering
        \begin{tikzpicture}[scale=1, transform shape]

        %%%%%%%%%% vertices and text

        \node[vertex] (u) at (1,5) {};
        \node[] () at (0.7,5) {$v_1$};

        \node[vertex] (x1) at (1.5,5.5) {};
        \node[] () at (1.3,5.8) {$p^{v_1,v_2}_1$};
        \node[vertex] (x2) at (2,5.5) {};
        \node[] () at (2.5,5.8) {$p^{v_1,v_2}_2$};

        \node[vertex] (v) at (2.5,5) {};
        \node[] () at (2.8,5) {$v_2$};


        %%%%%%%%% edges / arcs

        \draw[] (u)--(x1)--(x2)--(v)--(u);

        \end{tikzpicture}
        \caption{XOR gadget $\hat{O}(v_1,v_2)$.}
        \label{fig:acyclic:lb_or_gadget}
    \end{figure}


    \proofsubparagraph{Root vertex.}
    We start our construction by introducing a vertex $r$, which we refer to as the \emph{root vertex}.
    Furthermore, we attach a leaf $r'$ to $r$.



    \proofsubparagraph{Block and Variable Gadgets.}
    For every variable $x_i$ and every bag with $x_i \in B_j$, where $j \in [j_1,j_2]$,
    construct a \emph{block gadget} $\hat{B}_{i,j}$.
    Here we consider two cases, depending on whether $x_i \in V_1$ or $x_i \in V_2$.

    If $x_i \in V_1$, then we construct the block gadget as depicted in \cref{fig:acyclic:lb_block_gadget}.
    In order to do so, we introduce vertices $a, a', \chi_1, \chi_2, y_1, y_2, b_1, b_2$.
    Then, we attach leaves $l_1$ and $l_2$ to $\chi_1$ and $\chi_2$ respectively.
    Next, we add edges from $a$ to $\chi_1$ and $y_1$, from $a'$ to $\chi_2$ and $y_2$,
    from the root vertex $r$ to $\chi_1$ and $\chi_2$, as well as the edge $\{b_1,b_2\}$.
    Finally, we add the XOR gadgets $\hat{O}(a,b_1)$, $\hat{O}(a',b_2)$, $\hat{O}(\chi_1,\chi_2)$,
    and $\hat{O}(y_1,y_2)$.
    Next, for $j \in [j_1,j_2-1]$, we serially connect the block gadgets $\hat{B}_{i,j}$ and $\hat{B}_{i,j+1}$ so that the vertex $a'$ of $\hat{B}_{i,j}$ is the vertex $a$ of $\hat{B}_{i,j+1}$,
    thus resulting in the ``path'' $\hat{P}_i$ consisting of such serially connected block gadgets, called a \emph{variable gadget}.
    For an illustration see \cref{fig:acyclic:lb_serially_connected}.
    Intuitively, for any optimal matching $M$,
    it will hold that either $a, a' \notin V_M$ or $a, a' \in V_M$.
    In the latter case, it will hold that $|V_M \cap \{\chi_1,\chi_2\}| = |V_M \cap \{y_1,y_2\}| = 1$,
    resulting in $4$ choices for a total of $5$;
    we map each choice with an assignment where $x_i$ receives a value in $[5]$.


    \begin{figure}[ht]
        \centering
          \begin{subfigure}[b]{0.4\linewidth}
          \centering
            \begin{tikzpicture}[scale=0.75, transform shape]

            %%%%%%%%%% vertices and text

            \node[vertex] (a) at (0,5) {};
            \node[] () at (0,5.3) {$a$};

            \node[gray_vertex] (x1) at (1.5,6) {};
            \node[] () at (1.5,6.3) {$\chi_1$};

            \node[gray_vertex] (x2) at (4.5,6) {};
            \node[] () at (4.5,6.3) {$\chi_2$};

            \node[vertex] (y1) at (1.5,4) {};
            \node[] () at (1.5,4.3) {$y_1$};

            \node[vertex] (y2) at (4.5,4) {};
            \node[] () at (4.5,4.3) {$y_2$};

            \node[vertex] (a') at (6,5) {};
            \node[] () at (6,5.3) {$a'$};

            \node[vertex] (b1) at (2.5,2.5) {};
            \node[] () at (2.5,2.8) {$b_1$};
            \node[vertex] (b2) at (3.5,2.5) {};
            \node[] () at (3.5,2.8) {$b_2$};

            \node[gray_vertex] (r) at (3,7.5) {};
            \node[] () at (3,7.8) {$r$};


            %%%%%%%%% edges / arcs

            \draw[] (r)--(x1);
            \draw[] (r)--(x2);
            \draw[] (a)--(x1);
            \draw[] (x2)--(a');
            \draw[dashed] (x1)--(x2);
            \draw[] (a)--(y1);
            \draw[] (y2)--(a');
            \draw[dashed] (y1)--(y2);
            \draw[] (b1)--(b2);
            \draw[dashed] (a) edge [bend right] (b1);
            \draw[dashed] (a') edge [bend left] (b2);

            \end{tikzpicture}
            \caption{Block gadget.}
            \label{fig:acyclic:lb_block_gadget}
          \end{subfigure}
        \begin{subfigure}[b]{0.4\linewidth}
        \centering
            \begin{tikzpicture}[scale=0.5, transform shape]

            \node[vertex] (a) at (0,5) {};

            \node[gray_vertex] (x1) at (1.5,6) {};

            \node[gray_vertex] (x2) at (4.5,6) {};

            \node[vertex] (y1) at (1.5,4) {};

            \node[vertex] (y2) at (4.5,4) {};

            \node[vertex] (a') at (6,5) {};

            \node[vertex] (b1) at (2.5,2.5) {};
            \node[vertex] (b2) at (3.5,2.5) {};

            \begin{scope}[shift={(6,0)}]

                \node[gray_vertex] (x1') at (1.5,6) {};

                \node[gray_vertex] (x2') at (4.5,6) {};

                \node[vertex] (y1') at (1.5,4) {};

                \node[vertex] (y2') at (4.5,4) {};

                \node[vertex] (b1') at (2.5,2.5) {};
                \node[vertex] (b2') at (3.5,2.5) {};

                \node[vertex] (a'') at (6,5) {};

            \end{scope}


            \node[gray_vertex] (r) at (6,8.5) {};


            %%%%%%%%% edges / arcs

            \draw[] (r) edge [bend right] (x1);
            \draw[] (r) edge [bend right] (x2);
            \draw[] (r) edge [bend left] (x1');
            \draw[] (r) edge [bend left] (x2');
            \draw[] (a)--(x1);
            \draw[] (x2)--(a');
            \draw[dashed] (x1)--(x2);
            \draw[] (a)--(y1);
            \draw[] (y2)--(a');
            \draw[dashed] (y1)--(y2);
            \draw[] (b1)--(b2);
            \draw[dashed] (a) edge [bend right] (b1);
            \draw[dashed] (a') edge [bend left] (b2);
            \draw[] (a')--(x1');
            \draw[] (x2')--(a'');
            \draw[dashed] (x1')--(x2');
            \draw[] (a')--(y1');
            \draw[] (y2')--(a'');
            \draw[dashed] (y1')--(y2');
            \draw[] (b1')--(b2');
            \draw[dashed] (a') edge [bend right] (b1');
            \draw[dashed] (a'') edge [bend left] (b2');


            \end{tikzpicture}
            \caption{Serially connected block gadgets.}
            \label{fig:acyclic:lb_serially_connected}
          \end{subfigure}
        \caption{Case where $x_i \in V_1$. Gray vertices have a leaf attached, dashed edges denote XOR gadgets.}
        \label{fig:acyclic:lb}
        \end{figure}

        As for the case where $x_i \in V_2$, then the block gadget $\hat{B}_{i,j}$
        is constructed as follows.
        We introduce $6$ vertices $u^{i,j}, u^{i,j}_1, u^{i,j}_2, u^{i,j}_3, u^{i,j}_4, u^{i,j}_5$,
        we add an edge from $u^{i,j}$ to all vertices in $\setdef{u^{i,j}_k}{k \in [5]}$,
        and for all $1 \le k_1 < k_2 \le 5$ we add an XOR gadget
        $\hat{O}(u^{i,j}_{k_1},u^{i,j}_{k_2})$.
        Next, for $j \in [j_1,j_2-1]$, for all \emph{distinct} $k_1,k_2 \in [5]$,
        we add an XOR gadget $\hat{O}(u^{i,j}_{k_1},u^{i,j+1}_{k_2})$,
        and we call the \emph{variable gadget} $\hat{P}_i$ this sequence of serially connected block gadgets.



        \proofsubparagraph{Constraint Gadget.}
        This gadget is responsible for determining constraint satisfaction,
        based on the choices made in the rest of the graph.
        Let $c$ be a constraint of $\psi$, where $b(c) = j$ and
        $c$ involves variables $x_{i_1},x_{i_2},x_{i_3}, x_{i_4} \in B_j$.
        Consider the set $\mathcal{S}_c$ of the at most $5^4$ satisfying assignments of $c$.
        We construct the \emph{constraint gadget} $\hat{C}_c$ as follows.
        \begin{itemize}
            \item For each $\sigma \in \mathcal{S}_c$ construct a vertex $v_{c,\sigma}$.

            \item For distinct $\sigma_1,\sigma_2 \in \mathcal{S}_c$,
            introduce an XOR gadget $\hat{O}(v_{c,\sigma_1},v_{c,\sigma_2})$.

            \item Introduce a vertex $v_c$ and add edges between $v_c$ and all vertices $\setdef{v_{c,\sigma}}{\sigma \in \mathcal{S}_c}$.

            \item For each $\sigma \in \mathcal{S}_c$ and $\alpha \in [4]$,
            consider the following cases:
            \begin{itemize}
                \item if $x_{i_\alpha} \in V_1$ and $\sigma(x_{i_\alpha}) = 1$,
                then add an XOR gadget between $v_{c,\sigma}$ and vertices $\{b_1, b_2, \chi_2, y_2\}$ of $\hat{B}_{i,j}$,

                \item if $x_{i_\alpha} \in V_1$ and $\sigma(x_{i_\alpha}) = 2$,
                then add an XOR gadget between $v_{c,\sigma}$ and vertices $\{b_1, b_2, \chi_1, y_2\}$ of $\hat{B}_{i,j}$,

                \item if $x_{i_\alpha} \in V_1$ and $\sigma(x_{i_\alpha}) = 3$,
                then add an XOR gadget between $v_{c,\sigma}$ and vertices $\{a, a',y_1, y_2\}$ of $\hat{B}_{i,j}$,

                \item if $x_{i_\alpha} \in V_1$ and $\sigma(x_{i_\alpha}) = 4$,
                then add an XOR gadget between $v_{c,\sigma}$ and vertices $\{b_1, b_2, \chi_2, y_1\}$ of $\hat{B}_{i,j}$,

                \item if $x_{i_\alpha} \in V_1$ and $\sigma(x_{i_\alpha}) = 5$,
                then add an XOR gadget between $v_{c,\sigma}$ and vertices $\{b_1, b_2, \chi_1, y_1\}$ of $\hat{B}_{i,j}$,

                \item if $x_{i_\alpha} \in V_2$,
                then add an XOR gadget between $v_{c,\sigma}$ and all vertices in $\setdef{u^{i,j}_k}{k \in [5] \setminus \{ \sigma(x_{i_\alpha}) \}}$.
            \end{itemize}
        \end{itemize}

        This completes the construction of $G$.
        We set $L$ to be the number of XOR gadgets introduced in $G$,
        and we set $\ell = 1 + L + 2L_1 + L_2 + m$,
        where $L_i$ denotes the number of block gadgets constructed due to variables belonging to $V_i$,
        and $m$ is the number of constraints of $\psi$.
        In that case, $(G,\ell)$ is the constructed instance of \AcyclicM.



        \begin{lemma}\label{lem:acyclic:lb:csp->acyclic}
            If $\psi$ is satisfiable,
            then $G$ has an acyclic matching of size $\ell$.
        \end{lemma}

        \begin{nestedproof}
            Let $\sigma \colon X \to [5]$ be a satisfying assignment for $\psi$, that is,
            $\sigma$ satisfies all constraints $c_1, \ldots, c_m$.
            We will construct a matching $M \subseteq E(G)$ of size $|M| \geq \ell$ such that $G[V_M]$ is a forest.
            \begin{itemize}
                \item First, add to $M$ the edge $\{r,r'\}$.

                \item Then, for every XOR gadget $\hat{O}(v_1,v_2)$ in $G$, include in $M$ the edge between
                its two private vertices $\{p^{v_1,v_2}_1,p^{v_1,v_2}_2\}$.

                \item Next, for every constraint gadget $\hat{C}_c$, since $\sigma$ is a satisfying assignment,
                there exists some vertex $v_{c,\sigma_c}$ with $\sigma_c$ being the restriction of $\sigma$
                to the variables involved in $c$.
                Add in $M$ the edge $\{v_c,v_{c,\sigma_c}\}$.

                \item For each $x_i \in V_2$ and $j \in [t]$ such that $x_i \in B_j$
                we select into $M$ the edge $\{u^{i,j}, u^{i,j}_k\}$, where $\sigma(x_i) = k$.

                \item Finally, for each $x_i \in V_2$ and $j \in [t]$ such that $x_i \in B_j$,
                we include in $M$ the following edges from the block gadget $\hat{B}_{i,j}$:
                \begin{itemize}
                    \item if $\sigma(x_i) = 1$, the edges $\{ y_1, a \}$ and $\{ \chi_1, l_1 \}$,

                    \item if $\sigma(x_i) = 2$, the edges $\{ y_1, a \}$ and $\{ \chi_2, l_2 \}$,

                    \item if $\sigma(x_i) = 3$, the edges $\{ b_1, b_2 \}$ and $\{ \chi_1, l_1 \}$,

                    \item if $\sigma(x_i) = 4$, the edges $\{ y_2, a' \}$ and $\{ \chi_1, l_1 \}$,

                    \item if $\sigma(x_i) = 5$, the edges $\{ y_2, a' \}$ and $\{ \chi_2, l_2 \}$.
                \end{itemize}
            \end{itemize}
            Consequently, $M$ is a matching of size $1 + L + m + L_2 + 2L_1 = \ell$.

            We claim that $G[V_M]$ is a forest.
            Consider any constraint gadget $\hat{C}_c$,
            and let $v_{c,\sigma_c}$ such that $\{v_c, v_{c,\sigma_c}\} \in M$.
            Notice that since $\sigma_c$ is the restriction of $\sigma$ to its involved variables,
            one can easily verify that none of the vertices that $v_{c,\sigma_c}$ has an XOR gadget with belongs to $V_M$.
            Consequently, the vertices of the constraint gadgets belonging to $V_M$ induce acyclic components.

            It remains to argue that the graph induced by $r$, $r'$, and the vertices of the block gadgets in $V_M$ is acyclic.
            Notice that this is indeed the case for the block gadgets corresponding to variables in $V_2$,
            thus in the following we focus on the block gadgets corresponding to variables in $V_1$.
            Let $x_i \in V_1$ and $j \in [t]$ such that $x_i \in B_j$.
            First, notice that the graph induced by the vertices of any single block gadget $\hat{B}_{i,j}$ in $V_M$ along with $\{r,r'\}$
            is acyclic and vertices $a$ and $a'$ are disconnected.
            Furthermore, for $j_1, j_2 \in [t]$ and distinct $x_{i_1},x_{i_2} \in V_1$,
            notice that the block gadgets $\hat{B}_{i_1,j_1}$ and $\hat{B}_{i_2,j_2}$
            are disconnected in $G[V_M] - r$, thus if $G[V_M]$ contains a cycle,
            then this cycle involves only block gadgets corresponding to a single variable $x_i$.
            Finally, it holds that any two sequential block gadgets (as in \cref{fig:acyclic:lb_serially_connected})
            are connected in $G - r$ via their single common vertex which appears as $a$ in the first and $a'$ in the second gadget.
            Consequently, any cycle in $G[V_M]$ involving at least two block gadgets $\hat{B}_{i,j_1}$ and $\hat{B}_{i,j_2}$ with $j_1 < j_2$
            must contain $r$, the vertex $a'$ of $\hat{B}_{i,j_1}$, and the vertex $a$ of $\hat{B}_{i,j_2}$.
            Given that the vertices $a$ and $a'$ of any single block gadget are disconnected in $G[V_M] - r$,
            it follows that vertex $a'$ of $\hat{B}_{i,j_1}$ and $a$ of $\hat{B}_{i,j_2}$
            are disconnected in $G[V_M] - r$, thus no cycle exists in $G[V_M]$.
            This completes the proof.
        \end{nestedproof}

        \begin{lemma}
            If $G$ has an acyclic matching of size $\ell$,
            then $\psi$ admits a monotone satisfying multi-assignment which is consistent for $V_2$.
        \end{lemma}

        \begin{nestedproof}
            We start with the following claim.

            \begin{claim}\label{claim:acyclic:lb:acyclic->csp:maximum_matching}
                There exists an acyclic matching $M_{\max}$ of $G$ of maximum size such that
                for any XOR gadget $\hat{O}(v_1,v_2)$ of $G$ it holds that $|\{v_1,v_2\} \cap V_{M_{\max}}| \le 1$
                and $\{p^{v_1,v_2}_1,p^{v_1,v_2}\} \in M_{\max}$.
            \end{claim}

            \begin{claimproof}
                Let $M_{\max}$ be a maximum acyclic matching of $G$, and assume that $G$ contains an XOR gadget $\hat{O}(v_1,v_2)$.
                We show that we can obtain an acyclic matching $M'_{\max}$ of $G$ of the same size
                such that $|\{v_1,v_2\} \cap V_{M'_{\max}}| \le 1$ and $\{p^{v_1,v_2}_1,p^{v_1,v_2}\} \in M'_{\max}$.
                To this end, we consider different cases depending on the vertices of the XOR gadget that belong to $V_{M_{\max}}$.

                If $v_1,v_2 \notin V_{M_{\max}}$, then $\{p^{v_1,v_2}_1,p^{v_1,v_2}_2\} \in M_{\max}$ as
                $M_{\max}$ is of maximum size.

                Now assume that $|\{v_1,v_2\} \cap V_{M_{\max}}| = 1$, and without loss of generality $v_1 \in V_{M_{\max}}$ (the other case is analogous).
                Let $e \in M_{\max}$ be the edge incident to $v_1$.
                If $e \neq \{v_1,p^{v_1,v_2}_1\}$, then it holds that $\{p^{v_1,v_2}_1,p^{v_1,v_2}_2\} \in M_{\max}$ as $M_{\max}$ is of maximum size.
                Otherwise it holds that $p^{v_1,v_2}_2 \notin V_{M_{\max}}$ and the acyclic matching $M'_{\max} = (M_{\max} \setminus \{e\}) \cup \{\{p^{v_1,v_2}_1,p^{v_1,v_2}_2\}\}$
                has the same size.

                Lastly, assume that $v_1,v_2 \in V_{M_{\max}}$.
                Notice that in that case, either $p^{v_1,v_2}_1 \notin V_{M_{\max}}$ or $p^{v_1,v_2}_2 \notin V_{M_{\max}}$, as otherwise $G[V_{M_{\max}}]$
                contains a cycle.
                If $\{v_1,v_2\} \in M_{\max}$, then $M'_{\max} = (M_{\max} \setminus \{\{v_1,v_2\}\}) \cup \{\{p^{v_1,v_2}_1,p^{v_1,v_2}_2\}\}$
                is an acyclic matching of the same size.
                Otherwise, it holds that for either $v_1$ or $v_2$ there exists an edge $e \in M_{\max}$ incident to it
                that does not belong to the XOR gadget $\hat{O}(v_1,v_2)$.
                Assume without loss of generality that $e$ is incident to $v_1$, and let $e' \in M_{\max}$ be the edge incident to $v_2$ belonging to $M_{\max}$.
                Then, the acyclic matching $M'_{\max} = (M_{\max} \setminus \{e'\}) \cup \{\{p^{v_1,v_2}_1,p^{v_1,v_2}_2\}\}$ is of the same size.
            \end{claimproof}


            Armed with the previous claim, we can now prove the existence of an acyclic matching in $G$ with specific properties.

            \begin{claim}\label{claim:acyclic:lb:acyclic->csp:matching_properties}
                There exists an acyclic matching $M$ of maximum size of $G$ such that,
                for any XOR gadget $\hat{O}(v_1,v_2)$ of $G$ it holds that $|\{v_1,v_2\} \cap V_M| \le 1$ and $\{p^{v_1,v_2}_1,p^{v_1,v_2}\} \in M$.
                Furthermore, it holds that
                \begin{itemize}
                    \item $\{r,r'\} \in M$,

                    \item for all constraint gadgets $\hat{C}_c$,
                    there exists $\sigma \in \mathcal{S}_c$ such that $\{v_c, v_{c,\sigma_c}\} \in M$,

                    \item for all block gadgets $\hat{B}_{i,j}$,
                    \begin{itemize}
                        \item if $x_i \in V_1$ then
                        $| \{ \{\chi_1,l_1\}, \{\chi_2,l_2\} \} \cap M| =
                        |\{\{ y_1, a \}, \{ y_2, a' \}, \{ b_1, b_2 \}\} \cap M| = 1$,

                        \item otherwise if $x_i \in V_2$ there exists $k \in [5]$ such that $\{u^{i,j}, u^{i,j}_k\} \in M$.
                    \end{itemize}
                \end{itemize}
            \end{claim}

            \begin{claimproof}
                Let $M_{\max}$ be an acyclic matching of $G$ of maximum size as in
%--------------- EDO EXO KANEI PATENTA ---------------%
                \iflncs
                the previous claim,
                \else
                \cref{claim:acyclic:lb:acyclic->csp:maximum_matching},
                \fi
                where $|M_{\max}| \ge \ell$ since by assumption $G$ has an acyclic matching of size at least $\ell$.
                Consider the following reduction rule.

                \proofsubparagraph{Rule~$(\dagger)$.}
                Let $\{u,v\} \in M_{\mathsf{in}}$, and let $l \neq v$ be a leaf attached to $u$ in $G$, while $v$ is not.
                Then, replace $M_{\mathsf{in}}$ with $M_{\mathsf{out}} = (M_{\mathsf{in}} \setminus \{\{u,v\}\}) \cup \{\{u,l\}\}$.

                It is easy to see that in Rule~$(\dagger)$ if $M_{\mathsf{in}}$ is an acyclic matching,
                then so is $M_{\mathsf{out}}$, while the size of the two matchings is the same.
                Let $M$ denote the acyclic matching obtained after exhaustively applying Rule~$(\dagger)$ in $M_{\max}$.
                Notice that $|M| \ge \ell$, while for any XOR gadget $\hat{O}(v_1,v_2)$ of $G$ it holds that $|\{v_1,v_2\} \cap V_M| \le 1$
                and $\{p^{v_1,v_2}_1,p^{v_1,v_2}\} \in M$.

                Notice that due to Rule~$(\dagger)$, if $r \in V_M$, then $\{r,r'\} \in M$.
                Furthermore, for any edge in $M \setminus \{\{r,r'\}\}$ that is not due to an XOR gadget
                it holds that either both of its endpoints belong to the vertices of a block gadget,
                or they both belong to the vertices of a constraint gadget;
                that is due to the fact that any connected pair of vertices from both block and constraint gadgets
                or from different block gadgets is connected via an XOR gadget.

                Notice that in the constraint gadget $\hat{C}_c$,
                all vertices $\setdef{v_{c,\sigma}}{\sigma \in \mathcal{S}_c}$ are pairwise connected via XOR gadgets,
                thus at most one of them belongs to $V_M$.
                Along with the previous paragraph, it follows that either $V_M$ contains no vertex from $\hat{C}_c$,
                or there exists $\sigma_c \in \mathcal{S}_c$ such that $\{v_c,v_{c,\sigma_c}\} \in M$.

                Now consider the block gadget $\hat{B}_{i,j}$, where $j \in [t]$ such that $x_i \in B_j$.
                Assume first that $x_i \in V_1$,
                and notice that if $\chi_1 \in V_M$, then $\{\chi_1,l_1\} \in M$ and $\chi_2 \notin V_M$ (similarly for $\chi_2$).
                Furthermore, one can easily verify that $|\{\{ y_1, a \}, \{ y_2, a' \}, \{ b_1, b_2 \}\} \cap M| \le 1$.
                If on the other hand $x_i \in V_2$, then
                since vertices $\setdef{u^{i,j}_k}{k \in [5]}$ are pairwise connected via XOR gadgets,
                it holds that either $V_M$ contains no vertex of $\hat{B}_{i,j}$,
                or there exists $k \in [5]$ such that $\{u^{i,j}, u^{i,j}_k\} \in M$.

                Recall that $|M| \ge \ell = 1 + L + 2L_1 + L_2 + m$,
                where $L$ is the number of XOR gadgets in $G$,
                $L_i$ is the number of block gadgets constructed due to variables belonging to $V_i$,
                and $m$ is the number of constraints of $\psi$.
                Due to the discussion so far and summing over all constraint and block gadgets,
                $M$ contains at most $\ell$ edges.
                Consequently, it follows that the bounds obtained in the previous paragraphs are tight for all constraint and block gadgets,
                which implies that the inclusions are as described in the statement and $\{r,r'\} \in M$.
            \end{claimproof}


            Now let $M$ be an acyclic matching of $G$ as in
%--------------- EDO EXO KANEI PATENTA ---------------%
            \iflncs
            the previous claim.
            \else
            \cref{claim:acyclic:lb:acyclic->csp:matching_properties}.
            \fi
            Consider a multi-assignment $\sigma \colon X \times [t] \to [5]$ over all pairs $x_i \in X$ and $j \in [t]$
            with $x_i \in B_j$.
            In particular, consider two cases.
            If $x_i \in V_2$ and $\{u^{i,j}, u^{i,j}_k\} \in M$, then $\sigma(x_i,j) = k \in [5]$.
            Otherwise (i.e., $x_i \in V_1$),
            if for the edges of $\hat{B}_{i,j}$ it holds that
            \begin{itemize}
                \item $\{y_1,a\},  \{\chi_1,l_1\} \in M$, then $\sigma(x_i,j) = 1$,
                \item $\{y_1,a\},  \{\chi_2,l_2\} \in M$, then $\sigma(x_i,j) = 2$,
                \item $\{b_1,b_2\} \in M$,                then $\sigma(x_i,j) = 3$,
                \item $\{y_2,a'\}, \{\chi_1,l_1\} \in M$, then $\sigma(x_i,j) = 4$,
                \item $\{y_2,a'\}, \{\chi_2,l_2\} \in M$, then $\sigma(x_i,j) = 5$.
            \end{itemize}
            Notice that $\sigma$ is well-defined due to
%--------------- EDO EXO KANEI PATENTA ---------------%
            \iflncs
            the previous claim.
            \else
            \cref{claim:acyclic:lb:acyclic->csp:matching_properties}.
            \fi

            \begin{claim}
                It holds that $\sigma$ is monotone-increasing, as well as consistent for all $x_i \in V_2$.
            \end{claim}

            \begin{claimproof}
                Regarding the consistency for $x_i \in V_2$,
                notice that it follows due to the XOR gadgets between vertices belonging to sequential
                block gadgets $\hat{B}_{i,j}$ and $\hat{B}_{i,j+1}$.
                In the following we argue about the monotonicity.
                Assume that $x_i \in V_1$ and $j \in [j_1,j_2-1]$,
                where $B_{j_1}$ and $B_{j_2}$ denote the first and last bag of the decomposition containing $x_i$ respectively.


                We first argue that if for $\hat{B}_{i,j}$ it holds that
                $\{y_2 , a'\} \in M$,
                then for $\hat{B}_{i,j+1}$ it holds that
                $\{y_2 , a'\} \in M$.
                Indeed, since vertices $a'$ and $a$ of the first and the latter block gadget coincide,
                it follows that for $\hat{B}_{i,j+1}$ we have $\{y_1, a\}, \{b_1,b_2\} \notin M$,
                and since $|\{\{ y_1, a \}, \{ y_2, a' \}, \{ b_1, b_2 \}\} \cap M| = 1$ for any block gadget,
                the statement follows.

                Now we argue that if for $\hat{B}_{i,j}$ it holds that
                $\{y_2 , a'\}, \{\chi_2,l_2\} \in M$,
                then for $\hat{B}_{i,j+1}$ it holds that
                $\{y_2 , a'\}, \{\chi_2,l_2\} \in M$.
                In this case, due to the previous paragraph we have that in $\hat{B}_{i,j+1}$,
                $\{y_2 , a'\} \in M$ as well as $a \in V_M$,
                implying that $\chi_1 \notin V_M$ (as otherwise $G[V_M]$ has a cycle),
                which in turn implies that $\{\chi_2 , l_2\} \in M$.
                Consequently, it holds that if $\sigma (x_i,j) = 5$ then $\sigma (x_i,j+1) = 5$,
                while in conjunction with the previous paragraph we get that if $\sigma (x_i,j) = 4$
                then $\sigma (x_i,j+1) \in \{4,5\}$.

                Next, it is easy to see that if in $\hat{B}_{i,j}$ we have that $\{b_1 ,b_2 \} \in M$,
                then in $\hat{B}_{i,j+1}$ we have that $\{y_1 , a \} \notin M$;
                indeed, in $\hat{B}_{i,j}$ since $b_2 \in V_M$ it holds that $a' \notin V_M$,
                however this implies that for $\hat{B}_{i,j+1}$  it holds that $a \notin V_M$.
                Consequently, if $\sigma (x_i,j) = 3$ then $\sigma (x_i,j+1) \in \{3,4,5\}$.

                Finally assume that in $\hat{B}_{i,j}$
                we have that $\{y_2 ,a'\} , \{\chi_2 , l_2\} \in M$.
                In that case, it holds that in $\hat{B}_{i,j+1}$ we have $\chi_1 \notin V_M$
                (otherwise $G[V_M]$ contains a cycle),
                implying that $\{\chi_2, l_2\} \in M$.
                Consequently, we have that if $\sigma (x_i,j) = 2$ then
                $\sigma (x_i,j+1) \neq 1$.
            \end{claimproof}

            It remains to argue that $\sigma$ is a satisfying multi-assignment for $\psi$.
            Consider a constraint $c$ with $b(c)=j$,
            involving four variables from $V_{i_1}, V_{i_2}, V_{i_3}, V_{i_4}$.
            Let $\sigma_c \in \mathcal{S}_c$ such that $\{v_c, v_{c,\sigma_c}\} \in M$.
            We claim that $\sigma_c$ must be consistent with $\sigma$.
            Indeed, if $\sigma_c$ assigns value $k \in [5]$ to $x_i \in V_2$,
            then it follows that $(\setdef{u^{i,j}_k}{k \in [5]} \setminus \{u^{i,j}_{\sigma(x_i, j)}\}) \notin V_M$
            due to the associated XOR gadgets, thus $\{u^{i,j}, u^{i,j}_{\sigma(x_i, j)}\} \in M$.
            On the other hand, if $\sigma$ assigns value $k \in [5]$ to $x_i \in V_2$,
            then
            \begin{itemize}
                \item if $k=1$, then in $\hat{B}_{i,j}$ we have that $b_1,b_2,\chi_2,y_2 \notin V_M$, implying that $\{y_1,a\}, \{\chi_1,l_1\} \in M$,
                \item if $k=2$, then in $\hat{B}_{i,j}$ we have that $b_1,b_2,\chi_1,y_2 \notin V_M$, implying that $\{y_1,a\}, \{\chi_2,l_2\} \in M$,
                \item if $k=3$, then in $\hat{B}_{i,j}$ we have that $a,a',y_1,y_2 \notin V_M$, implying that $\{b_1,b_2\} \in M$,
                \item if $k=4$, then in $\hat{B}_{i,j}$ we have that $b_1,b_2,\chi_2,y_1 \notin V_M$, implying that $\{y_2,a'\}, \{\chi_1,l_1\} \in M$,
                \item if $k=5$, then in $\hat{B}_{i,j}$ we have that $b_1,b_2,\chi_1,y_1 \notin V_M$, implying that $\{y_2,a'\}, \{\chi_1,l_1\} \in M$.
            \end{itemize}
            This completes the proof.
        \end{nestedproof}

        Finally, to bound the pathwidth of the graph $G$,
        start with the decomposition of the primal graph of $\psi$ and
        in each $B_j$ replace each $x_i \in V_2 \cap B_j$ with all the vertices of $\hat{B}_{i,j}$.
        We further add in $B_j$ all the vertices of $\hat{B}_{i,j+1}$ for $x_i \in V_2 \cap B_{j+1}$,
        as well as the vertices $r$ and $r'$.
        For each constraint $c$, let $b(c)=j$ and add into $B_j$ all the
        vertices of $\hat{C}_c$ and any private vertex of an XOR gadget involving a vertex of $\hat{C}_c$,
        for a total of at most $1 + 5^4 + 2\binom{5^4}{2} + 8 \cdot 5^4$ vertices.
        So far each bag contains $p + \bO(\log p)$ vertices.
        To cover the remaining vertices,
        for each $j \in [t]$ we replace the bag $B_j$ with a sequence of bags such that
        all of them contain the vertices we have added to $B_j$ so far,
        the first bag contains the vertices $a$ belonging to all $\hat{B}_{i,j}$ for all $x_i \in V_1 \cap B_j$ and
        the last bag contains the vertices $a'$ belonging to all $\hat{B}_{i,j}$ for all $x_i \in V_1 \cap B_j$.
        We insert a sequence of $\bO(p)$ bags between these two,
        at each step adding all vertices apart from $a$ of a block gadget $\hat{B}_{i,j}$ and then removing
        all vertices of the same block gadget apart from $a'$, for $x_i \in V_1 \cap B_j$.
\end{proof}
