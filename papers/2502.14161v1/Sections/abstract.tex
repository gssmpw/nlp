We revisit the (structurally) parameterized complexity of \textsc{Induced Matching} and \textsc{Acyclic Matching},
two problems where we seek to find a maximum independent set of edges whose endpoints induce,
respectively, a matching and a forest.
Chaudhary and Zehavi~[WG '23] recently studied these problems parameterized by treewidth, denoted by $\mathrm{tw}$.
We resolve several of the problems left open in their work and extend their results as follows:
(i) for \textsc{Acyclic Matching}, Chaudhary and Zehavi gave an algorithm of running time
$6^{\mathrm{tw}}n^{\mathcal{O}(1)}$ and a lower bound of
$(3-\varepsilon)^{\mathrm{tw}}n^{\mathcal{O}(1)}$ (under the SETH);
we close this gap by, on the one hand giving a more careful analysis of their
algorithm showing that its complexity is actually $5^{\mathrm{tw}} n^{\mathcal{O}(1)}$,
and on the other giving a pw-SETH-based lower bound showing that this running time cannot be improved
(even for pathwidth),
(ii) for \textsc{Induced Matching} we show that their $3^{\mathrm{tw}} n^{\mathcal{O}(1)}$
algorithm is optimal under the pw-SETH (in fact improving over this for pathwidth is \emph{equivalent} to falsifying the pw-SETH)
by adapting a recent reduction for \textsc{Bounded Degree Vertex Deletion},
(iii) for both problems we give FPT algorithms with single-exponential dependence when
parameterized by clique-width and in particular for \textsc{Induced Matching}
our algorithm has running time $3^{\mathrm{cw}} n^{\mathcal{O}(1)}$,
which is optimal under the pw-SETH from our previous result.

% ------------------------------ ALTERNATIVELY ------------------------------
% We revisit the (structurally) parameterized complexity of \textsc{Induced Matching} and \textsc{Acyclic Matching},
% two problems where we seek to find a maximum independent set of edges whose endpoints induce,
% respectively, a matching and a forest.
% A recent work by Chaudhary and Zehavi~[WG '23] shows that the parameterization by the treewidth $\mathrm{tw}$
% of the input graph yields FPT algorithms of running times $3^{\mathrm{tw}} n^{\mathcal{O}(1)}$ and $6^{\mathrm{tw}} n^{\mathcal{O}(1)}$ respectively,
% and they ask whether their algorithms are optimal.
% We first show that, with some slightly more careful analysis, the latter algorithm has running time
% $5^{\mathrm{tw}} n^{\mathcal{O}(1)}$ instead.
% Our main contribution however is to show that any further improvement in any of them would refute the SETH,
% even for the parameterization by pathwidth, thus rendering both algorithms optimal.
% Moving on, we study both problems when parameterized by the clique-width $\mathrm{cw}$ of the input graph.
% Here, we develop an optimal (modulo SETH) algorithm for \textsc{Induced Matching} of running time
% $3^{\mathrm{cw}} n^{\mathcal{O}(1)}$, while for \textsc{Acyclic Matching} we develop an FPT algorithm of
% single-exponential parametric dependence by making use of the framework developed by Bergougnoux and Kant{\'e}~[TCS].
