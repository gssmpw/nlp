
\begin{abstract}
% 機械学習アルゴリズムとそのハイパーパラメータの最適な組み合わせの選択は高性能な機械学習システム開発において重要である.
Selecting the optimal combination of a machine learning (ML) algorithm and its hyper-parameters is crucial for the development of high-performance ML systems.
%
% 一方で,機械学習アルゴリズムとハイパーパラメータの組み合わせは膨大であるため,網羅的な検証では多くの時間を必要とする.
However, since the combination of ML algorithms and hyper-parameters is enormous, the exhaustive validation requires a significant amount of time.
%
% 既存研究の主要なアプローチの一つとしてベイズ最適化に基づく方法が広く研究されている.
Many existing studies use Bayesian optimization (BO) for accelerating the search.
%
% しかし,機械学習アルゴリズム毎に異なるハイパーパラメータ空間が存在するため,効率的な探索手法を構築することが難しい.
On the other hand, a significant difficulty is that, in general, there exists a different hyper-parameter space for each one of candidate ML algorithms. 
%
BO-based approaches typically build a surrogate model independently for each hyper-parameter space, by which sufficient observations are required for all candidate ML algorithms. 
%
% 本稿では全ての機械学習アルゴリズムのハイパーパラメータ空間を共有の潜在空間上へと写像し,その潜在空間上でのベイズ最適化を提案する.
In this study, our proposed method embeds different hyper-parameter spaces into a shared latent space, in which a surrogate multi-task model for BO is estimated.
%
% この方法では,共有潜在空間上で各機械学習アルゴリズムの観測結果を共有することができるため,少ない観測をより効率的に活かした探索が期待できる.
This approach can share information of observations from different ML algorithms by which efficient optimization is expected with a smaller number of total observations.
%
% また,潜在空間推定のための敵対的正則化を用いた事前学習と,効果的な事前学習済みモデルの選択のためのランキング学習についても提案する.
We further propose the pre-training of the latent space embedding with an adversarial regularization, and a ranking model for selecting an effective pre-trained embedding for a given target dataset.
%
% 計算機実験ではOpenMLで利用可能なデータセットを用いて提案手法の有効性を示す.
Our empirical study demonstrates effectiveness of the proposed method through datasets from OpenML. 
\end{abstract}