\section{Related Work}
% \paragraph{Domain Adaptation}
% 1) time domain not fully explored
% 2) explored but mixture of expert very unique
% 3) lack of multilingual settings


\subsection{Domain Adaptation}

% what is Domain Adaptation
% [CITE] some Domain Adaptation work
% Time can be also view as a domain.
% However, time domain not fully explored.
%  use unsupervised domain adaptation to address data feature shift~\cite{he2023domain,ott2022domain} and label shift~\cite{he2023domain} over time, but these works are limited to time series data.
% a series of work [cites] .. (time as domain, text data). 
% The mixture of experts' as a time domain adaptation method...

Domain adaptation (DA)~\cite{daume2006domain,blitzer2006domain,ben2010theory,farahani2021brief} is a set of model optimization and data augmentation methods to promote model performance, assuming that data distributions change between training and test steps.
DA has several major directions to improve text classification robustness, including pivot features~\cite{blitzer2006domain,li2022cross}, instance weighting~\cite{jiang2007instance,Lv2023Review}, and domain adversaries~\cite{Ganin2015Unsupervised,Kong2024Unsupervised,zeng2024Unsupervised}. 
However, very few studies have treated time as domains and developed new domain adaptation methods to model temporal shifts.
Our study treats time as domains and develops a multi-source adaptation approach, \textit{MoTE}, to learn time and promote model generalizability.

% \textit{\textbf{Time} can also be viewed as a domain}, where data shifts over time, bringing a challenge for model's temporal generalizability. 
% This brings a need for temporal domain adaptation, which has not been fully explored. 
%However, the temporal domain adaptation has not been fully explored. 

Previous works have considered adapting temporal effects in text classifiers, such as continuously pre-training language models~\cite{rottger2021temporal, agarwal2022temporal, shang2022improving} and diachronic word embeddings~\cite{huang2019neural, rajaby2021time,dhingra2022time}. 
However, domain adaptation has not been fully explored in those studies.
Several recent works have employed domain adaptation to address temporal shifts~\cite{he2023domain,ott2022domain} on structured data (sensor data) by the pivot feature approach that sets the feature space of the target domain as a pivot and aligns feature vectors of the source domain towards the pivot.
However, such approaches may not be applicable to the unstructured text data, which has high dimensional features and sequential dependencies –– the focus of our study.

% but these approaches have been primarily limited to time series data. 
%A series of works have considered time as a domain in the context of text data. 
In contrast, our study proposes a multi-source domain adaptation approach (MoTE) to model temporal effects into classification models. Particularly, the existing studies primarily focus on English data leaving multilingual classification scenarios underexplored, which has been examined in our study.


\subsection{Multilingual Classification}

% \paragraph{Multilingual Classification}
% 1) classification model
% 
% 2) lack of temporal consideration,
%for temporal effect, they only consider english scenorior
% 3) temporal study explored English, but lack multilingual settings

The remarkable success of language models has led to significant advancements in multilingual text classification, addressing challenges such as multilingual long-text classification~\cite{chalkidis2023chatgpt} and parameter-efficient multilingual classification~\cite{razuvayevskaya2024comparison}.
For example, in order to solve the challenge of multilingual long text classification,~\cite{chalkidis2022lexglue} uses a hierarchical attention mechanism to improve the context window of pre-trained language models. 
Additionally, researchers have leveraged the advanced capabilities of multilingual LLMs~\cite{xue2021mt5,ma2021contributions} for multilingual classification in both few-shot~\cite{wang2020Generalizing} and zero-shot ~\cite{yin2019benchmarking} settings.
As an instance, recent studies have evaluated the performance of ChatGPT~\cite{lai2023chatgpt} and m-GPT~\cite{shliazhko2023mgpt} on multilingual text classification tasks in a zero-shot setting, demonstrating the generalization capability of these large language models on unseen multilingual datasets.



However, due to the significant disparity between English and other languages in high-quality corpus data, existing state-of-the-art approaches typically leverage the language model's English capability to enhance classification in other languages, as exemplified by machine translation-augmented text classification~\cite{king2024using} and cross-lingual in-context learning~\cite{cueva2024adaptive}. 
Consequently, recent studies only focus on the temporal adaptation of English classification task~\cite{agarwal2022temporal,dhingra2022time}, the impact of temporal shifts and trends in non-English languages on multilingual text classification performance remains largely unexplored.
In contrast, our work propose the Mixture of Temporal Experts (MoTE), aims to investigate the temporal shifts in multilingual text classification data.