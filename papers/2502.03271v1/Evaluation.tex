\begin{table}[t]
\centering
\caption{
{Bugs identified by \TN; note that we have only listed the bugs that have been reported for over three months as of January 1, 2025.}
}
\resizebox{\columnwidth}{!}
{
\begin{threeparttable}
\renewcommand\arraystretch{1.1}
\setlength{\columnsep}{1pt}
\begin{tabular}{l r r l c c} 
 \toprule
 Package & Version & Stars & Bug Types$^\ddagger$ and Numbers & Status$^*$ & Patched$^\dagger$ \\ %[0.5ex] 
 \midrule
 candle-core & 0.4.1 & 13.2k & Con $\rightarrow$ Gen: I:3 (\texttt{as}) & \CIRCLE & - \\
 \hline
 py-spy & 0.3.14 & 11k & Con $\rightarrow$ Con: I:1 (\texttt{as}) & \Circle & - \\
 \hline
 fyrox-core & 0.27.0 & 7.1k & \parbox{5cm}{Con $\rightarrow$ Gen: I:1 (\texttt{as}) \\ Gen $\rightarrow$ Con: II:4 (\texttt{as}), III:2 (\texttt{as})} & \CIRCLE & \checkmark \\
 \hline
 gfx-backend-gl & 0.9.0 & 5.2k & Con $\rightarrow$ Gen: I:1 (\texttt{as}) & \CIRCLE & - \\
 \hline
 silicon & 0.5.2 & 3.1k & Con $\rightarrow$ Con: II:1 (\texttt{as}) & \Circle & - \\
 \hline
 webrender & 0.61.0 & 3k & Con $\rightarrow$ Con: I:2 (\texttt{as}) & \CIRCLE & - \\
 \hline
 spl-token-swap & 3.0.0 & 2.3k & Con $\rightarrow$ Gen: I:1 (\texttt{as}), III:1 (\texttt{as}) & \CIRCLE & - \\
 \hline
 scryer-prolog & 0.9.4 & 1.9k & Con $\rightarrow$ Con: I:6 (\texttt{transmute}) & \CIRCLE & - \\
 \hline
 libafl & 0.10.1 & 1.6k & Con $\rightarrow$ Con: I:3 (\texttt{as}), III:4 (\texttt{as}) & \CIRCLE & \checkmark \\
 \hline
 mesalink & 1.1.0 & 1.5k & Gen $\rightarrow$ Con: II:1 (\texttt{as}) & \Circle & - \\
 \hline
 fontdue & 0.8.0 & 1.2k & Gen $\rightarrow$ Con: I:1 (\texttt{transmute}) & \CIRCLE & - \\
 \hline
 pprof & 0.13.0 & 1k & \parbox{3cm}{Con $\rightarrow$ Con: II:1 (\texttt{as}) \\ Con $\rightarrow$ Gen: I:1 (\texttt{as})} & \CIRCLE & \checkmark \\
 \hline
 rendy-core & 0.5.1 & 814 & Gen $\rightarrow$ Con: II:2 (\texttt{as}) & \LEFTcircle & - \\
 \hline
 rendy-util & 0.4.1 & 814 & Gen $\rightarrow$ Con: II:2 (\texttt{as}) & \LEFTcircle & - \\
 \hline
 sciter-rs & 0.5.58 & 784 & Gen $\rightarrow$ Con: I:1 (\texttt{as}) & \CIRCLE & - \\
 \hline
 rosrust & 0.9.11 & 728 & Gen $\rightarrow$ Con: II:3 (\texttt{as}) III:1 (\texttt{as}) & \LEFTcircle & - \\
 \hline
 cortex-m & 0.7.7 & 669 & Gen $\rightarrow$ Con: II:1 (\texttt{as}) & \CIRCLE & \checkmark \\
 \hline
 rafx-base & 0.0.15 & 574 & Gen $\rightarrow$ Con: II:2 (\texttt{as}) & \LEFTcircle & - \\ 
 \hline
 xous & 0.9.50 & 500 & Con $\rightarrow$ Gen: I:2 (\texttt{as}), III:2 (\texttt{as}) & \CIRCLE & \checkmark \\
 \bottomrule
\end{tabular}
\begin{tablenotes}
    \item $^*$ Bug status: \Circle: Reported; \CIRCLE: Confirmed by Vendors; \LEFTcircle: Verified by PoC.
    \item $^\dagger$ Patch status: \checkmark: already patched; -: unpatched.
    \item $^\ddagger$ Bug type: Con: concrete type; Gen: generic type; [I, II, III]: bug type.
\end{tablenotes}
\end{threeparttable}
}
\label{top1k-zerodays}
\end{table}

\section{Evaluation}


\noindent{\bf Dataset Collection.} We gathered packages from \texttt{crates.io}, the Rust community's crate registry. To ensure comprehensive bug detection, The dataset consists of the packages ranked by download counts and the number of GitHub stars (as of September 1, 2024), and each of them has more than 500 stars. 
Regarding package size, the largest package contains 510k LoC, with an average package size of 9k LoC.

\vspace{0.05in}
\noindent{\bf Experiment Setup.} We built \TN{} and conducted experiments on a server with 48-core Intel Xeon CPU ES-2630 and 256 GB memory. The server was deployed with Ubuntu 22.04 and \texttt{rustc} 1.72.0-nightly. For each package, we set the preparation (dependencies resolution and compilation) time threshold to 20 minutes and \TN detection time threshold to 2 minutes. We ran \TN on the 3,000 packages for detection. 




\subsection{Bug Detection Results} \label{evaluation:effectiveness}



\begin{figure}[t]
    \centering
    \includegraphics[width=\linewidth]{Figure/cdf_plot.pdf}
    \caption{Detection time of \TN.}
    \label{fig:cdf-execution}
\end{figure}



First, we ran \TN on the RUSTSEC dataset of existing type confusion bugs 
{as of September 1, 2024 (see \autoref{rustsec-details})}.
\TN is capable of detecting all existing \bugs of three types. Second, {we ran \TN with 16-thread to scan the 3,000 packages.} {It ran 18 hours in total; the majority of time was spent on resolving dependencies and compilation. The detection time only took about 80 mins and 94 bugs were reported. For the 3,000 packages, 618 (20.6\%) failed due to compiler version, 37 (1.2\%) did not have proper cargo metadata, and 309 (10.3\%) failed with custom build options, which cannot be resolved automatically and required manual work. All the remaining 2,036 packages (67.9\%) could successfully compile within 20 minutes. Among them, 282 did not have \texttt{/bin} or \texttt{/lib} targets for detection, so TypePulse did not execute on them. For 1,754 packages that TypePulse ran detection on, the average detection time was 1.71s. Among the 1,754 packages, 1,483 were completed within 1s; 1,652 were completed within 5s; only 59 needed more than 10s to complete; 5 packages did not finish execution within 2 minutes. The CDF figure of the detection time is shown in~\autoref{fig:cdf-execution}.} {We manually verified these bug reports and confirmed that {71} of them were true positive, indicating an overall precision of {75.5\%}.} {The information on packages and bugs that have either been confirmed or reported for over three months is provided in \autoref{top1k-zerodays}}. Many bugs we detected are not trivial. {Among the 71 type confusion bugs detected by \TN, there are {50} bugs found in packages with more than 1,000 stars on GitHub}. These packages are well known in the Rust community and are usually maintained well by professional teams. For example, {\tt candle-core}~\cite{candle-core} is developed by a famous AI company (Hugging Face)~\cite{HuggingFace}. Therefore, our results suggest that even domain experts may write error-prone code in Rust.




\begin{table}[]
\centering
\caption{{The results of detectors on top 3,000 packages.}}
\footnotesize
\begin{threeparttable}
\begin{tabular}{lcrrrr}
\toprule
\multicolumn{1}{c}{\multirow{2}{*}{Detector}}                                & \multirow{2}{*}{Metrics} & \multicolumn{3}{c}{Bug Types}                                            & \multicolumn{1}{c}{\multirow{2}{*}{Overall}} \\ \cline{3-5}
\multicolumn{1}{c}{}                                                         &                          & \multicolumn{1}{c}{I} & \multicolumn{1}{c}{II} & \multicolumn{1}{c}{III} & \multicolumn{1}{c}{}                         \\ \hline
\multirow{3}{*}{TypePulse}                                                   & TP & {32} & {24} & {15} & {71}    \\
                                                                             & FP & {6}                     & {4}                      & {13}                       & {23}    \\
                                                                             & Precision                & {84.2\%}                     & {85.7\%}                      & {53.6\%}                       & {75.5\%}    \\ \hline
\multirow{3}{*}{\begin{tabular}[c]{@{}l@{}}TypePulse\\ w/o IPA\end{tabular}} & TP                       & {32}                     & {24}                      & {13}                       & {69}   \\
                                                                             & FP                       & {11}                     & {6}                      & {13}                       & {30}    \\
                                                                             & Precision                & {74.4\%}                     & {80\%}                      & {50\%}                       & {69.7\%}    \\
\bottomrule
\end{tabular}
\begin{tablenotes}
\item IPA: Interprocedural Analysis.
\end{tablenotes}
\end{threeparttable}
\label{detection-table}
\end{table}



\vspace{0.05in}
\noindent{\bf Capability to Find New Bugs.} 
To facilitate the resolution of these bugs, we also created a Proof of Concept (PoC) to report identified issues to package maintainers, which requires locating undefined behaviors in packages based on the diagnostic information produced by \TN. To trigger potential bugs, we need to create the appropriate \textit{src\_ty} and \textit{dst\_ty} before invoking the problematic conversion. We consider two distinct scenarios. First, when the source and destination variables are openly accessible, such as through public members within a class, we can directly create these variables. Second, when the source and destination variables are private and not directly accessible, {a constructor is required to initialize the variables. Additionally, \TN helps find feasible constructor functions, which also accelerates the PoC generation}.






At the time of writing, we have documented all the issues ({71}) that we have manually confirmed, and {32} of these have received acknowledgment responses from the package maintainers. After notifying the maintainers, we further reported these issues to the \rs advisories and CVE database. So far, we have received {six \rs IDs and one CVE ID}, and are awaiting confirmation and releases for the remaining ones. 
\looseness=-1



\subsection{False Positive Analysis} \label{evaluation:fp}


According to~\autoref{detection-table}, there are {23} instances of false positives. There are 2 arising from misidentifying function visibilities and {3} from developers' tricks to prevent invalid pointer exposure. Here we investigate the remaining {18} cases that arise from misinterpretation during the \tc analysis. It shows that \TN cannot understand complex condition semantics in checks, especially when the check is implemented in an unusual way. Considering Listing \ref{listing:xous} for example, \TN considers the method \texttt{to\_str} could construct an illegal string from an array of \texttt{u8}, which might contain a non-utf8 character. After manually studying the type constructor function (\texttt{new}) and the method used to insert characters into the string (\texttt{push}), we find that the characters in the string are restricted to be utf8 by the encoding at line 15 (\texttt{encode\_utf8}). This complex encoding makes it difficult for \TN to avoid false alarms in detecting Mismatched Scope Bugs. So far, \TN can only detect standard developer-enforced check patterns and several unsafe APIs with clear safety documentation. It is challenging for \TN to understand such encoding operations. We will extend the capability of \TN to interpret \tc in future work.






\begin{lstlisting}[
language=rust, 
style=lst,
caption=The false positive case of Mismatched Scope bug.,
label=listing:xous,
mathescape=false]
impl<const N: usize> String<N> {
  pub fn new() -> String<N> {
    String { bytes: [0; N], len: 0 }
  }
  pub fn to_str(&self) -> &str {
    unsafe { 
    str::from_utf8_unchecked(&self.bytes[0..]) }
  }
  pub fn push(&mut self, ch: char) -> Result<..> {
    match ch.len_utf8() {
      1 => { ... }
      _ => {
        let mut bytes: usize = 0;
        let mut data: [u8; 4] = [0; 4];
        let subslice = ch.encode_utf8(&mut data);
        // ...
\end{lstlisting}












\subsection{Impacts of Interprocedural Analysis} \label{interimpacts}

We conduct an ablation study to demonstrate \TN{}'s capabilities in reducing false positives by disabling interprocedural analysis (refer to the second row in~\autoref{detection-table}). \TN{} reduces 6 more false positives and detects 2 more true positives with interprocedural analysis enabled. Scanning 3k packages, \TN obtains the results with precision {75.5\%}, consisting of {71} True Positives and {23} False Positives. After disabling the interprocedural analysis, \TN's precision is reduced to only {69.7\%}, consisting of {69} True Positives and {30} False Positives. This comparison result highlights that interprocedural analysis can significantly enhance precision by analyzing the context across functions. First, it can detect the positive cases relying on external functions to decide the bit patterns -- 2 more mismatched scope bugs were detected. Second, it can reduce the false alarms in cases with developer-enforced checks -- {7 more False Positives were reduced}. For example, in the package of \texttt{arrow-buffer}, disabling interprocedural analysis will cause four more false positive cases of misalignment bugs. The Listing \ref{listing:arrow-buffer} shows a false positive case mitigated by interprocedural analysis. \TN first locates suspicious type conversion at line 20 since it finds that the pointer of \texttt{buffer} is cast to arbitrary generic type. However, the generic type cannot be controlled by attackers since \texttt{as\_slice} is a method relying on \texttt{BufferBuilder} and casting \texttt{MutableBuffer} to generic type. The constructor function of \texttt{MutableBuffer} cannot be located by traditional interprocedural analysis because it is not the caller of the method. \TN implements the functionality to find out the constructor functions by matching the type to the ones returned from other functions. In this example, \TN locates the function \texttt{with\_capacity}, which returns \texttt{MutableBuffer} as the constructor function. After analyzing the constructor function, \TN finds that the constructor function already guarantees the alignment of type with \texttt{Layout::from\_size\_align}; therefore, this should be a false alarm. Without interprocedural analysis, \TN cannot detect the developer-enforced check implemented in the constructor function.





\begin{lstlisting}[
language=rust, 
style=lst,
caption=The false positive case resolved by interprocedural analysis.,
label=listing:arrow-buffer,
mathescape=false]
impl MutableBuffer {
  #[inline]
  pub fn with_capacity(capacity: usize) -> Self {
    let layout = Layout::
      from_size_align(capacity, ALIGNMENT).unwrap();
    let data = match layout.size() {
      0 => dangling_ptr(),
      _ => {
        let raw_ptr = unsafe { 
          std::alloc::alloc(layout) 
        };
    // ...
    Self { data, len: 0, layout }
// ...
impl<T: ArrowNativeType> BufferBuilder<T> {
  #[inline]
  pub fn as_slice(&self) -> &[T] {
    // ...
    unsafe { std::slice::from_raw_parts(
      self.buffer.as_ptr() as _, self.len) }
// ...
\end{lstlisting}






\subsection{Comparison with Existing Tools} \label{evaluation:comparison}
To the best of our knowledge, \TN is the first bug detection tool to systematically detect \bugs in Rust, 
so we are unable to find similar tools to perform an {\it apple-to-apple} comparison. We choose the tools that are able to partially detect \bugs for comparison: Clippy\cite{clippy} and Rudra\cite{Yechan2021Rudra}. We run them on the packages listed in \autoref{top1k-zerodays} and compare the results. The comparison results are shown in \autoref{comparison}. Additionally, we compare the performance of \TN to the Rust's type system on the existing type confusion bugs in RUSTSEC dataset (see \autoref{rustsec-details}).













\vspace{0.05in}
\noindent{\bf Comparison with Clippy.}
Clippy is a static analysis tool that implements more than 650 types of lints~\cite{Lintsoft25online} to detect common errors in the Rust program. 
Clippy supports 2 types of lints to find the unsound usages of \texttt{as} and \texttt{transmute}. 
First, it can check whether \texttt{as} can lead to a misaligned pointer (\texttt{cast\_ptr\_alignment}\cite{castptralignment}). 
Second, it can check whether \texttt{transmute} occurs between types of different Application Binary Interfaces (ABIs) (\texttt{unsound\_collection\_transmute}~\cite{unsoundcollectiontransmute}). 
The version of Clippy with which we compare is 0.1.72, and \autoref{comparison} reveals that Clippy identifies only {10} relevant bugs, all of which are misalignment issues (Type I) restricted to $Con \rightarrow Con$ ({21} bugs in total).
{Regarding the remaining 11 misalignment bugs, one warning is intentionally suppressed by developers, four arise from overlooking variations in ABIs, and six involve \texttt{transmute}.}
We summarize two main reasons for Clippy's limitation as follows.
Firstly, Clippy fails to identify potential bugs involving generic types, as the two lint checks are only carried out when both the \textit{source type} and \textit{dest type} are concrete. Secondly, Clippy does not have a comprehensive approach to identifying type conversion errors. It is observed that Clippy's checks vary between \texttt{as} and \texttt{transmute}; misalignment checks are applied to \texttt{as} but omitted for \texttt{transmute}. For \btwo bugs, checks are conducted exclusively on \texttt{transmute}.
Furthermore, upon reviewing developer comments, it is observed that developers often disregard the warnings due to their belief that Clippy generates a significant number of false positive cases.






\begin{table}[t]
\centering
\caption{{Comparison with Clippy and Rudra}.}
% \xz{this table: the numbers of \TN should be red?}
\footnotesize
\begin{tabular}{c c c c c} 
 \toprule
 Detector & Type I & Type II & Type III & Overall \\ %[0.5ex] 
 \midrule
 Clippy & {10} & 0 & 0 & {10} \\ 
 % \hline
 Rudra & 0 & 0 & 0 & 0 \\
 % \hline
 \TN & {32} & {24} & {15} & {71} \\
 % \hline
 \bottomrule
\end{tabular}
\label{comparison}
\end{table}





\vspace{0.05in}
\noindent{\bf Comparison with Rudra.} 
Rudra\cite{Yechan2021Rudra} is the bug detector that can be used to capture memory safety bugs from Rust packages. 
We compared \TN with Rudra for two reasons. 
First, Rudra is the state-of-the-art memory safety bug detector, reporting 51.6\% of all memory safety bugs. 
Second, Rudra claims it can find the bugs of uninitialized memory exposure, which is the result of \btwo bugs. 
%
Our evaluation results show that Rudra can find five bugs of uninitialized memory.
However, none of them occurs in the type conversion process,
% The five Type II bugs reported by Rudra and the 11 Type II bugs reported by \TN are mutually exclusive,
which means it is not effective in detecting type confusion bugs.
% So it totally doesn't work on this task.
% The main reason is that Rudra checks inappropriate patterns. 
The main reason is that,
for the patterns of type conversion, 
Rudra only detects the terminators at the MIR level.
Unfortunately, \texttt{transmute} has been translated to statements rather than terminators since 2021. 
Moreover, Rudra implements the dataflow checker of \texttt{transmute} but excludes \texttt{as}. 
As a result, Rudra is not able to detect type confusion bugs effectively.


\vspace{0.05in}
\noindent{\bf Comparison with Rust Type System.}
{
We also conduct the experiment to elaborate if  
the existing Rust type system can effectively detect type confusion bugs in \texttt{unsafe} code regions.
%\TN is
% more reliable than the Rust type system (\texttt{1.72.0-nightly}) in detecting type confusion bugs, especially in \texttt{unsafe}. 
%
We compare the positive cases of type confusion bugs detected by \TN and the Rust type system.
%
Since Rust considers \texttt{unsafe} keyword to be required for raw pointer dereferences and unsafe APIs, removing \texttt{unsafe} will introduce syntax errors.
%
For the purpose of calculating the precision, we count these unremovable \texttt{unsafe} as positive bug detections by Rust type system.
%
If there are multiple \texttt{unsafe} blocks in a single function, we count it as one. In other words, the functions with \texttt{unsafe} but no type confusion bugs being reported should be considered as false positive cases.
%\
%\xz{CHECK: for the purposes of calculating the true positives and false positives.}
%
% Note that we manually add \texttt{unsafe} to the function body of ?? cases since some type confusion bugs were reported on the raw pointer.
% %
% However, we followed the definition of Rust that only dereference will trigger bugs on raw pointer.
%
%Following this method to set up the experiment, 
We use all vulnerable files including the 32 existing bugs (\autoref{rustsec-details}) 
%\xz{should fix all refs to appendix}
as the benchmark and find that both \TN and the type system can detect all the existing type confusion bugs; however, the type system produces 116 false positive cases while \TN only produces 3.}
% \xz{seems unfinished?}

\noindent{\bf Summary.}
Our evaluation confirms that \TN is the most effective tool to detect type confusion bugs in Rust. 
Existing state-of-the-art Rust bug detection tools such as Clippy and Rudra are not as effective,
since they are not designed for detecting type confusion bugs.  
Our experiment also explains that the current Rust type system is not sufficient to check the unsafe type conversion, highlighting the necessity of \TN.



\subsection{Impacts of Type Confusion Bugs} \label{evaluation:sec-impacts}

{
Memory errors might occur due to type confusion bugs, especially if the target type allows access to memory that the source type cannot reach.
The type confusion bugs discovered by \TN have different security implications. 
Among them, {28} trigger panics, {24} cause uninitialized memory access, {8} lead to out-of-bounds access, {7} construct illegal types, and 4 can generate data race issues}.
%
Using case studies of the \texttt{pprof} package~\cite{pprof}, a popular Rust-based CPU profiler with 1.3k stars on Github and 159 crates.io dependents, we show the impacts of bugs and how it helps diagnose Rust performance bottlenecks. The bug was reported and confirmed by the developers on Github. 
%
Besides, we also provide other case studies and corresponding PoCs in \autoref{sec:appendixB}.
% which has been assigned with a RUSTSEC-ID. 



\begin{lstlisting}[
language=rust, 
style=lst,
caption=Misaligned bug found in the \texttt{pprof} package.,
label=listing:pprof_bug1,
mathescape=false]
impl<'a, T> Iterator for TempFdArrayIterator<'a, T> {
  type Item = &'a T;
  fn next(&mut self) -> Option<Self::Item> {
    // ...
    let length = self.file_vec.len()/size_of::<T>();
    let ts = unsafe {
      slice::from_raw_parts(
        self.file_vec.as_ptr() as *const T,
        length)
      };

pub fn build(&self) -> Result<Report> {
    let mut hash_map = HashMap::new();
    match self.profiler.write().as_mut() {
      Err(err) => {...}
      Ok(profiler) => {
        profiler.data.try_iter()?
          .for_each(|entry| { .. }
\end{lstlisting}


\vspace{0.05in}
\noindent{\bf \Bone Bugs}.
% We detect a bug
As shown in Listing \ref{listing:pprof_bug1}, \TN detects a \bone bug (line 8) in the \texttt{next} function implemented on \texttt{TempFdArrayIterator}. 
% How does the bug happen?
When the unsafe \texttt{slice::from\_raw\_parts} is called (line 11), it assumes the caller meets safety contracts. The raw pointer must be aligned, non-null, and point to \texttt{length} bytes of initialized values\cite{fromrawparts}. Violating these safety contracts causes a panic. 
% How is the bug triggered in real-world cases?
One way to invoke the \texttt{next} function is to build reports from a running profiler; it will iterate each entry to process the data and write them into reports (line 12). 
The generic type \texttt{T} in \texttt{TempFdArrayIterator} is decided by the item stored in the \texttt{profiler.data} (line 17), which is \texttt{UnresolvedFrames}. \texttt{UnresolvedFrames} is a representation of an event backtrace, and it is a self-defined \texttt{struct} type with fields of \texttt{u64}, \texttt{usize}, an array of \texttt{u8}. 
Since the \texttt{file\_vec} is aligned to 1 byte, any type that has a larger alignment than 1 byte can lead to the \bone bug and crash here.

% start introducing how ecosystem could be easily affected by this bug
Famous Rust-based applications like GreptimeDB\cite{greptimedb} (4.1k stars on GitHub) are affected by this bug. GreptimeDB, a time series database storing logs, events, and CPU usage, crashes when calling pprof (\texttt{report::ReportBuilder::build}) to build reports. This panic can obstruct queries or data writes, causing real-time network monitoring applications to miss identifying high CPU usage, leading to network performance decline.
%
{For example, given a GreptimeDB-based public network server that provides the interface of event backtrace, the attacker could initialize the data \texttt{T} with the type aligned to 2 bytes, causing the network server panic and rendering it unavailable to users.}
% How many applications are affected? 
After reviewing related GitHub issues, we traced the 	\texttt{ReportBuilder::build} code pattern and searched on GitHub. 
% \xz{line 30 is not ReportBuilder::Build!}
Over 230 code files show similar patterns and may be affected.





