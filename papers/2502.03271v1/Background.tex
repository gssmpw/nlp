%-------------------------------------------------------------------------------
\section{Background}
\label{sec:bg}

\subsection{Rust Basics}

\noindent{\bf Generic Types and Traits.}
Rust provides the flexibility of code reuse and type safety with generic types and traits \cite{GenericD71:online, traits}. By writing code in a type-agnostic manner, generic types allow functions, methods, or data structures (e.g., struct) to operate on multiple types, which are represented by the placeholder (e.g., \texttt{T}, \texttt{U}). Since the generic types will be replaced by the actual types at the compile-time (i.e., monomorphization), it can prevent the \bugs at run-time. In addition, traits can be used to specify the constraints on generic types. By applying a trait to a generic type, Rust ensures that the type used to replace the generic type should also implement the required methods or characteristics defined by the trait. For example, the function \texttt{fn display<T: Copy>(input: T)} ensures that only the type implementing the \texttt{Copy} trait can be initialized as the argument \texttt{input}.

\vspace{0.05in}
\noindent{\bf Safe vs. Unsafe Rust.}
The central concept of safe Rust is to confirm memory {\it ownership} during compilation, where the compiler checks both the access and the {\it lifetime} of memory-allocated objects (or values). Moreover, safe Rust permits the {\it borrowing} of a value (i.e., making a reference to it) throughout the lifetime of the owner variable. In contrast, \texttt{unsafe} is used to highlight code segments that perform tasks that are not ensured by the compiler, placing the responsibility on developers to prevent memory safety issues. In Rust, there are five specific actions necessitating the \texttt{unsafe} keyword~\cite{unsafeRu19online}: dereferencing raw pointers, invoking unsafe functions, altering or accessing mutable static variables, defining unsafe traits, and executing inline assembly. Each of these actions could breach Rust's safety assurances. Unsafe Rust is crucial because it allows developers to interface with low-level system APIs, libraries written in other languages, or hardware directly.  


\vspace{0.05in}
\noindent{\bf Undefined Behaviors.}
Undefined behavior (UB) refers to the program whose outcome is not prescribed by the language's specification, which means that the language standard does not define what should happen if the UB occurs. In most cases, the result can only be decided by hardware and architectures, leading to inconsistent consequences in different environments. The outcomes of UB are unpredictable, ranging from security vulnerabilities to incorrect compiler optimization and code generation. The backend of the compiler might perform the optimization based on the assumption that the UB will not occur. Therefore, we should prevent UB from happening. Rust clearly defines scenarios that might trigger undefined behaviors~\cite{ubreference}, such as dereferencing null pointers, accessing out-of-bounds array elements, and data races when mutating shared data without synchronization. The design of \texttt{unsafe} help us narrow down the culprit of UB to the code related to unsafe code.


\subsection{Type Conversion in Rust}
Type conversion from the source type (\texttt{src\_ty}) to the destination type (\texttt{dst\_ty}) consists of two steps: 
\BC{1} {\it Conversion}, which involves altering or reinterpreting the bit pattern of a variable from one type (\texttt{src\_ty}) to a new type (\texttt{dst\_ty}), and \BC{2} {\it Access}, which involves accessing the variable as the new type (\texttt{dst\_ty}).
\texttt{rustc} limits developers to performing only explicit type conversions to maintain safety with compile-time verifications. These conversions can be implemented through type casting, transmute operations, and traits~\cite{traits}. Given that traits are typically handled by casting and transmute methods, our discussion will focus solely on casting and transmute operations.


\lstdefinestyle{lst}{
    float=tp,
    floatplacement=tbp,
    % abovecaptionskip=0.01in,
    numbers=left, 
    numberstyle=\scriptsize, 
    numbersep = 5pt,
    framexleftmargin = 0in,
    framexrightmargin = 0in,
    xleftmargin = 0.18in,
    xrightmargin = 0.1in,
    basicstyle=\ttfamily\scriptsize, 
    frame=lines,
    showtabs=true,
    showspaces=true,
    showstringspaces=false,
    literate={\ }{{\ }}1,
}
% \lstset{belowskip=0.17in}

\begin{lstlisting}[
language=rust, 
style=lst,
caption=Type conversion between pointers in \texttt{unsafe} code.,
label=listing:motivation1,
mathescape=true]
fn main() {
  let source_ty: u8 = 1;
  // compile error: non-primitive cast
  let dest_ty = &source_ty as &u32;  
  // Alternative 1: as
  let tmp_ty= &source_ty as *const u8 as *const u32;
  let dest_ty = unsafe {&*tmp_ty};
  // Alternative 2: transmute
  let dest_ty = unsafe { 
    transmute::<&u8, &u32>(&source_ty) 
  };
}
\end{lstlisting}

\vspace{0.05in}
\noindent{\bf Casting Operation.}
The type-casting operation depends on the keyword \texttt{as}, which is mainly used for secure and direct type conversions, including conversions between basic data types and raw pointers. The use of \texttt{as} generally involves straightforward bit manipulations or adjustments in values. For example, when an \texttt{f32} is converted to an \texttt{i32}, the fractional component of the floating point number is removed. Consequently, using \texttt{as} can result in data truncation or loss. It's important to note that under the stringent regulations established by \texttt{rustc}, \texttt{as} can be employed in the {\tt safe} code.

\vspace{0.05in}
\noindent{\bf Transmute Operation.}
Compared to \texttt{as}, the \texttt{transmute} function facilitates more intricate and dangerous transformations. Essentially, \texttt{transmute} performs a bitwise copy from one type to another without altering the bit pattern of the value. However, it modifies the interpretation of these bits by \texttt{rustc}. For instance, it allows for the direct conversion of an \texttt{i32}'s bit pattern to an \texttt{f32}, despite their fundamentally different representations. The validity of the original bit pattern in the new type is not assured, making \texttt{transmute} extremely risky and prone to causing undefined behaviors. Consequently, \texttt{transmute} necessitates the use of \texttt{unsafe} code.

\vspace{0.05in}
\noindent{\bf Type Conversion between Pointers.}
In Rust, pointer-type conversions can be achieved through casting and the use of transmute operations. However, Rust imposes various restrictions depending on the pointer types involved, meaning that some conversions are safely handled by Safe Rust, while others require the use of the \texttt{unsafe} keyword. For example, in Listing~\ref{listing:motivation1}, Safe Rust prohibits direct conversion between reference types (line 3), forcing developers to resort to two methods within \texttt{unsafe} code. The first method involves converting the reference of the original type to a raw pointer, followed by its conversion to another raw pointer (line 5). To acquire a reference of the new type, developers must first dereference the raw pointer and then form a new reference (line 6). As dereferencing a raw pointer is not allowed in Safe Rust, the use of \texttt{unsafe} code becomes unavoidable. Alternatively, the \texttt{transmute} function can be used directly to convert references (line 9), which must be used with the \texttt{unsafe} keyword due to its inherent risks. Such conversions in Unsafe Rust are prone to type conversion errors because the memory address remains the same for both pointers~(\texttt{\&src\_ty} and \texttt{\&dst\_ty}), while only the interpretation of the type changes.




\begin{table}[t] \label{rustsec-table}
\centering
\caption{The details of 32 reported Type Confusion Bugs on RustSec advisories (2019-2024).} \label{rustsec-details}
\vspace{0.05in}
\footnotesize
\begin{tabular}{c c c c c}
\toprule
  Year   & Type I & Type II  & Type III & Others \\ \midrule
2019 & 2019-0035 & - & 2019-0028 & - \\ \hline
2020 & 
\begin{tabular}[c]{@{}c@{}}2020-0035\\ 2020-0050\end{tabular} & 
\begin{tabular}[c]{@{}c@{}}2020-0029\\ 2020-0078\\ 2020-0079\\ 2020-0080\\ 2020-0081\\ \end{tabular} & 
\begin{tabular}[c]{@{}c@{}}2020-0029\\ 2020-0165 \end{tabular} &
\begin{tabular}[c]{@{}c@{}}2020-36317\\ 2020-0073\\ 2020-0164\end{tabular} \\ \hline
2021 & 
\begin{tabular}[c]{@{}c@{}}2021-0120\\ 2021-0121\\ 2021-0145\end{tabular} & 
\begin{tabular}[c]{@{}c@{}}2021-0021\\ 2021-0035\end{tabular} & 
\begin{tabular}[c]{@{}c@{}}2021-0019\\ 2021-0089\end{tabular} & 2021-0044 \\ \hline
2022 & 2022-0041 & 
\begin{tabular}[c]{@{}c@{}}2022-0052\\ 2022-0074\end{tabular} & 
2022-0092 & 
\begin{tabular}[c]{@{}c@{}}2022-0034\\ 2022-0078\end{tabular} \\ \hline
2023 & - & - & 
\begin{tabular}[c]{@{}c@{}}2023-0015\\ 2023-0055\end{tabular}  & - \\ \hline
2024 & - & 2024-0347 & 
\begin{tabular}[c]{@{}c@{}}2024-0001\end{tabular}  & - \\ 
\bottomrule
\end{tabular}
\end{table}




\section{Overview}

\subsection{Motivating Examples}
\label{sec:ty-bug}
We investigate all bug reports related to \bug{s} in the RUSTSEC advisories in the last five years~\cite{rustsec}. There are 32 reports but only 26 \bugs; The remaining six bugs are out of consideration because they are related to other memory safety problems, such as Use-After-Free, which can be handled by existing tools~\cite{Qin2020ReplicationPF, Yechan2021Rudra, Zhuohua2021MirChecker, safedrop}. The Bug IDs are listed in \autoref{rustsec-details}. We categorize the 26 bugs into three types based on their root causes, which are \bone, \btwo, and \bthree. All 26 bugs are related to pointer type conversion (i.e., references and raw pointers) in Unsafe blocks. In this section, we introduce the three bug types and their security impacts.


\vspace{0.05in}
\noindent\textbf{Type I: \Bone Bug.} The first type of bug occurs when the type is converted to another type leading to alignment violation. The alignment of a type specifies the valid memory address at which the type should be stored.  Given the alignment value as \texttt{n}, the type must be stored only at the address of a multiple of \texttt{n}. Some types have a fixed alignment regardless of the target architectures, while others could be platform-specific.  For example, the type of \texttt{i32} has both 4-byte alignments on the 32-bit or 64-bit target. In contrast, the types of \texttt{usize} and \texttt{isize} are aligned to 4 bytes on the 32-bit target and 8 bytes on the 64-bit target. The alignment requirement can be easily violated with pointer-type conversion since two pointers remain in the same memory address. Although the memory address is aligned for \texttt{src\_ty}, it might not be aligned for \texttt{dst\_ty} if not handled carefully. In the Listing \ref{listing:motivation-type1}, the method \texttt{fill\_bytes} allows the slice of \texttt{u8} to be cast onto the slice of \texttt{u32} (line 9). Since \texttt{u8} is aligned to 1 byte, the slice of \texttt{dest} can be stored at the arbitrary memory address. When it turns out to be accessed as \texttt{u32}, it is not guaranteed that the memory address can be multiple of 4 since \texttt{u32} is aligned to 4 bytes, leading to the misaligned pointer dereference. Note that developers of \texttt{rand\_core} consider that the issue could be avoided by limiting the target architectures to \texttt{x86} or \texttt{x86\_64} only since these architectures are designed to tolerate misaligned memory access. However, the alignment requirement is enforced by the compiler instead of these target architectures. Once the Rust compiler verifies that safe code adheres to alignment rules, it generates optimized machine code based on this assurance. However, if unsafe code violates these rules, it can cause undefined behavior or crash the program.
\looseness=-1



\lstdefinestyle{lst}{
    float=tp,
    floatplacement=tbp,
    % abovecaptionskip=0.01in,
    numbers=left, 
    numberstyle=\scriptsize, 
    numbersep = 5pt,
    framexleftmargin = 0in,
    framexrightmargin = 0in,
    xleftmargin = 0.18in,
    xrightmargin = 0.1in,
    basicstyle=\ttfamily\scriptsize, 
    frame=lines,
    showtabs=true,
    showspaces=true,
    showstringspaces=false,
    literate={\ }{{\ }}1,
}

\begin{lstlisting}[
language=rust, 
style=lst,
caption=A misalignment bug in \texttt{rand\_core} that casts bytes slices to integer slices (RUSTSEC-2019-0035 \cite{rs-2019-0035}).,
label=listing:motivation-type1,
mathescape=true]
#[cfg(any(target_arch = "x86",
  target_arch = "x86_64"))]
fn fill_bytes(&mut self, dest: &mut [u8]) {
  // ...

  while filled < end_direct {
    let dest_u32: &mut R::Results = unsafe { 
      &mut *(dest[filled..].as_mut_ptr() as 
      *mut <R as BlockRngCore>::Results) };
    self.core.generate(dest_u32);
    filled += self.results.as_ref().len() * 4;
    self.index = self.results.as_ref().len();
  }
  // ...
}
\end{lstlisting}




\vspace{0.05in}
\noindent\textbf{Type II: Inconsistent Layout Bug.} \label{motiv:bugII} The second type of bug occurs when \texttt{src\_ty} and \texttt{dst\_ty} have different memory layouts. 
%
In Listing \ref{listing:motivation-type2}, the method \texttt{as\_ref} allows casting between \texttt{Table} and \texttt{TableSlice} and returns the reference to the new type (line 20). 
%
However, when the \texttt{struct} type in Rust inherits the default representation (e.g. \texttt{repr(Rust)}), the compiler may reorder the memory layout, such as the fields of \texttt{struct}. 
%
The results of GDB show that after the raw pointer to \texttt{Table} is converted to the \texttt{TableSlice}, the fields \texttt{rows} of \texttt{Table} and one of \texttt{TableSlice} point to different memory addresses (line 26 and line 28), leading to inconsistent lengths of \texttt{rows} (line 27 and line 29).
%
It could impact applications that rely on the value (e.g., \texttt{rows.length}). 
For example, when applications plan to print all the data stored in table format to the terminal, the API (\texttt{TableSlice::print\_tty}) converts \texttt{Table} to \texttt{TableSlice} first and iterates the data stored in \texttt{rows}. 
%
Since iterating slice relies on the length (line 29) while the number of elements is actually only one (line 27), printing table leads to invalid memory access and segmentation fault, which has been reported in RUSTSEC-2022-0074.



\begin{lstlisting}[
language=rust, 
style=lst,
caption=An inconsistent layout bug in \texttt{prettytable-rs} that casts a \texttt{\&Vec} to \texttt{\&[T]} (RUSTSEC-2022-0074 \cite{rs-2022-0074}).,
label=listing:motivation-type2,
mathescape=false]
pub struct Table {
  format: Box<TableFormat>,
  titles: Box<Option<Row>>,
  rows: Vec<Row>,
}

pub struct TableSlice<'a> {
  format: &'a TableFormat,
  titles: &'a Option<Row>,
  rows: &'a [Row],
}

impl<'a> AsRef<TableSlice<'a>> for Table
fn as_ref(&self) -> &TableSlice<'a> {
  unsafe {
    let s = &mut *(
      (self as *const Table) as *mut Table
    );
    s.rows.shrink_to_fit();
    &*(self as *const Table as *const TableSlice<'a>)
  }
}

// From GDB results
// $8 as &Table, $7 as &TableSlice
p &$8.rows       // 0x7ff..82f0
p &$8.rows.len   // 1
p &$7.rows       // 0x7ff..82e0
p $7.rows.length // 93825009397280!
\end{lstlisting}







\vspace{0.05in}
\noindent\textbf{Type III: Mismatched Scope Bug.} The third type of bug occurs when we break the invariant by creating an invalid bit pattern or modifying the mutability of types. In Listing \ref{listing:motivation-type-3}, the trait \texttt{ComponentBytes} is designed to provide a method \texttt{as\_bytes\_mut} to modify the type \texttt{T} as byte slice. The type \texttt{T} could be any type that implements the traits \texttt{Copy}, \texttt{Send}, \texttt{Sync}, and lifetime bound \texttt{static}. However, the trait and lifetime bounds here cannot prevent the issues caused by the problematic type conversion implemented in \texttt{as\_bytes\_mut}. It allows casting between mutable raw pointers (\texttt{slice.as\_mut\_ptr() as *mut u8}) to create an invalid state for types since two pointers are pointing to overlapping memory. Safe Rust enforces aliasing rules, where mutable and immutable references can not exist simultaneously, while unsafe Rust allows the rule to be bypassed, as shown in the exploit (line 21 - 27). The attacker creates an immutable reference pointing to the static string as the type \texttt{T}. With \texttt{as\_bytes\_mut}, the mutable raw pointer to the slice of string casts to the mutable raw pointer of \texttt{u8} type. Since the function returns a mutable reference to a slice of \texttt{u8}, the attacker is allowed to modify any values in the slice of \texttt{u8}. However, the mutable reference \texttt{bytes} and immutable reference \texttt{component} point to the same data, breaking the aliasing rules of safe Rust. While the attacker modifies the value in \texttt{bytes}, he also changes the value of \texttt{component}, which should not be mutated originally. One security consequence of modifying immutable data is data races. In a multi-threaded environment, if the immutable object can be modified through a mutable reference while other threads are reading it, the outcome could be unpredicted or even lead to a program crash. In addition, applications usually rely on the static variable for security checks or maintaining a global state, which means that mutating the immutable data can also help attackers bypass security checks. \looseness=-1




\begin{lstlisting}[
language=rust, 
style=lst,
caption=A mismatched scope bug in \texttt{rgb} that allows viewing and modifying data of any type wrapped in \texttt{ComponentSlice<T>} as bytes (RUSTSEC-2020-0029\cite{rs-2020-0029}).,
label=listing:motivation-type-3,
mathescape=false]
pub trait ComponentBytes<T: Copy+Send+Sync+'static>
  where Self: ComponentSlice<T> {
  fn as_bytes_mut(&mut self) -> &mut [u8] {
    let slice = self.as_mut_slice();
    unsafe {
      slice::from_raw_parts_mut(
        slice.as_mut_ptr() as *mut _, ..)
    }
  }
}

impl<T> ComponentSlice<T> for [RGB<T>] { .. }

impl<T> RGB<T> {
  pub const fn new(r: T, g: T, b: T) -> Self {
    Self {b, g, r}
  }
}

// exploit for type III bug
let component: &'static str = "Hello, World!";
let new_rgb = RGB::new(component, .., ..);
let mut rgb_arr = [new_rgb; 3];
let bytes = rgb_arr.as_bytes_mut();
bytes[0] += component.len() as u8;
// now, we can modify static memory
println!("{}", rgb_arr[0]);
\end{lstlisting}





\subsection{Challenges and Insights}
\label{sec:challenge}
The Rust type system's features, such as ownership, trait bounds, and generic types, present challenges that cannot be directly addressed using existing methods in C and C++. We use the example of the \bthree bug to illustrate the three challenges.

\vspace{0.05in}
\noindent{\bf Interprocedural Type Conversion.} \label{motive:call-graph-construction}
{In Listing \ref{listing:motivation-type-3}, \TN identifies that the \texttt{as\_bytes\_mut} function performs a risky type conversion from a generic type to \texttt{u8} on line 7, potentially leading to a type confusion bug. To confirm the presence of this bug, \TN must determine if the type is converted between functions. Line 3 indicates that type \texttt{self} is initialized by a constructor function then passed to the current function. However, finding the constructor is a challenging problem that traditional call graphs cannot address. For instance, the type of \texttt{[RGB<T>]} implements \texttt{ComponentSlice} (refer to line 12) but must be initialized via the \texttt{new} function shown on line 15. Traditional call graphs can locate callers of \texttt{as\_bytes\_mut}, but the type constructor is not typically a direct caller (see lines 22 and 24). To address this issue, we identify the constructor by matching converted types to the return types of external functions via a new data structure called Property Graph}.














\vspace{0.05in}
\noindent{\bf Generic Type Resolution.} \label{motiv:generic-analysis}
{To predict the concrete types that can initialize generic type \texttt{T}, we first analyze trait bounds. The generic type \texttt{T}} is constrained by four trait bounds: \texttt{Copy}, \texttt{Send}, \texttt{Sync}, and \texttt{'static}.
Suppose we only enumerate the types directly involved in these trait bounds; we only get some internal types, such as \texttt{RGB<ComponentType>} and \texttt{BGR<ComponentType>}.
Since \texttt{ComponentBytes} is a public trait, the user can initialize it with external user-defined types, such as \texttt{str} for the generic type (line 21). 
To address such an issue, we extend {\it \analysisone} to {resolve the trait bounds}.
First, if these traits are also bounded by other traits, we need to parse the dependencies recursively.
Second, if the function is public,
we must consider all the primitive types and composite types that external users could
initialize. Therefore,
we collect traits and primitive types defined in standard libraries. {For the composite types, we build the \texttt{struct} type from primitive fields}. Finally, {\bugdetector can leverage the trait bounds to generate a type set. The type set includes all possible types that may be implicitly implemented and satisfy the trait bounds.}




\vspace{0.05in}
\noindent{\bf Alias Analysis.} \label{motiv:pointer-analysis}
{To verify the existence of the bug, it is crucial for us to determine whether the pointer alias remains valid after the type conversion (whether we can access \texttt{rgb\_arr} after line 24)}. 
The pointer type conversion on line 7 is translated to \texttt{\_8 = move \_9 as *mut u8 (PtrToPtr)} in the Mid-level Intermediate Representation (MIR \cite{MIR}) of the Rust compiler.
%\xz{plz explain MIR/HIR in background}
\texttt{move} represents the transfer of ownership of a value to another, which means \texttt{\_9} is not accessible anymore.
However, the previous alias of \texttt{\_9} still point to the same memory address.
Therefore, \texttt{slice} on line 4 is still accessible after the ownership of its mutable pointer (\texttt{\_9}) is transferred, leading to \bthree bug when accessing \texttt{rgb\_arr[0]} (line 27).
To precisely identify the alias relationship between the pointers, we analyze whether the pointer's ownership is transferred based on different forms of instruction in the Rust program.
In the Listing \ref{listing:motivation-type-3}, \analysistwo helps us verify that the parameter (\texttt{\&mut self}) points to the same memory location as the \texttt{u8} pointer (\texttt{\&mut [u8]}), and whether the parameter remains accessible after returning.
\looseness=-1


%\xz{how does it work exactly? not clear; cannot get the insight (done)}



% {done}