\section{Conclusion}

In this paper, we develop \TN, the {\it first} static analysis tool for detecting \bugs in Rust. 
\TN focuses on detecting the three most common categories of \bugs --- \bone, \btwo, and \bthree.
%
\TN detected {71} previously unknown bugs from the top 3,000 Rust packages. 
This number surpasses the number of \bugs documented in the last five years in RustSec,
which shows the effectiveness of \TN.
The identified bugs were reported to the developers, who have confirmed {32} of these issues.
%
We also compare \TN with existing Rust bug detection tools, 
and perform case studies to demonstrate the security implications of the identified bugs.
\TN will be open-sourced to facilitate future research.





\section*{Acknowledgment}
We thank our shepherd and the reviewers for their insightful
feedback. This work is partially supported by ONR grant N00014-23-1-2122 and the IDIA P3 Faculty Fellowship from George Mason University.
% \xz{update}

% \newpage
\section*{Open Science}

To promote transparency and reproducibility in our research, the data artifacts of this paper will be made publicly available, including source code, detected bugs, and related github issues we reported. We disclose only those issues that have been acknowledged and resolved by developers. Issues that remain unresolved at the time of writing are not included in detail.
All data is available on Zenodo: \url{https://zenodo.org/records/14750104}.



\section*{Ethics Considerations}
We take ethics seriously in this project.
All Rust repositories we tested in the paper are publicly accessible on Github.
During evaluations of our work, \TN identified several previously unknown type confusion vulnerabilities in widely used software. In each case, we followed a responsible disclosure policy, and reported our discovered vulnerabilities to the developers. We also submitted our findings to the CVE program and the RustSec Advisory Database. 
We did not disclose those issues to anyone else.
All the examples mentioned in the paper are the issues that have been acknowledged and fixed. 
The RustSec IDs issued are:
RUSTSEC-2023-0046, RUSTSEC-2023-0047, RUSTSEC-2024-0408, RUSTSEC-2024-0424, RUSTSEC-2024-0426, RUSTSEC-2024-0431;
The CVE ID is currently in the \texttt{reserved} status at the time of writing and will be released later.
% \xz{we have one CVE?}