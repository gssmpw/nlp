\section{Discussion}

\noindent{\bf Rust vs. C++ on Type Confusion Bugs.}
{
The distinction of compiler features, type systems, and memory management in Rust and C++ lead to different types of \bugs and new challenges in detecting them. First, }while pointer-type conversion is the primary cause of bugs in both languages, the \btwo bug and the \bthree bug mentioned in this paper do not occur in C++. This is because the C++ compiler does not rearrange the memory layout of composite types by default (thus no \textit{unstable\_ty}), and it also does not impose strict requirements on the bit-pattern as Rust does.
From a different angle, \bugs in C++, which consistently occur when downcasting an object from a parent class to a child class, does not exist in Rust.
This is because Rust lacks the object characteristics needed for such issues.
Rust introduces the concept of a trait object~\cite{traitobjects}, comparable to C++ objects, to define shared behaviors. However, since trait objects do not involve inheritance, they prevent object-based \bugs.
{Second, before resolving generic types, extracting traits completely is more difficult in Rust since Rust's traits can be implicitly bounded while C++'s concept must be explicit.
Lastly, Rust's implicit memory management brings convenience to developers by avoiding manual management. However, it actually makes pointer alias analysis harder. To verify whether the pointer is still valid, the detector must ensure whether the memory is automatically dropped or the ownership is transferred}.
\looseness=-1


\vspace{0.05in}
\noindent{\bf Complementing Rust type system.}
{
\TN is necessary since expanding the type system protection from Safe Rust to include \texttt{unsafe} is insufficient, as shown in \autoref{evaluation:comparison}.
%
The type system's safety assurances are not derived from executing type checks. Rather, Safe Rust prevents the presence of an \textit{invalid type} right from the beginning.
%
Consider the example of a misaligned pointer: Safe Rust prevents developers from converting a pointer to one aligned with larger byte sizes (refer to Listing~\ref{listing:motivation1}). Due to this stringent rule, the compiler confidently assumes that misaligned pointers are not present in Safe Rust. Consequently, optimizations and code generation within the compiler's backend are carried out based on this assumption.
%
A pointer generated within \texttt{unsafe} is the only entity capable of circumventing these rules. 
Even if the pointer from \texttt{unsafe} is converted to a reference, the compiler will still treat this reference as reliable. This flawed assumption can result in undefined behavior. %, impacting code generation.
Therefore, a tool like \TN that aids in identifying an invalid type becomes essential for Rust developers.
}


\vspace{0.05in}
\noindent{\bf Mitigations of Type Confusion Bugs.}
We summarize common ways to fix the bugs detected by \TN.


\noindent\emph{Type I: misalignment bugs.}
Misaligned references are prohibited in safe code, but the safety of misaligned raw pointers depends on access methods. We propose two strategies to prevent misalignment issues before dereferencing: (1) use \texttt{read\_unaligned}~\cite{readunalign} or \texttt{write\_unaligned}~\cite{writeunalign}, which handle misaligned pointers, or (2) create a new aligned pointer. Most functions require aligned pointers; \texttt{read\_unaligned} creates an aligned duplicate by copying data (\texttt{copy\_nonoverlapping}) and casting to \texttt{u8}. Alternatively, developers can manually create a new pointer by adding an offset (\texttt{ptr.add}) to align the address. 


\noindent\emph{Type II: inconsistent layout bugs.}
To avoid the \btwo bug, we need to ensure the type's memory layout is consistent and stable. If \textit{dst\_ty} is a primitive type with initialized memory in consecutive bytes, \textit{src\_ty} must not have uninitialized bytes. Although most code avoids inconsistent type conversion, it can occur during generic type conversion. 
% The buggy function is a public API, and callers pass \texttt{struct} as the parameter, causing the bug. 
%
To prevent bugs in generic types, we can apply trait bounds --- list the types that can be legally converted and implement the trait on them. This ensures callers use only the defined types as parameters. A well-known trait implementing this concept is \texttt{bytemuck::Pod}~\cite{Podinbyt5online}.
%
For \texttt{struct}-to-\texttt{struct} conversions, developers often wrongly assume stable memory layout. To ensure stability, developers need to annotate types for conversion with \texttt{repr(C)} or \texttt{repr(transparent)}~\cite{Typelayo51online}. 


\noindent\emph{Type III: mismatched scope bugs.} 
Such bugs often occur in exposed APIs with generic type conversion. 
Developers should limit types and use trait bounds to restrict conversion and validate values before converting.
Libraries provide \texttt{unsafe} APIs like \texttt{from\_*\_unchecked} (e.g., \texttt{str::from\_utf8\_unchecked}~\cite{strutf8unchecked}) for type conversion, whose safety must be ensured by callers. 
Callers must validate that source type values are appropriate for destination types. 
For instance, \texttt{str::from\_utf8\_unchecked} requires UTF-8 valid input, 
unlike the safe \texttt{std::from\_utf8}~\cite{strutf8}, which checks this. 
Developers often use \texttt{from\_utf8\_unchecked} over \texttt{from\_utf8} to avoid overhead, but safe functions should be used in critical security scenarios.




\section{Limitations and Future Work}
% In this section, we summarize two limitations from the perspectives of false positive cases and the scope of our current design. 
{
As detailed in \autoref{evaluation:fp}, \TN has limitations in accurately interpreting different implementations of developer-enforced checks, leading to false positive cases.
%
% This issue presents two primary challenges. 
Some check patterns are inherently implicit and difficult to formalize. For example, calling size check functions but actually examining padding bytes.
Understanding this intricacy requires a deep contextual insights. 
Additionally, assessing the value within the check condition is particularly challenging when relying solely on static tools. 
Without incorporating dynamic analysis, assuring the accuracy of security checks becomes difficult, especially in large software projects. 
In future work, we aim to integrate symbolic execution in access checks, such as implementing constraints to confirm that a pointer's memory address cannot be a multiple of the type size before \TN flags misalignment issues.
}
{Furthermore, with a path limitation of 1 for interprocedural analysis, TypePulse struggles to grasp the context in extended call chains,
which we aim to address in future work.}