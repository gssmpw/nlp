\section{Methods}
\label{sec:methods}

We conducted interviews with mental health clinicians to explore how they would design health information technologies (HITs) that support value-based mental healthcare.
Methodologically, we were inspired by work in speculative design to imagine futures where VBC is mandated, and then brainstorm with participants how HITs could support VBC outcomes data storage, collection, and use \cite{hockenhull_speculative_2021, wong_speculative_2018}. 
In this section, we detail the study procedures, including participant recruitment (Section \ref{sec:methods:participants}), background information (Section \ref{sec:methods:participants-backgrounds}), how data was collected and analyzed (Section \ref{sec:methods:data}), and our positionality (Section \ref{sec:methods:positionality}). 
All study procedures were approved by the coauthors' institutional review board (IRB). 

\subsection{Participant Recruitment}
\label{sec:methods:participants}
We enrolled as participants mental health clinicians, specifically practicing psychiatrists, clinical psychologists, licensed clinical social workers (LCSWs), and licensed mental health counselors (LMHCs).
We intentionally recruited providers from these different clinical orientations to gather different perspectives on designing HITs \cite{mental_health_america_types_2024}. 
Participants were recruited via a combination of convenience, purposive, and snowball sampling \cite{etikan_comparison_2015, goodman_snowball_1961}.
Specifically, a recruitment email and flier were sent to staff working at academic medical centers across the United States. 
\rev{Recruitment emails were often forwarded to providers who worked in smaller, private practices or community health settings, to help us gain perspectives from mental health clinicians working in diverse settings, treating different types of patients.} 
Within the qualitative tradition \cite{braun_one_2021}, our goal for this work was not to gather perspectives representative of mental health clinicians as a whole, but instead to deep dive with our participants into the complexities of designing HITs that support VBC.

\subsection{Participants' Backgrounds}
\label{sec:methods:participants-backgrounds}

\rev{Table \ref{tab:participants} summarizes background information for the 30 mental health clinicians who participated in the study.
This background information was collected during an intake survey, which was administered after participants provided informed consent for our study.
Apart from data collected within this intake survey, we often asked participants during our study interviews to provide background information regarding their current payment arrangements.
Most of our participants took traditional, fee-for-service payments (public and private), or asked their private practice patients to pay for care out-of-pocket.
A few participants (eg, SW28) worked in health systems transitioning to value-based payments.
Many participants were unfamiliar with VBC.
}

\begin{table*}[t]
\begin{tabular}{ll}
\toprule
Number of participants & 30 mental health clinicians \\ 
\midrule
Clinical training           & 13 Clinical Psychology \rev{(CP)} \\
                            & 6 Psychiatry \rev{(PS)} \\    
                            & 8 Clinical Social Work \rev{(SW)} \\
                            & 2 Mental Health Counseling \rev{(MC)} \\
                            & 1 Family and Marriage Therapist \rev{(FT)} \\
\midrule
Practice setting            & 16 Academic Medical Center \\
                            & 14 Private Practice \\
                            & 5 Community Mental Health Center \\
                            & 2 Employee Assistance Program \\
\midrule
Geographic location (in the USA)    & 26 Northeast \\
                                    & 2 Southeast \\
                                    & 2 West Coast \\
\bottomrule
\end{tabular}
\caption{Background information of the study participants. Participants could list multiple practice settings.
\rev{Clinical training abbreviations (eg, ``CP'') are used within Section \ref{sec:findings}.}
}
% \Description{A table summarizing the backgrounds of the 30 participants we interviewed in this study. The table describes the clinical training of participants, the practice setting, and geographic location (in the United States) of each participant.}
\label{tab:participants}
\vspace{-5pt}
\end{table*}

\subsection{Data Collection and Analysis}
\label{sec:methods:data}

All participants were asked to provide informed consent after being provided complete information about the study procedures.
Interviews were held via Zoom over two 1-hour sessions attended by the first three authors, and participants were reimbursed \$30 per hour for their time.
The first session was a semi-structured interview where we asked clinicians about their current care practices, specifically how they used data -- defined broadly, collected with or without technology -- in care.
We specifically asked participants about their perspectives on \textit{measurement-based care} (MBC), the practice of collecting and using data in care that would power HITs supporting VBC \cite{kilbourne_measuring_2018}.
We then asked participants further questions about how they used this data to measure care outcomes, how technology was involved in this process, and whether providers were accountable to achieve certain care outcomes.
Interview questions were broad to allow for on-the-spot adaptation and probing \cite{barriball_collecting_1994}.

In the second session, participants completed two design prompts.
These prompts were motivated by work in speculative design \cite{hockenhull_speculative_2021, wong_influence_2008}, to imagine futures where MBC and VBC were mandated and to understand how clinicians would collect and report outcomes data as a part of these programs.
The first prompt asked participating clinicians to imagine a world where they were mandated to use outcomes data as a part of care, and to brainstorm what data they would prioritize.
The second prompt was motivated by the five-star quality rating system used by the United States Center for Medicare \& Medicaid services (CMS) \cite{center_for_medicare__medicaid_services_five-star_2022}.
Participating clinicians were asked to imagine that as a part of VBC, CMS wanted to design ``mental health quality star ratings'' to measure patient outcomes and care quality across clinics and health systems.
Participants were asked to brainstorm what data should be included in this new star rating program.
After responding to each prompt, we discussed with participants the data they included in their responses, and asked probing questions to further understand how HITs could support data storage, collection, and use.
Full interview guides can be found in Appendix \ref{appendix:guide}.

Interviews were recorded with participants' permission, transcribed by a professional service, and de-identified.
Transcripts were analyzed using a reflexive thematic analysis approach adopted from \cite{braun_using_2006}.
This approach combined both inductive and deductive elements.
Codes and themes arose from the data, but were guided by our research interests and the literature \cite{braun_one_2021}, specifically the stages of preparation, collection, and action from Li et al. \cite{li_stage-based_2010}.
The first author qualitatively coded all transcripts.
Codes were iteratively refined, resulting in a final codebook, and all transcripts were recoded using the final codebook.
Themes were developed from the codes by the first author, with support from the second and third authors who also participated in the interviews and validated that the themes represented participants' views.
The codebook used to generate each theme can be found in Appendix \ref{appendix:codebook}.

\subsection{Positionality}
\label{sec:methods:positionality}

The first, second, and third authors are graduate students in computer and information science. 
These authors recruited participants, collected, and analyzed all of the data. 
One author is a clinical researcher and practicing mental health clinician who worked with the first author on the study protocols, and did not participate in the study. 
Another author is a health policy researcher, who is an expert on both digital mental health and value-based care.
The final author is a researcher in computing and information science. 
All authors were based in the United States, and thus our findings and perspectives are greatly informed, and potentially limited by, our knowledge of the United States healthcare system.