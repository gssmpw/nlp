\section{Introduction}
\label{sec:intro}

Health information technologies (HITs) are transforming how health services collect, share, and use data.
Electronic health records (EHRs) collect clinical data on provided treatments and patients' health, which can be aggregated and shared with regulators \cite{birkhead_uses_2015}.
Personal devices, such as smartphones and wearables, create opportunities to bring data on behavior, physiology, and well-being from everyday life into clinical care \cite{mohr_personal_2017, torous_new_2017}.
These data streams are being repurposed to transform how we pay for health services through \textit{value-based care} (VBC) programs, where healthcare providers (eg, hospitals, clinicians) are paid based upon the ``value'' of care they deliver to patients \cite{world_economic_forum_moment_2023, lewis_value-based_2023}.
VBC programs are implemented by paying providers for effectively managing patients' health, instantiated by collecting data to quantify the \textit{quality of care} providers deliver \cite{world_economic_forum_moment_2023, donabedian_quality_1988}. 
VBC is well-motivated: healthcare spending is on the rise -- \rev{for many reasons, including increased prescription drug and medical device costs, increased service utilization, and a global aging population \cite{martin_national_2023, stucki_what_2023} -- but increased spending is not always associated with improved health outcomes \cite{gunja_us_2023}}.
VBC programs incentivize healthcare providers to deliver treatments that simultaneously improve health while reducing service utilization and cost \cite{mcclellan_improving_2017}. 
But, these new financial arrangements raise a variety of sociotechnical questions.
For example, what data quantifies the quality of care, and how should this data be collected and managed?
How should quality data hold health systems accountable to improve patient outcomes?
How do we design HITs that support this process?

In this work, we explored these questions in a specific area of healthcare with longstanding quality challenges: mental healthcare.
The prevalence of mental health disorders and need for treatment continues to rise \cite{wolf_scoping_2024, santomauro_global_2021, xue_mental_2024, mcbain_mental_2023}.
Despite the need for high quality mental health services, there is a large gap between evidence-based practices and delivered care \cite{institute_of_medicine_improving_2006}, and mental healthcare has been much slower to improve services compared to other healthcare specialties \cite{pincus_quality_2016}.
A recent review paper found that only 27\% of published clinical reports describe adequate adherence to mental health clinical practice guidelines \cite{bauer_review_2002}.
Data from the United States suggests that only half of publicly insured patients receive appropriate follow-up care after a mental health-related emergency department visit \cite{the_national_committee_for_quality_assurance_follow-up_2022}, \rev{and there are an inadequate number of inpatient psychiatric beds and/or care providers available to treat patients \cite{american_psychiatric_association_psychiatric_2022, kaiser_family_foundation_mental_2024}.
Simultaneously, some psychiatric hospitals keep patients in care longer then medically necessary \cite{silver-greenberg_how_2024}.}
Researchers and policymakers have proposed that VBC programs could incentivize health systems to reduce these quality gaps by creating financial incentives to improve patient outcomes \cite{hobbs_knutson_driving_2021}.
Given these challenges, in 2021, the National Committee for Quality Assurance, or NCQA -- a leading organization in the United States responsible for assessing care quality -- proposed a quality measurement framework for value-based mental healthcare \cite{niles_behavioral_2021}.
With this proposal, the NCQA placed a shared responsibility on policymakers, health insurance companies, and healthcare providers to coordinate the definition, measurement, and management of quality.
Through systems of \textit{joint accountability}, the NCQA recognized that these different stakeholders must work together to improve mental health outcomes.
But, the NCQA's proposal stopped short of defining what mental health outcomes these programs should use, how outcomes data should be collected, and how accountability should be shared.

We therefore studied the design of HITs that support outcomes data storage, collection, and use in VBC with an important stakeholder in this process: mental health providers, specifically practicing clinicians.
We focused this study on mental health clinicians since they will play an essential role in realizing VBC. 
Mental health clinicians use their expertise to decide what treatments patients receive, they collect data on patients' progress in treatment, and collected data is transformed into the quality metrics that will determine how clinicians are reimbursed for their services.
Thus, VBC holds clinicians financially accountable to make treatment decisions based upon specific quality metrics, and assumes that improving metrics will improve patients' health.
Mental health clinicians have found these new forms of accountability challenging.
For example, less than 20\% of mental health clinicians practice \textit{measurement-based care} -- the process of routinely collecting data to measure treatment outcomes and inform decision-making \cite{fortney_tipping_2017, kilbourne_measuring_2018} -- citing concerns that data collection is burdensome, data collected within MBC do not effectively measure care outcomes across patients \cite{tauscher_what_2021, barkham_routine_2023}, and data could be used punitively to influence clinicians' pay \cite{lewis_implementing_2019, desimone_impact_2023}. 
These tensions create opportunities to study with mental health clinicians what data better measure care outcomes, how this data should be collected, and the extent to which data could be repurposed to create more accountable care.

In this work, we contribute findings from a set of interviews with 30 \rev{U.S.-based} mental health clinicians to explore the design space for HITs that support (1) outcomes data specification, (2) collection, and (3) use as a part of value-based mental healthcare.
These three areas were inspired by Li et al's \textit{stage-based model of personal informatics} \cite{li_stage-based_2010}, specifically the stages of \textit{preparation, collection, and action}, applied in this work to study the HITs and people (mental health clinicians) that will prepare, collect, and use outcomes data in value-based mental healthcare.
Our findings center mental health clinicians' perspectives in the design of HITs supporting value-based care.
Specifically, participating clinicians advocated that HITs store functional and engagement outcomes for value-based mental healthcare, which were perceived as the proximal outcomes to provided services (Section \ref{sec:findings:preparation}); called for investments in HITs that reduce the burden of collecting outcomes data (Section \ref{sec:findings:collection}); and believed that outcomes data would need to hold providers, health insurers, and social services jointly accountable to improve care (Section \ref{sec:findings:action}).
We conclude with implications for research developing (Section \ref{sec:discussion:tech}) and designing (Section \ref{sec:discussion:design}) HITs to better align stakeholders' -- including payers, clinicians, and social services -- data needs for both VBC and patient care.
