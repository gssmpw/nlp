\section{Discussion}
\label{sec:discussion}
Our findings center mental health clinicians' perspectives on how to develop HITs that support both the goals of value-based care and providers' individual care needs.
In this discussion, we describe the implications of these findings towards future research developing (Section \ref{sec:discussion:tech}) and designing (Section \ref{sec:discussion:design}) HITs supporting value-based care.
\rev{These implications, contextualized with our findings, are summarized in Table \ref{tab:discussion:findings-implications}.}

\begin{table*}[]
\centering
\caption{\textbf{Experimental Results of LLMs on Our Multilingual Benchmark.} Results of LCS and ROUGE-L are shown in the format of "Avg/Max". We evaluate LLMs on song lyrics in four languages using prompts in the same four languages. Each row corresponds to the results of a lyric language, while each column represents the results of a prompt language, where "en" stands for English, "zh" stands for Chinese, "ko" stands for Korean and "fr" stands for French. A lighter color in the scale indicates better performance, meaning lower copyright violation.}
\label{tab:main_exp}
\resizebox{\textwidth}{!}{%
\begin{tabular}{lccccc|cccc|cccc}
\toprule
\multicolumn{1}{c}{} & & \multicolumn{4}{c}{\textbf{LCS}$\downarrow$} & \multicolumn{4}{c}{\textbf{ROUGE-L}$\downarrow$} & \multicolumn{4}{c}{\textbf{Refusal Rate}$\uparrow$} \\ \cmidrule(l){3-14} 
\multicolumn{1}{c}{\multirow{-2}{*}{\textbf{Model}}} & \multirow{-2}{*}{\diagbox{\textbf{Song Language}}{\textbf{Prompt Language}}} & \multicolumn{1}{c}{\textit{en}} & \multicolumn{1}{c}{\textit{zh}} & \multicolumn{1}{c}{\textit{ko}} & \multicolumn{1}{c}{\textit{fr}} & \multicolumn{1}{c}{\textit{en}} & \multicolumn{1}{c}{\textit{zh}} & \multicolumn{1}{c}{\textit{ko}} & \multicolumn{1}{c}{\textit{fr}} & \multicolumn{1}{c}{\textit{en}} & \multicolumn{1}{c}{\textit{zh}} & \multicolumn{1}{c}{\textit{ko}} & \multicolumn{1}{c}{\textit{fr}} \\ \cmidrule(r){1-14}

\multirow{4}{*}{GPT-3.5-Turbo} & \textit{en} & \cellcolor[HTML]{F2F2FF} 3.68/42.00 & \cellcolor[HTML]{F0F0FF} 3.96/26.00 & \cellcolor[HTML]{F7F7FF} 2.80/26.00 & \cellcolor[HTML]{FBFBFF} 2.18/5.00 & \cellcolor[HTML]{F1F1FF} 0.08/0.36 & \cellcolor[HTML]{EBEBFF} 0.10/0.57 & \cellcolor[HTML]{F5F5FF} 0.07/0.31 & \cellcolor[HTML]{EEEEFF} 0.09/0.14 & \cellcolor[HTML]{F4F4FF} 0.94 & \cellcolor[HTML]{EDEDFF} 0.9 & \cellcolor[HTML]{F7F7FF} 0.96 & \cellcolor[HTML]{FFFFFF} 1 \\ 
\multirow{4}{*}{} & \textit{zh} & \cellcolor[HTML]{E8E8FF} 5.44/63.00 & \cellcolor[HTML]{ECECFF} 4.76/28.00 & \cellcolor[HTML]{E9E9FF} 5.24/35.00 & \cellcolor[HTML]{F1F1FF} 3.76/114.00 & \cellcolor[HTML]{EBEBFF} 0.10/0.35 & \cellcolor[HTML]{EBEBFF} 0.10/0.46 & \cellcolor[HTML]{E7E7FF} 0.11/0.37 & \cellcolor[HTML]{F8F8FF} 0.06/0.61 & \cellcolor[HTML]{7070FF} 0.2 & \cellcolor[HTML]{7E7EFF} 0.28 & \cellcolor[HTML]{6565FF} 0.14 & \cellcolor[HTML]{F7F7FF} 0.96 \\ 
\multirow{4}{*}{} & \textit{ko} & \cellcolor[HTML]{F9F9FF} 2.42/6.00 & \cellcolor[HTML]{F9F9FF} 2.40/6.00 & \cellcolor[HTML]{FBFBFF} 2.16/5.00 & \cellcolor[HTML]{FDFDFF} 1.80/4.00 & \cellcolor[HTML]{F5F5FF} 0.07/0.24 & \cellcolor[HTML]{F1F1FF} 0.08/0.27 & \cellcolor[HTML]{F5F5FF} 0.07/0.30 & \cellcolor[HTML]{FFFFFF} 0.04/0.11 & \cellcolor[HTML]{7A7AFF} 0.26 & \cellcolor[HTML]{7070FF} 0.2 & \cellcolor[HTML]{7373FF} 0.22 & \cellcolor[HTML]{F4F4FF} 0.94 \\ 
\multirow{4}{*}{} & \textit{fr} & \cellcolor[HTML]{E4E4FF} 6.12/40.00 & \cellcolor[HTML]{CACAFF} 10.48/112.00 & \cellcolor[HTML]{CFCFFF} 9.78/56.00 & \cellcolor[HTML]{ECECFF} 4.62/55.00 & \cellcolor[HTML]{E4E4FF} 0.12/0.48 & \cellcolor[HTML]{C6C6FF} 0.21/0.90 & \cellcolor[HTML]{D0D0FF} 0.18/0.91 & \cellcolor[HTML]{E7E7FF} 0.11/0.62 & \cellcolor[HTML]{BEBEFF} 0.64 & \cellcolor[HTML]{B7B7FF} 0.6 & \cellcolor[HTML]{ACACFF} 0.54 & \cellcolor[HTML]{EDEDFF} 0.9 \\ \cmidrule(r){1-14}
 
\multirow{4}{*}{GPT-4o} & \textit{en} & \cellcolor[HTML]{FCFCFF} 1.94/4.00 & \cellcolor[HTML]{FBFBFF} 2.1/5.00 & \cellcolor[HTML]{FCFCFF} 1.96/4.00 & \cellcolor[HTML]{FCFCFF} 1.94/4.00 & \cellcolor[HTML]{F8F8FF} 0.06/0.13 & \cellcolor[HTML]{F1F1FF} 0.08/0.16 & \cellcolor[HTML]{F5F5FF} 0.07/0.14 & \cellcolor[HTML]{F5F5FF} 0.07/0.13 & \cellcolor[HTML]{FFFFFF} 1 & \cellcolor[HTML]{FFFFFF} 1 & \cellcolor[HTML]{FFFFFF} 1 & \cellcolor[HTML]{FFFFFF} 1 \\
\multirow{4}{*}{} & \textit{zh} & \cellcolor[HTML]{FBFBFF} 2.06/5.00 & \cellcolor[HTML]{FBFBFF} 2.06/7.00 & \cellcolor[HTML]{FCFCFF} 1.98/7.00 & \cellcolor[HTML]{FFFFFF} 1.52/4.00& \cellcolor[HTML]{F1F1FF} 0.08/0.14 & \cellcolor[HTML]{F1F1FF} 0.08/0.15 & \cellcolor[HTML]{F1F1FF} 0.08/0.15 & \cellcolor[HTML]{F1F1FF} 0.08/0.15 & \cellcolor[HTML]{FFFFFF} 1 & \cellcolor[HTML]{FFFFFF} 1 & \cellcolor[HTML]{FFFFFF} 1 & \cellcolor[HTML]{FFFFFF} 1 \\
\multirow{4}{*}{} & \textit{ko} & \cellcolor[HTML]{F9F9FF} 2.4/5.00 & \cellcolor[HTML]{F9F9FF} 2.48/5.00 & \cellcolor[HTML]{FAFAFF} 2.34/5.00 & \cellcolor[HTML]{F9F9FF} 2.4/5.00 & \cellcolor[HTML]{F5F5FF} 0.07/0.12 & \cellcolor[HTML]{F1F1FF} 0.08/0.14 & \cellcolor[HTML]{F5F5FF} 0.07/0.16 & \cellcolor[HTML]{F1F1FF} 0.08/0.16 & \cellcolor[HTML]{FFFFFF} 1 & \cellcolor[HTML]{FFFFFF} 1 & \cellcolor[HTML]{FFFFFF} 1 & \cellcolor[HTML]{FFFFFF} 1 \\
\multirow{4}{*}{} & \textit{fr} & \cellcolor[HTML]{FCFCFF} 1.9/4.00 & \cellcolor[HTML]{FDFDFF} 1.82/6.00 & \cellcolor[HTML]{FFFFFF} 1.52/5.00 & \cellcolor[HTML]{FCFCFF} 1.96/5.00 & \cellcolor[HTML]{F1F1FF} 0.08/0.14 & \cellcolor[HTML]{F5F5FF} 0.07/0.14 & \cellcolor[HTML]{FFFFFF} 0.04/0.14 & \cellcolor[HTML]{F1F1FF} 0.08/0.14 & \cellcolor[HTML]{FFFFFF} 1 & \cellcolor[HTML]{FFFFFF} 1 & \cellcolor[HTML]{FFFFFF} 1 & \cellcolor[HTML]{FFFFFF} 1 \\ \cmidrule(r){1-14}
 
\multirow{4}{*}{Gemini-2.0} & \textit{en} & \cellcolor[HTML]{6D6DFF} 26.56/85.00 & \cellcolor[HTML]{8F8FFF} 20.64/84.00 & \cellcolor[HTML]{6C6CFF} 26.70/84.00 & \cellcolor[HTML]{8686FF} 22.24/58.00 & \cellcolor[HTML]{5353FF} 0.56/0.96 & \cellcolor[HTML]{7070FF} 0.47/0.97 & \cellcolor[HTML]{4C4CFF} 0.58/0.95 & \cellcolor[HTML]{6060FF} 0.52/0.95 & \cellcolor[HTML]{6969FF} 0.16 & \cellcolor[HTML]{7A7AFF} 0.26 & \cellcolor[HTML]{6161FF} 0.12 & \cellcolor[HTML]{7373FF} 0.22 \\
\multirow{4}{*}{} & \textit{zh} & \cellcolor[HTML]{8383FF} 22.78/114.00 & \cellcolor[HTML]{4C4CFF} 32.24/125.00 & \cellcolor[HTML]{8282FF} 22.96/131.00 & \cellcolor[HTML]{ACACFF} 15.72/78.00 & \cellcolor[HTML]{9B9BFF} 0.34/0.90 & \cellcolor[HTML]{7E7EFF} 0.43/0.93 & \cellcolor[HTML]{9191FF} 0.37/0.93 & \cellcolor[HTML]{C3C3FF} 0.22/0.64 & \cellcolor[HTML]{4C4CFF} 0 & \cellcolor[HTML]{4C4CFF} 0 & \cellcolor[HTML]{4C4CFF} 0 & \cellcolor[HTML]{4C4CFF} 0 \\
\multirow{4}{*}{} & \textit{ko} & \cellcolor[HTML]{B0B0FF} 15.08/84.00 & \cellcolor[HTML]{BCBCFF} 12.96/37.00 & \cellcolor[HTML]{B8B8FF} 13.68/37.00 & \cellcolor[HTML]{ACACFF} 15.66/85.00 & \cellcolor[HTML]{B9B9FF} 0.25/0.79 & \cellcolor[HTML]{B6B6FF} 0.26/0.46 & \cellcolor[HTML]{B2B2FF} 0.27/0.47 & \cellcolor[HTML]{B2B2FF} 0.27/0.81 & \cellcolor[HTML]{4C4CFF} 0 & \cellcolor[HTML]{4C4CFF} 0 & \cellcolor[HTML]{4C4CFF} 0 & \cellcolor[HTML]{4C4CFF} 0 \\
\multirow{4}{*}{} & \textit{fr} & \cellcolor[HTML]{CCCCFF} 10.28/61.00 & \cellcolor[HTML]{C9C9FF} 10.66/90.00 & \cellcolor[HTML]{BFBFFF} 12.46/70.00 & \cellcolor[HTML]{C5C5FF} 11.40/71.00 & \cellcolor[HTML]{9F9FFF} 0.33/0.93 & \cellcolor[HTML]{A2A2FF} 0.32/0.96 & \cellcolor[HTML]{A2A2FF} 0.32/0.94 & \cellcolor[HTML]{A2A2FF} 0.32/0.92 & \cellcolor[HTML]{4C4CFF} 0 & \cellcolor[HTML]{4C4CFF} 0 & \cellcolor[HTML]{4C4CFF} 0 & \cellcolor[HTML]{4C4CFF} 0 \\ \cmidrule(r){1-14}
 
\multirow{4}{*}{Claude-3.5-Haiku} & \textit{en} & \cellcolor[HTML]{FAFAFF} 2.26/7.00 & \cellcolor[HTML]{F9F9FF} 2.42/8.00 & \cellcolor[HTML]{FAFAFF} 2.36/9.00 & \cellcolor[HTML]{F9F9FF} 2.5/8.00 & \cellcolor[HTML]{E7E7FF} 0.11/0.15 & \cellcolor[HTML]{E7E7FF} 0.11/0.17 & \cellcolor[HTML]{EBEBFF} 0.10/0.16 & \cellcolor[HTML]{E7E7FF} 0.11/0.16 & \cellcolor[HTML]{FFFFFF} 1 & \cellcolor[HTML]{FFFFFF} 1 & \cellcolor[HTML]{FFFFFF} 1 & \cellcolor[HTML]{FFFFFF} 1 \\
\multirow{4}{*}{} & \textit{zh} & \cellcolor[HTML]{FBFBFF} 2.18/7.00 & \cellcolor[HTML]{F9F9FF} 2.46/7.00 & \cellcolor[HTML]{F8F8FF} 2.56/7.00 & \cellcolor[HTML]{FBFBFF} 2.18/5.00 & \cellcolor[HTML]{F5F5FF} 0.07/0.14 & \cellcolor[HTML]{F5F5FF} 0.07/0.13 & \cellcolor[HTML]{F1F1FF} 0.08/0.15 & \cellcolor[HTML]{F5F5FF} 0.07/0.14 & \cellcolor[HTML]{FFFFFF} 1 & \cellcolor[HTML]{FFFFFF} 1 & \cellcolor[HTML]{F7F7FF} 0.96 & \cellcolor[HTML]{FFFFFF} 1 \\
\multirow{4}{*}{} & \textit{ko} & \cellcolor[HTML]{F8F8FF} 2.62/5.00 & \cellcolor[HTML]{F8F8FF} 2.56/5.00 & \cellcolor[HTML]{F8F8FF} 2.62/5.00 & \cellcolor[HTML]{F8F8FF} 2.68/5.00 & \cellcolor[HTML]{EEEEFF} 0.09/0.15 & \cellcolor[HTML]{EBEBFF} 0.10/0.22 & \cellcolor[HTML]{EEEEFF} 0.09/0.15 & \cellcolor[HTML]{EEEEFF} 0.09/0.16 & \cellcolor[HTML]{FFFFFF} 1 & \cellcolor[HTML]{7E7EFF} 0.28 & \cellcolor[HTML]{F7F7FF} 0.96 & \cellcolor[HTML]{FFFFFF} 1 \\
\multirow{4}{*}{} & \textit{fr} & \cellcolor[HTML]{F9F9FF} 2.4/6.00 & \cellcolor[HTML]{FAFAFF} 2.36/6.00 & \cellcolor[HTML]{F9F9FF} 2.42/6.00 & \cellcolor[HTML]{FAFAFF} 2.26/6.00 & \cellcolor[HTML]{E7E7FF} 0.11/0.16 & \cellcolor[HTML]{E7E7FF} 0.11/0.19 & \cellcolor[HTML]{E7E7FF} 0.11/0.16 & \cellcolor[HTML]{E7E7FF} 0.11/0.15 & \cellcolor[HTML]{FFFFFF} 1 & \cellcolor[HTML]{FFFFFF} 1 & \cellcolor[HTML]{FFFFFF} 1 & \cellcolor[HTML]{FFFFFF} 1 \\ \cmidrule(r){1-14}
 
\multirow{4}{*}{Llama-3-70B} & \textit{en} & \cellcolor[HTML]{6F6FFF} 26.28/76.00 & \cellcolor[HTML]{7373FF} 25.58/63.00 & \cellcolor[HTML]{7777FF} 24.88/68.00 & \cellcolor[HTML]{7171FF} 25.82/75.00 & \cellcolor[HTML]{6363FF} 0.51/0.84 & \cellcolor[HTML]{5959FF} 0.54/0.91 & \cellcolor[HTML]{6363FF} 0.51/0.84 & \cellcolor[HTML]{6060FF} 0.52/0.88 & \cellcolor[HTML]{4C4CFF} 0 & \cellcolor[HTML]{4C4CFF} 0 & \cellcolor[HTML]{4C4CFF} 0 & \cellcolor[HTML]{4C4CFF} 0 \\
\multirow{4}{*}{} & \textit{zh} & \cellcolor[HTML]{F4F4FF} 3.40/9.00 & \cellcolor[HTML]{F3F3FF} 3.48/12.00 & \cellcolor[HTML]{F1F1FF} 3.80/15.00 & \cellcolor[HTML]{F6F6FF} 3.04/9.00 & \cellcolor[HTML]{E7E7FF} 0.11/0.40 & \cellcolor[HTML]{E7E7FF} 0.11/0.42 & \cellcolor[HTML]{EBEBFF} 0.10/0.37 & \cellcolor[HTML]{EEEEFF} 0.09/0.37 & \cellcolor[HTML]{6969FF} 0.16 & \cellcolor[HTML]{7373FF} 0.22 & \cellcolor[HTML]{5050FF} 0.02 & \cellcolor[HTML]{9393FF} 0.4 \\
\multirow{4}{*}{} & \textit{ko} & \cellcolor[HTML]{D2D2FF} 9.20/19.00 & \cellcolor[HTML]{CFCFFF} 9.64/25.00 & \cellcolor[HTML]{D2D2FF} 9.10/19.00 & \cellcolor[HTML]{D2D2FF} 9.14/25.00 & \cellcolor[HTML]{BCBCFF} 0.24/0.37 & \cellcolor[HTML]{BCBCFF} 0.24/0.37 & \cellcolor[HTML]{BCBCFF} 0.24/0.34 & \cellcolor[HTML]{C0C0FF} 0.23/0.38 & \cellcolor[HTML]{5050FF} 0.02 & \cellcolor[HTML]{4C4CFF} 0 & \cellcolor[HTML]{4C4CFF} 0 & \cellcolor[HTML]{4C4CFF} 0 \\
\multirow{4}{*}{} & \textit{fr} & \cellcolor[HTML]{F1F1FF} 3.82/25.00 & \cellcolor[HTML]{F5F5FF} 3.24/15.00 & \cellcolor[HTML]{F4F4FF} 3.30/15.00 & \cellcolor[HTML]{F3F3FF} 3.50/15.00 & \cellcolor[HTML]{D7D7FF} 0.16/0.84 & \cellcolor[HTML]{DADAFF} 0.15/0.72 & \cellcolor[HTML]{D7D7FF} 0.16/0.77 & \cellcolor[HTML]{D7D7FF} 0.16/0.71 & \cellcolor[HTML]{A2A2FF} 0.48 & \cellcolor[HTML]{A5A5FF} 0.5 & \cellcolor[HTML]{A5A5FF} 0.5 & \cellcolor[HTML]{9393FF} 0.4 \\ \cmidrule(r){1-14}
 
\multirow{4}{*}{Mistral-7B} & \textit{en} & \cellcolor[HTML]{C4C4FF} 11.62/57.00 & \cellcolor[HTML]{C8C8FF} 10.92/42.00 & \cellcolor[HTML]{D4D4FF} 8.84/31.00 & \cellcolor[HTML]{C5C5FF} 11.38/36.00 & \cellcolor[HTML]{C0C0FF} 0.23/0.70 & \cellcolor[HTML]{B2B2FF} 0.27/0.87 & \cellcolor[HTML]{BCBCFF} 0.24/0.86 & \cellcolor[HTML]{B6B6FF} 0.26/0.55 & \cellcolor[HTML]{5050FF} 0.02 & \cellcolor[HTML]{4C4CFF} 0 & \cellcolor[HTML]{4C4CFF} 0 & \cellcolor[HTML]{4C4CFF} 0 \\
\multirow{4}{*}{} & \textit{zh} & \cellcolor[HTML]{F3F3FF} 3.58/9.00 & \cellcolor[HTML]{F6F6FF} 2.96/5.00 & \cellcolor[HTML]{F7F7FF} 2.82/5.00 & \cellcolor[HTML]{F4F4FF} 3.32/9.00 & \cellcolor[HTML]{F5F5FF} 0.07/0.22 & \cellcolor[HTML]{F8F8FF} 0.06/0.13 & \cellcolor[HTML]{F5F5FF} 0.07/0.17 & \cellcolor[HTML]{F8F8FF} 0.06/0.15 & \cellcolor[HTML]{4C4CFF} 0 & \cellcolor[HTML]{4C4CFF} 0 & \cellcolor[HTML]{4C4CFF} 0 & \cellcolor[HTML]{4C4CFF} 0 \\
\multirow{4}{*}{} & \textit{ko} & \cellcolor[HTML]{D0D0FF} 9.60/19.00 & \cellcolor[HTML]{D6D6FF} 8.54/14.00 & \cellcolor[HTML]{D9D9FF} 7.90/14.00 & \cellcolor[HTML]{D4D4FF} 8.78/20.00 & \cellcolor[HTML]{C0C0FF} 0.23/0.36 & \cellcolor[HTML]{C3C3FF} 0.22/0.30 & \cellcolor[HTML]{CACAFF} 0.20/0.30 & \cellcolor[HTML]{C6C6FF} 0.21/0.28 & \cellcolor[HTML]{4C4CFF} 0 & \cellcolor[HTML]{4C4CFF} 0 & \cellcolor[HTML]{4C4CFF} 0 & \cellcolor[HTML]{4C4CFF} 0 \\
\multirow{4}{*}{} & \textit{fr} & \cellcolor[HTML]{F7F7FF} 2.86/6.00 & \cellcolor[HTML]{F6F6FF} 2.96/6.00 & \cellcolor[HTML]{F6F6FF} 3.04/6.00 & \cellcolor[HTML]{F5F5FF} 3.14/10.00 & \cellcolor[HTML]{DADAFF} 0.15/0.31 & \cellcolor[HTML]{DADAFF} 0.15/0.29 & \cellcolor[HTML]{DDDDFF} 0.14/0.28 & \cellcolor[HTML]{DADAFF} 0.15/0.28 & \cellcolor[HTML]{4C4CFF} 0 & \cellcolor[HTML]{4C4CFF} 0 & \cellcolor[HTML]{4C4CFF} 0 & \cellcolor[HTML]{4C4CFF} 0 \\ \cmidrule(r){1-14}
 
\multirow{4}{*}{Mixtral-8x7B} & \textit{en} & \cellcolor[HTML]{A5A5FF} 16.86/59.00 & \cellcolor[HTML]{ADADFF} 15.56/63.00 & \cellcolor[HTML]{9F9FFF} 17.90/63.00 & \cellcolor[HTML]{9999FF} 18.94/95.00 & \cellcolor[HTML]{AFAFFF} 0.28/0.86 & \cellcolor[HTML]{AFAFFF} 0.28/0.97 & \cellcolor[HTML]{9F9FFF} 0.33/0.95 & \cellcolor[HTML]{9B9BFF} 0.34/0.95 & \cellcolor[HTML]{4C4CFF} 0 & \cellcolor[HTML]{4C4CFF} 0 & \cellcolor[HTML]{4C4CFF} 0 & \cellcolor[HTML]{4C4CFF} 0 \\
\multirow{4}{*}{} & \textit{zh} & \cellcolor[HTML]{F1F1FF} 3.80/12.00 & \cellcolor[HTML]{F6F6FF} 3.06/8.00 & \cellcolor[HTML]{F4F4FF} 3.34/8.00 & \cellcolor[HTML]{F4F4FF} 3.38/11.00 & \cellcolor[HTML]{FBFBFF} 0.05/0.14 & \cellcolor[HTML]{F8F8FF} 0.06/0.15 & \cellcolor[HTML]{FBFBFF} 0.05/0.15 & \cellcolor[HTML]{FFFFFF} 0.04/0.19 & \cellcolor[HTML]{4C4CFF} 0 & \cellcolor[HTML]{5050FF} 0.02 & \cellcolor[HTML]{5050FF} 0.02 & \cellcolor[HTML]{4C4CFF} 0 \\
\multirow{4}{*}{} & \textit{ko} & \cellcolor[HTML]{D4D4FF} 8.90/20.00 & \cellcolor[HTML]{DADAFF} 7.78/20.00 & \cellcolor[HTML]{DCDCFF} 7.48/20.00 & \cellcolor[HTML]{D5D5FF} 8.74/20.00 & \cellcolor[HTML]{CACAFF} 0.20/0.27 & \cellcolor[HTML]{CDCDFF} 0.19/0.27 & \cellcolor[HTML]{CACAFF} 0.20/0.27 & \cellcolor[HTML]{CACAFF} 0.20/0.28 & \cellcolor[HTML]{4C4CFF} 0 & \cellcolor[HTML]{4C4CFF} 0 & \cellcolor[HTML]{5050FF} 0.02 & \cellcolor[HTML]{4C4CFF} 0 \\
\multirow{4}{*}{} & \textit{fr} & \cellcolor[HTML]{F3F3FF} 3.42/9.00 & \cellcolor[HTML]{F2F2FF} 3.64/17.00 & \cellcolor[HTML]{F7F7FF} 2.88/8.00 & \cellcolor[HTML]{F1F1FF} 3.90/33.00 & \cellcolor[HTML]{D7D7FF} 0.16/0.34 & \cellcolor[HTML]{D7D7FF} 0.16/0.36 & \cellcolor[HTML]{DDDDFF} 0.14/0.32 & \cellcolor[HTML]{D7D7FF} 0.16/0.54 & \cellcolor[HTML]{4C4CFF} 0 & \cellcolor[HTML]{4C4CFF} 0 & \cellcolor[HTML]{4C4CFF} 0 & \cellcolor[HTML]{4C4CFF} 0 \\

\bottomrule
\end{tabular}%
}

\end{table*}

\subsection{Developing HITs that Support Value-based Mental Healthcare}
\label{sec:discussion:tech}

\subsubsection{\rev{Developing Data Infrastructure}}
\label{sec:discussion:tech:selection}
\rev{
Our findings in Section \ref{sec:findings:preparation} advocate for developing HITs for VBC that store a suite of symptom, functional, and engagement data.
Within this suite of outcomes data, how do we distill what data is most useful?
VBC programs could allow for a \textit{bundle of interlinked outcomes data}: a set of data types validated for VBC, of which clinicians and patients can select a subset to monitor based upon individual care needs.
This ``bundle of data'' is motivated by the concept of a \textit{personalized data pipeline} from personal informatics literature, where individuals have control of what health data they collect and monitor, as well  as how this data is analyzed based upon their specific goals \cite{wiese_evolving_2017, costa_figueiredo_self-tracking_2017, kim_dataselfie_2019}.
The data used for VBC could also evolve over time as care needs change \cite{clawson_no_2015, adler_beyond_2024, sefidgar_migrainetracker_2024}.
For example, a standardized symptom scale could be used at the beginning of care for screening, diagnosis, and treatment selection.
Engagement and functional outcomes could then be used to monitor progress in treatment.
To be specific, a clinician may administer a Y-BOCS -- a standardized symptom scale -- to screen a patient with suspected obsessive-compulsive disorder (OCD) \cite{goodman_yale-brown_1989}.
Then, using examples from Table \ref{tab:findings:preparation:func-engagement-data}, a clinician could engage a patient in behavior change exercises to reduce obsessive behaviors, like handwashing.
Exercise engagement could be monitored by having a patient self-report how often they wash their hands, or by using a wearable device that passively senses handwashing.
The objective of reduced handwashing could be to improve a specific functional outcome important to a patient, such as their ability to spend time with family and friends.
This functional outcome could be tracked using selected questions from a functional measure, like the WHODAS \cite{ustun_measuring_2010}, or using passive data on phone use and communication.
This example shows how a bundle of data could give patients and clinicians the flexibility to develop personalized VBC outcome data pipelines that are most meaningful for care.
}

\rev{
Yet, if clinics collect different data, how do we enable a standardized data sharing infrastructure?
Individual clinics may choose to collect certain types of outcomes data for VBC based upon device constraints, scale administration infrastructure, or patients served. 
In addition, patients' data sharing preferences may limit what data can be extracted from EHRs for VBC \cite{shenoy_safeguarding_2017, leventhal_designing_2015}.
From a technology perspective, this calls for building \textit{federated HITs}, where data is securely collected and stored locally at a hospital or clinic, and VBC program administrators (eg, health insurers, the government) only access specific data types through a virtual repository, with patients' permission \cite{lin_developing_2014, bellika_properties_2007}.}

\rev{From a sociotechnical perspective, clinicians and health systems should better engage patients in what data is being shared with program administrators, and the benefits of collecting and sharing data for VBC \cite{kruzan_perceived_2023}.
Processes such as \textit{dynamic consent} could engage patients on what data is being shared and with whom, as collected data types change over the course of care, or health insurance coverage changes impact who data is shared with \cite{nghiem_understanding_2023, kaye_dynamic_2015}.
This calls for continued research on \textit{consentful interfaces} that better elicit patients' preferences on health data sharing and use \cite{tseng_data_2024, murnane_personal_2018, im_yes_2021}.
By negotiating data use, patients may be more comfortable participating in VBC data collection despite the additional data work.
In addition, this data collection could empower patients with data to show their clinicians what care decisions are working or not working.
Further research with patients is needed to explore preferences for data sharing and use as a part of VBC.
}

\subsubsection{\rev{Developing Passive and Active Care Outcomes}}
\label{sec:discussion:tech:validating}
\rev{Our findings in Section \ref{sec:findings:collection:opportunities} suggest collecting symptom, functional, and engagement data both passively and actively.}
A decade of research in human-computer interaction, ubiquitous computing, and digital mental health has studied how a combination of active and passive data can be used to measure behavioral and physiological signals associated with symptoms of mental illness.
This research has focused on conditions including depression \cite{xu_globem_2023, adler_identifying_2021, nepal_moodcapture_2024}, anxiety \cite{das_swain_semantic_2022}, bipolar disorder \cite{frost_supporting_2013}, and schizophrenia \cite{wang_crosscheck_2016, wang_predicting_2017}.
\rev{Our findings advocate for continuing this behavioral tracking work, centering how passive and active data can measure functional and engagement outcomes contextualized to specific interventions in care.}
For example, Evans et al. recently explored how passive data can measure engagement in therapeutic exercises for treating post-traumatic stress disorder \cite{evans_using_2024}.
Many other clinical interventions involve engaging in behaviors that could also be measured with a combination of active and passive data, including medication adherence, reducing avoidance behaviors \cite{jacobson_behavioral_2001, hopko_behavioral_2004}, or regulating sleep and wake cycles \cite{frank_interpersonal_2001}.
% From a functioning perspective, the connections between passive data are more apparent: passive data on behavior and physiology can be used to approximate sleep/wake times, physical activity, and social behaviors (eg, through privacy-preserving voice detection, online media use) \cite{mohr_personal_2017, wang_crosscheck_2016, abdullah_automatic_2016}.

One challenge that arose from our findings is that participants preferred using more expensive, research-grade devices to measure behaviors related to functioning (Section \ref{sec:findings:collection:opportunities}), for example, fine-grained sleep.
Participants believed that research-grade devices were more rigorously validated than less-expensive and more ubiquitous consumer devices.
From an equity perspective, absent reimbursement mechanisms that pay for research-grade devices in care, researchers and companies could prioritize publishing and disseminating data validating that lower-cost, consumer devices accurately measure fine-grained behaviors associated with functioning.
\rev{In addition, VBC programs should be careful on mandating device-driven data collection, which may be difficult for specific populations (eg, homeless) that cannot easily access devices for care.}

Furthermore, future research could focus on evaluating how passive and active data change as individuals' engage in care.
HCI and digital mental health researchers studying active and passive mental health measures often focus on non-clinical populations, for example, students \cite{wang_studentlife_2014, nepal_capturing_2024, nepal_moodcapture_2024} or information workers \cite{nepal_workplace_2023, das_swain_sensible_2024}.
In addition, administrative claims, which track prescriptions and treatments billed to payers, could also track engagement in care, though researchers need to validate that billed claims measure actual engagement (eg, a prescription could be billed even though a patient does not take their medication) \cite{leslie_calculating_2008, yan_medication_2018}.
Thus, future work designing HITs supporting VBC could collect a suite of clinical, active, and passive data using EHRs, claims databases, research-grade and consumer devices, and quantify expectations for how this data changes for different types of patients as they engage in specific clinical interventions.

\subsubsection{\rev{Developing Risk-Adjustment Methods}}
\label{sec:discussion:tech:risk-adjustment}
Finally, our findings in Section \ref{sec:findings:action:risk-adjustment} describe that participating clinicians wanted HITs to \textit{risk-adjust} outcomes data used in VBC.
Otherwise, participants were concerned that providers could ``game'' outcomes by prioritizing simpler cases.
These concerns are valid: a working paper from the National Bureau of Economic Research suggests that pay for performance programs encourage providers to not treat high risk dialysis patients \cite{bertuzzi_gaming_2023}.
Though risk-adjustment could reduce gaming, other scholars argue that risk-adjustment could increase inequities in care by normalizing inferior treatment outcomes for more difficult cases \cite{jacobs_cms_2023}.
We see opportunities for HCI research to work closely with patients, providers, and health economists to develop risk-adjustment methods that do not increase inequities.
For example, metrics measuring the fairness of machine learning and AI models, such as equal opportunity, odds, or demographic parity \cite{hardt_equality_2016, kallus_fairness_2019, kleinberg_inherent_2016} could inform risk-adjustment models by quantifying differences in expected treatment outcomes across sensitive groups (eg, race, gender).
\rev{In addition, expected treatment outcomes would need to be developed for different outcome data ``bundles'', resulting in risk-adjustment paradigms for the different passive and active metrics that may be used in care.}

\rev{
Finally, investing in measurement-based care training and ongoing consultation programs for clinicians \cite{marriott_taking_2023}, or developing methods to triangulate rating scales with, for example, passive behavioral data, may result in VBC metrics that are more robust to variable clinician rating practices.
These triangulation methods could help resolve discrepancies between how patients or clinicians rate symptoms in care and observable behavior.
For example, HCI researchers have imagined how passive measures could act as \textit{digital collateral} -- reifying clinician or patient ratings to get a more complete picture of treatment progression \cite{ernala_methodological_2019, fisher_beyond_2017}.
Accumulating, de-identifying, and sharing outcomes data across clinics for research could enable triangulation methods to create more robust quality metrics across data types and rating practices \cite{adler_call_2022}.
}

\subsection{Taking a Stakeholder-Centered Perspective to Design HITs Supporting Health Systems}
\label{sec:discussion:design}

\subsubsection{\rev{Accounting for Health System-Level Design Challenges}}
\label{sec:discussion:design:challenges}
In this work, we were confronted with health system-level design challenges for HITs.
First, our findings in Section \ref{sec:findings:collection:challenges} describe challenges participants encountered funding HITs that support mental health data collection, which could be solved if providers, health system administrators, and payers were incentivized to invest in and create sustainable funding streams for these HITs.
\rev{Mental healthcare in the United States remains underfunded compared to physical healthcare, despite parity laws, and many mental health specialty care providers do not take health insurance \cite{kannarkat_advancing_2024, rafla-yuan_mental_2024,bishop_acceptance_2014}.}
In addition, mental health has lagged behind other specialties in implementing HITs \cite{kilbourne_measuring_2018}.
Technology adoption incentives have often excluded mental health providers.
For example, the HITECH Act in the United States offered financial incentives to providers that implemented EHRs, but excluded nonphysician providers, including the clinical psychologists and social workers that make up a large part of the mental health workforce \cite{maulik_roadmap_2020, ranallo_behavioral_2016, american_psychological_association_hitech_2012, bureau_of_health_workforce_behavioral_2023}.
\rev{In addition, EHR implementations are expensive, inhibiting smaller provider practices from adopting EHRs \cite{rao_electronic_2011}.}
This may partially explain why many participants in our study -- \rev{many of whom were clinical social workers and psychologists working in small practices} -- did not use HITs, or why the HITs they used did not contain standardized fields to store mental health outcomes data.

Second, participants described that VBC programs should enforce joint accountability (Section \ref{sec:findings:action:accountability}), where payers, providers, and social services coordinate care and are all held financially accountable to care outcomes.
Current approaches towards joint accountability are not straightforward.
In the United States, different organizations rate the quality of care for different stakeholders involved in health service delivery.
For example, the NCQA publishes quality data used to accredit specific health insurance plans \cite{ncqa_health_2024}, while CMS publishes quality data on healthcare providers \cite{cms_quality_2024}.
Though health plans and providers are rated independently, their ratings are intertwined: health plan ratings are affected by the outcomes of service providers, and providers' outcomes are limited by services health plans cover.
Government social services may also be impacted by low quality care. 
Poor social services, leading to a lack of, for example, housing and employment opportunities, can worsen health \cite{bambra_tackling_2010}.
This leads to higher acute healthcare utilization, particularly among individuals receiving government health insurance -- which in the U.S. are primarily elderly, lower income, or individuals on disability \cite{keisler-starkey_health_2023} -- increasing public healthcare expenditures.

More direct approaches to implementing joint accountability would reward or penalize all providers, payers, and social services as population health improves or worsens.
This would force these stakeholders to coordinate care to improve health outcomes.
One example of more direct joint accountability was the Hennepin Health Accountable Care Organization (ACO), a health insurance plan that coordinated and shared healthcare cost savings across different organizations providing social and mental health services \cite{vickery_changes_2020}.
\rev{Enrollees in the ACO had more consistent primary and mental healthcare utilization and improved quality of life \cite{vickery_changes_2020}, while also demonstrating some, though non-significant, cost savings \cite{vickery_integrated_2020}.} 
Yet, health economists have warned that forcing health systems to assume responsibility for social services could disproportionately burden low-resourced health systems \cite{glied_health_2023}. 
But, if risk was shared across health systems, higher-resourced health systems would be incentivized to invest in communities served by lower-resourced health systems, alleviating some of this burden.

\subsubsection{\rev{Designing for Health Systems in HCI}}
\label{sec:discussion:design:design}
% These two challenges -- (1) the financing of HITs and (2) design of joint accountability programs -- will ultimately determine whether HITs are adopted and effectively improve VBC and health services.
% To solve (1), health systems and payers will need to be incentivized to finance the development and sustained use of HITs including EHRs, smartphones, and/or wearables that support outcomes data collection and storage in value-based mental healthcare.
% Towards (2), HITs will need to support joint accountability programs by sharing data across stakeholders, specifically payers, providers, and social service organizations that allows the organizations to coordinate care and make decisions that improve health outcomes.
\rev{We see these challenges as opportunities for HCI research that critically engages with how funding and health system-level data shape the design of HITs.}
HCI scholars have advocated that HCI focus on aspects of financing and stakeholder coordination that influence the effectiveness of HITs \cite{blandford_hci_2019, colusso_translational_2019}. 
\rev{
For example, the challenges we have identified could be framed as a \textit{goal misalignment} challenge \cite{kirchner_they_2021, schroeder_examining_2020}, where the data collection goals for VBC program administrators -- monitoring health system-level outcomes and costs -- are not always aligned with the goals of patients and clinicians, to monitor individual-level progress in care.
Personal informatics researchers have shown that individuals are more likely to engage in data collection when it aligns with their specific goals \cite{epstein_lived_2015}, and designing technologies to account for the diversity of users' goals can improve the data collection experience \cite{sefidgar_migrainetracker_2024}.
If patients and clinicians can personalize or negotiate collected outcomes data with VBC program administrators, we may be able to better align system- and individual-level data collection aims \cite{gulotta_fostering_2016}.
}

To align goals, Forlizzi argues that HCI take a service-oriented approach to design, which they call \textit{stakeholder-centered design}, to account for the needs of stakeholders and their interactions with technologies \cite{forlizzi_moving_2018}. 
In this work, we centered one specific stakeholder: mental health clinicians.
Future work taking a stakeholder-centered design perspective could uncover how interactions between patients, payers, providers, and social service organizations contribute to the financing and effective use of HITs in VBC. 
Methods from \textit{service design} could center each stakeholder's design requirements for HITs, and how HITs can support interactions within joint accountability programs \cite{hwang_societal-scale_2024, forlizzi_promoting_2013}.
For example, \textit{stakeholder mapping} could map the power dynamics of stakeholders involved in financing HITs \cite{newcombe_client_2003}.
\textit{Service blueprinting} could then identify interactions with HITs across stakeholders in joint accountability programs.
Surfacing multi-stakeholder perspectives are essential towards understanding why a specific HIT may or may not be funded or adopted.
Such methods would allow technology designers to infuse stakeholders' perspectives into the initial development of HITs, build technologies that effectively support VBC, and improve care outcomes for \textit{all} patients.

\subsection{Limitations}

These findings reflect our interpretation of the literature joined with the perspectives of 30 mental health clinicians.
They should not be interpreted to reflect mental health clinicians as a whole.
In addition, we only interviewed mental health clinicians as participants.
A broader perspective of value-based mental healthcare would include the views of other stakeholders, including but not limited to patients, health system administrators, payers, and social service administrators.
We plan to include these perspectives in future work.
\rev{Participants and the authors were based in the United States.
Findings and implications are thus biased towards a U.S. perspective.
In addition, the majority of participants worked in academic medical centers, which are not representative medical centers and clinics across the United States \cite{fisher_academic_2019}.
We often asked participants during the interviews about their current payment arrangements (eg, insurance, out-of-pocket pay), but structured data to support our findings were not collected.}

\subsection{Conclusion}

Our findings illuminate opportunities to design health information technologies that support value-based mental healthcare.
\rev{
Specifically, our findings advocate for flexibility in HIT development, allowing healthcare providers and patients choice to collect and store outcomes data for VBC most aligned with their specific care goals.
Simultaneously, future HCI research could take a more health systems-level perspective towards HIT design, by engaging the multiple stakeholders involved in VBC who influence the effectiveness of these technologies.
}
We hope these findings chart a path towards developing technologies that effectively support healthcare payment programs, clinicians' practice, improve health service delivery, and patient outcomes.

\begin{acks}
D.A. is supported by a National Science Foundation Graduate Research Fellowship under Grant No. DGE-2139899, a Digital Life Initiative Doctoral Fellowship, and a Siegel PiTech PhD Impact Fellowship.
Transcription and participant reimbursement costs were supported by a multi-investigator seed grant through the Cornell Academic Integration Program, awarded to T.C.
Publication costs were funded by a National Science Foundation Grant No. 2212351 awarded to T.C.
A.V.M is supported by NIMH Grant No. K23MH120505.
Any opinions, findings, and conclusions or recommendations expressed in this material are those of the authors and do not reflect the views of the funders.
\end{acks}