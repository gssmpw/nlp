\clearpage
\section{Interview Guide}
\label{appendix:guide}

\subsection{Session 1: Formative Interview}

\begin{enumerate}
    \item Can you describe your practice, including, but not limited to, the types of patients you see, the conditions you treat, and the treatments you offer?
    \item In this study we are interested in learning about how mental health specialty care providers use or do not use measurements during routine patient encounters to monitor patient progress and inform treatment, also known as measurement-based care, or MBC for short. What do you think about MBC? 
    \item Do you use measurements to inform your practice? If so...
    \begin{enumerate}
        \item Why do you practice measurement-based care? Is it mandated?
        \item What measures do you use? For what patients specific patients, or specific treatments?
        \item How do you collect these measures? 
        \begin{enumerate}
            \item Pen and paper? Is technology involved? 
            \item Do you use patient self-report?
            \item What informs these data collection decisions?
        \end{enumerate}
        \item How do these measures inform your practice?
        \begin{enumerate}
            \item Do you review measures with your patients?
            \item Is technology involved?
        \end{enumerate}
        \item Do you find MBC helpful? Not helpful? Why? What barriers do you face towards using MBC to improve care?
    \end{enumerate}
    \item If you do not use measurement-based care, why do you not use MBC?
    \begin{enumerate}
        \item Have you considered using MBC in your practice?
        \item Are there other practices you use that are like MBC?
    \end{enumerate}
    \item How do you think about measuring outcomes as a part of care? 
    \begin{enumerate}
        \item How do you measure that a patient is improving or deteriorating?
        \item How is technology involved/not involved in this process?
        \item Do your patients ``graduate'' or ``complete'' treatment? What data informs treatment completion or graduation?
    \end{enumerate}
    \item Do others (eg, administrators) in your organization review this information. If so, what do they do with this information?
    \begin{enumerate}
        \item Is there any systems-level oversight?
        \item Evaluation of patient progress?
        \item Accountability to achieve certain care outcomes?
        \item Does technology facilitate this process? Is there technology involved in care coordination, or sharing data for oversight? Why or why not?
    \end{enumerate}

\end{enumerate}

\subsection{Session 2: Brainstorming Session}

 In this session, we will have you brainstorm responses to two prompts on an online slide deck. We will complete each prompt one at a time. After you respond to each prompt, we will review your responses and ask you follow-up questions. Do you have any questions before we begin?

\begin{enumerate}
    \item We want you to imagine that your healthcare system has asked you to start monitoring patient outcomes as a part of your treatment, in order to maximize the care you deliver for your patients. 
    Your department chair, or a supervisor, may review these outcome measures with you at routine check-ins. 
    Please brainstorm a list of measures that you would use to monitor and maximize treatment outcomes for the patients you are treating. 
    Think about measures you would collect and monitor across routine patient encounters. 
    These measures can include both measures you currently collect in your practice, or measures you do not collect, but think would be helpful for monitoring and maximizing patient outcomes. 
    Feel free to brainstorm a list of measures, or draw diagrams or visuals, whatever is most helpful to communicate your ideas. 
    Once you are done, we will have a conversation about what you came up with.
    [Provide example measures, if helpful, such as:]
    \begin{enumerate}
        \item Standardized symptom measures (eg, patient reported outcome measures, like the PHQ-9, GAD-7, etc.)
        \item Quality of life measures
        \item Psychosocial measures
        \item Health behaviors
        \item Idiographic measures, focusing on specific patient goals
        \item Patient satisfaction/experience with treatment
        \item Working alliance
        \item Measures of comorbid conditions (connecting physical and mental health)
        \item Psychiatric events (eg, hospitalization)
    \end{enumerate}
    [Can also discuss:]
    \begin{enumerate}
        \item When would these measures be taken? At the beginning, during, or after the visit?
        \item At what frequency should these measures be taken?
        \item How would these measures be recorded? Self-reported by a patient? Rated by a clinician?
        \item Where would these measures be stored? In the EHR? Another system?
    \end{enumerate}

    \item We want you to now imagine that as part of a push for value-based care, the Center for Medicare \& Medicaid Services, or CMS, are creating “mental health quality star ratings” to measure patient outcomes and care quality across clinics and health systems. 
    CMS wants your help, and is seeking feedback from you, actual clinicians treating patients, to understand how these measures can be designed to most accurately reflect patient outcomes. 
    These ratings may be partially based upon measures that you, the clinician, gather during routine clinical encounters with your patient. 
    In this session, you will have an opportunity to shape how these mental health star ratings are created.
    Please brainstorm measures that your clinic, or department, would use to monitor patient outcomes across your entire clinic/department. 
    These measures would be shared with CMS. 
    Feel free to brainstorm a list of measures, or draw diagrams or visuals, whatever is most helpful to communicate your ideas. Once you are done, we will have a conversation about what you came up with.
\end{enumerate}

This concludes the study. Thank you for participating.