\subsection{Preparation: What Outcomes Data Should HITs Store?}
\label{sec:findings:preparation}

A foundation of building HITs for value-based mental healthcare are determining the standardized outcomes data these technologies should store.
Participating clinicians recognized the value of standardized outcomes data. 
As SW28 mentioned, \textit{``I think we have to have some concrete thing that's going to say, `You're getting better. This treatment is working.' ''}
But, participants believed it would be challenging to identify a limited set of outcomes data to use for VBC, even for a single patient or within a single disorder.
Participants mentioned how patients often present in care with multiple symptoms co-occurring across disorders, and collecting data to track all of their symptoms was burdensome:

\begin{quote}
    \textit{``You can't ask questions about absolutely everything. Sometimes you find the patient talks about how they're anxious about their parents, their family and their friends. I give them a longer anxiety scale that hits social anxiety, school anxiety, separation anxiety. 
    But I'm being forced to do all of these assessments and I'm not getting a really good reason why other than because you have to.''} (SW38)
\end{quote}

Given these complexities, we weighed with participants what outcomes data they preferred to use within VBC, and identified two themes.
First, drawing upon their clinical experience, participants believed that symptom scales -- for example, self-reported depression scales -- would be difficult to use. Participants described that symptom scales were difficult to interpret across patients, and did not accommodate patients who identify with different cultural backgrounds (Section \ref{sec:findings:preparation:symptoms}).
Instead, participants preferred using a combination of functional and engagement outcomes data that they believed better reflected patients' goals for care, were more closely connected to treatment, and were relevant across patients presenting with different disorders or symptoms (Section \ref{sec:findings:preparation:func-engagement}).
\rev{
By functional data, participants referred to data that quantified a patient's ability to participate in day-to-day life, including their cognition, mobility, ability to work, and maintain healthy relationships.
By engagement, participants referred to patients' engagement in treatment, including their ability to practice skills or behavior change exercises learned in care, take prescribed medication, or make safety plans for harmful (eg, suicidal) behaviors.
Participants mainly imagined forms of data captured within clinical encounters. 
We discuss in Section \ref{sec:findings:collection} participants' perspectives on using data captured both within and outside of clinical encounters for VBC.
}

\subsubsection{Challenges Using Symptom Scales as Outcomes Data in VBC}
\label{sec:findings:preparation:symptoms}

We began our interviews asking participants about using standardized symptom scales as outcomes data, as existing quality metrics and measurement-based care programs advocate for collecting standardized symptom scales \cite{morden_health_2022, jacobs_aligning_2023}.
These scales quantify symptom severity for specific mental health disorders, and include self-reported symptom scales such as the PHQ-9 for major depressive disorder, or GAD-7 for generalized anxiety disorder.
Symptom scores are added together to provide an overall measure of treatment progress, and could be shared with regulators as outcomes data in VBC.
Participants believed symptom scales were useful for communicating patients' diagnoses for \textit{``insurance repayment''} (SW55) because they give \textit{``a common language''} (CP51).
CP51 also believed symptom scales were useful for understanding if \textit{``there are specific clusters of symptom coming together to understand 
if I have an intervention that targets those symptoms''} (CP51).
But, our participants were uncomfortable using symptom scores as outcomes data within VBC, because patients have challenges interpreting and reporting symptoms.
SW58 explained:

\begin{quote}
    \textit{
    ``A score on the PHQ-9 can get worse because of external factors. 
    Somebody loses a job, gets a divorce, their child is sick, these things can happen that make stress or depression feel much harder to deal with.
    But it doesn't mean that the client is getting worse.''} (SW58)
\end{quote}

We further probed participants about factors that distort symptom scores.
For self-reported scales, some participants described internal and environmental factors that affect self-reporting.
SW37 mentioned how patients may have \textit{``a literacy issue and do not fully gather the meaning of all the questions''} or \textit{``do not feel comfortable fully disclosing their answers. We will see a discrepancy sometimes between how they fill out the form with their doctor and how they fill it out as a mental health professional.''}
CP30, a child and adolescent psychologist, stated that some children \textit{``just tend to kind of rate symptoms on the higher end.''}
One participant, a psychiatrist, described their own reporting behaviors to explain why symptom scores are difficult to interpret at face-value:

\begin{quote}
    \textit{``If I were to take a PHQ I would probably score highly on it, not because I'm depressed, but because when the questions say `You spend a lot of days not wanting to get out of bed,' or `You overeat,' and I'm like, `Yeah, I do, but I'm not doing it because I'm depressed, I'm doing it because I am lazy.' 
    The context is important.
    The hard numbers taken out of context aren't fully accurate.''} (PS25)
\end{quote}

Aside from self-reports, our participants also mentioned how it was difficult to interpret clinician-rated symptom scales.
SW38 would give patients \textit{``baseline assessments and my colleagues would be like, `Oh my god, the patient is so depressed. We have to give them this really extreme, very intense treatment.' ''}, but the participant challenged their colleagues to see symptom scores as \textit{``a piece of a whole picture''}.
Participants described how they would cross-reference symptom scales with other providers to improve their understanding of patients.
One participant, who treated patients with emergent psychotic symptoms, stated that they spend \textit{``30 minutes dissecting what patients have said in different providers' offices, trying to figure out if they've crossed the threshold to a first psychotic episode.''} (CP35).
Another participant mentioned that providers would report, for the same patient, different levels of symptoms quite frequently:

\begin{quote}
    \textit{ ``There's been multiple times where I am rating somebody at lower risk and another clinician rates them at higher risk and it's the same day and program. What do you do about that? How do you work? How do you provide the right treatment?''} (SW28)
 \end{quote}

Participants also believed that symptom scales did not accommodate patients from different cultures.
One participant mentioned how \textit{``there is stigma around sharing one's mental health and so in the hospital system where I work, there are people from so many different cultural backgrounds''}, and that symptom scales would be \textit{``aligned and more accurate for people who are open and coming from a cultural background where there's open discussion of mental health and symptomology''} (SW37).
Another participant, described that symptom scales were not developed inclusively, and the \textit{``evidence-based is pretty self-selecting''} (CP34).
We probed this participant further to understand how this might impact using symptom scores as outcomes data:

\begin{quote}
    \textit{``I feel mixed about this because the way we have developed these measures and the people on whom they have been developed for. They're just not always accurate, inclusive, or culturally appropriate. 
    I don't really see a world where they fully capture the clinical picture for somebody.''} (CP34)
\end{quote}


\subsubsection{Participants Preferred using Functional and Engagement Outcomes Data}
\label{sec:findings:preparation:func-engagement}

The prior section describes various challenges participating clinicians saw using symptom scales as outcomes data supporting VBC.
Given this, we asked our participants for their perceptions regarding alternative types of outcomes data that could be used.
Some participants mentioned using measures of patient satisfaction, but perceived that satisfaction could be biased by aspects of care unrelated to health outcomes, such as \textit{``if the patient likes the hospital's food''} (PS23).
We also asked our participants if care utilization data extracted from EHRs -- such as psychiatric hospitalizations -- were useful outcomes data, but participants were wary to create a culture that discourages utilization: \textit{``If I'm being measured on how many of my clients go to the emergency room, I don't care. Not that I don't care, but, if going to the emergency room was the best decision for that client, what am I going to do?''} (SW50).
Other participants brought up working alliance scales -- that measure the patient-clinician relationship -- but described alliance not as an outcome of care, but an important aspect \textit{``at the beginning of care because you're trying to get that buy-in for treatment''} (CP51).

Instead, participants advocated for using a combination of functional and engagement data as outcome measures, and saw these data types as more aligned with patients' care goals \rev{(examples in Table \ref{tab:findings:preparation:func-engagement-data})}. 
CP30 mentioned how impaired functioning was \textit{``why a lot of people seek treatment. They feel like something is messing up their life in some way. Their goal is to be able to go to school, hang out with friends, spend time with family, whatever it is.''}
Another participant, who treated individuals living with obsessive-compulsive disorder (OCD), found that \textit{``symptom relief itself is not terribly motivating for most people. If you are hamstrung by fears of household chemicals, nobody wakes up in the morning and says, `Oh, boy, I can't wait to get used to these household chemicals.' ''} but instead they \textit{``really focus more on functional gains''} and \textit{``my goal is to get somebody out of the house and interacting with friends''} (FT60).

\begin{table*}[t]
\begin{tabular}{l|l}
\toprule
\textbf{Functional data} quantifying a patient's ability & \textbf{Engagement data} showing participation in \\
to participate in everyday life & skills, exercises, or behaviors relevant for care \\
\midrule
\tabitem Attending school & \tabitem Leaving the house, if fearful of doing so \\
\tabitem Maintaining healthy physiological stress & \tabitem Medication adherence  \\
\tabitem Spending time with family and friends & \tabitem Reducing obsessive handwashing  \\
\tabitem Attending work & \tabitem Communicating suicidal ideation safety plans \\
\tabitem Staying physically active & \tabitem Going to the gym \\
\tabitem Consistent eating behavior & \tabitem Eating a meal \\
\bottomrule
\end{tabular}
\caption{\rev{Examples of functional and engagement data from our findings. 
Engagement data (eg, eating a meal) was often described as a proximal outcome of care and functioning a distal outcome (eg, consistent eating behavior).}
}
% \Description{A table providing definitions and examples of the functional and engagement outcomes data described in Section 4.1.2. There are two columns. In the left column, the first row states the definition of "functional outcomes" data, and then the second row has a bulleted list of functional outcomes described by our participants. In the right column, the first row states the definition of "engagement outcomes" data, and then the second row has a bulleted list of engagement outcomes described by our participants.}
\label{tab:findings:preparation:func-engagement-data}
\end{table*}

We asked participants to explain why functional improvements were not captured by symptom scales.
In other words, if functioning improves, why should we not expect symptoms to decrease?
Participants explained that symptom reduction was not the singular outcome of treatment, but treatment intends to improve functioning even if symptoms persist. 
For example, PS24, a psychiatrist treating patients living with schizophrenia, mentioned how they \textit{``tend to have very chronic patients where the goal isn't to get rid of symptoms, but the goal might be to make symptoms interfere with their life less''} and their patients may \textit{``have ongoing voices and paranoia, but they've gotten to the point where they're able to ignore the voices and attend work''}
SW58 agreed, stating that \textit{``when I think about somebody that experiences psychosis or bipolar disorder or depression, you may have this for your whole life. If the goal is to have fewer symptoms, am I setting you up to fail from the start?''} and they work with patients to understand \textit{``given what your life is, how do you want to live? Maybe there's specific things, maybe you want to go back to school.''}.

Furthermore, participants believed that functional outcomes were likely to improve if their patients engaged in care, and saw engagement as the most proximal outcome of care.
For example, many of our participants were psychotherapists who asked patients to practice specific skills or change behavior as a part of treatment.
FT60 mentioned how they \textit{``had somebody who was washing their hands a hundred times a day and driving his family nuts with accommodations''} and they had their patient \textit{``use judicial safety behaviors to play with his daughter who's crawling around on the floor and then take a shower afterwards.''}
A few months into treatment the \textit{``patient is still washing his hands a hundred times a day, but he and his family are tons happier than they were. They're raving about how well they're functioning and working together now}'' (FT60).
CP45 mentioned how for patients with \textit{``panic disorder, I'd want to have some behavioral data on what they are avoiding or how frequently they are getting out of the house, depending on the specifics of that person.''}
Another participant mentioned how, by tracking engagement, they might feel more confident that \textit{``someone having passive suicidal thoughts would have no intent to act on them''} because they \textit{``have a supportive family that they communicate to and a safety plan in place''} (CP46).
CP43 saw treatment as successful if patients consistently engaged in care, even if symptoms were not fully reduced:

\begin{quote}
   \textit{``Their [symptom] scores do cut in half, but don't move much beyond that and stay relatively stable. 
   If they practice their skills and those are well-developed, they got what I am aiming to provide for them.
   Sure their scores aren't zero, but that might just be because of their personality, environment, social context.''} (CP43)
\end{quote}

Unlike symptom scales, participants saw functioning and engagement as \textit{``trans-diagnostic''} (CP42), measuring care outcomes across patients experiencing different symptoms or disorders.
FT60 qualified that measuring symptoms were not irrelevant, but called for \textit{``a shift from symptom-focused metrics to patient-focused metrics, which can include the symptoms.''}
CP33 wanted to prioritize engagement outcomes for complex cases, giving an example of \textit{``a patient who had comorbid substance use disorder, PTSD [post-traumatic stress disorder], borderline personality disorder, there's a lot of suicidality, a lot of very, very intense mood, depression and anxiety''} that \textit{``those intense things, really intense urges, really intense depression, that didn't go away''} but the patient \textit{``developed trust and she kept coming to therapy. She missed, maybe, four sessions all year. Those are therapeutic gains. She internalized some hope that progress is possible.''}
CP30 further explained the importance of engagement and functional outcomes across conditions:

\begin{quote}

\textit{``If someone has one depressive episode, they will likely have another episode.
Someone who has generalized anxiety disorder may always be a more anxious person. 
Someone with obsessive-compulsive disorder may always be vulnerable to intrusive thoughts. 
It doesn't mean they've failed treatment if they can tolerate the anxiety or cope with the depression, go to work, get out of bed, shower, do the things you have to do, using the skills you learned in therapy.''} (CP30)
\end{quote}

