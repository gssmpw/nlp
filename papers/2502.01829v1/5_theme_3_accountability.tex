\subsection{Action: How Should Outcomes Data be Used in VBC?}
\label{sec:findings:action}

The prior section suggests opportunities to invest in HITs, and use active/passive data to improve clinicians' engagement in VBC outcomes data collection.
Once outcomes data has been collected, VBC programs use this data to create financial incentives that hold providers accountable to achieve specific care outcomes.
Participants, generally, recognized the need for more accountability.
One participant explained that \textit{``there are incentives to keep your patient caseload the same when you're in private practice, because it's a lot of work to do intakes, and you get comfortable with the people you see. 
And so, if there's a piece of your reimbursement that's tied to meeting an outcome and then discharging and starting anew, it also holds you more accountable''} (CP35).
Participants also raised that VBC could give patients more control over care decisions.
FT60 stated that \textit{``many providers convince patients that they're failing treatment''} and CP35 continued: \textit{``it's really hard to know, as a consumer, whether or not you're seeing somebody whose skills actually back up what they say. 
Value-based care could help you steer whether or not you go to somebody.''}

In this section, we describe both challenges and opportunities participants' perceived towards using outcomes data in VBC.
First, participants stressed that outcomes data would need to hold providers, healthcare payers, and social service organizations jointly accountable if VBC were to fulfill its promise of improving mental health outcomes (Section \ref{sec:findings:action:accountability}).
Second, participants voiced the need for HITs to implement risk-adjustments to outcomes data, otherwise clinicians may prioritize treating simpler patient cases that inflate care outcomes (Section \ref{sec:findings:action:risk-adjustment}).

\subsubsection{Using Outcomes Data for Joint Accountability in VBC}
\label{sec:findings:action:accountability}

After participants brainstormed what outcomes data HITs should store, we asked them how this data should be used in VBC programs to improve care.
Specifically, we were interested in who should be held accountable: payers, providers, or other entities?
Participating clinicians quickly pointed out that it would be very difficult to attribute the outcomes of care to any one specific entity.
Though providers would love to take credit if outcomes did improve, that was not always possible:

\begin{quote}
    \textit{``I don't really care what's causing the improvement. 
    If that's due to my intervention, great. 
    And if not, still great because they're feeling better. 
    But I think that tying outcomes to something very specific is too complex. 
    There's too many extraneous and confusing variables to ever do that.''} (CP51)
\end{quote}

Participants gave many examples highlighting the need for \textit{joint accountability}, where responsibility for care outcomes is shared across different providers, or external entities that influence mental health.
For example, CP46 worked in an adolescent inpatient psychiatry unit treating patients in crisis. 
In their view, crisis care may not translate into long-term outcomes, explaining that \textit{``patients may feel totally better because they're in the hospital removed from all the stress and problems of life. Once they leave, I suspect their symptoms would increase. One week or month here doesn't solve the patient's way of approaching life''} (CP46).
To solve this challenge, PS23 believed that clinicians providing inpatient and outpatient services -- the long-term care patients receive after discharge from inpatient -- should be held jointly accountable: \textit{``you could look at across the system, but may not be able to look at for the individual provider. From the patients with depression that we treat as inpatients who then go to our outpatient setting, 85\% still meet criteria for remission after one month. That tells us, okay, we as a health system are doing something right.''}
PS53 believed that physical health providers should be accountable for mental health outcomes.
They stated that \textit{``as I'm doing community psych, I'm learning more that outcomes involve physical care, especially if people can't move around, so we need integration''} (PS53).

Outside of holding providers jointly accountable, participants also described how external entities greatly influenced care outcomes.
One participant raised how health insurers should be held accountable, because \textit{``insurers reimburse clinicians so poorly, so the care is not going to be high quality a lot of the time. It's almost like this circular reasoning issue''} (CP34).
Another participant thought that social services, for example housing or education authorities, should also be held accountable for poor mental health outcomes.
They stated that \textit{``in settings like city hospitals, I wish there was more measurement around what interventions have been effective in reducing stressors related to housing, food, and educational access''} (CP34).
Because social factors, like housing, could influence care, some participants described taking a more active role in linking patients to social services.
SW57 mentioned how \textit{``some patients might be looking for mental health housing, so I'll say, `listen, if you want to bring the paperwork into your next session and spend your session with that, we can absolutely do that.' ''}
CP46 and PS23, though, both expressed frustration that social factors could lead to poor care outcomes, and providers, not social services, may be held accountable.
As they expressed:

\begin{quote}
    \textit{``A patient could be in foster care, lose their foster home and become stuck in inpatient care for two months. That outcome, the length of stay, has nothing to do with how much better they were and everything about the systems serving them.}'' (CP46)
\end{quote}

\begin{quote}
    \textit{``We get frustrated when we see a 30-day readmit but then we understand the patient is homeless and it's 30 degrees outside and someone stole their medication.''} (PS23)
\end{quote}

\subsubsection{Risk-Adjusting Outcomes Data to Encourage Fair Treatment}
\label{sec:findings:action:risk-adjustment}

To penalize and reward stakeholders within VBC, participants voiced that the outcomes data HITs share with payers or quality monitoring organizations need to be \textit{risk-adjusted}: adjusting expected care outcomes based upon the difficulty of the patient case.
Otherwise, participants believed that VBC could dis-incentivize clinicians from treating tougher cases to inflate outcome metrics.
CP42 stated \textit{``how long one person needs treatment differs from how long another person needs treatment. And having a strict outcome can mean that people aren't getting enough treatment.''}
Another participant echoed this concern, saying \textit{``it's kind of unfair if you've got someone treating more severe people to be like, `Oh, you suck at your job because you couldn't get your people down to that level' So the challenge is, what do you want your outcome to be?''} (CP43).
FT60 put these terms more starkly:

\begin{quote}
    \textit{``I'm extremely nervous about the impact on care because it's going to turn the clinician against the patient in favor of boosting their scores. 
    I'm fine with outcome measures being exposed to the consumer so that they can make an intelligent decision as to where they want seek care. 
    But I have real concerns about using them for reimbursement criteria or access to care as a consequence.''} (FT60)
\end{quote}

We approached our participants to understand how HITs should perform risk-adjustments to reduce these potential harms.
If outcome metrics were symptom or functional scores, one participant thought about using \textit{``individual changes or change scores rather than absolute zeros. A person who comes in really significantly depressed who moves mild to moderately depressed [on a symptom scale] is a big change. But if you just use absolute scores, that might not reflect that treatment works''} (CP35).
CP42 suggested that change scores should be relative to patients' baselines, stating that \textit{``if one patient's symptom severity were a 10, but they came into me starting at a 24 that would show amazing improvement. Whereas if someone's usually the happiest person in the world and they're now feeling a little depressed, that might be a notable change''} (CP42).

We also asked participants about specific factors risk-adjustment models should account for to estimate expected care outcomes.
PS32 thought models should account for other conditions patients are living with, stating that they can affect mental health outcomes data: \textit{``I have a patient who needed to have bariatric surgery because it's hard to manage their appetite. They have sleep and energy problems that are related to the chronic pain and fibromyalgia. All those other conditions besides depression already bring the PHQ pretty close to a 10, if not higher than a 10}'' (PS32).
Another participant mentioned that models should account for patients' history of mental illness.
Specifically, that they \textit{``tell people starting on a medication that if this is the first time they've been treated for depression or anxiety, I recommend once your symptoms have been alleviated that you stay on the medication for about six months. If they come off the medication at that time there's a 30\% risk of relapse. If this is your second time with an episode of depression or anxiety, the chances of a third relapse are 70\%. The goalposts get moved a little bit''} (PS23).
In addition, PS23 also raised how social factors influence care outcomes, because \textit{``in the last place I worked, 70\% of our patients were homeless. These are people with a high level of needs, a high level of trauma and stress.''}
CP33 agreed, saying that \textit{``I wouldn't expect someone who's coming in with a really severe depression, who has multiple stressors, and maybe less resilience factors, fewer support system, all kinds of things, to come out of that at the same place as someone who has had a relatively supportive and stable household.''}


In addition to using these factors to moderate expectations, participants also believed these factors should moderate the expected length of treatment, because \textit{``I wouldn't necessarily expect progress to happen in the same way or in the same timeframe''} (CP33).
One participant stated that \textit{``obviously we hope that all of our clients improve. Maybe, over a longer period of time we start to see improvement because you had a long period of therapy. But I don't know that timeframe''} (CP44).
Yet, not all participants were convinced that patients need to be in care for a long period of time to see improvement.
As one participant stated:

\begin{quote}
    \textit{``I would expect change. I would 100\% disabuse the notion that you need to be in therapy for years to see progress and instead show that a lot of people can get better from just a few sessions.''} (CP43)
\end{quote}



