\section{Findings}
\label{sec:findings}

Our findings highlight three themes exploring the design of health information technologies (HITs) that support outcomes data preparation, collection, and use within value-based mental healthcare.
(1) With regards to \textit{preparation}, participants preferred that HITs store functional and engagement outcomes for VBC, as compared to symptom scales or other outcomes data, because they believed functional and engagement outcomes were most directly tied to provided mental health services (Section \ref{sec:findings:preparation}).
(2) Participants also perceived that data \textit{collection} could be improved by investing in HITs that support standardized fields to collect mental health outcomes data, and saw opportunities for devices collecting both active and passive data to improve data collection (Section \ref{sec:findings:collection}).
(3) Finally, participants emphasized that \textit{actions} with outcomes data must hold payers, providers, and social services jointly accountable to care outcomes, and outcomes data need to be risk-adjusted, otherwise providers may prioritize easier to treat patients that inflate outcome metrics (Section \ref{sec:findings:action}).
Throughout our findings, participants are referred to with a unique identifier (eg, CP30) to maintain anonymity.
\rev{These identifiers indicate participants' clinical training (CP = Clinical Psychology, PS = Psychiatry, SW = Social Work, MC = Mental Health Counseling, FT = Family and Marriage Therapist, see Table \ref{tab:participants}).}
Participants referred to individuals receiving mental health services as both ``patients'' and ``clients'', and we use these terms interchangeably in our findings.


\subsection{Preparation: What Outcomes Data Should HITs Store?}
\label{sec:findings:preparation}

A foundation of building HITs for value-based mental healthcare are determining the standardized outcomes data these technologies should store.
Participating clinicians recognized the value of standardized outcomes data. 
As SW28 mentioned, \textit{``I think we have to have some concrete thing that's going to say, `You're getting better. This treatment is working.' ''}
But, participants believed it would be challenging to identify a limited set of outcomes data to use for VBC, even for a single patient or within a single disorder.
Participants mentioned how patients often present in care with multiple symptoms co-occurring across disorders, and collecting data to track all of their symptoms was burdensome:

\begin{quote}
    \textit{``You can't ask questions about absolutely everything. Sometimes you find the patient talks about how they're anxious about their parents, their family and their friends. I give them a longer anxiety scale that hits social anxiety, school anxiety, separation anxiety. 
    But I'm being forced to do all of these assessments and I'm not getting a really good reason why other than because you have to.''} (SW38)
\end{quote}

Given these complexities, we weighed with participants what outcomes data they preferred to use within VBC, and identified two themes.
First, drawing upon their clinical experience, participants believed that symptom scales -- for example, self-reported depression scales -- would be difficult to use. Participants described that symptom scales were difficult to interpret across patients, and did not accommodate patients who identify with different cultural backgrounds (Section \ref{sec:findings:preparation:symptoms}).
Instead, participants preferred using a combination of functional and engagement outcomes data that they believed better reflected patients' goals for care, were more closely connected to treatment, and were relevant across patients presenting with different disorders or symptoms (Section \ref{sec:findings:preparation:func-engagement}).
\rev{
By functional data, participants referred to data that quantified a patient's ability to participate in day-to-day life, including their cognition, mobility, ability to work, and maintain healthy relationships.
By engagement, participants referred to patients' engagement in treatment, including their ability to practice skills or behavior change exercises learned in care, take prescribed medication, or make safety plans for harmful (eg, suicidal) behaviors.
Participants mainly imagined forms of data captured within clinical encounters. 
We discuss in Section \ref{sec:findings:collection} participants' perspectives on using data captured both within and outside of clinical encounters for VBC.
}

\subsubsection{Challenges Using Symptom Scales as Outcomes Data in VBC}
\label{sec:findings:preparation:symptoms}

We began our interviews asking participants about using standardized symptom scales as outcomes data, as existing quality metrics and measurement-based care programs advocate for collecting standardized symptom scales \cite{morden_health_2022, jacobs_aligning_2023}.
These scales quantify symptom severity for specific mental health disorders, and include self-reported symptom scales such as the PHQ-9 for major depressive disorder, or GAD-7 for generalized anxiety disorder.
Symptom scores are added together to provide an overall measure of treatment progress, and could be shared with regulators as outcomes data in VBC.
Participants believed symptom scales were useful for communicating patients' diagnoses for \textit{``insurance repayment''} (SW55) because they give \textit{``a common language''} (CP51).
CP51 also believed symptom scales were useful for understanding if \textit{``there are specific clusters of symptom coming together to understand 
if I have an intervention that targets those symptoms''} (CP51).
But, our participants were uncomfortable using symptom scores as outcomes data within VBC, because patients have challenges interpreting and reporting symptoms.
SW58 explained:

\begin{quote}
    \textit{
    ``A score on the PHQ-9 can get worse because of external factors. 
    Somebody loses a job, gets a divorce, their child is sick, these things can happen that make stress or depression feel much harder to deal with.
    But it doesn't mean that the client is getting worse.''} (SW58)
\end{quote}

We further probed participants about factors that distort symptom scores.
For self-reported scales, some participants described internal and environmental factors that affect self-reporting.
SW37 mentioned how patients may have \textit{``a literacy issue and do not fully gather the meaning of all the questions''} or \textit{``do not feel comfortable fully disclosing their answers. We will see a discrepancy sometimes between how they fill out the form with their doctor and how they fill it out as a mental health professional.''}
CP30, a child and adolescent psychologist, stated that some children \textit{``just tend to kind of rate symptoms on the higher end.''}
One participant, a psychiatrist, described their own reporting behaviors to explain why symptom scores are difficult to interpret at face-value:

\begin{quote}
    \textit{``If I were to take a PHQ I would probably score highly on it, not because I'm depressed, but because when the questions say `You spend a lot of days not wanting to get out of bed,' or `You overeat,' and I'm like, `Yeah, I do, but I'm not doing it because I'm depressed, I'm doing it because I am lazy.' 
    The context is important.
    The hard numbers taken out of context aren't fully accurate.''} (PS25)
\end{quote}

Aside from self-reports, our participants also mentioned how it was difficult to interpret clinician-rated symptom scales.
SW38 would give patients \textit{``baseline assessments and my colleagues would be like, `Oh my god, the patient is so depressed. We have to give them this really extreme, very intense treatment.' ''}, but the participant challenged their colleagues to see symptom scores as \textit{``a piece of a whole picture''}.
Participants described how they would cross-reference symptom scales with other providers to improve their understanding of patients.
One participant, who treated patients with emergent psychotic symptoms, stated that they spend \textit{``30 minutes dissecting what patients have said in different providers' offices, trying to figure out if they've crossed the threshold to a first psychotic episode.''} (CP35).
Another participant mentioned that providers would report, for the same patient, different levels of symptoms quite frequently:

\begin{quote}
    \textit{ ``There's been multiple times where I am rating somebody at lower risk and another clinician rates them at higher risk and it's the same day and program. What do you do about that? How do you work? How do you provide the right treatment?''} (SW28)
 \end{quote}

Participants also believed that symptom scales did not accommodate patients from different cultures.
One participant mentioned how \textit{``there is stigma around sharing one's mental health and so in the hospital system where I work, there are people from so many different cultural backgrounds''}, and that symptom scales would be \textit{``aligned and more accurate for people who are open and coming from a cultural background where there's open discussion of mental health and symptomology''} (SW37).
Another participant, described that symptom scales were not developed inclusively, and the \textit{``evidence-based is pretty self-selecting''} (CP34).
We probed this participant further to understand how this might impact using symptom scores as outcomes data:

\begin{quote}
    \textit{``I feel mixed about this because the way we have developed these measures and the people on whom they have been developed for. They're just not always accurate, inclusive, or culturally appropriate. 
    I don't really see a world where they fully capture the clinical picture for somebody.''} (CP34)
\end{quote}


\subsubsection{Participants Preferred using Functional and Engagement Outcomes Data}
\label{sec:findings:preparation:func-engagement}

The prior section describes various challenges participating clinicians saw using symptom scales as outcomes data supporting VBC.
Given this, we asked our participants for their perceptions regarding alternative types of outcomes data that could be used.
Some participants mentioned using measures of patient satisfaction, but perceived that satisfaction could be biased by aspects of care unrelated to health outcomes, such as \textit{``if the patient likes the hospital's food''} (PS23).
We also asked our participants if care utilization data extracted from EHRs -- such as psychiatric hospitalizations -- were useful outcomes data, but participants were wary to create a culture that discourages utilization: \textit{``If I'm being measured on how many of my clients go to the emergency room, I don't care. Not that I don't care, but, if going to the emergency room was the best decision for that client, what am I going to do?''} (SW50).
Other participants brought up working alliance scales -- that measure the patient-clinician relationship -- but described alliance not as an outcome of care, but an important aspect \textit{``at the beginning of care because you're trying to get that buy-in for treatment''} (CP51).

Instead, participants advocated for using a combination of functional and engagement data as outcome measures, and saw these data types as more aligned with patients' care goals \rev{(examples in Table \ref{tab:findings:preparation:func-engagement-data})}. 
CP30 mentioned how impaired functioning was \textit{``why a lot of people seek treatment. They feel like something is messing up their life in some way. Their goal is to be able to go to school, hang out with friends, spend time with family, whatever it is.''}
Another participant, who treated individuals living with obsessive-compulsive disorder (OCD), found that \textit{``symptom relief itself is not terribly motivating for most people. If you are hamstrung by fears of household chemicals, nobody wakes up in the morning and says, `Oh, boy, I can't wait to get used to these household chemicals.' ''} but instead they \textit{``really focus more on functional gains''} and \textit{``my goal is to get somebody out of the house and interacting with friends''} (FT60).

\begin{table*}[t]
\begin{tabular}{l|l}
\toprule
\textbf{Functional data} quantifying a patient's ability & \textbf{Engagement data} showing participation in \\
to participate in everyday life & skills, exercises, or behaviors relevant for care \\
\midrule
\tabitem Attending school & \tabitem Leaving the house, if fearful of doing so \\
\tabitem Maintaining healthy physiological stress & \tabitem Medication adherence  \\
\tabitem Spending time with family and friends & \tabitem Reducing obsessive handwashing  \\
\tabitem Attending work & \tabitem Communicating suicidal ideation safety plans \\
\tabitem Staying physically active & \tabitem Going to the gym \\
\tabitem Consistent eating behavior & \tabitem Eating a meal \\
\bottomrule
\end{tabular}
\caption{\rev{Examples of functional and engagement data from our findings. 
Engagement data (eg, eating a meal) was often described as a proximal outcome of care and functioning a distal outcome (eg, consistent eating behavior).}
}
% \Description{A table providing definitions and examples of the functional and engagement outcomes data described in Section 4.1.2. There are two columns. In the left column, the first row states the definition of "functional outcomes" data, and then the second row has a bulleted list of functional outcomes described by our participants. In the right column, the first row states the definition of "engagement outcomes" data, and then the second row has a bulleted list of engagement outcomes described by our participants.}
\label{tab:findings:preparation:func-engagement-data}
\end{table*}

We asked participants to explain why functional improvements were not captured by symptom scales.
In other words, if functioning improves, why should we not expect symptoms to decrease?
Participants explained that symptom reduction was not the singular outcome of treatment, but treatment intends to improve functioning even if symptoms persist. 
For example, PS24, a psychiatrist treating patients living with schizophrenia, mentioned how they \textit{``tend to have very chronic patients where the goal isn't to get rid of symptoms, but the goal might be to make symptoms interfere with their life less''} and their patients may \textit{``have ongoing voices and paranoia, but they've gotten to the point where they're able to ignore the voices and attend work''}
SW58 agreed, stating that \textit{``when I think about somebody that experiences psychosis or bipolar disorder or depression, you may have this for your whole life. If the goal is to have fewer symptoms, am I setting you up to fail from the start?''} and they work with patients to understand \textit{``given what your life is, how do you want to live? Maybe there's specific things, maybe you want to go back to school.''}.

Furthermore, participants believed that functional outcomes were likely to improve if their patients engaged in care, and saw engagement as the most proximal outcome of care.
For example, many of our participants were psychotherapists who asked patients to practice specific skills or change behavior as a part of treatment.
FT60 mentioned how they \textit{``had somebody who was washing their hands a hundred times a day and driving his family nuts with accommodations''} and they had their patient \textit{``use judicial safety behaviors to play with his daughter who's crawling around on the floor and then take a shower afterwards.''}
A few months into treatment the \textit{``patient is still washing his hands a hundred times a day, but he and his family are tons happier than they were. They're raving about how well they're functioning and working together now}'' (FT60).
CP45 mentioned how for patients with \textit{``panic disorder, I'd want to have some behavioral data on what they are avoiding or how frequently they are getting out of the house, depending on the specifics of that person.''}
Another participant mentioned how, by tracking engagement, they might feel more confident that \textit{``someone having passive suicidal thoughts would have no intent to act on them''} because they \textit{``have a supportive family that they communicate to and a safety plan in place''} (CP46).
CP43 saw treatment as successful if patients consistently engaged in care, even if symptoms were not fully reduced:

\begin{quote}
   \textit{``Their [symptom] scores do cut in half, but don't move much beyond that and stay relatively stable. 
   If they practice their skills and those are well-developed, they got what I am aiming to provide for them.
   Sure their scores aren't zero, but that might just be because of their personality, environment, social context.''} (CP43)
\end{quote}

Unlike symptom scales, participants saw functioning and engagement as \textit{``trans-diagnostic''} (CP42), measuring care outcomes across patients experiencing different symptoms or disorders.
FT60 qualified that measuring symptoms were not irrelevant, but called for \textit{``a shift from symptom-focused metrics to patient-focused metrics, which can include the symptoms.''}
CP33 wanted to prioritize engagement outcomes for complex cases, giving an example of \textit{``a patient who had comorbid substance use disorder, PTSD [post-traumatic stress disorder], borderline personality disorder, there's a lot of suicidality, a lot of very, very intense mood, depression and anxiety''} that \textit{``those intense things, really intense urges, really intense depression, that didn't go away''} but the patient \textit{``developed trust and she kept coming to therapy. She missed, maybe, four sessions all year. Those are therapeutic gains. She internalized some hope that progress is possible.''}
CP30 further explained the importance of engagement and functional outcomes across conditions:

\begin{quote}

\textit{``If someone has one depressive episode, they will likely have another episode.
Someone who has generalized anxiety disorder may always be a more anxious person. 
Someone with obsessive-compulsive disorder may always be vulnerable to intrusive thoughts. 
It doesn't mean they've failed treatment if they can tolerate the anxiety or cope with the depression, go to work, get out of bed, shower, do the things you have to do, using the skills you learned in therapy.''} (CP30)
\end{quote}


\subsection{Collection: How Can HITs Support Outcomes Data Collection?}
\label{sec:findings:collection}

Our first set of findings describe that participants preferred HITs prioritize storing functional and engagement data as outcomes in VBC.
We then explored with participating clinicians how this data should be collected.
Our related work suggests that existing mental health data are not collected by clinicians because they perceive scale administration as burdensome, administration takes time away from treatment, and \rev{mental health clinicians are not always trained to use outcomes data in care}.
Participants affirmed that data was not collected, stating that \textit{``it's hard to get the buy-in from clinicians who don't have that initial training if they have no reason to do it''} and \textit{``if you are a clinician who does not care and doesn't have buy-in, it's really easy to let it slide off''} (CP35).
PS25 stated that \textit{`it feels like it's hard to seamlessly integrate scales into a session.''}
Another participant stated that data collection is \textit{``extra work and we're not getting paid for it''} (CP48).

Given these complexities, we asked participants about the barriers they saw towards engaging clinicians in data collection, and how HITs could improve outcomes data collection.
First, our participants described existing challenges using HITs to collect and manage outcomes data (Section \ref{sec:findings:collection:challenges}).
Specifically, they described that current clinical data infrastructure, namely electronic health records (EHRs), were not designed to collect mental health data, and it was difficult to acquire funding to improve data infrastructure.
Participants also believed that to engage clinicians in VBC data collection, any mandated data collection should be client-specific, easy to administer, and relevant to decision-making.
In addition, participants saw opportunities (Section \ref{sec:findings:collection:opportunities}) for active and passive data, collected via devices (eg, smartphones, tablets, wearables), to improve engagement in data collection.
But, participants believed that for passive data to be used as VBC outcomes, there would need to be evidence demonstrating that passive data accurately measures the outcomes of care.
Participants also raised practical challenges towards active and passive data use.
Participants wanted to understand who would pay for data collection devices and clinicians' time spent interpreting data, how this data would be integrated into clinical records, and were concerned that prioritizing data collected via devices could increase inequities in care. 

\subsubsection{Existing Challenges Collecting Outcomes Data}
\label{sec:findings:collection:challenges}

We probed participants to understand current challenges towards engaging clinicians in outcomes data collection.
Participants described challenges with existing HITs, specifically the electronic health records (EHRs) used in clinical settings to store and share patient health information, including outcomes data in VBC.
Many participants explained how EHRs were not built for mental health data collection.
For example, CP35 mentioned that \textit{``in the medical center, there's just so much red tape. I know they're trying to integrate outcome measures into our EHR system, but it's so challenging.''}
Participants often operated outside of the EHRs used by other clinicians in their health system.
Therefore, they would \textit{``transcribe scales in the EHR ourselves''}, so that other care providers could access patients' mental health data, but this was \textit{``so open to human error of typing in the numbers''} and thus \textit{``while I want scales to be written in the discharge summary paperwork other providers are getting, that doesn't always happen''} (CP46).
SW49 had personal experience advocating for investments in mental health data collection infrastructure.
Before becoming a clinician, they had worked as the director of data analytics and research at a health system serving patients living with eating disorders. 
They described: 

\begin{quote}
    \textit{``We had built data infrastructure, we were getting data on a weekly basis for all of our own patients across the system, and we were just starting to really incorporate that data into treatment planning and reporting data at the end of care to various stakeholders.
    I was there for three and a half years, and then my position was eliminated in a merger.
    This is an unfortunate aspect of mental health in particular, and especially in the eating disorder space. 
    The margins are really thin and nobody wants to invest in data.''} (SW49)
\end{quote}

Given these challenges, many of our participants chose to not use EHRs for data collection.
For example, SW49 described how their clinic would use \textit{``analog means (eg, recording and sharing measures on paper), which were terrible, but the analog means were more successful.''}
Some participants, specifically those in private practice, could not afford EHRs, and used other software for collecting and managing data.
CP34 described that they \textit{``have an Excel sheet for every client} and \textit{I just plug scale scores in, and then I graph them over time''}.
Another participant, SW17, practiced a form of psychotherapy, called dialectical behavioral therapy (DBT), that requires patients to fill out detailed ``diary cards'' before each clinical encounter. 
They described how \textit{``for every patient we do an EHR note''} but \textit{``DBT ends up being very detailed. I only put general data in the body of my EHR note. But for the actual diary cards, I give patients a paper binder and I keep the binders in my office in a locked cabinet''} (SW17).
Since clinicians do not enter detailed data into EHRs, the data entered into the EHR were often incomplete, missing important information from clinical encounters.
One participant mentioned how missing data could be harmful:

\begin{quote}
    \textit{``A lot of people get lazy about using the EHR and might just type in their note something general like, `the score was this' but not actually record all the individual answers.
    It's important to know where people scored on specific symptoms. 
    For example, on a depression scale, you want to know, are people scoring really highly on suicidality?''} (PS25)
\end{quote}

The prior paragraphs detail challenges but also the necessity to invest in and design HITs for collecting and managing mental health data, supporting data sharing between clinicians and other stakeholders as a part of VBC.
Aside from improving these HITs, participants also described that they would be more likely to engage in outcomes data collection if data were more effectively tied to care.
For example, SW38 described how mandated data were often not relevant for their patients.
They stated that \textit{``some of the assessments I have to do because we're a community-based clinic, but I don't really want to ask a 15-year-old about their heroin use habits if that's not something that is relevant, but I have to.''}
Another participant worried about the burden of outcomes data collection, stating that \textit{``if CMS were to require this data, it's important to do an audit of clinicians' other paperwork requirements when they consider the cost of adding these measures''} (CP42).
Therefore, to encourage data collection, participants wanted scales to be client-specific, easy to administer, and relevant for decision-making.
CP34 stated that the scales they choose to use are \textit{``a jumping off point for interventions. They take very little time, they probably take 30 seconds at the beginning of every session. And they're just really integrated into each session as a way to check in and give the person ownership over what they feel like is bothering them the most.''}


\subsubsection{Opportunities for Technology to Improve Outcomes Data Collection}
\label{sec:findings:collection:opportunities}

Given these challenges, we brainstormed with participants how functional and engagement outcomes could be patient-specific, easy to administer, and integrated into care.
% We were particularly interested in how technology could be useful, aside from EHR improvements.
Participants raised how they used technology to collect patient-specific active data relevant for care.
CP44 stated that they \textit{``had clients FaceTime loved ones for meals''} to demonstrate engagement in care.
Another participant mentioned how \textit{``if a patient's goal was, `I need to exercise more'} their patient \textit{``could take a photo at the gym''} (SW37).
Other participants thought active data collection could be difficult to enforce.
CP34 stated that \textit{``in most of my training it's been hard to even get people into treatment. I did some EMA data collection in grad school with people who were using substances, and it was just really, really challenging.''}

Some participants identified opportunities for passive data to reduce the burden of outcomes data collection.
For example, PS23 described how, for their patients, \textit{``panic is probably the one thing that you can see a lot of for PTSD, where you end up having physiological stress from your illness. A wearable will show an increase in heart rate, an increase in blood pressure, perhaps an increase in sweating, breathing, and respiration rate.''}
Another participant wrote that activity data could be useful \textit{``because whether we're talking about somebody who's depressed or we're talking about somebody where there's some health adherence problems, let's say it's following a cardiac healthy lifestyle or even anxiety, physical activity may be relevant''} (CP45).
PS53 was interested in collecting language data because \textit{``language is what we use to treat and diagnose.''} 

Despite participants' interest in passive data, they did not believe that passive data, at face-value, could be an outcome measure in VBC.
Even though passive data could make data collection easier, participants saw that interpreting passive data could be challenging.
CP42 stated that \textit{``I have no training in interpreting sleep data to know what's normal versus not''}.
Another participant mentioned how \textit{``on the physiological data, I thought about going back to discrepancies between what patients say or how they're behaving with me. I would need to reflect on, either with them or by myself with my supervisors, and say, `What does this data mean?'} (PS53).
Participants also raised privacy concerns with passive data collection, stating \textit{``patients have to be down with the device collecting the data and a clinician seeing all of that data just from a privacy perspective''} (CP42) and for language data that there would be \textit{``some resistance, from clinicians actually more than patients being recorded and things like that. It can be high-liability information''} (PS53).

Another consideration regarding the use of passive data in VBC was validation: that passive measurement tools are able to accurately measure care outcomes.
For example, if there were a VBC functional outcome focused on quantifying sleep improvements, participants wanted assurances that devices could accurately measure sleep.
CP35 mentioned that their clinic chose to use more expensive research-grade actigraphs, versus consumer activity monitors, because they believed that consumer devices were not as well validated.
Specifically, they said that \textit{``we used actigraphs. And not just like your Apple Watch, but well-validated actigraphs, because I learned your smartwatches are not well validated for telling you if you're sleeping when you're supposed to be sleeping''} though \textit{``it would be a lot easier if you could use what somebody is already wearing''} (CP35).
Another participant was unwilling to use passive data, and would prefer using self-reported scales in the absence of rigorous validation:

\begin{quote}
     \textit{``I've been intrigued by the promise and disappointed by the execution of devices. I hear from sleep researchers that unless you're getting a very expensive device that's really closely tracking you, your Apple Watch is not doing a great job estimating how deep is your actual sleep. 
     It's probably capturing general trends. 
     Unless the technology improves, I'd be really okay with just having a self-report.''} (CP43)
\end{quote}

Participants raised other practical challenges towards integrating both active and passive data into VBC.
CP35 raised that the need for payment mechanisms to reimburse for devices and interpreting data, stating that currently \textit{``we didn't bill separately for the watch. The clinic operated at a loss if the patient didn't bring the watch back.''}
Another participant wanted sleep data to be integrated into existing health records, stating \textit{``I would love if a portal integrated sleep data''} (CP42).
Other participants worried that an over-reliance on devices for active and passive data collection would cater to higher resourced individuals who could afford devices and share data.
CP34 stated that \textit{``the quality of the data really does, in my opinion, skew sometimes towards the higher resourced individuals''} and SW28 mentioned that \textit{``there's value in behavioral markers. But, in our setting, which is a hospital, a lot of patients are not going to have an Apple Watch or a smartphone. Patients don't even have WiFi.''}


\subsection{Action: How Should Outcomes Data be Used in VBC?}
\label{sec:findings:action}

The prior section suggests opportunities to invest in HITs, and use active/passive data to improve clinicians' engagement in VBC outcomes data collection.
Once outcomes data has been collected, VBC programs use this data to create financial incentives that hold providers accountable to achieve specific care outcomes.
Participants, generally, recognized the need for more accountability.
One participant explained that \textit{``there are incentives to keep your patient caseload the same when you're in private practice, because it's a lot of work to do intakes, and you get comfortable with the people you see. 
And so, if there's a piece of your reimbursement that's tied to meeting an outcome and then discharging and starting anew, it also holds you more accountable''} (CP35).
Participants also raised that VBC could give patients more control over care decisions.
FT60 stated that \textit{``many providers convince patients that they're failing treatment''} and CP35 continued: \textit{``it's really hard to know, as a consumer, whether or not you're seeing somebody whose skills actually back up what they say. 
Value-based care could help you steer whether or not you go to somebody.''}

In this section, we describe both challenges and opportunities participants' perceived towards using outcomes data in VBC.
First, participants stressed that outcomes data would need to hold providers, healthcare payers, and social service organizations jointly accountable if VBC were to fulfill its promise of improving mental health outcomes (Section \ref{sec:findings:action:accountability}).
Second, participants voiced the need for HITs to implement risk-adjustments to outcomes data, otherwise clinicians may prioritize treating simpler patient cases that inflate care outcomes (Section \ref{sec:findings:action:risk-adjustment}).

\subsubsection{Using Outcomes Data for Joint Accountability in VBC}
\label{sec:findings:action:accountability}

After participants brainstormed what outcomes data HITs should store, we asked them how this data should be used in VBC programs to improve care.
Specifically, we were interested in who should be held accountable: payers, providers, or other entities?
Participating clinicians quickly pointed out that it would be very difficult to attribute the outcomes of care to any one specific entity.
Though providers would love to take credit if outcomes did improve, that was not always possible:

\begin{quote}
    \textit{``I don't really care what's causing the improvement. 
    If that's due to my intervention, great. 
    And if not, still great because they're feeling better. 
    But I think that tying outcomes to something very specific is too complex. 
    There's too many extraneous and confusing variables to ever do that.''} (CP51)
\end{quote}

Participants gave many examples highlighting the need for \textit{joint accountability}, where responsibility for care outcomes is shared across different providers, or external entities that influence mental health.
For example, CP46 worked in an adolescent inpatient psychiatry unit treating patients in crisis. 
In their view, crisis care may not translate into long-term outcomes, explaining that \textit{``patients may feel totally better because they're in the hospital removed from all the stress and problems of life. Once they leave, I suspect their symptoms would increase. One week or month here doesn't solve the patient's way of approaching life''} (CP46).
To solve this challenge, PS23 believed that clinicians providing inpatient and outpatient services -- the long-term care patients receive after discharge from inpatient -- should be held jointly accountable: \textit{``you could look at across the system, but may not be able to look at for the individual provider. From the patients with depression that we treat as inpatients who then go to our outpatient setting, 85\% still meet criteria for remission after one month. That tells us, okay, we as a health system are doing something right.''}
PS53 believed that physical health providers should be accountable for mental health outcomes.
They stated that \textit{``as I'm doing community psych, I'm learning more that outcomes involve physical care, especially if people can't move around, so we need integration''} (PS53).

Outside of holding providers jointly accountable, participants also described how external entities greatly influenced care outcomes.
One participant raised how health insurers should be held accountable, because \textit{``insurers reimburse clinicians so poorly, so the care is not going to be high quality a lot of the time. It's almost like this circular reasoning issue''} (CP34).
Another participant thought that social services, for example housing or education authorities, should also be held accountable for poor mental health outcomes.
They stated that \textit{``in settings like city hospitals, I wish there was more measurement around what interventions have been effective in reducing stressors related to housing, food, and educational access''} (CP34).
Because social factors, like housing, could influence care, some participants described taking a more active role in linking patients to social services.
SW57 mentioned how \textit{``some patients might be looking for mental health housing, so I'll say, `listen, if you want to bring the paperwork into your next session and spend your session with that, we can absolutely do that.' ''}
CP46 and PS23, though, both expressed frustration that social factors could lead to poor care outcomes, and providers, not social services, may be held accountable.
As they expressed:

\begin{quote}
    \textit{``A patient could be in foster care, lose their foster home and become stuck in inpatient care for two months. That outcome, the length of stay, has nothing to do with how much better they were and everything about the systems serving them.}'' (CP46)
\end{quote}

\begin{quote}
    \textit{``We get frustrated when we see a 30-day readmit but then we understand the patient is homeless and it's 30 degrees outside and someone stole their medication.''} (PS23)
\end{quote}

\subsubsection{Risk-Adjusting Outcomes Data to Encourage Fair Treatment}
\label{sec:findings:action:risk-adjustment}

To penalize and reward stakeholders within VBC, participants voiced that the outcomes data HITs share with payers or quality monitoring organizations need to be \textit{risk-adjusted}: adjusting expected care outcomes based upon the difficulty of the patient case.
Otherwise, participants believed that VBC could dis-incentivize clinicians from treating tougher cases to inflate outcome metrics.
CP42 stated \textit{``how long one person needs treatment differs from how long another person needs treatment. And having a strict outcome can mean that people aren't getting enough treatment.''}
Another participant echoed this concern, saying \textit{``it's kind of unfair if you've got someone treating more severe people to be like, `Oh, you suck at your job because you couldn't get your people down to that level' So the challenge is, what do you want your outcome to be?''} (CP43).
FT60 put these terms more starkly:

\begin{quote}
    \textit{``I'm extremely nervous about the impact on care because it's going to turn the clinician against the patient in favor of boosting their scores. 
    I'm fine with outcome measures being exposed to the consumer so that they can make an intelligent decision as to where they want seek care. 
    But I have real concerns about using them for reimbursement criteria or access to care as a consequence.''} (FT60)
\end{quote}

We approached our participants to understand how HITs should perform risk-adjustments to reduce these potential harms.
If outcome metrics were symptom or functional scores, one participant thought about using \textit{``individual changes or change scores rather than absolute zeros. A person who comes in really significantly depressed who moves mild to moderately depressed [on a symptom scale] is a big change. But if you just use absolute scores, that might not reflect that treatment works''} (CP35).
CP42 suggested that change scores should be relative to patients' baselines, stating that \textit{``if one patient's symptom severity were a 10, but they came into me starting at a 24 that would show amazing improvement. Whereas if someone's usually the happiest person in the world and they're now feeling a little depressed, that might be a notable change''} (CP42).

We also asked participants about specific factors risk-adjustment models should account for to estimate expected care outcomes.
PS32 thought models should account for other conditions patients are living with, stating that they can affect mental health outcomes data: \textit{``I have a patient who needed to have bariatric surgery because it's hard to manage their appetite. They have sleep and energy problems that are related to the chronic pain and fibromyalgia. All those other conditions besides depression already bring the PHQ pretty close to a 10, if not higher than a 10}'' (PS32).
Another participant mentioned that models should account for patients' history of mental illness.
Specifically, that they \textit{``tell people starting on a medication that if this is the first time they've been treated for depression or anxiety, I recommend once your symptoms have been alleviated that you stay on the medication for about six months. If they come off the medication at that time there's a 30\% risk of relapse. If this is your second time with an episode of depression or anxiety, the chances of a third relapse are 70\%. The goalposts get moved a little bit''} (PS23).
In addition, PS23 also raised how social factors influence care outcomes, because \textit{``in the last place I worked, 70\% of our patients were homeless. These are people with a high level of needs, a high level of trauma and stress.''}
CP33 agreed, saying that \textit{``I wouldn't expect someone who's coming in with a really severe depression, who has multiple stressors, and maybe less resilience factors, fewer support system, all kinds of things, to come out of that at the same place as someone who has had a relatively supportive and stable household.''}


In addition to using these factors to moderate expectations, participants also believed these factors should moderate the expected length of treatment, because \textit{``I wouldn't necessarily expect progress to happen in the same way or in the same timeframe''} (CP33).
One participant stated that \textit{``obviously we hope that all of our clients improve. Maybe, over a longer period of time we start to see improvement because you had a long period of therapy. But I don't know that timeframe''} (CP44).
Yet, not all participants were convinced that patients need to be in care for a long period of time to see improvement.
As one participant stated:

\begin{quote}
    \textit{``I would expect change. I would 100\% disabuse the notion that you need to be in therapy for years to see progress and instead show that a lot of people can get better from just a few sessions.''} (CP43)
\end{quote}



