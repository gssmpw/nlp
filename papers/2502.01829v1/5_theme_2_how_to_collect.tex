\subsection{Collection: How Can HITs Support Outcomes Data Collection?}
\label{sec:findings:collection}

Our first set of findings describe that participants preferred HITs prioritize storing functional and engagement data as outcomes in VBC.
We then explored with participating clinicians how this data should be collected.
Our related work suggests that existing mental health data are not collected by clinicians because they perceive scale administration as burdensome, administration takes time away from treatment, and \rev{mental health clinicians are not always trained to use outcomes data in care}.
Participants affirmed that data was not collected, stating that \textit{``it's hard to get the buy-in from clinicians who don't have that initial training if they have no reason to do it''} and \textit{``if you are a clinician who does not care and doesn't have buy-in, it's really easy to let it slide off''} (CP35).
PS25 stated that \textit{`it feels like it's hard to seamlessly integrate scales into a session.''}
Another participant stated that data collection is \textit{``extra work and we're not getting paid for it''} (CP48).

Given these complexities, we asked participants about the barriers they saw towards engaging clinicians in data collection, and how HITs could improve outcomes data collection.
First, our participants described existing challenges using HITs to collect and manage outcomes data (Section \ref{sec:findings:collection:challenges}).
Specifically, they described that current clinical data infrastructure, namely electronic health records (EHRs), were not designed to collect mental health data, and it was difficult to acquire funding to improve data infrastructure.
Participants also believed that to engage clinicians in VBC data collection, any mandated data collection should be client-specific, easy to administer, and relevant to decision-making.
In addition, participants saw opportunities (Section \ref{sec:findings:collection:opportunities}) for active and passive data, collected via devices (eg, smartphones, tablets, wearables), to improve engagement in data collection.
But, participants believed that for passive data to be used as VBC outcomes, there would need to be evidence demonstrating that passive data accurately measures the outcomes of care.
Participants also raised practical challenges towards active and passive data use.
Participants wanted to understand who would pay for data collection devices and clinicians' time spent interpreting data, how this data would be integrated into clinical records, and were concerned that prioritizing data collected via devices could increase inequities in care. 

\subsubsection{Existing Challenges Collecting Outcomes Data}
\label{sec:findings:collection:challenges}

We probed participants to understand current challenges towards engaging clinicians in outcomes data collection.
Participants described challenges with existing HITs, specifically the electronic health records (EHRs) used in clinical settings to store and share patient health information, including outcomes data in VBC.
Many participants explained how EHRs were not built for mental health data collection.
For example, CP35 mentioned that \textit{``in the medical center, there's just so much red tape. I know they're trying to integrate outcome measures into our EHR system, but it's so challenging.''}
Participants often operated outside of the EHRs used by other clinicians in their health system.
Therefore, they would \textit{``transcribe scales in the EHR ourselves''}, so that other care providers could access patients' mental health data, but this was \textit{``so open to human error of typing in the numbers''} and thus \textit{``while I want scales to be written in the discharge summary paperwork other providers are getting, that doesn't always happen''} (CP46).
SW49 had personal experience advocating for investments in mental health data collection infrastructure.
Before becoming a clinician, they had worked as the director of data analytics and research at a health system serving patients living with eating disorders. 
They described: 

\begin{quote}
    \textit{``We had built data infrastructure, we were getting data on a weekly basis for all of our own patients across the system, and we were just starting to really incorporate that data into treatment planning and reporting data at the end of care to various stakeholders.
    I was there for three and a half years, and then my position was eliminated in a merger.
    This is an unfortunate aspect of mental health in particular, and especially in the eating disorder space. 
    The margins are really thin and nobody wants to invest in data.''} (SW49)
\end{quote}

Given these challenges, many of our participants chose to not use EHRs for data collection.
For example, SW49 described how their clinic would use \textit{``analog means (eg, recording and sharing measures on paper), which were terrible, but the analog means were more successful.''}
Some participants, specifically those in private practice, could not afford EHRs, and used other software for collecting and managing data.
CP34 described that they \textit{``have an Excel sheet for every client} and \textit{I just plug scale scores in, and then I graph them over time''}.
Another participant, SW17, practiced a form of psychotherapy, called dialectical behavioral therapy (DBT), that requires patients to fill out detailed ``diary cards'' before each clinical encounter. 
They described how \textit{``for every patient we do an EHR note''} but \textit{``DBT ends up being very detailed. I only put general data in the body of my EHR note. But for the actual diary cards, I give patients a paper binder and I keep the binders in my office in a locked cabinet''} (SW17).
Since clinicians do not enter detailed data into EHRs, the data entered into the EHR were often incomplete, missing important information from clinical encounters.
One participant mentioned how missing data could be harmful:

\begin{quote}
    \textit{``A lot of people get lazy about using the EHR and might just type in their note something general like, `the score was this' but not actually record all the individual answers.
    It's important to know where people scored on specific symptoms. 
    For example, on a depression scale, you want to know, are people scoring really highly on suicidality?''} (PS25)
\end{quote}

The prior paragraphs detail challenges but also the necessity to invest in and design HITs for collecting and managing mental health data, supporting data sharing between clinicians and other stakeholders as a part of VBC.
Aside from improving these HITs, participants also described that they would be more likely to engage in outcomes data collection if data were more effectively tied to care.
For example, SW38 described how mandated data were often not relevant for their patients.
They stated that \textit{``some of the assessments I have to do because we're a community-based clinic, but I don't really want to ask a 15-year-old about their heroin use habits if that's not something that is relevant, but I have to.''}
Another participant worried about the burden of outcomes data collection, stating that \textit{``if CMS were to require this data, it's important to do an audit of clinicians' other paperwork requirements when they consider the cost of adding these measures''} (CP42).
Therefore, to encourage data collection, participants wanted scales to be client-specific, easy to administer, and relevant for decision-making.
CP34 stated that the scales they choose to use are \textit{``a jumping off point for interventions. They take very little time, they probably take 30 seconds at the beginning of every session. And they're just really integrated into each session as a way to check in and give the person ownership over what they feel like is bothering them the most.''}


\subsubsection{Opportunities for Technology to Improve Outcomes Data Collection}
\label{sec:findings:collection:opportunities}

Given these challenges, we brainstormed with participants how functional and engagement outcomes could be patient-specific, easy to administer, and integrated into care.
% We were particularly interested in how technology could be useful, aside from EHR improvements.
Participants raised how they used technology to collect patient-specific active data relevant for care.
CP44 stated that they \textit{``had clients FaceTime loved ones for meals''} to demonstrate engagement in care.
Another participant mentioned how \textit{``if a patient's goal was, `I need to exercise more'} their patient \textit{``could take a photo at the gym''} (SW37).
Other participants thought active data collection could be difficult to enforce.
CP34 stated that \textit{``in most of my training it's been hard to even get people into treatment. I did some EMA data collection in grad school with people who were using substances, and it was just really, really challenging.''}

Some participants identified opportunities for passive data to reduce the burden of outcomes data collection.
For example, PS23 described how, for their patients, \textit{``panic is probably the one thing that you can see a lot of for PTSD, where you end up having physiological stress from your illness. A wearable will show an increase in heart rate, an increase in blood pressure, perhaps an increase in sweating, breathing, and respiration rate.''}
Another participant wrote that activity data could be useful \textit{``because whether we're talking about somebody who's depressed or we're talking about somebody where there's some health adherence problems, let's say it's following a cardiac healthy lifestyle or even anxiety, physical activity may be relevant''} (CP45).
PS53 was interested in collecting language data because \textit{``language is what we use to treat and diagnose.''} 

Despite participants' interest in passive data, they did not believe that passive data, at face-value, could be an outcome measure in VBC.
Even though passive data could make data collection easier, participants saw that interpreting passive data could be challenging.
CP42 stated that \textit{``I have no training in interpreting sleep data to know what's normal versus not''}.
Another participant mentioned how \textit{``on the physiological data, I thought about going back to discrepancies between what patients say or how they're behaving with me. I would need to reflect on, either with them or by myself with my supervisors, and say, `What does this data mean?'} (PS53).
Participants also raised privacy concerns with passive data collection, stating \textit{``patients have to be down with the device collecting the data and a clinician seeing all of that data just from a privacy perspective''} (CP42) and for language data that there would be \textit{``some resistance, from clinicians actually more than patients being recorded and things like that. It can be high-liability information''} (PS53).

Another consideration regarding the use of passive data in VBC was validation: that passive measurement tools are able to accurately measure care outcomes.
For example, if there were a VBC functional outcome focused on quantifying sleep improvements, participants wanted assurances that devices could accurately measure sleep.
CP35 mentioned that their clinic chose to use more expensive research-grade actigraphs, versus consumer activity monitors, because they believed that consumer devices were not as well validated.
Specifically, they said that \textit{``we used actigraphs. And not just like your Apple Watch, but well-validated actigraphs, because I learned your smartwatches are not well validated for telling you if you're sleeping when you're supposed to be sleeping''} though \textit{``it would be a lot easier if you could use what somebody is already wearing''} (CP35).
Another participant was unwilling to use passive data, and would prefer using self-reported scales in the absence of rigorous validation:

\begin{quote}
     \textit{``I've been intrigued by the promise and disappointed by the execution of devices. I hear from sleep researchers that unless you're getting a very expensive device that's really closely tracking you, your Apple Watch is not doing a great job estimating how deep is your actual sleep. 
     It's probably capturing general trends. 
     Unless the technology improves, I'd be really okay with just having a self-report.''} (CP43)
\end{quote}

Participants raised other practical challenges towards integrating both active and passive data into VBC.
CP35 raised that the need for payment mechanisms to reimburse for devices and interpreting data, stating that currently \textit{``we didn't bill separately for the watch. The clinic operated at a loss if the patient didn't bring the watch back.''}
Another participant wanted sleep data to be integrated into existing health records, stating \textit{``I would love if a portal integrated sleep data''} (CP42).
Other participants worried that an over-reliance on devices for active and passive data collection would cater to higher resourced individuals who could afford devices and share data.
CP34 stated that \textit{``the quality of the data really does, in my opinion, skew sometimes towards the higher resourced individuals''} and SW28 mentioned that \textit{``there's value in behavioral markers. But, in our setting, which is a hospital, a lot of patients are not going to have an Apple Watch or a smartphone. Patients don't even have WiFi.''}

