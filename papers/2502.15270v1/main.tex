\documentclass[sigconf,nonacm]{acmart}
% \usepackage{usenix}

\let\Bbbk\relax
% mypacakge 
\usepackage{amsmath}
\usepackage{amssymb,amsfonts}
\usepackage{textcomp}
\usepackage{xcolor}
\usepackage{url}
\usepackage{hyperref}
\usepackage{seqsplit}

\definecolor{yellow}{RGB}{255,255,153}
\definecolor{grey}{RGB}{224,224,224}
\definecolor{green}{RGB}{0,100,0}
\newcommand{\highlight}{\cellcolor{grey}}

\pagestyle{plain} % removes running headers
\usepackage{algorithm}
\usepackage{algorithmic}
\usepackage{graphicx}
\usepackage{subfigure} 
\usepackage{fancybox}
\usepackage{xspace}
\usepackage{threeparttable}
\usepackage{multirow}
\usepackage{enumitem}
\usepackage{seqsplit}
\usepackage{pifont}
\usepackage{caption}
\renewcommand{\algorithmicrequire}{ \textbf{Input:}} %Use Input in the format of Algorithm
\renewcommand{\algorithmicensure}{ \textbf{Output:}} %UseOutput in the format of Algorithm


\iffalse
\addtolength{\floatsep}{-2mm}
\addtolength{\dblfloatsep}{-4mm}
\addtolength{\textfloatsep}{-4mm}  % single column figure margin
\addtolength{\dbltextfloatsep}{-3mm}  % double column figure margin
\addtolength{\abovecaptionskip}{-2mm}
\addtolength{\belowcaptionskip}{-2mm}
\fi

\newcommand{\framework}{\textsc{IDRadar}}

\newcommand{\tabincell}[2]{\begin{tabular}{@{}#1@{}}#2\end{tabular}}
\newcommand{\haoyu}[1]{{\color{red} HW: #1}}
\newcommand{\todo}[1]{\textcolor{blue}{(TODO: #1)}}
\newcommand{\wc}[1]{\textcolor{purple}{(wc: #1)}}
\newcommand{\yj}[1]{\textcolor{blue}{(yj: #1)}}
\newcommand{\dzk}[1]{{\color{orange} #1}}
\newcommand{\tm}[1]{{\color{magenta} tm: #1}}


% result number
% approach
\newcommand{\datasetsize}{1,814 }
\newcommand{\totalbrand}{250 }
\newcommand{\totalsensitiveproperties}{272 }
\newcommand{\totalsensitivepropertiescase}{8,192 }
\newcommand{\totalsensitivesettings}{104 }
\newcommand{\totalsensitivesettingscase}{3,620 }
\newcommand{\totaldangerousproperties}{244 }
\newcommand{\totaldangerouspropertiescase}{3,477 }
\newcommand{\totaldangeroussettings}{39 }
\newcommand{\totaldangeroussettingscase}{1,336 }
\newcommand{\totalsensitivedevices}{1,406 }
\newcommand{\totaldangerousdevices}{1,112 }
\newcommand{\totalvulnerablecase}{4,813 }
% dynamic test
\newcommand{\dynamicmodels}{314 }
\newcommand{\dynamicbrands}{22 }
\newcommand{\dynamicproperties}{51 }
\newcommand{\dynamicpropertiescase}{160 }
\newcommand{\dynamicsettings}{15 }
\newcommand{\dynamicsettingscase}{400 }
\newcommand{\dynamicpropertyandsetting}{66 }
\newcommand{\dynamictotal}{216 }
\newcommand{\dynamicvulnerablecase}{560 }

\begin{document}
\title{On the (In)Security of Non-resettable Device Identifiers in Custom Android Systems}

\author{Zikan Dong}
% \email{dongzikan@bupt.edu.cn}
\affiliation{%
  \institution{Beijing University of Posts and Telecommunications}
  \city{Beijing}           
  \country{China}
}


\author{Liu Wang}
\affiliation{%
  \institution{Beijing University of Posts and Telecommunications}
  \city{Beijing}           
  \country{China}
}

\author{Guoai Xu}
% \email{xga@hit.edu.cn}
\affiliation{%
  \institution{Harbin Institute of Technology, Shenzhen}
  \city{Shenzhen}           
  \country{China}
}

\author{Haoyu Wang}
\authornotemark[2]
% \email{haoyuwang@hust.edu.cn}
\affiliation{%
  \institution{Huazhong University of Science and Technology}
  \city{Wuhan}           
  \country{China}
}


\begin{abstract}
User tracking is critical in the mobile ecosystem, which relies on device identifiers to build clear user profiles. 
In earlier ages, Android allowed easy access to non-resettable device identifiers like device serial numbers and IMEI by third-party apps for user tracking.
As privacy concerns grew, Google has tightened restrictions on these identifiers in native Android.
Despite this, stakeholders in custom Android systems seek consistent and stable user tracking capabilities across different system and device models, and they have introduced covert channels (e.g., system properties and settings) in customized systems to access identifiers, which undoubtedly increases the risk of user privacy breaches.
This paper examines the introduction of non-resettable identifiers through system customization and their vulnerability due to poor access control.
We present {\framework}, a scalable and accurate approach for identifying vulnerable properties and settings on custom Android ROMs.
Applying our approach to \datasetsize custom ROMs, we have identified \totalsensitivepropertiescase system properties and \totalsensitivesettingscase settings that store non-resettable identifiers, with \totaldangerouspropertiescase properties and \totaldangeroussettingscase settings lacking adequate access control, which can be abused by third-party apps to track users without permissions. Our large-scale analysis can identify a large number of security issues which are two orders of magnitude greater than existing techniques.
We further investigate the root causes of these access control deficiencies.
Validation on 32 devices through the remote real-device testing service confirmed our results.
Additionally, we observe that the vulnerable properties and settings usually occur in devices of the same OEMs. 
We have reported our findings to the respective vendors and received positive confirmations.
Our work underscores the need for greater scrutiny of covert access channels to device identifiers and better solutions to safeguard user privacy during system customizations.

\end{abstract}

\maketitle

\documentclass[../main.tex]{subfiles}
\graphicspath{{../images/}}
\makeatletter
\def\input@path{{../images/}}
\makeatother
\begin{document}
\section{Introduction}
\begin{figure}
\centering
\begin{tikzpicture}
\node[inner sep=0pt] (ws) at (0, 0) {
\includegraphics[height=.4\textwidth, trim={10cm 0 10cm 0},clip]{world_space.png}};
\node[inner sep=0pt] (cs) at (6,0) {\includegraphics[height=.4\textwidth, trim={10cm 1cm 10cm 4cm},clip]{conf_space.png}};
\end{tikzpicture}
\vspace{-5pt}
\label{fig:pbrm_intro}
\caption{\textbf{Left}: Shows world space obstacles as grey spheres. Robots start and goal configuration is colored red and green, respectively. Configurations along the computed path are colored transparent blue. \textbf{Right:} Mapped world space scenario to configuration space. Obstacle region is the grey mesh. Red spheres are collision-free regions computed by the neural SCDF. The optimized shortest path in the convex corridor is the blue curve.}
\vspace{-25pt}
\end{figure}
Motion planning is the problem of finding a collision-free trajectory that connects a given start and goal configuration. The planning takes place in the configuration space of the robot. For single body robots, like mobile robots or drones, the configuration space and the world space are usually the same. This simplifies the planning, since explicit obstacle representations are available which enables geometrical tools like separating hyperplanes, smallest distance to obstacles etc., to be used when designing motion planning algorithms. For multi-body robots like manipulators, the situation is completely different. The world space obstacles are usually mapped to non-convex regions, and to make the problem even harder, the mapping is usually not known. Forming explicit representations of the obstacle region in the configuration space is usually too expensive or intractable. Despite all of this, sampling based planners are used with great success, which mainly is due to their use of implicit representations of the obstacle region. The basic idea is to construct a graph in the configuration space that covers and connects the collision-free region. From this graph, a path can be extracted that connects a given start and goal configuration. The approach is computationally expensive, since the graph is constructed with the smallest geometrical building block available, points, which represents a collision-check. Furthermore, the extracted paths from the graph are non-smooth and jagged due to the stochastic nature of the approach. This adds an additional post-processing step to the process, where the paths are shortcutted and smoothened, before the path can be used for tracking. Clearly a lot of time is invested to form this graph and produce smooth paths. Thus, if the obstacles start to move, then all of this work is done in no use, since all points that make up this graph need to be re-verified, which is simply too time consuming to be done in real time.
\\\\
In this work, we want to address the existing drawbacks of the sampling based planners. Our main contribution is an improved motion planner where each vertex in the graph covers a collision-free region in the form of a sphere instead of a point and where the edges are formed with neighboring intersecting spheres. This representation has the advantage of instead of returning piecewise linear paths, returning a sequence of overlapping spheres, i.e. a convex corridor, that connects a given start and goal configuration, illustrated in Figure \ref{fig:pbrm_intro}. This convex corridor allows us to use convex optimization to produce smooth trajectories, instead of computationally expensive post-processing methods. The representation further allows us to estimate the coverage of the collision-free space, which gives us awareness and feedback in the offline roadmap construction phase. Finally, our representation is simple to adapt to moving obstacles, simply requery for the new radii and recheck for intersections. 
\\\\
The spherical collision-free regions are formed using a signed distance function (SDF), which is a function that returns the smallest distance from an arbitrary point to the boundary of an obstacle. As the name implies, the distance is signed, thus if the point is inside the obstacle it is negative otherwise positive. If the distance is positive, a sphere with radius equal to the distance is guaranteed to cover a collision-free region. Using an SDF in motion planning is not new, but what is novel about our approach is that we express the distance in the configuration space instead of the world space and by doing so allows us to form these convex collision-free regions. We refer to the resulting SDF as a signed configuration distance function (SCDF). Computing an SCDF analytically is non-trivial, our approach is therefore to parameterize the SCDF with a deep neural network and learn the mapping by supervised learning. Our resulting neural SCDF can compute distances for different parameter values of obstacle shapes and we also show how multiple distances can be combined, thus making our approach flexible.
\section{Related work}
Motion planning algorithms can roughly be divided into three families, grid-based, sampling based and optimization based methods. Grid-based methods (GBM) discretize the planning space from which a graph is then compiled. A standard search method is A$^\star$ \citep{a_star}, which is classified as an \textit{informed} search method, since it employs a heuristic function to speed up the search. A$^\star$ guarantees to return an optimal path at the level of discretization used. GBMs usually discretize the planning space by a regular lattice and this limits the GBMs to problems with low dimensionality due to the curse of dimensionality. Thus, GBMs are usually limited to single-body robots where the degrees of freedom (DOF) are low. To overcome the inherent scaling problem with the GBMs, stochastic methods are usually used for multi-body robots. These methods are termed as sampling-based methods (SBM) and core members within this family are the rapidly-exploring random trees (RRT) \citep{rrt} and the probabilistic roadmap (PRM) \citep{prm}. RRT grows a tree from the start configuration and explores the collision-free region in a rapid way until it is able to connect to the goal region. RRT is usually improved by bi-directional planning \citep{rrt_connect}, i.e. an additional tree is grown from the goal configuration and the trees are tested for connection after any tree has been expanded. RRT is a single-query method, thus it searches for a path from scratch each time it is queried. Contrary to this, PRM is a multi-query method, which solves for multiple queries without starting from scratch. PRM does this by creating a roadmap (graph) that covers the collision-free space as an offline step. The graph is then used to solve for multiple queries. PRMs are used in cases where the environment does not change since the extra offline step is too computationally costly and needs to be re-done if the environment is changed. In our work, we address this inherent issue by using a different roadmap representation. Our vertices in the graph cover a collision-free region in the form of spheres and we form the edges by checking for intersecting spheres. If something in the environment changes, we recompute the spheres radii and recheck the intersections, without relying on collision detection. We use a trained neural network to compute the sphere radius, therefore querying for the radius can be done fast, hence our representation enables the PRM for dynamic environments.
\\\\
In the recent decades, optimization based methods (OBM) \citep{chomp, schulman, itomp, stomp} have been introduced as an alternative to SBM for multi-body robots. Like the SBM, the OBMs scale well to higher dimensional problems and produce smoother motion. It is common to use a SDF in the optimization since it is a smooth function, thus enabling gradient-based methods. However, the standard way of expressing the SDF is in world space. The distance therefore needs to be mapped to the configuration space by the forward kinematics. This mapping makes the optimization problem a non-linear program (NLP), which is computationally expensive to solve. Recently, a different approach has been proposed. In \cite{mp_gcs} motion planning is formulated as a convex optimization problem by using the graph of convex sets framework \citep{gcs}. The underlying idea is to decompose the collision-free space into intersecting convex sets from which a convex optimization problem is formulated. In cases where an explicit representation of the obstacles in the configuration space exists, like for single-body robots, creating collision-free convex regions can be done fast \citep{iris}. For multi-body robots, this is non-trivial. Existing work does this successfully \citep{iris_nlp, iris_c} by an optimization based approach, but the methods are still too time consuming to be used in the presence of moving obstacles. Our approach is instead to use deep learning to learn an SDF expressed in the configuration space. With this, we can query for shortest distances to the collision boundary, which allows us to expand spherical regions which are collision-free. Our approach is fast and therefore enables our suggested roadmap planner to be used in dynamic environments.
\\\\
Recent research has focused on learning collision detection \citep{fk_kernel_distance, diffco, graphdistnet} by predicting the signed distance between the robot links and the surrounding obstacles in the world space. The learned SDF is used in trajectory optimization but since the distance is expressed in the world space, the problem becomes an NLP and therefore takes a long time to solve. We take a novel approach and suggest to instead express the signed distance in the configuration space. This allows us to improve the PRM at the same time as it enables convex optimization for trajectory optimization, which runs faster and is more reliable than NLP solvers. In \cite{cspf} a learned signed distance function in the configuration space is proposed similar to our approach. However, their approach is restricted to point cloud representations, while we propose to represent the obstacles as parameterized geometric shapes, e.g. spheres. Furthermore, we also show how to use our learned SCDF to improve an existing roadmap planner.
\section{Problem formulation}
A robot is located in the world space, $\W \subset \R^3 $. The unique location of the robot is given by its configuration $\q \in \C$, where $\C$ is the configuration space. The set of points covered by the robots bodies at a certain configuration is expressed as $\B(\q) \subset \W$. The robot is surrounded by $\NrObst$ obstacles $\O = \bigcup_{i=1}^{\NrObst} \O_i$, where  $\O_i \subset \W$. The representation of the obstacle in the configuration space is the set $\C\O_i = \{\q \in \C \: |\: \B(\q) \cap \O_i \neq \emptyset \}$. The obstacle space is formed as $\Co = \bigcup_{i=1}^{\NrObst} \C \O_i$. The complement is referred to as the free space, $\Cf = \C \setminus \Co$. The path planning problem is a tuple, ($\Cf$, $\qStart$, $\qGoal$), where we want to connect a query pair, consisting of a start, $\qStart$, and goal configuration, $\qGoal$, with a geometric path, $\q(s): [0, 1] \mapsto \Cf$, such that $\q(0)=\qStart$ and $\q(1)=\qGoal$, or report correctly when such a path does not exist.
\end{document}

\section{Basic Background: Supervised Learning and the PAC Model}
\label{sec:background}

At this point almost everyone has heard of machine learning (ML). Anyone likely to stumble upon this article will have also heard of its most influential special case, supervised learning, and those theoretically inclined will also be familiar with the PAC model. Nonetheless, I will set the stage by  recapping the basics.

\subsection{Basics of Supervised Learning}%Let's set the stage in any case

\emph{Supervised Learning} is the task of ``coming up'' with a function $f: \X \to \Y$ to ``explain'' or ``fit'' a sequence of input/output examples   $(x_1,y_1), \ldots, (x_n,y_n)$, with $x_i \in \X$ and $y_i \in \Y$.  Here $\X$ is a \emph{data domain} consisting of \emph{datapoints} $x \in \X$, $\Y$ is a \emph{label set} consisting of \emph{labels} $y \in \Y$, and the sequence $(x_1,y_1),\ldots,(x_n,y_n)$ is the \emph{training data} consisting of \emph{labeled examples (a.k.a. samples)}~$(x_i,y_i)$.  I~will refer to the chosen function $f$ as a \emph{predictor}, and to $n$ as the \emph{sample size}. A \emph{learning algorithm} takes as input training data, and outputs (some representation of) a predictor $f \in \Y^\X$.\footnote{Note that this describes the usual \emph{batch}, a.k.a.~\emph{offline}, setting of supervised learning. I do not discuss other paradigms such as online or active learning in this article.} 



Success in supervised learning is defined as \emph{generalization} to  future examples: For a typical \emph{test example}  $(x_{\tst},y_{\tst})$, the predicted label $y'_{\tst}=f(x_{\tst})$ should ``equal'' $y_{\tst}$, perhaps approximately. We usually assume the test example is drawn from the same  ``source'' as the training data  --- commonly, i.i.d.~from the same distribution. The quality of the prediction is quantified by $\ell(y'_{\tst},y_{\tst})$, where $\ell:~\Y~\times~\Y \to \RR_{\geq 0}$ is a \emph{loss function} chosen as part of the problem definition. Common loss functions include the 0-1 loss $\ell_{0-1}(y',y) = [y' \neq y]$ for \emph{classification} problems,\footnote{The notation $[P]$ denotes $1$ when predicate $P$ is true, and denotes $0$ when $P$ is false.} as well as the absolute loss $|y'-y|$ or squared loss $(y'-y)^2$ for \emph{regression problems} featuring $\Y  \sse \RR$.

Nontrivial generalization properties are typically only possible if one assumes something about the data.\footnote{The need for such an assumption is formalized by the  \emph{no free lunch theorems} of supervised learning \cite{wolpert_connection_1992,wolpert_lack_1996,schaffer_conservation_1994}.} The Bayesian approach to  machine learning, common in many applications, assumes some parametric form for the distribution generating the data, and postulates a prior on the parameters. This is not the approach I will take in this article. Instead, I will focus on the frequentist --- and some would say ``worst-case'' or ``adversarial'' ---  approach that is common in the computational learning theory community, embodied by the PAC model. Here we assume that the (training and test) data can be explained, perhaps approximately, by a function in some ``simple enough to learn'' class of functions $\H \sse \Y^\X$, often called the \emph{hypotheses}. Equivalently, we  seek a predictor which explains the unseen data roughly  as well as the best hypothesis $h^* \in \H$, whether or not we assume that $h^*$ itself provides a perfect explanation.



 \paragraph{Common Algorithmic Templates.} Perhaps the best known general-purpose supervised learning algorithm is \emph{empirical risk minimization (ERM)}, which chooses as its predictor a hypothesis $f \in \H$ minimizing $\frac{1}{n} \sum_{i=1}^n \ell(f(x_i),y_i)$ --- a quantity called the \emph{training error}, \emph{empirical error}, or \emph{empirical risk} of $f$. %\footnote{When multiple hypotheses minimize the empirical risk, we assume ERM breaks ties arbitrarily.}
A common template for generalizing ERM involves adding a \emph{regularization term} $\psi(f)$ to the  objective function, typically chosen to measure some notion of ``hypothesis complexity.'' An algorithm instantiating this template is known as a \emph{structural risk minimizer (SRM)}, and chooses as its predictor the hypothesis $f \in \H$ minimizing the \emph{structural risk} $\frac{1}{n} \sum_{i=1}^n \ell(f(x_i),y_i) + \psi(f)$. Other well-known algorithms, such as gradient descent and its variations,  can frequently be interpreted as approximate implementations of ERM or SRM.


\paragraph{Proper vs Improper Learning.} A learning algorithm is said to be \emph{proper} if its predictor $f$ is always chosen from the hypothesis class, i.e., $f \in \H$, otherwise it is said to be \emph{improper}. ERM  is an example of a proper learning algorithm, as are SRM algorithms of the form described above.  In the \emph{proper regime} of learning, algorithms are required to be proper. This article will be concerned with the more flexible \emph{improper regime} (a.k.a \emph{representation-independent learning}), where no such constraint is placed on the learner. In other words, all we care about is predictive power at test time, rather than any insights derived from the functional form or representation of the predictor~itself.


\subsection{The PAC Model}
A standard mathematical setup for evaluation of supervised learning algorithms, at least in the theoretical computer science community, is Valiant's \emph{Probably Approximately Correct (PAC) model} of learning (see e.g.~\cite{kearns_introduction_1994,mohri_foundations_2018}). Here, we assume there is an unknown distribution $\D$ on $\X \times \Y$ from which training and test data are  drawn.  Specifically, the labeled datapoints of the training set  $(x_1,y_1), \ldots, (x_n,y_n)$, as well as the test data  $(x_\tst,y_\tst)$, are i.i.d.~from $\D$. Often it is assumed that $\D$ lies in some class of distributions of interest. The \emph{true expected loss}, or simply \emph{loss}, of a predictor $f: \X \to \Y$ is the expected loss it incurs on draws from $\D$, written $L_\D(f) = \Ex_{(x,y) \sim \D} \ell(f(x),y)$.


There are two main ``settings'' in PAC learning. The  \emph{realizable setting} only requires that the data be perfectly explained by some hypothesis in $\H$. More generally, the \emph{agnostic setting} makes no assumption relating the data to the hypotheses, but shifts the goalposts as necessary to allow nontrivial guarantees: the expected loss at test time is evaluated only ``relative'' to that of the best hypothesis $h^* \in \H$. There are other settings which make more nuanced assumptions, such as $\D$ being of a particular parametric form or its support living in some (unknown) lower-dimensional space, etc. I will mostly discuss the realizable and agnostic settings in this article, those being the simplest and most studied from a theoretical perspective. %TODO:We will briefly discuss other settings in Section ??

The PAC model demands high probability guarantees of learners, in the worst case over distributions of interest. Consider first the realizable setting, where $\D$ is such that $\min_{h \in \H} L_{\D}(h) = 0$. A PAC learner has \emph{error} $\epsilon=\epsilon(n)$ and \emph{confidence} $\delta=\delta(n)$ if, when training data consists of $n$ i.i.d~samples from a realizable distribution $\D$, it produces a predictor $f$  satisfying $L_\D(f) \leq \epsilon$ with probability at least $1-\delta$. In the agnostic setting, where $\D$ can be arbitrary, we require $L_\D(f) - \min_{h \in \H} L_\D(h) \leq \epsilon$ with probability $1-\delta$.

In both the realizable and agnostic settings, we look for PAC learners with small $\epsilon$ and $\delta$ as a function of the sample size $n$. An equivalent perspective looks at the sample complexity $m(\epsilon,\delta)$, which is the minimum sample size which guarantees error  at most $\epsilon$ with probability at least $1-\delta$. We say a problem is \emph{PAC learnable} if its PAC sample complexity is finite whenever $\epsilon,\delta > 0$.

For most PAC learning problems, learnability and sample complexity are characterized in terms of a  ``dimension'' of the hypothesis class. Most prominently this is the \emph{VC dimension} for binary classification, the \emph{fat shattering dimension} for agnostic regression, and the \emph{DS dimension} for multiclass classification (see \cite{anthony_neural_1999,daniely_optimal_2014,brukhim_characterization_2022}). Treatment of these is beyond the scope of this article. The unfamiliar reader need not worry, however,  as dimensions will feature only tangentially in our~discussion.




%\paragraph{Learning settings: Realizable, Agnostic, etc.} In learning theory, evaluating a supervised learning algorithm requires specifying a data model and an objective. We will leave the details of the data model flexible for now, to allow for both the PAC model and the adversarial transductive model. Nonetheless we will describe two variations, which we call ``settings'', which cut across different models. The  \emph{realizable setting}  requires only that the data be perfectly explained by some hypothesis $h \in \H$ --- i.e., there exists a hypothesis which is guaranteed to suffer a loss of $0$ on training and test data. The performance of the learning algorithm is its expected loss at test time for some ``worst case'' realizable instance. More generally, the \emph{agnostic setting} makes no assumption relating the data to the hypotheses, but shifts the goalposts as necessary to allow nontrivial guarantees: the expected loss at test time is evaluated only ``relative'' to that of the best hypothesis $h^* \in \H$, again for some ``worst case'' instance. There are other settings which make more nuanced assumptions about the data, such as it is drawn from a distribution of a particular parametric form, or that it lives in some (unknown) lower-dimensional space, etc. We will mostly discuss the realizable and agnostic settings, those being the simplest and most studied from a theoretical perspective.




%%% Local Variables:
%%% mode: latex
%%% TeX-master: "learning_matching"
%%% End:

\section{Experiments}\label{sec_exp}
%\hp{Accelerating IM simulation~\cite{tang2015influence}}

% \begin{itemize}
%     \item 6.1. Problem setting of three COPs, including the general model and three specific CO problems 
%     \item 6.2. Experiment Setting (hyperparameters, details of training, evaluation, and test) 写在appendix里吧
%     \item 6.3. Performance analysis 这个要占半页
% \end{itemize}

%\hp{need to think of a way to compress these tables / visuals.} 

%\hp{\cancel{Baselines}; hyperparamters; \cancel{metrics}; etc.}

With theoretical guarantees on the existence and convergence of NE for ACCES games, we are also interested in how our proposed algorithm CCDO-RL works empirically. To evaluate this, we conduct experiments of CCDO-RL on three distinct ACCES game instances introduced in Section \ref{sub_exp_ins} and analyze the performance of CCDO-RL in Section \ref{sub_train_eval}. Section 6.2.1 aims to empirically demonstrate the convergence (Figures \ref{fig_exploit_20} and \ref{fig_exploit_50}) of the algorithm CCDO-RL over realistic CO problems, and show its consistency with Theorem \ref{CCDOA}. Section 6.2.2 intends to show the average reward (to seen training graphs) as well as the generalizability (to unseen test graphs) of the combinatorial player in real-world ACCES games (shown in Tables \ref{tab_aver}, and \ref{tab_gene}).

\subsection{Three Instances of ACCES Games} \label{sub_exp_ins}
% \hp{This para does not make much sense. Need to follow the framework in the Preliminaries section.}
% For combinatorial optimization problems in real-world applications, situations are more complicated and intractable due to changeable environmental or physical parameters. The form of parameter sets is very crucial because different types have different solvability and computation complexity. Forms of parameter sets mainly contain discrete sets, interval sets \cite{buchheim2018robust} like polyhedral and ellipsoid, probability distributions \cite{carlsson2018wasserstein}, and variable functions \cite{krause2008robust}.

% In reality, these parameters are often impacted by some common factors, such as conditions of weather, transportation, and individual personalities. \cite{kalimeris2019robust} proposed an assumption that real instances (e.g. demands in CVRP, coverages in CSP) 
%Considering affected or attacked COPs, the real instance $\{\theta_{i}\}$ always relied on the estimated value $\{\hat{\theta}_{i}$\} and the variation determined by independent factors $\{g_{i}\}$ and environment/physical parameters/attacker actions $\{\eta\}$. The concrete parameter influence model is stated as follows:

We consider a certain COP which is parameterized with $\{\theta_{i}\}$, where $i$ is the index of nodes (such as a target in security games) -- e.g., such parameters can be interpreted as attack probability of targets.
%coverage radius, customer's demands, or attack probability of targets. 
In real-world applications, we often need to estimate such parameters before solving the COPs. Unfortunately, the estimation $\{\hat{\theta}_{i}\}$ often bears a gap to the true value $\{\theta_{i}\}$, which derives from e.g. environment (aleatoric) uncertainty, model (epistemic) uncertainty, or an attacker trying to manipulate the defender's utility. We use a generic model to formulate this gap:
\begin{equation}\label{linrob}
    \theta_{i} = \hat{\theta}_{i} + y \cdot \tau_{i},
\end{equation}
where $y$ represents the strategy of the nature/attacker, $\tau_{i}$ is the environment factors like weather and transportation conditions, or human subjective factors like the preference of the attacker. 
Such abstraction can represent a wide range of ACCES games, such as facility location covering problems \cite{an2020battery, TIRKOLAEE2020340}, CVRP \cite{vehiclerouting.ch8,dinh2018exact, FLORIO20231081}, security patrolling (OP) \citep{xu2021robust}, and influence maximization problem \cite{kalimeris2019robust}. We describe three instances of ACCES games based on the model (\ref{linrob}).%Based on this model (\ref{linrob}), we focus on three combinatorial optimization problems with attacks or environmental/physical influence.

% \hp{Hard to follow. We should point out what are the two players, what are X, Y, u etc}

\textbf{Adversarial Covering Salesman Problem (ACSP):} In a map of cities, every city $i$ has a coverage $\theta_{i}$. A salesman finds the shortest path such that all cities are visited or covered, with $\theta_{i}$ influenced by physical factors $\tau_i$ and transportation parameters $y$ based on Eq.(\ref{linrob}). The salesman is Player 1 where $X$ consists of the feasible paths of the salesman. Nature is Player 2 with $Y$ = $[0, 1]^K \ni y, K \in \mathbb{N}$. The utility function of Player 1 $u$ is the opposite of the total traveling distance.

\textbf{Adversarial Capacitated Vehicle Routing Problem (ACVRP):} A vehicle with a constrained capacity of goods finds the shortest path under the worst case with the $i_{th}$ customer's demand $\theta_i$ changed by environmental factors $\tau_i$ and weather parameter $y$ on Eq.(\ref{linrob}). The vehicle is Player 1 where $X$ is the set of the feasible path $x$. Nature is Player 2 where $Y$ is $[0, 1]^K \ni y, K \in \mathbb{N}$. The utility function of Player 1  $u$ is the opposite of total delivery distance satisfying all the demands of customers.


\textbf{Patrolling Game (PG):} The patrolling game is described in the introduction.

For all the problem instances, we run our algorithm on two problem sizes: 20 nodes and 50 nodes. The detailed description and problem parameters of the three game instances are in Appendix \ref{app_ex_para_set}.

% Similarly, in the vehicle route problem (VRP), conditions with correlated parameters arouse broad attention from scholars \cite{vehiclerouting.ch8,dinh2018exact,FLORIO20231081}. \cite{dinh2018exact} considered the demand correlation by geographical proximity of nodes, described by some independent random variables in the fractional form. \cite{FLORIO20231081} utilized 'external factors' to stand for unknown covariates affecting all demands and presented a Bayesian model to learn correlations. Further more, about IM problems, \cite{kalimeris2019robust} combined node features and uncertain hyperparameters to fit the influence probability on each edge.

% \subsection{Training CCDO-RL}

% For all the problems, CCDO-RL adopts the REINFORCE algorithm with an attention-based encoder-decoder framework \cite{kool2018attention} (used as an inductive graph representation component) to learn a (generalizable) COP solver for one player (protagonist), and PPO \cite{schulman2017proximal} to train a policy for the other player (adversary) whose strategy space is continuous. CCDO-RL is trained with 50 epochs on a set of 10,000 graphs (with 20 or 50 nodes). The hyperparameters of CCDO-RL are specified in Appendix \ref{app_ex_para_set} (Table \ref{tab_hyper_ccdorl}). Our code is included as supplementary material for ease of reproduction. 
% % \hp{need to specify hyperparas}

\subsection{Performance of CCDO-RL}\label{sub_train_eval}

Two aspects are evaluated for the performance of CCDO-RL, i.e., i) Convergence to NE (Section \ref{sub_per_conver}) exploring whether CCDO-RL can compute the NE, and ii) Protagonist policy's average reward and generalizability (Section \ref{sub_per_rob}). Generalizability refers to the ability of RL models trained on previously seen graphs (problem instances), to perform well on a new set of unseen test graphs. The model’s usability is enhanced by generalizability, rather than focusing solely on the average reward, which is a critical motivation in the literature on RL for COPs \citep{khalil2017learning, kool2018attention}.

For all the problems, CCDO-RL adopts the REINFORCE algorithm with an attention-based encoder-decoder framework \citep{kool2018attention} (used as an inductive graph representation component) to learn a generalizable COP solver for Player 1 (protagonist), and PPO to train a policy for Player 2 (adversary) whose strategy space is continuous. CCDO-RL is trained on a set of 10,000 graphs (with 20 or 50 nodes). The hyperparameters of CCDO-RL are specified in Appendix \ref{app_ex_para_set} (Table \ref{tab_hyper_ccdorl}). Our code is included as supplementary material and will be open-sourced for ease of reproduction. 

% \textbf{Training.} For all the problems, CCDO-RL adopts the REINFORCE algorithm with attention-based encoder-decoder framework \cite{kool2018attention} (used as an inductive graph representation component) to learn a (generalizable) COP solver for one player (protagonist), and PPO \cite{schulman2017proximal} to train a policy for the other player (adversary) whose strategy space is continuous. CCDO-RL is trained with 50 epochs on a set of 10,000 graphs (with 20 or 50 nodes). 

% \hp{We should first present results about convergence as it is mostly aligned with the theory.}

\subsubsection{Convergence to NE} \label{sub_per_conver}

Exploitability is a common metric to describe the closeness to true NE by calculating the sum of performance distances between each new best response and subgame NE, i.e. $\sum_{i=1,2} U(\pi_{i,k}^{br}, \sigma_{-i,k}) - U(\sigma)$ in the general two-player game. Since our game is zero-sum, the calculation is as follows:
\begin{equation*}
   \text{Exploitability}(\sigma) = \max_{\pi_1 \in \Sigma_1} U(\pi_1, \sigma_{2}) - \min_{\pi_2 \in \Sigma_2} U(\sigma_1, \pi_2).
\end{equation*}
From Figure \ref{fig_exploit_20}, we can see that CCDO-RL can converge to approximate NE in 25 iterations or less (in the PG setting), reaching 0.05 in ACSP, 0.10 in ACVRP, and 0.03 in PG with 20 nodes. Similar results are observed in problems with 50 nodes (see Figure \ref{fig_exploit_50} in Appendix \ref{app_exp}). These results validate the effectiveness of CCDO-RL in finding the NE for various types of games.

%Similarly, the exploitability of three COPs in 50 nodes is provided in the appendix \ref{app_exp}.
\vspace{-\baselineskip}
\begin{figure}[htbp]
	\centering
    \subfigure[ACSP20]{
    \label{csp20_nashconv}
    \includegraphics[scale=0.20]{Figures/nashconv_log_csp20_sm_7.eps}
    }
    \subfigure[ACVRP20]{
    \label{cvrp20_nashconv}%文中引用该图片代号
    \includegraphics[scale=0.20]{Figures/nashconv_log_svrp20_sm_7.eps}
    }
    \subfigure[PG20]{
    \label{opsa20_nashconv}
    \includegraphics[scale=0.20]{Figures/nashconv_log_pg20_sm_7.eps}
    }
    \caption{Exploitability curve of CCDO-RL on three games of 20 nodes}
    \label{fig_exploit_20}
\end{figure}
\vspace{-\baselineskip}
\subsubsection{Average reward and Generalizability of Combinatorial player} \label{sub_per_rob}
% \subsubsection{Robustness and Generalizability of Protagonist Policy} \label{sub_per_rob}
%\hp{CCDO-RL being better in these following metrics is only kind of a by-product.}

% \textbf{Evaluation.} The learned policies are then tested on 200 graphs, where 100 of them are randomly selected from the 10,000 training graphs, and the other 100 are unseen graphs. 
% We use two metrics to evaluate the performance of different policies for the protagonist player: \textbf{Average proportional loss} $R-$ describes the policy overfitting degree \citep{lanctot2017unified}; \textbf{Reward} evaluates the performance of the protagonist with the adversary under three COPs.  
% \begin{eqnarray}
%         &R- = (\hat{D} - \hat{O}) / \hat{D}.
% \end{eqnarray}
% in which $\hat{D}$ is the mean value of the diagonals and $\hat{O}$ is the mean value of the off-diagonals in the payoff matrix provided in the Appendix \ref{app_exp}.

% Because the protagonist policy is trained against a powerful adversary under our ACCES game setting, the obtained policy is naturally robust against adversarial perturbations. This subsection sheds a bit of light on this perspective and quantifies the extent of robustness of CCDO-RL as well as the ability of RL to generalize to unseen test graphs.

\textbf{Evaluation.} The learned policies are tested on 200 graphs, with 100 being randomly selected from the 10,000 training graphs (to show the average reward), and the other 100 being unseen graphs (to test policy generalization). We evaluate the performance of the protagonist with the adversary under three COPs. For each COP, the performance is considered both on the 20-node and 50-node map.
% We use two metrics to evaluate the performance of different policies for the protagonist player: \textbf{Average proportional loss} $R-$ describes the policy overfitting degree \citep{lanctot2017unified}; \textbf{Reward} evaluates the performance of the protagonist with the adversary under three COPs.

\textbf{Baselines.} There are heuristic algorithms for each game instance (Heuristic in Table \ref{tab_aver} and \ref{tab_gene}) and a single-player RL algorithm. For ACVRP, we adopt the Tabu Search algorithm (Tabu) \citep{li2020improved} as the heuristic algorithm, which is widely applied in the routing problem. For ACSP, the common benchmark local search algorithm, LS2 \citep{golden2012generalized}, is used. For PG, we choose the greedy algorithm as the baseline. The "RL against Stoc" algorithm in Tables \ref{tab_aver} and \ref{tab_gene} is identical to the protagonist model in CCDO-RL but trained in environments with stochastic adversarial perturbations.

% \textbf{Baselines.} There are a heuristic algorithms for each game instance {\color{red} (Heuristic mentioned in the Table \ref{tab_aver} and \ref{tab_gene})} and a single-player RL algorithm. For ACVRP, we adopt the Clarke-Wright (CW) algorithm \citep{pichpibul2013heuristic} and the Tabu Search algorithm (Tabu) \citep{li2020improved} as heuristics, which are applied widely in the routing problem. For ACSP, two common benchmark local search algorithms, LS1 and LS2 \citep{golden2012generalized}, are used. For PG, we choose a local search algorithm \citep{vansteenwegen2009iterated} and the greedy algorithm as the heuristic baselines. {\color{red} The "RL  against Stoc" algorithm referred to Tables \ref{tab_aver} and \ref{tab_gene}} is identical to the protagonist model in CCDO-RL {\color{red} but trained on environments with stochastic adversarial perturbations.} 

\textbf{Average Reward.}  As illustrated in Table \ref{tab_aver}, our algorithm achieves a better average reward than baselines (10.08\% improvement on average of all settings against two baselines), regardless of CO instance or problem size, when confronting the adversary trained by CCDO-RL. In the setting of CSP-20 nodes, the average reward is improved by 46.98\% compared to the heuristic and by 7.14\% compared with the RL against Stoc. For the 50-node setting, the improvements are 45.91\% and 5.28\% respectively. Similarly, the improvements in contrast to Heuristic and RL against Stoc are as follows: 1.72\% and 3.01\%  for CVRP-20 nodes, 0.75\% and 4.46\% for CVRP-50 nodes, 4.17\% and 1.48\% for PG-20 nodes, and 10.60\% and 4.38\% for PG-50 nodes.

\textbf{Generalizability.} From Table \ref{tab_gene}, CCDO-RL continues to achieve a better average reward when facing the adversary, demonstrating that the learned RL policies generalize well to unseen graphs. Even though the non-RL baselines do have access to the graph structures and other problem information of the unseen problem instances, CCDO-RL can obtain comparable performances without re-training on the new problem instances. The improvements versus Heuristic and RL against Stoc are 46.61\% and 7.02\% for CSP-20 nodes, 42.24\% and 3.94\% for CSP-50 nodes, 1.12\% and 1.56\% for CVRP-20 nodes, 0.90\% and 5.05\% for CVRP-50 nodes, 5.35\% and 2.40\% for PG-20 nodes, and 12.17\% and 10.33\% for PG-50 nodes. Even when confronting the stochastic adversary, CCDO shows superior generalizability compared to two baselines across three COPs, with average improvements of 6.31\%, 3.42\%, and 3.95\% respectively. Detailed results are provided in Appendix \ref{app_exp} (Tables \ref{tab_csp_full_20} - \ref{tab_op_full_50}). 
% The model’s usability is enhanced by the ability to generalize rather than focusing solely on the average reward, which is a critical motivation of the RL for combinatorial optimization literature \citep{khalil2017learning, kool2018attention}.  

\begin{remark}
    In CO problems (or more broadly, operations research and economics), it is known that achieving solution quality improvements against strong baselines (e.g., the RL methods trained with a stochastic adversary) is very challenging, and the margins are usually small \citep{kool2018attention}, sometimes even less than 1\%. However, these “tiny” marginal improvements in profits keep small business owners in the real world alive. Last, the improvement depends a lot on the problem settings, and we show that sometimes the improvement can be much more significant.
\end{remark}
\vspace{-\baselineskip}
% \textbf{Performance analysis.} The robustness results of CCDO-RL for ACSP are shown in Table \ref{tab_csp}. We have the following observations: 1) On both of the 100 seen/unseen graphs, single-player RL performs better than heuristic algorithms no matter whether attacked or not. (2) When confronting the adversary trained by CCDO-RL, CCDO-RL exceeds RL by 0.25 and 0.24 on the training set, and by 0.25 and 0.18 on the test set, respectively under the 20-node and 50-node graphs. This demonstrates the robustness of CCDO-RL. 3) Compared to the performance of the training set with that of the test set, we can see that RL and CCDO-RL both maintain a certain degree of generalization. Similar results for ACVRP (Table \ref{tab_cvrp}) and SPG (Table \ref{tab_op}) are provided in Appendix \ref{app_exp}. 

\begin{table}[ht]
  \caption{Average reward against CCDO-RL's adversary (on seen graphs)}
  \vspace{\baselineskip}
  \label{tab_aver}
  \centering
  \small
  \begin{tabular}{lllllll}
    \toprule
    \multirow{2}{*}{method} & \multicolumn{2}{c}{ACSP (Mean$\pm$Std)} & \multicolumn{2}{c}{ACVRP (Mean$\pm$Std)} & \multicolumn{2}{c}{PG (Mean$\pm$Std)} \\
    \cmidrule(r){2-3} \cmidrule{4-5} \cmidrule(r){6-7}
                            & 20 nodes & 50 nodes & 20 nodes & 50 nodes & 20 nodes & 50 nodes\\
    \midrule
    Heuristic & 6.13$\pm$1.20 & 7.55$\pm$1.42 & 7.65$\pm$1.23  & 13.38$\pm$1.70 & 2.64$\pm$1.03 & 4.53$\pm$1.84   \\
    RL against Stoc    & 3.50$\pm$0.47  & 4.55$\pm$0.62  & 7.55$\pm$1.16  & 13.90$\pm$1.63 & 2.71$\pm$0.90 & 4.80$\pm$2.18   \\
    CCDO-RL   & $\pmb{3.25}$$\pm$0.42 & $\pmb{4.31}$$\pm$0.51  & $\pmb{7.42}$$\pm$1.21  & $\pmb{13.28}$$\pm$1.52 &  $\pmb{2.75}$$\pm$0.87 & $\pmb{5.01}$$\pm$1.91  \\
    \bottomrule
  \end{tabular}
\end{table}
\vspace{-\baselineskip}

\begin{table}[htp]
  \caption{Generalizability against CCDO-RL's adversary (on unseen graphs)}
  \vspace{\baselineskip}
  \label{tab_gene}
  \centering
  \small
  \begin{threeparttable}
  \begin{tabular}{lllllll}
    \toprule
    \multirow{2}{*}{method} & \multicolumn{2}{c}{ACSP (Mean$\pm$Std)} & \multicolumn{2}{c}{ACVRP (Mean$\pm$Std)} & \multicolumn{2}{c}{PG (Mean$\pm$Std)} \\
    \cmidrule(r){2-3} \cmidrule{4-5} \cmidrule(r){6-7}
                            & 20 nodes & 50 nodes & 20 nodes & 50 nodes & 20 nodes & 50 nodes\\
    \midrule
    Heuristic & 6.20$\pm$1.33 & 7.60$\pm$1.37   & 7.64$\pm$1.30  & 13.27$\pm$1.87 & 2.43$\pm$0.98 & 4.19$\pm$1.69    \\
    RL against Stoc  & 3.56$\pm$0.37  & 4.57$\pm$0.58  & 7.67$\pm$1.30  & 13.85$\pm$1.53 &  2.50$\pm$0.95 & 4.26$\pm$2.17 \\
    CCDO-RL   & $\pmb{3.31}$$\pm$0.35 & $\pmb{4.39}$$\pm$0.52  & $\pmb{7.55}$$\pm$1.28  & $\pmb{13.15}$$\pm$1.59 & $\pmb{2.56}$$\pm$0.92 & $\pmb{4.70}$$\pm$1.94\\

    \bottomrule
  \end{tabular}
  \begin{tablenotes}
      \footnotesize
      \item[1] For the average reward of ACSP and ACVRP, smaller is better while for that of PG larger is better.
  \end{tablenotes}
  \end{threeparttable}
\end{table}
\vspace{-\baselineskip}
% two heuristics and one RL
% \begin{table}[ht]
%   \caption{{\color{red} Average reward of CCDO-RL (on seen graphs). For the value of CSP and CVRP, larger is better while for that of PG smaller is better.}}
%   \label{tab_aver}
%   \centering
%   \small
%   \begin{tabular}{lllllll}
%     \toprule
%     \multirow{2}{*}{method} & \multicolumn{2}{c}{CSP (Mean$\pm$Std)} & \multicolumn{2}{c}{CVRP (Mean$\pm$Std)} & \multicolumn{2}{c}{PG (Mean$\pm$Std)} \\
%     \cmidrule(r){2-3} \cmidrule{4-5} \cmidrule(r){6-7}
%                             & 20 nodes & 50 nodes & 20 nodes & 50 nodes & 20 nodes & 50 nodes\\
%     \midrule
%     Baseline 1 & 4.52$\pm$0.71  & 5.98$\pm$0.94 & 7.64$\pm$1.56  & 13.49$\pm$2.10 & 2.71$\pm$1.10 & 1.82$\pm$1.40   \\
%     Baseline 2 & 6.13$\pm$1.20 & 7.55$\pm$1.42   & 7.65$\pm$1.23  & 13.38$\pm$1.70 & 2.64$\pm$1.03 & 1.47$\pm$0.99  \\
%     RL {\color{red}against Stoc}    & 3.50$\pm$0.47  & 4.55$\pm$0.62  & 7.55$\pm$1.16  & 13.90$\pm$1.63 & 2.71$\pm$0.90 & 1.54$\pm$1.03   \\
%     CCDO-RL   & $\pmb{3.25}$$\pm$0.42 & $\pmb{4.31}$$\pm$0.51  & $\pmb{7.42}$$\pm$1.21  & $\pmb{13.28}$$\pm$1.52 &  $\pmb{2.75}$$\pm$0.87 & $\pmb{1.87}$$\pm$1.22  \\
%     \bottomrule
%   \end{tabular}
% \end{table}


% \begin{table}[htp]
%   \caption{{\color{red}Generalizability of CCDO-RL (on unseen graphs)}}
%   \label{tab_gene}
%   \centering
%   \small
%   \begin{threeparttable}
%   \begin{tabular}{lllllll}
%     \toprule
%     \multirow{2}{*}{method} & \multicolumn{2}{c}{CSP (Mean$\pm$Std)} & \multicolumn{2}{c}{CVRP (Mean$\pm$Std)} & \multicolumn{2}{c}{PG (Mean$\pm$Std)} \\
%     \cmidrule(r){2-3} \cmidrule{4-5} \cmidrule(r){6-7}
%                             & 20 nodes & 50 nodes & 20 nodes & 50 nodes & 20 nodes & 50 nodes\\
%     \midrule
%     Baseline 1 & 4.53$\pm$0.79  & 5.95$\pm$0.96 & 7.55$\pm$1.39  & 13.35$\pm$2.04 & 2.52$\pm$1.08 & $\pmb{1.86}$$\pm$1.44  \\
%     Baseline 2 & 6.20$\pm$1.33 & 7.60$\pm$1.37   & 7.64$\pm$1.3  & 13.27$\pm$1.87 & 2.43$\pm$0.98 & 1.52$\pm$1.20    \\
%     RL {\color{red}against Stoc}  & 3.56$\pm$0.37  & 4.57$\pm$0.58  & 7.67$\pm$1.30  & 13.85$\pm$1.53 &  2.50$\pm$0.95 & 1.03$\pm$5.05 \\
%     CCDO-RL   & $\pmb{3.31}$$\pm$0.35 & $\pmb{4.39}$$\pm$0.52  & $\pmb{7.55}$$\pm$1.28  & $\pmb{13.15}$$\pm$1.59 & $\pmb{2.56}$$\pm$0.92 & 1.35$\pm$5.09\\

%     \bottomrule
%   \end{tabular}
%   \begin{tablenotes}
%       \footnotesize
%       \item[1] For the value of CSP and CVRP, larger is better while for that of PG smaller is better.
%   \end{tablenotes}
%   \end{threeparttable}
% \end{table}

\begin{table}[ht!]
\centering
\caption{\textbf{Super Resolution Performance Results.} Our proposed WGAN EEG Spatial Upsampling method significantly outperforms a baseline of Bicubic Interpolation commonly used in EEG upsampling pipelines.}
\label{tab:results}
\resizebox{0.8\linewidth}{!}{%
\begin{tabular}{@{}cccccc@{}}
\toprule
\multirow{2}{*}{\textbf{Dataset}} & \multirow{2}{*}{\textbf{Scale}} & \multicolumn{2}{c}{\textbf{Bicubic}} & \multicolumn{2}{c}{\textbf{WGAN}} \\ \cmidrule(l){3-6} 
                      &   & \textbf{MSE} & \textbf{MAE} & \textbf{MSE}    & \textbf{MAE}   \\
\toprule
\multirow{2}{*}{Val}  & 2 & 3.71E7       & 3.89E3       & \textbf{2.01E3} & \textbf{24.38} \\
                      & 4 & 7.23E7       & 6.42E3       & \textbf{8.53E3} & \textbf{63.83} \\
\midrule
\multirow{2}{*}{Test} & 2 & 3.75E7       & 3.91E3       & \textbf{2.06E3} & \textbf{24.66} \\
                      & 4 & 7.30E7       & 6.45E3       & \textbf{8.68E3} & \textbf{64.39} \\
\bottomrule
\end{tabular}%
}
\end{table}
This work identifies signal collapse as a critical bottleneck in one-shot neural network pruning. Performance loss in pruned networks is due to \textbf{signal collapse} in addition to the removal of critical parameters. We propose \textbf{REFLOW} (\textbf{Re}storing \textbf{F}low of \textbf{Low}-variance signals), a simple yet effective method that mitigates signal collapse without computationally expensive weight updates. By focusing on signal preservation, REFLOW highlights the importance of mitigating signal collapse in sparse networks and enables magnitude pruning to match or surpass state-of-the-art one-shot pruning methods such as CHITA, CBS, and WF.

REFLOW consistently achieves state-of-the-art accuracy across diverse architectures, restoring ResNeXt-101 from under 4.1\% to 78.9\% top-1 accuracy at 80\% sparsity on ImageNet. Its lightweight design makes it a practical solution for both research and deployment, delivering high-quality sparse models without the overhead of traditional approaches. These findings challenge the traditional emphasis on weight selection strategies and underscore the critical role of signal propagation for achieving high-quality sparse networks in the context of one-shot pruning.




\section{Related Work} \label{sec:related}

% \textbf{Adversarial Attack}
\textbf{Attacks on SLAM.} 
%With the rise of machine learning, 
The robustness of computer vision systems is being actively investigated. With the emergence of adversarial images in the digital domain by adding optimized noise directly to images~\cite{szegedy2013intriguing,carlini2017towards}, researchers find that such attacks also exist physically in the real world \cite{eykholt2018robust,song2018physical,zhao2019seeing}. To fill the gap between attacks in the digital and physical worlds, recent studies have demonstrated that attacks on real-world computer vision systems are practical \cite{eykholt2018robust,li2019adversarial,man2020ghostimage,sharif2016accessorize,zhao2019seeing,zhou2018invisible}. However, attacks on traditional computer vision methods such as SLAM are relatively less explored. \cite{yoshida2022adversarial} proposes an attack against the scan matching algorithm in LiDAR-based SLAM, while most SLAMs in AR/VR devices rely on different sensors like RGB/depth cameras and IMUs. \cite{ikram2022perceptual} and \cite{chen2024adversary} mislead visual SLAM by poisoning the images with special patterns, and \cite{wang2021can} causes the camera to fail using infrared light. In our work, we demonstrate attacks on Visual-Inertial SLAM (VI-SLAM) by perturbing the IMU readings, rather than cameras, and showing its impact on XR user experience. 

\textbf{Acoustic Injection Attacks.} Among various physical attacks, acoustic injection attacks are attractive due to their low cost. Son~\etal~\cite{son2015rocking} were the first to introduce acoustic attacks on MEMS gyroscopes, demonstrating how these attacks could lead to sensor denial-of-service and result in drone crashes. WALNUT~\cite{trippel2017walnut} expanded on this by developing output biasing and control attacks that enable precise manipulation of MEMS accelerometer outputs using modulated sound waves. Wang et al.~\cite{wang2017sonic} demonstrated a sonic gun, showcasing the vulnerability of various smart devices (\eg drones and self-balancing vehicles) to acoustic attacks. Tu et al. \cite{tu2018injected} designed side-swing and switching attacks to alter the outputs of MEMS gyroscopes and accelerometers. Furthermore, Ji et al. \cite{ji2021poltergeist} fool the object detectors by applying acoustic attack to the image stabilizers commonly used in modern cameras. However, none of the existing works study the relationship between the acoustic injections and SLAM outputs on recent XR devices. 

% \zijian{Do we need one session about security in AR/VR?}
% \yicheng{TODO}
%\jiasi{cite the AIVR paper (UMass Amherst?) paper is we have not already. They add IMU perturbation but w/o SLAM, iirc} \yicheng{Cited}

\textbf{XR Security and Privacy.} 
%Security and privacy concerns in XR systems have gained significant attention. 
For single-user XR systems, researchers have demonstrated various side-channel attacks to extract sensitive information (\eg keystrokes) through video feeds~\cite{ling2019know}, head movements~\cite{nair2023unique, slocum2023going}, architectural hints~\cite{zhang2023its,shang2020arspy}, power usage~\cite{li2024dangers}, and EM side-channel leakages~\cite{al2021vr}. In multi-user XR systems, Su et al.~\cite{su2024remote} use avatar motion data to infer keystrokes in shared VR environments. Slocum et al.~\cite{slocum2024doesn} reveal vulnerabilities in the shared state frameworks of multi-user AR. Similarly, Lebeck et al.~\cite{lebeck2017securing} highlight risks like deceptive virtual objects and emphasize access control for managing shared physical and virtual spaces. Ruth et al.~\cite{ruth2019secure} further propose a secure multi-user AR framework focusing on content sharing and permissions.
Chandio et al.~\cite{chandio2024stealthy} %introduced a multi-modal spatiotemporal attack that 
simultaneously manipulated visual and inertial sensors to disrupt XR pose estimation. However, their study evaluated the attack using offline datasets and assumed the attacker's capability to manipulate IMU data streams through acoustic means, without real experiments. Ours is the first to demonstrate acoustic injection attacks on recent XR devices, like the Hololens 2, in the real world.
 


\section*{Conclusion}
This paper aims to enhance our understanding of the computational complexity of computing various Shapley value variants. We found that for various ML models --- including decision trees, regression tree ensembles, weighted automata, and linear regression --- both local and global interventional and baseline SHAP can be computed in polynomial time under HMM modeled distributions. This extends popular algorithms, such as TreeSHAP, beyond their empirical distributional scope. We also establish strict complexity gaps between the various SHAP variants (baseline, interventional, and conditional) and prove the intractability of computing SHAP for tree ensembles and neural networks in simplified scenarios. Overall, we present SHAP as a versatile framework whose complexity depends on four key factors: \begin{inparaenum}[(i)] \item model type, \item SHAP variant, \item distribution modeling approach, \item and local vs. global explanations\end{inparaenum}. We believe this perspective provides deeper insight into the computational complexity of SHAP, paving the way for future work.




%We believe that our framework provides a more intricate understanding of SHAP computation complexity across different models, distributions, and variants, paving the way for further research.

Our work opens promising directions for future research. First, expanding our computational analysis to other SHAP-related metrics, such as asymmetric SHAP~\citep{frye20} and SAGE~\citep{covert2020understanding}, would be valuable. Additionally, we aim to explore more expressive distribution classes and relaxed assumptions beyond those in Section \ref{sec:tractable} while maintaining tractable SHAP computation. Finally, when exact computation is intractable (Section \ref{sec:intractable}), investigating the approximability of SHAP metrics through approximation and parameterized complexity theory~\citep{downey2012parameterized} is an important direction.

%Our work opens several promising avenues for future research on the computational properties of explainable AI methods, with a particular focus on SHAP. First, it would be interesting to broaden the computational analysis conducted in this work to include other popular SHAP-related metrics in the literature, such as asymmetric SHAP \cite{frye20} and SAGE \cite{covert2020understanding}. Also, in the future, we aim to explore more expressive distribution classes and relaxed distributional assumptions—extending beyond those examined in Section \ref{sec:tractable} —that still yield tractable SHAP computation. Finally, when exact computation proves intractable (Section \ref{sec:intractable}), it is worthwhile to theoretically investigate the question of the approximability of computing the SHAP metrics across various configurations, through the lens of approximation and parametrized complexity theory \cite{arora2009computational}.

%This paper aims to deepen our understanding of the computational complexity involved in obtaining different Shapley value variants. We found that for a variety of ML models, including decision trees, tree ensembles for regression, weighted automata, and linear regression models — computing both local and global interventional and baseline SHAP can be done in polynomial time when distributions are modeled by HMMs. This extends the distributional scope of popular algorithms like TreeSHAP, which is limited to empirical distributions. Additionally, we demonstrate a strict complexity gap between SHAP variants, showing that interventional and baseline SHAP can be strictly easier to compute than conditional SHAP. Despite these positive results, we uncovered intractability for various SHAP variants in neural networks and tree ensembles. Finally, we provided generalized complexity relations across SHAP variants. We believe that our framework offers a deeper understanding of the complexity involved in computing SHAP across various variants, models, distributions, as well as in both local and global computations, laying the groundwork for future research.

% \newpage
% This research focuses on analyzing the internal representations of numerical values in Large Language Models and does not involve human subjects, sensitive data, or direct societal impact.



\newpage
\bibliographystyle{ACM-Reference-Format}
\bibliography{base}
\begin{appendices}
\renewcommand{\thetable}{A\arabic{table}}
\counterwithin{table}{section}
\section{Appendices}
The full list of examined primary publications is enumerated in Table\ref{tab:flep}.

\begin{table}[ht!]
\caption{Full List Of Examined Publications
\label{tab:flep}}
\begin{adjustbox}{width=\linewidth, center}
\centering
\begin{tabular}{llll}
\toprule
\textbf{Year} & \textbf{Venue Type} & \textbf{Venue}               & \textbf{Title}              \\

\textsc{Conference} & & &\\   
\midrule
2023          & Conference          & AINA                         & A Vulnerability Detection Method for SDN with Optimized Fuzzing\citep{10.1007/978-3-031-28451-9_46}                                                                            \\
2022          & Conference          & ICAIS                        & Test Traffic Control Based on REST API for Software-Defined Networking\citep{10.1007/978-3-031-06788-4_40}                                                                     \\
2021          & Conference          & SOSR                         & Tardis: A Fault-Tolerant Design for Network Control Planes\citep{10.1145/3482898.3483355}                                                                                      \\
2021          & Conference          & NISS                         & SDN Control Plane Security: Attacks and Mitigation Techniques\citep{10.1145/3454127.3456612}                                                                                   \\
2021          & Conference          & DSN                          & A Comprehensive Study of Bugs in Software Defined Networks\citep{9505089}                                                                                                      \\
2021          & Conference          & NTMS                         & SDN Security through System Call Learning\citep{9432640}                                                                                                                       \\
2020          & Conference          & CIC                          & \begin{tabular}[c]{@{}l@{}}ParaSDN: An Access Control Model for SDN Applications based on Parameterized Roles and \\ Permissions\citep{9319021}\end{tabular}                   \\
2020          & Conference          & SP                           & Unexpected Data Dependency Creation and Chaining: A New Attack to SDN\citep{9152642}                                                                                           \\
2020          & Conference          & CSPS                         & Deep Learning Based Detection Method for SDN Malicious Applications\citep{10.1007/978-981-13-6508-9_13}                                                                        \\
2020          & Conference          & INFOCOM                      & AudiSDN: Automated Detection of Network Policy Inconsistencies in Software-Defined Networks\citep{9155378}                                                                     \\
2020          & Conference          & WISA                         & FSF: Code Coverage-Driven Fuzzing for Software-Defined Networking\citep{10.1007/978-3-030-39303-8_4}                                                                           \\
2020          & Conference          & CAV                          & Towards Model Checking Real-World Software-Defined Networks\citep{10.1007/978-3-030-53291-8_8}                                                                                 \\
2019          & Conference          & IM                           & Mining Software Repositories for Predictive Modelling of Defects in SDN Controller\citep{8717837}                                                                              \\
2019          & Conference          & IM                           & Thinking inside the Box: Differential Fault Localization for SDN Control Plane\citep{8717815}                                                                                  \\
2018          & Conference          & NFV-SDN                      & RE-CHECKER: Towards Secure RESTful Service in Software-Defined Networking\citep{8725649}                                                                                       \\
2018          & Conference          & ICNP                         & INDAGO: A New Framework For Detecting Malicious SDN Applications\citep{8526819}                                                                                                \\
% 2018          & Chapter             &                              & Security Analysis of FloodLight, ZeroSDN, Beacon and POX SDN Controllers\citep{Ilyas2018}                                                                                      \\
2018          & Conference          & CCS                          & AIM-SDN: Attacking Information Mismanagement in SDN-datastores\citep{10.1145/3243734.3243799}                                                                                  \\
2018          & Conference          & ICC                          & SENAD: Securing Network Application Deployment in Software Defined Networks\citep{8422405}                                                                                     \\
% 2018          & Chapter             &                              & Security Analysis of SDN Routing Applications\citep{Sagare2018}                                                                                                                \\
2017          & Conference          & NSDI                         & Automated Bug Removal for Software-Defined Networks\citep{10.5555/3154630.3154688}                                                                                             \\
2017          & Conference          & MASCOTS                      & Testing Black-Box SDN Applications with Formal Behavior Models\citep{8107437}                                                                                                  \\
2017          & Conference          & ICC                          & Controller DAC: Securing SDN controller with dynamic access control\citep{7997249}                                                                                             \\
2017          & Conference          & ISSREW                       & Analytics-Enhanced Automated Code Verification for Dependability of Software-Defined Networks\citep{8109275}                                                                   \\
2017          & Conference          & SIGCOMM                      & BigBug: Practical Concurrency Analysis for SDN\citep{10.1145/3050220.3050230}                                                                                                  \\
2017          & Conference          & ICCNT                        & Requirement analysis for abstracting security in software-defined network\citep{8204161}                                                                                       \\
2017          & Conference          & CNSM                         & An empirical study of software reliability in SDN controllers\citep{8256002}                                                                                                   \\
2017          & Conference          & BWCCA                        & A Comprehensive Security Analysis Checksheet for OpenFlow Networks\citep{10.1007/978-3-319-49106-6_22}                                                                         \\
2017          & Conference          & RAID                         & BEADS: Automated Attack Discovery in OpenFlow-Based SDN Systems\citep{10.1007/978-3-319-66332-6_14}                                                                            \\
2017          & Conference          & MIWAI                        & DREAD-R: Severity Assessment of ONOS SDN Controller\citep{10.1007/978-3-319-69456-6_27}                                                                                        \\
2016          & Conference          & CODASPY                      & SHIELD: An Automated Framework for Static Analysis of SDN Applications\citep{10.1145/2876019.2876026}                                                                          \\
2016          & Conference          & CODASPY                      & The Smaller, the Shrewder: A Simple Malicious Application Can Kill an Entire SDN Environment\citep{10.1145/2876019.2876024}                                                    \\
2016          & Conference          & ISSREW                       & Programming the Network: Application Software Faults in Software-Defined Networks\citep{7789391}                                                                               \\
2016          & Conference          & ICSP                         & BuDDI: Bug detection, debugging, and isolation middlebox for software-defined network controllers\citep{7818438}                                                               \\
2016          & Conference          & ICIEV                        & SDN testing and debugging tools: A survey\citep{7760078}                                                                                                                       \\
2015          & Conference          & ICUFN                        & Secure your Northbound SDN API\citep{7182679}                                                                                                                                  \\
2015          & Conference          & TSA                          & Model-Based Testing of SDN Firewalls: A Case Study\citep{7335947}                                                                                                              \\
2015          & Conference          & SIGCOMM                          & Automated Network Repair with Meta Provenance\citep{10.1145/2834050.2834112}                                                                                               \\
2015          & Conference          & GPCE                         & Safer SDN programming through Arbiter\citep{10.1145/2936314.2814218}                                                                                                           \\
2015          & Conference          & NetSoft                      & Design and deployment of secure, robust, and resilient SDN controllers\citep{7258233}                                                                                          \\
2014          & Conference          & CONEXT                       & Controller-agnostic SDN Debugging\citep{10.1145/2674005.2674993}                                                                                                               \\
2014          & Conference          & PLDI                         & VeriCon: towards verifying controller programs in software-defined networks\citep{10.1145/2666356.2594317}                                                                     \\
2014          & Conference          & NSDI                         & Tierless programming and reasoning for software-defined networks\citep{10.5555/2616448.2616496}                                                                                \\
2014          & Conference          & CONEXT                       & Model Based Black-Box Testing of SDN Applications\citep{10.1145/2680821.2680828}                                                                                               \\
2014          & Conference          & SIGCOMM                      & Troubleshooting blackbox SDN control software with minimal causal sequences\citep{10.1145/2740070.2626304}                                                                     \\
2013          & Conference          & PLDI                         & Machine-verified network controllers\citep{10.1145/2499370.2462178}                                                                                                            \\
2012          & Conference          & NSDI                         & A NICE Way to Test OpenFlow Applications\citep{180591}                                                                                                                         \\
              &                     &                              &                                                                                                                                                                                \\
\textsc{Journal} & & &\\ \hline
2023          & Journal             & WWW                          & DACAS: integration of attribute-based access control for northbound interface security in SDN\citep{Liu2023}                                                                   \\
2022          & Journal             & ToN                          & A Framework for Policy Inconsistency Detection in Software-Defined Networks\citep{9681706}                                                                                     \\
2021          & Journal             & CSUR                         & Application Threats to Exploit Northbound Interface Vulnerabilities in Software Defined Networks\citep{10.1145/3453648}                                                        \\
2021          & Journal             & JPDC                         & \begin{tabular}[c]{@{}l@{}}SEAPP: A secure application management framework based on REST API access control in SDN-enabled \\ cloud environment\citep{HU2021108}\end{tabular} \\
2020          & Journal             & IJCNA                        & Security in SDN: A comprehensive survey\citep{CORREACHICA2020102595}                                                                                                           \\
2019          & Journal             & COMST                        & Fault Management in Software-Defined Networking: A Survey\citep{8456508}                                                                                                       \\
2019          & Journal             & COMST                        & A Survey on Network Verification and Testing With Formal Methods: Approaches and Challenges\citep{8453007}                                                                     \\
2019          & Journal             & JAMT                         & Modeling and Verifying Basic Modules of Floodlight\citep{Xiang2019}                                                                                                            \\
2018          & Journal             & TNSM                         & Assessing the Maturity of SDN Controllers With Software Reliability Growth Models\citep{8386840}                                                                               \\
2018          & Journal             & JSAC                         & MORPH: An Adaptive Framework for Efficient and Byzantine Fault-Tolerant SDN Control Plane\citep{8490892}                                                                       \\
2017          & Journal             & COMST                        & A Survey on Fault Management in Software-Defined Networks\citep{7959044}                                                                                                       \\
2016          & Journal             & IJCNA                        & Secure and dependable software defined networks\citep{AKHUNZADA2016199}                                                                                                        \\
2016          & Journal             & MONET                        & Security in Software-Defined Networking: Threats and Countermeasures\citep{Shu2016}                                                                                            \\

\bottomrule

\end{tabular}
\end{adjustbox}
\end{table}
\end{appendices}


\end{document}