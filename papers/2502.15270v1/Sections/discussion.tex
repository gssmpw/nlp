\section{Discussion}
\label{sec:discussion}

\subsection{Mitigation}
To counteract the privacy risks posed by additional system properties and settings containing device non-resettable identifiers, we propose the following mitigation suggestions:

\noindent \textbf{More Explicit System Settings Access Control Methods.}
While Android has implemented relatively clear and comprehensive access control approaches for system properties through SELinux, the access control methods for system settings are much more ambiguous in comparison.
In the official Android documentation, there is far less content about system settings compared to system properties. 
We could not even find clear explanations for access control of system settings in the documentation, and the only related information we found was in the comments of the AOSP source code.
Therefore, we suggest that Google consider implementing additional access control mechanisms for system settings and tightening policies for newly added settings, so they are not accessible to third-party apps by default.

\noindent \textbf{Transparency of Covert Sensitive System Properties and Settings.}
While Google provides access control channels for different supply-chain actors (e.g., vendors and ODMs) in system customization when setting SELinux rules~\cite{vendorprop}, the number of actors involved in system customization far exceed these.
This results in access control enforcers may be unaware of the introduction of certain system properties or settings, such as the two sensitive system settings introduced by Baidu SDK.
We recommend that all actors involved in system customization provide detailed reports to access control enforcers when introducing sensitive system properties or settings. 
This would allow for comprehensive and effective access control to safeguard sensitive non-resettable device identifiers.

\noindent \textbf{Conducting Thorough Security Testing before Release.}
Upon verification, the ROMs of 1,516 devices in our dataset were confirmed to be supported by Google Play~\cite{playsupported}, indicating they passed Google's tests and should have undergone further testing by the device vendors before commercial deployment.
However, our findings show that testing for sensitive system properties and settings is inadequate.
We propose a simple yet effective testing method based on the characteristics of non-resettable identifiers to help reduce cases where sensitive properties or settings lack proper access control. 
This method involves two steps: First, after initializing the custom system, thoroughly use the device to ensure all sensitive properties and settings are initialized, then record their values. 
Second, factory reset the device and repeat the process, and identify properties and settings whose values remain unchanged.
Due to the nature of device non-resettable identifiers, which remain unchanged after a factory reset, this method can identify potential sensitive system properties and settings, and it can also handle cases where device identifiers are stored in encrypted form.



\subsection{Limitations}
\label{sec:limitations}

This paper has several limitations. 
First, we identified the system properties and settings present in custom Android systems by locating their usage, this approach, while addressing feasibility, also introduced constraints. 
Our approach does not allow us to accurately determine whether a system property or setting exists on the device, nor can it identify the specific circumstances under which they would be present.
Despite we have employed all possible methods to eliminate potential false positives, it is undeniable that some of them still persist.
Furthermore, our reliance on static analysis combined with the extensive code base of the custom system precludes us from obtaining the ground truth. 
This limitation greatly hinders our verification of whether all access behaviors of system properties and settings have been captured, and to ascertain if any sensitive ones have been missed.
Additionally, we only considered code written in Java language and did not take into account the access behaviors of system properties and settings in other languages, such as native code, which may have resulted in false negatives in our research.
Secondly, due to limitations in devices and financial resources, we could only use remote real-device testing services to conduct dynamic testing and validate our results. 
As mentioned in~\S\ref{sec:result}, it is difficult to find devices in remote real-device testing services that are exactly the same as those in our custom system dataset, making it impossible to generate PoCs for all the vulnerabilities we identified.
Additionally, the state of devices in testing services is complex, and we cannot confirm whether the devices have been fully initialized to reveal all sensitive system properties and settings. 
Moreover, these devices typically do not have SIM cards installed, which may result in the absence of sensitive system properties and settings related to SIM cards (e.g., IMSI, ICCID).