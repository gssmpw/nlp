\section{Results}
\label{sec:result}

\subsection{Dataset}

\begin{table*}[th]
\caption{The Distribution of Our Custom System Dataset and Summary of Our Results.}
\label{tab:dataset}
\begin{threeparttable}
\resizebox{1\linewidth}{!}{
\begin{tabular}{|c|c|ccccccccc|cc|cc|cc|}
\hline
\multirow{2}{*}{Brand} & \multirow{2}{*}{\# Devices} & \multicolumn{9}{c|}{Android Version}                   & \# Sensitive & \# (\%) Vulnerable & \# Sensitive & \# (\%) Vulnerable & \# (\%) Sensitive & \# (\%) Vulnerable \\ 
                       &                             & Pre-v6 & v7  & v8  & v9  & v10 & v11 & v12 & v13 & v14 & Properties   & Properties         & Settings     & Settings           & Devices           & Devices            \\ \hline \hline
Alps                   & 26                            & \multicolumn{1}{c|}{3}      & \multicolumn{1}{c|}{3}   & \multicolumn{1}{c|}{7}   & \multicolumn{1}{c|}{4}   & \multicolumn{1}{c|}{5}   & \multicolumn{1}{c|}{0}   & \multicolumn{1}{c|}{3}   & \multicolumn{1}{c|}{1}   & 0   & 94           & 40 (43\%)          & 24           & 5 (21\%)           & 10 (38\%)         & 7 (27\%)           \\ \hline
Asus                   & 55                            & \multicolumn{1}{c|}{1}      & \multicolumn{1}{c|}{6}   & \multicolumn{1}{c|}{10}  & \multicolumn{1}{c|}{11}  & \multicolumn{1}{c|}{11}  & \multicolumn{1}{c|}{3}   & \multicolumn{1}{c|}{3}   & \multicolumn{1}{c|}{10}  & 0   & 36           & 22 (61\%)          & 35           & 3 (9\%)            & 12 (22\%)         & 6 (11\%)           \\ \hline
Digma                  & 25                            & \multicolumn{1}{c|}{0}      & \multicolumn{1}{c|}{1}   & \multicolumn{1}{c|}{20}  & \multicolumn{1}{c|}{4}   & \multicolumn{1}{c|}{0}   & \multicolumn{1}{c|}{0}   & \multicolumn{1}{c|}{0}   & \multicolumn{1}{c|}{0}   & 0   & 20           & 3 (15\%)           & 35           & 0 (0\%)            & 25 (100\%)        & 1 (4\%)            \\ \hline
Huawei                 & 29                            & \multicolumn{1}{c|}{2}      & \multicolumn{1}{c|}{1}   & \multicolumn{1}{c|}{0}   & \multicolumn{1}{c|}{5}   & \multicolumn{1}{c|}{7}   & \multicolumn{1}{c|}{5}   & \multicolumn{1}{c|}{9}   & \multicolumn{1}{c|}{0}   & 0   & 6            & 1 (17\%)           & 5            & 1 (20\%)           & 5 (17\%)          & 1 (3\%)            \\ \hline
Lenovo                 & 99                            & \multicolumn{1}{c|}{21}     & \multicolumn{1}{c|}{9}   & \multicolumn{1}{c|}{8}   & \multicolumn{1}{c|}{26}  & \multicolumn{1}{c|}{12}  & \multicolumn{1}{c|}{13}  & \multicolumn{1}{c|}{6}   & \multicolumn{1}{c|}{4}   & 0   & 529          & 337 (64\%)         & 182          & 51 (28\%)          & 85 (86\%)         & 75 (76\%)          \\ \hline
Meizu                  & 40                            & \multicolumn{1}{c|}{10}     & \multicolumn{1}{c|}{6}   & \multicolumn{1}{c|}{8}   & \multicolumn{1}{c|}{6}   & \multicolumn{1}{c|}{2}   & \multicolumn{1}{c|}{4}   & \multicolumn{1}{c|}{0}   & \multicolumn{1}{c|}{3}   & 1   & 218          & 125 (56\%)         & 142          & 44 (31\%)          & 39 (97\%)         & 37 (93\%)          \\ \hline
Motorola               & 144                           & \multicolumn{1}{c|}{9}      & \multicolumn{1}{c|}{11}  & \multicolumn{1}{c|}{17}  & \multicolumn{1}{c|}{24}  & \multicolumn{1}{c|}{21}  & \multicolumn{1}{c|}{13}  & \multicolumn{1}{c|}{33}  & \multicolumn{1}{c|}{16}  & 0   & 416          & 162 (39\%)         & 146          & 55 (38\%)          & 118 (82\%)        & 113 (78\%)         \\ \hline
Nokia                  & 88                            & \multicolumn{1}{c|}{4}      & \multicolumn{1}{c|}{1}   & \multicolumn{1}{c|}{5}   & \multicolumn{1}{c|}{17}  & \multicolumn{1}{c|}{30}  & \multicolumn{1}{c|}{10}  & \multicolumn{1}{c|}{9}   & \multicolumn{1}{c|}{12}  & 0   & 363          & 266 (73\%)         & 100          & 36 (36\%)          & 49 (56\%)         & 40 (45\%)          \\ \hline
Nubia                  & 29                            & \multicolumn{1}{c|}{0}      & \multicolumn{1}{c|}{3}   & \multicolumn{1}{c|}{3}   & \multicolumn{1}{c|}{8}   & \multicolumn{1}{c|}{3}   & \multicolumn{1}{c|}{5}   & \multicolumn{1}{c|}{4}   & \multicolumn{1}{c|}{3}   & 0   & 78           & 46 (59\%)          & 87           & 25 (28\%)          & 29 (100\%)        & 21 (72\%)          \\ \hline
Oneplus                & 45                            & \multicolumn{1}{c|}{2}      & \multicolumn{1}{c|}{0}   & \multicolumn{1}{c|}{0}   & \multicolumn{1}{c|}{3}   & \multicolumn{1}{c|}{8}   & \multicolumn{1}{c|}{17}  & \multicolumn{1}{c|}{8}   & \multicolumn{1}{c|}{7}   & 0   & 195          & 106 (54\%)         & 78           & 47 (60\%)          & 38 (84\%)         & 36 (80\%)          \\ \hline
Oppo                   & 64                            & \multicolumn{1}{c|}{6}      & \multicolumn{1}{c|}{2}   & \multicolumn{1}{c|}{8}   & \multicolumn{1}{c|}{9}   & \multicolumn{1}{c|}{10}  & \multicolumn{1}{c|}{5}   & \multicolumn{1}{c|}{6}   & \multicolumn{1}{c|}{18}  & 0   & 384          & 229 (60\%)         & 145          & 25 (17\%)          & 61 (95\%)         & 59 (92\%)          \\ \hline
Poco                   & 22                            & \multicolumn{1}{c|}{0}      & \multicolumn{1}{c|}{0}   & \multicolumn{1}{c|}{0}   & \multicolumn{1}{c|}{0}   & \multicolumn{1}{c|}{3}   & \multicolumn{1}{c|}{8}   & \multicolumn{1}{c|}{6}   & \multicolumn{1}{c|}{5}   & 0   & 363          & 69 (19\%)          & 120          & 69 (56\%)          & 22 (100\%)        & 22 (100\%)         \\ \hline
Qti                    & 42                            & \multicolumn{1}{c|}{0}      & \multicolumn{1}{c|}{0}   & \multicolumn{1}{c|}{0}   & \multicolumn{1}{c|}{2}   & \multicolumn{1}{c|}{2}   & \multicolumn{1}{c|}{6}   & \multicolumn{1}{c|}{28}  & \multicolumn{1}{c|}{4}   & 0   & 326          & 132 (40\%)         & 173          & 93 (54\%)          & 41 (98\%)         & 40 (95\%)          \\ \hline
Realme                 & 65                            & \multicolumn{1}{c|}{0}      & \multicolumn{1}{c|}{0}   & \multicolumn{1}{c|}{1}   & \multicolumn{1}{c|}{8}   & \multicolumn{1}{c|}{19}  & \multicolumn{1}{c|}{11}  & \multicolumn{1}{c|}{10}  & \multicolumn{1}{c|}{15}  & 1   & 396          & 192 (48\%)         & 137          & 37 (27\%)          & 65 (100\%)        & 65 (100\%)         \\ \hline
Redmi                  & 59                            & \multicolumn{1}{c|}{0}      & \multicolumn{1}{c|}{0}   & \multicolumn{1}{c|}{0}   & \multicolumn{1}{c|}{0}   & \multicolumn{1}{c|}{7}   & \multicolumn{1}{c|}{10}  & \multicolumn{1}{c|}{11}  & \multicolumn{1}{c|}{26}  & 5   & 673          & 124 (18\%)         & 208          & 100 (48\%)         & 59 (100\%)        & 54 (92\%)          \\ \hline
Samsung                & 158                           & \multicolumn{1}{c|}{20}     & \multicolumn{1}{c|}{11}  & \multicolumn{1}{c|}{13}  & \multicolumn{1}{c|}{28}  & \multicolumn{1}{c|}{27}  & \multicolumn{1}{c|}{25}  & \multicolumn{1}{c|}{15}  & \multicolumn{1}{c|}{19}  & 0   & 534          & 226 (42\%)         & 524          & 212 (40\%)         & 154 (97\%)        & 134 (85\%)         \\ \hline
Xiaomi                 & 146                           & \multicolumn{1}{c|}{9}      & \multicolumn{1}{c|}{12}  & \multicolumn{1}{c|}{24}  & \multicolumn{1}{c|}{11}  & \multicolumn{1}{c|}{25}  & \multicolumn{1}{c|}{26}  & \multicolumn{1}{c|}{4}   & \multicolumn{1}{c|}{29}  & 6   & 1598         & 441 (28\%)         & 647          & 281 (43\%)         & 137 (94\%)        & 116 (79\%)         \\ \hline
Zte                    & 28                            & \multicolumn{1}{c|}{2}      & \multicolumn{1}{c|}{2}   & \multicolumn{1}{c|}{3}   & \multicolumn{1}{c|}{11}  & \multicolumn{1}{c|}{5}   & \multicolumn{1}{c|}{3}   & \multicolumn{1}{c|}{2}   & \multicolumn{1}{c|}{0}   & 0   & 153          & 116 (76\%)         & 188          & 57 (30\%)          & 23 (82\%)         & 15 (54\%)          \\ \hline
Other (232)            & 650                           & \multicolumn{1}{c|}{71}     & \multicolumn{1}{c|}{42}  & \multicolumn{1}{c|}{198} & \multicolumn{1}{c|}{148} & \multicolumn{1}{c|}{80}  & \multicolumn{1}{c|}{65}  & \multicolumn{1}{c|}{30}  & \multicolumn{1}{c|}{14}  & 2   & 1810         & 840 (46\%)         & 644          & 195 (30\%)         & 434 (67\%)        & 270 (42\%)         \\ \hline
Total (250)            & 1814                          & \multicolumn{1}{c|}{160}    & \multicolumn{1}{c|}{110} & \multicolumn{1}{c|}{325} & \multicolumn{1}{c|}{325} & \multicolumn{1}{c|}{277} & \multicolumn{1}{c|}{229} & \multicolumn{1}{c|}{187} & \multicolumn{1}{c|}{186} & 15  & 8192         & 3477 (42\%)        & 3620         & 1336 (37\%)        & 1406 (78\%)       & 1112 (61\%)        \\ \hline
\end{tabular}
}
\begin{tablenotes}
	\footnotesize
	\item Note: we separately summarize the 18 brands with more than 20 devices each, while the remaining 232 brands were combined in the statistics. In addition, due to the small \\number of ROMs before Android 6, we combine the statistics for ROMs from Android 6 and earlier versions. We also combine the sub-versions, such as Android 8.0 and 8.1. \\The percentages in the Vulnerable Properties and Settings columns are calculated by \# Vulnerable/\# Sensitive, and the percentages in the Sensitive/Vulnerable Devices \\columns are calculated in comparison to the \# Devices.
\end{tablenotes}
\end{threeparttable}
\end{table*}

To facilitate our large-scale investigation, we collected custom Android ROMs from the public project Android Dumps~\cite{android_dumps}, which hosts a wide range of Android stock ROMs.
In October 2023, we collected all system ROMs from this source, resulting in a dataset of \datasetsize ROMs from \totalbrand vendors.
The vendors, device numbers, and Android API versions of these ROMs are summarised in~\autoref{tab:dataset}.
Our dataset primarily includes ROMs from major Android vendors such as Samsung, Xiaomi, Huawei, Lenovo, 
which constitute the majority of the dataset.
It also includes ROMs from smaller vendors like Gionee and Blackview, usually with only 2-3 ROMs each.
Notably, our dataset includes Google’s ROMs, such as those for Pixel devices, which feature Google's customizations. 
The ROMs in the dataset span Android versions from 4.0.3 to 14, with build dates ranging from 2013 to 2023.
This diverse dataset enables us to thoroughly explore the covert channels used to access non-resettable device identifiers in custom Android systems.
Note that the ROMs from Android Dumps have already been parsed into specific file collections rather than packaged as image files, allowing us to directly analyze the files without additional unpacking work. 

\subsection{Custom System Properties/Settings with Non-Resettable Identifiers}
\label{sec:staticresult}


Applying {\framework} to all custom ROMs we collected, we successfully analyzed over 600K APK and JAR files within these custom systems, generating over 270GB of result data. 
This data includes accessed system properties and settings, as well as extensive contextual code information.
After applying the filtering process described in~\S\ref{sec:approachfilter}, we identified about 30K potential cases.
Over the course of five days, two experienced Android researchers completed the manual verification on these cases. 
During this process, we observed that different ROMs from the same brand often shared repeated cases, including identical system property or setting names and contextual code.
We believe that these properties and settings are the same across ROMs of the same brand. 
Consequently, we avoid duplicate manual reviews of these cases, which significantly reduces our workload.
As a result, we confirmed a total of \totalsensitivepropertiescase system properties and \totalsensitivesettingscase system settings containing non-resettable device identifiers, including \totalsensitiveproperties unique system properties and \totalsensitivesettings system settings\footnote{The same system property or setting may recur across different ROMs, resulting in a total count significantly higher than the number of unique types.}.
We found at least one instance of these in \totalsensitivedevices custom ROMs. 
The detailed results are presented in~\autoref{tab:dataset}, with a breakdown by identifier type available in~\autoref{tab:typesplit} in appendices.
For clarity, we will refer to system properties and settings that store non-resettable device identifiers as \textbf{sensitive system properties and settings} in the remainder of the text.
Our results reveal that sensitive system properties and settings introduced through system customizations are widely prevalent.

We also examined the evolution of sensitive system properties/settings across different Android versions.
As shown in~\autoref{fig:staticversion}, from the early Android versions up to Android 10, the average of sensitive system properties and settings in custom systems gradually increased, with the most significant growth occurring in Android 10. 
We speculate that this increase is related to Android's tightening policies on device identifiers, particularly the complete prohibition of third-party apps from using non-resettable device identifiers in Android 10.
After Android 10, there was a slight decline, possibly influenced by heightened regulatory scrutiny from both societal and academic spheres.
It is worth noting that our dataset includes only 15 custom ROMs for Android 14, so the small sample size may have skewed the data for this version, making it an outlier compared to the overall trend.

Additionally, we investigate which supply chain actors utilize covert channels to access identifiers and their purposes for doing so. 
We identified these actors by analyzing package names associated with the code accessing sensitive system properties and settings. We then focused on frequently occurring package names for a detailed case study, manually examining their code.
Our findings reveal that custom system developers (e.g., Oppo~\cite{oppo}) and hardware providers (e.g., MediaTek~\cite{mediatek}) use these properties within system components to monitor device status. 
For example, MediaTek's telephony component checks the IMSI or ICCID stored in system properties to determine if a SIM card is present.
Additionally, various SDKs in system apps leverage these properties and settings. 
This includes SDKs from device vendors like Xiaomi's~\cite{xiaomi} analytics and crash reporting SDKs, Lenovo's~\cite{lenovo} push SDK, and third-party SDKs such as Baidu's~\cite{baidu} map SDK and Alibaba's~\cite{alibaba} security SDK.
They use these identifiers for device identification, transmitting user usage data, crash reports, and identifier to remote endpoints.

Furthermore, our analysis shows that in higher versions of Android, supply chain actors increasingly prefer using custom system properties or settings instead of official APIs to access device identifiers. 
This trend reinforces our hypothesis that, despite having system-level permissions, these actors are affected by system modifications that disrupt stable user tracking via official APIs. 
As a result, they turn to covert channels to ensure consistent tracking.

\noindent \textbf{Our Findings.}
\textit{A substantial number of system properties and settings storing non-resettable device identifiers are present in custom systems across various brands and system versions. 
The introduction of these properties/settings is likely driven by the need for convenient access to identifiers, in response to restrictions on their usage.}


\begin{figure}[h]
  \centering
  \resizebox{1\linewidth}{!}{
  \includegraphics[width=\textwidth]{Graphs/android_version_0109.pdf}
  }
  \caption{The Average of System Properties/Settings and  Percentage of Devices across Different System Versions.}
  \label{fig:staticversion}
\end{figure}

\subsection{Vulnerable Implementations}
Next, we analyzed the access control policies for these sensitive system properties and system settings following the method described in~\S\ref{sec:approachaccess} to see if they have proper access control.
For the \totalsensitivepropertiescase system properties and \totalsensitivesettingscase system settings, we identified \totaldangerouspropertiescase system properties and \totaldangeroussettingscase system settings without proper access control, allowing any third-party apps to access these identifiers. We found at least one such system property or setting in \totaldangerousdevices custom ROMs.
Similarly, for the sake of simplicity, we refer to sensitive system properties and settings lacking effective access control as \textbf{vulnerable system properties and settings} in the following.
Our results indicate that vulnerable system properties and settings are prevalent in custom systems.
It is worth mentioning that we did not find any such vulnerabilities in Google's systems, demonstrating Google's robust code security that aligns with findings from previous studies~\cite{elsabagh2020firmscope, hou2022large}.
In terms of the trend across system versions, it is also quite similar to that of sensitive system properties and settings, as shown in~\autoref{fig:staticversion}.
Before Android 10, due to the relatively small number of sensitive system properties and settings, the difficulty of implementing access control was lower, resulting in more effective access control.
After Android 10, as the number of sensitive system properties and settings increased, the implementation of access control policies became more prone to errors.
Furthermore, we analyzed some of these cases with vulnerable access control, and summarized some of the possible reasons that may lead to vulnerable implementations.

\noindent \textbf{Overlooked System Properties and Settings.}
We believe that the most direct cause of vulnerable access control is the oversight during development, where appropriate access control policies were not deployed for sensitive system properties and settings.
For example, in the custom ROM of BRAND-A (Android version 14), we have reported four vulnerable system properties ``xx.xx.xx.imei1'', ``xx.xx.xx.imei2'', ``xx.xx.xx.meid'', and ``xx.xx.xx.sn'', which have been confirmed by the vendors.
Our analysis of all property contexts in this system revealed that the most common matching context name for these properties is ``*'', the default property context type. 
This rule can match any system property and also allows any third-party apps to access such system properties. 
In this example, we consider that the developers forgot to add a property context for this sensitive system property, resulting in the vulnerable access control.
There are a total of 844 similar cases in our results where the most matching property context for a sensitive system property is the default context (i.e., the property name in the security context is ``*''). 
On the other hand, most of the vulnerable system settings (i.e., 95\%) we discovered are not defined (overlooked) in the Settings class. 
This lack of definition allows third-party apps to access these sensitive settings, leading to access control issues.

Another reason sensitive system properties and settings may be overlooked is the large number of supply chain actors involved in custom systems.
When establishing access control policies, developers may focus on the custom properties and settings introduced by major supply chain actors, including device vendors and hardware providers.
However, they might miss some covert third-party actors, like software vendors, who also introduce sensitive properties and settings, leading to vulnerabilities.
We use the two sensitive system settings, ``xxx\_xxx\_i'' and ``xxx.xxx.deviceid'', introduced by Baidu~\cite{baidu}, as an example.
The ``xxx\_xxx\_i'' setting stores the device's plaintext IMEI, while ``xxx.xxx.deviceid'' contains an encrypted key-value pair that includes the IMEI and another identifier generated using the android\_id~\cite{androidid}. 
The encryption uses AES in CBC mode, but with a key hard-coded in the code, and the final string is encoded in base64.
These two settings are widely distributed across custom systems.
Our analysis revealed that access to these sensitive settings primarily occurs within the Baidu SDK used in many system apps, such as app stores, map apps, browsers, and weather apps.
In our results, we identified a total of 3,120 apps that access these settings. 
The widespread presence of these sensitive settings in custom systems, combined with the lack of access control in all the systems we analyzed, has significantly impacted our results (as shown in~\autoref{tab:dataset} and~\autoref{fig:staticversion}).


\noindent \textbf{Overly Complex Access Control Rules.}
SELinux policies in custom systems may contain a large number of rules due to the extensive customizations.
For example, in the custom ROM of BRAND-B (Android version 14), there are over 2,500 property contexts and more than 50,000 lines of specific SELinux policy files.
We have reported two vulnerable system properties ``xxx.xxx.xxx.btmac'' and ``xxx.xxx.xxx.wifimac'' identified in this ROM, which was also confirmed by device vendors.
These properties belong to ``radio\_prop'' type, which is included in six type attribute sets and linked to dozens of policy rules.
One of these rules allows ``untrusted\_app'' to access it, resulting in the access control issue for this sensitive property.
Therefore, we believe that overly complex access control rules contribute to such vulnerabilities.
On the one hand, as in the case of the above example, involving too many type attribute sets can increase the probability of introducing access control vulnerabilities. 
This is because a mistake in any of these sets can lead to access control errors for the system property. 
On the other hand, an excessive number of rules degrade the auditability of access control, making it more difficult to identify existing access control issues.

\noindent \textbf{Additional Access Channels for System Properties.}
Generally, the methods for accessing system properties or settings are through system APIs or commands, as mentioned in~\S\ref{sec:background1}.
However, in our findings, we discovered that some sensitive system properties are accessed by system services and returned by methods within these system services, thereby introducing additional access channels for sensitive system properties.
To investigate such issues, we conducted additional analysis on the access behaviors of sensitive system properties within system services. 
Since only specific system files are loaded as system services~\cite{systemservice}, we can identify potential access behaviors of sensitive system properties within system services by examining the file paths where these behaviors occur.
After identifying the potential targets, we further confirm whether the behavior occurs within a system service by examining the parent class of the class where the sensitive system property behavior is located. 
Typically, classes that host system services typically extend the \texttt{\seqsplit{com.android.server.SystemService}} class or the Binder's Stub class~\cite{binder}  (any class using Binder as a receiver also subclasses this class).
Finally, we manually analyzed the obtained results to: 
(1) confirm whether the class containing the sensitive system property access behavior was part of a system service; 
(2) determine whether the sensitive system property was accessed within a method of the system service and returned as a method result; 
(3) investigate whether the method was a public method and whether there were any access control policies in place to prevent access by third-party apps.
As a result, we found eight related cases.

\begin{figure}[h]
  \centering
  \resizebox{1\linewidth}{!}{
  \includegraphics[width=\linewidth]{Graphs/system_service_2.pdf}
  }
  \caption{The Code of the Vulnerable System Service.}
  \label{fig:service}
\end{figure}

An example is the system service named ``xxxService'' in the custom ROM of BRAND-C (Android version 12).
Within this service, there is a method with the signature ``public String generatePostString()''. The code for this method and its related methods is shown in~\autoref{fig:service}.
The method ``generatePostString'' further calls a method named ``generateImeiString'', as shown in Step~\ding{202}. 
This method reads the value of the sensitive but not vulnerable system property named ``xxx.xxx.xxx\_imei1'', as illustrated in Step~\ding{203}. 
The method ``generatePostString'' retrieves the device's IMEI, puts it into a JSONObject, and then returns the content of sensitive system property with other device information, as shown in Step~\ding{204}.
Since ``xxxService'' is a system service, and the ``generatePostString'' method is public with no additional access control, untrusted third-party apps can access this method and retrieve the device's IMEI.
One exploitation method for this type of case is to access the system service directly via Java reflection and Binder mechanism, which does not require the application context.
The additional channels for accessing sensitive system properties introduced in system services without access control result in the vulnerabilities.

\noindent \textbf{Our Findings.}
\textit{Vulnerable sensitive system properties and settings are prevalent in custom Android systems, and the reasons can be attributed to extensive and complex system customizations, as well as developer oversights.}

\subsection{Result Verification}
We further verify the accuracy of the vulnerable system properties and settings we identified, by generating Proof of Concepts (PoCs) on real smartphones.
Due to the limited availability of testing devices from various manufacturers and devices, we chose to take advantage of remote real-device testing services provided by device vendors and commercial platform (i.e., Alibaba Cloud~\cite{aliyun}). 
These services provide access to a wide range of models through remote desktop access, allowing users to upload and run test apps.
However, in our dataset, there is only one version of the custom ROM for each device, corresponding to a specific system version. 
The remote testing services might not have the exact brand and device we need, and even if they do, the system version might differ from our dataset.
Therefore, within the practical constraints, we are unable to generate PoCs for all identified vulnerabilities. 
As an alternative, we found 32 devices in the remote real-device testing service that have the same device model and Android system version as those in our dataset (although the specific build versions differ, such as the custom ROM compiled in September 2023 in our dataset versus one compiled in October 2023 in the testing service). 
We used these devices to evaluate the accuracy of our detection results.
Specifically, we developed a testing app without any permission requests that functions as a third-party app to retrieve all system properties and settings on the device, and check for the presence of any vulnerable system properties or settings.
We manually compared the contents of the found properties and settings with the non-resettable device identifiers provided in the devices' settings app, confirming that they indeed contained non-resettable identifiers.

In summary, our approach identified 130 vulnerable properties and settings across the 32 devices, and we successfully generated PoCs for 102 of them. 
After verification, all 102 properties and settings were confirmed to store non-resettable device identifiers.
For the remaining 28 cases, our manual verification shows that they are primarily caused by one system settings, ``xxx\_device\_mac''.
The cases related to ``xxx\_device\_mac'' were caused by the MediaTek~\cite{mtk} telephony service component. 
Upon analyzing its code, we found that it attempts to obtain the device MAC address from various sources, with accessing ``xxx\_device\_mac'' being one of the methods.
These cases can only be triggered under specific configurations, thus we cannot generate PoCs for them automatically.
We will discuss these limitations in~\S\ref{sec:limitations}.
Overall, the on-device dynamic evaluation demonstrates the high accuracy of our approach.

\noindent \textbf{Our Findings.}
\textit{Dynamic exploration of real-world devices shows the great accuracy of our approach. We can generate PoCs for most of the vulnerable system properties or settings automatically, while the remaining ones can only be triggered under specific conditions.
}

\subsection{Comparison with State-of-the-art}
We further compare our approach with U2-I2~\cite{meng2023post}, the most closely related work.
In terms of research scale, U2-I2 covered only 13 devices, identifying 30 vulnerable system properties and settings. 
In contrast, our dataset and the total number of identified issues are two orders of magnitude larger, allowing for a comprehensive examination of non-resettable identifiers in system properties and settings, with broader coverage in both time scale and device diversity.
Besides, we further compare the capability of these two approaches for analyzing the same ROM.
Upon reviewing its open-source artifact~\cite{gdprartifact}, we find that the test devices have been anonymized, making it impossible for us to conduct a comparison against their results.
Additionally, we are unable to apply the U2-I2 method on remote real-devices, as factory resets, which are necessary for the U2-I2 method, cannot be performed on these devices.
Ultimately, we have gathered four real devices that can be used for fair comparison. For these devices, their ROMs are included in our database.
Further, we conducted dynamic testing on these devices using the U2-I2 tool.
Specifically, following the guidelines, we searched for non-resettable identifiers by running the U2-I2 tester app before and after factory reset of devices and comparing the results.
We ran the ``GETID'' functionality in the tool, which is relevant to our research objectives and is used to detect non-resettable identifiers in system properties and settings.
During the testing, we found that the process required considerable manual effort. 
Since a factory reset was involved, we first had to restore rooted test device to prevent it from becoming bricked (i.e., can no longer start normally).
Additionally, parts of the tool’s testing process, including the factory reset, re-enabling of ADB debugging post-reset, determination of system properties/settings, and final result comparison, could not be automated.
In particular, during the result comparison, many items remained unchanged after the reset, even after excluding the very short ones. 
It is necessary to check the device's non-resettable identifiers in the settings app and compare them with the unchanged system properties and settings.

As a result, U2-I2 identified three non-resettable identifiers in the system properties and settings across four devices, while our method not only detected these three cases but also found two more identifiers in system settings on two devices, namely ``xxx\_sim\_imsi'' and ``xxx\_sim\_iccid''.
These two system settings are associated with a feature related to mobile data and only appear after activating specific functions on the device. 
Due to the lack of complete feature setup in the U2-I2 method, it failed to detect them.
This highlights U2-I2's limitation in detecting issues that rely on specific preconditions (e.g., activating a feature or performing an action), whereas our static approach is capable of uncovering such vulnerabilities.
However, we identified one system setting (the aforementioned ``xxx\_device\_mac'') through our method that was not found on the device, indicating a potential false positive.
In summary, our method not only replicates U2-I2’s findings with acceptable accuracy loss, but also uncovers additional vulnerabilities.

\noindent \textbf{Our Findings.}
\textit{
Compared to existing work, our experiments operate on a much larger scale without any limitation, enabling a thorough examination of non-resettable identifiers across diverse devices.
Even testing on the same devices, our method can uncover more vulnerabilities that were not detected by U2-I2. It suggests the high scalability and accuracy of our approach.
}

\subsection{Brand-Wide Recurring Vulnerabilities}
As mentioned in~\ref{sec:staticresult}, we observed that the same vulnerabilities often appeared on multiple devices from the same brand.
Inspired by this, we extended our testing to \dynamicmodels new devices of the brands we have analyzed, using the remote real-device service.
Through this method, we expect to identify more vulnerable devices, even their system ROMs are not available.

\begin{table}[t]
\caption{The Device Information of Extended Evaluation.}

\label{tab:dynamic}
\begin{threeparttable}
\resizebox{1\linewidth}{!}{
\begin{tabular}{|c|c|cccccccccc|c|c|}
\hline
\multirow{2}{*}{Brand} & \multirow{2}{*}{\# Devices} & \multicolumn{10}{c|}{Android Version}                 & \multirow{2}{*}{\# VP} & \multirow{2}{*}{\# VS} \\
                       &                             & v6 & v7 & v8 & v9 & v10 & v11 & v12 & v13 & v14 & v15 &                        &                        \\ \hline
Huawei                 & 29/11                       & 0  & 0  & 3  & 3  & 20  & 0   & 3   & 0   & 0   & 0   & 0                      & 22                     \\ \hline
Motorola               & 7/7                         & 1  & 3  & 0  & 0  & 0   & 3   & 0   & 0   & 0   & 0   & 18                     & 11                     \\ \hline
Oppo                   & 53/31                       & 0  & 0  & 0  & 2  & 5   & 10  & 6   & 17  & 13  & 0   & 4                      & 56                     \\ \hline
Oneplus                & 11/7                        & 0  & 0  & 0  & 0  & 0   & 0   & 0   & 0   & 9   & 2   & 0                      & 15                     \\ \hline
Samsung                & 35/21                       & 0  & 2  & 2  & 3  & 3   & 0   & 10  & 9   & 6   & 0   & 12                     & 50                     \\ \hline
Vivo                   & 110/97                      & 0  & 0  & 4  & 2  & 7   & 24  & 18  & 28  & 26  & 1   & 36                     & 183                    \\ \hline
Xiaomi                 & 20/9                        & 0  & 0  & 0  & 0  & 0   & 0   & 0   & 12  & 8   & 0   & 4                      & 9                      \\ \hline
Redmi                  & 23/13                       & 0  & 0  & 0  & 0  & 0   & 1   & 2   & 17  & 3   & 0   & 10                     & 12                     \\ \hline
Other (14)             & 26/20                       & 1  & 6  & 3  & 4  & 4   & 4   & 3   & 0   & 0   & 1   & 76                     & 42                     \\ \hline
Total (22)             & 314/216                     & 2  & 11 & 12 & 14 & 39  & 42  & 42  & 83  & 65  & 4   & 160                    & 400                    \\ \hline
\end{tabular}
}
\begin{tablenotes}
	\footnotesize
	\item Note: we separately summarize the 8 brands with more than 5 devices each, whi-\\le the remaining 14 brands were combined in the statistics. The ``Devices'' column \\contains two numbers: the total number of devices tested and the number of \\devices with vulnerable system properties or settings. The last two columns, ``VP'' \\and ``VS'', represent vulnerable properties and vulnerable settings, respectively.
\end{tablenotes}
\end{threeparttable}
\end{table}




Our dynamic testing results, summarized in~\autoref{tab:dynamic}, showed that the test on \dynamicmodels different devices across \dynamicbrands brands identified \dynamicpropertiescase vulnerable system properties and \dynamicsettingscase vulnerable system settings among them.
This further confirmed our findings that the same vulnerabilities do exist across different devices from the same brand. We speculate that this may be due to code reuse within the same brand or among upstream suppliers.
The most prevalent vulnerable system settings were still ``xxx\_xxx\_i'' and ``xxx.xxx.deviceid'', both introduced by Baidu.
After decrypting their content, we found that on devices with higher Android versions (higher than 11), the Baidu SDK typically cannot obtain the device's IMEI. 
We speculate that on these newer versions, the Baidu SDK primarily relies on device identifiers generated from the ``android\_id'' for user tracking.
By conducting these dynamic tests, we demonstrated that similar vulnerable system properties and settings issues indeed exist across devices of the same brand.

We have timely disclosed the vulnerabilities identified in dynamic testing to vendors.
In general, vendors require corresponding evidences (e.g., PoCs) to confirm the vulnerabilities. Thus, we only reported the vulnerabilities that we have generated PoCs in real devices with system versions greater than 12, to the corresponding vendors.
We did not report issues in devices with older Android versions, because vendors typically focus on the security of recently released products, and issues present in lower versions may have already been fixed in higher versions.
Currently, two vendors have confirmed our findings, including 82 vulnerable system properties and 2 vulnerable settings. 
They have already fixed the related issues in the new system updates. 
The security vulnerability reports we sent to other vendors are still under review.

\noindent \textbf{Our Findings.}
\textit{We observed that vulnerable system properties/settings often repeatedly appear in devices from the same OEMs. 
By black-box testing of \dynamicmodels devices, we indeed found many recurring vulnerable properties and settings without ROMs.
We further reported our findings to respective vendors and received acknowledgment.}




\subsection{Ethical Consideration}
We have reported all our findings to the relevant parties and assisted them in addressing the issues. 
To date, all responding manufacturers have completed the issue fixes.
Additionally, we anonymized the brand names mentioned in the specific issue descriptions, as well as the detailed names of vulnerable system properties and settings in our paper.
There are two primary reasons. 
First, we are waiting for the remaining vendors to confirm our findings and address the security issues, followed by an additional 90-day confidentiality period, recommended by responsible disclosure. 
Second, and of greater concern, it is the fact that although vendors typically promptly address confirmed issues and strive to push security updates to users, a significant number of users continue to use older models that no longer receive system updates or may refuse to update their custom systems. 
These circumstances could result in custom Android systems on user devices containing the system properties or settings containing device non-resettable identifiers that we discovered.
Since our findings are derived from analyzing a large number of custom systems from various brands worldwide, if malicious developers exploit our discoveries to attempt to collect device non-resettable identifiers through the system properties and settings on user devices, it could pose a significant threat to user privacy.
Therefore, we chose to anonymize our specific findings to reduce the risk of compromising user privacy.