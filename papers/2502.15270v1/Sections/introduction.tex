\section{Introduction}
\label{sec:introduction}

User tracking is pivotal in the Android ecosystem, enabling apps to deliver personalized user experiences, targeted advertising, user behavior analysis, performance monitoring, and market segmentation~\cite{kollnig2022iphones, binns2018third, razaghpanah2018apps, lerner2016internet, nath2015madscope, yang2020comparative}. 
This practice allows developers to optimize app functionality, enhance marketing strategies, and deliver tailored content to users, thereby significantly contributing to the overall effectiveness and user satisfaction of mobile apps.
In Android, identifiers, which are unique strings or numbers, are crucial for tracking by distinguishing users or devices and are often collected with specific user data.
The uniqueness of identifiers allows apps to link behaviors collected at different times, locations, and even across different apps to the same user, enabling them to build a clear user profile and provide better services.

Android identifiers can be classified based on their scope of usage. 
\textit{App identifiers} are unique to each app and are used to track user behavior within a single app, while \textit{Device identifiers} are used to track user behavior across multiple apps on the same device. 
Device identifiers can be further categorized based on their persistence into \textit{user-resettable} and \textit{non-resettable} types. 
User-resettable identifiers, like advertising IDs~\cite{advertisingid}, can be reset by the user, providing control over privacy. 
In contrast, non-resettable identifiers, such as the International Mobile Equipment Identity (IMEI), are permanently tied to the device and provide continuous tracking even after a device reset, making them preferable for tracking, but their high stability and extensive reach come with significant privacy concerns.
If compromised, these identifiers could lead to severe breaches of user privacy, potentially exposing extensive personal data to malicious actors.
Additionally, non-resettable device identifiers are widely used for user tracking not only by third-party apps, but also by various supply chain actors in the Android ecosystem, such as OS developers, hardware vendors, and pre-installed system apps~\cite{lyons2023log, gamba2020analysis, schindler2022privacy, leith2021mobile, liu2021android, reardon201950, demetriou2016free, chen2014information}.
This widespread use highlights the essential requirement for robust protective measures to prevent their compromise and ensure user privacy.

As privacy concerns grow, the management of non-resettable device identifiers in Android has undergone significant updates to improve user privacy.
Before Android 10, apps could access these identifiers through specific official APIs after obtaining the required permissions and user consent.
With the release of Android 10 in 2019, access to these identifiers was restricted to apps with privileged permissions, which are not available to third-party apps.
Google now recommends third-party apps use user-resettable advertising IDs for advertising and analytics purposes.
Concurrently, many countries have enacted legislation (e.g., GDPR~\cite{gdpr} and CCPA~\cite{ccpa}) to regulate the collection and use of personal data.

However, some supply-chain actors, including device and hardware vendors as well as software providers, often introduce \textit{covert channels} to facilitate the retrieval of non-resettable identifiers~\cite{meng2023post, zhou2022uncovering}.
For instance, storing these identifiers in \textit{system properties and settings}, which function as global variables in the Android system, allows easy access to them.
One possible reason for this could be that these supply chain actors desire to maintain consistent, stable, and convenient tracking capabilities across different versions or modifications of the system. 
Introducing covert access channels can enable them to access these stable device identifiers, regardless of changes made to the official APIs or other system components by Google and system customizers.
However, this practice raises privacy concerns, as non-resettable identifiers could be leaked through these channels, creating new attack surfaces, if not properly managed.
These channels should be secured with access control measures equivalent to those protecting the identifiers themselves, necessitating stringent access control policies during system customization.
Regrettably, the customization introduced by vendors tends to introduce vulnerabilities such as improper security configurations and outdated protections, as highlighted in many existing studies~\cite{possemato2021trust, hou2022large, elsabagh2020firmscope, liu2022customized, el2021dissecting, li2021android}.
Inadequate protection of these identifiers can lead to unauthorized user tracking, impacting user privacy.

The security of custom Android systems has been widely studied by the research community.
Many researchers have focused on pre-installed apps in custom systems, studying privacy concerns or additional attack surfaces introduced by these apps~\cite{blazquez2021trouble, gamba2020analysis, elsabagh2020firmscope}.
There has also been research into the timeliness and effectiveness of applying Google's security updates in custom environments~\cite{hou2022large, liu2022customized, el2021dissecting}.
These studies, although have touched on the privacy issues within custom Android systems, they primarily consider the official channels for user tracking and do not take into account the covert channels that can return non-resettable device identifiers, introduced by the system customizations.
To the best of our knowledge, the only existing work in the research community that mentions these channels is U2-I2~\cite{meng2023post}, which employs dynamic testing on real devices to assess the mismanagement of these channels. 
However, their approach limits its research breadth.
U2-I2 requires a factory reset of devices, which confines them to using physical devices (even cannot use the widely accessible remote real-devices as alternatives). 
Consequently, expanding the scope of experiments necessitates purchasing a large number of devices, making it cost-prohibitive. 
Additionally, their method involves considerable manual costs, which also limits the test coverage.
As a result, it covered only 13 smartphones from 9 vendors, identifying 23 vulnerable system properties and 7 settings. 

In the work, we aim to provide a large-scale, accurate, and expandable investigation of non-resettable device identifiers in the wild, including the covert channels (i.e., system properties and settings), their usage, and the involved vulnerabilities.
To this end, we present {\framework}, an end-to-end approach that proceeds in three steps.
Firstly, 
we target the usage of system properties and settings in custom ROMs\footnote{An Android ROM is a firmware image file that contains the operating system and software for Android devices, enabling users to install or update on their devices.}
to identify all relevant instances in custom Android systems.
This begins with a pilot study to catalog methods of accessing system properties and settings in custom systems, followed by a static analysis method that leverages control-flow and data-flow analysis to pinpoint their usage points.
Secondly, we filter the above results by employing two heuristic approaches to identify potential system property and setting candidates that store non-resettable device identifiers.
We further manually verify the content of these properties and settings through the analysis of contextual code information.
Finally, we examine SELinux policies and system frameworks in custom systems to identify the access control for system properties and settings, uncovering vulnerable ones that store non-resettable device identifiers without adequate access control, i.e., can be abused by third-party apps without permissions.

Applying our methodology to \datasetsize custom Android systems from \totalbrand different Original Equipment Manufacturers (OEMs), 
we identified \totalsensitivepropertiescase system properties and \totalsensitivesettingscase system settings that store non-resettable device identifiers. 
Among these, \totaldangerouspropertiescase system properties and \totaldangeroussettingscase system settings across \totaldangerousdevices custom Android systems were found to lack proper access control measures.
Our detailed analysis of these vulnerable cases revealed major reasons for such vulnerabilities, including overlooked system properties and settings, overly complex access control rules, and additional access channels for system properties. 
To validate our results, we developed a demo app that requires no permissions and can access the device identifiers stored in these system properties and settings. We tested this app on 32 devices available through remote real-device testing services, and identified 130 vulnerable system properties and settings.
Of them, 102 can be directly triggered on these real devices and the remaining 28 have specific trigger conditions, confirming the accuracy of our detected results.
We further conducted a detailed comparison with U2-I2~\cite{meng2023post} on four real Android devices. The result shows that we can indeed identify more identifiers in systems properties and settings, which highlights U2-I2’s limitation in detecting issues that rely on specific preconditions (e.g., activating a feature or performing an action), whereas our static approach is capable of performing system-wide comprehensive analysis.
We also observed that vulnerable implementations usually recur across devices from the same OEMs. We then extended our research to a further \dynamicmodels real devices of \dynamicbrands OEMs using the same testing service. 
Even without analyzing these specific systems, we were able to directly trigger \dynamicpropertiescase vulnerable system properties and \dynamicsettingscase vulnerable system settings that store non-resettable device identifiers, affecting a total of \dynamictotal devices. 
We have reported the identified issues to the respective device vendors, and 84 vulnerabilities have been confirmed as of this writing. 

In summary, our work presents these major contributions:

\begin{itemize}
    \item We conduct a large scale study on custom Android systems to examine covert channels (i.e., custom system properties and settings) for accessing non-resettable device identifiers. 
    Our study presents an overall landscape of the privacy issues introduced by these covert channels.

    \item We develop an end-to-end analysis pipeline to thoroughly investigate the covert channels on a large dataset of custom Android ROMs, successfully identifying thousands of system properties and settings that store non-resettable device identifiers, many of which are vulnerable to exploitation by third-party apps. We also analyze the root cause of these vulnerabilities.

    \item We leverage remote real-device testing services to validate our approach, demonstrating that it is possible for third-party apps to retrieve non-resettable device identifiers from real devices without requesting any permissions. We have reported these to the respective vendors and receive positive confirmation.
    
\end{itemize}

The source code of our tool is publicly available~\cite{opensource}.