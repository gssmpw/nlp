\section{Related Work}
\label{sec:related}


\noindent \textbf{User Identifiers and Tracking Practices.}
Many existing researches have studied user identifiers and third-party tracking practices in mobile apps.
Razaghpanah et al.~\cite{razaghpanah2018apps} conducted a global study on mobile tracking ecosystems using the Lumen Privacy Monitor app. 
They identified 2,121 third-party advertising and tracking services, highlighting the heavy reliance on device identifiers for user tracking and the dominance of a few companies in the ecosystem.
Leith et al.~\cite{leith2021mobile} investigated data transmissions from mobile operating systems to manufacturers, revealing that device identifiers and other user data were sent to back-end servers even when users had configured their systems for minimal data sharing.
Kollnig et al.~\cite{kollnig2022iphones} studied 24,000 Android and iOS apps to compare the performance of these operating systems in terms of user privacy, noting the widespread use of user identifiers in both.
While these researchers examined the use of device identifiers for user tracking by Android apps and the system, they did not explore the use of identifiers beyond those provided by the documented system APIs.

\noindent \textbf{System Customization.}
Numerous previous studies have examined the security issues arising from the customizations from manufacturers.
Gamba et al.~\cite{gamba2020analysis} conducted a large-scale study of pre-installed apps on Android devices from over 200 vendors. 
They uncovered relationships between supply chain actors and found that these apps often exhibit invasive or even malicious behaviors, including backdoor access to sensitive data and services.
Lyons et al.~\cite{lyons2023log} systematically investigated the sensitive information stored in Android system logs. 
They discovered that many device identifiers and user behavior data were present in the logs, and that some high-privileged apps, including those from OEMs and large commercial companies, were collecting this information.
El-Rewini et al.~\cite{el2021dissecting} conducted a large-scale study on residual APIs—unused custom APIs left in the custom Android codebase. 
Analyzing 628 ROMs from seven major vendors, they found that these residual APIs are widespread and pose significant security risks, such as access control anomalies and potential exploitation.
These studies have explored specific types of issues introduced by system customizations, but none of them have involved the additional channels for obtaining device identifiers in custom systems.

Most relevant to our work, U2-I2~\cite{meng2023post} conducted a systematic study on system-level protection for non-resettable identifiers.
They revealed through small-scale dynamic analysis that in custom Android systems, many identifiers, including those in system properties and settings, are not adequately protected, potentially leading to leakage.
In addition to the seven well-known identifiers mentioned in~\S\ref{sec:background0}, they identified other non-resettable identifiers related to hardware (e.g., screen, NFC, and camera), which can remain constant after a factory reset and are similarly unprotected.
However, due to the limitations of dynamic methods, they are unable to perform large-scale analysis.
In contrast, our static approach requires only system ROMs, making it far more efficient and scalable. 
