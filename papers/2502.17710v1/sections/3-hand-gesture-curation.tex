\begin{table*}[h]
\centering
\small
\resizebox{2\columnwidth}{!}{
\begin{tabular}{lllll}
%\setlength{\tabcolsep}{1mm}
\toprule
\textbf{Gesture name} & \textbf{Country} & \textbf{Cultural Meaning} & \textbf{Specific Scenarios (to avoid)} & \textbf{Rating} \\
\midrule
Horns & Brazil & Infidelity & Professional meetings, formal events & Off/Obs (4/5) \\
Fig Sign & Indonesia & Female genitalia & All public spaces, workplace & Hate (1/5), Off/Obs (4/5) \\
Five Fathers & Saudi Arabia & Maternal insult & Family gatherings, business settings & Off/Obs (4/5) \\
Quenelle & France & Nazi-like salute & Public spaces, Jewish communities & Hate (4/5), Off/Obs (1/5) \\
Shocker & USA & Female objectification & Professional settings, mixed company & Off/Obs (5/5) \\
OK & Turkey & Homophobic & LGBTQ+ spaces, public forums & Hate (5/5) \\
\bottomrule
\end{tabular}
}
\caption{Examples of aggregated annotations from \offHandsDataset. Rating shows the number of annotators (out of 5) who assigned each label, where Off/Obs = Offensive/Obscene and Hate = Hateful.}
\label{tab:examples}
\vspace{-1.2em}
\end{table*}


\section{\offHandsDataset: Dataset Construction}

\label{sec:data}
We curate \offHandsDataset, a dataset focused on identifying and documenting gestures that may be considered offensive or inappropriate across different regions. We employ two approaches to collect data: (1) identifying \textit{offensive} gestures across different regions using documented online sources (\S\ref{sec:data:ssec:seed_curation}), and (2) identifying regions where gestures considered offensive in the US are \textit{not offensive} elsewhere, using LLM-generated suggestions  (\S\ref{sec:data:ssec:US_study}). All gesture-country pairs are human validated (\S\ref{sec:data:ssec:annotation_framework}). 



\subsection{Curating Offensive Gesture Data}
\label{sec:data:ssec:seed_curation}

We manually curated a set of 25 emblematic gestures\footnote{The 25 gestures are: ok gesture, thumbs up, fig sign, horns gesture, index finger pointing, forearm jerk, open palm, chin flick, pinched fingers, V sign, quenelle, Serbian salute, crossed fingers, middle finger, finger snapping, L sign, beckoning sign, using left hand, touching head, showing sole/feet, cutis, three-finger salute, five fathers, wanker, and shocker. Note: The `Hitler/Nazi Salute' was deliberately excluded as preliminary tests showed AI systems universally rejected its mention or description.} by consolidating information from numerous travel advisory boards, cultural exchange programs, workplace etiquette resources, and existing anthropological studies. These sources documented the countries where each gesture is considered offensive, resulting in 181 distinct culturally sensitive gestures-country pairs across 76 countries.\footnote{Full list of sources will be released with the dataset.} We use these country boundaries as proxies for culture, despite their limitations, following similar existing work in computational studies \cite{Wilson2016CulturalIO, jha2023seegull, romero2024cvqa}.


For each gesture, we extract the \textit{canonical name} from its corresponding Wikipedia page title and collect all \textit{alternate names} mentioned on the page, including those in English and other languages. We also record the \textit{physical description} provided on Wikipedia to ensure annotators can fully understand each gesture, even if a specific name is unfamiliar. To further support annotation, we collect two images per gesture (50 total) from Wikipedia, Wikimedia, and CC-BY-4.0 licensed sources, cropping each to focus on the gesture.


\subsection{Western-Centric Interpretations} 
\label{sec:data:ssec:US_study}
To investigate potential western-centric biases in AI systems \cite{stochasticparrot, prabhakaran2022cultural}, we collected offensiveness interpretations of all 25 gestures from USA and Canada.\footnote{We define `West' as `Northern American' subregion of UN geoscheme} %()\saadia{Why is Mexico excluded? I'm still hesitant about this use of "North America."}. 

To complement our initial set focused on gestures considered \textit{offensive} across different regions, we leveraged LLMs (\texttt{GPT-4} and \texttt{Claude 3.5 Sonnet}) to identify countries where gestures offensive in USA might be \textit{culturally acceptable} elsewhere. We used LLMs for such suggestions due to inherent reporting biases in human-curated sources, which predominantly document where gestures are unacceptable rather than explicitly listing where they are acceptable. Unsurprisingly, LLMs had low precision in suggesting such regions; however, this still helped identify regions where these gestures are not offensive, as well as additional countries where they are offensive, thus enriching our dataset.

Our final set comprises of 288 gesture-country pairs (43 from USA and Canada\footnote{7 gestures were offensive in USA from our initial set}, and 64 from LLMs) spanning across 25 gestures and 85 countries. We collect annotations for \textbf{all} of these pairs. 

\subsection{Annotator Regions}

Since collecting country-level annotations for each of the 85 countries would be prohibitively complex, we define cultural in-groups using the United Nations geoscheme's 22 geographical subregions.\footnote{Northern, Eastern, Middle, Southern, and Western Africa; Caribbean, Central and South America, and Northern America; Central, Eastern, South-eastern, Southern, and Western Asia; Eastern, Northern, Southern, and Western Europe; Australia and New Zealand; and Melanesia, Micronesia, and Polynesia are the 22 UN regions from \url{https://unstats.un.org/unsd/methodology/m49/}} This grouping provides finer granularity than continent-level, but more practical than country level. Within each in-group, we select annotators exclusively from countries represented in our dataset, ensuring cultural relevance while maintaining practical scalability. Our final set spans 18 of these subregions.






\subsection{Annotation Framework}
\label{sec:data:ssec:annotation_framework}
For each gesture-country pair, annotators were presented with the gesture name, alternate names, physical description, country name and 2 images of the gesture. The annotators provided:
%\saadia{can you use a specific example to illustrate what they would see for each textual input? Or a figure, reviewers often want to be shown annotation interfaces?}
\begin{enumerate}[itemsep=0pt, topsep=2pt,]
\item An \textbf{Offensiveness label} (Hateful, Offensive, Rude, Not Offensive, or Unsure)
\item \textbf{Confidence rating} on a 5-point Likert scale
\item \textbf{Free-text cultural meaning} of the gesture
\item \textbf{Specific contexts or scenarios} where the gesture is considered offensive or appropriate 
\end{enumerate}

The offensiveness scale categorizes gestures as: \textit{Hateful} (if hateful towards specific groups), \textit{Offensive/Obscene} (offensive and disturbing in general, but not targetting any group), \textit{Rude/Impolite/Inappropriate/Disrespectful} (minor transgressions, but best avoided), \textit{Not Offensive/Appropriate/No Meaning} (acceptable/neutral), or \textit{Unsure} (with justification).
Following prior work \citep{sap2019risk}, we instructed annotators to label whether gestures could be seen as offensive by others, considering religious and cultural significance, generational sensitivities, historical usage contexts, and minority perspectives, in contrast to asking if \textit{they} were offended themselves.


We recruited 268 annotators via Prolific\footnote{\url{https://www.prolific.com/}} from 18 UN geoscheme regions and 51 countries (112 female, 158 male, 2 undisclosed). Each annotator evaluated 5-7 gestures from their subregion, with 5 cultural in-group annotations per gesture-country pair. Details on the annotation scheme, IRB approval, and fair pay are in Appendix \ref{app:annotation_details}. 





\subsection{Dataset Characteristics}
\label{sec:data:char}
Our final dataset comprises \textbf{288 gesture-country pairs} spanning across \textbf{25 gestures} and \textbf{85 countries}, with an average of \textbf{4.89 annotations per pair}, yielding \textbf{a total of 1,408 annotations}.\footnote{Post filtering to remove spam annotations.} The most severe harm types identified are gender-based harassment (sexual harassment 7.64\%, infidelity 3.47\%) and discriminatory content (antisemitism 2.43\%, homophobia 2.08\%, white supremacy 1.04\%, ableism 0.69\%). The dataset also includes hostile behavior (11.11\%) and obscene gestures (9.38\%). Please refer to Appendix \ref{app:data_characteristics} for more examples, inter-annotator agreement, offensiveness ratings and confidence score distributions. 


\begin{table}[h]
\centering
\small
\begin{tabular}{@{}lcr@{}}
\toprule
\textbf{Category} & \textbf{Gesture-Country} & \textbf{Annotation} \\
\textbf{} & \textbf{Pairs} & \textbf{Tuples} \\
\midrule
Hateful & 57 & 285 \\
Offensive & 145 & 713 \\
Rude & 169 & 832 \\
Generally Off. & 221 & 1,087 \\
Not Offensive & 165 & 808 \\
\midrule
\textbf{Total} & \textbf{228} & \textbf{1408}  \\
\bottomrule
\end{tabular}
\caption{Dataset analysis by annotation category. We introduce a `\textit{Generally offensive}' category that groups all offensive-type annotations (hateful, offensive, or rude).}
\label{tab:gesture_summary}
\vspace{-1.5em}
\end{table}



\paragraph{Thresholding} 
Since interpretations of offensiveness are known to be subjective \cite{prabhakaran2021releasing, sap2022annotators, ross2017measuring}, we avoid majority voting. Instead, we use configurable thresholds $\theta_\text{category}>=n$, requiring at least $n$ annotators to mark a gesture-country pair in that category, following prior work \citep{bhatt2022re,jha2023seegull}.  Throughout the paper, we use \{$\theta_\text{Gen. Off}\geq3$ or $\theta_\text{Hateful}\geq1$\}, meaning a gesture-country pair is considered offensive if at least 3 annotators mark it as generally offensive or if at least 1 annotator marks it as hateful.\footnote{We use a lower threshold for hateful annotations because, while they provide valuable information about potential harm to specific groups, gestures rarely receive more than one hateful annotation (Figure \ref{fig:data_threshold} in Appendix \ref{app:annotation_details}).} Using this threshold, we find that $n=10/25$ gestures have benign or positive US Interpretations. 
Please refer to Appendix \ref{app:threshold_2} for similar results with different threshold of $\theta_\text{Gen. Off}=5$. 
