\section{Related Work and Discussion}

\paragraph{Nonverbal Behavior across Cultures}
Nonverbal behavior encompasses gestures, facial expressions, posture, proxemics (space use), haptics (touch), and vocalics (tone, pitch) \cite{Knapp1972NonverbalCI, Matsumoto2013NonverbalCS}--all of which vary significantly across cultures. In \textit{contact} cultures like  Latin America and the Middle East, people engage in closer proximity interactions than in Northern America or Northern Europe \cite{hall1963system, sorokowska2017preferred}; direct eye contact is encouraged in Western countries like France but considered disrespectful in parts of Asia, such as Japan \cite{argyle1994gaze}.  Gestures, in particular, pose a high risk of misinterpretation. 
They can be broadly classified into emblematic gestures--also known as symbolic gestures--which have distinct, culture-dependent meanings \cite{Matsumoto2012CulturalSA}, and co-verbal gestures (or speech illustrators), which accompany speech and follow more universal patterns \cite{mcneill1992hand}. Unlike co-verbal gestures, emblematic gestures function independently and are especially prone to cross-cultural misinterpretation \cite{matsumoto2013cultural, kendon2004gesture}. Our work focuses solely on emblematic gestures. 
%By studying emblematic gestures, we aim to advance understanding of cultural nuances in nonverbal AI safety.




\paragraph{Cultural Unawareness as a Safety Concern}
Current AI safety research primarily focuses on explicit threats like violence and NSFW content \cite{Rando2022RedTeamingTS, Schramowski2022SafeLD, Yang2023SneakyPromptJT, Liu2023MMSafetyBenchAB}, employing strategies such as safety training \cite{Huang2023ASO, Shen2023DoAN}, red-teaming \cite{Ganguli2022RedTL, Liu2024ArondightRT, Ge2023MARTIL}, safety modules \cite{touvron2023llama, liu2024safety}, and risk taxonomies \cite{wang2023not, Brahman2024TheAO, vidgen2024introducing}. However, they often overlook cultural contexts \cite{sambasivan2021re}, as demonstrated by our findings of widespread cultural unawareness in current AI systems.



\paragraph{Western-Centric Biases in AI Systems}
AI systems exhibit Western-centric biases \cite{stochasticparrot, Masoud2023CulturalAI, prabhakaran2022cultural}, favoring Western perspectives while misinterpreting or underrepresenting non-Western cultural elements \cite{bhatt2022re, zhou2022richer, basu2023inspecting}. Our results align with these observations -- all evaluated models show better detection of US-offensive gestures compared to those offensive in other cultures. 
These skews likely stem from biased training data \cite{ferrara2023fairness, suresh2021framework} and problematic AI development practices \cite{mehrabi2021survey, belenguer2022ai}. Potential mitigation strategies include finetuning on culturally-specific datasets \cite{dwivedi2023eticor, li2024culturellm}, and increased participation of local experts in model development \cite{Kirk2024ThePA}. 




\paragraph{Contextual Reasoning for Cultural Norms}
Visual interpretation of cultural norms, particularly non-verbal gestures, presents unique challenges compared to traditional offensive content detection. While both language \cite{Gehman2020RealToxicityPromptsEN, Jain2024PolygloToxicityPromptsME} and visual \cite{Arora2023ADAMAXBasedOO, Shidaganti2023DeepLD} safety systems rely on large-scale curated datasets, gesture interpretation requires nuanced cultural understanding. Recent work suggests contextual information can improve offensive content detection \cite{zhou2023cobra, yerukola2024pope}. However, our Country+Scene evaluation reveals that additional scene context had no effect on LLMs and actually degraded T2I and VLM performance, highlighting fundamental limitations in current cross-modal contextual reasoning approaches.

