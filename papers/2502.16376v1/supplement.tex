%File: formatting-instructions-latex-2025.tex
%release 2025.0
\documentclass[letterpaper]{article} % DO NOT CHANGE THIS
\usepackage{aaai25}  % DO NOT CHANGE THIS
\usepackage{times}  % DO NOT CHANGE THIS
\usepackage{helvet}  % DO NOT CHANGE THIS
\usepackage{courier}  % DO NOT CHANGE THIS
\usepackage[hyphens]{url}  % DO NOT CHANGE THIS
\usepackage{graphicx} % DO NOT CHANGE THIS
\urlstyle{rm} % DO NOT CHANGE THIS
\def\UrlFont{\rm}  % DO NOT CHANGE THIS
\usepackage{natbib}  % DO NOT CHANGE THIS AND DO NOT ADD ANY OPTIONS TO IT
\usepackage{caption} % DO NOT CHANGE THIS AND DO NOT ADD ANY OPTIONS TO IT
\frenchspacing  % DO NOT CHANGE THIS
\setlength{\pdfpagewidth}{8.5in}  % DO NOT CHANGE THIS
\setlength{\pdfpageheight}{11in}  % DO NOT CHANGE THIS
%
% These are recommended to typeset algorithms but not required. See the subsubsection on algorithms. Remove them if you don't have algorithms in your paper.
\usepackage{algorithmic}
\usepackage{threeparttable}
\usepackage{diagbox}
\usepackage{multirow}
\usepackage{booktabs} % For better table formatting
\usepackage{subcaption} % For subtable environment
%%%%%%%%%%%---SETME-----%%%%%%%%%%%%%
%replace @@ with the submission number submission site.
\newcommand{\thiswork}{INF$^2$\xspace}
%%%%%%%%%%%%%%%%%%%%%%%%%%%%%%%%%%%%


%\newcommand{\rev}[1]{{\color{olivegreen}#1}}
\newcommand{\rev}[1]{{#1}}


\newcommand{\JL}[1]{{\color{cyan}[\textbf{\sc JLee}: \textit{#1}]}}
\newcommand{\JW}[1]{{\color{orange}[\textbf{\sc JJung}: \textit{#1}]}}
\newcommand{\JY}[1]{{\color{blue(ncs)}[\textbf{\sc JSong}: \textit{#1}]}}
\newcommand{\HS}[1]{{\color{magenta}[\textbf{\sc HJang}: \textit{#1}]}}
\newcommand{\CS}[1]{{\color{navy}[\textbf{\sc CShin}: \textit{#1}]}}
\newcommand{\SN}[1]{{\color{olive}[\textbf{\sc SNoh}: \textit{#1}]}}

%\def\final{}   % uncomment this for the submission version
\ifdefined\final
\renewcommand{\JL}[1]{}
\renewcommand{\JW}[1]{}
\renewcommand{\JY}[1]{}
\renewcommand{\HS}[1]{}
\renewcommand{\CS}[1]{}
\renewcommand{\SN}[1]{}
\fi

%%% Notion for baseline approaches %%% 
\newcommand{\baseline}{offloading-based batched inference\xspace}
\newcommand{\Baseline}{Offloading-based batched inference\xspace}


\newcommand{\ans}{attention-near storage\xspace}
\newcommand{\Ans}{Attention-near storage\xspace}
\newcommand{\ANS}{Attention-Near Storage\xspace}

\newcommand{\wb}{delayed KV cache writeback\xspace}
\newcommand{\Wb}{Delayed KV cache writeback\xspace}
\newcommand{\WB}{Delayed KV Cache Writeback\xspace}

\newcommand{\xcache}{X-cache\xspace}
\newcommand{\XCACHE}{X-Cache\xspace}


%%% Notions for our methods %%%
\newcommand{\schemea}{\textbf{Expanding supported maximum sequence length with optimized performance}\xspace}
\newcommand{\Schemea}{\textbf{Expanding supported maximum sequence length with optimized performance}\xspace}

\newcommand{\schemeb}{\textbf{Optimizing the storage device performance}\xspace}
\newcommand{\Schemeb}{\textbf{Optimizing the storage device performance}\xspace}

\newcommand{\schemec}{\textbf{Orthogonally supporting Compression Techniques}\xspace}
\newcommand{\Schemec}{\textbf{Orthogonally supporting Compression Techniques}\xspace}



% Circular numbers
\usepackage{tikz}
\newcommand*\circled[1]{\tikz[baseline=(char.base)]{
            \node[shape=circle,draw,inner sep=0.4pt] (char) {#1};}}

\newcommand*\bcircled[1]{\tikz[baseline=(char.base)]{
            \node[shape=circle,draw,inner sep=0.4pt, fill=black, text=white] (char) {#1};}}

%
% These are are recommended to typeset listings but not required. See the subsubsection on listing. Remove this block if you don't have listings in your paper.
\usepackage{newfloat}
\usepackage{listings}
\DeclareCaptionStyle{ruled}{labelfont=normalfont,labelsep=colon,strut=off} % DO NOT CHANGE THIS
\lstset{%
	basicstyle={\footnotesize\ttfamily},% footnotesize acceptable for monospace
	numbers=left,numberstyle=\footnotesize,xleftmargin=2em,% show line numbers, remove this entire line if you don't want the numbers.
	aboveskip=0pt,belowskip=0pt,%
	showstringspaces=false,tabsize=2,breaklines=true}
\floatstyle{ruled}
\newfloat{listing}{tb}{lst}{}
\floatname{listing}{Listing}
%
% Keep the \pdfinfo as shown here. There's no need
% for you to add the /Title and /Author tags.
\pdfinfo{
/TemplateVersion (2025.1)
}

% DISALLOWED PACKAGES
% \usepackage{authblk} -- This package is specifically forbidden
% \usepackage{balance} -- This package is specifically forbidden
% \usepackage{color (if used in text)
% \usepackage{CJK} -- This package is specifically forbidden
% \usepackage{float} -- This package is specifically forbidden
% \usepackage{flushend} -- This package is specifically forbidden
% \usepackage{fontenc} -- This package is specifically forbidden
% \usepackage{fullpage} -- This package is specifically forbidden
% \usepackage{geometry} -- This package is specifically forbidden
% \usepackage{grffile} -- This package is specifically forbidden
% \usepackage{hyperref} -- This package is specifically forbidden
% \usepackage{navigator} -- This package is specifically forbidden
% (or any other package that embeds links such as navigator or hyperref)
% \indentfirst} -- This package is specifically forbidden
% \layout} -- This package is specifically forbidden
% \multicol} -- This package is specifically forbidden
% \nameref} -- This package is specifically forbidden
% \usepackage{savetrees} -- This package is specifically forbidden
% \usepackage{setspace} -- This package is specifically forbidden
% \usepackage{stfloats} -- This package is specifically forbidden
% \usepackage{tabu} -- This package is specifically forbidden
% \usepackage{titlesec} -- This package is specifically forbidden
% \usepackage{tocbibind} -- This package is specifically forbidden
% \usepackage{ulem} -- This package is specifically forbidden
% \usepackage{wrapfig} -- This package is specifically forbidden
% DISALLOWED COMMANDS
% \nocopyright -- Your paper will not be published if you use this command
% \addtolength -- This command may not be used
% \balance -- This command may not be used
% \baselinestretch -- Your paper will not be published if you use this command
% \clearpage -- No page breaks of any kind may be used for the final version of your paper
% \columnsep -- This command may not be used
% \newpage -- No page breaks of any kind may be used for the final version of your paper
% \pagebreak -- No page breaks of any kind may be used for the final version of your paperr
% \pagestyle -- This command may not be used
% \tiny -- This is not an acceptable font size.
% \vspace{- -- No negative value may be used in proximity of a caption, figure, table, section, subsection, subsubsection, or reference
% \vskip{- -- No negative value may be used to alter spacing above or below a caption, figure, table, section, subsection, subsubsection, or reference

\setcounter{secnumdepth}{0} %May be changed to 1 or 2 if section numbers are desired.

% The file aaai25.sty is the style file for AAAI Press
% proceedings, working notes, and technical reports.
%

% Title

% Your title must be in mixed case, not sentence case.
% That means all verbs (including short verbs like be, is, using,and go),
% nouns, adverbs, adjectives should be capitalized, including both words in hyphenated terms, while
% articles, conjunctions, and prepositions are lower case unless they
% directly follow a colon or long dash
\title{Does Your AI Agent Get You? A Personalizable Framework for Approximating Human Models from Argumentation-based Dialogue Traces}
\author{
    Supplement}
\affiliations{
    %Afiliations
    % \textsuperscript{\rm 1}Association for the Advancement of Artificial Intelligence\\
    % If you have multiple authors and multiple affiliations
    % use superscripts in text and roman font to identify them.
    % For example,

    % Sunil Issar\textsuperscript{\rm 2},
    % J. Scott Penberthy\textsuperscript{\rm 3},
    % George Ferguson\textsuperscript{\rm 4},
    % Hans Guesgen\textsuperscript{\rm 5}
    % Note that the comma should be placed after the superscript
    % Washington University in St. Louis\\
    % email address must be in roman text type, not monospace or sans serif
    % \{t.yinxu, v.stylianos, wyeoh\}@wustl.edu
%
% See more examples next
}

%Example, Single Author, ->> remove \iffalse,\fi and place them surrounding AAAI title to use it
\iffalse
\title{My Publication Title --- Single Author}
\author {
    Author Name
}
\affiliations{
    Affiliation\\
    Affiliation Line 2\\
    name@example.com
}
\fi

\iffalse
%Example, Multiple Authors, ->> remove \iffalse,\fi and place them surrounding AAAI title to use it
\title{My Publication Title --- Multiple Authors}
\author {
    % Authors
    First Author Name\textsuperscript{\rm 1,\rm 2},
    Second Author Name\textsuperscript{\rm 2},
    Third Author Name\textsuperscript{\rm 1}
}
\affiliations {
    % Affiliations
    \textsuperscript{\rm 1}Affiliation 1\\
    \textsuperscript{\rm 2}Affiliation 2\\
    firstAuthor@affiliation1.com, secondAuthor@affilation2.com, thirdAuthor@affiliation1.com
}
\fi


% REMOVE THIS: bibentry
% This is only needed to show inline citations in the guidelines document. You should not need it and can safely delete it.
\usepackage{bibentry}
% END REMOVE bibentry

\begin{document}

\maketitle


\subsection{Baselines in Comparative Evaluation}

\squishlist
\item \textit{$H\!M_1$}: An argumentation-based method for updating probability distributions of human models based on argument graphs \cite{hunter2015modelling}.
Inspired by the redistribution function, we apply this concept to our model distribution update.  

At each time step $t_i$, when an argument $A_i$ is presented by either the agent or the human, we perform the following naive update on the probability distribution: 
    \begin{equation}
	P^{t_i}_{h}(m)= 
	\begin{cases}
		P^{t_i - 1}_{h}(m)+  P^{t_i - 1}_{h}\left(h_{A_i}(m)\right) & \text { if } m \models A_i \\ 
		0 & \text { if } m \not \models A_i
	\end{cases}
	\label{eq:redistribution}
\end{equation}
where $h_{A_i}(m)=m \backslash\{\alpha\}$ and $\alpha$ is of the form $A_i$. 

\item \textit{$H\!M_2$}: An enhanced version of Hunter's $H\!M_1$ that utilizes the argument structure for updating the distribution \cite{hunter2015modelling}. Specifically, consider an argument graph $G$ where $\operatorname{Attacks}(G)$ represents the set of attack relations in $G$. For instance, if $A_1 = \langle \{a\}, \{a\} \rangle$ and $A_2 =  \langle \{b, b \rightarrow \neg a\}, \{\neg a\} \rangle$, $A_2$ is a counterargument of $A_1$, indicating that $(A_2, A_1) \in \operatorname{Attacks}(G)$. In this way, this method first applies Equation~\eqref{eq:redistribution} and then proceeds with the following update:
     \begin{equation}
     P^{t_i}_{h}(m)= 
     \begin{cases}
     P^{t_i}_{h}(m)+  P^{t_i }_{h}\left(h_{\Phi}(m)\right) & \text { if } m \models \Phi \\ 
     0 & \text { if } m \not \models \Phi
     \end{cases}
     \end{equation}
     where $\Phi = \{ \neg B \mid (B, A_i) \in \operatorname{Attacks}(G) \text{ or } (A_i, B) \in \operatorname{Attacks}(G)\}$.
    
\item \textit{$H\!A$}: A state-of-the-art method for learning probability distributions of arguments by \citet{hunter2016persuasion}. 
This baseline method updates the belief in each argument throughout the dialogue by considering the initial probability of each argument and the human's confidence in their arguments. 
Specifically, for each argument $A_i$, the final distribution is:
\begin{equation}
      P(A_i)\!\!= \begin{cases}
      0.2 & x_i = \alpha, \exists B \in \operatorname{Opp}(A_i), P(B)>0.5 \\ 
      0.2 & x_i = \eta, \exists B \in \operatorname{Pro}(A_i), P(B)>0.5 \\ 
      0.8 & x_i = \alpha, \forall B \in \operatorname{Opp}(A_i), P(B) \leq 0.5 \\ 
      \sigma_i & x_i = \eta, \forall B \in \operatorname{Pro}(A_i), P(B) \leq 0.5
      \end{cases}
      \end{equation}
where $\operatorname{Opp}(A_i)= \{A_{i+1} \mid \exists i, x_{i+1} = \eta\}$ and $\operatorname{Pro}(A_i) = \{A_j \mid \exists j, i<j, x_j = \alpha, (A_j, A_i) \in  \operatorname{Attacks}(G)\}$.
\squishend

\begin{example}
Suppose there are four possible models, $\mathcal{M} = \{m_1, m_2, m_3, m_4\}$, with a uniform prior distribution $P^{t_0}_h(m_1) = P^{t_0}_h(m_2) = P^{t_0}_h(m_3) = P^{t_0}_h(m_4) = 0.25$. At time step $t_1$, the agent asserts the argument $A_1 = \langle \{a\}, \{a\} \rangle$. At the next timestep $t_2$, the human presents the argument $A_2 =  \langle \{b, b \rightarrow \neg a\}, \{\neg a\} \rangle$ with confidence value $\sigma_2 = 0.6$. Applying the redistribution mechanisms in $H\!M_1$ and $H\!M_2$ in timesteps $t_1$ and $t_2$, respectively, the results are shown in Table \ref{table:example_of_hunter}. By applying $H\!A$ after timestep $t_2$, we have $P(A_1) = 0.2$ and $P(A_2) = 0.6$.
\begin{table}[t]
\centering \small
\begin{tabular}{ccccc}
\toprule[2pt]
& $m_1$ & $m_2$ & $m_3$ & $m_4$ \\ \midrule[2pt]
$a$                 & True  & True  & False & False \\ [1ex]
$b$                 & True  & False & True  & False \\ [1ex]
$P^{t_0}_{h}(m_i)$    & $0.25$   & $0.25$   & $0.25$   & $0.25$  \\ [1ex]
$P^{t_1}_{h}(m_i)$      & $0.5$   & $0.5$   &  $0$    &  $0$     \\
$P^{t_2}_{h}(m_i)$      & $0$   & $0$   &  $1$    &  $0$     \\
\bottomrule[2pt]
\end{tabular}
\caption{An example of the baseline methods.}
\vspace{-1em}
\label{table:example_of_hunter}
\end{table}
\end{example}

\subsection{Details of Human Model Approximation}


We compared the Spearman’s rank correlation distributions in round $k = \{2,3,4,5\}$ of human model rankings where parameters are learned from the previous $k-1$ rounds among Persona and its ablations and the two baselines. Table \ref{table:overall_model_comparison} displays the distribution of Spearman's rank correlation coefficients in each round, showing that Persona performed better than all the other methods in all rounds. Compared to \textit{Generic} and \textit{SBU}, the results demonstrate that incorporating both personalization and the weighting function increases the accuracy of model approximation. Notably, by considering the distribution within $[0.25, 1]$, Persona significantly outperformed $H\!M_1$ and $H\!M_2$ across all rounds.


We also performed paired Student's $t$-tests to compare various methods, with Table \ref{table:all_rounds_p_value} displaying the $p$-values that assess the hypothesis that method $X$ outperforms method $Y$ in human model approximation across different rounds. The results show that Persona consistently and statistically significantly outperforms all other methods in almost all rounds, with the exception of Round 5. The reason is that while Persona does better than \textit{Generic} and \textit{SBU} in the high correlation range, they both do better in the medium positive correlation range of $[0.25, 0.75)$ shown in Table \ref{table:round5_model_comparison}. However, the improvements of Persona and its ablation variants \textit{Generic} and \textit{SBU} over state-of-the-art baselines in every round are statistically significant, with $p$-values smaller than 0.05. 
These findings demonstrate Persona's ability to leverage existing data to personalize parameters, thereby enhancing human model estimation accuracy in subsequent rounds beyond state-of-the-art baselines and ablation variants. 
Notably, even the non-personalized ablation variants consistently outperform all baselines, further validating our approach.

\vspace{-1.5em}

\begin{table}[H]
\centering 
\begin{subtable}[t]{\columnwidth}
\centering
\resizebox{1.\columnwidth}{!}{
\begin{tabular}{cccccc}
\toprule[2pt]
Round $2$ & \textit{Persona} & \textit{Generic} & \textit{SBU} & \textit{$H\!M_1$} & \textit{$H\!M_2$}\\ \midrule[2pt] 
$[-1, -0.75)$ & $0.059$    & $0.065$  & $0.065$ & $0.141$ & $0.152$ \\ [1ex]
$[-0.75, -0.25)$ &  $0.147$  & $0.141$  & $0.152$ & $0.185$ & $0.130$   \\ [1ex]
$[-0.25, 0.25)$  &   $0.233$ & $0.234$   & $0.207$ & $0.266$ & $0.234$  \\ [1ex]
$[0.25, 0.75)$ & $0.185$ & $0.207$ & $0.250$ & $0.147$ & $0.239$ \\ [1ex]
$[0.75, 1]$ & $0.375$ & $0.353$ & $0.326$ & $0.261$ & $0.245$ \\ \bottomrule[2pt]
\end{tabular}
}
\caption{Comparison of model estimation in Round 2.}
\label{table:round2_model_comparison}
\end{subtable}

\vspace{0.2em} % Adds some vertical space between subtables

\begin{subtable}[t]{\columnwidth}
\centering
\resizebox{1.\columnwidth}{!}{
\begin{tabular}{cccccc}
\toprule[2pt]
Round $3$ & \textit{Persona} & \textit{Generic} & \textit{SBU} & \textit{$H\!M_1$} & \textit{$H\!M_2$}\\ \midrule[2pt] 
$[-1, -0.75)$ & $0.059$    & $0.076$  & $0.076$ & $0.087$ & $0.081$ \\ [1ex]
$[-0.75, -0.25)$ &  $0.114$  & $0.120$  & $0.114$ & $0.179$ & $0.163$   \\ [1ex]
$[-0.25, 0.25)$  &   $0.217$ & $0.212$   & $0.234$ & $0.223$ & $0.245$  \\ [1ex]
$[0.25, 0.75)$ & $0.217$ & $0.234$ & $0.223$ & $0.261$ & $0.245$ \\ [1ex]
$[0.75, 1]$ & $0.391$ & $0.359$ & $0.353$ & $0.250$ & $0.266$ \\ \bottomrule[2pt]
\end{tabular}}
\caption{Comparison of model estimation in Round 3.}
\label{table:round3_model_comparison}
\end{subtable}

\vspace{0.2em}

\begin{subtable}[t]{\columnwidth}
\centering
\resizebox{1.\columnwidth}{!}{
\begin{tabular}{cccccc}
\toprule[2pt]
Round $4$ & \textit{Persona} & \textit{Generic} & \textit{SBU} & \textit{$H\!M_1$} & \textit{$H\!M_2$}\\ \midrule[2pt] 
$[-1, -0.75)$ & $0.049$    & $0.054$  & $0.054$ & $0.087$ & $0.071$ \\ [1ex]
$[-0.75, -0.25)$ &  $0.130$  & $0.130$  & $0.147$ & $0.158$ & $0.168$   \\ [1ex]
$[-0.25, 0.25)$  &   $0.152$ & $0.158$   & $0.152$ & $0.293$ & $0.185$  \\ [1ex]
$[0.25, 0.75)$ & $0.261$ & $0.321$ & $0.266$ & $0.217$ & $0.212$ \\ [1ex]
$[0.75, 1]$ & $0.408$ & $0.337$ & $0.380$ & $0.245$ & $0.364$ \\ \bottomrule[2pt]
\end{tabular}}
\caption{Comparison of model estimation in Round 4.}
\label{table:round4_model_comparison}
\end{subtable}

\vspace{0.2em}

\begin{subtable}[t]{\columnwidth}
\centering
\resizebox{1.\columnwidth}{!}{
\begin{tabular}{cccccc}
\toprule[2pt]
Round $5$ & \textit{Persona} & \textit{Generic} & \textit{SBU} & \textit{$H\!M_1$} & \textit{$H\!M_2$}\\ \midrule[2pt] 
$[-1, -0.75)$ & $0.060$    & $0.060$  & $0.060$ & $0.043$ & $0.065$ \\ [1ex]
$[-0.75, -0.25)$ &  $0.076$  & $0.081$  & $0.081$ & $0.136$ & $0.136$   \\ [1ex]
$[-0.25, 0.25)$  &   $0.147$ & $0.119$   & $0.130$ & $0.207$ & $0.185$  \\ [1ex]
$[0.25, 0.75)$ & $0.250$ & $0.288$ & $0.277$ & $0.277$ & $0.255$ \\ [1ex]
$[0.75, 1]$ & $0.467$ & $0.451$ & $0.451$ & $0.337$ & $0.359$ \\ \bottomrule[2pt]
\end{tabular}}
\caption{Comparison of model estimation in Round 5.}
\vspace{-0.5em}
\label{table:round5_model_comparison}
\end{subtable}

\caption{The distributions of Spearman’s rank correlation coefficients in model approximation in Round $k$ ($k = 2, 3, 4, 5$) of human model rankings where parameters are learned from the first $k-1$ rounds. Note that for participants with only four interactions, the results for Round 5 are identical to those of Round 4.}
\label{table:overall_model_comparison}


\end{table}



% Our user study demonstrates the effectiveness of our proposed framework in argumentation-based dialogue scenarios. The results highlight the framework's ability to personalize human model approximation as a probability distribution and estimate argument beliefs, leading to increased participant confidence. Additionally, post-study questionnaire responses corroborate these findings, with participants expressing high satisfaction with the interaction and the quality of Blitzcrank's explanations, as shown in Table~\ref{table:comprehension_and_satisfaction}.

\subsection{Post-study Results}
In our scenarios, participants were divided into Group A, who ended the conversation themselves, and Group B, where Blitzcrank ended the dialogue. Group A confirmed confidence levels across four rounds, while Group B did so over five rounds. The results in Table~\ref{table:t-test_confidence} provide compelling evidence that the confidence in the AI assistant increases, indicating that participants' confidence grows as the dialogue progresses and the assistant provides more relevant and persuasive arguments.

Finally, Table~\ref{table:comprehension_and_satisfaction} shows the post-study questionnaire responses further corroborate these findings, with participants reporting high levels of satisfaction with the interaction and the quality of Blitzcrank's arguments.




\begin{table}[H]
\centering
\begin{subtable}[t]{\columnwidth}
\centering
\resizebox{1.\columnwidth}{!}{ 
\begin{tabular}{cccccc}
\toprule[2pt]
\diagbox{$X$}{Round 2}{$Y$}& \textit{Persona} & \textit{Generic} & \textit{SBU} & \textit{$H\!M_1$} & \textit{$H\!M_2$} \\ \midrule[2pt]
\textit{Persona} & -- & $0.044$ & $0.047$ & $2.408 \times 10^{-6}$ & $5.730 \times 10^{-5}$ \\ [1ex]
\textit{Generic}   & $0.956$    & --  & $0.187$ &  $8.210 \times 10^{-6}$ & $1.907 \times 10^{-4}$\\ [1ex]
\textit{SBU}    &  $0.953$  & $0.813$  & -- & $4.685 \times 10^{-5}$  & $5.659 \times 10^{-4}$ \\ [1ex]
\textit{$H\!M_1$}     & $1$   & $1$  & $1$  & -- & $0.908$ \\ [1ex]
\textit{$H\!M_2$}   &  $1$   & $1$  & $0.999$ & $0.092$ & --  \\ \bottomrule[2pt]
\end{tabular}
}
\caption{The $p$-values that $X$ outperforms $Y$ in Round 2.}
\label{table:round2_p_value}
\end{subtable}

\vspace{0.2em}

\begin{subtable}[t]{\columnwidth}
\centering
\resizebox{1.\columnwidth}{!}{ 
\begin{tabular}{cccccc}
\toprule[2pt]
\diagbox{$X$}{Round 3}{$Y$}& \textit{Persona} & \textit{Generic} & \textit{SBU} & \textit{$H\!M_1$} & \textit{$H\!M_2$} \\ \midrule[2pt]
\textit{Persona} & -- & $0.004$ & $0.004$ & $3.254 \times 10^{-4}$ & $0.002$ \\ [1ex]
\textit{Generic}   & $0.996$    & --  & $0.231$ &  $0.003$ & $0.021$\\ [1ex]
\textit{SBU}    &  $0.997$  & $0.769$  & -- & $0.005$  & $0.026$ \\ [1ex]
\textit{$H\!M_1$}     & $1$   & $0.997$  & $0.995$  & -- & $0.776$ \\ [1ex]
\textit{$H\!M_2$}   &  $0.998$   & $0.979$  & $0.974$ & $0.224$ & --  \\ \bottomrule[2pt]
\end{tabular}
}
\caption{The $p$-values that $X$ outperforms $Y$ in Round 3.}
\label{table:round3_p_value}
\end{subtable}

\vspace{0.2em}


\begin{subtable}[t]{\columnwidth}
\centering
\resizebox{1.\columnwidth}{!}{ 
\begin{tabular}{cccccc}
\toprule[2pt]
\diagbox{$X$}{Round 4}{$Y$}& \textit{Persona} & \textit{Generic} & \textit{SBU} & \textit{$H\!M_1$} & \textit{$H\!M_2$} \\ \midrule[2pt]
\textit{Persona} & -- & $0.010$ & $0.047$ & $3.760 \times 10^{-5}$ & $0.006$ \\ [1ex]
\textit{Generic}   & $0.990$    & --  & $0.748$ &  $6.182 \times 10^{-4}$ & $0.030$\\ [1ex]
\textit{SBU}    &  $0.953$  & $0.252$  & -- & $5.905 \times 10^{-4}$  & $0.022$ \\ [1ex]
\textit{$H\!M_1$}     & $1$   & $1$  & $1$  & -- & $0.916$ \\ [1ex]
\textit{$H\!M_2$}   &  $0.994$   & $0.969$  & $0.978$ & $0.084$ & --  \\ \bottomrule[2pt]
\end{tabular}
}
\caption{The $p$-values that $X$ outperforms $Y$ in Round 4.}
\label{table:round4_p_value}
\end{subtable}

\vspace{0.2em}
\begin{subtable}[t]{\columnwidth}
\centering
\resizebox{1.\columnwidth}{!}{ 
\begin{tabular}{cccccc}
\toprule[2pt]
\diagbox{$X$}{Round 5}{$Y$}& \textit{Persona} & \textit{Generic} & \textit{SBU} & \textit{$H\!M_1$} & \textit{$H\!M_2$} \\ \midrule[2pt]
\textit{Persona} & -- & $0.640$ & $0.426$ & $0.006$ & $0.001$ \\ [1ex]
\textit{Generic}   & $0.360$    & --  & $0.153$ &  $0.003$ & $5.790 \times 10^{-4}$\\ [1ex]
\textit{SBU}    &  $0.574$  & $0.867$  & -- & $0.007$  & $0.001$ \\ [1ex]
\textit{$H\!M_1$}     & $0.994$   & $0.997$  & $0.993$  & -- & $0.225$ \\ [1ex]
\textit{$H\!M_2$}   &  $0.999$   & $0.999$  & $0.999$ & $0.775$ & --  \\ \bottomrule[2pt]
\end{tabular}
}
\caption{The $p$-values that $X$ outperforms $Y$ in Round 5.}
\label{table:round5_p_value}
\end{subtable}

\caption{The $p$-values from paired Student’s t-tests assessing the
hypothesis that $X$ outperforms $Y$ in round $k$ ($k = 2,3,4,5$) in Experiment 2.1}
\label{table:all_rounds_p_value}
\end{table}


\begin{table}[H]
\centering
\resizebox{1.\columnwidth}{!}{ 
\begin{tabular}{lcc}
\toprule[2pt]
& Group A (Four rounds) & Group B (Five rounds) \\
\midrule[2pt]
$p_{1,2}$ & $4.199 \times 10^{-15}$ & $1.774 \times 10^{-20}$ \\
$p_{2,3}$ & $3.412 \times 10^{-5}$ & $0.005$ \\
$p_{3,4}$ & $0.02$ & $0.016$ \\
$p_{4,5}$ & -- & $0.043$ \\
% \midrule
% \textbf{Stat. Significance} & \multicolumn{2}{c}{$p < 0.05$ indicates significance} \\
\bottomrule[2pt]
\end{tabular}
}
\caption{The $p$-values of comparing confidence values between interaction rounds. Specifically, $p_{i,j}$ indicates the $p$-value for the hypothesis that the confidence increases from round $i$ to round $j$. }
\label{table:t-test_confidence}
\end{table}

\begin{table}[H]
\centering \small
\begin{tabular}{cc}
\toprule[2pt]
                               & All Participants \\ \midrule[2pt]
Comprehension Score (out of 5) & 3.32             \\ [1ex]
Satisfaction Score (out of 5)  & 3.12             \\ \bottomrule[2pt]
\end{tabular}
\caption{Comprehension score and satisfaction score.}
\label{table:comprehension_and_satisfaction}
\end{table}

\bibliography{aaai25}

\end{document}