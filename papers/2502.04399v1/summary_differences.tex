\documentclass[11pt]{letter} % Default font size of the document, change to 10pt to fit more text


\usepackage{newcent} % Default font is the New Century Schoolbook PostScript font 
%\usepackage{helvet} % Uncomment this (while commenting the above line) to use the Helvetica font

% Margins
\topmargin=-1in % Moves the top of the document 1 inch above the default
\textheight=8.5in % Total height of the text on the page before text goes on to the next page, this can be increased in a longer letter
\oddsidemargin=-10pt % Position of the left margin, can be negative or positive if you want more or less room
\textwidth=6.5in % Total width of the text, increase this if the left margin was decreased and vice-versa

\let\raggedleft\raggedright % Pushes the date (at the top) to the left, comment this line to have the date on the right
\usepackage{enumitem}
\usepackage{url}

\begin{document}

%----------------------------------------------------------------------------------------
%	ADDRESSEE SECTION
%----------------------------------------------------------------------------------------

\begin{letter}{
% Prof.~Rolf Stadler, Prof.~Prosper Chemouil, Prof.~Pan Hui,\\
% Prof.~Noura Limam, Prof.~Wolfgang Kellerer, Prof.~Yonggang Wen\\
% Guest editors of IEEE Journal on Selected Areas in Communications,\\
% Special Issue on Advances in Artificial Intelligence and Machine Learning for Networking \\
% % 123 Pleasant Lane \\
% % City, State 12345}
}

%----------------------------------------------------------------------------------------
%	YOUR NAME & ADDRESS SECTION
%----------------------------------------------------------------------------------------

\begin{center}
\Large Online Location Planning for AI-Defined Vehicles: Optimizing Joint Tasks of Order Serving and Spatio-Temporal Heterogeneous Model Fine-Tuning\\
\vspace{5mm}
{\large

        Boken Zheng, Bo Rao, Tianxiang Zhu, 
        Chee~Wei~Tan, ~{\em Member,~IEEE,}
        Jingpu~Duan,~{\em Member,~IEEE,}
        Zhou~Zhi,~{\em Member,~IEEE,}
        Xu~Chen,~{\em Senior Member,~IEEE,}
        and Xiaoxi~Zhang, ~{\em Member,~IEEE}% <-this % stops a space
}
% \large\bf Dr. Xiaoxi Zhang \\ % Your name
% %\vspace{20pt} \hrule height 1pt % If you would like a horizontal line separating the name from the address, uncomment the line to the left of this text
% Carnegie Mellon University, NASA Research Park \\ Moffett Field, CA 94035 \\ (878) 999-3355 % Your address and phone number
\end{center} 
%\vfill




%\signature{Xiaoxi Zhang} % Your name for the signature at the bottom

%----------------------------------------------------------------------------------------
%	LETTER CONTENT SECTION
%----------------------------------------------------------------------------------------

\opening{Dear Editors,} 
 
I am delighted to submit an extended research article entitled ``Online Location Planning for AI-Defined Vehicles: Optimizing Joint Tasks of Order Serving and Spatio-Temporal Heterogeneous Model Fine-Tuning'' for publication in the IEEE Transactions on Mobile Computing. 

I confirm that this work is currently not under consideration for publication elsewhere, and that a conference version of this paper appears at the IEEE/ACM International Symposium on Quality of Service (IWQoS) 2024.
%which is cited as reference [1] in this submission and can be found at \url{https://arxiv.org/abs/2003.05649}. 

The major differences between this submission and the IWQoS paper are:

\begin{enumerate}[label=(\arabic*)]
    % \item Our IWQoS paper exclusively delved into the single-agent version of the proposed algorithm, focusing on one Micro Grid (MG) at a time. In contrast, we extend our approach to the multi-agent scenario, wherein multiple MGs interact with each other concurrently. We also make the scenario more practical compared with the conference version, by considering EV's uncertain mobility across mutiple MGs and incorporate it into drivers' reaction functions.
\item Our IWQoS paper was a short paper, similar to an extended abstract. It was initially concise, lacking in details. However, we have carefully expanded the content of the article, adding a wealth of additional details that are easily comprehensible, resulting in a more coherent and logical flow of ideas. Our problem model has been significantly expanded, taking into account model fine-tuning tasks with spatio-temporal heterogeneity (as highlighted in the sections of introduction and Problem Formulation). The previous IWQoS paper only considered two tasks: vehicle order serving and data collection in the context of the Internet of Vehicles, and in fact, we did not consider how to use this data. In this paper, we indicate the purpose of these data - for model fine-tuning. In the current problem scenario, we have studied the impact of data freshness and data volume on the accuracy and runtime of model fine-tuning. Compared to the IWQoS paper, we have considered more practical, diverse, and certainly more complex scenarios. This scenario combines crowd sensing, model fine-tuning, and smart city technology to explore the ability to empower smart cities with big models, which is in line with the current and future needs of smart cities.
    \item We have added a comparative experiment under different distributions of PoI and visualized these conditions, which clearly demonstrates the superior performance of our approach compared to the baseline algorithms. The result, which is show in Section V, not only showcases the effectiveness of our method but also reinforces its robustness. 
    \item We dedicated substantial effort to improving readability. E.g., we added a notion table, revised some explanations, introduced additional figures depicting the distribution of orders, and augmented content to reinforce the overall logic of the paper.
\end{enumerate}

I enclose the paper abstract below for your convenience:

Advances in artificial intelligence (AI) including foundation models (FMs), are increasingly transforming human society, with smart city driving the evolution of urban living. %Collecting urban data through crowdsensing plays a distinctive role in the development of smart cities. 
Meanwhile, vehicle crowdsensing (VCS) has emerged as a key enabler, leveraging vehicles' mobility and sensor-equipped capabilities. In particular, ride-hailing vehicles can effectively facilitate flexible data collection and contribute towards urban intelligence, despite resource limitations. Therefore, this work explores a promising scenario, where edge-assisted vehicles perform joint tasks of order serving and the emerging foundation model fine-tuning using various urban data.
However, integrating the VCS AI task with the conventional order serving task is challenging, due to their inconsistent spatio-temporal characteristics: (i) The distributions of ride orders and data point-of-interests (PoIs) may not coincide in geography, both following a priori unknown patterns; (ii) they have distinct forms of temporal effects, i.e., prolonged waiting makes orders become instantly invalid while data with increased staleness gradually reduces its utility for model fine-tuning.
%a larger age-of-information (AoI) has lower utility. 
%Exploiting these insights, this work takes the first attempt to optimize the joint tasks of order serving and model fine-tuning using spatio-temporal heterogeneous urban data. %AoI for VCS into order serving, the common main task of ride-hailing vehicles. 
To overcome these obstacles, we propose an online framework based on multi-agent reinforcement learning (MARL) with careful augmentation. A new quality-of-service (QoS) metric is designed to characterize and balance the utility of the two joint tasks, under the effects of varying data volumes and staleness. We also integrate graph neural networks (GNNs) with MARL to enhance state representations, capturing graph-structured, time-varying dependencies among vehicles and across locations. Extensive experiments on our testbed simulator, utilizing various real-world foundation model fine-tuning tasks and the New York City Taxi ride order dataset, demonstrate the advantage of our proposed method.

Thank you for your consideration.

\closing{Sincerely yours,\\ Xiaoxi Zhang}


\encl{Review Manuscript submitted to IEEE TMC} % List your enclosed documents here, comment this out to get rid of the "encl:"

%----------------------------------------------------------------------------------------

\end{letter}

\end{document}