\begin{abstract}
% Recently, smart city has become a new model for urban management and development.
% Crowdsensing (CS), an important way for obtaining urban data, plays an important role in the construction of smart cities. 
Collecting urban data through crowdsensing plays a distinctive role in the development of smart cities. Given the high mobility and sensor-carrying capability, vehicle crowdsensing (VCS) has become a significant part of urban crowdsensing tasks. Ride-hailing vehicles, which are widely distributed in cities, can be a powerful tool for carrying out VCS. However, dispatching the vehicles to jointly benefit VCS and order serving is challenging, as the goals of these two tasks may not be consistent or even conflict. The distribution of ride orders and the distribution of point-of-interests (PoIs) may not coincide in time and geography, and the distribution follows a priori unknown patterns. In addition, these orders and data PoIs have distinct forms of timeliness: prolonged waiting makes orders invalid and data with a larger age-of-information (AoI) has lower utility. Exploiting these insights, this work takes the first attempt to integrate AoI for VCS into order serving, the common main task of ride-hailing vehicles. We propose an online framework by extending multi-agent reinforcement learning (MARL) with careful augmentation to optimize the profit of order-serving and the data utility of crowdsensing. A new quality-of-service (QoS) metric is designed to characterize the utility of the two joint tasks, and formal mathematical modeling drives our MARL design. In particular, to effectively capture the graph-structured dependencies among vehicles and address the dynamic state space caused by the variable size of order and PoI sets, we integrated graph neural networks (GNN) to enhance state representations. We developed a simulator and conducted extensive experiments utilizing the New York City Taxi dataset and various PoI distributions. Experimental results in various settings demonstrate the advantage of our method in QoS improvement.
\end{abstract}

\begin{IEEEkeywords}
smart city, vehicle crowdsensing, points-of-interest, age-of-information, multi-agent reinforcement learning, graph neural networks
\end{IEEEkeywords}