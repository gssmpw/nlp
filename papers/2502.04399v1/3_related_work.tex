\section{Related Work}
In this section, we provide insightful existing works in five related categories of research areas. We also highlight the key differences between these works and our study, in each paragraph.

% \xiaoxi{Boken, in this section, for each category, please search and add 1-sentence-introduction per paper, for at least 2 papers published in IEEE Trans. Service Computing (TSC) from 2023-2024. If no highly related ones are found in TSC, then try IEEE ToN, TPDS, or TMC.}

\smallskip
\noindent\textbf{Vehicle Crowdsensing (VCS).}
In VCS systems, vehicles move within defined ranges and collect data for performing VCS tasks, e.g., machine learning driven traffic monitoring. Various studies consider optimizing different metrics for these tasks, such as the volume of data collected, the geographic fairness of the collection, or the sensing coverage. For example, Fan \emph{et al.}~\cite{INFO21} formulate the VCS problem as a stochastic dynamic program and achieve fine-grained spatio-temporal sensing coverage at the minimum long-term cost. Other works~\cite{INFO23Privacy}, \cite{INFO22Privacy}, \cite{TITS2020Privacy} focus on protecting the privacy of the vehicles participating in the crowdsensing tasks. In~\cite{vehiclecrowdsensingdeepapproach}, researchers explore the optimal selection of mobile vehicles using a deep learning-based offline algorithm that predicts vehicle mobility. Recently, to overcome the constraint of ground-only operations, unmanned aerial vehicles (UAVs) have been incorporated for data collection, collaborating with ground vehicles. Yu \emph{et al.}~\cite{ICDE23} maximize the amount of collected data and minimize energy consumption through the cooperation between unmanned ground vehicles (UGVs) and UAVs. 
%\broken{
Furthermore, the EHTA framework\cite{lu2024ehta} introduces an environmentally-aware heterogeneous task allocation mechanism to ensure fairness and efficiency in vehicular crowdsensing.
%} 
% \bo{
In terms of traffic monitoring, Li \emph{et al.}~\cite{li2019privacy} introduced a novel security model to define the misbehavior of malicious drivers in traffic monitoring via fog-assisted vehicular crowdsensing.
% }
Notably, our approach diverges from these studies as we study a new scenario where ride-hailing vehicles can gain utilities in joint tasks of order serving and model fine-tuning.

\smallskip
\noindent\textbf{Model Fine-tuning in Vehicular Networks.}
% \broken{
Numerous studies have explored the integration of machine learning (ML) tasks within vehicular networks, demonstrating their potential to enhance intelligent transportation systems. For instance, ML algorithms have been employed for tasks such as real-time traffic prediction\cite{sun2020machine,chawla2024real}, dynamic routing optimization\cite{kumar2023dynamic}, and autonomous vehicle decision-making\cite{shi2020automated}. These studies underscore the growing reliance on vehicular networks for executing computationally intensive tasks in distributed environments. Recent works have increasingly focused on fine-tuning models for vehicular networks, particularly leveraging RSUs and edge devices for collaborative learning. GIOV \cite{xie2024giov} proposed a federated learning framework using RSUs to fine-tune models for adaptive vehicular applications, achieving enhanced performance in resource-constrained environments. Similarly, GAI-IOV \cite{xie2024gai} explored the deployment of pre-trained generative models for personalized content delivery in connected vehicles, addressing privacy concerns through local fine-tuning mechanisms. Otoum \emph{et al.}~\cite{otoum2022transfer} introduced an energy-efficient strategy for fine-tuning models across vehicular networks, focusing on minimizing latency and resource consumption during updates.In parallel, fine-tuning UFMs has proven transformative in addressing specific challenges in urban environments. For example, GeoSAM \cite{sultan2023geosam} enhances the SAM \cite{kirillov2023segment} model for mobility infrastructure segmentation by integrating automated visual prompt generation, showcasing its potential for urban planning. RingMo-SAM \cite{yan2023ringmo} customizes SAM’s prompt encoder for multi-source remote sensing segmentation, significantly improving performance with complex urban datasets. Additionally, GeoCLIP \cite{vivanco2024geoclip} fine-tunes the CLIP \cite{radford2021learning} framework to align images with geographic coordinates for global-scale geo-localization, utilizing innovative GPS encoding and hierarchical representations to enhance performance even with sparse training data.In contrast to these studies, our work introduces a novel scenario that synergistically benefits a diverse array of smart city applications. We not only mathematically formulate the online joint optimization of order-serving and vehicular model fine-tuning, capturing the intricate dynamics between vehicles and their urban environments, but also consider critical factors often overlooked. Specifically, our approach emphasizes the age of fine-tuning data, recognizing its significance in maintaining model relevance in dynamic urban contexts, and examines the impact of data volume on fine-tuning accuracy, ensuring robust performance even under constrained conditions. This holistic perspective enhances both the practicality and effectiveness of vehicular model fine-tuning in real-world applications.
% }
% \xiaoxi{Boken, please introduce papers for this category. First, introduce a bunch of studies on (any type of) machine learning tasks performed by vehicles. Next, introduce those that consider foundation model fine-tuning on vehicles or in RSUs; for these highly-related works, introduce 1-2 sentences per paper and at least 3 papers in total. Then, emphasize those that explicitly consider/implement fine-tuning for UFMs in vehicles; also, introduce at least 3 works one by one for these UFM vehicular fine-tuning studies. Besides, at the end of the paragraph, emphasize how our work differs from them.}

\smallskip
\noindent\textbf{Vehicle Dispatching.}
%(我们不考虑路径的选择,提一下做订单和车matching和已经决定路径的相关工作。我们的工作怎么怎么样,我们和它们的区别。其他的工作没考虑什么,方法上的不同,在哪里不适用,缺陷。把drl相关的工作放在第三段介绍)
Recently, with the rapid development of ride-hailing platforms, such as Didi~\cite{DidiChuxing2020}
%~\xiaoxi{cite}
and Uber~\cite{Uber2020}, more and more private car owners are opting to become ride-hailing drivers. In a large-scale fleet system, it makes sense to reasonably dispatch vehicles to maintain a balance between supply and demand. Traditionally, vehicle dispatching problems are commonly approached as classic combinatorial optimization problems, exemplified by the Traveling Salesman Problem (TSP) or the Vehicle Routing Problem (VRP). 
% \broken{
For instance, Zhang \emph{et al.}~\cite{KDD17} propose a novel combinatorial optimization model for the dispatching problem maximizing the global success rate by optimally matching drivers and riders, thus enhancing overall travel efficiency and improving the user experience.
% }
% \bo{
In~\cite{zhang2021data}, a data-driven optimization method is deployed on the transportation network company's side to efficiently and effectively schedule the vehicles’cruising plan.
% }
% \xiaoxirev{with a goal of xxxx maximization}; 
% \broken{
Yuen \emph{et al.}~\cite{WWW19} propose performing dynamic programming to help drivers find the route most likely to pick up passengers and maximize the probability of finding additional compatible customers while minimizing detours beyond a permissible threshold.
% }
% \xiaoxirev{and maximize/minimize xxxx.}
Other works may choose different optimization goals such as vehicle cruising time minimization~\cite{Ubiquitous18} or constraints such as energy consumption limits~\cite{vehicle_dispatch_modle_based},\cite{ge2023towards}. 
%A model-based method~\cite{Ubiquitous18} establishes mathematical models for order distribution, and applies operations research and optimization methods to minimize vehicle cruising time. Another work~\cite{vehicle_dispatch_modle_based} proposes a model predictive control (MPC) approach, optimizing order dispatching and fleet rebalancing while considering energy constraints. 
These works require a relatively complex mathematical model to model the distribution of orders or environmental dynamics, e.g., traffic or weather conditions, and their common focus is order serving provided to passengers. Our research distinguishes itself from theirs as our vehicles also bear the responsibility of data collection and model fine-tuning. Another difference is that, some of the above works consider route planning, but our work does not focus on specific route arrangements. Instead, as shown in Fig. \ref{fig:system overview}, our controlled decision variables are the next locations each vehicle should move to, rather than specific roads. This finer-grained dispatching, compared with planning routes, enhances the flexibility for idle vehicles to adapt with changing environments and increases the probability to maximize task performing utilities. 
%上面提到的这些工作有些是在考虑为车辆接送乘客做出路径规划和用时,而我们的工作中并不关注具体的路径安排。在我们的场景中车辆调度到一个区域内,可以与同一个区域内的订单匹配并订单立即生效,而不关心车到达乘客所在具体的位置的路线调度。上面这些工作需要相对复杂的数学模型对订单的分布、交通状况、路网拓扑甚至天气状况建立数学模型。同时由于调度决策是一个在线优化问题,需要利用现有信息实时决策,为优化模型的求解带来困难。在车辆调度的工作中,核心的关注点是为乘客停供服务,我们的工作与之不同的点在于我们的问题场景中车辆还兼顾采集数据的职责,存在两种任务的权衡。


%With the development of reinforcement learning, deep reinforcement learning has shown better performance than traditional methods in vehicle scheduling problems with dynamic changes in urban environments. Some studies \cite{INFO18,TITS20} regard the dispatch center as an agent, which can obtain real-time information about vehicles and orders and make dispatch decisions for vehicles. In large-scale fleet scheduling, treating the dispatch center as an agent may lead to rapid expansion of the status and action dimensions. Therefore, multi-agent reinforcement learning is also widely used in vehicle scheduling problems. Lin \emph{et al.} \cite{KDD18} propose a contextual multi-agent reinforcement learning framework which can capture the complicated stochastic demand-supply variations in high-dimensional space to achieve coordination among a large number of agents. Sun \emph{et al.}\cite{KDD22} treat each driver as an agent and solve joint order dispatching and driver repositioning problem by reinforcement learning. But too many agents also bring problems to cooperation between agents.\\

\noindent\textbf{Deep Reinforcement Learning (DRL).}
%(减少强化学习算法的介绍,用一两句介绍。后面介绍在smart city中的应用,在车辆调度和crowdsensing的介绍)
DRL is a powerful technique that uses neural networks to simulate or evaluate the actions of agents that play actions in sequential timesteps through the feedback from environments and revealed states~\cite{introduce_drl}. In recent years, a series of research works applied DRL to VCS or vehicle dispatching systems~\cite{TITS22}. For instance, in \cite{INFO20}, a 
% \xiaoxi{Raobo, please explicitly describe the DRL algorithm such as PPO, SAC?)} \bo{done}
IMPALA-based~\cite{espeholt2018impala} framework is proposed for multi-task-oriented VCS, which features a centralized control and distributed execution system. Ding \emph{et al.}~\cite{INFO21VEHICLES} use graph neural networks to extract road network information and combine it with reinforcement learning to select routes for taxis participating in the VCS task. In these studies, the data distribution is established at the initial moment, and vehicles continuously collect data from the environment while in motion. However, in our scenario, vehicles are dynamically directed to arrive at data PoIs, where urban data can be generated and discarded in real time. We design novel reward and optimization metrics to incorporate data freshness as a pivotal factor influencing the utility of UFM fine-tuning task. In related areas such as vehicle dispatching, DRL is also widely used and shows superior performance. Some studies~\cite{INFO18, TITS20} regard the dispatch center as an agent responsible for making dispatch decisions for vehicles. Lin \emph{et al.}~\cite{KDD18} propose a contextual multi-agent reinforcement learning framework that can capture the complicated stochastic demand-supply variations in high-dimensional space. Sun \emph{et al.}~\cite{KDD22} treat each driver as an agent and solve joint order dispatching and driver repositioning problems by reinforcement learning. 
% \bo{
In SA-MADRL~\cite{liu2024multi}, each idle mobile charging station or each electric vehicle with insufficient electricity obtains the surrounding situation regarding the charging supply and charging demand through V2V communications, and then uses a Q-network trained by multi-agent deep reinforcement learning method to make the scheduling decision independently.
% } 
These DRL-based vehicle dispatching studies do not consider model fine-tuning tasks fulfilled by vehicles, and thus their problems do not have the challenge of addressing spatio-temporal heterogeneity in either data or fine-tuning tasks.

\smallskip
\noindent\textbf{Graph Neural Networks (GNNs)}
Given the dynamics and potential complex-structured items in the environments, DRL algorithms may benefit from enhanced state representation. Graph Neural Networks (GNNs) is an advantageous tool for state representation to extract the graph-structured dependencies across different entities in DRL environments. Recently, Li \emph{et al.}~\cite{liyihong} nicely integrate Heterogeneous Graph Neural Network (HAN) techniques and attention mechanisms into an MARL framework for minimizing the aggregate completion time for machine learning tasks in distributed edge cluster serving systems. You \emph{et al.}~\cite{you2024raccoon} jointly optimize the content recommendation and caching in IVI systems, and they propose a GNN for user-item prediction to capture the intricate topological relationship between different agents, outperforming the conventional MARL strategies in agent coordination. 
% \broken{
For instance, Pamuklu \emph{et al.}\cite{pamuklu2023heterogeneous} propose a heterogeneous GNN-RL-based approach for task offloading in multi-UAV networks, where a novel GNN is used to model the relationships between vertices via graph network chains. MAGIC\cite{niu2021multi} employs Graph Attention Networks (GATs) to model and manage the interactions between agents in agent-agent communication. Rathore \emph{et al.}\cite{rathore2023gnn} introduce a modified graph structure that incorporates both vehicle topology and reputation estimates from Roadside Units (RSUs), leveraging GNN to enhance the reinforcement learning process by enabling vehicles to fuse sensor data and peer-reported safety messages for more accurate reputation estimation.  In contrast to prior work, which focuses on static or predefined relationships in task-specific graphs, our approach introduces a dynamic state space representation that evolves with time and context. By leveraging dynamic topology graph, we model the complex interactions between vehicles, orders, PoIs, and grids within an urban environment. Our method captures the full complexity of the environment, enabling better performance across a wider range of tasks.
% }
% \xiaoxi{For instance, xxxxxx............. Boken, please search and find at least 3 more references that combine GNN with DRL. Finally, emphasize the differences between ours and these works.}
