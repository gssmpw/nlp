\documentclass[lettersize, journal]{IEEEtran}

\usepackage{amsmath,amsfonts}
% \usepackage{algorithmic}
\usepackage{algpseudocode}
\usepackage{algorithm}
\usepackage{array}
%\usepackage[caption=false,font=normalsize,labelfont=sf,textfont=sf]{subfig}
\usepackage{textcomp}
\usepackage{stfloats}
\usepackage{url}
\usepackage{verbatim}
\usepackage{graphicx}
\usepackage{cite}

\usepackage{amssymb}
\usepackage{mathrsfs}
\usepackage{graphicx}
\usepackage{xcolor}
\usepackage{booktabs} 
\usepackage{tabularx}
\usepackage{float}
\usepackage{multirow}
\usepackage{threeparttable}
\usepackage{subfigure}
%\usepackage{caption}
\usepackage{subcaption}
\usepackage{url}

%\graphicspath{{./figures/}}

\newcommand{\broken}[1]{{\bf \color{blue} [broken: #1]}}
\newcommand{\xiaoxi}[1]{{\bf \color{red} [xiaoxi: #1]}}
\newcommand{\xiaoxirev}[1]{{\bf \color{violet} #1}}
\newcommand{\bo}[1]{{\bf \color{green} [bo: #1]}}
% correct bad hyphenation here
\hyphenation{op-tical net-works semi-conduc-tor}

\pagestyle{plain}
\def\BibTeX{{\rm B\kern-.05em{\sc i\kern-.025em b}\kern-.08em
    T\kern-.1667em\lower.7ex\hbox{E}\kern-.125emX}}
\begin{document}

\title{Can You Do Both? Balancing Order Serving and Crowdsensing for Ride-Hailing Vehicles \textcolor{red}{TBD}}
\title{Online Location Planning for AI-Defined Vehicles: Optimizing Joint Tasks of Order Serving and Spatio-Temporal Heterogeneous Model Fine-Tuning}

\author{
	Bokeng~Zheng*, Bo~Rao*, 
        Tianxiang~Zhu,
        Chee Wei Tan,~\IEEEmembership{Member,~IEEE,}
        Jingpu~Duan,~\IEEEmembership{Member,~IEEE,}
        Zhi~Zhou,~\IEEEmembership{Member,~IEEE,}
        Xu~Chen,~\IEEEmembership{Senior Member,~IEEE,}
        Xiaoxi~Zhang,~\IEEEmembership{Member,~IEEE}
		
  
	% \IEEEauthorblockA{\IEEEauthorrefmark{1}Sun Yat-sen University, Guangzhou, China}
 
 %        \IEEEauthorblockA{\IEEEauthorrefmark{2}Southern University of Science and Technology, Shenzhen, China}
        
 %        \IEEEauthorblockA{ \IEEEauthorrefmark{3}Pengcheng Laboratory, Shenzhen, China}
	% \IEEEauthorblockA{
 % zhutx23@mail2.sysu.edu.cn, \{zhangxx89,zhouzhi9,chenxu35\}@mail.sysu.edu.cn,\ duanjp@pcl.ac.cn
 %    }
\thanks{Boken Zheng, B. Rao, T. Zhu, Z. Zhou, X. Chen, and X. Zhang are with the School of Computer Science and Engineering, Sun Yat-sen University, Guangzhou 510006, China~(E-mail: \{zhengbk6,raob5,zhutx23\}@mail2.sysu.edu.cn; \{zhouzhi9,chenxu35,zhangxx89\}@mail.sysu.edu.cn).}
\thanks{Chee Wei Tan is with the College of Computing and Data Science, Nanyang Technological University, Singapore (E-mail: cheewei.tan@ntu.edu.sg).}
\thanks{J. Duan is with the Department of Communications, Peng Cheng Laboratory, Shenzhen 518066, China (E-mail: duanjp@pcl.ac.cn).}
\thanks{* indicates the co-first authors; Xiaoxi Zhang is the corresponding author.}
% \thanks{D. Yu is with Institute of Intelligent Computing, School of Computer Science and Technology, Shandong University, Qingdao, 266237, P.R. China (E-mail: dxyu@sdu.edu.cn).}
% \thanks{Y. Wu is with the School of Cyberspace Security, Dongguan University of
% Technology, Dongguan 523808, China (E-mail: wuyu@dgut.edu.cn).}
% \thanks{D. Guo is with the Key Laboratory of
% Information System Engineering, National University of Defense Technology, Changsha, Hunan 410073, China (E-mail: dekeguo@nudt.edu.cn).}
%\thanks{Xiaoxi Zhang is the corresponding author.}	
\thanks{A previous version appears at IWQoS 2024 as a short paper. 
This journal article contains substantial new results about the scenario, problem formulation, and experiments.
%proof, especially on the multi-agent scenario.
}
}

% \author{Michael~Shell,~\IEEEmembership{Member,~IEEE,}
%          John~Doe,~\IEEEmembership
% }%         and~Jane~Doe,~\IEEEmembership{Life~Fellow,~IEEE}% <-this % stops a space
% \thanks{M. Shell was with the Department
% of Electrical and Computer Engineering, Georgia Institute of Technology, Atlanta,
% GA, 30332 USA e-mail: (see http://www.michaelshell.org/contact.html).}% <-this % stops a space
% \thanks{J. Doe and J. Doe are with Anonymous University.}% <-this % stops a space
% \thanks{Manuscript received April 19, 2005; revised August 26, 2015.}}


% The paper headers
%\markboth{Journal of \LaTeX\ Class Files,~Vol.~14, No.~8, August~2015}%
%{Shell \MakeLowercase{\textit{et al.}}: Bare Demo of IEEEtran.cls for IEEE Journals}
%\IEEEpubid{0000--0000/00\$00.00~\copyright~2021 IEEE}
\IEEEpubidadjcol
    \IEEEoverridecommandlockouts
\maketitle


\begin{abstract}
Advances in artificial intelligence (AI) including foundation models (FMs), are increasingly transforming human society, with smart city driving the evolution of urban living. %Collecting urban data through crowdsensing plays a distinctive role in the development of smart cities. 
Meanwhile, vehicle crowdsensing (VCS) has emerged as a key enabler, leveraging vehicles' mobility and sensor-equipped capabilities. In particular, ride-hailing vehicles can effectively facilitate flexible data collection and contribute towards urban intelligence, despite resource limitations. Therefore, this work explores a promising scenario, where edge-assisted vehicles perform joint tasks of order serving and the emerging foundation model fine-tuning using various urban data.
However, integrating the VCS AI task with the conventional order serving task is challenging, due to their inconsistent spatio-temporal characteristics: (i) The distributions of ride orders and data point-of-interests (PoIs) may not coincide in geography, both following a priori unknown patterns; (ii) they have distinct forms of temporal effects, i.e., prolonged waiting makes orders become instantly invalid while data with increased staleness gradually reduces its utility for model fine-tuning.
%a larger age-of-information (AoI) has lower utility. 
%Exploiting these insights, this work takes the first attempt to optimize the joint tasks of order serving and model fine-tuning using spatio-temporal heterogeneous urban data. %AoI for VCS into order serving, the common main task of ride-hailing vehicles. 
To overcome these obstacles, we propose an online framework based on multi-agent reinforcement learning (MARL) with careful augmentation. A new quality-of-service (QoS) metric is designed to characterize and balance the utility of the two joint tasks, under the effects of varying data volumes and staleness. We also integrate graph neural networks (GNNs) with MARL to enhance state representations, capturing graph-structured, time-varying dependencies among vehicles and across locations. Extensive experiments on our testbed simulator, utilizing various real-world foundation model fine-tuning tasks and the New York City Taxi ride order dataset, demonstrate the advantage of our proposed method.
% \xiaoxi{Please replace the crowdsensing task to Scenario I: large-model fine-tuning tasks and change the AoI metric to test accuracy. The large-model driven task can be for entertainment supported by future IVI systems. For each vehicle, it can collect data at a PoI, either does local fine-tuning (meaning it has to pull from the RSU and store a large-language model and LoRA adapters), or uploads data to the RSU for fine-tuning at the RSU. The input data for fine-tuning has temporal-spacial heterogeneity. Specifically, the data at different PoI locations are non-IID and have different data sizes. Therefore, the sequence of locations each vehicle arrives to collect the data affects the accuracy, similar to conventional training. Here, we consider fine-tuning because vehicles can't do entire model training due to resource limits.} 

%\xiaoxi{Scenario II, which is continuous FT: In this map, the PoIs include two types of points: data locations and model RSU locations. The fine-tuned models are large and stored at RSUs. For any given geo-distributed RSU server, after a previous vehicle that finishes fine-tuning and updates the large model to the server when it moves near the RSU. it'd be better if another vehicle that has collected some data and does fine-tuning on the road to arrive sooner to use the updated fine-tuned model, in proximity to this RSU, for continuous fine-tuning and pushing back the fined-tuned adapters to the RSU. The entire model will be updated at the RSU immediately, and the old model won't be stored. If a vehicle comes late, the fetched model at the RSU might be stale, and thus the accuracy will be lower and the fine-tuned adapters may be discarded without collecting any reward. After a period of time, the model may be refreshed, once a new vehicle pushes its newly fine-tuned model.}
\end{abstract}

\begin{IEEEkeywords}
    smart city, vehicle crowdsensing, fine-tuning, foundation models, urban data, spatio-temporal heterogeneous, multi-agent reinforcement learning, graph neural networks.
\end{IEEEkeywords}

% Note that keywords are not normally used for peerreview papers.
% \begin{IEEEkeywords}
% Multi-Arm Bandit, Reinforcement Learning, Upper Confidence Bound, Smoke cessation.
% \end{IEEEkeywords}

% \IEEEpeerreviewmaketitle

\section{Introduction}
Recommender systems are essential components of contemporary digital landscape, enabling personalized services across a diverse range of fields, including e-commerce, social media, and entertainment \cite{zhang2023robust}. The data in RS generally consist of both structural information (\textit{e.g.}, user-item interactions) and textual information (\textit{e.g.}, user attributes and item descriptions).
% Based on technology and utilization of data, RS can be divided into three categories: content-based RS, collaborative filtering RS and hybrid filtering RS~\cite{adomavicius2005toward}.
\begin{figure}[ht]
    \centering
    \includegraphics[width=\linewidth]{figures/overview-yuanhao.pdf}
    \caption{An overview of GFM-based RS. Compared with GNN-based or LLM-based RS, GFM-based RS are positioned as integrating both approaches to create more comprehensive recommendations.}
    \label{fig:overview}
\end{figure}
With the rapid development of graph learning, GNN-based methods have emerged as an important technology in RS, which can further enhance the collaborative signals of collaborative filtering and extend the signals to higher-order structures and external knowledge~\cite{wu2022graph}. However, due to the inherent structural bias, they struggle to handle textual information.
This is where the powerful capabilities of large language models, which have made significant impacts in the field of natural language processing (NLP) and come into play in the realm of RS~\cite{yang2023palr,zhai2024actions}. Leveraging the advanced text capabilities of LLM, these methods efficiently capture user and item textual information while integrating world knowledge for improved recommendations. However, their reasoning limitations restrict the collaborative signals they can comprehend.
Inspired by the success of LLM in the NLP field, the graph domain has also been undergoing transformation, leading to the emergence of graph foundation models (GFMs)~\cite{liu2023towards}. By integrating GNN and LLM technologies, GFM-based RS can efficiently utilize data to align user preferences and make more precise recommendations with minimized bias, as depicted in Figure~\ref{fig:overview}. By appropriately integrating key information from both graph structures and text, GFM-based RS hold significant potential to emerge as a new paradigm in RS.
% \section{Taxonomy}

% As illustrated by Fig. \ref{}, the typical process of vision models based time series analysis has five components: (1) normalization/scaling; (2) time series to image transformation; (3) image modeling; (4) image to time series recovery; and (5) task processing. In the rest of this paper, we will discuss the typical methods for each of these components. The detailed taxonomy of the methods are summarized in Table \ref{tab.taxonomy}.

%Typical step: normalization/scaling, transformation, vision modeling, task-specific head, inverse transformation (for tasks that output time series, e.g., forecasting, generation, imputation, anomaly detection). Normalization is to fit the arbitrary range of time series values to RGB representation.

\begin{figure*}[!t]
\centering
\includegraphics[width=1.0\textwidth]{fig/fig_3.pdf}
% \vspace{-1em}
\caption{An illustration of different methods for imaging time series with a sample (length=336) from the \textit{Electricity} benchmark dataset \protect\cite{nie2023time}. (a)(c)(d)(e)(f) %are univariate methods.
visualize the same variate. (b) visualizes all 321 variates. Filterbank is omitted due to its %high
similarity to STFT.}\label{fig.tsimage}
\vspace{-0.2cm}
\end{figure*}

\begin{table*}[t]
\centering
\scriptsize
\setlength{\tabcolsep}{2.7pt}{
% \begin{tabular}{llllllllllll}
\begin{tabular}{llcccccccccl}
\toprule[1pt]
\multirow{2}{*}{Method} & \multirow{2}{*}{TS-Type} & \multirow{2}{*}{Imaging} & \multicolumn{5}{c}{Imaged Time Series Modeling} & \multirow{2}{*}{TS-Recover} & \multirow{2}{*}{Task} & \multirow{2}{*}{Domain} & \multirow{2}{*}{Code}\\ \cmidrule{4-8}
 & & & Multi-modal & Model & Pre-trained & Fine-tune & Prompt & & & & \\ \midrule
\cite{silva2013time} & UTS & RP & \xmark & \texttt{K-NN} & \xmark & \xmark & \xmark & \xmark & Classification & General & \xmark\\
\cite{wang2015encoding} & UTS & GAF & \xmark & \texttt{CNN} & \xmark & \cmark$^{\flat}$ & \xmark & \cmark & Classification & General & \xmark\\
\cite{wang2015imaging} & UTS & GAF & \xmark & \texttt{CNN} & \xmark & \cmark$^{\flat}$ & \xmark & \cmark & Multiple & General & \xmark\\
% \multirow{2}{*}{\cite{wang2015imaging}} & \multirow{2}{*}{UTS} & \multirow{2}{*}{GAF} & \multirow{2}{*}{\xmark} & \multirow{2}{*}{\texttt{CNN}} & \multirow{2}{*}{\xmark} & \multirow{2}{*}{\cmark$^{\flat}$} & \multirow{2}{*}{\xmark} & \multirow{2}{*}{\cmark} & Classification & \multirow{2}{*}{General} & \multirow{2}{*}{\xmark}\\
% & & & & & & & & & \& Imputation & & \\
\cite{ma2017learning} & MTS & Heatmap & \xmark & \texttt{CNN} & \xmark & \cmark$^{\flat}$ & \xmark & \cmark & Forecasting & Traffic & \xmark\\
\cite{hatami2018classification} & UTS & RP & \xmark & \texttt{CNN} & \xmark & \cmark$^{\flat}$ & \xmark & \xmark & Classification & General & \xmark\\
\cite{yazdanbakhsh2019multivariate} & MTS & Heatmap & \xmark & \texttt{CNN} & \xmark & \cmark$^{\flat}$ & \xmark & \xmark & Classification & General & \cmark\textsuperscript{\href{https://github.com/SonbolYb/multivariate_timeseries_dilated_conv}{[1]}}\\
MSCRED \cite{zhang2019deep} & MTS & Other ($\S$\ref{sec.othermethod}) & \xmark & \texttt{ConvLSTM} & \xmark & \cmark$^{\flat}$ & \xmark & \xmark & Anomaly & General & \cmark\textsuperscript{\href{https://github.com/7fantasysz/MSCRED}{[2]}}\\
\cite{li2020forecasting} & UTS & RP & \xmark & \texttt{CNN} & \cmark & \cmark & \xmark & \xmark & Forecasting & General & \cmark\textsuperscript{\href{https://github.com/lixixibj/forecasting-with-time-series-imaging}{[3]}}\\
\cite{cohen2020trading} & UTS & LinePlot & \xmark & \texttt{Ensemble} & \xmark & \cmark$^{\flat}$ & \xmark & \xmark & Classification & Finance & \xmark\\
% \cite{du2020image} & UTS & Spectrogram & \xmark & \texttt{CNN} & \xmark & \cmark$^{\flat}$ & \xmark & \xmark & Classification & Finance & \xmark\\
\cite{barra2020deep} & UTS & GAF & \xmark & \texttt{CNN} & \xmark & \cmark$^{\flat}$ & \xmark & \xmark & Classification & Finance & \xmark\\
% \cite{barra2020deep} & UTS & GAF & \xmark & \texttt{VGG-16} & \xmark & \cmark$^{\flat}$ & \xmark & \xmark & Classification & Finance & \xmark\\
% \cite{cao2021image} & UTS & RP & \xmark & \texttt{CNN} & \xmark & \cmark$^{\flat}$ & \xmark & \xmark & Classification & General & \xmark\\
VisualAE \cite{sood2021visual} & UTS & LinePlot & \xmark & \texttt{CNN} & \xmark & \cmark$^{\flat}$ & \xmark & \cmark & Forecasting & Finance & \xmark\\
% VisualAE \cite{sood2021visual} & UTS & LinePlot & \xmark & \texttt{CNN} & \xmark & \cmark$^{\flat}$ & \xmark & \xmark & Img-Generation & Finance & \xmark\\
\cite{zeng2021deep} & MTS & Heatmap & \xmark & \texttt{CNN,LSTM} & \xmark & \cmark$^{\flat}$ & \xmark & \cmark & Forecasting & Finance & \xmark\\
% \cite{zeng2021deep} & MTS & Heatmap & \xmark & \texttt{SRVP} & \xmark & \cmark$^{\flat}$ & \xmark & \cmark & Forecasting & Finance & \xmark\\
AST \cite{gong2021ast} & UTS & Spectrogram & \xmark & \texttt{DeiT} & \cmark & \cmark & \xmark & \xmark & Classification & Audio & \cmark\textsuperscript{\href{https://github.com/YuanGongND/ast}{[4]}}\\
TTS-GAN \cite{li2022tts} & MTS & Heatmap & \xmark & \texttt{ViT} & \xmark & \cmark$^{\flat}$ & \xmark & \cmark & Ts-Generation & Health & \cmark\textsuperscript{\href{https://github.com/imics-lab/tts-gan}{[5]}}\\
SSAST \cite{gong2022ssast} & UTS & Spectrogram & \xmark & \texttt{ViT} & \cmark$^{\natural}$ & \cmark & \xmark & \xmark & Classification & Audio & \cmark\textsuperscript{\href{https://github.com/YuanGongND/ssast}{[6]}}\\
MAE-AST \cite{baade2022mae} & UTS & Spectrogram & \xmark & \texttt{MAE} & \cmark$^{\natural}$ & \cmark & \xmark & \xmark & Classification & Audio & \cmark\textsuperscript{\href{https://github.com/AlanBaade/MAE-AST-Public}{[7]}}\\
AST-SED \cite{li2023ast} & UTS & Spectrogram & \xmark & \texttt{SSAST,GRU} & \cmark & \cmark & \xmark & \xmark & EventDetection & Audio & \xmark\\
\cite{jin2023classification} & UTS & %Multiple
LinePlot & \xmark & \texttt{CNN} & \cmark & \cmark & \xmark & \xmark & Classification & Physics & \xmark\\
ForCNN \cite{semenoglou2023image} & UTS & LinePlot & \xmark & \texttt{CNN} & \xmark & \cmark$^{\flat}$ & \xmark & \xmark & Forecasting & General & \xmark\\
Vit-num-spec \cite{zeng2023pixels} & UTS & Spectrogram & \xmark & \texttt{ViT} & \xmark & \cmark$^{\flat}$ & \xmark & \xmark & Forecasting & Finance & \xmark\\
% \cite{wimmer2023leveraging} & MTS & LinePlot & \xmark & \texttt{CLIP,LSTM} & \cmark & \cmark & \xmark & \xmark & Classification & Finance & \xmark\\
ViTST \cite{li2023time} & MTS & LinePlot & \xmark & \texttt{Swin} & \cmark & \cmark & \xmark & \xmark & Classification & General & \cmark\textsuperscript{\href{https://github.com/Leezekun/ViTST}{[8]}}\\
MV-DTSA \cite{yang2023your} & UTS\textsuperscript{*} & LinePlot & \xmark & \texttt{CNN} & \xmark & \cmark$^{\flat}$ & \xmark & \cmark & Forecasting & General & \cmark\textsuperscript{\href{https://github.com/IkeYang/machine-vision-assisted-deep-time-series-analysis-MV-DTSA-}{[9]}}\\
TimesNet \cite{wu2023timesnet} & MTS & Heatmap & \xmark & \texttt{CNN} & \xmark & \cmark$^{\flat}$ & \xmark & \cmark & Multiple & General & \cmark\textsuperscript{\href{https://github.com/thuml/TimesNet}{[10]}}\\
ITF-TAD \cite{namura2024training} & UTS & Spectrogram & \xmark & \texttt{CNN} & \cmark & \xmark & \xmark & \xmark & Anomaly & General & \xmark\\
\cite{kaewrakmuk2024multi} & UTS & GAF & \xmark & \texttt{CNN} & \cmark & \cmark & \xmark & \xmark & Classification & Sensing & \xmark\\
HCR-AdaAD \cite{lin2024hierarchical} & MTS & RP & \xmark & \texttt{CNN,GNN} & \xmark & \cmark$^{\flat}$ & \xmark & \xmark & Anomaly & General & \xmark\\
FIRTS \cite{costa2024fusion} & UTS & Other ($\S$\ref{sec.othermethod}) & \xmark & \texttt{CNN} & \xmark & \cmark$^{\flat}$ & \xmark & \xmark & Classification & General & \cmark\textsuperscript{\href{https://sites.google.com/view/firts-paper}{[11]}}\\
% \multirow{2}{*}{FIRTS \cite{costa2024fusion}} & \multirow{2}{*}{UTS} & Spectrogram & \multirow{2}{*}{\xmark} & \multirow{2}{*}{\texttt{CNN}} & \multirow{2}{*}{\xmark} & \multirow{2}{*}{\cmark$^{\flat}$} & \multirow{2}{*}{\xmark} & \multirow{2}{*}{\xmark} & \multirow{2}{*}{Classification} & \multirow{2}{*}{General} & \multirow{2}{*}{\cmark\textsuperscript{\href{https://sites.google.com/view/firts-paper}{[2]}}}\\
%  & & \& GAF,RP,MTF & & & & & & & & & \\
% \cite{homenda2024time} & UTS\textsuperscript{*} & Multiple & \xmark & \texttt{CNN} & \xmark & \cmark$^{\flat}$ & \xmark & \xmark & Classification & General & \xmark\\
CAFO \cite{kim2024cafo} & MTS & RP & \xmark & \texttt{CNN,ViT} & \xmark & \cmark$^{\flat}$ & \xmark & \xmark & Explanation & General & \cmark\textsuperscript{\href{https://github.com/eai-lab/CAFO}{[12]}}\\
% \multirow{2}{*}{CAFO \cite{kim2024cafo}} & \multirow{2}{*}{MTS} & \multirow{2}{*}{RP} & \multirow{2}{*}{\xmark} & \texttt{ShuffleNet,ResNet} & \multirow{2}{*}{\cmark} & \multirow{2}{*}{\cmark} & \multirow{2}{*}{\xmark} & \multirow{2}{*}{\xmark} & Classification & \multirow{2}{*}{General} & \multirow{2}{*}{\cmark}\\
%  & & & & \texttt{MLP-Mixer,ViT} & & & & & \& Explanation & & \\
ViTime \cite{yang2024vitime} & UTS\textsuperscript{*} & LinePlot & \xmark & \texttt{ViT} & \cmark$^{\natural}$ & \cmark & \xmark & \cmark & Forecasting & General & \cmark\textsuperscript{\href{https://github.com/IkeYang/ViTime}{[13]}}\\
ImagenTime \cite{naiman2024utilizing} & MTS & Other ($\S$\ref{sec.othermethod}) & \xmark & %\texttt{Diffusion}
\texttt{CNN} & \xmark & \cmark$^{\flat}$ & \xmark & \cmark & Ts-Generation & General & \cmark\textsuperscript{\href{https://github.com/azencot-group/ImagenTime}{[14]}}\\
TimEHR \cite{karami2024timehr} & MTS & Heatmap & \xmark & \texttt{CNN} & \xmark & \cmark$^{\flat}$ & \xmark & \cmark & Ts-Generation & Health & \cmark\textsuperscript{\href{https://github.com/esl-epfl/TimEHR}{[15]}}\\
VisionTS \cite{chen2024visionts} & UTS\textsuperscript{*} & Heatmap & \xmark & \texttt{MAE} & \cmark & \cmark & \xmark & \cmark & Forecasting & General & \cmark\textsuperscript{\href{https://github.com/Keytoyze/VisionTS}{[16]}}\\ \midrule
InsightMiner \cite{zhang2023insight} & UTS & LinePlot & \cmark & \texttt{LLaVA} & \cmark & \cmark & \cmark & \xmark & Txt-Generation & General & \xmark\\
\cite{wimmer2023leveraging} & MTS & LinePlot & \cmark & \texttt{CLIP,LSTM} & \cmark & \cmark & \xmark & \xmark & Classification & Finance & \xmark\\
% \cite{dixit2024vision} & UTS & Spectrogram & \cmark & \texttt{GPT4o,Gemini} & \cmark & \xmark & \cmark & \xmark & Classification & Audio & \xmark\\
\multirow{2}{*}{\cite{dixit2024vision}} & \multirow{2}{*}{UTS} & \multirow{2}{*}{Spectrogram} & \multirow{2}{*}{\cmark} & \texttt{GPT4o,Gemini} & \multirow{2}{*}{\cmark} & \multirow{2}{*}{\xmark} & \multirow{2}{*}{\cmark} & \multirow{2}{*}{\xmark} & \multirow{2}{*}{Classification} & \multirow{2}{*}{Audio} & \multirow{2}{*}{\xmark}\\
 & & & & \& \texttt{Claude3} & & & & & & & \\
\cite{daswani2024plots} & MTS & LinePlot & \cmark & \texttt{GPT4o,Gemini} & \cmark & \xmark & \cmark & \xmark & Multiple & General & \xmark\\
TAMA \cite{zhuang2024see} & UTS & LinePlot & \cmark & \texttt{GPT4o} & \cmark & \xmark & \cmark & \xmark & Anomaly & General & \xmark\\
\cite{prithyani2024feasibility} & MTS & LinePlot & \cmark & \texttt{LLaVA} & \cmark & \cmark & \cmark & \xmark & Classification & General & \cmark\textsuperscript{\href{https://github.com/vinayp17/VLM_TSC}{[17]}}\\
\bottomrule[1pt]
\end{tabular}}
\vspace{-0.25cm}
\caption{Taxonomy of vision models on time series. The top panel includes single-modal models. The bottom panel includes multi-modal models. {\bf TS-Type} denotes type of time series. {\bf TS-Recover} denotes %whether time series recovery ($\S$\ref{sec.processing}) has been performed.
recovering time series from predicted images ($\S$\ref{sec.processing}). \textsuperscript{*}: %the model has been %applied on MTSs by %processing %modeling the individual UTSs of each MTS.
the method has been used to model the individual UTSs of an MTS. $^{\natural}$: a new pre-trained model was proposed in the work. $^{\flat}$: %without using a pre-trained model, fine-tune means training from scratch.
when pre-trained models were unused, ``Fine-tune'' refers to train a task-specific model from scratch. %In the
{\bf Model} column: \texttt{CNN} could be regular CNN, ResNet, VGG-Net, %U-Net,
{\em etc.}}\label{tab.taxonomy}
% The code only include verified official code from the authors.
\vspace{-0.3cm}
\end{table*}

\begin{table*}[t]
\centering
\small
\setlength{\tabcolsep}{2.9pt}{
\begin{tabular}{l|l|l|l}\hline
% \toprule[1pt]
\rowcolor{gray!20}
{\bf Method} & {\bf TS-Type} & {\bf Advantages} & {\bf Limitations}\\ \hline
Line Plot ($\S$\ref{sec.lineplot}) & UTS, MTS & matches human perception of time series & limited to MTSs with a small number of variates\\ \hline
Heatmap ($\S$\ref{sec.heatmap}) & UTS, MTS & straightforward for both UTSs and MTSs & the order of variates may affect their correlation learning\\ \hline
Spectrogram ($\S$\ref{sec.spectrogram}) & UTS & encodes the time-frequency space & limited to UTSs; needs a proper choice of window/wavelet\\ \hline
GAF ($\S$\ref{sec.gaf}) & UTS & encodes the temporal correlations in a UTS & limited to UTSs; $O(T^{2})$ time and space complexity\\ \hline% for long time series\\ \hline
% RP ($\S$\ref{sec.rp}) & UTS & flexibility in image size by tuning $m$ and $\tau$ & limited to UTSs; the pattern has a threshold-dependency\\ \hline
RP ($\S$\ref{sec.rp}) & UTS & flexibility in image size by tuning $m$ and $\tau$ & limited to UTSs; information loss after thresholding\\ \hline
% \bottomrule[1pt]
\end{tabular}}
\vspace{-0.2cm}
\caption{Summary of the five primary methods for transforming time series to images. {\bf TS-Type} denotes type of time series.}\label{tab.tsimage}
\vspace{-0.2cm}
\end{table*}

\section{Time Series To Image Transformation}\label{sec.tsimage}

% This section summarizes 5 major methods for imaging time series ($\S$\ref{sec.lineplot}-$\S$\ref{sec.rp}). We also discuss some other methods ($\S$\ref{sec.othermethod}) and how to model MTS with these methods ($\S$\ref{sec.modelmts}).
This section summarizes the methods for imaging time series ($\S$\ref{sec.lineplot}-$\S$\ref{sec.othermethod}) and their extensions to encode MTSs ($\S$\ref{sec.modelmts}).

% This section summarizes 5 major methods for transforming time series to images, including Line Plot, Heatmap, Spetrogram, GAF and RP, and several minor methods. We discuss their pros and cons and how to deal with MTS.

% This section discusses the advantages and limitations of different methods for time series to image transformation (invertible, efficiency, information preservation, MTS, long-range time series, parametric, etc.).

%\subsection{Methods}

\vspace{-0.08cm}

\subsection{Line Plot}\label{sec.lineplot}

Line Plot is a straightforward way for visualizing UTSs for human analysis ({\em e.g.}, stocks, power consumption, {\em etc.}). As illustrated by Fig. \ref{fig.tsimage}(a), the simplest approach is to draw a 2D image with x-axis representing %the time horizon
time steps and y-axis representing %the magnitude of the normalized time series.
time-wise values, %A line is used to connect all values of the series over time.
with a line connecting all values of the series over time. This image can be %represented by either three-channel pixels or single-channel pixels
either three-channel ({\em i.e.}, RGB) or single-channel as the colors may not %provide additional information
be informative %\cite{cohen2020trading,sood2021visual,jin2023classification,zhang2023insight,zhuang2024see}.
\cite{cohen2020trading,sood2021visual,jin2023classification,zhang2023insight}. ForCNN \cite{semenoglou2023image} even uses a single 8-bit integer to represent each pixel for black-white images. So far, there is no consensus on whether other graphical components, such as legend, grids and tick labels, could provide extra benefits in any task. For example, ViTST \cite{li2023time} finds these components are superfluous in a classification task, while TAMA \cite{zhuang2024see} finds grid-like auxiliary lines help enhance anomaly detection.

In addition to the regular Line Plot, MV-DTSA \cite{yang2023your} and ViTime \cite{yang2024vitime} divide an image into $h\times L$ grids, %where $h$ is the number of rows and $L$ is the number of columns,
and %introduced
define a function to map each time step of a UTS to a grid, producing a grid-like Line Plot. Also, we include methods that use Scatter Plot \cite{daswani2024plots,prithyani2024feasibility} in this category because %the only difference between a Scatter Plot and a Line Plot is whether the time-wise values are connected by lines.
a Scatter Plot resembles a Line Plot but doesn't connect %time-wise values
data points with a line. By comparing them, \cite{prithyani2024feasibility} finds a Line Plot could induce better time series classification.

For MTSs, we defer the discussion on Line Plot to $\S$\ref{sec.modelmts}.

% For MTS, some methods use the channel-independence assumption proposed in \cite{nie2023time} and represent each variate in MTS with an individual Line Plot \cite{yang2023your,yang2024vitime}. ViTST \cite{li2023time} also uses an individual Line Plot per variate, but colors different lines and assembles all plots to form a bigger image. The method in \cite{wimmer2023leveraging} plots %the time series of
% all variates in a single Line Plot and distinguish them by %use different
% types of lines ({\em e.g.}, solid, dashed, dotted, {\em etc.}). %to distinguish them.
% However, these methods only work for a small number of variates. For example, in \cite{wimmer2023leveraging}, there are only 4 variates in its financial MTSs.

%\cite{li2023time} space-costly because of blank pixels. scatter plot.

%Invertible with a numeric prediction head \cite{sood2021visual}. It fits tasks such as forecasting, imputation, etc.

\vspace{-0.08cm}

\subsection{Heatmap}\label{sec.heatmap}

As shown in Fig. \ref{fig.tsimage}(b), Heatmap is a 2D visualization of the magnitude of the values in a matrix using color. %The variation of color represents the intensity of each value. %Therefore,
It has been used to %directly
represent the matrix of an MTS, {\em i.e.}, $\mat{X} \in \mathbb{R}^{d\times T}$, as a one-channel $d\times T$ image \cite{li2022tts,yazdanbakhsh2019multivariate}. Similarly, TimEHR \cite{karami2024timehr} represents an {\em irregular} MTS, where the intervals between time steps are uneven, as a $d\times H$ Heatmap image by grouping the uneven time steps into $H$ even time bins. In \cite{zeng2021deep}, a different method is used for visualizing a 9-variate financial %time series.
MTS. It reshapes the 9 variates at each time step to a $3\times 3$ Heatmap image, and uses the sequence of images to forecast future %image
frames, achieving %time series
%MTS
time series forecasting. In contrast, VisionTS \cite{chen2024visionts} uses Heatmap to visualize UTSs. %instead.
Similar to TimesNet \cite{wu2023timesnet}, it first segments a length-$T$ UTS into $\lfloor T/P\rfloor$ length-$P$ subsequences, where $P$ is a parameter representing a periodicity of the UTS. Then the subsequences are stacked into a $P\times \lfloor T/P\rfloor$ matrix, %and duplicated 3 times to produce a 3-channel
with 3 duplicated channels, to produce a grayscale image %which serves as an
input to %a vision foundation model.
an LVM. To encode MTSs, VisionTS adopts the channel independence assumption \cite{nie2023time} and individually models each variate in an MTS.

\vspace{0.2cm}

\noindent{\bf Remark.} Heatmap can be used to visualize matrices of various forms. It is also used for matrices generated by the subsequent methods ({\em e.g.}, Spectrogram, GAF, RP) in this section. In this paper, the name Heatmap refers specifically to images that use color to visualize the (normalized) values in UTS $\mat{x}$ or MTS $\mat{X}$ without performing other transformations.

%\cite{chen2024visionts,karami2024timehr} bin version of TSH \cite{karami2024timehr}, DE and STFT \cite{naiman2024utilizing} (DE can be used for constructing RP), rearrange variates for video version of TSH \cite{zeng2021deep}.

%\vspace{0.2cm}

\subsection{Spectrogram}\label{sec.spectrogram}

A {\em spectrogram} is a visual representation of the spectrum of frequencies of a signal as it varies with time, which are extensively used for analyzing audio signals \cite{gong2021ast}. Since audio signals are a type of UTS, spectrogram can be considered as a method for imaging a UTS. As shown in Fig. \ref{fig.tsimage}(c), a common format is a 2D heatmap image with x-axis representing time steps and y-axis representing frequency, {\em a.k.a.} a time-frequency space. %The color at each point
Each pixel in the image represents the (logarithmic) amplitude of a specific frequency at a specific time point. Typical methods for %transforming a UTS to
producing a spectrogram include {\bf Short-Time Fourier Transform (STFT)} \cite{griffin1984signal}, {\bf Wavelet Transform} \cite{daubechies1990wavelet}, and {\bf Filterbank} \cite{vetterli1992wavelets}.

\vspace{0.2cm}

\noindent{\bf STFT.} %Discrete Fourier transform (DFT) can be used to represent a UTS signal %$\mat{x}=[x_{1}, ..., x_{T}]$
%$\mat{x}\in\mathbb{R}^{1\times T}$ as a sum of sinusoidal components. The output of the transform is a function of frequency $f(w)$, describing the intensity of each constituent frequency $w$ of the entire UTS. 
Discrete Fourier transform (DFT) can be used to describe the intensity $f(w)$ of each constituent frequency $w$ of a UTS signal $\mat{x}\in\mathbb{R}^{1\times T}$. However, $f(w)$ has no time dependency. It cannot provide dynamic information such as when a specific frequency appear in the UTS. STFT addresses this deficiency by sliding a window function $g(t)$ over the time steps in %the UTS,
$\mat{x}$, and computing the DFT within each window by
\begin{equation}\label{eq.stft}
\small
\begin{aligned}
f(w,\tau) = \sum_{t=1}^{T}x_{t}g(t - \tau)e^{-iwt}
\end{aligned}
\end{equation}
where $w$ is frequency, $\tau$ is the position of the window, $f(w,\tau)$ describes the intensity of frequency $w$ at time step $\tau$.

%With a proper selection of the
By selecting a proper window function $g(\cdot)$ ({\em e.g.}, Gaussian/Hamming/Bartlett window), %({\em e.g.}, Gaussian window, Hamming window, Bartlett window), %{\em etc.}),
a 2D spectrogram ({\em e.g.}, Fig. \ref{fig.tsimage}(c)) can be drawn via a heatmap on the squared values $|f(w,\tau)|^{2}$, with $w$ as the y-axis, and $\tau$ as the x-axis. For example, \cite{dixit2024vision} uses STFT based spectrogram as an input to LMMs %\hh{do you mean LVMs? check}
for time series classification.

%Fourier transform is a powerful data analysis tool that represents any complex signal as a sum of sines and cosines and transforms the signal from the time domain to the frequency domain. However, Fourier transform can only show which frequencies are present in the signal, but not when these frequencies appear. The STFT divides original signal into several parts using a sliding window to fix this problem. STFT involves a sliding window for extracting frequency components within the window.

\vspace{0.2cm}

\noindent{\bf Wavelet Transform.} %Like Fourier transform, %\hh{this paragraph needs a citation}
Continuous Wavelet Transform (CWT) uses the inner product to measure the similarity between a signal function $x(t)$ and an analyzing function. %In STFT (Eq.~\eqref{eq.stft}), the analyzing function is a windowed exponential $g(t - \tau)e^{-iwt}$.
%In CWT,
The analyzing function is a {\em wavelet} $\psi(t)$, where the typical choices include Morse wavelet, Morlet wavelet, %Daubechies wavelet, %Beylkin wavelet, 
{\em etc.} %The
CWT compares $x(t)$ to the shifted and scaled ({\em i.e.}, stretched or shrunk) versions of the wavelet, and output a CWT coefficient by
\begin{equation}\label{eq.cwt}
\small
\begin{aligned}
c(s,\tau) = \int_{-\infty}^{\infty}x(t)\frac{1}{s}\psi^{*}(\frac{t - \tau}{s})dt
\end{aligned}
\end{equation}
where $*$ denotes complex conjugate, $\tau$ is the time step to shift, and $s$ represents the scale. In practice, a discretized version of CWT in Eq.~\eqref{eq.cwt} is implemented for UTS $[x_{1}, ..., x_{T}]$.

It is noteworthy that the scale $s$ controls the frequency encoded in a wavelet -- a larger $s$ leads to a stretched wavelet with a lower frequency, and vice versa. As such, by varying $s$ and $\tau$, a 2D spectrogram ({\em e.g.}, Fig. \ref{fig.tsimage}(d)) can be drawn %, often with a heatmap
on $|c(s,\tau)|$, where $s$ is the y-axis and $\tau$ is the x-axis. Compared to STFT, which uses a fixed window size, Wavelet Transform allows variable wavelet sizes -- a larger size %region
for more precise low frequency information. 
%Usually, $s$ and $\tau$ vary dependently -- a larger $s$ leads to a stretched wavelet that shifts slowly, {\em i.e.}, a smaller $\tau$. This property %of CWT
%yields a spectrogram that balances the resolutions of frequency %$s$
%and time, %$\tau$,
%which is an advantage over the fixed time resolution in STFT.
% Thus, both of the methods in %\cite{du2020image}
% \cite{namura2024training} and \cite{zeng2023pixels} choose CWT (with Morlet wavelet) to generate the spectrogram.
Thus, the methods in \cite{du2020image,namura2024training,zeng2023pixels} choose CWT (with Morlet wavelet) to generate the spectrogram.

%A wavelet is a wave-like oscillation that has zero mean and is localized in both time and frequency space.

\vspace{0.2cm}

\noindent{\bf Filterbank.} This method %is relevant to
resembles STFT and is often used in processing audio signals. Given an audio signal, it firstly goes through a {\em pre-emphasis filter} to boost high frequencies, which helps improve the clarity of the signal. Then, STFT is applied on the signal. %with a sliding window $g(t)$ of size $k$ that shifts in a fixed stride $\tau$. %where the adjacent windows may overlap in $k$ time length.
%Finally, filterbank features are computed by applying multiple ``triangle-shaped'' filters spaced on the Mel-scale to the STFT output $f(w, \tau)$. %where Mel-scale is a method to make the filters more discriminative on lower frequencies, %than higher frequencies,
%imitating the non-linear human ear perception of sound.
Finally, multiple ``triangle-shaped'' filters spaced on a Mel-scale are applied to the STFT power spectrum $|f(w, \tau)|^{2}$ to extract frequency bands. The outcome filterbank features $\hat{f}(w, \tau)$ can be used to yield a spectrogram with $w$ as the y-axis, and $\tau$ as the x-axis.

%Filterbank was introduced in AST \cite{gong2021ast} with %$k$=25ms
Filterbank was adopted in AST \cite{gong2021ast} with 
a 25ms Hamming window that shifts every 10ms for classifying audio signals using Vision Transformer (ViT). It then becomes widely used in the follow-up works such as SSAST \cite{gong2022ssast}, MAE-AST \cite{baade2022mae}, and AST-SED \cite{li2023ast}, as summarized in Table \ref{tab.taxonomy}.



%Use MLP to predict TS directly \cite{zeng2023pixels}.

%\vspace{0.2cm}

% \vspace{0.2cm}

\subsection{Gramian Angular Field (GAF)}\label{sec.gaf}

GAF was introduced for classifying UTSs using CNNs %using %image based CNNs
by \cite{wang2015encoding}. It was then extended %with an extension
to an imputation task in \cite{wang2015imaging}. Similarly, \cite{barra2020deep} applied GAF for financial time series forecasting.

Given a UTS $\mat{x}\in\mathbb{R}^{1\times T}$, %$[x_{1}, ..., x_{T}]$,
the first step %before GAF
is to rescale each $x_{t}$ to a value $\tilde{x}_{t}$ %in the interval of
within $[0, 1]$ (or $[-1, 1]$). %by min-max normalization.
This range enables mapping $\tilde{x}_{t}$ to polar coordinates by $\phi_{t}=\text{arccos}(\tilde{x}_{i})$, with a radius $r=t/N$ encoding the time stamp, where $N$ is a constant factor to regularize the span of the polar coordinates. %system. Then,
Two types of GAF, Gramian Sum Angular Field (GASF) and Gramian Difference Angular Field (GADF) are defined as
\begin{equation}\label{eq.gaf}
\small
\begin{aligned}
&\text{GASF:}~~\text{cos}(\phi_{t} + \phi_{t'})=x_{t}x_{t'} - \sqrt{1 - x_{t}^{2}}\sqrt{1 - x_{t'}^{2}}\\
&\text{GADF:}~~\text{sin}(\phi_{t} - \phi_{t'})=x_{t'}\sqrt{1 - x_{t}^{2}} - x_{t}\sqrt{1 - x_{t'}^{2}}
\end{aligned}
\end{equation}
which exploits the pairwise temporal correlations in the UTS. Thus, the outcome is a $T\times T$ matrix $\mat{G}$ with $\mat{G}_{t,t'}$ specified by either type in Eq.~\eqref{eq.gaf}. A GAF image is a heatmap on $\mat{G}$ with both axes representing time, as illustrated by Fig. \ref{fig.tsimage}(e).

% Invertible.

% \vspace{0.2cm}

\subsection{Recurrence Plot (RP)}\label{sec.rp}

%RP \cite{eckmann1987recurrence} is a method to encode a UTS into an image that aims to capture the periodic patterns in the UTS by using its reconstructed {\em phase space}. The phase space of a UTS $[x_{1}, ..., x_{T}]$ can be reconstructed by {\em time delay embedding}, which is a set of new vectors $\mat{v}_{1}$, ..., $\mat{v}_{l}$ with

RP \cite{eckmann1987recurrence} encodes a UTS into an image that captures its periodic patterns by using its reconstructed {\em phase space}. The phase space of %a UTS %$[x_{1}, ..., x_{T}]$
$\mat{x}\in\mathbb{R}^{1\times T}$ can be reconstructed by {\em time delay embedding} -- a set of new vectors $\mat{v}_{1}$, ..., $\mat{v}_{l}$ with
\begin{equation}\label{eq.de}
\small
\begin{aligned}
\mat{v}_{t}=[x_{t}, x_{t+\tau}, x_{t+2\tau}, ..., x_{t+(m-1)\tau}]\in\mathbb{R}^{m\tau},~~~1\le t \le l
\end{aligned}
\end{equation}
where $\tau$ is the time delay, $m$ is the dimension of the phase space, both %of which
are hyperparameters. Hence, $l=T-(m-1)\tau$. With vectors $\mat{v}_{1}$, ..., $\mat{v}_{l}$, an RP image %is constructed by measuring
measures their pairwise distances, results in an $l\times l$ image whose element
\begin{equation}\label{eq.rp}
\small
\begin{aligned}
\text{RP}_{i,j}=\Theta(\varepsilon - \|\mat{v}_{i} - \mat{v}_{j}\|),~~~1\le i,j\le l
\end{aligned}
\end{equation}
where $\Theta(\cdot)$ is the Heaviside step function, $\varepsilon$ is a threshold, and $\|\cdot\|$ is a norm function such as $\ell_{2}$ norm. Eq.~\eqref{eq.rp} %states RP produces a heatmap image on a binary matrix with $\text{RP}_{i,j}=1$ if $\mat{v}_{i}$ and $\mat{v}_{j}$ are sufficiently similar.
generates a binary matrix with $\text{RP}_{i,j}=1$ if $\mat{v}_{i}$ and $\mat{v}_{j}$ are sufficiently similar, producing a black-white image ({\em e.g.}, Fig. \ref{fig.tsimage}(f)).% ({\em e.g.}, a periodic pattern).

An advantage of RP is its flexibility in image size by tuning $m$ and $\tau$. Thus it has been used for time series classification %\cite{cao2021image},
\cite{silva2013time,hatami2018classification}, forecasting \cite{li2020forecasting}, anomaly detection \cite{lin2024hierarchical} and %feature-wise
explanation \cite{kim2024cafo}. Moreover, the method in \cite{hatami2018classification}, and similarly in HCR-AdaAD \cite{lin2024hierarchical}, omit the thresholding in Eq.~\eqref{eq.rp} and uses $\|\mat{v}_{i} - \mat{v}_{j}\|$ to produce continuously valued images %in a classification task
to avoid information loss.


% \vspace{0.2cm}

\subsection{Other Methods}\label{sec.othermethod}

%There are some less commonly used methods. For example, in
Additionally, %there are some peripheral methods. %In addition to GAF,
\cite{wang2015encoding} introduces Markov Transition Field (MTF) for imaging a UTS. %$\mat{x}\in\mathbb{R}^{1\times T}$. 
%MTF first assigns each $x_{t}$ to one of $Q$ quantile bins, then builds a $Q\times Q$ Markov transition matrix $\mat{M}$ {\em s.t.} $\mat{M}_{i,j}$ represents the frequency %with which
%of the case when a point $x_{t}$ in the $i$-th bin is followed by a point $x_{t'}$ in the $j$-th bin, {\em i.e.}, $t=t'+1$. Matrix $\mat{M}$ serves as the input of a heatmap image.
MTF is a matrix $\mat{M}\in\mathbb{R}^{Q\times Q}$ encoding the transition probabilities over time segments, where $Q$ is the number of segments. %Moreover,
ImagenTime \cite{naiman2024utilizing} stacks the delay embeddings $\mat{v}_{1}$, ..., $\mat{v}_{l}$ in Eq.~\eqref{eq.de} to an $l\times m\tau$ matrix for visualizing UTSs. %It also uses a variant of STFT.
% The method in \cite{homenda2024time} introduces five different 2D images by counting, rearranging, replicating the values in a UTS. 
MSCRED \cite{zhang2019deep} uses heatmaps on the $d\times d$ correlation matrices of MTSs with $d$ variates for anomaly detection. 
Furthermore, some methods use a mixture of imaging methods by stacking different transformations. \cite{wang2015imaging} stacks GASF, GADF, MTF to a 3-channel image. %Similarly,
FIRTS \cite{costa2024fusion} builds a 3-channel image by stacking GASF, MTF and RP. %the GASF, MTF, RP representations of each UTS.
%\cite{jin2023classification} combines Line Plot with Constant-Q Transform (CQT) \cite{brown1991calculation}, a method related to wavelet transform ($\S$\ref{sec.spectrogram}), to generate 2-channel images.
The mixture methods encode a UTS with multiple views and were found more robust than single-view images in these works for %time series
classification tasks.

\subsection{How to Model MTS}\label{sec.modelmts}

In the above methods, Heatmap ($\S$\ref{sec.heatmap}) can be %directly
used to visualize the %2D
variate-time matrices, $\mat{X}$, of MTSs ({\em e.g.}, Fig. \ref{fig.structure}(b)), where correlated variates %are better to
should be spatially close to each other. Line Plot ($\S$\ref{sec.lineplot}) can be used to visualize MTSs by plotting all variates in the same image \cite{wimmer2023leveraging,daswani2024plots} or combining all univariate images to compose a bigger %1-channel
image \cite {li2023time}, but these methods only work for a small number of variates. Spectrogram ($\S$\ref{sec.spectrogram}), GAF ($\S$\ref{sec.gaf}), and RP ($\S$\ref{sec.rp}) were designed specifically for UTSs. For these methods and Line Plot, which are not straightforward %for MTS transformation,
in imaging MTSs, the general approaches %to use them %for MTS
include using channel independence assumption to model each variate individually \cite{nie2023time}, %like VisionTS \cite{chen2024visionts},
or stacking the images of $d$ variates to form a $d$-channel image %as did by
\cite{naiman2024utilizing,kim2024cafo}. %\cite{prithyani2024feasibility,naiman2024utilizing,kim2024cafo}.
However, the latter does not fit some vision models pre-trained on RGB images which requires 3-channel inputs (more discussions are deferred to $\S$\ref{sec.processing}).

\vspace{0.2cm}

\noindent{\bf Remark.} As a summary, Table \ref{tab.tsimage} recaps the salient advantages and limitations of the five primary imaging methods that are introduced in this section.

% \hh{can we have a table (e.g., rows are different imaging methods and columns are a few desirable propoerties) or a short paragraph to discuss/summarize/compare the strenths and weakness of different imaging methods for ts? This might bring some structure/comprehension to this section (as opposed to, e.g., some reviewer might complain that what we do here is a laundry list)}

\section{Imaged Time Series Modeling}\label{sec.model}

With image representations, time series analysis can be readily performed with vision models. This section discusses such solutions from %traditional vision models %($\S$\ref{sec.cnns})
%to the recent large vision models %($\S$\ref{sec.lvms})
%and large multimodal models.% ($\S$\ref{sec.lmms}).
the traditional models to the SOTA models.

\begin{figure*}[!t]
\centering
\includegraphics[width=0.9\textwidth]{fig/fig_2.pdf}
% \vspace{-1em}
\caption{An illustration of different modeling strategies on imaged time series in (a)(b)(c) and task-specific heads in (d).}\label{fig.models}
\vspace{-0.2cm}
\end{figure*}

\subsection{Conventional Vision Models}\label{sec.cnns}

%Similar to
Following traditional %methods on
image classification, \cite{silva2013time} applies a K-NN classifier on the RPs of time series, \cite{cohen2020trading} applies an ensemble of fundamental classifiers such as %linear regression, SVM, Ada Boost, {\em etc.}
SVM and AdaBoost on the Line Plots %images
for time series classification. As an image encoder, %a typical encoder, %of images,
CNNs have been %extensively
widely used for learning image representations. %\cite{he2016deep}.
Different from using 1D CNNs on sequences %UTS or MTS
\cite{bai2018empirical}, %regular
2D or 3D CNNs can be applied on imaged time series as shown in Fig. \ref{fig.models}(a). %to learn time series representations by encoding their image transformations.
For example, %standard
regular CNNs have been used on Spectrograms \cite{du2020image}, tiled CNNs have been used on GAF images \cite{wang2015encoding,wang2015imaging}, dilated CNNs have been used on Heatmap images \cite{yazdanbakhsh2019multivariate}. More frequently, ResNet \cite{he2016deep}, Inception-v1 \cite{szegedy2015going}, and VGG-Net \cite{simonyan2014very} have been used on Line Plots \cite{jin2023classification,semenoglou2023image}, Heatmap images \cite{zeng2021deep}, RP images \cite{li2020forecasting,kim2024cafo}, GAF images \cite{barra2020deep,kaewrakmuk2024multi}, 
% Heatmaps \cite{zeng2021deep}, RPs \cite{li2020forecasting,kim2024cafo}, GAFs \cite{barra2020deep,kaewrakmuk2024multi},
and even a mixture of GAF, MTF and RP images \cite{costa2024fusion}. In particular, for time series generation tasks, %a diffusion model with U-Nets \cite{naiman2024utilizing} and GAN frameworks of CNNs \cite{li2022tts,karami2024timehr} have also been explored.%investigated.
GAN frameworks of CNNs \cite{li2022tts,karami2024timehr} and a diffusion model with U-Nets \cite{naiman2024utilizing} have also been explored.

Due to their small to medium sizes, these models are often trained from scratch using task-specific training data. %per task using the task's training set. %of time series images.
Meanwhile, fine-tuning {\em pre-trained vision models}  %such as those pre-trained on ImageNet, %\cite{deng2009imagenet}, 
have already been found promising in cross-modality knowledge transfer for time series anomaly detection \cite{namura2024training}, forecasting \cite{li2020forecasting} and classification \cite{jin2023classification}.

% \cite{li2020forecasting} uses ImageNet pretrained CNNs.

\subsection{Large Vision Models (LVMs)}\label{sec.lvms}

Vision Transformer (ViT) \cite{dosovitskiy2021image} has %given birth to
inspired the development of %some
modern LVMs %large vision models (LVMs)
such as %DeiT \cite{touvron2021training}, 
Swin \cite{liu2021swin}, BEiT \cite{bao2022beit}, and MAE \cite{he2022masked}. %Given an input image, ViT splits it
As Fig. \ref{fig.models}(b) shows, ViT splits an %input
image into {\em patches} of fixed size, then embeds each patch and augments it with a positional embedding. The %resulting
vectors of patches are processed by a Transformer %encoder
as if they were token embeddings. Compared to CNNs, ViTs are less data-efficient, but have higher capacity. %Consequently,
Thus, %the
{\em pre-trained} ViTs have been explored for modeling %the images of time series.
imaged time series. For example, AST \cite{gong2021ast} fine-tunes DeiT \cite{touvron2021training} on the filterbank spetrogram of audios %signals
for classification tasks and finds %using
ImageNet-pretrained DeiT is remarkably effective in knowledge transfer. The fine-tuning paradigm has also been %similarly
adopted in \cite{zeng2023pixels,li2023time} but with different pre-trained models %initializations
such as Swin by \cite{li2023time}. 
VisionTS \cite{chen2024visionts} %explains
attributes %the superiority of LVMs
LVMs' superiority over LLMs in knowledge transfer %over LLMs %as an outcome of
to the small gap between the pre-trained images and imaged time series. %the patterns learned from the large-scale pre-trained images and the patterns in the images of time series.
It %also
finds that with one-epoch fine-tuning, MAE becomes the SOTA time series forecasters on %many
some benchmark datasets.

Similar to %build
time series foundation models %\cite{das2024decoder,goswami2024moment,ansari2024chronos,shi2024time}, %such as TimesFM \cite{das2024decoder}, MOMENT \cite{goswami2024moment}, Chronos \cite{ansari2024chronos} and Time-MoE \cite{shi2024time},
such as TimesFM \cite{das2024decoder}, %and MOMENT \cite{goswami2024moment}, 
there are some initial efforts in pre-training ViT architectures with imaged time series. Following AST, SSAST \cite{gong2022ssast} introduced a %joint discriminative and generative
%masked spectrogram patch prediction self-supervised learning framework
masked spectrogram patch prediction framework for pre-training ViT on a large dataset -- AudioSet-2M. Then it becomes a backbone of some follow-up works such as AST-SED \cite{li2023ast} for sound event detection. %To be effective for UTSs,
For UTSs, ViTime \cite{yang2024vitime} generates a large set of Line Plots of synthetic UTSs for pre-training ViT, which was found superior over TimesFM in zero-shot forecasting tasks on benchmark datasets.

\subsection{Large Multimodal Models (LMMs)}\label{sec.lmms}

%As Large Multimodal Models (LMMs)
As LMMs %are getting
get growing attentions, some %of the
notable LMMs, such as LLaVA \cite{liu2023visual}, Gemini \cite{team2023gemini}, GPT-4o \cite{achiam2023gpt} and Claude-3 \cite{anthropic2024claude}, have been explored to consolidate the power of LLMs %on time series
and LVMs in time series analysis. 
Since LMMs support multimodal input via prompts, methods in this thread typically prompt LMMs with the textual and imaged representations of time series, %textual representation of time series and their %image transformations, transformed images,
%then instruct LMMs
and instructions on what tasks to perform ({\em e.g.}, Fig. \ref{fig.models}(c)).

InsightMiner \cite{zhang2023insight} is a pioneer work that uses the LLaVA architecture to generate %textual descriptions about
texts describing the trend of each input UTS. It extracts the trend of a UTS by Seasonal-Trend decomposition, encodes the Line Plot of the trend, and concatenates the embedding of the Line Plot with the embeddings of a textual instruction, which includes a sequence of numbers representing the UTS, {\em e.g.}, ``[1.1, 1.7, ..., 0.3]''. The concatenated embeddings are taken by a language model for generating trend descriptions. %It also fine-tunes a few layers with the generated texts to align LLaVA checkpoints with time series domain.
Similarly, \cite{prithyani2024feasibility} adopts the LLaVA architecture, but for MTS classification. An MTS is encoded by %a sequence of
the visual %token
embeddings of the stacked Line Plots of all variates. %meanwhile
%The method also stacks
%The time series of all variate are also stacked in a prompt % of all variates in a prompt
The matrix of the MTS is also verbalized in a prompt 
as the textual modality. %By manipulating token embeddings,
By integrating token embeddings, both %of these %works propose to
methods fine-tune some layers of the LMMs with some synthetic data.

Moreover, zero-shot and in-context learning performance of several commercial LMMs have been evaluated for audio classification \cite{dixit2024vision}, anomaly detection \cite{zhuang2024see}, and some synthetic tasks \cite{daswani2024plots}, where the image %({\em e.g.}, spectrograms, Line Plots)
and textual representations of a query %UTS or MTS
time series are integrated into a prompt. For in-context learning, these methods inject the images of a few example time series and their labels ({\em e.g.}, classes) %({\em e.g.}, classes, normal status)
into an instruction to prompt LMMs for assisting the prediction of the query time series.

\subsection{Task-Specific Heads}\label{sec.task}

%With the image embedding of a time series, the next step is to produce its prediction.
For classification tasks, most of the methods in Table \ref{tab.taxonomy} adopt a fully connected (FC) layer or multilayer perceptron (MLP) to transform an embedding into a probability distribution over all classes. For forecasting tasks, there are two approaches: (1) using a $d_{e}\times W$ MLP/FC layer to directly predict (from the $d_{e}$-dimensional embedding) the time series values in a future time window of size $W$ \cite{li2020forecasting,semenoglou2023image}; (2) predicting the pixel values that represent the future part of the time series and then recovering the time series from the predicted image \cite{yang2023your,chen2024visionts,yang2024vitime} ($\S$\ref{sec.processing} discusses the recovery methods). Imputation and generation tasks resemble forecasting %in the sense of predicting
as they also predict time series values. Thus approach (2) has been used for imputation \cite{wang2015imaging} and generation \cite{naiman2024utilizing,karami2024timehr}. %LMMs have been used for classification, text generation, and anomaly detection. For these tasks,
When using LMMs for classification, text generation, and anomaly detection, most of the methods prompt LMMs to produce the desired outputs in textual answers, circumventing task-specific heads \cite{zhang2023insight,dixit2024vision,zhuang2024see}.

%Forecasting: MLP, FC to predict numerical values using embeddings. Imputation of images (TSH). Classification: MLP, FC using embeddings.

\section{Pre-Processing and Post-Processing}\label{sec.processing}

To be successful in using vision models, some subtle design desiderata %to be considered
include {\bf time series normalization}, {\bf image alignment} and {\bf time series recovery}.

\vspace{0.2cm}

\noindent{\bf Time Series Normalization.} Vision models are usually trained on %images after Gaussian normalization (GN).
standardized images. To be aligned, the images introduced in $\S$\ref{sec.tsimage} should be normalized with a controlled mean and standard deviation, as did by \cite{gong2021ast} on spectrograms. In particular, as Heatmap is built on raw time series values, the commonly used Instance Normalization (IN) \cite{kim2022reversible} can be applied on the time series as suggested by VisionTS \cite{chen2024visionts} since IN share similar merits as Standardization. %although min-max normalization was used by \cite{karami2024timehr,zeng2021deep}.
Using Line Plot requires a proper range of y-axis. In addition to rescaling time series %by min-max or GN
\cite{zhuang2024see}, ViTST \cite{li2023time} introduced several methods to remove extreme values from the plot. GAF requires min-max normalization on its input, as it transforms time series values withtin $[0, 1]$ to polar coordinates ({\em i.e.}, arccos). In contrast, input to RP is usually normalization-free as an $\ell_{2}$ norm is involved in Eq.~\eqref{eq.rp} before thresholding.%for a comparison with a threshold.

\vspace{0.2cm}

\noindent{\bf Image Alignment.} When using pre-trained models, it is imperative to fit the image size to the input requirement of the models. This is especially true for Transformer based models as they use a fixed number of positional embeddings to encode the spacial information of image patches. For 3-channel RGB images such as Line Plot, it is straightforward to meet a pre-defined size by adjusting the resolution when producing the image. For images built upon matrices such as Heatmap, Spectrogram, GAF, RP, the number of channels and matrix size need adjustment. For the channels, one method is to duplicate a matrix to 3 channels \cite{chen2024visionts}, another way is to average the weights of the 3-channel patch embedding layer into a 1-channel layer \cite{gong2021ast}. For the image size, bilinear interpolation is a common method to resize input images \cite{chen2024visionts}. Alternatively, AST \cite{gong2021ast} %use cut and bilinear interpolation on
resizes the positional embeddings instead of the images to fit the model to a desired input size. However, the interpolation in these methods may either alter the time series or the spacial information in positional embeddings.

% single-channel (UTS), RGB channel (UTS), duplicate channels (UTS), multi-channel (MTS).

%Bilinear interpolation.

%Correlated variates are better to be spatially close to each other.

%\subsection{Pre-training}

\vspace{0.2cm}

\noindent{\bf Time Series Recovery.} As stated in $\S$\ref{sec.task}, tasks such as forecasting, imputation and generation requires predicting time series values. For models that predict pixel values of images, post-processing involves recovering time series from the predicted images. Recovery from Line Plots is tricky, it requires locating pixels that %correspond to
represent time series and mapping them back to the original values. This can be done by manipulating a grid-like Line Plot as introduced in \cite{yang2023your,yang2024vitime}, which has a recovery function. In contrast, recovery from Heatmap is straightforward as it directly stores the predicted time series values \cite{zeng2021deep,chen2024visionts}. Spectrogram is underexplored in these tasks and it remains open on how to recover time series from it. The existing work \cite{zeng2023pixels} uses Spectrogram for forecasting only with an MLP head that directly predicts time series. %predicts time series values.
GAF supports accurate recovery by an inverse mapping from polar coordinates to normalized time series \cite{wang2015imaging}. However, RP lost time series information during thresholding (Eq.~\ref{eq.rp}), thus may not fit recovery-demanded tasks without using an {\em ad-hoc} prediction head.


% Line Plot was regarded as matrices with rows and columns for mapping in \cite{sood2021visual}.


%\section{Tasks and Time Series Recovery}

%\subsection{Task-Specific Head}

% \subsection{Time Series Recovery}




The GFM-based RS effectively utilize the technological complementarity of GNN and LLM. GNNs struggle to model textual information, while the reasoning capabilities of LLMs do not support their comprehension of higher-order structural information. These two technologies complement each other's shortcomings in GFM, which emerges as a future opportunity in the field of recommendations. For example, LLMGR~\cite{guo2024integrating} injects the embeddings learned by GNN into the token embedding sequence of LLM, and adapts the GFM to the recommendation task through two-stage fine-tuning. LLMRG~\cite{wang2023enhancing} constructs inference graphs and divergence graphs based on user interaction history using LLM, which are then encoded by GNN for recommendations. DALR~\cite{peng2024denoising} aligns the embeddings encoded by GNN and those encoded by LLM in various ways, using the aligned embeddings for subsequent recommendations.

In this survey, we comprehensively investigate the relevant work of GFM-based RS, and provide a clear taxonomy based on the synergistic relationship between the graph and LLM in GFM: \textbf{Graph-augmented LLM}, \textbf{LLM-augmented graph} and \textbf{graph-LLM harmonization}.
Graph-augmented LLM methods can be viewed as utilizing the structural information of the graph to aid the knowledge obtained from LLM pre-training for recommendations. LLM-augmented graph methods, on the other hand, is led by the structural information of the graph, with the world knowledge of LLM serving as auxiliary information. Graph-LLM harmonization methods involve the equal transformation of these two types of information in the representation space. 

% As an evergreen topic in both academia and industry, numerous surveys have been conducted on recommender systems \cite{gao2023survey,wu2024survey}. The former provides a comprehensive review of graph-based recommender systems, representing traditional methodologies, while the latter offers an overview of LLM based recommender systems, representing a new paradigm. While the previous two surveys offer detailed insights into the respective technologies, they were unaware of the rapid development of GFM in the field of recommendations. Therefore, our survey offers a broader perspective for extensive research related to recommendations.

As an evergreen topic in both academia and industry, RS have been the subject of numerous surveys (e.g., \cite{gao2023survey,wu2024survey,liu2023towards,li2023survey}). \cite{gao2023survey,wu2024survey} focus on specific methodologies, such as GNN-based RS or the more recent LLM-based RS. \cite{li2023survey} concentrates on utilizing LLM to enhance graphs for tackling tasks related to graphs. However, the field is rapidly evolving with GFMs emerging as a crucial technique of the RS research. \cite{liu2023towards} systematically outlines the existing GFMs from the perspectives of pre-training and adaptation, while overlooking the recommendation which is one of the significant downstream tasks for GFM. This survey provides a timely and comprehensive overview that covers the landscape of GFM-based recommender systems.

The contributions of this survey can be summarized in the following aspects:\textbf{1)} \textit{Pioneering overview}: Our survey fills the blank in comprehensive work in the field of GFM-based RS. \textbf{2)} \textit{Clear taxonomy}: The comprehensive survey presents a well-structured taxonomy of GFM-based RS, allowing future work to be easily categorized within the corresponding branches. \textbf{3)} \textit{Promising outlook}: We present the challenges and future research directions in this field, which can serve as a valuable reference for research in this rapidly evolving area.
% This survey provides the first systematic review of graph foundation models for recommendation, offering several key contributions to the field:  

% 1. \textbf{A Novel Classification Framework}: We propose a comprehensive framework to categorize the GFM into three paradigms: Graph-augmented LLMs, LLM-augmented graphs, and LLM-graph harmonization in recommendation. This taxonomy provides a clear roadmap for understanding the field and guiding future research.

% 2. \textbf{Methodological Review}: We conduct an in-depth analysis of methodologies within each paradigm, discussing their theoretical foundations, design strategies, and real-world applications. Representative studies are examined to highlight their contributions to solving key recommendation challenges.

% 3. \textbf{Challenges and Future Directions}: Through meticulous literature synthesis, we unveil major challenges in this field, such as alignment of representations, computational efficiency, scalability, and integration complexity. Simultaneously, we spotlight prospective avenues for future research, including adaptive integration methods, cross-modal fusion, and efficient large-scale deployment strategies. Our analysis and insights aim to both address these current challenges and inspire future innovation, guiding researchers to unlock the full potential of integrating graph and LLM technologies.

\section{Related Work}
\label{sec:rw}

Our work lies at the intersection of three lines of inquiry: research on technologies supporting health services (Section \ref{sec:rw:tech-services}), mental health data collection and storage (Section \ref{sec:rw:data}), and value-based mental healthcare (Section \ref{sec:rw:vbc}).

\subsection{Designing Technologies for Health Services}
\label{sec:rw:tech-services}

In this work, we studied technologies that support value-based care and the delivery of \textit{health services}, which encompass the people, organizations, and technology involved in healthcare delivery \cite{issues_working_1994, sanford_schwartz_chapter_2017}.
These people and organizations include \textit{healthcare providers}, the clinicians or hospital systems that provide treatments or preventive care (the ``services''); as well as \textit{healthcare payers}, the government agencies or private health insurance companies that pay for health services.
We review specific technologies supporting mental health services in Section \ref{sec:rw:data}.
To design technologies for health services, we need to confront more than the hardware or software capabilities of a specific technology, or the effectiveness of interventions that use technologies to improve health outcomes.
We also need to confront sociotechnical factors that affect the implementation and effectiveness of these technologies in real-world care. 
Norman and Stappers categorize sociotechnical factors that affect technology implementation as political, economic, cultural, organizational, and structural \cite{norman_designx_2015}.
Blandford states that, for health services specifically, HCI scholars should \textit{``consider stages (of identifying technical possibilities or early adopters and planning for adoption and diffusion) that are rarely discussed in HCI, but that are necessary to deliver real impact from HCI innovations in healthcare''} \cite{blandford_hci_2019}.
Thus, we were motivated to improve the design of technologies supporting health services by understanding factors that affect their implementation and adoption in care.

Recently, HCI scholars have considered adopting ideas from health services research to improve both the design and effectiveness of health technologies.
Scholars have considered how HCI research can integrate aspects of \textit{implementation science} -- the health services field examining the real-world adoption of evidence-based interventions \cite{lyon_bridging_2023}. 
Interviews with HCI and implementation science researchers uncovered that HCI tends to de-prioritize factors that influence long-term adoption of technologies in their initial design, including the financial incentives that affect adoption, and an understanding of how technologies support providers after implementation \cite{dopp_aligning_2020}.
Moreover, HCI scholars have stated that if technologies are to impact real-world care, HCI researchers should focus on how technology is consumed in care, including developing an understanding of the technical and market incentives to use new tools \cite{colusso_translational_2019}.
Inspired by this work, we considered these aspects of adoption in the initial design of technologies that support value-based mental healthcare.
Specifically, we considered how technologies can support healthcare providers -- practicing clinicians -- including how these technologies can be integrated into clinicians' workflows to support care, and the financial incentives that influence HIT adoption as a part of value-based care.

\subsection{Health Information Technologies for Collecting and Storing Mental Health Data}
\label{sec:rw:data}

HCI, health informatics, and mental health researchers have collaborated to build health information technologies (HITs) for collecting and storing mental health data.
In this work, we focus on three categories of mental health data: clinical data, active data, and passive data.
\textit{Clinical data} can be retrieved from \textit{electronic health records} (EHRs), which record information collected during clinical visits including patient demographics, diagnoses, health and family history, treatments provided, and unstructured clinical notes \cite{birkhead_uses_2015}.
\rev{That said, to protect patient privacy, not all mental health data may be contained within the EHR, and exporting EHR data for VBC may require patient consent \cite{shenoy_safeguarding_2017, leventhal_designing_2015}.}
Clinical data can also be retrieved from \textit{administrative claims databases}, which log diagnostic, treatment, and medication information used to bill healthcare payers \cite{karve_prospective_2009, davis_can_2016}.
Clinics or hospitals may also collect measures of patient satisfaction to understand patients' perceptions of their care \cite{carr-hill_measurement_1992}.

\textit{Active data} require active patient or clinician engagement to be collected, and can be collected with technologies that support digital surveys (eg, smartphones, iPads, computers, \rev{patient portals}) and pen-and-paper questionnaires.
This data include validated self-reported \textit{measures of mental health symptoms}, which quantify symptom presence and/or severity for specific mental health disorders, such as the PHQ-9 for major depressive disorder \cite{kroenke_phq-9_2001}, or the GAD-7 for generalized anxiety disorder \cite{spitzer_brief_2006}.
Active data can also include clinician-rated scales, collected during clinical interviews \cite{andersen_brief_1986}.
Outside of symptoms, self-reported and clinician-rated measures can also quantify \textit{functioning}, as mental health symptoms can impair functioning including cognition, mobility, self-care, and sociality \cite{ustun_measuring_2010}. 
Self-reported measures can also quantify how well patients and their mental health clinicians collaborate towards shared goals, complete tasks, and bond, called \textit{working alliance} \cite{hatcher_development_2006}.
The discussed scales typically quantify persistent symptoms or functional impairment.
Researchers have used everyday devices, such as smartphones, to collect more in-the-moment symptoms via questionnaires called ecological momentary assessments (EMAs) \cite{wang_crosscheck_2016, hsieh_using_2008}.
EMAs can also collect \textit{engagement data}, measuring, for example, medication adherence, or participation in behavioral interventions, such as mindfulness exercises \cite{militello_digital_2022, klasnja_how_2011}.
Active data can be stored in clinical records, like an EHR, but significant investments have not been made to build structured EHR fields for storing active data \cite{pincus_quality_2016}.

In addition to active data, sensors embedded in devices (eg, smartphones, wearables) and online platforms have created opportunities to collect \textit{passive data} -- data collected with little-to-no effort -- on behavior and physiology \cite{nghiem_understanding_2023}.
Passive data can be used to estimate signals related to functioning, including social behaviors, mobility, and sleep \cite{mohr_personal_2017, saeb_relationship_2016, saeb_scalable_2017}, and more recently, researchers have investigated if passive data can measure engagement in therapeutic exercises \cite{evans_using_2024}.
Prior work has also studied whether passive data can estimate symptom severity \cite{adler_measuring_2024, das_swain_semantic_2022, meyerhoff_evaluation_2021, currey_digital_2022}.
The use of passive data in treatment is limited: \rev{while passive data can be collected within EHRs \cite{apple_healthcare_2024, metrohealth_track_2024, pennic_novant_2015}, established clinical guidelines for passive data use in care do not exist, and use is often limited to patients who are motivated to share passive data with their healthcare provider \cite{nghiem_understanding_2023}}.

It is challenging to identify what mental health data are most relevant to HITs in certain contexts, given their variety.
Li et al. proposed a 5-stage model to work through these challenges, specifically in the context of \textit{personal informatics systems}, where users collect data for self-reflection and gaining self-knowledge.
These five stages are preparation, collection, integration, reflection, and action \cite{li_stage-based_2010}.
In this work, we study how HITs can support mental health outcomes data as a part of value-based mental healthcare, inspired by three out of these five stages, specifically \textit{preparation}, understanding what data to collect; \textit{collection}, gathering data; and \textit{action}, how data is used.
We focus on these three stages because they capture existing challenges to design HITs that support VBC, which we review in Section \ref{sec:rw:vbc}.

\subsection{Value-based Mental Healthcare}
\label{sec:rw:vbc}
The World Economic Forum defines \textit{value-based care} (VBC) as a \textit{``patient-centric way to design and manage health systems''} and \textit{``align industry stakeholders around the shared objective of improving health outcomes delivered to patients at a given cost''} \cite{world_economic_forum_value_2017}.
VBC intends to change how healthcare is paid for, away from \textit{fee-for-service} payment models -- where payers reimburse providers for the number of services they provide -- towards paying for services if they deliver ``value'' to the healthcare system \cite{brown_key_2017}.
In practice, VBC is implemented by paying providers a set rate for managing patients' health, sharing savings if specific cost or utilization targets are met, and/or by offering financial incentives for payers and providers based upon \textit{quality measures}, which quantify the ``value'' of care \cite{world_economic_forum_moment_2023, health_care_payment_learning__action_network_alternative_2017}.
These changes shift some of the financial risk of healthcare from payers to providers.
In fee-for-service models, providers continue to be paid as they provide more services.
In VBC, providers may lose money if services cost more than set rates, specific cost/utilization targets are not met, or if care quality suffers \cite{novikov_historical_2018, health_care_payment_learning__action_network_alternative_2017}.

Standardized quality measures guide payers and providers to deliver services that improve health outcomes and reduce cost.
% Quality measures can be derived from administrative claims, EHRs, and patient self-report; are validated for their reliability and validity; importance for improving quality; feasibility to collect; and are certified by country-specific organizations like the NCQA in the United States, or the National Institute for Health and Care Excellence (NICE) in the UK \cite{center_for_medicare__medicaid_services_your_2021, national_institute_for_health_and_care_excellence_nice_2019}
The Donabedian model categorizes quality measures into three areas: (1) \textit{structure} -- the material, human, and organizational resources used in care (eg, the ratio of patients to providers); (2) \textit{process} -- the services provided in care (eg, the percentage of patients receiving immunizations); and (3) \textit{outcomes} -- measuring the effectiveness of care (eg, surgical mortality rates) \cite{donabedian_quality_1988, endeshaw_healthcare_2020,agency_for_healthcare_research_and_quality_types_2015}.
% Each category of measures has strengths and weaknesses.
While structure and process measures are more actionable -- hospital systems can hire more staff, or modify care practices -- their relationship to outcomes can be ambiguous \cite{quentin_measuring_2019}. 
In contrast, outcome measures most clearly represent the goals of care, but can be biased by factors outside of providers' direct control, including co-occurring health conditions that complicate treatment success \cite{lilienfeld_why_2013, quentin_measuring_2019}.
To reduce bias, statisticians apply a \textit{risk-adjustment} to outcome measures, using regression to model expected care outcomes observed in real-world data, based upon variables known to moderate treatment effects \cite{lane-fall_outcomes_2013}.
The quality of provided health services for a specific patient can then be determined based upon whether a patient's health outcomes exceed or underperform expectations.

Mental healthcare has faced specific challenges implementing VBC.
Some of these challenges can be attributed to ambiguity on how to design health information technologies (HITs) that store outcomes data tying provided services to value \cite{world_economic_forum_value_2017}.
\textit{Preparation challenges} revolve around identifying standardized outcome metrics to store in HITs.
Current quality monitoring programs incentivize using symptom scales as standardized care outcomes \cite{morden_health_2022}.
Patients often experience a unique constellation of symptoms that cut across multiple disorders (eg, major depressive disorder and generalized anxiety disorder) \cite{boschloo_network_2015, cramer_comorbidity_2010, barkham_routine_2023}, making it difficult to identify a limited set of symptom scales to track outcomes across patients.
Given these challenges, researchers have proposed using other data types as an alternative to symptom scales within VBC \cite{hobbs_knutson_driving_2021, oslin_provider_2019}. 
For example, scholars and healthcare providers have argued that functional and engagement outcomes may be a promising alternative to symptom scales. 
Engagement is the proximal outcome of many mental health treatments, improved functioning is often more important to patients than symptom reduction, and functional outcomes measure treatment progress across patients living with different mental health symptoms or disorders \cite{stewart_can_2017, tauscher_what_2021, pincus_quality_2016}.

In terms of \textit{data collection}, it is estimated that less than 20\% of mental health clinicians practice measurement-based care (MBC) -- the process of collecting, planning, and adjusting treatment based on outcomes data -- specifically symptom scales \cite{zimmerman_why_2008, fortney_tipping_2017}, despite evidence that MBC improves outcomes \cite{barkham_routine_2023}. 
MBC is usually implemented by having patients routinely self-report symptoms during clinical encounters using validated symptom scales, like the PHQ-9 for depression, or the GAD-7 for anxiety \cite{wray_enhancing_2018}.
Mental health clinicians choose to not practice MBC for many reasons. 
Electronic health records (EHRs) often do not have standardized fields to support symptom data collection, clinicians perceive that symptom scale administration disrupts the therapeutic relationship, and clinicians are often not paid to administer symptom scales \cite{lewis_implementing_2019, desimone_impact_2023, oslin_provider_2019}.
These barriers call for work centering mental health providers in designing HITs that effectively engage providers in outcomes data collection.

\textit{Action} challenges stem from both perceptions of how outcomes data could be used in care, and challenges towards attributing accountability for care.
For example, clinicians are often not trained to use outcomes data in care, and worry that they will be held accountable and penalized if outcomes data reveal that their patients are not improving \cite{lewis_implementing_2019, desimone_impact_2023}.
There are also concerns that outcomes data could be gamed: biased reporting that artificially inflates performance metrics \cite{kilbourne_measuring_2018}.
In addition, it is difficult in mental healthcare to attribute accountability to specific actors (eg, specific providers) in care systems.
Mental healthcare is often ``siloed'' from physical healthcare, though both physical and mental health outcomes are strongly intertwined (eg, individuals living with schizophrenia suffer from chronic physical health conditions) \cite{pincus_quality_2016}.
Thus, existing value-based mental healthcare programs may hold both physical and mental health clinicians \textit{jointly accountable} by sharing cost savings across different types of providers \cite{hobbs_knutson_driving_2021}.

Taken together, this prior work demonstrates challenges designing HITs that support value-based mental healthcare.
Integral to the design of these HITs are mental health clinicians, who are asked to participate in outcomes data collection, which clinicians have found challenging, and will be held financially accountable to the outcomes data HITs store.
Given these challenges, this work centers mental health clinicians' perspectives on how to design HITs that support value-based mental healthcare.
By centering clinicians' perspectives, we looked to gain a deeper understanding of their workflows and incentives to adopt HITs, and integrate this knowledge into the design and development of HITs supporting value-based care. 
The following section details the methodology used in this study.
\section{Problem Formulation}
\subsection{System Overview}
\label{subsec:sys-overview}

%\xiaoxi{Please replace the crowdsensing task to Scenario I: large-model fine-tuning tasks and change the AoI metric to test accuracy. The large-model driven task can be for entertainment supported by future IVI systems. For each vehicle, it can collect data at a PoI, either does local fine-tuning (meaning it has to pull from the RSU and store a large-language model and LoRA adapters), or uploads data to the RSU for fine-tuning at the RSU. The input data for fine-tuning has temporal-spacial heterogeneity. Specifically, the data at different PoI locations are non-IID and have different data sizes. Therefore, the sequence of locations each vehicle arrives to collect the data affects the accuracy, similar to conventional training. Here, we consider fine-tuning because vehicles can't do entire model training due to resource limits.} 

%\xiaoxi{Section \ref{subsec:sys-overview} should describe a problem overview with system architecture, where AI-defined ride-hailing vehicles can do a traditional task, which is serving orders by picking up passengers, and performing AI tasks when moving around. The considered AI tasks include LLM tasks that need continuous fine-tuning and inference for various future infotainment services, e.g., large-model based recommendation or generative AR/VR gaming. Given the storage and computation capacities, vehicles can do light-weight LoRA-based fine-tuning when encountering task shift or inference accuracy decay.}
\begin{table}[t]

\caption{Key Notations with Description.}
\label{table:notation}
\centering
\begin{threeparttable}
\begin{tabular}{c p{6cm}}
\toprule
Notation & Explanation \\
\midrule
{$m$, $M$, $\mathcal{M}$} & Vehicle index, number of vehicles, Vehicle set \\

\midrule
{$g$, $G$, $\mathcal{G}$} & Grid index, number of grids, Grid set\\

\midrule
{$t$, $T$, $\mathcal{T}$} & Time slot index, number of time slots, time slot set\\

\midrule
{$o$, $\mathcal{O}_t$} & Order index, Order set\\

\midrule
{$p$, $\mathcal{P}_t$} & PoI index, PoI set\\

\midrule
{$g_t^m$} & Index of grid vehicle $m$ is located\\

\midrule
{$\mathcal{N}(g)$} & Neighboring grids set of grid $g$\\

\midrule
{$\sigma(o)$} & Price of order $o$\\

\midrule
{$d\left(p\right)$} & Data volume of PoI $p$\\

\midrule
{$\eta$} & The rank of the low-rank matrices in LoRA\\

\midrule
{$\lambda_{t}^p$} & Data AoI of PoI $p$\\

\midrule
{$u_{t}^p$} & The data utility of the PoI $p$\\

\midrule
{$\textbf{x}$, $\textbf{y}$, $\textbf{z}$} & Vehicle dispatching, order accepting, data collecting\\


\midrule
{$s_{m,t}$, $a_{m,t}$, $r_{m,t}$} & State, action and reward of vehicle $m$\\

\bottomrule
\end{tabular}
\begin{tablenotes}   
        \footnotesize               
        \item[1] $\bullet$ We omit descriptions such as ``at time slot $t$".     
      \end{tablenotes}           
    \end{threeparttable}

\end{table}

In this section, we provide a detailed description of our scenario. We consider a vehicular network, consisting of a large number of moving ride-hailing vehicles that are managed by a cloud platform (such as DiDi or Urber) within a certain geographical range and can interact with RSUs equipped with edge servers. Beyond the routine operations of picking up and dropping off passengers, each vehicle actively engages in urban sensing tasks. Leveraging the installed professional sensors and smartphones, the vehicles can collect data from PoIs distributed across the designated area. We define $\mathcal{M} \triangleq \left\{m | m = 1,2,\cdots, M\right\}$ to represent the set of vehicles in the system. The activity range of each vehicle is limited to the target area. Similar to many prior studies \cite{KDD18, KDD22}, we discretize the target area into $G$ grids, represented by set $\mathcal{G} \triangleq \left\{g|g = 1, 2, \cdots, G \right\}$. Similarly, the time horizon is also divided into several discrete time slots, represented by $\mathcal{T} \triangleq \left\{t| t = 1, 2, \cdots, T \right\}$. The set of orders and the set of PoIs at time slot $t$ are $\mathcal{O}_t$ and $\mathcal{P}_t$, respectively. At time slot \( t \), the set of orders is denoted as \( \mathcal{O}_t \), and the set of Points of Interest (PoIs) is represented as \( \mathcal{P}_t \). 
% \broken{
To account for task-specific requirements, \( \mathcal{P}_t \) can be decomposed into subsets, where each subset corresponds to a distinct type of data PoI associated with a particular task. Formally, this decomposition can be expressed as:  
\begin{align}
    \mathcal{P}_t = \bigcup_{k \in \mathcal{K}} \mathcal{P}_{t,k}, \quad \mathcal{P}_{t,k} \cap \mathcal{P}_{t,j} = \emptyset, \, \forall k \neq j,
\end{align}
where \( \mathcal{K} \) is the set of task types, and \( \mathcal{P}_{t,k} \) denotes the subset of PoIs related to task \( k \). For instance, \( \mathcal{P}_{t,1} \) represent PoIs contributing to Vision Transformers (ViT)\cite{dosovitskiy2020image} based tasks such as image classification, while \( \mathcal{P}_{t,2} \) corresponds to PoIs relevant to SAM-based tasks like image segmentation. 
% }
% \xiaoxi{Should be further decompose $\mathcal{P}_t$ into subsets, each of which represents a set of data PoIs for a distinct task, e.g., ViT-based tasks should be different from BERT-based??}
Since the number of orders and PoIs change at each time slot, the size of $\mathcal{O}_t$ and $\mathcal{P}_t$ can be distinct across $t$. We consider vehicles that currently not serving orders or collecting data as available vehicles. For grid $g$, we use $\mathcal{O}_t^g$, $\mathcal{M}_t^g$ and $\mathcal{P}_t^g$ respectively to represent the set of orders, the set of available vehicles, and the set of PoIs in grid $g$ at time slot $t$. The index of the grid (interchangeable with location in this paper) where vehicle $m$ is located at time slot $t$ is $g_t^m$, where we have $g_t^m \in \mathcal{G}$. To maximize the effectiveness of order-serving and data collection, each available vehicle $m$ needs to decide whether to accept an order $o \in \mathcal{O}_t^{g_t^m}$, collect data from a PoI $p \in \mathcal{P}_t^{g_t^m}$, or travel to another grid. Once an available vehicle 
$m$ accepts an order or collects data from a PoI, it becomes unavailable and will revert to being available again after completing passenger drop-off or data collection. For convenience, some of the important notations used in the paper are listed in Table \ref{table:notation}.

\subsection{Primer on Model Fine-tuning with LoRA}
\label{ssec:primer-lora}
% \broken{
Low-Rank Adaptation (LoRA) is a powerful technique designed to efficiently fine-tune large pre-trained models, without the need to retrain all model parameters. Traditional fine-tuning methods often involve updating the entire model’s weights, which can be computationally expensive and resource-intensive, especially with massive models. LoRA addresses this by observing that the difference between the pre-trained weights and the fine-tuned weights often lies in a low-dimensional subspace. Thus, LoRA introduces a low-rank approximation to model these changes, significantly reducing the number of parameters that need to be updated. The benefits of LoRA include a substantial reduction in memory and computational costs, as well as the ability to adapt large models to new tasks with minimal overhead.
Formally, LoRA modifies the weight matrix $W_0 \in \mathbb{R}^{d \times k}$ of a pre-trained model by introducing a low-rank update, represented as $W_0 + \Delta W = W_0 + BA$, where $B \in \mathbb{R}^{d \times \eta}$ and $A \in \mathbb{R}^{\eta \times k}$, with $\eta \ll \min(d, k)$. The matrices $B$ and $A$ are the only trainable parameters, while the original weights $W_0$ are frozen.  The model input is denoted as $x$, and the output is represented as $h$. The forward pass incorporating the LoRA module is given by:
\begin{align}
h = W_0 x + \gamma \eta BA x,
\end{align}
where \( \gamma \) is a scaling factor, and \( \eta \) is the rank of the low-rank matrices. This formulation shows how LoRA modifies the original model by adding a low-rank update in parallel to the pre-trained weights. The rank \( \eta \) controls the capacity of the low-rank adaptation, allowing for efficient fine-tuning with minimal changes to the original model. Additionally, LoRA preserves computational efficiency during inference, making it particularly useful for deployment in resource-constrained environments.
% }
% \bo{Need to add: $x$ is the model input, $h$ is output}
% \xiaoxi{Boken, please add a basic introduction (a short paragraph) to describe the LoRA technique (its goal, benefits, and so on), and then add some basic formulations for the readers to learn what LoRA is.}


\subsection{QoS and Optimization Modeling for Vehicular Joint Tasks}
\label{ssec:optimization-qos}

Subsequently, we define the Accumulated Driver Income (ADI), the Accumulated Data Utility (ADU), and the QoS. Following this, we give the mathematical form of the problem and the optimization goal.

\noindent\textbf{Definition 1. (ADI)} We use $\mathcal{O}_t^{'g}$ to denote the set of orders accepted by vehicles in grid $g$ at time slot $t$. For order $o$ within the set $\mathcal{O}_t^{'g}$, we denote by $\sigma(o)$ the price of order $o$. ADI represents the total income of all drivers over all time slots, so the expression for ADI is as follows: 
\begin{align}
ADI 
    = \sum\limits^{T}_{t=1} \sum\limits^{G}_{g=1}\sum\limits^{}_{o \in \mathcal{O}_t^{'g}} \sigma(o).
\end{align}

There are several PoIs distributed within each grid, and vehicles can collect data from these PoIs. Each PoI $p \in \mathcal{P}_t$  is associated with a certain volume of data $d(p)$ that needs to be collected. To maintain data integrity and prevent excessive bandwidth usage, we assume that an available vehicle can collect data from only one PoI at a time, and the collection process continues until the pending data volume at the PoI is reduced to 0. 




%%%%%%%%%%%%% AoI and CDV %%%%%%%%%%%%%%%%%%%%%%%%%%%


% In numerous tasks emphasizing the timeliness of data, age-of-information (AoI) is commonly employed to characterize data freshness \cite{introduce_AoI1, introduce_AoI2}. In this study, we utilize the elapsed time from data generation to the current time slot as the representation of data AoI. In detail, $\lambda_t^{p}$ represents the AoI of the PoI $p$ at time slot $t$. We establish that the AoI for a newly generated PoI is 1, with the AoI incrementing by 1 for each subsequent time slot, i.e.,
% \begin{align}
%     \lambda_{t+1}^p = \lambda_t^p + 1.
%     \label{equ:aoi increase}
% \end{align}
% When the AoI reaches a certain threshold, the PoI data will be cleared.

% We define CDV to denote the total volume of data collected by vehicles over all time slots, which is expressed as:
% \begin{align}
% CDV = \sum\limits^{T}_{t=1} \sum\limits^{G}_{g=1}
% \sum\limits^{}_{p \in \mathcal{P}_t^{'g}}
% b\left(p\right).
% \end{align}
%{and the volume of data collected in a grid will not be greater than the volume of data in the grid, i.e.:
%\begin{align}
%\sum \limits^{}_{m \in {\mathcal{C}_t^g}}d_t^{m} \leq b_t^g,\quad %\forall g \in \mathcal{G} \quad \forall t \in \mathcal{T}.
%\end{align}
%}



% \textbf{Definition 3. (AoI)} In many tasks that focus on data timeliness, AoI is often used to describe the freshness of data. In this paper, we use the time elapsed from the generation of the data to the current time slot to represent the AoI of the data. In detail, $\lambda_t^{p}$ represents the AoI of the PoI $p$ at time slot $t$. We set that the AoI of the newly generated PoI is 1, and the AoI increases by 1 for each time slot, i.e.,
% \begin{align}
%     \lambda_{t+1}^p = \lambda_t^p + 1.
%     \label{equ:aoi increase}
% \end{align}

%We define an indicator $u$ to comprehensively evaluate the volume and AoI of the collected data. The utility of the data collected from PoI $p$ is denoted as $u_t^p$, which is the ratio of data volume to AoI, i.e.,
% \begin{align}
% \label{equation: data utility}
% u_t^{p} = \frac{d(p)}{\lambda_t^{p}}.
% \end{align}


%%%%%%%%%%%%% END: AoI and CDV %%%%%%%%%%%%%%%%%%%%%%%%%%%




\noindent\textbf{Definition 2. (ADU)} We define ADU as the sum of the data utility collected by all vehicles in all time slots. The expression for ADU is defined as:
\begin{align}
ADU =  \sum\limits^{T}_{t=1} \sum\limits^{G}_{g=1} \sum\limits^{}_{p \in {\mathcal{P}}_t^{'g}} u_t^p,
\end{align}
where $\mathcal{P}_t^{'g}$ is the set of PoIs collected in grid $g$ at time slot $t$. We use ADU to indicate the quality of VCS in the following sections.

\begin{figure}[t]
\hspace{-1cm}  % 向左调整图片的位置
\subfigure[Accuracy as a function of data freshness or AOI for two different tasks.]{
\label{fig: sub.1 aoi}
\includegraphics[width=0.5\textwidth]{figure/aoi.png}}
\subfigure[Accuracy as a function of data volume for the same tasks.]{
\label{fig: sub.2 data_volume}
\hspace{-0.4cm}  % 向左调整图片的位置
\includegraphics[width=0.5\textwidth]{figure/data_volume.png}}
\caption{The impact of data freshness and data volume on the fine-tuning accuracy of different tasks under different UFMs.}
\label{fig: impact_on_fine-tune}
\end{figure}

\noindent{\bf Freshness and quantity based data utility.} Many existing studies tend to overlook the importance of data freshness in model fine-tuning, yet we explicitly take into account the effects of using various freshness degrees of collected data to fine-tune the foundation model using LoRA techniques, and we focus on the inference accuracy of such fine-tuned models varying data freshness. Intuitively, the freshness of data should hold significant relevance in numerous real-time VCS tasks \cite{the_reason_why_data_freshness_is_important}. For instance, in traffic monitoring, the most recent traffic data is more beneficial for intelligent transportation systems \cite{a_example_for_the_reason_why_data_freshness_is_important}. But these works are not foundation fine-tuning tasks and the task performance has different metrics from ours. 
% \broken{
To verify this intuition, we conducted a series of experiments on various large model fine-tuning tasks, including LoRA-based fine-tuning for image classification using ViT and image segmentation using SAM. As Fig.~\ref{fig: impact_on_fine-tune} show, the relationship between fine-tuning accuracy and data freshness exhibits a complex and often unpredictable pattern. Specifically, accuracy tends to follow a concave or even linear decay as data freshness decreases, though the exact functional form is task-dependent and difficult to predict. This underscores the challenge of modeling data freshness in real-world applications. To address this uncertainty, we propose leveraging DRL to learn optimal fine-tuning strategies for varying data freshness. Additionally, our experiments reveal that accuracy improves with the volume of data, but the relationship is not linear. As the amount of data increases, the fine-tuning accuracy shows a logarithmic growth, and beyond a certain threshold, further increases in data volume yield diminishing returns. This phenomenon is consistent with the scaling laws for neural language models\cite{kaplan2020scaling}, which highlights the diminishing impact of additional training data after reaching a critical dataset size. These results emphasize the importance of considering both data freshness and quantity when fine-tuning UFMs, and highlight the potential of DRL in optimizing these factors for improved task performance.
% }
% \xiaoxi{Please elaborate that we did our own experiments to verify this intuition for various large model fine-tuning tasks, and highlight the specific function forms (e.g., maybe concave or convex, or linear decay) are uncertain and hard to predict, so as to motivate using DRL to learn. Maybe put two figures showing the curves of acc w.r.t. staleness of two different tasks here; then put two figures of curves of acc w.r.t. data volumes of two corresponding tasks here.} %Considering these factors, we establish an inverse relationship between data utility and AoI in equation (\ref{equation: data utility}) to depict the impact of AoI. 

\smallskip
\noindent{\bf Spatio-temporal heterogeneity and task differences.} 
% ~\broken{
Data collection for fine-tuning tasks exhibits significant spatio-temporal heterogeneity and task-specific variations. Spatially, different PoIs are associated with distinct distributions of collected datasets, influenced by factors such as regional activity patterns, environmental features, and task-specific requirements. These datasets are further tied to varying types of tasks and corresponding base models, such as image classification using ViT or image segmentation with SAM. This spatial diversity necessitates careful consideration of the data's regional context when fine-tuning models. Temporally, the heterogeneity arises from the varying arrival times of vehicles at PoIs. As vehicles may visit a PoI at different times, the freshness of the collected data naturally declines over time, as discussed in the previous section. This temporal decay in data freshness introduces additional complexity to tasks that rely on timely and accurate data for model fine-tuning. Balancing these temporal factors is critical for maintaining the relevance and utility of the collected data.
% }
% \xiaoxi{First, please emphasize different locations of data PoI have different distribution of fine-tuning datasets. Also, these datasets may be associated with different tasks and base models (ViT, SAM, etc.) Then, the temporal heterogeneity is reflected in that vehicles may come to any given PoI at different times; if they come later, the data freshness may be decreased, as introduced in the above paragraph. Then, talk about the utility function of order serving is different from that of fine-tuning tasks:}
For better comprehension, we consider the price of an order as the utility of the order. The utility of an order remains constant from the time of its creation until its expiration, during which the utility that a driver can obtain by accepting the order is equivalent to the order's price. If the order expires, its utility abruptly drops to 0. The variation trends in the utility of orders and PoIs are distinct, implying that accepting orders or collecting data at suitable times and locations may potentially minimize the impact between the profit generated from passenger transportation and the utility derived from data collection. 
% \begin{figure}[t]
% \centerline{\includegraphics[width=0.6\linewidth]{figure/difference rewrd.png}}
% \caption{The value of orders and data changes over time.
% }
% \label{fig: order and data change}
% \end{figure}


\noindent\textbf{Definition 3. (QoS)} To finally capture the overall performance of order-serving and fine-tuning tasks, we propose a QoS function to evaluate the overall utility of all the vehicles in the network that accomplish both order serving and model fine-tuning tasks, i.e.,
\begin{align}
QoS = \alpha ADI + \beta ADU, \label{equ: qos def}
\end{align}
where $\alpha$ and $\beta$ are importance factors that balance the contributions of ADI and ADU, respectively. These weights are determined collaboratively by stakeholders, such as government entities or organizations overseeing vehicle-sensing coordination tasks and ride-hailing companies responsible for passenger service. By setting these hyperparameters in advance, the system adapts to operational priorities and ensures alignment with strategic objectives.

In our settings, the operations that each available vehicle can perform are denoted as \textbf{x}, \textbf{y}, and \textbf{z}, corresponding to dispatching, order acceptance, and data collection, respectively. The dispatching decision \textbf{x} exerts a certain impact on both \textbf{y} and \textbf{z}, and has a long-term impact on QoS. Specifically, in real-world applications, the distribution of orders in the city is non-uniform, and a sequence of dispatch decisions for a vehicle will allocate the vehicle to grids with varying order quantities. The vehicle assigned to grids with a lower supply-demand ratio has more opportunity to match an order successfully. Although the distribution of PoIs is often related to the type of VCS task, in some cases the distribution of PoIs is not uniform and the data utility of different PoIs in different grids is also different. We express the mathematical form of our optimization problem as follows.
\begin{align}
\underset{\mathbf{\textbf{x},\textbf{y},\textbf{z}}}{\text{Max}}
 &\hspace{3mm}
 {QoS(\textbf{x}, \textbf{y}, \textbf{z})}& \label{max obj}\\
 \text{S.t.} 
 &\quad x_t^{m, g} \in \left\{0,1\right\}, \forall m \in \mathcal{M}, \ t \in \mathcal{T}, \ g \in \mathcal{N}(g_t^m) \label{cst: x binary}\\
 & \quad y_t^{m} \in \left\{0,1\right\}, \forall m \in \mathcal{M}, \ t \in \mathcal{T} \label{cst: y binary}\\
 & \quad z_t^{m} \in \left\{0,1\right\}, \forall m \in \mathcal{M}, \ t \in \mathcal{T} \label{cst: z binary}\\
 & 0 \leq \sum\limits^{}_{g \in {\mathcal{N}(g_t^m)} } x_t^{m, g} \leq 1, \forall m \in \mathcal{M}, \ t \in \mathcal{T} \label{cst: one x} \\
 & \sum\limits^{}_{g \in {\mathcal{N}(g_t^m)} } x_t^{m, g} + y_t^m + z_t^m = 1, \forall m \in \mathcal{M}, \ t \in \mathcal{T} \label{cst: one choice}
\end{align}

Here, constraints (\ref{cst: x binary}), (\ref{cst: y binary}), and (\ref{cst: z binary}) denote that vehicle dispatching decision $x_t^{m,g}$, order-accepting decision $y_t^{m}$, and data-collecting decision $z_t^m$ are binary. In constraint (\ref{cst: x binary}), $g_t^m$ represents the grid where vehicle $m$ is currently located, and $\mathcal{N}(g)$ represents the set of grids adjacent to grid $g$ (including grid $g$ itself). For each vehicle, $x_t^{m,g}$ is only positive (= 1) when vehicle $m$ is dispatched to grid $g$ at time slot $t$, otherwise it is 0 (constraint (\ref{cst: x binary})). When $x_t^{m,g_t^m} = 1$, it means that vehicle $m$ decides to stay in its current grid $g_t^m$. $y_t^{m}$ is 1 only when $m$ decide to accept an order at time slot $t$, otherwise 0 (constraint (\ref{cst: y binary})). Similarly, $z_t^{m}$ is 1 only when $m$ decides to collect data from a PoI at time slot $t$, otherwise 0 (constraint (\ref{cst: z binary})).
Inequality (\ref{cst: one x}) denotes that vehicle $m$ has only one dispatch destination at a time slot. Equation (\ref{cst: one choice}) denotes that at each time slot, vehicles choose one of three choices: traveling to the dispatching destination $g \in \mathcal{N}(g_t^m)$, accepting an order, and collecting data. Note that the vehicles we mention in this section refer to the ones that are available at the current time slot.

Our mathematical models (\ref{max obj})-(\ref{cst: one choice}) describe the joint ride-hailing vehicles dispatching and crowdsensing scenarios. This is an online and sequential decision-making problem where decisions need to be made in real-time based on available information. Reinforcement learning emerges as a superior approach for addressing such problems. Our method is introduced in the next section.





 

%\section{PROPOSED SOLUTION: GNN-BASED MARL}
\section{Proposed Solution: GNN-enhanced MARL}
%\xiaoxi{Boken, please carefully read this entire section, and then modify anything that can be enhanced and used in your current implementations.}

In this section, we provide a comprehensive description of the algorithms designed to address the challenges posed by the joint scenarios of ride-hailing vehicle dispatching and crowdsensing. Our algorithm leverages MARL and integrates with other technologies to optimize the QoS as defined in (\ref{max obj}). 

% First, we assume that the joint optimization problem is a Markov decision process (MDP) and introduce the design of state, action, and reward functions in reinforcement learning. After that, we will introduce mean-field multi-agent reinforcement learning to solve the problem of training a large-scale number of agents.
\subsection{Algorithm Overview}
Due to the complexity of joint optimizing decision variables for fleets in urban environments, we employ a decentralized optimization framework. For each time slot, decision variables $\textbf{x}$, $\textbf{y}$ and $\textbf{z}$ can be independently determined by each available vehicle. Optimizing the QoS becomes feasible if the framework can learn the statistical distribution of orders and PoIs while gaining insights into their time sensitivity. The utilization of a distributed optimization framework effectively mitigates the challenge of the decision space rapidly expanding with the size of the vehicle set. The demand for online optimization and distributed decision-making motivates us to use MARL, renowned for its excellent performance in long-term and coupled decision-making \cite{why_use_MARL}. We illustrate the structure of our framework in Fig. \ref{fig: module intro}. Our system is based on the actor-critic MARL algorithm, with each agent making independent decisions. Additionally, we incorporate GNN for raw state embedding. 
% \broken{
Furthermore, the framework integrates the RankTuner module, enabling dynamic adjustment of LoRA ranks to balance fine-tuning accuracy and efficiency. 
% }
In the subsequent sections, we provide a concise overview of key settings such as states, actions, and rewards, followed by an explanation of GNN-based state embedding. 
\begin{figure*}[t]
\centerline{\includegraphics[width=1.0\linewidth]{figure/fig/rank_framework.png}}
\caption{GNN-based MARL framework. It includes the environment, a GNN embedding module for processing raw state information, an actor-critic module for decision-making, a replay buffer for experience storage, and the RankTuner module for dynamically adjusting LoRA ranks to balance fine-tuning accuracy and efficiency. These components work together to enable agents to make independent, informed decisions while optimizing their actions based on the dynamic environment.
% \xiaoxi{Please summarize the names and usages of the key modules and their in-between relationships in a few sentences in this caption.}
}
\label{fig: module intro}
\end{figure*}
\subsection{MARL Statement}
Formally, we model the joint optimization problem as a Markov game, which is represented by a
tuple $(\mathcal{S}, \mathcal{A}, \mathcal{P}, \mathcal{R})$, where $\mathcal{S}$, $\mathcal{A}$, $\mathcal{R}$ and $\mathcal{P}$ are the set of states, actions, rewards, and state transition probability. We define the important components in the MARL framework as follows.
% Since fleet sizes tend to be large, treating a cloud platform as an agent to manage all the vehicles will create a large state space and action space, leading to poor training results and affecting
% the algorithm’s scalability.
% To avoid this problem, we consider each vehicle as an agent and design the algorithm accordingly.
% Each agent is trained centrally and takes actions distributedly.

% Since both ride-haling and for-hire vehicles are managed by the platform or company, and the decision of each vehicle have an impact on the decisions of other vehicles. It is natural to think of a cloud platform as an agent that makes decisions about all vehicles, as most work does. However, since fleet sizes tend to be large, treating a cloud platform as an agent to manage all the vehicles will create a large state space and action space, leading to poor training results and affecting algorithm's scalability. To avoid this problem, we consider each vehicle as an agent and design the algorithm accordingly. Each agent is trained centrally and takes actions distributedly.

% We consider the state transfer of the vehicle as a Markov decision process, represented by a tuple $(\mathcal{S}, \mathcal{O}, \mathcal{A},  \mathcal{P}, \mathcal{R})$, where $\mathcal{S}$, $\mathcal{O}$, $\mathcal{A}$, $\mathcal{R}$ are the set of states, observations, actions and reward, and $\mathcal{P}$ is state transition probability. 

\noindent\textbf{Agent.} We consider a vehicle as an agent. As our objective is to optimize the overall income and data utility of all vehicles, each vehicle can be considered as a homogeneous agent performing cooperative tasks. We still use $\mathcal{M}$ to represent the set of agents.

\noindent\textbf{State space.} Intuitively, a global environmental state should encompass factors such as the distribution of drivers, the distribution of orders, the distribution of PoIs, the current time slot $t$, the generation time and estimated travel time for each order, and the data volume and AoI for each PoI. At the beginning of each time slot $t$, each vehicle $m$ gets a local state correlated with the global environment state $s_t \in \mathcal{S}$, which can be written as ${s_{m,t}}$. The challenge arises in determining whether to aggregate all the relevant factors into a large state space for every agent or to partition the state space into subspaces for each agent. Both approaches prove to be inefficient. Moreover, the continuous changes in the number of orders and PoIs lead to a dynamic change in the state space dimension over time. Selecting a fixed dimension for the state space becomes impractical, posing challenges for implementing the MARL algorithm. To tackle these challenges, we leverage Relational Graph Convolutional Networks (R-GCN) \cite{R-GCN} to encode the features of each agent. The output of R-GCN, denoted as $s^{\prime}_{m,t}$, serves as the embedding state of each agent, integrating the raw state and reducing the raw state dimension to a fixed dimension. The concrete representation of the state space and the detailed process of state embedding will be presented in the subsequent subsection.

% \textbf{Observation space.} 
% At the begin of each time slot $t$, each vehicle $m$ gets a local observation correlated with the environment state $s_t \in \mathcal{S}$, which can be written as ${o_t^m}$. Intuitively, the observation ${o_t^m}$ of vehicle $m$ should include the index of grid $g$ where vehicle $m$ is located, the current time slot $t$, the number of orders to be matched in grid $g$, the generation time and the estimated travel time of each order, the number of uncollected PoIs, the data volume and AoI of each PoI, the number of idle vehicles, and the data collection rate of vehicle $m$. 


% \textbf{Observation space.} At the begin of each time slot, each vehicle $m$ gets a local observation correlated with the environment state $s_t \in \mathcal{S}$, which can be written as $o_t^m = \left\{g_t^m, f_{t}^m, \big |\mathcal{W}_t^{g_t^m} \big |, \big | \mathcal{C}_t^{g_t^m} \big | , t, n\right\}$. In observation $o_t^m$, $g_t^m$ denotes the index of grid that vehicle $n$ is in at time slot $t$; $f_{t}^m$ is a flag bit used to indicate whether vehicle $m$ is serving the order or not, if vehicle $m$ is serving an order then $f_{t}^m$ is 1, otherwise 0; $\big |\mathcal{W}_t^{g_t^m} \big |$ and $\big | \mathcal{C}_t^{g_t^m} \big |$ denotes the number of orders waiting to be served and vehicles which are idle respectively in grid $g_t^m$; $t$ is the time slot and $m$ is the index of the vehicle.

\noindent\textbf{Action space.}
At every time slot, each available agent $m$ takes an action $a_{m,t}$ according to its policy after getting the embedding state $s^{\prime}_{m,t}$. The action $a_{m,t}$ indicates whether the agent should be dispatched to a neighboring grid, remain in the current grid and whether it should accept an order or collect data from a PoI. We denote all agents' joint action as $a_t = \left\{a_{1,t}, a_{2,t}, ..., a_{M,t}\right\}$.


\noindent\textbf{State transition probability.}
Based on the environmental state $s_t$ and the joint action $a_t$ of all agents, the environmental state will transit to the next state $s_{t+1}$ with probability $p(s_{t+1}|s_t, a_t)$.

\noindent\textbf{Reward.}
We calculate an immediate reward for each available agent according to its action. Given the three distinct action types, we formulate distinct reward functions corresponding to each action type.

\begin{itemize}
\item[$\bullet$] If dispatching agent $m$ to a neighboring grid at time slot $t$, we calculate the immediate reward of $m$ as: 
\begin{align}
    r_{m,t} = 0.
\end{align}
In this paper, remaining in the current grid is conceptualized as a special form of dispatching. We assign a reward of 0 for dispatching since dispatching doesn't directly yield rewards, although it can influence subsequent actions. Simultaneously, this is implemented to discourage agents from repeatedly dispatching between specific grids to gain rewards. Despite the reward value being 0 for dispatching, our expectation is that, through MARL, agents can still learn the influence of dispatching on order-serving and data collection, enabling a more effective dispatching policy.
% \begin{align}
% & {\delta}_t^g = \big |\mathcal{W}_t^{g} \big | - \big | \mathcal{C}_t^{g} \big | + b_t^{g},\\
% & r_t^n = max({\delta}_t^{des} - {\delta}_t^{g_t^m}, 0),
% \end{align}
% where $des$ represents the index of the destination grid of vehicle $n$. From (14) and (15), it can be seen that the agent will receive a positive reward when the sum of the supply-demand gap and the amount of data at the destination is larger than those at the origin. 
\end{itemize}

\begin{itemize}
\item[$\bullet$] If agent $m$ decides to accept an order at time slot $t$, the reward of $m$ is written as: 
\begin{align}
& r_{m,t} = \alpha \cdot \sigma(o_t^m),
\end{align}
where $o_t^m$ represents the order that be accepted by $m$, and $\sigma(o_t^m)$ represents the price of $o_t^m$. Here, $\alpha$ is a weight, which is the same as in expression (\ref{equ: qos def}).
\end{itemize}

\begin{itemize}
\item[$\bullet$] 
% \broken{
If agent $m$ decides to collect data from a PoI at time slot $t$, the reward of $m$ is written as: 
\begin{align}
& r_{m,t} = \beta u_t^{p_t^m} = \beta f_k(d_t^{p_{t,k}^m},{\lambda}_t^{{p_{t,k}^m}}).
\end{align}
Here, $r_{m,t}$ equals the utility of the data collected by $m$, and $p_t^m$ is the PoI collected by agent $m$ at time slot $t$ with task $k$. $\beta$ is also a weight like $\alpha$. The function \( f_k(d, \lambda) \) encapsulates the fine-tuning accuracy under the combined influence of two key factors: \( d \), the amount of collected data, and \( \lambda \), the degree of data freshness (also referred to as AoI). The subscript \( k \) indicates the task-specific nature of the function, as different tasks may exhibit unique dependencies on data quantity and freshness. This function quantifies how the accuracy of the fine-tuned model varies based on these two variables. As both data quantity and freshness directly impact the utility function, the interplay between these variables determines the model’s fine-tuning performance, with \( f(d, \lambda) \) capturing this dependency in a nuanced and task-specific manner.
% }
\end{itemize}

%(订单、数据。aoi对reward的随时间的contribution图)
%(添加新的方法,新的技术)

%In the reinforcement learning setup above, we consider each vehicle as an agent. Using algorithms based on the Centralized training with decentralized execution (CTDE) architecture, such as MAPPO, each agent can make independent decisions. However, in practice, the size of the fleet in an area is often more than dozens, or even hundreds. In a system with hundreds of agents, it is often difficult for ordinary multi-agent reinforcement learning to achieve satisfactory results because the actions of each agent may interfere with other agents. Learning for efficient coordination in large-scale multi-agent systems suffers from the problem of the curse of dimensionality due to the exponential growth of agent interactions. Mean-Field (MF)-based methods address this issue by transforming the interactions within the whole system into a single agent played with the average effect of its neighbors \cite{AAAI23}.
\subsection{State Embedding}
Reviewing the structure and influencing factors of our optimization problem, the arrival of ride orders and the generation of PoIs follow a prior but unknown distribution. To ensure adaptability to the dynamic and highly stochastic environment, each agent requires a policy network with high generalization, utilizing ample state information to formulate decisions. These decisions aim to optimize the overall QoS throughout numerous time slots from a long-term perspective. Therefore, continuous monitoring of the states of vehicles, orders, and PoIs is crucial for each agent. First, we define the features of raw state for the agent $m$, namely $s_{m,t} = \left\{ \mathcal{I}_{m,t}, \mathcal{U}_{m,t}, v_{m,t}, h_{m,t} \right\}$, serving as the input of RGCN and is formalized as follows.

\begin{itemize}
\item[$\bullet$] 
\textbf{Order feature set $\mathcal{I}_{m,t}$} contains some entities $i_{m,t} \in \mathcal{I}_{m,t}$, each of which represents the states of a order. Specifically, $i_{m,t}$ is the concatenation of the price, generation time, estimated travel time, origin, and destination of an order within the grid where $m$ is located. The feature set $\mathcal{I}_{m,t}$ not only provides information on the number and price of orders within the grid but also indicates the impact of orders on vehicle distribution due to their different destinations and travel times.
\end{itemize}

\begin{itemize}
\item[$\bullet$] 
\textbf{PoI feature set ${\mathcal{U}}_{m,t}$} is a set contains entities $u_{m,t} \in \mathcal{U}_{m,t}$. Each element $u_{m,t}$ represents the data volume, AoI, and generation location of a PoI within the grid where $m$ is located. By capturing the data volume and AoI of each PoI, the set ${\mathcal{P}}_{m,t}$ effectively characterizes the present data utility associated with individual PoIs.
\end{itemize}

\begin{itemize}
\item[$\bullet$] 
\textbf{Vehicle feature vector $v_{m,t}$} represents the location, working states (serving orders, collecting data or idle), current time slot, and index of agent $m$ (using one-hot encoding). 
\end{itemize}

\begin{itemize}
\item[$\bullet$] 
\textbf{Grid feature vector $h_{m,t}$} contains the number of orders, available vehicles, and PoIs in the current grid where the agent $m$ is located. To distinguish different grids, $h_{m,t}$ includes the grid index encoded by one-hot. Grid feature reflects the distribution of orders, vehicles, and PoIs in the environment.
\end{itemize}

In our MARL framework, if each agent $m$ can only observe the local features $\left\{ \mathcal{I}_{m,t}, \mathcal{U}_{m,t}, v_{m,t}, h_{m,t}\right\}$, there is an increased susceptibility to becoming ensnared in local optima \cite{liyihong}. A global representation of features is thus needed. However, a straightforward concatenation of all $s_{m,t}$ across all agents into a global state would result in a rapid expansion of the agent's input dimension, potentially compromising algorithm performance. In addition, since the size of the order set and PoI set are not fixed, we cannot specify a fixed input dimension for the agent. We try to use GNN to solve these problems. In our approach, we embed graph neural networks into the agent's policy network, enhancing the fusion and interaction between the agent's state and the environment's state. Below we define the topology graph used by GNN.

\begin{figure}[t]
\centering  %图片全局居中
\subfigure[Vehicles, orders, and PoIs in grids $g_1$, $g_2$, $g_3$ and $g_4$.]{
\label{fig: sub.1 distribution}
\includegraphics[width=0.3\textwidth]{figure/fig/grid.png}}
\subfigure[Topology graph based on (a).]{
\label{fig: sub.2 topo}
\includegraphics[width=0.32\textwidth]{figure/topology_example2}}
\caption{An example of constructing a topology graph.}
\label{fig: example of topo}
\end{figure}

\noindent\textbf{Definition 4. (Topology Graph.)} Due to the advantages of GNN in topology-based node information transferring and information processing, we represent vehicles, orders, PoIs, and grids as nodes within a topology graph. The topology graph is used to describe the relationship between urban states and vehicles. Our topology graph is $Gr(\mathcal{N}_t,\mathcal{E}_t)$, where the node set $\mathcal{N}_t$ consists of order nodes $\mathcal{O}_t$, PoI nodes $\mathcal{P}_t$, grid nodes $\mathcal{G}$, vehicle nodes $\mathcal{M}$, and a shortcut node. We abuse the notation of the order, PoI, grid, and vehicle indices to denote the corresponding nodes as well. $\mathcal{E}_t$ is the set of edges formed by connecting nodes, and its definition is as follows. As shown in Fig. \ref{fig: example of topo}, a vehicle node $m \in \mathcal{M}$ connects with a grid node $g \in \mathcal{G}$ only if vehicle $m$ is within the range of gird $g$. If the generation location of order $o \in \mathcal{O}_t$ is grid $g$, node $o$ is connected to node $g$. Similarly, PoI nodes are interconnected with grid nodes using a similar criterion. Grid node $g$ is only connected to the grid nodes which are its neighboring grid. Specifically, a shortcut node is introduced, linked to each grid node, expediting the information propagation within the graph neural network. Through the connection of grid nodes, we can capture the connectivity and topological structure between grids. Meanwhile, GNN facilitates each vehicle node's awareness of the features associated with orders and PoIs.

The topology graph contains multiple types of nodes, so we need to distinguish between different types of nodes. Since the R-GCN model can process topology graphs with different types of nodes and generate embedding information for the nodes, we use R-GCN to generate the state representation for agents. 
The information propagation is that each node $n \in \mathcal{N}_t$ passes the features as messages to its neighboring node and $n$ aggregates the features which are from its neighbors. The aggregation method and update steps of our R-GCN follow the steps in \cite{R-GCN}. The propagation model of our R-GCN is described as:
\begin{align}
    & {Gr}_{(0)}^{shortcut} = \textbf{0} \\
    & {Gr}_{(0)}^{\mathcal{M}} = \cup_{m = 1}^M v_{m,t}  \\
    & {Gr}_{(0)}^{\mathcal{G}} = \cup_{m = 1}^M h_{m,t}  \\
    & {Gr}_{(0)}^{\mathcal{O}} = \cup_{m = 1}^M \mathcal{I}_{m,t} \\
    & {Gr}_{(0)}^{\mathcal{P}} = \cup_{m = 1}^M \mathcal{U}_{m,t} 
\end{align}

The initial input ${Gr}_{(0)}$ is written as: 
\begin{align}
    {Gr}_{(0)} = \left\{ {Gr}_{(0)}^{shortcut}, {Gr}_{(0)}^{\mathcal{M}}, {Gr}_{(0)}^{\mathcal{G}}, {Gr}_{(0)}^{\mathcal{O}},{Gr}_{(0)}^{\mathcal{P}} \right \}.
\end{align} 

The node embedding is propagated in each layer $l$, i.e., ${Gr_{(l)}} = f({Gr_{(l-1)}})$, where $f(\cdot)$ represents the graph convolution network aggregating the features of each node with its neighbors. After $L$ layers of graph message passing, we get the final graph embedding $Gr_{(L-1)}$. We then map $Gr^{\mathcal{M}}_{(L- 1)}$ to the corresponding agents as the input of their actor networks.
\begin{table}[ht]
\centering
\caption{Impact of Rank on Fine-Tuning Accuracy and Time}
\resizebox{\linewidth}{!}{
\begin{tabular}{|c|c|c|}
\hline
\textbf{Rank} & \textbf{Fine-Tune Time (Normalized)} & \textbf{Accuracy Factor} \\
\hline
1 & 1.00  & 0.70     \\
2 & 1.45  & 0.76     \\
3 & 3.71  & 0.98     \\
4 & 5.05  & 0.99     \\
5 & 6.00  & 0.99365  \\
6 & 5.31  & 1.00     \\
\hline
\end{tabular}
}
\label{tab:rank}
\end{table}
\subsection{Heuristic-based Rank Selection Integrated with MARL}
% \broken{
Since LoRA is employed for task fine-tuning, selecting an appropriate rank is crucial. According to relevant studies~\cite{bai2024federated}, a larger rank generally leads to better fine-tuning accuracy but at the cost of increased fine-tuning time. An interesting and important trade-off here is that if choosing a larger rank greedily for obtaining a higher accuracy per task, the longer fine-tuning time will very likely reduce the expected total number of fine-tuning tasks that a vehicle can accomplish in the entire time span. Therefore, the best rank of a fine-tuning adaptor should be carefully chosen to achieve the total utility. 

Our experiments, as summarized in Table~\ref{tab:rank}, show that the choice of rank significantly impacts both accuracy (in terms of accuracy discounts compared to the best accuracy) and fine-tuning time across different tasks, such as image classification and image segmentation. To address this, we propose the RankTuner, a dynamic adjustment mechanism designed to determine the most suitable rank for fine-tuning when the optimal rank is unknown. The RankTuner operates as follows: Initially, a rank is randomly selected as the benchmark. During each iteration, the algorithm compares the current ADU to the previous round. If the ADU improves, the algorithm maintains the current direction (increasing or decreasing the rank) to further explore better settings. If the ADU decreases, the algorithm reverts to the previous rank and switches the direction. The rank is constrained within a predefined allowable range to ensure feasibility. The pseudocode for the RankTuner is presented in Algorithm~\ref{alg:rank_tuner}.
% }

\begin{algorithm}
\caption{A Rank Selection Algorithm (RankTuner)}
\label{alg:rank_tuner}
\begin{algorithmic}[1]
\State \textbf{Input:} Allowed rank range $[\eta_{\text{min}}, \eta_{\text{max}}]$, initial rank $\eta_0$, initial direction $d$ (\texttt{+1} or \texttt{-1}).
\State Initialize $\eta \leftarrow \eta_0$, $d \leftarrow \texttt{+1}$, previous ADU $\text{ADU}_{\text{prev}} \leftarrow 0$.
\For{each fine-tuning iteration}
    \State Fine-tune model with rank $rank$ and compute current ADU $\text{ADU}_{\text{curr}}$.
    \If{$\text{ADU}_{\text{curr}} > \text{ADU}_{\text{prev}}$}
        \State $\eta \leftarrow \eta + d$ {\Comment{Keep direction and adjust rank.}}
    \Else
        \State $d \leftarrow -d$ {\Comment{Reverse direction.}}
        \State $\eta \leftarrow \eta + d$ {\Comment{Revert to previous rank.}}
    \EndIf
    \State $\eta \leftarrow \max(\eta_{\text{min}}, \min(\eta, \eta_{\text{max}}))$ 
    \State Update $\text{ADU}_{\text{prev}} \leftarrow \text{ADU}_{\text{curr}}$.
\EndFor
\end{algorithmic}
\end{algorithm}

The RankTuner 
% \xiaoxi{We don't commonly use separated words to name an algorithm. Please see the revised wording of the algorithm title. I changed the name to RankTuner. I also revised the title of Section IV-D.}
effectively balances fine-tuning accuracy and time efficiency by dynamically adapting the rank based on real-time feedback. This mechanism ensures that the fine-tuning process remains efficient and yields high-quality results across various tasks and environmental conditions.



\subsection{Training}
% \begin{algorithm}[t]
% \caption{Pseudocode: GNN-MAPPO} 
% \label{alg1} \begin{algorithmic}
% \STATE Initialize policy network $\pi_\theta$ and value network $V_\phi$
% \STATE Initialize a memory buffer $D$
% \FOR{episode $i$ = $1$, $2$, $\dots$, $max\underline{~}episode$}
% \STATE Initialize environment
% \FOR{time-slot $t$ = $1$, $2$, $\dots$, $T$}
% \STATE Construct topology graph $Gra(\mathcal{N}_t, \mathcal{E}_t)$
% \FOR{for agent $m$ = $1$, $2$, $\dots$, $M$}
% % \STATE Agent $m$ get its raw state  $s_{m,t}$
% \STATE Input raw state $s_{m,t}$ into RGCN and output the embedding state $s^{'}_{m,t}$
% \STATE Agent $m$ executes action according to $\pi_\theta(a_{m,t} |s^{'}_{m,t})$
% \STATE Get the reward $r_{m,t}$
% \ENDFOR
% \STATE Get the next state $s_{t+1}$
% \ENDFOR
% \ENDFOR
 
		
% 	\end{algorithmic} 
% \end{algorithm}
Our MARL model is based on multi-agent proximal policy optimization (MAPPO) \cite{neurips22}. 
% PPO is a stable on-policy learning algorithm that can handle the large combinatorial action space. We treat agents as homogeneous agents (agents have
% identical state and action spaces), so all agents share actor and critical network parameters during the training process.
Each agent has a flag bit to indicate whether the agent is available or not. We ignore the output of the action by non-available agents. Due to the fewer dispatching destinations in the boundary grids than in the non-boundary grids, we mask the corresponding dispatching action for agents in boundary grids. For every policy update step, we collect a batch of trajectories from the environment and compute the loss function according to \cite{PPO}. Our MARL model is trained online because online training has a higher utilization rate of sampled data. 
% Agents produce an action $a^i$ \tianxiang{what is i} from the state $s^i$ to jointly optimize the discounted accumulated reward $J(\theta) = \mathbb{E}_{A_t, S_t}({\sum \limits^{T}_{t}} \gamma^t R(S_t, A_t))$ by using a policy ${\pi}_\theta (a^i
% |s^i)$ parameterized by $\theta$, where $R(S_t, A_t)$ denotes the shared reward function. 
We apply some implementation techniques, including Generalized Advantage Estimation (GAE) with advantage normalization and value-clipping. 


% \subsection{Mean Field MARL}
% In the vehicles dispatching and VSC task, agents interact with other agents by choosing when to dispatch, take orders, or collect data. Obviously, the actions made by each agent affect the state of the environment. In terms of order taking, when many vehicles are dispatched to the same target area, the ratio of idle vehicles to orders (or the ratio of supply and demand) in that area will become larger, resulting in many arriving vehicles not being able to successfully receive orders. For VCS tasks, the amount of data that needs to be collected in PoI decreases with the collection of vehicles. When a group of vehicles choose to collect data, there may be no more available data to collect. In traditional multi-agent reinforcement learning, in order to simulate the interaction and coordinate the action, the joint action of all agents and the environmental state are required to calculate the state action value, i.e., $Q(s_t,a_t)$, where $a_t=\left\{a_t^1, a_t^2, ..., a_t^M\right\}$. Since each vehicle is regarded as an agent in our reinforcement learning scenario setting, when the number of agents expands, the expansion of the joint action space will follow, which brings difficulties to the coordination of agents. Learning for efficient coordination in large-scale multi-agent systems suffers from the problem of the curse of dimensionality due to the exponential growth of agent interactions. 
% \begin{figure}[t]
% \centerline{\includegraphics[width=1.0\linewidth]{figure/marl_algorithm.png}}
% \caption{Mean Field RL}
% \label{fig}
% \end{figure}

% Mean-Field (MF)-based methods address this issue by transforming the interactions within the whole system into a single agent played with the average effect of its neighbors \cite{AAAI23}.
% Since vehicles are entities with specific locations on the map, it is natural to define the neighbor vehicles of vehicle $m$ as a set of vehicles which is in the current grid and the neighboring grid, and we denote this set as ${nei}_t^m$. The actions of each agent are encoded with one-hot codes in the discrete action space. The state of the simulation environment is $s^t = \left\{o^t_1, o^t_2, ..., o^t_m\right\}$, i.e., the splicing of the observations of each vehicle. We calculate the mean action $\hat{a}^t_m$ based on the neighborhood ${nei}_t^m$ of agent m as
% \begin{align}
%     \hat{a}^t_m = \frac{1}{\big |{{nei}_t^n} \big |} \sum\limits^{}_{k \in {{nei}_t^n}} a^t_k,
% \end{align}
% and the state-action value of each agent is written as 
% \begin{align}
%     Q_t^m(s_t,a_t) =  Q_t^m(s_t, a_t^m, \hat{a}_t^m).
% \end{align}
% The actor is trained by the sampled policy gradient:
% \begin{align}
% \nabla_{{\theta}_i}\mathcal{J}({\theta}_i) = \nabla_{{\theta}_i}log {\pi}_{{\theta}_i}(s)Q_{{\phi}^i}(s, a_i,\hat{a}_i) \big|_{a=\pi_{{\theta}_i}(s)},
% \end{align}
% and update the parameter of the critic network by minimizing the loss $\mathscr{L}$ as follows:
% \begin{align}
% \mathcal{L}({\phi}_i) = \sum{(y_i - Q_{{\phi}^i}(s, a_i, \hat{a}_j))}^2 \\
% y_i = r_i + \gamma {V_{{\phi}_i}}^{MF}(s),
% \end{align}
% where $\theta_i$, $\phi_i$, ${V_{{\phi}_i}}^{MF}(s)$ is the parameter of actor network, the parameter of critic network and mean field value function.
%{Mean-Field (MF)-based methods address this issue by transforming the interactions within the whole system into a single agent played with the average effect of its neighbors \cite{AAAI23}. 由于车辆是在地图上有具体位置的实体,我们很自然地将每辆车的邻居车辆定义为在目前区域以及邻居区域的车辆的集合,将这个集合记作${nei}_t^n$. 在离散的动作空间中用one-hot码编码每个智能体的动作,我们将时隙t时刻所有车辆的联合动作表示为$a^t = {a^t_1, a^t_2, ..., a^t_N}$。环境的状态为$s^t = {o^t_1, o^t_2, ..., a^t_N}$,即每辆车的observation的拼接。We calculate the mean action $a^t_i$ based on the neighborhood ${nei}_t^n$ of agent i as}
%{
%\begin{align}
%& a^t_i = \frac{1}{N^i} \sum,
%\end{align}
%}

%(图进行修改,状态包括什么,数据utility,三种动作用文字描述,文字用框框住,图中体现提到的指标)
%(还可以画出设置数据分布示意图)





\section{Evaluation}

% \saidur{Working on it}




\begin{table*}[!t]
% \small
\centering
\caption{Summary of Results for EMBER Domain-IL Experiments.}
\vspace{-0.2cm}
\label{tab:ember_DIL}

\begin{tabular}{p{1.1cm}|l|c|c|c|c|c|c|c} 

% \toprule 

\multirow{2}{*}{\textbf{Group}} & \multirow{2}{*}{\textbf{Method}} & \multicolumn{7}{c}{\textbf{Budget}} \\ \cline{3-9}

&  & 1K & 10K & 50K & 100K & 200K & 300K & 400K \\ \midrule

\multirow{3}{*}{Baselines} 
& Joint  & \multicolumn{7}{c}{96.4$\pm$0.3} \\ 
& None   & \multicolumn{7}{c}{93.1$\pm$0.1} \\ 
& GRS    & 93.6$\pm$0.3 & 94.1$\pm$1.3 & 95.3$\pm$0.2 & 95.3$\pm$0.7 & 95.9$\pm$0.1 & 95.8$\pm$0.6 & 96.0$\pm$0.3 \\ 
\midrule

\multirow{4}{*}{\parbox{0.7cm}{Prior \\ Work}} 
& ER~\cite{er}     & 80.6$\pm$0.1 & 73.5$\pm$0.5 & 70.5$\pm$0.3 & 69.9$\pm$0.1 & 70.0$\pm$0.1 & 70.0$\pm$0.1 & 70.0$\pm$0.1 \\ 
& AGEM~\cite{agem}   & 80.5$\pm$0.1 & 73.6$\pm$0.2 & 70.4$\pm$0.3 & 70.0$\pm$0.1 & 70.0$\pm$0.2 & 70.0$\pm$0.1 & 70.0$\pm$0.1 \\ 
& GR~\cite{gr}     & \multicolumn{7}{c}{93.1$\pm$0.2} \\ 
& RtF~\cite{rtf}    & \multicolumn{7}{c}{93.2$\pm$0.2} \\ 
& BI-R~\cite{BIR}   & \multicolumn{7}{c}{93.4$\pm$0.1} \\ 
\midrule

\multirow{4}{*}{\system}      
& \system-R         & \textbf{93.7$\pm$0.1} & \textbf{94.7$\pm$0.1} & \textbf{95.4$\pm$0.1} & \textbf{95.3$\pm$0.6} & \textbf{96.0$\pm$0.1} & \textbf{96.1$\pm$0.1} & \textbf{96.1$\pm$0.1} \\ 
& \system-U         & \textbf{93.6$\pm$0.2} & 94.0$\pm$0.2 & 95.1$\pm$0.1 & \textbf{95.3$\pm$0.1} & 95.5$\pm$0.1 & 95.7$\pm$0.1 & 95.8$\pm$0.1 \\  \cline{2-9}
& MADAR$^{\theta}$-R & \textbf{93.6$\pm$0.1} & \textbf{94.4$\pm$0.3} & \textbf{95.3$\pm$0.2} & \textbf{95.8$\pm$0.1} & \textbf{96.1$\pm$0.1} & \textbf{96.1$\pm$0.1} & \textbf{96.1$\pm$0.1} \\ 
& MADAR$^{\theta}$-U & 93.5$\pm$0.2 & 94.1$\pm$0.2 & 94.9$\pm$0.1 & 95.2$\pm$0.2 & 95.6$\pm$0.1 & 95.7$\pm$0.1 & 95.7$\pm$0.1 \\ 

\bottomrule

\end{tabular}
\vspace{-0.2cm}
\end{table*}









\begin{figure}[!t]
    \centering
    \begin{subfigure}{0.485\linewidth}
        \centering
        \includegraphics[width=1.0\linewidth]{figures_TIFS/EMBER_IFS_DIL_RATIO.pdf}
        \label{fig:EMBER_DIL_IFS_R}
        \vspace{-0.4cm}
        \caption{MADAR Ratio}
    \end{subfigure}
    \hfill
    \begin{subfigure}{0.485\linewidth}
        \centering
        \includegraphics[width=1.0\linewidth]{figures_TIFS/EMBER_IFS_DIL_UNIFORM.pdf}
        \label{fig:EMBER_DIL_IFS_U}
        \vspace{-0.4cm}
        \caption{MADAR Uniform}
    \end{subfigure}
    \vfill
    \begin{subfigure}{0.485\linewidth}
        \centering
        \includegraphics[width=1.0\linewidth]{figures_TIFS/EMBER_AWS_DIL_RATIO.pdf}
        \label{fig:EMBER_DIL_AWS_R}
        \vspace{-0.4cm}
        \caption{MADAR$^\theta$ Ratio}
    \end{subfigure}
    \hfill
    \begin{subfigure}{0.485\linewidth}
        \centering
        \includegraphics[width=1.0\linewidth]{figures_TIFS/EMBER_AWS_DIL_UNIFORM.pdf}
        \label{fig:EMBER_DIL_AWS_U}
        \vspace{-0.4cm}
        \caption{MADAR$^\theta$ Uniform}
    \end{subfigure}

    \caption{EMBER Domain-IL: Comparison of the MADAR-R, MADAR-U, MADAR$^\theta$-R, and MADAR$^\theta$-U with Joint baseline.}
    \label{fig:ember_DIL}
    \vspace{-0.3cm}
\end{figure}





\begin{table*}[!t]
\centering
\caption{Summary of Results for AZ Domain-IL Experiments.}
\vspace{-0.3cm}
\label{tab:az_DIL}
\begin{tabular}{p{1.1cm}|l|c|c|c|c|c|c|c} 

% \toprule 

\multirow{2}{*}{\textbf{Group}} & \multirow{2}{*}{\textbf{Method}} & \multicolumn{7}{c}{\textbf{Budget}} \\ \cline{3-9}

&  & 1K & 10K & 50K & 100K & 200K & 300K & 400K \\ \midrule

\multirow{3}{*}{Baselines} 
& Joint  & \multicolumn{7}{c}{97.3$\pm$0.1} \\ 
& None   & \multicolumn{7}{c}{94.4$\pm$0.1} \\ 
& GRS    & 95.3$\pm$0.1 & 96.4$\pm$0.1 & 96.9$\pm$0.1 & 97.1$\pm$0.1 & 97.1$\pm$0.1 & 97.2$\pm$0.1 & 97.2$\pm$0.1 \\ 
\midrule

\multirow{4}{*}{\parbox{0.7cm}{Prior \\ Work}} 
& ER~\cite{er}     & 40.4$\pm$0.1 & 40.1$\pm$0.1 & 41.1$\pm$0.2 & 42.6$\pm$0.1 & 44.0$\pm$0.1 & 45.9$\pm$0.1 & 48.6$\pm$1.1 \\ 
& AGEM~\cite{agem}   & 45.4$\pm$0.1 & 47.4$\pm$0.2 & 49.2$\pm$0.2 & 53.7$\pm$0.6 & 54.2$\pm$0.3 & 54.8$\pm$0.4 & 56.7$\pm$0.3 \\ 
& GR~\cite{gr}     & \multicolumn{7}{c}{93.3$\pm$0.4} \\ 
& RtF~\cite{rtf}     & \multicolumn{7}{c}{93.4$\pm$0.2} \\ 
& BI-R~\cite{BIR}     & \multicolumn{7}{c}{93.5$\pm$0.1} \\ 
\midrule

\multirow{4}{*}{\system}      
& \system-R         & \textbf{95.8$\pm$0.1} & \textbf{96.6$\pm$0.1} & \textbf{96.9$\pm$0.1} & \textbf{97.0$\pm$0.1} & \textbf{97.0$\pm$0.1} & \textbf{97.0$\pm$0.1} & \textbf{97.0$\pm$0.1} \\ 
& \system-U         & \textbf{95.7$\pm$0.1} & 95.5$\pm$0.1 & 95.2$\pm$0.2 & 95.2$\pm$0.1 & 95.4$\pm$0.1 & 95.8$\pm$0.2 & 96.3$\pm$0.2 \\ \cline{2-9}
& MADAR$^{\theta}$-R & \textbf{95.8$\pm$0.2} & \textbf{96.6$\pm$0.1} & \textbf{96.9$\pm$0.1} & \textbf{96.9$\pm$0.1} & \textbf{97.1$\pm$0.1} & \textbf{97.1$\pm$0.1} & \textbf{97.2$\pm$0.1} \\ 
& MADAR$^{\theta}$-U & 95.6$\pm$0.1 & 96.1$\pm$0.1 & 96.6$\pm$0.1 & 96.8$\pm$0.1 & \textbf{97.0$\pm$0.1} & \textbf{97.1$\pm$0.1} & \textbf{97.1$\pm$0.1} \\ 

\bottomrule

\end{tabular}
\vspace{-0.3cm}
\end{table*}



\begin{figure}[!t]
    \centering
    \begin{subfigure}{0.485\linewidth}
        \centering
        \includegraphics[width=1.0\linewidth]{figures_TIFS/AZ_IFS_DIL_RATIO.pdf}
        \label{fig:AZ_DIL_IFS_R}
        \vspace{-0.4cm}
        \caption{MADAR Ratio}
    \end{subfigure}
    \hfill
    \begin{subfigure}{0.485\linewidth}
        \centering
        \includegraphics[width=1.0\linewidth]{figures_TIFS/AZ_IFS_DIL_UNIFORM.pdf}
        \label{fig:AZ_DIL_IFS_U}
        \vspace{-0.4cm}
        \caption{MADAR Uniform}
    \end{subfigure}
    \hfill
    \begin{subfigure}{0.485\linewidth}
        \centering
        \includegraphics[width=1.0\linewidth]{figures_TIFS/AZ_AWS_DIL_RATIO.pdf}
        \label{fig:AZ_DIL_AWS_R}
        \vspace{-0.4cm}
        \caption{MADAR$^\theta$ Ratio}
    \end{subfigure}
    \hfill
    \begin{subfigure}{0.485\linewidth}
        \centering
        \includegraphics[width=1.0\linewidth]{figures_TIFS/AZ_AWS_DIL_UNIFORM.pdf}
        \label{fig:AZ_DIL_AWS_U}
        \vspace{-0.4cm}
        \caption{MADAR$^\theta$ Uniform}
    \end{subfigure}

    \caption{AZ Domain-IL: Comparison of the MADAR-R, MADAR-U, MADAR$^\theta$-R, and MADAR$^\theta$-U with Joint baseline.}
    \label{fig:az_DIL}
    \vspace{-0.3cm}
\end{figure}






% \subsection{Experimental Setup, Datasets, and Baselines}


We present the results of our \system\ framework and MADAR$^\theta$ in the Domain-IL, Class-IL, and Task-IL scenarios using the EMBER and AZ datasets discussed in Section~\ref{sec:dataset}. To denote our techniques, we use the following abbreviations: {\bf \system-R} for \system-Ratio, {\bf \system-U} for \system-Uniform, {\bf MADAR$^\theta$-R} for MADAR$^\theta$-Ratio, and {\bf MADAR$^\theta$-U} for MADAR$^\theta$-Uniform.

For all three scenarios, we compare \system\ against widely studied replay-based continual learning (CL) techniques, including experience replay (ER)\cite{er}, average gradient episodic memory (AGEM)\cite{agem}, deep generative replay (GR)\cite{gr}, Replay-through-Feedback (RtF)\cite{rtf}, and Brain-inspired Replay (BI-R)\cite{BIR}. Additionally, we evaluate \system\ against iCaRL\cite{icarl}, a replay-based method specifically designed for Class-IL. For the Class-IL and Task-IL scenarios, we additionally compare \system\ with Task-specific Attention Modules in Lifelong Learning (TAMiL)\cite{tamil}. Furthermore, we benchmark MADAR against MalCL\cite{malcl}, a method specifically designed for Class-IL. Notably, most recent work focuses primarily on Class-IL and Task-IL scenarios, limiting direct comparisons in the Domain-IL scenario. In our results tables, the best-performing methods and those within the error margin of the top results are highlighted. 

%Finally, we built upon the codebase provided by \cite{continual-learning-malware} for implementation and evaluation.


% In this study, we utilize large-scale malware datasets, including the EMBER dataset~\cite{ember}, a widely recognized benchmark for Windows malware classification, and two Android malware datasets derived from AndroZoo~\cite{AndroZoo}, which were specifically curated for this research. Our approach is evaluated against two primary baselines:

% \begin{smitemize}
%     \item \textbf{None}: A baseline where the model is trained sequentially on each new task without employing any continual learning (CL) techniques, serving as an informal lower bound.
%     \item \textbf{Joint}: A baseline where the model is trained on both new and previously seen data at each step, representing an informal upper bound. While resource-intensive, the \textbf{Joint} baseline consistently achieves robust performance.
% \end{smitemize}

% Additionally, we introduce a third baseline: \textbf{Global Reservoir Sampling (GRS)}. This method is based on reservoir sampling~\cite{vitter1985random} and builds upon prior work by \cite{continual-learning-malware}. GRS provides an unbiased representation of class distributions and serves as a strong benchmark for comparing our diversity-aware approach.




% We now present the results of our \system framework for both \system and MADAR$^\theta$ in the Domain-IL, Class-IL, and Task-IL scenarios for EMBER and AZ datasets. We use the following four abbreviations to denote our techniques---{\bf \system-R} for \system-Ratio, {\bf ~\system-U} for \system-Uniform, {\bf MADAR$^\theta$-R} for MADAR$^\theta$-Ratio, and {\bf ~MADAR$^\theta$-U} for MADAR$^\theta$-Uniform.  For all three scenarios, we compare \system\ with the most widely studied replay-based CL techniques: experience replay (ER)~\cite{er}, average gradient episodic memory (AGEM)~\cite{agem}, deep generative replay (GR)~\cite{gr}, Replay-through-Feedback (RtF)~\cite{rtf}, and Brain-inspired Replay (BI-R)~\cite{BIR}. In addition, we compare \system\ with iCaRL~\cite{icarl}, a replay-based technique specifically designed for Class-IL. Furthermore, we compare \system with Task-specific Attention Modules in Lifelong learning (TAMiL)~\cite{bhat2023task} which is designed for Class-IL and Task-IL scenarios. In addition, we also compare MADAR with MalCL~\cite{malcl} specifically designed for Class-IL. We observe that recent works mostly focus on Class-IL and Task-IL scenarios which limits what we can compare with in the Domain-IL scenario. The results of the best-performing method, as well as those within the error range of the best results, are highlighted in the results tables. We built upon the code of the prior work by \cite{continual-learning-malware}.

% In this study, we use large-scale Windows and Android malware datasets: EMBER~\cite{ember}, a Windows malware dataset from prior work, recognized as a standard benchmark for malware classification, and two new Android malware datasets derived from AndroZoo~\cite{AndroZoo}, specifically assembled for this research.

% We adopt two baselines for comparison: {\em None} and {\em Joint}.  {\em None} sequentially trains the model on each new task without any CL techniques, serving as an informal minimum baseline. By contrast, {\em Joint} uses all new and prior data for training at each step, acting as an informal maximum baseline. Despite its resource demands, {\em Joint} ensures strong performance throughout the dataset. We also introduce an additional baseline -- Global Reservoir Sampling (GRS) built upon {\em reservoir sampling}~\cite{vitter1985random} and \cite{continual-learning-malware}. GRS provides an unbiased sampling of the underlying class distributions and serves as a strong point of comparison for our diversity-aware approach.

% In this study, we utilize large-scale malware datasets, including the EMBER dataset~\cite{ember}, a widely used benchmark for Windows malware classification, and two Android malware datasets derived from AndroZoo~\cite{AndroZoo}, specifically assembled for this research. We compare our approach against two baselines: {\em None}, where the model is trained sequentially on each new task without any CL techniques, serving as an informal lower bound; and {\em Joint}, which trains on both new and previous data at each step, representing an informal upper bound. Although resource-intensive, {\em Joint} ensures consistently strong results. Additionally, we introduce another baseline -- Global Reservoir Sampling (GRS), an approach based on {\em reservoir sampling}~\cite{vitter1985random} and \cite{continual-learning-malware}, which provides an unbiased representation of class distributions and serves as a strong point of comparison for our diversity-aware approach.


\subsection{Domain-IL}
\label{domainilexps}

%% #of training samples --> 674994
%As shown in Table~\ref{tab:combined_DIL}, a



In EMBER, we have 12 tasks, each representing the monthly data distribution spanning January--December 2018. Our results, detailed in Table~\ref{tab:ember_DIL}, provide a comprehensive view of each method's performance, reported as the average accuracy over all tasks $\mathbf{\overline{AP}}$. Additionally, Figure~\ref{fig:ember_DIL} illustrates the progression of average accuracy over time compared to the \textit{Joint} baseline. 

The informal lower and upper performance bounds for this configuration are approximated by the \textit{None} and \textit{Joint} methods, achieving $\mathbf{\overline{AP}}$ scores of 93.1\% and 96.4\%, respectively. Meanwhile, \textit{GRS} serves as a strong baseline, providing unbiased sampling without incorporating sample diversity awareness.

% In EMBER, we have 12 tasks, each representing the monthly data distribution spanning January--December 2018. Our results, detailed in Table~\ref{tab:ember_DIL}, present a nuanced view of each method's performance, reported as the average accuracy over all tasks $\mathbf{\overline{AP}}$. In addition, Figure~\ref{fig:ember_DIL} represents the progression of average accuracy as the task progresses compared with {joint} baseline. The informal lower and upper performance bounds for this configuration can be approximated by the {\em None} and {\em Joint} methods, which get $\mathbf{\overline{AP}}$ of 93.1\% and 96.4\%, respectively. Meanwhile, {\em GRS} represents a strong baseline for unbiased sampling without awareness of sample diversity.

At a lower budget of 1K, \system-R, \system-U, and MADAR$^\theta$-R exhibit competitive performance, all achieving $\mathbf{\overline{AP}}$ of over $93.6$\%, significantly outperforming prior approaches. This highlights their ability to effectively utilize limited resources. In particular, \system-R achieves the highest accuracy at this budget, with $\mathbf{\overline{AP}}$ of $93.7\%$.

As the memory budget increases, the performance of all \system\ and MADAR$^\theta$ variants improves steadily. At a budget of 200K, \system-R and MADAR$^\theta$-R achieve an impressive $\mathbf{\overline{AP}}$ of $96.0\%$ and $96.1\%$, respectively, closely approaching the $96.4\%$ achieved by the \textit{Joint} baseline, which utilizes over 670K samples. Uniform strategies, including \system-U and MADAR$^\theta$-U, are only slightly behind, with $\mathbf{\overline{AP}}$ values of $95.5\%$ and $95.6\%$, respectively.

% At lower budget of 1K, GRS, \system-R, and \system-U exhibit competitive performance, all significantly better than prior work with $\mathbf{\overline{AP}}$ above $93.6$\%, indicating their effective utilization of limited resources. ER and AGEM performed far below even the \emph{None} baseline, while GR could only match it. For higher budgets, GRS and \system\ methods all show excellent performance. At a 200K budget, \system-R yields $\mathbf{\overline{AP}}$ of $96.0$\%, close to the $96.4$\% reached by the Joint baseline that used over 670K samples. GRS is competitive, while Uniform strategies are only slightly behind.




\begin{table*}[!t]
\centering
\caption{Summary of Results for EMBER Class-IL Experiments.}
\vspace{-0.3cm}
\label{tab:ember_CIL}
\begin{tabular}{p{1.1cm}|l|c|c|c|c|c|c|c} 

% \toprule 

\multirow{2}{*}{\textbf{Group}} & \multirow{2}{*}{\textbf{Method}} & \multicolumn{7}{c}{\textbf{Budget}} \\ \cline{3-9}

&  & 100 & 500 & 1K & 5K & 10K & 15K & 20K \\ \midrule

\multirow{3}{*}{Baselines} 
& Joint  & \multicolumn{7}{c}{86.5$\pm$0.4} \\ 
& None   & \multicolumn{7}{c}{26.5$\pm$0.2} \\ 
& GRS    & 51.9$\pm$0.4 & 70.3$\pm$0.5 & 75.4$\pm$0.7 & 82.0$\pm$0.2 & 83.5$\pm$0.1 & 84.3$\pm$0.3 & 84.6$\pm$0.2 \\ \midrule

\multirow{6}{*}{\parbox{0.7cm}{Prior \\ Work}} 
& TAMiL~\cite{tamil}  & 32.2$\pm$0.3 & 33.1$\pm$0.2 & 35.3$\pm$0.2 & 36.7$\pm$0.1 & 38.2$\pm$0.3 & 37.2$\pm$0.2 & 38.8$\pm$0.2 \\ 
& iCaRL~\cite{icarl}  & 53.9$\pm$0.7 & 58.7$\pm$0.7 & 60.0$\pm$1.0 & 63.9$\pm$1.2 & 64.6$\pm$0.8 & 65.5$\pm$1.0 & 66.8$\pm$1.1 \\ 
& ER~\cite{er}     & 27.5$\pm$0.1 & 27.8$\pm$0.1 & 28.0$\pm$0.1 & 27.9$\pm$0.1 & 28.0$\pm$0.1 & 28.0$\pm$0.1 & 28.2$\pm$0.1 \\ 
& AGEM~\cite{agem}   & 27.3$\pm$0.1 & 27.4$\pm$0.1 & 27.7$\pm$0.1 & 28.5$\pm$0.1 & 28.2$\pm$0.1 & 28.3$\pm$0.1 & 28.2$\pm$0.1 \\ 
& GR~\cite{gr}     & \multicolumn{7}{c}{26.8$\pm$0.2} \\ 
& RtF~\cite{rtf}   & \multicolumn{7}{c}{26.5$\pm$0.1} \\ 
& BI-R~\cite{BIR}   & \multicolumn{7}{c}{26.9$\pm$0.1} \\ 
& MalCL~\cite{malcl}   & \multicolumn{7}{c}{54.5$\pm$0.3} \\ 
\midrule

\multirow{4}{*}{\system} 
& \system-R & \textbf{68.0$\pm$0.4} & 73.6$\pm$0.2 & 76.0$\pm$0.3 & 81.5$\pm$0.2 & 83.2$\pm$0.2 & 83.8$\pm$0.2 & 84.0$\pm$0.2 \\ 
& \system-U & 66.4$\pm$0.4 & \textbf{76.5$\pm$0.2} & \textbf{79.4$\pm$0.4} & \textbf{83.8$\pm$0.2} & \textbf{84.8$\pm$0.1} & \textbf{85.5$\pm$0.1} & \textbf{85.8$\pm$0.3} \\ \cline{2-9}
& MADAR$^{\theta}$-R & {\bf 67.9$\pm$0.3} & 72.7$\pm$0.5 & 72.7$\pm$0.5 & 81.7$\pm$0.2 & 83.2$\pm$0.1 & 83.9$\pm$0.1 & 84.5$\pm$0.2 \\ 
& MADAR$^{\theta}$-U & 67.5$\pm$0.3 & {\bf 76.4$\pm$0.4} & {\bf 78.5$\pm$0.4} & {\bf 84.1$\pm$0.1} & {\bf 85.3$\pm$0.1} & {\bf 85.8$\pm$0.2} & {\bf 86.2$\pm$0.2} \\ 

\bottomrule

\end{tabular}
\vspace{-0.2cm}
\end{table*}



\begin{figure}[!t]
    \centering
    \begin{subfigure}{0.485\linewidth}
        \centering
        \includegraphics[width=1.0\linewidth]{figures_TIFS/EMBER_CIL_IFS_RATIO.pdf}
        \label{fig:EMBER_CIL_IFS_R}
        \vspace{-0.4cm}
        \caption{MADAR Ratio}
    \end{subfigure}
    \hfill
    \begin{subfigure}{0.485\linewidth}
        \centering
        \includegraphics[width=1.0\linewidth]{figures_TIFS/EMBER_CIL_IFS_UNIFORM.pdf}
        \label{fig:EMBER_CIL_IFS_U}
        \vspace{-0.4cm}
        \caption{MADAR Uniform}
    \end{subfigure}
    \vfill
    \begin{subfigure}{0.485\linewidth}
        \centering
        \includegraphics[width=1.0\linewidth]{figures_TIFS/EMBER_CIL_AWS_RATIO.pdf}
        \label{fig:EMBER_CIL_AWS_R}
        \vspace{-0.4cm}
        \caption{MADAR$^\theta$ Ratio}
    \end{subfigure}
    \hfill
    \begin{subfigure}{0.485\linewidth}
        \centering
        \includegraphics[width=1.0\linewidth]{figures_TIFS/EMBER_CIL_AWS_UNIFORM.pdf}
        \label{fig:EMBER_CIL_AWS_U}
        \vspace{-0.4cm}
        \caption{MADAR$^\theta$ Uniform}
    \end{subfigure}

    \caption{EMBER Class-IL: Comparison of the MADAR-R, MADAR-U, MADAR$^\theta$-R, and MADAR$^\theta$-U with Joint baseline.}
    \label{fig:ember_CIL}
    \vspace{-0.3cm}
\end{figure}


For the experiments with AZ-Domain, we consider 9 tasks, each representing a yearly data distribution from 2008 to 2016. The performance of each method is presented in Table~\ref{tab:az_DIL} as $\mathbf{\overline{AP}}$ and compared to two baselines: \textit{None}, which achieves $94.4\%$, and \textit{Joint}, which reaches $97.3\%$. Additionally, Figure~\ref{fig:az_DIL} illustrates the progression of average accuracy across tasks, highlighting the comparison with the \textit{Joint} baseline.

Similar to the results observed with EMBER, our MADAR techniques consistently outperform prior methods such as ER, AGEM, GR, RtF, and BI-R across all budget levels. For lower budgets, such as 1K, \system-R achieves $\mathbf{\overline{AP}}$ of $95.8\%$ and coming within 1.5\% of the \textit{Joint} baseline.

At higher budgets, ranging from 100K to 400K, \system-R continues to demonstrate high $\mathbf{\overline{AP}}$ scores of up to $97.0\%$, closely matching GRS and only marginally below the \textit{Joint} baseline, which requires significantly more training samples (680K). Notably, MADAR$^\theta$-R exhibits comparable performance, reaching a peak $\mathbf{\overline{AP}}$ of $97.2\%$ at the highest budget level, further underscoring the efficacy of our diversity-aware approach.



% For the experiments with AZ-Domain, we have 9 tasks, each representing a year from 2008 to 2016. The performance of each method is shown in Table~\ref{tab:az_DIL} as $\mathbf{\overline{AP}}$ and compared with two baselines: {\em None} at $94.4\pm0.1$ and {\em Joint} at $97.3\pm0.1$. Additionally, Figure~\ref{fig:az_DIL} illustrates the progression of average accuracy as tasks progress, compared to the \textit{Joint} baseline. 

% As with EMBER, we find that our MADAR techniques greatly surpass previous methods like ER, AGEM, GR, RtF, and BI-R for every budget level. For lower budgets like 1K, \system-R slightly outperforms GRS and is within 1.5\% of {\em Joint}. For higher budgets (100K-400K), \system-R perform well -- in line with GRS and just slightly below {\em Joint}, which requires 680K training samples. 


% In summary, our results empirically depict the effectiveness of MADAR's diversity-aware sample selection in maximizing the efficiency and effectiveness of a malware classifier in Domain-IL. \system-R is either better or on par with GRS and significantly better than prior work.

In summary, these results empirically demonstrate the effectiveness of MADAR's diversity-aware sample selection in enhancing the efficiency and accuracy of malware classification in Domain-IL scenarios. \system-R and MADAR$^\theta$-R, in particular, consistently either yield on-par or outperform GRS while delivering significant improvements over prior methods.












\begin{table*}[!t]
\centering
\caption{Summary of Results for AZ Class-IL Experiments.}
\vspace{-0.3cm}
\label{tab:az_CIL}
\begin{tabular}{p{1.1cm}|l|c|c|c|c|c|c|c} 

% \toprule 

\multirow{2}{*}{\textbf{Group}} & \multirow{2}{*}{\textbf{Method}} & \multicolumn{7}{c}{\textbf{Budget}} \\ \cline{3-9}

&  & 100 & 500 & 1K & 5K & 10K & 15K & 20K \\ \midrule

\multirow{3}{*}{Baselines} 
& Joint  & \multicolumn{7}{c}{94.2$\pm$0.1} \\ 
& None   & \multicolumn{7}{c}{26.4$\pm$0.2} \\ 
& GRS    & 43.8$\pm$0.7 & 62.9$\pm$0.8 & 70.2$\pm$0.4 & 83.0$\pm$0.3 & 86.4$\pm$0.2 & 88.2$\pm$0.2 & 89.1$\pm$0.2 \\ \midrule

\multirow{6}{*}{\parbox{0.7cm}{Prior \\ Work}} 
& TAMiL~\cite{tamil}  & 53.4$\pm$0.3 & 55.2$\pm$0.3 & 57.6$\pm$0.3 & 60.8$\pm$0.2 & 63.5$\pm$0.1 & 65.3$\pm$0.5 & 67.7$\pm$0.3 \\ 
& iCaRL~\cite{icarl}  & 43.6$\pm$1.2 & 54.9$\pm$1.0 & 61.7$\pm$0.7 & 77.2$\pm$0.4 & 81.5$\pm$0.6 & 83.4$\pm$0.5 & 84.6$\pm$0.5 \\ 
& ER~\cite{er}     & 50.8$\pm$0.7 & 58.3$\pm$0.6 & 58.9$\pm$0.2 & 59.2$\pm$0.8 & 62.9$\pm$0.7 & 63.1$\pm$0.5 & 64.2$\pm$0.4 \\ 
& AGEM~\cite{agem}   & 27.3$\pm$0.7 & 28.0$\pm$1.4 & 27.1$\pm$0.3 & 28.0$\pm$0.6 & 28.2$\pm$1.0 & 29.8$\pm$2.6 & 28.0$\pm$0.8 \\ 
& GR~\cite{gr}     & \multicolumn{7}{c}{22.7$\pm$0.3} \\ 
& RtF~\cite{rtf}    & \multicolumn{7}{c}{22.9$\pm$0.3} \\ 
& BI-R~\cite{BIR}   & \multicolumn{7}{c}{23.4$\pm$0.2} \\ 
& MalCL~\cite{malcl}   & \multicolumn{7}{c}{59.8$\pm$0.4} \\ 
\midrule

\multirow{4}{*}{\system} 
& \system-R & \textbf{59.4$\pm$0.6} & 67.8$\pm$0.9 & 71.9$\pm$0.5 & 82.9$\pm$0.2 & 86.3$\pm$0.1 & 88.2$\pm$0.2 & 89.1$\pm$0.1 \\ 
& \system-U & 57.3$\pm$0.5 & \textbf{70.4$\pm$0.4} & \textbf{76.2$\pm$0.2} & \textbf{86.8$\pm$0.1} & \textbf{89.8$\pm$0.1} & \textbf{91.0$\pm$0.1} & \textbf{91.5$\pm$0.1} \\ \cline{2-9}
& MADAR$^{\theta}$-R & {\bf 58.8$\pm$0.3} & 66.2$\pm$0.7 & 71.0$\pm$0.7 & 81.2$\pm$0.3 & 85.1$\pm$0.2 & 86.9$\pm$0.2 & 88.1$\pm$0.1 \\ 
& MADAR$^{\theta}$-U & 58.5$\pm$0.7 & {\bf 70.1$\pm$0.2} & {\bf 74.7$\pm$0.2} & {\bf 85.5$\pm$0.1} & {\bf 88.7$\pm$0.1} & {\bf 90.3$\pm$0.2} & {\bf 90.7$\pm$0.1} \\ 

\bottomrule

\end{tabular}
\vspace{-0.2cm}
\end{table*}








\begin{figure}[!t]
    \centering
    \begin{subfigure}{0.485\linewidth}
        \centering
        \includegraphics[width=1.0\linewidth]{figures_TIFS/AZ_CIL_IFS_RATIO.pdf}
        \label{fig:AZ_CIL_IFS_R}
        \vspace{-0.4cm}
        \caption{MADAR Ratio}
    \end{subfigure}
    \hfill
    \begin{subfigure}{0.485\linewidth}
        \centering
        \includegraphics[width=1.0\linewidth]{figures_TIFS/AZ_CIL_IFS_UNIFORM.pdf}
        \label{fig:AZ_CIL_IFS_U}
        \vspace{-0.4cm}
        \caption{MADAR Uniform}
    \end{subfigure}
    \vfill
    \begin{subfigure}{0.485\linewidth}
        \centering
        \includegraphics[width=1.0\linewidth]{figures_TIFS/AZ_CIL_AWS_RATIO.pdf}
        \label{fig:AZ_CIL_AWS_R}
        \vspace{-0.4cm}
        \caption{MADAR$^\theta$ Ratio}
    \end{subfigure}
    \hfill
    \begin{subfigure}{0.485\linewidth}
        \centering
        \includegraphics[width=1.0\linewidth]{figures_TIFS/AZ_CIL_AWS_UNIFORM.pdf}
        \label{fig:AZ_CIL_AWS_U}
        \vspace{-0.4cm}
        \caption{MADAR$^\theta$ Uniform}
    \end{subfigure}

    \caption{AZ Class-IL: Comparison of the MADAR-R, MADAR-U, MADAR$^\theta$-R, and MADAR$^\theta$-U with Joint baseline.}
    \label{fig:az_CIL}
    \vspace{-0.3cm}
\end{figure}





\subsection{Class-IL}
\label{classilexps}



In this set of experiments with EMBER, we consider 11 tasks, starting with 50 classes (representing distinct malware families) in the initial task, and incrementally adding five new classes in each subsequent task. Table~\ref{tab:ember_CIL} presents the performance of each method, measured by average accuracy $\mathbf{\overline{AP}}$. The \textit{None} and \textit{Joint} baselines achieve $\mathbf{\overline{AP}}$ values of $26.5\%$ and $86.5\%$, respectively, providing informal lower and upper bounds. Figure~\ref{fig:ember_CIL} illustrates the progression of average accuracy across tasks, showing how the \system\ and MADAR$^\theta$ methods compare to the \textit{Joint} baseline.

At a very low budget of just 100 samples, \system-R achieves a notable $\mathbf{\overline{AP}}$ of $68.0\%$, outperforming GRS and prior methods by a significant margin. As the budget increases, \system-U emerges as the top performer, achieving $\mathbf{\overline{AP}}$ values of $76.5\%$ and $79.4\%$ at 1K and 10K budgets, respectively, surpassing all other methods, including GRS. 

%For example, at a 10K budget, \system-U outperforms GRS, which achieves $83.5\%$, with an $\mathbf{\overline{AP}}$ of $84.8\%$.

At higher budgets, \system-U and MADAR$^\theta$-U continue to excel, with MADAR$^\theta$-U achieving the best results overall. At a 20K budget, MADAR$^\theta$-U reaches an $\mathbf{\overline{AP}}$ of $86.2\%$, nearly equaling the \textit{Joint} baseline, which uses over {\bf 150 times} more training samples. These results clearly demonstrate the effectiveness of MADAR's diversity-aware sample selection and the effectiveness of \system-U and MADAR$^\theta$-U in leveraging limited resources.

In contrast, prior methods such as ER, AGEM, GR, RtF, and BI-R fail to exceed 30\% $\mathbf{\overline{AP}}$, while more advanced techniques like TAMiL and MalCL achieve only $38.2\%$ and $54.8\%$, respectively. Even iCaRL, designed specifically for Class-IL, achieves only $64.6\%$, further highlighting the significant advantage of our approaches in the malware domain.


% In this set of experiments with EMBER, we have 11 tasks, where the initial task starts with 50 classes---one for each of 50 malware families---and five classes are added in each subsequent task. The performance of these methods, detailed in Table~\ref{tab:az_CIL}, is measured by average accuracy $\mathbf{\overline{AP}}$ with {\em None} and {\em Joint} training baselines at an $\mathbf{\overline{AP}}$ of $26.5\pm0.2$ and $86.5\pm0.4$, respectively. Additionally, Figure~\ref{fig:ember_CIL} illustrates the progression of average accuracy across tasks, highlighting the comparison with the \textit{Joint} baseline. 

% For a very low budget of 100 samples, \system methods greatly outperform GRS, with \system-R getting 16\% higher $\mathbf{\overline{AP}}$. For more reasonable budgets, however, the uniform variant \system-U offers the best performance. For example, with a 10K budget, \system-U yields at least 84.8\% $\mathbf{\overline{AP}}$, which is better than GRS at 83.5\% $\mathbf{\overline{AP}}$. They also fare far better than all prior works, with ER, AGEM, GR, RtF, and BI-R below 30\%, TAMiL at 38.2\%, MalCL at 54.8\% and iCaRL at only 64.6\%. These poor results for the prior methods are in line with other findings in the malware domain~\cite{continual-learning-malware}. For a budget of 20K, \system-U reaches $85.8\pm0.3$, nearly as good as the Joint baseline that uses a maximum budget over 150 times larger.



In the Class-IL setting of AZ-Class, we consider 11 tasks. The summary results of all experiments are provided in Table~\ref{tab:az_CIL}, with comparisons against the \textit{None} and \textit{Joint} baselines, which achieve $\mathbf{\overline{AP}}$ scores of $26.4\%$ and $94.2\%$, respectively. Figure~\ref{fig:az_CIL} illustrates the progression of average accuracy across tasks, showing how each method performs relative to the \textit{Joint} baseline.

As shown in Table~\ref{tab:az_CIL}, among the prior methods, iCaRL performs best across most budget configurations, outperforming techniques such as MalCL, TAMiL, ER, AGEM, GR, RtF, and BI-R. Therefore, we focus on comparing MADAR's performance with iCaRL. At a low budget of 100 samples, iCaRL and GRS achieve less than $44\%$ $\mathbf{\overline{AP}}$, while all MADAR methods surpass $57\%$. In particular, \system-R and MADAR$^\theta$-R achieve $\mathbf{\overline{AP}}$ scores of $59.4\%$ and $58.8\%$, respectively, highlighting their efficiency at low-resource levels.

As the budget increases, all methods improve, but \system-U consistently delivers the best results. At a budget of 1K, \system-U achieves the highest $\mathbf{\overline{AP}}$ at $70.4\%$, followed closely by MADAR$^\theta$-U at $70.1\%$. This trend continues as budgets increase, with \system-U outperforming all other methods, achieving $\mathbf{\overline{AP}}$ scores of $89.8\%$ at 10K and $91.5\%$ at 20K. Compared to GRS, which achieves $90.1\%$ at 20K, and iCaRL, which trails at $84.6\%$, \system-U demonstrates clear superiority. MADAR$^\theta$-U also performs GRS reaching $90.7\%$ at 20K.



% We have 11 tasks for the Class-IL setting of AZ-Class. The summary results of all the experiments are shown in Table~\ref{tab:az_CIL} and benchmarked against {\em None} and {\em Joint} with $\mathbf{\overline{AP}}$ of $26.4\pm0.2$ and $94.2\pm0.1$, respectively. Figure~\ref{fig:az_CIL} illustrates the progression of average accuracy across tasks, highlighting the comparison with the \textit{Joint} baseline. 


% As we can from Table~\ref{tab:az_CIL} that, among TAMiL, iCaRL, ER, AGEM, GR, RtF, and BI-R, iCaRL outperforms in most of the budget configurations. Therefore, we discuss the results of MADAR in comparison with iCaRL. For a low budget of 100, iCaRL and GRS get less than 44\%, while all MADAR methods achieve over 57\%. As budgets increase, all methods improve, with \system-U offering the best results at every budget from 1K to 20K. At 20K, it reaches $91.5\pm0.1\%$, which is 1.4\% higher than GRS and 6.9\% higher than iCaRL.



In summary, our experiments demonstrate the effectiveness of \system's diversity-aware replay techniques in the Class-IL setting for both EMBER and AZ datasets. While GRS shows significant improvement with larger budgets, \system's uniform variants consistently outperform it across all budget levels. These results underscore \system's ability to mitigate catastrophic forgetting and enhance malware classification performance, even in resource-constrained environments.

% In summary, our experiments clearly demonstrate the effectiveness of \system's diversity-aware replay techniques in Class-IL for both EMBER and AZ datasets. Additionally, while GRS shows significant improvement with an increased budget, the uniform variants of \system  are more effective at every budget level. \system  significantly improves performance in malware classification by mitigating catastrophic forgetting, and they do so using fewer resources.












\begin{table*}[!t]
\centering
\caption{Summary of Results for EMBER Task-IL Experiments.}
\vspace{-0.3cm}
\label{tab:ember_TIL}
\begin{tabular}{p{1.1cm}|l|c|c|c|c|c|c|c} 

% \toprule 

\multirow{2}{*}{\textbf{Group}} & \multirow{2}{*}{\textbf{Method}} & \multicolumn{7}{c}{\textbf{Budget}} \\ \cline{3-9}

&  & 100 & 500 & 1K & 5K & 10K & 15K & 20K \\ \midrule

\multirow{3}{*}{Baselines} 
& Joint  & \multicolumn{7}{c}{97.0$\pm$0.3} \\ 
& None   & \multicolumn{7}{c}{74.6$\pm$0.7} \\ 
& GRS    & 86.9$\pm$0.3 & 87.4$\pm$0.3 & 93.6$\pm$0.3 & 94.4$\pm$0.2 & 94.7$\pm$0.3 & 94.9$\pm$0.1 & 95.0$\pm$0.1 \\ \midrule

\multirow{6}{*}{\parbox{0.7cm}{Prior \\ Work}} 
& TAMiL~\cite{tamil}  & 72.8$\pm$0.1 & 81.5$\pm$0.3 & 86.9$\pm$0.2 & 88.1$\pm$0.3 & 90.3$\pm$0.1 & 93.2$\pm$0.3 & 94.2$\pm$0.7 \\ 
& ER~\cite{er}     & 67.4$\pm$0.3 & 84.9$\pm$0.2 & 89.5$\pm$0.5 & 93.9$\pm$0.2 & 94.8$\pm$0.2 & 95.2$\pm$0.1 & 95.4$\pm$0.1 \\ 
& AGEM~\cite{agem}   & 79.6$\pm$0.2 & 81.7$\pm$0.2 & 83.8$\pm$0.4 & 84.9$\pm$0.2 & 86.1$\pm$0.2 & 88.9$\pm$0.2 & 89.3$\pm$0.1 \\ 
& GR~\cite{gr}     & \multicolumn{7}{c}{79.8$\pm$0.3} \\ 
& RtF~\cite{rtf}    & \multicolumn{7}{c}{77.8$\pm$0.2} \\ 
& BI-R~\cite{BIR}   & \multicolumn{7}{c}{87.2$\pm$0.3} \\ \midrule

\multirow{4}{*}{\system} 
& \system-R & 92.1$\pm$0.2 & 92.3$\pm$0.9 & 93.8$\pm$0.2 & 94.2$\pm$0.1 & 94.8$\pm$0.2 & {\bf 95.7$\pm$0.2} & {\bf 95.6$\pm$0.1} \\ 
& \system-U & {\bf 93.4$\pm$0.2} & {\bf 93.7$\pm$0.3} & {\bf 93.9$\pm$0.3} & {\bf 94.8$\pm$0.2} & {\bf 95.6$\pm$0.1} & {\bf 95.7$\pm$0.1} & {\bf 95.8$\pm$0.2} \\ \cline{2-9}
& MADAR$^{\theta}$-R & {\bf 93.1$\pm$0.2} & {\bf 93.3$\pm$0.1} & {\bf 93.6$\pm$0.1} & 94.3$\pm$0.1 & 94.6$\pm$0.2 & 94.8$\pm$0.2 & 94.7$\pm$0.3 \\ 
& MADAR$^{\theta}$-U & {\bf 93.2$\pm$0.1} & 93.1$\pm$0.2 & {\bf 93.8$\pm$0.2} & {\bf 94.4$\pm$0.1} & {\bf 94.8$\pm$0.1} & {\bf 95.3$\pm$0.2} & {\bf 95.5$\pm$0.3} \\ 

\bottomrule

\end{tabular}
\vspace{-0.3cm}
\end{table*}



\begin{figure}[!t]
    \centering
    \begin{subfigure}{0.485\linewidth}
        \centering
        \includegraphics[width=1.0\linewidth]{figures_TIFS/EMBER_TIL_IFS_RATIO.pdf}
        \label{fig:EMBER_TIL_IFS_R}
        \vspace{-0.4cm}
        \caption{MADAR Ratio}
    \end{subfigure}
    \hfill
    \begin{subfigure}{0.485\linewidth}
        \centering
        \includegraphics[width=1.0\linewidth]{figures_TIFS/EMBER_TIL_IFS_UNIFORM.pdf}
        \label{fig:EMBER_TIL_IFS_U}
        \vspace{-0.4cm}
        \caption{MADAR Uniform}
    \end{subfigure}
    \vfill
    \begin{subfigure}{0.485\linewidth}
        \centering
        \includegraphics[width=1.0\linewidth]{figures_TIFS/EMBER_TIL_AWS_RATIO.pdf}
        \label{fig:EMBER_TIL_AWS_R}
        \vspace{-0.4cm}
        \caption{MADAR$^\theta$ Ratio}
    \end{subfigure}
    \hfill
    \begin{subfigure}{0.485\linewidth}
        \centering
        \includegraphics[width=1.0\linewidth]{figures_TIFS/EMBER_TIL_AWS_UNIFORM.pdf}
        \label{fig:EMBER_TIL_AWS_U}
        \vspace{-0.4cm}
        \caption{MADAR$^\theta$ Uniform}
    \end{subfigure}

    \caption{EMBER Task-IL: Comparison of the MADAR-R, MADAR-U, MADAR$^\theta$-R, and MADAR$^\theta$-U with Joint baseline.}
    \label{fig:ember_TIL}
    \vspace{-0.3cm}
\end{figure}

























\subsection{Task-IL}
\label{taskilexps-ember}


In this set of experiments with EMBER, we consider 20 tasks, with 5 new classes added in each task. The summarized results are shown in Table~\ref{tab:ember_TIL}, where performance is reported as the average accuracy over all tasks ($\mathbf{\overline{AP}}$). It is worth noting that Task-IL is considered the easiest scenario in continual learning~\cite{van2022three, BIR}. The \textit{None} and \textit{Joint} methods serve as informal lower and upper bounds, achieving $\mathbf{\overline{AP}}$ scores of $74.6\%$ and $97\%$, respectively. Figure~\ref{fig:ember_TIL} visualizes the progression of average accuracy across tasks, highlighting comparisons with the \textit{Joint} baseline.

As shown in Table~\ref{tab:ember_TIL}, ER consistently outperforms TAMiL, A-GEM, GR, RtF, and BI-R across all budget configurations and even surpasses GRS in some cases. However, \system\ variants significantly outperform all prior methods, particularly under lower budget constraints (100–1K). For example, \system-U achieves the highest $\mathbf{\overline{AP}}$ of $93.4\%$ and $93.7\%$ at budgets of 100 and 1K, respectively, outperforming GRS and all other approaches. Similarly, MADAR$^\theta$-U performs competitively, with $\mathbf{\overline{AP}}$ of $93.2\%$ at a 100 budget and $93.8\%$ at 1K.

As the budget increases, the performance gap among \system, ER, and GRS narrows; however, \system\ variants continue to either outperform or match other techniques. Notably, the \system-U variant of MADAR achieves the best overall performance at a budget of 20K, attaining a $\mathbf{\overline{AP}}$ of $95.8\%$, which closely approaches the \textit{Joint} baseline. Similarly, \system-R yields $\mathbf{\overline{AP}}$ of $95.6\%$ at 20K.



% In this set of experiments with EMBER, we have 20 tasks with 5 new classes in each task. Table~\ref{tab:ember_TIL} shows a summarized view of this set of experiments, where the performances are presented as the average accuracy over all tasks ($\mathbf{\overline{AP}}$). Note that Task-IL is considered the easiest scenario of continual learning~\cite{van2022three, BIR}. The {\em None} and {\em Joint} methods, which are the informal lower and upper bounds of this configuration, attain $\overline{AP}$ of $74.6\%$ and $\overline{AP}$ of $97.03\%$, respectively. Figure~\ref{fig:ember_TIL} illustrates the progression of average accuracy across tasks, showing how each method performs relative to the \textit{Joint} baseline.

% As we can see from Table~\ref{tab:combined_TIL}, ER outperforms TAMiL, A-GEM, GR, RtF, and BI-R in all budget configurations and outperforms GRS for few configurations. \system, on the other hand, outperforms all the prior methods significantly in lower budget constraints ($100$–$1K$). For instance, \system-U reaches $\mathbf{\overline{AP}}$ of 93.9\% with only 1K replay samples, compared with 93.6\% for GRS. The performance gap among MADAR, ER, and GRS gets closer as the budget increases; however, \system  variants continue to either outperform or perform on par with other techniques. In particular, the \system-U variant of MADAR outperforms all the other techniques and attains $\mathbf{\overline{AP}}$ of 95.8\% with a 20K replay budget, which is close to joint level performance.


Task-IL for AZ consists of 20 tasks, each with 5 non-overlapping classes. The results are summarized in Table~\ref{tab:az_TIL} and benchmarked against the \textit{None} and \textit{Joint} baselines, which achieve $\mathbf{\overline{AP}}$ values of $74.5\%$ and $98.8\%$, respectively. Figure~\ref{fig:az_TIL} illustrates the progression of average accuracy across tasks, showing how each method performs relative to the \textit{Joint} baseline.

As seen in Table~\ref{tab:az_TIL}, ER consistently outperforms TAMiL, AGEM, GR, RtF, BI-R, and GRS across most budget configurations, making it a strong baseline for comparison. At a low budget of 100 samples, \system-U achieves $\mathbf{\overline{AP}}$ of $88.1\%$, which is 4.5\% higher than ER's performance. Similarly, MADAR$^\theta$-U demonstrates competitive performance, achieving $87.9\%$ at the same budget.

As the budget increases, \system-U continues to deliver the best performance, reaching $\mathbf{\overline{AP}}$ scores of $94.5\%$ at a 1K budget and $98.1\%$ at a 10K budget, outperforming all other methods, including ER and GRS. At the highest budget of 20K, \system-U achieves an $\mathbf{\overline{AP}}$ of $98.7\%$, surpassing ER by 1.2\% and nearly matching the \textit{Joint} baseline. Notably, MADAR$^\theta$-U also performs well, achieving $98.1\%$. In contrast, \system-R and MADAR$^\theta$-R perform slightly lower but remain competitive, with $\mathbf{\overline{AP}}$ values of $97.9\%$ and $96.9\%$ at a 20K budget, respectively. These results indicate that ratio-based methods, while effective, are slightly less robust than uniform sampling in this scenario.

In summary, \system-U and MADAR$^\theta$-U consistently demonstrate better performance across most of the budget levels, particularly excelling at low-resource settings and achieving near-optimal results at higher budgets. These findings underscore the effectiveness of \system\ framework in Task-IL scenarios and their ability to approach joint-level performance with significantly fewer resources.


% Task-IL for AZ contains 20 tasks, each with 5 non-overlapping classes. Our results are shown in Table~\ref{tab:az_TIL}, compared against the {\em None} and {\em Joint} benchmarks, with $\mathbf{\overline{AP}}$ of 74.5\% and 98.8\%, respectively. Figure~\ref{fig:az_TIL} illustrates the progression of average accuracy across tasks, showing how each method performs relative to the \textit{Joint} baseline. As with EMBER, ER outperforms TAMiL, AGEM, GR, RtF, BI-R, and GRS for most budgets, so we use it for comparison. For a low budget of 100, \system-U achieves an $\overline{AP}$ of 88.1\%, 4.5\% higher than that of ER. For a higher budget of 20K, \system-U attains an $\overline{AP}$ of 98.7\%, which is 1.2\% higher than that of ER and very close to the joint level performance of 98.8\%.


% Overall, mirroring the success seen with the EMBER dataset, our proposed techniques also surpass previous work in Task-IL in the context of the AZ-Class dataset. Additionally, while ER and GRS shows significant improvement with an increased budget, the uniform variant of IFS of MADAR is more effective at every budget level.








\begin{table*}[!t]
\centering
\caption{Summary of Results for AZ Task-IL Experiments.}
\vspace{-0.3cm}
\label{tab:az_TIL}
\begin{tabular}{p{1.1cm}|l|c|c|c|c|c|c|c} 

% \toprule 

\multirow{2}{*}{\textbf{Group}} & \multirow{2}{*}{\textbf{Method}} & \multicolumn{7}{c}{\textbf{Budget}} \\ \cline{3-9}

&  & 100 & 500 & 1K & 5K & 10K & 15K & 20K \\ \midrule

\multirow{3}{*}{Baselines} 
& Joint  & \multicolumn{7}{c}{98.8$\pm$0.2} \\ 
& None   & \multicolumn{7}{c}{74.5$\pm$0.2} \\ 
& GRS    & 85.2$\pm$0.1 & 89.2$\pm$0.2 & 90.8$\pm$0.1 & 91.6$\pm$0.2 & 93.5$\pm$0.1 & 93.9$\pm$0.1 & 95.2$\pm$0.1 \\ \midrule

\multirow{6}{*}{\parbox{0.7cm}{Prior \\ Work}} 
& TAMiL  & 80.5$\pm$0.4 & 85.3$\pm$0.6 & 91.5$\pm$0.2 & 92.1$\pm$0.1 & 93.5$\pm$0.1 & 94.0$\pm$0.2 & 94.8$\pm$0.2 \\ 
& ER     & 83.6$\pm$0.2 & 90.2$\pm$0.1 & 92.3$\pm$0.3 & 95.6$\pm$0.1 & 96.2$\pm$0.1 & 96.8$\pm$0.2 & 97.5$\pm$0.2 \\ 
& AGEM   & 76.7$\pm$0.5 & 82.8$\pm$0.2 & 85.3$\pm$0.1 & 85.6$\pm$0.2 & 86.7$\pm$0.2 & 88.9$\pm$0.2 & 91.3$\pm$0.3 \\ 
& GR     & \multicolumn{7}{c}{75.6$\pm$0.2} \\ 
& RtF    & \multicolumn{7}{c}{74.2$\pm$0.3} \\ 
& BI-R   & \multicolumn{7}{c}{85.4$\pm$0.2} \\ \midrule

\multirow{4}{*}{\system} 
& \system-R & 86.0$\pm$0.3 & 90.3$\pm$0.2 & 92.4$\pm$0.1 & 95.8$\pm$0.2 & 96.7$\pm$0.1 & 97.1$\pm$0.1 & 97.9$\pm$0.2 \\ 
& \system-U & {\bf 88.1$\pm$0.3} & {\bf 92.9$\pm$0.2} & {\bf 94.5$\pm$0.3} & {\bf 97.2$\pm$0.2} & {\bf 98.1$\pm$0.1} & {\bf 98.5$\pm$0.1} & {\bf 98.7$\pm$0.1} \\ \cline{2-9}
& MADAR$^{\theta}$-R & 87.3$\pm$0.3 & {\bf 90.6$\pm$0.2} & 93.2$\pm$0.2 & 95.7$\pm$0.2 & 95.9$\pm$0.1 & 96.6$\pm$0.1 & 96.9$\pm$0.1 \\ 
& MADAR$^{\theta}$-U & {\bf 87.9$\pm$0.2} & {\bf 90.8$\pm$0.2} & {\bf 93.6$\pm$0.1} & {\bf 96.2$\pm$0.3} & {\bf 97.2$\pm$0.2} & {\bf 97.5$\pm$0.2} & {\bf 98.1$\pm$0.1} \\ 

\bottomrule

\end{tabular}
\vspace{-0.3cm}
\end{table*}



\begin{figure}[!t]
    \centering
    \begin{subfigure}{0.45\linewidth}
        \centering
        \includegraphics[width=1.0\linewidth]{figures_TIFS/AZ_TIL_IFS_RATIO.pdf}
        \label{fig:AZ_TIL_IFS_R}
        \vspace{-0.4cm}
        \caption{MADAR Ratio}
    \end{subfigure}
    \hfill
    \begin{subfigure}{0.45\linewidth}
        \centering
        \includegraphics[width=1.0\linewidth]{figures_TIFS/AZ_TIL_IFS_UNIFORM.pdf}
        \label{fig:AZ_TIL_IFS_U}
        \vspace{-0.4cm}
        \caption{MADAR Uniform}
    \end{subfigure}
    \vfill
    \begin{subfigure}{0.45\linewidth}
        \centering
        \includegraphics[width=1.0\linewidth]{figures_TIFS/AZ_TIL_AWS_RATIO.pdf}
        \label{fig:AZ_TIL_AWS_R}
        \vspace{-0.4cm}
        \caption{MADAR$^\theta$ Ratio}
    \end{subfigure}
    \hfill
    \begin{subfigure}{0.45\linewidth}
        \centering
        \includegraphics[width=1.0\linewidth]{figures_TIFS/AZ_TIL_AWS_UNIFORM.pdf}
        \label{fig:AZ_TIL_AWS_U}
        \vspace{-0.4cm}
        \caption{MADAR$^\theta$ Uniform}
    \end{subfigure}

    \caption{AZ Task-IL: Comparison of the MADAR-R, MADAR-U, MADAR$^\theta$-R, and MADAR$^\theta$-U with Joint baseline.}
    \label{fig:az_TIL}
    \vspace{-0.3cm}
\end{figure}


\subsection{Analysis and Discussion}\label{diss}


Our results demonstrate that MADAR yields markedly better performances compared to previous methods for both the EMBER and AZ datasets across all CL settings. This clearly indicates that diversity-aware replay is effective in preserving the stability of a CL-based system for malware classification, while prior CL techniques largely fail to achieve acceptable performance.


\paragraphX{\bf MADAR in low-budget settings.} In Domain-IL, MADAR achieves competitive performance even with a 1K budget, surpassing prior work by over 3 percentage points in EMBER and AZ. At higher budgets, ratio-based selection (\system-R and MADAR$^{\theta}$-R) achieves near Joint baseline performance (96.4\% in EMBER and 97.3\% in AZ) while using significantly fewer resources. This demonstrates MADAR’s efficiency in leveraging limited samples to achieve robust classification.


\paragraphX{\bf MADAR is both effective and scalable.} Traditional CL methods, including ER and AGEM, experience significant performance degradation as tasks increase. In contrast, MADAR maintains high accuracy across 20 Task-IL tasks, with \system-U achieving 95.8\% in EMBER and 98.7\% in AZ at a 20K budget, nearly matching the {\em Joint} baseline.




\paragraphX{\bf Ratio vs. Uniform Budgeting.} A consistent trend across our experiments is that ratio-based selection performs best in Domain-IL, whereas uniform-based selection is superior in Class-IL and Task-IL. MADAR$^{\theta}$-U reaches 91.5\% in AZ at 20K, significantly outperforming iCaRL and TAMiL. Furthermore, in EMBER, \system-U achieves near {\em Joint} baseline performance at just a 5K budget, underscoring the effectiveness of uniform selection in class-incremental settings. Intuitively, this makes sense because ratio budgeting for binary classification in the Domain-IL setting naturally captures the contributions of each family to the overall malware distribution. Additionally, since there are many small families in the Domain-IL datasets, uniformly sampling from them consumes budget while offering little improvement in malware coverage. In contrast, our Class-IL and Task-IL experiments perform classification across families, which is better supported by Uniform budgeting to maintain class balance and ensure coverage over all families. Moreover, in most settings we can leverage efficient representations using MADAR$^\theta$ to scale the approach regardless of feature dimension without significant loss of performance.



\paragraphX{\bf GRS remains a strong baseline at high budgets.} While MADAR consistently outperforms GRS in low-resource settings, GRS performs comparably at higher budgets, particularly in Domain-IL. This suggests that diversity-aware replay is most impactful when the number of available samples per class is limited, whereas uniform selection provides sufficient representation at larger budgets.















\if 0
Our results demonstrate that MADAR yields markedly better performances compared to previous methods for both the EMBER and AZ datasets across all CL settings. This clearly indicates that diversity-aware replay is effective in preserving the stability of a CL-based system for malware classification, while prior CL techniques largely fail to achieve acceptable performance.


In the Domain-IL scenario, MADAR consistently achieves better performance than all other methods, particularly at lower budgets. For example, MADAR's uniform and ratio variants surpass other methods with $\mathbf{\overline{AP}}$ values exceeding $93.6\%$ in EMBER and $95.7\%$ in AZ at a 1K budget. As the memory budget increases, the ratio-based variants (\system-R and MADAR$^\theta$-R) excel, approaching the \textit{Joint} baselines of $96.4\%$ for EMBER and $97.3\%$ for AZ. Notably, these results are achieved with significantly fewer replay samples compared to the \textit{Joint} baseline, highlighting MADAR's efficiency in leveraging limited resources.


In the Class-IL scenario, MADAR achieves remarkable improvements over prior methods, including iCaRL and TAMiL, on both EMBER and AZ datasets. For EMBER, \system-U achieves near \textit{Joint} baseline performance with a budget as low as 5K, outperforming iCaRL  method with fewer resources. Similarly, in AZ, MADAR$^\theta$-U reaches an impressive $\mathbf{\overline{AP}}$ of $91.5\%$ at a 20K budget, significantly surpassing prior techniques. Across both datasets, uniform variants (\system-U and MADAR$^\theta$-U) consistently outperform other methods, demonstrating their effectiveness in managing resources and adapting to evolving class distributions.


In the Task-IL scenario, MADAR outperforms prior methods by a significant margin for both the EMBER and AZ datasets, confirming that diversity-aware replay is effective for this scenario. For EMBER, \system-U achieves $\mathbf{\overline{AP}}$ values of $95.8\%$ at a 20K budget, effectively matching \textit{Joint} performance with a fraction of the resources. For AZ, MADAR$^\theta$-U attains $98.7\%$ at 20K, further underscoring the efficacy of diversity-aware techniques in resource-constrained settings.These findings highlight that the MADAR framework, particularly the uniform variant, not only matches but often exceeds the effectiveness of existing techniques, confirming its robustness across various budget levels in Task-IL.


The Ratio variants worked better for Domain-IL experiments, while Uniform variants worked well in Class-IL and Task-IL. Intuitively, this makes sense because ratio budgeting for binary classification in the Domain-IL setting naturally captures the contributions of each family to the overall malware distribution. Additionally, since there are many small families in the Domain-IL datasets, uniformly sampling from them consumes budget while offering little improvement in malware coverage. In contrast, our Class-IL and Task-IL experiments perform classification across families, which is better supported by Uniform budgeting to maintain class balance and ensure coverage over all families. Moreover, in most settings we can leverage efficient representations using MADAR$^\theta$ to scale the approach regardless of feature dimension without significant loss of performance.


Our results show that GRS performs very well, in some cases closer to the performances of MADAR. Indeed, uniform random sampling should be expected to be a strong baseline, since it provides an unbiased estimate of the true underlying distribution. MADAR is particularly effective in Class-IL and Task-IL, and for lower budgets in Domain-IL, while GRS generally performs as well as MADAR in higher-budget Domain-IL settings. We hypothesize that MADAR's diversity-aware approach is more important when the number of samples per class is limited. In our Domain-IL experiments, larger budgets enable a sufficient representation of the distributions of both classes with uniform selection, making MADAR useful only at smaller budget sizes. 
\fi 

















\section{Discussion}\label{sec:discussion}



\subsection{From Interactive Prompting to Interactive Multi-modal Prompting}
The rapid advancements of large pre-trained generative models including large language models and text-to-image generation models, have inspired many HCI researchers to develop interactive tools to support users in crafting appropriate prompts.
% Studies on this topic in last two years' HCI conferences are predominantly focused on helping users refine single-modality textual prompts.
Many previous studies are focused on helping users refine single-modality textual prompts.
However, for many real-world applications concerning data beyond text modality, such as multi-modal AI and embodied intelligence, information from other modalities is essential in constructing sophisticated multi-modal prompts that fully convey users' instruction.
This demand inspires some researchers to develop multimodal prompting interactions to facilitate generation tasks ranging from visual modality image generation~\cite{wang2024promptcharm, promptpaint} to textual modality story generation~\cite{chung2022tale}.
% Some previous studies contributed relevant findings on this topic. 
Specifically, for the image generation task, recent studies have contributed some relevant findings on multi-modal prompting.
For example, PromptCharm~\cite{wang2024promptcharm} discovers the importance of multimodal feedback in refining initial text-based prompting in diffusion models.
However, the multi-modal interactions in PromptCharm are mainly focused on the feedback empowered the inpainting function, instead of supporting initial multimodal sketch-prompt control. 

\begin{figure*}[t]
    \centering
    \includegraphics[width=0.9\textwidth]{src/img/novice_expert.pdf}
    \vspace{-2mm}
    \caption{The comparison between novice and expert participants in painting reveals that experts produce more accurate and fine-grained sketches, resulting in closer alignment with reference images in close-ended tasks. Conversely, in open-ended tasks, expert fine-grained strokes fail to generate precise results due to \tool's lack of control at the thin stroke level.}
    \Description{The comparison between novice and expert participants in painting reveals that experts produce more accurate and fine-grained sketches, resulting in closer alignment with reference images in close-ended tasks. Novice users create rougher sketches with less accuracy in shape. Conversely, in open-ended tasks, expert fine-grained strokes fail to generate precise results due to \tool's lack of control at the thin stroke level, while novice users' broader strokes yield results more aligned with their sketches.}
    \label{fig:novice_expert}
    % \vspace{-3mm}
\end{figure*}


% In particular, in the initial control input, users are unable to explicitly specify multi-modal generation intents.
In another example, PromptPaint~\cite{promptpaint} stresses the importance of paint-medium-like interactions and introduces Prompt stencil functions that allow users to perform fine-grained controls with localized image generation. 
However, insufficient spatial control (\eg, PromptPaint only allows for single-object prompt stencil at a time) and unstable models can still leave some users feeling the uncertainty of AI and a varying degree of ownership of the generated artwork~\cite{promptpaint}.
% As a result, the gap between intuitive multi-modal or paint-medium-like control and the current prompting interface still exists, which requires further research on multi-modal prompting interactions.
From this perspective, our work seeks to further enhance multi-object spatial-semantic prompting control by users' natural sketching.
However, there are still some challenges to be resolved, such as consistent multi-object generation in multiple rounds to increase stability and improved understanding of user sketches.   


% \new{
% From this perspective, our work is a step forward in this direction by allowing multi-object spatial-semantic prompting control by users' natural sketching, which considers the interplay between multiple sketch regions.
% % To further advance the multi-modal prompting experience, there are some aspects we identify to be important.
% % One of the important aspects is enhancing the consistency and stability of multiple rounds of generation to reduce the uncertainty and loss of control on users' part.
% % For this purpose, we need to develop techniques to incorporate consistent generation~\cite{tewel2024training} into multi-modal prompting framework.}
% % Another important aspect is improving generative models' understanding of the implicit user intents \new{implied by the paint-medium-like or sketch-based input (\eg, sketch of two people with their hands slightly overlapping indicates holding hand without needing explicit prompt).
% % This can facilitate more natural control and alleviate users' effort in tuning the textual prompt.
% % In addition, it can increase users' sense of ownership as the generated results can be more aligned with their sketching intents.
% }
% For example, when users draw sketches of two people with their hands slightly overlapping, current region-based models cannot automatically infer users' implicit intention that the two people are holding hands.
% Instead, they still require users to explicitly specify in the prompt such relationship.
% \tool addresses this through sketch-aware prompt recommendation to fill in the necessary semantic information, alleviating users' workload.
% However, some users want the generative AI in the future to be able to directly infer this natural implicit intentions from the sketches without additional prompting since prompt recommendation can still be unstable sometimes.


% \new{
% Besides visual generation, 
% }
% For example, one of the important aspect is referring~\cite{he2024multi}, linking specific text semantics with specific spatial object, which is partly what we do in our sketch-aware prompt recommendation.
% Analogously, in natural communication between humans, text or audio alone often cannot suffice in expressing the speakers' intentions, and speakers often need to refer to an existing spatial object or draw out an illustration of her ideas for better explanation.
% Philosophically, we HCI researchers are mostly concerned about the human-end experience in human-AI communications.
% However, studies on prompting is unique in that we should not just care about the human-end interaction, but also make sure that AI can really get what the human means and produce intention-aligned output.
% Such consideration can drastically impact the design of prompting interactions in human-AI collaboration applications.
% On this note, although studies on multi-modal interactions is a well-established topic in HCI community, it remains a challenging problem what kind of multi-modal information is really effective in helping humans convey their ideas to current and next generation large AI models.




\subsection{Novice Performance vs. Expert Performance}\label{sec:nVe}
In this section we discuss the performance difference between novice and expert regarding experience in painting and prompting.
First, regarding painting skills, some participants with experience (4/12) preferred to draw accurate and fine-grained shapes at the beginning. 
All novice users (5/12) draw rough and less accurate shapes, while some participants with basic painting skills (3/12) also favored sketching rough areas of objects, as exemplified in Figure~\ref{fig:novice_expert}.
The experienced participants using fine-grained strokes (4/12, none of whom were experienced in prompting) achieved higher IoU scores (0.557) in the close-ended task (0.535) when using \tool. 
This is because their sketches were closer in shape and location to the reference, making the single object decomposition result more accurate.
Also, experienced participants are better at arranging spatial location and size of objects than novice participants.
However, some experienced participants (3/12) have mentioned that the fine-grained stroke sometimes makes them frustrated.
As P1's comment for his result in open-ended task: "\emph{It seems it cannot understand thin strokes; even if the shape is accurate, it can only generate content roughly around the area, especially when there is overlapping.}" 
This suggests that while \tool\ provides rough control to produce reasonably fine results from less accurate sketches for novice users, it may disappoint experienced users seeking more precise control through finer strokes. 
As shown in the last column in Figure~\ref{fig:novice_expert}, the dragon hovering in the sky was wrongly turned into a standing large dragon by \tool.

Second, regarding prompting skills, 3 out of 12 participants had one or more years of experience in T2I prompting. These participants used more modifiers than others during both T2I and R2I tasks.
Their performance in the T2I (0.335) and R2I (0.469) tasks showed higher scores than the average T2I (0.314) and R2I (0.418), but there was no performance improvement with \tool\ between their results (0.508) and the overall average score (0.528). 
This indicates that \tool\ can assist novice users in prompting, enabling them to produce satisfactory images similar to those created by users with prompting expertise.



\subsection{Applicability of \tool}
The feedback from user study highlighted several potential applications for our system. 
Three participants (P2, P6, P8) mentioned its possible use in commercial advertising design, emphasizing the importance of controllability for such work. 
They noted that the system's flexibility allows designers to quickly experiment with different settings.
Some participants (N = 3) also mentioned its potential for digital asset creation, particularly for game asset design. 
P7, a game mod developer, found the system highly useful for mod development. 
He explained: "\emph{Mods often require a series of images with a consistent theme and specific spatial requirements. 
For example, in a sacrifice scene, how the objects are arranged is closely tied to the mod's background. It would be difficult for a developer without professional skills, but with this system, it is possible to quickly construct such images}."
A few participants expressed similar thoughts regarding its use in scene construction, such as in film production. 
An interesting suggestion came from participant P4, who proposed its application in crime scene description. 
She pointed out that witnesses are often not skilled artists, and typically describe crime scenes verbally while someone else illustrates their account. 
With this system, witnesses could more easily express what they saw themselves, potentially producing depictions closer to the real events. "\emph{Details like object locations and distances from buildings can be easily conveyed using the system}," she added.

% \subsection{Model Understanding of Users' Implicit Intents}
% In region-sketch-based control of generative models, a significant gap between interaction design and actual implementation is the model's failure in understanding users' naturally expressed intentions.
% For example, when users draw sketches of two people with their hands slightly overlapping, current region-based models cannot automatically infer users' implicit intention that the two people are holding hands.
% Instead, they still require users to explicitly specify in the prompt such relationship.
% \tool addresses this through sketch-aware prompt recommendation to fill in the necessary semantic information, alleviating users' workload.
% However, some users want the generative AI in the future to be able to directly infer this natural implicit intentions from the sketches without additional prompting since prompt recommendation can still be unstable sometimes.
% This problem reflects a more general dilemma, which ubiquitously exists in all forms of conditioned control for generative models such as canny or scribble control.
% This is because all the control models are trained on pairs of explicit control signal and target image, which is lacking further interpretation or customization of the user intentions behind the seemingly straightforward input.
% For another example, the generative models cannot understand what abstraction level the user has in mind for her personal scribbles.
% Such problems leave more challenges to be addressed by future human-AI co-creation research.
% One possible direction is fine-tuning the conditioned models on individual user's conditioned control data to provide more customized interpretation. 

% \subsection{Balance between recommendation and autonomy}
% AIGC tools are a typical example of 
\subsection{Progressive Sketching}
Currently \tool is mainly aimed at novice users who are only capable of creating very rough sketches by themselves.
However, more accomplished painters or even professional artists typically have a coarse-to-fine creative process. 
Such a process is most evident in painting styles like traditional oil painting or digital impasto painting, where artists first quickly lay down large color patches to outline the most primitive proportion and structure of visual elements.
After that, the artists will progressively add layers of finer color strokes to the canvas to gradually refine the painting to an exquisite piece of artwork.
One participant in our user study (P1) , as a professional painter, has mentioned a similar point "\emph{
I think it is useful for laying out the big picture, give some inspirations for the initial drawing stage}."
Therefore, rough sketch also plays a part in the professional artists' creation process, yet it is more challenging to integrate AI into this more complex coarse-to-fine procedure.
Particularly, artists would like to preserve some of their finer strokes in later progression, not just the shape of the initial sketch.
In addition, instead of requiring the tool to generate a finished piece of artwork, some artists may prefer a model that can generate another more accurate sketch based on the initial one, and leave the final coloring and refining to the artists themselves.
To accommodate these diverse progressive sketching requirements, a more advanced sketch-based AI-assisted creation tool should be developed that can seamlessly enable artist intervention at any stage of the sketch and maximally preserve their creative intents to the finest level. 

\subsection{Ethical Issues}
Intellectual property and unethical misuse are two potential ethical concerns of AI-assisted creative tools, particularly those targeting novice users.
In terms of intellectual property, \tool hands over to novice users more control, giving them a higher sense of ownership of the creation.
However, the question still remains: how much contribution from the user's part constitutes full authorship of the artwork?
As \tool still relies on backbone generative models which may be trained on uncopyrighted data largely responsible for turning the sketch into finished artwork, we should design some mechanisms to circumvent this risk.
For example, we can allow artists to upload backbone models trained on their own artworks to integrate with our sketch control.
Regarding unethical misuse, \tool makes fine-grained spatial control more accessible to novice users, who may maliciously generate inappropriate content such as more realistic deepfake with specific postures they want or other explicit content.
To address this issue, we plan to incorporate a more sophisticated filtering mechanism that can detect and screen unethical content with more complex spatial-semantic conditions. 
% In the future, we plan to enable artists to upload their own style model

% \subsection{From interactive prompting to interactive spatial prompting}


\subsection{Limitations and Future work}

    \textbf{User Study Design}. Our open-ended task assesses the usability of \tool's system features in general use cases. To further examine aspects such as creativity and controllability across different methods, the open-ended task could be improved by incorporating baselines to provide more insightful comparative analysis. 
    Besides, in close-ended tasks, while the fixing order of tool usage prevents prior knowledge leakage, it might introduce learning effects. In our study, we include practice sessions for the three systems before the formal task to mitigate these effects. In the future, utilizing parallel tests (\textit{e.g.} different content with the same difficulty) or adding a control group could further reduce the learning effects.

    \textbf{Failure Cases}. There are certain failure cases with \tool that can limit its usability. 
    Firstly, when there are three or more objects with similar semantics, objects may still be missing despite prompt recommendations. 
    Secondly, if an object's stroke is thin, \tool may incorrectly interpret it as a full area, as demonstrated in the expert results of the open-ended task in Figure~\ref{fig:novice_expert}. 
    Finally, sometimes inclusion relationships (\textit{e.g.} inside) between objects cannot be generated correctly, partially due to biases in the base model that lack training samples with such relationship. 

    \textbf{More support for single object adjustment}.
    Participants (N=4) suggested that additional control features should be introduced, beyond just adjusting size and location. They noted that when objects overlap, they cannot freely control which object appears on top or which should be covered, and overlapping areas are currently not allowed.
    They proposed adding features such as layer control and depth control within the single-object mask manipulation. Currently, the system assigns layers based on color order, but future versions should allow users to adjust the layer of each object freely, while considering weighted prompts for overlapping areas.

    \textbf{More customized generation ability}.
    Our current system is built around a single model $ColorfulXL-Lightning$, which limits its ability to fully support the diverse creative needs of users. Feedback from participants has indicated a strong desire for more flexibility in style and personalization, such as integrating fine-tuned models that cater to specific artistic styles or individual preferences. 
    This limitation restricts the ability to adapt to varied creative intents across different users and contexts.
    In future iterations, we plan to address this by embedding a model selection feature, allowing users to choose from a variety of pre-trained or custom fine-tuned models that better align with their stylistic preferences. 
    
    \textbf{Integrate other model functions}.
    Our current system is compatible with many existing tools, such as Promptist~\cite{hao2024optimizing} and Magic Prompt, allowing users to iteratively generate prompts for single objects. However, the integration of these functions is somewhat limited in scope, and users may benefit from a broader range of interactive options, especially for more complex generation tasks. Additionally, for multimodal large models, users can currently explore using affordable or open-source models like Qwen2-VL~\cite{qwen} and InternVL2-Llama3~\cite{llama}, which have demonstrated solid inference performance in our tests. While GPT-4o remains a leading choice, alternative models also offer competitive results.
    Moving forward, we aim to integrate more multimodal large models into the system, giving users the flexibility to choose the models that best fit their needs. 
    


\section{Conclusion}\label{sec:conclusion}
In this paper, we present \tool, an interactive system designed to help novice users create high-quality, fine-grained images that align with their intentions based on rough sketches. 
The system first refines the user's initial prompt into a complete and coherent one that matches the rough sketch, ensuring the generated results are both stable, coherent and high quality.
To further support users in achieving fine-grained alignment between the generated image and their creative intent without requiring professional skills, we introduce a decompose-and-recompose strategy. 
This allows users to select desired, refined object shapes for individual decomposed objects and then recombine them, providing flexible mask manipulation for precise spatial control.
The framework operates through a coarse-to-fine process, enabling iterative and fine-grained control that is not possible with traditional end-to-end generation methods. 
Our user study demonstrates that \tool offers novice users enhanced flexibility in control and fine-grained alignment between their intentions and the generated images.



% \input{data/appendix_B}

% use section* for acknowledgment
\section*{Acknowledgment}

The authors would like to thank anonymous reviewers for their insightful comments and suggestions. This work was supported by the National Science Foundation of Chinagrant(No.62102460 and No.U20A20159), Guangzhou Science and Technology Plan Project (No. 202201011392), Guangdong Basic and Applied Basic Research Foundation (No. 2023A1515012982), Young Outstanding Award under the Zhujiang Talent Plan of Guangdong Province, Guangdong Basic and Applied Basic Research Foundation (No. 2023B1515120058), and Guangzhou Basic and Applied Basic Research Program (No. 2024A04J6367). Xiaoxi Zhang is the corresponding author.


%The authors would like to thank anonymous reviewers for their insightful comments and suggestions. 
% This work was supported by the Key-Area Research and Development Program of Guangdong Province (2021B0101400001), National Science Foundation of China (62102460, U20A20159, 6197243), Guangzhou Science and Technology Plan Project (202201011392, 2024A04J6367), Young Outstanding Award under the Zhujiang Talent Plan of Guangdong Province, Guangdong Basic and Applied Basic Research Foundation (No. 2023A1515012982, 2021B151520008). %, and the Program for Guangdong Introducing Innovative and Entrepreneurial Teams (2017ZT07X355). 

%Xiaoxi Zhang is the corresponding author.




\ifCLASSOPTIONcaptionsoff
  \newpage
\fi


\bibliographystyle{IEEEtran}
\bibliography{ref}
%\appendices
\newpage
% biography section
% 
% If you have an EPS/PDF photo (graphicx package needed) extra braces are
% needed around the contents of the optional argument to biography to prevent
% the LaTeX parser from getting confused when it sees the complicated
% \includegraphics command within an optional argument. (You could create
% your own custom macro containing the \includegraphics command to make things
% simpler here.)
%\begin{IEEEbiography}[{\includegraphics[width=1in,height=1.25in,clip,keepaspectratio]{mshell}}]{Michael Shell}
% or if you just want to reserve a space for a photo:
\begin{IEEEbiography}
[{\includegraphics[width=1in,height=1.25in,clip,keepaspectratio]{figure/fig/broken.jpg}}]{Bokeng Zheng}
is currently pursuing a bachelor’s degree with the School of Computer Science and Engineering, Sun Yat-sen University. His research interests include edge computing and the application of fine-tuning.
\end{IEEEbiography}
\vskip -2\baselineskip plus -1fil
\begin{IEEEbiography}
[{\includegraphics[width=1in,height=1.25in,clip,keepaspectratio]{figure/raobo.jpg}}]{Bo Rao}
received his bachelor’s degree from the School of Electronics and Information, South China University of Technology in 2022. He is currently pursuing a master’s degree with the School of Computer Science and Engineering, Sun Yat-sen University. His research interests include edge computing and the application of reinforcement learning.
\end{IEEEbiography}
\vskip -2\baselineskip plus -1fil
\begin{IEEEbiography}
[{\includegraphics[width=1in,height=1.25in,clip,keepaspectratio]{figure/ztx_3.jpg}}]{Tianxiang Zhu}
received his bachelor’s degree from the College of Software Engineering, Sichuan University in 2021. He is currently pursuing a master’s degree with the School of Computer Science and Engineering, Sun Yat-sen University. His research interests include reinforcement learning and intelligent manufacturing.
\end{IEEEbiography}
%\vskip -2\baselineskip plus -1fil

% \begin{IEEEbiography}[{\includegraphics[width=1in,height=1.25in,clip,keepaspectratio]{figure/youyf.png}}]{Yufei You}
% received her bachelor’s degree from the College of Computer Science and Technology, Hainan University in 2022. She is currently pursuing her master’s degree with the School of Computer Science and Engineering, Sun Yat-sen University. Her research interests include reinforcement learning and intelligent manufacturing.
% \end{IEEEbiography}
% \vskip -2\baselineskip plus -1fil

% \begin{IEEEbiography}[{\includegraphics[width=1in,height=1.25in,clip,keepaspectratio]{figure/Yihong_Li.jpg}}]{Yihong Li}
% received his bachelor’s degree from the School of Information Management, Sun Yat-sen University in 2021. He is currently pursuing a master’s degree with the School of Computer Science and Engineering, Sun Yat-sen University. His research interests include machine learning systems and networking.
% \end{IEEEbiography}
% \vskip -2\baselineskip plus -1fil
\vskip -2\baselineskip plus -1fil
\begin{IEEEbiography}[{\includegraphics[width=1in,height=1.25in,clip,keepaspectratio]{figure/djp.jpg}}]{Jingpu Duan}
 received the B.E. degree from the Huazhong University of Science and Technology, Wuhan, China, in 2013, and the Ph.D. degree from the University of Hong Kong, Hong Kong, China, in 2018. He is currently a Research Assistant Professor with the Institute of Future Networks, Southern University of Science and Technology, Shenzhen, China. He also works with the Department of Communications, Pengcheng Laboratory, Shenzhen, China. His research interest includes designing and implementing high-performance networking systems.
\end{IEEEbiography}
\vskip -2\baselineskip plus -1fil

\begin{IEEEbiography}[{\includegraphics[width=1in,height=1.25in,clip,keepaspectratio]{figure/chee.png}}]{Chee Wei Tan}
received an M.A. and Ph.D. in
Electrical Engineering from Princeton University. He
is an Associate Professor of Computer Science and
Engineering at Nanyang Technological University.
He conducts research in networks, distributed optimization, and generative AI. Dr. Tan has served as IEEE Distinguished Lecturer and Editor for IEEE Transactions on Cognitive Communications and Networking, IEEE/ACM Transactions on Networking,
and IEEE Transactions on Communications. He has
received the Princeton University Wu Prize for Excellence, the Google Faculty Award, and several teaching excellence awards.
He was selected twice for the U.S. National Academy of Engineering ChinaAmerica Frontiers of Engineering Symposium. He is a Co-Chair of the
Cognitive Radio and AI-Enabled Networks Symposium at IEEE GLOBECOM 2025 and a member of ACM Learning at Scale Extended Steering Committee.
\end{IEEEbiography}
\vskip -2\baselineskip plus -1fil
% \begin{IEEEbiography}[{\includegraphics[width=1in,height=1.25in,clip,keepaspectratio]{figure/author-chao.jpeg}}]{Chao Yu}
% received the Ph.D. degree in computer science from the University of Wollongong, Australia, in 2014. He is currently an Associate Professor with the School of Computer Science and Engineering, Sun Yat-sen University, Guangzhou, China. He has published more than 100 articles in prestigious journals, such as IEEE Transactions on Neural Networks and Learning Systems, IEEE Transactions on Cybernetics, and IEEE Transactions on Vehicular Technology. His research interests include multi-agent systems and reinforcement learning.
% \end{IEEEbiography}

% \vskip -2\baselineskip plus -1fil
% \begin{IEEEbiography}[{\includegraphics[width=1in,height=1.25in,clip,keepaspectratio]{figure/dongxiao_yu.jpg}}]{Dongxiao Yu}
% received the BSc degree in 2006 from the School of Mathematics, Shandong University and the PhD degree in 2014 from the Department of Computer Science, The University of Hong Kong. He became an associate professor in the School of Computer Science and Technology, Huazhong University of Science and Technology, in 2016. He is currently a professor in the School of Computer Science and Technology, Shandong University. His research interests include wireless networks, distributed computing and graph algorithms.
% \end{IEEEbiography}
% % \vspace{11pt}
% \vskip -2\baselineskip plus -1fil
% \begin{IEEEbiography}[{\includegraphics[width=1in,height=1.25in,clip,keepaspectratio]{figure/yu_wu.jpg}}]{Yu Wu}
% received the B.S. and M.A. degrees from Tsinghua University in 2006 and 2009, respectively, and the Ph.D. degree from The University of Hong Kong in 2013. He worked as a Post-Doctoral Fellow with Arizona State University from 2013 to 2015. He is currently an Associate Professor with the Dongguan University of Technology and a Researcher with the Peng Cheng Laboratory. His research interests include blockchain, edge computing, and the Internet of Things.
% \end{IEEEbiography}
% % \vspace{11pt}
% \vskip -2\baselineskip plus -1fil
\begin{IEEEbiography}[{\includegraphics[width=1in,height=1.25in,clip,keepaspectratio]{figure/zhi_zhou.jpg}}]{Zhi Zhou} 
 received the B.S., M.E., and Ph.D. degrees in 2012, 2014, and 2017, respectively, all from the School of Computer Science and Technology at Huazhong University of Science and Technology (HUST), Wuhan, China. He is currently an associate professor in the School of Data and Computer Science at Sun Yat-sen University, Guangzhou, China. In 2016, he was a visiting scholar at University of Gottingen. He was nominated for the ¨2019 China Computer Federation CCF Outstanding Doctoral Dissertation Award, the sole recipient of the 2018 ACM Wuhan Hubei Computer Society Doctoral Dissertation Award, and a recipient of the Best Paper Award of IEEE UIC 2018. His research interests include edge computing, cloud computing, and distributed systems.
\end{IEEEbiography}
% \vspace{11pt}
% \vskip -2\baselineskip plus -1fil
% \begin{IEEEbiography}[{\includegraphics[width=1in,height=1.25in,clip,keepaspectratio]{figure/deke_guo.jpg}}]{Deke Guo} 
%  received the B.S. degree in industry engineering from the Beijing University of Aeronautics and Astronautics, Beijing, China, in 2001, and the Ph.D. degree in management science and engineering from the National University of Defense Technology, Changsha, China, in 2008. He is currently a Full Professor with the College of System Engineering, National University of Defense Technology. His research interests include distributed systems, software-defined networking, data center networking, wireless and mobile systems, and interconnection networks. He is a senior member of the IEEE and a member of the ACM.
% \end{IEEEbiography}
 
% \vspace{-20cm}
\vskip -2\baselineskip plus -1fil
\begin{IEEEbiography}[{\includegraphics[width=1in,height=1.25in,clip,keepaspectratio]{figure/Xu.pdf}}]{Xu Chen}
received the Ph.D. degree in information engineering from the Chinese University of Hong Kong in 2012. He is a Full Professor with Sun Yat-sen University, Guangzhou, China, Director of Institute of Advanced Networking and Computing Systems, and the Vice Director of the National and Local Joint Engineering Laboratory of Digital Home. He was a Post-Doctoral Research Associate with Arizona State University, Tempe, USA, from 2012 to 2014, and a Humboldt Scholar Fellow with the Institute of Computer Science, University of Goettingen, Germany, from 2014 to 2016. He was a recipient of the Prestigious Humboldt Research Fellowship awarded by Alexander von Humboldt Foundation of Germany, the 2014 Hong Kong Young Scientist Runner-Up Award, the 2020 IEEE Computer Society Best Paper Awards Runner-Up, the 2017 IEEE Communication Society Asia–Pacific Outstanding Young Researcher Award, the 2017 IEEE ComSoc Young Professional Best Paper Award, the Honorable Mention Award of 2010 IEEE international conference on Intelligence and Security Informatics, the Best Paper Runner-Up Award of 2014 IEEE International Conference on Computer Communications (INFOCOM), and the Best Paper Award of 2017 IEEE International Conference on Communications. He is currently an Area Editor of IEEE Open Journal of the Communications Society, an Associate Editor of the IEEE Transactions Wireless Communications, IEEE Internet of Things Journal, IEEE Transactions on Vehicular Technology, and IEEE Journal on Selected Areas in Communications (JSAC) Series on Network Softwarization and Enablers.
\end{IEEEbiography}
%\vfill

% You can push biographies down or up by placing
% a \vfill before or after them. The appropriate
% use of \vfill depends on what kind of text is
% on the last page and whether or not the columns
% are being equalized.

\vskip -2\baselineskip plus -1fil
\begin{IEEEbiography}[{\includegraphics[width=1in,height=1.25in,clip,keepaspectratio]{figure/zxx.jpg}}]{Xiaoxi Zhang} received the B.E. degree in electronics and information engineering from the Huazhong University of Science and Technology in 2013 and the Ph.D. degree in computer science from The University of Hong Kong in 2017. She was a Post-Doctoral Researcher with the Department of Electrical and Computer Engineering, Carnegie Mellon University, during 2017-2020. She is currently an Associate Professor with the School of Computer Science and Engineering, Sun Yat-sen University. She was a recipient of the Young Outstanding Award of the Guangdong Province, Best Student Paper Award of IEEE MSN 2024, Best Paper Award of IEEE BigCom 2024, and Best Paper Nominee of IEEE/ACM IWQoS 2023. She is currently an Area Editor of Elsevier Computer Networks, an Associate editor of IEEE Networking Letters, and a Guest editor of MDPI Symmetry in Optimization Theory and Its Applications. She is broadly interested in optimization, algorithm design, and system implementation for networked systems, including cloud and edge computing networks, distributed machine learning systems, and vehicular networks.
\end{IEEEbiography}
\vskip -2\baselineskip plus -1fil
%\input{data/appendix_A}

% \vskip -2.5\baselineskip plus -1fil
\begin{IEEEbiographynophoto}{Krithika Iyer} is a computing PhD candidate at the Kahlert School of Computing and 
the Scientific Computing and Imaging Institute at the University of Utah in the Image Analysis track. She received her B.E. in Electronics \& Telecommunication in 2015. Her research interests include medical image analysis, probabilistic modeling, deep learning, and statistical shape modeling.\end{IEEEbiographynophoto}
\vskip -2.5\baselineskip plus -1fil
\begin{IEEEbiographynophoto}{Mokshagna Sai Teja Karanam} is a computing PhD student at the Kahlert School of Computing and the Scientific Computing and Imaging Institute at the University of Utah in the Image Analysis track. He received his B.E. in Computer Science in 2020 and his Masters in 2024. His research interests include deep learning, computer vision and statistical shape modeling.\end{IEEEbiographynophoto}
\vskip -2.5\baselineskip plus -1fil
\begin{IEEEbiographynophoto}{Shireen Elhabian} is a faculty member at the Kahlert School of Computing and the Scientific Computing and Imaging Institute at the University of Utah. Her research focuses on enhancing diagnostic accuracy through deep learning, probabilistic modeling, and advanced computer vision algorithms. She has published over 100 peer-reviewed publications in prestigious journals and conferences, including IEEE-TMI, MedIA, ICLR, CVPR, ICCV, MICCAI, IPMI, AAAI. \end{IEEEbiographynophoto}
\vskip -3\baselineskip plus -1fil


\end{document}


