% biography section
% 
% If you have an EPS/PDF photo (graphicx package needed) extra braces are
% needed around the contents of the optional argument to biography to prevent
% the LaTeX parser from getting confused when it sees the complicated
% \includegraphics command within an optional argument. (You could create
% your own custom macro containing the \includegraphics command to make things
% simpler here.)
%\begin{IEEEbiography}[{\includegraphics[width=1in,height=1.25in,clip,keepaspectratio]{mshell}}]{Michael Shell}
% or if you just want to reserve a space for a photo:
\begin{IEEEbiography}
[{\includegraphics[width=1in,height=1.25in,clip,keepaspectratio]{figure/fig/broken.jpg}}]{Bokeng Zheng}
is currently pursuing a bachelor’s degree with the School of Computer Science and Engineering, Sun Yat-sen University. His research interests include edge computing and the application of fine-tuning.
\end{IEEEbiography}
\vskip -2\baselineskip plus -1fil
\begin{IEEEbiography}
[{\includegraphics[width=1in,height=1.25in,clip,keepaspectratio]{figure/raobo.jpg}}]{Bo Rao}
received his bachelor’s degree from the School of Electronics and Information, South China University of Technology in 2022. He is currently pursuing a master’s degree with the School of Computer Science and Engineering, Sun Yat-sen University. His research interests include edge computing and the application of reinforcement learning.
\end{IEEEbiography}
\vskip -2\baselineskip plus -1fil
\begin{IEEEbiography}
[{\includegraphics[width=1in,height=1.25in,clip,keepaspectratio]{figure/ztx_3.jpg}}]{Tianxiang Zhu}
received his bachelor’s degree from the College of Software Engineering, Sichuan University in 2021. He is currently pursuing a master’s degree with the School of Computer Science and Engineering, Sun Yat-sen University. His research interests include reinforcement learning and intelligent manufacturing.
\end{IEEEbiography}
%\vskip -2\baselineskip plus -1fil

% \begin{IEEEbiography}[{\includegraphics[width=1in,height=1.25in,clip,keepaspectratio]{figure/youyf.png}}]{Yufei You}
% received her bachelor’s degree from the College of Computer Science and Technology, Hainan University in 2022. She is currently pursuing her master’s degree with the School of Computer Science and Engineering, Sun Yat-sen University. Her research interests include reinforcement learning and intelligent manufacturing.
% \end{IEEEbiography}
% \vskip -2\baselineskip plus -1fil

% \begin{IEEEbiography}[{\includegraphics[width=1in,height=1.25in,clip,keepaspectratio]{figure/Yihong_Li.jpg}}]{Yihong Li}
% received his bachelor’s degree from the School of Information Management, Sun Yat-sen University in 2021. He is currently pursuing a master’s degree with the School of Computer Science and Engineering, Sun Yat-sen University. His research interests include machine learning systems and networking.
% \end{IEEEbiography}
% \vskip -2\baselineskip plus -1fil
\vskip -2\baselineskip plus -1fil
\begin{IEEEbiography}[{\includegraphics[width=1in,height=1.25in,clip,keepaspectratio]{figure/djp.jpg}}]{Jingpu Duan}
 received the B.E. degree from the Huazhong University of Science and Technology, Wuhan, China, in 2013, and the Ph.D. degree from the University of Hong Kong, Hong Kong, China, in 2018. He is currently a Research Assistant Professor with the Institute of Future Networks, Southern University of Science and Technology, Shenzhen, China. He also works with the Department of Communications, Pengcheng Laboratory, Shenzhen, China. His research interest includes designing and implementing high-performance networking systems.
\end{IEEEbiography}
\vskip -2\baselineskip plus -1fil

\begin{IEEEbiography}[{\includegraphics[width=1in,height=1.25in,clip,keepaspectratio]{figure/chee.png}}]{Chee Wei Tan}
received an M.A. and Ph.D. in
Electrical Engineering from Princeton University. He
is an Associate Professor of Computer Science and
Engineering at Nanyang Technological University.
He conducts research in networks, distributed optimization, and generative AI. Dr. Tan has served as IEEE Distinguished Lecturer and Editor for IEEE Transactions on Cognitive Communications and Networking, IEEE/ACM Transactions on Networking,
and IEEE Transactions on Communications. He has
received the Princeton University Wu Prize for Excellence, the Google Faculty Award, and several teaching excellence awards.
He was selected twice for the U.S. National Academy of Engineering ChinaAmerica Frontiers of Engineering Symposium. He is a Co-Chair of the
Cognitive Radio and AI-Enabled Networks Symposium at IEEE GLOBECOM 2025 and a member of ACM Learning at Scale Extended Steering Committee.
\end{IEEEbiography}
\vskip -2\baselineskip plus -1fil
% \begin{IEEEbiography}[{\includegraphics[width=1in,height=1.25in,clip,keepaspectratio]{figure/author-chao.jpeg}}]{Chao Yu}
% received the Ph.D. degree in computer science from the University of Wollongong, Australia, in 2014. He is currently an Associate Professor with the School of Computer Science and Engineering, Sun Yat-sen University, Guangzhou, China. He has published more than 100 articles in prestigious journals, such as IEEE Transactions on Neural Networks and Learning Systems, IEEE Transactions on Cybernetics, and IEEE Transactions on Vehicular Technology. His research interests include multi-agent systems and reinforcement learning.
% \end{IEEEbiography}

% \vskip -2\baselineskip plus -1fil
% \begin{IEEEbiography}[{\includegraphics[width=1in,height=1.25in,clip,keepaspectratio]{figure/dongxiao_yu.jpg}}]{Dongxiao Yu}
% received the BSc degree in 2006 from the School of Mathematics, Shandong University and the PhD degree in 2014 from the Department of Computer Science, The University of Hong Kong. He became an associate professor in the School of Computer Science and Technology, Huazhong University of Science and Technology, in 2016. He is currently a professor in the School of Computer Science and Technology, Shandong University. His research interests include wireless networks, distributed computing and graph algorithms.
% \end{IEEEbiography}
% % \vspace{11pt}
% \vskip -2\baselineskip plus -1fil
% \begin{IEEEbiography}[{\includegraphics[width=1in,height=1.25in,clip,keepaspectratio]{figure/yu_wu.jpg}}]{Yu Wu}
% received the B.S. and M.A. degrees from Tsinghua University in 2006 and 2009, respectively, and the Ph.D. degree from The University of Hong Kong in 2013. He worked as a Post-Doctoral Fellow with Arizona State University from 2013 to 2015. He is currently an Associate Professor with the Dongguan University of Technology and a Researcher with the Peng Cheng Laboratory. His research interests include blockchain, edge computing, and the Internet of Things.
% \end{IEEEbiography}
% % \vspace{11pt}
% \vskip -2\baselineskip plus -1fil
\begin{IEEEbiography}[{\includegraphics[width=1in,height=1.25in,clip,keepaspectratio]{figure/zhi_zhou.jpg}}]{Zhi Zhou} 
 received the B.S., M.E., and Ph.D. degrees in 2012, 2014, and 2017, respectively, all from the School of Computer Science and Technology at Huazhong University of Science and Technology (HUST), Wuhan, China. He is currently an associate professor in the School of Data and Computer Science at Sun Yat-sen University, Guangzhou, China. In 2016, he was a visiting scholar at University of Gottingen. He was nominated for the ¨2019 China Computer Federation CCF Outstanding Doctoral Dissertation Award, the sole recipient of the 2018 ACM Wuhan Hubei Computer Society Doctoral Dissertation Award, and a recipient of the Best Paper Award of IEEE UIC 2018. His research interests include edge computing, cloud computing, and distributed systems.
\end{IEEEbiography}
% \vspace{11pt}
% \vskip -2\baselineskip plus -1fil
% \begin{IEEEbiography}[{\includegraphics[width=1in,height=1.25in,clip,keepaspectratio]{figure/deke_guo.jpg}}]{Deke Guo} 
%  received the B.S. degree in industry engineering from the Beijing University of Aeronautics and Astronautics, Beijing, China, in 2001, and the Ph.D. degree in management science and engineering from the National University of Defense Technology, Changsha, China, in 2008. He is currently a Full Professor with the College of System Engineering, National University of Defense Technology. His research interests include distributed systems, software-defined networking, data center networking, wireless and mobile systems, and interconnection networks. He is a senior member of the IEEE and a member of the ACM.
% \end{IEEEbiography}
 
% \vspace{-20cm}
\vskip -2\baselineskip plus -1fil
\begin{IEEEbiography}[{\includegraphics[width=1in,height=1.25in,clip,keepaspectratio]{figure/Xu.pdf}}]{Xu Chen}
received the Ph.D. degree in information engineering from the Chinese University of Hong Kong in 2012. He is a Full Professor with Sun Yat-sen University, Guangzhou, China, Director of Institute of Advanced Networking and Computing Systems, and the Vice Director of the National and Local Joint Engineering Laboratory of Digital Home. He was a Post-Doctoral Research Associate with Arizona State University, Tempe, USA, from 2012 to 2014, and a Humboldt Scholar Fellow with the Institute of Computer Science, University of Goettingen, Germany, from 2014 to 2016. He was a recipient of the Prestigious Humboldt Research Fellowship awarded by Alexander von Humboldt Foundation of Germany, the 2014 Hong Kong Young Scientist Runner-Up Award, the 2020 IEEE Computer Society Best Paper Awards Runner-Up, the 2017 IEEE Communication Society Asia–Pacific Outstanding Young Researcher Award, the 2017 IEEE ComSoc Young Professional Best Paper Award, the Honorable Mention Award of 2010 IEEE international conference on Intelligence and Security Informatics, the Best Paper Runner-Up Award of 2014 IEEE International Conference on Computer Communications (INFOCOM), and the Best Paper Award of 2017 IEEE International Conference on Communications. He is currently an Area Editor of IEEE Open Journal of the Communications Society, an Associate Editor of the IEEE Transactions Wireless Communications, IEEE Internet of Things Journal, IEEE Transactions on Vehicular Technology, and IEEE Journal on Selected Areas in Communications (JSAC) Series on Network Softwarization and Enablers.
\end{IEEEbiography}
%\vfill

% You can push biographies down or up by placing
% a \vfill before or after them. The appropriate
% use of \vfill depends on what kind of text is
% on the last page and whether or not the columns
% are being equalized.

\vskip -2\baselineskip plus -1fil
\begin{IEEEbiography}[{\includegraphics[width=1in,height=1.25in,clip,keepaspectratio]{figure/zxx.jpg}}]{Xiaoxi Zhang} received the B.E. degree in electronics and information engineering from the Huazhong University of Science and Technology in 2013 and the Ph.D. degree in computer science from The University of Hong Kong in 2017. She was a Post-Doctoral Researcher with the Department of Electrical and Computer Engineering, Carnegie Mellon University, during 2017-2020. She is currently an Associate Professor with the School of Computer Science and Engineering, Sun Yat-sen University. She was a recipient of the Young Outstanding Award of the Guangdong Province, Best Student Paper Award of IEEE MSN 2024, Best Paper Award of IEEE BigCom 2024, and Best Paper Nominee of IEEE/ACM IWQoS 2023. She is currently an Area Editor of Elsevier Computer Networks, an Associate editor of IEEE Networking Letters, and a Guest editor of MDPI Symmetry in Optimization Theory and Its Applications. She is broadly interested in optimization, algorithm design, and system implementation for networked systems, including cloud and edge computing networks, distributed machine learning systems, and vehicular networks.
\end{IEEEbiography}
\vskip -2\baselineskip plus -1fil