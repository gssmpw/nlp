\section{Related Work}
Since \arrival\ is contained in $\NP \cap \CoNP$, it also naturally fits into the complexity class $\TFNP$, which contains total problems with efficiently verifiable solutions. In fact, after a series of results for containment in subclasses of $\TFNP$~\cite{karthikc.s.DidTrainReach2017, gartnerARRIVALNextStop2018}, we now know that \arrival\ is contained in $\UEOPL$~\cite{fearnleyUniqueEndPotential2020}.

Going into a slightly different direction, Gärtner et al.\@~\cite{gartnerARRIVALNextStop2018} proved that \arrival\ is also contained in $\UP$ and $\CoUP$, the analogues of $\NP$ and $\CoNP$ with unique solutions. 
In fact, it would not be hard to rederive this using our reduction to $\ell_1$-contraction: The idea is that the fixed point acts as an efficient certificate for both YES- and NO-instances and it must be unique due to the contraction property.

It is also known that \arrival\ can be solved in polynomial time on some restricted graph classes. For example, a result due to Priezzhev et al.\@~\cite{priezzhevEulerianWalkersModel1996} implies that \arrival\ can be solved by simulation in polynomial time on Eulerian graphs. Other results include polynomial-time algorithms on tree-like multigraphs~\cite{augerPolynomialTimeAlgorithm2022} and path-like multigraphs with many tokens~\cite{augerGeneralizedARRIVALProblem2023}. 

Finally, further variants of \arrival\ have been studied in the past, including a stochastic variant~\cite{websterStochasticArrivalProblem2022}, a recursive variant~\cite{websterRecursiveArrivalProblem2023}, as well as variants with one or two players~\cite{fearnleyReachabilitySwitchingGames2021}.

\paragraph*{Comparison to \ssg\ }

As mentioned before, both our results show parallels between \arrival\ and Simple Stochastic Games (\ssg). We will briefly discuss the connections between the two problems. 
Similarly to \arrival, \ssg\ is contained in $\NP \cap \CoNP$~\cite{condonComplexityStochasticGames1992} and even $\UP \cap \CoUP$~\cite{chatterjeeReductionParityGames2011a}, but no polynomial-time algorithm is known. The state-of-the-art algorithm due to Ludwig~\cite{ludwigSubexponentialRandomizedAlgorithm1995} runs in randomized subexponential time $2^{\bigO(\sqrt{n}\log n)}$.

Dohrau et al.\@~\cite{dohrauARRIVALZeroPlayerGraph2017} already wondered about similarities between \arrival\ and \ssg\ when they first introduced \arrival. Since then, both problems were shown to reduce to the problem of finding a Tarski fixed point~\cite{etessamiTarskiTheoremSupermodular2020, gartnerSubexponentialAlgorithmARRIVAL2021}, and to be contained in the complexity class $\UEOPL$~\cite{fearnleyUniqueEndPotential2020}. Both our results for \arrival\ further extend this list of similarities: Our upper bound of $2^{\bigO(k \log^2 n)}$ for \arrival\ on $n$-vertex graphs of treewidth $k$ is comparable to a similar bound for \ssg\ due to Chatterjee et al.\@~\cite{chatterjeeFasterAlgorithmTurnbased2023}. Moreover, \ssg\ reduces to finding an approximate fixed point of an $\ell_\infty$-contracting function~\cite{condonComplexityStochasticGames1992}, which is analogous to our reduction from \arrival\ to $\ell_1$-contraction. 

Both \arrival\ and \ssg\ also admit polynomial-time algorithms on graphs with a bounded feedback vertex set~\cite{gartnerSubexponentialAlgorithmARRIVAL2021, augerFindingOptimalStrategies2014}. In fact, in the case of \arrival, we will discuss this in more detail in Section~\ref{ssec:subexponential_algo}.