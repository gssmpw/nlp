%%%%%%%% ICML 2025 EXAMPLE LATEX SUBMISSION FILE %%%%%%%%%%%%%%%%%

\documentclass{article}

% Recommended, but optional, packages for figures and better typesetting:
\usepackage{microtype}
\usepackage{graphicx}
\usepackage{subfigure}
\usepackage{booktabs} % for professional tables
\usepackage[T1]{fontenc}

% hyperref makes hyperlinks in the resulting PDF.
% If your build breaks (sometimes temporarily if a hyperlink spans a page)
% please comment out the following usepackage line and replace
% \usepackage{icml2025} with \usepackage[nohyperref]{icml2025} above.
\usepackage{hyperref}


% Attempt to make hyperref and algorithmic work together better:
\newcommand{\theHalgorithm}{\arabic{algorithm}}

% Use the following line for the initial blind version submitted for review:

% \usepackage{icml2025}

% If accepted, instead use the following line for the camera-ready submission:
\usepackage[accepted]{icml2025}

% For theorems and such
\usepackage{amsmath}
\usepackage{amssymb}
\usepackage{mathtools}
\usepackage{amsthm}

% if you use cleveref..
\usepackage[capitalize,noabbrev]{cleveref}

%%%%%%%%%%%%%%%%%%%%%%%%%%%%%%%%
% THEOREMS
%%%%%%%%%%%%%%%%%%%%%%%%%%%%%%%%
\theoremstyle{plain}
\newtheorem{theorem}{Theorem}[section]
\newtheorem{proposition}[theorem]{Proposition}
\newtheorem{lemma}[theorem]{Lemma}
\newtheorem{corollary}[theorem]{Corollary}
\theoremstyle{definition}
\newtheorem{definition}[theorem]{Definition}
\newtheorem{assumption}[theorem]{Assumption}
\theoremstyle{remark}
\newtheorem{remark}[theorem]{Remark}

% Todonotes is useful during development; simply uncomment the next line
%    and comment out the line below the next line to turn off comments
%\usepackage[disable,textsize=tiny]{todonotes}
\usepackage[textsize=tiny]{todonotes}

% The \icmltitle you define below is probably too long as a header.
% Therefore, a short form for the running title is supplied here:
\icmltitlerunning{Twilight: Adaptive Attention Sparsity with Hierarchical Top-$p$ Pruning}

% Some additional packages
\usepackage{graphicx}
\usepackage{multirow}

% Collborators.
\newcommand{\cf}[1]{{\color{black} #1}}
\newcommand{\mingyu}[1]{\textcolor{purple}{[]}}
\newcommand{\jiaming}[1]{{\color{black} #1}}

\begin{document}

\twocolumn[
\icmltitle{Twilight: Adaptive Attention Sparsity with Hierarchical Top-$p$ Pruning}

% It is OKAY to include author information, even for blind
% submissions: the style file will automatically remove it for you
% unless you've provided the [accepted] option to the icml2025
% package.

% List of affiliations: The first argument should be a (short)
% identifier you will use later to specify author affiliations
% Academic affiliations should list Department, University, City, Region, Country
% Industry affiliations should list Company, City, Region, Country

% You can specify symbols, otherwise they are numbered in order.
% Ideally, you should not use this facility. Affiliations will be numbered
% in order of appearance and this is the preferred way.
% \icmlsetsymbol{equal}{*}

\begin{icmlauthorlist}
\icmlauthor{Chaofan Lin}{thu}
\icmlauthor{Jiaming Tang}{mit}
\icmlauthor{Shuo Yang}{ucb}
\icmlauthor{Hanshuo Wang}{sqz}
\icmlauthor{Tian Tang}{thu}
\icmlauthor{Boyu Tian}{thu}
\icmlauthor{Ion Stoica}{ucb}
%\icmlauthor{}{sch}
\icmlauthor{Song Han}{mit}
\icmlauthor{Mingyu Gao}{thu}
%\icmlauthor{}{sch}
%\icmlauthor{}{sch}
\end{icmlauthorlist}

\icmlaffiliation{thu}{Tsinghua
University}
\icmlaffiliation{mit}{MIT}
\icmlaffiliation{ucb}{UC Berkerly}
\icmlaffiliation{sqz}{Shanghai Qi Zhi Institute}
% \icmlcorrespondingauthor{Ion Stocia}{songhan@mit.edu}
% \icmlcorrespondingauthor{Song Han}{songhan@mit.edu}
\icmlcorrespondingauthor{Mingyu Gao}{gaomy@tsinghua.edu.cn}
% \icmlcorrespondingauthor{Firstname2 Lastname2}{first2.last2@www.uk}

% You may provide any keywords that you
% find helpful for describing your paper; these are used to populate
% the "keywords" metadata in the PDF but will not be shown in the document
\icmlkeywords{Large Language Model, Sparse Attention, Decode, KV Cache}

\vskip 0.3in
]

% this must go after the closing bracket ] following \twocolumn[ ...

% This command actually creates the footnote in the first column
% listing the affiliations and the copyright notice.
% The command takes one argument, which is text to display at the start of the footnote.
% The \icmlEqualContribution command is standard text for equal contribution.
% Remove it (just {}) if you do not need this facility.

\printAffiliationsAndNotice{}  % leave blank if no need to mention equal contribution
% \printAffiliationsAndNotice{\icmlEqualContribution} % otherwise use the standard text.

Large language model (LLM)-based agents have shown promise in tackling complex tasks by interacting dynamically with the environment. 
Existing work primarily focuses on behavior cloning from expert demonstrations and preference learning through exploratory trajectory sampling. However, these methods often struggle in long-horizon tasks, where suboptimal actions accumulate step by step, causing agents to deviate from correct task trajectories.
To address this, we highlight the importance of \textit{timely calibration} and the need to automatically construct calibration trajectories for training agents. We propose \textbf{S}tep-Level \textbf{T}raj\textbf{e}ctory \textbf{Ca}libration (\textbf{\model}), a novel framework for LLM agent learning. 
Specifically, \model identifies suboptimal actions through a step-level reward comparison during exploration. It constructs calibrated trajectories using LLM-driven reflection, enabling agents to learn from improved decision-making processes. These calibrated trajectories, together with successful trajectory data, are utilized for reinforced training.
Extensive experiments demonstrate that \model significantly outperforms existing methods. Further analysis highlights that step-level calibration enables agents to complete tasks with greater robustness. 
Our code and data are available at \url{https://github.com/WangHanLinHenry/STeCa}.
%!TEX root = gcn.tex
\section{Introduction}
Graphs, representing structural data and topology, are widely used across various domains, such as social networks and merchandising transactions.
Graph convolutional networks (GCN)~\cite{iclr/KipfW17} have significantly enhanced model training on these interconnected nodes.
However, these graphs often contain sensitive information that should not be leaked to untrusted parties.
For example, companies may analyze sensitive demographic and behavioral data about users for applications ranging from targeted advertising to personalized medicine.
Given the data-centric nature and analytical power of GCN training, addressing these privacy concerns is imperative.

Secure multi-party computation (MPC)~\cite{crypto/ChaumDG87,crypto/ChenC06,eurocrypt/CiampiRSW22} is a critical tool for privacy-preserving machine learning, enabling mutually distrustful parties to collaboratively train models with privacy protection over inputs and (intermediate) computations.
While research advances (\eg,~\cite{ccs/RatheeRKCGRS20,uss/NgC21,sp21/TanKTW,uss/WatsonWP22,icml/Keller022,ccs/ABY318,folkerts2023redsec}) support secure training on convolutional neural networks (CNNs) efficiently, private GCN training with MPC over graphs remains challenging.

Graph convolutional layers in GCNs involve multiplications with a (normalized) adjacency matrix containing $\numedge$ non-zero values in a $\numnode \times \numnode$ matrix for a graph with $\numnode$ nodes and $\numedge$ edges.
The graphs are typically sparse but large.
One could use the standard Beaver-triple-based protocol to securely perform these sparse matrix multiplications by treating graph convolution as ordinary dense matrix multiplication.
However, this approach incurs $O(\numnode^2)$ communication and memory costs due to computations on irrelevant nodes.
%
Integrating existing cryptographic advances, the initial effort of SecGNN~\cite{tsc/WangZJ23,nips/RanXLWQW23} requires heavy communication or computational overhead.
Recently, CoGNN~\cite{ccs/ZouLSLXX24} optimizes the overhead in terms of  horizontal data partitioning, proposing a semi-honest secure framework.
Research for secure GCN over vertical data  remains nascent.

Current MPC studies, for GCN or not, have primarily targeted settings where participants own different data samples, \ie, horizontally partitioned data~\cite{ccs/ZouLSLXX24}.
MPC specialized for scenarios where parties hold different types of features~\cite{tkde/LiuKZPHYOZY24,icml/CastigliaZ0KBP23,nips/Wang0ZLWL23} is rare.
This paper studies $2$-party secure GCN training for these vertical partition cases, where one party holds private graph topology (\eg, edges) while the other owns private node features.
For instance, LinkedIn holds private social relationships between users, while banks own users' private bank statements.
Such real-world graph structures underpin the relevance of our focus.
To our knowledge, no prior work tackles secure GCN training in this context, which is crucial for cross-silo collaboration.


To realize secure GCN over vertically split data, we tailor MPC protocols for sparse graph convolution, which fundamentally involves sparse (adjacency) matrix multiplication.
Recent studies have begun exploring MPC protocols for sparse matrix multiplication (SMM).
ROOM~\cite{ccs/SchoppmannG0P19}, a seminal work on SMM, requires foreknowledge of sparsity types: whether the input matrices are row-sparse or column-sparse.
Unfortunately, GCN typically trains on graphs with arbitrary sparsity, where nodes have varying degrees and no specific sparsity constraints.
Moreover, the adjacency matrix in GCN often contains a self-loop operation represented by adding the identity matrix, which is neither row- nor column-sparse.
Araki~\etal~\cite{ccs/Araki0OPRT21} avoid this limitation in their scalable, secure graph analysis work, yet it does not cover vertical partition.

% and related primitives
To bridge this gap, we propose a secure sparse matrix multiplication protocol, \osmm, achieving \emph{accurate, efficient, and secure GCN training over vertical data} for the first time.

\subsection{New Techniques for Sparse Matrices}
The cost of evaluating a GCN layer is dominated by SMM in the form of $\adjmat\feamat$, where $\adjmat$ is a sparse adjacency matrix of a (directed) graph $\graph$ and $\feamat$ is a dense matrix of node features.
For unrelated nodes, which often constitute a substantial portion, the element-wise products $0\cdot x$ are always zero.
Our efficient MPC design 
avoids unnecessary secure computation over unrelated nodes by focusing on computing non-zero results while concealing the sparse topology.
We achieve this~by:
1) decomposing the sparse matrix $\adjmat$ into a product of matrices (\S\ref{sec::sgc}), including permutation and binary diagonal matrices, that can \emph{faithfully} represent the original graph topology;
2) devising specialized protocols (\S\ref{sec::smm_protocol}) for efficiently multiplying the structured matrices while hiding sparsity topology.


 
\subsubsection{Sparse Matrix Decomposition}
We decompose adjacency matrix $\adjmat$ of $\graph$ into two bipartite graphs: one represented by sparse matrix $\adjout$, linking the out-degree nodes to edges, the other 
by sparse matrix $\adjin$,
linking edges to in-degree nodes.

%\ie, we decompose $\adjmat$ into $\adjout \adjin$, where $\adjout$ and $\adjin$ are sparse matrices representing these connections.
%linking out-degree nodes to edges and edges to in-degree nodes of $\graph$, respectively.

We then permute the columns of $\adjout$ and the rows of $\adjin$ so that the permuted matrices $\adjout'$ and $\adjin'$ have non-zero positions with \emph{monotonically non-decreasing} row and column indices.
A permutation $\sigma$ is used to preserve the edge topology, leading to an initial decomposition of $\adjmat = \adjout'\sigma \adjin'$.
This is further refined into a sequence of \emph{linear transformations}, 
which can be efficiently computed by our MPC protocols for 
\emph{oblivious permutation}
%($\Pi_{\ssp}$) 
and \emph{oblivious selection-multiplication}.
% ($\Pi_\SM$)
\iffalse
Our approach leverages bipartite graph representation and the monotonicity of non-zero positions to decompose a general sparse matrix into linear transformations, enhancing the efficiency of our MPC protocols.
\fi
Our decomposition approach is not limited to GCNs but also general~SMM 
by 
%simply 
treating them 
as adjacency matrices.
%of a graph.
%Since any sparse matrix can be viewed 

%allowing the same technique to be applied.

 
\subsubsection{New Protocols for Linear Transformations}
\emph{Oblivious permutation} (OP) is a two-party protocol taking a private permutation $\sigma$ and a private vector $\xvec$ from the two parties, respectively, and generating a secret share $\l\sigma \xvec\r$ between them.
Our OP protocol employs correlated randomnesses generated in an input-independent offline phase to mask $\sigma$ and $\xvec$ for secure computations on intermediate results, requiring only $1$ round in the online phase (\cf, $\ge 2$ in previous works~\cite{ccs/AsharovHIKNPTT22, ccs/Araki0OPRT21}).

Another crucial two-party protocol in our work is \emph{oblivious selection-multiplication} (OSM).
It takes a private bit~$s$ from a party and secret share $\l x\r$ of an arithmetic number~$x$ owned by the two parties as input and generates secret share $\l sx\r$.
%between them.
%Like our OP protocol, o
Our $1$-round OSM protocol also uses pre-computed randomnesses to mask $s$ and $x$.
%for secure computations.
Compared to the Beaver-triple-based~\cite{crypto/Beaver91a} and oblivious-transfer (OT)-based approaches~\cite{pkc/Tzeng02}, our protocol saves ${\sim}50\%$ of online communication while having the same offline communication and round complexities.

By decomposing the sparse matrix into linear transformations and applying our specialized protocols, our \osmm protocol
%($\prosmm$) 
reduces the complexity of evaluating $\numnode \times \numnode$ sparse matrices with $\numedge$ non-zero values from $O(\numnode^2)$ to $O(\numedge)$.

%(\S\ref{sec::secgcn})
\subsection{\cgnn: Secure GCN made Efficient}
Supported by our new sparsity techniques, we build \cgnn, 
a two-party computation (2PC) framework for GCN inference and training over vertical
%ly split
data.
Our contributions include:

1) We are the first to explore sparsity over vertically split, secret-shared data in MPC, enabling decompositions of sparse matrices with arbitrary sparsity and isolating computations that can be performed in plaintext without sacrificing privacy.

2) We propose two efficient $2$PC primitives for OP and OSM, both optimally single-round.
Combined with our sparse matrix decomposition approach, our \osmm protocol ($\prosmm$) achieves constant-round communication costs of $O(\numedge)$, reducing memory requirements and avoiding out-of-memory errors for large matrices.
In practice, it saves $99\%+$ communication
%(Table~\ref{table:comm_smm}) 
and reduces ${\sim}72\%$ memory usage over large $(5000\times5000)$ matrices compared with using Beaver triples.
%(Table~\ref{table:mem_smm_sparse}) ${\sim}16\%$-

3) We build an end-to-end secure GCN framework for inference and training over vertically split data, maintaining accuracy on par with plaintext computations.
We will open-source our evaluation code for research and deployment.

To evaluate the performance of $\cgnn$, we conducted extensive experiments over three standard graph datasets (Cora~\cite{aim/SenNBGGE08}, Citeseer~\cite{dl/GilesBL98}, and Pubmed~\cite{ijcnlp/DernoncourtL17}),
reporting communication, memory usage, accuracy, and running time under varying network conditions, along with an ablation study with or without \osmm.
Below, we highlight our key achievements.

\textit{Communication (\S\ref{sec::comm_compare_gcn}).}
$\cgnn$ saves communication by $50$-$80\%$.
(\cf,~CoGNN~\cite{ccs/KotiKPG24}, OblivGNN~\cite{uss/XuL0AYY24}).

\textit{Memory usage (\S\ref{sec::smmmemory}).}
\cgnn alleviates out-of-memory problems of using %the standard 
Beaver-triples~\cite{crypto/Beaver91a} for large datasets.

\textit{Accuracy (\S\ref{sec::acc_compare_gcn}).}
$\cgnn$ achieves inference and training accuracy comparable to plaintext counterparts.
%training accuracy $\{76\%$, $65.1\%$, $75.2\%\}$ comparable to $\{75.7\%$, $65.4\%$, $74.5\%\}$ in plaintext.

{\textit{Computational efficiency (\S\ref{sec::time_net}).}} 
%If the network is worse in bandwidth and better in latency, $\cgnn$ shows more benefits.
$\cgnn$ is faster by $6$-$45\%$ in inference and $28$-$95\%$ in training across various networks and excels in narrow-bandwidth and low-latency~ones.

{\textit{Impact of \osmm (\S\ref{sec:ablation}).}}
Our \osmm protocol shows a $10$-$42\times$ speed-up for $5000\times 5000$ matrices and saves $10$-2$1\%$ memory for ``small'' datasets and up to $90\%$+ for larger ones.

\section{Related Work}
\label{related}
\textbf{Copyright infringement in text-to-image models.}
Recent research \cite{carlini2023extracting, somepalli2023diffusion, somepalli2023understanding, gu2023memorization, wang2024replication, wen2024detecting, chiba2024probabilistic} highlights the potential for copyright infringement in text-to-image models. These models are trained on vast datasets that often include copyrighted material, which could inadvertently be memorized by the model during training \cite{vyas2023provable, ren2024copyright}. Additionally, several studies have pointed out that synthetic images generated by these models might violate IP rights due to the inclusion of elements or styles that resemble copyrighted works \cite{poland2023generative, wang2024stronger}. Specifically, models like stable diffusion \cite{Rombach_2022_CVPR} may generate images that bear close resemblances to copyrighted artworks, thus raising concerns about IP infringement \cite{shi2024rlcp}. 

\textbf{Image infringement detection and mitigation.}
The current mainstream infringing image detection methods primarily measure the distance or invariance in pixel or embedding space \cite{carlini2023extracting, somepalli2023diffusion, shi2024rlcp, wang2021bag, wang2024image}. For example, \citeauthor{carlini2023extracting} uses the $L_2$ norm to retrieve memorized images. \citeauthor{somepalli2023diffusion} use SSCD \cite{pizzi2022self}, which learns the invariance of image transformations to identify memorized prompts by comparing the generated images with the original training ones. \citeauthor{zhang2018unreasonable} compare image similarity using the LPIPS distance, which aligns with human perception but has limitations in capturing certain nuances. \citeauthor{wang2024image} transform the replication level of each image replica pair into a probability density function. Studies \cite{wen2024detecting, wang2024evaluating} have shown that these methods have lower generalization capabilities and lack interpretability because they do not fully align with human judgment standards. For copyright infringement mitigation, the current approaches mainly involve machine unlearning to remove the model's memory of copyright information \cite{bourtoule2021machine, nguyen2022survey, kumari2023ablating, zhang2024forget} or deleting duplicated samples from the training data \cite{webster2023duplication, somepalli2023understanding}. These methods often require additional model training. On the other hand, \citeauthor{wen2024detecting} have revealed the overfitting of memorized samples to specific input texts and attempt to modify prompts to mitigate the generation of memorized data. Similarly, \citeauthor{wang2024evaluating} leverage LVLMs to detect copyright information and use this information as negative prompts to suppress the generation of infringing images.














\section{Bringing Top-$p$ Sampling to Sparse Attention\label{sec:top_p}}

In this section, we formulate the current sparse attention methods and re-examine the root causes of the problems. We argue that to mathematically approximate the attention output, the goal is to select a minimal set of indices such that the sum of their attention scores meets a certain threshold. Therefore, we propose that top-$p$ sampling should be used instead of top-$k$ to filter out the critical tokens.

\subsection{Problem Formulation\label{sec:formulate}}

We start by formulating the sparse attention. Consider the attention computation during the decoding phase, where we have the query vector $Q \in \mathbb{R}^{1 \times d}$, and the key-value cache $K, V \in \mathbb{R}^{n \times d}$. Here, $d$ denotes the head dimension, and $n$ represents the context length. 
% The standard decoding attention can be formulated as
% \begin{equation}
%     O = \text{softmax} \bigg( \frac{Q \cdot K^T}{\sqrt{d}} \bigg) V = WV
% \end{equation}
% where $W \in \mathbb{R}^{1\times n}$ (sometimes denoted as $P$) represents the (normalized) attention weights. 
% Sparse attention, on the other hand, only loads a subset of tokens from the KV cache and computes the partial attention, which can be represented using a mask matrix.
\begin{definition}[Sparse Attention]
Let $\mathcal{I}$ be the set of selected indices, the output of the sparse attention equals to
\begin{equation}
    \hat{O} = \text{softmax} \bigg( \frac{Q \cdot K^T}{\sqrt{d}} \bigg) \Lambda_{\mathcal{I}} V = W \Lambda_{\mathcal{I}} V
\end{equation}
where $\Lambda_{\mathcal{I}} \in \mathbb{R}^{n \times n},
    \Lambda_{\mathcal{I}}[i, j] = 
    \begin{cases}
    1 & \text{if } i=j \text{ and } i \in \mathcal{I} \\
    0 & \text{otherwise}
    \end{cases}
$.
\end{definition}

% As mentioned above, there are two major types of top-$k$-based methods: static (query-agnostic) and dynamic (query-aware) methods. In our formulation, the difference lies on whether $\mathcal{I}$ depends on the query vector each time.

% \mingyu{Citation needed. Again, you need to add citations when you write the draft, not later. Because it is difficult for you to know where a citation is needed afterwards.}

\cf{
To minimize the output error $\Vert O-\hat{O}\Vert$, we need to carefully select the subset of tokens that are used in the sparse attention computation. However, directly optimizing this objective function without loading the full KV cache is challenging. Earlier research has shown that the distribution of V is relatively smooth \cite{atom}, which implies that the bound is relatively tight.
\begin{equation}
\label{eq:error}
\begin{aligned}
\mathcal{L} = \Vert O - \hat{O}\Vert &= \Vert W(\Lambda_\mathcal{I} - \mathbf{1}^{n \times n}) V\Vert \\
&\le \Vert W(\Lambda_\mathcal{I} - \mathbf{1}^{n \times n}) \Vert \cdot \Vert V \Vert
\end{aligned}
\end{equation}
Therefore, the objective becomes minimizing $\Vert W(\Lambda_{\mathcal{I}} - \mathbf{1}_{n\times n})\Vert = 1 - \sum_{i \in \mathcal{I}} W[i]$, which means selecting a subset of tokens that maximizes the sum of attention weights. If we fix the number of the subset, i.e. $|\mathcal{I}|$, then we have the oracle top-$k$ attention:

\begin{definition}[Oracle Top-$k$ Sparse Attention] Given the budget $B$,
\label{def:topk}
\begin{equation}
    \mathcal{I} = \arg\max_{\mathcal{I}} \sum_{i=1}^n W \Lambda_{\mathcal{I}}\ \ \ \text{s.t.} \ |\mathcal{I}| = B
\end{equation}
\end{definition}
}

The oracle top-$k$ attention serves as a theoretical upperbound of current sparse methods.

% \begin{figure*}[ht]
% \begin{center}
% \centerline{\includegraphics[width=2\columnwidth]{figures/distrib1.pdf}}
% \caption{Diverse distributions observed in attention weights. \textbf{The leftmost image} illustrates a "Flat" distribution \textbf{(Diffuse Attention)}, where the weights are uniformly distributed. \textbf{The middle image} depicts a "Peaked" distribution \textbf{(Focused Attention)}, where the weights are concentrated on the head and tail tokens. When overlaid, the differences between these distributions become readily apparent.}
% \label{fig:distrib}
% \end{center}
% \end{figure*}

\begin{figure*}[t]
  \centering
    \includegraphics[width=0.666\columnwidth]{figures/diffuse.pdf}
    \includegraphics[width=0.666\columnwidth]{figures/focus.pdf}
    \includegraphics[width=0.666\columnwidth]{figures/overlap.pdf}
  \caption{Diverse distributions observed in attention weights. \textbf{The leftmost image} illustrates a "Flat" distribution \textbf{(Diffuse Attention)}, where the weights are uniformly distributed. \textbf{The middle image} depicts a "Peaked" distribution \textbf{(Focused Attention)}, where the weights are concentrated on the head and tail tokens. When overlaid, the differences between these distributions become readily apparent.}
  \label{fig:distrib}
\end{figure*}

\subsection{Rethink the Problem of Top-$k$}
\cf{The Achilles’ heel of top-$k$ attention, as we described earlier, is the dilemma in determining a uniform budget $B$. A larger $B$ leads to inefficiency, while a smaller $B$ results in accuracy loss. We find that this predicament is quite similar to the one encountered in the sampling phase of large language models (LLMs). During the sampling phase, the model samples the final output token from a predicted probability distribution. Nucleus sampling \cite{holtzman2019curious}, or top-$p$ sampling, was proposed to address the problem that top-$k$ sampling cannot adapt to different next-word distributions.}

\cf{Motivated by this insight, we examine the distributions of attention weights more closely. \autoref{fig:distrib} displays two different types of attention weight distributions in real LLMs mentioned in \autoref{fig:teaser}. In \autoref{eq:error}, we demonstrated that the output error can be related to the sum of attention weights. It is straightforward to observe that, when comparing a flat distribution to a peaked one, a greater number of tokens must be selected in the flat distribution to reach the same cumulative threshold. Therefore, we argue that \textbf{the core reason for budget dynamism is the dynamic nature of attention weight distributions at runtime.} Drawing inspiration from top-$p$ sampling, we introduce top-$p$ sparse attention by directly apply threshold to the sum of attention weights.}

% We believe that \textbf{the core reason for the dynamic budgets is the dynamic distributions of attention weights at runtime}, which motivates us to replace top-$k$ in current sparse attention algorithm with top-$p$.


% However, we found that \texttt{top-$k$} \mingyu{why such style set?} using a fixed budget $B$, making it impossible for existing algorithms to leverage adaptive sparsity. 
% \mingyu{Grammar issue; this sentence misses the predicate (the verb).}
% As Figure \mingyu{?} shows, the distributions of attention weights vary across different attention heads, layers and queries. Using a uniformed $B$ leads to either inefficiency due to a larger $B$ or accuracy lose due to a smaller $B$. This brings challenges when we deploy these algorithms in serving systems, where


% \begin{figure}[ht]
% \begin{center}
% \centerline{\includegraphics[width=\columnwidth]{figures/distrib.png}}
% \caption{Different distributions which not only appear in next word probability distribution of LLM sampling, but also in attention weights distribution. For a “Flat” distribution, the weights are more uniform and top-$k$ should use a larger budget. For a “Peaked” distribution, the weights is more skewed and top-$k$ can achieve the same probability sum with a less budget.}
% \label{fig:distr}
% \end{center}
% \end{figure}

\begin{definition}[Oracle Top-$p$ Sparse Attention] Given the threshold $p$,
\label{def:topp}
\begin{equation}
    \mathcal{I} = \arg\min_{\mathcal{I}} |\mathcal{I}|\ \ \ \ \text{s.t.} \ \sum_{i=1}^n W \Lambda_{\mathcal{I}} \ge p
\end{equation}
\end{definition}

% Top-$p$ is inherently adaptive to different distributions since it directly goes to our ultimate goal, i.e. the sum of normalized attention weights.

Comparing to top-$k$, top-$p$ is advantageous because it provides a theoretical upperbound of error in \autoref{eq:error} by $(1-p) \cdot \Vert V \Vert$. Under this circumstance, top-$p$ reduces the budget as low as possible, making it both efficient and adaptive to different distributions.

% We evaluated the oracle top-$p$ in two respects: efficiency and its adaptive budget capability. For the former, we use cosine similarity as the metrics with a 10k retrieval prompt, as \autoref{fig:cos} shows. The relative error for top-$p$ with $p=0.95$ was found to be around the theoretical bound, proving its effective control of error. For top-$k$, the results were also good with $B=128$, but degradation was observed with $B=16$, indicating under-selection. We successfully observe the budget dyxnamism in \autoref{fig:dynamism}, which demonstrates that top-$p$ sparse attention can adaptively adjust attention sparsity at runtime. Given this capability, we will focus on using top-$p$ to address the budget problem in top-$k$ in the next section.

% Additionally, we found that real-practice top-$k$ methods, such as Quest \cite{tang2024quest}, performed worse than the oracle top-$k$ with the same budget, highlighting the limitations of fixed-budget approaches

% \mingyu{You have these many formal definitions and notations, but at the end you do not have a formal proof to show Def 3.3 is better than 3.2 to match the error minimization of Eq (4). While intuitively this is true, showing it explicitly would be better.}

% \begin{figure}[h]
% \begin{center}
% \centerline{\includegraphics[width=\columnwidth]{figures/cos.pdf}}
% \caption{Relative output error measured by cosine similarity in each layer.}
% \label{fig:cos}
% \end{center}
% \vskip -0.3in
% \end{figure}

% \begin{figure}[ht]
% \begin{center}
% \centerline{\includegraphics[width=\columnwidth]{figures/dynamism.pdf}}
% \caption{Dynamic budgets observed in oracle top-$p$ attention. We observe the dynamism across four dimensions: different \textbf{prompts (tasks)}, different \textbf{queries} with the same prompt, different \textbf{layers} in the same query, and different \textbf{heads} in the same layer.}
% \label{fig:dynamism}
% \end{center}
% \end{figure}
\section{Twilight}
% \cf{TODO(Chaofan): Section 4 needs refactor.}
\begin{figure*}[t]
\begin{center}
\centerline{\includegraphics[width=\columnwidth*2]{figures/arch.pdf}}
% \mingyu{Nits: to beautify, all figures in the paper should have similar font size \emph{after} embedded in the paper. That is, even you use the same font size when drawing them, you also need to make sure they have the same scaling ratio when included in the paper. Since the final absolute widths are fixed (single-column or double-column), this means you need to be careful how wide your figures should be when drawing them.}
\caption{Architecture of Twilight. Twilight is built on certain existing algorithm and serves as its optimizer. It computes self-attention in three steps. First, \textbf{Token Selector} select critical tokens using the strategy of base algorithm under a relaxed budget. Then, \textbf{Twilight Pruner} prunes the selected token indices via top-$p$ thresholding. Finally, the optimized token indices are passed to \textbf{Sparse Attention Kernel} to perform attention computation.}
\label{fig:arch}
\end{center}
\end{figure*}

In the previous section, we demonstrated that top-$p$ attention can adaptively control the budget while ensuring that the sum of normalized attention weights meets a certain threshold $p$. Our primary goal is to use top-$p$ to endow more existing algorithms with adaptive attention sparsity, rather than simply inventing another sparse attention, which is motivated by two main reasons: On one hand, despite their budget-related challenges, existing sparse algorithms have achieved significant success in current serving systems ~\cite{vllm, sglang}, thanks to their effective token selection strategies. These strategies can be readily reused and enhanced with adaptive sparsity.
On the other hand, we anticipate that future sparse attention methods may still employ top-$k$ selection. By developing a general solution like ours, we aim to automatically equip these future methods with adaptive attention sparsity, thereby improving their efficiency and adaptability without requiring extensive redesign. Consequently, we initially positioned our system, Twilight, as an \textbf{optimizer} for existing algorithms.

However, deploying top-$p$ to different existing sparse attention algorithms faces majorly three challenges, both algorithm-wise and system-wise.

% \mingyu{I think such challenges are better put at the end of the last section rather than here. They belong to the high-level discussions rather than our concrete designs, i.e., they are not unique to Twilight.}

\textbf{(C1) Not all algorithms are suitable for top-$p$.} Top-$p$ imposes strict constraints on the layout of attention weights. For example, simply replacing top-$k$ with top-$p$ in Quest \cite{tang2024quest} would not work, as Quest performs max pooling on weights with a per-page layout (16 tokens per page). Additionally, some other methods \cite{yang2024tidaldecodefastaccuratellm, liu2024retrievalattention} do not use attention weights to select critical tokens at all.

\textbf{(C2) It's harder to estimate weights for top-$p$ than top-$k$.} 
% Section \ref{sec:top_p} shows great potential of top-$p$ pruning method. However, either oracle top-$k$ or oracle top-$p$ attention can only save the load of $V$ since we need to load all keys to compute full attention weights \cite{sheng2023flexgen}. Numerous top-$k$ based methods dedicate to find ways to better estimate attention weights without loading full keys through either loading less channels \cite{ribar2023sparq, yang2024post, zhang2024pqcache} or less tokens \cite{tang2024quest}. To summarize, their core purpose is to represent the $K$ cache in a lower-precision way. For example, DS \cite{yang2024post} compresses $K$ cache to 2-bit with only half channels, therefore saving $1/16$ memory I/O. However, 
The precision requirement of top-$p$ is higher than that of top-$k$, because the former requires a certain degree of numerical accuracy while the latter only demands ordinality. 
% \mingyu{There is no explanation in the paper for this claim. The table also directly says so without reasons.} 
\autoref{table:prune_cmp} provides a basic comparison of top-$k$, top-$p$, and full attention. The precision requirement of top-$p$ attention lies between the other two, which leads us to reconsider the appropriate precision choice for compressing the $K$ cache.

\textbf{(C3) System-level optimizations are needed.} Since our work is the first work introduce top-$p$ to attention weights, many algorithms need to be efficiently implemented in hardware, including efforts on both efficient parallel algorithm designs and efficient kernel optimizations.
% \mingyu{This is a very vague discussion that offers little info to readers. Try to be more specific, e.g., what (extra or untraditional) computations are needed (e.g., a prefix sum?), and their impls are not widely explored, etc.}

In Section \ref{sec:hp}, we address \textbf{C1} by proposing a unified pruning framework for sparse attention. In 
 Section \ref{sec:kernel}, we mitigate the runtime overhead of by efficient kernel implementations (top-$p$, SpGEMV, Attention) and 4-bit quantization of $K$ cache, addressing \textbf{C2} and \textbf{C3}. Lastly, in Section \ref{sec:disc}, we analyze the overhead of Twilight and discuss some topics.
\begin{table}[t]
\caption{Comparing of different pruning methods on attention weights. "Normalize" indicates \texttt{softmax}.}
\label{table:prune_cmp}
% \mingyu{This table seems important, but none of the information in it has been explained anywhere in the paper. You cannot let the readers to derive these characteristics by themselves.}
\begin{center}
\begin{small}
\resizebox{\linewidth}{!}{
\begin{tabular}{ccccc}
\toprule
\textbf{Methods} & \textbf{Efficiency} & \textbf{Precision} & \textbf{Output} & \textbf{Need} \\
 & & \textbf{Requirement} & \textbf{Accuracy} & \textbf{Normalize?} \\
\midrule
Top-$k$ & High & Low  & Median & $\times$ \\
Top-$p$ & High & Median & High & $\surd$\\
Full Attn.   & Low  & High & High & $\surd$ \\
\bottomrule
\end{tabular}
}
\end{small}
\end{center}
\end{table}

\subsection{Hierarchical Pruning with Select-then-Prune Architecture\label{sec:hp}}

% Despite various designs in different algorithms we mentioned before, there is a core commonality we can observe from the formulation in Section \ref{sec:formulate} \textemdash most \mingyu{for emdash, you either have spaces both before and after (then you need to use \{\} after the command (like \textemdash{}) to ensure the space is not removed), or have no spaces both before and after.} sparse algorithms select a subset of tokens.

% Based on this, we abstract the base algorithm into a black-box \textbf{Token Selector}, which uses some metadata to select critical tokens. 
% \mingyu{I think you can improve this description like this. You start by reminding that existing algorithms have the difficulty of choosing a proper budget, either too conservative (good accuracy, bad efficiency) or too aggressive (bad accuracy, good efficiency). So we make it a two-step process like this. Now Token Selector only needs to be conservative, and we have a Pruner to help. This achieves both good accuracy and good efficiency, ...\\
% Essentially in this paragraph you are missing the point that why this would work. It works because the Selector is conservative. (A2) in the next paragraph touches on this, but it is a bit too late, and also not detailed enough to show how it resolves the tradeoff.}
% And we treat top-$p$ as a \textbf{Pruner}, which optimizes the indices dumped by Token Selector by further pruning unimportant tokens, consisting of the hierarchical \textbf{Select-then-Prune} architecture we propose as the left side of \autoref{fig:arch} illustrates. This hierarchical architecture also share commonalities to the LLM sampling stage, where we usually use a mixture of top-$k$ and top-$p$ to sample final token (Similarly, in LLM sampling, we also first apply top-$k$ then apply top-$p$ according to the implementation of open-source LLM engines like vLLM \cite{vllm}).

% Comparing to the original algorithm, inserting a pruner in the middle has the following advantages. \textbf{(A1) The pruner makes the algorithm efficient.} As we mentioned above, existing algorithms suffer from inefficiency and always mistakenly select some less important tokens. Top-$p$ pruning is more accurate than Token Selector which is based on top-$k$, which can further prune these tokens. \textbf{(A2) The pruner makes the algorithm adaptive to budget.} In this architecture, due to the presence of pruner, the Token Selector is allowed to use a very conservative budget like $\frac{1}{4}$ sparsity (which equals to $8192$ in $32k$ context length). Note that the latency of Token Selector is irrelevant to the budget in most algorithms. Then the pruner will automatically lower the budget across different heads, layers and queries, solving the problem that the budget is difficult to determine.

Recall that existing algorithms face the challenge of choosing a proper budget: either over-selection or under-selection. To address this, we propose a two-step process. We first abstract the base algorithm into a black-box \textbf{Token Selector}, which uses some metadata to select a subset of critical tokens. And we allow the Token Selector to a conservative budget (e.g. $1/4$ sparsity), as we have a \textbf{Pruner} after that to further optimize the selected indices by pruning unimportant tokens. This hierarchical \textbf{Select-then-Prune} architecture is illustrated on the left side of \autoref{fig:arch}.

% This design shares similarities with the LLM sampling stage, where a mixture of top-$k$ and top-$p$ is commonly used to sample the final token (similarly, in LLM sampling, we first apply top-$k$ and then top-$p$, as implemented in open-source LLM engines like vLLM \cite{vllm}).
% Compared to the original algorithm, inserting a pruner in the middle offers the following advantages:

% \textbf{(A1) The pruner makes the algorithm efficient.} Existing algorithms often suffer from inefficiency due to mistakenly selecting less important tokens. The top-$p$ pruner is more accurate than the Token Selector, which is based on top-$k$, allowing it to further prune these unnecessary tokens.
% \textbf{(A2) The pruner makes the algorithm adaptive to budget.} In this architecture, the Token Selector is allowed to use a very conservative budget, such as 41​ sparsity (equivalent to 8192 tokens in a 32k context length). Importantly, the latency of the Token Selector is largely independent of the budget in most algorithms. The pruner then dynamically adjusts the budget across different heads, layers, and queries, solving the problem of determining the optimal budget.

\subsection{Efficient Kernel Implementation\label{sec:kernel}}

\subsubsection{Efficient SpGEMV with 4-bit Quantization of Key Cache}

As previously analyzed, the precision requirement of top-$p$ lies between top-$k$ and full attention. For top-$k$, many works \cite{yang2024post, zhang2024pqcache} push the compression to a extreme low-bit (1-bit/2-bit). For full attention, SageAttention \cite{zhang2025sageattention} is proposed recently as a 8-bit accurate attention by smoothing $K$ and per-block quantization. In this work, we find 4-bit strikes a balance between accuracy and efficiency, making it the ideal choice to calculate estimated attention weights for top-$p$. And we implement an efficient sparse GEMV (SpGEMV) kernel based on FlashInfer.
% \cite{ye2025flashinfer} which loads $K$ in a scattered manner, aligning with the design of Paged $K$ cache \cite{vllm}. 
We maintain an extra INT4 asymmetrically quantized $K$ cache in GPU as \autoref{fig:arch} shows. The INT4 $K$ vectors are unpacked and dequantized in shared memory which reduces I/O between global memory and shared memory to at most $1/4$, leading to a considerable end-to-end speedup.

% \mingyu{I do not see much novelty from this part. Are we just using an existing impl from FlashInfer, or do we have some new contributions? If the former, this subsubsection should be put \emph{after} those in which we have contributions in this subsection. If the latter, explicitly highlight the contributions.}

\subsubsection{Efficient Top-$p$ Kernel via Binary Search}

\begin{algorithm}[tb]
   \caption{Top-$p$ via Binary Search}
   \label{alg:binary}
\begin{algorithmic}
   \STATE {\bfseries Input:} Normalized attention weights $W \in \mathbb{R}^{BS \times H \times N}$, Threshold of TopP $p$, Hyper-parameter $\epsilon$
   \STATE {\bfseries Output:} Indices $I$, Mask $M \in \{0, 1\}^{BS \times H \times N}$
   \STATE \textbf{Initialize:} $l = 0$, $r = \max(W)$, $m = (l + r) / 2$;
   \REPEAT
   \STATE $W_0 =\text{where}(W < m, 0.0, W)$;
   \STATE $W_1 =\text{where}(W \le l, \text{INF}, W)$;
   \STATE $W_2 =\text{where}(W > r, \text{-INF}, W)$;
   \STATE $s=\text{sum}(W_0)$; //\texttt{ Compute the sum over current threshold } $m$;
   \IF{$s \ge p$}
   \STATE $l = m$;
   \ELSE
   \STATE $r = m$;
   \ENDIF
   \UNTIL{$\max(W_2) - \min(W_1) \ge \epsilon$}
   \STATE Select indices $I$ or mask $M$ where $W \ge l$;
   \STATE \textbf{return }{$I$, $M$};
\end{algorithmic}
\end{algorithm}

As we mentioned before, our top-$p$ method is motivated by the top-$p$ sampling, which also takes up a portion of decode latency. Therefore, our efficient kernel is modified from the top-$p$ sampling kernel from FlashInfer \cite{ye2025flashinfer}, a high performance kernel library for LLM serving.

A brute-force way to do top-$p$ sampling is to sort the elements by a descending order and accumulate them until the sum meets the threshold, which is quite inefficient in parallel hardwares like modern GPU. Our kernel adopts a parallel-friendly binary search algorithm as \cref{alg:binary}.
% \mingyu{Fix: ref name is missing. Also, consistently use either ref or autoref or cref throughout the paper} described.

% However, since our scenario is a bit different from token sampling, there are still some problems when adapting this algorithm to Twilight. First, as our top-$p$ is performed on attention weights, with different shape and layout comparing with the logits distribution on vocabulary, we need to redesign the parallel strategy including blocks/threads launching. \mingyu{So this is the problem. What is your solution?}
% Second, we can fuse top-$p$ with GEMV kernel which reduces kernel launch time and reuse some median results such as maximum which are already computed in the softmax part.

% \subsubsection{Efficient Sparse Attention with Awareness of Head Dynamism}

% Top-$p$ pruner brings head-wise dynamic budgets, which also brings some system challenges especially in the attention kernel. Traditional Sparse(Paged) Attention kernel allocates uniformed computation resources to all heads, leading to computation inefficiency. 
% Other head-wise dynamic budget works also face the same challenge. DuoAttention \cite{xiao2024duo} packs the retrieval heads and streaming heads separately and computes them in two steps. AdaKV \cite{feng2025adakvoptimizingkvcache} adopts a flattened KV cache and reuses \texttt{flash\_attn\_varlen}. These methods are either not general enough or not efficient enough to be used in Twilight. 
% FlashInfer \cite{ye2025flashinfer} deeply investigates the load balancing problem, but only for requests with dynamic lengths. To build an attention kernel with awareness of head-wise dynamism, Twilight borrows the idea from AdaKV \cite{feng2025adakvoptimizingkvcache}, reusing the load balancing algorithm in FlashInfer by flattening the head dimension.
% \mingyu{I am not sure about whether the last sentence (just one sentence) is sufficiently clear to explain how we did it. It seems very abstract. But maybe this is the style of AI papers in contrast to system papers. Use your own judgment.}

\subsection{Overhead Analysis and Discussion\label{sec:disc}}

\textbf{Runtime Overhead.} The runtime of Twilight-optimized algorithm consists of three parts according to the pipeline in \autoref{fig:arch}: $T_{\text{Token Selector}} + T_{\text{Twight Pruner}} + T_{\text{Sparse Attention}}$. Comparing to the baseline without Twilight, our method introduces an extra latency term $T_{\text{Twight Pruner}}$ but reduces $T_{\text{Sparse Attention}}$ because it further reduces its I/O. Our hierarchical architecture naturally fits the hierarchical sparsity, where the number of tokens gradually decreases as the precision increases. Suppose the Token Selector has a $\frac{1}{16}$ sparsity, then the theoretical speed up can be formulated as 

$$
    \frac{N/16 + B_0}{N/16 + B_0/4 + B_1}
$$

where $B_0 = |I_0|$ is the budget of Token Selector, $B_1 = |I_1|$ is the budget after pruned by Twilight. Suppose $B_0 = N/4, B_1 = N/64$, then the speed up is approximately $2 \times$. Here we omit the overhead of top-$p$ since SpGEMV dominates the latency when $B_0 = N/8 \sim N/4$.

% \begin{figure}[ht]
% \begin{center}
% \centerline{\includegraphics[width=\columnwidth]{figures/time_io.pdf}}
% % \mingyu{The figure, e.g., its x and y axes and the several boxes, is not easy to understand. Maybe more explanation is needed.\\
% % Some of the texts are too small to read.}
% % \caption{A theoretical time cost model for Sparse Attention with Twilight.}
% \label{fig:time_model}
% \end{center}
% \end{figure}
% \mingyu{INT4 vs. int4 (and similarly FP16, fp16, etc.); be consistent. Check the whole paper}

% \textbf{Memory Overhead.} As \autoref{fig:arch} shows, Twilight introduces an extra INT4 quantized key cache, which brings a $\frac{1}{2} \times \frac{1}{4} = \frac{1}{8}$ extra KV cache memory overhead. However, this additional overhead doesn't appear in all cases. On one hand, some base algorithms also maintain an INT4 key cache like DS \cite{yang2024post}, which is already included in the metadata part. On the other hand, there are some recent efforts \cite{zhang2024sageattention2} explore the INT4 full attention. This brings chances that we directly involve the estimated attention weights calculated by INT4 key cache in the attention computation, which allows us not to maintain the original FP16 key cache. Moreover, there are some optimizations when GPU memory becomes a bottleneck, like offloading and selective quantization (Only maintain extra INT4 key cache for hot tokens), which we leave as future works.

\textbf{Integrate with Serving System.} Since our system design naturally aligns with PagedAttention \cite{vllm}, Twilight can be 
% \mingyu{``can be'' is something you have not done. So what is the implementation now? Have we already done the integration? Do we want to briefly discuss the implementation?} 
seamlessly integrated into popular serving systems like vLLM~\cite{vllm} and SGLang~\cite{sglang}. Prefix sharing and multi-phase attention \cite{lin2024parrot, sglang, zhu2024relayattention, ye-etal-2024-chunkattention, cascade-inference} also become common techniques in modern serving systems, which also fit Twilight since we use paged-level or token-level sparse operations and can achieve flexible computation flow.
\begin{table*}[tb]
\centering
\caption{Demographics of Participant Clients: Previous Art Therapy Sessions indicates the number of times the client has previously participated in art therapy; Familiarity with Traditional Drawing reflects the client's level of experience with traditional drawing techniques (0-not familiar; 1-very familiar); Familiarity with Digital Drawing reflects the client's level of experience with digital drawing techniques (0-not familiar; 1-very familiar); Participation Purposes reflects the reasons clients choose to engage in the activity.}
\vspace{-3mm}
\label{tab:clients}
\small
\resizebox{1\linewidth}{!}{
\begin{tabular}{cccccccccc}
\toprule
\textbf{ID} & \textbf{Gender} & \textbf{Age} & \textbf{Education} & \textbf{Region} & \parbox[t]{2.5cm}{\centering\textbf{Previous Art Therapy Sessions}} & \parbox[t]{3cm}{\centering\textbf{Familiarity with Traditional Drawing}} & \parbox[t]{2cm}{\centering\textbf{Familiarity with Digital Drawing}} & \parbox[t]{2cm}{\centering\textbf{Therapist Assignment}} & \parbox[t]{2.5cm}{\centering\textbf{Participation Purposes}} \\
\midrule
C1  & Female & 37  & Bachelor's & China/Shanghai & 0                            & 1                                   & 0.25  &T3 & Personal Growth                   \\
C2  & Female & 35  & Bachelor's & China/Shenzhen & 3                            & 0.5                                   & 0.5   &T3 & Career Development and Family                 \\
C3  & Female & 28  & Master's   & China/Hebei    & 2                            & 0.75                                  & 0.75   &T3  & Family and Emotional Management                \\
C4  & Female & 36  & Bachelor's & China/Beijing  & 10                           & 0.75                                   & 0   &T3  &Career Development                \\
C5  & Male   & 28  & Master's   & Germany       & 0                            & 1                                   & 0.75   &T3   &  Emotional Management and Personal Growth                       \\
C6  & Other  & 26  & Associate's & China/Heilongjiang & 1                            & 0.5                                   & 0.25  &T5  & Emotional Exploration and Intimate Relationships                           \\
C7  & Female & 23  & Master's   & China/Shanghai & 0                            & 1                                   & 1     &T5     &  Intimate Relationships                    \\
C8  & Female & 20  & Bachelor's & China/Shenzhen & 0                            & 0.5                                   & 0.5    &T5   &  Emotional Management and Intimate Relationships                       \\
C9  & Female & 25  & Bachelor's & China/Guangxi  & 4                            & 0                                   & 0.5    &T5    &  Self-Expression and Emotional Exploration                      \\
C10 & Male   & 23  & Master's   & China/Shenzhen & 0                            & 0.75                                   & 0.5   &T5   &             Self-Expression and Social Skills             \\
C11 & Female & 26  & Master's   & China/Hangzhou & 0                            & 0.5                                   & 0.25    &T4  &        Emotional Management, Social Skills and Intimate Relationships                 \\
C12 & Female & 26  & Master's   & China/Shanghai & 2                            & 0.75                                   & 0.5    &T4   &                   Stress Relieving and Intimate Relationships  \\
C13 & Female & 30  & Master's   & China/Dalian   & 0                            & 0.5                                   & 0.25   &T4    &             Family and Emotional Management            \\
C14 & Female & 19  & Bachelor's & China/Chongqing & 0                            & 0.25                                   & 0.25   &T4  &                Personal Growth and Self-Exploration           \\
C15 & Male   & 27  & Bachelor's & China/Beijing  & 0                            & 0.25                                  & 0.25   &T4    &                 Stress Relieving and Personal Growth        \\
C16 & Female & 22  & Bachelor's & China/Shandong & 0                            & 0.5                                   & 0.25   &T1     &              Emotional Management and Social Skills       \\
C17 & Male   & 38  & Master's   & China/Sichuan  & 0                            & 0.75                                   & 0.75   &T1     &                    Personal Growth      \\
C18 & Female & 40  & Master's   & China/Beijing  & 20                           & 1                                   & 0.75    &T1      &               Stress Relieving and Emotional Management          \\
C19 & Female & 28  & Bachelor's & China/Guangzhou & 0                            & 0.5                                   & 0   &T1       &                 Future Career Planning and Personal Growth      \\
C20 & Male   & 25  & Master's   & China/Guangzhou & 0                            & 1                                   & 1   &T1        &                    Academic Pressure Relieving   \\
C21 & Male   & 24  & Master's   & China/Hubei    & 0                            & 0                                   & 0   &T2        &                Childhood Family and Dreams Exploration  \\
C22 & Female & 24  & Master's   & China/Shenzhen & 0                            & 0.25                                   & 0.25    &T2  &                Emotional Management and Personal Growth     \\
C23 & Male   & 25  & Master's   & China/Zhejiang & 10                           & 0.5                                   & 0.5    &T2   &                  Emotional Development and Self-Expression        \\
C24 & Male & 55  & Bachelor's & Dubai& 0 & 0.5& 0.5&T2 &                           Emotional Management \\
\bottomrule

\end{tabular}}
\Description{The table 2 describes 24 participants in art therapy sessions. The participants are from diverse locations, including China (Shanghai, Shenzhen, Hebei, Beijing, Heilongjiang, Guangxi, Hangzhou, Chongqing, Shandong, Sichuan, Hubei, and Zhejiang), Germany, and Dubai. The ages range from 19 to 55 years old, with varying levels of education from associate degrees to master's degrees and bachelor's degrees. Their familiarity with traditional drawing techniques ranges from no familiarity to very familiar, while their familiarity with digital drawing techniques also varies across the spectrum. The participants have attended between 0 and 20 previous art therapy sessions and are assigned to different therapists identified by codes T1 to T5.Participation Purposes reflects the reasons clients choose to engage in the activity}
\end{table*}

\section{Field study}
Using \name{} as both a novel system to study and a research tool to study with, we aim to explore how a human-AI system support clients' art therapy homework in their daily settings (\textbf{RQ1}) and how such a system could mediate therapist-client collaboration surrounding art therapy homework (\textbf{RQ2}). To this end, we conducted a field deployment involving 24 recruited clients and five therapists over the course of one month.



%参与者与实验的setup
    %参与者招募
        % 我们招募的途径:To recruit our clients, we distributed digital recruitment flyers through social media platforms.
        % 海报上描述了什么:The recruitment flyer described the art therapy activities as "promoting self-exploration using a digital software".
        % 我首先要求参与者填写pre-问卷,这个问卷主要包括descriptions of the art therapy activities, demographic information, the number of art therapy sessions they attended, familiarity with digital drawing, and specific needs for the art therapy activities.
        % Participants were included in this study with the aim of reducing stress and anxiety, fostering personal growth, improving emotional regulation, and strengthening social skills.
        % 此外,we tried to selection of participants based on their regions, occupations, the types of devices they used, and the number of times they participated in art therapy.
        % finally, 有27名参与者开始使用这个系统,其中有3名参与者drop out因为缺乏时间
\subsection{Participants and Study Procedure}
\subsubsection{Participants}

The five therapists who participated in the field evaluation were the same ones from our contextual study (see \autoref{tab:expert}). Each therapist was compensated at their regular hourly rate.
For client recruitment, we distributed digital flyers through social media platforms, describing the art therapy activities as an "online art therapy experience promoting self-exploration using a digital software." This aligns with the common goal of art therapy sessions, which are widely used to promote self-exploration for all clients, beyond treating mental illness~\cite{kahn1999art, riley2003family}.

Participants first completed a pre-questionnaire, which provided an overview of the activities and collected demographics, and prior experiences with art therapy experience and with digital drawing---to ensure that we include both novices and experienced user---and their personal goals for participation. 
The therapists guided the recruitment and screening of the the clients, and included individuals seeking for reducing stress, fostering personal growth, enhancing emotional regulation, and strengthening social skills. The therapists excluded individuals with serious mental health conditions to minimize ethical risks.
%Based on the therapists' advice, clients with goals such as reducing stress and anxiety, fostering personal growth, enhancing emotional regulation, and strengthening social skills were included, avoiding ethical concerns related to clinically diagnosed mental health conditions. 
%We also considered participants' regions, device types, drawing familiarity, and prior art therapy experience to create a balanced selection.

In total, 27 clients began using \name{}, but 3 withdrew early due to scheduling conflicts. The final group of 24 clients (C1-C24; 8 self-identified males, 15 self-identified females, 1 identifying as other; aged 19-55) completed the study (client demographics are detailed in the~\autoref{tab:clients}). Clients who completed the full process were compensated with \$37, others were compensated with a prorated fee.
Our study protocol was approved by the institutional research ethics board, and all participant names in this paper have been changed to pseudonyms. Participants reviewed and signed informed consent forms before taking part, acknowledging their understanding of the study.

% The five therapists participated in the field evaluation were the ones who also participated in our contextual study (see \autoref{tab:expert}).
% Five art therapists were compensated with their regular hourly rate.
% For the clients recruitment, we distributed digital recruitment flyers through social media platforms. 
% The recruitment flyer described the art therapy activities as ``online art therapy experience promoting self-exploration using a digital software''.
% This is due to that this is a common goal for art therapy sessions, since Art therapy activities are not only effective in treating mental illness but also widely promote self-exploration for every clients, as commonly integrated into practice~\cite{kahn1999art,riley2003family}.
% First, participants completed a pre-questionnaire that provided an overview of the art therapy activities and gathered details such as their demographics, the number of art therapy sessions they've attended, familiarity with digital drawing, and any specific needs they hoped to address.
% Following that, based on the advices from the therapists, clients were included with the goal of reducing stress and anxiety, fostering personal growth, enhancing emotional regulation, and strengthening social skills.
% The therapists suggest so since they agree that these therapeutic goals would be beneficial for eavery day therapy clients and would could It might avoid the potential ethical and safety risks associated with clinically diagnosed mental health issue.
% Further, we selected participants based on a balance of their regions, the types of devices they used, the familiarity with drawing and their prior experience with art therapy. 

% In total, 27 clients began using \name{}, but 3 withdrew from the study at the early stage due to scheduling conflicts.
% Finally, 24 clients (C1-C24; 8 self-identified males, 15 self-identified females, 1 identifying as other; aged 19-55) completed our field study. 
% APPENDIX shows the specific client demographics.
% We compensated clients based on their level of involvement, with those who completed the full one-month study receiving 200 RMB as a bonus, and clients who dropped out receiving a prorated fee according to the duration of their participation.

% Our protocol was approved by the institutional research ethics board, and all names in this paper have been changed to pseudonyms.
% Also, before participating in the activity, participants carefully reviewed and signed the informed consent form, acknowledging their understanding.

%在与治疗师协商讨论下,这些用户被分到5位治疗师(see Table),其中T2有4位来访者,其余治疗师有5位来访者。
%这个研究. .
%在活动开始前,我们邀请每位参与者开展了一场介绍session. 主要是目的是介绍活动目的与流程,并且演示如何使用\name{},并且为每位来访者可以接触到系统的URL的链接;
%介绍活动结束后,来访者被鼓励有规律地去自行探索使用\name{};
%每隔一周,我们会安排治疗师与来访者进行线上一对一的session。我们会鼓励治疗师在线上一对一session之前提前review来访者的使用数据,并通过即时通讯软件与我们交流review之后的洞见与想法。
%在线上一对一session时,在不干扰治疗师艺术治疗实践的基础上,我们鼓励治疗师在线上一对一session时利用这些数据。在艺术创作阶段,来访者可以通过分享屏幕的方式使用系统的第一个阶段进行创作并与治疗师进行讨论交流,在session快结束前治疗师会给来访者推荐家庭作业。
%在session结束后,治疗师会在治疗师系统上安排家庭作业并给予来访者的个人赠言。此外,来访者在结束线上session后可以按照治疗师的推荐完成家庭作业或者自行探索使用系统。
\subsubsection{Procedures}

Clients were distributed in coordination with the five therapists, as shown in \autoref{tab:expert}. T2 was assigned four clients, while the other therapists each had five clients. The field study consisted of two main activities: (1) three online in-session activities, where clients had one-on-one conversations and collaborated with the therapist, and (2) unstructured between-session activities, where clients practiced therapy homework using \name{} following the therapist’s recommendations.
Before the study, we held online introductory sessions to familiarize the clients with \name{}, and provided both demonstrations and hands-on exploration on their preferred devices. Similarly, we offered online training for therapists on customizing and reviewing homework, while allowing them to explore both the therapist-facing and client-facing applications. After the session, clients were encouraged to regularly explore \name{}.
Two weeks into the study, we scheduled weekly one-on-one online sessions between therapists and clients, each lasting approximately 60 minutes. Therapists were encouraged to review the clients' homework history using \autoref{fig:ui}(c) before each session. During the online session, therapists used this data to inform their practices without interrupting the flow of therapy. We encouraged clients in advance, to create artworks during the Art-making Phase~(\autoref{fig:qual_results}(a)), sharing screens and discussing their creations with the therapist, but did not interfere with the therapeutic process.

%Clients also used \autoref{fig:qual_results}(a) to create artwork, sharing their screens and discussing their creations with the therapist.

At the end of each session, therapists recommended homework tasks based on insights gained during the conversation. After the session, therapists might customize homework agents, including customizing conversational principles, assigning homework tasks, and providing personal messages through \autoref{fig:ui}~(d). Clients could then either complete the assigned homework or engage in self-exploration using \name{} between sessions.

% Clients were distributed In coordination with the five therapists, as shown by \autoref{tab:clients}: T2 was assigned with four clients, while each of the other therapists was assigned with five clients.
% The procedure for the field study consisted of two activities: (1) three online in-session activities where they have one-on-one conversation and collaboration with the therapist and (2) unstructured between-session activities where they perform therapy homework practices either upon recommendations of usage from the therapist or volunteerily use it in their daily lives.
% Before the study, we conducted an introductory session for each client to explain the activities, demonstrate how to use \name{}, and provide access to \name{} via a URL on their preferred devices.
% After the introductory session, the clients were encouraged to explore the use of \name{} on a regular basis.

% After two weeks of self-exploration, we started scheduling weekly one one-on-one online sessions between the therapists and the clients.
% Therapists were encouraged to review clients' homework history using \autoref{fig:ui}~(c) before the online session.
% During the online one-on-one session, we encouraged therapists to use these history data without interfering with their art therapy practices. 
% Also, they would utilize \autoref{fig:ui}~(a) to create their artwork by sharing their screens and discussing their artworks with therapists. 
% Before the end of the session, the therapist would recommend the homework tasks for the client based on the insights gained from the one-on-one session.
% After the online session ends, therapists would customize homework agents, including modifying or updating the conversational agent principles, assigning homework tasks and providing therapist's messages to the client through \autoref{fig:system}~(d). 
% Correspondingly, clients could either complete the homework or engage in self-exploration using \name{} between sessions.

% 对于异步session场景数据收集下,所有来访者使用系统的图像以及对话记录等日志数据以及治疗师在治疗师系统中使用定制功能的日志数据在保存在数据库中。
% 此外,我们鼓励来访者和治疗师通过即时通讯软件发送给我们images以及comments关于使用系统的实践以及感受。
% 对于线上session的场景数据收集,首先,online sessions were audio- and video-recorded.
% 此外,at the end of each online session, we conducted a 5-minute interview with therapists, mainly to collect their practices and experiences about the session.
% Upon concluding all the sessions,我们与治疗师以及来访者开展了约为30分钟的semi-structured interview to 探索ai agents如何支持艺术治疗场景的家庭作业(RQ1)以及AI agents如何mediate 治疗师与来访者合作(RQ2). We used 治疗师与来访者在 the trial period使用系统的log 数据以及他们的反馈作为stimuli 去问特定的使用实践的问题。
% With participants' consent, we recorded the interviews and transcribed them for thematic analysis.
% First, two researchers conducted collaborative inductive coding. They initially annotated the transcript to identify relevant quotes, key concepts, and recurring patterns in the data. These findings were further developed through regular discussions, leading to a detailed coding scheme aligned with the research questions. Quotes were then coded and clustered into a hierarchy of emerging themes, continually reviewed, and refined in recurrent meetings, where exemplar quotes were also selected for presenting each theme and sub-theme. 
% Also, we collected the log data from 治疗师和来访者 作为证据以及examples for the thematic analysis results.

\subsection{Data Gathering Methods} 

For between-sessions, we stored all homework-related data in a database, including artwork, dialogue, usage logs, as well as information on homework customization such as conversational principles, tasks, and personal messages.
We encouraged participants to use personal messaging (WeChat) to share pictures and comments about on-the-spot experience and feelings after homework with \name{} to compensate for semi-structured interviews.
During online sessions, we recorded audio and video. 
The researchers did not observe the therapy session in live, but reviewed post hoc, as the therapists believed a third party's presence could affect a client's emotional expression and the therapist-client dynamic.
After each session, we conducted a brief 5-minute interview with the therapists to gather their insights and feelings.

Upon the completion of the final one-on-one sessions, we conducted 30-minute semi-structured interviews with both therapists and clients. These interviews aimed to explore how \name{} supported art therapy homework in clients' daily lives (\textbf{RQ1}) and how therapists and clients collaborated surrounding art therapy homework (\textbf{RQ2}). We used feedback and homework outcomes from the trial period to ask targeted questions about their practices.
With participants' consent, we recorded and transcribed the brief 5-minute interviews and the 30-minute interviews for thematic analysis~\cite{braun2006using}. This analysis also included the personal messages shared by the participants about their on-the-spot experiences.
%we recorded and transcribed the interviews for thematic analysis. 
Two researchers then engaged in inductive coding, annotating transcripts to identify relevant quotes, key concepts, and patterns. They developed a detailed coding scheme through regular discussions, grouping quotes into a hierarchical structure of themes and sub-themes. Exemplar quotes were selected to represent each theme. We also used homework history (e.g., images or conversation data) and customization data (e.g., homework dialogue principle data) as evidences or examples to back up the findings in our thematic analysis.



% In between sessions, all homework history data~(e.g., artwork, creative process data and dialogue data) and history data on homework customization~(e.g., principles of conversational agents, homework tasks and personal messages) were stored in the database.
% In addition, we encouraged clients and therapists to send us images and comments about their experiences and feelings when using \name{} via an instant messaging app.
% For online in-sessions, the sessions were first audio- and video-recorded.
% At the end of each in-session, we conducted a brief 5-minute interview with the therapists to gather insights into their practices and feelings during the session.
% Upon concluding all the sessions, we conducted approximately 30-minute semi-structured interviews with both the therapists and the clients to explore how \name{} support art therapy homework in clients' daily settings~(\textbf{RQ1}), and how therapists tailored the homework and tracked the homework history surrounding art therapy homework~(\textbf{RQ2}). 
% Further, we employed the homework outcomes and feedback from both therapists and clients during the trial period as stimuli to ask specific questions about their practices. 

% With participants' consent, we recorded the interviews and transcribed them for thematic analysis~\cite{braun2006using}.
% Initially, two researchers engaged in collaborative inductive coding. They began by annotating the transcript to highlight relevant quotes, key concepts, and recurring patterns in the data. Through regular discussions, they expanded these insights into a detailed coding scheme that aligned with their research questions. The quotes were then systematically coded and grouped into a hierarchical structure of emerging themes, which were continuously reviewed and refined during recurring meetings. During these discussions, exemplar quotes were also chosen to represent each theme and sub-theme.
% We also gathered homework history and customization data, including artworks and conversation records from both therapists and clients, as evidence and examples to support the results of the thematic analysis.

\begin{figure*}[tb]
  \centering
  \includegraphics[width=\linewidth]{images/findings_1.png}
  \vspace{-7mm}
  \caption{Overview of The Homework Engagement of Clients with \name{}: (a) Homework Activity Date Distribution; (b) Accumulated Homework Activity Hourly Distribution of the Day; (c) Usage of AI Brushes in Artworks; 
  }
  \Description{Figure 5 contains three sub-figures. Figure 5a shows the Homework Activity Date Distribution for 24 clients over a four-week period, using seven different shades of purple to represent varying levels of participation in the homework sessions. Figure 5b illustrates the frequency of AI brush usage during clients' homework art-making, with the top 20 most frequently used brushes highlighted in larger font. Figure 5c depicts the distribution of homework sessions across different times of the day, revealing that clients tend to engage in homework sessions more frequently in the afternoon and evening.}
  \label{fig:quan_results}
\end{figure*}




\section{Conclusion }
This paper introduces the Latent Radiance Field (LRF), which to our knowledge, is the first work to construct radiance field representations directly in the 2D latent space for 3D reconstruction. We present a novel framework for incorporating 3D awareness into 2D representation learning, featuring a correspondence-aware autoencoding method and a VAE-Radiance Field (VAE-RF) alignment strategy to bridge the domain gap between the 2D latent space and the natural 3D space, thereby significantly enhancing the visual quality of our LRF.
Future work will focus on incorporating our method with more compact 3D representations, efficient NVS, few-shot NVS in latent space, as well as exploring its application with potential 3D latent diffusion models.

% for arxiv ver we remove impact
% \section*{Impact Statement}
This paper presents work whose goal is to advance the field of 
Machine Learning. There are many potential societal consequences 
of our work, none which we feel must be specifically highlighted here.

\bibliography{main}
\bibliographystyle{icml2025}


%%%%%%%%%%%%%%%%%%%%%%%%%%%%%%%%%%%%%%%%%%%%%%%%%%%%%%%%%%%%%%%%%%%%%%%%%%%%%%%
%%%%%%%%%%%%%%%%%%%%%%%%%%%%%%%%%%%%%%%%%%%%%%%%%%%%%%%%%%%%%%%%%%%%%%%%%%%%%%%
% APPENDIX
%%%%%%%%%%%%%%%%%%%%%%%%%%%%%%%%%%%%%%%%%%%%%%%%%%%%%%%%%%%%%%%%%%%%%%%%%%%%%%%
%%%%%%%%%%%%%%%%%%%%%%%%%%%%%%%%%%%%%%%%%%%%%%%%%%%%%%%%%%%%%%%%%%%%%%%%%%%%%%%
\section{Hard Threshold of EAC}\label{threshhold}
In constructing a weighted-gradient saliency map, the value of \(\gamma\) determines the number of the dimensions we select where important feature anchors are located. As the value of \(\gamma\) increases, the number of selected dimensions decreases, requiring the editing information to be compressed into a smaller space during the compression process. 
During compression, it is desired for the compression space to be as small as possible to preserve the general abilities of the model. However, reducing the compression space inevitably increases the loss of editing information, which reduces the editing performance of the model.
Therefore, to ensure editing performance in a single editing scenario, different values of \(\gamma\) are determined for various models, methods, and datasets. Fifty pieces of knowledge were randomly selected from the dataset, and reliability, generalization, and locality were measured after editing. The averages of these metrics were then taken as a measure of the editing performance of the model.
Table~\ref{value} presents the details of \(\gamma\), while Table~\ref{s} illustrates the corresponding editing performance before and after the introduction of EAC. $P_{x}$ denotes the value below which x\% of the values in the dataset.


\begin{table}[!htb]
\caption{The value of $\gamma$.}
\centering
\resizebox{0.45\textwidth}{!}{
\begin{tabular}{lcccc}
\toprule
\textbf{Datasets} & \textbf{Model} & \textbf{ROME} & \textbf{MEMIT} \\
\midrule
\multirow{2}{*}{\textbf{ZSRE}} & GPT-2 XL & $P_{80}$ & $P_{80}$ \\
 & LLaMA-3 (8B) & $P_{90}$ & $P_{95}$ \\
\midrule
\multirow{2}{*}{\textbf{COUNTERFACT}} & GPT-2 XL & $P_{85}$ & $P_{85}$ \\
 & LLaMA-3 (8B) & $P_{95}$ & $P_{95}$ \\
\bottomrule
\end{tabular}}
\label{value}
\end{table}


\begin{table}[!htb]
\caption{The value of $\gamma$.}
\centering
\resizebox{\textwidth}{!}{%
\begin{tabular}{lccccccccccccc}
\toprule
\multirow{1}{*}{Dataset} & \multirow{1}{*}{Method} & \multicolumn{3}{c}{\textbf{GPT-2 XL}} & \multicolumn{3}{c}{\textbf{LLaMA-3 (8B)}} \\
\cmidrule(lr){3-5} \cmidrule(lr){6-8}
& & \multicolumn{1}{c}{Reliability} & \multicolumn{1}{c}{Generalization} & \multicolumn{1}{c}{Locality} & \multicolumn{1}{c}{Reliability} & \multicolumn{1}{c}{Generalization} & \multicolumn{1}{c}{Locality} \\
\midrule
\multirow{1}{*}{ZsRE} & ROME & 1.0000 & 0.9112 & 0.9661 & 1.0000 & 0.9883 & 0.9600  \\
& ROME-EAC & 1.0000 & 0.8923 & 0.9560 & 0.9933 & 0.9733 & 0.9742  \\
\cmidrule(lr){2-8}
& MEMIT & 0.6928 & 0.5208 & 1.0000 & 0.9507 & 0.9333 & 0.9688  \\
& MEMIT-EAC & 0.6614 & 0.4968 & 0.9971 & 0.9503 & 0.9390 & 0.9767  \\
\midrule
\multirow{1}{*}{CounterFact} & ROME & 1.0000 & 0.4200 & 0.9600 & 1.0000 & 0.3600 & 0.7800  \\
& ROME-EAC & 0.9800 & 0.3800 & 0.9600 & 1.0000 & 0.3200 & 0.8800  \\
\cmidrule(lr){2-8}
& MEMIT & 0.9000 & 0.2200 & 1.0000 & 1.0000 & 0.3800 & 0.9500  \\
& MEMIT-EAC & 0.8000 & 0.1800 & 1.0000 & 1.0000 & 0.3200 & 0.9800  \\
\bottomrule
\end{tabular}%
}
\label{s}
\end{table}

\section{Optimization Details}\label{b}
ROME derives a closed-form solution to achieve the optimization:
\begin{equation}
\text{minimize} \ \| \widehat{W}K - V \| \ \text{such that} \ \widehat{W}\mathbf{k}_* = \mathbf{v}_* \ \text{by setting} \ \widehat{W} = W + \Lambda (C^{-1}\mathbf{k}_*)^T.
\end{equation}

Here \( W \) is the original matrix, \( C = KK^T \) is a constant that is pre-cached by estimating the uncentered covariance of \( \mathbf{k} \) from a sample of Wikipedia text, and \( \Lambda = (\mathbf{v}_* - W\mathbf{k}_*) / ( (C^{-1}\mathbf{k}_*)^T \mathbf{k}_*) \) is a vector proportional to the residual error of the new key-value pair on the original memory matrix.

In ROME, \(\mathbf{k}_*\) is derived from the following equation:
\begin{equation}
\mathbf{k}_* = \frac{1}{N} \sum_{j=1}^{N} \mathbf{k}(x_j + s), \quad \text{where} \quad \mathbf{k}(x) = \sigma \left( W_{f_c}^{(l^*)} \gamma \left( a_{[x],i}^{(l^*)} + h_{[x],i}^{(l^*-1)} \right) \right).
\end{equation}

ROME set $\mathbf{v}_* = \arg\min_z \mathcal{L}(z)$, where the objective $\mathcal{L}(z)$ is:
\begin{equation}
\frac{1}{N} \sum_{j=1}^{N} -\log \mathbb{P}_{G(m_{i}^{l^*}:=z))} \left[ o^* \mid x_j + p \right] + D_{KL} \left( \mathbb{P}_{G(m_{i}^{l^*}:=z)} \left[ x \mid p' \right] \parallel \mathbb{P}_{G} \left[ x \mid p' \right] \right).
\end{equation}

\section{Experimental Setup} \label{detail}

\subsection{Editing Methods}\label{EM}

In our experiments, Two popular editing methods including ROME and MEMIT were selected as baselines.

\textbf{ROME} \cite{DBLP:conf/nips/MengBAB22}: it first localized the factual knowledge at a specific layer in the transformer MLP modules, and then updated the knowledge by directly writing new key-value pairs in the MLP module.

\textbf{MEMIT} \cite{DBLP:conf/iclr/MengSABB23}: it extended ROME to edit a large set of facts and updated a set of MLP layers to update knowledge.

The ability of these methods was assessed based on EasyEdit~\cite{DBLP:journals/corr/abs-2308-07269}, an easy-to-use knowledge editing framework which integrates the released codes and hyperparameters from previous methods.

\subsection{Editing Datasets}\label{dat}
In our experiment, two popular model editing datasets \textsc{ZsRE}~\cite{DBLP:conf/conll/LevySCZ17} and \textsc{CounterFact}~\cite{DBLP:conf/nips/MengBAB22} were adopted.

\textbf{\textsc{ZsRE}} is a QA dataset using question rephrasings generated by back-translation as the equivalence neighborhood.
Each input is a question about an entity, and plausible alternative edit labels are sampled from the top-ranked predictions of a BART-base model trained on \textsc{ZsRE}.

\textbf{\textsc{CounterFact}} accounts for counterfacts that start with low scores in comparison to correct facts. It constructs out-of-scope data by substituting the subject entity for a proximate subject entity sharing a predicate. This alteration enables us to differentiate between superficial wording changes and more significant modifications that correspond to a meaningful shift in a fact. 

\subsection{Metrics for Evaluating Editing Performance}\label{Mediting performance}
\paragraph{Reliability} means that given an editing factual knowledge, the edited model should produce the expected predictions. The reliability is measured as the average accuracy on the edit case:
\begin{equation}
\mathbb{E}_{(x'_{ei}, y'_{ei}) \sim \{(x_{ei}, y_{ei})\}} \mathbf{1} \left\{ \arg\max_y f_{\theta_{i}} \left( y \mid x'_{ei} \right) = y'_{ei} \right\}.
\label{rel}
\end{equation}

\paragraph{Generalization} means that edited models should be able to recall the updated knowledge when prompted within the editing scope. The generalization is assessed by the average accuracy of the model on examples uniformly sampled from the equivalence neighborhood:
\begin{equation}
\mathbb{E}_{(x'_{ei}, y'_{ei}) \sim N(x_{ei}, y_{ei})} \mathbf{1} \left\{ \arg\max_y f_{\theta_{i}} \left( y \mid x'_{ei} \right) = y'_{ei} \right\}.
\label{gen}
\end{equation}

\paragraph{Locality} means that the edited model should remain unchanged in response to prompts that are irrelevant or the out-of-scope. The locality is evaluated by the rate at which the edited model's predictions remain unchanged compared to the pre-edit model.
\begin{equation}
\mathbb{E}_{(x'_{ei}, y'_{ei}) \sim O(x_{ei}, y_{ei})} \mathbf{1} \left\{ f_{\theta_{i}} \left( y \mid x'_{ei} \right) = f_{\theta_{i-1}} \left( y \mid x'_{ei} \right) \right\}.
\label{loc}
\end{equation}

\subsection{Downstream Tasks}\label{pro}

Four downstream tasks were selected to measure the general abilities of models before and after editing:
\textbf{Natural language inference (NLI)} on the RTE~\cite{DBLP:conf/mlcw/DaganGM05}, and the results were measured by accuracy of two-way classification.
\textbf{Open-domain QA} on the Natural Question~\cite{DBLP:journals/tacl/KwiatkowskiPRCP19}, and the results were measured by exact match (EM) with the reference answer after minor normalization as in \citet{DBLP:conf/acl/ChenFWB17} and \citet{DBLP:conf/acl/LeeCT19}.
\textbf{Summarization} on the SAMSum~\cite{gliwa-etal-2019-samsum}, and the results were measured by the average of ROUGE-1, ROUGE-2 and ROUGE-L as in \citet{lin-2004-rouge}.
\textbf{Sentiment analysis} on the SST2~\cite{DBLP:conf/emnlp/SocherPWCMNP13}, and the results were measured by accuracy of two-way classification.

The prompts for each task were illustrated in Table~\ref{tab-prompt}.

\begin{table*}[!htb]
% \small
\centering
\begin{tabular}{p{0.95\linewidth}}
\toprule

NLI:\\
\{\texttt{SENTENCE1}\} entails the \{\texttt{SENTENCE2}\}. True or False? answer:\\

\midrule

Open-domain QA:\\
Refer to the passage below and answer the following question. Passage: \{\texttt{DOCUMENT}\} Question: \{\texttt{QUESTION}\}\\

\midrule

Summarization:\\
\{\texttt{DIALOGUE}\} TL;DR:\\

\midrule


Sentiment analysis:\\
For each snippet of text, label the sentiment of the text as positive or negative. The answer should be exact 'positive' or 'negative'. text: \{\texttt{TEXT}\} answer:\\

\bottomrule
\end{tabular}
\caption{The prompts to LLMs for evaluating their zero-shot performance on these general tasks.}
\label{tab-prompt}
\end{table*}

\subsection{Hyper-parameters for Elastic Net}\label{hy}

In our experiment, we set \(\lambda = 5 \times 10^{-7} \), \(\mu = 5 \times 10^{-1} \) for GPT2-XL\cite{radford2019language} and \(\lambda = 5 \times 10^{-7} \), \(\mu = 1 \times 10^{-3} \) for LLaMA-3 (8B)\cite{llama3}.

\begin{figure*}[!hbt]
  \centering
  \includegraphics[width=0.5\textwidth]{figures/legend_edit.pdf}
  \vspace{-4mm}
\end{figure*}

\begin{figure*}[!hbt]
  \centering
  \subfigure{
  \includegraphics[width=0.23\textwidth]{figures/ROME-GPT2XL-CF-edit.pdf}}
  \subfigure{
  \includegraphics[width=0.23\textwidth]{figures/ROME-LLaMA3-8B-CF-edit.pdf}}
  \subfigure{
  \includegraphics[width=0.23\textwidth]{figures/MEMIT-GPT2XL-CF-edit.pdf}}
  \subfigure{
  \includegraphics[width=0.23\textwidth]{figures/MEMIT-LLaMA3-8B-CF-edit.pdf}}
  \caption{Edited on CounterFact, editing performance of edited models using the ROME~\cite{DBLP:conf/nips/MengBAB22} and MEMIT~\cite{DBLP:conf/iclr/MengSABB23} on GPT2-XL~\cite{radford2019language} and LLaMA-3 (8B)~\cite{llama3}, as the number of edits increases before and after the introduction of EAC.}
  \vspace{-4mm}
  \label{edit-performance-cf}
\end{figure*}

\begin{figure*}[!hbt]
  \centering
  \includegraphics[width=0.75\textwidth]{figures/legend.pdf}
  \vspace{-4mm}
\end{figure*}

\begin{figure*}[!htb]
  \centering
  \subfigure{
  \includegraphics[width=0.23\textwidth]{figures/ROME-GPT2XL-CounterFact.pdf}}
  \subfigure{
  \includegraphics[width=0.23\textwidth]{figures/ROME-LLaMA3-8B-CounterFact.pdf}}
  \subfigure{
  \includegraphics[width=0.23\textwidth]{figures/MEMIT-GPT2XL-CounterFact.pdf}}
  \subfigure{
  \includegraphics[width=0.23\textwidth]{figures/MEMIT-LLaMA3-8B-CounterFact.pdf}}
  \caption{Edited on CounterFact, performance on general tasks using the ROME~\cite{DBLP:conf/nips/MengBAB22} and MEMIT~\cite{DBLP:conf/iclr/MengSABB23} on GPT2-XL~\cite{radford2019language} and LLaMA-3 (8B)~\cite{llama3}, as the number of edits increases before and after the introduction of EAC.}
  \vspace{-4mm}
  \label{task-performance-cf}
\end{figure*}

\section{Experimental Results}\label{app}

\subsection{Results of Editing Performance}\label{cf-performance}
Applying CounterFact as the editing dataset, Figure~\ref{edit-performance-cf} presents the editing performance of the ROME~\cite{DBLP:conf/nips/MengBAB22} and MEMIT~\cite{DBLP:conf/iclr/MengSABB23} methods on GPT2-XL~\cite{radford2019language} and LLaMA-3 (8B)~\cite{llama3}, respectively, as the number of edits increases before and after the introduction of EAC.
The dashed line represents the ROME or MEMIT, while the solid line represents the ROME or MEMIT applying the EAC.


\subsection{Results of General Abilities}\label{cf-ability}
Applying CounterFact as the editing dataset, Figure~\ref{task-performance-cf} presents the performance on general tasks of edited models using the ROME~\cite{DBLP:conf/nips/MengBAB22} and MEMIT~\cite{DBLP:conf/iclr/MengSABB23} methods on GPT2-XL~\cite{radford2019language} and LLaMA-3 (8B)~\cite{llama3}, respectively, as the number of edits increases before and after the introduction of EAC. 
The dashed line represents the ROME or MEMIT, while the solid line represents the ROME or MEMIT applying the EAC.

\subsection{Results of Larger Model}\label{13 B}
To better demonstrate the scalability and efficiency of our approach, we conducted experiments using the LLaMA-2 (13B)~\cite{DBLP:journals/corr/abs-2307-09288}.
Figure~\ref{13B-edit} presents the editing performance of the ROME~\cite{DBLP:conf/nips/MengBAB22} and MEMIT~\cite{DBLP:conf/iclr/MengSABB23} methods on LLaMA-2 (13B)
~\cite{DBLP:journals/corr/abs-2307-09288}, as the number of edits increases before and after the introduction of EAC.
Figure~\ref{13B} presents the performance on general tasks of edited models using the ROME and MEMIT methods on LLaMA-2 (13B), as the number of edits increases before and after the introduction of EAC.
The dashed line represents the ROME or MEMIT, while the solid line represents the ROME or MEMIT applying the EAC.

\begin{figure*}[!hbt]
  \centering
  \includegraphics[width=0.5\textwidth]{figures/legend_edit.pdf}
  \vspace{-4mm}
\end{figure*}

\begin{figure*}[!hbt]
  \centering
  \subfigure{
  \includegraphics[width=0.23\textwidth]{figures/ROME-LLaMA2-13B-ZsRE-edit.pdf}}
  \subfigure{
  \includegraphics[width=0.23\textwidth]{figures/MEMIT-LLaMA2-13B-ZsRE-edit.pdf}}
  \subfigure{
  \includegraphics[width=0.23\textwidth]{figures/ROME-LLaMA2-13B-CF-edit.pdf}}
  \subfigure{
  \includegraphics[width=0.23\textwidth]{figures/MEMIT-LLaMA2-13B-CF-edit.pdf}}
  \caption{Editing performance of edited models using the ROME~\cite{DBLP:conf/nips/MengBAB22} and MEMIT~\cite{DBLP:conf/iclr/MengSABB23} on LLaMA-2 (13B)~\cite{DBLP:journals/corr/abs-2307-09288}, as the number of edits increases before and after the introduction of EAC.}
  \vspace{-4mm}
  \label{13B-edit}
\end{figure*}

\begin{figure*}[!hbt]
  \centering
  \includegraphics[width=0.75\textwidth]{figures/legend.pdf}
  \vspace{-4mm}
\end{figure*}

\begin{figure*}[!htb]
  \centering
  \subfigure{
  \includegraphics[width=0.23\textwidth]{figures/ROME-LLaMA2-13B-ZsRE.pdf}}
  \subfigure{
  \includegraphics[width=0.23\textwidth]{figures/MEMIT-LLaMA2-13B-ZsRE.pdf}}
  \subfigure{
  \includegraphics[width=0.23\textwidth]{figures/ROME-LLaMA2-13B-CounterFact.pdf}}
  \subfigure{
  \includegraphics[width=0.23\textwidth]{figures/MEMIT-LLaMA2-13B-CounterFact.pdf}}
  \caption{Performance on general tasks using the ROME~\cite{DBLP:conf/nips/MengBAB22} and MEMIT~\cite{DBLP:conf/iclr/MengSABB23} on LLaMA-2 (13B)~\cite{DBLP:journals/corr/abs-2307-09288}, as the number of edits increases before and after the introduction of EAC.}
  \vspace{-4mm}
  \label{13B}
\end{figure*}

\section{Analysis of Elastic Net}
\label{FT}
It is worth noting that the elastic net introduced in EAC can be applied to methods beyond ROME and MEMIT, such as FT~\cite{DBLP:conf/emnlp/CaoAT21}, to preserve the general abilities of the model.
Unlike the previously mentioned fine-tuning, FT is a model editing approach. It utilized the gradient to gather information about the knowledge to be updated and applied this information directly to the model parameters for updates.
Similar to the approaches of ROME and MEMIT, which involve locating parameters and modifying them, the FT method utilizes gradient information to directly update the model parameters for editing. Therefore, we incorporate an elastic net during the training process to constrain the deviation of the edited matrix.
Figure~\ref{ft} shows the sequential editing performance of FT on GPT2-XL and LLaMA-3 (8B) before and after the introduction of elastic net.
The dashed line represents the FT, while the solid line represents the FT applying the elastic net.
The experimental results indicate that when using the FT method to edit the model, the direct use of gradient information to modify the parameters destroys the general ability of the model. By constraining the deviation of the edited matrix, the incorporation of the elastic net effectively preserves the general abilities of the model.

\begin{figure*}[t]
  \centering
  \subfigure{
  \includegraphics[width=0.43\textwidth]{figures/legend_FT.pdf}}
\end{figure*}

\begin{figure*}[t]%[!ht]
  \centering
  \subfigure{
  \includegraphics[width=0.22\textwidth]{figures/FT-GPT2XL-ZsRE.pdf}}
  \subfigure{
  \includegraphics[width=0.22\textwidth]{figures/FT-GPT2XL-CounterFact.pdf}}
  \vspace{-2mm}
  \caption{Edited on the ZsRE or CounterFact datasets, the sequential editing performance of FT~\cite{DBLP:conf/emnlp/CaoAT21} and FT with elastic net on GPT2-XL before and after the introduction of elastic net.}
  \vspace{-2mm}
  \label{ft}
\end{figure*}


% You can have as much text here as you want. The main body must be at most $8$ pages long.
% For the final version, one more page can be added.
% If you want, you can use an appendix like this one.  

% The $\mathtt{\backslash onecolumn}$ command above can be kept in place if you prefer a one-column appendix, or can be removed if you prefer a two-column appendix.  Apart from this possible change, the style (font size, spacing, margins, page numbering, etc.) should be kept the same as the main body.

%%%%%%%%%%%%%%%%%%%%%%%%%%%%%%%%%%%%%%%%%%%%%%%%%%%%%%%%%%%%%%%%%%%%%%%%%%%%%%%
%%%%%%%%%%%%%%%%%%%%%%%%%%%%%%%%%%%%%%%%%%%%%%%%%%%%%%%%%%%%%%%%%%%%%%%%%%%%%%%


\end{document}


% This document was modified from the file originally made available by
% Pat Langley and Andrea Danyluk for ICML-2K. This version was created
% by Iain Murray in 2018, and modified by Alexandre Bouchard in
% 2019 and 2021 and by Csaba Szepesvari, Gang Niu and Sivan Sabato in 2022.
% Modified again in 2023 and 2024 by Sivan Sabato and Jonathan Scarlett.
% Previous contributors include Dan Roy, Lise Getoor and Tobias
% Scheffer, which was slightly modified from the 2010 version by
% Thorsten Joachims & Johannes Fuernkranz, slightly modified from the
% 2009 version by Kiri Wagstaff and Sam Roweis's 2008 version, which is
% slightly modified from Prasad Tadepalli's 2007 version which is a
% lightly changed version of the previous year's version by Andrew
% Moore, which was in turn edited from those of Kristian Kersting and
% Codrina Lauth. Alex Smola contributed to the algorithmic style files.