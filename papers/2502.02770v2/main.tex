%%%%%%%% ICML 2025 EXAMPLE LATEX SUBMISSION FILE %%%%%%%%%%%%%%%%%

\documentclass{article}

% Recommended, but optional, packages for figures and better typesetting:
\usepackage{microtype}
\usepackage{graphicx}
\usepackage{subfigure}
\usepackage{booktabs} % for professional tables
\usepackage[T1]{fontenc}

% hyperref makes hyperlinks in the resulting PDF.
% If your build breaks (sometimes temporarily if a hyperlink spans a page)
% please comment out the following usepackage line and replace
% \usepackage{icml2025} with \usepackage[nohyperref]{icml2025} above.
\usepackage{hyperref}


% Attempt to make hyperref and algorithmic work together better:
\newcommand{\theHalgorithm}{\arabic{algorithm}}

% Use the following line for the initial blind version submitted for review:

% \usepackage{icml2025}

% If accepted, instead use the following line for the camera-ready submission:
\usepackage[accepted]{icml2025}

% For theorems and such
\usepackage{amsmath}
\usepackage{amssymb}
\usepackage{mathtools}
\usepackage{amsthm}

% if you use cleveref..
\usepackage[capitalize,noabbrev]{cleveref}

%%%%%%%%%%%%%%%%%%%%%%%%%%%%%%%%
% THEOREMS
%%%%%%%%%%%%%%%%%%%%%%%%%%%%%%%%
\theoremstyle{plain}
\newtheorem{theorem}{Theorem}[section]
\newtheorem{proposition}[theorem]{Proposition}
\newtheorem{lemma}[theorem]{Lemma}
\newtheorem{corollary}[theorem]{Corollary}
\theoremstyle{definition}
\newtheorem{definition}[theorem]{Definition}
\newtheorem{assumption}[theorem]{Assumption}
\theoremstyle{remark}
\newtheorem{remark}[theorem]{Remark}

% Todonotes is useful during development; simply uncomment the next line
%    and comment out the line below the next line to turn off comments
%\usepackage[disable,textsize=tiny]{todonotes}
\usepackage[textsize=tiny]{todonotes}

% The \icmltitle you define below is probably too long as a header.
% Therefore, a short form for the running title is supplied here:
\icmltitlerunning{Twilight: Adaptive Attention Sparsity with Hierarchical Top-$p$ Pruning}

% Some additional packages
\usepackage{graphicx}
\usepackage{multirow}

% Collborators.
\newcommand{\cf}[1]{{\color{black} #1}}
\newcommand{\mingyu}[1]{\textcolor{purple}{[]}}
\newcommand{\jiaming}[1]{{\color{black} #1}}

\begin{document}

\twocolumn[
\icmltitle{Twilight: Adaptive Attention Sparsity with Hierarchical Top-$p$ Pruning}

% It is OKAY to include author information, even for blind
% submissions: the style file will automatically remove it for you
% unless you've provided the [accepted] option to the icml2025
% package.

% List of affiliations: The first argument should be a (short)
% identifier you will use later to specify author affiliations
% Academic affiliations should list Department, University, City, Region, Country
% Industry affiliations should list Company, City, Region, Country

% You can specify symbols, otherwise they are numbered in order.
% Ideally, you should not use this facility. Affiliations will be numbered
% in order of appearance and this is the preferred way.
% \icmlsetsymbol{equal}{*}

\begin{icmlauthorlist}
\icmlauthor{Chaofan Lin}{thu}
\icmlauthor{Jiaming Tang}{mit}
\icmlauthor{Shuo Yang}{ucb}
\icmlauthor{Hanshuo Wang}{sqz}
\icmlauthor{Tian Tang}{thu}
\icmlauthor{Boyu Tian}{thu}
\icmlauthor{Ion Stoica}{ucb}
%\icmlauthor{}{sch}
\icmlauthor{Song Han}{mit}
\icmlauthor{Mingyu Gao}{thu}
%\icmlauthor{}{sch}
%\icmlauthor{}{sch}
\end{icmlauthorlist}

\icmlaffiliation{thu}{Tsinghua
University}
\icmlaffiliation{mit}{MIT}
\icmlaffiliation{ucb}{UC Berkerly}
\icmlaffiliation{sqz}{Shanghai Qi Zhi Institute}
% \icmlcorrespondingauthor{Ion Stocia}{songhan@mit.edu}
% \icmlcorrespondingauthor{Song Han}{songhan@mit.edu}
\icmlcorrespondingauthor{Mingyu Gao}{gaomy@tsinghua.edu.cn}
% \icmlcorrespondingauthor{Firstname2 Lastname2}{first2.last2@www.uk}

% You may provide any keywords that you
% find helpful for describing your paper; these are used to populate
% the "keywords" metadata in the PDF but will not be shown in the document
\icmlkeywords{Large Language Model, Sparse Attention, Decode, KV Cache}

\vskip 0.3in
]

% this must go after the closing bracket ] following \twocolumn[ ...

% This command actually creates the footnote in the first column
% listing the affiliations and the copyright notice.
% The command takes one argument, which is text to display at the start of the footnote.
% The \icmlEqualContribution command is standard text for equal contribution.
% Remove it (just {}) if you do not need this facility.

\printAffiliationsAndNotice{}  % leave blank if no need to mention equal contribution
% \printAffiliationsAndNotice{\icmlEqualContribution} % otherwise use the standard text.

\begin{abstract}
Out-of-distribution (OOD) detection and OOD generalization are widely studied in Deep Neural Networks (DNNs), yet their relationship remains poorly understood. We empirically show that the degree of Neural Collapse (NC) in a network layer is inversely related with these objectives: stronger NC improves OOD detection but degrades generalization, while weaker NC enhances generalization at the cost of detection. This trade-off suggests that a single feature space cannot simultaneously achieve both tasks. To address this, we develop a theoretical framework linking NC to OOD detection and generalization. We show that entropy regularization mitigates NC to improve generalization, while a fixed Simplex Equiangular Tight Frame (ETF) projector enforces NC for better detection. Based on these insights, we propose a method to control NC at different DNN layers. In experiments, our method excels at both tasks across OOD datasets and DNN architectures. 

\begin{comment}   

Out-of-distribution (OOD) detection and OOD generalization are critical for deploying machine learning models in real-world scenarios. While substantial progress has been made in addressing these problems independently, few works have attempted to tackle them jointly. However, existing methods often rely on auxiliary OOD training data and primarily focus on covariate-shifted OOD data that share labels with in-distribution (ID) data. In contrast, we tackle the more realistic and challenging task of jointly detecting and generalizing to semantic OOD data with disjoint labels from the ID data, without auxiliary OOD training data.
Achieving both objectives simultaneously is inherently difficult due to a fundamental conflict — OOD generalization requires enhanced transferability, while OOD detection necessitates the inhibition of transfer.
To address this, we leverage insights from neural collapse (NC) — a phenomenon in deep networks where top-layer representations suppress feature variability and adopt a Simplex Equiangular Tight Frame (ETF) structure, impairing transferability. By controlling NC, we unify OOD detection and generalization: preventing NC enhances OOD transfer while inducing NC improves OOD detection.
Our proposed method excels at both tasks across various OOD datasets and architectures. 

\end{comment}


\end{abstract}
\section{Introduction}
\label{sec:intro}

Foundational models (FMs)~\cite{zhang2024data, zhou2023comprehensive} have shown remarkable progress in the healthcare domain, enabling professional-like assessment of disease diagnosis, treatment decision-making, and monitoring~\cite{zhang2023text, wang2022medclip, lu2023mi-zero}. 
Examples include LLaVA-Med~\cite{li2023llava}, Med-PaLM Multimodal~\cite{tu2024towards}, and Med-Flamingo~\cite{moor2023med}, have demonstrated their capacity on question answering, medical image analysis, and report generation.
These studies follow a predominant top-down model development strategy that requires upstream developers to collect data and train models for downstream tasks. 
Consequently, the developed model capabilities are heavily dependent on the training data, limiting their generalization performance in diverse clinical scenarios. 
For instance, Med-Gemini~\cite{yang2024advancing} reveals promising general capabilities in report generation while it lags behind state-of-the-art (SoTA) models on classification tasks, especially for out-of-domain applications. 
This indicates that while the generalizability of the foundation model is promising, more solutions are expected to meet the various specialized clinical needs.

To address these challenges, multi-center data centralization becomes essential to enhance model capacity and robustness across varied clinical scenarios~\cite{rajpurkar2022ai}. 
Centralizing distributed data can significantly improve model training and inference performance.
However, the process of medical data storage, transfer, and aggregation among centers requires extra efforts to ensure data security and system interoperability~\cite{bradford2020international}.
Moreover, a growing concern for patient privacy makes large-scale multi-center data sharing particularly challenging. 
While efforts like federated learning~\cite{wen2023survey, li2020review} can achieve good model performance on local data, the need for synchronized system coordination presents significant challenges, as clients are unable to update asynchronously. This limitation greatly restricts the practical capability of such approaches.
As a result, without a flexible collaboration, medical community still struggles to fully utilize the isolated data and local computation resources for comprehensive medical AI model development. 
To address this dilemma, open-source platforms encourage public data sharing and knowledge integration~\cite{markiewicz2021openneuro, zenodo}.
However, these platforms focus solely on raw data sharing while seldom providing collaborative model training or cooperation between different institutions.
Recently, collaborative learning has emerged as a viable approach for enhancing multi-model robustness~\cite{boulemtafes2020review}. 
For instance, software-like model development~\cite{raffel2023building} mimics software engineering practices by introducing structured workflows, enabling merging, version control, and continuous model integration.
Under this design, model ability can be strengthened with incremental knowledge updates similar to the version updating in software development. 

Although collaborative learning provides a multi-model collaboration, two key challenges remain in the leakage of raw data during collaboration~\cite{huang2023lorahub} and the synchronization of multiple collaborators~\cite{mcmahan2017communication} in the medical AI community. It is still challenging to integrate decentralized, privacy-sensitive data across institutions, leading to under-utilized insights and fragmented knowledge sharing~\cite{kaissis2020secure, rajpurkar2022ai, abdullah2021ethics}.
 To address these challenges, inspired by the collaborative software development, we propose \textbf{Med}ical \textbf{Fo}undation Models Me\textbf{rg}ing (\textbf{MedForge}), a cooperative workflow enabling continuously community-driven foundation model (FM) development.
MedForge enables a lightweight manner for individual centers to share their knowledge among multiple centers, minimizing the burden of data transmission and integration while enhancing model robustness.
Meanwhile, MedForge facilitates asynchronous and flexible collaboration, allowing individual centers to continuously update and improve medical FMs without the need for real-time synchronization.
Similar to open-source software development, MedForge incrementally updates medical knowledge and follows a sustainable model development scheme. 
This key design emphasizes a bottom-up construction of a multi-task medical FM, allowing downstream users to collaboratively build, refine, and update the upstream model according to their local resources. Our major contributions of MedForge are as below: 
\begin{enumerate}
    \item[$\bullet$] We introduce a collaborative workflow to promote the merging scheme of open-source software development. Our proposed MedForge allows distributed clinical centers to asynchronously contribute to comprehensive medical model construction while reducing transmitting costs among centers and avoiding the leakage of raw data, thus enhancing the utilization of private resources in the healthcare system. 
    \item[$\bullet$] We propose two effective knowledge-merging strategies for the asynchronous branch contribution. The MedForge-Fusion strategy updates the plugin module parameters of the main model during the merging phase, whereas the MedForge-Mixture strategy integrates the output of the plugin module by memorizing each contributor's coefficient. These strategies make MedForge more flexible and versatile. MedForge-Fusion is friendly to implement, while the MedForge-Mixture offers better performance and robustness.
    \item[$\bullet$]  We comprehensively evaluate model merging strategies to accumulate medical knowledge among multiple branch plugin modules. MedForge yields superior performance on medical classification tasks compared to other collaborative baselines across multiple datasets. We demonstrate the robustness of MedForge by shuffling the task order and evaluating various configurations of plugin modules and dataset distillation methods.
\end{enumerate}



\section{Related Works}

\subsection{Parental Homework Involvement in Education}

Parental homework involvement is commonly defined as \textit{parents' monitoring, supervision, and participation in their children's schoolwork and academic performance} \cite{pomerantz2007whom}. Research has shown that such involvement plays a crucial role in children's academic success, motivation, and well-being \cite{patall2008parent, dettmers2019antecedents, cooper1989synthesis}. While much of the literature focuses on Western contexts, studies in China highlight unique dynamics. Chinese parents often engage more directly, employing motivational strategies such as reasoning, monitoring, and even criticism to ensure academic success \cite{kim2013parents}. These behaviours align with broader cultural expectations in China, where academic achievement is highly valued, and parents feel a strong responsibility to support their children's education.

Some studies have also examined how Chinese parents adapted their involvement during the COVID-19 pandemic, where concerns over online education led to increased parental engagement \cite{wang2021parental}. Research has identified various types of parental involvement in Chinese families, ranging from supportive to disengaged, with the former most closely linked to academic success \cite{gan2019parental}. %These findings underscore the importance of understanding homework involvement in non-Western contexts.
%Further research has explored how Chinese parents adapted their involvement to specific contexts. For instance, Wang et al. \cite{wang2021parental} examined parental homework involvement during the COVID-19 pandemic, finding that Chinese parents, particularly mothers, became even more engaged in their children's learning due to concerns over the efficacy of online education. Similarly, Gan et al. \cite{gan2019parental} identified different types of parental involvement in Chinese families, ranging from supportive to disengaged, with supportive involvement being the most closely linked to academic success. 
These findings highlight the importance of understanding homework involvement in non-Western contexts.% how Chinese parents navigate the complexities of homework involvement in a rapidly changing educational landscape.

%Additionally, Lau et al. \cite{lau2011parental} and Liu et al. \cite{liu2019parental} examined parental involvement in younger children and its long-term impacts on academic readiness and emotional regulation. Their findings emphasize that early parental support in homework-related activities significantly influences children's long-term academic self-efficacy and emotional well-being, laying the groundwork for continued parental engagement during later school years.

The study of parental homework involvement typically relied on self-report surveys  (e.g., EMBU \cite{arrindell1999development}, QPH \cite{dumont2014quality}), interviews, and observations. While these methods provide valuable insights, they often fail to capture the nuanced emotions and behaviours that occur during real-world homework involvement, where conflicts may arise that are not evident in the presence of a human observer (due to the \textit{Hawthorne Effect} \cite{adair1984hawthorne}). Additionally, reliance on self-reported data introduces bias and may not fully reflect the subtleties of everyday involvement. For example, high-level categorizations of involvement, such as Gan et al.'s four types of involvement \cite{gan2019parental}, may miss the nuanced ways these interactions occur in day-to-day life.


\subsection{Emotional Experiences and Parent-child Conflicts During Homework Involvement}


While parental homework involvement is generally associated with positive educational outcomes, it can also lead to emotional strain and conflicts within families. Nnamani et al. \cite{nnamani2020impact} found that although parental involvement positively impacts students' emotional adjustment and academic performance, the emotional burden on parents often goes unnoticed. This emotional toll is particularly evident in cultures where academic success is strongly emphasized, as is the case in China. Kim et al. \cite{kim2020dyadic} developing Dyadic Mirror, a wearable smart mirror that provides parents with a second-person live-view of their own expressions as seen by their child during face-to-face interactions. Studies have shown that parents experience tension when balancing the desire to foster autonomy with the need to control their children's learning \cite{cunha2015parents}. The emotional stress parents feel during homework sessions can negatively affect children, creating a feedback loop of stress and conflict \cite{moe2018brief}. This dynamic is particularly significant in cultures where academic achievement is heavily emphasized, as in China \cite{nnamani2020impact}.


Homework involvement can exacerbate family conflicts, especially during adolescence, as parents strive to ensure academic success. Solomon et al. \cite{solomon2002helping} explored how homework involvement can become a source of conflict and found that the pressure parents feel to ensure their children's academic success can exacerbate tensions, often turning homework sessions into battlegrounds where unresolved issues about control and expectations surface. This finding is echoed by Suarez et al. \cite{suarez2022parental}, who reported high levels of family conflict and stress during the COVID-19 pandemic, a time when parental involvement in homework increased dramatically due to school closures and remote learning. 

Above findings underscore the complexity of parental homework involvement, where well-intentioned efforts to support academic achievement can inadvertently result in emotional strain and conflict. Our research aims to further unpack these emotional dynamics and explore how they are intertwined with parental behaviours and conflicts during homework involvement in Chinese families.


\subsection{Technology Supported Parent-Child Interaction}

Technological interventions in HCI have demonstrated the potential to enhance parent-child interactions in educational settings. Liu et al. \cite{liu2024he} explored the use of image-based generative AI in family expressive arts therapy. Fan et al. \cite{fan2019character} developed a tangible system for improving literacy in children with dyslexia, while Zhang et al. \cite{zhang2022storybuddy} introduced an AI-driven storytelling tool to balance parental involvement in learning activities. Although such technologies support collaborative learning, they have largely overlooked the specific challenges of homework involvement. Kalanadhabhatta et al. \cite{kalanadhabhatta2024playlogue} developed a dataset for analyzing adult-child conversations during play, demonstrating the potential of systematic conversation analysis in understanding parent-child interactions. 


In one of the few studies addressing this gap, Kerawalla et al. \cite{kerawalla2007exploring} examined a tablet-based platform designed to enhance parental understanding of classroom methods. This study is one of the few that addresses the role of technology in supporting homework parental involvement, showing the potential for digital tools to improve educational outcomes. Similarly, recent innovations like EduChat \cite{dan2023educhat}, an LLM-based educational chatbot, highlight the potential of AI in offering personalized support for both parents and children. Yu et al. \cite{yu2021parental} proposed a framework for parental mediation in children's use of creation-oriented educational media, and outlined three dimensions of mediation—creative, preparative, and administrative—offering insights for designing media that fosters creative learning while involving parents in the process. 



While technology-supported applications have made progress in facilitating parent-child interactions through storytelling, literacy development, and specific learning activities, a significant gap remains in addressing homework involvement, a crucial but underexplored aspect of parent-child interaction. Most research, focused on Western contexts \cite{pomerantz2007whom, patall2008parent, dettmers2019antecedents}, prioritizes academic outcomes, often overlook the unique emotional and behavioural dynamics that arise during homework involvement in non-Western cultures, particularly in China \cite{kim2013parents, gan2019parental, wang2021parental}. Additionally, reliance on self-report methods \cite{gan2019parental, patall2008parent} introduces bias and fails to capture real-time interactions.




Our study distinguishes itself in several ways: (1) we focus on the Chinese cultural context, shaped by unique parental expectations and pressures \cite{kim2013parents, suarez2022parental}; (2) unlike prior work that largely depends on subjective data, we utilize audio recordings of real-world homework sessions for a richer and more objective analysis; (3) by exploring the interplay of parental behaviours, emotions, and conflicts, we aim to deepen understanding of the complexities in homework involvement, contributing valuable insights to the family education research and designing technologies to improving parenting practices in China.

\section{Bringing Top-$p$ Sampling to Sparse Attention\label{sec:top_p}}

In this section, we formulate the current sparse attention methods and re-examine the root causes of the problems. We argue that to mathematically approximate the attention output, the goal is to select a minimal set of indices such that the sum of their attention scores meets a certain threshold. Therefore, we propose that top-$p$ sampling should be used instead of top-$k$ to filter out the critical tokens.

\subsection{Problem Formulation\label{sec:formulate}}

We start by formulating the sparse attention. Consider the attention computation during the decoding phase, where we have the query vector $Q \in \mathbb{R}^{1 \times d}$, and the key-value cache $K, V \in \mathbb{R}^{n \times d}$. Here, $d$ denotes the head dimension, and $n$ represents the context length. 
% The standard decoding attention can be formulated as
% \begin{equation}
%     O = \text{softmax} \bigg( \frac{Q \cdot K^T}{\sqrt{d}} \bigg) V = WV
% \end{equation}
% where $W \in \mathbb{R}^{1\times n}$ (sometimes denoted as $P$) represents the (normalized) attention weights. 
% Sparse attention, on the other hand, only loads a subset of tokens from the KV cache and computes the partial attention, which can be represented using a mask matrix.
\begin{definition}[Sparse Attention]
Let $\mathcal{I}$ be the set of selected indices, the output of the sparse attention equals to
\begin{equation}
    \hat{O} = \text{softmax} \bigg( \frac{Q \cdot K^T}{\sqrt{d}} \bigg) \Lambda_{\mathcal{I}} V = W \Lambda_{\mathcal{I}} V
\end{equation}
where $\Lambda_{\mathcal{I}} \in \mathbb{R}^{n \times n},
    \Lambda_{\mathcal{I}}[i, j] = 
    \begin{cases}
    1 & \text{if } i=j \text{ and } i \in \mathcal{I} \\
    0 & \text{otherwise}
    \end{cases}
$.
\end{definition}

% As mentioned above, there are two major types of top-$k$-based methods: static (query-agnostic) and dynamic (query-aware) methods. In our formulation, the difference lies on whether $\mathcal{I}$ depends on the query vector each time.

% \mingyu{Citation needed. Again, you need to add citations when you write the draft, not later. Because it is difficult for you to know where a citation is needed afterwards.}

\cf{
To minimize the output error $\Vert O-\hat{O}\Vert$, we need to carefully select the subset of tokens that are used in the sparse attention computation. However, directly optimizing this objective function without loading the full KV cache is challenging. Earlier research has shown that the distribution of V is relatively smooth \cite{atom}, which implies that the bound is relatively tight.
\begin{equation}
\label{eq:error}
\begin{aligned}
\mathcal{L} = \Vert O - \hat{O}\Vert &= \Vert W(\Lambda_\mathcal{I} - \mathbf{1}^{n \times n}) V\Vert \\
&\le \Vert W(\Lambda_\mathcal{I} - \mathbf{1}^{n \times n}) \Vert \cdot \Vert V \Vert
\end{aligned}
\end{equation}
Therefore, the objective becomes minimizing $\Vert W(\Lambda_{\mathcal{I}} - \mathbf{1}_{n\times n})\Vert = 1 - \sum_{i \in \mathcal{I}} W[i]$, which means selecting a subset of tokens that maximizes the sum of attention weights. If we fix the number of the subset, i.e. $|\mathcal{I}|$, then we have the oracle top-$k$ attention:

\begin{definition}[Oracle Top-$k$ Sparse Attention] Given the budget $B$,
\label{def:topk}
\begin{equation}
    \mathcal{I} = \arg\max_{\mathcal{I}} \sum_{i=1}^n W \Lambda_{\mathcal{I}}\ \ \ \text{s.t.} \ |\mathcal{I}| = B
\end{equation}
\end{definition}
}

The oracle top-$k$ attention serves as a theoretical upperbound of current sparse methods.

% \begin{figure*}[ht]
% \begin{center}
% \centerline{\includegraphics[width=2\columnwidth]{figures/distrib1.pdf}}
% \caption{Diverse distributions observed in attention weights. \textbf{The leftmost image} illustrates a "Flat" distribution \textbf{(Diffuse Attention)}, where the weights are uniformly distributed. \textbf{The middle image} depicts a "Peaked" distribution \textbf{(Focused Attention)}, where the weights are concentrated on the head and tail tokens. When overlaid, the differences between these distributions become readily apparent.}
% \label{fig:distrib}
% \end{center}
% \end{figure*}

\begin{figure*}[t]
  \centering
    \includegraphics[width=0.666\columnwidth]{figures/diffuse.pdf}
    \includegraphics[width=0.666\columnwidth]{figures/focus.pdf}
    \includegraphics[width=0.666\columnwidth]{figures/overlap.pdf}
  \caption{Diverse distributions observed in attention weights. \textbf{The leftmost image} illustrates a "Flat" distribution \textbf{(Diffuse Attention)}, where the weights are uniformly distributed. \textbf{The middle image} depicts a "Peaked" distribution \textbf{(Focused Attention)}, where the weights are concentrated on the head and tail tokens. When overlaid, the differences between these distributions become readily apparent.}
  \label{fig:distrib}
\end{figure*}

\subsection{Rethink the Problem of Top-$k$}
\cf{The Achilles’ heel of top-$k$ attention, as we described earlier, is the dilemma in determining a uniform budget $B$. A larger $B$ leads to inefficiency, while a smaller $B$ results in accuracy loss. We find that this predicament is quite similar to the one encountered in the sampling phase of large language models (LLMs). During the sampling phase, the model samples the final output token from a predicted probability distribution. Nucleus sampling \cite{holtzman2019curious}, or top-$p$ sampling, was proposed to address the problem that top-$k$ sampling cannot adapt to different next-word distributions.}

\cf{Motivated by this insight, we examine the distributions of attention weights more closely. \autoref{fig:distrib} displays two different types of attention weight distributions in real LLMs mentioned in \autoref{fig:teaser}. In \autoref{eq:error}, we demonstrated that the output error can be related to the sum of attention weights. It is straightforward to observe that, when comparing a flat distribution to a peaked one, a greater number of tokens must be selected in the flat distribution to reach the same cumulative threshold. Therefore, we argue that \textbf{the core reason for budget dynamism is the dynamic nature of attention weight distributions at runtime.} Drawing inspiration from top-$p$ sampling, we introduce top-$p$ sparse attention by directly apply threshold to the sum of attention weights.}

% We believe that \textbf{the core reason for the dynamic budgets is the dynamic distributions of attention weights at runtime}, which motivates us to replace top-$k$ in current sparse attention algorithm with top-$p$.


% However, we found that \texttt{top-$k$} \mingyu{why such style set?} using a fixed budget $B$, making it impossible for existing algorithms to leverage adaptive sparsity. 
% \mingyu{Grammar issue; this sentence misses the predicate (the verb).}
% As Figure \mingyu{?} shows, the distributions of attention weights vary across different attention heads, layers and queries. Using a uniformed $B$ leads to either inefficiency due to a larger $B$ or accuracy lose due to a smaller $B$. This brings challenges when we deploy these algorithms in serving systems, where


% \begin{figure}[ht]
% \begin{center}
% \centerline{\includegraphics[width=\columnwidth]{figures/distrib.png}}
% \caption{Different distributions which not only appear in next word probability distribution of LLM sampling, but also in attention weights distribution. For a “Flat” distribution, the weights are more uniform and top-$k$ should use a larger budget. For a “Peaked” distribution, the weights is more skewed and top-$k$ can achieve the same probability sum with a less budget.}
% \label{fig:distr}
% \end{center}
% \end{figure}

\begin{definition}[Oracle Top-$p$ Sparse Attention] Given the threshold $p$,
\label{def:topp}
\begin{equation}
    \mathcal{I} = \arg\min_{\mathcal{I}} |\mathcal{I}|\ \ \ \ \text{s.t.} \ \sum_{i=1}^n W \Lambda_{\mathcal{I}} \ge p
\end{equation}
\end{definition}

% Top-$p$ is inherently adaptive to different distributions since it directly goes to our ultimate goal, i.e. the sum of normalized attention weights.

Comparing to top-$k$, top-$p$ is advantageous because it provides a theoretical upperbound of error in \autoref{eq:error} by $(1-p) \cdot \Vert V \Vert$. Under this circumstance, top-$p$ reduces the budget as low as possible, making it both efficient and adaptive to different distributions.

% We evaluated the oracle top-$p$ in two respects: efficiency and its adaptive budget capability. For the former, we use cosine similarity as the metrics with a 10k retrieval prompt, as \autoref{fig:cos} shows. The relative error for top-$p$ with $p=0.95$ was found to be around the theoretical bound, proving its effective control of error. For top-$k$, the results were also good with $B=128$, but degradation was observed with $B=16$, indicating under-selection. We successfully observe the budget dyxnamism in \autoref{fig:dynamism}, which demonstrates that top-$p$ sparse attention can adaptively adjust attention sparsity at runtime. Given this capability, we will focus on using top-$p$ to address the budget problem in top-$k$ in the next section.

% Additionally, we found that real-practice top-$k$ methods, such as Quest \cite{tang2024quest}, performed worse than the oracle top-$k$ with the same budget, highlighting the limitations of fixed-budget approaches

% \mingyu{You have these many formal definitions and notations, but at the end you do not have a formal proof to show Def 3.3 is better than 3.2 to match the error minimization of Eq (4). While intuitively this is true, showing it explicitly would be better.}

% \begin{figure}[h]
% \begin{center}
% \centerline{\includegraphics[width=\columnwidth]{figures/cos.pdf}}
% \caption{Relative output error measured by cosine similarity in each layer.}
% \label{fig:cos}
% \end{center}
% \vskip -0.3in
% \end{figure}

% \begin{figure}[ht]
% \begin{center}
% \centerline{\includegraphics[width=\columnwidth]{figures/dynamism.pdf}}
% \caption{Dynamic budgets observed in oracle top-$p$ attention. We observe the dynamism across four dimensions: different \textbf{prompts (tasks)}, different \textbf{queries} with the same prompt, different \textbf{layers} in the same query, and different \textbf{heads} in the same layer.}
% \label{fig:dynamism}
% \end{center}
% \end{figure}
\section{Twilight}
% \cf{TODO(Chaofan): Section 4 needs refactor.}
\begin{figure*}[t]
\begin{center}
\centerline{\includegraphics[width=\columnwidth*2]{figures/arch.pdf}}
% \mingyu{Nits: to beautify, all figures in the paper should have similar font size \emph{after} embedded in the paper. That is, even you use the same font size when drawing them, you also need to make sure they have the same scaling ratio when included in the paper. Since the final absolute widths are fixed (single-column or double-column), this means you need to be careful how wide your figures should be when drawing them.}
\caption{Architecture of Twilight. Twilight is built on certain existing algorithm and serves as its optimizer. It computes self-attention in three steps. First, \textbf{Token Selector} select critical tokens using the strategy of base algorithm under a relaxed budget. Then, \textbf{Twilight Pruner} prunes the selected token indices via top-$p$ thresholding. Finally, the optimized token indices are passed to \textbf{Sparse Attention Kernel} to perform attention computation.}
\label{fig:arch}
\end{center}
\end{figure*}

In the previous section, we demonstrated that top-$p$ attention can adaptively control the budget while ensuring that the sum of normalized attention weights meets a certain threshold $p$. Our primary goal is to use top-$p$ to endow more existing algorithms with adaptive attention sparsity, rather than simply inventing another sparse attention, which is motivated by two main reasons: On one hand, despite their budget-related challenges, existing sparse algorithms have achieved significant success in current serving systems ~\cite{vllm, sglang}, thanks to their effective token selection strategies. These strategies can be readily reused and enhanced with adaptive sparsity.
On the other hand, we anticipate that future sparse attention methods may still employ top-$k$ selection. By developing a general solution like ours, we aim to automatically equip these future methods with adaptive attention sparsity, thereby improving their efficiency and adaptability without requiring extensive redesign. Consequently, we initially positioned our system, Twilight, as an \textbf{optimizer} for existing algorithms.

However, deploying top-$p$ to different existing sparse attention algorithms faces majorly three challenges, both algorithm-wise and system-wise.

% \mingyu{I think such challenges are better put at the end of the last section rather than here. They belong to the high-level discussions rather than our concrete designs, i.e., they are not unique to Twilight.}

\textbf{(C1) Not all algorithms are suitable for top-$p$.} Top-$p$ imposes strict constraints on the layout of attention weights. For example, simply replacing top-$k$ with top-$p$ in Quest \cite{tang2024quest} would not work, as Quest performs max pooling on weights with a per-page layout (16 tokens per page). Additionally, some other methods \cite{yang2024tidaldecodefastaccuratellm, liu2024retrievalattention} do not use attention weights to select critical tokens at all.

\textbf{(C2) It's harder to estimate weights for top-$p$ than top-$k$.} 
% Section \ref{sec:top_p} shows great potential of top-$p$ pruning method. However, either oracle top-$k$ or oracle top-$p$ attention can only save the load of $V$ since we need to load all keys to compute full attention weights \cite{sheng2023flexgen}. Numerous top-$k$ based methods dedicate to find ways to better estimate attention weights without loading full keys through either loading less channels \cite{ribar2023sparq, yang2024post, zhang2024pqcache} or less tokens \cite{tang2024quest}. To summarize, their core purpose is to represent the $K$ cache in a lower-precision way. For example, DS \cite{yang2024post} compresses $K$ cache to 2-bit with only half channels, therefore saving $1/16$ memory I/O. However, 
The precision requirement of top-$p$ is higher than that of top-$k$, because the former requires a certain degree of numerical accuracy while the latter only demands ordinality. 
% \mingyu{There is no explanation in the paper for this claim. The table also directly says so without reasons.} 
\autoref{table:prune_cmp} provides a basic comparison of top-$k$, top-$p$, and full attention. The precision requirement of top-$p$ attention lies between the other two, which leads us to reconsider the appropriate precision choice for compressing the $K$ cache.

\textbf{(C3) System-level optimizations are needed.} Since our work is the first work introduce top-$p$ to attention weights, many algorithms need to be efficiently implemented in hardware, including efforts on both efficient parallel algorithm designs and efficient kernel optimizations.
% \mingyu{This is a very vague discussion that offers little info to readers. Try to be more specific, e.g., what (extra or untraditional) computations are needed (e.g., a prefix sum?), and their impls are not widely explored, etc.}

In Section \ref{sec:hp}, we address \textbf{C1} by proposing a unified pruning framework for sparse attention. In 
 Section \ref{sec:kernel}, we mitigate the runtime overhead of by efficient kernel implementations (top-$p$, SpGEMV, Attention) and 4-bit quantization of $K$ cache, addressing \textbf{C2} and \textbf{C3}. Lastly, in Section \ref{sec:disc}, we analyze the overhead of Twilight and discuss some topics.
\begin{table}[t]
\caption{Comparing of different pruning methods on attention weights. "Normalize" indicates \texttt{softmax}.}
\label{table:prune_cmp}
% \mingyu{This table seems important, but none of the information in it has been explained anywhere in the paper. You cannot let the readers to derive these characteristics by themselves.}
\begin{center}
\begin{small}
\resizebox{\linewidth}{!}{
\begin{tabular}{ccccc}
\toprule
\textbf{Methods} & \textbf{Efficiency} & \textbf{Precision} & \textbf{Output} & \textbf{Need} \\
 & & \textbf{Requirement} & \textbf{Accuracy} & \textbf{Normalize?} \\
\midrule
Top-$k$ & High & Low  & Median & $\times$ \\
Top-$p$ & High & Median & High & $\surd$\\
Full Attn.   & Low  & High & High & $\surd$ \\
\bottomrule
\end{tabular}
}
\end{small}
\end{center}
\end{table}

\subsection{Hierarchical Pruning with Select-then-Prune Architecture\label{sec:hp}}

% Despite various designs in different algorithms we mentioned before, there is a core commonality we can observe from the formulation in Section \ref{sec:formulate} \textemdash most \mingyu{for emdash, you either have spaces both before and after (then you need to use \{\} after the command (like \textemdash{}) to ensure the space is not removed), or have no spaces both before and after.} sparse algorithms select a subset of tokens.

% Based on this, we abstract the base algorithm into a black-box \textbf{Token Selector}, which uses some metadata to select critical tokens. 
% \mingyu{I think you can improve this description like this. You start by reminding that existing algorithms have the difficulty of choosing a proper budget, either too conservative (good accuracy, bad efficiency) or too aggressive (bad accuracy, good efficiency). So we make it a two-step process like this. Now Token Selector only needs to be conservative, and we have a Pruner to help. This achieves both good accuracy and good efficiency, ...\\
% Essentially in this paragraph you are missing the point that why this would work. It works because the Selector is conservative. (A2) in the next paragraph touches on this, but it is a bit too late, and also not detailed enough to show how it resolves the tradeoff.}
% And we treat top-$p$ as a \textbf{Pruner}, which optimizes the indices dumped by Token Selector by further pruning unimportant tokens, consisting of the hierarchical \textbf{Select-then-Prune} architecture we propose as the left side of \autoref{fig:arch} illustrates. This hierarchical architecture also share commonalities to the LLM sampling stage, where we usually use a mixture of top-$k$ and top-$p$ to sample final token (Similarly, in LLM sampling, we also first apply top-$k$ then apply top-$p$ according to the implementation of open-source LLM engines like vLLM \cite{vllm}).

% Comparing to the original algorithm, inserting a pruner in the middle has the following advantages. \textbf{(A1) The pruner makes the algorithm efficient.} As we mentioned above, existing algorithms suffer from inefficiency and always mistakenly select some less important tokens. Top-$p$ pruning is more accurate than Token Selector which is based on top-$k$, which can further prune these tokens. \textbf{(A2) The pruner makes the algorithm adaptive to budget.} In this architecture, due to the presence of pruner, the Token Selector is allowed to use a very conservative budget like $\frac{1}{4}$ sparsity (which equals to $8192$ in $32k$ context length). Note that the latency of Token Selector is irrelevant to the budget in most algorithms. Then the pruner will automatically lower the budget across different heads, layers and queries, solving the problem that the budget is difficult to determine.

Recall that existing algorithms face the challenge of choosing a proper budget: either over-selection or under-selection. To address this, we propose a two-step process. We first abstract the base algorithm into a black-box \textbf{Token Selector}, which uses some metadata to select a subset of critical tokens. And we allow the Token Selector to a conservative budget (e.g. $1/4$ sparsity), as we have a \textbf{Pruner} after that to further optimize the selected indices by pruning unimportant tokens. This hierarchical \textbf{Select-then-Prune} architecture is illustrated on the left side of \autoref{fig:arch}.

% This design shares similarities with the LLM sampling stage, where a mixture of top-$k$ and top-$p$ is commonly used to sample the final token (similarly, in LLM sampling, we first apply top-$k$ and then top-$p$, as implemented in open-source LLM engines like vLLM \cite{vllm}).
% Compared to the original algorithm, inserting a pruner in the middle offers the following advantages:

% \textbf{(A1) The pruner makes the algorithm efficient.} Existing algorithms often suffer from inefficiency due to mistakenly selecting less important tokens. The top-$p$ pruner is more accurate than the Token Selector, which is based on top-$k$, allowing it to further prune these unnecessary tokens.
% \textbf{(A2) The pruner makes the algorithm adaptive to budget.} In this architecture, the Token Selector is allowed to use a very conservative budget, such as 41​ sparsity (equivalent to 8192 tokens in a 32k context length). Importantly, the latency of the Token Selector is largely independent of the budget in most algorithms. The pruner then dynamically adjusts the budget across different heads, layers, and queries, solving the problem of determining the optimal budget.

\subsection{Efficient Kernel Implementation\label{sec:kernel}}

\subsubsection{Efficient SpGEMV with 4-bit Quantization of Key Cache}

As previously analyzed, the precision requirement of top-$p$ lies between top-$k$ and full attention. For top-$k$, many works \cite{yang2024post, zhang2024pqcache} push the compression to a extreme low-bit (1-bit/2-bit). For full attention, SageAttention \cite{zhang2025sageattention} is proposed recently as a 8-bit accurate attention by smoothing $K$ and per-block quantization. In this work, we find 4-bit strikes a balance between accuracy and efficiency, making it the ideal choice to calculate estimated attention weights for top-$p$. And we implement an efficient sparse GEMV (SpGEMV) kernel based on FlashInfer.
% \cite{ye2025flashinfer} which loads $K$ in a scattered manner, aligning with the design of Paged $K$ cache \cite{vllm}. 
We maintain an extra INT4 asymmetrically quantized $K$ cache in GPU as \autoref{fig:arch} shows. The INT4 $K$ vectors are unpacked and dequantized in shared memory which reduces I/O between global memory and shared memory to at most $1/4$, leading to a considerable end-to-end speedup.

% \mingyu{I do not see much novelty from this part. Are we just using an existing impl from FlashInfer, or do we have some new contributions? If the former, this subsubsection should be put \emph{after} those in which we have contributions in this subsection. If the latter, explicitly highlight the contributions.}

\subsubsection{Efficient Top-$p$ Kernel via Binary Search}

\begin{algorithm}[tb]
   \caption{Top-$p$ via Binary Search}
   \label{alg:binary}
\begin{algorithmic}
   \STATE {\bfseries Input:} Normalized attention weights $W \in \mathbb{R}^{BS \times H \times N}$, Threshold of TopP $p$, Hyper-parameter $\epsilon$
   \STATE {\bfseries Output:} Indices $I$, Mask $M \in \{0, 1\}^{BS \times H \times N}$
   \STATE \textbf{Initialize:} $l = 0$, $r = \max(W)$, $m = (l + r) / 2$;
   \REPEAT
   \STATE $W_0 =\text{where}(W < m, 0.0, W)$;
   \STATE $W_1 =\text{where}(W \le l, \text{INF}, W)$;
   \STATE $W_2 =\text{where}(W > r, \text{-INF}, W)$;
   \STATE $s=\text{sum}(W_0)$; //\texttt{ Compute the sum over current threshold } $m$;
   \IF{$s \ge p$}
   \STATE $l = m$;
   \ELSE
   \STATE $r = m$;
   \ENDIF
   \UNTIL{$\max(W_2) - \min(W_1) \ge \epsilon$}
   \STATE Select indices $I$ or mask $M$ where $W \ge l$;
   \STATE \textbf{return }{$I$, $M$};
\end{algorithmic}
\end{algorithm}

As we mentioned before, our top-$p$ method is motivated by the top-$p$ sampling, which also takes up a portion of decode latency. Therefore, our efficient kernel is modified from the top-$p$ sampling kernel from FlashInfer \cite{ye2025flashinfer}, a high performance kernel library for LLM serving.

A brute-force way to do top-$p$ sampling is to sort the elements by a descending order and accumulate them until the sum meets the threshold, which is quite inefficient in parallel hardwares like modern GPU. Our kernel adopts a parallel-friendly binary search algorithm as \cref{alg:binary}.
% \mingyu{Fix: ref name is missing. Also, consistently use either ref or autoref or cref throughout the paper} described.

% However, since our scenario is a bit different from token sampling, there are still some problems when adapting this algorithm to Twilight. First, as our top-$p$ is performed on attention weights, with different shape and layout comparing with the logits distribution on vocabulary, we need to redesign the parallel strategy including blocks/threads launching. \mingyu{So this is the problem. What is your solution?}
% Second, we can fuse top-$p$ with GEMV kernel which reduces kernel launch time and reuse some median results such as maximum which are already computed in the softmax part.

% \subsubsection{Efficient Sparse Attention with Awareness of Head Dynamism}

% Top-$p$ pruner brings head-wise dynamic budgets, which also brings some system challenges especially in the attention kernel. Traditional Sparse(Paged) Attention kernel allocates uniformed computation resources to all heads, leading to computation inefficiency. 
% Other head-wise dynamic budget works also face the same challenge. DuoAttention \cite{xiao2024duo} packs the retrieval heads and streaming heads separately and computes them in two steps. AdaKV \cite{feng2025adakvoptimizingkvcache} adopts a flattened KV cache and reuses \texttt{flash\_attn\_varlen}. These methods are either not general enough or not efficient enough to be used in Twilight. 
% FlashInfer \cite{ye2025flashinfer} deeply investigates the load balancing problem, but only for requests with dynamic lengths. To build an attention kernel with awareness of head-wise dynamism, Twilight borrows the idea from AdaKV \cite{feng2025adakvoptimizingkvcache}, reusing the load balancing algorithm in FlashInfer by flattening the head dimension.
% \mingyu{I am not sure about whether the last sentence (just one sentence) is sufficiently clear to explain how we did it. It seems very abstract. But maybe this is the style of AI papers in contrast to system papers. Use your own judgment.}

\subsection{Overhead Analysis and Discussion\label{sec:disc}}

\textbf{Runtime Overhead.} The runtime of Twilight-optimized algorithm consists of three parts according to the pipeline in \autoref{fig:arch}: $T_{\text{Token Selector}} + T_{\text{Twight Pruner}} + T_{\text{Sparse Attention}}$. Comparing to the baseline without Twilight, our method introduces an extra latency term $T_{\text{Twight Pruner}}$ but reduces $T_{\text{Sparse Attention}}$ because it further reduces its I/O. Our hierarchical architecture naturally fits the hierarchical sparsity, where the number of tokens gradually decreases as the precision increases. Suppose the Token Selector has a $\frac{1}{16}$ sparsity, then the theoretical speed up can be formulated as 

$$
    \frac{N/16 + B_0}{N/16 + B_0/4 + B_1}
$$

where $B_0 = |I_0|$ is the budget of Token Selector, $B_1 = |I_1|$ is the budget after pruned by Twilight. Suppose $B_0 = N/4, B_1 = N/64$, then the speed up is approximately $2 \times$. Here we omit the overhead of top-$p$ since SpGEMV dominates the latency when $B_0 = N/8 \sim N/4$.

% \begin{figure}[ht]
% \begin{center}
% \centerline{\includegraphics[width=\columnwidth]{figures/time_io.pdf}}
% % \mingyu{The figure, e.g., its x and y axes and the several boxes, is not easy to understand. Maybe more explanation is needed.\\
% % Some of the texts are too small to read.}
% % \caption{A theoretical time cost model for Sparse Attention with Twilight.}
% \label{fig:time_model}
% \end{center}
% \end{figure}
% \mingyu{INT4 vs. int4 (and similarly FP16, fp16, etc.); be consistent. Check the whole paper}

% \textbf{Memory Overhead.} As \autoref{fig:arch} shows, Twilight introduces an extra INT4 quantized key cache, which brings a $\frac{1}{2} \times \frac{1}{4} = \frac{1}{8}$ extra KV cache memory overhead. However, this additional overhead doesn't appear in all cases. On one hand, some base algorithms also maintain an INT4 key cache like DS \cite{yang2024post}, which is already included in the metadata part. On the other hand, there are some recent efforts \cite{zhang2024sageattention2} explore the INT4 full attention. This brings chances that we directly involve the estimated attention weights calculated by INT4 key cache in the attention computation, which allows us not to maintain the original FP16 key cache. Moreover, there are some optimizations when GPU memory becomes a bottleneck, like offloading and selective quantization (Only maintain extra INT4 key cache for hot tokens), which we leave as future works.

\textbf{Integrate with Serving System.} Since our system design naturally aligns with PagedAttention \cite{vllm}, Twilight can be 
% \mingyu{``can be'' is something you have not done. So what is the implementation now? Have we already done the integration? Do we want to briefly discuss the implementation?} 
seamlessly integrated into popular serving systems like vLLM~\cite{vllm} and SGLang~\cite{sglang}. Prefix sharing and multi-phase attention \cite{lin2024parrot, sglang, zhu2024relayattention, ye-etal-2024-chunkattention, cascade-inference} also become common techniques in modern serving systems, which also fit Twilight since we use paged-level or token-level sparse operations and can achieve flexible computation flow.
\begin{table*}[t]
\centering
\tiny
\begin{tabular}{|M{1.2cm}|M{0.7cm}|M{1cm}|M{1cm}|M{1cm}|M{0.8cm}|M{1.2cm}|M{0.7cm}|M{1cm}|M{1cm}|M{1cm}|M{0.8cm}|}
\hline\hline
Model & \#GPU & \#Strategies & Search Time(/s) & Simulation Time(/s) & E2E Time(/s) & Model & \#GPU & \#Strategies & Search Time(/s) & Simulation Time(/s) & E2E Time(/s) \\ \hline
\multirow{4}{*}{Llama-2-7B} & 64 & 23348 & 0.06 & 49.7 & 51.0 & \multirow{4}{*}{Llama-2-13B} & 64 & 23400 & 0.05 & 58.1 & 59.3 \\ \cline{2-6} \cline{8-12} 
 & 256 & 14372 & 0.05 & 43.5 & 44.4 &  & 256 & 13552 & 0.03 & 49.9 & 50.8 \\ \cline{2-6} \cline{8-12} 
 & 1024 & 8856 & 0.04 & 41.8 & 42.2 &  & 1024 & 8920 & 0.02 & 51.0 & 51.7 \\ \cline{2-6} \cline{8-12} 
 & 4096 & 4700 & 0.03 & 33.0 & 33.2 &  & 4096 & 4720 & 0.02 & 44.1 & 44.3 \\ \hline
\multirow{4}{*}{Llama-2-70B} & 64 & 53264 & 0.1 & 68.8 & 75.0 & \multirow{4}{*}{Llama-3-8B} & 64 & 23348 & 0.05 & 48.3 & 49.6 \\ \cline{2-6} \cline{8-12} 
 & 256 & 31440 & 0.06 & 57.7 & 60.9 &  & 256 & 14372 & 0.04 & 42.0 & 42.8 \\ \cline{2-6} \cline{8-12} 
 & 1024 & 20152 & 0.05 & 57.4 & 59.6 &  & 1024 & 8856 & 0.03 & 40.9 & 41.3 \\ \cline{2-6} \cline{8-12} 
 & 4096 & 10948 & 0.04 & 63.2 & 65.0 &  & 4096 & 4700 & 0.03 & 32.7 & 32.9 \\ \hline
\multirow{4}{*}{Llama-3-70B} & 64 & 53264 & 0.1 & 66.8 & 71.8 & \multirow{4}{*}{GLM-67B} & 64 & 20528 & 0.04 & 19.3 & 20.6 \\ \cline{2-6} \cline{8-12} 
 & 256 & 31440 & 0.07 & 56.3 & 59.6 &  & 256 & 12132 & 0.03 & 16.6 & 17.4 \\ \cline{2-6} \cline{8-12} 
 & 1024 & 20152 & 0.05 & 55.5 & 57.6 &  & 1024 & 7948 & 0.02 & 16.9 & 17.3 \\ \cline{2-6} \cline{8-12} 
 & 4096 & 10948 & 0.04 & 62.4 & 63.4 &  & 4096 & 4196 & 0.02 & 21.3 & 21.5 \\ \hline
\multirow{2}{*}{GLM-130B} & 64 & 33540 & 0.06 & 22.4 & 52.4 & \multirow{2}{*}{GLM-130B} & 1024 & 11976 & 0.03 & 16.7 & 18.2 \\ \cline{2-6} \cline{8-12} 
 & 256 & 18776 & 0.04 & 17.2 & 19.4 &  & 4096 & 6040 & 0.02 & 19.2 & 20.1 \\ \hline\hline
\end{tabular}%
\caption{
    The search space and the time cost for \sysname on Heterogeneous GPUs.
  For the pictures of time cost, the light color without hatches represents the time spent searching, while the deep color with hatches represents the time spent simulating.
  We can observe that it only takes \sysname\ about 1 minute to complete the end-to-end simulation. 
}
\label{tab:exp:cost}
\end{table*}

\section{Experiments}\label{sec:exp}


%In this section, we first evaluate \sysname's cost model accuracy under different settings to build the basis for the search in \S\ref{sec:exp:accuracy}.
%We show the search space of \sysname, and the search time cost for the search in \S\ref{sec:exp:cost}.
%Then, t
To prove \sysname's optimal search ability on MegatronLM, we did a comparative analysis between \sysname\ and experts on MegatronLM in \S\ref{sec:exp:expert}.
%After that, we compare \sysname with existing auto-parallel frameworks, including Alpa, Galvatron, etc., in \S\ref{sec:exp:comparison}.
Finally, we evaluate \sysname to search for the finance-optimal plan under different settings in \S\ref{sec:exp:finance}.

%\subsection{Cost Model Accuracy}\label{sec:exp:accuracy}
%



\section{Cost Analysis}\label{sec:exp:cost}

\sssec{Method}.
We did a cost analysis to show the gap between the large search space and the search efficiency of the \sysname.
We selected Llama-2 models (7B, 13B, and 70B) with 64, 256, 1024, and 4096 GPUs.
Then, for all the settings, we implemented \sysname\ on it and recorded the searched strategy number along with the end-to-end time (search time and simulation time)


\sssec{Result}. As shown in Table \ref{tab:exp:cost}, the number of explored strategies grows exponentially with model size. For smaller models like Llama-7B, even with 4096 GPUs, the search space remains relatively small. However, for larger models such as Llama-70B, the search space nearly triples compared to Llama-7B under the same GPU configuration. The end-to-end time reveals that the simulation phase is the main bottleneck, which may take 1 minute to execute on average. While the search time only takes less than 1 second to execute on average. This highlights the need for optimizing the simulation process, particularly in large-scale settings, while \sysname’s search algorithm remains efficient and scalable across different configurations.




\begin{figure*}[thbp]
  \centering
    \subfloat{\includegraphics[width=0.4\textwidth]{figs/fig-expert-legend.pdf}}\\
    \addtocounter{subfigure}{-1}

    \begin{minipage}{\textwidth}
    {\centering{\hspace{2.8cm}A800\hspace{4cm}H100\hspace{4.2cm}H800}}
    \end{minipage}

    \raisebox{0.8cm}{\rotatebox[origin=c]{90}{Llama-2}}
    \subfloat[7B]{\includegraphics[width=0.106\textwidth]{figs/fig-expert-A800-llama2-7b.pdf}}
    \subfloat[13B]{\includegraphics[width=0.106\textwidth]{figs/fig-expert-A800-llama2-13b.pdf}}
    \subfloat[70B]{\includegraphics[width=0.106\textwidth]{figs/fig-expert-A800-llama2-70b.pdf}}
    \subfloat[7B]{\includegraphics[width=0.106\textwidth]{figs/fig-expert-H100-llama2-7b.pdf}}
    \subfloat[13B]{\includegraphics[width=0.106\textwidth]{figs/fig-expert-H100-llama2-13b.pdf}}
    \subfloat[70B]{\includegraphics[width=0.106\textwidth]{figs/fig-expert-H100-llama2-70b.pdf}}
    \subfloat[7B]{\includegraphics[width=0.106\textwidth]{figs/fig-expert-H800-llama2-7b.pdf}}
    \subfloat[13B]{\includegraphics[width=0.106\textwidth]{figs/fig-expert-H800-llama2-13b.pdf}}
    \subfloat[70B]{\includegraphics[width=0.106\textwidth]{figs/fig-expert-H800-llama2-70b.pdf}}
    \\
    \raisebox{0.8cm}{\rotatebox[origin=c]{90}{Llama-3}}
    \subfloat[8B]{\includegraphics[width=0.16\textwidth]{figs/fig-expert-A800-llama3-8b.pdf}}
    \subfloat[70B]{\includegraphics[width=0.16\textwidth]{figs/fig-expert-A800-llama3-70b.pdf}}
    \subfloat[8B]{\includegraphics[width=0.16\textwidth]{figs/fig-expert-H100-llama3-8b.pdf}}
    \subfloat[70B]{\includegraphics[width=0.16\textwidth]{figs/fig-expert-H100-llama3-70b.pdf}}
    \subfloat[8B]{\includegraphics[width=0.16\textwidth]{figs/fig-expert-H800-llama3-8b.pdf}}
    \subfloat[70B]{\includegraphics[width=0.16\textwidth]{figs/fig-expert-H800-llama3-70b.pdf}}
    \\
    \raisebox{0.8cm}{\rotatebox[origin=c]{90}{GLM}}
    \subfloat[67B]{\includegraphics[width=0.16\textwidth]{figs/fig-expert-A800-glm-67b.pdf}}
    \subfloat[130B]{\includegraphics[width=0.16\textwidth]{figs/fig-expert-A800-glm-130b.pdf}}
    \subfloat[67B]{\includegraphics[width=0.16\textwidth]{figs/fig-expert-H100-glm-67b.pdf}}
    \subfloat[130B]{\includegraphics[width=0.16\textwidth]{figs/fig-expert-H100-glm-130b.pdf}}
    \subfloat[67B]{\includegraphics[width=0.16\textwidth]{figs/fig-expert-H800-glm-67b.pdf}}
    \subfloat[130B]{\includegraphics[width=0.16\textwidth]{figs/fig-expert-H800-glm-130b.pdf}}
  \caption{
  We compare \sysname's searched optimal plan's throughput with expert's proposed plan's throughput in single-GPU setting.
  }
  \label{fig:expert:throughput}
  \vspace{-10pt}
\end{figure*}

\subsection{Mode-1: Comparison with Expert Plans}\label{sec:exp:expert}

\sssec{Method}.
To prove the \sysname's ability to search the optimal strategy on MegatronLM, we compared \sysname\ with an expert.
We first selected three models with different parameter sizes (7 model settings in total): Llama-2 (7B, 13B, and 70B), Llama-3 (8B, 70B), and GLM (67B, 130B).
Then, we offer 4 GPU number settings: 32, 128, 256, and 1024.
Next, we asked six experts to craft a parallel strategy for each setting (different models and different GPU settings, overall $7\times 4=28$ settings) based on their expert experience.
Each participant has over six years of industry machine learning service or training experience.
Then, we ran each of the six participants' parallel strategies for each setting on MegatronLM and picked the optimal one (one with the largest throughput) among the six expert-crated strategies as the expert-optimal strategy.
At last, we run \sysname\ to search the optimal parallel strategy automatically and compare the \sysname's parallel strategy's throughput with the expert-optimal parallel strategy's throughput.

\sssec{Results}.
As shown in Fig. \ref{fig:expert:throughput}, \sysname demonstrates its ability to automatically generate parallel strategies that match or exceed expert-tuned plans across various model configurations. This highlights \sysname's capability to generalize and optimize without manual intervention.

\par A key finding is that \sysname consistently matches or outperforms manually designed strategies, showing that its automated search can achieve results on par with domain experts. This adaptability extends across diverse hardware and model types, while specific setups often constrain expert-tuned plans. \sysname dynamically adjusts to different configurations, optimizing parallel strategies based on the specific training environment.

\par Another important observation is \sysname’s flexibility in combining different parallelism techniques—data, tensor, and pipeline. While expert strategies often focus on one type of parallelism, \sysname optimally balances multiple forms, leading to superior performance, especially for large-scale models. This hybrid approach is likely the key to future parallelism strategies, where flexibility and adaptation are critical.
%\subsection{Comparison with Other Schemes}\label{sec:exp:comparison}

\begin{table}[h!]
\centering
\caption{GPT-3 Model Specification}
\label{tab:gpt-3}
\begin{tabular}{ccccc}
\hline
\#params & Hidden size & \#layers & \#heads & \#gpus \\ \hline\hline
350M & 1024 & 24 & 16 & 1 \\ 
1.3B & 2048 & 24 & 32 & 4 \\ 
2.6B & 2560 & 32 & 32 & 8 \\ 
6.7B & 4096 & 32 & 32 & 16 \\ 
15B & 5120 & 48 & 32 & 32 \\ 
39B & 8192 & 48 & 64 & 64 \\ \hline\hline
\end{tabular}
\end{table}


\begin{table}[h!]
\centering
\caption{LLaMA Model Specification}
\label{tab:llama}
\begin{tabular}{ccccc}
\hline
\#params & Hidden size & \#layers & \#heads & \#gpus \\ \hline\hline
7B & 4096 & 32 & 32 & 8 \\
13B & 5120 & 40 & 40 & 16 \\
33B & 6656 & 60 & 52 & 32 \\
70B & 8192 & 80 & 64 & 64 \\ \hline\hline
\end{tabular}
\end{table}

\begin{table}[h!]
\centering
\caption{GShard MoE Model Specification}
\label{tab:moe}
\begin{tabular}{cccccc}
\hline
\#params & Hidden size & \#layers & \#heads & \#experts & \#gpus \\ \hline\hline
380M & 768 & 8 & 16 & 8 & 1 \\
1.3B & 768 & 16 & 16 & 16 & 4 \\
2.4B & 1024 & 16 & 16 & 16 & 8 \\
10B & 1536 & 16 & 16 & 32 & 16 \\
27B & 2048 & 16 & 32 & 48 & 32 \\
70B & 2048 & 32 & 32 & 64 & 64 \\ \hline\hline
\end{tabular}
\end{table}

\sssec{Models and training workflows}.
For our experiments, we target three types of models: GPT-3, LLaMA, and a Mixture of Experts (MoE) model. These models represent a range of architectures, from homogeneous to heterogeneous, providing a comprehensive evaluation of our parallelism strategies. 

\par \textbf{GPT-3} (see Table \ref{tab:gpt-3}) is a homogeneous Transformer-based language model comprising many stacked layers. Its model parallelization plan has been extensively studied and optimized in various research efforts. \textbf{LLaMA} (see Table \ref{tab:llama}) is another advanced Transformer-based model designed for language modeling, with a focus on efficiency and performance in both pre-training and fine-tuning phases. \textbf{MoE} models (see Table \ref{tab:moe}), such as GShard, combine dense and sparse architectures by incorporating a mixture of expert layers. These layers replace the feed-forward layers in every few Transformer layers, making them highly adaptable to different computational environments.

\par To study the scalability and efficiency of training large models, we follow standard machine learning practices by scaling the model size proportionally with the number of GPUs, as reported in Table 4. For GPT-3, we increase the hidden size and the number of layers concurrently with the number of GPUs, following the methodology used in previous studies. For the MoE model, we primarily increase the number of experts, which is crucial for leveraging the model's sparse architecture and optimizing performance across multiple GPUs. For LLaMA, we adjust the model's depth (number of layers) and width (hidden size) to ensure it scales effectively with the available GPU resources.

\par In each experiment, we adopt the recommended global batch size per established ML practices to maintain consistent statistical behavior across different model configurations. We then fine-tune the micro-batch size for each model and system configuration to maximize overall system performance, with gradient accumulation applied across micro-batches.

\sssec{Baselines}. For each model, we compare our system, \sysname, against strong baselines, including Alpa and Galvatron, and manually designed strategies using Megatron-LM.

\par \textbf{Alpa} is chosen as one of the baselines due to its automated parallelization capabilities, particularly for large-scale models. Alpa utilizes a combination of intra-operator and inter-operator parallelism to optimize the training process. We configure Alpa to its best settings by following the guidelines provided in their documentation and research papers. Alpa is known for its comprehensive strategy space, which includes various parallelism paradigms such as data parallelism, tensor parallelism, and pipeline parallelism.

\par \textbf{Galvatron} is another baseline we employ, noted for its efficient transformer training over multiple GPUs using automatic parallelism. Galvatron incorporates multiple popular parallelism dimensions and automatically discovers the most efficient hybrid parallelism strategy through a decision tree decomposition and a dynamic programming search algorithm. We perform a grid search to determine the optimal configurations for Galvatron, ensuring that we fully leverage its capabilities.

\par \textbf{Megatron-LM} serves as the manually designed baseline, specifically for GPT-like models. Megatron-LM v2 is a state-of-the-art system that combines data parallelism, pipeline parallelism, and manually designed operator parallelism (denoted as TMP). This combination is controlled by three integer parameters that specify the degrees of parallelism assigned to each technique. Following the guidance from their research, we conduct a thorough grid search of these parameters and report the best configuration results. While Megatron-LM is highly specialized for GPT-like models, it does not support other models in our evaluation due to its lack of flexibility in handling different architectures.

Our comparison does not include open-source systems like \textbf{FlexFlow} and \textbf{Tofu} due to their limitations. FlexFlow lacks support for essential operators such as layer normalization and mixed-precision operators, and Tofu only supports single-node execution and is not open-sourced. Given these theoretical and practical constraints, we do not expect FlexFlow or Tofu to outperform the state-of-the-art manual baselines in our evaluation.

In summary, our evaluation includes \sysname, Alpa for its automated strategy space, Galvatron for its efficient hybrid parallelism discovery, and manually tuned Megatron-LM for its specialization in GPT-like models. This comprehensive approach thoroughly compares different parallelism strategies and model architectures.

\sssec{Evaluation metrics}. We measure training throughput in our evaluation. We evaluate the system's weak scaling when increasing the model size and the number of GPUs. Following \cite{narayanan2021efficient}, we use the aggregated peta floating-point operations per second (PFLOPS) of the whole cluster as the metric. After proper warmup, we measure it by running a few batches with dummy data. All our results (including those in later sections) have a standard deviation within 0.5\%, so we skip the error bars in our figures.

\sssec{GPT-3 results}.
\textcolor{red}{To be done}

\sssec{Llama results}.
\textcolor{red}{To be done}

\sssec{MoE results}.
\textcolor{red}{To be done}

\subsection{Mode-2: Heterogeneous GPU Search}

\begin{figure}[t]
  \centering
    \subfloat{\includegraphics[width=0.48\textwidth]{figs/fig-heter-legend.pdf}}\\
    \addtocounter{subfigure}{-1}
    
    \subfloat[Llama-2-7B]{\includegraphics[width=0.16\textwidth]{figs/fig-heter-llama2-7b.pdf}}
    \subfloat[Llama-2-13B]{\includegraphics[width=0.16\textwidth]{figs/fig-heter-llama2-13b.pdf}}
    \subfloat[Llama-2-70B]{\includegraphics[width=0.16\textwidth]{figs/fig-heter-llama2-70b.pdf}}
    \\

    \subfloat[Llama-3-8B]{\includegraphics[width=0.24\textwidth]{figs/fig-heter-llama3-8b.pdf}}
    \subfloat[Llama-3-70B]{\includegraphics[width=0.24\textwidth]{figs/fig-heter-llama3-70b.pdf}}
    \\

    \subfloat[GLM-67B]{\includegraphics[width=0.24\textwidth]{figs/fig-heter-glm-67b.pdf}}
    \subfloat[GLM-130B]{\includegraphics[width=0.24\textwidth]{figs/fig-heter-glm-130b.pdf}}
  \caption{
  For the heterogeneous GPU search scene, we compare expert-designed strategies's throughput with \sysname-searched strategies.
  The results prove the that \sysname achieves better throughput in heterogeneous scene.
  }
  \label{fig:exp:heter}
\end{figure}

% Please add the following required packages to your document preamble:
% \usepackage{graphicx}
\begin{table}[t]
\centering
\resizebox{0.5\textwidth}{!}{%
\begin{tabular}{c|cccc}
\hline
Model & H100 & H800 & A800 & Heter. \\ \hline\hline
Llama-2-7B & 10148287 & 9024716 & 3966756 & 5240609 \\
Llama-2-13B & 5721253 & 4937998 & 2187876 & 3040095 \\
Llama-2-70B & 1233850 & 1174362 & 458719 & 654206 \\
Llama-3-8B & 9167338 & 7610698 & 3586433 & 4660743 \\
Llama-3-70B & 1129568 & 1079507 & 425660 & 626050 \\
GLM-67B & 1288107 & 1218933 & 483384 & 699978 \\
GLM-130B & 508377 & 491088 & 202137 & 300193 \\ \hline\hline
\end{tabular}%
}
\caption{
We compare heterogeneous GPU with single-GPU search's optimal strategies' throughput.
The experiment is conducted with 1024 GPUs.
And the heterogeneous GPU setting is activated with A800 and H100.
}
\label{tab:exp:heter}
\end{table}

\sssec{Method}.
To evaluate \sysname's performance in heterogeneous GPU environments, we conducted a comprehensive comparison of \sysname-searched strategies and expert-designed strategies under heterogeneous GPU configurations. 
We use \sysname in the two GPU-heterogeneous environments with Nvidia H100 and A800 activated for search.
Also, we follow the design of \S\ref{sec:exp:expert}, we recruit six experts to craft a heterogeneous parallel strategy for each setting, and we picked the optimal one as the expert-designed strategy.
We offer 4 GPU number settings: 64, 256, 1024, and 4096.

Besides that, we also compared the heterogeneous GPU setting with single GPU setting in the same GPU number setting (1024).
We compare the throughput between the different settings (only A100, H100, H800, and heterogeneous settings)

\sssec{Results}.
As shown in Fig. \ref{fig:exp:heter}, our experiments reveal that \sysname consistently achieves higher throughput than expert-tuned configurations, particularly with larger models. \sysname’s approach dynamically balances data, tensor, and pipeline parallelism across heterogeneous GPUs, a task often challenging for manual tuning. This adaptability highlights the efficiency of automated strategies, especially in cloud-based or distributed environments where GPU types may vary. Overall, \sysname’s heterogeneous GPU search framework offers a scalable, cost-effective solution for optimizing model training in heterogeneous hardware contexts.

Table \ref{tab:exp:heter} shows the heterogeneous GPU setting compared with a single GPU setting.
Though a heterogeneous GPU setting strategy can not beat the performance of a single-GPU setting strategy, \sysname's searched strategy can nearly match with them.
\subsection{Mode-3: Evaluation Performance on Financial Cost}\label{sec:exp:finance}

%\sssec{Models and training workflows}.

\sssec{Search pools for GPU}. To comprehensively evaluate the financial cost performance of \sysname, we incorporate a variety of GPU types commonly used by major cloud service providers. Our search pools include the following GPU models: NVIDIA H100, A800 and H800.

These GPUs represent a range of performance capabilities and costs, providing a realistic and comprehensive basis for evaluating the financial efficiency of our system. By including these diverse GPU options, we can simulate the decision-making process of users who leverage cloud-based GPU resources, allowing us to optimize for both time and financial cost under various configurations.

\begin{figure}[t]
  \centering
    \subfloat[Per Throu. Llama-70B]{\includegraphics[width=0.24\textwidth]{figs/fig-money-per-Llama-2-70B.pdf}}
    \subfloat[Overall Throu. Llama-70B]{\includegraphics[width=0.24\textwidth]{figs/fig-money-all-Llama-2-70B.pdf}}
    \\
    \subfloat[Per Throu. GLM-67B]{\includegraphics[width=0.24\textwidth]{figs/fig-money-per-GLM-67B.pdf}}
    \subfloat[Overall Throu. GLM-67B]{\includegraphics[width=0.24\textwidth]{figs/fig-money-all-GLM-67B.pdf}}
    \\
    \subfloat[Per Throu. GLM-130B]{\includegraphics[width=0.24\textwidth]{figs/fig-money-per-GLM-130B.pdf}}
    \subfloat[Overall Throu. GLM-130B]{\includegraphics[width=0.24\textwidth]{figs/fig-money-all-GLM-130B.pdf}}
  \caption{
  We list the optimal line of \sysname.
  }
  \label{fig:money}
\end{figure}
\section{Conclusion}
We introduced \methodname, an effective training framework defending against MIAs for LLMs. The extensive experiments demonstrate its robustness in protecting privacy while maintaining strong language modeling performance across various datasets and architectures. Although our study focuses on fine-tuning due to computational constraints, \methodname can be seamlessly applied to large-scale pretraining, as done in prior selective pretraining work~\cite{lin2024not}. By categorizing tokens and treating them appropriately, \methodname opens a novel pathway for MIA defense. Future work can explore improved token selection strategies and multi-objective training approaches.
% for arxiv ver we remove impact
% \input{content/7-acknowledgement-and-impact}

\bibliography{main}
\bibliographystyle{icml2025}


%%%%%%%%%%%%%%%%%%%%%%%%%%%%%%%%%%%%%%%%%%%%%%%%%%%%%%%%%%%%%%%%%%%%%%%%%%%%%%%
%%%%%%%%%%%%%%%%%%%%%%%%%%%%%%%%%%%%%%%%%%%%%%%%%%%%%%%%%%%%%%%%%%%%%%%%%%%%%%%
% APPENDIX
%%%%%%%%%%%%%%%%%%%%%%%%%%%%%%%%%%%%%%%%%%%%%%%%%%%%%%%%%%%%%%%%%%%%%%%%%%%%%%%
%%%%%%%%%%%%%%%%%%%%%%%%%%%%%%%%%%%%%%%%%%%%%%%%%%%%%%%%%%%%%%%%%%%%%%%%%%%%%%%
\newpage
\appendix
\onecolumn
\section{Full Results on Longbench}
\label{appendix}
% \renewcommand{\arraystretch}{1.2} % 设置行高
\begin{table*}[ht]
\setlength{\tabcolsep}{2.5pt} % 设置列间距
\caption{\textbf{Result on Longbench.} The highest score in each task is marked in bold (except for "Full"). We also note the relative error of Twilight when integrated with the corresponding base algorithm. Green indicates an increase in score, while red indicates a decrease.}
\label{table:longbench}
    \centering
    \scalebox{0.69}{
    \begin{tabular}{lcccccccccccccc}
        \toprule
        \multirow{2}*{\textbf{Methods}} &
        \multirow{2}*{\textbf{Budget }} &
        \multicolumn{2}{c}{\textbf{Single-Doc. QA}} & \multicolumn{3}{c}{\textbf{Multi-Doc. QA}} & \multicolumn{3}{c}{\textbf{Summarization}} & \multicolumn{1}{c}{\textbf{Few-shot}} & \multicolumn{2}{c}{\textbf{Code}} & \multicolumn{1}{c}{\textbf{Synthetic}} & \multirow{2}*{\textbf{Avg. Score}}  \\
        \cmidrule(lr){3-4}\cmidrule(lr){5-7}\cmidrule(lr){8-10} \cmidrule(lr){11-11} \cmidrule(lr){12-13} \cmidrule(lr){14-14} 
        & & \textit{Qasper} & \textit{MF-en} & \textit{HotpotQA} & \textit{2WikiMQA} &  \textit{Musique} & \textit{GovReport} & \textit{QMSum} & \textit{MultiNews} & \textit{TriviaQA} &  \textit{LCC} & \textit{Repobench-P} & \textit{PR-en} \\
        \midrule
        \multicolumn{15}{c}{\textsc{Longchat-7B-32k}} \\
        \midrule
        \multirow{2}*{Full} & 32k & 29.48 & 42.11 & 30.97 & 23.74 & 13.11 & 31.03 & 22.77 & 26.09 & 83.25 & 30.50 & 52.70 & 55.62 & 36.78 \\
         & \textbf{Twilight (Avg. 146)} & 31.74 & \textbf{43.91} & 33.59 & \textbf{25.65} & \textbf{13.93} & 32.19 & \textbf{23.15} & 26.30 & 85.14 & 34.50 & 54.98 & 57.12 & 38.52\textcolor{teal}{(+4.7\%)}\\
        \midrule
        \multirow{5}*{Quest}
         & 256 & 26.00 & 32.83 & 23.23 & 22.14 & 7.45 & 22.64 & 20.98 & 25.05 & 67.40 & 33.60 & 48.70 & 45.07 & 31.26 \\
      & 1024 & 31.63 & 42.36 & 30.47 & 24.42 & 10.11 & 29.94 & 22.70 & 26.39 & 84.21 & 34.5 & 51.52 & 53.95 & 36.85 \\
       & 4096 & 29.77 & 42.71 & 32.94 & 23.94 & 13.24 & 31.54 & 22.86 & 26.45 & 84.37 & 31.50 & 53.17 & 55.52 & 37.33 \\
        & 8192 & 29.34 & 41.70 & 33.27 & 23.46 & 13.51 & 31.18 & 23.02 & 26.48 & 84.70 & 30.00 & 53.02 & 55.57 & 37.10 \\
             & \textbf{Twilight (Avg. 131)} & 31.95 & 43.28 & 31.62 & 24.87 & 13.48 & \textbf{32.21} & 22.79 & 26.33 & 84.93 & 33.50 & 54.86 & 56.70 & 38.04\textcolor{teal}{(+2.5\%)} \\
        \midrule
    \multirow{5}*{DS}
         & 256 & 28.28 & 39.78 & 27.10 & 20.75 & 9.34 & 29.68 & 21.79 & 25.69 & 83.97 & 32.00 & 52.01 & 53.44 & 35.32 \\
      & 1024 & 30.55 & 41.27 & 30.85 & 21.87 & 7.27 & 26.82 & 22.95 & 26.51 & 83.22 & 31.50 & 53.23 & 55.50 & 35.96 \\
       & 4096 & 28.95 & 41.90 & 32.52 & 23.65 & 8.07 & 29.68 & 22.75 & \textbf{26.55} & 83.34 & 30.00 & 52.77 & 55.48 & 36.31 \\
        & 8192 & 29.05 & 41.42 & 31.79 & 22.95 & 12.50 & 30.44 & 22.50 & 26.43 & 83.63 & 30.50 & 52.87 & 55.33 & 36.62 \\
             & \textbf{Twilight (Avg. 126)} & \textbf{32.34} & 43.89 & \textbf{34.67} & 25.43 & 13.84 & 31.88 & 23.01 & 26.32 & \textbf{85.29} & \textbf{35.50} & \textbf{55.03} & \textbf{57.27} & \textbf{38.71}\textcolor{teal}{(+5.7\%)} \\
        \midrule
        \multicolumn{15}{c}{\textsc{Llama-3.1-8B-Instruct}} \\
        \midrule
        \multirow{2}*{Full} & 128k & 46.17 & 53.33 & 55.36 & 43.95 & 27.08 & 35.01 & 25.24 & 27.37 & 91.18 & 99.50 & 62.17 & 57.76 & 52.01 \\
         & \textbf{Twilight (Avg. 478)} & 43.08 & 52.99 & 52.22 & 44.83 & 25.79 & 34.21 & \textbf{25.47} & 26.98 & 91.85 & \textbf{100.00} & \textbf{64.06} & 58.22 & 51.64\textcolor{red}{(-0.7\%)} \\
        \midrule
        \multirow{5}*{Quest}
         & 256 & 24.65 & 37.50 & 30.12 & 23.60 & 12.93 & 27.53 & 20.11 & 26.59 & 65.34 & 95.00 & 49.70 & 45.27 & 38.20 \\
      & 1024 & 38.47 & 49.32 & 47.43 & 38.48 & 20.59 & 33.71 & 23.67 & 26.60 & 81.94 & 99.50 & 60.78 & 52.96 & 47.79 \\
       & 4096 & 43.97 & 53.64 & 51.94 & 42.54 & 24.00 & 34.34 & 24.36 & 26.75 & 90.96 & 99.50 & 62.03 & 55.49 & 50.79 \\
        & 8192 &\textbf{44.34} & 53.25 & 54.72 & 44.84 & \textbf{25.98} & 34.62 & 24.98 & 26.70 & 91.61 & \textbf{100.00} & 62.02 & 54.20 & 51.44 \\
         & \textbf{Twilight (Avg. 427)} & 43.44 & 53.2 & 53.77 & 43.56 & 25.42 & 34.39 & 25.23 & 26.99 & 91.25 & 100.0 & 63.55 & 58.06 & 51.57\textcolor{teal}{(+0.3\%)} \\
        \midrule
    \multirow{5}*{DS}
         & 256 & 38.24 & 49.58 & 43.38 & 31.98 & 15.52 & 33.40 & 24.06 & 26.86 & 84.41 & 99.50 & 53.28 & 48.64 & 45.74 \\
      & 1024 & 42.97 & \textbf{54.65} & 51.75 & 33.92 & 20.39 & 34.50 & 24.92 & 26.71 & \textbf{92.81} & 99.50 & 62.66 & 48.37 & 49.43 \\
       & 4096 & 43.50 & 53.17 & 54.21 & 44.70 & 23.14 & \textbf{34.73} & 25.40 & 26.71 & 92.78 & 99.50 & 62.59 & 51.31 & 50.98 \\
        & 8192 & 43.82 & 53.71 & 54.19 & \textbf{45.13} & 23.72 & 34.27 & 24.98 & 26.69 & 91.61 & \textbf{100.00} & 62.40 & 52.87 & 51.14 \\
             & \textbf{Twilight (Avg. 446)} & 43.08 & 52.89 & \textbf{54.68} & 44.86 & 24.88 & 34.09 & 25.20 & \textbf{27.00} & 91.20 & \textbf{100.00} & 63.95 & \textbf{58.93} & \textbf{51.73}\textcolor{teal}{(+1.2\%)} \\
\bottomrule
\end{tabular}
}
\end{table*}

% You can have as much text here as you want. The main body must be at most $8$ pages long.
% For the final version, one more page can be added.
% If you want, you can use an appendix like this one.  

% The $\mathtt{\backslash onecolumn}$ command above can be kept in place if you prefer a one-column appendix, or can be removed if you prefer a two-column appendix.  Apart from this possible change, the style (font size, spacing, margins, page numbering, etc.) should be kept the same as the main body.

%%%%%%%%%%%%%%%%%%%%%%%%%%%%%%%%%%%%%%%%%%%%%%%%%%%%%%%%%%%%%%%%%%%%%%%%%%%%%%%
%%%%%%%%%%%%%%%%%%%%%%%%%%%%%%%%%%%%%%%%%%%%%%%%%%%%%%%%%%%%%%%%%%%%%%%%%%%%%%%


\end{document}


% This document was modified from the file originally made available by
% Pat Langley and Andrea Danyluk for ICML-2K. This version was created
% by Iain Murray in 2018, and modified by Alexandre Bouchard in
% 2019 and 2021 and by Csaba Szepesvari, Gang Niu and Sivan Sabato in 2022.
% Modified again in 2023 and 2024 by Sivan Sabato and Jonathan Scarlett.
% Previous contributors include Dan Roy, Lise Getoor and Tobias
% Scheffer, which was slightly modified from the 2010 version by
% Thorsten Joachims & Johannes Fuernkranz, slightly modified from the
% 2009 version by Kiri Wagstaff and Sam Roweis's 2008 version, which is
% slightly modified from Prasad Tadepalli's 2007 version which is a
% lightly changed version of the previous year's version by Andrew
% Moore, which was in turn edited from those of Kristian Kersting and
% Codrina Lauth. Alex Smola contributed to the algorithmic style files.