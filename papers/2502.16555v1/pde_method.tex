%\subsection{Timescales analysis}
%Here, we present timescales analysis for \Cref{eq:glow_hybrid,eq:fluid_model_eqs}. \Cref{tab:ts_analysis}~summarizes different ion and electron timescales for a RF-GDP with $p_0=1$ Torr.  Considering the plasma oscillation frequencies for electron $\omega_{e} = \sqrt{q_e^2 n_e /m_e/\epsilon_0}$ and ions $\omega_{i} = \sqrt{q_e^2 n_i /m_i/\epsilon_0}$ we can write $\omega_{e} / \omega_{i} = \of{m_i / m_e}^{1/2} \of{n_e / n_i}^{1/2}$. For GDPs, typically $n_e\sim n_i \sim 10^{17}$, $f_{e} \approx 8.9 \times 10^{9}$ $s^{-1}$ and $f_{i} \approx 3.3 \times 10^{7}$ $s^{-1}$. Electrons have faster time scales than ions due to the small mass ratio between electrons and ions, $m_e/m_i = 1.36\times 10^{-5}$. The input voltage oscillation period is set to $7.374 \times 10^{-7}$ s for all RF-GDP cases considered in this paper.
%\begin{table}[htb]
%	\centering
%	\resizebox{\textwidth}{!}{
%	\begin{tabular}{||c|c|c|c||}
%		\hline 
%		Formulation & Species  & Term & $\Delta t$ [s]\\
%		\hline
%		\multirow{6}{*}{Fluid} & \multirow{3}{*}{Ions} 		& $\nabla_{\vect{x}} \cdot \of{\mu_i \vect{E}_{\text{}} n_i}$ -- advection 		 & $\frac{\Delta x}{\mu_i \norm{\vect{E}_{\text{max}}}} \approx \mathcal{O}(10^{-9}) $ \\ [0.05cm]		
%							   &  					   		& $\nabla_{\vect{x}} \cdot \of{D_i \nabla_{\vect{x}} n_i}$ -- diffusion  & $\frac{\Delta x^2}{ 2 D_i } \approx \mathcal{O}(10^{-7}) $  \\ [0.05cm]
%							   &                       		& $k_i n_0 n_e$ -- reaction   											 & $ \frac{\Delta n_i}{k_{i,\text{max}} n_0 n_e} \approx \mathcal{O}(10^{-9})$   \\ [0.05cm]
%							   \cline{2-4}
%							   & \multirow{3}{*}{Electrons} & $\nabla_{\vect{x}} \cdot \of{\mu_e \vect{E} n_e}$ -- advection 		 & $\frac{\Delta x}{\mu_e \norm{\vect{E}_{\text{max}}}} \approx \mathcal{O}(10^{-11})$  \\ [0.1cm]		
%							   &  					   		& $\nabla_{\vect{x}} \cdot \of{D_e \nabla_{\vect{x}} n_e}$ -- diffusion  & $\frac{\Delta x^2}{ 2 D_e } \approx \mathcal{O}(10^{-12})$  \\ [0.1cm]
%							   &                       		& $k_i n_0 n_e$ -- reaction   											 & $\frac{\Delta n_e}{k_{i,\text{max}} n_0 n_e} \approx \mathcal{O}(10^{-11})$   \\ [0.1cm]
%		\hline
%		\hline
%		\multirow{6}{*}{Hybrid} & \multirow{3}{*}{Ions} 	& $\nabla_{\vect{x}} \cdot \of{\mu_i \vect{E} n_i}$ -- advection 		 & $\frac{\Delta x}{\mu_i \norm{\vect{E}_{\text{max}}}} \approx \mathcal{O}(10^{-9})$  \\		[0.1cm]
%								&  					   		& $\nabla_{\vect{x}} \cdot \of{D_i \nabla_{\vect{x}} n_i}$ -- diffusion  & $\frac{\Delta x^2}{ 2 D_i } \approx \mathcal{O}(10^{-7}) $  \\ 	[0.1cm]
%								&                       	& $k_i n_0 n_e$ -- reaction   											 & $\frac{\Delta n_e}{k_{i,\text{max}} n_0 n_e} \approx \mathcal{O}(10^{-9}) $   \\ [0.1cm]
%								\cline{2-4}
%								&\multirow{3}{*}{Electron BTE} & $\vect{v} \cdot \nabla_{\vect{x}} f$ -- $\vect{x}$-advection 		 				 & $\frac{\Delta x}{\norm{\vect{v}}} \approx \mathcal{O}(10^{-12})$\\ [0.1cm]
%								& 							   & $\frac{q_e \vect{E}}{m_e} \cdot \nabla_{\vect{v}} f$ -- $\vect{v}$-advection 		 & $\frac{\norm{\Delta \vect v} m_e}{q_e\norm{\vect{E}_{\text{max}}}} \approx \mathcal{O}(10^{-12})$\\ 	[0.1cm]
%								& 							   &  $k_i n_0 n_e = n_0 \int_{\vect{v}} \norm{\vect v}^3 \sigma_{\text{ion}}\of{\norm{\vect v}} f\of{\vect{v}} \diff{\vect{v}}$ -- reaction & $\frac{\Delta n_e}{k_{i,\text{max}} n_0 n_e} \approx \mathcal{O}(10^{-11})$ \\ [0.1cm]
%		\hline
%	\end{tabular}}
%\caption{A summary of the timescales for advection, diffusion, and reaction terms for the fluid and the hybrid formulations, assuming an explicit time integration scheme.  Here, we assume $\frac{\Delta x}{L}\approx \mathcal{O}(10^{-3})$, $\frac{\norm{\Delta \vect{v}}}{\norm{\vect v}} \approx \mathcal{O}(10^{-3})$, $\norm{\vect E_{\text{max}}}=8\times10^{4}$ [V/m], $k_{i,\text{max}}=10^{-14}$ [$m^3/s$], and the transport coefficients summarized in \Cref{t:model_parameters} with $p_0=$  1 Torr case \label{tab:ts_analysis}.}
%\end{table}


%taking $n_e \sim n_i$.
%The drift-diffusion approximation provides a closed expression for the mass continuity flux terms given by $\vect{v}_{e}n_{e}$ and $\vect{v}_{i}n_{i}$. Hence, continuity equations for electrons have significantly faster timescales compared to the ions.
\subsection{Numerical solver for the fluid approximation}
\label{subsec:fluid_solver}
As our focus is on the numerics of the hybrid solver, here we give a high-level description of the fluid solver. We are not claiming that this is the most efficient solver for this problem. Instead, we focus on comparing the converged time-periodic solutions between the fluid and the hybrid approaches. For the results of the fluid solver, we performed self-convergence tests in space and time. Now let us describe the solver for the fluid model described in \Cref{eq:fluid_model_eqs}. We use a Chebyshev collocation method for the spatial discretization. These collocation points are clustered to the electrodes. This is ideal for efficiently resolving sharp gradients of the electric field, species densities, and the electron energy in the sheath.  
%sheath profiles with sharp gradients near the electrodes. 
For the time discretization, we found that the operator-split approximation-based semi-implicit methods for solving~\Cref{e:fl_b,e:fl_c,e:fl_d} have timestep size restrictions determined by the electron timescales (see \Cref{subsubsec:ts_analysis}). Therefore, the system given in \Cref{eq:fluid_model_eqs} is solved as a coupled non-linear system with fully implicit time integration. We use Newton's method with direct factorization of the assembled Jacobian to solve for the Newton step. 
%The overall computational complexity for a single timestep is $\mathcal{O}(N_x^3)$ where $N_x$ denotes the number of collocation points in space. 
%The direct LU factorization used in the Newton step has $\mathcal{O}(N_x^3)$ computational complexity, but we found it quite robust. Also, the factorization costs are insignificant for the target problem sizes, $N_x=$ 200 to 400.
%The use of direct factorization is justified by the problem size (i.e., $N_x=200$ for medium resolution run, and $N_x=400$ for high resolution run).  



%Note that using Chebyshev collocation points naturally creates an adaptive grid in space where the higher resolution points are closer to the glow discharge walls. The above is useful to resolve the sheath region efficiently. 

\subsection{Hybrid fluid-BTE solver}
\label{subsec:pde_solver}
Here, we discuss the discretization of~\Cref{eq:glow_hybrid}. First, let us consider the velocity space discretization of \Cref{eq:hybrid_cnt_b}. The rate of change in the EDF due to the velocity space advection and collisions is given by \Cref{eq:bte-0d}. For the velocity space, we use spherical coordinates with a mixed Galerkin and collocation scheme for the angular direction and a Galerkin scheme for the radial coordinate discretization. We discretize $f(\vect{v}, t)$ as follows:
\begin{equation}
	f(\vect{v}, t) = \sum_{klm} f_{klm}\of{t} \Phi_{klm}\of{\vect{v}} \text{ where } \Phi_{klm}\of{\vect{v}}  = \underbrace{\phi_k\of{v}}_{\text{B-Spline basis}} \overbrace{Y_{lm}\of{v_\theta, v_\phi}}^{\tiny\text{spherical harmonics}}, \label{eq:f_expansion}
\end{equation} where $\biggl\{\phi_{k}\of{v}\biggr\}_{k=0}^{N_r-1}$ are cubic B-splines defined on a regular grid, and $Y_{lm}\of{\vtheta, \vphi}$ are defined as 
\begin{align}
	Y_{lm}\of{\vtheta,\vphi} &= U_{lm} P^{|m|}_l\of{\cos\of{\vtheta}} \alpha_m\of{\vphi} \text{ , } \nonumber \\
	U_{lm} = 
	\begin{cases}
		(-1)^m \sqrt{2} \sqrt{\frac{2l+1}{4\pi} \frac{(l-|m|)!}{(l+|m|)!}}, &m\neq 0, \\
		\sqrt{\frac{2l+1}{4\pi}}, &m = 0,
	\end{cases} \text{ , } &
	\alpha_m\of{\vphi}  =
	\begin{cases}
		\sin\of{|m|\phi}, &m < 0, \\
		1, &m = 0,\\
		\cos\of{m\phi}, &m > 0.
	\end{cases} \label{eq:sph_harmonics}
\end{align}
The $l$ index is also referred to as a polar mode, and the $m$ index as an azimuthal mode. Now, assume that $f_{klm}\of{t=0}=0,\  \forall m>0$. By aligning $\vect{E}$ to the velocity space z-axis, so that $\vect{E} = E \vect{\hat{e}_z} =  E \of{\cos(\vtheta) \vect{\hat{e}_r} - \sin(\vtheta)\vect{\hat{e}_\theta}}$ we ensure that the $\vect{E}$ acceleration excites only the polar modes~\Cref{eq:f_expansion}. This fact and the isotropic scattering assumption ensure that the BTE solutions preserve the azimuthal symmetry in the velocity space and thus $f_{klm}\of{t}$ remains zero $\forall t>0,\ m>0$. This essentially reduces the representation from 1D3V to 1D2V. Therefore, the EDF representation only requires $\{Y_{l0}\}_{l}$ modes. For this reason, we drop the azimuthal index $v_{\phi}$ in the remainder of the paper. Let $p$, $k$ denote the indices of the basis functions along the radial coordinate; $q$, $l$ denote the indices in the polar angle; and $\phi_p\of{v}$, $\phi_k\of{v}$ denote the test and trial B-splines in radial coordinate. Under these definitions, the discretized weak form of the collision operator is given by


%\Cref{eq:hybrid_cnt_a} is discretized using a Chebyshev collocation method, and we use the same Chebyshev collocation scheme for spatial discretization of~\Cref{eq:hybrid_cnt_b}. Next, using spherical coordinates, we describe the $\vect{v}$-space discretization of~\Cref{eq:hybrid_cnt_b}. We use a mixed Galerkin and collocation scheme for the angular direction and a Galerkin scheme for the radial coordinate discretization. Let $\{x_i\}^{^{N_x}}_{i=1}$ and $\{\vtheta_a\}^{^{N_\vtheta}}_{a=1}$ be collocation points in $x$, and $\vtheta$ coordinates. The semi-discrete form of~\Cref{eq:hybrid_cnt_b} is given by 
%\begin{subequations}
%	\begin{align}
%		\partial_t f(t, x, v, \vtheta_a) + v\cos\of{\vtheta_a} \partial_x f(t, x, v, \vtheta_a) &= \of{\frac{\partial f}{\partial t}}_{\text{$\vect{v}$-space}}\of{t, x, v, \vtheta_a} \label{eq:bte_1d_semi_discrete}, \\
%		\of{\frac{\partial f}{\partial t}}_{\text{$\vect{v}$-space}} &= \frac{q_e E}{m_e} \of{\cos(\vtheta) \partial_v f - \sin(\vtheta) \frac{1}{v} \partial_{\vtheta}f } + C_{en}(f) \label{eq:bte_vspace}.
%	\end{align}
%\end{subequations} We use a Galerkin discretization scheme on spherical coordinates for~\Cref{eq:bte_vspace}. For~\Cref{eq:bte_vspace}, we use compactly supported B-splines in the radial coordinate with real-valued spherical harmonics in the angular directions for EDF approximation. We use uniformly spaced knots in the radial coordinate to define a cubic B-spline basis. The EDF representation is given by 
%\begin{equation}
%	f(\vect{v}, t) = \sum_{klm} f_{klm}\of{t} \Phi_{klm}\of{\vect{v}} \text{ where } \Phi_{klm}\of{\vect{v}}  = \underbrace{\phi_k\of{v}}_{\text{B-Spline basis}} \overbrace{Y_{lm}\of{v_\theta, v_\phi}}^{\tiny\text{spherical harmonics}}. \label{eq:f_expansion} 
%\end{equation} The spherical harmonics basis functions are defined as 
%\begin{align}
%	Y_{lm}\of{\vtheta,\vphi} &= U_{lm} P^{|m|}_l\of{\cos\of{\vtheta}} \alpha_m\of{\vphi} \text{ , } \nonumber \\
%	U_{lm} = 
%	\begin{cases}
%		(-1)^m \sqrt{2} \sqrt{\frac{2l+1}{4\pi} \frac{(l-|m|)!}{(l+|m|)!}}, &m\neq 0, \\
%		\sqrt{\frac{2l+1}{4\pi}}, &m = 0,
%	\end{cases} \text{ , } &
%	\alpha_m\of{\vphi}  =
%	\begin{cases}
%		\sin\of{|m|\phi}, &m < 0, \\
%		1, &m = 0,\\
%		\cos\of{m\phi}, &m > 0.
%	\end{cases} \label{eq:sph_harmonics}
%\end{align} 
%With an aligned $\vect{E}$ field to $\vect{v}$-space z-axis, $\vect{E} = E \vect{\hat{e}_z} =  E \of{\cos(\vtheta) \vect{\hat{e}_r} - \sin(\vtheta)\vect{\hat{e}_\theta}}$ ensures that the $\vect{E}$ acceleration excites only the polar modes~\Cref{eq:f_expansion}. This fact and the isotropic scattering assumption ensure that the BTE solutions preserve the azimuthal symmetry in $\vect{v}$-space. Therefore, the EDF representation only requires $\{Y_{l0}\}_{l}$ modes. For this reason, we drop the azimuthal index $v_{\phi}$ in the remainder of the paper. Let $p$, $k$ denote indices of the basis functions along the radial coordinate, $q$, $l$ denote indices in the polar angle, and $\phi_p\of{v}$, $\phi_k\of{v}$ denote the test and trial B-splines in radial coordinate. Under these definitions, the discretized weak form of the collision operator is given by
\begin{multline}
    n_0 {[\vect C_{en}]}^{pq}_{kl} = n_0 \myint_{\reals^+} v^2 \sigma_T\of{v} \phi_{k}\of{v} \delta_{ql} \of{\phi_{p}\of{u} \delta_{q0}  -\phi_{p}\of{v}} \diff{v}, \\
    \text{ for } p,k \in \{0,\hdots, N_r-1\} \text{ and } q,l \in \{0, \hdots,  N_l-1\} \label{eq:c_en_mat_1d}.
\end{multline}
For non-zero heavy temperature, \Cref{eq:c_en_mat_1d} gets an additional correction term, given by
\begin{multline}
	n_0 T_0 {[\vect C_{T}]}^{pq}_{kl} = \frac{n_0 T_0 k_B}{m_0} \delta_{q0} \myint_{\reals^+} v^3 \sigma_T\of{v} \partial_v \phi_p\of{v} \partial_v \phi_k\of{v} \diff{v}, \\
	\text{ for } p,k \in \{0,\hdots, N_r-1\} \text{ and } q,l \in \{0, \hdots,  N_l-1\} \label{eq:c_T0_mat_1d}.
\end{multline} In \Cref{eq:c_en_mat_1d} and \Cref{eq:c_T0_mat_1d} $\sigma_T$ denotes the total collisional cross-section, $n_0$ and $T_0$ denotes the background neutral density and temperature. In our case, we have momentum transfer and ionization collisions. For multiple collisions, the effective collision operator is given by $\vect C_{en}^{\text{effec.}} = \sum_{i\in {\text{collisions}}}$ $n_i \vect{C_{en}} \of{\sigma_i} + n_0 T_0 \vect C_{T}\of{\sigma_0}$ where $0$ index denotes the ground state heavy species, and $\sigma_0$ denotes the total cross-section for the momentum transfer collisions. The discretized velocity space acceleration operator is given by
\begin{multline}
	\small
	[\vect A_v]^{pq}_{kl} = \int_{\reals^+}  
		\delta_{(q+1) l} v^2 \phi_p\of{v} \of{A_M\of{l} \partial_v\phi_k\of{v} + A_D\of{l}\frac{1}{v} \phi_k\of{v}} + \\
		\delta_{(q-1) l} v^2 \phi_p\of{v} \of{B_M\of{l} \partial_v\phi_k\of{v} + B_D\of{l}\frac{1}{v} \phi_k\of{v}} 
	 \diff{v}, \\
	 \text{ for } p,k \in \{0,\hdots, N_r-1\} \text{ and } q,l \in \{0, \hdots,  N_l-1\} \label{eq:adv_v_ws},
\end{multline} where $A_M$, $A_D$, $B_M$, and $B_D$ are coefficients defined by
\begin{equation}
A_M(l) = \frac{l}{\sqrt{4l^2-1}}\text{, } B_M(l) = \frac{l+1}{\sqrt{4l^2-1}}\text{, } A_D(l) = \frac{l^2}{\sqrt{4l^2-1}}\text{, } B_D(l) = \frac{l(l-1)}{\sqrt{4l^2-1}}. 	
\end{equation}
To summarize, the velocity space discretized BTE is given by
\begin{equation}
	\partial_t \of{\vect M_{v,\vtheta} \vect f} = \of{\vect{C}_{en} + E \vect A_v}\vect{f} \label{eq:discretized_bte_vspace},
\end{equation} where $\vect M_{v,\vtheta}$, $\vect C_{en}$, and $\vect A_v$ $\in \reals^{N_rN_l \times N_rN_l} $ denote the Galerkin mass matrix, electron-heavy collisions and the velocity space acceleration operators respectively. More details on the velocity space discretization can be found in \cite{fernando0DBTE}. 

Now we discuss the spatial discretization of \Cref{eq:hybrid_cnt_b}. We use a Chebyshev collocation scheme for the spatial discretization of~\Cref{eq:hybrid_cnt_b}. The spherical harmonic representation is inadequate to impose the $\vtheta$ discontinuous boundary conditions specified in~\Cref{eq:hybrid_cnt_b_bdy}. This is resolved by a mixed Galerkin and collocation representation in $\vtheta$. Let $\{x_i\}^{^{N_x}-1}_{i=0}$ and $\{\vtheta_a\}^{^{N_\vtheta}-1}_{a=0}$ be collocation points in $x$, and $\vtheta$ coordinates. With these collocation points, the EDF representation is given by
\begin{multline}
	f(x_i, v, \vtheta_a, t) = \sum_{k=0}^{N_r-1} f_{k}\of{x_i, \vtheta_a, t} \phi_{k}\of{v} = \sum_{k=0}^{N_r-1} f_{ika}\of{t} \phi_{k}\of{v}, \\ \text{ where } f_{ika}(t) \equiv f_k(x_i, \vtheta_a, t), 
\end{multline}
and the semi-discrete form of~\Cref{eq:hybrid_cnt_b} is given by 
\begin{equation}
\partial_t f(t, x, v, \vtheta_a) + v\cos\of{\vtheta_a} \partial_x f(t, x, v, \vtheta_a) = \of{\partial_t f}_{\text{$\vect{v}$-space}}\of{t, x, v, \vtheta_a} \label{eq:bte_1d_semi_discrete}.
\end{equation} Let $\vect{P}_{S}$ denotes the $\vtheta$ ordinates to spherical harmonics projection, and $\vect{P}_{O}$ denotes the spherical harmonics to $\vtheta$ ordinates projection operators. These operators are defined by
\begin{multline}
	\vect P_S = \vect I_v \otimes \vect T_S \text{ , } \vect P_O = \vect I_v \otimes \vect T_O \text{ where } 
	[\vect T_{S}]^{q}_{a} = Y_{q}\of{\vtheta_a} w_a \text{ and }\\
	[\vect T_{O}]^{a}_{q} = Y_{q}\of{\vtheta_a}, 
	\text{ for } q \in \{0, \hdots, N_l-1\}\text{, } a\in \{0, \hdots, N_\vtheta-1\}\label{eq:proj_ops} .
\end{multline} Here, $\otimes$ denotes the matrix Kronecker product, $\vect I_v\in \reals^{N_r\times N_r}$ denotes the identity matrix in $v$ coordinate, and $w_a$ denotes the quadrature weights for the spherical basis projection. We use a Chebyshev collocation method in space, a Galerkin discretization in the $v$ coordinate, and a mixed Galerkin and collocation scheme in the $\vtheta$ coordinate for the discretization of~\Cref{eq:bte_1d_semi_discrete}. Let $\vect{D}_x$ denotes the discrete Chebyshev-basis derivative operator and $\vect{A}_x$ denotes the $v$ coordinate Galerkin projection of the spatial advection term in~\Cref{eq:bte_1d_semi_discrete}. The $\vect{A}_x$ operator is given by
\begin{multline}
	\vect{A}_x = \vect{D}_{\vtheta} \otimes \vect{G}_{v} \text{ where,}\quad
	[\vect D_\vtheta]_{ab} = \delta_{ab} \cos\of{\vtheta_a} \text{ and } \\ [\vect{G}_{v}]_{pk} =\int_{\reals^+} v^3 \phi_{p}\of{v} \phi_{k}\of{v}\diff{v}, \\
	\text{ for } p,k \in \{0,\hdots, N_r-1\} \text{ and } a, b \in \{0, \hdots, N_{\vtheta}-1\}
	\label{eq:A_x}.
\end{multline} By putting everything together, with $x$, $v$, and $\vtheta$ discretized BTE is given by
\begin{multline}
   \partial_t \vect F + \vect A_x \vect F \vect D_x^T = \vect P_{O}\vect C_{en} \vect P_{S} \vect{F} + \vect P_O \vect A_v \vect P_{S} \of{\vect E \pdot \vect{F}} \text{ where } \\
   \vect F \in \reals^{N_r N_\vtheta \times N_x}\text{, }\vect{P}_s
   \in \reals^{N_rN_l \times N_r N_\vtheta}\text{, }\\
   \vect{C}_{en}\text{, }\vect{A}_v \in \reals^{N_r N_l \times N_r N_l}\text{, }
   \vect{P}_O \in \reals^{N_r N_\vtheta \times N_r N_l}\text{, }
   \vect{D}_x \in \reals^{N_x \times N_x}\text{, }\vect{A}_x\in \reals^{N_rN_\vtheta \times N_r N_\vtheta}, \\
   [\vect F]_{i,:} = f_{ika}\text{, } 
   \text{ for } i \in \{0, \hdots ,N_x-1\} \text{, } k \in \{0, \hdots N_r-1\} \text{, and } a \in \{0, \hdots N_\vtheta-1\}    \label{eq:bte_1d_discrete}.
\end{multline} Here, $\vect{E} \in \reals^{N_x}$ and $\pdot$ denotes the column wise element product. Also, it is important to note that the operators in \Cref{eq:bte_1d_discrete} is properly scaled by Galerkin mass matrices, i.e., from here on, $\vect{G}_v \equiv \vect{M}^{-1}_v \vect{G}_v$, $\vect{A}_x \equiv \vect{D}_\vtheta \otimes \vect G_v$, $\vect{C}_{en} \equiv \vect M^{-1}_{v,\vtheta} \vect C_{en}$, and $\vect{A}_v \equiv M^{-1}_{v,\vtheta} \vect A_v$, where $\vect{M}_v$ and $\vect{M}_{v,\vtheta}$ denote the standard Galerkin mass matrices~\cite{ciarlet2002finite} in $v$ and $v, \vtheta $ coordinates respectively.  \Cref{eq:hybrid_cnt_a,eq:hybrid_cnt_c} are discretized with the same Chebyshev collocation scheme we used for~\Cref{eq:hybrid_cnt_b}. Let $\vect{k}_i\in \reals^{N_r N_\vtheta}$ denotes the discretized ionization rate coefficient operator, $\vect{u}\in\reals^{N_r N_\vtheta}$ denotes the zeroth-order moment operator where $\vect{n}_e = \vect{u}^T \vect{F}$, and $\vect{L}\in\reals^{N_x \times N_x}$ denotes the discretized electrostatics operator where $\vect{E}=\vect{L}\of{\vect{n}_i-\vect{n}_e}$. Then, the discretized hybrid model is given by
\begin{subequations}
	\begin{align}
		&\partial_t \vect n_i  + \vect D_x \vect J_i \of{\vect E, \vect n_i} = \of{\vect{k}^T_i \vect F} n_0 \vect n_e \label{eq:hybrid_discrete_a},\\
		&\partial_t \vect F + \vect A_x \vect F \vect D_x^T = \vect P_{O}\vect C_{en} \vect P_{S} \vect{F}  +  \vect P_O \vect A_v \vect P_{S} \of{\vect E \pdot \vect{F}}   \label{eq:hybrid_discrete_b},\\
		&\vect{E} = \vect{L}\of{\vect n_i - \vect n_e}, \text{ where } \vect n_e = \vect u^T \vect F \label{eq:hybrid_discrete_c}.
	\end{align}\label{eq:hybrid_discrete}
\end{subequations}The total number of unknowns for the discretized system is $N_x(1 + N_{r} N_{\vtheta})$ where $N_x$, $N_r$, and $N_{\vtheta}$ denote the number of Chebyshev collocation points in $x$, the number of B-spline basis used in $v$, and the number of collocation points in $\vtheta$.



%As mentioned, For \Cref{eq:hybrid_cnt_b}, we use a Chebyshev collocation method in $\vect{x}$-space, a Galerkin scheme in the $\vect{v}$-space radial direction, and a mixed Galerkin collocation scheme in $\vtheta$ coordinate.  The left-hand side of \Cref{eq:bte_1d_semi_discrete} acts on each ordinate separately, while the right-hand side $\vect{v}$-space operator in \Cref{eq:bte_vspace} couples all $\vtheta$ ordinates.
%The EDF is represented using $f(t, x_i, v, \vtheta_a) = \sum_{k} f_{ika}(t) \phi_k\of{v}$, and the fully discretized system is given by
%\begin{multline}
%    \partial_t \vect F + \vect A_x \vect F \vect D_x^T = \vect P_{O}\of{\vect C_{en}  + \vect{E} \pdot \vect A_v  }  \vect P_{S} \vect{F} \\\text{ where } [\vect F]_{i,:} = f_{ika} \text{ for } k \in \{1, \hdots N_r\}, a \in \{1, \hdots N_\vtheta\}  \label{eq:bte_1d_discrete}.
%\end{multline} In the above, $\vect D_x$ denotes the discrete spatial derivative operator, $\vect{A}_x$ denotes the $v$ coordinate Galerkin projection of the spatial advection term, $\vect{P}_{S}$ denotes the $\vtheta$ ordinates to spherical harmonics projection, and $\vect{P}_{O}$ denotes the spherical harmonics to $\vtheta$ ordinates operator. These operators are defined by 
%% given by $g\of{\vtheta} = \sum_l g_l Y_{l}\of{\vtheta}$, where $g_l = \frac{1}{2\pi}\int_{0}^{\pi} g\of{\vtheta}Y_l\of{\vtheta} \sin\of{\vtheta}\diff{\vtheta} \approx \sum_a g\of{\vtheta_a} Y_l\of{\vtheta_a} w_a$.
%\begin{align}
%    [\vect A_x]^{pa}_{kb} = \delta_{ab} \cos\of{\vtheta_a} \int_{\reals^+} v^3 \phi_{p}\of{v} \phi_{k}\of{v}\diff{v}, \label{eq:A_x}\\
%    [\vect P_{S}]^{pq}_{ka} = \delta_{pk} Y_{q}\of{\vtheta_a} w_a, \text{\ \ and\ \ }
%    [\vect P_{O}]^{pa}_{kq} = \delta_{pk} Y_{q}\of{\vtheta_a} \label{eq:proj_ops}.
%\end{align} Here, $w_a$ denotes the quadrature weights for the spherical basis projection. In summary, the discretized hybrid model is given by 
%\begin{subequations}
%\begin{align}
%	&\partial_t \vect n_i  + \vect D_x \vect J_i \of{\vect E\of{\vect n_i, \vect F}, \vect n_i} = \vect{\dot{S}}\of{\vect F} = \of{\vect{k}^T_i \vect F} n_0 \vect n_e \label{eq:hybrid_discrete_a},\\
%	&\partial_t \vect F + \vect A_x \vect F \vect D_x^T = \vect P_{O}\of{\vect C_{en}  + \vect{E} \of{\vect n_i, \vect F} \pdot \vect A_v  }  \vect P_{S} \vect{F} \label{eq:hybrid_discrete_b},\\
%	&\vect{E} = \vect{L}\of{\vect n_i - \vect n_e}, \text{ where } \vect n_e = \vect u^T \vect F \label{eq:hybrid_discrete_c}.
%\end{align}\label{eq:hybrid_discrete}
%\end{subequations} Here, $\vect{k}_i$ denotes the ionization rate coefficient operator, $\vect{L}$ denotes the discretized operator for electrostatics, and $\vect u$ denotes the zeroth-order moment operator such that $\vect n_e = \vect u^T \vect F$. 
%\Cref{eq:hybrid_discrete_a,eq:hybrid_discrete_b,eq:hybrid_discrete_c} are the discrete form of~\Cref{eq:glow_hybrid}. The total number of unknowns for the discretized system is given by $N_x(1 + N_{r} N_{\vtheta})$ where $N_x$, $N_r$, and $N_{\vtheta}$ denote the number of Chebyshev collocation points in $\vect{x}$-space, the number of B-spline basis used in $v$, and the number of ordinates in $\vtheta$.

\subsubsection{Timescales analysis}
\label{subsubsec:ts_analysis}
Here, we present timescales analysis for \Cref{eq:hybrid_discrete}. \Cref{tab:ts_analysis}~summarizes the main ion and electron timescales for a RF-GDP with $p_0=1$ Torr. The input voltage oscillation period is set to $7.374 \times 10^{-8}$ s for all RF-GDP cases considered in this paper. The timescale for the collision operator is given by
\begin{multline}
	\Delta t \frac{\norm{\vect{C}_{en} \vect{F}}}{\norm{\vect{F}}} \leq \Delta t \norm{\vect{C}_{en}} \leq \Delta t n_0\max\of{\norm{\vect{v}}\sigma_{0}, \norm{\vect{v}}\sigma_{i}}\leq \frac{\norm{\Delta \vect{F}}}{\norm{\vect{F}}}\\ \implies
	\Delta t \leq \frac{\norm{\Delta \vect{F}}}{\norm{\vect{F}}} \frac{1}{n_0\max\of{\norm{\vect{v}}\sigma_{0}, \norm{\vect{v}}\sigma_{i}}}= \frac{\norm{\Delta \vect{F}}}{\norm{\vect{F}}} \frac{1}{n_0\max\of{\norm{\vect{v}}\sigma_{0}}} \label{eq:coll_timescale}.
\end{multline} The reaction and collision terms determine the fastest timescale for ions and electrons. However, as shown in \Cref{tab:ts_analysis}, ions have much slower timescales than electrons. Therefore, the electron timescale determines the electrostatic timescale. A change in the electron number density $n_e(\vect{x}, t)$ is caused by the BTE spatial advection and reaction terms. While the velocity space advection is not directly coupled to $n_e(\vect{x}, t)$, it is indirectly coupled through the collision term. This is due to the EDF tails being populated due to the velocity space advection, and high-energy electrons are likely to trigger ionization collisions. Therefore, the electrostatic and the BTE coupling timescale is governed by the position-velocity space advection and the reaction term timescales. The collisional timescale gives the relative change in the EDF due to collisions. Typically, Since $\sigma_0 \geq \sigma_{\text{i}}$, the collisional timescale is determined by the momentum transfer cross-section. 
%Operator split schemes that decouple position space and velocity space operators

\begin{table}[tbhp]
	\centering
	\resizebox{\textwidth}{!}{
		\begin{tabular}{||c|c|c||}
			\hline 
			Species  & Term & $\Delta t$ [s]\\
			\hline
			\multirow{3}{*}{Ions} 	& $\nabla_{\vect{x}} \cdot \of{\mu_i \vect{E} n_i}$ -- advection 		 & $\frac{\Delta x}{\mu_i \norm{\vect{E}_{\text{max}}}} \approx \mathcal{O}(10^{-9})$  \\		[0.1cm]
			& $\nabla_{\vect{x}} \cdot \of{D_i \nabla_{\vect{x}} n_i}$ -- diffusion  & $\frac{\Delta x^2}{ 2 D_i } \approx \mathcal{O}(10^{-7}) $  \\ 	[0.1cm]
			& $k_i n_0 n_e$ -- reaction   											 & $\frac{\Delta n_i}{n_i}\frac{n_i}{k_{i,\text{max}} n_0 n_e} \approx \mathcal{O}(10^{-10})$   \\ [0.1cm]
			\cline{2-3}
			\multirow{3}{*}{Electron BTE}  & $\vect{v} \cdot \nabla_{\vect{x}} f$ -- $\vect{x}$-advection 		 				 & $\frac{\Delta x}{\norm{\vect{v}}} \approx \mathcal{O}(10^{-12})$\\ [0.1cm]
			& $\frac{q_e \vect{E}}{m_e} \cdot \nabla_{\vect{v}} f$ -- $\vect{v}$-advection 		 & $\frac{\norm{\Delta \vect v} m_e}{q_e\norm{\vect{E}_{\text{max}}}} \approx \mathcal{O}(10^{-12})$\\ 	[0.1cm]
			&  $k_i n_0 n_e = n_0 \int_{\vect{v}} \norm{\vect v}^3 \sigma_{i}\of{\norm{\vect v}} f\of{\vect{v}} \diff{\vect{v}}$ -- reaction & $\frac{\Delta n_e}{n_e}\frac{1}{k_{i,\text{max}} n_0 } \approx \mathcal{O}(10^{-12})$ \\ [0.1cm]
			&$ \vect{C}_{en}\vect{F}$ -- collisions  & $\frac{\norm{\Delta \vect{F}}}{\norm{\vect{F}}} \frac{1}{n_0 \max\of{ \norm{\vect{v}} \sigma_0}} \approx \mathcal{O}(10^{-13})$\\ [0.2cm]
			\cline{2-3}
			\multirow{2}{*}{Driving voltage} & &\\[0.01cm]
											 & $V(t) = V_0 \sin\of{2\pi \zeta t}$ & $1/\zeta=7.374 \times 10^{-8}$ \\ [0.1cm]
			\hline
	\end{tabular}}
	\caption{A summary of the timescales for advection, diffusion, and reaction terms for the hybrid formulation, assuming an explicit time integration scheme.  Here, we assume $\frac{\Delta x}{L}\approx \mathcal{O}(10^{-3})$, $\frac{\norm{\Delta \vect{v}}}{\norm{\vect v}} \approx \mathcal{O}(10^{-3})$, $\frac{\norm{\Delta\vect{F}}}{\norm{\vect{F}}} \approx \mathcal{O}(10^{-3})$, and $\frac{\Delta n_e}{n_e} \approx \mathcal{O}(10^{-3})$. The maximum absolute quantities occurs closer to the electrodes with $\norm{\vect E_{\text{max}}}=8\times10^{4}$ [V/m], $k_{i,\text{max}}=10^{-14}$ [$m^3/s$], and the corresponding ionization degree $\frac{n_i}{n_e}\approx\mathcal{O}(10^2)$. For the other transport coefficients, we use the values summarized in \Cref{t:model_parameters} with the $p_0=$ 1 Torr case. \label{tab:ts_analysis}}
\end{table}
%Considering the plasma oscillation frequencies for electrons $\omega_{e} = \sqrt{q_e^2 n_e /m_e/\epsilon_0}$ and ions $\omega_{i} = \sqrt{q_e^2 n_i /m_i/\epsilon_0}$ we can write $\omega_{e} / \omega_{i} = \of{m_i / m_e}^{1/2} \of{n_e / n_i}^{1/2}$. For RF-GDPs, typically $n_e \sim n_i \sim 10^{17}$ $m^{-3}$, $f_{e} \approx 8.9 \times 10^{9}$ $s^{-1}$ and $f_{i} \approx 3.3 \times 10^{7}$ $s^{-1}$. Therefore, electrons have a higher plasma frequency than ions. This is primarily due to the small mass ratio between electrons and ions, $m_e/m_i = 1.36\times 10^{-5}$. 

\subsubsection{Hybrid solver time integration}
Typically, we want to evolve the system to the order of thousand cycles or $10^{-5}$ seconds, and require ten million timesteps. An implicit scheme with an iterative solve is possible as the matrix-vector products can be also done in an optimal way, but designing a good preconditioner is not trivial. Instead we opt for an operator split scheme.
As we can see in~\Cref{tab:ts_analysis}, the ions have slower timescales compared to electrons. Hence, we use an operator split scheme to decouple ion and electron transport equations. Since electrons determine the electrostatics coupling timescale, \Cref{eq:hybrid_discrete_a} is solved with decoupled electrostatics.
%The ion and electron plasma frequencies determine the $\vect{E}$ field coupling time scales for \Cref{eq:hybrid_discrete_a,eq:hybrid_discrete_b}. This is mainly due to the electron-to-ion mass ratio. Therefore, ions respond slowly compared to the electrons for a given perturbation in the electric field. Hence, \Cref{eq:hybrid_discrete_a} is solved with frozen $\vect{E}$ and $\vect{F}$ values. 

\par \textbf{Heavies evolution}: For a given state $(\vect n_i^n, \vect F^n)$ at time $t=t_n$, the ion transport equation is solved implicitly with lagged $\vect F^n$ and $\vect E^n = \vect L (\vect n^n_i - \vect u^T \vect F^n)$. This coupling results in a linear system to be solved for the ions state at time $t_{n+1}=t_n +  \Delta t$, where $\Delta t$ denotes the time step size. So, the ions update in our operator-split scheme is given by
\begin{equation}
	\frac{\vect n^{n+1}_i -\vect n^n_i}{\Delta t} + \vect{D}_x \vect J_i \of{\vect E^n, \vect n^{n+1}_i}= \of{\vect k_i^T \vect F^n} n_0 \of{\vect u^T \vect F^n} \label{eq:fluid_step}.
\end{equation}


%The electrons are strongly coupled to the electric field through~\Cref{eq:hybrid_discrete_c}. Therefore, \Cref{eq:hybrid_discrete_b,eq:hybrid_discrete_c} are solved together.
%to account for the $\vect{E}$ field variation due to electron transport.

\par \textbf{Electron evolution}: We consider the evolution of \Cref{eq:hybrid_discrete_b,eq:hybrid_discrete_c} assuming a fixed $\vect{n}_i$. We consider two alternative schemes for the electrons: a semi-implicit and a fully-implicit one. The semi-implicit scheme compute the EDF at time $t + \Delta t$ is given by
\begin{subequations}
	\begin{empheq}[left =\text{\small semi-implicit} \empheqlbrace]{align}
        &\frac{\vect{F}^{n+1/2} - \vect{F}^n}{\Delta t/2} =  -\vect A_x \vect F^{n+1/2} \vect D_x^T, t \in \of{t_n , t_n + \frac{\Delta t}{2}} \label{eq:bte_semi_implicit_xspace0}, \\
        &\vect{E}^{1/2}=\vect L \of{\vect n_i - \vect u^T \vect F^{n + 1/2}} \label{eq:bte_semi_implicit_efield},\\
        & \frac{\vect{F}^{*} - \vect{F}^{n+1/2}}{\Delta t} = \vect P_{O}\vect C_{en}\vect P_{S} \vect{F}^*  + \nonumber \\
        &\qquad \qquad \vect P_O \vect A_v \vect P_{S} \of{\vect{E}^{1/2} \pdot \vect{F}^*}, t \in \of{t_n, t_n + \Delta t}, \label{eq:bte_semi_implicit_vspace}\\
        &\frac{\vect{F}^{n+1} - \vect{F}^*}{\Delta t / 2} =  -\vect A_x \vect F^{n+1} \vect D_x^T, t \in \of{t_{n+1/2} , t_{n+1/2} + \frac{\Delta t}{2}} \label{eq:bte_semi_implicit_xspace1}.
    \end{empheq} \label{eq:bte_semi_implicit}
\end{subequations}
The overview of the semi-implicit scheme is summarized below. 
\begin{itemize}
\item \textbf{Spatial advection}: \Cref{eq:bte_semi_implicit_xspace0,eq:bte_semi_implicit_xspace1} corresponds to the spatial advection of electrons. We use the the eigen decomposition of $\vect{G}_v=\vect{U}\vect{\Lambda}U^{-1}$ to diagonalize \Cref{eq:bte_semi_implicit_xspace0,eq:bte_semi_implicit_xspace1}. Since $\vect{G}_v$ is a product of two positive definite matrices, it has real positive eigenvalues. Upon diagonalization, \Cref{eq:bte_semi_implicit_xspace0,eq:bte_semi_implicit_xspace1} become series of decoupled $N_rN_\vtheta$ spatial advection equations where the advection velocity is given by $\vect{a} = \Lambda_i \cos(\vtheta_j) \vect{\hat{e}}_x$ for $i=\{0,\hdots N_r-1\}$ and $j=\{0,\hdots N_\vtheta-1\}$ where $N_r$ and $N_\vtheta$ denotes the number of B-splines and $\vtheta$ ordinates used. For each $i$ and $j$, we precompute and store the inverted $\left(\vect{I}_x + \frac{1}{2} \Delta t \Lambda_i \cos(\vtheta_j) \vect D_x \right)^{-1} = \vect Q_{ij}$ operators and use them for spatial advection. Here, $\vect{I}_x\in \reals^{N_x\times N_x}$ denotes the identity matrix in the $x$ dimension. This direct inversion is performed only once and reused throughout the time evolution. The storage cost for the inverted matrices is given by $\mathcal{O}(N_r N_\vtheta N_x^2)$. \Cref{alg:bte_spatial_adv} presents the overview of the BTE spatial advection step. In the implementation, the application of $\vect Q_{ij}$ is performed as a batched general matrix-matrix multiplication (GEMM).
%
\item \textbf{Electrostatics}: \Cref{eq:bte_semi_implicit_efield} computes the updated electric field $\vect{E}^{1/2}$ following the spatial advection. Here, $\vect{L}\in \reals^{N_x \times N_x}$ denotes the discretized $\nabla_{\vect{x}} \Delta^{-1}_{\vect x}$ operator which is computed once and reused throughout the time evolution. 
%
\item \textbf{Velocity space update}: We use the generalized minimal residual method (GMRES) to solve~\Cref{eq:bte_semi_implicit_vspace}. In~\Cref{eq:bte_semi_implicit_vspace}, the discretized EDF is spatially decoupled, and the velocity space left-hand side operator varies spatially and is given by
\begin{multline}
	\biggl(\vect{I}_{v,\vtheta} -\Delta t \vect P_{O}\of{\vect C_{en}  + \bigl[\vect{E}^{1/2}\bigr]_{i} \vect A_v  }  \vect P_{S}\biggr) \vect F^* = \vect F^{n + 1/2}  \\ \text{ where } \vect E^{1/2} \in \reals^{N_x}, 
	\text{ for } i \in \{0,\hdots, N_x - 1\} \label{eq:vspace_batched_system}.
\end{multline} Here, $\vect{I}_{v,\vtheta}$ denotes the identity operator in the velocity space, and $[\vect{E}^{1/2}]_i$ denotes the electric field evaluated at the $i$-th spatial collocation point. Therefore, the solve of~\Cref{eq:bte_semi_implicit_vspace} reduces to a $N_x$ decoupled linear systems. For the GMRES solve, these linear systems are solved as a batched system. The batched right-hand side evaluation is summarized in \Cref{alg:vspace_action}. We use a domain decomposition preconditioner for the GMRES solve. Let $E_0$ denotes an upper bound on the electric field such that $\norm{\vect{E}\of{\vect x, t}} \leq E_0 \forall t$ and $-E_0 = S_0 < S_1 < \hdots < S_{N_c}=E_0$ be a time independent non-overlapping partition of $(-E_0, E_0)$. We precompute a series of operators $\{H_p\}_{p=1}^{N_c}$ where each $\vect H_p\in \reals^{N_r N_l}$ is given by
\begin{equation}
	\vect H_p = \of{\vect I -\Delta t \of {\vect C_{en} + \frac{1}{2}\of{S_{p-1} + S_{p}} \vect A_v}}^{-1} 
	\text{ for } p\in\{1,\hdots N_c\}.
\end{equation} Here, $\vect I\in \reals^{N_r N_l}$ denotes the identity matrix. For a specified $\vect{E} \in \reals^{N_x}$, we define $N_c$ partitions where partition $w_k$ is given by $w_k=\{i\quad | \  S_{k-1} \leq [\vect{E}]_i < S_{k}\}$ for $k\in \{1,\hdots,N_c\}$. We refer to this as the preconditioner setup step, which is performed once per each timestep. The precondition operator evaluation is summarized in \Cref{alg:vspace_action_precon}.
%$\{\vect H_p\}_{p=1}^{N_c}$
\end{itemize}

\begin{algorithm}[!tbhp]
	\begin{algorithmic}[1]
		\Require $\vect{F}^n\in \reals^{N_rN_\vtheta \times N_x}$ -- EDF at time $t_n$, \nonumber\\
		$\biggl\{\vect{Q}_{ij} = \left(\vect{I}_x + \frac{1}{2} \Delta t \Lambda_i \cos(\vtheta_j) \vect D_x \right)^{-1}\biggr\}_{(i,j)\in \{0,..,N_r-1\}\times\{0,..,N_\vtheta-1\}}$ -- inverted spatial advection operators, and $\vect{G}_v = \vect{U} \vect \Lambda \vect U^{-1}$
		\Ensure Solution at time $t_n + \Delta t/2$, $\vect{F}^{1/2} \in \reals^{N_r N_\vtheta \times N_x}$ 
		\State $\vect Y \leftarrow \text{unfold}(\vect F, (N_r, N_\vtheta, N_x))$ \Comment{$\vect Y \in \reals^{N_r \times N_\vtheta\times N_x}$}
		\State $[\vect Y]_{ijk} \leftarrow [\vect U^{-1}]_{im} \times_{m} [\vect F^n]_{mjk}$ \Comment{contraction on $m$ index}
		\For{$i \in \{0,\hdots, N_r-1\}$}
		\For{$j \in \{0,\hdots, N_\vtheta-1\}$}
		\rlap{\smash{$\left.\begin{array}{@{}c@{}}\\{}\\{}\end{array}\color{black}\right\}%
				\color{black}\begin{tabular}{l} Batched GEMM kernels \end{tabular}$}}
		\State $\vect Y_{ijk} \leftarrow \vect Q_{ijkl} \times_l \vect Y_{ijl}$ \Comment{$\vect Q_{ij}\in \reals^{N_x \times N_x}$ and contraction on $l$ index}
		\EndFor
		\EndFor
		\State $[\vect Y]_{ijk} \leftarrow [\vect U]_{im} \times_{m} [\vect F^n]_{mjk}$ \Comment{contraction on $m$ index}
		\State $\vect F^{1/2} \leftarrow \text{fold}(\vect Y , (N_rN_\vtheta \times N_x))$ \Comment{$\vect F^{1/2} \in \reals^{N_rN_\vtheta \times N_x}$}
		\State \Return $\vect{F}^{1/2}$
	\end{algorithmic}
	\caption{BTE spatial advection. \label{alg:bte_spatial_adv}}
\end{algorithm}
\begin{algorithm}
	\begin{algorithmic}[1]
		\Require $\vect F \in \reals^{N_r N_\vtheta \times N_x}$, $\vect E \in \reals^N_x$ -- electric field, $\vect C_en$, $\vect A_v$, $\vect P_O$, $\vect P_S$, and $\Delta t$
		\Ensure $\biggl(\vect{I}_{v,\vtheta} -\Delta t \vect P_{O}\of{\vect C_{en}  + \bigl[\vect{E}\bigr]_{i} \vect A_v  }  \vect P_{S}\biggr) \vect F$ -- operator action on $\vect F$
		\State $\vect F_S \leftarrow \vect P_S \vect F$
		\State $\vect F_C \leftarrow \vect C_{en} \vect F_s$ \quad and \quad $\vect F_A \leftarrow \vect A_v \vect F_s$
		\State $\vect G  \leftarrow  \vect F - \Delta t \vect P_O \of{\vect F_C + \vect E \pdot \vect F_A}$
		\State \Return $\vect G$
	\end{algorithmic}
	\caption{BTE velocity space operator action. \label{alg:vspace_action}}
\end{algorithm}
\begin{algorithm}
	\begin{algorithmic}[1]
		\Require $\vect F $, $\{\vect H_p\}_{p=1}^{N_c}$ -- precondition operators, $\{w_p\}_{p=1}^{N_c}$ -- domain decomposition, $\vect P_S $, $\vect P_O$
		\Ensure Preconditioner action $: \reals^{N_r N_\vtheta \times N_x} \rightarrow \reals^{N_r N_\vtheta \times N_x}$
		\State $\vect G \leftarrow \vect 0 $ \Comment{$\vect G \in \reals^{NrN_l \times N_x}$}
		\State $\vect F_S \leftarrow \vect P_S \vect F$ \Comment{$\vect F_s \in \reals^{NrN_l \times N_x}$}
		\For {$p \in \{1,\hdots, N_c\}$ }
			\State $\vect G[:, w_p] \leftarrow \vect H_p \vect F_S[:, w_p]$
		\EndFor
		\State $\vect G \leftarrow \vect P_O \vect G$
		\State \Return $\vect G$ \Comment{$\vect G \in \reals^{N_r N_\vtheta \times N_x}$}
	\end{algorithmic}
	\caption{BTE velocity space GMRES preconditioner. \label{alg:vspace_action_precon}}
\end{algorithm}

%\Cref{eq:bte_semi_implicit_xspace0} updates $\vect{F}^{n+1/2}$ to capture the spatial advection, i.e., the translation of electrons. Next, \Cref{eq:bte_semi_implicit_efield} is used to compute the updated electric field $\vect{E}^{1/2}$, \Cref{eq:bte_semi_implicit_vspace} is solved for the $\vect{v}$-space advection and collisions, and finally  \Cref{eq:bte_semi_implicit_xspace1} repeats the spatial advection and computes the $\vect{F}^{n+1}$. To solve the linear system in \Cref{eq:bte_semi_implicit_vspace}, we use the generalized minimum residual method (GMRES). We use pre-factored right-hand side operators for a predefined $E_c\in \of{-E_0 , E_0}$ coarse grid values as a preconditioner for the GMRES solve. Here, $E_0$ denotes an upper bound on  the electric field where $\norm{\vect{E}\of{t,\vect x}} \leq E_0$ $\forall t$. We use the eigen decomposition of $\vect{A}_x=\vect{U}\vect{\Lambda}U^{-1}$ to diagonalize \Cref{eq:bte_semi_implicit_xspace0,eq:bte_semi_implicit_xspace1}. Upon diagonalization, \Cref{eq:bte_semi_implicit_xspace0,eq:bte_semi_implicit_xspace1} become series of decoupled $N_rN_\vtheta$ advection equations where the advection velocity is given by $\vect{a} = \Lambda_i \cos(\vtheta_j) \vect{\hat{e}}_x$ for $i=\{1,\hdots N_r\}$ and $j=\{1,\hdots N_\vtheta\}$ where $N_r$ and $N_\vtheta$ denotes the number of B-splines and $\vtheta$ ordinates used. For the above, we pre-compute and store inverted $N_r N_\vtheta$ operators and use them directly for solving~\Cref{eq:bte_semi_implicit_xspace0,eq:bte_semi_implicit_xspace1}. The decoupled $\vect{E}$, $\vect{x}$-space, and $\vect{v}$-space solves introduce timestep size restrictions that arise from the electron plasma oscillation frequency. To address this timestep size restriction, we propose a fully-implicit scheme given by  

The semi-implicit approach has a timestep size restriction that arises from the decoupled electrostatics, spatial, and velocity space solves. This is determined by the BTE spatial advection and reaction timescales (see \Cref{subsubsec:ts_analysis}). To address this timestep size restriction, we propose a fully-implicit scheme for the BTE, which is given by 
\begin{subequations}
	\begin{empheq}[left =\text{\small fully-implicit} \empheqlbrace]{align}
		\frac{\vect F^{n+1} - \vect F^n}{\Delta t}  &= -\vect A_x \vect F^{n+1} \vect D_x^T + \vect P_{O}\vect C_{en} \vect P_{S} \vect{F}^{n+1} + \nonumber \\
		& \qquad\qquad \vect P_O \vect A_v \vect P_S \of {\vect E(\vect F^{n+1}) \pdot \vect F^{n+1}}     \label{eq:bte_fully_implicit_a}, \\
		\vect{E}\of{\vect{F}} &= \vect L \of{\vect n_i - \vect u^T \vect F} \label{eq:bte_fully_implicit_b}.
	\end{empheq} \label{eq:bte_fully_implicit}
\end{subequations} 
We use Newton's method  with line search to solve the nonlinear system by~\Cref{eq:bte_fully_implicit}. The application of the Jacobian, evaluated at $\vect F$, on a given input $\vect X$ is given by
\begin{multline}
	%\vect{J_F} \vect X = -\vect A_x \vect X \vect D_x^T + \vect P_O \of{ \vect C_{en} + \vect E\of{\vect F} \pdot \vect A_v} \vect P_S \vect X  +  \vect P_O \of{\vect E \of{\vect X} \pdot \vect A_v }\vect P_S \vect F \label{eq:1dbte_jac_action}.
	\vect{J_F} \vect X = -\vect A_x \vect X \vect D_x^T + \vect P_O \vect C_{en} \vect P_S \vect X  + \\ \vect P_O \vect A_v \vect P_S \of{E\of{\vect F} \pdot \vect X}  +  \vect P_O \vect A_v \vect P_S \of{\vect E \of{\vect X} \pdot \vect F} \label{eq:1dbte_jac_action}.
\end{multline}
Computing the linear step is, however, a challenge. If we try to assemble the Jacobian, we end up with a dense phase space operator with a prohibitive storage cost. Instead, we use a matrix-free GMRES solver, but it turns out the Jacobian is highly ill-conditioned. The solution is to use preconditioning. For this, we propose to use the linear operator corresponding to the update of the semi-implicit scheme defined in~\Cref{eq:bte_semi_implicit} as a preconditioner. We discuss the efficiency of the solver in the results section.


%This is a challenging system to solve. The key challenges include fully coupled phase space solve, operator assembly and storage cost is expensive, and the BTE equation becomes non-linear due to the $\vect{E}$ field coupling. Also, if we were to use an LU factorization for the Newton step, the cost would be $\mathcal{O}\of{N_r^3 N_x^3 N_{\vtheta}^3}$ which is infeasible for the problem sizes we consider. To mitigate these challenges, we use a matrix-free Newton's method to evolve~\Cref{eq:bte_fully_implicit} where the Jacobian action is given by
%We use GMRES to perform the Jacobian solve for the Newton iteration. We use~\Cref{eq:bte_semi_implicit} as a preconditioner for the GMRES solve of~\Cref{eq:bte_fully_implicit}. 
%We use an operator split similar to semi-implicit scheme as a preconditioner for the fully-implicit solve. 

\subsubsection{Hybrid solver complexity}
Recall that $N_x$, $N_r$, and $N_{\vtheta}$ denote the number of Chebyshev collocation points in position space, B-spline basis functions in $v$, and discrete ordinate points in $\vtheta$. This will result in $N_x(1 + N_r N_{\vtheta})$ degrees of freedom for the hybrid solver. Recall, $N_l$ denotes the number of spherical harmonics used for the mixed Galerkin and collocation discretization scheme used in $\vtheta$ coordinate.The velocity space operators $\vect{C}_{en}$, $\vect{A}_v$ are stored as dense matrices. Operators $\vect{A}_x$, $\vect P_S$, $\vect P_O$ are not assembled, and their actions are performed with tensor contractions with the stored $\vect D_\vtheta$, $\vect{G}_v$, $\vect T_S$, and $\vect T_O$ operators. These operators are stored as dense matrices. \Cref{tab:hybrid_comp_complexity}~summarizes all discretized BTE operators and their dimensions. It also provides the storage complexities for these operators and the computational complexity for their actions.
%As mentioned, the $\vect{v}$-space advection-collision given by~\Cref{eq:bte_vspace} is discretized using $N_l$ spherical harmonics in $\vtheta$. Therefore, this appears in the $\vect{A}_v$ and $\vect C_{en}$ costs. \Cref{tab:hybrid_comp_complexity}~summarizes complexity for storage and action of key BTE operators. 
\begin{table}[!tbhp]
	\centering
	\resizebox{\textwidth}{!}{
	\begin{tabular}{||c|c|c|c|c||}
		\hline
		Operator & Dimensions &  Storage & Action on $\small \vect{X} \in \reals^{N_r\times N_\vtheta \times N_x}$, $\vect X_S \in \reals^{N_r\times N_l \times N_x}$ & Complexity\\
		\hline
		%$\vect{F}$ -- degrees of freedom & $\mathcal{O} (N_x(1+N_rN_{\vtheta}))$ & --\\
		$\vect{D}_x$ & $N_x \times N_x $  & $\mathcal{O} (N_x^2)$ & $[\vect Y]_{ijk} = [\vect D_x]_{im} \times_m [\vect X]_{jkm}$  & $\mathcal{O}\of{N_x^2 N_r N_{\vtheta}}$ \\ [0.05cm]
		$\vect{A}_x=\vect D_\vtheta \otimes \vect G_v$ & $N_rN_\vtheta \times N_r N_\vtheta $  & $\mathcal{O} (N_r^2 + N_{\vtheta})$ & $[\vect Y]_{ijk} = \vect [D_\vtheta]_{jj} \pdot_j [\vect G_v]_{im} \times_m [\vect X]_{mjk}$ & $\mathcal{O} \of{N_x N_{\vtheta} N_r^2}$ \\ [0.05cm]
		$\vect P_S=\vect I_v \otimes \vect T_S$ & $N_r N_l \times N_v N_\vtheta$ & $\mathcal{O}(N_{\vtheta} N_l)$ & $[\vect Y]_{ijk} = [\vect T_S]_{jm} \times_m [\vect X]_{imk}$ & $\mathcal{O}\of{N_x N_r N_l N_\vtheta}$ \\ [0.05cm]
		$\vect P_O=\vect I_v \otimes \vect T_O$ & $N_r N_\vtheta \times N_r N_l$ & $\mathcal{O}(N_{\vtheta} N_l)$ & $[\vect Y]_{ijk} = [\vect T_O]_{jm} \times_m [\vect X_S]_{imk}$ & $\mathcal{O}\of{N_x N_r N_l N_\vtheta}$ \\ [0.05cm]
		$\vect{C}_{en}$ & $N_r N_l \times N_r N_l$ & $\mathcal{O}(N_r^2 N_l^2)$ & $[\vect Y]_{ijk} = [\vect C_{en}]_{ijlm} \times_{lm} [\vect X_S]_{lmk}$ & $\mathcal{O}\of{N_x N_r^2 N_l^2}$ \\ [0.05cm]
		$\vect{A}_{v}$ & $N_r N_l \times N_r N_l$ & $\mathcal{O} (N_r^2 N_l^2)$ & $[\vect Y]_{ijk} = [\vect A_{v}]_{ijlm} \times_{lm} [\vect X_S]_{lmk}$ & $\mathcal{O}\of{N_x N_r^2 N_l^2}$ \\ [0.05cm]
		\hline
	\end{tabular}}
\caption{Summary of storage and computational complexity for the discretized BTE operators. Here $\times_k$ denotes the contraction along index $k$, and $\pdot_k$ denotes element wise product along index $k$. \label{tab:hybrid_comp_complexity}}
\end{table} 

The computational complexities for $\vect A_x \vect F \vect D_x^T$ evaluation is given by $T_{\vect{x}\text{-rhs}} = \mathcal{O}\big(N_r N_{\vtheta} N_x \of{N_r + N_x}\big)$,  and the spatial advection solve in \Cref{alg:bte_spatial_adv} is $T_{\vect{x}\text{-solve}} = \mathcal{O}\of{N_r N_{\vtheta} N_x \of{N_r + N_x}}$. The computation cost for $P_O\of{\vect C_{en} + \vect E\pdot \vect A_v} P_S$ is $T_{\vect{v}\text{-rhs}} = \mathcal{O} \of{\of{2 N_\vtheta + N_r N_l} N_r N_l N_x}$. Application of the preconditioner has a similar costs so that $T_{\vect{v}\text{-precond}} = T_{\vect{v}\text{-rhs}}$. Hence, the overall computational cost for a single BTE time-step for the semi-implicit scheme is given by 
\begin{equation}
	T_{\text{semi-implicit}} =  T_{\vect{x}\text{-solve}} + k(T_{\vect{v}\text{-precond}} + T_{\vect{v}\text{-rhs}}) = T_{\vect{x}\text{-solve}} + 2kT_{\vect{v}\text{-rhs}} \label{eq:semi_imp_ts_cost}.
\end{equation} 
Here, $2kT_{\vect{v}\text{-rhs}}$ denotes the velocity space GMRES solver cost, and $k$ denotes the GMRES iterations for convergence. 
For the fully-implicit scheme, the right-hand side evaluation cost of the Jacobian action given in~\Cref{eq:1dbte_jac_action} is $T_{\vect{vx}\text{-jac}} = T_{\vect{x}\text{-rhs}} + T_{\vect{v}\text{-rhs}} + T_{\vect{E}\text{-solve}}$. Here, $\vect E$ solve cost is $T_{\vect{E}\text{-solve}} = \mathcal{O}(N_r N_\vtheta N_x + N_x^2)$, and since $(T_{\vect{x}\text{-rhs}} + T_{\vect{v}\text{-rhs}}) >> T_{\vect E\text{-solve}}$ it is omitted in the analysis. The operator split preconditioning cost for the fully implicit scheme is $T_{\vect{vx}\text{-precond}} = T_{\vect{x}\text{-rhs}} + T_{\vect{v}\text{-rhs}}$. Taken together, the computational complexity for a single timestep solve of the fully-implicit scheme is given by
\begin{equation}
T_{\text{fully-implicit}} = k(T_{\vect{vx}\text{-precond}} + T_{\vect{vx}\text{-jac}}) = 2k(T_{\vect{x}\text{-rhs}} + T_{\vect{v}\text{-rhs}}) \label{eq:bte_imp_complexity}.
\end{equation} Here, $k$ denotes the total Newton-GMRES iterations for convergence.
Let $(\Delta t_F,k_F)$ and $(\Delta t_S,k_S)$ tuples denote the timestep size and the number of iterations for the described fully-implicit and semi-implicit schemes respectively. The cost ratio $S$ of the semi-implicit to fully-implicit scheme is given by
\begin{equation}
	\frac{\Delta t_F \of{T_{\vect{x}\text{-rhs}} + 2k_S T_{\vect{v}\text{-rhs}}}}{\Delta t_S 2k_F \of{T_{\vect{x}\text{-rhs}} + T_{\vect{v}\text{-rhs}}}} \label{eq:ts_efficiency}.
\end{equation} 
Due to the nonlinearity and ill-conditioning of~\Cref{eq:bte_fully_implicit} it is hard to determine the effectiveness of the fully-implicit scheme. It depends on the preconditioned linear solves and the underlying prefactors in the complexity estimates, some of which are related to memory accesses and hardware performance. Next, we describe a series of numerical experiments to compare the two schemes. %Typically, we need $\Delta t_S \sim \frac{1}{\omega_{e}} < \Delta t_F$.% and the fully-implicit scheme is more efficient than the semi-implicit scheme, when $S > 1$.


\subsection{Implementation}
\label{subsec:implement_details}
We have implemented all the presented algorithms in Python and we release them in a library we call~\bte. We use \texttt{NumPy} and \texttt{CuPy} libraries for all the linear algebra operations. The above libraries provide an interface to basic linear algebra subprograms (BLAS) for CPU and GPU architectures. Both hybrid and the fluid model time integration supports GPU acceleration with \texttt{CuPy}. We use \texttt{CuPy} GMRES solver with \texttt{LinearOperator} class to specify operator and preconditioner actions. The batched GEMM operations are performed using the \texttt{einsum} function. \bte~is available at \url{https://github.com/ut-padas/boltzmann.git}.