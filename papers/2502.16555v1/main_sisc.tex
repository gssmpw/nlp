% SIAM Article Template
\documentclass[final]{siamart250211}

% Information that is shared between the article and the supplement
% (title and author information, macros, packages, etc.) goes into
% ex_shared.tex. If there is no supplement, this file can be included
% directly.

%
% --- inline annotations
%
\newcommand{\red}[1]{{\color{red}#1}}
\newcommand{\todo}[1]{{\color{red}#1}}
\newcommand{\TODO}[1]{\textbf{\color{red}[TODO: #1]}}
% --- disable by uncommenting  
% \renewcommand{\TODO}[1]{}
% \renewcommand{\todo}[1]{#1}



\newcommand{\VLM}{LVLM\xspace} 
\newcommand{\ours}{PeKit\xspace}
\newcommand{\yollava}{Yo’LLaVA\xspace}

\newcommand{\thisismy}{This-Is-My-Img\xspace}
\newcommand{\myparagraph}[1]{\noindent\textbf{#1}}
\newcommand{\vdoro}[1]{{\color[rgb]{0.4, 0.18, 0.78} {[V] #1}}}
% --- disable by uncommenting  
% \renewcommand{\TODO}[1]{}
% \renewcommand{\todo}[1]{#1}
\usepackage{slashbox}
% Vectors
\newcommand{\bB}{\mathcal{B}}
\newcommand{\bw}{\mathbf{w}}
\newcommand{\bs}{\mathbf{s}}
\newcommand{\bo}{\mathbf{o}}
\newcommand{\bn}{\mathbf{n}}
\newcommand{\bc}{\mathbf{c}}
\newcommand{\bp}{\mathbf{p}}
\newcommand{\bS}{\mathbf{S}}
\newcommand{\bk}{\mathbf{k}}
\newcommand{\bmu}{\boldsymbol{\mu}}
\newcommand{\bx}{\mathbf{x}}
\newcommand{\bg}{\mathbf{g}}
\newcommand{\be}{\mathbf{e}}
\newcommand{\bX}{\mathbf{X}}
\newcommand{\by}{\mathbf{y}}
\newcommand{\bv}{\mathbf{v}}
\newcommand{\bz}{\mathbf{z}}
\newcommand{\bq}{\mathbf{q}}
\newcommand{\bff}{\mathbf{f}}
\newcommand{\bu}{\mathbf{u}}
\newcommand{\bh}{\mathbf{h}}
\newcommand{\bb}{\mathbf{b}}

\newcommand{\rone}{\textcolor{green}{R1}}
\newcommand{\rtwo}{\textcolor{orange}{R2}}
\newcommand{\rthree}{\textcolor{red}{R3}}
\usepackage{amsmath}
%\usepackage{arydshln}
\DeclareMathOperator{\similarity}{sim}
\DeclareMathOperator{\AvgPool}{AvgPool}

\newcommand{\argmax}{\mathop{\mathrm{argmax}}}     



%
\setlength\unitlength{1mm}
\newcommand{\twodots}{\mathinner {\ldotp \ldotp}}
% bb font symbols
\newcommand{\Rho}{\mathrm{P}}
\newcommand{\Tau}{\mathrm{T}}

\newfont{\bbb}{msbm10 scaled 700}
\newcommand{\CCC}{\mbox{\bbb C}}

\newfont{\bb}{msbm10 scaled 1100}
\newcommand{\CC}{\mbox{\bb C}}
\newcommand{\PP}{\mbox{\bb P}}
\newcommand{\RR}{\mbox{\bb R}}
\newcommand{\QQ}{\mbox{\bb Q}}
\newcommand{\ZZ}{\mbox{\bb Z}}
\newcommand{\FF}{\mbox{\bb F}}
\newcommand{\GG}{\mbox{\bb G}}
\newcommand{\EE}{\mbox{\bb E}}
\newcommand{\NN}{\mbox{\bb N}}
\newcommand{\KK}{\mbox{\bb K}}
\newcommand{\HH}{\mbox{\bb H}}
\newcommand{\SSS}{\mbox{\bb S}}
\newcommand{\UU}{\mbox{\bb U}}
\newcommand{\VV}{\mbox{\bb V}}


\newcommand{\yy}{\mathbbm{y}}
\newcommand{\xx}{\mathbbm{x}}
\newcommand{\zz}{\mathbbm{z}}
\newcommand{\sss}{\mathbbm{s}}
\newcommand{\rr}{\mathbbm{r}}
\newcommand{\pp}{\mathbbm{p}}
\newcommand{\qq}{\mathbbm{q}}
\newcommand{\ww}{\mathbbm{w}}
\newcommand{\hh}{\mathbbm{h}}
\newcommand{\vvv}{\mathbbm{v}}

% Vectors

\newcommand{\av}{{\bf a}}
\newcommand{\bv}{{\bf b}}
\newcommand{\cv}{{\bf c}}
\newcommand{\dv}{{\bf d}}
\newcommand{\ev}{{\bf e}}
\newcommand{\fv}{{\bf f}}
\newcommand{\gv}{{\bf g}}
\newcommand{\hv}{{\bf h}}
\newcommand{\iv}{{\bf i}}
\newcommand{\jv}{{\bf j}}
\newcommand{\kv}{{\bf k}}
\newcommand{\lv}{{\bf l}}
\newcommand{\mv}{{\bf m}}
\newcommand{\nv}{{\bf n}}
\newcommand{\ov}{{\bf o}}
\newcommand{\pv}{{\bf p}}
\newcommand{\qv}{{\bf q}}
\newcommand{\rv}{{\bf r}}
\newcommand{\sv}{{\bf s}}
\newcommand{\tv}{{\bf t}}
\newcommand{\uv}{{\bf u}}
\newcommand{\wv}{{\bf w}}
\newcommand{\vv}{{\bf v}}
\newcommand{\xv}{{\bf x}}
\newcommand{\yv}{{\bf y}}
\newcommand{\zv}{{\bf z}}
\newcommand{\zerov}{{\bf 0}}
\newcommand{\onev}{{\bf 1}}

% Matrices

\newcommand{\Am}{{\bf A}}
\newcommand{\Bm}{{\bf B}}
\newcommand{\Cm}{{\bf C}}
\newcommand{\Dm}{{\bf D}}
\newcommand{\Em}{{\bf E}}
\newcommand{\Fm}{{\bf F}}
\newcommand{\Gm}{{\bf G}}
\newcommand{\Hm}{{\bf H}}
\newcommand{\Id}{{\bf I}}
\newcommand{\Jm}{{\bf J}}
\newcommand{\Km}{{\bf K}}
\newcommand{\Lm}{{\bf L}}
\newcommand{\Mm}{{\bf M}}
\newcommand{\Nm}{{\bf N}}
\newcommand{\Om}{{\bf O}}
\newcommand{\Pm}{{\bf P}}
\newcommand{\Qm}{{\bf Q}}
\newcommand{\Rm}{{\bf R}}
\newcommand{\Sm}{{\bf S}}
\newcommand{\Tm}{{\bf T}}
\newcommand{\Um}{{\bf U}}
\newcommand{\Wm}{{\bf W}}
\newcommand{\Vm}{{\bf V}}
\newcommand{\Xm}{{\bf X}}
\newcommand{\Ym}{{\bf Y}}
\newcommand{\Zm}{{\bf Z}}

% Calligraphic

\newcommand{\Ac}{{\cal A}}
\newcommand{\Bc}{{\cal B}}
\newcommand{\Cc}{{\cal C}}
\newcommand{\Dc}{{\cal D}}
\newcommand{\Ec}{{\cal E}}
\newcommand{\Fc}{{\cal F}}
\newcommand{\Gc}{{\cal G}}
\newcommand{\Hc}{{\cal H}}
\newcommand{\Ic}{{\cal I}}
\newcommand{\Jc}{{\cal J}}
\newcommand{\Kc}{{\cal K}}
\newcommand{\Lc}{{\cal L}}
\newcommand{\Mc}{{\cal M}}
\newcommand{\Nc}{{\cal N}}
\newcommand{\nc}{{\cal n}}
\newcommand{\Oc}{{\cal O}}
\newcommand{\Pc}{{\cal P}}
\newcommand{\Qc}{{\cal Q}}
\newcommand{\Rc}{{\cal R}}
\newcommand{\Sc}{{\cal S}}
\newcommand{\Tc}{{\cal T}}
\newcommand{\Uc}{{\cal U}}
\newcommand{\Wc}{{\cal W}}
\newcommand{\Vc}{{\cal V}}
\newcommand{\Xc}{{\cal X}}
\newcommand{\Yc}{{\cal Y}}
\newcommand{\Zc}{{\cal Z}}

% Bold greek letters

\newcommand{\alphav}{\hbox{\boldmath$\alpha$}}
\newcommand{\betav}{\hbox{\boldmath$\beta$}}
\newcommand{\gammav}{\hbox{\boldmath$\gamma$}}
\newcommand{\deltav}{\hbox{\boldmath$\delta$}}
\newcommand{\etav}{\hbox{\boldmath$\eta$}}
\newcommand{\lambdav}{\hbox{\boldmath$\lambda$}}
\newcommand{\epsilonv}{\hbox{\boldmath$\epsilon$}}
\newcommand{\nuv}{\hbox{\boldmath$\nu$}}
\newcommand{\muv}{\hbox{\boldmath$\mu$}}
\newcommand{\zetav}{\hbox{\boldmath$\zeta$}}
\newcommand{\phiv}{\hbox{\boldmath$\phi$}}
\newcommand{\psiv}{\hbox{\boldmath$\psi$}}
\newcommand{\thetav}{\hbox{\boldmath$\theta$}}
\newcommand{\tauv}{\hbox{\boldmath$\tau$}}
\newcommand{\omegav}{\hbox{\boldmath$\omega$}}
\newcommand{\xiv}{\hbox{\boldmath$\xi$}}
\newcommand{\sigmav}{\hbox{\boldmath$\sigma$}}
\newcommand{\piv}{\hbox{\boldmath$\pi$}}
\newcommand{\rhov}{\hbox{\boldmath$\rho$}}
\newcommand{\upsilonv}{\hbox{\boldmath$\upsilon$}}

\newcommand{\Gammam}{\hbox{\boldmath$\Gamma$}}
\newcommand{\Lambdam}{\hbox{\boldmath$\Lambda$}}
\newcommand{\Deltam}{\hbox{\boldmath$\Delta$}}
\newcommand{\Sigmam}{\hbox{\boldmath$\Sigma$}}
\newcommand{\Phim}{\hbox{\boldmath$\Phi$}}
\newcommand{\Pim}{\hbox{\boldmath$\Pi$}}
\newcommand{\Psim}{\hbox{\boldmath$\Psi$}}
\newcommand{\Thetam}{\hbox{\boldmath$\Theta$}}
\newcommand{\Omegam}{\hbox{\boldmath$\Omega$}}
\newcommand{\Xim}{\hbox{\boldmath$\Xi$}}


% Sans Serif small case

\newcommand{\Gsf}{{\sf G}}

\newcommand{\asf}{{\sf a}}
\newcommand{\bsf}{{\sf b}}
\newcommand{\csf}{{\sf c}}
\newcommand{\dsf}{{\sf d}}
\newcommand{\esf}{{\sf e}}
\newcommand{\fsf}{{\sf f}}
\newcommand{\gsf}{{\sf g}}
\newcommand{\hsf}{{\sf h}}
\newcommand{\isf}{{\sf i}}
\newcommand{\jsf}{{\sf j}}
\newcommand{\ksf}{{\sf k}}
\newcommand{\lsf}{{\sf l}}
\newcommand{\msf}{{\sf m}}
\newcommand{\nsf}{{\sf n}}
\newcommand{\osf}{{\sf o}}
\newcommand{\psf}{{\sf p}}
\newcommand{\qsf}{{\sf q}}
\newcommand{\rsf}{{\sf r}}
\newcommand{\ssf}{{\sf s}}
\newcommand{\tsf}{{\sf t}}
\newcommand{\usf}{{\sf u}}
\newcommand{\wsf}{{\sf w}}
\newcommand{\vsf}{{\sf v}}
\newcommand{\xsf}{{\sf x}}
\newcommand{\ysf}{{\sf y}}
\newcommand{\zsf}{{\sf z}}


% mixed symbols

\newcommand{\sinc}{{\hbox{sinc}}}
\newcommand{\diag}{{\hbox{diag}}}
\renewcommand{\det}{{\hbox{det}}}
\newcommand{\trace}{{\hbox{tr}}}
\newcommand{\sign}{{\hbox{sign}}}
\renewcommand{\arg}{{\hbox{arg}}}
\newcommand{\var}{{\hbox{var}}}
\newcommand{\cov}{{\hbox{cov}}}
\newcommand{\Ei}{{\rm E}_{\rm i}}
\renewcommand{\Re}{{\rm Re}}
\renewcommand{\Im}{{\rm Im}}
\newcommand{\eqdef}{\stackrel{\Delta}{=}}
\newcommand{\defines}{{\,\,\stackrel{\scriptscriptstyle \bigtriangleup}{=}\,\,}}
\newcommand{\<}{\left\langle}
\renewcommand{\>}{\right\rangle}
\newcommand{\herm}{{\sf H}}
\newcommand{\trasp}{{\sf T}}
\newcommand{\transp}{{\sf T}}
\renewcommand{\vec}{{\rm vec}}
\newcommand{\Psf}{{\sf P}}
\newcommand{\SINR}{{\sf SINR}}
\newcommand{\SNR}{{\sf SNR}}
\newcommand{\MMSE}{{\sf MMSE}}
\newcommand{\REF}{{\RED [REF]}}

% Markov chain
\usepackage{stmaryrd} % for \mkv 
\newcommand{\mkv}{-\!\!\!\!\minuso\!\!\!\!-}

% Colors

\newcommand{\RED}{\color[rgb]{1.00,0.10,0.10}}
\newcommand{\BLUE}{\color[rgb]{0,0,0.90}}
\newcommand{\GREEN}{\color[rgb]{0,0.80,0.20}}

%%%%%%%%%%%%%%%%%%%%%%%%%%%%%%%%%%%%%%%%%%
\usepackage{hyperref}
\hypersetup{
    bookmarks=true,         % show bookmarks bar?
    unicode=false,          % non-Latin characters in AcrobatÕs bookmarks
    pdftoolbar=true,        % show AcrobatÕs toolbar?
    pdfmenubar=true,        % show AcrobatÕs menu?
    pdffitwindow=false,     % window fit to page when opened
    pdfstartview={FitH},    % fits the width of the page to the window
%    pdftitle={My title},    % title
%    pdfauthor={Author},     % author
%    pdfsubject={Subject},   % subject of the document
%    pdfcreator={Creator},   % creator of the document
%    pdfproducer={Producer}, % producer of the document
%    pdfkeywords={keyword1} {key2} {key3}, % list of keywords
    pdfnewwindow=true,      % links in new window
    colorlinks=true,       % false: boxed links; true: colored links
    linkcolor=red,          % color of internal links (change box color with linkbordercolor)
    citecolor=green,        % color of links to bibliography
    filecolor=blue,      % color of file links
    urlcolor=blue           % color of external links
}
%%%%%%%%%%%%%%%%%%%%%%%%%%%%%%%%%%%%%%%%%%%



%% SIAM Shared Information Template
% This is information that is shared between the main document and any
% supplement. If no supplement is required, then this information can
% be included directly in the main document.


% Packages and macros go here
\usepackage{lipsum}
\usepackage{amsfonts}
\usepackage{graphicx}
\usepackage{epstopdf}
%\usepackage{algorithmic}

%%%%% custom settings %%%%%%
\usepackage{tikz, multirow, makecell, booktabs}
\usetikzlibrary{shapes.geometric, arrows}

\usepackage{bm,enumitem,comment}
\usepackage[linesnumbered,ruled,vlined]{algorithm2e}
\renewcommand{\thealgocf}{\thesection.\arabic{algocf}}
\SetAlgoCaptionSeparator{ }

\usepackage[left=1.72in, right=1.65in, top=1.37in, bottom=1.37in]{geometry}
% \usepackage[draft,notref,notcite]{showkeys}
% \makeatletter
% \renewcommand*\showkeyslabelformat[1]{%
%   \llap{\fbox{\normalfont\fontsize{7}{10}\selectfont\ttfamily#1}\hspace{1.5em}}}
% \makeatother

\newcommand{\vertiii}[1]{{\left\vert\kern-0.25ex\left\vert\kern-0.25ex\left\vert #1 \right\vert\kern-0.25ex\right\vert\kern-0.25ex\right\vert}}
%%%%%%%%%%%%%%%%%%%%%%

\ifpdf
  \DeclareGraphicsExtensions{.eps,.pdf,.png,.jpg}
\else
  \DeclareGraphicsExtensions{.eps}
\fi

% Add a serial/Oxford comma by default.
\newcommand{\creflastconjunction}{, and~}

% Used for creating new theorem and remark environments
\newsiamremark{remark}{Remark}
\newsiamremark{hypothesis}{Hypothesis}
\crefname{hypothesis}{Hypothesis}{Hypotheses}
\newsiamthm{claim}{Claim}

% Sets running headers as well as PDF title and authors
\headers{Deep collocation method with error control}{M. Weng, Z. Mao, and J. Shen}

% Title. If the supplement option is on, then "Supplementary Material"
% is automatically inserted before the title.
\title{Deep collocation method: A framework for solving PDEs using neural networks with error control %\thanks{Submitted to the editors DATE.}
}

% Authors: full names plus addresses.
\author{Mingxing Weng\thanks{School of Mathematical Sciences, Shanghai Jiao Tong University,
	Shanghai 200240, China; School of Mathematical Science, Eastern Institute of Technology, Ningbo, 
	Zhejiang 315200, China
	(mxweng22@sjtu.edu.cn)}
\and Zhiping Mao\thanks{School of Mathematical Science, Eastern Institute of Technology, Ningbo, 
		Zhejiang 315200, China
  		(zmao@eitech.edu.cn, jshen@eitech.edu.cn).}
\and Jie Shen\footnotemark[2]
}

\usepackage{amsopn}
\DeclareMathOperator{\diag}{diag}


%%% Local Variables: 
%%% mode:latex
%%% TeX-master: "ex_article"
%%% End: 

\ifpdf
\DeclareGraphicsExtensions{.eps,.pdf,.png,.jpg}
\else
\DeclareGraphicsExtensions{.eps}
\fi

% Add a serial/Oxford comma by default.
%\newcommand{\creflastconjunction}{, and~}

% Used for creating new theorem and remark environments
%\newsiamremark{remark}{Remark}
%\newsiamremark{hypothesis}{Hypothesis}
%\crefname{hypothesis}{Hypothesis}{Hypotheses}
%\newsiamthm{claim}{Claim}
%\newsiamremark{fact}{Fact}
%\crefname{fact}{Fact}{Facts}

% Sets running headers as well as PDF title and authors
\headers{\tiny{\textsc{Boltzsim}: A fast solver for the 1D-space electron Boltzmann equation}}{\tiny{M. Fernando, J. Almgren-Bell, T. Oliver, R. Morser, P. Varghese, L. Raja, and G. Biros}}

% Title. If the supplement option is on, then "Supplementary Material"
% is automatically inserted before the title.
%\title{An Example Article\thanks{Submitted to the editors DATE.
%		\funding{This work was funded by the Fog Research Institute under contract no.~FRI-454.}}}
\title{\textsc{Boltzsim}: A fast solver for the 1D-space electron Boltzmann equation with applications to radio-frequency glow discharge plasmas\thanks{Submitted to the editors February 23, 2025. \funding{This work was funded by the U.S. Department of Energy, National Nuclear Security Administration award number DE-NA0003969}}}

% Authors: full names plus addresses.
% Authors: full names plus addresses.
\author{ Milinda Fernando\thanks{Oden Institute, The University of Texas at Austin (\email{milinda@oden.utexas.edu}, \email{jalmgrenbell@utexas.edu}, \email{oliver@oden.utexas.edu}, \email{rmoser@oden.utexas.edu}, \email{varghese@mail.utexas.edu}, \email{lraja@mail.utexas.edu}, \email{biros@oden.utexas.edu}).}
		\and James Almgren-Bell\footnotemark[1]
		\and Todd Oliver\footnotemark[1] 
		\and Robert Moser\footnotemark[1]
		\and Philip Varghese\footnotemark[1]
		\and Laxminarayan Raja\footnotemark[1]
		\and George Biros\footnotemark[1]}

%\usepackage{amsopn}
%\DeclareMathOperator{\diag}{diag}

% Optional PDF information
\ifpdf
\hypersetup{
  pdftitle={\textsc{Boltzsim}: A fast solver for the 1D-space electron Boltzmann equation with applications to radio-frequency glow discharge plasmas},
  pdfauthor={M. Fernando, J. Almgren-Bell, T. Oliver, R. Morser, P. Varghese, L. Raja, and G. Biros}
}
\fi



% The next statement enables references to information in the
% supplement. See the xr-hyperref package for details.

%\externaldocument[][nocite]{ex_supplement}

% FundRef data to be entered by SIAM
%<funding-group specific-use="FundRef">
%<award-group>
%<funding-source>
%<named-content content-type="funder-name"> 
%</named-content> 
%<named-content content-type="funder-identifier"> 
%</named-content>
%</funding-source>
%<award-id> </award-id>
%</award-group>
%</funding-group>

\begin{document}
\maketitle

% REQUIRED
\begin{abstract}
\begin{abstract}  
Test time scaling is currently one of the most active research areas that shows promise after training time scaling has reached its limits.
Deep-thinking (DT) models are a class of recurrent models that can perform easy-to-hard generalization by assigning more compute to harder test samples.
However, due to their inability to determine the complexity of a test sample, DT models have to use a large amount of computation for both easy and hard test samples.
Excessive test time computation is wasteful and can cause the ``overthinking'' problem where more test time computation leads to worse results.
In this paper, we introduce a test time training method for determining the optimal amount of computation needed for each sample during test time.
We also propose Conv-LiGRU, a novel recurrent architecture for efficient and robust visual reasoning. 
Extensive experiments demonstrate that Conv-LiGRU is more stable than DT, effectively mitigates the ``overthinking'' phenomenon, and achieves superior accuracy.
\end{abstract}  
\end{abstract}

% REQUIRED
\begin{keywords}
Boltzmann equation, Galerkin approach, Multi-term expansion, Low-temperature plasma, glow discharge plasma
\end{keywords}

% REQUIRED
\begin{MSCcodes}
35Q20, 82B40, 82D10
\end{MSCcodes}


\section{Introduction} \label{sec:intro}
\section{Introduction}


\begin{figure}[t]
\centering
\includegraphics[width=0.6\columnwidth]{figures/evaluation_desiderata_V5.pdf}
\vspace{-0.5cm}
\caption{\systemName is a platform for conducting realistic evaluations of code LLMs, collecting human preferences of coding models with real users, real tasks, and in realistic environments, aimed at addressing the limitations of existing evaluations.
}
\label{fig:motivation}
\end{figure}

\begin{figure*}[t]
\centering
\includegraphics[width=\textwidth]{figures/system_design_v2.png}
\caption{We introduce \systemName, a VSCode extension to collect human preferences of code directly in a developer's IDE. \systemName enables developers to use code completions from various models. The system comprises a) the interface in the user's IDE which presents paired completions to users (left), b) a sampling strategy that picks model pairs to reduce latency (right, top), and c) a prompting scheme that allows diverse LLMs to perform code completions with high fidelity.
Users can select between the top completion (green box) using \texttt{tab} or the bottom completion (blue box) using \texttt{shift+tab}.}
\label{fig:overview}
\end{figure*}

As model capabilities improve, large language models (LLMs) are increasingly integrated into user environments and workflows.
For example, software developers code with AI in integrated developer environments (IDEs)~\citep{peng2023impact}, doctors rely on notes generated through ambient listening~\citep{oberst2024science}, and lawyers consider case evidence identified by electronic discovery systems~\citep{yang2024beyond}.
Increasing deployment of models in productivity tools demands evaluation that more closely reflects real-world circumstances~\citep{hutchinson2022evaluation, saxon2024benchmarks, kapoor2024ai}.
While newer benchmarks and live platforms incorporate human feedback to capture real-world usage, they almost exclusively focus on evaluating LLMs in chat conversations~\citep{zheng2023judging,dubois2023alpacafarm,chiang2024chatbot, kirk2024the}.
Model evaluation must move beyond chat-based interactions and into specialized user environments.



 

In this work, we focus on evaluating LLM-based coding assistants. 
Despite the popularity of these tools---millions of developers use Github Copilot~\citep{Copilot}---existing
evaluations of the coding capabilities of new models exhibit multiple limitations (Figure~\ref{fig:motivation}, bottom).
Traditional ML benchmarks evaluate LLM capabilities by measuring how well a model can complete static, interview-style coding tasks~\citep{chen2021evaluating,austin2021program,jain2024livecodebench, white2024livebench} and lack \emph{real users}. 
User studies recruit real users to evaluate the effectiveness of LLMs as coding assistants, but are often limited to simple programming tasks as opposed to \emph{real tasks}~\citep{vaithilingam2022expectation,ross2023programmer, mozannar2024realhumaneval}.
Recent efforts to collect human feedback such as Chatbot Arena~\citep{chiang2024chatbot} are still removed from a \emph{realistic environment}, resulting in users and data that deviate from typical software development processes.
We introduce \systemName to address these limitations (Figure~\ref{fig:motivation}, top), and we describe our three main contributions below.


\textbf{We deploy \systemName in-the-wild to collect human preferences on code.} 
\systemName is a Visual Studio Code extension, collecting preferences directly in a developer's IDE within their actual workflow (Figure~\ref{fig:overview}).
\systemName provides developers with code completions, akin to the type of support provided by Github Copilot~\citep{Copilot}. 
Over the past 3 months, \systemName has served over~\completions suggestions from 10 state-of-the-art LLMs, 
gathering \sampleCount~votes from \userCount~users.
To collect user preferences,
\systemName presents a novel interface that shows users paired code completions from two different LLMs, which are determined based on a sampling strategy that aims to 
mitigate latency while preserving coverage across model comparisons.
Additionally, we devise a prompting scheme that allows a diverse set of models to perform code completions with high fidelity.
See Section~\ref{sec:system} and Section~\ref{sec:deployment} for details about system design and deployment respectively.



\textbf{We construct a leaderboard of user preferences and find notable differences from existing static benchmarks and human preference leaderboards.}
In general, we observe that smaller models seem to overperform in static benchmarks compared to our leaderboard, while performance among larger models is mixed (Section~\ref{sec:leaderboard_calculation}).
We attribute these differences to the fact that \systemName is exposed to users and tasks that differ drastically from code evaluations in the past. 
Our data spans 103 programming languages and 24 natural languages as well as a variety of real-world applications and code structures, while static benchmarks tend to focus on a specific programming and natural language and task (e.g. coding competition problems).
Additionally, while all of \systemName interactions contain code contexts and the majority involve infilling tasks, a much smaller fraction of Chatbot Arena's coding tasks contain code context, with infilling tasks appearing even more rarely. 
We analyze our data in depth in Section~\ref{subsec:comparison}.



\textbf{We derive new insights into user preferences of code by analyzing \systemName's diverse and distinct data distribution.}
We compare user preferences across different stratifications of input data (e.g., common versus rare languages) and observe which affect observed preferences most (Section~\ref{sec:analysis}).
For example, while user preferences stay relatively consistent across various programming languages, they differ drastically between different task categories (e.g. frontend/backend versus algorithm design).
We also observe variations in user preference due to different features related to code structure 
(e.g., context length and completion patterns).
We open-source \systemName and release a curated subset of code contexts.
Altogether, our results highlight the necessity of model evaluation in realistic and domain-specific settings.





%
\section{Background}
\label{sec:background}
\section{Background}\label{sec:backgrnd}

\subsection{Cold Start Latency and Mitigation Techniques}

Traditional FaaS platforms mitigate cold starts through snapshotting, lightweight virtualization, and warm-state management. Snapshot-based methods like \textbf{REAP} and \textbf{Catalyzer} reduce initialization time by preloading or restoring container states but require significant memory and I/O resources, limiting scalability~\cite{dong_catalyzer_2020, ustiugov_benchmarking_2021}. Lightweight virtualization solutions, such as \textbf{Firecracker} microVMs, achieve fast startup times with strong isolation but depend on robust infrastructure, making them less adaptable to fluctuating workloads~\cite{agache_firecracker_2020}. Warm-state management techniques like \textbf{Faa\$T}~\cite{romero_faa_2021} and \textbf{Kraken}~\cite{vivek_kraken_2021} keep frequently invoked containers ready, balancing readiness and cost efficiency under predictable workloads but incurring overhead when demand is erratic~\cite{romero_faa_2021, vivek_kraken_2021}. While these methods perform well in resource-rich cloud environments, their resource intensity challenges applicability in edge settings.

\subsubsection{Edge FaaS Perspective}

In edge environments, cold start mitigation emphasizes lightweight designs, resource sharing, and hybrid task distribution. Lightweight execution environments like unikernels~\cite{edward_sock_2018} and \textbf{Firecracker}~\cite{agache_firecracker_2020}, as used by \textbf{TinyFaaS}~\cite{pfandzelter_tinyfaas_2020}, minimize resource usage and initialization delays but require careful orchestration to avoid resource contention. Function co-location, demonstrated by \textbf{Photons}~\cite{v_dukic_photons_2020}, reduces redundant initializations by sharing runtime resources among related functions, though this complicates isolation in multi-tenant setups~\cite{v_dukic_photons_2020}. Hybrid offloading frameworks like \textbf{GeoFaaS}~\cite{malekabbasi_geofaas_2024} balance edge-cloud workloads by offloading latency-tolerant tasks to the cloud and reserving edge resources for real-time operations, requiring reliable connectivity and efficient task management. These edge-specific strategies address cold starts effectively but introduce challenges in scalability and orchestration.

\subsection{Predictive Scaling and Caching Techniques}

Efficient resource allocation is vital for maintaining low latency and high availability in serverless platforms. Predictive scaling and caching techniques dynamically provision resources and reduce cold start latency by leveraging workload prediction and state retention.
Traditional FaaS platforms use predictive scaling and caching to optimize resources, employing techniques (OFC, FaasCache) to reduce cold starts. However, these methods rely on centralized orchestration and workload predictability, limiting their effectiveness in dynamic, resource-constrained edge environments.



\subsubsection{Edge FaaS Perspective}

Edge FaaS platforms adapt predictive scaling and caching techniques to constrain resources and heterogeneous environments. \textbf{EDGE-Cache}~\cite{kim_delay-aware_2022} uses traffic profiling to selectively retain high-priority functions, reducing memory overhead while maintaining readiness for frequent requests. Hybrid frameworks like \textbf{GeoFaaS}~\cite{malekabbasi_geofaas_2024} implement distributed caching to balance resources between edge and cloud nodes, enabling low-latency processing for critical tasks while offloading less critical workloads. Machine learning methods, such as clustering-based workload predictors~\cite{gao_machine_2020} and GRU-based models~\cite{guo_applying_2018}, enhance resource provisioning in edge systems by efficiently forecasting workload spikes. These innovations effectively address cold start challenges in edge environments, though their dependency on accurate predictions and robust orchestration poses scalability challenges.

\subsection{Decentralized Orchestration, Function Placement, and Scheduling}

Efficient orchestration in serverless platforms involves workload distribution, resource optimization, and performance assurance. While traditional FaaS platforms rely on centralized control, edge environments require decentralized and adaptive strategies to address unique challenges such as resource constraints and heterogeneous hardware.



\subsubsection{Edge FaaS Perspective}

Edge FaaS platforms adopt decentralized and adaptive orchestration frameworks to meet the demands of resource-constrained environments. Systems like \textbf{Wukong} distribute scheduling across edge nodes, enhancing data locality and scalability while reducing network latency. Lightweight frameworks such as \textbf{OpenWhisk Lite}~\cite{kravchenko_kpavelopenwhisk-light_2024} optimize resource allocation by decentralizing scheduling policies, minimizing cold starts and latency in edge setups~\cite{benjamin_wukong_2020}. Hybrid solutions like \textbf{OpenFaaS}~\cite{noauthor_openfaasfaas_2024} and \textbf{EdgeMatrix}~\cite{shen_edgematrix_2023} combine edge-cloud orchestration to balance resource utilization, retaining latency-sensitive functions at the edge while offloading non-critical workloads to the cloud. While these approaches improve flexibility, they face challenges in maintaining coordination and ensuring consistent performance across distributed nodes.


%
\section{Methodology} \label{sec:methodology}
\section{Research Methodology}~\label{sec:Methodology}

In this section, we discuss the process of conducting our systematic review, e.g., our search strategy for data extraction of relevant studies, based on the guidelines of Kitchenham et al.~\cite{kitchenham2022segress} to conduct SLRs and Petersen et al.~\cite{PETERSEN20151} to conduct systematic mapping studies (SMSs) in Software Engineering. In this systematic review, we divide our work into a four-stage procedure, including planning, conducting, building a taxonomy, and reporting the review, illustrated in Fig.~\ref{fig:search}. The four stages are as follows: (1) the \emph{planning} stage involved identifying research questions (RQs) and specifying the detailed research plan for the study; (2) the \emph{conducting} stage involved analyzing and synthesizing the existing primary studies to answer the research questions; (3) the \emph{taxonomy} stage was introduced to optimize the data extraction results and consolidate a taxonomy schema for REDAST methodology; (4) the \emph{reporting} stage involved the reviewing, concluding and reporting the final result of our study.

\begin{figure}[!t]
    \centering
    \includegraphics[width=1\linewidth]{fig/methodology/searching-process.drawio.pdf}
    \caption{Systematic Literature Review Process}
    \label{fig:search}
\end{figure}

\subsection{Research Questions}
In this study, we developed five research questions (RQs) to identify the input and output, analyze technologies, evaluate metrics, identify challenges, and identify potential opportunities. 

\textbf{RQ1. What are the input configurations, formats, and notations used in the requirements in requirements-driven
automated software testing?} In requirements-driven testing, the input is some form of requirements specification -- which can vary significantly. RQ1 maps the input for REDAST and reports on the comparison among different formats for requirements specification.

\textbf{RQ2. What are the frameworks, tools, processing methods, and transformation techniques used in requirements-driven automated software testing studies?} RQ2 explores the technical solutions from requirements to generated artifacts, e.g., rule-based transformation applying natural language processing (NLP) pipelines and deep learning (DL) techniques, where we additionally discuss the potential intermediate representation and additional input for the transformation process.

\textbf{RQ3. What are the test formats and coverage criteria used in the requirements-driven automated software
testing process?} RQ3 focuses on identifying the formulation of generated artifacts (i.e., the final output). We map the adopted test formats and analyze their characteristics in the REDAST process.

\textbf{RQ4. How do existing studies evaluate the generated test artifacts in the requirements-driven automated software testing process?} RQ4 identifies the evaluation datasets, metrics, and case study methodologies in the selected papers. This aims to understand how researchers assess the effectiveness, accuracy, and practical applicability of the generated test artifacts.

\textbf{RQ5. What are the limitations and challenges of existing requirements-driven automated software testing methods in the current era?} RQ5 addresses the limitations and challenges of existing studies while exploring future directions in the current era of technology development. %It particularly highlights the potential benefits of advanced LLMs and examines their capacity to meet the high expectations placed on these cutting-edge language modeling technologies. %\textcolor{blue}{CA: Do we really need to focus on LLMs? TBD.} \textcolor{orange}{FW: About LLMs, I removed the direct emphase in RQ5 but kept the discussion in RQ5 and the solution section. I think that would be more appropriate.}

\subsection{Searching Strategy}

The overview of the search process is exhibited in Fig. \ref{fig:papers}, which includes all the details of our search steps.
\begin{table}[!ht]
\caption{List of Search Terms}
\label{table:search_term}
\begin{tabularx}{\textwidth}{lX}
\hline
\textbf{Terms Group} & \textbf{Terms} \\ \hline
Test Group & test* \\
Requirement Group & requirement* OR use case* OR user stor* OR specification* \\
Software Group & software* OR system* \\
Method Group & generat* OR deriv* OR map* OR creat* OR extract* OR design* OR priorit* OR construct* OR transform* \\ \hline
\end{tabularx}
\end{table}

\begin{figure}
    \centering
    \includegraphics[width=1\linewidth]{fig/methodology/search-papers.drawio.pdf}
    \caption{Study Search Process}
    \label{fig:papers}
\end{figure}

\subsubsection{Search String Formulation}
Our research questions (RQs) guided the identification of the main search terms. We designed our search string with generic keywords to avoid missing out on any related papers, where four groups of search terms are included, namely ``test group'', ``requirement group'', ``software group'', and ``method group''. In order to capture all the expressions of the search terms, we use wildcards to match the appendix of the word, e.g., ``test*'' can capture ``testing'', ``tests'' and so on. The search terms are listed in Table~\ref{table:search_term}, decided after iterative discussion and refinement among all the authors. As a result, we finally formed the search string as follows:


\hangindent=1.5em
 \textbf{ON ABSTRACT} ((``test*'') \textbf{AND} (``requirement*'' \textbf{OR} ``use case*'' \textbf{OR} ``user stor*'' \textbf{OR} ``specifications'') \textbf{AND} (``software*'' \textbf{OR} ``system*'') \textbf{AND} (``generat*'' \textbf{OR} ``deriv*'' \textbf{OR} ``map*'' \textbf{OR} ``creat*'' \textbf{OR} ``extract*'' \textbf{OR} ``design*'' \textbf{OR} ``priorit*'' \textbf{OR} ``construct*'' \textbf{OR} ``transform*''))

The search process was conducted in September 2024, and therefore, the search results reflect studies available up to that date. We conducted the search process on six online databases: IEEE Xplore, ACM Digital Library, Wiley, Scopus, Web of Science, and Science Direct. However, some databases were incompatible with our default search string in the following situations: (1) unsupported for searching within abstract, such as Scopus, and (2) limited search terms, such as ScienceDirect. Here, for (1) situation, we searched within the title, keyword, and abstract, and for (2) situation, we separately executed the search and removed the duplicate papers in the merging process. 

\subsubsection{Automated Searching and Duplicate Removal}
We used advanced search to execute our search string within our selected databases, following our designed selection criteria in Table \ref{table:selection}. The first search returned 27,333 papers. Specifically for the duplicate removal, we used a Python script to remove (1) overlapped search results among multiple databases and (2) conference or workshop papers, also found with the same title and authors in the other journals. After duplicate removal, we obtained 21,652 papers for further filtering.

\begin{table*}[]
\caption{Selection Criteria}
\label{table:selection}
\begin{tabularx}{\textwidth}{lX}
\hline
\textbf{Criterion ID} & \textbf{Criterion Description} \\ \hline
S01          & Papers written in English. \\
S02-1        & Papers in the subjects of "Computer Science" or "Software Engineering". \\
S02-2        & Papers published on software testing-related issues. \\
S03          & Papers published from 1991 to the present. \\ 
S04          & Papers with accessible full text. \\ \hline
\end{tabularx}
\end{table*}

\begin{table*}[]
\small
\caption{Inclusion and Exclusion Criteria}
\label{table:criteria}
\begin{tabularx}{\textwidth}{lX}
\hline
\textbf{ID}  & \textbf{Description} \\ \hline
\multicolumn{2}{l}{\textbf{Inclusion Criteria}} \\ \hline
I01 & Papers about requirements-driven automated system testing or acceptance testing generation, or studies that generate system-testing-related artifacts. \\
I02 & Peer-reviewed studies that have been used in academia with references from literature. \\ \hline
\multicolumn{2}{l}{\textbf{Exclusion Criteria}} \\ \hline
E01 & Studies that only support automated code generation, but not test-artifact generation. \\
E02 & Studies that do not use requirements-related information as an input. \\
E03 & Papers with fewer than 5 pages (1-4 pages). \\
E04 & Non-primary studies (secondary or tertiary studies). \\
E05 & Vision papers and grey literature (unpublished work), books (chapters), posters, discussions, opinions, keynotes, magazine articles, experience, and comparison papers. \\ \hline
\end{tabularx}
\end{table*}

\subsubsection{Filtering Process}

In this step, we filtered a total of 21,652 papers using the inclusion and exclusion criteria outlined in Table \ref{table:criteria}. This process was primarily carried out by the first and second authors. Our criteria are structured at different levels, facilitating a multi-step filtering process. This approach involves applying various criteria in three distinct phases. We employed a cross-verification method involving (1) the first and second authors and (2) the other authors. Initially, the filtering was conducted separately by the first and second authors. After cross-verifying their results, the results were then reviewed and discussed further by the other authors for final decision-making. We widely adopted this verification strategy within the filtering stages. During the filtering process, we managed our paper list using a BibTeX file and categorized the papers with color-coding through BibTeX management software\footnote{\url{https://bibdesk.sourceforge.io/}}, i.e., “red” for irrelevant papers, “yellow” for potentially relevant papers, and “blue” for relevant papers. This color-coding system facilitated the organization and review of papers according to their relevance.

The screening process is shown below,
\begin{itemize}
    \item \textbf{1st-round Filtering} was based on the title and abstract, using the criteria I01 and E01. At this stage, the number of papers was reduced from 21,652 to 9,071.
    \item \textbf{2nd-round Filtering}. We attempted to include requirements-related papers based on E02 on the title and abstract level, which resulted from 9,071 to 4,071 papers. We excluded all the papers that did not focus on requirements-related information as an input or only mentioned the term ``requirements'' but did not refer to the requirements specification.
    \item \textbf{3rd-round Filtering}. We selectively reviewed the content of papers identified as potentially relevant to requirements-driven automated test generation. This process resulted in 162 papers for further analysis.
\end{itemize}
Note that, especially for third-round filtering, we aimed to include as many relevant papers as possible, even borderline cases, according to our criteria. The results were then discussed iteratively among all the authors to reach a consensus.

\subsubsection{Snowballing}

Snowballing is necessary for identifying papers that may have been missed during the automated search. Following the guidelines by Wohlin~\cite{wohlin2014guidelines}, we conducted both forward and backward snowballing. As a result, we identified 24 additional papers through this process.

\subsubsection{Data Extraction}

Based on the formulated research questions (RQs), we designed 38 data extraction questions\footnote{\url{https://drive.google.com/file/d/1yjy-59Juu9L3WHaOPu-XQo-j-HHGTbx_/view?usp=sharing}} and created a Google Form to collect the required information from the relevant papers. The questions included 30 short-answer questions, six checkbox questions, and two selection questions. The data extraction was organized into five sections: (1) basic information: fundamental details such as title, author, venue, etc.; (2) open information: insights on motivation, limitations, challenges, etc.; (3) requirements: requirements format, notation, and related aspects; (4) methodology: details, including immediate representation and technique support; (5) test-related information: test format(s), coverage, and related elements. Similar to the filtering process, the first and second authors conducted the data extraction and then forwarded the results to the other authors to initiate the review meeting.

\subsubsection{Quality Assessment}

During the data extraction process, we encountered papers with insufficient information. To address this, we conducted a quality assessment in parallel to ensure the relevance of the papers to our objectives. This approach, also adopted in previous secondary studies~\cite{shamsujjoha2021developing, naveed2024model}, involved designing a set of assessment questions based on guidelines by Kitchenham et al.~\cite{kitchenham2022segress}. The quality assessment questions in our study are shown below:
\begin{itemize}
    \item \textbf{QA1}. Does this study clearly state \emph{how} requirements drive automated test generation?
    \item \textbf{QA2}. Does this study clearly state the \emph{aim} of REDAST?
    \item \textbf{QA3}. Does this study enable \emph{automation} in test generation?
    \item \textbf{QA4}. Does this study demonstrate the usability of the method from the perspective of methodology explanation, discussion, case examples, and experiments?
\end{itemize}
QA4 originates from an open perspective in the review process, where we focused on evaluation, discussion, and explanation. Our review also examined the study’s overall structure, including the methodology description, case studies, experiments, and analyses. The detailed results of the quality assessment are provided in the Appendix. Following this assessment, the final data extraction was based on 156 papers.

% \begin{table}[]
% \begin{tabular}{ll}
% \hline
% QA ID & QA Questions                                             \\ \hline
% Q01   & Does this study clearly state its aims?                  \\
% Q02   & Does this study clearly describe its methodology?        \\
% Q03   & Does this study involve automated test generation?       \\
% Q04   & Does this study include a promising evaluation?          \\
% Q05   & Does this study demonstrate the usability of the method? \\ \hline
% \end{tabular}%
% \caption{Questions for Quality Assessment}
% \label{table:qa}
% \end{table}

% automated quality assessment

% \textcolor{blue}{CA: Our search strategy focused on identifying requirements types first. We covered several sources, e.g., ~\cite{Pohl:11,wagner2019status} to identify different formats and notations of specifying requirements. However, this came out to be a long list, e.g., free-form NL requirements, semi-formal UML models, free-from textual use case models, UML class diagrams, UML activity diagrams, and so on. In this paper, we attempted to primarily focus on requirements-related aspects and not design-level information. Hence, we generalised our search string to include generic keywords, e.g., requirement*, use case*, and user stor*. We did so to avoid missing out on any papers, bringing too restrictive in our search strategy, and not creating a too-generic search string with all the aforementioned formats to avoid getting results beyond our review's scope.}


%% Use \subsection commands to start a subsection.



%\subsection{Study Selection}

% In this step, we further looked into the content of searched papers using our search strategy and applied our inclusion and exclusion criteria. Our filtering strategy aimed to pinpoint studies focused on requirements-driven system-level testing. Recognizing the presence of irrelevant papers in our search results, we established detailed selection criteria for preliminary inclusion and exclusion, as shown in Table \ref{table: criteria}. Specifically, we further developed the taxonomy schema to exclude two types of studies that did not meet the requirements for system-level testing: (1) studies supporting specification-driven test generation, such as UML-driven test generation, rather than requirements-driven testing, and (2) studies focusing on code-based test generation, such as requirement-driven code generation for unit testing.




%\subsection{Timescales analysis}
%Here, we present timescales analysis for \Cref{eq:glow_hybrid,eq:fluid_model_eqs}. \Cref{tab:ts_analysis}~summarizes different ion and electron timescales for a RF-GDP with $p_0=1$ Torr.  Considering the plasma oscillation frequencies for electron $\omega_{e} = \sqrt{q_e^2 n_e /m_e/\epsilon_0}$ and ions $\omega_{i} = \sqrt{q_e^2 n_i /m_i/\epsilon_0}$ we can write $\omega_{e} / \omega_{i} = \of{m_i / m_e}^{1/2} \of{n_e / n_i}^{1/2}$. For GDPs, typically $n_e\sim n_i \sim 10^{17}$, $f_{e} \approx 8.9 \times 10^{9}$ $s^{-1}$ and $f_{i} \approx 3.3 \times 10^{7}$ $s^{-1}$. Electrons have faster time scales than ions due to the small mass ratio between electrons and ions, $m_e/m_i = 1.36\times 10^{-5}$. The input voltage oscillation period is set to $7.374 \times 10^{-7}$ s for all RF-GDP cases considered in this paper.
%\begin{table}[htb]
%	\centering
%	\resizebox{\textwidth}{!}{
%	\begin{tabular}{||c|c|c|c||}
%		\hline 
%		Formulation & Species  & Term & $\Delta t$ [s]\\
%		\hline
%		\multirow{6}{*}{Fluid} & \multirow{3}{*}{Ions} 		& $\nabla_{\vect{x}} \cdot \of{\mu_i \vect{E}_{\text{}} n_i}$ -- advection 		 & $\frac{\Delta x}{\mu_i \norm{\vect{E}_{\text{max}}}} \approx \mathcal{O}(10^{-9}) $ \\ [0.05cm]		
%							   &  					   		& $\nabla_{\vect{x}} \cdot \of{D_i \nabla_{\vect{x}} n_i}$ -- diffusion  & $\frac{\Delta x^2}{ 2 D_i } \approx \mathcal{O}(10^{-7}) $  \\ [0.05cm]
%							   &                       		& $k_i n_0 n_e$ -- reaction   											 & $ \frac{\Delta n_i}{k_{i,\text{max}} n_0 n_e} \approx \mathcal{O}(10^{-9})$   \\ [0.05cm]
%							   \cline{2-4}
%							   & \multirow{3}{*}{Electrons} & $\nabla_{\vect{x}} \cdot \of{\mu_e \vect{E} n_e}$ -- advection 		 & $\frac{\Delta x}{\mu_e \norm{\vect{E}_{\text{max}}}} \approx \mathcal{O}(10^{-11})$  \\ [0.1cm]		
%							   &  					   		& $\nabla_{\vect{x}} \cdot \of{D_e \nabla_{\vect{x}} n_e}$ -- diffusion  & $\frac{\Delta x^2}{ 2 D_e } \approx \mathcal{O}(10^{-12})$  \\ [0.1cm]
%							   &                       		& $k_i n_0 n_e$ -- reaction   											 & $\frac{\Delta n_e}{k_{i,\text{max}} n_0 n_e} \approx \mathcal{O}(10^{-11})$   \\ [0.1cm]
%		\hline
%		\hline
%		\multirow{6}{*}{Hybrid} & \multirow{3}{*}{Ions} 	& $\nabla_{\vect{x}} \cdot \of{\mu_i \vect{E} n_i}$ -- advection 		 & $\frac{\Delta x}{\mu_i \norm{\vect{E}_{\text{max}}}} \approx \mathcal{O}(10^{-9})$  \\		[0.1cm]
%								&  					   		& $\nabla_{\vect{x}} \cdot \of{D_i \nabla_{\vect{x}} n_i}$ -- diffusion  & $\frac{\Delta x^2}{ 2 D_i } \approx \mathcal{O}(10^{-7}) $  \\ 	[0.1cm]
%								&                       	& $k_i n_0 n_e$ -- reaction   											 & $\frac{\Delta n_e}{k_{i,\text{max}} n_0 n_e} \approx \mathcal{O}(10^{-9}) $   \\ [0.1cm]
%								\cline{2-4}
%								&\multirow{3}{*}{Electron BTE} & $\vect{v} \cdot \nabla_{\vect{x}} f$ -- $\vect{x}$-advection 		 				 & $\frac{\Delta x}{\norm{\vect{v}}} \approx \mathcal{O}(10^{-12})$\\ [0.1cm]
%								& 							   & $\frac{q_e \vect{E}}{m_e} \cdot \nabla_{\vect{v}} f$ -- $\vect{v}$-advection 		 & $\frac{\norm{\Delta \vect v} m_e}{q_e\norm{\vect{E}_{\text{max}}}} \approx \mathcal{O}(10^{-12})$\\ 	[0.1cm]
%								& 							   &  $k_i n_0 n_e = n_0 \int_{\vect{v}} \norm{\vect v}^3 \sigma_{\text{ion}}\of{\norm{\vect v}} f\of{\vect{v}} \diff{\vect{v}}$ -- reaction & $\frac{\Delta n_e}{k_{i,\text{max}} n_0 n_e} \approx \mathcal{O}(10^{-11})$ \\ [0.1cm]
%		\hline
%	\end{tabular}}
%\caption{A summary of the timescales for advection, diffusion, and reaction terms for the fluid and the hybrid formulations, assuming an explicit time integration scheme.  Here, we assume $\frac{\Delta x}{L}\approx \mathcal{O}(10^{-3})$, $\frac{\norm{\Delta \vect{v}}}{\norm{\vect v}} \approx \mathcal{O}(10^{-3})$, $\norm{\vect E_{\text{max}}}=8\times10^{4}$ [V/m], $k_{i,\text{max}}=10^{-14}$ [$m^3/s$], and the transport coefficients summarized in \Cref{t:model_parameters} with $p_0=$  1 Torr case \label{tab:ts_analysis}.}
%\end{table}


%taking $n_e \sim n_i$.
%The drift-diffusion approximation provides a closed expression for the mass continuity flux terms given by $\vect{v}_{e}n_{e}$ and $\vect{v}_{i}n_{i}$. Hence, continuity equations for electrons have significantly faster timescales compared to the ions.
\subsection{Numerical solver for the fluid approximation}
\label{subsec:fluid_solver}
As our focus is on the numerics of the hybrid solver, here we give a high-level description of the fluid solver. We are not claiming that this is the most efficient solver for this problem. Instead, we focus on comparing the converged time-periodic solutions between the fluid and the hybrid approaches. For the results of the fluid solver, we performed self-convergence tests in space and time. Now let us describe the solver for the fluid model described in \Cref{eq:fluid_model_eqs}. We use a Chebyshev collocation method for the spatial discretization. These collocation points are clustered to the electrodes. This is ideal for efficiently resolving sharp gradients of the electric field, species densities, and the electron energy in the sheath.  
%sheath profiles with sharp gradients near the electrodes. 
For the time discretization, we found that the operator-split approximation-based semi-implicit methods for solving~\Cref{e:fl_b,e:fl_c,e:fl_d} have timestep size restrictions determined by the electron timescales (see \Cref{subsubsec:ts_analysis}). Therefore, the system given in \Cref{eq:fluid_model_eqs} is solved as a coupled non-linear system with fully implicit time integration. We use Newton's method with direct factorization of the assembled Jacobian to solve for the Newton step. 
%The overall computational complexity for a single timestep is $\mathcal{O}(N_x^3)$ where $N_x$ denotes the number of collocation points in space. 
%The direct LU factorization used in the Newton step has $\mathcal{O}(N_x^3)$ computational complexity, but we found it quite robust. Also, the factorization costs are insignificant for the target problem sizes, $N_x=$ 200 to 400.
%The use of direct factorization is justified by the problem size (i.e., $N_x=200$ for medium resolution run, and $N_x=400$ for high resolution run).  



%Note that using Chebyshev collocation points naturally creates an adaptive grid in space where the higher resolution points are closer to the glow discharge walls. The above is useful to resolve the sheath region efficiently. 

\subsection{Hybrid fluid-BTE solver}
\label{subsec:pde_solver}
Here, we discuss the discretization of~\Cref{eq:glow_hybrid}. First, let us consider the velocity space discretization of \Cref{eq:hybrid_cnt_b}. The rate of change in the EDF due to the velocity space advection and collisions is given by \Cref{eq:bte-0d}. For the velocity space, we use spherical coordinates with a mixed Galerkin and collocation scheme for the angular direction and a Galerkin scheme for the radial coordinate discretization. We discretize $f(\vect{v}, t)$ as follows:
\begin{equation}
	f(\vect{v}, t) = \sum_{klm} f_{klm}\of{t} \Phi_{klm}\of{\vect{v}} \text{ where } \Phi_{klm}\of{\vect{v}}  = \underbrace{\phi_k\of{v}}_{\text{B-Spline basis}} \overbrace{Y_{lm}\of{v_\theta, v_\phi}}^{\tiny\text{spherical harmonics}}, \label{eq:f_expansion}
\end{equation} where $\biggl\{\phi_{k}\of{v}\biggr\}_{k=0}^{N_r-1}$ are cubic B-splines defined on a regular grid, and $Y_{lm}\of{\vtheta, \vphi}$ are defined as 
\begin{align}
	Y_{lm}\of{\vtheta,\vphi} &= U_{lm} P^{|m|}_l\of{\cos\of{\vtheta}} \alpha_m\of{\vphi} \text{ , } \nonumber \\
	U_{lm} = 
	\begin{cases}
		(-1)^m \sqrt{2} \sqrt{\frac{2l+1}{4\pi} \frac{(l-|m|)!}{(l+|m|)!}}, &m\neq 0, \\
		\sqrt{\frac{2l+1}{4\pi}}, &m = 0,
	\end{cases} \text{ , } &
	\alpha_m\of{\vphi}  =
	\begin{cases}
		\sin\of{|m|\phi}, &m < 0, \\
		1, &m = 0,\\
		\cos\of{m\phi}, &m > 0.
	\end{cases} \label{eq:sph_harmonics}
\end{align}
The $l$ index is also referred to as a polar mode, and the $m$ index as an azimuthal mode. Now, assume that $f_{klm}\of{t=0}=0,\  \forall m>0$. By aligning $\vect{E}$ to the velocity space z-axis, so that $\vect{E} = E \vect{\hat{e}_z} =  E \of{\cos(\vtheta) \vect{\hat{e}_r} - \sin(\vtheta)\vect{\hat{e}_\theta}}$ we ensure that the $\vect{E}$ acceleration excites only the polar modes~\Cref{eq:f_expansion}. This fact and the isotropic scattering assumption ensure that the BTE solutions preserve the azimuthal symmetry in the velocity space and thus $f_{klm}\of{t}$ remains zero $\forall t>0,\ m>0$. This essentially reduces the representation from 1D3V to 1D2V. Therefore, the EDF representation only requires $\{Y_{l0}\}_{l}$ modes. For this reason, we drop the azimuthal index $v_{\phi}$ in the remainder of the paper. Let $p$, $k$ denote the indices of the basis functions along the radial coordinate; $q$, $l$ denote the indices in the polar angle; and $\phi_p\of{v}$, $\phi_k\of{v}$ denote the test and trial B-splines in radial coordinate. Under these definitions, the discretized weak form of the collision operator is given by


%\Cref{eq:hybrid_cnt_a} is discretized using a Chebyshev collocation method, and we use the same Chebyshev collocation scheme for spatial discretization of~\Cref{eq:hybrid_cnt_b}. Next, using spherical coordinates, we describe the $\vect{v}$-space discretization of~\Cref{eq:hybrid_cnt_b}. We use a mixed Galerkin and collocation scheme for the angular direction and a Galerkin scheme for the radial coordinate discretization. Let $\{x_i\}^{^{N_x}}_{i=1}$ and $\{\vtheta_a\}^{^{N_\vtheta}}_{a=1}$ be collocation points in $x$, and $\vtheta$ coordinates. The semi-discrete form of~\Cref{eq:hybrid_cnt_b} is given by 
%\begin{subequations}
%	\begin{align}
%		\partial_t f(t, x, v, \vtheta_a) + v\cos\of{\vtheta_a} \partial_x f(t, x, v, \vtheta_a) &= \of{\frac{\partial f}{\partial t}}_{\text{$\vect{v}$-space}}\of{t, x, v, \vtheta_a} \label{eq:bte_1d_semi_discrete}, \\
%		\of{\frac{\partial f}{\partial t}}_{\text{$\vect{v}$-space}} &= \frac{q_e E}{m_e} \of{\cos(\vtheta) \partial_v f - \sin(\vtheta) \frac{1}{v} \partial_{\vtheta}f } + C_{en}(f) \label{eq:bte_vspace}.
%	\end{align}
%\end{subequations} We use a Galerkin discretization scheme on spherical coordinates for~\Cref{eq:bte_vspace}. For~\Cref{eq:bte_vspace}, we use compactly supported B-splines in the radial coordinate with real-valued spherical harmonics in the angular directions for EDF approximation. We use uniformly spaced knots in the radial coordinate to define a cubic B-spline basis. The EDF representation is given by 
%\begin{equation}
%	f(\vect{v}, t) = \sum_{klm} f_{klm}\of{t} \Phi_{klm}\of{\vect{v}} \text{ where } \Phi_{klm}\of{\vect{v}}  = \underbrace{\phi_k\of{v}}_{\text{B-Spline basis}} \overbrace{Y_{lm}\of{v_\theta, v_\phi}}^{\tiny\text{spherical harmonics}}. \label{eq:f_expansion} 
%\end{equation} The spherical harmonics basis functions are defined as 
%\begin{align}
%	Y_{lm}\of{\vtheta,\vphi} &= U_{lm} P^{|m|}_l\of{\cos\of{\vtheta}} \alpha_m\of{\vphi} \text{ , } \nonumber \\
%	U_{lm} = 
%	\begin{cases}
%		(-1)^m \sqrt{2} \sqrt{\frac{2l+1}{4\pi} \frac{(l-|m|)!}{(l+|m|)!}}, &m\neq 0, \\
%		\sqrt{\frac{2l+1}{4\pi}}, &m = 0,
%	\end{cases} \text{ , } &
%	\alpha_m\of{\vphi}  =
%	\begin{cases}
%		\sin\of{|m|\phi}, &m < 0, \\
%		1, &m = 0,\\
%		\cos\of{m\phi}, &m > 0.
%	\end{cases} \label{eq:sph_harmonics}
%\end{align} 
%With an aligned $\vect{E}$ field to $\vect{v}$-space z-axis, $\vect{E} = E \vect{\hat{e}_z} =  E \of{\cos(\vtheta) \vect{\hat{e}_r} - \sin(\vtheta)\vect{\hat{e}_\theta}}$ ensures that the $\vect{E}$ acceleration excites only the polar modes~\Cref{eq:f_expansion}. This fact and the isotropic scattering assumption ensure that the BTE solutions preserve the azimuthal symmetry in $\vect{v}$-space. Therefore, the EDF representation only requires $\{Y_{l0}\}_{l}$ modes. For this reason, we drop the azimuthal index $v_{\phi}$ in the remainder of the paper. Let $p$, $k$ denote indices of the basis functions along the radial coordinate, $q$, $l$ denote indices in the polar angle, and $\phi_p\of{v}$, $\phi_k\of{v}$ denote the test and trial B-splines in radial coordinate. Under these definitions, the discretized weak form of the collision operator is given by
\begin{multline}
    n_0 {[\vect C_{en}]}^{pq}_{kl} = n_0 \myint_{\reals^+} v^2 \sigma_T\of{v} \phi_{k}\of{v} \delta_{ql} \of{\phi_{p}\of{u} \delta_{q0}  -\phi_{p}\of{v}} \diff{v}, \\
    \text{ for } p,k \in \{0,\hdots, N_r-1\} \text{ and } q,l \in \{0, \hdots,  N_l-1\} \label{eq:c_en_mat_1d}.
\end{multline}
For non-zero heavy temperature, \Cref{eq:c_en_mat_1d} gets an additional correction term, given by
\begin{multline}
	n_0 T_0 {[\vect C_{T}]}^{pq}_{kl} = \frac{n_0 T_0 k_B}{m_0} \delta_{q0} \myint_{\reals^+} v^3 \sigma_T\of{v} \partial_v \phi_p\of{v} \partial_v \phi_k\of{v} \diff{v}, \\
	\text{ for } p,k \in \{0,\hdots, N_r-1\} \text{ and } q,l \in \{0, \hdots,  N_l-1\} \label{eq:c_T0_mat_1d}.
\end{multline} In \Cref{eq:c_en_mat_1d} and \Cref{eq:c_T0_mat_1d} $\sigma_T$ denotes the total collisional cross-section, $n_0$ and $T_0$ denotes the background neutral density and temperature. In our case, we have momentum transfer and ionization collisions. For multiple collisions, the effective collision operator is given by $\vect C_{en}^{\text{effec.}} = \sum_{i\in {\text{collisions}}}$ $n_i \vect{C_{en}} \of{\sigma_i} + n_0 T_0 \vect C_{T}\of{\sigma_0}$ where $0$ index denotes the ground state heavy species, and $\sigma_0$ denotes the total cross-section for the momentum transfer collisions. The discretized velocity space acceleration operator is given by
\begin{multline}
	\small
	[\vect A_v]^{pq}_{kl} = \int_{\reals^+}  
		\delta_{(q+1) l} v^2 \phi_p\of{v} \of{A_M\of{l} \partial_v\phi_k\of{v} + A_D\of{l}\frac{1}{v} \phi_k\of{v}} + \\
		\delta_{(q-1) l} v^2 \phi_p\of{v} \of{B_M\of{l} \partial_v\phi_k\of{v} + B_D\of{l}\frac{1}{v} \phi_k\of{v}} 
	 \diff{v}, \\
	 \text{ for } p,k \in \{0,\hdots, N_r-1\} \text{ and } q,l \in \{0, \hdots,  N_l-1\} \label{eq:adv_v_ws},
\end{multline} where $A_M$, $A_D$, $B_M$, and $B_D$ are coefficients defined by
\begin{equation}
A_M(l) = \frac{l}{\sqrt{4l^2-1}}\text{, } B_M(l) = \frac{l+1}{\sqrt{4l^2-1}}\text{, } A_D(l) = \frac{l^2}{\sqrt{4l^2-1}}\text{, } B_D(l) = \frac{l(l-1)}{\sqrt{4l^2-1}}. 	
\end{equation}
To summarize, the velocity space discretized BTE is given by
\begin{equation}
	\partial_t \of{\vect M_{v,\vtheta} \vect f} = \of{\vect{C}_{en} + E \vect A_v}\vect{f} \label{eq:discretized_bte_vspace},
\end{equation} where $\vect M_{v,\vtheta}$, $\vect C_{en}$, and $\vect A_v$ $\in \reals^{N_rN_l \times N_rN_l} $ denote the Galerkin mass matrix, electron-heavy collisions and the velocity space acceleration operators respectively. More details on the velocity space discretization can be found in \cite{fernando0DBTE}. 

Now we discuss the spatial discretization of \Cref{eq:hybrid_cnt_b}. We use a Chebyshev collocation scheme for the spatial discretization of~\Cref{eq:hybrid_cnt_b}. The spherical harmonic representation is inadequate to impose the $\vtheta$ discontinuous boundary conditions specified in~\Cref{eq:hybrid_cnt_b_bdy}. This is resolved by a mixed Galerkin and collocation representation in $\vtheta$. Let $\{x_i\}^{^{N_x}-1}_{i=0}$ and $\{\vtheta_a\}^{^{N_\vtheta}-1}_{a=0}$ be collocation points in $x$, and $\vtheta$ coordinates. With these collocation points, the EDF representation is given by
\begin{multline}
	f(x_i, v, \vtheta_a, t) = \sum_{k=0}^{N_r-1} f_{k}\of{x_i, \vtheta_a, t} \phi_{k}\of{v} = \sum_{k=0}^{N_r-1} f_{ika}\of{t} \phi_{k}\of{v}, \\ \text{ where } f_{ika}(t) \equiv f_k(x_i, \vtheta_a, t), 
\end{multline}
and the semi-discrete form of~\Cref{eq:hybrid_cnt_b} is given by 
\begin{equation}
\partial_t f(t, x, v, \vtheta_a) + v\cos\of{\vtheta_a} \partial_x f(t, x, v, \vtheta_a) = \of{\partial_t f}_{\text{$\vect{v}$-space}}\of{t, x, v, \vtheta_a} \label{eq:bte_1d_semi_discrete}.
\end{equation} Let $\vect{P}_{S}$ denotes the $\vtheta$ ordinates to spherical harmonics projection, and $\vect{P}_{O}$ denotes the spherical harmonics to $\vtheta$ ordinates projection operators. These operators are defined by
\begin{multline}
	\vect P_S = \vect I_v \otimes \vect T_S \text{ , } \vect P_O = \vect I_v \otimes \vect T_O \text{ where } 
	[\vect T_{S}]^{q}_{a} = Y_{q}\of{\vtheta_a} w_a \text{ and }\\
	[\vect T_{O}]^{a}_{q} = Y_{q}\of{\vtheta_a}, 
	\text{ for } q \in \{0, \hdots, N_l-1\}\text{, } a\in \{0, \hdots, N_\vtheta-1\}\label{eq:proj_ops} .
\end{multline} Here, $\otimes$ denotes the matrix Kronecker product, $\vect I_v\in \reals^{N_r\times N_r}$ denotes the identity matrix in $v$ coordinate, and $w_a$ denotes the quadrature weights for the spherical basis projection. We use a Chebyshev collocation method in space, a Galerkin discretization in the $v$ coordinate, and a mixed Galerkin and collocation scheme in the $\vtheta$ coordinate for the discretization of~\Cref{eq:bte_1d_semi_discrete}. Let $\vect{D}_x$ denotes the discrete Chebyshev-basis derivative operator and $\vect{A}_x$ denotes the $v$ coordinate Galerkin projection of the spatial advection term in~\Cref{eq:bte_1d_semi_discrete}. The $\vect{A}_x$ operator is given by
\begin{multline}
	\vect{A}_x = \vect{D}_{\vtheta} \otimes \vect{G}_{v} \text{ where,}\quad
	[\vect D_\vtheta]_{ab} = \delta_{ab} \cos\of{\vtheta_a} \text{ and } \\ [\vect{G}_{v}]_{pk} =\int_{\reals^+} v^3 \phi_{p}\of{v} \phi_{k}\of{v}\diff{v}, \\
	\text{ for } p,k \in \{0,\hdots, N_r-1\} \text{ and } a, b \in \{0, \hdots, N_{\vtheta}-1\}
	\label{eq:A_x}.
\end{multline} By putting everything together, with $x$, $v$, and $\vtheta$ discretized BTE is given by
\begin{multline}
   \partial_t \vect F + \vect A_x \vect F \vect D_x^T = \vect P_{O}\vect C_{en} \vect P_{S} \vect{F} + \vect P_O \vect A_v \vect P_{S} \of{\vect E \pdot \vect{F}} \text{ where } \\
   \vect F \in \reals^{N_r N_\vtheta \times N_x}\text{, }\vect{P}_s
   \in \reals^{N_rN_l \times N_r N_\vtheta}\text{, }\\
   \vect{C}_{en}\text{, }\vect{A}_v \in \reals^{N_r N_l \times N_r N_l}\text{, }
   \vect{P}_O \in \reals^{N_r N_\vtheta \times N_r N_l}\text{, }
   \vect{D}_x \in \reals^{N_x \times N_x}\text{, }\vect{A}_x\in \reals^{N_rN_\vtheta \times N_r N_\vtheta}, \\
   [\vect F]_{i,:} = f_{ika}\text{, } 
   \text{ for } i \in \{0, \hdots ,N_x-1\} \text{, } k \in \{0, \hdots N_r-1\} \text{, and } a \in \{0, \hdots N_\vtheta-1\}    \label{eq:bte_1d_discrete}.
\end{multline} Here, $\vect{E} \in \reals^{N_x}$ and $\pdot$ denotes the column wise element product. Also, it is important to note that the operators in \Cref{eq:bte_1d_discrete} is properly scaled by Galerkin mass matrices, i.e., from here on, $\vect{G}_v \equiv \vect{M}^{-1}_v \vect{G}_v$, $\vect{A}_x \equiv \vect{D}_\vtheta \otimes \vect G_v$, $\vect{C}_{en} \equiv \vect M^{-1}_{v,\vtheta} \vect C_{en}$, and $\vect{A}_v \equiv M^{-1}_{v,\vtheta} \vect A_v$, where $\vect{M}_v$ and $\vect{M}_{v,\vtheta}$ denote the standard Galerkin mass matrices~\cite{ciarlet2002finite} in $v$ and $v, \vtheta $ coordinates respectively.  \Cref{eq:hybrid_cnt_a,eq:hybrid_cnt_c} are discretized with the same Chebyshev collocation scheme we used for~\Cref{eq:hybrid_cnt_b}. Let $\vect{k}_i\in \reals^{N_r N_\vtheta}$ denotes the discretized ionization rate coefficient operator, $\vect{u}\in\reals^{N_r N_\vtheta}$ denotes the zeroth-order moment operator where $\vect{n}_e = \vect{u}^T \vect{F}$, and $\vect{L}\in\reals^{N_x \times N_x}$ denotes the discretized electrostatics operator where $\vect{E}=\vect{L}\of{\vect{n}_i-\vect{n}_e}$. Then, the discretized hybrid model is given by
\begin{subequations}
	\begin{align}
		&\partial_t \vect n_i  + \vect D_x \vect J_i \of{\vect E, \vect n_i} = \of{\vect{k}^T_i \vect F} n_0 \vect n_e \label{eq:hybrid_discrete_a},\\
		&\partial_t \vect F + \vect A_x \vect F \vect D_x^T = \vect P_{O}\vect C_{en} \vect P_{S} \vect{F}  +  \vect P_O \vect A_v \vect P_{S} \of{\vect E \pdot \vect{F}}   \label{eq:hybrid_discrete_b},\\
		&\vect{E} = \vect{L}\of{\vect n_i - \vect n_e}, \text{ where } \vect n_e = \vect u^T \vect F \label{eq:hybrid_discrete_c}.
	\end{align}\label{eq:hybrid_discrete}
\end{subequations}The total number of unknowns for the discretized system is $N_x(1 + N_{r} N_{\vtheta})$ where $N_x$, $N_r$, and $N_{\vtheta}$ denote the number of Chebyshev collocation points in $x$, the number of B-spline basis used in $v$, and the number of collocation points in $\vtheta$.



%As mentioned, For \Cref{eq:hybrid_cnt_b}, we use a Chebyshev collocation method in $\vect{x}$-space, a Galerkin scheme in the $\vect{v}$-space radial direction, and a mixed Galerkin collocation scheme in $\vtheta$ coordinate.  The left-hand side of \Cref{eq:bte_1d_semi_discrete} acts on each ordinate separately, while the right-hand side $\vect{v}$-space operator in \Cref{eq:bte_vspace} couples all $\vtheta$ ordinates.
%The EDF is represented using $f(t, x_i, v, \vtheta_a) = \sum_{k} f_{ika}(t) \phi_k\of{v}$, and the fully discretized system is given by
%\begin{multline}
%    \partial_t \vect F + \vect A_x \vect F \vect D_x^T = \vect P_{O}\of{\vect C_{en}  + \vect{E} \pdot \vect A_v  }  \vect P_{S} \vect{F} \\\text{ where } [\vect F]_{i,:} = f_{ika} \text{ for } k \in \{1, \hdots N_r\}, a \in \{1, \hdots N_\vtheta\}  \label{eq:bte_1d_discrete}.
%\end{multline} In the above, $\vect D_x$ denotes the discrete spatial derivative operator, $\vect{A}_x$ denotes the $v$ coordinate Galerkin projection of the spatial advection term, $\vect{P}_{S}$ denotes the $\vtheta$ ordinates to spherical harmonics projection, and $\vect{P}_{O}$ denotes the spherical harmonics to $\vtheta$ ordinates operator. These operators are defined by 
%% given by $g\of{\vtheta} = \sum_l g_l Y_{l}\of{\vtheta}$, where $g_l = \frac{1}{2\pi}\int_{0}^{\pi} g\of{\vtheta}Y_l\of{\vtheta} \sin\of{\vtheta}\diff{\vtheta} \approx \sum_a g\of{\vtheta_a} Y_l\of{\vtheta_a} w_a$.
%\begin{align}
%    [\vect A_x]^{pa}_{kb} = \delta_{ab} \cos\of{\vtheta_a} \int_{\reals^+} v^3 \phi_{p}\of{v} \phi_{k}\of{v}\diff{v}, \label{eq:A_x}\\
%    [\vect P_{S}]^{pq}_{ka} = \delta_{pk} Y_{q}\of{\vtheta_a} w_a, \text{\ \ and\ \ }
%    [\vect P_{O}]^{pa}_{kq} = \delta_{pk} Y_{q}\of{\vtheta_a} \label{eq:proj_ops}.
%\end{align} Here, $w_a$ denotes the quadrature weights for the spherical basis projection. In summary, the discretized hybrid model is given by 
%\begin{subequations}
%\begin{align}
%	&\partial_t \vect n_i  + \vect D_x \vect J_i \of{\vect E\of{\vect n_i, \vect F}, \vect n_i} = \vect{\dot{S}}\of{\vect F} = \of{\vect{k}^T_i \vect F} n_0 \vect n_e \label{eq:hybrid_discrete_a},\\
%	&\partial_t \vect F + \vect A_x \vect F \vect D_x^T = \vect P_{O}\of{\vect C_{en}  + \vect{E} \of{\vect n_i, \vect F} \pdot \vect A_v  }  \vect P_{S} \vect{F} \label{eq:hybrid_discrete_b},\\
%	&\vect{E} = \vect{L}\of{\vect n_i - \vect n_e}, \text{ where } \vect n_e = \vect u^T \vect F \label{eq:hybrid_discrete_c}.
%\end{align}\label{eq:hybrid_discrete}
%\end{subequations} Here, $\vect{k}_i$ denotes the ionization rate coefficient operator, $\vect{L}$ denotes the discretized operator for electrostatics, and $\vect u$ denotes the zeroth-order moment operator such that $\vect n_e = \vect u^T \vect F$. 
%\Cref{eq:hybrid_discrete_a,eq:hybrid_discrete_b,eq:hybrid_discrete_c} are the discrete form of~\Cref{eq:glow_hybrid}. The total number of unknowns for the discretized system is given by $N_x(1 + N_{r} N_{\vtheta})$ where $N_x$, $N_r$, and $N_{\vtheta}$ denote the number of Chebyshev collocation points in $\vect{x}$-space, the number of B-spline basis used in $v$, and the number of ordinates in $\vtheta$.

\subsubsection{Timescales analysis}
\label{subsubsec:ts_analysis}
Here, we present timescales analysis for \Cref{eq:hybrid_discrete}. \Cref{tab:ts_analysis}~summarizes the main ion and electron timescales for a RF-GDP with $p_0=1$ Torr. The input voltage oscillation period is set to $7.374 \times 10^{-8}$ s for all RF-GDP cases considered in this paper. The timescale for the collision operator is given by
\begin{multline}
	\Delta t \frac{\norm{\vect{C}_{en} \vect{F}}}{\norm{\vect{F}}} \leq \Delta t \norm{\vect{C}_{en}} \leq \Delta t n_0\max\of{\norm{\vect{v}}\sigma_{0}, \norm{\vect{v}}\sigma_{i}}\leq \frac{\norm{\Delta \vect{F}}}{\norm{\vect{F}}}\\ \implies
	\Delta t \leq \frac{\norm{\Delta \vect{F}}}{\norm{\vect{F}}} \frac{1}{n_0\max\of{\norm{\vect{v}}\sigma_{0}, \norm{\vect{v}}\sigma_{i}}}= \frac{\norm{\Delta \vect{F}}}{\norm{\vect{F}}} \frac{1}{n_0\max\of{\norm{\vect{v}}\sigma_{0}}} \label{eq:coll_timescale}.
\end{multline} The reaction and collision terms determine the fastest timescale for ions and electrons. However, as shown in \Cref{tab:ts_analysis}, ions have much slower timescales than electrons. Therefore, the electron timescale determines the electrostatic timescale. A change in the electron number density $n_e(\vect{x}, t)$ is caused by the BTE spatial advection and reaction terms. While the velocity space advection is not directly coupled to $n_e(\vect{x}, t)$, it is indirectly coupled through the collision term. This is due to the EDF tails being populated due to the velocity space advection, and high-energy electrons are likely to trigger ionization collisions. Therefore, the electrostatic and the BTE coupling timescale is governed by the position-velocity space advection and the reaction term timescales. The collisional timescale gives the relative change in the EDF due to collisions. Typically, Since $\sigma_0 \geq \sigma_{\text{i}}$, the collisional timescale is determined by the momentum transfer cross-section. 
%Operator split schemes that decouple position space and velocity space operators

\begin{table}[tbhp]
	\centering
	\resizebox{\textwidth}{!}{
		\begin{tabular}{||c|c|c||}
			\hline 
			Species  & Term & $\Delta t$ [s]\\
			\hline
			\multirow{3}{*}{Ions} 	& $\nabla_{\vect{x}} \cdot \of{\mu_i \vect{E} n_i}$ -- advection 		 & $\frac{\Delta x}{\mu_i \norm{\vect{E}_{\text{max}}}} \approx \mathcal{O}(10^{-9})$  \\		[0.1cm]
			& $\nabla_{\vect{x}} \cdot \of{D_i \nabla_{\vect{x}} n_i}$ -- diffusion  & $\frac{\Delta x^2}{ 2 D_i } \approx \mathcal{O}(10^{-7}) $  \\ 	[0.1cm]
			& $k_i n_0 n_e$ -- reaction   											 & $\frac{\Delta n_i}{n_i}\frac{n_i}{k_{i,\text{max}} n_0 n_e} \approx \mathcal{O}(10^{-10})$   \\ [0.1cm]
			\cline{2-3}
			\multirow{3}{*}{Electron BTE}  & $\vect{v} \cdot \nabla_{\vect{x}} f$ -- $\vect{x}$-advection 		 				 & $\frac{\Delta x}{\norm{\vect{v}}} \approx \mathcal{O}(10^{-12})$\\ [0.1cm]
			& $\frac{q_e \vect{E}}{m_e} \cdot \nabla_{\vect{v}} f$ -- $\vect{v}$-advection 		 & $\frac{\norm{\Delta \vect v} m_e}{q_e\norm{\vect{E}_{\text{max}}}} \approx \mathcal{O}(10^{-12})$\\ 	[0.1cm]
			&  $k_i n_0 n_e = n_0 \int_{\vect{v}} \norm{\vect v}^3 \sigma_{i}\of{\norm{\vect v}} f\of{\vect{v}} \diff{\vect{v}}$ -- reaction & $\frac{\Delta n_e}{n_e}\frac{1}{k_{i,\text{max}} n_0 } \approx \mathcal{O}(10^{-12})$ \\ [0.1cm]
			&$ \vect{C}_{en}\vect{F}$ -- collisions  & $\frac{\norm{\Delta \vect{F}}}{\norm{\vect{F}}} \frac{1}{n_0 \max\of{ \norm{\vect{v}} \sigma_0}} \approx \mathcal{O}(10^{-13})$\\ [0.2cm]
			\cline{2-3}
			\multirow{2}{*}{Driving voltage} & &\\[0.01cm]
											 & $V(t) = V_0 \sin\of{2\pi \zeta t}$ & $1/\zeta=7.374 \times 10^{-8}$ \\ [0.1cm]
			\hline
	\end{tabular}}
	\caption{A summary of the timescales for advection, diffusion, and reaction terms for the hybrid formulation, assuming an explicit time integration scheme.  Here, we assume $\frac{\Delta x}{L}\approx \mathcal{O}(10^{-3})$, $\frac{\norm{\Delta \vect{v}}}{\norm{\vect v}} \approx \mathcal{O}(10^{-3})$, $\frac{\norm{\Delta\vect{F}}}{\norm{\vect{F}}} \approx \mathcal{O}(10^{-3})$, and $\frac{\Delta n_e}{n_e} \approx \mathcal{O}(10^{-3})$. The maximum absolute quantities occurs closer to the electrodes with $\norm{\vect E_{\text{max}}}=8\times10^{4}$ [V/m], $k_{i,\text{max}}=10^{-14}$ [$m^3/s$], and the corresponding ionization degree $\frac{n_i}{n_e}\approx\mathcal{O}(10^2)$. For the other transport coefficients, we use the values summarized in \Cref{t:model_parameters} with the $p_0=$ 1 Torr case. \label{tab:ts_analysis}}
\end{table}
%Considering the plasma oscillation frequencies for electrons $\omega_{e} = \sqrt{q_e^2 n_e /m_e/\epsilon_0}$ and ions $\omega_{i} = \sqrt{q_e^2 n_i /m_i/\epsilon_0}$ we can write $\omega_{e} / \omega_{i} = \of{m_i / m_e}^{1/2} \of{n_e / n_i}^{1/2}$. For RF-GDPs, typically $n_e \sim n_i \sim 10^{17}$ $m^{-3}$, $f_{e} \approx 8.9 \times 10^{9}$ $s^{-1}$ and $f_{i} \approx 3.3 \times 10^{7}$ $s^{-1}$. Therefore, electrons have a higher plasma frequency than ions. This is primarily due to the small mass ratio between electrons and ions, $m_e/m_i = 1.36\times 10^{-5}$. 

\subsubsection{Hybrid solver time integration}
Typically, we want to evolve the system to the order of thousand cycles or $10^{-5}$ seconds, and require ten million timesteps. An implicit scheme with an iterative solve is possible as the matrix-vector products can be also done in an optimal way, but designing a good preconditioner is not trivial. Instead we opt for an operator split scheme.
As we can see in~\Cref{tab:ts_analysis}, the ions have slower timescales compared to electrons. Hence, we use an operator split scheme to decouple ion and electron transport equations. Since electrons determine the electrostatics coupling timescale, \Cref{eq:hybrid_discrete_a} is solved with decoupled electrostatics.
%The ion and electron plasma frequencies determine the $\vect{E}$ field coupling time scales for \Cref{eq:hybrid_discrete_a,eq:hybrid_discrete_b}. This is mainly due to the electron-to-ion mass ratio. Therefore, ions respond slowly compared to the electrons for a given perturbation in the electric field. Hence, \Cref{eq:hybrid_discrete_a} is solved with frozen $\vect{E}$ and $\vect{F}$ values. 

\par \textbf{Heavies evolution}: For a given state $(\vect n_i^n, \vect F^n)$ at time $t=t_n$, the ion transport equation is solved implicitly with lagged $\vect F^n$ and $\vect E^n = \vect L (\vect n^n_i - \vect u^T \vect F^n)$. This coupling results in a linear system to be solved for the ions state at time $t_{n+1}=t_n +  \Delta t$, where $\Delta t$ denotes the time step size. So, the ions update in our operator-split scheme is given by
\begin{equation}
	\frac{\vect n^{n+1}_i -\vect n^n_i}{\Delta t} + \vect{D}_x \vect J_i \of{\vect E^n, \vect n^{n+1}_i}= \of{\vect k_i^T \vect F^n} n_0 \of{\vect u^T \vect F^n} \label{eq:fluid_step}.
\end{equation}


%The electrons are strongly coupled to the electric field through~\Cref{eq:hybrid_discrete_c}. Therefore, \Cref{eq:hybrid_discrete_b,eq:hybrid_discrete_c} are solved together.
%to account for the $\vect{E}$ field variation due to electron transport.

\par \textbf{Electron evolution}: We consider the evolution of \Cref{eq:hybrid_discrete_b,eq:hybrid_discrete_c} assuming a fixed $\vect{n}_i$. We consider two alternative schemes for the electrons: a semi-implicit and a fully-implicit one. The semi-implicit scheme compute the EDF at time $t + \Delta t$ is given by
\begin{subequations}
	\begin{empheq}[left =\text{\small semi-implicit} \empheqlbrace]{align}
        &\frac{\vect{F}^{n+1/2} - \vect{F}^n}{\Delta t/2} =  -\vect A_x \vect F^{n+1/2} \vect D_x^T, t \in \of{t_n , t_n + \frac{\Delta t}{2}} \label{eq:bte_semi_implicit_xspace0}, \\
        &\vect{E}^{1/2}=\vect L \of{\vect n_i - \vect u^T \vect F^{n + 1/2}} \label{eq:bte_semi_implicit_efield},\\
        & \frac{\vect{F}^{*} - \vect{F}^{n+1/2}}{\Delta t} = \vect P_{O}\vect C_{en}\vect P_{S} \vect{F}^*  + \nonumber \\
        &\qquad \qquad \vect P_O \vect A_v \vect P_{S} \of{\vect{E}^{1/2} \pdot \vect{F}^*}, t \in \of{t_n, t_n + \Delta t}, \label{eq:bte_semi_implicit_vspace}\\
        &\frac{\vect{F}^{n+1} - \vect{F}^*}{\Delta t / 2} =  -\vect A_x \vect F^{n+1} \vect D_x^T, t \in \of{t_{n+1/2} , t_{n+1/2} + \frac{\Delta t}{2}} \label{eq:bte_semi_implicit_xspace1}.
    \end{empheq} \label{eq:bte_semi_implicit}
\end{subequations}
The overview of the semi-implicit scheme is summarized below. 
\begin{itemize}
\item \textbf{Spatial advection}: \Cref{eq:bte_semi_implicit_xspace0,eq:bte_semi_implicit_xspace1} corresponds to the spatial advection of electrons. We use the the eigen decomposition of $\vect{G}_v=\vect{U}\vect{\Lambda}U^{-1}$ to diagonalize \Cref{eq:bte_semi_implicit_xspace0,eq:bte_semi_implicit_xspace1}. Since $\vect{G}_v$ is a product of two positive definite matrices, it has real positive eigenvalues. Upon diagonalization, \Cref{eq:bte_semi_implicit_xspace0,eq:bte_semi_implicit_xspace1} become series of decoupled $N_rN_\vtheta$ spatial advection equations where the advection velocity is given by $\vect{a} = \Lambda_i \cos(\vtheta_j) \vect{\hat{e}}_x$ for $i=\{0,\hdots N_r-1\}$ and $j=\{0,\hdots N_\vtheta-1\}$ where $N_r$ and $N_\vtheta$ denotes the number of B-splines and $\vtheta$ ordinates used. For each $i$ and $j$, we precompute and store the inverted $\left(\vect{I}_x + \frac{1}{2} \Delta t \Lambda_i \cos(\vtheta_j) \vect D_x \right)^{-1} = \vect Q_{ij}$ operators and use them for spatial advection. Here, $\vect{I}_x\in \reals^{N_x\times N_x}$ denotes the identity matrix in the $x$ dimension. This direct inversion is performed only once and reused throughout the time evolution. The storage cost for the inverted matrices is given by $\mathcal{O}(N_r N_\vtheta N_x^2)$. \Cref{alg:bte_spatial_adv} presents the overview of the BTE spatial advection step. In the implementation, the application of $\vect Q_{ij}$ is performed as a batched general matrix-matrix multiplication (GEMM).
%
\item \textbf{Electrostatics}: \Cref{eq:bte_semi_implicit_efield} computes the updated electric field $\vect{E}^{1/2}$ following the spatial advection. Here, $\vect{L}\in \reals^{N_x \times N_x}$ denotes the discretized $\nabla_{\vect{x}} \Delta^{-1}_{\vect x}$ operator which is computed once and reused throughout the time evolution. 
%
\item \textbf{Velocity space update}: We use the generalized minimal residual method (GMRES) to solve~\Cref{eq:bte_semi_implicit_vspace}. In~\Cref{eq:bte_semi_implicit_vspace}, the discretized EDF is spatially decoupled, and the velocity space left-hand side operator varies spatially and is given by
\begin{multline}
	\biggl(\vect{I}_{v,\vtheta} -\Delta t \vect P_{O}\of{\vect C_{en}  + \bigl[\vect{E}^{1/2}\bigr]_{i} \vect A_v  }  \vect P_{S}\biggr) \vect F^* = \vect F^{n + 1/2}  \\ \text{ where } \vect E^{1/2} \in \reals^{N_x}, 
	\text{ for } i \in \{0,\hdots, N_x - 1\} \label{eq:vspace_batched_system}.
\end{multline} Here, $\vect{I}_{v,\vtheta}$ denotes the identity operator in the velocity space, and $[\vect{E}^{1/2}]_i$ denotes the electric field evaluated at the $i$-th spatial collocation point. Therefore, the solve of~\Cref{eq:bte_semi_implicit_vspace} reduces to a $N_x$ decoupled linear systems. For the GMRES solve, these linear systems are solved as a batched system. The batched right-hand side evaluation is summarized in \Cref{alg:vspace_action}. We use a domain decomposition preconditioner for the GMRES solve. Let $E_0$ denotes an upper bound on the electric field such that $\norm{\vect{E}\of{\vect x, t}} \leq E_0 \forall t$ and $-E_0 = S_0 < S_1 < \hdots < S_{N_c}=E_0$ be a time independent non-overlapping partition of $(-E_0, E_0)$. We precompute a series of operators $\{H_p\}_{p=1}^{N_c}$ where each $\vect H_p\in \reals^{N_r N_l}$ is given by
\begin{equation}
	\vect H_p = \of{\vect I -\Delta t \of {\vect C_{en} + \frac{1}{2}\of{S_{p-1} + S_{p}} \vect A_v}}^{-1} 
	\text{ for } p\in\{1,\hdots N_c\}.
\end{equation} Here, $\vect I\in \reals^{N_r N_l}$ denotes the identity matrix. For a specified $\vect{E} \in \reals^{N_x}$, we define $N_c$ partitions where partition $w_k$ is given by $w_k=\{i\quad | \  S_{k-1} \leq [\vect{E}]_i < S_{k}\}$ for $k\in \{1,\hdots,N_c\}$. We refer to this as the preconditioner setup step, which is performed once per each timestep. The precondition operator evaluation is summarized in \Cref{alg:vspace_action_precon}.
%$\{\vect H_p\}_{p=1}^{N_c}$
\end{itemize}

\begin{algorithm}[!tbhp]
	\begin{algorithmic}[1]
		\Require $\vect{F}^n\in \reals^{N_rN_\vtheta \times N_x}$ -- EDF at time $t_n$, \nonumber\\
		$\biggl\{\vect{Q}_{ij} = \left(\vect{I}_x + \frac{1}{2} \Delta t \Lambda_i \cos(\vtheta_j) \vect D_x \right)^{-1}\biggr\}_{(i,j)\in \{0,..,N_r-1\}\times\{0,..,N_\vtheta-1\}}$ -- inverted spatial advection operators, and $\vect{G}_v = \vect{U} \vect \Lambda \vect U^{-1}$
		\Ensure Solution at time $t_n + \Delta t/2$, $\vect{F}^{1/2} \in \reals^{N_r N_\vtheta \times N_x}$ 
		\State $\vect Y \leftarrow \text{unfold}(\vect F, (N_r, N_\vtheta, N_x))$ \Comment{$\vect Y \in \reals^{N_r \times N_\vtheta\times N_x}$}
		\State $[\vect Y]_{ijk} \leftarrow [\vect U^{-1}]_{im} \times_{m} [\vect F^n]_{mjk}$ \Comment{contraction on $m$ index}
		\For{$i \in \{0,\hdots, N_r-1\}$}
		\For{$j \in \{0,\hdots, N_\vtheta-1\}$}
		\rlap{\smash{$\left.\begin{array}{@{}c@{}}\\{}\\{}\end{array}\color{black}\right\}%
				\color{black}\begin{tabular}{l} Batched GEMM kernels \end{tabular}$}}
		\State $\vect Y_{ijk} \leftarrow \vect Q_{ijkl} \times_l \vect Y_{ijl}$ \Comment{$\vect Q_{ij}\in \reals^{N_x \times N_x}$ and contraction on $l$ index}
		\EndFor
		\EndFor
		\State $[\vect Y]_{ijk} \leftarrow [\vect U]_{im} \times_{m} [\vect F^n]_{mjk}$ \Comment{contraction on $m$ index}
		\State $\vect F^{1/2} \leftarrow \text{fold}(\vect Y , (N_rN_\vtheta \times N_x))$ \Comment{$\vect F^{1/2} \in \reals^{N_rN_\vtheta \times N_x}$}
		\State \Return $\vect{F}^{1/2}$
	\end{algorithmic}
	\caption{BTE spatial advection. \label{alg:bte_spatial_adv}}
\end{algorithm}
\begin{algorithm}
	\begin{algorithmic}[1]
		\Require $\vect F \in \reals^{N_r N_\vtheta \times N_x}$, $\vect E \in \reals^N_x$ -- electric field, $\vect C_en$, $\vect A_v$, $\vect P_O$, $\vect P_S$, and $\Delta t$
		\Ensure $\biggl(\vect{I}_{v,\vtheta} -\Delta t \vect P_{O}\of{\vect C_{en}  + \bigl[\vect{E}\bigr]_{i} \vect A_v  }  \vect P_{S}\biggr) \vect F$ -- operator action on $\vect F$
		\State $\vect F_S \leftarrow \vect P_S \vect F$
		\State $\vect F_C \leftarrow \vect C_{en} \vect F_s$ \quad and \quad $\vect F_A \leftarrow \vect A_v \vect F_s$
		\State $\vect G  \leftarrow  \vect F - \Delta t \vect P_O \of{\vect F_C + \vect E \pdot \vect F_A}$
		\State \Return $\vect G$
	\end{algorithmic}
	\caption{BTE velocity space operator action. \label{alg:vspace_action}}
\end{algorithm}
\begin{algorithm}
	\begin{algorithmic}[1]
		\Require $\vect F $, $\{\vect H_p\}_{p=1}^{N_c}$ -- precondition operators, $\{w_p\}_{p=1}^{N_c}$ -- domain decomposition, $\vect P_S $, $\vect P_O$
		\Ensure Preconditioner action $: \reals^{N_r N_\vtheta \times N_x} \rightarrow \reals^{N_r N_\vtheta \times N_x}$
		\State $\vect G \leftarrow \vect 0 $ \Comment{$\vect G \in \reals^{NrN_l \times N_x}$}
		\State $\vect F_S \leftarrow \vect P_S \vect F$ \Comment{$\vect F_s \in \reals^{NrN_l \times N_x}$}
		\For {$p \in \{1,\hdots, N_c\}$ }
			\State $\vect G[:, w_p] \leftarrow \vect H_p \vect F_S[:, w_p]$
		\EndFor
		\State $\vect G \leftarrow \vect P_O \vect G$
		\State \Return $\vect G$ \Comment{$\vect G \in \reals^{N_r N_\vtheta \times N_x}$}
	\end{algorithmic}
	\caption{BTE velocity space GMRES preconditioner. \label{alg:vspace_action_precon}}
\end{algorithm}

%\Cref{eq:bte_semi_implicit_xspace0} updates $\vect{F}^{n+1/2}$ to capture the spatial advection, i.e., the translation of electrons. Next, \Cref{eq:bte_semi_implicit_efield} is used to compute the updated electric field $\vect{E}^{1/2}$, \Cref{eq:bte_semi_implicit_vspace} is solved for the $\vect{v}$-space advection and collisions, and finally  \Cref{eq:bte_semi_implicit_xspace1} repeats the spatial advection and computes the $\vect{F}^{n+1}$. To solve the linear system in \Cref{eq:bte_semi_implicit_vspace}, we use the generalized minimum residual method (GMRES). We use pre-factored right-hand side operators for a predefined $E_c\in \of{-E_0 , E_0}$ coarse grid values as a preconditioner for the GMRES solve. Here, $E_0$ denotes an upper bound on  the electric field where $\norm{\vect{E}\of{t,\vect x}} \leq E_0$ $\forall t$. We use the eigen decomposition of $\vect{A}_x=\vect{U}\vect{\Lambda}U^{-1}$ to diagonalize \Cref{eq:bte_semi_implicit_xspace0,eq:bte_semi_implicit_xspace1}. Upon diagonalization, \Cref{eq:bte_semi_implicit_xspace0,eq:bte_semi_implicit_xspace1} become series of decoupled $N_rN_\vtheta$ advection equations where the advection velocity is given by $\vect{a} = \Lambda_i \cos(\vtheta_j) \vect{\hat{e}}_x$ for $i=\{1,\hdots N_r\}$ and $j=\{1,\hdots N_\vtheta\}$ where $N_r$ and $N_\vtheta$ denotes the number of B-splines and $\vtheta$ ordinates used. For the above, we pre-compute and store inverted $N_r N_\vtheta$ operators and use them directly for solving~\Cref{eq:bte_semi_implicit_xspace0,eq:bte_semi_implicit_xspace1}. The decoupled $\vect{E}$, $\vect{x}$-space, and $\vect{v}$-space solves introduce timestep size restrictions that arise from the electron plasma oscillation frequency. To address this timestep size restriction, we propose a fully-implicit scheme given by  

The semi-implicit approach has a timestep size restriction that arises from the decoupled electrostatics, spatial, and velocity space solves. This is determined by the BTE spatial advection and reaction timescales (see \Cref{subsubsec:ts_analysis}). To address this timestep size restriction, we propose a fully-implicit scheme for the BTE, which is given by 
\begin{subequations}
	\begin{empheq}[left =\text{\small fully-implicit} \empheqlbrace]{align}
		\frac{\vect F^{n+1} - \vect F^n}{\Delta t}  &= -\vect A_x \vect F^{n+1} \vect D_x^T + \vect P_{O}\vect C_{en} \vect P_{S} \vect{F}^{n+1} + \nonumber \\
		& \qquad\qquad \vect P_O \vect A_v \vect P_S \of {\vect E(\vect F^{n+1}) \pdot \vect F^{n+1}}     \label{eq:bte_fully_implicit_a}, \\
		\vect{E}\of{\vect{F}} &= \vect L \of{\vect n_i - \vect u^T \vect F} \label{eq:bte_fully_implicit_b}.
	\end{empheq} \label{eq:bte_fully_implicit}
\end{subequations} 
We use Newton's method  with line search to solve the nonlinear system by~\Cref{eq:bte_fully_implicit}. The application of the Jacobian, evaluated at $\vect F$, on a given input $\vect X$ is given by
\begin{multline}
	%\vect{J_F} \vect X = -\vect A_x \vect X \vect D_x^T + \vect P_O \of{ \vect C_{en} + \vect E\of{\vect F} \pdot \vect A_v} \vect P_S \vect X  +  \vect P_O \of{\vect E \of{\vect X} \pdot \vect A_v }\vect P_S \vect F \label{eq:1dbte_jac_action}.
	\vect{J_F} \vect X = -\vect A_x \vect X \vect D_x^T + \vect P_O \vect C_{en} \vect P_S \vect X  + \\ \vect P_O \vect A_v \vect P_S \of{E\of{\vect F} \pdot \vect X}  +  \vect P_O \vect A_v \vect P_S \of{\vect E \of{\vect X} \pdot \vect F} \label{eq:1dbte_jac_action}.
\end{multline}
Computing the linear step is, however, a challenge. If we try to assemble the Jacobian, we end up with a dense phase space operator with a prohibitive storage cost. Instead, we use a matrix-free GMRES solver, but it turns out the Jacobian is highly ill-conditioned. The solution is to use preconditioning. For this, we propose to use the linear operator corresponding to the update of the semi-implicit scheme defined in~\Cref{eq:bte_semi_implicit} as a preconditioner. We discuss the efficiency of the solver in the results section.


%This is a challenging system to solve. The key challenges include fully coupled phase space solve, operator assembly and storage cost is expensive, and the BTE equation becomes non-linear due to the $\vect{E}$ field coupling. Also, if we were to use an LU factorization for the Newton step, the cost would be $\mathcal{O}\of{N_r^3 N_x^3 N_{\vtheta}^3}$ which is infeasible for the problem sizes we consider. To mitigate these challenges, we use a matrix-free Newton's method to evolve~\Cref{eq:bte_fully_implicit} where the Jacobian action is given by
%We use GMRES to perform the Jacobian solve for the Newton iteration. We use~\Cref{eq:bte_semi_implicit} as a preconditioner for the GMRES solve of~\Cref{eq:bte_fully_implicit}. 
%We use an operator split similar to semi-implicit scheme as a preconditioner for the fully-implicit solve. 

\subsubsection{Hybrid solver complexity}
Recall that $N_x$, $N_r$, and $N_{\vtheta}$ denote the number of Chebyshev collocation points in position space, B-spline basis functions in $v$, and discrete ordinate points in $\vtheta$. This will result in $N_x(1 + N_r N_{\vtheta})$ degrees of freedom for the hybrid solver. Recall, $N_l$ denotes the number of spherical harmonics used for the mixed Galerkin and collocation discretization scheme used in $\vtheta$ coordinate.The velocity space operators $\vect{C}_{en}$, $\vect{A}_v$ are stored as dense matrices. Operators $\vect{A}_x$, $\vect P_S$, $\vect P_O$ are not assembled, and their actions are performed with tensor contractions with the stored $\vect D_\vtheta$, $\vect{G}_v$, $\vect T_S$, and $\vect T_O$ operators. These operators are stored as dense matrices. \Cref{tab:hybrid_comp_complexity}~summarizes all discretized BTE operators and their dimensions. It also provides the storage complexities for these operators and the computational complexity for their actions.
%As mentioned, the $\vect{v}$-space advection-collision given by~\Cref{eq:bte_vspace} is discretized using $N_l$ spherical harmonics in $\vtheta$. Therefore, this appears in the $\vect{A}_v$ and $\vect C_{en}$ costs. \Cref{tab:hybrid_comp_complexity}~summarizes complexity for storage and action of key BTE operators. 
\begin{table}[!tbhp]
	\centering
	\resizebox{\textwidth}{!}{
	\begin{tabular}{||c|c|c|c|c||}
		\hline
		Operator & Dimensions &  Storage & Action on $\small \vect{X} \in \reals^{N_r\times N_\vtheta \times N_x}$, $\vect X_S \in \reals^{N_r\times N_l \times N_x}$ & Complexity\\
		\hline
		%$\vect{F}$ -- degrees of freedom & $\mathcal{O} (N_x(1+N_rN_{\vtheta}))$ & --\\
		$\vect{D}_x$ & $N_x \times N_x $  & $\mathcal{O} (N_x^2)$ & $[\vect Y]_{ijk} = [\vect D_x]_{im} \times_m [\vect X]_{jkm}$  & $\mathcal{O}\of{N_x^2 N_r N_{\vtheta}}$ \\ [0.05cm]
		$\vect{A}_x=\vect D_\vtheta \otimes \vect G_v$ & $N_rN_\vtheta \times N_r N_\vtheta $  & $\mathcal{O} (N_r^2 + N_{\vtheta})$ & $[\vect Y]_{ijk} = \vect [D_\vtheta]_{jj} \pdot_j [\vect G_v]_{im} \times_m [\vect X]_{mjk}$ & $\mathcal{O} \of{N_x N_{\vtheta} N_r^2}$ \\ [0.05cm]
		$\vect P_S=\vect I_v \otimes \vect T_S$ & $N_r N_l \times N_v N_\vtheta$ & $\mathcal{O}(N_{\vtheta} N_l)$ & $[\vect Y]_{ijk} = [\vect T_S]_{jm} \times_m [\vect X]_{imk}$ & $\mathcal{O}\of{N_x N_r N_l N_\vtheta}$ \\ [0.05cm]
		$\vect P_O=\vect I_v \otimes \vect T_O$ & $N_r N_\vtheta \times N_r N_l$ & $\mathcal{O}(N_{\vtheta} N_l)$ & $[\vect Y]_{ijk} = [\vect T_O]_{jm} \times_m [\vect X_S]_{imk}$ & $\mathcal{O}\of{N_x N_r N_l N_\vtheta}$ \\ [0.05cm]
		$\vect{C}_{en}$ & $N_r N_l \times N_r N_l$ & $\mathcal{O}(N_r^2 N_l^2)$ & $[\vect Y]_{ijk} = [\vect C_{en}]_{ijlm} \times_{lm} [\vect X_S]_{lmk}$ & $\mathcal{O}\of{N_x N_r^2 N_l^2}$ \\ [0.05cm]
		$\vect{A}_{v}$ & $N_r N_l \times N_r N_l$ & $\mathcal{O} (N_r^2 N_l^2)$ & $[\vect Y]_{ijk} = [\vect A_{v}]_{ijlm} \times_{lm} [\vect X_S]_{lmk}$ & $\mathcal{O}\of{N_x N_r^2 N_l^2}$ \\ [0.05cm]
		\hline
	\end{tabular}}
\caption{Summary of storage and computational complexity for the discretized BTE operators. Here $\times_k$ denotes the contraction along index $k$, and $\pdot_k$ denotes element wise product along index $k$. \label{tab:hybrid_comp_complexity}}
\end{table} 

The computational complexities for $\vect A_x \vect F \vect D_x^T$ evaluation is given by $T_{\vect{x}\text{-rhs}} = \mathcal{O}\big(N_r N_{\vtheta} N_x \of{N_r + N_x}\big)$,  and the spatial advection solve in \Cref{alg:bte_spatial_adv} is $T_{\vect{x}\text{-solve}} = \mathcal{O}\of{N_r N_{\vtheta} N_x \of{N_r + N_x}}$. The computation cost for $P_O\of{\vect C_{en} + \vect E\pdot \vect A_v} P_S$ is $T_{\vect{v}\text{-rhs}} = \mathcal{O} \of{\of{2 N_\vtheta + N_r N_l} N_r N_l N_x}$. Application of the preconditioner has a similar costs so that $T_{\vect{v}\text{-precond}} = T_{\vect{v}\text{-rhs}}$. Hence, the overall computational cost for a single BTE time-step for the semi-implicit scheme is given by 
\begin{equation}
	T_{\text{semi-implicit}} =  T_{\vect{x}\text{-solve}} + k(T_{\vect{v}\text{-precond}} + T_{\vect{v}\text{-rhs}}) = T_{\vect{x}\text{-solve}} + 2kT_{\vect{v}\text{-rhs}} \label{eq:semi_imp_ts_cost}.
\end{equation} 
Here, $2kT_{\vect{v}\text{-rhs}}$ denotes the velocity space GMRES solver cost, and $k$ denotes the GMRES iterations for convergence. 
For the fully-implicit scheme, the right-hand side evaluation cost of the Jacobian action given in~\Cref{eq:1dbte_jac_action} is $T_{\vect{vx}\text{-jac}} = T_{\vect{x}\text{-rhs}} + T_{\vect{v}\text{-rhs}} + T_{\vect{E}\text{-solve}}$. Here, $\vect E$ solve cost is $T_{\vect{E}\text{-solve}} = \mathcal{O}(N_r N_\vtheta N_x + N_x^2)$, and since $(T_{\vect{x}\text{-rhs}} + T_{\vect{v}\text{-rhs}}) >> T_{\vect E\text{-solve}}$ it is omitted in the analysis. The operator split preconditioning cost for the fully implicit scheme is $T_{\vect{vx}\text{-precond}} = T_{\vect{x}\text{-rhs}} + T_{\vect{v}\text{-rhs}}$. Taken together, the computational complexity for a single timestep solve of the fully-implicit scheme is given by
\begin{equation}
T_{\text{fully-implicit}} = k(T_{\vect{vx}\text{-precond}} + T_{\vect{vx}\text{-jac}}) = 2k(T_{\vect{x}\text{-rhs}} + T_{\vect{v}\text{-rhs}}) \label{eq:bte_imp_complexity}.
\end{equation} Here, $k$ denotes the total Newton-GMRES iterations for convergence.
Let $(\Delta t_F,k_F)$ and $(\Delta t_S,k_S)$ tuples denote the timestep size and the number of iterations for the described fully-implicit and semi-implicit schemes respectively. The cost ratio $S$ of the semi-implicit to fully-implicit scheme is given by
\begin{equation}
	\frac{\Delta t_F \of{T_{\vect{x}\text{-rhs}} + 2k_S T_{\vect{v}\text{-rhs}}}}{\Delta t_S 2k_F \of{T_{\vect{x}\text{-rhs}} + T_{\vect{v}\text{-rhs}}}} \label{eq:ts_efficiency}.
\end{equation} 
Due to the nonlinearity and ill-conditioning of~\Cref{eq:bte_fully_implicit} it is hard to determine the effectiveness of the fully-implicit scheme. It depends on the preconditioned linear solves and the underlying prefactors in the complexity estimates, some of which are related to memory accesses and hardware performance. Next, we describe a series of numerical experiments to compare the two schemes. %Typically, we need $\Delta t_S \sim \frac{1}{\omega_{e}} < \Delta t_F$.% and the fully-implicit scheme is more efficient than the semi-implicit scheme, when $S > 1$.


\subsection{Implementation}
\label{subsec:implement_details}
We have implemented all the presented algorithms in Python and we release them in a library we call~\bte. We use \texttt{NumPy} and \texttt{CuPy} libraries for all the linear algebra operations. The above libraries provide an interface to basic linear algebra subprograms (BLAS) for CPU and GPU architectures. Both hybrid and the fluid model time integration supports GPU acceleration with \texttt{CuPy}. We use \texttt{CuPy} GMRES solver with \texttt{LinearOperator} class to specify operator and preconditioner actions. The batched GEMM operations are performed using the \texttt{einsum} function. \bte~is available at \url{https://github.com/ut-padas/boltzmann.git}.
%
\section{Results} \label{sec:results}

\begin{table*}[t]
\centering
\fontsize{11pt}{11pt}\selectfont
\begin{tabular}{lllllllllllll}
\toprule
\multicolumn{1}{c}{\textbf{task}} & \multicolumn{2}{c}{\textbf{Mir}} & \multicolumn{2}{c}{\textbf{Lai}} & \multicolumn{2}{c}{\textbf{Ziegen.}} & \multicolumn{2}{c}{\textbf{Cao}} & \multicolumn{2}{c}{\textbf{Alva-Man.}} & \multicolumn{1}{c}{\textbf{avg.}} & \textbf{\begin{tabular}[c]{@{}l@{}}avg.\\ rank\end{tabular}} \\
\multicolumn{1}{c}{\textbf{metrics}} & \multicolumn{1}{c}{\textbf{cor.}} & \multicolumn{1}{c}{\textbf{p-v.}} & \multicolumn{1}{c}{\textbf{cor.}} & \multicolumn{1}{c}{\textbf{p-v.}} & \multicolumn{1}{c}{\textbf{cor.}} & \multicolumn{1}{c}{\textbf{p-v.}} & \multicolumn{1}{c}{\textbf{cor.}} & \multicolumn{1}{c}{\textbf{p-v.}} & \multicolumn{1}{c}{\textbf{cor.}} & \multicolumn{1}{c}{\textbf{p-v.}} &  &  \\ \midrule
\textbf{S-Bleu} & 0.50 & 0.0 & 0.47 & 0.0 & 0.59 & 0.0 & 0.58 & 0.0 & 0.68 & 0.0 & 0.57 & 5.8 \\
\textbf{R-Bleu} & -- & -- & 0.27 & 0.0 & 0.30 & 0.0 & -- & -- & -- & -- & - &  \\
\textbf{S-Meteor} & 0.49 & 0.0 & 0.48 & 0.0 & 0.61 & 0.0 & 0.57 & 0.0 & 0.64 & 0.0 & 0.56 & 6.1 \\
\textbf{R-Meteor} & -- & -- & 0.34 & 0.0 & 0.26 & 0.0 & -- & -- & -- & -- & - &  \\
\textbf{S-Bertscore} & \textbf{0.53} & 0.0 & {\ul 0.80} & 0.0 & \textbf{0.70} & 0.0 & {\ul 0.66} & 0.0 & {\ul0.78} & 0.0 & \textbf{0.69} & \textbf{1.7} \\
\textbf{R-Bertscore} & -- & -- & 0.51 & 0.0 & 0.38 & 0.0 & -- & -- & -- & -- & - &  \\
\textbf{S-Bleurt} & {\ul 0.52} & 0.0 & {\ul 0.80} & 0.0 & 0.60 & 0.0 & \textbf{0.70} & 0.0 & \textbf{0.80} & 0.0 & {\ul 0.68} & {\ul 2.3} \\
\textbf{R-Bleurt} & -- & -- & 0.59 & 0.0 & -0.05 & 0.13 & -- & -- & -- & -- & - &  \\
\textbf{S-Cosine} & 0.51 & 0.0 & 0.69 & 0.0 & {\ul 0.62} & 0.0 & 0.61 & 0.0 & 0.65 & 0.0 & 0.62 & 4.4 \\
\textbf{R-Cosine} & -- & -- & 0.40 & 0.0 & 0.29 & 0.0 & -- & -- & -- & -- & - & \\ \midrule
\textbf{QuestEval} & 0.23 & 0.0 & 0.25 & 0.0 & 0.49 & 0.0 & 0.47 & 0.0 & 0.62 & 0.0 & 0.41 & 9.0 \\
\textbf{LLaMa3} & 0.36 & 0.0 & \textbf{0.84} & 0.0 & {\ul{0.62}} & 0.0 & 0.61 & 0.0 &  0.76 & 0.0 & 0.64 & 3.6 \\
\textbf{our (3b)} & 0.49 & 0.0 & 0.73 & 0.0 & 0.54 & 0.0 & 0.53 & 0.0 & 0.7 & 0.0 & 0.60 & 5.8 \\
\textbf{our (8b)} & 0.48 & 0.0 & 0.73 & 0.0 & 0.52 & 0.0 & 0.53 & 0.0 & 0.7 & 0.0 & 0.59 & 6.3 \\  \bottomrule
\end{tabular}
\caption{Pearson correlation on human evaluation on system output. `R-': reference-based. `S-': source-based.}
\label{tab:sys}
\end{table*}



\begin{table}%[]
\centering
\fontsize{11pt}{11pt}\selectfont
\begin{tabular}{llllll}
\toprule
\multicolumn{1}{c}{\textbf{task}} & \multicolumn{1}{c}{\textbf{Lai}} & \multicolumn{1}{c}{\textbf{Zei.}} & \multicolumn{1}{c}{\textbf{Scia.}} & \textbf{} & \textbf{} \\ 
\multicolumn{1}{c}{\textbf{metrics}} & \multicolumn{1}{c}{\textbf{cor.}} & \multicolumn{1}{c}{\textbf{cor.}} & \multicolumn{1}{c}{\textbf{cor.}} & \textbf{avg.} & \textbf{\begin{tabular}[c]{@{}l@{}}avg.\\ rank\end{tabular}} \\ \midrule
\textbf{S-Bleu} & 0.40 & 0.40 & 0.19* & 0.33 & 7.67 \\
\textbf{S-Meteor} & 0.41 & 0.42 & 0.16* & 0.33 & 7.33 \\
\textbf{S-BertS.} & {\ul0.58} & 0.47 & 0.31 & 0.45 & 3.67 \\
\textbf{S-Bleurt} & 0.45 & {\ul 0.54} & {\ul 0.37} & 0.45 & {\ul 3.33} \\
\textbf{S-Cosine} & 0.56 & 0.52 & 0.3 & {\ul 0.46} & {\ul 3.33} \\ \midrule
\textbf{QuestE.} & 0.27 & 0.35 & 0.06* & 0.23 & 9.00 \\
\textbf{LlaMA3} & \textbf{0.6} & \textbf{0.67} & \textbf{0.51} & \textbf{0.59} & \textbf{1.0} \\
\textbf{Our (3b)} & 0.51 & 0.49 & 0.23* & 0.39 & 4.83 \\
\textbf{Our (8b)} & 0.52 & 0.49 & 0.22* & 0.43 & 4.83 \\ \bottomrule
\end{tabular}
\caption{Pearson correlation on human ratings on reference output. *not significant; we cannot reject the null hypothesis of zero correlation}
\label{tab:ref}
\end{table}


\begin{table*}%[]
\centering
\fontsize{11pt}{11pt}\selectfont
\begin{tabular}{lllllllll}
\toprule
\textbf{task} & \multicolumn{1}{c}{\textbf{ALL}} & \multicolumn{1}{c}{\textbf{sentiment}} & \multicolumn{1}{c}{\textbf{detoxify}} & \multicolumn{1}{c}{\textbf{catchy}} & \multicolumn{1}{c}{\textbf{polite}} & \multicolumn{1}{c}{\textbf{persuasive}} & \multicolumn{1}{c}{\textbf{formal}} & \textbf{\begin{tabular}[c]{@{}l@{}}avg. \\ rank\end{tabular}} \\
\textbf{metrics} & \multicolumn{1}{c}{\textbf{cor.}} & \multicolumn{1}{c}{\textbf{cor.}} & \multicolumn{1}{c}{\textbf{cor.}} & \multicolumn{1}{c}{\textbf{cor.}} & \multicolumn{1}{c}{\textbf{cor.}} & \multicolumn{1}{c}{\textbf{cor.}} & \multicolumn{1}{c}{\textbf{cor.}} &  \\ \midrule
\textbf{S-Bleu} & -0.17 & -0.82 & -0.45 & -0.12* & -0.1* & -0.05 & -0.21 & 8.42 \\
\textbf{R-Bleu} & - & -0.5 & -0.45 &  &  &  &  &  \\
\textbf{S-Meteor} & -0.07* & -0.55 & -0.4 & -0.01* & 0.1* & -0.16 & -0.04* & 7.67 \\
\textbf{R-Meteor} & - & -0.17* & -0.39 & - & - & - & - & - \\
\textbf{S-BertScore} & 0.11 & -0.38 & -0.07* & -0.17* & 0.28 & 0.12 & 0.25 & 6.0 \\
\textbf{R-BertScore} & - & -0.02* & -0.21* & - & - & - & - & - \\
\textbf{S-Bleurt} & 0.29 & 0.05* & 0.45 & 0.06* & 0.29 & 0.23 & 0.46 & 4.2 \\
\textbf{R-Bleurt} & - &  0.21 & 0.38 & - & - & - & - & - \\
\textbf{S-Cosine} & 0.01* & -0.5 & -0.13* & -0.19* & 0.05* & -0.05* & 0.15* & 7.42 \\
\textbf{R-Cosine} & - & -0.11* & -0.16* & - & - & - & - & - \\ \midrule
\textbf{QuestEval} & 0.21 & {\ul{0.29}} & 0.23 & 0.37 & 0.19* & 0.35 & 0.14* & 4.67 \\
\textbf{LlaMA3} & \textbf{0.82} & \textbf{0.80} & \textbf{0.72} & \textbf{0.84} & \textbf{0.84} & \textbf{0.90} & \textbf{0.88} & \textbf{1.00} \\
\textbf{Our (3b)} & 0.47 & -0.11* & 0.37 & 0.61 & 0.53 & 0.54 & 0.66 & 3.5 \\
\textbf{Our (8b)} & {\ul{0.57}} & 0.09* & {\ul 0.49} & {\ul 0.72} & {\ul 0.64} & {\ul 0.62} & {\ul 0.67} & {\ul 2.17} \\ \bottomrule
\end{tabular}
\caption{Pearson correlation on human ratings on our constructed test set. 'R-': reference-based. 'S-': source-based. *not significant; we cannot reject the null hypothesis of zero correlation}
\label{tab:con}
\end{table*}

\section{Results}
We benchmark the different metrics on the different datasets using correlation to human judgement. For content preservation, we show results split on data with system output, reference output and our constructed test set: we show that the data source for evaluation leads to different conclusions on the metrics. In addition, we examine whether the metrics can rank style transfer systems similar to humans. On style strength, we likewise show correlations between human judgment and zero-shot evaluation approaches. When applicable, we summarize results by reporting the average correlation. And the average ranking of the metric per dataset (by ranking which metric obtains the highest correlation to human judgement per dataset). 

\subsection{Content preservation}
\paragraph{How do data sources affect the conclusion on best metric?}
The conclusions about the metrics' performance change radically depending on whether we use system output data, reference output, or our constructed test set. Ideally, a good metric correlates highly with humans on any data source. Ideally, for meta-evaluation, a metric should correlate consistently across all data sources, but the following shows that the correlations indicate different things, and the conclusion on the best metric should be drawn carefully.

Looking at the metrics correlations with humans on the data source with system output (Table~\ref{tab:sys}), we see a relatively high correlation for many of the metrics on many tasks. The overall best metrics are S-BertScore and S-BLEURT (avg+avg rank). We see no notable difference in our method of using the 3B or 8B model as the backbone.

Examining the average correlations based on data with reference output (Table~\ref{tab:ref}), now the zero-shoot prompting with LlaMA3 70B is the best-performing approach ($0.59$ avg). Tied for second place are source-based cosine embedding ($0.46$ avg), BLEURT ($0.45$ avg) and BertScore ($0.45$ avg). Our method follows on a 5. place: here, the 8b version (($0.43$ avg)) shows a bit stronger results than 3b ($0.39$ avg). The fact that the conclusions change, whether looking at reference or system output, confirms the observations made by \citet{scialom-etal-2021-questeval} on simplicity transfer.   

Now consider the results on our test set (Table~\ref{tab:con}): Several metrics show low or no correlation; we even see a significantly negative correlation for some metrics on ALL (BLEU) and for specific subparts of our test set for BLEU, Meteor, BertScore, Cosine. On the other end, LlaMA3 70B is again performing best, showing strong results ($0.82$ in ALL). The runner-up is now our 8B method, with a gap to the 3B version ($0.57$ vs $0.47$ in ALL). Note our method still shows zero correlation for the sentiment task. After, ranks BLEURT ($0.29$), QuestEval ($0.21$), BertScore ($0.11$), Cosine ($0.01$).  

On our test set, we find that some metrics that correlate relatively well on the other datasets, now exhibit low correlation. Hence, with our test set, we can now support the logical reasoning with data evidence: Evaluation of content preservation for style transfer needs to take the style shift into account. This conclusion could not be drawn using the existing data sources: We hypothesise that for the data with system-based output, successful output happens to be very similar to the source sentence and vice versa, and reference-based output might not contain server mistakes as they are gold references. Thus, none of the existing data sources tests the limits of the metrics.  


\paragraph{How do reference-based metrics compare to source-based ones?} Reference-based metrics show a lower correlation than the source-based counterpart for all metrics on both datasets with ratings on references (Table~\ref{tab:sys}). As discussed previously, reference-based metrics for style transfer have the drawback that many different good solutions on a rewrite might exist and not only one similar to a reference.


\paragraph{How well can the metrics rank the performance of style transfer methods?}
We compare the metrics' ability to judge the best style transfer methods w.r.t. the human annotations: Several of the data sources contain samples from different style transfer systems. In order to use metrics to assess the quality of the style transfer system, metrics should correctly find the best-performing system. Hence, we evaluate whether the metrics for content preservation provide the same system ranking as human evaluators. We take the mean of the score for every output on each system and the mean of the human annotations; we compare the systems using the Kendall's Tau correlation. 

We find only the evaluation using the dataset Mir, Lai, and Ziegen to result in significant correlations, probably because of sparsity in a number of system tests (App.~\ref{app:dataset}). Our method (8b) is the only metric providing a perfect ranking of the style transfer system on the Lai data, and Llama3 70B the only one on the Ziegen data. Results in App.~\ref{app:results}. 


\subsection{Style strength results}
%Evaluating style strengths is a challenging task. 
Llama3 70B shows better overall results than our method. However, our method scores higher than Llama3 70B on 2 out of 6 datasets, but it also exhibits zero correlation on one task (Table~\ref{tab:styleresults}).%More work i s needed on evaluating style strengths. 
 
\begin{table}%[]
\fontsize{11pt}{11pt}\selectfont
\begin{tabular}{lccc}
\toprule
\multicolumn{1}{c}{\textbf{}} & \textbf{LlaMA3} & \textbf{Our (3b)} & \textbf{Our (8b)} \\ \midrule
\textbf{Mir} & 0.46 & 0.54 & \textbf{0.57} \\
\textbf{Lai} & \textbf{0.57} & 0.18 & 0.19 \\
\textbf{Ziegen.} & 0.25 & 0.27 & \textbf{0.32} \\
\textbf{Alva-M.} & \textbf{0.59} & 0.03* & 0.02* \\
\textbf{Scialom} & \textbf{0.62} & 0.45 & 0.44 \\
\textbf{\begin{tabular}[c]{@{}l@{}}Our Test\end{tabular}} & \textbf{0.63} & 0.46 & 0.48 \\ \bottomrule
\end{tabular}
\caption{Style strength: Pearson correlation to human ratings. *not significant; we cannot reject the null hypothesis of zero corelation}
\label{tab:styleresults}
\end{table}

\subsection{Ablation}
We conduct several runs of the methods using LLMs with variations in instructions/prompts (App.~\ref{app:method}). We observe that the lower the correlation on a task, the higher the variation between the different runs. For our method, we only observe low variance between the runs.
None of the variations leads to different conclusions of the meta-evaluation. Results in App.~\ref{app:results}.
%
\section{Conclusions} \label{sec:conclusion}
\section{Conclusion}
In this work, we propose a simple yet effective approach, called SMILE, for graph few-shot learning with fewer tasks. Specifically, we introduce a novel dual-level mixup strategy, including within-task and across-task mixup, for enriching the diversity of nodes within each task and the diversity of tasks. Also, we incorporate the degree-based prior information to learn expressive node embeddings. Theoretically, we prove that SMILE effectively enhances the model's generalization performance. Empirically, we conduct extensive experiments on multiple benchmarks and the results suggest that SMILE significantly outperforms other baselines, including both in-domain and cross-domain few-shot settings.
%
\section{Future work}
\section{Future Work}
\noindent \textbf{Eliciting Confidence Preference Data.} There can be several different ways of eliciting relative confidence judgments. Prompts could allow for ties in confidence or compare confidence across more than two questions. Kahneman-Tversky Optimization (KTO)~\citep{Ethayarajh2024KTOMA} for LM alignment 
achieves DPO~\citep{Rafailov2023DirectPO} levels of performance by using binary signals of desirability for generations. We can apply KTO to confidence preference data generation by asking for binary signals—--confident or not—--and then converting these into relative judgments, ranking “not confident” answers below “confident” ones.\\\\
\noindent \textbf{Rank Aggregation.} In this work, we explore the most popular rank aggregation methods like Elo rating~\citep{elo_ratings}, Bradley-Terry~\citep{bradley_terry}, and TrueSkill~\citep{true_skill}. Another approach to rank aggregation is to represent preference data as a graph, with nodes as questions and directed edges reflecting match outcomes between questions. Since the outcome of some of these matchups can be inconsistent and non-transitive, algorithms like Rank Centrality~\citep{Negahban2012RankCR}, PageRank~\citep{Page1999ThePC}, and Minimum Feedback Arc Set~\citep{Vahidi2024MinimumWF} could be used to reduce cycles in the graph and better manage these inconsistencies.\\\\
\noindent \textbf{Confidence Estimation for Longform Generations.} While we benchmark on multiple-choice tasks, relative confidence estimation can also extend to longform generation. Log probabilities on answer tokens are commonly used for confidence estimation in multiple-choice tasks, but token-level uncertainty doesn't translate well to longform sequences. Moreover, there may be different levels of uncertainty associated with different aspects of a longform generation, e.g. how complete a generation, vs how factual it is, etc. Relative confidence estimation could produce fine-grained confidence scores for different attributes of a longform response by adjusting the prompt for confidence preferences accordingly.\\\\
\noindent \textbf{Alignment with Relative Confidence.} Works like~\cite{Tian2023FinetuningLM} explore using absolute confidence scores to align language models for different attributes such as factuality, without human annotations (RLAIF). Since relative confidences are more calibrated than absolute confidences, we can instead use relative confidences to construct preference pairs for aligning models on different attributes. \\\\
\noindent \textbf{Curriculum Learning with Difficulty Estimates.} We also explore generating relative confidence judgments without revealing model answers (Section~\ref{sec:results}). These scores correspond to difficulty ratings, which could inform curriculum learning by first training on lower-difficulty examples.



\bibliographystyle{siamplain}
\bibliography{bte}
\end{document}
