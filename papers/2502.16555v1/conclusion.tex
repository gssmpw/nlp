We presented the \bte~solver for the one-dimensional electron BTE for LTPs. \bte~u-ses a Eulerian approach with a Chebyshev-collocation in the position space. In the velocity space, we use a Galerkin discretization in the radial direction and a mixed spherical harmonics with discrete ordinate method for the angular directions. We compare two models for electron transport, fluid and BTE. In the fluid approximation, we rely on the continuity equations for species transport. For the hybrid approximation, we use continuity equations for ions while using the BTE for electron transport. In the fluid approximation, we use non-Maxwellian treatment of EDFs with tabulated kinetic coefficients. As expected, at lower pressures, we observe significant differences between the time-periodic profiles, computed based on the fluid and hybrid approximations. The above indicates that the kinetic treatment of electrons is crucial for accurate RF-GDPs simulation, and the widely-used tabulated interpolation of kinetic coefficients is inadequate to capture the spatial coupling effects on EDFs. 



%\begin{itemize}
%	\item What we did in this paper ? 
%	\item The kinetic treatment of electrons is important in LTPs. Alternative modeling approaches with tabulated kinetic coefficients makes significant errors. 
%	\item 
%\end{itemize}