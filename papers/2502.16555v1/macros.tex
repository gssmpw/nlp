%\newtheorem{theorem}{Theorem}[section]
%\newtheorem{definition}{Definition}[section]
\newcommand{\R}{\mathbb{R}}
\newcommand{\norm}[1]{\left\lVert#1\right\rVert}
\newcommand{\ppartial}[2]{\frac{\partial}{\partial{#2}}{#1}}


\newcommand{\myint}{\int\limits}
\newcommand{\diff}[1]{\, d#1}
\newcommand{\vect}[1]{\boldsymbol{#1}}
\usepackage{mleftright}
\newcommand{\of}[1]{\mleft( #1 \mright)}
\newcommand{\ddt}[1]{\partial_t #1}
\newcommand{\vth}{v_\textrm{th}}
\newcommand{\reals}{\mathbb{R}}
\newcommand{\integers}{\mathbb{Z}}
\newcommand{\RR}{\mathbb{R}}
\newcommand{\vr}{v}
\newcommand{\hf}{{\hat{f}}}
%\newcommand{\vtheta}{\theta_{\vect{v}}}
%\newcommand{\vphi}{\varphi_{\vect{v}}}
%\newcommand{\vr}{v_{r}}
\newcommand{\bte}{\textsc{Boltzsim}}
\newcommand{\vtheta}{{v_{\theta}}}
\newcommand{\vphi}{v_{\varphi}}
\newcommand{\utheta}{{u_{\theta}}}
\newcommand{\uphi}{u_{\varphi}}
\newcommand{\vomega}{v_{\omega}}
\newcommand{\vrunit}{\hat{\vect{v}}_{r}}
\newcommand{\vthetaunit}{\hat{\vect{v}}_{\theta}}
\newcommand{\vphiunit}{\hat{\vect{v}}_{\varphi}}
\newcommand{\lp}[2]{L^{(#1)}_{#2}}
\newcommand{\lph}[2]{\tilde{L}^{(#1)}_{#2}}
\newcommand{\bfun}[2]{\Phi^{(#1)}_{#2}}
\newcommand{\tfun}[2]{\Psi^{(#1)}_{#2}}
\newcommand{\nfun}[2]{N^{(#1)}_{#2}}
\newcommand{\maxp}[2]{P^{(#1)}_{#2}}
\newcommand{\lagp}[2]{L^{(#1)}_{#2}}
\newcommand{\diffm}[3]{D^{(#1)}_{#2,#3}}
\newcommand{\planck}{h}
\newcommand{\te}{{\tilde{e}}}
\newcommand{\bolsig}{\textsc{Bolsig+}\xspace}

\definecolor{codegreen}{rgb}{0,0.6,0}
\definecolor{codegray}{rgb}{0.5,0.5,0.5}
\definecolor{codepurple}{rgb}{0.58,0,0.82}
\definecolor{backcolour}{rgb}{0.95,0.95,0.92}

%Code listing style named "mystyle"
\lstdefinestyle{mystyle}{
  backgroundcolor=\color{backcolour}, commentstyle=\color{codegreen},
  keywordstyle=\color{magenta},
  numberstyle=\tiny\color{codegray},
  stringstyle=\color{codepurple},
  basicstyle=\ttfamily\footnotesize,
  breakatwhitespace=false,         
  breaklines=true,                 
  captionpos=b,                    
  keepspaces=true,                 
  numbers=left,                    
  numbersep=5pt,                  
  showspaces=false,                
  showstringspaces=false,
  showtabs=false,                  
  tabsize=2
}

%"mystyle" code listing set
\lstset{style=mystyle}

\definecolor{a1}{RGB}{228,26,28}
\definecolor{a2}{RGB}{55,126,184}
\definecolor{a3}{RGB}{77,175,74}
\definecolor{a4}{RGB}{152,78,163}
\definecolor{a5}{RGB}{255,127,0}
\definecolor{a6}{RGB}{166,86,40}
\definecolor{a7}{RGB}{166,86,40}
\definecolor{a8}{RGB}{247,129,191}

\makeatletter
\DeclareRobustCommand{\pdot}{\mathbin{\mathpalette\pdot@\relax}}
\newcommand{\pdot@}[2]{%
  \ooalign{%
    $\m@th#1\circ$\cr
    \hidewidth$\m@th#1\cdot$\hidewidth\cr
  }%
}

%\newcommand{\added}[1]{\textcolor{black}{{#1}}}

