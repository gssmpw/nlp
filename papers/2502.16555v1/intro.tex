%The plasma state can be considered as the fourth state of matter, and it is characterized by the collective behavior of its species (i.e., neutrals, ions, and electrons). Predictive mathematical modeling of plasmas is useful in many application domains in science and engineering.
Low-temperature plasmas (LTPs) find application in advan-ced manufacturing, semiconductor processing, the medical device industry, plasma-assisted combustion, and numerous other application domains. Modeling of low-temperature, weakly ionized plasmas can be challenging due to non-equilibrium chemistry, time-scale differences, and non-local electromagnetic coupling of charged particles. Electrons in LTPs often deviate from thermal equilibrium. Hence, their velocity space distribution functions differ from the Maxwell-Boltzmann distribution. Accurate modeling of LTPs requires solving the BTE for the electron distribution function (EDF). The BTE evolves the EDF dynamics and is coupled to ion and neutral species dynamics. Solving the BTE is a challenging task due to the dimensionality of the problem, the multiple time scales in various plasma processes, and dense collisional operators. Predictive simulation of LTPs relies on a hierarchy of models ranging from semi-analytical models to a fully kinetic treatment of plasma. Most existing LTP simulations rely on a model known as the ``fluid approximation'' that is derived from averaging the first three velocity space moments, mass, momentum, and energy of the BTE. The fluid approximation for the electron transport is based on assumptions on the EDF that may be invalid in certain LTP conditions. They also require approximating quantities that depend on the EDF, e.g., reaction rates, electron mobility, and electron diffusivity.
%, which is globally coupled with self-consistent electromagnetic effects. Representing an arbitrary EDF necessitates solving the electron BTE for specified LTP conditions.  

In this paper, we relax the fluid approximation for the electrons and discuss the numerical solution of the BTE in one spatial dimension and three velocity dimensions. We refer to this problem as the 1D3V BTE and the spatially homogeneous case as the 0D3V BTE. There are two main approaches for numerically solving the 1D3V BTE: Lagrangian and Eulerian. In the Lagrangian approach, a particle-in-cell (PIC) scheme is used for phase space advection, and the direct simulation Monte Carlo (DSMC) sampling approach is used to model particle collisions. In contrast, Eulerian solvers discretize the BTE in a fixed frame with standard PDE discretization techniques such as Galerkin, finite differences, and spectral methods. Eulerian BTE solvers are also referred to as deterministic BTE solvers because of the absence of sampling methods for particle collisions. In this paper, we focus on the Eulerian approach for solving spatially coupled electron BTE for LTPs. The developed methodology is applied to the kinetic treatment of electrons in low-temperature radio-frequency glow discharge plasmas (RF-GDPs). To our knowledge, existing work uses either PIC-DSMC or fluid approximations for RF-GDP simulations. We introduce a hybrid solver where the heavies are represented using a fluid approximation, and the electrons are by the BTE. We refer to the new solver as~\bte. Our contributions are summarized below. 
%GDPs have applications in plasma cleaning, material processing, and plasma chemistry. 
%The current existing modeling methodology for RF-GDPs relies on fluid models with drift-diffusion approximation for species transport. The developed BTE solver is used to create a hybrid model for RF-GDPs. In the proposed model, electrons are modeled using the BTE, and heavy species (i.e., ions) are modeled using the standard fluid approximation. The contributions of the presented work are summarized below.  

\begin{itemize}
	\item \textbf{Multi-term EDF approximation}: \bte~extends the traditional two-term EDF expansion with arbitrary multi-term EDF approximations for one-dimensional space electron BTE. 
	\item \textbf{Semi-implicit and fully implicit schemes}: We present novel numerical schemes for the time evolution of electron BTE, with semi-implicit and fully-implicit coupling with the self-consistent electric field generated by charged plasma species. We propose a domain decomposition-based preconditioning method for the implicit time integration of the BTE  for both semi and fully implicit schemes.  
	\item \textbf{Self-consistent hybrid model for RF-GDPs}: We propose a self-con-sistent fluid and BTE combined hybrid transport model for RF-GDPs. We use the BTE for the kinetic treatment of electrons and the fluid approximation for heavy species transport. 
	\item \textbf{Comparison between fluid and hybrid models}: A comparison between fluid and hybrid transport models is carried out for RF-GDPs, with argon ions and electrons. Using the RF-GDP setup, we verify our solver by comparing to an in-house PIC-DSMC code. 
	\item \textbf{Open source}: Our solver is implemented in the Python programming language with support for CPU and portable GPU acceleration. The solver is openly available at \url{https://github.com/ut-padas/boltzmann}. 
\end{itemize}
%In our experiments, we use a chemistry model in which the neutral number density is constant and with electron-neutral momentum transfer and ionization reactions. So, the heavies consist of ions. Our methodology, however, readily extends to more general chemistry models.  