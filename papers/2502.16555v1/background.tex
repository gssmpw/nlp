%There exist several Eulerian codes~\cite{hagelaar2005solving,hagelaar2015coulomb,frost1962rotational,COMSOL1998, pancheshnyi2008zdplaskin,morgan1990elendif} that use fixed two-term EDF approximations for solving spatially homogeneous BTE. Similar to~\cite{fernando0DBTE} the work in \cite{mutibolt} also supports multi-term EDF approximations. However, it does not support Coulomb interactions and transient solutions. The PIC-DSMC approach is widely used for solving spatially inhomogeneous BTE for various application settings~\cite{scanlon2010open, zabelok2015adaptive, frezzotti2011solving, oblapenko2020velocity, bartel2003modelling, lymberopoulos1994stochastic, levko2021vizgrain}. To the best of our knowledge, there has been limited work on spatially inhomogeneous Eulerian electron BTE solvers for LTPs as well as RF-GDPs. Also, the existing literature has limited discussion on solver algorithmic details and related complexity analysis for 1D3V LTP solvers.

We summarize existing methods for modeling GDPs. The fluid approximation is the most widely used approach to simulate RF-GDPs~\cite{boeuf1987numerical,barnes1988staggered,zhao2018numerical,young1993two, young1993comparative,panneer2015computational,kothnur2007simulation,liu2014numerical}. 
%The fluid approximation for species transport is derived from the BTE's mass, momentum, and energy moments evolution equations. The momentum continuity equation is further simplified using the so-called drift-diffusion~\cite{hill1986introduction} approximation. The drift-diffusion approximation eliminates the time evolution of the momentum continuity equation, which is indirectly coupled to the mass continuity equation. The work described by \cite{zhao2018numerical, liu2014numerical} presents a detailed study on the effect of the secondary electron emission coefficient using the one-dimensional fluid approximation of RF-GDPs. 
There have been several attempts for the two-dimensional modeling of RF-GDPs using the fluid approximation~\cite{young1993two, wilcoxson1996simulation}. 
%The work described in \cite{lin2001simulation} presents a reduced-order modeling approach for RF-GDPs, based on the solutions computed by the fluid approximation. 
Lagrangian particle-in-cell algorithms for evolving fluid equations for GDPs are discussed in~\cite{young1993two, young1993comparative}. The fluid approximation assumes that species distribution functions are isotropic. This is not the case, especially for the electrons closer to discharge walls due to a combination of boundary conditions and the strong electric field gradients in the sheath. Most BTE solvers are based on the PIC-DSMC approach~\cite{scanlon2010open, zabelok2015adaptive, frezzotti2011solving, oblapenko2020velocity, bartel2003modelling, lymberopoulos1994stochastic, levko2021vizgrain}. 
There have been several attempts~\cite{surendra1991particle, bogaerts1999hybrid, satake1997two, tsendin1995electron, yuan20171d} to develop more accurate models for GDPs, specifically using the BTE for electrons.
The authors in~\cite{surendra1991particle} use a PIC-DSMC approach to model species transport using the BTE but have limited details on the collisional process. A hybrid modeling approach that uses the fluid approximation for heavies and the BTE for electrons is proposed in~\cite{bogaerts1999hybrid} using a PIC-DSMC to simulate the BTE effects.
%In the above, their PIC-DSMC implementation requires an electric field as a spatiotemporal field, which is computed from the fluid approximation. 
In~\cite{satake1997two} the authors propose a two-dimensional hybrid model for RF-GDPs by evolving the BTE with a simplified Bhatnagar–Gross–Krook (BGK) collision operator. 
A hybrid transport model with two-term EDF approximations is proposed in~\cite{ yuan20171d,loffhagen2009advances}. The work given by~\cite{yuan20171d} uses an approximation scheme to update the two EDF components, and the work by~\cite{loffhagen2009advances} uses a quasi-stationary approximation of the EDF with a simplified collision operator. Both of these works do not solve the one-dimensional electron BTE.

To the best of our knowledge, there has been limited work on spatially inhomogeneous Eulerian electron BTE solvers for LTPs as well as RF-GDPs. Also, the existing literature has limited discussion on solver algorithmic details and related complexity analysis for 1D3V LTP solvers. Our work is based on a Eulerian 0D3V BTE solver we developed in our group~\cite{fernando0DBTE}. For related work please see the discussion in~\cite{fernando0DBTE}. Here we focus on 1D3V solvers.
% to simulate direct current GDPs.




 





%There are several modeling approaches for numerical simulation of RF glow discharge plasmas. Most commonly used approach being the fluid approximation for species transport. 

%. The above is derived from velocity space moments (i.e., mass, momentum and energy)
%Numerical simulation of RF glow dischar






%The existing methodology to solve the BTE can be categorized in to Eulerian (sometimes called deterministic solvers) and Lagrangian approaches. 
%
%There have been several work on Eulerian BTE solvers with 
%Eulerian and Lagrangian particle-in-cell with direct simulation Monte Carlo methods
%Existing modeling approaches can be categorized into following. 
%Existing models can be hierarchically organized based on their overall accuracy. 
%The highest fidelity model would be the use of BTE for transport of all species. The above 
%
%\begin{itemize}
%	\item ROM. 
%	\begin{itemize}
%		\item Model reduction based in fluid approximations \url{https://www.sciencedirect.com/science/article/pii/S0021999101968081}
%	\end{itemize}
%	\item Fluid approximations 
%	\begin{itemize}
%		\item Numerical model of rf glow discharges ; \url{https://journals.aps.org/pra/abstract/10.1103/PhysRevA.36.2782}
%		\item \url{https://www.sciencedirect.com/science/article/pii/0021999188901568}
%		\item SSE effects, - \url{https://iopscience.iop.org/article/10.1088/1674-1056/27/2/025201/pdf}
%		\url{https://pubs.aip.org/aip/pop/article-abstract/21/8/083511/955154/Numerical-study-of-effect-of-secondary-electron?redirectedFrom=fulltext}
%		\item 2D fluid solver \url{https://www.sciencedirect.com/science/article/pii/S000925099680008X}
%		\item PIC for fluid approximations. \url{https://ieeexplore-ieee-org.ezproxy.lib.utexas.edu/stamp/stamp.jsp?tp=&arnumber=277557}
%		\item \url{https://iopscience.iop.org/article/10.1088/0022-3727/26/5/010/pdf} comparison with different fluid models with MC simulations. 
%		\item Raja's paper ; \url{https://pubs.aip.org/aip/jap/article/118/24/243301/141536/Computational-modeling-of-the-effect-of-external}
%	\end{itemize}
%	\item PIC-DSMC codes
%	\begin{itemize}
%		\item Particle Simulations of Radio-Frequency
%		Glow Discharges \url{https://ieeexplore-ieee-org.ezproxy.lib.utexas.edu/stamp/stamp.jsp?tp=&arnumber=106808}
%		\item PIC-DSMC with hybrid -\url{https://iopscience.iop.org/article/10.1143/JJAP.38.4404/pdf}
%		\item \url{https://iopscience.iop.org/article/10.1143/JJAP.36.4789/pdf}
%		
%		\item 2D PIC-BTE with fluid model \url{https://iopscience.iop.org/article/10.1143/JJAP.36.4789/pdf}
%	\end{itemize}
%	\item Eulerian codes
%	\begin{itemize}
%		\item \url{https://iopscience.iop.org/article/10.1088/0963-0252/18/3/034006/pdf?casa_token=cwIYxFtzewkAAAAA:l2WqchzAyFzV3yfCvulW4b9js_TrRWcdZDbr9WxNEpRPqx_B_dtdLb6m2etoCvk7LIPRwWgMtEJpr6Ifc-EAK_UxgtuW}
%		\item{DC glow with 2-term BTE}-		\url{https://www.researchgate.net/publication/318292261_1D_kinetic_simulations_of_a_short_glow_discharge_in_helium/link/59b3bfe9a6fdcc3f889568c7/download?_tp=eyJjb250ZXh0Ijp7ImZpcnN0UGFnZSI6InB1YmxpY2F0aW9uIiwicGFnZSI6InB1YmxpY2F0aW9uIn19}
%		\item semi-analytical model \url{https://iopscience.iop.org/article/10.1088/0963-0252/4/2/004/pdf?casa_token=P1bFsSkkcRAAAAAA:9DVFZO--gLxH_gKCJ5FVdZ_hxAO_ipPBdlU76wYDOVzKsQevJaeBfGVWj-6u9nVA_R9aovGomzJOsOijwTMfEhjvySi9}
%		\item 
%	\end{itemize}
%\end{itemize}