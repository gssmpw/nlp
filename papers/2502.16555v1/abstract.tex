We present an algorithm for solving the one-dimensional space collisional Boltzmann transport equation (BTE) for electrons in low-temperature plasmas (LTPs). Modeling LTPs is useful in many applications, including advanced manufacturing, material processing, and hypersonic flows, to name a few. The proposed BTE solver is based on an Eulerian formulation. It uses Chebyshev collocation method in physical space and a combination of Galerkin and discrete ordinates in velocity space. We present self-convergence results and cross-code verification studies compared to an in-house particle-in-cell (PIC) direct simulation Monte Carlo (DSMC) code. \bte~is our open source implementation of the solver. Furthermore, we use \bte~to simulate radio-frequency glow discharge plasmas (RF-GDPs) and compare with an existing methodology that approximates the electron BTE. We compare these two approaches and quantify their differences as a function of the discharge pressure. The two approaches show an 80x, 3x, 1.6x, and 0.98x difference between cycle-averaged time periodic electron number density profiles at 0.1 Torr, 0.5 Torr, 1 Torr, and 2 Torr discharge pressures, respectively. As expected, these differences are significant at low pressures, for example less than 1 Torr. 

%the electron BTE with the widely used two temperature ``fluid'' approximations with tabulated coefficients. We report the differences between the electron BTE and fluid approximation of electron transport. As expected, these differences are significant in lower pressures (i.e., less than 200 mTorr). %especially in low-pressure RF-GDPs. 

%Most state-of-the-art methods for electron kinetics are based on Monte-Carlo sampling for collisions combined with Lagrangian particle-in-cell methods. We discuss
%an Eulerian solver that approximates the electron velocity distribution function using spherical harmonics (angular components)
%and B-splines (energy component). Our solver supports electron-heavy elastic and inelastic binary collisions, electron-electron
%Coulomb interactions, steady-state and transient dynamics, and an arbitrary nmber of angular terms in the electron distribution
%function. We report convergence results and compare our solver to two other codes: an in-house particle Monte-Carlo method; and
%Bolsig+, a state-of-the-art Eulerian solver for electron transport in LTPs. Furthermore, we use our solver to study the relaxation
%time scales of the higher-order anisotropic correction terms. Our code is open-source and provides an interface that allows coupling
%to multiphysics simulations of low-temperature plasmas.