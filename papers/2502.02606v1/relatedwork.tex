\section{Related Work}
\subsection{Carbon Footprint of Semiconductor Manufacturing}
The environmental impact of semiconductor fabrication has been extensively studied, with a primary focus on energy consumption and greenhouse gas (GHG) emissions. The studies in\cite{huang2016developing} and \cite{vasan2014carbon} conducted a life cycle assessment (LCA) of semiconductor manufacturing, revealing that the wafer fabrication stage alone contributes up to 80\% of total emissions due to the high-energy demands of ion implantation, photolithography, and chemical vapor deposition (CVD). The study highlighted that perfluorinated compounds (PFCs), a class of synthetic compounds containing thousands of chemicals formed from carbon chains with fluorine attached to these chains, used in plasma etching and chamber cleaning have global warming potentials (GWP) thousands of times higher than $CO_2$, making their reduction a critical challenge.

Similarly, the studied in \cite{huang2016developing} and \cite{boyd2011life} analyzed emissions from 300$mm$ wafer fabrication, estimating that each 5$nm$ process node chip requires over 450 $kWh$ of energy per wafer, translating to approximately 35–50 $kg$ $CO_2$ per wafer under a global average electricity grid mix. The study suggested moving to renewable energy sources in semiconductor foundries as an effective strategy to reduce emissions.

However, these studies provide macro-scale insights into wafer-level carbon footprints but lack transistor-level granularity. The per-transistor $CO_2$ footprint remains poorly quantified, limiting the precision of green computing benchmarks. Our work addresses this critical gap by introducing a mathematical framework to compute Carbon Per Transistor (CPT) across different processors.

\subsection{Power Dissipation and Energy Efficiency in Computing}
The operational energy consumption of transistors is a key factor in sustainable computing. The legendary work in \cite{dennard1974design} originally predicted that as transistor size shrinks, power consumption per transistor should decrease proportionally. However, due to power density limitations, modern processors face the end of Dennard Scaling, leading to increased dynamic and leakage power dissipation\cite{horowitz20141}.
Several studies have examined power efficiency trends in computing. The work in\cite{koomey2011growth} demonstrated that the energy efficiency of computing devices doubles approximately every 1.5 years, following an exponential improvement trend. The authors in\cite{hennessy2011computer} emphasized that while performance-per-watt has improved due to better architecture and process scaling, the total energy consumption of modern chips remains high due to the increasing transistor count in multi-core architectures. Moreover, the work in \cite{sungheetha2024adaptive} proposed chip-level optimizations, such as dynamic voltage scaling (DVS), near-threshold computing, and specialized AI accelerators, to reduce operational power consumption. Despite these efforts, no existing research explicitly quantifies $CO_2$ emissions at the transistor level. This study bridges that gap by combining power dissipation analysis with semiconductor manufacturing emissions, establishing a rigorous $CO_2$ footprint per transistor metric.

\subsection{Semiconductor Industry Efforts Toward Green Computing}
Leading semiconductor manufacturers have initiated efforts to reduce their carbon footprint. Taiwan Semiconductor Manufacturing Company Limited (TSMC) has committed to achieving net-zero emissions by 2050, with initiatives including 100\% renewable energy adoption and process efficiency improvements \cite{nagapurkar2023cleaner}. Additionally, Intel has pledged to achieve zero carbon fabs by 2040, focusing on carbon-neutral chip packaging and low-emission materials\cite{ivan2021practices}. Furthermore, AMD, and NVIDIA have introduced low-power AI accelerators, optimizing performance-per-watt to reduce the energy impact of data center AI workloads\cite{kang2024sustainability}. While these initiatives are commendable, they lack a standardized, quantifiable metric to assess the sustainability of individual transistor designs. The Carbon Per Transistor (CPT) formula proposed in this study provides such a metric, enabling comparative sustainability analysis across different architectures, process nodes, and device categories.

The related works reviewed above establish the need for a scientific metric quantifying per-transistor carbon emissions. Existing research has examined wafer-level emissions and energy use. Still, it lacks transistor-level granularity, processor power dissipation trends but does not compute $CO_2$ emissions per transistor, and Industry-led sustainability efforts but lacks a standardized $CO_2$ metric for green computing.

However, to our knowledge, no proposed Carbon Per Transistor (CPT) formula to calculate $CO_2$ emissions per transistor across manufacturing and operation was done or validated, and no established benchmark for sustainable semiconductor design exists yet. By bridging these research gaps, this work advances the state-of-the-art in carbon-conscious computing, paving the way for low-carbon semiconductor design and green computing metrics.