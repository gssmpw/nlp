% to be able to draw some self-contained figs
\usepackage{tikz}
\usepackage{amsmath}

% inlined bib file
\usepackage{filecontents}

% \documentclass[conference,compsoc]{IEEEtran}
% \IEEEoverridecommandlockouts
% \usepackage{epsfig,endnotes}
%%%%%%%%%%%%%%%%%%%%%%%%%%%%%%%%%%%%%%%%%%%%%%%%
\usepackage{svg}
\usepackage{mdframed}
\usepackage{graphicx}
\usepackage{algorithm}
\usepackage{algorithmicx}
\usepackage{gensymb}
\usepackage[utf8]{inputenc}
\DeclareUnicodeCharacter{2212}{-}
\usepackage{mathtools}
\usepackage{algpseudocode}
\usepackage{multirow}
\usepackage{amssymb}
\usepackage{url}
\usepackage{hyperref}
\usepackage{xcolor,colortbl}
\usepackage{caption}    % To create space between listing caption and listing
\usepackage{array}
\usepackage{booktabs}
\usepackage{pifont}
\usepackage{verbatim}
\usepackage{tikz}
\newtheorem{claim}{Claim}
\usepackage{makecell}
\newtheorem{definition}{Definition}
\usepackage{amsmath, amssymb, amsfonts}
\usepackage{enumitem}
\usepackage{siunitx}
\usepackage{subcaption}

\usepackage{listings}
\usepackage{xcolor}
\definecolor{green}{rgb}{0.01, 0.75, 0.24}
\definecolor{red}{rgb}{0.76, 0.23, 0.13}
\usepackage{pifont}% http://ctan.org/pkg/pifont
\newcommand{\cmark}{\textcolor{green}{\ding{51}}}%
\newcommand{\xmark}{\textcolor{red}{\ding{55}}}%

\definecolor{red}{HTML}{FF6666}
\definecolor{green}{HTML}{67AB9F}
\definecolor{purple}{HTML}{CCCCFF}

% Define custom style for highlighted lines
\lstdefinestyle{highlight}{
    backgroundcolor=\color{yellow},
}



\usepackage{pifont}% http://ctan.org/pkg/pifont
% \newcommand{\cmark}{\ding{51}}%
% \newcommand{\xmark}{\ding{55}}%

\renewcommand{\paragraph}[1]{\medskip \noindent {\bf #1.}}

\newcommand{\lstbg}[3][0pt]{{\fboxsep#1\colorbox{#2}{\strut #3}}}
\lstdefinelanguage{diff}{
  basicstyle=\ttfamily\small,
  morecomment=[f][\lstbg{red!20}]-,
  morecomment=[f][\lstbg{green!20}]+,
  morecomment=[f][\textit]{@@},
  %morecomment=[f][\textit]{---},
  %morecomment=[f][\textit]{+++},
}

%\newcommand{\hl}[1]{\textcolor{red}{\textbf{#1}}}
\newcommand{\hl}[1]{\textcolor{black}{#1}}
\newcommand{\oldhl}[1]{\textcolor{black}{#1}}


%\lstset { %
%    language=C++,
%    backgroundcolor=\color{black!5}, % set backgroundcolor
%    basicstyle=\footnotesize,% basic font setting
%} 
\definecolor{mGreen}{rgb}{0,0.58,0}
\definecolor{mGray}{rgb}{0.5,0.5,0.5}
\definecolor{mPurple}{rgb}{0.58,0,0.82}
\definecolor{backgroundColour}{rgb}{0.80,0.80,0.80}

\lstdefinelanguage{markdown}{
  morekeywords={},
  sensitive=false, 
  basicstyle=\footnotesize\ttfamily,
  columns=fullflexible,
  breaklines=true,      
  frame=single, 
  morecomment=[l]{\#},
  morestring=[b]",
}

\lstdefinestyle{CStyle2}{
  %backgroundcolor=\color{backgroundColour!5},  
  backgroundcolor=\color{white},
  basicstyle=\footnotesize\ttfamily,
  columns=fullflexible,
  breakatwhitespace=false,      
  breaklines=true,                
  captionpos=b,                    
  commentstyle=\color{mGreen}, 
  extendedchars=true,              
  frame=single,                   
  keepspaces=true,             
  keywordstyle=\color{blue},      
  language=c++,                 
  numbers=left,                
  numbersep=5pt,                   
  numberstyle=\tiny\color{mGray}, 
  rulecolor=\color{mGray},        
  showspaces=false,               
  showtabs=false,                 
  stepnumber=1,                  
  stringstyle=\color{magenta},    
  tabsize=1,                      
  title=\lstname,
  morecomment=[f][\lstbg{red!20}]-,
  morecomment=[f][\lstbg{green!20}]+,
  morecomment=[f][\textit]{@@},
  %morecomment=[f][\textit]{---},
}


\definecolor{mGreen}{rgb}{0,0.58,0}
\definecolor{mGray}{rgb}{0.5,0.5,0.5}
\definecolor{mPurple}{rgb}{0.58,0,0.82}
\definecolor{backgroundColour}{rgb}{0.80,0.80,0.80}
