
\section{Introduction}
In the past two decades, our computing systems have evolved and grown at an astounding rate. A side effect of this growth has been increased resource sharing and, with it, erosion of isolation boundaries. \textit{Multitenancy} has already been shown to be a significant security and privacy threat in shared cloud instances. VM boundaries can be invalidated either by software or hardware bugs~\cite{van2020lvi,islam2019spoiler,moghimi2020medusa,correia2018copycat} or by exploiting subtle information leakages at the hardware level~\cite{tpmfail}. 
%
%Emerging microVM solutions offer isolated VMs to ensure secure computing environments. For instance, Amazon's Nitro and more recent introduction of Firecracker aim to completely virtualize hardware resources and allow sharing using a lightweight solution which already powers lambda functions handling trillions of requests each month for AWS customers. While promising, the details of the isolation offered are not made public by AWS and have yet to be vetted by third parties. 

\paragraph{Microarchitectural Threats}
Arguably, one of the greatest security threats comes from attacks that target the implementation through side-channels or from hardware vulnerabilities. Such attacks started as a niche exploiting leakages through execution timing, power, and electromagnetic emanations but later evolved to exploit microarchitectural (\Mi ) leakages, e.g. through shared cache and memory subsystems, speculative execution, shared peripherals, etc. \Mi\ threats represent one of the most significant types of vulnerabilities since they can be carried out remotely with software access only. Prime examples of these threats are the early execution timing~\cite{kocher1996timing} and cache attacks~\cite{yarom2014flush+, liu2015last,liu2015last}, and later Meltdown, Spectre~\cite{kocher2019spectre}, and MDS attacks~\cite{moghimi2020medusa,canella2019fallout,vanbulck2020lvi}  which allow an unprivileged user to access privileged memory space breaking isolation mechanisms such as memory space isolation across processes, cores, browsers tabs and even virtual machines hosted on shared cloud instances. Active attacks, e.g. Rowhammer, have also proven effective in recovering sensitive information~\cite{kwong2020rambleed} and~\cite{mus2023jolt,adiletta2023mayhem}. While numerous practical countermeasures were proposed and implemented, there remains a massive attack surface unexplored. Indeed, 5 years after Meltdown was mitigated (August 2023), a new transient execution vulnerability, Downfall~\cite{moghimi2023downfall}, was discovered that exploits speculative data gathering and allows Meltdown-style data leakage and even injection across threads. 
%Further complicating analysis, FPGA and GPU systems are now incorporated in the IaaS clouds and devices virtualized. Integrated FPGAs were shown~\cite{weissman_jackhammer_2020}  to be used to target DRAMs, bypassing the memory controller, or to target PCI/DMA connections as well as I/O memory management units~\cite{iotlb} .

\paragraph{Lack of Access to Design Internals}
A significant factor contributing to the difficulty of evaluating the security of large-scale computer systems is that design details are rarely disclosed. Given only superficial interface definitions, researchers are forced to reverse engineering and black box analysis. While companies have access to the internals of their system, it is hard to argue that they are aware of their own designs either due to third-party IPs, mobility of engineers, and silos isolating their engineering teams from each other. IPs are orphaned with little superficial information surviving after only a few years of breaking institutional memory. These factors combined pose a great danger for \Mi\ security. 

The primary goal of the proposed work is to answer the following question: \textit{Can we use AI to automatically find brand-new vulnerabilities?}
%
In practical terms, can we build an AI agent that can discover the next Meltdown or Spectre vulnerabilities? Currently, there are intense efforts in the cybersecurity research community to deploy AI tools to scan Open Source Software (OSS) for known vulnerabilities, e.g. for detection in \Mi\ we have~\cite{doychev2015cacheaudit,jan2018microwalk,jan2022microwalk,cauligi2020constant,wang2019kleespectre,guarnieri2020spectector} and for patching~\cite{gupta2017deepfix,yasunaga2021break,tarlow2020learning,pearce2023examining,wu2023effective,garg2023rapgen} and~\cite{tol2023zeroleak}.

We take on a more challenging problem and investigate how we can build an AI Agent that constantly searches the target platform for brand new \Mi\ vulnerabilities. In a way, such an ability would bring true scalability and a tipping point since, if granted, we could surpass human abilities by creating as many AI Agents as we want by just throwing more cycles at the problem. In the hands of software/hardware vendors, such a tool would allow us to address vulnerabilities early on before the software advances deeper in the deployment pipeline. What is missing is the know-how to put such a system together i.e. a tool that can constantly analyze a hardware/software stack under popular configurations, identify and report found vulnerabilities, articulating cause and effect and severity of the vulnerability. In this work, we take inspiration from cybersecurity researchers on how they came up with new vulnerabilities:

\paragraph{Randomization} There is a healthy dose of manual or automated trial and error in discovering new vulnerabilities. In \Mi\, security fuzzing has become an indispensable tool to test randomized attack vectors and thereby identify or generate improved versions of vulnerabilities. For instance, Oleksenko et al.\cite{oleksenko2019specfuzz} developed SpecFuzz to test for speculative execution vulnerabilities. The tool combines dynamic simulation with conventional fuzzing for the identification of potential Spectre vulnerabilities. Another example is Transyther~\cite{moghimi2020medusa} , a mutational fuzzing tool that generates Meltdown variants and tests them to discover leaks. Transyther found a previously unknown transient execution attack through the word combining buffer in Intel CPUs~\cite{moghimi2020medusa}. In~\cite{jattke2022blacksmith}, Jattke et al. use fuzzing to discover non-uniform hammering patterns to make Rowhammer fault injection viable in a large class of DRAM devices. While effective, fuzzing, as currently practiced in \Mi\ security, only works in small domains and fails to scale to cover larger domains to discover new vulnerabilities. Indeed, SpecFuzz for Spectre v1 is only able to Spectre gadgets, and Transyther discovered the Medusa vulnerability since it is reachable with mild randomization from Meltdown variants. 


The discovery of the timing channel by Kocher~\cite{kocher1996timing} led to the discovery of cache-timing attacks~\cite{osvik2006cache}. Similarly, sharing in Branch Prediction Units (BPUs) led to the exploitation of secret dependent branching behavior to recover leakages~\cite{aciiccmez2006predicting}. These attacks led to \Mi\ Covert Channels that may be used intentionally to exfiltrate data, e.g. by signaling via cache access patterns and break isolation mechanisms. Covert-channels were first used by many researchers as an initial demonstration of the existence of a side-channel, with the channel rate providing a measure for the level of the leakage. Covert channels and manipulations in BPUs, in turn, became enablers for Transient Execution Attacks such as Meltdown, Spectre, and later MDS attacks. Further, the recent work~\cite{moghimi2023downfall} uses the Meltdown style data leakage and the LVI style~\cite{vanbulck2020lvi}  data injection mechanisms in the context of SIMD instructions to discover new vulnerabilities.

The x86 instruction set is a complex architecture that supports thousands of instructions, registers, and addressing modes, with each microarchitecture adding layers of optimizations for performance and efficiency. These optimizations, while beneficial, introduce complexities that can hide vulnerabilities, as seen with exploits like Meltdown and Spectre, which exploit unexpected microarchitectural behavior to expose sensitive data.
%
Traditional testing methods like random fuzzing are inadequate due to the vast number of instruction combinations and the specific, rare conditions that often trigger vulnerabilities. Complex features like out-of-order and speculative execution increase both performance and the difficulty of detecting flaws, making the discovery of microarchitectural vulnerabilities challenging.

An effective approach involves intelligent, feedback-based testing, where processor behavior under different conditions guides the search for vulnerabilities. This approach allows testing to focus on high-priority areas, improving efficiency and effectiveness. Feedback mechanisms can also adapt to new microarchitectures, adjusting their methods for each processor generation, an essential feature given the rapid evolution of hardware designs.
%
Machine Learning (ML) enhances this feedback-driven approach by identifying patterns in cache or power usage that indicate potential vulnerabilities. Over time, ML models improve, enabling more systematic and scalable vulnerability discovery across diverse processor designs.
%
RL further advances this approach, using a reward-based system to optimize instruction space exploration. RL agents prioritize instruction sequences that reveal anomalies, efficiently balancing exploration with exploiting known vulnerabilities, making them suitable for evolving architectures.

In summary, random fuzzing alone is insufficient for discovering vulnerabilities in modern x86 microarchitectures. Integrating feedback mechanisms with RL allows a more targeted, adaptable, and effective approach, essential for uncovering hidden vulnerabilities and maintaining security in rapidly advancing processor designs.

In this work, we make the following contributions:
\begin{enumerate}[nosep]
    \item We propose a novel approach to discovering microarchitectural vulnerabilities using RL.
    
    \item We develop a custom RL environment that simulates the execution of x86 instructions on a microarchitecture, allowing the agent to explore the instruction space.
    
    \item We find new transient execution based leakage mechanisms on Intel Skylake-X and Raptor Lake microarchitectures based on masked FP exceptions and MME/x87 transitions demonstrating the effectiveness of the RL agent in discovering vulnerabilities.

\end{enumerate}
