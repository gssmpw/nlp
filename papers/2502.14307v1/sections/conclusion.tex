\section{Conclusion}

% In this study, we developed an RL framework tailored for the detection of vulnerabilities within processor microarchitecture. We demonstrated that our RL agent can detect previously discovered vulnerabilities and discover unknown mechanisms of transient execution. Specifically, our agent discovered that observable transient execution can be triggered by masked exceptions which do not require any ucode assists or fault handling. Moreover, the transition between different instruction set extensions cause hardware exceptions to get lost meanwhile causing observable transient execution.


In this work, we proposed an RL framework for discovering \Mi vulnerabilities, demonstrating its capability to efficiently navigate the vast and complex instruction space of modern CPUs. The framework successfully rediscovered known vulnerabilities and uncovered new transient execution mechanisms, such as masked floating-point exceptions and MMX-to-x87 transitions, showcasing its potential as a powerful tool for hardware security research.

The proposed framework is notable for its adaptability across different \Mi, leveraging custom RL environments and performance counter feedback to guide exploration. By employing hierarchical action spaces, parallelized training, and dynamic environment generation, it addresses the scalability challenges inherent in exploring diverse instruction sets and architectures. Additionally, the use of transfer learning reduces the cost of applying the framework to new processor models, making it a versatile solution for vulnerability discovery.

Beyond CPUs, the framework can be extended to analyze other hardware platforms, such as GPUs and accelerators, by adapting the action space and observation metrics to the specific execution models and vulnerabilities of those platforms. The integration of GPU-specific metrics, for instance, could open new avenues for discovering side-channel vulnerabilities in high-performance computing environments.

% Despite its successes, the work leaves several areas for future exploration. Incorporating more diverse system configurations, such as secure enclaves and advanced speculative execution mitigations, could further enhance the framework's robustness. 
% Additionally, refining reward signals and leveraging unsupervised or self-supervised pretraining could improve the framework’s ability to detect subtle vulnerabilities in sparse or noisy reward environments.

Overall, the proposed RL framework represents a significant step forward in automating \Mi vulnerability discovery. By scaling efficiently across vulnerabilities, architectures, and hardware platforms, it lays the groundwork for a more systematic and adaptive approach to hardware security testing, ensuring that modern processors remain resilient against emerging threats.
