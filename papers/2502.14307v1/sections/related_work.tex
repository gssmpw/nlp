\section{Related Works}

\Mi vulnerability discovery has attracted significant attention, leading to the development of several tools and methodologies aimed at exposing speculative execution and side-channel vulnerabilities. 
%
Osiris~\cite{weber2021osiris} introduces a fuzzing-based framework that automates the discovery of timing-based \Mi side channels by using an instruction-sequence triple notation: reset instruction (setting the \Mi component to a known state), a trigger instruction (modifying the state based on secret-dependent operations), and a measurement instruction (extracting the secret by timing differences).
%
Transynther~\cite{moghimi2020medusa, moghimi2023downfall} automates exploring Meltdown-type attacks by synthesizing binarizes based on the known attack patterns. For the classification and root cause analysis of the generated attacks, Transynther uses performance counters and \Mi ``buffer grooming'' technique.
%
AutoCAT~\cite{luo2022autocat} automates the discovery of cache-based side-channel attacks on unknown cache structures using RL.
%
Several studies also focus on using hardware performance counters to detect speculative execution issues. For example,~\cite{ragab2021rage, oleksenko2023hide} use performance counters to monitor mis-speculation behavior. 
%
More recently, \cite{chakraborty2024shesha} proposed a particle swarm optimization based algorithm to discover unknown transient paths. Their main assumption is different instruction sets do not interfere with each other do not share the same resources, therefore, they can be analysed independently. In this dissertation, we show that this assumption limits the exploration of the instruction space and combining different instruction sets can lead to new mechanisms of transient execution.

Although, these tools have shown promise in detecting \Mi vulnerabities, they are limited in their ability to efficiently explore the large instruction space and the complex interactions between different instructions.
