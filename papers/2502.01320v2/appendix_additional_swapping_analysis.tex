\paragraph{Household Size} Household size is an important factor in determining which households are swapped. This is because the inclusion of race counts as flagging variables for determining reidentification risk implies that total household size is also a flagging variable (because total household size is the sum of race counts). Consistent with this, we observe that larger (more uncommon) household sizes are over-represented among the swapped households. For example, 2 person households are the most common household size in Alabama, making up $35.59\%$ of households. However, they make up only $15.69\%$ of the swapped households under a $2\%$ swap rate and $22.12\%$ of the swapped households under a $10\%$ swap rate. In contrast, $3.41\%$ of all households in Alabama are of size 6 or greater, but these households make up $21.85\%$ of the households swapped at a $2\%$ swap rate and $20.39\%$ of the households swapped at a $10\%$ swap rate. The full table of the distribution over household sizes for the state, the swapped households under a $2\%$ swap rate, and the swapped households under a $10\%$ swap rate is shown in Table~\ref{table:hh_dist} in Appendix~\ref{appendix:additional_tables_figures}.

\paragraph{Distribution of People by Race} Although swapping preserves household sizes and voting age counts, it often changes race and ethnicity counts. Therefore, swapping causes changes in the distribution of people by race or ethnicity across the state.
Figure~\ref{fig:map_w_al} illustrates the effects of swapping on white households by county in Alabama. Maps showing white, Black, Asian, and Hispanic population effects in Alabama, Wisconsin, Texas, Nevada, and Vermont are in Figures \ref{fig:al_full_maps}, \ref{fig:wi_full_maps}, \ref{fig:tx_full_maps}, \ref{fig:nv_full_maps}, \ref{fig:vt_full_maps} in Appendix~\ref{appendix:additional_tables_figures}.

We observe that counties that \emph{gain} significant numbers of a racial group are usually counties that have a relatively \emph{small} number of residents in that group before swapping, and counties that \emph{lose} significant numbers of a racial group are usually counties that have a relatively \emph{large} number of residents in that group before swapping. This is consistent with a view in which swapping effectively mixes nearby geographic regions, causing regression to the mean. We caution that this is only a loose rule of thumb and is sometimes violated. Another reason we stop short of finding a general pattern is that some of the maps showing which counties lose and gain people after swapping change noticeably depending on whether we use our standard implementation of swapping or our high variance implementation of swapping (these maps are the only results we generate that sometimes change meaningfully between the two swapping implementations).

Note that, while we observe that counties experiencing significant changes in swapping have particularly high or low populations, it is not the case that any county with a particularly high or low population experiences a significant change after swapping. For example, counties that have a large number of a certain group do not necessarily lose people in that group due to swapping. 

\begin{figure}
    \centering
    \includegraphics[scale=0.22]{AL_figures/w_pop_change_0.1_map.png}\includegraphics[scale=0.22]{AL_figures/w_pct_map.png}\hspace{5mm}  \includegraphics[scale=0.22]{AL_figures/as_pop_change_0.1_map.png}\includegraphics[scale=0.22]{AL_figures/as_pct_map.png}
    \caption{The effect of swapping on white and Asian population by county in Alabama. For reference, the percent white and percent Asian for each county is also shown.}
    \label{fig:map_w_al}
\end{figure}


\paragraph{Which Races are Being Swapped for Which?}
Each swap involves two households: the household that was targeted for swapping (because of its relatively high risk for reidentification) and the household that is selected as the target's partner. In Table \ref{tab:target_distribution}, we show the races of the households that are chosen as swap targets with a 10\% swap rate in Alabama. In Table \ref{tab:partner_distribution}, we show the distribution of the race of the partner household, for a given type of target household. Corresponding tables for a 2\% swap rate in Alabama are Tables \ref{tab:target_distribution_2pct} and \ref{tab:partner_distribution_2pct} in Appendix \ref{appendix:additional_tables_figures}.

We make three main observations:
\begin{enumerate}
    \item Looking at Table \ref{tab:target_distribution}, households with multiple races are strongly over-represented among the swap targets. Also, households with uncommon races are slightly over-represented among the swap targets.
    \item Looking at Table \ref{tab:partner_distribution}, the distribution of partners for any particular type of target is similar to the race distribution among all partner households, which in turn is similar to the overall race distribution across all households. We hypothesize this is because partners are selected largely randomly (except for the optimization for geographic proximity) from households of the same size and voting age count. Ignoring for a moment the slight dependencies between race and household size and age distribution, the observed distribution of partners is to be expected.
    \item After considering the first two effects, we observe households have an elevated likelihood of being swapped with a household with the same race. This is visible in the relatively large numbers on the diagonal of Table \ref{tab:partner_distribution}. This may be because there is geographic clustering of households with the same race. This may also be because there is a correlation between household size and household race. Such a correlation would mean that swap partners, which necessarily have the same size, are more likely to have the same race. Table~\ref{tab:household_size_race_corr} in Appendix~\ref{appendix:additional_tables_figures} illustrates this correlation.
\end{enumerate}

\begin{table}
    \centering
    Distribution of Races of Target Households for a 10\% Swap Rate
    \begin{tabular}{|r|c|c|c|c|c|c|c|c|}
    \hline
                  & W    & B    & AI/AN & AS  & H/PI & OTH  & 2+  & Multiple Races \\ \hline
    \% overall    & 70.8 & 23.9 & 0.3   & 0.7 & 0.0  & 1.1  & 0.6 & 2.5            \\ \hline
    \% of targets & 21.2 & 10.4 & 4.3   & 5.4 & 0.6  & 10.3 & 8.8 & 39.0           \\ \hline
    \end{tabular}

    \caption{This table shows the distribution of the race among targeted households, compared to the distribution of the race of all households.}
    \label{tab:target_distribution}
\end{table}

\begin{table}

\setlength\doublerulesep{5mm} 
    \centering
    Partner Distribution for a 10\% Swap Rate
    \begin{tabular}{|r|c|c|c|c|c|c|c|c|}
    \hline
    \diagbox[width=\dimexpr \textwidth/8+5\tabcolsep\relax, height=1cm]{Target}{Partner}
                   & \multicolumn{1}{c|}{W} & \multicolumn{1}{c|}{B} & \multicolumn{1}{c|}{AI/AN} & \multicolumn{1}{c|}{AS} & \multicolumn{1}{c|}{H/PI} & \multicolumn{1}{c|}{OTH} & \multicolumn{1}{c|}{2+} & \multicolumn{1}{c|}{Multiple Races} \\ \hline
    W              & 67.6                 & 27.1                 & 0.2                      & 1.2                   & 0.1                     & 3.1                    & 0.6                   & 5.5                               \\ \hline
    B              & 49.6                 & 46.2                 & 0.2                      & 1.0                   & 0.1                     & 2.3                    & 0.6                   & 6.6                               \\ \hline
    AI/AN          & 77.0                 & 19.9                 & 0.6                      & 0.5                   & 0.0                     & 1.1                    & 0.9                   & 2.0                               \\ \hline
    AS             & 69.2                 & 23.1                 & 0.1                      & 5.5                   & 0.0                     & 1.5                    & 0.6                   & 3.2                               \\ \hline
    H/PI           & 70.7                 & 24.9                 & 0.7                      & 0.5                   & 0.3                     & 2.8                    & 0.0                   & 1.4                               \\ \hline
    OTH            & 64.4                 & 26.1                 & 0.3                      & 0.7                   & 0.0                     & 7.8                    & 0.6                   & 2.9                               \\ \hline
    2+             & 71.3                 & 25.6                 & 0.5                      & 0.7                   & 0.0                     & 1.0                    & 0.8                   & 2.0                               \\ \hline
    Multiple Races & 70.3                 & 26.1                 & 0.3                      & 1.0                   & 0.0                     & 1.8                    & 0.5                   & 5.3                               \\ \hline\hline
    \% overall     & 70.8                 & 23.9                 & 0.3                      & 0.7                   & 0.0                     & 1.1                    & 0.6                   & 2.5                               \\ \hline
    \% of partners & 64.3                 & 26.6                 & 0.3                      & 1.1                   & 0.0                     & 2.5                    & 0.6                   & 4.4                               \\ \hline
    \end{tabular}
    \caption{For a swap target of a particular race, this table shows the distribution of the race of the partner household. The row labels are the race of the target household. Each row shows the distribution of the race of the partner household for that particular type of target household (each row is normalized to sum to 100). For reference, below is distribution of the household races among all households and among the households selected as partners.}
    \label{tab:partner_distribution}
\end{table}