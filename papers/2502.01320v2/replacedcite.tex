\section{Related Work}
\label{sec:related}
\paragraph*{Disclosure Avoidance in the US Decennial Census.}

The US Census Bureau has a statutory obligation to protect the privacy of respondents under Title 13 of the U.S. Code.
To do so, they have deployed a number of Disclosure Avoidance Systems (DASs) over the years ____.
Of note, for the purposes of this work, are two methods in particular: data swapping, used in 1990, 2000, and 2010; and TopDown, the Bureau's implementation of differential privacy, used in 2020.

Data swapping, introduced by ____, works by exchanging the values contained in particular cells in a dataset while leaving aggregate statistics unchanged.
For a survey on variants of swapping over the years, see ____.
At a high level, the swapping algorithm used by the Census Bureau swapped the records associated with one household with those of another household in another location.
While the exact details of this procedure are unknown, we compile a combination of reports and statements issued by the Census Bureau and related personnel ____ that collectively provide enough information to implement a close approximation to the original Census procedure.
We provide further details in \Cref{sec:swapping}.

Beginning in 2020, the Census Bureau began using a new DAS based on differential privacy ____.
Instead of swapping records, the Bureau's algorithm (known as TopDown) adds noise to tabulated counts, providing formal privacy guarantees, and then, in a {\em post-processing} phase, solves a constrained optimization problem to find ``nearby'' noisy counts satisfying hierarchical consistency and other requirements (\textit{e.g.}, non-negativity) ____. 
Unlike swapping, the algorithm itself is public knowledge; the realized values of the added noise are kept secret to protect privacy.

Our focus in this work is on the Decennial Census.
Disclosure avoidance is also used in the American Community Survey (ACS), which collects more detailed information on a sample of the population (see ____).
The Census Bureau uses a variety of disclosure avoidance techniques to release
ACS data, including swapping ____, ____.

\paragraph*{Analyzing Census DASs.}

Research on the effects of swapping in the Census, especially in comparison with differential privacy, is closely related to our work ____.
____ implements a version of swapping for the 2007--2011 American Community Survey in Alleghany County, PA.
____ provides results using both synthetic data at the tract level and privileged access to the true (unswapped) microdata through the Census Bureau.
In contrast, our methods make analyses like this possible \emph{without} privileged access, which may benefit future researchers, and they are sufficiently general to be used across the country.
____ compare the effects of swapping and differential privacy on both utility and privacy.
To do so, they create synthetic data by combining blocks from nine counties across the country from the 2010 Census.
They provide two plausible implementations of swapping which, like ____, operate by swapping individual records with an external pool of candidates that is representative of the US population.
In contrast, our swapping implementation operates on \textit{households}, as opposed to \textit{individuals}, which is consistent with practices in the 1990--2010 Decennial Censuses.

____ give a swapping algorithm that provides a differential privacy guarantee and analyze it on Census data.
In contrast, the swapping procedure used by the Census, which we aim to reconstruct here, has no formal differential privacy guarantees.
Because swapping requires access to household-level data at the block level, ____ evaluate their approach on the 1940 Census, for which the original microdata have been made public.
Our approach instead generates synthetic microdata for the 2010 Census ____, to which we then apply our swapping procedure.


While it is still fairly new, the TopDown algorithm and its close variants have been the subject of a number of studies ____.
Both ____ and ____ consider how TopDown will impact analyses related to voting rights.
Because the original TopDown algorithm is difficult to run reliably, we adopt a simpler variant (``ToyDown'') introduced by ____, similar to the ``Reasonably Good Algorithm'' of ____.
____ analyze the reliability of TopDown outputs for redistricting applications.
____ characterize the amount of noise introduced by the overall TopDown process.
We build upon some of their techniques to calibrate a key parameter of our algorithm against released TopDown data.
Beyond the uncertainty introduced by the DAS, ____ and ____ detail how other sources of error lead to inaccurate data and downstream policy impacts.

\paragraph*{Synthetic data generation.}

Our swapping implementation requires access to household-level microdata, which are in general only made public by the Census Bureau after 72 years.
Instead, we rely on \textit{synthetic} microdata for the 2010 Census, derived from a combination of Census data products using techniques from ____.
Synthetic data are not perfect: while they match tabular data released by the Census, they will differ in distribution and locations of households.
Moreover, disclosure avoidance techniques have already been used prior to the release of the data products on which they are based, meaning that when we run our swapping implementation, it will effectively be on \textit{already-swapped} data.
Still, prior work has had to rely on a variety of heuristics to obtain household-level datasets ____, and the implementation provided by ____ provides a principled way to sample synthetic data.
We provide further details on this approach in \Cref{fig:databases}.