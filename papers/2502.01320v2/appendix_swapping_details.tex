Beyond the main swapping implementation choices described in Section \ref{sec:swapping}, there are a few more minor choices we must make. Among these choices, we conduct a robustness check by creating two implementations of swapping, one standard implementation and one ``high-variance'' implementation.
\begin{itemize}
    \item To pick a swap partner, we do not deterministically pick the geographically closest potential swap partner; instead, we take the $k$ closest potential swap partners and randomly chose one. In our standard implementation of swapping, $k=10$. In our high variance swapping implementation, $k=100$.
    \item We know that the Census Bureau assigns each household to a tier from 1 (least risk) to 4 (greatest risk), where tier 4 households are swapped with probability 1 \citep{steel-zayatz}. To assign these tiers, we sort the households by reidentification risk (using the method above to determine reidentification risk) and split the list into tiers. We set the distribution of tier assignments such that (1) there are twice as many tier 3 households as tier 4 households, and three times as many tier 2 households as tier 4 households and (2) in expectation, we will swap all of the tier 4 households and half of the tier 3 households by the time we make the desired number of swaps (assuming the standard swap probabilities described below). For reference, using a $10\%$ swap rate, this means that $6.25\%$ of the households are in tier 4, $12.5\%$ are in tier 3, $18.75\%$ are in tier 2, and $62.5\%$ are in tier 1.\footnote{Since we iterate through households in order of their tier, households in tier 1 or 2 are very unlikely to be \emph{targeted} to be swapped, but still have a chance of being involved in a swap as the swap \emph{partner}.}
    \item The decision to swap a household is randomized, where the probability of a house being swapped depends on its risk tier. Concretely, we sort the households by tier from higher tier to lower tier, randomly shuffling the households within each tier. Then, we iterate through the households. Tier 4 households are swapped with probability 1 (following \citet{steel-zayatz}); tier 3 households are swapping with probability $p_3$, tier 2 households with probability $p_2$, and tier 1 households with probability $p_1$. We continue swapping until we have swapped a total number of households equal to the swap rate multiplied by the number of households in the state.
    Our standard implementation uses $p_3 = 0.6$, $p_2 = 0.3$, and $p_1 = 0.1$. Our high variance swapping implementation uses $p_3'=0.3$, $p_2'=p_2$, and $p_1'=p_1$. This change increases the expected number of households considered as swap targets before the swap rate is exhausted.
\end{itemize}