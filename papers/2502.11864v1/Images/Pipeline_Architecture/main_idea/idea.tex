\begin{figure}[t]
    \centering
    \begin{subfigure}[t]{0.2\columnwidth}
    \centering
        \includegraphics[width=\textwidth]{Images/Pipeline_Architecture/main_idea/rgbfront.jpg}
        \caption{Front\\view}
    \end{subfigure}
    \begin{subfigure}[t]{0.2\columnwidth}
        \centering
        \includegraphics[width=\textwidth,trim= 65 0 65 60,clip]{Images/Pipeline_Architecture/main_idea/rgb.jpg}
        \caption{Bird's eye view (BEV)}
    \end{subfigure}
    \begin{subfigure}[t]{0.2\columnwidth}
        \centering
        \includegraphics[width=\textwidth, trim= 65 0 65 60, clip]{Images/Pipeline_Architecture/main_idea/semantic_cs.jpg}
        \caption{Correct\\semantic segmentation BEV}
    \end{subfigure}
    \begin{subfigure}[t]{0.2\columnwidth}
        \centering
        \includegraphics[width=\textwidth, trim= 65 0 65 60,clip]{Images/Pipeline_Architecture/main_idea/semantic_cs_perturbed.jpg}
        \caption{Perturbed semantic segmentation BEV}
    \end{subfigure}
    \caption{An illustration of the main idea of our experiments in the CARLA driving simulator. An agent perceives its environment through a semantic segmentation mask in a bird's eye view, which we perturb in a controlled manner. This corresponds to perfectly quantifiable perceptual uncertainties, which we can provide to the agent. In our study, we investigate whether the agent benefits from this uncertainty information.}
    \label{fig:page1-fig}
\end{figure}