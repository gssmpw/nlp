\begin{figure*}[t]
    % \centering
    \hfill
    \begin{subfigure}{0.35\textwidth}
    \includegraphics[width=\textwidth]{images/diff_model_family.pdf}
    \caption{Different LLMs}\label{fig:eval_model_family}
    \end{subfigure}
    \hfill
    \begin{subfigure}{0.34\textwidth}
    \includegraphics[width=\textwidth]{images/diff_model_size.pdf}
    \caption{Different model sizes}\label{fig:eval_model_size}
    \end{subfigure}
    \hfill
    \begin{subfigure}{0.29\textwidth}
    \includegraphics[width=\textwidth]{images/diff_hallucination_rate.pdf}
    \caption{Different hallucination rates}\label{fig:eval_hallucination_rate}
    \end{subfigure}
    \hfill
     \vspace{-.5pc}
    \caption{
    For each method, we show performance variation across 17 LLMs (\ref{fig:eval_model_family}). We also show AUPRC scores across LLMs with different capacities (\ref{fig:eval_model_size}), as well as performance on data with different hallucination rates (\ref{fig:eval_hallucination_rate}). Note that the model used in Figure~\ref{fig:eval_hallucination_rate} is Llama3-8B.
    }
    \label{fig:experimental_result}
    \vspace{-1pc}
\end{figure*}