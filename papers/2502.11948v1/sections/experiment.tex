\subsection{Experimental Setup}




\paragraph{Models.}

To understand the impact of model family and capacity on entity-level hallucination detection, we experiment with 17 diverse LLMs, including \textbf{Llama3}-\{8B, 70B\}~\citep{grattafiori2024llama3herdmodels}, \textbf{Llama3.1}-8B, \textbf{Llama3.2}-3B, \textbf{Aquila2}-\{7B, 34B\}~\citep{zhang2024aquila2technicalreport}, \textbf{InternLM2}-\{7B, 20B\}~\citep{cai2024internlm2technicalreport}, \textbf{Qwen2.5}-\{7B, 32B\}~\citep{qwen2025qwen25technicalreport}, \textbf{Yi}-\{9B, 34B\}~\citep{ai2024yiopenfoundationmodels}, \textbf{phi-2}~\citep{gunasekar2023textbooksneed}, \textbf{Mistral}-7B~\citep{jiang2023mistral7b}, \textbf{Mixtral}-8x22B~\citep{jiang2024mixtralexperts}, and \textbf{Gemma2}-\{9B, 27B\}~\citep{gemmateam2024gemma2improvingopen}.


\section{Experimental Results}

\begin{figure}
    \center
    \includegraphics[width=1.0 \linewidth]{fig/research_question1.pdf}
    \caption{Visualization of \mname{} and other methods.}
    \label{experiment}
\end{figure}

\subsection{RQ 1: Does \mname{} Generate Faithful Explanations?}

We evaluate the generated explanations by \mname{} based on factual and counterfactual reasoning, as presented in Table~\ref{table:main}. \mname{} achieves the best performance across the metrics $Fid_{F},$ $Fid_{CF}$, $KL_{F}$, and $KL_{CF}$, while also maintaining the smallest size ($Size$), demonstrating that our explanations are both simple and effective. The baseline, denoted as $N_{audio}=5$, generates audio conditioned on the same textual input five times to observe the inherent variance, serving as the lower bound for $Fid_{F}$, $KL_{F}$. \mname{}'s factual audio nearly reaches the lower bound, indicating high performance. Furthermore, significant changes in $Fid_{CF}$ and $KL_{CF}$ under counterfactual perturbations confirm that the explanations are both sufficient and necessary. The \mname{} with factual and counterfactual losses in Eq.\eqref{total}, outperforms the variants \mname \ w/ Eq.~\eqref{fact} and \mname \ w/ Eq.~\eqref{cfact}, which apply only factual or counterfactual loss with a regularization term. This indicates that the two losses complement each other, enhancing overall performance. Furthermore, we evaluate \mname \ w/ Eq.~\eqref{audioexpl} using an averaged explanation mask, showing the robustness of explainability in describing the entire audio. In contrast, other baselines fail to generate meaningful counterfactual audio, lacking the optimization properties needed to enforce counterfactual explanations. 

The strong performance highlights the effectiveness of leveraging latent embedding vectors to generate explanations. While most baselines are designed to explain supervised learning models, they rely on vectors that represent the probability distribution of the final audio token. This approach, however, does not align well with the inference process of audio generation models. In extreme cases, such as top-$k$ sampling ($k$=250), the 250-th audio token could be sampled, leading to significant discrepancies between the gradients or probability-related information the token most likely predicted by the model. In contrast, our approach avoids dependency on the sampling process, allowing the model to produce more faithful explanations.

\subsection{RQ 2: How Well Do the Explanations from \mname{} Reflect the Generated Audio?}

We visualize the explanations generated by \mname{} and other baselines, as shown in Figure~\ref{experiment}. \mname{} demonstrates a clear advantage in focusing on key audio elements. Unlike other baselines, which often assign relatively high importance scores to less important tokens like `A' and `with', \mname{} consistently assigns higher importance scores to crucial tokens such as `ticktocks' and `music'. For instance, \mname{} assigns a notably high importance score of 0.96 to `music', emphasizing its ability to focus on significant input tokens. In contrast, other models like $\text{Grad-CAM-e}$ and AtMan distribute importance more broadly, including less relevant tokens. These results show that \mname{} consistently provides faithful explanations, aligning the generated audio with the essential components of the input text.

\begin{figure}
    \center
    \includegraphics[width=1.0 \linewidth]{fig/research_question2-2.pdf}
    \caption{Explanation generated by \mname{} for two audios created from a single prompt. (a) includes bird sounds, while (b) does not.}
    \label{rq2}
\end{figure}

Furthermore, when generating audio from a prompt containing multiple concepts, some words may be less prominently reflected. In such case, \mname{} provides adequate explanations for each specific audio, indicating whether each word from the prompt has been incorporated into the generated audio. As illustrated in Figure~\ref{rq2}, the difference between the two audios is that bird sounds are present in Figure~\ref{rq2}-(a) but absent in Figure~\ref{rq2}-(b).
\mname{} effectively describes the audios by assigning high importance scores of 0.98 and 0.99 to the token `Water,' which is the primary sound in both audios. \mname{} assigns a score of 0.54 to `birds,' while it assigns a score of 0.14, accurately reflecting the different audio characteristics in each case. These results show that \mname{} can provide explanations that are well-suited to the corresponding audio. Furthermore, these explanations serve as valuable insights for editing generated audio to better align with user intention. 

\subsection{RQ 3: How Can Explanations Help Understand AudioGen Behavior?} 

We explore the output patterns of AudioGen using the explanations generated by \mname{}. First, we investigate whether AudioGen can effectively handle sentences containing negations and double negations, as shown in Figure~\ref{rq3}. The explanations of the generated audios are presented in response to input prompts containing `without thunder' and `without no thunder.' In both cases, the generated audio includes the sound of thunder along with the rain. Using \mname{}, we observe that `without' and `without no' have lower importance compared to 'thunder' in the explanations. We hypothesize that this occurs because the training dataset lacks sufficient examples of negation and double negation. An examination of the AudioCaps dataset reveals a scarcity of such cases. Additionally, by aggregating tokens from the explanations, we identify the top and bottom 50 tokens in Table~\ref{tabler3} in the Appendix. Tokens with high importance are predominantly nouns, such as `thunder,' while those with low importance include sound descriptors like `distant,' as well as sequential expressions like `before.' Such analyses could be used to debug TAG models or to identify potential inherent biases in their behavior. 

\begin{figure}
    \center
    \includegraphics[width=1.0 \linewidth]{fig/research_question3-1.pdf}
    \caption{Explanations generated from negated prompts: (a) single negation, (b) double negation.}
    \label{rq3}
\end{figure}

\subsection{RQ 4: Does \mname{} Generate Explanations Efficiently?}

\begin{table}[!ht]
\centering
\begin{tabular}{lrr}
\toprule
Method& Memory (MB)$\downarrow$& Time (s)$\downarrow$ \\
\midrule
    $\text{Grad-CAM-e}$ & 8641.306 & 49.038  \\
    $\text{Grad-CAM-a}$ & 41655.848 & 62.276  \\
    AtMan & \textbf{5081.957} & \textbf{7.295}  \\
    Chefer et al. & 41684.969 & 52.166 \\ \hline
    \mname \ w/ Eq.~\eqref{fact} & 11980.894 & 36.639 \\
    \mname \ w/ Eq.~\eqref{cfact} & 11981.114 & 37.373 \\
    \mname \ w/ Eq.~\eqref{audioexpl} & 12001.931 & 63.198 \\
    \mname & 12001.931 & 63.198 \\
\bottomrule
\end{tabular}
\caption{Efficiency analysis of \mname{} and other baseline methods. The best results are highlighted in \textbf{bold}.}
\label{table:eff}
\end{table}

We evaluate the efficiency of explanation methods based on the average time and total GPU memory usage per explanation, as shown in Table~\ref{table:eff}. For GPU memory efficiency, the results rank in the following order: AtMan, $\text{Grad-CAM-e}$, \mname{}, $\text{Grad-CAM-a}$, and Chefer et al. For time efficiency, the order is AtMan, $\text{Grad-CAM-e}$, Chefer et al., $\text{Grad-CAM-a}$, and \mname{}. Although AtMan is the most efficient, its performance remains subpar due to its simplistic approach. $\text{Grad-CAM-e}$ demonstrates greater memory efficiency compared to $\text{Grad-CAM-a}$ and Chefer et al., as it tracks a shallower layer. While \mname{} requires additional computational time to train explanation masks, it achieves memory efficiency by reducing GPU storage and operates with $\mathcal{O}(Lk)$ complexity, ensuring linear scalability for large-scale tasks.

\paragraph{Evaluation Metrics.}

Entity-level hallucination detection can be formulated as a binary classification task. To evaluate performance, we use (1) \textbf{AUPRC} and (2) \textbf{AUROC}, which assess the relationship between entity-level hallucination labels $l$ and scores $y^e$. AUPRC captures precision-recall trade-offs, while AUROC evaluates true and false positive rates. Unlike AUROC, AUPRC disregards true negatives, emphasizing false positive reduction---a key advantage for hallucination detection, where true negatives often involve less informative entities like prepositions and conjunctions. 
We complement these metrics by also reporting (3) $\mathbf{F1}_\mathbf{Opt}$, (4) $\mathbf{Precision}_\mathbf{Opt}$, and (5) $\mathbf{Recall}_\mathbf{Opt}$, where $\mathrm{F1}_\mathrm{Opt}$ is the optimal F1 score among all possible threshold and $\mathrm{Precision}_\mathrm{Opt}$, and $\mathrm{Recall}_\mathrm{Opt}$ are corresponding Precision and Recall values.


\subsection{Experimental Results}\label{sec:result}



\paragraph{How do different uncertainty scores perform on entity-level hallucination detection?}

Table~\ref{tb:evaluation} presents the evaluation results for five uncertainty-based hallucination detection approaches using Llama3-8B. 
Likelihood, Entropy, and CCP exhibit low $\mathrm{Precision}_\mathrm{Opt}$ ($\approx$ overall hallucination rate) but achieve high $\mathrm{Recall}_\mathrm{Opt}$. This pattern suggests these methods over-predict hallucinations, making them less suitable for reliable detection. Their focus on individual token probabilities rather than contextual roles likely contributes to this limitation, indicating that \emph{hallucination detection is inherently context-dependent and requires uncertainty scores calibrated with contextual information.} 

SAR and Focus, which incorporate context information, show better overall performance. However, their lower $\mathrm{Recall}_\mathrm{Opt}$ indicates that the current methods for modeling context remain suboptimal, failing to capture some hallucinated content. These findings highlight the challenges in entity-level hallucination detection and the need for more advanced approaches that better integrate contextual information while achieving a balanced trade-off between precision and recall. 



\paragraph{How do different LLM families and capacity impact performance?}

Figure~\ref{fig:eval_model_family} summarizes the performance variation across all 17 different LLMs. The results indicate that while AUROC, AUPRC, and $\mathrm{F1}_\mathrm{Opt}$ scores vary across model families, \emph{the method used to compute uncertainty scores has a more significant impact on performance}. Notably, Focus consistently achieves the highest performance across all model families. Further evaluation details can be found in Appendix~\ref{ap:experiment}.

\begin{table*}
    \scriptsize
    \centering
    \begin{NiceTabular}{@{}l@{\hskip4pt}p{0.92\textwidth}@{}}
    \CodeBefore
\cellcolor{gray!20}{1-1,1-2}
\cellcolor{gray!20}{5-1,5-2}
\cellcolor{gray!20}{9-1,9-2}
\Body
    \toprule
    \multicolumn{2}{c}{\textbf{Case 1: Under-prediction of SAR}}\\
    \midrule
    \textbf{Groundtruth} & \texttt{
    [...] 
Diaz
started
his
{\color{red}political career}
as a
{\color{red}member}
of
{\color{red}the Sangguniang Bayan}
{\color{red}(municipal council)}
of
{\color{red}Santa Cruz}
in
{\color{red}1978}.
He
later
became
{\color{red}the Vice Mayor}
of
{\color{red}Santa Cruz}
in
{\color{red}1980}
and
was elected
as
{\color{red}the town's Mayor}
in
{\color{red}1988}.
[...]
    }\\
    \hdashline
    \textbf{Likelihood} & \texttt{
    [...]
\adjustbox{bgcolor={red!61.3}}{\strut Diaz}
\adjustbox{bgcolor={red!95.4}}{\strut started}
\adjustbox{bgcolor={red!0.4}}{\strut his}
\adjustbox{bgcolor={red!7.2}}{\strut political career}
\adjustbox{bgcolor={red!35.3}}{\strut as a}
\adjustbox{bgcolor={red!67.7}}{\strut member}
\adjustbox{bgcolor={red!0.6}}{\strut of}
\adjustbox{bgcolor={red!21.2}}{\strut the Sangguniang Bayan}
\adjustbox{bgcolor={red!24.8}}{\strut (municipal council)}
\adjustbox{bgcolor={red!5.2}}{\strut of}
\adjustbox{bgcolor={red!23.9}}{\strut Santa Cruz}
\adjustbox{bgcolor={red!71.2}}{\strut in}
\adjustbox{bgcolor={red!81.2}}{\strut 1978}
\adjustbox{bgcolor={red!28.5}}{\strut .}
\adjustbox{bgcolor={red!15.2}}{\strut He}
\adjustbox{bgcolor={red!90.6}}{\strut later}
\adjustbox{bgcolor={red!89.7}}{\strut became}
\adjustbox{bgcolor={red!48.9}}{\strut the Vice Mayor}
\adjustbox{bgcolor={red!6.6}}{\strut of}
\adjustbox{bgcolor={red!14.3}}{\strut Santa Cruz}
\adjustbox{bgcolor={red!62.6}}{\strut in}
\adjustbox{bgcolor={red!45.6}}{\strut 1980}
\adjustbox{bgcolor={red!31.4}}{\strut and}
\adjustbox{bgcolor={red!61.7}}{\strut was elected}
\adjustbox{bgcolor={red!42.9}}{\strut as}
\adjustbox{bgcolor={red!61.5}}{\strut the town's Mayor}
\adjustbox{bgcolor={red!2.0}}{\strut in}
\adjustbox{bgcolor={red!37.2}}{\strut 1988}
\adjustbox{bgcolor={red!99.5}}{\strut .}
[...]
    }\\
    \hdashline
    \textbf{SAR} & \texttt{
    [...]
\adjustbox{bgcolor={red!64.0}}{\strut Diaz}
\adjustbox{bgcolor={red!31.4}}{\strut started}
\adjustbox{bgcolor={red!0.1}}{\strut his}
\adjustbox{bgcolor={red!2.4}}{\strut political career}
\adjustbox{bgcolor={red!6.3}}{\strut as a}
\adjustbox{bgcolor={red!16.0}}{\strut member}
\adjustbox{bgcolor={red!0.1}}{\strut of}
\adjustbox{bgcolor={red!13.8}}{\strut the Sangguniang Bayan}
\adjustbox{bgcolor={red!6.7}}{\strut (municipal council)}
\adjustbox{bgcolor={red!0.6}}{\strut of}
\adjustbox{bgcolor={red!7.0}}{\strut Santa Cruz}
\adjustbox{bgcolor={red!23.7}}{\strut in}
\adjustbox{bgcolor={red!74.5}}{\strut 1978}
\adjustbox{bgcolor={red!3.6}}{\strut .}
\adjustbox{bgcolor={red!8.3}}{\strut He}
\adjustbox{bgcolor={red!53.3}}{\strut later}
\adjustbox{bgcolor={red!35.6}}{\strut became}
\adjustbox{bgcolor={red!28.4}}{\strut the Vice Mayor}
\adjustbox{bgcolor={red!1.3}}{\strut of}
\adjustbox{bgcolor={red!4.1}}{\strut Santa Cruz}
\adjustbox{bgcolor={red!19.8}}{\strut in}
\adjustbox{bgcolor={red!38.4}}{\strut 1980}
\adjustbox{bgcolor={red!7.1}}{\strut and}
\adjustbox{bgcolor={red!23.8}}{\strut was elected}
\adjustbox{bgcolor={red!14.5}}{\strut as}
\adjustbox{bgcolor={red!23.2}}{\strut the town's Mayor}
\adjustbox{bgcolor={red!0.4}}{\strut in}
\adjustbox{bgcolor={red!29.5}}{\strut 1988}
\adjustbox{bgcolor={red!53.6}}{\strut .}
[...]
    }\\
\midrule
\multicolumn{2}{c}{\textbf{Case 2: The type-filter of Focus and the limitations of uncertainty scores}}\\
    \midrule
\textbf{Groundtruth} & \texttt{
Taral Hicks
is
an
American
actress
and
singer,
born
on
September 21, 1974,
in
{\color{red}The Bronx, New York}.
[...]
She
later
{\color{red}transitioned}
to
{\color{red}acting},
appearing in
films
such as
"A Bronx Tale"
(1993),
"Just Cause"
(1995),
and
"Belly"
(1998).
[...]
}\\
\hdashline
\textbf{Likelihood} & \texttt{
\adjustbox{bgcolor={red!27.5}}{\strut Taral Hicks}
\adjustbox{bgcolor={red!64.8}}{\strut is}
\adjustbox{bgcolor={red!43.8}}{\strut an}
\adjustbox{bgcolor={red!25.7}}{\strut American}
\adjustbox{bgcolor={red!98.8}}{\strut actress}
\adjustbox{bgcolor={red!49.1}}{\strut and}
\adjustbox{bgcolor={red!26.2}}{\strut singer}
\adjustbox{bgcolor={red!91.9}}{\strut ,}
\adjustbox{bgcolor={red!76.3}}{\strut born}
\adjustbox{bgcolor={red!56.8}}{\strut on}
\adjustbox{bgcolor={red!34.6}}{\strut September 21, 1974}
\adjustbox{bgcolor={red!71.4}}{\strut ,}
\adjustbox{bgcolor={red!9.1}}{\strut in}
\adjustbox{bgcolor={red!21.4}}{\strut The Bronx, New York}
\adjustbox{bgcolor={red!65.0}}{\strut .}
[...]
\adjustbox{bgcolor={red!75.1}}{\strut She}
\adjustbox{bgcolor={red!76.5}}{\strut later}
\adjustbox{bgcolor={red!13.8}}{\strut transitioned}
\adjustbox{bgcolor={red!86.7}}{\strut to}
\adjustbox{bgcolor={red!1.1}}{\strut acting}
\adjustbox{bgcolor={red!15.8}}{\strut ,}
\adjustbox{bgcolor={red!8.7}}{\strut appearing in}
\adjustbox{bgcolor={red!45.0}}{\strut films}
\adjustbox{bgcolor={red!10.6}}{\strut such as}
\adjustbox{bgcolor={red!48.7}}{\strut "A Bronx Tale"}
\adjustbox{bgcolor={red!18.1}}{\strut (1993),}
\adjustbox{bgcolor={red!47.3}}{\strut "Just Cause"}
\adjustbox{bgcolor={red!2.6}}{\strut (1995),}
\adjustbox{bgcolor={red!4.8}}{\strut and}
\adjustbox{bgcolor={red!46.6}}{\strut "Belly"}
\adjustbox{bgcolor={red!1.7}}{\strut (1998).}
[...]
}\\
\hdashline
\textbf{Focus} & \texttt{
\adjustbox{bgcolor={red!5.3}}{\strut Taral Hicks}
\adjustbox{bgcolor={red!0.0}}{\strut is}
\adjustbox{bgcolor={red!0.0}}{\strut an}
\adjustbox{bgcolor={red!23.5}}{\strut American}
\adjustbox{bgcolor={red!40.9}}{\strut actress}
\adjustbox{bgcolor={red!0.0}}{\strut and}
\adjustbox{bgcolor={red!30.4}}{\strut singer}
\adjustbox{bgcolor={red!0.0}}{\strut ,}
\adjustbox{bgcolor={red!0.0}}{\strut born}
\adjustbox{bgcolor={red!0.0}}{\strut on}
\adjustbox{bgcolor={red!31.9}}{\strut September 21, 1974}
\adjustbox{bgcolor={red!0.0}}{\strut ,}
\adjustbox{bgcolor={red!0.0}}{\strut in}
\adjustbox{bgcolor={red!21.0}}{\strut The Bronx, New York}
\adjustbox{bgcolor={red!0.0}}{\strut .}
[...]
\adjustbox{bgcolor={red!0.0}}{\strut She}
\adjustbox{bgcolor={red!0.0}}{\strut later}
\adjustbox{bgcolor={red!0.0}}{\strut transitioned}
\adjustbox{bgcolor={red!0.0}}{\strut to}
\adjustbox{bgcolor={red!32.2}}{\strut acting}
\adjustbox{bgcolor={red!0.0}}{\strut ,}
\adjustbox{bgcolor={red!0.0}}{\strut appearing in}
\adjustbox{bgcolor={red!33.2}}{\strut films}
\adjustbox{bgcolor={red!0.0}}{\strut such as}
\adjustbox{bgcolor={red!43.6}}{\strut "A Bronx Tale"}
\adjustbox{bgcolor={red!25.3}}{\strut (1993),}
\adjustbox{bgcolor={red!42.5}}{\strut "Just Cause"}
\adjustbox{bgcolor={red!30.1}}{\strut (1995),}
\adjustbox{bgcolor={red!0.0}}{\strut and}
\adjustbox{bgcolor={red!44.2}}{\strut "Belly"}
\adjustbox{bgcolor={red!30.1}}{\strut (1998).}
[...]
}\\
\midrule
\multicolumn{2}{c}{\textbf{Case 3: Uncertainty propagation of Focus}}\\
    \midrule
    \textbf{Groundtruth} & \texttt{
    [...]
Fernandinho
began
his
professional career
with
{\color{red}Atletico Paranaense}
in
{\color{red}Brazil}
before
{\color{red}moving}
to
{\color{red}Ukrainian club}
{\color{red}Shakhtar Donetsk}
in
{\color{red}2005}.
[...]
He
is known
for
his
{\color{red}physicality},
{\color{red}tackling ability},
and
{\color{red}passing range},
and
is
{\color{red}widely regarded}
as
{\color{red}one of the best}
{\color{red}defensive midfielders}
in
{\color{red}the world}.
    }\\
    \hdashline
    \textbf{Likelihood} & \texttt{
    [...]
\adjustbox{bgcolor={red!23.6}}{\strut Fernandinho}
\adjustbox{bgcolor={red!78.3}}{\strut began}
\adjustbox{bgcolor={red!3.3}}{\strut his}
\adjustbox{bgcolor={red!47.7}}{\strut professional career}
\adjustbox{bgcolor={red!60.5}}{\strut with}
\adjustbox{bgcolor={red!25.3}}{\strut Atletico Paranaense}
\adjustbox{bgcolor={red!42.1}}{\strut in}
\adjustbox{bgcolor={red!97.7}}{\strut Brazil}
\adjustbox{bgcolor={red!91.5}}{\strut before}
\adjustbox{bgcolor={red!40.5}}{\strut moving}
\adjustbox{bgcolor={red!9.1}}{\strut to}
\adjustbox{bgcolor={red!65.8}}{\strut Ukrainian club}
\adjustbox{bgcolor={red!3.2}}{\strut Shakhtar Donetsk}
\adjustbox{bgcolor={red!20.1}}{\strut in}
\adjustbox{bgcolor={red!1.5}}{\strut 2005}
\adjustbox{bgcolor={red!19.5}}{\strut .}
[...]
\adjustbox{bgcolor={red!34.9}}{\strut He}
\adjustbox{bgcolor={red!28.2}}{\strut is known}
\adjustbox{bgcolor={red!0.0}}{\strut for}
\adjustbox{bgcolor={red!0.0}}{\strut his}
\adjustbox{bgcolor={red!61.0}}{\strut physicality}
\adjustbox{bgcolor={red!8.6}}{\strut ,}
\adjustbox{bgcolor={red!77.0}}{\strut tackling ability}
\adjustbox{bgcolor={red!0.6}}{\strut ,}
\adjustbox{bgcolor={red!12.7}}{\strut and}
\adjustbox{bgcolor={red!53.7}}{\strut passing range}
\adjustbox{bgcolor={red!22.5}}{\strut ,}
\adjustbox{bgcolor={red!34.9}}{\strut and}
\adjustbox{bgcolor={red!61.7}}{\strut is}
\adjustbox{bgcolor={red!50.0}}{\strut widely regarded}
\adjustbox{bgcolor={red!0.0}}{\strut as}
\adjustbox{bgcolor={red!0.8}}{\strut one of the best}
\adjustbox{bgcolor={red!3.1}}{\strut defensive midfielders}
\adjustbox{bgcolor={red!2.0}}{\strut in}
\adjustbox{bgcolor={red!1.5}}{\strut the world}
\adjustbox{bgcolor={red!27.0}}{\strut .}
    }\\
    \hdashline
    \textbf{Focus} & \texttt{
    [...]
\adjustbox{bgcolor={red!28.8}}{\strut Fernandinho}
\adjustbox{bgcolor={red!0.0}}{\strut began}
\adjustbox{bgcolor={red!0.0}}{\strut his}
\adjustbox{bgcolor={red!16.9}}{\strut professional career}
\adjustbox{bgcolor={red!0.0}}{\strut with}
\adjustbox{bgcolor={red!36.5}}{\strut Atletico Paranaense}
\adjustbox{bgcolor={red!0.0}}{\strut in}
\adjustbox{bgcolor={red!35.3}}{\strut Brazil}
\adjustbox{bgcolor={red!0.0}}{\strut before}
\adjustbox{bgcolor={red!0.0}}{\strut moving}
\adjustbox{bgcolor={red!0.0}}{\strut to}
\adjustbox{bgcolor={red!38.8}}{\strut Ukrainian club}
\adjustbox{bgcolor={red!31.1}}{\strut Shakhtar Donetsk}
\adjustbox{bgcolor={red!0.0}}{\strut in}
\adjustbox{bgcolor={red!31.3}}{\strut 2005}
\adjustbox{bgcolor={red!0.0}}{\strut .}
[...]
\adjustbox{bgcolor={red!0.0}}{\strut He}
\adjustbox{bgcolor={red!0.0}}{\strut is known}
\adjustbox{bgcolor={red!0.0}}{\strut for}
\adjustbox{bgcolor={red!0.0}}{\strut his}
\adjustbox{bgcolor={red!36.0}}{\strut physicality}
\adjustbox{bgcolor={red!0.0}}{\strut ,}
\adjustbox{bgcolor={red!15.3}}{\strut tackling ability}
\adjustbox{bgcolor={red!0.0}}{\strut ,}
\adjustbox{bgcolor={red!0.0}}{\strut and}
\adjustbox{bgcolor={red!15.3}}{\strut passing range}
\adjustbox{bgcolor={red!0.0}}{\strut ,}
\adjustbox{bgcolor={red!0.0}}{\strut and}
\adjustbox{bgcolor={red!0.0}}{\strut is}
\adjustbox{bgcolor={red!0.0}}{\strut widely regarded}
\adjustbox{bgcolor={red!0.0}}{\strut as}
\adjustbox{bgcolor={red!7.2}}{\strut one of the best}
\adjustbox{bgcolor={red!18.8}}{\strut defensive midfielders} 
\adjustbox{bgcolor={red!0.0}}{\strut in}
\adjustbox{bgcolor={red!14.5}}{\strut the world}
\adjustbox{bgcolor={red!0.0}}{\strut .}
    }\\
\bottomrule
    \end{NiceTabular}
    \vspace{-0.5pc}
    \caption{We sampled 3 instances from our dataset to demonstrate the differences across uncertainty scores. For label, entities colored in {\color{red}red} indicate hallucination. For uncertainty scores, entities \colorbox{red!50}{boxed in red} with different tints represent the degree of uncertainty. A lighter (darker) box indicates a lower (higher) uncertainty.}
    \label{tb:qualitative_analysis}
    \vspace{-1pc}
\end{table*}

Figure~\ref{fig:eval_model_size} presents the performance changes across different model sizes within six families: Llama3, Aquila2, InternLM2, Qwen2.5, Yi, and Gemma2, each comprising two size variants. The results reveal that, in most cases, using a larger model does not significantly enhance performance. The only exception is using Gemma on Focus, where the AUROC score improves by 0.12 between the 27B and 9B versions. Performance improvements for other model families and approaches remain marginal, typically below 0.01. These findings suggest that \emph{a larger LLM may not reflect its better capability of determining hallucination on its token probabilities.}




\paragraph{How does performance vary across different hallucination levels?}

We categorize \dataset into three groups based on the proportion of hallucinated entities in each generation (See Table~\ref{tb:statistic_hallu_rate} in Appendix~\ref{ap:experiment} for details). Figure~\ref{fig:eval_hallucination_rate} shows the $\mathrm{Precision}_\mathrm{Opt}$, $\mathrm{Recall}_\mathrm{Opt}$, and $\mathrm{F1}_\mathrm{Opt}$ scores across three groups. The results reveal that all methods struggle to detect hallucinated content when the hallucination rate is low, with $\mathrm{F1}_\mathrm{Opt}$ scores around 0.2. 
{Entropy and CCP exhibit a steep increase in $\mathrm{Recall}_\mathrm{Opt}$ compared to$\mathrm{Precision}_\mathrm{Opt}$ as the hallucination rate increases, suggesting their tendency to over-predict hallucinations, particularly in a high-hallucination scenario. In contrast, Focus achieves a small difference between the $\mathrm{Recall}_\mathrm{Opt}$ and $\mathrm{Precision}_\mathrm{Opt}$ when the hallucination rate is high, demonstrating its ability to balance precision-recall trade-offs while also highlighting the challenge of detecting sparse hallucination.}



\subsection{In-depth Analysis}\label{sec:qualitative_analysis}

To better understand the strengths and limitations of uncertainty scores for detecting hallucinated entities, we analyze cases where (1) all scores failed or misidentified hallucinations, and (2) scores varied in performance. We classify entities using thresholds for $\mathrm{F1}_\mathrm{Opt}$ and categorize false positives/negatives by POS, NER tags, and sentence positions (first, middle, or last six words). We then identify tags and positions where approaches excel or falter, visualizing samples with color-coded uncertainty scores to uncover patterns behind detection discrepancies (See Table~\ref{tb:qualitative_analysis}). Figure~\ref{fig:error_analysis} shows the FPR and FNR across NER tags and sentence positions (The result across POS tags is in Appendix~\ref{ap:experiment}). Our analysis focuses on Likelihood, SAR, and Focus, as SAR and Focus demonstrated the most effective performance in Section~\ref{sec:result}, and Likelihood serves as a straightforward baseline for comparison.

\begin{table*}[]
\caption{Error analysis of outputs from MLLM.  We classified the errors into five categories and counted the number of them. Note that a single response could contain multiple errors, so the sum of errors does not match the total output of MLLM.}
\Description{
The table presents an error analysis of outputs from an MLLM (Multimodal Large Language Model). The errors are categorized into five types: Wrong Character Recognition, Wrong Object Recognition, Nonexistent Objects and Texts, Misunderstanding User Input, and Inaccurate User Input. It also includes a count of outputs with no errors. The total number of outputs is also provided, and a single response can contain multiple errors. The following describes the findings. For Scene Description, there were 31 instances of Wrong Character Recognition, 6 instances of Wrong Object Recognition, 11 instances of Nonexistent Objects and Texts, no occurrences of Misunderstanding or Inaccurate User Input, 117 outputs with No Error, and a total of 164 outputs. For Q\&A Response, there were 9 instances of Wrong Character Recognition, 6 instances of Wrong Object Recognition, 15 instances of Nonexistent Objects and Texts, 5 instances of Misunderstanding User Input, 1 instance of Inaccurate User Input, 21 outputs with No Error, and a total of 53 outputs.
}
\label{tab:hallucinations}

\begin{tabular}{@{}cccccccc@{}}
\toprule
                  & \begin{tabular}[c]{@{}c@{}}Wrong\\ Character \\ Recognition\end{tabular} & \begin{tabular}[c]{@{}c@{}}Wrong\\ Object\\ Recognition\end{tabular} & \begin{tabular}[c]{@{}c@{}}Nonexistent\\ Objects and \\Texts\end{tabular} & \begin{tabular}[c]{@{}c@{}}Misunderstanding\\ User\\ Input\end{tabular} & \begin{tabular}[c]{@{}c@{}}Inaccurate\\ User\\ Input\end{tabular} & \begin{tabular}[c]{@{}c@{}}No\\ Error\end{tabular} & \begin{tabular}[c]{@{}c@{}}Total \\ output\end{tabular} \\ \midrule
Scene Description & 31                                                  & 6                                                                    & 11                                                                   & -                                                                     & -                                                          & 117                                                   & 164                                                          \\
Q\&A Response     & 9                                                   & 6                                                                    & 15                                                                   & 5                                                                     & 1                                                          & 21                                                    & 53                                                           \\ \bottomrule
\end{tabular}%
\end{table*}

\vspace{-0.1cm}
\paragraph{SAR under-predicts hallucinations due to unreliable token importance weighting.}
The left and middle plots of Figure~\ref{fig:error_analysis} show that SAR has the lowest FPR but the highest FNR across most tags, particularly for named entities, indicating a tendency to under-predict hallucinations. This occurs because SAR weights token importance based on sentence similarity without the token, which often remains unchanged even if the token is informative. The first case in Table~\ref{tb:qualitative_analysis} illustrates this: SAR assigns lighter shades to entities like the second ``Santa Cruz'' since removing either ``Santa'' or ``Cruz'' barely affects sentence similarity, despite the term's informativeness.



\vspace{-0.1cm}
\paragraph{The type-filter of Focus reduces FNR on name entities but sheds light on a bigger limitation of uncertainty-based hallucination score.}
The left and middle plots of Figure~\ref{fig:error_analysis} reveal that Focus performs differently for named and non-named entities. It achieves a low FNR but high FPR for named entities, and the opposite for non-named ones. This is because Focus filters for named entities based on POS and NER tags. While promising---since named entities often hallucinate (as shown in Figure~\ref{fig:linguistic_stat})---its high FPR suggests that its base score (the sum of Likelihood and Entropy) poorly distinguishes hallucinations, frequently assigning high uncertainty to named entities. The 2nd case in Table~\ref{tb:qualitative_analysis} illustrates this: Focus ignores function words like ``is'' and ``to,'' reducing FPR, but indiscriminately highlights named entities like ``American'' and ``A Bronx Tale,'' even when accurate.


\vspace{-0.1cm}
\paragraph{Uncertainty propagation of Focus alleviates the over-confidence nature of LLMs.}
The right plots in Figure~\ref{fig:error_analysis} show that LLMs are less confident when generating the first few words of a sentence and become over-confident as generation progresses, as indicated by a decrease in FPR and an increase in FNR for Likelihood. This contrasts with the typical distribution of hallucinations, which occur mostly in the middle and end of sentences (Section~\ref{sec:data_analysis}). Focus addresses this by propagating uncertainty scores based on attention, leading to a decrease of FNR over positions. However, its FPR increase over positions suggest that using attention scores to propagate uncertainty may wrongly penalize entities that are not over-confident. The 3rd case in Table~\ref{tb:qualitative_analysis} illustrates this: Likelihood assigns higher uncertainty to early words (\eg, ``Fernandinho began'') and lower scores to later words (\eg, ``Shakhtar Donetsk in 2005''), while Focus detects hallucinations at sentence ends by linking them to prior hallucinated content (\eg, ``Ukrainian club'').
