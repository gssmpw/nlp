To mitigate the impact of hallucination nature of LLMs, many studies propose detecting hallucinated generation through uncertainty estimation. However, these approaches predominantly operate at the sentence or paragraph level, failing to pinpoint specific spans or entities responsible for hallucinated content. This lack of granularity is especially problematic for long-form outputs that mix accurate and fabricated information. To address this limitation, we explore \emph{entity-level hallucination detection}. We propose a new data set, \dataset, which annotates hallucination at the entity level. Based on the dataset, we comprehensively evaluate uncertainty-based hallucination detection approaches across 17 modern LLMs. Our experimental results show that uncertainty estimation approaches focusing on individual token probabilities tend to over-predict hallucinations, while context-aware methods show better but still suboptimal performance. Through an in-depth qualitative study, we identify relationships between hallucination tendencies and linguistic properties and highlight important directions for future research.