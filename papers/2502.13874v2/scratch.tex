% In this paper, we will explain why spatial data requires special treatment, and how and when to semantically lift environmental data to a knowledge graph. We will present our KnowWhereGraph that contains a wide range of integrated datasets at the human-environment interface, introduce our application areas, and discuss geospatial enrichment services on top of our graph. Jointly, the graph and services will provide answers to questions such as ``what is here", ``what happened here before", and ``how does this region compare to \dots" for any region on earth within seconds.

% Catchy KWG
% KnowWhereGraph\footnote{\url{https://knowwheregraph.org/}} (KWG) is one of the largest, publicly available geospatial knowledge graphs in the world \cite{kwg-ai-mag,kwg-tr}.
% Use-case
% KWG generally supports applications in the food, agriculture, humanitarian relief, and energy sectors and their attendant supply chains; and more specifically supports environmental policy issues relative to interactions among agricultural sustainability, soil conservation practices, and farm labor; and delivery of emergency humanitarian aid, within the US and internationally. 
% Statistics
% To do so, KWG brings together over 30 datasets related to observations of natural hazards (e.g., hurricanes, wildfires, and smoke plumes), spatial characteristics related to climate (e.g., temperature, precipitation, and air quality), soil properties, crop and land-cover types, demographics, human health, and spatial representations of human-meaningful places, resulting in a knowledge graph with over 16 billion triples.
% Schema
% To integrate these data, we have added an additional layer: KWG uses a schema that provides a thorough and rich\footnote{The OWL ontology has over 300 classes and about 3,000 axioms.} ontological representation formally describing the relationships between the types of data. The geospatial integration is performed by a consistent alignment to a Discrete Global Grid (DGG) \cite{dgg}, where we partition the surface of the Earth into small squares. These squares form an approximation of the spatial extent of physical and regional phenomena within the graph and act as a geospatial backbone. 

% Enter KWG-Lite and KWP
% However, due to the size and complexity of the graph and its schema, this can result in a steep learning curve and usability obstacles for those who are not well versed in ontologies, knowledge graphs, or SPARQL \cite{sparql-tr}. Likewise, for certain use cases a deep or rich ontological representation is not necessary; for some users visualizing or navigating graph data values can be unintuitive; and finally, the use of the DGG can be a barrier itself, as it can result in long and expensive queries.

% For example, a central piece of our schema is depicted in Figure~\ref{fig:kwg-core-sosa-kernel}. Learning the specific value of an observation pertaining to a particular feature of interest is already a complex, multi-hop query. Figure~\ref{fig:chain} shows how a particular instance of a \textsf{Hazard} finally relates to an observation on its impact, which is encoded as a data type. This rather complex way of representing the data is necessary because some of the target applications of our graph are aimed at specialists who require a high level of detail. However, for less involved application use cases, we aim to simplify this process and reduce the conceptual barrier to entry, as well as improve performance for certain use cases.

% Such a pattern-based approach furthermore allows for relatively straightforward dataset integration. When integrating multiple datasets, it can be convenient to conceptualize them along the same (ontological) dimension. For example, understanding tabular data about a place as observations, similarly to how a hazard impacts a place (and the measurement thereof) is also an observation. This results in a predictable method for querying against the final, integrated schema of the knowledge graph.