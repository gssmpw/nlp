\section{Related Work}
\label{sec:rel}
% \todo[inline]{Assign: Meilin, Zilong and Kitty \\
% Core content: discuss those literatures that are relevant to KnowWhereGraph. Focus on how KnowWhereGraph is similar and different from these works. Some topics are identified as below. \\
% Length: 1.5 pages}
%%%%%
In this section, we discuss the role of spatial data infrastructures and (geo)portals in facilitating the development and deployment of geospatial knowledge graphs. We focus on how existing geospatial knowledge graphs compare with KnowWhereGraph. Finally, we discuss several key concepts, methodologies, and conventions used in this paper.  
%%%%%%%%%%%%%%%%%%%%%%%%%%%%%%%%%%%%%%%%%%%%%%%%%%%%%%%%%%%%%%%%%
\subsection{Spatial Data Infrastructures}
\label{ssec:sdi}
%%%%%
Central to discussions of geospatial data is the concept of Spatial Data Infrastructure (SDI), which encompasses the fundamental physical and organizational structures that enable the use of spatial data by multiple stakeholders \citep{rajabifard_role_2006}. Typical definitions invoke some combination of components---e.g., technology, standards, systems, people, policies, and data---aiming at facilitating data acquisition, storage, and sharing, assisting policy-making, and enhancing business operations \citep{hendriks_reconsidering_2012}. The five most typical activities performed within SDIs are identified as data discovery, access, registration, processing, and visualization. However, these activities often encounter semantic challenges, such as missing or unclear metadata, which can stem from both the original data and the processes involved \citep{janowicz2010semantic}. To overcome these challenges and extend the capabilities of SDIs, Linked Data, a term synonymous with knowledge graphs, has been proposed to semantically empower SDIs. Linked Data is increasingly recognized as the cornerstone of next-generation SDIs \citep{janowicz2010semantic, wiemann2016spatial, huang2019assessment, ronzhin2019next, huang2020towards}. KnowWhereGraph, presented in this paper, is one such initiative aiming at building a cross-disciplinary SDI using Linked Data technology. 

%%%%%%%%%%%%%%%%%%%%%%%%%%%%%%%%%%%%%%%%%%%%%%%%%%%%%%%%%%%%%%%%%
\subsection{(Geo)portals}
\label{ssec:gp}
%%%%%
The key activities identified within SDIs have driven the development of (geo)portals, defining the essential tasks these platforms are expected to perform. (Geo)portals are web-based geospatial resources that enable searching and accessing integrated geographic information \citep{maguire2005emergence,mai2020semantically}. By following the Open Geospatial Consortium (OGC) and the International Organization for Standardization (ISO) standards, (geo)portals have achieved functionalities such as metadata cataloging, data discovery, data visualization, data sharing, and data downloading. To address the challenge of topic heterogeneity resulting from multiple metadata standards, \citet{hu2015metadata} proposed Linked Data--driven (geo)portals with enhanced interoperability. Furthermore, \citet{jiang2019current} provided a summary and highlighted remaining challenges in existing (geo)portals, such as data updating, harmonization, and the need to address multidisciplinary demands in the era of big data.

(Geo)portals remain crucial for data sharing, visualization, and online management, especially within government agencies and research institutions. However, there is a growing need for (geo)portals to transition toward more comprehensive geospatial knowledge graphs, i,e., \textit{making data smart}, not applications \citep{janowicz2015data}. Such graphs provide more advanced analysis across domains to better manage the ever-growing volume of geospatial data. In this paper, we outline the strategies and methods employed by KnowWhereGraph to build an open and user-friendly platform. A key focus is on bridging the gap between data from different themes and sources that are typically stored separately in conventional (geo)portals.

%%%%%%%%%%%%%%%%%%%%%%%%%%%%%%%%%%%%%%%%%%%%%%%%%%%%%%%%%%%%%%%%%
\subsection{Geospatial Knowledge Graphs}
\label{ssec:gkg}
%%%%%
With the availability of large-scale encyclopedic knowledge graphs like DBpedia \citep{auer2007dbpedia} and YAGO \citep{suchanek2007yago}, there is an increasing need to link rich geospatial datasets for improved data discovery, integration, and knowledge extraction \citep{mai2022symbolic,regalia2018gnis,qi2023evkg}. Previous efforts, such as the LinkGeoData project \citep{auer2009linkedgeodata,stadler2012linkedgeodata}, have focused on transforming OpenStreetMap (OSM)\footnote{\url{https://www.openstreetmap.org/}} data into a graph structure, linking it with DBpedia, GeoNames\footnote{\url{https://www.geonames.org/}}, and other datasets. YAGO2 \citep{hoffart2013yago2} adds spatio-temporal dimensions to YAGO by including \emph{time} and \emph{location} in its original subject–property–object triples representation. YAGO2geo \citep{karalis2019extending} further extends YAGO2 by incorporating more detailed geospatial information, such as administrative data from three countries, geometries of administrative regions from the Global Administrative Areas dataset (GADM), and geographic features like lakes and streams from OSM. These advancements, for instance, representing cities with multipolygons instead of simple coordinates, significantly enhanced geospatial capabilities for more complex question-answering \citep{mai2021geographic}. However, these geospatial knowledge graphs have mainly focused on modeling and integrating a \textit{limited} set of regional identifiers, thus limiting their ability to cover a broader range of geospatial information, such as land-use delineation, climate zones, and hurricane trajectories. This paper introduces a novel combination of discrete global grid (DGG) and knowledge graph technology used by KnowWhereGraph to harmonize different types of geographic information across various themes.

Finally, it is worth noting the recent launch of ArcGIS Knowledge\footnote{\url{https://enterprise.arcgis.com/en/knowledge/}} in 2023, an enterprise knowledge graph tool that combines graph and spatial analytics. This tool enables the discovery and analysis of interlinked spatial and non-spatial, structured and unstructured data. Additionally, the latest release\footnote{\url{https://enterprise.arcgis.com/en/knowledge/latest/introduction/what-is-arcgis-knowledge.htm}} of ArcGIS Knowledge also allows users to connect existing graph databases to the ArcGIS platform. To align with these industry advancements, this paper also introduces plug-ins for both ArcGIS Pro and QGIS\footnote{\url{https://qgis.org/}}, allowing users to readily benefit from the rich and interlinked data provided by KnowWhereGraph. 

%Previous work on constructing geospatial knowledge graphs has mainly focused on two aspects, namely, how to represent geometries of geographic features and how to model their topological relations. The former includes YAGO2geo \citep{karalis2019extending}, a geospatial extension of the YAGO2 knowledge graph \citep{hoffart2013yago2}, 

%%%%%%%%%%%%%%%%%%%%%%%%%%%%%%%%%%%%%%%%%%%%%%%%%%%%%%%%%%%%%%%%%
\subsection{Convention and Background}
\label{ssec:conv}
%%%%%
% \todo[inline]{Are there any others}
%%%%%
We note a few conventions that are used throughout the paper and provide background discussions surrounding key techniques adopted in this paper.

\textbf{Schema Diagrams.} These are used to quickly and intuitively convey the expected structure of a knowledge graph. They are central to the Modular Ontology Modeling (MOMo) process \citep{momo-swj}, which was the initial knowledge modeling methodology used to develop KnowWhereGraph's schema, and they serve as our primary visual tool for both team collaboration and outreach concerning symbolic knowledge models. These diagrams are consistently used throughout our documentation, and they have a stable visual syntax.

Each schema diagram represents a labeled graph that indicates Web Ontology Language (OWL) entities and their (possible) relations. Nodes in these diagrams are categorized and visually distinguished as follows: (1) \emph{Classes} -- rectangular, gold, solid border, (2) \emph{Modules} -- rectangular, light blue, dashed border, (3) \emph{Controlled vocabularies} -- rectangular, purple, solid border, (4) \emph{Data types} -- oval, yellow, solid border. Arrows in the schema diagrams can be of two types: (1) \emph{Subclass relation} (\texttt{rdfs:subClassOf}) -- white-headed without labels, (2) \emph{Data}/\emph{object property} -- labeled with the name of the property. A through discussion of these constructs, along with the ontology modeling process that guided the construction of KnowWhereGraph's underlying graph, is provided by \citet{kwg-jws,kwg-fois,momo-swj}. Such schema diagrams can be found in Figures~\ref{fig:KnowWhereGraph-core-sosa-kernel}--\ref{fig:md-model}.

\textbf{URI and Prefix.}
Table~\ref{tab:namespaces} lists the various namespaces, in the format of URI (Uniform Resource Identifier), and their prefixes that we use within KnowWhereGraph, including the datatypes and ontologies that we re-use. Of particular importance are \texttt{kwg-ont}, which is used for KnowWhereGraph's ontology, and \texttt{kwgr}, which generally indicates individual entity or instance. For example, it is common to see the triple of the following form to indicate the statement that ``x (e.g., Thomas Fire) is an instance of type X (e.g., wildfire), defined by KnowWhereGraph ontology (Section \ref{sec:schema})". 
\begin{equation*}
    %<\texttt{kwgr:x}, \text{\qquad}\texttt{a}, \text{\qquad}\texttt{kwg-ont:X~}>
    \langle\texttt{kwgr:x, a, kwg-ont:X}\rangle
\end{equation*}

\begin{table}
    \centering
    \begin{tabular}{r|l}
Prefix   & Namespace \\\hline
% deo:     & $\langle$\url{http://url#}$\rangle$\\
geo:     & $\langle$\url{http://www.opengis.net/ont/geosparql#}$\rangle$\\
cdt:    & $\langle$\url{http://w3id.org/lindt/custom_datatypes#}$\rangle$\\
kwg-ont: & $\langle$\url{http://stko-kwg.geog.ucsb.edu/lod/ontology/}$\rangle$\\
kwgr:    & $\langle$\url{http://stko-kwg.geog.ucsb.edu/lod/resource/}$\rangle$\\
prov:    & $\langle$\url{http://www.w3.org/ns/prov#}$\rangle$\\
qudt:    & $\langle$\url{http://qudt.org/schema/qudt/}$\rangle$\\
sosa:    & $\langle$\url{http://www.w3.org/ns/sosa/}$\rangle$\\
time:    & $\langle$\url{http://www.w3.org/2006/time#}$\rangle$\\
xsd:     & $\langle$\url{http://www.w3.org/2001/XMLSchema#}$\rangle$
    \end{tabular}
    \caption{Namespaces used in KnowWhereGraph and in the various schema diagrams presented in this paper.} 
    \label{tab:namespaces}
\end{table}

% {\color{red} \textbf{Thematic vs. Spatial.}}

%A Discrete Global Grid (DGG) is a spatial reference system that partitions the Earth’s surface into a series of uniform, non-overlapping cells. 

\textbf{Discrete Global Grids. } A Discrete Global Grid (DGG) system is a framework composed of hierarchical levels of grids. The key components of a DGG include (1) \emph{cells} -- the basic unit, representing a fixed geographic region; (2) \emph{levels} -- corresponding to granularities of cells; and (3) \emph{coverings} -- a collection of cells that cover specific areas or shapes. Each cell in a DGG is uniquely identified by a stable cell ID, which encodes information about its level, relations with its parent/children, and neighboring cells from the same level. 

In KnowWhereGraph, we utilize Google's \emph{S2 Geometry} framework \citep{veach2017s2}, which is a hierarchical mosaic of spherical quadrilateral cells, as the DGG foundation. S2 cells are sequentially indexed using a Hilbert space-filling curve that traverses the surface of a unit sphere projected onto the six faces of a cube. This projection ensures a relatively even distribution of cells across the sphere's surface. Each S2 cell is identified by a unique 64-bit \texttt{S2CellID}, which encodes the cell’s location on the curve and its level within the S2 hierarchy. 
%The grid of cells at level 13 (approximately $1.27\ km^2$) serves as the primary spatial reference framework for linking spatial features in the graph. The implementation of S2 Geometry within KnowWhereGraph, detailed in \citep{stephen2024s2}, enriches diverse data sources topologically and enables the conflation of multi-scale raster and vector data.

%%%%%%%%%%%%%%%%%%%%%%%%%%%%%%%%%%%%%%%%%%%%%%%%%%%%%%%%%%%%%%%%%