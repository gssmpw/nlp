% This must be in the first 5 lines to tell arXiv to use pdfLaTeX, which is strongly recommended.
\pdfoutput=1
\pdfminorversion=5  % 降低 PDF 版本
\pdfobjcompresslevel=0  % 禁用对象压缩

% In particular, the hyperref package requires pdfLaTeX in order to break URLs across lines.

\documentclass[11pt]{article}

% Change "review" to "final" to generate the final (sometimes called camera-ready) version.
% Change to "preprint" to generate a non-anonymous version with page numbers.
\usepackage[preprint]{acl}
% \usepackage[review]{acl}
\usepackage{nicematrix}
% Standard package includes
\usepackage{times}
\usepackage{latexsym}
\usepackage{subfigure}
\usepackage{amsmath}
\usepackage{amsfonts}
\usepackage{enumitem}
\usepackage{multirow}
\usepackage{colortbl} % 用于表格颜色
\usepackage{xcolor}   % 用于定义颜色
\usepackage{booktabs}
\usepackage{adjustbox}
\usepackage{makecell}
\usepackage{hyperref}
\usepackage{tabularx}
\usepackage{booktabs}
\usepackage{array}
\usepackage[linesnumbered,ruled,vlined]{algorithm2e}

\usepackage{booktabs}
\newcommand{\Thickline}{\noalign{\hrule height 1.2pt}}
%%%%%%%%% 图位置
% \setlength{\abovecaptionskip}{0pt}    % 调整图片与标题间距
% \setlength{\belowcaptionskip}{0pt}    % 调整标题与后续文本间距
\setlength{\abovecaptionskip}{5pt}  % 图题上方的距离
\setlength{\belowcaptionskip}{-10pt}  % 图题下方的距离

%%%%%%%%%
% \usepackage{algorithm}
% \usepackage{algorithmic}
% \usepackage[ruled,vlined]{algorithm2e}
% For proper rendering and hyphenation of words containing Latin characters (including in bib files)
\usepackage[T1]{fontenc}
% For Vietnamese characters
% \usepackage[T5]{fontenc}
% See https://www.latex-project.org/help/documentation/encguide.pdf for other character sets

% This assumes your files are encoded as UTF8
\usepackage[utf8]{inputenc}

% This is not strictly necessary, and may be commented out,
% but it will improve the layout of the manuscript,
% and will typically save some space.
\usepackage{microtype}

% This is also not strictly necessary, and may be commented out.
% However, it will improve the aesthetics of text in
% the typewriter font.
\usepackage{inconsolata}

%Including images in your LaTeX document requires adding
%additional package(s)
\usepackage{graphicx}

% If the title and author information does not fit in the area allocated, uncomment the following
%
%\setlength\titlebox{<dim>}
%
% and set <dim> to something 5cm or larger.
\newcommand\ourmethod{\texttt{SARChat}\xspace}

\title{SARChat-Bench-2M: A Multi-Task Vision-Language Benchmark for SAR Image Interpretation}

% Author information can be set in various styles:
% For several authors from the same institution:
% \author{Author 1 \and ... \and Author n \\
%         Address line \\ ... \\ Address line}
% if the names do not fit well on one line use
%         Author 1 \\ {\bf Author 2} \\ ... \\ {\bf Author n} \\
% For authors from different institutions:
% \author{Author 1 \\ Address line \\  ... \\ Address line
%         \And  ... \And
%         Author n \\ Address line \\ ... \\ Address line}
% To start a separate ``row'' of authors use \AND, as in
% \author{Author 1 \\ Address line \\  ... \\ Address line
%         \AND
%         Author 2 \\ Address line \\ ... \\ Address line \And
%         Author 3 \\ Address line \\ ... \\ Address line}

% \author{First Author \\
%   Affiliation / Address line 1 \\
%   Affiliation / Address line 2 \\
%   Affiliation / Address line 3 \\
%   \texttt{email@domain} \\\And
%   Second Author \\
%   Affiliation / Address line 1 \\
%   Affiliation / Address line 2 \\
%   Affiliation / Address line 3 \\
%   \texttt{email@domain} \\}

\author{
 Zhiming~Ma$^{1,2\dag}$, Xiayang~Xiao$^{1,2\dag*}$, Sihao~Dong$^{3\dag}$, Peidong~Wang$^4$, HaiPeng~Wang$^{1*}$, Qingyun~Pan$^{5}$  \hspace{0.4em} \\
 $^1$The Key Laboratory for Information Science of Electromagnetic Waves (Ministry of Education), \\ School of Information Science and Technology, Fudan University, Shanghai, China \\
 $^2$China Mobile Internet Company Ltd., Guangzhou, China \\
 $^3$The School of Automation and Electrical Engineering, \\ Inner Mongolia University of Science and Technology, Baotou, China \\
 $^4$School of Computer Science and Engineering, Northeastern University, Shenyang, China \\
 $^5$China Mobile Group Guangdong Co., Ltd. Guangzhou Branch, Guangzhou, China \\
% \texttt{mazhiming312@outlook.com} \quad
% \texttt{xxx@xx} \\
% \texttt{xxx@xx} \quad
% \texttt{xxx@xx}\\
% \texttt{\{xxx@xx\}@cse.neu.edu.cn}\\
}


\begin{document}
\maketitle
\begin{abstract}
% Despite significant progress of Large Language Models (LLMs) in natural language processing and image understanding, their application in the interpretation of Synthetic Aperture Radar (SAR) remote sensing images remains constrained by the high threshold of domain-specific expertise and the scarcity of training data. To address these limitations, this study constructs the first large-scale multimodal dialogue dataset for SAR images, named \ourmethod-2M, comprising 2 million high-quality image-text pairs. This dataset encompasses diverse scenarios such as oceanic, terrestrial, and urban environments, with detailed annotations of fine-grained features including target shape, quantity, and location. The dataset supports multiple tasks, including scene classification, image captioning, visual question answering, object localization. Experimental validation on 16 mainstream vision-language models, including the Qwen2VL series, InternVL2.5 series, and LLaVA, confirms the efficacy of the dataset and establishes the first multi-task dialogue benchmark in the SAR domain. The associated code, models, and dataset have been made publicly available at https://github.com/JimmyMa99/\ourmethod to foster further advancements in SAR vision-language models.

% While Vision-Language Models (VLMs) have demonstrated remarkable capabilities in general-domain visual-language understanding, their application to Synthetic Aperture Radar (SAR) image interpretation remains limited due to domain expertise barriers and data scarcity. This paper introduces \ourmethod-2M, the first large-scale multimodal dialogue dataset for SAR imagery analysis, containing millions of high-quality image-text pairs across diverse environments including oceanic, terrestrial, and urban scenarios. The dataset provides comprehensive annotations of spatial-temporal features and supports multimodal tasks. Extensive evaluations on mainstream vision-language architectures (Qwen2VL, InternVL2.5, LLaVA) validate the dataset's effectiveness and establish the first comprehensive benchmark for SAR-oriented multimodal dialogue systems. Our work advances SAR visual-language understanding by publicly releasing the complete framework including dataset, models, and evaluation protocols at https://github.com/JimmyMa99/SARChat.

% In the field of synthetic aperture radar (SAR) remote sensing image interpretation, although Vision language models (VLMs) have made remarkable progress in natural language processing and image understanding, their applications still face challenges such as high thresholds of professional knowledge and a scarcity of training data. This paper innovatively proposes the first large-scale multimodal dialogue dataset for SAR images SARChat-Bench-2M, which contains approximately 2 million high-quality image-text pairs, comprehensively covering diverse scenarios such as oceans, land, and urban areas, and provides detailed annotations of fine-grained features such as the shape, quantity, and location of targets. This dataset not only supports several key tasks such as scene classification, image description generation, visual question answering, target localization, and detection, but also has unique innovative aspects: We developed a visual-language dataset and benchmark for the SAR domain, enabling and evaluating VLMs' capabilities in SAR image interpretation, which provides a paradigmatic framework for constructing multimodal datasets across various remote sensing vertical domains. Through experiments on 16 mainstream VLMs, the effectiveness of the dataset has been fully verified, and the first multi-task dialogue benchmark in the SAR field has been successfully established. The project will be released at \url{https://github.com/JimmyMa99/SARChat}, aiming to promote the in-depth development and wide application of SAR visual language models.

As a powerful all-weather Earth observation tool, synthetic aperture radar (SAR) remote sensing enables critical military reconnaissance, maritime surveillance, and infrastructure monitoring. Although Vision language models (VLMs) have made remarkable progress in natural language processing and image understanding, their applications remain limited in professional domains due to insufficient domain expertise. This paper innovatively proposes the first large-scale multimodal dialogue dataset for SAR images, named SARChat-2M, which contains approximately 2 million high-quality image-text pairs, encompasses diverse scenarios with detailed target annotations. This dataset not only supports several key tasks such as visual understanding and object detection tasks, but also serves as the first visual-language benchmark in the SAR domain. Through this work, we enable and evaluate VLMs' capabilities in SAR image interpretation, providing a paradigmatic framework for constructing multimodal datasets across various remote sensing vertical domains. Through experiments on 16 mainstream VLMs, the effectiveness of the dataset has been fully verified. The project will be released at 
\url{https://github.com/JimmyMa99/SARChat}.

% \url{https://anonymous.4open.science/r/SARChat-D0ED/}.

\end{abstract}

\def\thefootnote{\dag}\footnotetext{These authors contributed equally to this work.}\def\thefootnote{\arabic{footnote}}
\def\thefootnote{*}\footnotetext{Corresponding author. Xiayang~Xiao and Haipeng~Wang}
\def\thefootnote{*}\footnotetext{Contact email: mazhiming312@outlook.com or xyxiao20@fudan.edu.cn }
\def\thefootnote{\arabic{footnote}}

% \def\thefootnote{\dag}
% \footnotetext[1]{These authors contributed equally to this work.}

% \def\thefootnote{*}\footnotetext[2]{Corresponding author. Xiayang~Xiao and Haipeng~Wang}

\def\thefootnote{\arabic{footnote}}

\section{Introduction}
% In recent years, deep neural network, particularly Convolutional Neural Networks (CNNs) \citep{dosovitskiy2020image} and Vision Transformers (ViTs) \citep{lecun1998gradient}, have achieved remarkable progress in the field of remote sensing data analysis, significantly enhancing the processing efficiency and analytical accuracy of large-scale remote sensing datasets. However, existing research has predominantly focused on the extraction of visual features from images, while falling short in the deep semantic parsing of visual labels and general reasoning capabilities \citep{li2024vision}. This limitation constrains the applicability and interpretability of models in complex scenarios.
% Concurrently, LLMs and their extended versions have made significant achievements in the field of natural language processing. Vision-Language Models (VLMs), by integrating generative pre-training and instruction tuning strategies, have demonstrated robust zero-shot learning and generalization capabilities in multimodal interactive tasks \cite{dai2023instructblip}. Inspired by these advancements, researchers have begun to explore the deep integration of visual models with LLMs to construct novel vision-language models.
% Although preliminary achievements have been made in the remote sensing domain with models such as RSGPT \citep{hu2023rsgpt} and GeoChat \citep{kuckreja2024geochat}, these models are primarily designed for natural images and are not directly applicable to SAR images. SAR images, due to their unique scattering imaging mechanisms, present challenges such as blurred target edges, dispersed scattering speckles, and high sensitivity to orientation, which hinder intuitive interpretation. Moreover, there is currently a lack of large-scale, high-quality SAR image-text alignment datasets. Existing datasets are mostly focused on visual recognition tasks and lack accompanying textual descriptions, which restricts the application of VLMs in the SAR domain \citep{kuckreja2024geochat, cheng2022nwpu, zhang2023rs5m}.
In recent years, deep neural networks, notably CNNs \citep{lecun1998gradient} and ViTs \citep{dosovitskiy2020image}, have achieved remarkable progress in remote sensing data analysis, enhancing both processing efficiency and analytical accuracy. However, existing research mainly focuses on visual feature extraction, while lacking deep semantic parsing and reasoning capabilities \citep{li2024vision}, limiting model applicability in complex scenarios.

With the advancement of Large-Language Models (LLMs), Vision-Language Models (VLMs), through integrating pre-training and instruction tuning, have demonstrated robust zero-shot learning and generalization in multimodal tasks \cite{dai2023instructblip}. This has inspired researchers to explore the deep integration of visual models with LLMs.

Although models designed for optical remote sensing images, like RSGPT \citep{hu2023rsgpt} and GeoChat \citep{kuckreja2024geochat}, have shown preliminary achievements, they struggle to perform well in SAR applications. SAR images inherently pose significant interpretation challenges due to their scattering imaging mechanisms, characterized by blurred target edges, dispersed speckles, and orientation sensitivity. Meanwhile, existing SAR datasets primarily focus on visual recognition tasks \citep{kuckreja2024geochat, cheng2022nwpu, zhang2023rs5m}, leaving a critical shortage of large-scale, high-quality image-text alignment datasets. Both these intrinsic characteristics and data limitations impede the advancement of VLMs in the SAR domain.


\begin{figure*}[th]
    \centering
    \includegraphics[width=0.73\linewidth]{figures/fig1.pdf}
    \caption{\textbf{An overview of \ourmethod-Bench-2M.} The left figure demonstrates the representative tasks realized with the SAR image-text dataset, \ourmethod-2M, constructed in this paper. Validating the dataset's efficacy and superiority in supporting multi-task applications. The right figure presents the correlation radar charts and quantitative line graphs derived from the performance evaluation of 16 VLMs basing on this dataset, establishing the benchmark (\ourmethod-Bench) within this domain.}
    \label{fig1-overview}
    % \vspace{-1em}
\end{figure*}

% As shown in Figure \ref{fig1-overview}, the core contribution of this paper is the construction of a large-scale multimodal conversational dataset for SAR images, named \ourmethod-2M, and the establishment of a multimodal task-oriented benchmark for the SAR domain, \ourmethod-2M. The \ourmethod-2M dataset comprises approximately 2 million high-quality SAR image-text pairs, encompassing diverse scenarios such as maritime, terrestrial, and urban areas. It is characterized by fine-grained semantic descriptions and multi-scale image resolutions ranging from 0.3 to 10 meters. As illustrated in Figure\ref{fig1-overview}, through cross-modal representation learning, the dataset establishes a robust correspondence between SAR images and their textual descriptions, supporting multi-task learning, including scene classification, image captioning, visual question answering (VQA), visual localization, and object detection. Based on this, the \ourmethod-2M benchmark, constructed in this paper, covers six core tasks—classification, description, counting, localization, recognition, and reference—which systematically evaluate the performance of vision-language models in the SAR domain.
% The primary contributions of this paper are as follows:

% \noindent(1) The creation of the largest SAR remote sensing instruction-following dataset to date, \ourmethod-2M, which contains over 2 million high-quality image-text pairs and covers multi-scenario task-oriented dialogues, effectively addressing the knowledge scarcity of VLMs in the SAR domain.

% \noindent(2) The establishment of the SAR domain multimodal task-oriented benchmark, \ourmethod-2M, which includes six core tasks—classification, description, counting, localization, recognition, and reference-and systematically evaluates the performance of vision-language models in the SAR domain through multi-dimensional assessment metrics.

Current VLMs are primarily trained on conventional natural images without extensive fine-tuning for the SAR vertical domain. Despite their strong visual capabilities for natural images, these VLMs still have significant room for improvement in SAR image interpretation. Building upon the SARDet-100K dataset \citep{li2024sardet100k} with its rich SAR imagery and detection annotations, we construct \ourmethod-Bench-2M, a task-oriented SAR-specific image-text pair dataset, to address the insufficient SAR image interpretation capabilities of existing VLMs.

As shown in Figure \ref{fig1-overview}, we present \ourmethod-2M, a large-scale multimodal conversational dataset for SAR images, and establish \ourmethod-Bench, a comprehensive multimodal task-oriented benchmark for the SAR domain. The \ourmethod-2M dataset contains approximately 2 million high-quality SAR image-text pairs across maritime, terrestrial, and urban scenarios, featuring fine-grained semantic descriptions and multi-scale resolutions (0.3-10 meters). The dataset supports major vision-language tasks such as image captioning, VQA(Visual Question Answering), visual localization, and object detection. To systematically evaluate model performance in these domains, we design six specific benchmark tasks in \ourmethod-2M: \textbf{classification}, \textbf{fine-grained description}, \textbf{instance counting}, \textbf{spatial grounding}, \textbf{cross-modal identification}, and \textbf{referring}. To validate the effectiveness of our dataset and benchmark, we conduct extensive experiments by fine-tuning 16 state-of-the-art VLMs of varying parameter scales, including InternVL2.5, DeepSeekVL, GLM-Edge-V, and the mPLUG-Owl family. Through training on \ourmethod-2M, these visual language models (VLMs) acquire comprehensive multi-task capabilities in SAR interpretation, as demonstrated by our systematic evaluation on \ourmethod-Bench.

The primary contributions of this paper are as follows:
\begin{enumerate}
\item The construction of \ourmethod-2M, the largest SAR remote sensing instruction-following dataset to date, comprising over 2 million high-quality image-text pairs across multi-scenario task-oriented dialogues, alleviating the knowledge scarcity of VLMs in the SAR domain.

\item The development of \ourmethod-Bench, a comprehensive SAR domain multimodal benchmark encompassing six core tasks (classification, description, counting, localization, recognition, and refering), enabling systematic evaluation of vision-language models through multi-dimensional assessment metrics.

\item It pioneers a research paradigm applicable to the SAR field, providing reference ideas for the construction of models in other remote-sensing vertical domains. The methods and processes adopted in data collection, annotation, as well as model training and evaluation in this study have good generality and extensibility.
\end{enumerate}
%%%%%%%%%%%%%%%%%%%%%%%%%%%%%%%%%%%%%%%%%%%%%%%%%%%%%
\begin{figure*}[h]
    \centering
    \includegraphics[width=0.7\linewidth]{figures/2.pdf}
    \caption{\textbf{Construction of \ourmethod-2M dataset.} On the left, ten existing SAR detection benchmark datasets. The middle part is the SARDet-100K dataset, formed by integrating the ten datasets on the left. On the right, six core tasks constructed based on the dataset are presented, with each task corresponding to different task identifiers, operation steps, and relevant templates.}
    \label{fig2-data}
\end{figure*}

\section{Related Work}

\subsection{VLMs for Remote Sensing}
VLMs are capable of converting images into natural language descriptions and parsing the relationships between objects, demonstrating remarkable performance in tasks such as text-image retrieval, image captioning, and visual question answering. Recently, models like RemoteClip \citep{liu2024remoteclip} have been applied to the field of remote sensing images, primarily focusing on cross-modal retrieval and zero-shot classification. However, these models have not addressed tasks such as image description generation and visual grounding. The RSGPT model has achieved text description and visual question answering for remote sensing images, but it has not expanded to tasks such as classification and detection. The GeoChat model has advanced multi-task conversational processing of high-resolution remote sensing imagery, including scene classification, visual question answering, multi-turn dialogue, visual grounding, and reference object detection. However, these models, including GeoChat, predominantly rely on optical remote sensing training data, leading to suboptimal performance in SAR-specific interpretation tasks. EarthGPT \citep{zhang2024earthgpt} has extended the application of multimodal large language models to the remote sensing field through instruction tuning, but its performance in SAR image multi-task processing still needs improvement. Compared with natural images, the interpretation of SAR images is more challenging, which poses higher demands on the model's processing capabilities and adaptability.

\subsection{Remote Sensing Vision-Language Datasets}
% \noindent \textbf{Data Augmentation for LLMs} 
Remote sensing datasets are essential for models that interpret remote sensing imagery. Existing datasets such as UCM Captions \cite{7546397}, Sydney Captions \cite{qu2016deep}, RSICD \citep{lu2017exploring}, RSITMD \citep{yuan2022exploring}, and RSVG \citep{zhan2023rsvg} provide preliminary resources for studying the correlation between remote sensing images and text. However, these datasets are limited not only in scale but also in modality, containing only optical images without SAR data, leaving SAR interpretation capabilities largely unexplored. Although large-scale datasets like MillionAID \citep{long2021creating}, FMoW \citep{christie2018functional}, and BigEarthNet \citep{sumbul2019bigearthnet} exist, they lack text-image pairs. The RS5M  dataset \citep{zhang2023rs5m}, containing 5 million image-text pairs, is still limited to optical images. The MMRS-1M dataset \citep{zhang2024earthgpt}, which covers optical, infrared, and SAR modes, has a very low proportion of SAR image-text data. Therefore, this paper constructs the \ourmethod-2M dataset, which focuses on SAR images and contains over 2 million image-text pairs, covering tasks such as classification, detection, caption generation, VQA, and visual grounding.


\section{Data Construction and Description}

\subsection{The Procedure of Data Construction}

\subsubsection{Dataset Overall}
% \textbf{Supervised Fine-Tuning (SFT)}
% In this paper, as shown in Figure \ref{fig2-data}, \ourmethod-2M, a multi-task benchmark dataset centered on SAR images, is proposed. It contains 2 million multimodal dialogue samples, among which the training set has 1,836,912 samples and the test set has 226,636 samples, providing support for the reliability of model training and evaluation.


% This dataset covers six core tasks in the field of SAR image analysis, namely \textbf{classification}, \textbf{fine-grained description}, \textbf{instance counting}, \textbf{spatial grounding}, \textbf{cross-modal identification}, and \textbf{referring}. Specifically, it is constructed through multi-task generation and strict data screening based on the open-source SARDet-100K dataset\citep{li2024sardet100k}. On the basis of the dataset, systematic multimodal adaptation and language annotation expansion work are carried out on ten existing SAR detection benchmark datasets such as AIR-SARShip, HRSID, and MSAR. This process involves cross-modal representation learning, aiming to establish a stable correspondence between SAR images and text descriptions and achieve visual-language joint modeling. The finally formed large-scale benchmark dataset covers six key semantic categories, namely ships, tanks, bridges, ports, aircraft, and automobiles, and is accompanied by 2 million carefully collated annotations. For the detailed quantitative analysis of the dataset, please refer to Section 3.2.
% Based on the above-mentioned dataset, this study establishes the first evaluation framework for SAR remot-sensing visual-language models. This framework meets three key research requirements: first, pre-training of the basic model based on multi-task supervision; second, achieving domain adaptation across the SAR and language domains; third, conducting comprehensive performance evaluation using standardized indicators. 

As shown in Figure \ref{fig2-data}, we propose \ourmethod-2M, a multi-task dataset for SAR images, comprising 2 million multimodal dialogue samples (1,836,912 train and 226,636 test samples) to ensure robust model training and evaluation.

% The dataset encompasses six core SAR image analysis tasks: \textbf{classification}, \textbf{fine-grained description}, \textbf{instance counting}, \textbf{spatial grounding}, \textbf{cross-modal identification}, and \textbf{referring}. Built upon the SARDet-100K dataset \citep{li2024sardet100k}, it integrates multimodal adaptation and expanded language annotations from ten established SAR detection benchmarks, including AIR-SARShip, HRSID, and MSAR. Through cross-modal representation learning, the dataset establishes robust image-text correspondences across six semantic categories (ships, tanks, bridges, ports, aircraft, and automobiles) with 2 million curated annotations. Detailed quantitative analysis is presented in Section 3.2.
% The dataset covers six core SAR image analysis tasks: \textbf{classification}, \textbf{fine-grained description}, \textbf{instance counting}, \textbf{spatial grounding}, \textbf{cross-modal identification}, and \textbf{referring}. Based on the SARDet-100K dataset \citep{li2024sardet100k}, it incorporates multimodal adaptations and enhanced language annotations from ten established SAR detection benchmarks such as AIR-SARShip(1.0\&2.0)\citep{DBLP:journals/remotesensing/WangWZDW19a}, HRSID\citep{9127939}, MSAR\citep{chen2022large}, SADD\citep{rs13183690}, SAR-AIRcraf\citep{zhirui2023sar}, ShipDataset\citep{wang2019sar}, SSDD\citep{zhang2021sar}, OGSOD\citep{wang2023category}, and SIVED\citep{lin2023sived}. The dataset establishes image-text correspondences across six semantic categories (ships, tanks, bridges, ports, aircraft, and automobiles), generating 2 million carefully curated annotations for cross-modal learning.

% This study designs the first evaluation framework for SAR remote-sensing visual-language models, addressing three key requirements: multi-task supervised pre-training, cross-domain adaptation between SAR and language domains, and comprehensive performance assessment using standardized metrics.

Based on the SARDet-100K dataset \citep{li2024sardet100k}, it incorporates multimodal adaptations and enhanced language annotations from ten established SAR detection benchmarks such as AIR-SARShip(1.0\&2.0) \citep{DBLP:journals/remotesensing/WangWZDW19a}, HRSID \citep{9127939}, MSAR \citep{chen2022large}, SADD \citep{rs13183690}, SAR-AIRcraf \citep{zhirui2023sar}, ShipDataset \citep{wang2019sar}, SSDD \citep{zhang2021sar}, OGSOD \citep{wang2023category}, and SIVED \citep{lin2023sived}. The \ourmethod-2M covers six semantic categories (ships, tanks, bridges, ports, aircraft, and automobiles) and supports six core SAR image analysis tasks: \textbf{classification}, \textbf{fine-grained description}, \textbf{instance counting}, \textbf{spatial grounding}, \textbf{cross-modal identification}, and \textbf{referring}. These diverse tasks are designed to enhance VLMs' capabilities in SAR image interpretation, with 2 million carefully curated annotations for cross-modal learning.

% \vspace{-3ex}
% \begin{equation}
% L_{S} = -\mathbb{E}_{x,y\sim D^s}[\sum_{t=1}^{T}log(M_t(y_{t}|y_{<t},x)]
% \label{eq:sft}
% \end{equation}

% \vspace{-2ex}
% \begin{equation}
% % \hspace*{-0.8em}
% L_{S} = -\mathbb{E}_{D^S}[\sum_{k=1}^{T}log(M_t(y_{k}|y_{<k},x))]
% % L_{S} = -\mathbb{E}_{(x, y) \sim D^s}[\sum_{t=1}^{T}log(M_t(y_{t}|y_{<t},x))]
% \label{eq:sft}
% \end{equation}

% \noindent where $T$ represents the length of response $y$. From Equation \ref{eq:sft}, $L_{S}$ attains its minimum when the distribution of $M_t$ coincides with the conditional probability distribution $p(y|x)$ of the responses in $D^S$.

\subsubsection{Task Definition}
Based on the characteristics of SAR images and the core capabilities of the VLM, this study constructs an evaluation system consisting of six tasks. The definitions of each task are as follows:

% \noindent(1)\textbf{Classification:} As a fundamental task in SAR image interpretation, it requires the VLM to accurately discriminate target categories. This validates the model's ability to directly output semantic categories through visual perception and serves as a standard paradigm for evaluating the basic visual understanding ability of the VLM.

\noindent\textbf{(1) Classification:} Classification is a fundamental task in SAR image interpretation that evaluates the VLM's basic visual understanding through target category discrimination.

\noindent\textbf{(2) Fine-Grained Description:} The fine-grained description task focuses on both target category identification and geometric attribute analysis in SAR imagery. Beyond basic classification, it evaluates the VLM's capability to extract detailed morphological features and spatial orientations, demonstrating the model's proficiency in reasoning about SAR-specific spatial-geometric relationships.

\noindent\textbf{(3) Instance Counting:} This task requires accurate counting of multiple SAR targets while extracting their spatial coordinates and orientation information. The key challenge lies in preventing double-counting errors, particularly in complex scenes where multiple targets overlap. The model must maintain robust counting performance while handling various target densities and background complexities.

\noindent\textbf{(4) Spatial Grounding:} This task challenges the model to interpret and reason about complex spatial relationships between multiple targets in SAR imagery, including their relative positions, distances, and directional relationships. The key challenge lies in accurately understanding and describing diverse spatial configurations, especially in scenes with multiple interacting objects and varying spatial layouts. The model must demonstrate precise spatial reasoning abilities while handling complex multi-target scenarios and maintaining consistent performance across different scene compositions.


\noindent\textbf{(5) Cross-Modal Identification:} Given specified spatial coordinates, the VLM infers target attributes and generates comprehensive descriptions (size, morphology, direction, distance). This task examines the model's ability to fuse and reason about multimodal information in SAR interpretation.

\noindent\textbf{(6) Referring:} This reverse-reasoning task challenges the model to locate specific instances in SAR images from textual descriptions. The key challenge lies in bridging semantic-visual gaps while accurately determining target spatial orientations, requiring robust cross-modal reasoning capabilities across varied scene configurations.


\subsubsection{Task-Oriented Data Generation}

% \textbf{Seed data} Our seed data consists of two parts: 1.A small human labeled data which serves as an initial dataset to inform the model in sampling new data. 2.A review dataset, where the inputs are an instruction and a response integrated into a prompt template, and the outputs are the rationale for assigning a score and the final score. This dataset is designed to facilitate the model's proficiency in scoring data, thereby ensuring the seamless progression of the subsequent iteration process.

% \textbf{Seed Data} 
% \begin{enumerate}[itemsep=0pt,parsep=0pt,topsep=0pt,partopsep=0pt]
%     \item A small labeled dataset, which serves as the initial foundation for the model to sample new data.
%     \item A review dataset, where the inputs are an instruction and a response integrated into a prompt template, and the outputs are the rationale for assigning a score and the final score. This dataset is designed to facilitate the model's proficiency in scoring data, ensuring the seamless progression of the subsequent iteration process.
% \end{enumerate}
Based on the characteristics of the six tasks, this study designs a multimodal dialogue data generation scheme. The specific rules and implementation logic are as follows, with detailed templates provided in the Appendix \ref{example}:

\noindent\textbf{(0) Dataset Definitions}

Our dataset adopts a unified representation scheme across all visual-language tasks to ensure consistency and interpretability. The spatial information is uniformly encoded using the bounding box format \{<$x_1$><$y_1$><$x_2$><$y_2$>\}, where ($x_1$,$y_1$) and ($x_2$,$y_2$) denote the top-left and bottom-right coordinates respectively. Spatial relationships are structured through a standard 3×3 grid system (consisting of top-left, top, top-right, left, middle, right, bottom-left, bottom, bottom-right regions).

To explicitly specify different task requirements, we incorporate task-specific prompts: $[count]$ for Instance Counting Task, $[grounding]$ for Spatial Grounding Task, $[identify]$ for Cross-Modal Identification, and $[refer]$ for Referring Task. These prompts help guide the model's attention to the relevant aspects of each task.

These definitions form the foundational framework for our task formulations and evaluation metrics, enabling systematic assessment of visual-language models' capabilities.

\noindent\textbf{(1) Classification Task}

The Classification Task assesses the model's SAR image recognition capabilities through 20 distinct question-answer template pairs. Random template combinations enhance data diversity, with standardized notation for multi-target scenarios.

\noindent\textbf{(2) Fine-Grained Description Task}


Fine-Grained description evaluates the model's structured parsing of satellite imagery through comprehensive quality control. Following our dataset definitions, we filter images below 224×224 pixels and exclude targets with area ratio $R$ < 1\% (Equation \ref{eq:ratio}). Targets with aspect ratios exceeding 10:1 or out-of-bounds coordinates are removed. Size descriptions are categorized using area-ratio thresholds (small: <5\%, large: >30\%). We construct 40 interaction templates to accommodate multi-target scenarios. The calculation of $R$ is formulated as follows:

\begin{equation}
R = \frac{w_{box} \times h_{box}}{W_{img} \times H_{img}} \times 100%
\label{eq:ratio}
\end{equation}

where $w_{box}$ and $h_{box}$ denote the width and height of the target bounding box, respectively; $W_{img}$ and $H_{img}$ represent the width and height of the image.

\noindent\textbf{(3) Instance Counting Task}

As a fundamental component of our visual reasoning system, this task focuses on evaluating the model's quantitative counting capabilities. We designate 15 question templates with $[count]$ identifiers to specify the task requirements, while utilizing our unified bounding box format for structured output representation. The framework supports extended expressions for multi-instance scenarios through coordinate serialization. 



\noindent\textbf{(4) Spatial Grounding Task}


Spatial Grounding assessment evaluates the model's proficiency in characterizing structural relationships among multiple target objects. Leveraging our established grid system, we quantify spatial relationships through two primary mechanisms: relative distance metrics (with proximal threshold defined as $(W_{img}+H_{img})/8$) and directional relationships (encompassing horizontal, vertical, and diagonal orientations). The framework incorporates 15 spatial-relationship templates, each prefixed with $[grounding]$ identifiers, conforming to our unified spatial representation scheme.

\noindent\textbf{(5) Cross-Modal Identification} 

Cross-modal parsing evaluation employs a three-tier feature description system. Spatial positioning utilizes a 3×3 grid partitioning scheme for orientation description. Quantitative classification encompasses five-level size descriptions based on area-ratio  $R$ thresholds ($\geq$0.4:very large; $\geq$0.25:large; $\geq$0.1:medium; $\geq$0.03:small; $<$0.03:very small) and morphological analysis through bounding-box aspect ratios ($>$1.5:wide-body; 0.67$\leq$ratio$\leq$1.5:approximately square; $<$0.67:tall-body).

Feature integration combines spatial-size-morphological elements into comprehensive target profiles. The system implements 20 differential response templates with a dedicated $[identify]$ instruction identifier and structured output templates.

\noindent\textbf{(6) Referring Task}

Referring evaluates cross-modal correlation capabilities between natural language and image regions. Queries follow the pattern "Where is the \{category\}?", prefixed with $[refer]$ identifiers. The task outputs both precise bounding box coordinates and grid-based orientation descriptions, adhering to our unified spatial representation framework through nested parenthetical notation.

\subsection{Quantitative Analysis of Datasets} \label{sec:data_construction}
% \subsubsection{Analysis of Datasets Composition}
% \textbf{Review Seed data} 
% The dataset constructed in this study contains 198,676 training samples and 24,197 test samples, with a training/test split ratio of 8.1:1. To gain in-depth insights into the characteristics of the dataset, word-frequency analysis is first conducted on the model's response texts.  For details, please refer to \ref{fig4-word} in the Appendix.  Based on this preliminary finding, further quantitative analyses will be carried out from two dimensions: category distribution and object  morphology. 
The quantitative analysis in this study focuses on two key dimensions: category distribution and object morphological patterns.
% \begin{table*}[h]
%     \centering
%     \begin{tabular}{ccccc}
%         \toprule
%         \textbf{Category} & \textbf{Training Samples} & \textbf{Training Proportion} & \textbf{Test Samples} & \textbf{Test Proportion}  \\ 
%         \midrule
%         Ship & 93,373 & 46.98\% & 10,741 & 44.38\% \\
%         Aircraft & 40,705 & 20.48\% & 6,779 & 28.01\% \\
%         Car & 9,561 & 4.81\% & 1,230 & 5.08\% \\
%         Tank & 24,187 & 12.17\% & 1,773 & 7.33\% \\
%         Bridge & 27,615 & 13.89\% & 3,281 & 13.56\% \\
%         Harbor & 3,306 & 1.66\% & 399 & 1.65\% \\
%         \bottomrule
%     \end{tabular}
%     \caption{Category Distribution Statistics}
%     \label{tab1-clases}
% \end{table*}

% \begin{table}[h]
%     \centering
%     \small
%     \caption{Category Distribution Statistics}
%     \label{tab1-clases}
%     \begin{tabular}{ccc}
%         \toprule
%         \textbf{Category} & \textbf{Training} & \textbf{Test}  \\ 
%         \midrule
%         Ship & 93,373 (46.98\%) & 10,741 (44.38\%) \\
%         Aircraft & 40,705 (20.48\%) & 6,779 (28.01\%) \\
%         Car & 9,561 (4.81\%) & 1,230 (5.08\%) \\
%         Tank & 24,187 (12.17\%) & 1,773 (7.33\%) \\
%         Bridge & 27,615 (13.89\%) & 3,281 (13.56\%) \\
%         Harbor & 3,306 (1.66\%) & 399 (1.65\%) \\
%         \bottomrule
%     \end{tabular}
% \end{table}
\begin{table}[h]
    \centering
    \small

    \setlength{\heavyrulewidth}{1.5pt}
    \setlength{\lightrulewidth}{1.5pt}
    \begin{tabular}{ccc}
        \toprule
        \textbf{Category}&\textbf{Training}&\textbf{Test}\\ 
        \midrule
        Ship&93,373 (46.98\%)&10,741 (44.38\%)\\
        Aircraft&40,705 (20.48\%)&6,779 (28.01\%)\\
        Car&9,561 (4.81\%)&1,230 (5.08\%)\\
        Tank&24,187 (12.17\%)&1,773 (7.33\%)\\
        Bridge&27,615 (13.89\%)&3,281 (13.56\%)\\
        Harbor&3,306 (1.66\%)&399 (1.65\%)\\
        \bottomrule
    \end{tabular}
        \caption{Category Distribution Statistics}
    \label{tab1-clases}
\end{table}
\textbf{(1) Category Distribution Characteristics}

% The dataset constructed in this study exhibits distinct characteristics in the distribution of target categories. As can be seen from Table \ref{tab1-clases}, it has a significant long-tailed distribution feature. The ship category dominates both the training set and the test set, accounting for 46.98\% and 44.38\% respectively, which reflects that the number of samples of this category in the dataset is extremely abundant. 

% In sharp contrast, the harbor category accounts for less than 2\% in both the training set and the test set, belonging to the category with scarce samples. The proportion of the aircraft category in the test set has increased by 7.53\% compared with that in the training set. This change indicates that the distributions of different categories in the training set and the test set are not entirely consistent. Such differences may pose challenges to the generalization ability of the model and require special attention during the model training and evaluation processes. In addition, the proportions of categories such as cars, tanks, and bridges in the training set and the test set are at a medium level and relatively stable. Overall, the category distribution of this dataset is unbalanced, which not only brings challenges to the research on tasks such as target detection and recognition of different categories, but also provides an opportunity for the research on how to handle unbalanced data and improve the performance of the model for minority categories.

As shown in Table \ref{tab1-clases}, the ship category dominates both training and test sets (46.98\% and 44.38\% respectively), while the harbor category represents less than 2\%. A significant distribution shift is observed in the aircraft category, with a 7.53\% increase in the test set compared to the training set. Categories such as cars, tanks, and bridges maintain moderate and stable proportions across both sets. This class distribution aligns with real-world SAR imagery characteristics, where certain target types naturally appear more frequently than others due to the inherent nature of SAR remote sensing applications and operational scenarios.

% \begin{table}[htbp]
%     \centering
%     \begin{tabular}{lrr}
%     \toprule
%     Category & Train & Test \\
%     \midrule
%     Ship & 93342 (46.98\%) & 10738 (44.38\%) \\
%     Aircraft & 40698 (20.48\%) & 6778 (28.01\%) \\
%     Car & 9561 (4.81\%) & 1230 (5.08\%) \\
%     Tank & 24154 (12.16\%) & 1771 (7.32\%) \\
%     Bridge & 27615 (13.90\%) & 3281 (13.56\%) \\
%     Harbor & 3306 (1.66\%) & 399 (1.65\%) \\
%     \midrule
%     Total & 198676 (100\%) & 24197 (100\%) \\
%     \bottomrule
%     \end{tabular}
%     \caption{Category Distribution of SARChat-2M Dataset}
%     \label{tab:sarchat_distribution}
% \end{table}



\textbf{(2) Object Morphology Analysis}

% We quantify geometric characteristics through aspect ratio (AR):
% \begin{equation}
%     AR = \frac{h_{box}}{w_{box}} \times 100\%
% \end{equation}
% Table \ref{tab2-aspect_ratio} reveals a right-skewed distribution in the training set. The test set exhibits similar distributional characteristics. The small differences in central tendency measures between the two datasets (difference rates of -2.03\% for mean and -1.69\% for median) indicate good representative of the test set. However, the standard deviation (SD) in the test set decreased by 22.82\% compared to the training set,suggesting a more concentrated data distribution with relatively lower volatility. While this difference in dispersion does not affect the fundamental characteristics of the overall distribution, it warrants attention in subsequent model evaluations.

% In addition, this study conducts an exhaustive analysis of the key morphologies of the targets. As can be seen from Figure \ref{tab3-morphological_distribution}, the distribution intervals of the key target morphologies are as follows:

% \begin{table}[htbp]
%     \centering
%     \small
%     \caption{Morphological distribution}
%     \label{tab3-morphological_distribution}
%     \setlength{\heavyrulewidth}{1.5pt}
%     \setlength{\lightrulewidth}{1.5pt}
%     \begin{tabular}{p{4cm}p{3cm}}
%     \toprule
%     Shape & Number (\%) \\
%     \midrule
%     Roughly Square & 179652 (63.03\%) \\
%     Wide & 61154 (21.47\%) \\
%     Tall & 43976 (15.44\%) \\
%     \bottomrule
%     \end{tabular}
% \end{table}

This study quantify geometric characteristics using aspect ratio (AR):
\begin{equation}
    AR = \frac{h_{box}}{w_{box}} \times 100\%
\end{equation}

\newcolumntype{C}{>{\centering\arraybackslash}X}
\begin{table}[ht]
    \centering
    \small % 减小字体大小
    \setlength{\heavyrulewidth}{1.5pt}
    \setlength{\lightrulewidth}{1.5pt}
    \begin{tabularx}{\linewidth}{lCCC}
        \toprule
        Metric & Training Set & Test Set & Diff-Rate \\
        \midrule
        Mean & 1.28 & 1.26 & -0.02 \\
        Median & 1.062 & 1.05 & -0.017 \\
        SD & 1.18 & 0.91 & -0.22 \\
        \bottomrule
    \end{tabularx}
        \caption{Aspect Ratio Distribution Comparison}
    \label{tab2-aspect_ratio}
\end{table}

As shown in Table \ref{tab2-aspect_ratio}, the differences in central tendency between training and test sets are minimal (mean: -0.02, median: -0.017). The test set exhibits a 0.22 lower standard deviation, indicating a more concentrated distribution. The key morphological distribution intervals of targets are illustrated in Appendix \ref{Morphological}.





% \label{sec:data_construction}

% \noindent\textbf{Broad-bodied (AR $\leq$ 0.67):} It accounts for 18.14\% in the training set and 17.72\% in the test set. The similar proportions imply that the samples of this morphology are stably distributed in the two datasets. This stability is conducive to the model's stable learning of its features, reducing misjudgments.

% \noindent\textbf{Nearly square-shaped (0.67 $<$ AR $\leq$ 1.5):} It accounts for 39.67\% in the training set and as high as 59.37\% in the test set. This morphology dominates in both datasets, and the proportion in the test set has significantly increased by nearly 20 percentage points compared to that in the training set. This indicates that this morphology has a high frequency of occurrence among the target morphologies. The model needs to pay special attention to it, and in practical applications, it should have a stronger ability to recognize and adapt to this morphology.

% \noindent\textbf{Tall-bodied (AR $>$ 1.5):} It accounts for 31.96\% in the training set and 22.91\% in the test set. Although the proportion in the test set has decreased, it still maintains a certain ratio, indicating that it is representative among the target samples. During the model training, it is necessary to master the features of this morphology to ensure accurate recognition.

% Overall, the distribution of the key target morphologies in the dataset is dominated by the nearly square-shaped morphology, with certain proportions of the broad-bodied and tall-bodied morphologies. This distribution provides diverse samples for model training, which is helpful for enhancing the model's generalization ability. The significant increase in the proportion of the nearly square-shaped morphology in the test set poses higher requirements for the model's performance. During model training and optimization, special attention should be paid to its recognition and classification to improve the accuracy and reliability of the model in practical applications. The stable distribution of the broad-bodied and tall-bodied morphologies in the two datasets provides a stable learning and evaluation environment for the model, facilitating the model to accurately learn their features.

% Upon further in-depth analysis, as can be seen from Table \ref{tab3-vl-models}, this study delved into the aspect ratio distributions of different target categories and unearthed geometric morphology patterns with highly distinctive features, providing crucial information for subsequent research.

% (1) The bridge category has the highest average aspect ratio, reaching 1.56. Its shape is relatively evenly distributed among the broad-bodied, nearly-square, and tall-bodied types, accounting for 18\%, 45\%, and 37\% respectively. This balanced distribution comprehensively supports research on the impact of shape on model performance, facilitating the optimization of bridge target detection and recognition algorithms, and enabling the model to maintain stable performance when dealing with bridges of different shapes.

% (2) The ship category exhibits diverse shapes, with broad-bodied, nearly-square, and tall-bodied samples accounting for 28\%, 40\%, and 32\% respectively. This diverse distribution realistically reflects the actual shape characteristics of ships, providing abundant training materials for the model and enhancing the accuracy and robustness of the model in recognizing ship targets.

% (3) The distribution of the tank category is highly concentrated, with 94\% of the samples being nearly-square, while the proportions of broad-bodied and tall-bodied samples are extremely small, at 1.6\% and 4.1\% respectively. This characteristic makes the main features of tanks prominent and stable, facilitating rapid learning by the model, improving training efficiency, reducing misjudgment rates during recognition, and thus having great application value in military target detection.

% (4) The proportion distribution of the aircraft category is stable, with 87\% of the samples being nearly-square, and the proportions of broad-bodied and tall-bodied samples being 5\% and 6\% respectively. This stable distribution reflects the regularity of aircraft shapes. The model can learn the main features of aircraft from this, maintaining a high recognition accuracy in different scenarios.

% (5) For the car and port categories, the changes are moderate, with 60\%-68\% of the samples being nearly-square, and the rest distributed among the broad-bodied and tall-bodied categories. This distribution not only reflects the shape commonalities of the two types of targets but also retains certain differences, helping the model learn their main features and enhancing its generalization ability.

% Most critically, the distribution patterns of all categories are highly consistent between the training set and the test set. This indicates that the dataset has good stability and reliability, ensuring consistent model performance on different datasets. It provides a reliable basis for model performance evaluation, enhances the credibility and practicality of the model in practical applications, and also facilitates communication and comparison among researchers, promoting the development of research in related fields.

% \noindent\textbf{Distribution Analysis of Morphological Categories}

The dataset exhibits three distinct morphological categories based on aspect ratio (AR): broad-bodied (AR $\leq$ 0.67), nearly square-shaped (0.67 $<$ AR $\leq$ 1.5), and tall-bodied (AR $>$ 1.5). Detailed distribution analysis can be found in Appendix \ref{appendix:analys-mor}.

% Nearly square-shaped morphology dominates both datasets, accounting for 39.67\% in training and 59.37\% in test sets, indicating its prevalence in target morphologies. Broad-bodied shapes maintain stable distributions (18.14\% training, 17.72\% test), while tall-bodied shapes show a moderate decrease from training (31.96\%) to test (22.91\%) sets. This distribution diversity enhances the model's generalization capability, though the significant increase in nearly square-shaped samples in the test set demands particular attention during model optimization.

% \noindent\textbf{Category-Specific Morphological Patterns}

% As shown in Table \ref{tab3-vl-models}, each target category displays distinctive morphological characteristics. Bridges exhibit the highest average aspect ratio (1.56) with balanced distribution across all morphologies (18\% broad, 45\% square, 37\% tall). Ships demonstrate diverse shapes (28\% broad, 40\% square, 32\% tall), reflecting their real-world variability. Tanks and aircraft show highly concentrated distributions, with nearly square shapes dominating at 94\% and 87\% respectively, facilitating efficient model learning. Cars and ports maintain moderate distributions with 60-68\% nearly square shapes and balanced remaining proportions.

% The consistent distribution patterns between training and test sets across all categories ensure dataset stability and reliability, providing a solid foundation for model evaluation and practical applications. This structural consistency enhances the model's performance reliability and facilitates research comparability in the field.

% As shown in Table \ref{tab3-vl-models}, bridges exhibit the highest average aspect ratio (1.56) with diverse morphologies (18\% wide, 45\% roughly square, 37\% tall), while tanks and aircraft display concentrated distributions (94\% and 87\% roughly square shapes respectively). Ships show varied shapes (28\% wide, 40\% roughly square, 32\% tall), and cars/ports maintain moderate distributions with 60-68\% roughly square shapes. The consistent distributions between training and test sets ensure dataset reliability for model evaluation.
%%%%%%%%%%%%%%%%%%%%%%%%%%%%%%%%%%%







% \begin{table*}[htbp]
% \scriptsize
% \setlength{\tabcolsep}{3.5pt}  % 默认是 6pt
% \centering
% \begin{tabular}{cccccccccccccccc}

% \hline
% \multirow{3}{*}{Model} & \multirow{3}{*}{Param} & \multirow{3}{*}{\begin{tabular}[c]{@{}c@{}}Avg\\ score\end{tabular}} & \multicolumn{13}{c}{Tasks} \\
% \cline{4-16}
%  &  &  & \multirow{2}{*}{\begin{tabular}[c]{@{}c@{}}Only\\ count\end{tabular}} & \multicolumn{2}{c}{Instance Count} & \multirow{2}{*}{\begin{tabular}[c]{@{}c@{}}Abstract\\ position\end{tabular}} & \multicolumn{2}{c}{Spatial Ground} & \multicolumn{2}{c}{Cross-Modal ID} & \multicolumn{2}{c}{Multi-target Ref} & \multicolumn{2}{c}{Single-target Ref} & \multirow{2}{*}{Class} \\
% \cline{5-6} \cline{8-15}
%  &  &  &  & IoU=.25 & IoU=.5 &  & Multi & Single & Multi & Single & IoU=.25 & IoU=.5 & IoU=.25 & IoU=.5 &  \\
% \hline
% \multirow{4}{*}{InternVL2.5} & 8B & 92.79 & 74.14 & 61.37 & 52.17 & 81.25 & 62.25 & 87.91 & 98.84 & 98.98 & 37.49 & 23.46 & 74.86 & 60.13 & 97.25 \\
%  & 4B & 91.57 & 72.68 & 57.54 & 47.35 & 83.33 & 60.89 & 85.90 & 98.01 & 98.76 & 34.05 & 18.86 & 69.92 & 55.29 & 97.27 \\
%  & 2B & 90.55 & 71.52 & 54.11 & 44.22 & 50.00 & 60.81 & 81.92 & 97.79 & 98.63 & 27.05 & 13.91 & 68.50 & 52.16 & 96.69 \\
%  & 1B & 88.89 & 69.87 & 50.18 & 39.35 & 0.00 & 56.30 & 82.24 & 96.98 & 98.60 & 22.13 & 9.94 & 62.33 & 44.99 & 96.65 \\
% \hline
% \multirow{2}{*}{DeepSeekVL} & 7B & 88.99 & 20.66 & 8.49 & 4.19 & 64.29 & 65.32 & 85.78 & 98.97 & 99.05 & 28.75 & 13.66 & 64.34 & 48.84 & 93.23 \\
%  & 1.3B & 84.01 & 19.61 & 4.00 & 1.32 & 75.00 & 60.38 & 82.00 & 96.40 & 97.45 & 16.11 & 6.23 & 53.58 & 34.28 & 47.37 \\
% \hline
% Phi-3.5-vision & 4.2B & 92.06 & 72.69 & 57.48 & 47.60 & 62.50 & 58.85 & 87.29 & 98.93 & 98.59 & 31.65 & 17.16 & 70.95 & 55.70 & 96.42 \\
% \hline
% \multirow{2}{*}{GLM-Edge-V} & 2B & 90.20 & 71.59 & 51.97 & 40.37 & 42.86 & 59.15 & 86.33 & 97.54 & 98.60 & 24.15 & 10.66 & 65.57 & 46.46 & 97.39 \\
%  & 5B & 90.48 & 73.44 & 56.30 & 44.56 & 75.00 & 61.38 & 89.96 & 96.69 & 95.96 & 30.68 & 15.41 & 69.36 & 34.28 & 47.37 \\
% \hline
% \multirow{3}{*}{mPLUG-Owl3} & 2B & 90.32 & 67.56 & 41.56 & 28.83 & 75.00 & 45.65 & 97.58 & 98.95 & 99.42 & 14.91 & 5.42 & 50.46 & 30.16 & 98.31 \\
%  & 7B & 91.71 & 71.00 & 48.07 & 35.27 & 100.00 & 56.37 & 93.32 & 99.27 & 99.51 & 19.72 & 7.66 & 57.27 & 38.00 & 98.80 \\
%  & 1B & 89.68 & 67.03 & 38.64 & 24.98 & 75.00 & 44.07 & 97.19 & 98.72 & 98.87 & 11.86 & 4.12 & 44.34 & 24.02 & 98.06 \\
% \hline
% \multirow{2}{*}{Qwen2-VL} & 7B & 90.76 & 72.79 & 58.51 & 50.24 & 0.00 & 64.17 & 83.87 & 97.54 & 99.18 & 39.11 & 26.29 & 70.55 & 57.04 & 97.30 \\
%  & 2B & 90.27 & 69.63 & 53.62 & 45.47 & 50.00 & 59.04 & 78.49 & 97.55 & 99.26 & 32.60 & 20.12 & 65.31 & 51.53 & 96.88 \\
% \hline
% LLaVA & 7B & 91.21 & 71.89 & 56.89 & 46.80 & 57.14 & 62.70 & 85.79 & 97.84 & 98.42 & 30.81 & 15.48 & 71.89 & 56.70 & 96.90 \\
% \hline
% Yi-VL & 6B & 84.01 & 19.61 & 4.00 & 1.32 & 75.00 & 60.38 & 82.00 & 96.40 & 97.45 & 16.11 & 6.23 & 53.58 & 34.28 & 47.37 \\
% \hline
% \end{tabular}
% \caption{Performance comparison of different vision-language models}
% \label{tab3-vl-models}
% \end{table*}

\begin{table*}[htbp]
\scriptsize
\setlength{\tabcolsep}{2.5pt}

\centering
\begin{tabular}{ccccccccccccccccc}
\Thickline
\multirow{3}{*}{\textbf{Model}} & \multirow{3}{*}{\textbf{Param}} & \multirow{3}{*}{\begin{tabular}[c]{@{}c@{}}\textbf{Avg}\\ \textbf{score}\end{tabular}} & \multicolumn{14}{c}{\textbf{Tasks}} \\
\cline{4-17}
 &  &  & \multirow{2}{*}{\begin{tabular}[c]{@{}c@{}}\textbf{Only}\\ \textbf{count}\end{tabular}} & \multicolumn{2}{c}{\textbf{Instance Count}} & \multirow{2}{*}{\begin{tabular}[c]{@{}c@{}}\textbf{Abstract}\\ \textbf{position}\end{tabular}} & \multicolumn{2}{c}{\textbf{Spatial Ground}} & \multicolumn{2}{c}{\textbf{Cross-Modal ID}} & \multicolumn{2}{c}{\textbf{Multi-target Ref}} & \multicolumn{2}{c}{\textbf{Single-target Ref}} & \multirow{2}{*}{\textbf{Descript}} & \multirow{2}{*}{\textbf{Class}} \\
\cline{5-6} \cline{8-9} \cline{10-11} \cline{12-13} \cline{14-15}
 &  &  &  & \textbf{IoU=.25} & \textbf{IoU=.5} &  & \textbf{Multi} & \textbf{Single} & \textbf{Multi} & \textbf{Single} & \textbf{IoU=.25} & \textbf{IoU=.5} & \textbf{IoU=.25} & \textbf{IoU=.5} &  &  \\
\Thickline
\multirow{4}{*}{\textbf{InternVL2.5}} & \textbf{8B} & 92.79 & \textbf{74.14} & 61.37 & 52.17 & 81.25 & 62.25 & 87.91 & 98.84 & 98.98 & 37.49 & 23.46 & 74.86 & 60.13 & 63.43 & 97.25 \\
 & \textbf{4B} & 91.57 & 72.68 & 57.54 & 47.35 & 83.33 & 60.89 & 85.90 & 98.01 & 98.76 & 34.05 & 18.86 & 69.92 & 55.29 & 58.84 & 97.27 \\
 & \textbf{2B} & 90.55 & 71.52 & 54.11 & 44.22 & 50.00 & 60.81 & 81.92 & 97.79 & 98.63 & 27.05 & 13.91 & 68.50 & 52.16 & 56.36 & 96.69 \\
 & \textbf{1B} & 88.89 & 69.87 & 50.18 & 39.35 & 0.00 & 56.30 & 82.24 & 96.98 & 98.60 & 22.13 & 9.94 & 62.33 & 44.99 & 53.30 & 96.65 \\
\hline
\multirow{2}{*}{\textbf{DeepSeekVL}} & \textbf{7B} & 88.99 & 20.66 & 8.49 & 4.19 & 64.29 & 65.32 & 85.78 & 98.97 & 99.05 & 28.75 & 13.66 & 64.34 & 48.84 & 51.08 & 93.23 \\
 & \textbf{1.3B} & 84.01 & 19.61 & 4.00 & 1.32 & 75.00 & 60.38 & 82.00 & 96.40 & 97.45 & 16.11 & 6.23 & 53.58 & 34.28 & 44.44 & 47.37 \\
\hline
\textbf{Phi-3.5-vision} & \textbf{4.2B} & 92.06 & 72.69 & 57.48 & 47.60 & 62.50 & 58.85 & 87.29 & 98.93 & 98.59 & 31.65 & 17.16 & 70.95 & 55.70 & 59.95 & 96.42 \\
\hline
\multirow{2}{*}{\textbf{GLM-Edge-V}} & \textbf{2B} & 90.20 & 71.59 & 51.97 & 40.37 & 42.86 & 59.15 & 86.33 & 97.54 & 98.60 & 24.15 & 10.66 & 65.57 & 46.46 & 57.86 & 97.39 \\
 & \textbf{5B} & 90.48 & 73.44 & 56.30 & 44.56 & 75.00 & 61.38 & 89.96 & 96.69 & 95.96 & 30.68 & 15.41 & 69.36 & 51.81 & 61.45 & 98.02 \\
\hline
\multirow{3}{*}{\textbf{mPLUG-Owl3}} & \textbf{7B} & 91.71 & 71.00 & 48.07 & 35.27 & \textbf{100.00} & 56.37 & 93.32 & \textbf{99.27} & \textbf{99.51} & 19.72 & 7.66 & 57.27 & 38.00 & 54.65 & 98.80 \\
& \textbf{2B} & 90.32 & 67.56 & 41.56 & 28.83 & 75.00 & 45.65 & 97.58 & 98.95 & 99.42 & 14.91 & 5.42 & 50.46 & 30.16 & 41.76 & 98.31 \\
 & \textbf{1B} & 89.68 & 67.03 & 38.64 & 24.98 & 75.00 & 44.07 & 97.19 & 98.72 & 98.87 & 11.86 & 4.12 & 44.34 & 24.02 & 40.16 & 98.06 \\
\hline
\multirow{2}{*}{\textbf{Qwen2-VL}} & \textbf{7B} & 90.76 & \textbf{72.79} & 58.51 & 50.24 & 0.00 & 64.17 & 83.87 & 97.54 & 99.18 & 39.11 & 26.29 & 70.55 & 57.04 & 63.11 & 97.30 \\
 & \textbf{2B} & 90.27 & 69.63 & 53.62 & 45.47 & 50.00 & 59.04 & 78.49 & 97.55 & 99.26 & 32.60 & 20.12 & 65.31 & 51.53 & 55.20 & 96.88 \\
\hline
\textbf{LLaVA-1.5} & \textbf{7B} & 91.21 & 71.89 & 56.89 & 46.80 & 57.14 & 62.70 & 85.79 & 97.84 & 98.42 & 30.81 & 15.48 & 71.89 & 56.70 & 61.35 & 96.90 \\
\hline
\textbf{Yi-VL} & \textbf{6B} & 84.35 & 32.62 & 14.35 & 9.44 & 75.00 & 53.68 & 72.38 & 93.63 & 97.95 & 7.76 & 2.69 & 32.95 & 16.63 & 38.15 & 95.32 \\
\Thickline
\end{tabular}
\caption{Performance comparison of different vision-language models}
\label{tab3-vl-models}
\end{table*}

\section{SARChat-Bench Evaluation Method and Settings} 

This section details the evaluation methodology of \ourmethod-Bench, a standardized benchmark suite we designed for comprehensive assessment of VLMs in SAR interpretation. The benchmark covers six fundamental tasks that evaluate the model's core capabilities across information processing, target localization, and semantic understanding, providing multi-dimensional insights into visual-language model performance in the SAR domain. The evaluation framework ensures fair and thorough assessment of VLMs' capabilities across different SAR interpretation scenarios.

\subsection{Evaluation Metrics}

% \noindent(1) \textbf{Accuracy:} As a core evaluation metric, accuracy can intuitively and concisely reflect the degree of fit between the model's prediction results and the actual situation. Its calculation formula is as follows:

% \begin{equation}
%     Acc = \frac{TP}{TP + FP + FN} \times 100\%
% \end{equation}

% Here, TP (True Positive) is the number of samples the model correctly predicts as positive, reflecting its accurate recognition of positive samples. FP (False Positive) is the number of samples wrongly predicted as positive, showing misjudgment. FN (False Negative) is the number of samples wrongly predicted as negative, indicating omission of positive samples. This metric quantifies the model's overall prediction accuracy.

\noindent\textbf{(1) Accuracy:} A core metric reflecting model prediction fit, calculated as:

\begin{equation}
Acc = \frac{TP}{TP + FP + FN} \times 100\%
\end{equation}

where $TP$ denotes correct positive predictions, $FP$ represents false positive predictions, and $FN$ indicates false negative predictions.

\noindent\textbf{(2) Intersection over Union (IoU):} In tasks involving localization, identification, and reference, IoU is a key metric measuring the overlap between predicted and ground-truth bounding boxes (bbox). Higher IoU values indicate greater overlap and better localization performance. All IoU-related calculations in this paper are performed with thresholds of 0.25 and 0.5.

\noindent\textbf{(3) Overall Score Calculation:} 

\begin{equation}
S_m = \sum_{t \in T} a_{m,t} \times \frac{n_t}{\sum_{i \in T} n_i}
\end{equation}

Among them, $n_t$ represents the sample size of task $t$, $a_{m,t}$ denotes the accuracy of model $m$ on task $t$, and $T$ is the set of all tasks. The detailed calculation of $a_{m,t}$ for each task can be found in Appendix \ref{task-cal}.

% 这个是最简化的形式,直接展示模型总加权得分 $S_m$ 的计算方式,其中 $a_{m,t}$ 是模型 $m$ 在任务 $t$ 上的准确率,$n_t$ 是任务 $t$ 的计数,$T$ 是所有任务的集合。



\subsection{Assessment Methods}

This section elaborates on the specific evaluation method processes for six types of tasks.

% \noindent(1) \textbf{Instance Counting:} The key to the counting task is to obtain the number predicted by the model and the actual number in the label. Since it is a single class, the performance of the model can be intuitively evaluated by comparing the matching degree of the two.

% \noindent(2) \textbf{Spatial Grounding:}  Extract the bbox and category information of the model prediction and the label, calculate the accuracy rate based on IoU, and evaluate the recognition accuracy of the target position and category. At the same time, with the help of natural language processing technology, extract sentences containing multiple categories and abstract position information (such as "top", "bottom", etc.) from the model prediction and the label, compare the predicted abstract position with the true abstract position, calculate according to the accuracy formula, and comprehensively evaluate the positioning accuracy of the model from multiple dimensions.

% \noindent(3) \textbf{Cross-Modal Identification:} Extract the bbox of the model prediction and the label (the label for single-target and the label set for multi-target), and calculate the accuracy rate through IoU. Evaluate the single-target cross-modal identification performance in the single-target case, and in the multi-target case, compare the overlap between the predicted bounding box and the true bounding box to measure the cross-modal identification performance in complex scenarios.

% \noindent(4) \textbf{Referring:}Extract the bbox of the model prediction and the label (the label for single-target and the label set for multi-target), and calculate the accuracy rate based on IoU. Evaluate the single-target referring performance in the single-target case, and in the multi-target case, analyze multiple sets of data to comprehensively evaluate the performance and stability.

% \noindent(5) \textbf{Fine-Grained Description:} First, use a specific text segmentation algorithm to segment the long sentences in the model prediction and the label into short sentences. Then, use information extraction technology to extract category and position information from each short sentence and organize it into a set. Finally, compare the predicted set with the true set, calculate according to the accuracy formula, and evaluate the performance of the model.

% \noindent(6) \textbf{Classification:}

% Extract the categories predicted by the model and the true label categories, strictly compare them, calculate the classification accuracy rate according to the accuracy formula, accurately measure the accuracy and reliability of the model, and provide a key basis for the performance evaluation of the model.

\noindent(1) \textbf{Instance Counting:} Compare predicted and label object counts for single-class evaluation, where counting accuracy is measured by $Acc$ and object localization precision is evaluated using Intersection over Union ($IoU$).

\noindent\textbf{(2) Spatial Grounding:} Evaluate spatial accuracy through $IoU$-based bbox matching and abstract position analysis (e.g., "top", "bottom") from natural language descriptions.

\noindent\textbf{(3) Cross-Modal Identification:} Calculate $IoU$ between predicted and ground-truth bboxes for both single and multiple target scenarios to assess cross-modal matching capability.

\noindent\textbf{(4) Referring:} Assess referring accuracy through $IoU$ metrics in both single-target and multi-target contexts.

\noindent\textbf{(5) Fine-Grained Description:} Segment predictions and ground-truth into short phrases, extract category and position information, and compare content sets for detailed description evaluation.

\noindent\textbf{(6) Classification:} Compare predicted and ground-truth categories to assess classification accuracy $Acc$.

\section{Experiments and Analysis}

\subsection{Implementation Details}

% During the fine-tuning stage, a prudent training strategy is adopted. The model is trained for 1 epoch, and the batch size is set to 4 (the actual effective batch size is 32, with 2-GPU training considered and gradient accumulation steps set to 4). To optimize the model's performance, the LoRA (Low-Rank Adaptation) training method is employed, with parameter configurations of rank = 8 and alpha = 32. LoRA is applied to all linear layers in the model. 

% The initial learning rate is set to 1e-4, and a learning rate warm-up strategy with a ratio of 0.1 is implemented. To enhance training efficiency, the gradient checkpointing technology is enabled. All experiments are conducted on 2 NVIDIA A100 GPUs, and the training is carried out using bfloat16. 

During the fine-tuning stage, the model is trained for 1 epoch with a batch size of 4 (effective batch size 32 with gradient accumulation steps of 4). This study employs LoRA training method with rank=8 and alpha=32 on all linear layers. The learning rate is initialized at 1e-4 with a 0.1 warm-up ratio. All experiments are conducted on 2 NVIDIA A100 GPUs using bfloat16 precision. Training is implemented using the MS-SWIFT framework \citep{zhao2024swiftascalablelightweightinfrastructure} for efficient distributed training. In the most time-consuming case, the training process for a single model took up to 192 hours to complete.

\subsection{Benchmark Evaluation}
% To verify the effectiveness and practicality of the SARChat-2M dataset, this study conducts a series of benchmark experiments on trained models based on this dataset. Sixteen mainstream visual-language models, such as Qwen2VL, InternVL2.5, and LLaVA, are selected. These models possess multiple capabilities, adopt different architectures and training methods, and all of them are open-source, which is conducive to customization and optimization. In experiments of different tasks, each model exhibits diverse characteristics:


To verify the effectiveness and practicality of the \ourmethod-2M dataset, we conducted extensive experiments on SAR image interpretation tasks using sixteen mainstream visual-language models. Our preliminary analysis reveals that these models, despite their strong performance on natural images, struggle significantly with SAR image interpretation without domain-specific fine-tuning, demonstrating the critical importance of SAR-domain adaptation. Specifically, we conducted a detailed before-and-after analysis on InternVL2-8B, which achieved the best performance among all tested models, to quantitatively demonstrate the impact of SAR fine-tuning. The results are presented in Appendix~\ref{ftornot}.

As shown in Table \ref{tab3-vl-models}, the evaluated models include recent advances such as Qwen2-VL\citep{wang2024qwen2}, InternVL2.5\citep{chen2024expanding}, DeepSeekVL\citep{lu2024deepseekvl}, Phi-3.5-vision\citep{abdin2024phi}, GLM-Edge-V\citep{glm2024chatglm}, mPLUG-Owl3\citep{ye2024mplug}, Yi-VL\citep{young2024yi} and LLaVA-1.5\citep{liu2023llava}. In experiments of different tasks, each model exhibits diverse characteristics:

% \noindent(1)\textbf{Instance Counting:} This task requires the model to accurately identify the number of specific objects in the image. According to the experimental data, the InternVL2.5 series of models perform relatively outstandingly in this task, with the highest accuracy rate reaching 74.14\%. The highest accuracy rate of the QwenVL2 series of models is 72.79\%. However, the accuracy rates of most other models are lower than 60\%, which not only reflects that the counting task still poses great challenges to some models but also indicates that this dataset can well distinguish the performance of different models in this task.
\noindent\textbf{(1) Instance Counting} requires VLMs to identify the number of specific objects in the image. Two leading model families achieve state-of-the-art performance: InternVL2.5 and Qwen-VL2, reaching accuracies of 74.14\% and 72.79\% respectively. However, the accuracy of most other models fall below 60\%, highlighting both the challenging nature of the counting task and the dataset's effectiveness in differentiating model capabilities.

% \noindent(2)\textbf{Spatial Grounding:} This task mainly tests the model's ability to determine the spatial position of objects in the image. In the abstract position description section, the 7B model of the mPLUG-Owl3 series performs excellently, with an accuracy rate of 100\%, leading other series of models by a large margin. In terms of single-target positioning, the entire mPLUG-Owl3 series performs optimally, with the precision rate exceeding 90\%, while the positioning accuracy rates of other series of models are in the range of 80\%-85\%. In the multi-target positioning scenario, the accuracy rates of most models are approximately around 60\%. This indicates that accurately processing the spatial information of multiple targets is still a key area that many models need to focus on improving.

\noindent\textbf{(2) Spatial Grounding} evaluates models' capability in spatial localization. For abstract position descriptions, mPLUG-Owl3-7B achieves 100\% accuracy, significantly outperforming other models. The mPLUG-Owl3 family maintains superior performance (>90\%) in single-target localization, while other models achieve 80\%-85\%. However, in multi-target scenarios, most models' accuracy drops to approximately 60\%. These results suggest that accurate multi-target spatial information processing remains a crucial area for future model improvements.

\begin{figure*}[th]
    \centering    \includegraphics[width=0.73\linewidth]{figures/9.pdf}
    \caption{\textbf{Evaluation examples on \ourmethod-Bench.} VLM predictions are shown in green/red for correct/incorrect descriptions, with the ground truth in green and the predictions in red boxes. And [Human], [Bot], and [Check] icons denote user input, VLMs response, and standard output, respectively.}
    \label{fig9-sample}
    % \vspace{-1em}
\end{figure*}

\noindent\textbf{(3) Cross-Modal Identification} focuses on the model's ability to build precise connections between visual information and other modal information. In this experiment, the process from image recognition to text description is mainly concerned. The experimental data shows that for both single-target and multi-target tasks, the accuracy rates of most models exceed 90\%. Among them, the mPLUG-Owl3-7B model performs the best, with the accuracy rates of single-target and multi-target tasks reaching 99.27\% and 99.51\% respectively, fully demonstrating the powerful capabilities of large language models in cross-modal identification tasks.

% \noindent(4)\textbf{Referring:} The referring task requires the model to accurately locate the corresponding objects in the image based on the text description. The experimental evaluation shows that in the single-target referring task, the accuracy rates of all series of models are lower than 75\%; in the multi-target referring task, the accuracy rates are even lower than 40\%. This clearly indicates that most current models have significant deficiencies in processing the alignment and correlation between language information and visual information, and it is difficult to establish an accurate correspondence between text descriptions and specific objects in the image.

\noindent\textbf{(4) Referring} challenges models to precisely locate objects in SAR images based on textual descriptions. Our experiments reveal significant performance gaps: models achieve less than 75\% accuracy on single-target tasks and below 40\% on multi-target scenarios. These results highlight the current limitations in cross-modal alignment, particularly in establishing precise text-to-object correspondences within SAR imagery.

\noindent\textbf{(5) Fine-Grained Description} requires the model to provide detailed feature and attribute descriptions of the objects in the image. The experiment shows that the model accuracy rates are in the range of 40\%-63\%. Among them, models with larger parameters such as Qwen2-VL-7B and InternVL2.5-8B perform outstandingly and can give more detailed and accurate descriptions. In contrast, other models with smaller parameter sizes perform poorly, indicating that the accuracy rate of the fine-grained description task is highly sensitive to the model's parameter size.

% \noindent(6)\textbf{Classification:} The image classification task tests the model's ability to classify images into the correct categories according to their content. According to the table data, regardless of the parameter size, the average accuracy rates of series such as InternVL2.5, mPLUG-Owl3, Qwen2VL, as well as some models, exceed 96\%. The performance of large language models can be comparable to that of traditional CNN models.

\noindent\textbf{(6) Classification} evaluates models' ability to categorize images based on their content. According to the table data, regardless of parameter size, series such as InternVL2.5, mPLUG-Owl3, Qwen2-VL, and several other models achieve accuracy rates exceeding 96\%. The performance of these VLMs demonstrates competitiveness with traditional vision classification models.

\noindent \textbf{Summary:} We benchmark 16 mainstream VLMs on \ourmethod-Bench. Model size strongly affects fine-grained description performance but shows little impact on classification. While large models excel in cross-modal and class identify tasks and basic spatial grounding, they struggle with referring, counting, detailed descriptions, and multi-target spatial relationships.


\subsection{Edge-side models for SAR Applications}

% This study has open-sourced multiple edge- side small models with 1B parameters trained on \ourmethod-2M and provided the baselines of state -of-the-art open-source models. According to the data analysis in Table 4, small models with around 1B parameters exhibit diverse performance in different tasks. They perform excellently in cross-modal identification tasks, with the average accuracies of multi-object and single-object being approximately 96.98\% and 98.60\%, respectively. However, there is still room for improvement in some tasks such as referring tasks. In the future, users can add domain data for further training on these models to quickly adapt to different task requirements and reduce the development threshold. Meanwhile, after optimization, these edge-side small models can operate efficiently on satellite or ground-side edge devices, enabling real-time processing and analysis of SAR data, reducing the dependence on cloud or ground stations, and lowering data transmission costs and computational resource consumption.
This study has multiple edge-side models ($\leq$5B parameters) trained on \ourmethod-2M and evaluates their performances. According to Table \ref{tab3-vl-models}, it demonstrates that these models exhibit task-specific performance variations, achieving remarkable accuracy in cross-modal identification, while showing potential for improvement in referring tasks. These models support domain-specific fine-tuning for rapid task adaptation. After optimization, they can operate efficiently on satellite or ground-edge devices, enabling real-time SAR data processing while reducing dependence on cloud infrastructure and minimizing data transmission costs.
 


\subsection{Dialogue Visualizaion}

% Figure \ref{fig9-sample} shows the evaluation examples on the \ourmethod-Benchmark. It presents six different tasks and their related results. Each task section is presented in the form of a dialogue, with the question on the upper side and the model's response along with the corresponding verification result (marked with a green tick for correctness) on the lower side. For example, in Task 1, the question is "Looking at this image, which of the following categories are present: aircraft, bridge, car, harbor, ship, tank?", and the model responds "It is ship.", receiving feedback of being correct. Another example is Task 3, the instance-counting task, where the question is "[count] Count the ship in this satellite image.", and the model's response "There are 4 instances. \{...\}" is also verified as correct.

Figure \ref{fig9-sample} presents examples across six tasks from \ourmethod-Bench. The model completes these tasks with reasonable performance, and its coordinate predictions align with the ground truth annotations. In the spatial grounding task, the model identifies an additional ship that was not included in the original annotations, suggesting its potential in detecting previously unmarked targets in SAR imagery.

\section{Conclusion}

% In this paper, the SARChat-2M dataset is constructed. Comprising around two million high-quality image-text pairs, it encompasses diverse scenes and is annotated with fine-grained features. This dataset can support multiple tasks, thereby promoting the development of  VLMs in SAR remote sensing. Meanwhile, the SARChat-Bench benchmark testing framework is introduced, offering a systematic evaluation system for VLMs in the SAR domain. This framework overcomes the limitations of VLMs in integrating SAR remote-sensing knowledge, facilitating the further advancement of VLMs in this field. 
This research introduces \ourmethod-2M, a large-scale dataset of two million annotated SAR image-text pairs, addressing the scarcity of language-vision data in the SAR domain. The accompanying \ourmethod-Bench provides a systematic evaluation framework for assessing VLMs in SAR interpretation tasks, facilitating domain-specific knowledge integration and accelerating the development of SAR-oriented VLMs.

\newpage

\section*{Limitation}
Despite the comprehensive scale of \ourmethod-2M based on SARDet-100K dataset, the inherent annotation inconsistencies across different SAR data sources may lead to potential limitations. The varying annotation quality could result in missing targets or imprecise target delineation. Notably, there exist cases where VLMs successfully identify valid targets that were not originally annotated in the dataset, highlighting the annotation completeness challenge in the current benchmark construction.

\section*{Ethics Statement}
In this study, all SAR datasets and methodologies are used strictly for academic research purposes, adhering to their respective licenses and data usage agreements. While our research aims to advance the fundamental understanding of SAR image interpretation, we acknowledge that these technologies could potentially be applied to military or defense-related purposes. We emphasize that the responsible application of such technologies is crucial, and their deployment should strictly comply with relevant regulations and ethical guidelines. The research community should maintain ongoing discussions regarding the dual-use nature of SAR technologies to ensure their development serves beneficial purposes while minimizing potential misuse.

\newpage

%\section*{Acknowledgments}

%We thank the anonymous reviewers for their valuable comments. The work presented herein is gratefully acknowledged to be supported in part by the National Natural Science Foundation of China (Grant No. 62271153) and the Natural Science Foundation of Shanghai (Grant No. 22ZR1406700). 


\bibliography{custom}

% Custom bibliography entries only
% \bibliography{custom}



\clearpage
\appendix
\addcontentsline{toc}{chapter}{Appendix}
\section*{Appendix}

\section{Data structure analysis}

\subsection{Word Frequency Analysis of SARChat-2M}
As shown in Figure 4 , location words (such as center, middle, top) and target object words (such as ship, aircraft, tank) have the highest occurrence frequencies in SAR image descriptions, and the adjective "small" is the most frequently used descriptive word.
\begin{figure}[h]
    \centering
    \includegraphics[width=0.8\linewidth]{figures/3.png}
    \caption{Cloud Map of Word-frequency Distribution }
    \label{fig3-word}
\end{figure}

% \section{Data Content Analysis}
% \label{sec:app_data}

% Content analysis of data, for why it works on some aspects.
\begin{figure}[h]
    \centering
    \includegraphics[width=1.0\linewidth]{figures/4.png}
    \caption{The Proportion Distribution of Samples in the Training Set}
    \label{fig4-train-classes}
\end{figure}
\begin{figure}[h]
    \centering
    \includegraphics[width=1.0\linewidth]{figures/5.png}
    \caption{The Proportion Distribution of Samples in the Testing Set}
    \label{fig5-test-classes}
\end{figure}
% \noindent\textbf{(1)Analysis of Datasets Composition}
\subsection{Analysis of Datasets Composition}
\label{Morphological}


\noindent\textbf{Training Set Category Distribution}

As shown in Figure \ref{fig4-train-classes}, in the \ourmethod-2M training set, the category distribution shows significant differences. Among them, the "Ship" category has the largest proportion, reaching 46.98\%, followed by the "Aircraft" category, with a proportion of 20.48\%. These two categories account for the majority of the samples in the training set. It can be seen that the sample distribution in the training set is imbalanced, and the "Ship" and "Aircraft" categories dominate. This may enable the model to learn the features of these two categories more comprehensively during the training process. However, since the samples of other categories are relatively few, the model's ability to learn and generalize their features may be affected to a certain extent.

\noindent\textbf{Test Set Category Distribution}

As shown in Figure \ref{fig5-test-classes}, in the \ourmethod-2M test set, the "Ship" category has the largest proportion among all categories, reaching 44.38\%, and the "Aircraft" category ranks second, with a proportion of 28.01\%. The distribution trends of these two major categories in the test set are similar to those in the training set. This indicates that the test set has a certain similarity to the training set in terms of the overall category distribution and can be used to test the model's generalization ability on data with a similar distribution. However, the slight differences in the proportions also remind us to comprehensively consider various factors when evaluating the model's performance.

\noindent\textbf{Cross-modal Shape Distribution Analysis}

As shown in Figure \ref{fig6-cross_model}, in the cross-modal identification shape distribution, the "Roughly Square" shape has the largest proportion, with a quantity of 179,652. This shape has an absolute advantage among all shape categories. This means that in the cross-modal identification task, the number of samples of the "Roughly Square" shape is much larger than that of other shapes. The model may be more sensitive to this shape and tend to identify the target as the "Roughly Square" shape during the recognition process. Therefore, when training and optimizing the model, attention should be paid to improving the recognition ability of other shapes to achieve a more balanced recognition effect.

\begin{figure}[h]
\centering
\includegraphics[width=0.9\linewidth]{figures/6.png}
\small
\caption{Morphological distribution}
\label{fig6-cross_model}
\end{figure}
\begin{table*}[h]
    \small
    \setlength{\tabcolsep}{2.5pt}
    \setlength{\arrayrulewidth}{2.5pt} % 设置表格线条宽度为2.5pt
    \centering
 
    \begin{tabular}{ccccccccr}
        \toprule
        \multirow{2}{*}{Category} & \multirow{2}{*}{Dataset} & \multirow{2}{*}{Total Samples} & \multirow{2}{*}{Mean AR} & \multirow{2}{*}{Median AR} & \multirow{2}{*}{Std Dev} & \multicolumn{3}{c}{AR Distribution (\%)}\\
        \cmidrule(lr){7 - 9}
        &  &  &  &  &  & AR$\leq$0.67 & 0.67<AR$\leq$1.5 & AR>1.5 \\
        \midrule
        \multirow{2}{*}{Ship} & Train & 93,342 & 1.34 & 1.07 & 1.24 & 28.37 & 39.67 & 31.96 \\
        & Test & 10,738 & 1.308 & 1.026 & 1.10 & 29.34 & 39.82 & 30.84 \\
        \midrule
        \multirow{2}{*}{Aircraft} & Train & 40,698 & 1.074 & 1.047 & 0.32 & 5.85 & 87.67 & 6.48 \\
        & Test & 6,778 & 1.08 & 1.041 & 0.31 & 4.56 & 87.36 & 8.08 \\
        \midrule
        \multirow{2}{*}{Car} & Train & 9,561 & 1.23 & 1.08 & 0.56 & 13.18 & 60.07 & 26.75 \\
        & Test & 1,230 & 1.21 & 1.07 & 0.53 & 12.28 & 62.11 & 25.61 \\
        \midrule
        \multirow{2}{*}{Tank} & Train & 24,15 & 1.10 & 1.00 & 0.84 & 1.58 & 94.29 & 4.13 \\
        & Test & 1,771 & 1.09 & 1.00 & 0.29 & 1.41 & 94.36 & 4.23 \\
        \midrule
        \multirow{2}{*}{Bridge} & Train & 27,615 & 1.56 & 1.18 & 1.92 & 18.38 & 44.83 & 36.79 \\
        & Test & 3,281 & 1.568 & 1.2 & 1.24 & 18.01 & 44.59 & 37.4 \\
        \midrule
        \multirow{2}{*}{Harbor} & Train & 3,306 & 1.20 & 1.01 & 0.72 & 14.19 & 67.93 & 17.88 \\
        & Test & 399 & 1.23 & 1.01 & 0.81 & 15.04 & 68.42 & 16.54 \\
        \bottomrule
    \end{tabular}
       \caption{Analysis of Aspect Ratio of Different Types of Targets}
    \label{tab:analysis-aspect-ratio}
\end{table*}
\noindent\textbf{Distribution Analysis of Morphological Categories}

\label{appendix:analys-mor}
As shown in Table \ref{tab:analysis-aspect-ratio}, nearly square-shaped morphology dominates both datasets, accounting for 39.67\% in training and 59.37\% in test sets, indicating its prevalence in target morphologies. Broad-bodied shapes maintain stable distributions (18.14\% training, 17.72\% test), while tall-bodied shapes show a moderate decrease from training (31.96\%) to test (22.91\%) sets. This distribution diversity enhances the model's generalization capability, though the significant increase in nearly square-shaped samples in the test set demands particular attention during model optimization.

\noindent\textbf{Category-Specific Morphological Patterns}


As shown in Table \ref{tab:analysis-aspect-ratio}, each category displays distinctive morphological characteristics. Bridges exhibit the highest average aspect ratio (1.56) with balanced distribution across all morphologies (18\% broad, 45\% square, 37\% tall). Ships demonstrate diverse shapes (28\% broad, 40\% square, 32\% tall), reflecting their real-world variability. Tanks and aircraft show highly concentrated distributions, with nearly square shapes dominating at 94\% and 87\% respectively, facilitating efficient model learning. Cars and ports maintain moderate distributions with 60-68\% nearly square shapes and balanced remaining proportions.

% To more clearly demonstrate the proportion of the category structure in the constructed dataset, this paper supplements the following analysis of the dataset.

% As shown in Figure \ref{fig4-train-classes}, in the \ourmethod-2M training set, the category distribution shows significant differences. Among them, the “Ship” category has the largest proportion, reaching 46.98\%, followed by the “Aircraft” category, with a proportion of 20.48\%. These two categories account for the majority of the samples in the training set. It can be seen that the sample distribution in the training set is imbalanced, and the “Ship” and “Aircraft” categories dominate. This may enable the model to learn the features of these two categories more comprehensively during the training process. However, since the samples of other categories are relatively few, the model's ability to learn and generalize their features may be affected to a certain extent.


% As shown in Figure \ref{fig5-test-classes}, in the \ourmethod-2M test set, the “Ship” category has the largest proportion among all categories, reaching 44.38\%, and the “Aircraft” category ranks second, with a proportion of 28.01\%. The distribution trends of these two major categories in the test set are similar to those in the training set. This indicates that the test set has a certain similarity to the training set in terms of the overall category distribution and can be used to test the model's generalization ability on data with a similar distribution. However, the slight differences in the proportions also remind us to comprehensively consider various factors when evaluating the model's performance.

%  \begin{figure}[h]
%     \centering
%     \includegraphics[width=0.9\linewidth]{figures/6.png}
%     \caption{Morphological distribution}
%     \label{fig6-cross_model}
% \end{figure}

% As shown in Figure \ref{fig6-cross_model}, in the cross-modal identification shape distribution, the “Roughly Square” shape has the largest proportion, with a quantity of 179,652. This shape has an absolute advantage among all shape categories. This means that in the cross-modal identification task, the number of samples of the “Roughly Square” shape is much larger than that of other shapes. The model may be more sensitive to this shape and tend to identify the target as the “Roughly Square” shape during the recognition process. Therefore, when training and optimizing the model, attention should be paid to improving the recognition ability of other shapes to achieve a more balanced recognition effect.

% \noindent\textbf{Distribution Analysis of Morphological Categories}

% As shown in Table \ref{}, nearly square-shaped morphology dominates both datasets, accounting for 39.67\% in training and 59.37\% in test sets, indicating its prevalence in target morphologies. Broad-bodied shapes maintain stable distributions (18.14\% training, 17.72\% test), while tall-bodied shapes show a moderate decrease from training (31.96\%) to test (22.91\%) sets. This distribution diversity enhances the model's generalization capability, though the significant increase in nearly square-shaped samples in the test set demands particular attention during model optimization.

% \noindent\textbf{Category-Specific Morphological Patterns}

%  As shown in Table \ref{}, each target category displays distinctive morphological characteristics. Bridges exhibit the highest average aspect ratio (1.56) with balanced distribution across all morphologies (18\% broad, 45\% square, 37\% tall). Ships demonstrate diverse shapes (28\% broad, 40\% square, 32\% tall), reflecting their real-world variability. Tanks and aircraft show highly concentrated distributions, with nearly square shapes dominating at 94\% and 87\% respectively, facilitating efficient model learning. Cars and ports maintain moderate distributions with 60-68\% nearly square shapes and balanced remaining proportions.

% \noindent\textbf{(2)Analysis of Task based on Datasets Composition}
\subsection{Analysis of Task based on Datasets Composition}
% \textbf{Review Seed data} 
% The dataset constructed in this paper aims to provide support for unified multi-modal tasks. Therefore, this section focuses on elaborating the specific distribution of the tasks covered by the dataset. As shown in Figure \ref{fig7-task-train} and Figure \ref{fig8-task-test}, the dataset encompasses six main tasks. Among them, the tasks of Classification, Fine-Grained Description, and Instance Counting do not distinguish the number of targets, while the tasks of Spatial Grounding, Cross-Modal Identification, and Referring are further divided into single-object and multi-object cases according to the number of targets in the images. This classification provides diverse scenarios for model training and helps to enhance the model's generalization ability and its capacity to handle complex tasks.

Our dataset is designed to support unified multi-modal tasks through a comprehensive task taxonomy, as shown in Table \ref{tab:task_distribution}. It encompasses six primary tasks: Classification, Fine-Grained Description, Instance Counting, Spatial Grounding, Cross-Modal Identification, and Referring. Among these, the first three tasks are target-quantity independent, while Spatial Grounding, Cross-Modal Identification, and Referring are further categorized into single-object and multi-object variants. This systematic organization enables diverse training scenarios and enhances model generalization capabilities.

% \begin{table}[htbp]
%     \centering
%     \caption{Task type distribution in training and test sets}
%     \label{tab:task_distribution}
%     \resizebox{1.0\linewidth}{!}{
%         \begin{tabular}{lcc}
%             \Thickline
%             Task Type & Train & Test \\
%             \Thickline
%             Instance Counting & 95493 (5.2\%) & 11794 (5.2\%) \\
%             Spatial Grounding & 94456 (5.1\%) & 11608 (5.1\%) \\
%             Cross - Modal Identification & 1423548 (77.5\%) & 175565 (77.4\%) \\
%             Referring & 95486 (5.2\%) & 11703 (5.2\%) \\
%             Fine - Grained Description & 46141 (2.5\%) & 6032 (2.7\%) \\
%             Classification & 81788 (4.5\%) & 10024 (4.4\%) \\
%             \Thickline
%         \end{tabular}
%     }
% \end{table}
\begin{table}[htbp]
    \centering
  
    \resizebox{1.0\linewidth}{!}{
    \begin{tabular}{lcc}
    \noalign{\hrule height 1.5pt}
    Task Type & Train & Test \\
    \noalign{\hrule height 1.5pt}
    Instance Counting & 95493 (5.2\%) & 11794 (5.2\%) \\
    Spatial Grounding & 94456 (5.1\%) & 11608 (5.1\%) \\
    Cross - Modal Identification & 1423548 (77.5\%) & 175565 (77.4\%) \\
    Referring & 95486 (5.2\%) & 11703 (5.2\%) \\
    Fine - Grained Description & 46141 (2.5\%) & 6032 (2.7\%) \\
    Classification & 81788 (4.5\%) & 10024 (4.4\%) \\
    \noalign{\hrule height 1.5pt}
    \end{tabular}
    }
      \caption{Task type distribution in training and test sets}
    \label{tab:task_distribution}
\end{table}

\begin{figure}[h]
    \centering
    \includegraphics[width=0.85\linewidth]{figures/7.png}
    {\raggedright
    \caption{Train Task Distribution}
    \label{fig7-task-train}
    }
\end{figure}
 
 \begin{figure}[h]
    \centering
    \includegraphics[width=0.85\linewidth]{figures/8.png}
    \caption{Test Task Distribution}
    \label{fig8-task-test}
\end{figure}

% Specifically, as depicted in Figure \ref{fig7-task-train}, the training set contains a total of 1,836,912 data entries. The Cross-Modal Identification task has the largest amount of data, with 1,423,548 entries, accounting for 77.50\% and occupying a dominant position among all tasks. The large amount of data enables the model to fully learn the features and methods of cross-modal identification to deal with complex cross-modal scenarios. The Instance Counting task has 95,493 entries, accounting for 5.20\%, allowing the model to master the methods of identifying and counting the number of targets. The Spatial Grounding task has 94,456 entries, accounting for 5.14\%, enabling the model to learn the target-localization ability in different scenarios. The Referring task has 95,486 entries, accounting for 5.20\%, assisting the model in accurately associating referring objects. The Fine-Grained Description task has 46,141 entries, accounting for 2.51\%. The relatively small amount of data may affect the model's performance in description tasks to some extent, and it may be considered to appropriately increase the data volume in the follow-up. The Classification task has 81,788 entries, accounting for 4.45\%, providing a certain sample support for the model to learn classification features.

% As shown in Figure \ref{fig8-task-test}, the test set contains a total of 226,636 data entries. The Cross-Modal Identification task still has the largest amount of data, with 175,565 entries, accounting for 77.47\%, which is consistent with the distribution in the training set. This provides a reliable basis for model performance evaluation. By comparing the completion of this task in the training set and the test set, the learning effect of the model can be accurately judged. The Instance Counting task has 11,704 entries, accounting for 5.16\%. The Spatial Grounding task has 11,608 entries, accounting for 5.12\%. The Referring task has 11,703 entries, accounting for 5.16\%. The Fine-Grained Description task has 6,032 entries, accounting for 2.66\%. The Classification task has 10,024 entries, accounting for 4.42\%. The task distribution proportions in the test set are similar to those in the training set, further reflecting the stability and reliability of the dataset.
As illustrated in Figure \ref{fig7-task-train}, the training set comprises 1,836,912 entries. Cross-Modal Identification dominates with 1,423,548 entries (77.50\%), enabling robust cross-modal feature learning. Instance Counting and Referring tasks contain 95,493 (5.20\%) and 95,486 (5.20\%) entries respectively, while Spatial Grounding accounts for 94,456 entries (5.14\%). Fine-Grained Description includes 46,141 entries (2.51\%), with its relatively limited data volume potentially affecting model performance. The Classification task contains 81,788 entries (4.45\%).

The test set (Figure \ref{fig8-task-test}) maintains a parallel distribution across its 226,636 entries. Cross-Modal Identification remains dominant with 175,565 entries (77.47\%), followed by Instance Counting (11,704, 5.16\%), Referring (11,703, 5.16\%), Spatial Grounding (11,608, 5.12\%), Classification (10,024, 4.42\%), and Fine-Grained Description (6,032, 2.66\%). This consistent distribution ensures reliable model evaluation.

%%%%%%%

% Overall, the task distribution of this dataset is reasonable and diverse. The data-volume distribution of different tasks in the training set can meet the model's learning needs for different types of tasks, enabling the model to comprehensively master the key features and solutions of various tasks during the training process. The consistency of task distribution between the test set and the training set ensures the accuracy and reliability of model evaluation. However, the data volumes of the Fine-Grained Description task in both the training set and the test set are relatively small, which may have a certain impact on the model's performance in description tasks. In subsequent research, it is advisable to consider appropriately increasing its data volume. At the same time, for the Cross-Modal Identification task with a large proportion of data, it is necessary to pay attention to whether the model will over-fit and adopt effective regularization methods during the model training process to ensure the model's generalization ability. 

The dataset exhibits a well-balanced task distribution that supports comprehensive model training. While the consistency between training and test sets ensures reliable evaluation, two aspects warrant attention: the relatively limited data in Fine-Grained Description tasks may constrain model performance and could benefit from expansion, while the dominant proportion of Cross-Modal Identification data necessitates careful consideration of overfitting during training through appropriate regularization methods.

\subsection{Dataset Examples}

\label{example}

\setlength{\fboxsep}{4pt}  % 调整文字和框的距离
\setlength{\fboxrule}{0.5pt}  % 调整框线的粗细
\noindent\fbox{%
\begin{minipage}{0.46\textwidth}
\small\noindent\textbf{Classification Task:}
\vspace{1pt}
\hrule
\vspace{4pt}

\small\noindent\textbf{User:} {\ttfamily"Please identify the target categories in the figure (candidate categories: \{total category list\})"}

\small\noindent\textbf{AI:} {\ttfamily"The \{specific category\} targets can be identified in the figure"}
\end{minipage}%
}


\setlength{\fboxsep}{4pt}  % 调整文字和框的距离
\setlength{\fboxrule}{0.5pt}  % 调整框线的粗细
\noindent\fbox{%
\begin{minipage}{0.46\textwidth}
\small\noindent\textbf{Spatial Grounding Task:}
\vspace{1pt}
\hrule
\vspace{4pt}

\small\noindent\textbf{User:} {\ttfamily"[grounding] Can you describe the spatial arrangement of various features in this SAR image?"}

\small\noindent\textbf{AI:} {\ttfamily"In this SAR image, I can identify several objects and their locations: There is a \{category-1\} in the top right portion \{bbox\}. There are 2 \{category-2\} in the bottom left: one in the bottom left portion \{bbox\}, another in the middle left portion \{bbox\}. The \{category-1\} is far from the \{category-2\}."}
\end{minipage}%
}

\setlength{\fboxsep}{4pt}  % 调整文字和框的距离
\setlength{\fboxrule}{0.5pt}  % 调整框线的粗细
\noindent\fbox{%
\begin{minipage}{0.46\textwidth}
\small\noindent\textbf{Fine-Grained Description Task:}
\vspace{1pt}
\hrule
\vspace{4pt}

\small\noindent\textbf{User:} {\ttfamily"Please describe the content of the satellite image in detail"}

\small\noindent\textbf{AI:} {\ttfamily"In the figure, \{size\}\{category\} can be seen located at \{orientation\},"+"and there is also \{size\}\{category\} located at \{orientation\}"}
\end{minipage}%
}

\setlength{\fboxsep}{4pt}  % 调整文字和框的距离
\setlength{\fboxrule}{0.5pt}  % 调整框线的粗细
\noindent\fbox{%
\begin{minipage}{0.46\textwidth}
\small\noindent\textbf{Instance Counting Task:}
\vspace{1pt}
\hrule
\vspace{4pt}

\small\noindent\textbf{User:} {\ttfamily"[count] How many \{category\} are there in the image?""}

\small\noindent\textbf{AI:} {\ttfamily"There are 3 instances. \{bbox\}\{bbox\}\{bbox\}"}
\end{minipage}%
}

\setlength{\fboxsep}{4pt}  % 调整文字和框的距离
\setlength{\fboxrule}{0.5pt}  % 调整框线的粗细
\noindent\fbox{%
\begin{minipage}{0.46\textwidth}
\small\noindent\textbf{Cross-Modal Identification Task:}
\vspace{1pt}
\hrule
\vspace{4pt}

\small\noindent\textbf{User:} {\ttfamily"[identify] What object is located in \{bbox\}?"}

\small\noindent\textbf{AI:} {\ttfamily"This is a medium-sized \{category\} located in the top right portion of the image. It appears to be wide in shape."}
\end{minipage}%
}

\setlength{\fboxsep}{4pt}  % 调整文字和框的距离
\setlength{\fboxrule}{0.5pt}  % 调整框线的粗细
\noindent\fbox{%
\begin{minipage}{0.46\textwidth}
\small\noindent\textbf{Referring Task:}
\vspace{1pt}
\hrule
\vspace{4pt}

\small\noindent\textbf{User:} {\ttfamily"[refer] Where is the \{category\}?"}

\small\noindent\textbf{AI:} {\ttfamily"\{bbox\} (top right)."}
\end{minipage}%
}

\section{Motivation for the Dual-benchmark Setup}

In tasks such as spatial localization, cross-modal identification, and referring expression comprehension, this study constructs a dual-benchmark of single-target and multi-target data. The core research motivations are as follows:


\noindent\textbf{(1) Evaluation of the Model's Target Discrimination Ability}

The single-target scenario aims to test the model's basic recognition ability for independent targets. In contrast, the multi-target scenario focuses on examining the model's ability to separate and select targets in complex environments, especially when multiple targets exhibit similar features. This dual-benchmark design can effectively diagnose the performance differences of the model in scenarios with varying degrees of complexity.

\begin{table*}[ht]
    \small
    \setlength{\tabcolsep}{2.5pt}
    \setlength{\arrayrulewidth}{2.5pt} % 设置表格线条宽度为2.5pt
    \centering

    \begin{tabular}{lcccc}
        \Thickline
        \textbf{Performance} & \textbf{8B} & \textbf{4B} & \textbf{2B} & \textbf{1B} \\
        \Thickline
        Instance Counting Accuracy & 74.14 & 72.68 & 71.52 & 69.87 \\
        Instance Counting Accuracy (IoU = 0.25) & 61.37 & 57.54 & 54.11 & 50.18 \\
        Instance Counting Accuracy (IoU = 0.5) & 52.17 & 47.35 & 44.22 & 39.35 \\
        Spatial Grounding Accuracy & 62.25 & 60.89 & 60.81 & 56.30 \\
        Abstract Location in Spatial Grounding Accuracy & 81.25 & 83.33 & 50.00 & 0.00 \\
        Spatial Grounding Single Accuracy & 87.91 & 85.90 & 81.92 & 82.24 \\
        Cross-Modal Identification (Multi) Accuracy & 98.84 & 98.01 & 97.79 & 96.98 \\
        Cross-Modal Identification (Single) Accuracy & 98.98 & 98.76 & 98.63 & 98.60 \\
        Referring (Multi) Accuracy (IoU = 0.25) & 37.49 & 34.05 & 27.05 & 22.13 \\
        Referring (Multi) Accuracy (IoU = 0.5) & 23.46 & 18.86 & 13.91 & 9.94 \\
        Referring (Single) Accuracy (IoU = 0.25) & 74.86 & 69.92 & 68.50 & 62.33 \\
        Referring (Single) Accuracy (IoU = 0.5) & 60.13 & 55.29 & 52.16 & 44.99 \\
        Fine-Grained Description Accuracy & 63.43 & 58.84 & 56.36 & 53.30 \\
        Classification Accuracy & 97.25 & 97.27 & 96.69 & 96.65 \\
        \Thickline
    \end{tabular}
        \caption{Performance comparison across different model sizes}
    \label{tab:internvlperformance}
\end{table*}

\noindent \textbf{(2) Revelation of Target Association Understanding Issues}

In multi-target scenarios, the model usually needs to understand the spatial and semantic relationships between targets. By comparing the performance differences of the model in single-target and multi-target scenarios, it is possible to evaluate whether the model truly understands the descriptions of the positional relationships between targets. This helps to identify the limitations of the model when dealing with relative position descriptions such as "the vehicle on the left" and "the tank in the middle".

\noindent\textbf{(3) Exposure of Attention Mechanism Defects}

In multi-target scenarios, the model is highly prone to problems such as attention divergence or overlap. When there are multiple similar targets in an image, the model may have difficulty accurately locating the specific target described by the user. Through the comparison between single-target and multi-target scenarios, the deficiencies of the model in attention allocation can be clearly demonstrated.

\noindent \textbf{(4) Simulation of Real-world Application Scenarios}

Real-world applications cover both simple single-target scenarios and complex multi-target environments. The establishment of the dual-benchmark is more in line with real-world usage requirements, providing a more comprehensive dimension for model evaluation and helping to improve the applicability and reliability of the model in actual deployments. 

\section{Task-specific Performance Scoring}\label{task-cal}

To evaluate model performance on each task, the task-specific accuacy is caculate by Formula \ref{for5}. For each task $t$, the accuracy score $a_{m,t}$ of model $m$ is computed by averaging the accuracy scores across all subtasks:

\begin{equation}
\label{for5}
a_{m,t} = \frac{1}{k} \sum_{i=1}^{k} a_{m,t,i}
\end{equation}

where $a_{m,t}$ is the average accuracy of model $m$ on task $t$, $k$ is the number of subtasks, and $a_{m,t,i}$ is the accuracy of model $m$ on the $i$-th subtask of task $t$. This approach ensures that each subtask contributes equally to the overall task score.
\section{The Analysis of Model Size}
Based on the data analysis in Table \ref{tab:internvlperformance}, it can be concluded that for most task-related metrics, there is a trend of performance improvement as the model size increases from 1B to 8B. For example, the instance-counting accuracy rises from 69.87\% to 74.14\%, the spatial-grounding accuracy increases from 56.30\% to 62.25\%, the fine-grained description accuracy goes up from 53.30\% to 63.43\%, and the classification accuracy climbs from 96.65\% to 97.25\%. This indicates that an increase in model size is beneficial to enhancing the performance of these tasks.
However, the accuracy of abstract locations in the spatial-grounding task shows a unique trend of change. This metric increases from 0.00\% for the 1B model to 83.33\% for the 4B model, but then decreases to 81.25\% for the 8B model, not increasing monotonically with the model size. Evidently, the influence of model size on some specific tasks follows complex patterns. Therefore, when selecting a model, it is necessary to comprehensively consider the task type and model size to achieve optimal performance.

\begin{table*}[htbp]
\scriptsize
\setlength{\tabcolsep}{2.5pt}


\centering
\begin{tabular}{lcccccccccccc}
\Thickline
\multirow{2}{*}{\textbf{Model}} & \multirow{2}{*}{\begin{tabular}[c]{@{}c@{}}\textbf{Only}\\ \textbf{count}\end{tabular}} & \multicolumn{2}{c}{\textbf{Spatial Ground}} & \multicolumn{2}{c}{\textbf{Cross-Modal ID}} & \multicolumn{2}{c}{\textbf{Multi-target Ref}} & \multicolumn{2}{c}{\textbf{Single-target Ref}} & \multirow{2}{*}{\textbf{Descript}} & \multirow{2}{*}{\textbf{Class}} \\
\cline{3-4} \cline{5-6} \cline{7-8} \cline{9-10}
 & & \textbf{Multi} & \textbf{Single} & \textbf{Multi} & \textbf{Single} & \textbf{IoU=.25} & \textbf{IoU=.5} & \textbf{IoU=.25} & \textbf{IoU=.5} & & \\
\Thickline
\textbf{InternVL2.5(8B)} & 63.71 & 5.07 & 0.38 & 0.12 & 0.17 & 0 & 0 & 0 & 0 & 0.04 & 17.8 \\
\textbf{SARChat-InternVL2.5(8B)} & 74.14 & 62.25 & 87.91 & 98.84 & 98.98 & 37.49 & 23.46 & 74.86 & 60.13 & 63.43 & 97.25 \\
\Thickline
\end{tabular}
\caption{Performance comparison of InternVL2.5 before and after fine-tuning}
\label{normal-ft}
\end{table*}


\begin{figure*}[th]
    \centering    \includegraphics[width=0.88\linewidth]{figures/SAR_sample.pdf}
    \caption{\textbf{Simple Examples of ship detection in SAR images.} The ships appear as distinct bright spots in these SAR images, making them relatively easy to count even for VLMs without SAR-specific training.}
    \label{SAR-sample}
    % \vspace{-1em}
\end{figure*}

\section{Comparison before and after training}
\label{ftornot}
To evaluate the impact of the \ourmethod-2M training dataset, we conducted a comparative analysis on InternVL-2.5-8B—the best-performing model in \ourmethod-Bench—before and after fine-tuning. Our evaluation metrics reveal that without fine-tuning on \ourmethod-2M, the model fails to comprehend most SAR-specific targets, exhibiting near-zero performance on tasks involving target interpretation and description. The only exception is the Instance Counting Task, where InternVL2.5-8B achieves a baseline accuracy of 63.71\%. This relatively high performance can be attributed to the prevalence of ship-on-sea samples, where SAR imaging exhibits distinctive characteristics: the smooth sea surface creates specular reflection, causing most electromagnetic waves to scatter away from the sensor direction, resulting in weak returns that appear as dark areas. Meanwhile, ships' metallic structures and geometric features (such as dihedral and trihedral corner reflectors) generate strong backscattering, concentrating radar waves back to the sensor, thus appearing as bright spots. As shown in Figure \ref{SAR-sample}, these samples present relatively straightforward recognition scenarios, leading to higher counting accuracy scores. The comparative results between the base model and its fine-tuned version on SARChat-Bench are presented in Table \ref{normal-ft}. Overall, fine-tuning with the \ourmethod-2M dataset proves essential for enabling VLMs to interpret SAR imagery effectively.




\end{document}
