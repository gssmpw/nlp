\documentclass[sigconf,natbib=true]{acmart}
\let\Bbbk\relax
% \pdfobjcompresslevel=2

\usepackage{enumitem}
\usepackage{multicol}
\usepackage{amssymb}
\usepackage{amsmath}
\usepackage{graphicx}
\usepackage{xcolor}
\usepackage{amsfonts} 
\usepackage{multirow}
\usepackage{natbib}
\usepackage{url}
% \url{https://example.com/very-long-url-that-does-not-wrap}
\usepackage{balance}
% save space
% \usepackage{microtype}
 \setlength\floatsep{0.2\baselineskip plus 3pt minus 2pt} % distance between two floats
 \setlength\textfloatsep{0.2\baselineskip plus 3pt minus 2pt} % distance between floats on the top or the bottom and the text
 %\setlength{\textfloatsep}{2pt}
 \setlength\intextsep{0.2\baselineskip plus 3pt minus 2pt} % distance between floats inserted inside the text (using h) and the text
 \setlength\dbltextfloatsep{0.2\baselineskip plus 3pt minus 2pt} % distance between a float spanning both columns and the text
 \setlength\dblfloatsep{0.2\baselineskip plus 3pt minus 2pt} % distance between two floats spanning both columns.

\definecolor{mygray}{gray}{0.9} 
\definecolor{lightgray}{gray}{0.95} 
\newcommand{\ZKI}[1]{\textcolor{blue}{[dzq: #1]}}
\newcommand{\revision}[1]{\textcolor{black}{#1}}
%\newcommand{\revision}[1]{{{#1}}}



\long\def\appendixsiam{\par\setcounter{section}{0}\setcounter{subsection}{0}\setcounter{equation}{0}}
\AtBeginDocument{%
  \providecommand\BibTeX{{%
    Bib\TeX}}}

\copyrightyear{2025}
\acmYear{2025}
\setcopyright{acmlicensed}
\acmConference[WWW '25]{Proceedings of the ACM Web Conference 2025}{April 28-May 2, 2025}{Sydney, NSW, Australia}
\acmBooktitle{Proceedings of the ACM Web Conference 2025, April 28-May 2, 2025, Sydney, NSW, Australia}
\acmDOI{10.1145/3696410.3714726}
\acmISBN{979-8-4007-1274-6/25/04}

%% These commands are for a PROCEEDINGS abstract or paper.
% \acmConference[Conference acronym 'XX]{Make sure to enter the correct
  % conference title from your rights confirmation emai}{June 03--05,
  % 2018}{Woodstock, NY}
%%
%%  Uncomment \acmBooktitle if the title of the proceedings is different
%%  from ``Proceedings of ...''!
%%
%%\acmBooktitle{Woodstock '18: ACM Symposium on Neural Gaze Detection,
%%  June 03--05, 2018, Woodstock, NY}
% \acmISBN{978-1-4503-XXXX-X/18/06}


%%
%% Submission ID.
%% Use this when submitting an article to a sponsored event. You'll
%% receive a unique submission ID from the organizers
%% of the event, and this ID should be used as the parameter to this command.
%%\acmSubmissionID{123-A56-BU3}

%%
%% For managing citations, it is recommended to use bibliography
%% files in BibTeX format.
%%
%% You can then either use BibTeX with the ACM-Reference-Format style,
%% or BibLaTeX with the acmnumeric or acmauthoryear sytles, that include
%% support for advanced citation of software artefact from the
%% biblatex-software package, also separately available on CTAN.
%%
%% Look at the sample-*-biblatex.tex files for templates showcasing
%% the biblatex styles.
%%

%%
%% The majority of ACM publications use numbered citations and
%% references.  The command \citestyle{authoryear} switches to the
%% "author year" style.
%%
%% If you are preparing content for an event
%% sponsored by ACM SIGGRAPH, you must use the "author year" style of
%% citations and references.
%% Uncommenting
%% the next command will enable that style.
%%\citestyle{acmauthoryear}


%%
%% end of the preamble, start of the body of the document source.
\begin{document}

\title{IllusionCAPTCHA: A CAPTCHA based on Visual Illusion}

\author{Ziqi Ding}
\orcid{0009-0007-6257-1502}
\affiliation{%
  \institution{University of New South Wales}
  \city{Sydney}
  \country{Australia}
}
\email{ziqi.ding1@unsw.edu.au}

\author{Gelei Deng}
\orcid{0000-0002-0046-6674}
\affiliation{%
  \institution{Nanyang Technological University}
  \country{Singapore}
}
\email{gelei.deng@ntu.edu.sg}

\author{Yi Liu}
\orcid{0000-0002-4978-127X}
\affiliation{%
  \institution{Quantstamp}
  \country{Singapore}
}
\email{yi009@e.ntu.edu.sg}
\authornote{Corresponding author.}


\author{Junchen Ding}
\orcid{0009-0007-6531-8190}
\affiliation{%
  \institution{University of New South Wales}
  \city{Sydney}
  \country{Australia}
}
\email{jamison.ding@unsw.edu.au}

\author{Jieshan Chen}
\orcid{0000-0002-2700-7478}
\affiliation{%
  \institution{Data61, CSIRO}
  \city{Sydney}
  \country{Australia}
}
\email{jieshan.chen@data61.csiro.au}

\author{Yulei Sui}
\orcid{0000-0002-9510-6574}
\affiliation{%
  \institution{University of New South Wales}
  \city{Sydney}
  \state{New South Wales}
  \country{Australia}
}
\email{y.sui@unsw.edu.au}

\author{Yuekang Li}
\orcid{0000-0003-4382-0757}
\affiliation{%
  \institution{University of New South Wales}
  \city{Sydney}
  \country{Australia}
}
\email{yuekang.li@unsw.edu.au}

\renewcommand{\shortauthors}{Ziqi Ding et al.}

\begin{abstract}
CAPTCHAs have long been essential tools for protecting applications from automated bots. Initially designed as simple questions to distinguish humans from bots, they have become increasingly complex to keep pace with the proliferation of CAPTCHA-cracking techniques employed by malicious actors. However, with the advent of advanced large language models (LLMs), the effectiveness of existing CAPTCHAs is now being undermined.

To address this issue, we have conducted an empirical study to evaluate the performance of multimodal LLMs in solving CAPTCHAs and to assess how many attempts human users typically need to pass them. Our findings reveal that while LLMs can solve most CAPTCHAs, they struggle with those requiring complex reasoning type of CAPTCHA that also presents significant challenges for human users. Interestingly, our user study shows that the majority of human participants require a second attempt to pass these reasoning CAPTCHAs, a finding not reported in previous research.

\revision{Based on empirical findings, we present IllusionCAPTCHA, a novel security mechanism employing the "Human-Easy but AI-Hard" paradigm. This new CAPTCHA employs visual illusions to create tasks that are intuitive for humans but highly confusing for AI models. Furthermore, we developed a structured, step-by-step method that generates misleading options, which particularly guide LLMs towards making incorrect choices and reduce their chances of successfully solving CAPTCHAs. Our evaluation shows that IllusionCAPTCHA can effectively deceive LLMs 100\% of the time. Moreover, our structured design significantly increases the likelihood of AI errors when attempting to solve these challenges. Results from our user study indicate that 86.95\% of participants successfully passed the CAPTCHA on their first attempt, outperforming other CAPTCHA systems.}

\end{abstract}


% \vspace{-4mm}
\begin{CCSXML}
<ccs2012>
   <concept>
       <concept_id>10002978.10003022.10003026</concept_id>
       <concept_desc>Security and privacy~Web application security</concept_desc>
       <concept_significance>500</concept_significance>
       </concept>
 </ccs2012>
\end{CCSXML}

\ccsdesc[500]{Security and privacy~Web application security}



\keywords{CAPTCHAs, AI security, large language models, visual illusions}

\maketitle

\section{Introduction}

Deep Reinforcement Learning (DRL) has emerged as a transformative paradigm for solving complex sequential decision-making problems. By enabling autonomous agents to interact with an environment, receive feedback in the form of rewards, and iteratively refine their policies, DRL has demonstrated remarkable success across a diverse range of domains including games (\eg Atari~\citep{mnih2013playing,kaiser2020model}, Go~\citep{silver2018general,silver2017mastering}, and StarCraft II~\citep{vinyals2019grandmaster,vinyals2017starcraft}), robotics~\citep{kalashnikov2018scalable}, communication networks~\citep{feriani2021single}, and finance~\citep{liu2024dynamic}. These successes underscore DRL's capability to surpass traditional rule-based systems, particularly in high-dimensional and dynamically evolving environments.

Despite these advances, a fundamental challenge remains: DRL agents typically rely on deep neural networks, which operate as black-box models, obscuring the rationale behind their decision-making processes. This opacity poses significant barriers to adoption in safety-critical and high-stakes applications, where interpretability is crucial for trust, compliance, and debugging. The lack of transparency in DRL can lead to unreliable decision-making, rendering it unsuitable for domains where explainability is a prerequisite, such as healthcare, autonomous driving, and financial risk assessment.

To address these concerns, the field of Explainable Deep Reinforcement Learning (XRL) has emerged, aiming to develop techniques that enhance the interpretability of DRL policies. XRL seeks to provide insights into an agent’s decision-making process, enabling researchers, practitioners, and end-users to understand, validate, and refine learned policies. By facilitating greater transparency, XRL contributes to the development of safer, more robust, and ethically aligned AI systems.

Furthermore, the increasing integration of Reinforcement Learning (RL) with Large Language Models (LLMs) has placed RL at the forefront of natural language processing (NLP) advancements. Methods such as Reinforcement Learning from Human Feedback (RLHF)~\citep{bai2022training,ouyang2022training} have become essential for aligning LLM outputs with human preferences and ethical guidelines. By treating language generation as a sequential decision-making process, RL-based fine-tuning enables LLMs to optimize for attributes such as factual accuracy, coherence, and user satisfaction, surpassing conventional supervised learning techniques. However, the application of RL in LLM alignment further amplifies the explainability challenge, as the complex interactions between RL updates and neural representations remain poorly understood.

This survey provides a systematic review of explainability methods in DRL, with a particular focus on their integration with LLMs and human-in-the-loop systems. We first introduce fundamental RL concepts and highlight key advances in DRL. We then categorize and analyze existing explanation techniques, encompassing feature-level, state-level, dataset-level, and model-level approaches. Additionally, we discuss methods for evaluating XRL techniques, considering both qualitative and quantitative assessment criteria. Finally, we explore real-world applications of XRL, including policy refinement, adversarial attack mitigation, and emerging challenges in ensuring interpretability in modern AI systems. Through this survey, we aim to provide a comprehensive perspective on the current state of XRL and outline future research directions to advance the development of interpretable and trustworthy DRL models.
\section{Background}
\label{sec:background}

\noindent
In this section, we first overview the principles governing transformer architecture. Next, we present a concise overview of DP-SFGs, which we employ to map OTA circuits into transformer-friendly sequential data. Finally, we describe a precomputed LUT-based width estimator to translate DP-SFG parameters to transistor widths.
\vspace{-1mm}
\subsection{The transformer architecture}

\noindent
The transformer~\cite{vaswani_17} is viewed as one of the most promising deep learning architectures for sequential data prediction in NLP.  It relies on an attention mechanism that reveals interdependencies among sequence elements, even in long sequences. The architecture takes a series of inputs \((x_1, x_2, x_3, \ldots, x_n\)) and generates corresponding outputs \((y_1, y_2, y_3, \ldots, y_n\)).

\begin{figure}[b]
\vspace{-5mm}
\centering
\includegraphics[width=0.5\textwidth, bb=0 0 370 190]{fig/TransformermODEL.pdf}
\vspace{-5mm}
\caption{Architecture of a transformer.}
\label{fig:simpleTrans}
% \vspace{-2mm}
\end{figure}

The simplified architecture shown in Fig.~\ref{fig:simpleTrans} consists of $N$ identical stacked encoder blocks, followed by $N$ identical stacked decoder blocks. The encoder and decoder is fed by an input embedding block, which converts a discrete input sequence to a continuous representation for neural processing. Additionally, a positional encoding block encodes the relative or absolute positional details of each element in the sequence using sine-cosine encoding functions at different frequencies. This allows the model to comprehend the position of each element in the sequence, thus understanding its context. Each encoder block comprises a multi-head self-attention block and a position-wise feed-forward network (FFN); each decoder block, which has a similar structure to the encoder, consists of an additional multi-head cross-attention block, stacked between the multi-head self-attention and feed-forward blocks. The attention block tracks the correlation between elements in the sequence and builds a contextual representation of interdependencies using a scaled dot-product between the query ($Q$), key ($K$), and value ($V$) vectors:
\begin{equation}
\text{{Attention}}(Q, K, V) = \text{softmax}\left(\frac{QK^T}{\sqrt{d_k}}\right)V,
\end{equation}
where $d_k$ is the dimension of the query and key vectors. The FFN consists of two fully connected networks with an activation function and dropout after each network to avoid overfitting. The model features residual connections across the attention blocks and FFN to mitigate vanishing gradients and facilitate information flow.

\subsection{Driving-point signal flow graphs}

\noindent
The input data sequence to the transformer must encode information that relates the parameters of a circuit to its performance metrics.  Our method for representing circuit performance is based on the signal flow graph (SFG).  The classical SFG proposed by Mason~\cite{Mason53} provides a graph representation of linear time-invariant (LTI) systems, and maps on well to the analysis of linear analog circuits such as amplifiers. In our work, we employ the driving-point signal flow graph (DP-SFG)~\cite{ochoa_98,schmid_18}. The vertices of this graph are the set of excitations (voltage and current sources) in the circuit and internal states (e.g., voltages) in the circuit.  
% An edge is drawn between vertices that have an electrical relationship, and the weight on each edge is the gain of the edge;
An edge connects vertices with an electrical relationship, and the edge weight is the gain; 
for example, if a vertex $z$ has two incoming edges from vertices $x$ and $y$, with gains $a$ and $b$, respectively, then $z = ax + by$, using the principle of superposition in LTI systems.  To effectively use superposition to assess the impact of each node on every other node, the DP-SFG introduces auxiliary voltages at internal nodes of the circuit that are not connected to excitations. These auxiliary sources are structured to not to alter any currents or voltages in the original circuit, and simplifies the SFG formulation for circuit analysis.
% enable easy formulation of the SFG to analyze circuit behavior. 

\begin{figure}[t]
% \vspace{-6mm}
\centering
\includegraphics[width=0.9\linewidth, bb=0 0 320 140]{fig/DPSFG.pdf}
\vspace{-0.25cm}
\caption{~(a) Schematic and (b) DP-SFG for an active inductor.}
\label{fig:DP-SFG_ex}
\vspace{-5mm}
\end{figure}

Fig.~\ref{fig:DP-SFG_ex}(a) shows a circuit of an active inductor, which is an inductor-less circuit that replicates the behavior of an inductor over a certain range of frequencies. Fig.~\ref{fig:DP-SFG_ex}(b) shows the equivalent DP-SFG. In Section~\ref{sec:dp-sfg}, we provide a detailed explanation that shows how a circuit may be mapped to its equivalent DP-SFG. 


\ignore{
\subsection{Lookup table for MOSFET sizing}
\label{sec:LUT}

\noindent
As seen in Fig.~\ref{fig:DP-SFG_ex}, the edge weights in a DP-SFG include circuit parameters such as the transistor transconductance, $g_m$, and various capacitances in the circuit.  The circuit may be optimized to find values of these parameters that meet specifications, but ultimately these must be translated into physical transistor parameters such as the transistor width.   In older technologies, the square-law model for MOS transistors could be used to perform a translation between DP-SFG parameters and transistor widths, but square-law behavior is inadequate for capturing the complexities of modern MOS transistor models.
In this work, we use a precomputed lookup table (LUT) that rapidly performs the mapping to device sizes while incorporating the complexities of advanced MOS models.

\begin{figure}[htbp]
\vspace{-0.4cm}
\centering
\includegraphics[height=4cm]{fig/lut_fig_1.pdf}
\vspace{-0.55cm}
\caption{LUT generation using three DOFs, $V_{gs}$, $V_{ds}$ and $L$.}
\label{fig:lutgen}
\vspace{-0.1cm}
\end{figure}

The LUT is indexed by the $V_{gs}$, $V_{ds}$, and length $L$ of the transistor, and provides four outputs: the drain current ($I_d$), transconductance ($g_m$), source-drain conductance ($g_{ds}$), and drain-source capacitance ($C_{ds}$).
The entries of the LUT are computed by performing a nested DC sweep simulation across the three input indices for the MOSFET with a specific reference width, $W_{ref}$, as shown in Fig.~\ref{fig:lutgen}, and for each input combination, the four outputs are recorded.
\blueHL{Empirically, we see that the impact of $V_{sb}$ is small enough that it can be neglected, and therefore we set $V_{sb} = 0$ in the sweeps used to create the LUT.}

Our methodology uses this LUT, together with the $g_m/I_d$ methodology~\cite{silviera_96}, to translate circuit parameters predicted by the transformer to transistor widths. The cornerstone of this methodology relies on the inherent width independence of the ratio $g_m/I_d$ to estimate the unknown device width: this makes it feasible to use an LUT characterized for a reference width $W_{ref}$. 
We will elaborate on this procedure further in Section~\ref{sec:precomputedLUTs}, and show how the LUT, together with the $g_m/I_d$ method, can effectively estimate the device widths corresponding to the transformer outputs.
% \redHL{\sout{required to achieve equivalent DC operating characteristics within the circuit. Section III D \redHL{Do not hardcode section numbers!!} provides an in-depth explanation of the implementation details of this methodology.}}
}
\section{Threat Model}
In this paper, we outline our assumptions regarding the goals and capabilities of attackers.

\noindent\textbf{Attacker goals.} We assume that the adversary aims to automatically solve CAPTCHAs without human interactions, which could potentially lead to these results~\cite{che2021augmented,shi2020text}:
(1) Automating Actions: Gaining unauthorized access to websites, applications, or services to automate tasks (e.g. account creation, data scraping, or spamming). (2) Credential Harvesting: Exploiting CAPTCHA weaknesses to gain access to user accounts by defeating login protections. (3) Fraudulent Activities: Engaging in malicious activities like ticket scalping, purchasing limited-edition items, or bypassing purchase limits imposed by websites. (4) Disruption of Services: Creating bot networks that can flood websites with traffic, breaking CAPTCHAs to disrupt normal operations.


\noindent\textbf{Attacker capabilities.} We assume the attacker is restricted to interacting with the CAPTCHA through the graphical interface, without using techniques such as reverse engineering, JavaScript decompiling, or direct code analysis. In this work, we primarily consider that the adversary abuses the capabilities of multimodal LLMs with their reasoning capabilities and object recognition capabilities. These LLMs can be utilized not only to solve CAPTCHAs but also to automate the entire attack process—from selecting target websites to registering accounts—enabling a highly efficient and scalable attack pipeline.


% Please add the following required packages to your document preamble:
% \usepackage{multirow}
\begin{table}[t]
    \centering
    % \scriptsize
    % \footnotesize
    \tabcolsep=1.5pt
    \renewcommand{\arraystretch}{0.92} 
    \caption{Experimental results of applying the multi-model LLMs over the selected CAPTCHAs.}
 % Please add the following required packages to your document preamble:
% \usepackage{multirow}
% Please add the following required packages to your document preamble:
% \usepackage{multirow}
\resizebox{\linewidth}{!}{
    \begin{tabular}{cc|cc|cc}
    \hline
    \multicolumn{2}{c|}{\textbf{Method}}                                                            & \multicolumn{2}{c|}{\textbf{Zero-Shot}}                                  & \multicolumn{2}{c}{\textbf{COT}}                                         \\ \hline
    \multicolumn{2}{c|}{\textbf{Metric}}                                                            & \multicolumn{2}{c|}{\textbf{Success Rate}}                               & \multicolumn{2}{c}{\textbf{Success Rate}}                                \\ \hline
    \multicolumn{2}{c|}{\textbf{Model}}                                                             & \multicolumn{1}{c|}{\textbf{GPT4o}} & \textbf{Gemini} & \multicolumn{1}{c|}{\textbf{GPT4o}} & \textbf{Gemini} \\ \hline
    \multicolumn{1}{c|}{\multirow{4}{*}{\textbf{Text-based}}}  & \textbf{Simplest}          & 76.66\%                                    & 73.33\%                     & 90.00\%                                    & 83.33\%                     \\
    \multicolumn{1}{c|}{}                                              & \textbf{Overloaping}       & 66.66\%                                    & 60\%                        & 70.00\%                                    & 60.00\%                     \\
    \multicolumn{1}{c|}{}                                              & \textbf{Noise}             & 70.00\%                                    & 73.33\%                     & 73.33\%                                    & 66.66\%                     \\
    \multicolumn{1}{c|}{}                                              & \textbf{Noise+Overloaping} & 36.66\%                                    & 23.33\%                     & 50.00\%                                    & 43.33\%                     \\ \hline
    \multicolumn{1}{c|}{\multirow{2}{*}{\textbf{Image-based}}} & \textbf{reCAPTCHA}         & 40.00\%                                    & 33.33\%                     & 50.00\%                                    & 23.33\%                     \\
    \multicolumn{1}{c|}{}                                              & \textbf{hCAPTCHA}          & 40.00\%                                    & 36.66\%                     & 43.33\%                                    & 30.00\%                     \\ \hline
    \multicolumn{1}{c|}{\multirow{5}{*}{\textbf{Reasoning}}}   & \textbf{Angular}           & 13.33\%                                    & 0.00\%                      & 13.33\%                                    & 0.00\%                      \\
    \multicolumn{1}{c|}{}                                              & \textbf{Gobang}            & 0.00\%                                     & 0.00\%                      & 6.66\%                                     & 3.33\%                      \\
    \multicolumn{1}{c|}{}                                              & \textbf{IconCrush}         & 0.00\%                                     & 0.00\%                      & 16.66\%                                    & 10.00\%                     \\
    \multicolumn{1}{c|}{}                                              & \textbf{Space}             & 46.66\%                                    & 26.66\%                     & 53.33\%                                    & 26.66\%                     \\
    \multicolumn{1}{c|}{}                                              & \textbf{Space Reasoning}   & 33.33\%                                    & 20.00\%                     & 40.00\%                                    & 23.33\%                     \\ \hline
    \multicolumn{2}{c|}{\textbf{Average}}                                                           & \textbf{38.48\%}                           & \textbf{31.51\%}            & \textbf{46.06\%}                           & \textbf{33.63\%}            \\ \hline
    \end{tabular}
    }

\label{tab:Empirical-of-LLM}
\end{table}

\begin{table}[t]
    \centering
    % \scriptsize
    % \footnotesize
    \tabcolsep=1.5pt
    \renewcommand{\arraystretch}{0.92} 
    \caption{Experimental results of applying the multi-model LLMs over the selected CAPTCHAs.}
    \begin{tabular}{c|cccc}
    \hline
    \textbf{\# of Attempt}       & \multicolumn{1}{c|}{\textbf{1}} & \multicolumn{1}{c|}{\textbf{2}} & \multicolumn{1}{c|}{\textbf{3}} & \textbf{>3} \\ \hline
    \textbf{Text-based CAPTCHA}  & 47.82\%                                     & 39.13\%                                      & 8.69\%                                      & 4.34\%                     \\
    \textbf{Image-based CAPTCHA} & 30.43\%                                     & 56.52\%                                      & 4.34\%                                      & 8.69\%                     \\
    \textbf{Reasoning CAPTCHA}   & 21.73\%                                     & 43.47\%                                      & 21.73\%                                     & 13.04\%                    \\ \hline
    \textbf{Average}             & \textbf{33.33\%}                            & \textbf{46.37\%}                             & \textbf{11.59\%}                            & \textbf{8.69\%}            \\ \hline
    \end{tabular}
\label{tab:user-study}
\end{table}

\section{Empirical Study}
\label{sec:empirical_study}

We first conduct a systematic empirical study to assess the effectiveness of LLMs in identifying both traditional and modern CAPTCHAs. The full potential of LLMs in this area remains largely unexplored. Additionally, to address the knowledge gap among human users regarding CAPTCHAs, we designed a user study to assess user performance across various CAPTCHA challenges. This investigation is structured around two research questions:

\begin{itemize}[leftmargin=*]
    \setlength\itemsep{0pt}
    \item \textbf{RQ1 (Effectiveness):} How effective are LLMs in accurately solving CAPTCHAs, and what types of errors are they most likely to make?
    \item \textbf{RQ2 (User Study):} Are human users able to effectively resolve various categories of CAPTCHA challenges, and what specific obstacles do they encounter throughout the process?
\end{itemize}

In the following of this section, we address the two research questions through two sets of experiments. 

\subsection{Effectiveness of LLMs in Solving CAPTCHAs}



\noindent\textbf{CAPTCHA Categorization.} Different from other works~\cite{deng2024oedipus,searles2023empirical}, we cover all categories of visual CAPTCHAs. Within each category, there are different designs from different vendors. Therefore, we compiled widely deployed CAPTCHAs available online and organized them into the following detailed subcategories.

\begin{itemize}[leftmargin=*]
    \item \textbf{Text-based CAPTCHAs.} We collect different types of text-based CAPTCHAs that requires users to recognize a series of letters or characters. After survey, we conclude four types of them available now. (1) \textbf{Simplest Text-based CAPTCHAs}, shown in Figure~\ref{fig:Text-based}(a), is the simplest text-based CAPTCHAs, which is also the most popular CAPTCHAs online. This type of challenge can be solved easily by traditional CAPTCHA solvers. These typically feature clear, unaltered text, making them vulnerable to basic image recognition techniques. 
    (2) \textbf{Noisy Text-based CAPTCHAs}, shown in Figure~\ref{fig:Text-based}(b), introduce visual noise, such as random lines, dots, or distortions into the text CAPTCHA, which can interfere with traditional CAPTCHA solvers. Despite the added complexity, they still primarily ensure that users could recognize the contents within.
    (3) \textbf{Overlapping Text-based CAPTCHAs}, shown in Figure~\ref{fig:Text-based}(c), are a type of text-based CAPTCHAs that involve texts where characters are overlapped with each other at different angles. While this writing style is totally recognizable to humans, it is hard for traditional solvers~\cite{ye2018yet} that relies on segmentation strategies to solve. 
    (4) \textbf{Noise-enhanced Overlapping Text-based CAPTCHAs}, shown in Figure~\ref{fig:Text-based}(d), are the type of challenges combine both visual noise and overlapping texts, which significantly increases the difficulty for traditional CAPTCHA solvers to counter. 

    \item \textbf{Image-based CAPTCHAs.} In addition to the traditional text-based CAPTCHAs, more recent ones include images that tests the common sense of users as a type of challenge. We conclude two types of basic image-based CAPTCHAs.
    (1) \textbf{reCAPTCHA} presents users with tasks like selecting images (image classification) that contain specific objects, such as traffic lights or crosswalks, or verifying street signs. Vastly adopted by Google, it is the most common types of CAPTCHA that has been well researched. There are three versions of reCAPTCHAs, with similar image patterns but different underlying mechanisms to counter traditional automated solutions such as JavaScript reverse engineering. 
    (2) \textbf{hCAPTCHA} involves more detailed image recognition tasks, requiring users to have a stronger ability to understand the prompts (e.g., selecting images that contain wheels).
    
    \item \textbf{Reasoning-based CAPTCHAs} 
    are new emerging category of challenges that aims to counter the automated solvers powered by deep learning methods. After survey, we identify three types of reasoning-based CAPTCHAs.
    (1) \textbf{Rotation CAPTCHAs}, also known as Angular by their developers, require users to adjust an object’s orientation to align with a reference object. 
    As shown in Figure~\ref{fig:Reasoning-based}(a), users need to properly recognize the orientation of two different objects (the finger and the lamb in this example) to solve the challenge. There are two versions of Rotation CAPTCHAs available in the market now, both devleoped by Arkose Labs. 
    (2) \textbf{Bingo CAPTCHAs (Gobang \& IconCrush)} is a new type of reasoning-based CAPTCHA also developed by Arkose Labs.
    As seen in Figure~\ref{fig:Reasoning-based}(b), this type of challenge tasks users with identifying and rearranging elements on a board to create a line of matching items. The types of elements and the rules for manipulation can differ widely based on the provider. For instance, in Figure~\ref{fig:Reasoning-based}(b), users can swap any two items without restriction, while in Figure~\ref{fig:Reasoning-based}(c), swaps are limited to adjacent items, illustrating the range of variation in this type of CAPTCHA.
    (3) \textbf{3D Logical CAPTCHAs}, as demonstrated in Figure~\ref{fig:Reasoning-based}(d) and Figure~\ref{fig:Reasoning-based}(e), requires users to choose an object from a 3D environment. This process requires users to identify the logical relationships tied to attributes like shape, color, and orientation of the objects within the challenge. For instance, in Figure~\ref{fig:Reasoning-based}(d), users must identify the number 0 that aligns with the orientation of a yellow letter W, whereas Figure~\ref{fig:Reasoning-based}(e) asks users to select the larger object positioned to the left of a green object.
    
\end{itemize}





\noindent\textbf{Dataset Collection.} To rigorously assess the ability of LLMs to solve CAPTCHAs, we include the three types of CAPTCHAs as discussed in the Background: text-based, image-based and reasoning-based CAPTCHAs. In particular, we exclude audio CAPTCHAs due to their limited usage online~\cite{fanelle2020blind}, which is mainly for visually impaired people. Furthermore, our study emphasizes real-world scenarios, so all CAPTCHAs used were collected from website applications. Consequently, we built a dataset comprising three types of CAPTCHA (text-based CAPTCHAs shown in Figure~\ref{fig:Text-based}, image-based CAPTCHAs shown in Figure~\ref{fig:Image-based} and reasoning-based CAPTCHAs shown in Figure~\ref{fig:Reasoning-based} \revision{each containing 30 capthchas per subdivision.}.


\noindent \textbf{Methodology.}  To evaluate these CAPTCHAs, we employ two powerful LLMs (Gemini 1.5 pro 2.0 and GPT4-o) using both Zero-Shot and Chain-of-Thought (COT) methodologies. Each CAPTCHA category presents a unique set of challenges that require customized solution strategies. As a result, we utilize different LLM prompts to predict the outcomes of various CAPTCHAs, measuring success rates as our primary metric. We manually analyze each LLM response to ensure the accuracy of the results. In the zero-shot approach, a solution is considered correct only if the LLM outlines the exact procedure to solve the CAPTCHA. In contrast, in the CoT approach, a substep is deemed successful if the LLM's proposed solution for that specific sub-step is accurate.

\noindent\textbf{Result Analysis.} Table~\ref{tab:Empirical-of-LLM}—with GPT4o representing GPT-4o and Gemini representing Gemini 1.5 pro 2.0—presents our evaluation of LLMs' effectiveness in solving CAPTCHAs. Using a zero-shot approach, the LLMs successfully solve most text-based CAPTCHAs, except for those with overlapping characters or significant noise, while their accuracy drops to only 40\% for image-based CAPTCHAs. Moreover, LLMs face challenges with reasoning-based CAPTCHAs due to their limited reasoning capabilities; however, employing COT prompting significantly enhances their performance in identifying these types of CAPTCHAs. These findings show the growing threat that advancements in LLMs pose to web security, suggesting that current CAPTCHA methods may no longer be sufficiently secure.

\begin{center}
    \setlength{\fboxrule}{1pt}
    %\fbox
    \fcolorbox{lightgray}{mygray}{%
      \parbox{0.47\textwidth}{%
        \textbf{Answer to RQ1:}
        \emph{Our verification experiment reveals that (1) LLMs perform better on text-based CAPTCHAs compared to image-based and reasoning-based CAPTCHAs; and (2) although LLMs struggle with complex reasoning CAPTCHAs, their performance significantly improves when employing the Chain-of-Thought (CoT) strategy. This suggests that with reasoning chains, LLMs have the potential to overcome these challenges. Consequently, this indicates that current CAPTCHAs may no longer be as secure as intended.}
        }%
    }
\end{center}


\begin{figure*}[!t]
	\centering
    \includegraphics[width=0.9\linewidth]{Figure/Overview_of_IllusionCAPTCHA.eps}
	\caption{Overview of IllusionCAPTCHA}
     % \vspace{-10pt}
	\label{fig:Illusion-based}
\end{figure*}


\subsection{User Study}

To investigate user behavior, we designed a questionnaire-based study. Because some CAPTCHAs cannot be reliably retrieved from their original sources, we constructed the study by extracting CAPTCHA images directly from their native web applications and manually annotating the correct responses. All images were drawn from the dataset we mentioned before.

\noindent\textbf{User Study Settings.} Our questionnaire allocates each participant 1 minute to solve a CAPTCHA. If they cannot complete it within that time, they must attempt it again until they succeed. During this process, we record both successful and failed attempts. Finally, we have 23 human participants in our study.

\noindent\textbf{Result Analysis.} Table~\ref{tab:user-study} presents the results of our user study, revealing that most participants were unable to solve the CAPTCHA on their first attempt; notably, both image-based and reasoning-based CAPTCHAs proved particularly challenging, with some individuals requiring more than three attempts to successfully pass them.


\begin{center}
    \setlength{\fboxrule}{1pt}
    %\fbox
    \fcolorbox{lightgray}{mygray}{%
      \parbox{0.47\textwidth}{%
        \textbf{Answer to RQ2:}
        \emph{The result of our user study reveals that (1) While reasoning-based CAPTCHAs pose significant challenges for AI systems, they are also difficult for human users. Hence, these CAPTCHAs can easily frustrate users, leading to diminished patience during their attempts. (2) Human users frequently make the same mistakes as LLMs, highlighting the need to develop methods that can effectively distinguish between LLMs and human users.}
        }%
    }
\end{center}












\section{Methodology}

As illustrated in Figure~\ref{fig:Illusion-based}, IllusionCAPTCHA generates CAPTCHA challenges through a three-step process. 
First, it blends a base image with a user-defined prompt, such as ``huge forest,'' to create a visual illusion that obscures the original content. With the prompt, the output image will be looked like the things in the prompt, hiding its true content from base image. This results in images that, while recognizable to humans, can confuse AI systems. 
Second, multiple-choice options are generated based on the altered images, forming the CAPTCHA challenge options.
Our empirical study indicates that humans may occasionally make errors similar to those of LLMs, suggesting that relying solely on illusionary images may not be sufficient to distinguish human users from bots. 
Therefore, we incoperate the third step of ``Inducement Prompt'' to induce our LLM-based attackers to choose the intended choice. Moreover, we utilize multimodel question to increase difficulty for attackers but easy for human users to identify. Below we detail the design of IllusionCAPTCHA.
%

\subsection{Illusionary Image Generation}

\begin{figure}[!t]
	\centering
    \includegraphics[width=0.8\linewidth]{Figure/Picture_1.eps}
	\caption{An example of the original and illusionary image}
     % \vspace{-10pt}
	\label{fig:Illusion-example}
\end{figure}
\label{sec:method}
The first objective is to create illusionary images that are easily recognizable by humans but difficult for AI systems to identify. This process involves tackling two primary challenges: (1) maintaining the context of the base image, and (2) add disturbance to the image particularly effective for AI systems to interfere with their capabilities while maintaining recognizability for humans. 

To address the first challenge, we employ an illusion diffusion model~\cite{AP123}, which generates images by blending two different types of content. Built upon ControlNet~\cite{zhang2023adding}, a framework that allows precise control over image generation through conditional inputs, this model ensures that the resulting images remain accessible to human viewers while being challenging for automated systems to interpret.
Figure~\ref{fig:Illusion-example} shows how a normal image is transferred into an illusionary one.
However, not all generated images will effectively balance recognizability for humans while fooling AI vision. To overcome the second challenge, we first generate 50 sample images using different seeds, all within the range of 0 to 5, at a fixed illusion strength level of 1.5—an optimal value for human identification in this context. We then calculate the cosine similarity between each generated image and the base image, selecting the one with the lowest similarity, which can be seen as the most diffcult images for bots to identify.

To enhance the perceptibility of the generated images, we develop tailored strategies for two types of illusion-based CAPTCHAs: text-based CAPTCHAs and image-based CAPTCHAs. In the first scenario, the base image contains a clear, readable word embedded within an illusion. To ensure that human users can still recognize the text with minimal effort, we opt for simple, familiar English words such as ``day'' or ``sun.'' 
In the second scenario, the base image features a well-known, easily recognizable character or object, such as an iconic symbol or a famous location (e.g. ``Eiffel Tower''). This ensures that human users can quickly identify the content, even with added illusionary elements. 
These strategies aim to strike a balance between maintaining human usability and introducing complexity that misleads AI systems. 


\subsection{Options Setup} 

Our options have been meticulously crafted to safeguard against LLM-based attacks. In our CAPTCHA, we offer four distinct options. One option represents the correct answer usually the hidden content of our image, while another is the input prompt we utilize in the generation of our image. The remaining two options consist of detailed descriptions of our prompt part without the correct answer, intentionally crafted without referencing any content from our true answer.

Unlike traditional CAPTCHAs that require users to type text or select multiple images to answer a question, our CAPTCHA asks users to choose the correct description of an image. This design simplifies the process by offering a hint, making it easier for users to identify the correct answer without needing to click through multiple images.

Compared to text-based CAPTCHAs, ours is more user-friendly, as it avoids the challenges posed by vague images. Additionally, in contrast to hCAPTCHA and reCAPTCHA, our approach reduces the difficulty of making a selection. Unlike reasoning-based CAPTCHAs that require users to manipulate images, which can lead to frustration, our design eliminates the need for such interactions, further improving user experience.


\subsection{Inducement Prompt}

Building on our empirical study, we discover that both LLMs and human users tend to make similar errors when presented with certain types of CAPTCHAs. Additionally, human users often require a second attempt to pass the CAPTCHA successfully. As a result, relying on a single question to differentiate between AI and human users proves insufficient.
To address this issue, we designed a system that aims to lure potential attackers, such as multimodal LLMs, into selecting predictable, bot-like answers. Our CAPTCHA format uses multiple-choice questions, each offering four answer options.

Our strategy centers on the idea to trick the LLM-based adversary to select the option that describes the illusionary element added, which is the object that LLMs typically fails to capture. Research~\cite{hu2024bliva} has shown that LLMs typically describe images with long, detailed sentences. To exploit this, we include one option that features an intentionally elaborate, detailed description of the illusionary elements in the image (e.g., "a vast forest filled with birds, depicting a beautiful and serene scene").

Additionally, to reduce the difficulty for human users, we embed hints within the questions that guide them toward the correct answer. Therefore, these hints(e.g. Tell us the \textbf{true} and \textbf{detailed} answer of this image) are crafted to trigger hallucinations in LLMs, further increasing the likelihood that bots will select incorrect responses, although they are in the prompt that attacker sets before.








\section{Evaluation}
To evaluate the performance of our IllusionCAPTCHA, we have structured our evaluation around four research questions:

\begin{itemize}[noitemsep,leftmargin=*] 
\item \textbf{RQ3: Human Identification of Illusionary Images.} Can the illusionary images generated by IllusionCAPTCHA remain identifiable to human users? 

\item \textbf{RQ4: LLM Deception by Illusionary Content.} Can the illusionary content effectively deceive LLMs into selecting a false answer? 

\item \textbf{RQ5: Inducement Prompts Effectiveness.} Can the CAPTCHA structure we designed compel bots to make targeted choices? 

\item \textbf{RQ6: Human Attempts to Pass CAPTCHA.} How many attempts do human users require to successfully pass our designed CAPTCHA? 
\end{itemize}






\subsection{RQ3: Human Identification of Illusionary Images}

\noindent\textbf{Motivation.} In this section, we examine whether illusionary images can effectively convey information to human users, a critical factor since a CAPTCHA image must clearly communicate its intended message to its target audience.

\noindent\textbf{Method.} To address RQ3, we designed a questionnaire to assess human users' ability to identify illusionary images. The questionnaire comprises two types of images—text-based and image-based illusionary images—each consisting of five samples. All ten samples were generated using the method described in Section~\ref{sec:method}, and to avoid copyright issues, the base images were produced using a diffusion technique. Below, we provide the details of our questionnaire.



\begin{table}
  \centering
\caption{\textcolor{black}{Effect of duplicate code removal on structural metrics. (+ve) indicates positive impact; (-ve) indicates negative impact; (-) indicates metric remains unaffected, \textbf{bold} indicates statistical significance; \textit{italic} indicates improvement.}}
\label{Table:Metrics Suites and Metrics Tools Summary}
%\begin{sideways}
\begin{adjustbox}{width=1.0\textwidth,center}
%\begin{adjustbox}{width=\textheight,totalheight=\textwidth,keepaspectratio}
\begin{tabular}{lllll}\hline
\toprule
\bfseries Quality Attribute & \bfseries Metric & \bfseries Impact & \bfseries \textit{p}-value & \bfseries Cliff's delta ($\delta$) \\
\midrule
%\multicolumn{2}{l}{\textbf{\textit{Internal Quality Attribute }}}\\
%\midrule
Cohesion &  LCOM5  & +ve & \textit{\textbf{7.72e-41}} & 0.54 (Large)
%(Small)   
\\ 
Coupling &  \cellcolor{gray!30}CBO  & \cellcolor{gray!30}+ve & \cellcolor{gray!30}\textit{\textbf{9.49e-76}} & \cellcolor{gray!30}0.6 (Large)
%(Small) 
\\
         &  RFC & +ve & \textit{\textbf{1.25e-68}}  & 0.55 (Large)

\\
         &  \cellcolor{gray!30}NII & \cellcolor{gray!30}-ve &  \cellcolor{gray!30}\textbf{0} & \cellcolor{gray!30}0.47 (Large)

\\
         &  NOI & +ve & \textit{\textbf{0}}  & 0.26 (Small)

\\
Complexity &  \cellcolor{gray!30}CC & \cellcolor{gray!30}- & \cellcolor{gray!30}\textbf{0} & \cellcolor{gray!30}0.14 (Small)

\\
           &  WMC & +ve & \textit{\textbf{6.51e-70}} & 0.5 (Large)

\\
         &  \cellcolor{gray!30}NL &  \cellcolor{gray!30}- &  \cellcolor{gray!30}\textbf{3.92e-05}&  \cellcolor{gray!30}0.03 (Negligible)

\\
         &  NLE &  - & \textbf{0.004}  & 0.02 (Negligible)

\\
         &  \cellcolor{gray!30}HCPL & \cellcolor{gray!30}+ve &  \cellcolor{gray!30}\textit{\textbf{0}} &  \cellcolor{gray!30}0.14 (Negligible)

\\
         &  HDIF & +ve & \textit{\textbf{0}}  &  0.08 (Negligible)

\\
         &  \cellcolor{gray!30}HEFF & \cellcolor{gray!30}+ve &  \cellcolor{gray!30}\textit{\textbf{2.45e-271}} &  \cellcolor{gray!30}0.13 (Negligible)

\\
         &  HNDB & +ve &  \textit{\textbf{1.07e-266}} & 0.13  (Negligible)

\\
         &  \cellcolor{gray!30}HPL & \cellcolor{gray!30}+ve & \cellcolor{gray!30}\textit{\textbf{0}} & \cellcolor{gray!30}0.13  (Negligible)

\\
         &  HPV & +ve & \textit{\textbf{0}}  & 0.14  (Negligible)

\\
         &  \cellcolor{gray!30}HTRP & \cellcolor{gray!30}+ve &  \cellcolor{gray!30}\textit{\textbf{2.48e-271}} & \cellcolor{gray!30}0.13  (Negligible)

\\
         &  HVOL & +ve & \textit{\textbf{0}}  &  0.13  (Negligible)

\\
         &  \cellcolor{gray!30}MIMS & \cellcolor{gray!30}+ve & \cellcolor{gray!30}\textit{\textbf{7.23e-227}}  &  \cellcolor{gray!30}0.13  (Negligible)

\\
         &   MI & +ve &  \textit{\textbf{7.22e-227}} &  0.13  (Negligible)

\\
         &   \cellcolor{gray!30}MISEI & \cellcolor{gray!30}+ve & \cellcolor{gray!30}\textit{\textbf{0}} & \cellcolor{gray!30}0.16  (Small)

\\
         &   MISM &  +ve&  \textit{\textbf{0}}  & 0.16  (Small)

\\
Inheritance &   \cellcolor{gray!30}DIT & \cellcolor{gray!30}-ve & \cellcolor{gray!30}\textbf{3.81e-199} & \cellcolor{gray!30}0.6 (Large) 
 
\\
            &  NOC & +ve & \textbf{\textit{3.61e-130}} & 0.83 (Large)  

\\
            &  \cellcolor{gray!30}NOA & \cellcolor{gray!30}-ve & \cellcolor{gray!30}\textbf{2.37e-196}  & \cellcolor{gray!30}0.63 (Large)
 
\\ 
Design Size &  LOC & +ve & \textbf{\textit{0}} &   0.14 (Small)

\\
         &  \cellcolor{gray!30}TLOC & \cellcolor{gray!30}+ve &   \cellcolor{gray!30}\textit{\textbf{0}} &  \cellcolor{gray!30}0.16 (Small)

\\
            &  LLOC &  +ve &  \textbf{\textit{0}} &   0.13 (Negligible)

\\
         &  \cellcolor{gray!30}TLLOC & \cellcolor{gray!30}+ve & \cellcolor{gray!30}\textit{\textbf{0}}  &   \cellcolor{gray!30}0.15 (Small)

\\
            &   CLOC & - &  \textbf{1.43e-05} &   0.02 (Negligible)

\\
            &  \cellcolor{gray!30}NPM & \cellcolor{gray!30}- & \cellcolor{gray!30}\textbf{4.42e-193} & \cellcolor{gray!30}0.5 (Large)

\\
         &  NOS &  +ve&  \textit{\textbf{0}} &  0.07 (Negligible)

\\
         &  \cellcolor{gray!30}TNOS & \cellcolor{gray!30}+ve & \cellcolor{gray!30}\textit{\textbf{0}}  &  \cellcolor{gray!30}0.08 (Negligible)

\\
\bottomrule
\end{tabular}
\end{adjustbox}
%\end{sideways}
\end{table}
\begin{comment}
    


%\begin{sidewaystable}
\begin{table*}

\caption{The impact of duplicate code removal on quality metrics.}
\label{composites}


\fontsize{10}{14}\selectfont
	\tabcolsep=0.1cm
\resizebox{\textwidth}{!}{
\begin{tabular}{llcccccccccccccccccccccc} \toprule
\multicolumn{1}{c}{}                             & \multicolumn{1}{c}{}                          & \multicolumn{3}{c}{Cohesion}                                                                                                            & \multicolumn{5}{c}{Coupling}                                                                                                          & \multicolumn{4}{c}{Complexity}                                                                                                                                                         & \multicolumn{3}{c}{Inheritance}                                                         & \multicolumn{7}{c}{Design Size}                                                                                                                                                                                                                                                                         \\
\multicolumn{1}{c}{\multirow{-2}{*}{}} & \multicolumn{1}{c}{\multirow{-2}{*}{Measure}} & LCOM5                                        &                                          &                                          & CBO                                         & RFC                                         &     &    &                 & WMC                                         & CC                                         &                                  &                                          & DIT                                         & NOC                 &   NOA            & SLOC                                         & LLOC                                        & CLOC                &      NPM                                  &                                        &                                          &                                         \\ \hline
                                                 & Refactoring Impact                            & 0                                           & \cellcolor[HTML]{0350F8}1                   & \cellcolor[HTML]{0350F8}1                   & 0                                           & \cellcolor[HTML]{0350F8}-12                 & 0        & 0        & 0                   & \cellcolor[HTML]{0C56F7}-5                  & \cellcolor[HTML]{1B61F8}-3                  & \cellcolor[HTML]{1B61F8}-3                  & \cellcolor[HTML]{A3BBED}-1                  & 0                                           & 0                   & 0                   & \cellcolor[HTML]{7DA2F1}-2                  & 0                                           & 0                   & 7                                           & \cellcolor[HTML]{F8ADAD}2                   & \cellcolor[HTML]{0350F8}-20                 & 0                                           \\
                                                 & Behavior                                      & -                                           & \cellcolor[HTML]{0350F8}haut                & \cellcolor[HTML]{0350F8}haut                & -                                           & \cellcolor[HTML]{0350F8}bas                 & -        & -        & -                   & \cellcolor[HTML]{0C56F7}bas                 & \cellcolor[HTML]{1B61F8}bas                 & \cellcolor[HTML]{1B61F8}bas                 & \cellcolor[HTML]{A3BBED}bas                 & -                                           & -                   & -                   & \cellcolor[HTML]{7DA2F1}bas                 & -                                           & -                   & haut                                        & \cellcolor[HTML]{F8ADAD}haut                & \cellcolor[HTML]{0350F8}bas                 & -                                           \\
\multirow{-3}{*}{}                         & P-value ($\delta$)                            & 1 (N)                                       & \cellcolor[HTML]{0350F8}\textless{}0.05 (S) & \cellcolor[HTML]{0350F8}\textless{}0.05 (S) & 1 (N)                                       & \cellcolor[HTML]{0350F8}\textless{}0.05 (M) & 0.07 (N) & 0.07 (N) & \textless{}0.05 (N) & \cellcolor[HTML]{0C56F7}\textless{}0.05(S)  & \cellcolor[HTML]{1B61F8}\textless{}0.05 (S) & \cellcolor[HTML]{1B61F8}\textless{}0.05(S)  & \cellcolor[HTML]{A3BBED}\textless{}0.05(N)  & 0.08 (N)                                    & 0.23 (S)            & 0.19 (N)            & \cellcolor[HTML]{7DA2F1}0.06 (N)            & 0.16 (N)                                    & \textless{}0.05 (N) & 0.09 (S)                                    & \cellcolor[HTML]{F8ADAD}\textless{}0.05 (N) & \cellcolor[HTML]{0350F8}\textless{}0.05 (M) & 0.12 (N)                                    \\
 \bottomrule
\end{tabular}
}
\end{table*}
%\end{sidewaystable}

\end{comment}

% \noindent\textbf 

\begin{itemize}[leftmargin=*]
    \item \textbf{Perception of Illusion (Mandatory Question):} ``Do you notice any illusionary effect in this image?''
    
    \item \textbf{Uncertainty Clarification (Optional Question):} ``If you are uncertain, could you please explain why?''
    
    \item \textbf{Confidence Level (Mandatory Question):} ``If you answered `Yes' or `No' regarding the perception of an illusion, how confident are you in your response? Please rate on a scale from 1 (least confident) to 5 (most confident).''
    
    \item \textbf{Image Description (Mandatory Question):} ``What do you observe in this image?''
    
    \item \textbf{Description Confidence (Mandatory Question):} ``How confident are you in your description of the image? Rate from 1 (least confident) to 5 (most confident).''
\end{itemize}


\noindent\textbf{Result Analysis.} The key results from this survey are summarized in Table~\ref{tex:RQ1}, 10 participants taking part in this questionnaire. In terms of visibility, the data reveals that human users were able to accurately identify 83\% of illusionary text and 88\% of illusionary images on average. This suggests a relatively strong ability to recognize deceptive or distorted content in both formats of 
illusionary content.

Additionally, the confidence metric provides insight into the users' perception of their own performance. The majority of participants reported high levels of confidence in their selections, indicating that they believed they were making correct judgments, even when faced with illusionary or complex content. This confidence may play a crucial role in how users engage with tasks that involve visual and textual interpretation, highlighting the special structure of human vision.

\subsection{RQ-2 How do different configurations affect the effectiveness of LLMs?}
\label{sec:rq2}


\noindent 
\textbf{Impact of different example numbers.}
As previous studies~\cite{brown2020language,A3CodGen} have shown, the number of examples provided has a significant impact on LLMs' performance. 
To explore this, we adjust the number of examples while keeping other parameters and hyperparameters constant to ensure a fair comparison.
We do not conduct experiments in a zero-shot setting, as LLMs may generate unnormalized outputs without a prompt template, which would hinder automated extraction. 
From Fig.~\ref{fig:ablation}, we observe that as the number of examples increases, both the average token length and time cost rise sharply, while the improvement in Pass@k remains modest.
Based on these findings, we perform our ablation studies (Table~\ref{tab:rq1} and \ref{tab:ablation}) using a one-shot setting in \mytitle.


\noindent 
\textbf{Impact of different selection strategies.}
% Our case study reveals that the RAG method improves the performance of LLMs.
RAG retrieves relevant codes from a retrieval database and supplements this information for code generation~\cite{parvez2021retrieval}. 
To ensure a fair comparison, we set the number of examples to one and evaluated the results of RAG versus random selection on the same LLM (i.e., DeepSeek-V3). From Table~\ref{tab:ablation}, Pass@1 and Compile@1 are higher when RAG is enabled, indicating that it improves the effectiveness of code generation.


\begin{figure}[htbp]
    \centering
    \includegraphics[width=\linewidth]{figs/ablation.pdf}
    \caption{Performance of Qwen2.5-Coder-7B. The x-axis represents the number of shots.}
    \label{fig:ablation}
\end{figure}
\vspace{-0.2cm}

\noindent 
\textbf{Impact of Context Information.}
Since that relevant context typically enhances performance in other programming languages, we conduct an ablation study to examine the influence of context on the quality of LLM-generated contracts. Table~\ref{tab:ablation} shows that providing context information improves both Pass@1 and Compile@1. 
However, there is no clear correlation between gas fees, vulnerability rate, and the presence of context information.


% \vspace{-0.1cm}
\begin{table}[htbp]
    \centering
    \caption{Ablation study on the effect of RAG and Context on DeepSeek-V3's (one-shot) performance.}
    \resizebox{\linewidth}{!}
    {
        \begin{tabular}{cc|cccc}
        \toprule
        RAG & Context & Pass@1 & Compile@1 & Fee & Vul \\
        \midrule
        \ding{51} & \ding{51} & \textbf{21.72\%}& \textbf{53.35\%}&  \textbf{-7525}& 26.61\% \\ 
        \ding{55} & \ding{51} & 20.24\% & 51.08\% & 3828& \textbf{23.68\%}\\ 
        \ding{51} & \ding{55} & 21.28\% & 52.54\% & -708& 26.13\%\\
        \ding{55} & \ding{55} & 20.17\% & 50.32\% & 768& 26.83\%\\   
        \bottomrule
        \end{tabular}
    }
    \label{tab:ablation}
\end{table}
% \vspace{-0.4cm}
\subsection{RQ-3 Gas Efficiency and Scalability Analysis}
\label{sec:rq3}


\noindent
\textbf{Objective.}
In \Chain, we take \textit{Gas Fee} and \textit{Scalability} into consideration.
In blockchain query databases, the gas fee and scaleability are essential since the former will largely impact the practicality of the designed methods (i.e., a higher gas fee means more money spent when running queries) and the latter will impact compatibility (i.e., poor compatibility will lead to massive code modification when adjusting to another blockchain system).

\noindent
\textbf{Experimental Design.}
First, we investigate the impact of different data structures on gas fee and design four variants of \Chain. 
vChain+$_{F}$ represents vChain+ replicated and enhanced to support multimodal queries.
vChain+$_{O}$ is vChain+$_{F}$ without the off-chain query module.
MulChain$_{BT}$ is MulChain with its underlying data structure replaced with B\(+\)Tree for time range queries.
MulChain$_{BH}$ is MulChain with its underlying data structure replaced with our gas-efficient BHashTree for time range queries.
MulChain$_{T}$ is MulChain with its underlying data structure replaced with our verifiable trie for fuzzy queries.
This approach allows us to examine the individual effects of each component.


\noindent
\textbf{Results.} We discuss the results from the aspects of gas consumption and scalability, respectively.




\noindent
\textbf{\underline{Gas Consumption Analysis.}}
The average gas fees for BHashTree are much lower than those of vChain+$_{F}$, slightly lower than B\(+\)Tree-based methods, as illustrated in Fig.~\ref{fig:Gas Consumption}(a).
Fig.~\ref{fig:Gas Consumption}(b) presents the average gas fees of the trie in comparison to the accumulator from vChain+. 
Notably, the gas consumption of MulChain$_{T}$ exceeds that of vChain+$_{F}$ due to our strategic trade-off of space for time. 
We deem this trade-off acceptable, as the reduction in query latency is particularly valuable in the context of fuzzy queries on blockchains.




\begin{figure}[htbp]
    \centering
    \includegraphics[width=.7\linewidth]{figures/Gas_BHT_BT.pdf}
    % \captionsetup{skip=0pt}
    \caption{Gas Consumption}
    \label{fig:Gas Consumption}
\end{figure}


\noindent
\textbf{\underline{Scalability Analysis.}}
\Chain supports all six SQL primitives (i.e., insert, delete, update, simple, time range, fuzzy queries) on Ethereum and FISCO BCOS. 
In contrast, the CRUD Service of FISCO BCOS does not support time range and fuzzy queries. 
We test \Chain using time range queries on BTC and ETH datasets. 
From Fig.~\ref{fig:FISCO BCOS}(a), we can see that \Chain undergoes a decline of up to 3.78\% when the number of blocks grows.
In Fig.~\ref{fig:FISCO BCOS}(b), we observe that MulChain$_{BT}$ is faster on the BTC dataset than on ETH for block counts below 128 and above 1024. 
This performance difference is due to the varying timestamp densities of the two datasets and the initialization cost of the B\(+\)Tree.


\begin{figure}[htbp]
    \centering
    \includegraphics[width=.7\linewidth]{figures/FISCO_all.pdf}
    % \captionsetup{skip=0pt}
    \caption{Query Performance on FISCO BCOS}
    \label{fig:FISCO BCOS}
\end{figure}


\intuition{
{\bf Answer to RQ-3}: 
(1) The five data structures (i.e., accumulator of vChain+$_{F}$, vChain+$_{O}$, MulChain$_{BT}$, MulChain$_{BH}$ and MulChain$_T$) contribute substantially to \Chain, and combining them achieves the best performance of blockchain query on different scenarios.
(2) The gas fee of MulChain$_{T}$ exceeds that of vChain+$_{F}$ due to our strategic trade-off of space for time.
(3) \Chain supports blockchains based on Ethereum virtual machine and Hyperledger Fabric.
}
\begin{table*}[t!]
    \centering
    \tabcolsep=1.5pt
    \renewcommand{\arraystretch}{0.92} 
    \caption{Experimental results of RQ6}
    \begin{tabular}{ccccc}
    \hline
    \multicolumn{1}{c|}{\textbf{Attempt Times}}   & \multicolumn{1}{c|}{\textbf{First Attempt}} & \multicolumn{1}{c|}{\textbf{Second Attempt}} & \multicolumn{1}{c|}{\textbf{Third Attempt}} & \textbf{More-time Attempt} \\ \hline
    \multicolumn{1}{c|}{\textbf{IllusionCAPTCHA}} & 86.95\%                                     & 8.69\%                                       & 0.00\%                                      & 4.34\%                     \\ \hline
    \multicolumn{1}{l}{}                          & \multicolumn{1}{l}{}                        & \multicolumn{1}{l}{}                         & \multicolumn{1}{l}{}                        & \multicolumn{1}{l}{}      
    \end{tabular}
\label{tex:RQ4}
\end{table*}

\subsection{RQ4: LLM Deception by Illusionary Content} 

\noindent\textbf{Motivation.} In this section, we investigate whether illusionary content can effectively deceive the visual processing of LLMs, a critical requirement since a CAPTCHA image must successfully mislead AI systems.


\noindent\textbf{Method.} To rigorously test our generated illusionary content, we adopt the same settings as our empirical study in Section~\ref{sec:empirical_study}, employing 30 generated illusionary images. In contrast to our empirical study, this section aims to demonstrate that LLMs are unable to identify illusionary content. Additionally, unlike other studies, we require precise answers—for example, the correct response should be the name of a concrete bridge
rather than simply bridge.

\noindent\textbf{Result Analysis.} Table~\ref{tex:RQ2} presents the experimental results for LLMs in identifying both illusionary images and text. Our findings indicate that, under both Zero-Shot and COT reasoning settings, neither GPT nor Gemini successfully identified the illusionary images, achieving a 0\% success rate. Notably, when using COT, GPT was able to discern the shape of a hidden character within the image but failed to accurately name the character, even when provided with a hint. These results suggest that visual illusions are particularly challenging for current LLMs to identify, underscoring their effectiveness as natural CAPTCHAs.

\subsection{RQ5: Effectiveness of Inducement Prompts} 

\noindent\textbf{Motivation.} In this section, we explore whether our inducement prompts can effectively guide our intended attackers—GPT-4o and Gemini 1.5 pro 2.0—to select the options we designed.


\noindent\textbf{Method.} In this evaluation, we test GPT-4o and Gemini 1.5 Pro 2.0. We employ two prompt settings Zero-Shot and COT, to assess their performance. Additionally, we allow LLMs two attempts to identify CAPTCHAs, leveraging their ability to retain context across interactions. For this experiment, we utilize 30 IllusionaryCaptchas as the target images.

\noindent\textbf{Result Analysis.} From Table~\ref{tex:RQ3}, we can see that in both attempts, the LLMs consistently selected the option we predicted they would choose, suggesting that the models were identifying only the generated content and not focusing on what we intended human users to recognize. Additionally, we observed that the LLMs often selected the longest description of the images, indicating a tendency to overlook the core elements of the visual illusion. This behavior highlights a key limitation in the LLMs' ability to process visual context effectively, as they appear to prioritize the length or complexity of the descriptions rather than engaging with the nuanced visual details. This finding suggests that while LLMs perform well with textual analysis, they may struggle when tasked with interpreting visual content that requires deeper contextual understanding or inference, such as illusionary images.

\subsection{RQ6: Human Attempts to Pass CAPTCHA}

\textbf{Motivation.} One of the primary aims of our CAPTCHA is to facilitate easier identification of images by human users. Therefore, it is crucial to demonstrate that our CAPTCHA is more user-friendly. To achieve this, we need to assess the number of attempts required for human users to successfully pass the CAPTCHA.

\noindent\textbf{Method.} In this evaluation, we designed a questionnaire structure similar to the one used in Section~\ref{sec:empirical_study} consulting 23 participants to investigate how many attempts human users need to pass our IllusionCAPTCHA. 

\noindent\textbf{Result Analysis.} Table~\ref{tex:RQ4} presents the experimental results of our IllusionCAPTCHA for human users. In this survey, we consulted 23 participants, and we found that 86.95\% were able to pass the CAPTCHA on their first attempt, while 8.69\% succeeded on their second attempt. We also collected feedback on the reasons for failure and discovered that the primary reason participants could not pass was that they did not know the name of the character, although they recognized it as a character from television. Therefore, our CAPTCHA is more friendly for human users to identify, compared to current existing CAPTCHAs.


\section{Discussion}

\revision{In this section, we discuss the comparison to adversarial image-based techniques and address several challenges associated with real-world deployment.}

\revision{\noindent \textbf{Comparison to Adversarial Attacks.} Adversarial image-based techniques typically rely on the addition of carefully crafted noise to images. However, recent studies~\cite{wei2022towards} indicate that these methods often lack transferability and can be easily defeated by a novel LLM with enhanced visual capabilities. Our experimental results demonstrating the LLM's effectiveness in identifying adversarial images is available on our website~\cite{ourwebsite}.}

\revision{\noindent \textbf{Challenge of Cross-cultural Adaptability.} Our experiments reveal that individuals from different countries and age groups may exhibit varying abilities in identifying illusionary images due to cultural differences. To mitigate this issue, we propose incorporating common, everyday images—such as those of fruits, restaurants, and landscapes—to create illusionary images that are universally recognizable. By leveraging familiar objects, we aim to minimize the impact of cultural differences and ensure a consistent user experience across diverse demographics.}

\revision{\noindent \textbf{Challenge of Image Copyright.} In real-world deployment, copyright concerns may render certain images or terms (e.g., \textit{Mickey Mouse}) unsuitable for use. To mitigate these issues, we plan to employ a local AI system to generate images while carefully avoiding problematic words. This approach enables the creation of copyright-free images, thereby ensuring smoother and more compliant deployment in practical scenarios.}
We present RiskHarvester, a risk-based tool to compute a security risk score based on the value of the asset and ease of attack on a database. We calculated the value of asset by identifying the sensitive data categories present in a database from the database keywords. We utilized data flow analysis, SQL, and Object Relational Mapper (ORM) parsing to identify the database keywords. To calculate the ease of attack, we utilized passive network analysis to retrieve the database host information. To evaluate RiskHarvester, we curated RiskBench, a benchmark of 1,791 database secret-asset pairs with sensitive data categories and host information manually retrieved from 188 GitHub repositories. RiskHarvester demonstrates precision of (95\%) and recall (90\%) in detecting database keywords for the value of asset and precision of (96\%) and recall (94\%) in detecting valid hosts for ease of attack. Finally, we conducted an online survey to understand whether developers prioritize secret removal based on security risk score. We found that 86\% of the developers prioritized the secrets for removal with descending security risk scores.

\clearpage
\vfill\eject 
\bibliographystyle{ACM-Reference-Format}
\balance
\bibliography{refs}

\end{document}

